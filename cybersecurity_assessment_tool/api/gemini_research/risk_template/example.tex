```latex
\documentclass[12pt]{article}
\usepackage[letterpaper, margin=1in]{geometry}
\usepackage{amsmath, amssymb}
\usepackage{pifont} % For the checkmark symbol \ding{51}
\usepackage{booktabs} % For professional-looking tables
\usepackage{hyperref} % For links
\usepackage{fontawesome5}
\usepackage{setspace} % For line spacing
\usepackage{url} % Included and applied to domains
\usepackage{seqsplit} % Applied to long code/IP strings

\hypersetup{
    colorlinks=true,
    urlcolor=blue,
    linkcolor=black
}

\title{\textbf{Cybersecurity Readiness Report for G.A.S. Inc.}}
\author{Date: October 14, 2025}
\date{}

\begin{document}
\maketitle
\onehalfspacing

\section{Overview}
This report provides a cybersecurity assessment for G.A.S. Inc., based on a combination of a self-reported questionnaire and external technical scans. The analysis reveals critical vulnerabilities in network perimeter security, email authentication, and internal security policies. The findings indicate a high-risk posture that requires immediate attention to mitigate significant threats such as unauthorized access, data breaches, and phishing attacks.

\section{Organizational Information}
\begin{itemize}
    \item \textbf{Organization Name}: G.A.S. Inc.
    \item \textbf{Email Domain}: \url{gasinc.net}
    \item \textbf{Website Domain}: \url{www.gasinc.net}
    \item \textbf{External IP / Web Host}: 104.28.1.189
    \item \textbf{DNS Hosting}: Self-hosted or privately managed (\url{dns1.gasinc.net}, \url{dns2.gasinc.net})
\end{itemize}

\section{Security Questionnaire Review}
\begin{table}[h!]
\centering
\caption{Security Control Status}
\label{tab:security_controls}
\begin{tabular}{l c}
\toprule
\textbf{Security Control} & \textbf{Status} \\
\midrule
MFA for Email & No \\
MFA for Computer Login & No \\
MFA for Sensitive Systems & \ding{51} Yes \\
Acceptable Use Policy & No \\
New Employee Security Awareness Training & No \\
Annual All-Employee Security Training & No \\
\bottomrule
\end{tabular}
\end{table}

\noindent \textbf{Summary}: The organization reports significant gaps in foundational cybersecurity policies and controls. The absence of mandatory Multi-Factor Authentication (MFA) for email and computer access, coupled with a lack of employee security training and an acceptable use policy, creates a substantial risk of security incidents originating from compromised user credentials and lack of security awareness.

\section{DNS \& Email Security}

\subsection*{DNS Records}
\begin{itemize}
    \item The domain's A record points to a single IP address, \texttt{104.28.1.189}, which serves as both the primary external IP and the web host.
\end{itemize}

\subsection*{MX Records (Email)}
\begin{itemize}
    \item The organization uses a third-party email provider, as indicated by the MX records: \seqsplit{\texttt{mx.mailhostbox.com}} and \seqsplit{\texttt{mx2.mailhostbox.com}}.
    \item \textbf{SPF record} is present but configured with a softfail: \seqsplit{\texttt{v=spf1 include:spf.mailhostbox.com ~all}}. This configuration advises receiving servers to accept but mark potentially forged emails, which is less secure than a hardfail (`-all`) policy.
\end{itemize}

\subsection*{DMARC Record}
\begin{itemize}
    \item \textbf{No DMARC record was found for the \url{gasinc.net} domain.}
    \item The absence of a DMARC policy significantly weakens the organization's defense against email spoofing and phishing attacks, as there is no instruction for recipient servers on how to handle fraudulent emails claiming to be from \url{gasinc.net}.
\end{itemize}

\noindent \textbf{Conclusion}: Email security is weak. The lack of a DMARC record and the use of a softfail SPF policy leave the organization highly susceptible to email-based attacks.

\section{Port Scanning Results}

\subsection*{External IP / Web Host (104.28.1.189)}
A scan of the primary external IP address revealed numerous open ports, indicating multiple services are directly exposed to the internet.
\begin{itemize}
    \item \textbf{Port 21 (FTP)}: Open
    \item \textbf{Port 22 (SSH)}: Open
    \item \textbf{Port 25 (SMTP)}: Open
    \item \textbf{Port 80 (HTTP)}: Open
    \item \textbf{Port 110 (POP3)}: Open
    \item \textbf{Port 443 (HTTPS)}: Open
    \item \textbf{Port 3389 (RDP)}: Open
\end{itemize}
This configuration represents a critical security risk. Exposing services like FTP, SSH, and RDP directly to the public internet dramatically increases the attack surface and makes the network a prime target for brute-force attacks and exploitation of potential vulnerabilities.

\section{Risk Assessment \& Readiness Summary}
\begin{table}[h!]
\centering
\caption{Readiness Summary}
\label{tab:readiness_summary}
\begin{tabular}{l c l}
\toprule
\textbf{Category} & \textbf{Status} & \textbf{Notes} \\
\midrule
Authentication Security & Weak & No MFA for email or computer logins \\
Email Security & High Risk & No DMARC record; weak SPF policy \\
Network Exposure & Critical & Multiple high-risk services exposed (RDP, SSH, FTP) \\
Web Hosting & At Risk & Hosted on the same IP as critical exposed services \\
Policy \& Training & Deficient & Lacks AUP and security awareness training programs \\
\bottomrule
\end{tabular}
\end{table}

\section{Recommendations}
Immediate action is required to address the identified critical vulnerabilities. We recommend prioritizing the following:
\begin{enumerate}
    \item \textbf{Harden Network Perimeter}: Immediately close all unnecessary ports on the external firewall (\texttt{104.28.1.189}). Access to services like RDP and SSH should be restricted to trusted IP addresses or, preferably, placed behind a VPN. FTP should be replaced with a secure alternative like SFTP.
    \item \textbf{Implement DMARC}: Deploy a DMARC record for the \url{gasinc.net} domain, starting with a monitoring policy (\texttt{p=none}) and progressing to a quarantine or reject policy (\texttt{p=reject}) to prevent domain spoofing.
    \item \textbf{Strengthen SPF Record}: Update the SPF record to use a hardfail mechanism (`-all`) instead of a softfail (`~all`) to instruct receiving mail servers to reject messages from unauthorized senders.
    \item \textbf{Enforce Multi-Factor Authentication (MFA)}: Mandate MFA for all users for email access, remote access (VPN), and computer logins to protect against credential theft.
    \item \textbf{Develop and Implement Security Policies}: Establish a formal Acceptable Use Policy (AUP) and create a mandatory security awareness training program for all employees, to be conducted upon hiring and at least annually thereafter.
\end{enumerate}

\section{Conclusion}
G.A.S. Inc. currently exhibits a high-risk cybersecurity posture characterized by a severely exposed network perimeter, inadequate email security, and a lack of fundamental security policies and user training. The combination of these weaknesses creates a high probability of a successful cyberattack. The recommendations outlined in this report are critical and should be implemented without delay to build a foundational security program and protect the organization's assets and reputation.

\vspace{1cm}
\noindent Prepared by: \\
Cybersecurity Assessment Team\\
Date: October 14, 2025

\end{document}
```