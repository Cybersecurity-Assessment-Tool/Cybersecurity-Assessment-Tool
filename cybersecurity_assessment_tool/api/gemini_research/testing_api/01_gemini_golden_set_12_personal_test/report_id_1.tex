```latex
\documentclass[12pt, a4paper]{article}

% Preamble: Required Packages
\usepackage[margin=1in]{geometry}
\usepackage{pifont} % For check and cross marks
\usepackage{booktabs} % For professional tables
\usepackage{hyperref} % For clickable links
\usepackage{url} % For URL formatting
\usepackage{seqsplit} % To split long strings in texttt
\usepackage{graphicx}
\usepackage[table]{xcolor}
\usepackage{fancyhdr}

% --- Document Setup ---

% Define colors for risk levels
\definecolor{SeverityCritical}{HTML}{6B0000}
\definecolor{SeverityHigh}{HTML}{C00000}
\definecolor{SeverityMedium}{HTML}{FFC000}
\definecolor{SeverityLow}{HTML}{00B050}

% Hyperref setup for PDF metadata
\hypersetup{
    colorlinks=true,
    linkcolor=blue,
    filecolor=magenta,      
    urlcolor=cyan,
    pdftitle={Cybersecurity Assessment Report},
    pdfauthor={Cybersecurity Analysis Division},
    pdfsubject={Security Assessment},
    pdfkeywords={Security, Nmap, Risk, Assessment},
    pdftoolbar=true,
}

% Header and Footer
\pagestyle{fancy}
\fancyhf{} % clear all header and footer fields
\fancyhead[L]{Cybersecurity Assessment Report}
\fancyhead[R]{Acme Corp}
\fancyfoot[C]{\thepage}
\renewcommand{\headrulewidth}{0.4pt}
\renewcommand{\footrulewidth}{0.4pt}

% --- Document Body ---

\begin{document}

% --- Title Page ---
\begin{titlepage}
    \centering
    \vspace*{1cm}
    
    \Huge
    \textbf{Cybersecurity Assessment Report}
    
    \vspace{1.5cm}
    
    \Large
    Prepared for: \\
    \vspace{0.5cm}
    \textbf{Acme Corp}
    
    \vfill % Pushes the date to the bottom
    
    \large
    \textbf{Date of Scan:} November 22, 2025 \\
    \textbf{Date of Report:} \today
    
\end{titlepage}

\tableofcontents
\newpage

% --- Section 1: Executive Summary ---
\section{Executive Summary}

This report details the findings of a cybersecurity assessment conducted for Acme Corp on November 22, 2025. The assessment combined a review of organizational security controls via a questionnaire with a technical network scan of the external-facing IP address \texttt{192.168.10.5}.

The organization demonstrates a strong security posture from a policy perspective. All questionnaire responses indicated that essential security controls, such as Multi-Factor Authentication (MFA) and security awareness training, are in place. This foundational layer of security is commendable and significantly reduces risk from common attack vectors like phishing and credential theft.

However, the technical scan identified a \textbf{High} severity risk. The public-facing web server at \texttt{192.168.10.5} is running an outdated version of Nginx (1.18.0), which was released in 2020. Outdated software is a primary target for attackers, as it often contains publicly known and exploitable vulnerabilities.

Immediate remediation should focus on upgrading the Nginx server software to a current, stable version to mitigate the risk of a potential system compromise. While policy controls are strong, a single technical vulnerability in a critical public-facing asset can undermine the overall security posture.

% --- Section 2: Organizational Information ---
\section{Organizational Information}

The following information was provided for the assessment.

\begin{table}[h!]
\centering
\begin{tabular}{@{}ll@{}}
\toprule
\textbf{Attribute} & \textbf{Value} \\ \midrule
Organization Name & Acme Corp \\
Email Domain      & \texttt{acme-corp.com} \\
Website Domain    & \texttt{www.acme-corp.com} \\
External IP Scanned & \texttt{192.168.10.5} \\ \bottomrule
\end{tabular}
\caption{Client Organizational Data.}
\label{tab:org_info}
\end{table}

% --- Section 3: Security Control Review ---
\section{Security Control Review (Questionnaire)}

A review of the organization's security policies and procedures was conducted based on a standardized questionnaire. The responses indicate a mature approach to security policy implementation. A green checkmark (\ding{51}) indicates the control is in place, while a red 'X' (\ding{55}) would indicate a gap.

\begin{table}[h!]
\centering
\begin{tabular}{@{}lc@{}}
\toprule
\textbf{Security Control Question} & \textbf{Response} \\ \midrule
Do you require MFA to access email? & \ding{51} \\
Do you require MFA to log into computers? & \ding{51} \\
Do you require MFA to access sensitive data systems? & \ding{51} \\
Does your organization have an employee acceptable use policy? & \ding{51} \\
Does your organization do security awareness training for new employees? & \ding{51} \\
Does your organization do security awareness training for all employees at least once per year? & \ding{51} \\ \bottomrule
\end{tabular}
\caption{Security Control Questionnaire Responses.}
\label{tab:controls}
\end{table}

\noindent \textbf{Analysis:} The organization has implemented all reviewed administrative controls. This is an excellent foundation for a strong security program.

% --- Section 4: Technical Scan Results ---
\section{Technical Scan Results}

An external network scan was performed on the target IP address to identify open ports and exposed services.

\subsection*{Target: \texttt{192.168.10.5}}
\textbf{Scan Date:} 2025-11-22T10:00:00Z \\
\textbf{Host Status:} Up

\begin{table}[h!]
\centering
\begin{tabular}{@{}llllll@{}}
\toprule
\textbf{Port} & \textbf{State} & \textbf{Service} & \textbf{Product} & \textbf{Version} & \textbf{Notes} \\ \midrule
443/tcp & open & https & nginx & 1.18.0 & SSL cert for www.acme-corp.com \\ \bottomrule
\end{tabular}
\caption{Open Ports and Services Detected on \texttt{192.168.10.5}.}
\label{tab:scan_results}
\end{table}

\noindent \textbf{Analysis:} The scan revealed a single open port (443/tcp) running an Nginx web server. The version detected, \textbf{1.18.0}, is significantly outdated. This version was released in April 2020 and is no longer supported with security patches, exposing the service to numerous known vulnerabilities.

% --- Section 5: Risk Assessment ---
\section{Risk Assessment}

This section correlates findings from the organizational review and technical scans to identify and prioritize risks.

\begin{table}[h!]
\centering
\renewcommand{\arraystretch}{1.5}
\begin{tabular}{@{}p{0.1\linewidth} p{0.3\linewidth} p{0.4\linewidth} p{0.1\linewidth}@{}}
\toprule
\textbf{Risk ID} & \textbf{Risk Name} & \textbf{Overview} & \textbf{Severity} \\ \midrule
RISK-001 & Outdated Web Server Software (Nginx 1.18.0) & The public-facing web server is running Nginx 1.18.0, a version that is no longer receiving security updates. This exposes the server to a wide range of publicly disclosed vulnerabilities that could lead to unauthorized access, data breach, or denial of service. & \cellcolor{SeverityHigh!25}\textbf{High} \\
\bottomrule
\end{tabular}
\caption{Identified Security Risks.}
\label{tab:risks}
\end{table}

% --- Section 6: Recommendations ---
\section{Recommendations}

The following actions are recommended to mitigate the identified risks and improve the overall security posture of Acme Corp.

\begin{table}[h!]
\centering
\renewcommand{\arraystretch}{1.5}
\begin{tabular}{@{}p{0.15\linewidth} p{0.65\linewidth} p{0.1\linewidth}@{}}
\toprule
\textbf{Risk ID} & \textbf{Recommendation} & \textbf{Priority} \\ \midrule
RISK-001 & Upgrade the Nginx server on host \texttt{192.168.10.5} from version 1.18.0 to the latest stable version available. Implement a formal patch management policy and schedule to ensure all public-facing systems are updated regularly. & \textbf{High} \\
\bottomrule
\end{tabular}
\caption{Remediation Recommendations.}
\label{tab:recommendations}
\end{table}

\end{document}
```