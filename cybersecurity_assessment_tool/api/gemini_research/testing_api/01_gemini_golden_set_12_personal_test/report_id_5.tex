```latex
\documentclass[12pt]{article}

% --- PACKAGES ---
\usepackage[margin=1in]{geometry}
\usepackage{pifont} % For checkmarks and crosses
\usepackage{booktabs} % For professional tables
\usepackage{hyperref} % For clickable links
\usepackage{url}      % For URL formatting
\usepackage{seqsplit} % For splitting long strings
\usepackage{graphicx} % For logo (placeholder)
\usepackage{xcolor}   % For colors in text

% --- DOCUMENT METADATA ---
\hypersetup{
    colorlinks=true,
    linkcolor=blue,
    filecolor=magenta,      
    urlcolor=cyan,
    pdftitle={Cybersecurity Assessment Report},
    pdfauthor={Cybersecurity Analysis Division},
    pdfsubject={Security Posture Analysis},
    pdfkeywords={Security, Nmap, Risk, Assessment},
}

% --- CUSTOM COMMANDS ---
\newcommand{\yes}{\ding{51}}
\newcommand{\no}{\ding{55}}
\newcommand{\unknown}{?}

\begin{document}

% --- TITLE PAGE ---
\begin{titlepage}
    \centering
    \vspace*{1cm}
    
    \rule{\textwidth}{1.5pt}\vspace*{0.5cm}
    \Huge\textbf{Cybersecurity Assessment Report}
    \vspace*{0.5cm}\rule{\textwidth}{1.5pt}
    
    \vspace{2cm}
    
    {\Large \textbf{Prepared For:} \\ EdgeCorp}
    
    \vspace{1.5cm}
    
    {\Large \textbf{Report Date:} \\ \today}
    
    \vfill
    
    {\large \textbf{Generated By:} \\ Cybersecurity Analysis Division}
    
\end{titlepage}

\tableofcontents
\newpage

% --- 1. EXECUTIVE SUMMARY ---
\section{Executive Summary}
This report provides a comprehensive analysis of the cybersecurity posture of EdgeCorp, based on network scans, organizational data, and a review of pre-existing risks. The assessment identified critical vulnerabilities that require immediate attention.

The primary finding is the direct exposure of the Remote Desktop Protocol (RDP) on TCP port 3389 for the host at \texttt{10.10.10.51}. This finding expands upon a previously identified risk on a separate host (\texttt{10.10.10.50}), indicating a systemic configuration issue rather than an isolated incident.

This critical technical vulnerability is severely compounded by a complete lack of Multi-Factor Authentication (MFA) across all essential services, including email, computer logins, and access to sensitive data. The combination of exposed RDP and weak authentication creates a high-probability attack vector for ransomware and unauthorized access.

Furthermore, foundational security controls, such as an acceptable use policy and security awareness training, are not confirmed to be in place. These gaps in governance weaken the organization's overall defense against common cyber threats.

Immediate remediation is required to close the exposed network services and implement MFA. Strategic initiatives should focus on developing and enforcing core security policies to build a more resilient security foundation.

% --- 2. ORGANIZATIONAL INFORMATION ---
\section{Organizational Information}
The following information was provided for the assessment. This data forms the basis for understanding the organizational context of the technical findings.

\begin{itemize}
    \item \textbf{Organization Name:} EdgeCorp
    \item \textbf{Email Domain:} \texttt{edge.com}
    \item \textbf{Website Domain:} \url{http://edge.com}
    \item \textbf{Assessed External IP:} \texttt{10.10.10.51}
\end{itemize}

% --- 3. SECURITY CONTROL REVIEW ---
\section{Security Control Review}
A review of administrative and policy-based security controls was conducted via a questionnaire. The results below highlight significant gaps in foundational security practices. A red cross (\no) indicates a negative response and a critical control gap. A question mark (?) indicates an "Unknown" response, which represents a potential gap that requires verification.

\begin{table}[h!]
\centering
\caption{Security Controls Questionnaire Results}
\begin{tabular}{p{0.7\linewidth} c}
\toprule
\textbf{Control Question} & \textbf{Status} \\
\midrule
Do you require MFA to access email? & \textcolor{red}{\no} \\
Do you require MFA to log into computers? & \textcolor{red}{\no} \\
Do you require MFA to access sensitive data systems? & \textcolor{red}{\no} \\
Does your organization have an employee acceptable use policy? & \unknown \\
Does your organization do security awareness training for new employees? & \unknown \\
Does your organization do security awareness training for all employees at least once per year? & \unknown \\
\bottomrule
\end{tabular}
\end{table}

% --- 4. TECHNICAL SCAN RESULTS ---
\section{Technical Scan Results}
An external network scan was performed on the target IP address to identify open ports and exposed services.

\begin{itemize}
    \item \textbf{Target IP:} \texttt{10.10.10.51}
    \item \textbf{Scan Tool:} Nmap
    \item \textbf{Host Status:} Up
\end{itemize}

The following table details the open ports discovered on the target system.

\begin{table}[h!]
\centering
\caption{Open Ports on \texttt{10.10.10.51}}
\begin{tabular}{l l l l}
\toprule
\textbf{Port} & \textbf{State} & \textbf{Service Name} & \textbf{Description} \\
\midrule
3389/tcp & Open & \texttt{ms-wbt-server} & Microsoft Remote Desktop Protocol (RDP) \\
\bottomrule
\end{tabular}
\end{table}

\subsection{Analysis of Technical Findings}
The scan confirms that TCP port 3389 is open, exposing the Microsoft Remote Desktop Protocol (RDP) service directly to the internet. RDP is a primary target for attackers who use brute-force password guessing, credential stuffing, and exploitation of known vulnerabilities to gain unauthorized access to internal networks.

% --- 5. CONSOLIDATED RISK ASSESSMENT ---
\section{Consolidated Risk Assessment}
This section correlates the findings from the security control review, the technical scan, and pre-existing risk data to provide a holistic view of the current risk posture.

\begin{table}[h!]
\centering
\caption{Identified Security Risks}
\begin{tabular}{p{0.2\linewidth} p{0.6\linewidth} p{0.1\linewidth}}
\toprule
\textbf{Risk Name} & \textbf{Overview} & \textbf{Severity} \\
\midrule
\textbf{Systemic RDP Exposure} & The RDP service is exposed to the internet on multiple hosts (\texttt{10.10.10.50}, \texttt{10.10.10.51}). This allows attackers to directly target a critical remote access service. & \textbf{Critical} \\
\addlinespace
\textbf{Lack of Multi-Factor Authentication} & MFA is not enforced for any critical access points (email, computer login, sensitive data). This significantly increases the risk of account compromise via stolen or weak credentials. & \textbf{Critical} \\
\addlinespace
\textbf{Policy \& Training Gaps} & Core security governance controls, including an acceptable use policy and security awareness training, are not confirmed to be in place. This leads to inconsistent security practices and a workforce that is more susceptible to social engineering attacks. & \textbf{High} \\
\bottomrule
\end{tabular}
\end{table}

% --- 6. RECOMMENDATIONS ---
\section{Recommendations}
The following actions are recommended to mitigate the identified risks. Recommendations are prioritized based on severity and potential impact.

\subsection{Immediate Priority (Containment)}
\begin{enumerate}
    \item \textbf{Block RDP Access:} Immediately implement a firewall rule to block all inbound internet traffic to TCP port 3389 on host \texttt{10.10.10.51}.
    \item \textbf{Verify Existing Blocks:} Confirm that a similar firewall rule is in place and effective for host \texttt{10.10.10.50}.
\end{enumerate}

\subsection{High Priority (Remediation)}
\begin{enumerate}
    \setcounter{enumi}{2}
    \item \textbf{Implement MFA:} Procure and deploy an MFA solution. Prioritize enforcement on the following systems in order:
    \begin{itemize}
        \item External-facing services (e.g., VPN, OWA).
        \item Email (e.g., Office 365, G Suite).
        \item Access to all sensitive data systems.
        \item All standard user and privileged computer logins.
    \end{itemize}
    \item \textbf{Deploy Secure Remote Access:} If remote access is required, implement a Virtual Private Network (VPN) solution that requires MFA for authentication. All RDP access should occur exclusively through the secure VPN tunnel.
\end{enumerate}

\subsection{Medium Priority (Strengthening Governance)}
\begin{enumerate}
    \setcounter{enumi}{4}
    \item \textbf{Develop Acceptable Use Policy (AUP):} Create and enforce an AUP that clearly defines the rules for using company IT assets, data, and networks.
    \item \textbf{Establish Security Awareness Training:} Implement a mandatory security awareness training program for all new hires and conduct annual refresher training for all employees. The training should cover topics such as phishing, password security, and social engineering.
\end{enumerate}

\end{document}
```