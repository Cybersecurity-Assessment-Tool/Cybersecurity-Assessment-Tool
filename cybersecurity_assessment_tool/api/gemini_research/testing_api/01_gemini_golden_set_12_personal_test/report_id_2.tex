```latex
\documentclass[12pt]{article}

% 1. DOCUMENT SETUP & PACKAGES
\usepackage[margin=1in]{geometry}
\usepackage{pifont} % For checkmarks and crosses
\usepackage{booktabs} % For professional tables
\usepackage{hyperref} % For clickable links
\usepackage{url} % For URL formatting
\usepackage{seqsplit} % For splitting long strings in tt font
\usepackage{graphicx}
\usepackage{xcolor}

% --- Document Metadata ---
\title{Cybersecurity Assessment Report}
\author{Cybersecurity Analysis Division}
\date{\today}

% --- Hyperref Setup ---
\hypersetup{
    colorlinks=true,
    linkcolor=blue,
    filecolor=magenta,      
    urlcolor=cyan,
    pdftitle={Cybersecurity Assessment Report},
    pdfpagemode=FullScreen,
}

\begin{document}

\maketitle
\hrule
\begin{center}
\textbf{CONFIDENTIAL} \\
\vspace{5mm}
Prepared for: \textbf{[Organization Name]}
\end{center}
\hrule
\vspace{10mm}

\tableofcontents
\newpage

% 2. EXECUTIVE SUMMARY
\section{Executive Summary}

This report details the findings of a cybersecurity assessment which synthesized data from a network scan, an organizational security questionnaire, and a review of pre-existing risks. The assessment reveals a critically deficient security posture that exposes the organization to significant threats, including data breach, ransomware, and unauthorized access.

Key findings of critical severity include:
\begin{itemize}
    \item \textbf{Exposed Vulnerable Service:} An internet-facing FTP server is running a dangerously outdated version of \texttt{vsftpd} (2.3.4) from 2011, which is known to contain a critical backdoor vulnerability. The service is further misconfigured to allow anonymous access.
    \item \textbf{Complete Lack of Multi-Factor Authentication (MFA):} MFA is not enforced for any system, including email, computer logins, or access to sensitive data. This represents a fundamental failure in identity and access management.
    \item \textbf{Absence of Foundational Security Policies:} The organization lacks a basic employee acceptable use policy and does not conduct any form of security awareness training.
    \item \textbf{Use of End-of-Life Systems:} Workstations are running the Windows 7 operating system, which is no longer supported by Microsoft and does not receive security updates.
\end{itemize}

Immediate and decisive action is required to remediate these vulnerabilities. This report provides a consolidated risk assessment and a prioritized list of actionable recommendations to mitigate the identified threats and strengthen the organization's overall security posture.

% 3. ORGANIZATIONAL INFORMATION
\section{Organizational Information}

The following information was provided for the assessment. Placeholders are used where data was unavailable.

\begin{tabular}{@{}ll}
    \toprule
    \textbf{Attribute} & \textbf{Value} \\
    \midrule
    Organization Name & \textbf{[Organization Name]} \\
    Email Domain & \texttt{[Domain]} \\
    Website Domain & Unknown \\
    External IP Address & \texttt{10.0.0.15} \\
    \bottomrule
\end{tabular}

% 4. SECURITY CONTROL REVIEW (QUESTIONNAIRE)
\section{Security Control Review}

A review of the organization's security controls was conducted via a questionnaire. The responses indicate a complete absence of fundamental security practices. A red 'X' (\ding{55}) signifies a negative response, highlighting a control gap.

\begin{table}[h!]
\centering
\caption{Security Control Questionnaire Results}
\begin{tabular}{@{}p{0.8\linewidth}c@{}}
    \toprule
    \textbf{Control Question} & \textbf{Response} \\
    \midrule
    Do you require MFA to access email? & \textcolor{red}{\ding{55}} \\
    Do you require MFA to log into computers? & \textcolor{red}{\ding{55}} \\
    Do you require MFA to access sensitive data systems? & \textcolor{red}{\ding{55}} \\
    \addlinespace
    Does your organization have an employee acceptable use policy? & \textcolor{red}{\ding{55}} \\
    \addlinespace
    Does your organization do security awareness training for new employees? & \textcolor{red}{\ding{55}} \\
    Does your organization do security awareness training for all employees at least once per year? & \textcolor{red}{\ding{55}} \\
    \bottomrule
\end{tabular}
\end{table}

\paragraph{Analysis:} The lack of MFA for email and sensitive systems is a critical risk, making the organization highly susceptible to phishing and credential stuffing attacks. The absence of security policies and training creates a high-risk environment where employees are unaware of their security responsibilities, increasing the likelihood of human error leading to a security incident.

% 5. TECHNICAL SCAN RESULTS
\section{Technical Scan Results}

An external network scan was performed on the provided IP address to identify exposed services and potential vulnerabilities.

\begin{itemize}
    \item \textbf{Target IP:} \texttt{10.0.0.15}
    \item \textbf{Scan Date:} \today
\end{itemize}

The scan identified one open port with a critically vulnerable service.

\begin{table}[h!]
\centering
\caption{Open Port Analysis}
\begin{tabular}{@{}lllll@{}}
    \toprule
    \textbf{Port} & \textbf{State} & \textbf{Service} & \textbf{Version} & \textbf{Details} \\
    \midrule
    21/tcp & Open & FTP & vsftpd 2.3.4 & Anonymous FTP login allowed. \\
    \bottomrule
\end{tabular}
\end{table}

\paragraph{Analysis:} The File Transfer Protocol (FTP) service is running \textbf{vsftpd version 2.3.4}. This specific version, released in 2011, contains a well-known and critical backdoor vulnerability (\textbf{CVE-2011-2523}). If exploited, this vulnerability grants an attacker a command shell on the underlying server, leading to a complete system compromise. The configuration allowing anonymous login further lowers the barrier for an attacker, as no credentials are required to interact with this vulnerable service.

% 6. CONSOLIDATED RISK ASSESSMENT
\section{Consolidated Risk Assessment}

The following table synthesizes findings from the security control review, technical scan, and pre-existing risk data into a prioritized list.

\begin{table}[h!]
\centering
\caption{Summary of Identified Risks}
\begin{tabular}{@{}p{0.15\linewidth}p{0.6\linewidth}l@{}}
    \toprule
    \textbf{Risk Name} & \textbf{Description} & \textbf{Severity} \\
    \midrule
    \textbf{Vulnerable FTP Service} & A public-facing FTP server is running a version with a known remote code execution backdoor (CVE-2011-2523) and allows anonymous access. & \textbf{Critical} \\
    \addlinespace
    \textbf{Lack of MFA} & Multi-Factor Authentication is not enforced on any system, leaving accounts vulnerable to compromise via stolen credentials. & \textbf{Critical} \\
    \addlinespace
    \textbf{Absence of Security Policies \& Training} & The lack of an acceptable use policy and security awareness training program exposes the organization to significant risk from insider threats and human error. & \textbf{High} \\
    \addlinespace
    \textbf{End-of-Life OS} & Workstations are running Windows 7, an unsupported operating system that no longer receives security patches, making them easy targets for malware. & \textbf{Medium} \\
    \bottomrule
\end{tabular}
\end{table}

% 7. RECOMMENDATIONS
\section{Recommendations}

The following actions are recommended to mitigate the identified risks. They are prioritized based on severity and potential impact.

\subsection{Immediate Actions (Critical Risks)}
\begin{enumerate}
    \item \textbf{Remediate Vulnerable FTP Service:}
    \begin{itemize}
        \item \textbf{Immediately} take the FTP server offline to prevent exploitation.
        \item Conduct a forensic analysis of the server to check for signs of existing compromise.
        \item If a file transfer service is required, decommission the current server and deploy a new, fully patched server using a secure protocol such as SFTP (SSH File Transfer Protocol).
        \item Ensure anonymous access is disabled.
    \end{itemize}
    \item \textbf{Implement Multi-Factor Authentication (MFA):}
    \begin{itemize}
        \item Prioritize the deployment of MFA for all remote access, administrator accounts, and email.
        \item Develop a plan to roll out MFA for all employee logins to computers and sensitive systems.
    \end{itemize}
\end{enumerate}

\subsection{High-Priority Actions}
\begin{enumerate}
    \setcounter{enumi}{2} % Continue numbering
    \item \textbf{Develop Security Policies:}
    \begin{itemize}
        \item Draft and implement a formal Employee Acceptable Use Policy that outlines the rules for using company IT assets.
        \item Develop a corresponding password policy that enforces complexity and length requirements.
    \end{itemize}
    \item \textbf{Establish Security Awareness Training:}
    \begin{itemize}
        \item Implement a mandatory security training program for all new hires.
        \item Conduct annual security awareness training for all staff, covering topics such as phishing, social engineering, and data handling best practices.
    \end{itemize}
\end{enumerate}

\subsection{Medium-Priority Actions}
\begin{enumerate}
    \setcounter{enumi}{4} % Continue numbering
    \item \textbf{Upgrade End-of-Life Systems:}
    \begin{itemize}
        \item Create an inventory of all workstations running Windows 7.
        \item Develop and execute a plan to upgrade or replace these systems with a modern, supported operating system (e.g., Windows 11).
    \end{itemize}
\end{enumerate}

\end{document}
```