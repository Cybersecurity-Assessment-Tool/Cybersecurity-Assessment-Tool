```latex
\documentclass[12pt]{article}

% Preamble: Required Packages
\usepackage[margin=1in]{geometry}
\usepackage{pifont} % For checkmarks and crosses
\usepackage{booktabs} % For professional tables
\usepackage{hyperref} % For clickable links and TOC
\usepackage{url} % For formatting URLs
\usepackage{seqsplit} % For splitting long strings in texttt
\usepackage{graphicx}
\usepackage{xcolor}
\usepackage{fancyhdr}

% --- Document Setup ---
\hypersetup{
    colorlinks=true,
    linkcolor=blue,
    filecolor=magenta,      
    urlcolor=cyan,
    pdftitle={Cybersecurity Assessment Report},
    pdfpagemode=FullScreen,
}

% Define colors for severity
\definecolor{sev_critical}{HTML}{990000}
\definecolor{sev_high}{HTML}{DD0000}
\definecolor{sev_medium}{HTML}{FF8C00}
\definecolor{sev_low}{HTML}{F0E68C}

% Header and Footer
\pagestyle{fancy}
\fancyhf{}
\fancyhead[L]{Cybersecurity Assessment Report}
\fancyhead[R]{Localhost}
\fancyfoot[C]{\thepage}
\renewcommand{\headrulewidth}{0.4pt}
\renewcommand{\footrulewidth}{0.4pt}

% --- Document Start ---
\begin{document}

% --- Title Page ---
\begin{titlepage}
    \centering
    \vspace*{1cm}
    \Huge
    \textbf{Cybersecurity Assessment Report}
    
    \vspace{1.5cm}
    \Large
    Prepared for: \\
    \vspace{0.5cm}
    \textbf{Localhost}
    
    \vfill
    
    \large
    Date of Report: \today
    
    \vspace{0.8cm}
    \small
    This report contains sensitive information. Distribution should be limited to authorized personnel only.
\end{titlepage}

\tableofcontents
\newpage

% --- Section 1: Executive Summary ---
\section{Executive Summary}
This report provides a cybersecurity assessment for \textbf{Localhost}, based on network scans, organizational data, and a review of existing risks. The analysis was conducted on \today.

The assessment reveals a \textbf{critical risk posture}. A network scan confirmed that an administrative service (SSH on port 22) is exposed on the external IP address \texttt{127.0.0.1}. This finding validates a pre-existing critical risk, "Localhost Exposed," which has a CVSS score of 10.0.

Furthermore, a review of security controls indicates a significant gap: Multi-Factor Authentication (MFA) is not confirmed to be in use for email, computer logins, or access to sensitive systems. The absence of this fundamental control dramatically increases the risk of unauthorized access, especially when combined with an exposed administrative service.

Immediate remediation is required to address the exposed service and implement MFA across the organization to mitigate the high likelihood of a security breach.

% --- Section 2: Organizational Information ---
\section{Organizational Information}
The following details were provided for the assessment. Placeholders are used where data was unavailable.

\begin{table}[h!]
\centering
\begin{tabular}{@{}ll@{}}
\toprule
\textbf{Attribute} & \textbf{Value} \\ \midrule
Organization Name & \textbf{Localhost} \\
Email Domain & \texttt{[Domain]} \\
Website Domain & N/A \\
External IP Address & \texttt{127.0.0.1} \\ \bottomrule
\end{tabular}
\caption{Client Organizational Data.}
\label{tab:org_data}
\end{table}

% --- Section 3: Security Control Review ---
\section{Security Control Review}
A review of key security controls was performed based on a questionnaire. Responses of "No" or "N/A" (Not Applicable/Not Answered) are treated as control gaps, as their protective status cannot be confirmed.

\begin{table}[h!]
\centering
\begin{tabular}{@{}p{0.6\textwidth}cc@{}}
\toprule
\textbf{Control Question} & \textbf{Response} & \textbf{Assessment} \\ \midrule
Do you require MFA to access email? & N/A \hspace{0.5cm} \ding{55} & \textcolor{sev_critical}{\textbf{Critical Gap}} \\
Do you require MFA to log into computers? & N/A \hspace{0.5cm} \ding{55} & \textcolor{sev_high}{\textbf{High Risk}} \\
Do you require MFA to access sensitive data systems? & N/A \hspace{0.5cm} \ding{55} & \textcolor{sev_critical}{\textbf{Critical Gap}} \\ \bottomrule
\end{tabular}
\caption{Security Controls Questionnaire Analysis. The checkmark (\ding{51}) indicates a positive control, while the cross (\ding{55}) indicates a control gap.}
\label{tab:controls}
\end{table}

The consistent lack of confirmed MFA implementation is a major security weakness. Credential theft, a common attack vector, could lead to a complete system compromise without this secondary authentication layer.

% --- Section 4: Technical Scan Results ---
\section{Technical Scan Results}
An external network scan was conducted to identify exposed services and potential vulnerabilities.

\begin{itemize}
    \item \textbf{Scan Target:} \texttt{127.0.0.1}
    \item \textbf{Scan Date:} Scan data imported on \today.
\end{itemize}

The following table details the open ports discovered on the target system.

\begin{table}[h!]
\centering
\begin{tabular}{@{}lllll@{}}
\toprule
\textbf{Port} & \textbf{Protocol} & \textbf{State} & \textbf{Service} & \textbf{Details} \\ \midrule
22 & TCP & open & ssh & Service version not enumerated. \\
   &   &   &   & \textit{\textcolor{sev_high}{High Risk: SSH is a common target for brute-force attacks.}} \\
\bottomrule
\end{tabular}
\caption{Open Ports Detected on \texttt{127.0.0.1}.}
\label{tab:nmap_results}
\end{table}

\subsection{Analysis of Technical Findings}
The scan confirms that the Secure Shell (SSH) service on port 22 is open to the public. SSH is a powerful administrative protocol, and its direct exposure is highly discouraged. This finding directly validates the critical risk identified in Section 5.

% --- Section 5: Consolidated Risk Assessment ---
\section{Consolidated Risk Assessment}
This section synthesizes findings from the security control review, technical scan, and pre-existing risk data into a consolidated list of identified risks.

\begin{table}[h!]
\centering
\begin{tabular}{@{}p{0.2\textwidth}p{0.1\textwidth}p{0.55\textwidth}p{0.15\textwidth}@{}}
\toprule
\textbf{Risk Name} & \textbf{Severity} & \textbf{Description} & \textbf{Affected Asset} \\ \midrule
\textbf{Localhost Exposed} & \textcolor{sev_critical}{\textbf{Critical (10.0)}} & A critical service (SSH on port 22) is exposed to the public internet, making it a primary target for automated attacks and unauthorized access attempts. This was validated by the network scan. & \texttt{127.0.0.1} \\
\addlinespace
\textbf{Lack of Multi-Factor Authentication} & \textcolor{sev_high}{\textbf{High}} & The absence of enforced MFA for email, endpoints, and sensitive systems means a single compromised password could lead to a significant data breach or system takeover. & Organization-wide \\
\bottomrule
\end{tabular}
\caption{Summary of Identified Risks.}
\label{tab:risks}
\end{table}

% --- Section 6: Recommendations ---
\section{Recommendations}
The following prioritized recommendations are provided to address the identified risks and improve the overall security posture of \textbf{Localhost}.

\subsection{Priority 1: Remediate Exposed SSH Service (Critical)}
Immediate action is required to mitigate the risk associated with the publicly accessible SSH port.
\begin{itemize}
    \item \textbf{Immediate Action:} Use a firewall to restrict all access to TCP port 22. Only allow connections from specific, trusted IP addresses (IP whitelisting).
    \item \textbf{Strengthen Configuration:} If external access is a business necessity, disable password-based authentication entirely and enforce the use of strong cryptographic keys (e.g., ED25519) for authentication.
    \item \textbf{Monitoring:} Implement robust logging and alerting for all SSH login attempts (both successful and failed) to detect suspicious activity.
\end{itemize}

\subsection{Priority 2: Implement Multi-Factor Authentication (High)}
Deploying MFA is one of the most effective controls to prevent unauthorized access.
\begin{itemize}
    \item \textbf{Email and Collaboration:} Immediately enforce MFA for all user accounts on the primary email and collaboration platform.
    \item \textbf{Administrative Access:} Require MFA for all administrative access, including SSH, server management consoles, and network device administration.
    \item \textbf{Sensitive Systems:} Roll out MFA for all applications and systems that process or store sensitive or critical data.
\end{itemize}

\subsection{Priority 3: Conduct Comprehensive Vulnerability Assessment (Medium)}
The initial scan was limited in scope. A deeper analysis is necessary to uncover further weaknesses.
\begin{itemize}
    \item \textbf{Authenticated Scanning:} Perform an internal, authenticated vulnerability scan on critical assets to identify missing patches, insecure software versions, and misconfigurations that are not visible from the outside.
    \item \textbf{Regular Cadence:} Establish a regular schedule for vulnerability scanning (e.g., quarterly) to ensure new risks are identified and remediated in a timely manner.
\end{itemize}

\end{document}
```