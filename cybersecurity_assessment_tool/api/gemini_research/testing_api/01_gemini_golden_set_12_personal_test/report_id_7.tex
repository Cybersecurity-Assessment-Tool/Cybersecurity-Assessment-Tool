```latex
\documentclass[12pt, a4paper]{article}

% Preamble: Required Packages and Document Setup
\usepackage[margin=1in]{geometry}
\usepackage{pifont} % For checkmarks and crosses (dingbats)
\usepackage{booktabs} % For professional-looking tables
\usepackage{hyperref} % For clickable links
\usepackage{url} % For formatting URLs
\usepackage{seqsplit} % For splitting long strings like IPs
\usepackage{graphicx}
\usepackage{xcolor}
\usepackage{fancyhdr}

% --- Document Metadata and Hyperref Setup ---
\hypersetup{
    colorlinks=true,
    linkcolor=blue,
    filecolor=magenta,      
    urlcolor=cyan,
    pdftitle={Cybersecurity Posture Assessment Report},
    pdfauthor={Cybersecurity Analyst},
    pdfsubject={Security Assessment},
    pdfkeywords={Cybersecurity, Risk, Analysis},
    bookmarks=true
}

% --- Custom Colors and Commands ---
\definecolor{darkblue}{rgb}{0.0, 0.0, 0.55}
\definecolor{darkred}{rgb}{0.55, 0.0, 0.0}
\newcommand{\yes}{\textcolor{darkgreen}{\ding{51}}} % Green checkmark
\newcommand{\no}{\textcolor{darkred}{\ding{55}}}   % Red X

% --- Header and Footer ---
\pagestyle{fancy}
\fancyhf{} % Clear all header and footer fields
\fancyhead[L]{Cybersecurity Posture Assessment}
\fancyhead[R]{For: FutureNet}
\fancyfoot[C]{\thepage}
\renewcommand{\headrulewidth}{0.4pt}
\renewcommand{\footrulewidth}{0.4pt}

% --- Document Start ---
\begin{document}

% --- Title Page ---
\begin{titlepage}
    \centering
    \vspace*{1cm}
    
    \includegraphics[width=0.3\textwidth]{example-image-a} % Placeholder for company logo
    
    \vspace{1.5cm}
    
    {\Huge \bfseries Cybersecurity Posture Assessment Report\par}
    
    \vspace{1.5cm}
    
    {\Large \textbf{Prepared For:}\\ FutureNet\par}
    
    \vspace{2cm}
    
    {\large \today\par}
    
    \vfill
    
    {\large \textit{This report contains sensitive information and should be handled with care. Access is restricted to authorized personnel only.}\par}
    
\end{titlepage}

\tableofcontents
\newpage

% ===================================================================
% SECTION 1: EXECUTIVE SUMMARY
% ===================================================================
\section{Executive Summary}

This report details the findings of a cybersecurity posture assessment conducted for FutureNet. The analysis is based on a combination of technical network scanning, a review of organizational security controls via a questionnaire, and an evaluation of pre-existing risks.

The overall security posture of FutureNet is determined to be at a \textbf{CRITICAL} risk level. This assessment is primarily driven by fundamental and severe gaps in foundational security controls. The complete absence of Multi-Factor Authentication (MFA) across all critical systems, including email and remote access, represents a significant vulnerability. This is compounded by a lack of a formal security awareness training program, leaving the organization highly susceptible to social engineering and phishing attacks.

Technical analysis identified an externally exposed Secure Shell (SSH) service. While a common administrative tool, its exposure to the public internet without the protection of MFA or IP-based access controls creates a direct and high-risk pathway for attackers to compromise internal systems, especially if credentials are stolen.

Immediate and decisive action is required to address these deficiencies. The recommendations outlined in this report prioritize the implementation of MFA, the establishment of a security awareness program, and the securing of all internet-facing services.

% ===================================================================
% SECTION 2: ORGANIZATIONAL INFORMATION
% ===================================================================
\section{Organizational Information}

The following details were provided for the assessment. This information helps to establish the context and scope of the review.

\begin{tabular}{@{}ll}
\toprule
\textbf{Attribute} & \textbf{Value} \\
\midrule
Organization Name & FutureNet \\
Email Domain & \texttt{future.net} \\
Website Domain & \texttt{future.net} \\
External IP Address & \seqsplit{\texttt{2001:db8::1}} \\
\bottomrule
\end{tabular}

% ===================================================================
% SECTION 3: SECURITY CONTROL REVIEW
% ===================================================================
\section{Security Control Review (Questionnaire Analysis)}

A security questionnaire was completed to evaluate the implementation of key administrative and preventative controls. The responses indicate critical gaps in the organization's security framework.

\begin{table}[h!]
\centering
\caption{Security Controls Questionnaire Results}
\begin{tabular}{@{}p{0.75\linewidth}c@{}}
\toprule
\textbf{Control Question} & \textbf{Response} \\
\midrule
Do you require MFA to access email? & \no \\
Do you require MFA to log into computers? & \no \\
Do you require MFA to access sensitive data systems? & \no \\
Does your organization have an employee acceptable use policy? & \no \\
Does your organization do security awareness training for new employees? & \no \\
Does your organization do security awareness training for all employees at least once per year? & \no \\
\bottomrule
\end{tabular}
\end{table}

\subsection*{Analysis of Control Gaps}
The consistent "No" responses across all questions highlight a lack of foundational security practices.
\begin{itemize}
    \item \textbf{Multi-Factor Authentication (MFA):} The absence of MFA for email, computer logins, and sensitive systems is the most severe finding. A single compromised password could lead to a full-scale breach of corporate data and systems.
    \item \textbf{Security Policies \& Training:} The lack of an acceptable use policy and any form of security awareness training creates a high-risk environment. Employees are likely unaware of cyber threats and their role in protecting the organization, making them easy targets for phishing and other social engineering attacks.
\end{itemize}

% ===================================================================
% SECTION 4: TECHNICAL SCAN RESULTS
% ===================================================================
\section{Technical Scan Results}

An external network scan was performed against the provided IP address to identify accessible services.

\begin{itemize}
    \item \textbf{Target IP Address:} \seqsplit{\texttt{2001:db8::1}}
    \item \textbf{Scan Date:} \today
\end{itemize}

\begin{table}[h!]
\centering
\caption{Open Ports Detected}
\begin{tabular}{@{}llll@{}}
\toprule
\textbf{Port} & \textbf{State} & \textbf{Service} & \textbf{Notes} \\
\midrule
22/tcp & open & ssh & Secure Shell (SSH) for remote administration. \\
\bottomrule
\end{tabular}
\end{table}

\subsection*{Analysis of Technical Findings}
The scan identified that port 22 (SSH) is open to the public internet. SSH is a powerful administrative protocol that provides command-line access to a server. Exposing this service externally is a common practice but carries significant risk if not properly secured. In the context of the control gaps identified in Section 3 (specifically, the lack of MFA), this open port becomes a \textbf{critical vulnerability}. An attacker who obtains valid credentials through phishing or other means could gain direct access to the organization's infrastructure without being challenged by a second authentication factor.

% ===================================================================
% SECTION 5: RISK ASSESSMENT SUMMARY
% ===================================================================
\section{Risk Assessment Summary}

The following table synthesizes the findings from the control review and technical scan into a prioritized list of identified risks.

\begin{table}[h!]
\centering
\caption{Identified Risks and Severity}
\begin{tabular}{@{}p{0.1\linewidth}p{0.25\linewidth}p{0.45\linewidth}l@{}}
\toprule
\textbf{Risk ID} & \textbf{Risk Name} & \textbf{Description} & \textbf{Severity} \\
\midrule
RISK-001 & No Multi-Factor Authentication (MFA) & The absence of MFA on all systems allows an attacker with a single stolen password to gain unauthorized access to email, servers, and sensitive data. & \textcolor{darkred}{Critical} \\
\addlinespace
RISK-002 & No Security Awareness Training & Employees are not trained to identify or report phishing and other cyber threats, making the organization highly vulnerable to initial compromise. & \textcolor{darkred}{Critical} \\
\addlinespace
RISK-003 & Exposed SSH Service without MFA & The SSH management port is open to the internet. This, combined with no MFA (RISK-001), creates a direct path for attackers to compromise a key server. & \textcolor{darkred}{High} \\
\addlinespace
RISK-004 & No Acceptable Use Policy (AUP) & The lack of a formal AUP creates ambiguity regarding security responsibilities and acceptable employee behavior, increasing the risk of insider threat and policy violations. & \textcolor{orange}{Medium} \\
\bottomrule
\end{tabular}
\end{table}

% ===================================================================
% SECTION 6: RECOMMENDATIONS
% ===================================================================
\section{Recommendations}

The following actions are recommended to mitigate the identified risks. These are prioritized based on severity and potential impact.

\subsection*{RISK-001: Implement Multi-Factor Authentication (Critical)}
\begin{itemize}
    \item \textbf{Immediate Action:} Enable MFA on the email platform (e.g., Office 365, Google Workspace) for all users immediately. This is the single most effective control to prevent business email compromise.
    \item \textbf{Short-Term Action:} Enforce MFA on all remote access systems, including VPNs and the exposed SSH service. For SSH, prioritize public key authentication and disable passwords.
    \item \textbf{Long-Term Action:} Develop a roadmap to enforce MFA for all applications, especially those containing sensitive or financial data.
\end{itemize}

\subsection*{RISK-002: Establish a Security Awareness Program (Critical)}
\begin{itemize}
    \item \textbf{Immediate Action:} Enroll all employees in a foundational security awareness training course that covers phishing, password hygiene, and social engineering.
    \item \textbf{Short-Term Action:} Conduct a baseline phishing simulation to gauge employee susceptibility. Use the results to tailor future training.
    \item \textbf{Long-Term Action:} Implement a continuous training program where all employees receive training at least annually, and new hires are trained as part of their onboarding process.
\end{itemize}

\subsection*{RISK-003: Secure the Exposed SSH Service (High)}
\begin{itemize}
    \item \textbf{Immediate Action:} If SSH access is not required from the public internet, block port 22 at the network firewall.
    \item \textbf{Short-Term Action:} If external access is required, restrict access to a whitelist of trusted IP addresses. Concurrently, enforce the use of SSH keys for authentication and completely disable password-based authentication in the SSH server configuration.
\end{itemize}

\subsection*{RISK-004: Develop an Acceptable Use Policy (Medium)}
\begin{itemize}
    \item \textbf{Short-Term Action:} Draft a formal Acceptable Use Policy (AUP) that clearly defines the rules for using company technology, data, and internet access.
    \item \textbf{Long-Term Action:} Require all current and new employees to read and formally acknowledge the AUP. Review and update the policy annually.
\end{itemize}

\end{document}
```