```latex
\documentclass[12pt]{article}

% Preamble: Required Packages
\usepackage[margin=1in]{geometry}
\usepackage{pifont} % For checkmarks and crosses (\ding)
\usepackage{booktabs} % For professional tables
\usepackage{hyperref} % For hyperlinks
\usepackage{url} % For URL formatting
\usepackage{seqsplit} % To split long strings in tt font
\usepackage{graphicx}
\usepackage[T1]{fontenc}
\usepackage{xcolor}
\usepackage{fancyhdr}

% --- Document Metadata ---
\title{Cybersecurity Posture Assessment Report}
\author{Cybersecurity Analysis Division}
\date{\today}

% --- Custom Commands & Settings ---
\hypersetup{
    colorlinks=true,
    linkcolor=blue,
    filecolor=magenta,      
    urlcolor=cyan,
    pdftitle={Cybersecurity Posture Assessment Report},
    pdfpagemode=FullScreen,
}

% Define colors for severity
\definecolor{critseverity}{HTML}{6B0504}
\definecolor{highseverity}{HTML}{A31621}
\definecolor{medseverity}{HTML}{D76A03}
\definecolor{lowseverity}{HTML}{F4D35E}
\definecolor{infoseverity}{HTML}{4381C1}

% Checkmark and Cross definitions
\newcommand{\yes}{\ding{51}}
\newcommand{\no}{\ding{55}}

% Header and Footer
\pagestyle{fancy}
\fancyhf{}
\fancyhead[L]{Cybersecurity Posture Assessment}
\fancyhead[R]{\textbf{FixedIt LLC}}
\fancyfoot[C]{\thepage}

\begin{document}

\maketitle
\thispagestyle{empty}
\newpage

\tableofcontents
\newpage

% ==============================================================================
% 1. Executive Overview
% ==============================================================================
\section{Executive Overview}

This report provides a comprehensive cybersecurity assessment for \textbf{FixedIt LLC}, synthesizing data from a technical network scan, a security controls questionnaire, and a review of pre-existing risks. The analysis aims to provide a clear and actionable overview of the organization's current security posture.

The assessment reveals a positive and proactive security posture. The organization attests to having strong administrative controls, including universal Multi-Factor Authentication (MFA) and a robust security awareness training program. 

A key finding of this assessment is the successful remediation of a previously identified risk. The prior risk, "Unencrypted Web Server," was associated with an open Port 80. The current technical scan confirms that \textbf{Port 80 is now closed} on the target system (\texttt{192.168.0.5}), effectively mitigating this vulnerability. No other open ports or active services were discovered during the scan.

While the current state is strong, security is an ongoing process. Recommendations in this report focus on maintaining this robust posture through continuous monitoring and formal documentation of remediated risks.

% ==============================================================================
% 2. Organizational Information
% ==============================================================================
\section{Organizational Information}

The following details were provided for the assessment. This information forms the basis of the analysis and defines the scope of the review.

\begin{itemize}
    \item \textbf{Organization Name:} FixedIt LLC
    \item \textbf{Primary Email Domain:} \texttt{fixedit.com}
    \item \textbf{External IP Assessed:} \texttt{192.168.0.5}
\end{itemize}

% ==============================================================================
% 3. Security Control Review
% ==============================================================================
\section{Security Control Review}

The following table summarizes the organization's self-reported status of key administrative and technical security controls. A response of "Yes" (\yes) indicates the control is in place, while "No" (\no) would indicate a potential gap.

\begin{table}[h!]
\centering
\caption{Security Controls Questionnaire Results}
\begin{tabular}{p{0.8\linewidth} c}
\toprule
\textbf{Control Question} & \textbf{Response} \\
\midrule
Do you require MFA to access email? & \yes \\
Do you require MFA to log into computers? & \yes \\
Do you require MFA to access sensitive data systems? & \yes \\
Does your organization have an employee acceptable use policy? & \yes \\
Does your organization do security awareness training for new employees? & \yes \\
Does your organization do security awareness training for all employees at least once per year? & \yes \\
\bottomrule
\end{tabular}
\end{table}

\subsection*{Analysis}
The responses indicate a strong foundation of security policies and controls. The consistent implementation of MFA across critical systems significantly reduces the risk of unauthorized access via compromised credentials. The established security awareness program is crucial for defending against social engineering and phishing attacks. No gaps were identified from this questionnaire.

% ==============================================================================
% 4. Technical Scan Results
% ==============================================================================
\section{Technical Scan Results}

A network scan was performed on the designated external IP address to identify open ports and exposed services.

\begin{itemize}
    \item \textbf{Target IP:} \texttt{192.168.0.5}
    \item \textbf{Scan Date:} Not specified in scan data.
\end{itemize}

\begin{table}[h!]
\centering
\caption{Port Scan Findings for \texttt{192.168.0.5}}
\begin{tabular}{llll}
\toprule
\textbf{Port} & \textbf{State} & \textbf{Service} & \textbf{Version} \\
\midrule
80/tcp & closed & http & N/A \\
\bottomrule
\end{tabular}
\end{table}

\subsection*{Analysis}
The technical scan found no open ports on the target system. The fact that port 80 (HTTP) is explicitly reported as closed is a significant finding, as it directly contradicts a previously known risk. This indicates that proactive steps have been taken to secure the network perimeter. The absence of open ports drastically reduces the external attack surface.

% ==============================================================================
% 5. Correlated Risk Assessment
% ==============================================================================
\section{Correlated Risk Assessment}

This section synthesizes findings from all data sources to provide a holistic view of the organization's risk landscape. The status of a pre-existing vulnerability was evaluated against the new technical scan data.

\begin{table}[h!]
\centering
\caption{Summary of Identified Risks}
\begin{tabular}{p{0.25\linewidth} p{0.45\linewidth} p{0.2\linewidth}}
\toprule
\textbf{Risk Name} & \textbf{Description & Correlation} & \textbf{Status} \\
\midrule
\textcolor{medseverity}{\textbf{Unencrypted Web Server}} & A previously identified risk where Port 80 was open, potentially exposing unencrypted HTTP traffic. \textbf{Correlation:} The new technical scan confirms Port 80 is now \textbf{closed}, mitigating this threat. & \textcolor{green!50!black}{\textbf{Remediated}} \\
\addlinespace
\textcolor{infoseverity}{\textbf{Administrative Controls}} & The organization reports strong administrative controls, including mandatory MFA and security awareness training. \textbf{Correlation:} This is a significant strength that reduces overall organizational risk. & \textcolor{blue!60!black}{\textbf{Strength Identified}} \\
\bottomrule
\end{tabular}
\end{table}

% ==============================================================================
% 6. Recommendations
% ==============================================================================
\section{Recommendations}

Based on the analysis, the organization's security posture is strong. The following recommendations are focused on maintaining and formalizing this high standard of security.

\begin{enumerate}
    \item \textbf{Formalize Risk Management Updates} \\
    \textit{Priority: High} \\
    The "Unencrypted Web Server" risk has been successfully mitigated. It is crucial to formally update the internal risk register to reflect this change. This ensures that documentation accurately represents the current security posture and that resources are not wasted tracking a resolved issue.

    \item \textbf{Implement Continuous Monitoring} \\
    \textit{Priority: Medium} \\
    To maintain the secure state of the network perimeter, establish a regular, automated scanning schedule (e.g., weekly or monthly). This practice will provide early warnings if new services are exposed or if configurations change, allowing for rapid response before they can be exploited.

    \item \textbf{Maintain and Review Security Controls} \\
    \textit{Priority: Medium} \\
    The strong administrative controls (MFA, training) are a cornerstone of the organization's defense. Continue to enforce these policies rigorously. Conduct an annual review of these controls to ensure they remain effective and aligned with evolving threats and business needs.
\end{enumerate}

\end{document}
```