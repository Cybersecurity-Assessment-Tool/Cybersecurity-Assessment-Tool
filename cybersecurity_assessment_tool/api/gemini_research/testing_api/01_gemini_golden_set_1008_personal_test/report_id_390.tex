```latex
\documentclass[12pt]{article}

% Preamble: Required Packages
\usepackage[margin=1in]{geometry}
\usepackage{pifont} % For checkmarks and crosses
\usepackage{booktabs} % For professional tables
\usepackage{hyperref} % For clickable links
\usepackage{url} % For formatting URLs
\usepackage{seqsplit} % For splitting long strings in texttt
\usepackage[utf8]{inputenc}

% Document Metadata
\title{Cybersecurity Posture Assessment Report}
\author{Cybersecurity Analyst}
\date{\today}

% Hyperref Setup
\hypersetup{
    colorlinks=true,
    linkcolor=blue,
    filecolor=magenta,      
    urlcolor=cyan,
    pdftitle={Cybersecurity Posture Assessment Report},
    pdfpagemode=FullScreen,
}

\begin{document}

\maketitle
\thispagestyle{empty}
\newpage
\tableofcontents
\newpage

% --- 1. Executive Summary ---
\section{Executive Summary}
This report provides a comprehensive cybersecurity assessment for \textbf{Terraform Global}, based on a technical network scan, a review of organizational security controls, and an analysis of pre-existing risk data. The assessment was conducted on \today.

The analysis reveals critical deficiencies in identity and access management controls, specifically a complete lack of Multi-Factor Authentication (MFA) across all key systems. Additionally, a foundational policy gap was identified due to the absence of an employee Acceptable Use Policy (AUP). These administrative gaps present a significantly higher risk to the organization than the technical findings.

On the technical side, the network scan of the target host \texttt{192.168.0.5} did not reveal any open ports or active vulnerabilities. This finding contradicts a pre-existing risk concerning an unencrypted web server on Port 80. This suggests the pre-existing risk may be outdated or has been remediated.

Immediate remediation should focus on implementing MFA and establishing a formal AUP to mitigate the most severe risks of account compromise and insider threat.

% --- 2. Organizational Information ---
\section{Organizational Information}
The following details were provided for the assessment:
\begin{itemize}
    \item \textbf{Organization Name:} Terraform Global
    \item \textbf{Email Domain:} \texttt{TerraformGlobal.org}
    \item \textbf{Website Domain:} \url{www.TerraformGlobal.org}
    \item \textbf{External IP Address:} \texttt{48.42.194.224}
\end{itemize}

% --- 3. Security Control Review ---
\section{Security Control Review}
A review of administrative and policy-based security controls was conducted via a questionnaire. The results highlight significant gaps in fundamental security practices. A summary of the findings is presented in Table \ref{tab:controls}.

\begin{table}[h!]
\centering
\caption{Organizational Security Control Assessment}
\label{tab:controls}
\begin{tabular}{@{}lcc@{}}
\toprule
\textbf{Control Question} & \textbf{Response} & \textbf{Assessment} \\
\midrule
Do you require MFA to access email? & \ding{55} & Critical Gap \\
Do you require MFA to log into computers? & \ding{55} & Critical Gap \\
Do you require MFA to access sensitive data systems? & \ding{55} & Critical Gap \\
Does your organization have an employee acceptable use policy? & \ding{55} & High Risk \\
Does your organization do security awareness training for new employees? & \ding{51} & Best Practice Met \\
Does your organization do security awareness training for all employees annually? & \ding{51} & Best Practice Met \\
\bottomrule
\end{tabular}
\end{table}

\paragraph{Analysis:} The lack of MFA for email, computer logins, and sensitive data access is a critical vulnerability. It significantly increases the likelihood of a successful account takeover attack via phishing or credential stuffing. The absence of an Acceptable Use Policy creates ambiguity for employees and weakens the organization's ability to enforce secure behavior. While the security awareness training program is a positive control, its effectiveness is diminished without these foundational security measures in place.

% --- 4. Technical Scan Results ---
\section{Technical Scan Results}
A network scan was performed on the specified target to identify potential vulnerabilities.

\subsection{Scan Details}
\begin{itemize}
    \item \textbf{Scanner Used:} Nmap
    \item \textbf{Target IP Address:} \texttt{192.168.0.5}
    \item \textbf{Target Status:} Host was responsive (up).
\end{itemize}

\subsection{Port Scan Findings}
The scan results for the target host are detailed in Table \ref{tab:ports}. No open ports or vulnerable services were identified during this assessment.

\begin{table}[h!]
\centering
\caption{Port Scan Results for \texttt{192.168.0.5}}
\label{tab:ports}
\begin{tabular}{@{}llll@{}}
\toprule
\textbf{Port} & \textbf{State} & \textbf{Service} & \textbf{Product / Version} \\
\midrule
80/tcp & closed & http & N/A \\
\bottomrule
\end{tabular}
\end{table}

\paragraph{Analysis:} The scan indicates that Port 80 (HTTP) is closed on the target system. This is a positive security finding, as it prevents unencrypted web traffic. This result directly contradicts the pre-existing risk titled "Unencrypted Web Server," suggesting that the risk has either been remediated or was based on outdated information.

% --- 5. Overall Risk Assessment ---
\section{Risk Assessment}
This section synthesizes findings from the security control review, technical scan, and pre-existing risk data to provide a holistic view of the organization's current risk posture. The identified risks are summarized in Table \ref{tab:risks}.

\begin{table}[h!]
\centering
\caption{Summary of Identified Risks}
\label{tab:risks}
\begin{tabular}{@{}p{0.3\linewidth}p{0.5\linewidth}l@{}}
\toprule
\textbf{Risk Name} & \textbf{Description} & \textbf{Severity} \\
\midrule
\textbf{Lack of Multi-Factor Authentication (MFA)} & The absence of MFA for email, endpoints, and sensitive data systems exposes the organization to a high risk of unauthorized access through account compromise. & \textbf{Critical} \\
\addlinespace
\textbf{Missing Acceptable Use Policy (AUP)} & Without a formal AUP, there are no defined rules for employee use of corporate assets, increasing the risk of data misuse, policy violations, and insider threats. & \textbf{High} \\
\addlinespace
\textbf{Unencrypted Web Server (Potentially Remediated)} & A pre-existing risk indicated an open Port 80. The current scan shows this port is closed, suggesting remediation. This requires verification across the environment. & \textbf{Low} \\
\bottomrule
\end{tabular}
\end{table}

% --- 6. Recommendations ---
\section{Recommendations}
The following actionable recommendations are provided to address the identified risks. They are prioritized based on severity and potential impact.

\subsection{Priority 1: Implement Multi-Factor Authentication (Critical)}
\begin{itemize}
    \item \textbf{Action:} Deploy a robust MFA solution across the organization.
    \item \textbf{Details:} Prioritize the rollout in the following order:
    \begin{enumerate}
        \item All administrative and privileged accounts.
        \item All user accounts for email access (e.g., Office 365, Google Workspace).
        \item All remote access solutions (e.g., VPN).
        \item All systems containing sensitive data.
    \end{enumerate}
    \item \textbf{Justification:} This is the single most effective control to prevent unauthorized account access and will drastically reduce the organization's attack surface.
\end{itemize}

\subsection{Priority 2: Develop and Enforce an Acceptable Use Policy (High)}
\begin{itemize}
    \item \textbf{Action:} Draft, approve, and implement a formal Acceptable Use Policy (AUP).
    \item \textbf{Details:} The policy should clearly define the rules for using company networks, devices, and data. All employees must be required to read and formally acknowledge the policy. This AUP should be integrated into the existing security awareness training program.
    \item \textbf{Justification:} An AUP establishes a baseline for secure behavior, reduces legal liability, and empowers the organization to enforce its security standards.
\end{itemize}

\subsection{Priority 3: Verify Risk Remediation (Informational)}
\begin{itemize}
    \item \textbf{Action:} Conduct a formal review of the "Unencrypted Web Server" risk.
    \item \textbf{Details:} Confirm that Port 80 is closed on all relevant production web servers and that all web traffic is encrypted using TLS/SSL (HTTPS). Update the central risk register to formally close this item once verified.
    \item \textbf{Justification:} Maintaining an accurate risk register is crucial for an effective security program. This ensures that resources are focused on current, relevant threats.
\end{itemize}

\end{document}
```