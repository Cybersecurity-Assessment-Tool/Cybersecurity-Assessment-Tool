```latex
\documentclass[12pt]{article}

% Preamble: Required Packages
\usepackage[margin=1in]{geometry}
\usepackage{pifont} % For checkmarks and crosses
\usepackage{booktabs} % For professional tables
\usepackage{hyperref} % For clickable links
\usepackage{url} % For formatting URLs
\usepackage{seqsplit} % To split long strings in tt font
\usepackage{xcolor} % For colors
\usepackage{graphicx} % For potential images/logos
\usepackage{fancyhdr} % For headers and footers

% --- Document Setup ---
\hypersetup{
    colorlinks=true,
    linkcolor=blue,
    filecolor=magenta,      
    urlcolor=cyan,
}

% Define severity colors
\definecolor{critical}{HTML}{990000}
\definecolor{high}{HTML}{D14302}
\definecolor{medium}{HTML}{E5A50A}
\definecolor{low}{HTML}{3E8E41}

% Header and Footer
\pagestyle{fancy}
\fancyhf{}
\lhead{Cybersecurity Assessment Report}
\rhead{\textbf{Urban Jungle Planning}}
\cfoot{\thepage}

% --- Document Start ---
\begin{document}

% --- Title Page ---
\begin{titlepage}
    \centering
    \vspace*{1cm}
    \includegraphics[width=0.4\textwidth]{example-image-a} % Placeholder for company logo
    
    \vspace{1.5cm}
    
    \Huge
    \textbf{Cybersecurity Posture Assessment Report}
    
    \vspace{1.5cm}
    
    \Large
    Prepared for: \textbf{Urban Jungle Planning}
    
    \vspace{2cm}
    
    \normalsize
    Report Date: \today
    
    \vfill
    
    \small
    \textit{This report contains sensitive information and should be handled with care. Distribution is restricted to authorized personnel only.}
    
\end{titlepage}

\tableofcontents
\newpage

% --- Section 1: Executive Summary ---
\section{Executive Summary}

This report provides a comprehensive analysis of the cybersecurity posture for \textbf{Urban Jungle Planning}. The assessment is based on a correlation of network scan data, a security controls questionnaire, and a review of pre-existing risks.

The overall security posture reveals several critical and high-risk vulnerabilities that require immediate attention. Key findings include:

\begin{itemize}
    \item \textbf{Critical Database Exposure:} A MySQL database server is directly exposed to the network on port 3306. This service is running an outdated and unsupported version of MySQL (5.7.33), which no longer receives security updates, posing a significant risk of data breach.
    
    \item \textbf{Critical Gaps in Administrative Controls:} The organization lacks fundamental security policies and practices. Specifically, there is no requirement for Multi-Factor Authentication (MFA) on employee computers, no formal Acceptable Use Policy (AUP), and no annual security awareness training for all employees. These gaps significantly increase the risk of credential compromise, insider threats, and susceptibility to social engineering attacks.
\end{itemize}

This report details these findings and provides prioritized, actionable recommendations to mitigate the identified risks and strengthen the organization's overall security framework. We urge management to review the recommendations in Section \ref{sec:recommendations} and implement them promptly.

% --- Section 2: Organizational Information ---
\section{Organizational Information}

The following information was provided for the assessment.

\begin{table}[h!]
\centering
\begin{tabular}{@{}ll@{}}
\toprule
\textbf{Attribute} & \textbf{Value} \\ \midrule
Organization Name    & \textbf{Urban Jungle Planning} \\
Email Domain         & \texttt{UrbanJunglePlanning.org} \\
Website Domain       & \url{www.UrbanJunglePlanning.org} \\
External IP Address  & \texttt{208.111.144.143} \\ \bottomrule
\end{tabular}
\caption{Client Organizational Details}
\label{tab:org_info}
\end{table}

% --- Section 3: Security Control Review ---
\section{Security Control Review}

A review of administrative and technical security controls was conducted via a questionnaire. The results highlight significant gaps in foundational security practices. A "Yes" (\ding{51}) indicates a control is in place, while a "No" (\ding{55}) indicates a control gap.

\begin{table}[h!]
\centering
\begin{tabular}{@{}p{0.7\linewidth}c@{}}
\toprule
\textbf{Security Control Question} & \textbf{Status} \\ \midrule
Do you require MFA to access email? & \ding{51} \\
Do you require MFA to log into computers? & \textcolor{red}{\ding{55}} \\
Do you require MFA to access sensitive data systems? & \ding{51} \\
Does your organization have an employee acceptable use policy? & \textcolor{red}{\ding{55}} \\
Does your organization do security awareness training for new employees? & \ding{51} \\
Does your organization do security awareness training for all employees at least once per year? & \textcolor{red}{\ding{55}} \\ \bottomrule
\end{tabular}
\caption{Security Controls Questionnaire Results}
\label{tab:controls}
\end{table}

\subsection*{Analysis of Control Gaps}
The identified gaps represent high-risk areas:
\begin{itemize}
    \item \textbf{No MFA for Computer Logins:} The absence of MFA on endpoints is a critical weakness. If an employee's password is stolen, an attacker can gain full access to their computer, potentially leading to lateral movement across the network and access to sensitive data.
    \item \textbf{No Acceptable Use Policy (AUP):} Without an AUP, there are no clear guidelines for employees regarding the safe and appropriate use of company assets. This increases the risk of unintentional data exposure, malware infections, and misuse of resources.
    \item \textbf{No Annual Security Training:} Cyber threats evolve constantly. Training only new hires is insufficient. Without annual refresher training, employees are more likely to fall victim to modern phishing, ransomware, and other social engineering attacks.
\end{itemize}

% --- Section 4: Technical Scan Results ---
\section{Technical Scan Results}

An external network scan was performed to identify exposed services.

\subsection*{Host: \texttt{172.16.50.20}}
The scan identified the following open port on the target system.

\begin{table}[h!]
\centering
\begin{tabular}{@{}lllll@{}}
\toprule
\textbf{Port} & \textbf{State} & \textbf{Service} & \textbf{Product} & \textbf{Version} \\ \midrule
3306/tcp      & open           & mysql            & MySQL            & 5.7.33           \\ \bottomrule
\end{tabular}
\caption{Open Ports and Services Detected}
\label{tab:scan_results}
\end{table}

\subsection*{Analysis of Technical Findings}
\begin{itemize}
    \item \textbf{Exposed Database Service:} Port 3306 is the default port for MySQL. Exposing a database server directly to the network is extremely dangerous. It allows attackers to perform brute-force attacks against database credentials, exploit vulnerabilities in the database software, and potentially exfiltrate or destroy sensitive data.
    \item \textbf{Unsupported Software Version:} The identified version, \textbf{MySQL 5.7.33}, reached its official End-of-Life (EOL) in October 2023. This means it no longer receives security patches from the vendor. Any new vulnerabilities discovered in this version will remain unpatched, leaving the server permanently vulnerable to exploitation.
\end{itemize}

% --- Section 5: Risk Assessment Summary ---
\section{Risk Assessment Summary}

The following table synthesizes findings from the security control review, technical scan, and pre-existing risk data into a prioritized list.

\begin{table}[h!]
\centering
\resizebox{\textwidth}{!}{%
\begin{tabular}{@{}lp{0.4\linewidth}p{0.3\linewidth}@{}}
\toprule
\textbf{Risk / Vulnerability} & \textbf{Overview} & \textbf{Affected Systems} \\ \midrule
\textbf{\textcolor{critical}{Unsupported Database Software}} & The MySQL server is running version 5.7.33, which is End-of-Life and no longer receives security updates. & Server at \texttt{172.16.50.20} \\
\addlinespace
\textbf{\textcolor{critical}{Exposed Database Service}} & The MySQL database port (3306) is open to the network, inviting brute-force attacks and direct exploitation attempts. & Server at \texttt{172.16.50.20} \\
\addlinespace
\textbf{\textcolor{high}{Lack of Endpoint MFA}} & The absence of MFA on computer logins allows an attacker with a stolen password to gain full access to an employee's workstation and network resources. & All employee workstations \\
\addlinespace
\textbf{\textcolor{high}{Missing Acceptable Use Policy}} & The lack of a formal policy creates ambiguity for employees regarding safe technology use, increasing the risk of insider threats and accidental breaches. & Organization-wide policy \\
\addlinespace
\textbf{\textcolor{high}{Inadequate Security Training}} & Without mandatory annual training, employees' ability to recognize and respond to modern cyber threats diminishes over time. & All employees \\ \bottomrule
\end{tabular}
}
\caption{Consolidated Risk Register}
\label{tab:risk_register}
\end{table}

% --- Section 6: Recommendations ---
\section{Recommendations}
\label{sec:recommendations}

The following actions are recommended to mitigate the identified risks. They are prioritized based on severity and potential impact.

\subsection*{Immediate Priority (Critical Risks)}
\begin{enumerate}
    \item \textbf{Restrict Access to Database Port:} Immediately implement firewall rules to block all public access to TCP port 3306 on server \texttt{172.16.50.20}. Access should be restricted to only trusted internal IP addresses or completely firewalled from the outside, requiring a VPN for administrative access.
    
    \item \textbf{Plan and Execute Database Upgrade:} Initiate a project to upgrade the MySQL 5.7.33 instance to a currently supported version (e.g., MySQL 8.x). This is essential to ensure the system receives critical security patches.
\end{enumerate}

\subsection*{High Priority (High Risks)}
\begin{enumerate}
    \setcounter{enumi}{2} % Continue numbering
    \item \textbf{Deploy Multi-Factor Authentication (MFA) for Endpoints:} Enforce MFA for all employee computer logins. This can be achieved through solutions like Windows Hello for Business, Duo, or other third-party identity providers.
    
    \item \textbf{Develop and Implement an Acceptable Use Policy (AUP):} Create a formal AUP that clearly defines the rules and responsibilities for employees when using company technology and data. This policy should be signed by all employees.
    
    \item \textbf{Establish a Mandatory Annual Security Training Program:} Implement a security awareness training program that is mandatory for all employees on an annual basis. The training should cover current threats such as phishing, ransomware, and secure data handling.
\end{enumerate}

% --- Document End ---
\end{document}
```