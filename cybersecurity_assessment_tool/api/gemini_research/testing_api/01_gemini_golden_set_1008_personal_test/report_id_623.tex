```latex
\documentclass[12pt]{article}

% Preamble: Required Packages
\usepackage[margin=1in]{geometry}
\usepackage{pifont} % For checkmarks and crosses
\usepackage{booktabs} % For professional tables
\usepackage{hyperref} % For clickable links
\usepackage{url} % For URL formatting
\usepackage{seqsplit} % For splitting long strings without spaces
\usepackage{graphicx}
\usepackage{xcolor}

% Hyperref Setup
\hypersetup{
    colorlinks=true,
    linkcolor=blue,
    filecolor=magenta,      
    urlcolor=cyan,
    pdftitle={Cybersecurity Assessment Report},
    pdfpagemode=FullScreen,
}

% Document Title Block
\title{Cybersecurity Assessment Report \\ \large For: Aeon Pharmaceuticals}
\author{Cybersecurity Analysis Division}
\date{\today}

\begin{document}

\maketitle
\tableofcontents
\newpage

% --- 1. Executive Summary ---
\section{Executive Summary}
This report provides a comprehensive cybersecurity assessment for Aeon Pharmaceuticals, synthesizing data from network scans, organizational questionnaires, and pre-existing risk registers. The analysis reveals critical deficiencies in fundamental security controls, alongside technically identified vulnerabilities.

The most significant findings include a systemic lack of Multi-Factor Authentication (MFA) across all key assets, including email, endpoints, and sensitive data systems. This represents a critical vulnerability that drastically increases the risk of account compromise and subsequent data breach. Furthermore, the absence of foundational policies and training for new employees weakens the organization's human firewall.

Technical scans confirmed a pre-identified critical risk, "Localhost Exposed," associated with an open management port on a scanned system. This finding, combined with the policy and identity management gaps, indicates a fragile security posture that requires immediate and decisive remediation. This report outlines prioritized, actionable recommendations to address these findings and strengthen the overall security posture of the organization.

% --- 2. Organizational Information ---
\section{Organizational Information}
The following details were provided for the assessment. This information is used to establish the context and scope of the review.

\begin{tabular}{@{}ll}
\toprule
\textbf{Attribute} & \textbf{Value} \\
\midrule
Organization Name & \textbf{Aeon Pharmaceuticals} \\
Email Domain      & \seqsplit{\texttt{AeonPharmaceuticals.com}} \\
Website Domain    & \seqsplit{\url{www.AeonPharmaceuticals.com}} \\
External IP Address & \seqsplit{\texttt{7.75.204.125}} \\
\bottomrule
\end{tabular}

% --- 3. Security Control Review ---
\section{Security Control Review}
A review of the organization's security controls was conducted via a questionnaire. The responses indicate significant gaps in essential security practices, particularly concerning identity and access management and employee security awareness.

\begin{table}[h!]
\centering
\caption{Security Controls Questionnaire Results}
\begin{tabular}{@{}p{0.8\linewidth}c@{}}
\toprule
\textbf{Control Question} & \textbf{Response} \\
\midrule
Do you require MFA to access email? & \ding{55} \\
Do you require MFA to log into computers? & \ding{55} \\
Do you require MFA to access sensitive data systems? & \ding{55} \\
Does your organization have an employee acceptable use policy? & \ding{55} \\
Does your organization do security awareness training for new employees? & \ding{55} \\
Does your organization do security awareness training for all employees at least once per year? & \ding{51} \\
\bottomrule
\end{tabular}
\end{table}

\subsection*{Analysis of Control Gaps}
The five "No" responses (\ding{55}) highlight critical deficiencies:
\begin{itemize}
    \item \textbf{Lack of MFA:} The absence of MFA for email, computer logins, and sensitive systems is a severe weakness. Stolen credentials alone are sufficient for an attacker to gain widespread access.
    \item \textbf{Policy and Training Gaps:} The lack of an acceptable use policy and security training for new hires creates an environment where employees are unaware of their security responsibilities, making them more susceptible to social engineering attacks.
\end{itemize}

% --- 4. Technical Scan Results ---
\section{Technical Scan Results}
A network scan was performed to identify open ports and services on the target system. The results confirm the presence of an open management service.

\begin{itemize}
    \item \textbf{Target IP Address:} \texttt{127.0.0.1}
    \item \textbf{Scan Status:} Host is UP.
\end{itemize}

\begin{table}[h!]
\centering
\caption{Open Ports Detected on \texttt{127.0.0.1}}
\begin{tabular}{@{}llll@{}}
\toprule
\textbf{Port} & \textbf{State} & \textbf{Inferred Service} & \textbf{Notes} \\
\midrule
22/tcp & open & SSH (Secure Shell) & Management protocol. Access should be strictly controlled. \\
\bottomrule
\end{tabular}
\end{table}
\subsection*{Analysis of Technical Findings}
The scan identified that port 22 (SSH) is open on the loopback interface (\texttt{127.0.0.1}). While not directly exposed to the internet, an open port on localhost can be exploited by malicious software already on the machine for privilege escalation or lateral movement. This finding directly correlates with and validates the existing risk documented in Input 3, "Localhost Exposed."

% --- 5. Consolidated Risk Assessment ---
\section{Consolidated Risk Assessment}
The following table synthesizes findings from the security control review, technical scans, and the existing risk register into a consolidated view of the primary risks facing the organization.

\begin{table}[h!]
\centering
\caption{Summary of Identified Risks}
\begin{tabular}{@{}p{0.25\linewidth}p{0.5\linewidth}l@{}}
\toprule
\textbf{Risk Title} & \textbf{Description} & \textbf{Severity} \\
\midrule
\textbf{Systemic Lack of MFA} & The absence of Multi-Factor Authentication for email, endpoints, and sensitive data systems allows for account takeover using only compromised credentials. & \textbf{Critical} \\
\addlinespace
\textbf{Localhost Exposed} & Port 22 (SSH) is open on the loopback interface of a key system. This aligns with a pre-existing registered risk with a CVSS score of 10.0. & \textbf{Critical} \\
\addlinespace
\textbf{Inadequate Security Policies \& Onboarding} & The lack of an Acceptable Use Policy and security training for new hires significantly increases the risk of insider threat and successful social engineering attacks. & \textbf{High} \\
\bottomrule
\end{tabular}
\end{table}

% --- 6. Recommendations ---
\section{Recommendations}
Based on the consolidated risk assessment, the following prioritized actions are recommended to mitigate the identified vulnerabilities and improve the overall security posture of Aeon Pharmaceuticals.

\subsection*{Priority 1: Immediate Actions (Critical Risks)}
\begin{enumerate}
    \item \textbf{Deploy Multi-Factor Authentication (MFA):} Immediately begin a phased rollout of MFA across the organization. Prioritize the following systems:
    \begin{itemize}
        \item Email (e.g., Office 365, Google Workspace)
        \item VPN and other remote access solutions
        \item All systems hosting sensitive or regulated data
        \item Privileged user and administrator accounts
    \end{itemize}
    \item \textbf{Remediate "Localhost Exposed" Vulnerability:} Investigate the system at \texttt{127.0.0.1} to determine the purpose of the open SSH port. If it is not required for a specific application function, the service should be disabled. If required, firewall rules must be implemented to ensure it is only accessible by the intended application or service.
\end{enumerate}

\subsection*{Priority 2: Foundational Improvements (High Risks)}
\begin{enumerate}
    \setcounter{enumi}{2} % Continue numbering
    \item \textbf{Develop and Implement an Acceptable Use Policy (AUP):} Create a formal AUP that clearly defines the rules and responsibilities for all employees when using company technology and data. This policy should be reviewed and signed by all staff.
    \item \textbf{Integrate Security into Employee Onboarding:} Develop a mandatory security awareness training module for all new hires. This training should cover the AUP, phishing identification, password hygiene, and incident reporting procedures.
\end{enumerate}

\subsection*{Priority 3: Proactive Measures}
\begin{enumerate}
    \setcounter{enumi}{4} % Continue numbering
    \item \textbf{Conduct Comprehensive Vulnerability Scanning:} The initial scan was limited. A comprehensive, credentialed vulnerability scan should be performed across all critical assets to identify missing patches, insecure configurations, and outdated software versions that were not visible in the initial network scan.
\end{enumerate}

\end{document}
```