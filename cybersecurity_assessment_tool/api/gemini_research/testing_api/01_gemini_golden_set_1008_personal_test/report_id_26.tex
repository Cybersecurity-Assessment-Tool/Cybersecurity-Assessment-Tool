```latex
\documentclass[12pt]{article}

% --- PACKAGES ---
\usepackage[margin=1in]{geometry}
\usepackage{pifont} % For checkmarks and crosses
\usepackage{booktabs} % For professional tables
\usepackage[hidelinks]{hyperref} % For clickable links
\usepackage{url} % For URL formatting
\usepackage{seqsplit} % For splitting long strings in texttt
\usepackage[table]{xcolor} % For coloring table cells
\usepackage{graphicx}

% --- DOCUMENT INFORMATION ---
\title{Cybersecurity Assessment Report}
\author{Cybersecurity Analysis Division}
\date{\today}

% --- CUSTOM COMMANDS & SETTINGS ---
\newcommand{\yes}{\ding{51}}
\newcommand{\no}{\ding{55}}

% Define severity colors
\definecolor{sevCritical}{HTML}{990000}
\definecolor{sevHigh}{HTML}{D16300}
\definecolor{sevMedium}{HTML}{D1A600}
\definecolor{sevLow}{HTML}{339900}

\hypersetup{
    pdftitle={Cybersecurity Assessment Report for Falcon Heavy},
    pdfauthor={Cybersecurity Analysis Division},
    pdfsubject={Security Assessment},
    pdfkeywords={Security, Risk, Assessment, Nmap, Vulnerability}
}

% --- DOCUMENT START ---
\begin{document}

\maketitle
\thispagestyle{empty}
\newpage

\tableofcontents
\newpage

% ===================================================================
\section{Executive Summary}
% ===================================================================

This report details the findings of a cybersecurity assessment conducted for \textbf{Falcon Heavy}. The assessment combined a technical network scan, a review of existing risks, and an analysis of organizational security controls based on a questionnaire.

The assessment identified several critical and high-risk vulnerabilities that expose the organization to significant threats, including unauthorized access, data breaches, and malware infections. Key findings include:

\begin{itemize}
    \item \textbf{Critical FTP Vulnerability:} An externally facing FTP server is running a severely outdated version of \texttt{vsftpd} (2.3.4), which is known to contain a critical backdoor vulnerability. Furthermore, the server is configured to allow anonymous logins, permitting unauthenticated access.
    \item \textbf{Critical Lack of Multi-Factor Authentication (MFA):} MFA is not enforced for accessing email, which is a primary target for account takeover and phishing attacks.
    \item \textbf{High-Risk Procedural Gaps:} MFA is also not required for computer logins, and new employees do not receive mandatory security awareness training, leaving the organization vulnerable to credential theft and social engineering.
\end{itemize}

The overall security posture is considered weak due to these fundamental control gaps. Immediate remediation of the identified critical risks is strongly recommended to reduce the likelihood of a security incident.

% ===================================================================
\section{Organizational Information}
% ===================================================================

The following information was provided by the organization for this assessment.

\begin{tabular}{@{}ll}
    \toprule
    \textbf{Attribute} & \textbf{Value} \\
    \midrule
    Organization Name & \textbf{Falcon Heavy} \\
    Email Domain & \texttt{FalconHeavy.com} \\
    Website Domain & \url{www.FalconHeavy.com} \\
    External IP Address & \texttt{94.3.167.67} \\
    \bottomrule
\end{tabular}

% ===================================================================
\section{Security Control Review}
% ===================================================================

The following table summarizes the organization's responses to the security controls questionnaire. Items marked with \no{} represent significant gaps in the security framework and are correlated with findings in the Risk Assessment section.

\begin{tabular}{@{}p{0.6\textwidth} c p{0.2\textwidth}@{}}
    \toprule
    \textbf{Control Question} & \textbf{Response} & \textbf{Assessment} \\
    \midrule
    Do you require MFA to access email? & \no & \textcolor{red}{\textbf{Critical Gap}} \\
    Do you require MFA to log into computers? & \no & \textcolor{orange}{High Risk} \\
    Do you require MFA to access sensitive data systems? & \yes & Control in Place \\
    Does your organization have an employee acceptable use policy? & \yes & Control in Place \\
    Does your organization do security awareness training for new employees? & \no & \textcolor{orange}{High Risk} \\
    Does your organization do security awareness training for all employees at least once per year? & \yes & Control in Place \\
    \bottomrule
\end{tabular}

% ===================================================================
\section{Technical Scan Results}
% ===================================================================

An external network scan was performed to identify open ports and exposed services.

\begin{itemize}
    \item \textbf{Target IP Address:} \texttt{10.0.0.15}
    \item \textbf{Scan Tool:} Nmap
\end{itemize}

\subsection{Open Ports and Services}
The following table details the services discovered during the scan.

\begin{tabular}{@{}lllll@{}}
    \toprule
    \textbf{Port} & \textbf{State} & \textbf{Service} & \textbf{Version} & \textbf{Details} \\
    \midrule
    21/tcp & open & ftp & vsftpd 2.3.4 & Anonymous FTP login allowed. \\
    \bottomrule
\end{tabular}

\subsection{Technical Analysis}
The scan revealed a critical finding:
\begin{enumerate}
    \item \textbf{Outdated and Vulnerable Service:} The FTP service is running \texttt{vsftpd 2.3.4}. This version was released in 2011 and is widely known to be vulnerable to a backdoor (CVE-2011-2523), which allows for remote command execution.
    \item \textbf{Insecure Configuration:} The service permits anonymous FTP logins. This allows any external party to connect to the server and potentially access, upload, or download files without authentication, posing a severe risk of data leakage or malware implantation.
\end{enumerate}

% ===================================================================
\section{Risk Assessment Summary}
% ===================================================================

This section synthesizes findings from the technical scan, security control review, and pre-existing risk data into a unified list. Each risk is assigned a severity level to guide prioritization.

\begin{tabular}{@{}lp{0.3\textwidth}p{0.4\textwidth}l@{}}
    \toprule
    \textbf{ID} & \textbf{Risk Name} & \textbf{Description} & \textbf{Severity} \\
    \midrule
    \rowcolor{sevCritical!25}
    RISK-001 & Vulnerable FTP Service & The public-facing FTP server runs \texttt{vsftpd 2.3.4}, which contains a known remote code execution backdoor (CVE-2011-2523). & \color{sevCritical}{\textbf{Critical}} \\
    
    \rowcolor{sevCritical!25}
    RISK-002 & Anonymous FTP Access & The FTP server is configured to allow anonymous login, permitting unauthenticated access to the file system. & \color{sevCritical}{\textbf{Critical}} \\
    
    \rowcolor{sevCritical!25}
    RISK-003 & No MFA for Email & Lack of MFA on email accounts significantly increases the risk of account compromise via phishing or credential stuffing. & \color{sevCritical}{\textbf{Critical}} \\
    
    \rowcolor{sevHigh!25}
    RISK-004 & No MFA for Computer Logins & The absence of MFA on workstations allows an attacker with stolen credentials to gain direct access to internal systems. & \color{sevHigh}{\textbf{High}} \\
    
    \rowcolor{sevHigh!25}
    RISK-005 & No Security Training for New Hires & New employees are not trained on security policies, making them highly susceptible to social engineering and phishing attacks. & \color{sevHigh}{\textbf{High}} \\
    
    \rowcolor{sevMedium!25}
    RISK-006 & Outdated Windows Policy & (Pre-existing) Workstations are running Windows 7, which is an unsupported OS and does not receive security updates. & \color{sevMedium}{\textbf{Medium}} \\
    \bottomrule
\end{tabular}

% ===================================================================
\section{Recommendations}
% ===================================================================

The following actions are recommended to mitigate the identified risks. They are prioritized by severity.

\subsection{Critical Priority Recommendations}
\begin{description}
    \item[RISK-001/002: Remediate Vulnerable FTP Server]
    \begin{itemize}
        \item \textbf{Immediate Action:} Take the FTP server offline immediately to prevent exploitation.
        \item \textbf{Short-Term:} If FTP is a business necessity, upgrade the \texttt{vsftpd} software to the latest stable version and disable anonymous access.
        \item \textbf{Long-Term:} Migrate from FTP to a more secure file transfer protocol, such as SFTP (SSH File Transfer Protocol) or FTPS (FTP over SSL/TLS).
    \end{itemize}
    
    \item[RISK-003: Implement MFA for Email]
    \begin{itemize}
        \item \textbf{Immediate Action:} Enforce MFA for all users, especially administrators and executives, on the \texttt{FalconHeavy.com} email system.
    \end{itemize}
\end{description}

\subsection{High Priority Recommendations}
\begin{description}
    \item[RISK-004: Implement MFA for Endpoints]
    \begin{itemize}
        \item \textbf{Action:} Deploy an MFA solution for all workstation and server logins to protect against unauthorized access from compromised credentials.
    \end{itemize}
    
    \item[RISK-005: Mandate New Hire Security Training]
    \begin{itemize}
        \item \textbf{Action:} Integrate mandatory security awareness training into the standard onboarding process for all new employees before they are granted system access.
    \end{itemize}
\end{description}

\subsection{Medium Priority Recommendations}
\begin{description}
    \item[RISK-006: Upgrade End-of-Life Operating Systems]
    \begin{itemize}
        \item \textbf{Action:} Continue with the planned project to upgrade all Windows 7 workstations to a supported operating system, such as Windows 10 or 11, to ensure they receive security patches.
    \end{itemize}
\end{description}

\end{document}
```