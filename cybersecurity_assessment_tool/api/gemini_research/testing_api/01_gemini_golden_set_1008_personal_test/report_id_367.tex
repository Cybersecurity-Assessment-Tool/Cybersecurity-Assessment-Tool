```latex
\documentclass[12pt]{article}

% Preamble: Required Packages
\usepackage[margin=1in]{geometry}
\usepackage{pifont} % For checkmarks and crosses
\usepackage{booktabs} % For professional tables
\usepackage{hyperref} % For hyperlinks
\usepackage{url}      % For URL formatting
\usepackage{seqsplit} % For splitting long strings like IPs

% Document Metadata
\title{Cybersecurity Posture Assessment Report}
\author{Cybersecurity Analysis Division}
\date{\today}

\begin{document}

\maketitle
\thispagestyle{empty}
\newpage
\tableofcontents
\newpage

% --- 1. Executive Summary ---
\section*{1. Executive Summary}

This report details the findings of a cybersecurity posture assessment for \textbf{Cinder \& Ash}. The assessment incorporated a review of organizational security controls, an external network scan, and an analysis of pre-existing risks.

The organization has implemented foundational security controls, such as requiring Multi-Factor Authentication (MFA) for email and computer access. However, several critical and high-risk gaps were identified that significantly increase the organization's risk exposure.

Key findings include:
\begin{itemize}
    \item \textbf{Critical Control Gap:} Multi-Factor Authentication is not required for accessing sensitive data systems.
    \item \textbf{High-Risk Policy Gaps:} The organization lacks a formal employee Acceptable Use Policy and does not conduct any security awareness training.
    \item \textbf{Technical Finding:} An external scan identified a publicly accessible Secure Shell (SSH) service on port 22. While common for remote administration, this service is a primary target for attackers and must be rigorously secured.
\end{itemize}

Immediate remediation is recommended to address these deficiencies, focusing on implementing MFA for sensitive systems, establishing a security training program, and reviewing the configuration of the exposed SSH service.

% --- 2. Organizational Information ---
\section*{2. Organizational Information}

The following information was provided for the assessment.

\begin{tabular}{@{}ll}
    \toprule
    \textbf{Attribute} & \textbf{Value} \\
    \midrule
    Organization Name & \textbf{Cinder \& Ash} \\
    Email Domain & \texttt{CinderAsh.com} \\
    Website Domain & \url{www.CinderAsh.com} \\
    External IP Address & \texttt{232.208.88.66} \\
    \bottomrule
\end{tabular}

% --- 3. Security Control Review ---
\section*{3. Security Control Review}

A review of administrative and organizational security controls was conducted via a questionnaire. The responses reveal significant gaps in policy and user awareness, which are foundational elements of a robust security program. A summary of the findings is presented in Table 1.

\begin{table}[h!]
\centering
\caption{Organizational Security Control Status}
\begin{tabular}{@{}p{0.75\linewidth}c@{}}
    \toprule
    \textbf{Control Question} & \textbf{Response} \\
    \midrule
    Do you require MFA to access email? & \ding{51} \\
    Do you require MFA to log into computers? & \ding{51} \\
    Do you require MFA to access sensitive data systems? & \textbf{\color{red}\ding{55}} \\
    Does your organization have an employee acceptable use policy? & \textbf{\color{red}\ding{55}} \\
    Does your organization do security awareness training for new employees? & \textbf{\color{red}\ding{55}} \\
    Does your organization do security awareness training for all employees at least once per year? & \textbf{\color{red}\ding{55}} \\
    \bottomrule
\end{tabular}
\end{table}

% --- 4. Technical Scan Results ---
\section*{4. Technical Scan Results}

An external network scan was performed to identify publicly accessible services. The scan targeted the IPv6 address associated with the organization.

\begin{itemize}
    \item \textbf{Target IP Address:} \seqsplit{\texttt{2001:db8::1}}
    \item \textbf{Scan Date:} As per scan metadata.
\end{itemize}

The scan identified one open port, as detailed in Table 2.

\begin{table}[h!]
\centering
\caption{Open Port Analysis}
\begin{tabular}{@{}llll@{}}
    \toprule
    \textbf{Port} & \textbf{State} & \textbf{Service} & \textbf{Analysis} \\
    \midrule
    22/tcp & Open & SSH (inferred) & Secure Shell is used for remote administration. \\
    & & & Its exposure requires strong authentication \\
    & & & and configuration to prevent unauthorized access. \\
    \bottomrule
\end{tabular}
\end{table}

\textbf{Note:} The scan did not retrieve service version information. A more detailed, authenticated scan is recommended to identify potential vulnerabilities associated with the specific SSH server software in use.

% --- 5. Consolidated Risk Assessment ---
\section*{5. Consolidated Risk Assessment}

The following risks have been identified by correlating the security control gaps and technical findings. As no pre-existing vulnerabilities were documented, all risks listed below are new findings from this assessment.

\begin{table}[h!]
\centering
\caption{Identified Risk Summary}
\begin{tabular}{@{}lp{0.3\linewidth}p{0.4\linewidth}l@{}}
    \toprule
    \textbf{ID} & \textbf{Risk Name} & \textbf{Description} & \textbf{Severity} \\
    \midrule
    RISK-001 & No MFA for Sensitive Data & Lack of MFA on critical systems allows an attacker with stolen credentials to gain direct access to sensitive information. & \textbf{Critical} \\
    \addlinespace
    RISK-002 & Lack of Security Awareness Training & Employees are not trained to recognize or respond to social engineering or phishing attacks, making them a vulnerable entry point. & \textbf{High} \\
    \addlinespace
    RISK-003 & No Acceptable Use Policy & Without a formal policy, there is no enforceable standard for employee behavior on corporate assets, increasing the risk of insider threat. & \textbf{High} \\
    \addlinespace
    RISK-004 & Exposed SSH Service & The publicly accessible SSH port is a high-value target for brute-force and credential stuffing attacks. & \textbf{Medium} \\
    \bottomrule
\end{tabular}
\end{table}

% --- 6. Recommendations ---
\section*{6. Recommendations}

The following actions are recommended to mitigate the identified risks and improve the overall security posture of \textbf{Cinder \& Ash}.

\begin{enumerate}
    \item \textbf{(RISK-001) Implement MFA for Sensitive Systems:}
    Immediately prioritize the deployment and enforcement of MFA across all applications, databases, and systems that store or process sensitive data. This is the single most effective control to mitigate the risk of credential compromise.

    \item \textbf{(RISK-002) Establish a Security Awareness Program:}
    Develop and implement a mandatory security awareness training program. This program should include onboarding training for new hires and annual refresher courses for all staff, covering topics such as phishing, password security, and data handling.

    \item \textbf{(RISK-003) Develop an Acceptable Use Policy (AUP):}
    Create and enforce a formal AUP that clearly defines the rules for using company IT resources. All employees must read and acknowledge this policy as a condition of being granted access to corporate systems.

    \item \textbf{(RISK-004) Secure the Exposed SSH Service:}
    Conduct an immediate review of the SSH service on \seqsplit{\texttt{2001:db8::1}}.
    \begin{itemize}
        \item If public access is not required, restrict access to trusted IP addresses using a firewall.
        \item If public access is required, enforce key-based authentication and disable password-based logins.
        \item Ensure logging is enabled and monitored for anomalous access patterns.
    \end{itemize}
\end{enumerate}

\end{document}
```