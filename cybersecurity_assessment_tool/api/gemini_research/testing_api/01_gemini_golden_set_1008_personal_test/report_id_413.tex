```latex
\documentclass[12pt, a4paper]{article}

% Preamble: Required Packages
\usepackage[margin=1in]{geometry}
\usepackage{pifont} % For checkmarks and crosses
\usepackage{booktabs} % For professional tables
\usepackage{hyperref} % For clickable links
\usepackage{url} % For formatting URLs
\usepackage{seqsplit} % For splitting long strings without breaking
\usepackage{graphicx}
\usepackage{xcolor}

% --- Document Setup ---
\hypersetup{
    colorlinks=true,
    linkcolor=blue,
    filecolor=magenta,      
    urlcolor=cyan,
    pdftitle={Cybersecurity Posture Report},
    pdfpagemode=FullScreen,
}

\newcommand{\yes}{\ding{51}}
\newcommand{\no}{\ding{55}}

% --- Document Body ---
\begin{document}

% --- Title Page ---
\begin{titlepage}
    \centering
    \vfill
    \huge
    \textbf{Cybersecurity Posture Report}
    \vspace{1.5cm}
    \Large
    \textbf{Prepared for: Ironclad Logistics}
    \vspace{2cm}
    \normalsize
    \textbf{Date:} \today \\
    \textbf{Report ID:} CSR-2023-451
    \vfill
    \textit{This report contains sensitive information and should be handled with care.}
\end{titlepage}

\tableofcontents
\newpage

% --- Executive Summary ---
\section*{Executive Summary}
This report provides a comprehensive analysis of the cybersecurity posture for \textbf{Ironclad Logistics}, based on a correlation of network scan data, organizational security controls, and pre-existing risk information.

The assessment identified several critical and high-risk vulnerabilities that require immediate attention. The most significant findings include:
\begin{itemize}
    \item \textbf{Systemic Remote Desktop Protocol (RDP) Exposure:} The network scan confirmed a new instance of an open RDP port on an internal system (\texttt{10.10.10.51}), adding to a previously identified risk on another machine. This pattern suggests a systemic configuration issue, creating a significant attack vector for ransomware and unauthorized access.
    \item \textbf{Critical Gaps in Multi-Factor Authentication (MFA):} MFA is not enforced for accessing email or for logging into company computers. This lack of fundamental security control drastically increases the risk of account compromise and lateral movement within the network.
    \item \textbf{Insufficient Employee Security Training:} New employees do not receive security awareness training upon being hired, leaving the organization vulnerable to social engineering and phishing attacks from day one of a new employee's tenure.
\end{itemize}

These findings, when combined, create a high-likelihood scenario for a security breach. An attacker could exploit the lack of MFA to compromise an email account, use stolen credentials to access an internal machine via the exposed RDP, and then move laterally. Immediate remediation of these issues is strongly recommended to reduce the organization's risk profile.

% --- Organizational Information ---
\section*{1. Organizational Information}
This section details the organizational data provided for the assessment.
\begin{itemize}
    \item \textbf{Organization Name:} Ironclad Logistics
    \item \textbf{Email Domain:} \texttt{IroncladLogistics.com}
    \item \textbf{Website Domain:} \url{www.IroncladLogistics.com}
    \item \textbf{Known External IP:} \texttt{8.157.232.16}
\end{itemize}

% --- Security Control Review ---
\section*{2. Security Control Review}
The following table summarizes the organization's responses to a security questionnaire. "No" answers indicate significant gaps in the security framework and are highlighted for review.

\begin{table}[h!]
\centering
\caption{Security Controls Questionnaire Analysis}
\begin{tabular}{p{0.5\linewidth} c p{0.3\linewidth}}
\toprule
\textbf{Control Question} & \textbf{Response} & \textbf{Analyst Note} \\
\midrule
Does your organization require MFA to access email? & \no & \textbf{Critical Risk.} A primary defense against Business Email Compromise (BEC) and phishing is missing. \\
\addlinespace
Does your organization require MFA to log into computers? & \no & \textbf{High Risk.} Lack of endpoint MFA allows attackers with stolen credentials to easily gain system access. \\
\addlinespace
Does your organization require MFA to access sensitive data systems? & \yes & Good practice. This control should be expanded to all critical systems. \\
\addlinespace
Does your organization have an employee acceptable use policy? & \yes & Foundational policy is in place. \\
\addlinespace
Does your organization do security awareness training for new employees? & \no & \textbf{High Risk.} New hires are a common target and represent an immediate vulnerability without initial training. \\
\addlinespace
Does your organization do security awareness training for all employees at least once per year? & \yes & Good practice for maintaining security awareness. \\
\bottomrule
\end{tabular}
\end{table}

% --- Technical Scan Results ---
\section*{3. Technical Scan Results}
An Nmap scan was conducted to identify open ports and services on the target system.

\begin{itemize}
    \item \textbf{Target IP Address:} \texttt{10.10.10.51}
    \item \textbf{Scan Status:} Host is up.
\end{itemize}

\begin{table}[h!]
\centering
\caption{Open Ports on \texttt{10.10.10.51}}
\begin{tabular}{c c c p{0.5\linewidth}}
\toprule
\textbf{Port} & \textbf{State} & \textbf{Service} & \textbf{Analyst Note} \\
\midrule
3389/tcp & open & \texttt{ms-wbt-server} & This is the standard port for Microsoft Remote Desktop Protocol (RDP). Exposing RDP without proper controls is a critical vulnerability often exploited by ransomware groups. \\
\bottomrule
\end{tabular}
\end{table}

\subsection*{Correlation with Existing Risks}
This finding is highly significant as it corroborates a pre-existing risk (\textit{RDP Exposure} on \texttt{10.10.10.50}). Discovering a second machine with the same vulnerability indicates a systemic issue rather than an isolated misconfiguration.

% --- Risk Assessment Summary ---
\section*{4. Risk Assessment Summary}
The following table synthesizes findings from the security control review, technical scan, and pre-existing risk data into a prioritized list.

\begin{table}[h!]
\centering
\caption{Synthesized Risk Register}
\begin{tabular}{p{0.1\linewidth} p{0.2\linewidth} p{0.4\linewidth} p{0.15\linewidth}}
\toprule
\textbf{ID} & \textbf{Risk Name} & \textbf{Description} & \textbf{Severity} \\
\midrule
\textbf{R-01} & Systemic RDP Exposure & RDP is exposed on multiple internal systems (\texttt{10.10.10.50}, \texttt{10.10.10.51}). This is a primary vector for ransomware and direct network compromise. & \textbf{Critical} \\
\addlinespace
\textbf{R-02} & Lack of Foundational MFA & MFA is not enforced on email or computer logins, making credential theft highly impactful and leaving core systems vulnerable to unauthorized access. & \textbf{Critical} \\
\addlinespace
\textbf{R-03} & Inadequate Employee Onboarding Security & New employees are not provided with security awareness training, creating an immediate and ongoing human-layer vulnerability. & \textbf{High} \\
\bottomrule
\end{tabular}
\end{table}

% --- Recommendations ---
\section*{5. Recommendations}
Based on the analysis, the following actions are recommended to mitigate the identified risks. Recommendations are prioritized by severity.

\subsection*{Immediate Actions (Critical Priority)}
\begin{enumerate}
    \item \textbf{Remediate RDP Exposure:} Immediately restrict all access to TCP port 3389 on \texttt{10.10.10.50} and \texttt{10.10.10.51} from any untrusted network. If remote access is a business requirement, it \textbf{must} be placed behind a Virtual Private Network (VPN) that requires MFA.
    \item \textbf{Enforce MFA on Email:} Enable MFA for all user email accounts immediately. This is the single most effective control to prevent account takeovers and Business Email Compromise (BEC) attacks.
\end{enumerate}

\subsection*{High Priority Actions}
\begin{enumerate}
    \setcounter{enumi}{2}
    \item \textbf{Deploy MFA for Endpoint Logins:} Plan and execute the rollout of MFA for all employee computer and server logins. This strengthens internal security and contains the impact of stolen credentials.
    \item \textbf{Integrate Security into Onboarding:} Develop and mandate a security awareness training module for all new employees, to be completed during their first week. This closes a critical gap in the employee lifecycle.
\end{enumerate}

\subsection*{General Recommendations}
\begin{enumerate}
    \setcounter{enumi}{4}
    \item \textbf{Conduct Comprehensive Vulnerability Scanning:} Perform authenticated scans on the internal network and unauthenticated scans on the external perimeter to identify any other exposed services or vulnerabilities.
    \item \textbf{Review and Update Policies:} Review the Acceptable Use Policy to ensure it explicitly covers remote access protocols and MFA requirements.
\end{enumerate}

\end{document}
```