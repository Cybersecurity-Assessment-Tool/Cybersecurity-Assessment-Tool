```latex
\documentclass[12pt]{article}

% Preamble: Required Packages
\usepackage[margin=1in]{geometry}
\usepackage{pifont} % For checkmarks and crosses
\usepackage{booktabs} % For professional tables
\usepackage{hyperref} % For clickable links
\usepackage{url} % For formatting URLs
\usepackage{seqsplit} % For splitting long strings to prevent overflow
\usepackage{xcolor} % For colors
\usepackage{graphicx}
\usepackage{framed}

% Document Metadata
\title{Cybersecurity Posture Assessment Report}
\author{Cybersecurity Analysis Division}
\date{\today}

\begin{document}

\maketitle
\thispagestyle{empty}
\clearpage

\tableofcontents
\clearpage

% --- 1. Executive Overview ---
\section{Executive Overview}

This report provides a comprehensive cybersecurity posture assessment for \textbf{Foresight Strategies}. The analysis is based on a synthesis of technical network scan data, a review of organizational security controls via a questionnaire, and an evaluation of previously documented risks.

The assessment reveals a mixed security posture. On a positive note, a previously identified risk concerning an unencrypted web server on port 80 appears to have been successfully remediated, as our technical scan confirmed the port is now closed. This indicates progress in vulnerability management.

However, significant gaps were identified in foundational security policies and procedures. The most critical risk is the absence of Multi-Factor Authentication (MFA) for email access, which exposes the organization to a high likelihood of account compromise and subsequent data breaches. Additional high-risk gaps include the lack of a formal employee acceptable use policy and the failure to provide security awareness training to new hires during their onboarding process.

Immediate action should be focused on implementing MFA for email and developing the missing security policies to mitigate these critical and high-risk exposures.

% --- 2. Organizational Information ---
\section{Organizational Information}

The following details were provided for the assessment. This information is used to establish the context and scope of the review.

\begin{framed}
\begin{tabular}{@{}ll}
    \textbf{Organization Name:} & \textbf{Foresight Strategies} \\
    \textbf{Email Domain:} & \texttt{ForesightStrategies.org} \\
    \textbf{Website Domain:} & \url{www.ForesightStrategies.org} \\
    \textbf{External IP Address:} & \texttt{124.186.23.27} \\
\end{tabular}
\end{framed}

% --- 3. Security Control Review (Questionnaire) ---
\section{Security Control Review}

An analysis of the organization's security questionnaire was performed to evaluate the implementation of key administrative and technical controls. The table below summarizes the responses and provides an assessment of each control's status. Gaps in these controls often represent significant organizational risk.

\begin{table}[h!]
\centering
\caption{Security Controls Questionnaire Analysis}
\begin{tabular}{@{}p{0.55\textwidth} c p{0.2\textwidth}@{}}
\toprule
\textbf{Control Question} & \textbf{Response} & \textbf{Assessment} \\
\midrule
Do you require MFA to access email? & \ding{55} & \textcolor{red}{\textbf{Critical Gap}} \\
Do you require MFA to log into computers? & \ding{51} & Implemented \\
Do you require MFA to access sensitive data systems? & \ding{51} & Implemented \\
Does your organization have an employee acceptable use policy? & \ding{55} & \textcolor{orange}{\textbf{High Risk}} \\
Does your organization do security awareness training for new employees? & \ding{55} & \textcolor{orange}{\textbf{High Risk}} \\
Does your organization do security awareness training for all employees at least once per year? & \ding{51} & Implemented \\
\bottomrule
\end{tabular}
\\
\vspace{0.2cm}
\small{\textit{Key: \ding{51} = Yes (Control in place), \ding{55} = No (Control gap identified)}}
\end{table}

% --- 4. Technical Scan Results ---
\section{Technical Scan Results}

A network scan was conducted to identify open ports and exposed services on the target system. The results provide a technical snapshot of the host's external-facing posture.

\begin{itemize}
    \item \textbf{Target IP Address:} \texttt{192.168.0.5}
    \item \textbf{Scan Tool:} Nmap
\end{itemize}

\begin{table}[h!]
\centering
\caption{Network Port Scan Findings}
\begin{tabular}{@{}lllll@{}}
\toprule
\textbf{Port} & \textbf{State} & \textbf{Service} & \textbf{Product/Version} & \textbf{Notes} \\
\midrule
80/tcp & closed & http & N/A & \begin{tabular}[t]{@{}l@{}}This port was found to be closed. This \\ finding suggests that a previously identified \\ risk of an unencrypted web server has \\ been remediated.\end{tabular} \\
\bottomrule
\end{tabular}
\end{table}

% --- 5. Consolidated Risk Assessment ---
\section{Consolidated Risk Assessment}

This section synthesizes findings from the security control review, technical scan, and pre-existing risk data into a consolidated list. Risks are prioritized based on their potential impact on the organization.

\begin{table}[h!]
\centering
\caption{Summary of Identified Risks}
\begin{tabular}{@{}p{0.25\textwidth} p{0.45\textwidth} p{0.15\textwidth}@{}}
\toprule
\textbf{Risk Name} & \textbf{Description} & \textbf{Severity} \\
\midrule
\textbf{Email Account Compromise via Missing MFA} & The lack of MFA on email accounts makes them highly susceptible to takeover through phishing or credential stuffing attacks, which can lead to data breaches and financial fraud. & \textcolor{red}{\textbf{Critical}} \\
\addlinespace
\textbf{Lack of Employee Acceptable Use Policy (AUP)} & Without a formal AUP, employees lack clear guidelines on the secure and acceptable use of company assets, increasing the risk of insider threat and accidental data exposure. & \textcolor{orange}{\textbf{High}} \\
\addlinespace
\textbf{Inadequate New Hire Security Onboarding} & New employees are not receiving security awareness training upon joining, leaving a critical window where they are more vulnerable to social engineering and policy violations. & \textcolor{orange}{\textbf{High}} \\
\addlinespace
\textbf{Unencrypted Web Server (Remediated)} & A previously documented risk stated that port 80 was open. Our technical scan confirmed this port is now \textbf{closed}. The risk is considered remediated. & \textcolor{green}{\textbf{Informational}} \\
\bottomrule
\end{tabular}
\end{table}

% --- 6. Recommendations ---
\section{Recommendations}

Based on the consolidated risk assessment, the following actions are recommended to improve the cybersecurity posture of \textbf{Foresight Strategies}. Recommendations are prioritized by severity.

\subsection{Immediate Priority (Critical)}
\begin{enumerate}
    \item \textbf{Enforce MFA for Email Access:} Immediately deploy and enforce MFA for all user access to the email system (\texttt{ForesightStrategies.org}). This is the single most effective control to mitigate the risk of business email compromise.
\end{enumerate}

\subsection{High Priority}
\begin{enumerate}
    \setcounter{enumi}{1}
    \item \textbf{Develop and Implement an Acceptable Use Policy (AUP):} Create a formal AUP that clearly defines the rules and responsibilities for all employees when using corporate IT resources. Ensure all employees read and acknowledge the policy.
    \item \textbf{Integrate Security Training into New Hire Onboarding:} Develop a mandatory security awareness training module for all new employees. This training should be a required part of the onboarding process before system access is granted.
\end{enumerate}

\subsection{Administrative Actions}
\begin{enumerate}
    \setcounter{enumi}{3}
    \item \textbf{Update Risk Register:} The internal risk register should be updated to formally close the risk associated with the "Unencrypted Web Server" on port 80, reflecting its successful remediation.
\end{enumerate}

\end{document}
```