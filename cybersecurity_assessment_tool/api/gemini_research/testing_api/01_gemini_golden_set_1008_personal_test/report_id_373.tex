```latex
\documentclass[12pt]{article}

% --- PACKAGES ---
\usepackage[margin=1in]{geometry}
\usepackage{pifont} % For checkmarks and crosses
\usepackage{booktabs} % For professional tables
\usepackage{hyperref} % For clickable links and TOC
\usepackage{url} % For formatting URLs
\usepackage{seqsplit} % For splitting long text strings
\usepackage[T1]{fontenc}

% --- DOCUMENT METADATA ---
\hypersetup{
    colorlinks=true,
    linkcolor=black,
    urlcolor=blue,
    pdftitle={Cybersecurity Assessment Report},
    pdfauthor={Cybersecurity Analyst},
    pdfsubject={Security Analysis}
}

\title{Cybersecurity Assessment Report \\ \large For: \textbf{Top Tier}}
\author{Cybersecurity Analyst}
\date{November 22, 2025}

\begin{document}

\maketitle
\thispagestyle{empty}
\newpage

\tableofcontents
\thispagestyle{empty}
\newpage

% --- EXECUTIVE SUMMARY ---
\section{Executive Summary}
\pagestyle{headings}
This report details the findings of a cybersecurity assessment conducted on November 22, 2025. The assessment combined a technical network scan, a review of organizational security controls, and an analysis of existing risks.

The overall security posture of \textbf{Top Tier} is considered to be at a \textbf{High Risk} level. Several critical and high-risk vulnerabilities were identified that require immediate attention.

Key findings include:
\begin{itemize}
    \item \textbf{Critical Lack of Multi-Factor Authentication (MFA):} The organization has not implemented MFA for email, computer logins, or access to sensitive data systems. This represents a critical vulnerability, as a single compromised password could lead to a widespread system breach.
    \item \textbf{Outdated Web Server Software:} The external-facing web server at \texttt{192.168.10.5} is running Nginx version 1.18.0, a version released in 2020. This software is significantly outdated and is known to be vulnerable to multiple publicly disclosed security flaws (CVEs).
    \item \textbf{Insufficient Security Training:} While new employees receive security training, there is no mandatory annual refresher course for all staff. This gap allows security knowledge to become stale, increasing susceptibility to evolving threats like sophisticated phishing attacks.
\end{itemize}

Immediate remediation of these issues is strongly recommended to reduce the organization's attack surface and mitigate the risk of a significant security incident. Detailed recommendations are provided in Section \ref{sec:recommendations}.

% --- ORGANIZATIONAL INFORMATION ---
\section{Organizational Information}
The following information was provided for the assessment.

\begin{tabular}{@{}ll}
\toprule
\textbf{Attribute} & \textbf{Value} \\
\midrule
Organization Name & \textbf{Top Tier} \\
Email Domain & \texttt{TopTier.org} \\
Website Domain & \url{www.TopTier.org} \\
External IP Address & \texttt{175.78.237.172} \\
\bottomrule
\end{tabular}

% --- SECURITY CONTROL REVIEW ---
\section{Security Control Review}
A review of administrative and organizational security controls was conducted via a questionnaire. The responses indicate significant gaps in foundational security practices. "No" answers highlight areas of high risk that weaken the organization's defensive capabilities.

\begin{table}[h!]
\centering
\caption{Security Controls Questionnaire Results}
\begin{tabular}{@{}p{0.8\linewidth}c@{}}
\toprule
\textbf{Control Question} & \textbf{Response} \\
\midrule
Do you require MFA to access email? & \ding{55} \\
Do you require MFA to log into computers? & \ding{55} \\
Do you require MFA to access sensitive data systems? & \ding{55} \\
Does your organization have an employee acceptable use policy? & \ding{51} \\
Does your organization do security awareness training for new employees? & \ding{51} \\
Does your organization do security awareness training for all employees at least once per year? & \ding{55} \\
\bottomrule
\end{tabular}
\end{table}

\noindent \textbf{Analysis:} The complete absence of Multi-Factor Authentication is a critical deficiency. Email is a primary target for attackers, and without MFA, compromised credentials can lead directly to data breaches and further internal attacks. The lack of annual security training for all employees perpetuates a high-risk environment where staff may not recognize modern social engineering tactics.

% --- TECHNICAL SCAN RESULTS ---
\section{Technical Scan Results}
An Nmap scan was performed to identify open ports and running services on the target system.

\subsection{Target: \texttt{192.168.10.5}}
The scan identified one open port on the target host.

\begin{table}[h!]
\centering
\caption{Open Ports and Services on \texttt{192.168.10.5}}
\begin{tabular}{@{}lllll@{}}
\toprule
\textbf{Port} & \textbf{State} & \textbf{Service} & \textbf{Product} & \textbf{Version} \\
\midrule
443/tcp & open & https & nginx & 1.18.0 \\
\bottomrule
\end{tabular}
\end{table}

\noindent \textbf{Analysis:} The host is running a web server (nginx) on port 443 (HTTPS). The detected version, \textbf{1.18.0}, is severely outdated. This version is past its end-of-life and is susceptible to numerous known vulnerabilities. Running outdated software on an internet-facing service presents a high risk of compromise.

% --- RISK ASSESSMENT ---
\section{Risk Assessment}
The following table summarizes the key risks identified during this assessment, derived from correlating the security control gaps and technical findings. No pre-existing vulnerabilities were reported.

\begin{table}[h!]
\centering
\caption{Summary of Identified Risks}
\begin{tabular}{@{}lp{0.5\linewidth}l@{}}
\toprule
\textbf{Risk Name} & \textbf{Overview} & \textbf{Severity} \\
\midrule
\textbf{Widespread Lack of MFA} & The absence of MFA on email, endpoints, and sensitive systems allows an attacker with valid credentials to gain unauthorized access without additional checks. & \textbf{Critical} \\
\\
\textbf{Outdated Web Server} & The Nginx server (v1.18.0) is outdated and has multiple known vulnerabilities, making it a prime target for automated attacks and exploitation. & \textbf{High} \\
\\
\textbf{Insufficient Security Training} & Lack of annual training makes employees more vulnerable to social engineering and phishing, which is the primary method for initial credential theft. & \textbf{High} \\
\bottomrule
\end{tabular}
\end{table}

% --- RECOMMENDATIONS ---
\section{Recommendations}
\label{sec:recommendations}
The following actions are recommended to address the identified risks and improve the overall security posture of the organization.

\begin{enumerate}
    \item \textbf{Implement Multi-Factor Authentication (Critical):}
    \begin{itemize}
        \item \textbf{Action:} Immediately begin a phased rollout of a mandatory MFA policy.
        \item \textbf{Priority 1:} Enable MFA for all email accounts (e.g., via O365 or Google Workspace controls).
        \item \textbf{Priority 2:} Enforce MFA for remote access (VPN) and all administrative accounts.
        \item \textbf{Priority 3:} Extend MFA requirements to all user endpoints and applications accessing sensitive data.
    \end{itemize}
    \vspace{0.5cm}
    \item \textbf{Establish a Patch Management Program (High):}
    \begin{itemize}
        \item \textbf{Action:} Immediately upgrade the Nginx server at \texttt{192.168.10.5} to the latest stable version.
        \item \textbf{Policy:} Develop and implement a formal patch management policy. This policy should mandate that all critical internet-facing systems are patched within 14-30 days of a vendor releasing a security update.
    \end{itemize}
    \vspace{0.5cm}
    \item \textbf{Enhance Security Awareness Program (High):}
    \begin{itemize}
        \item \textbf{Action:} Institute a mandatory annual security awareness training program for all employees, contractors, and executives.
        \item \textbf{Content:} The training should cover current threats, including phishing, ransomware, and proper data handling.
        \item \textbf{Testing:} Supplement the training with periodic, unannounced phishing simulation tests to measure effectiveness and identify areas for improvement.
    \end{itemize}
\end{enumerate}

\end{document}
```