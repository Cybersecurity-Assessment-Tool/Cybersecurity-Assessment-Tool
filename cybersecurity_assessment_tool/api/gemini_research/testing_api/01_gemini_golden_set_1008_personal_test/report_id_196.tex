```latex
\documentclass[12pt]{article}

% Preamble: Required Packages
\usepackage[margin=1in]{geometry}
\usepackage{pifont} % For \ding
\usepackage{booktabs} % For professional tables
\usepackage{hyperref} % For clickable links and TOC
\usepackage{url} % For formatting URLs
\usepackage{seqsplit} % For splitting long strings in texttt
\usepackage{xcolor} % For coloring text
\usepackage{graphicx} % For potential logos/images

% Hyperref Setup
\hypersetup{
    colorlinks=true,
    linkcolor=blue,
    filecolor=magenta,      
    urlcolor=cyan,
    pdftitle={Cybersecurity Assessment Report},
    pdfpagemode=FullScreen,
}

% Custom Commands
\newcommand{\yes}{\textcolor{green}{\ding{51}}}
\newcommand{\no}{\textcolor{red}{\ding{55}}}

\begin{document}

% --- Title Page ---
\begin{titlepage}
    \centering
    \vspace*{1cm}
    \Huge\textbf{Cybersecurity Assessment Report}
    \vspace{1.5cm}
    \Large
    \begin{tabular}{@{}l@{\quad}l}
        \textbf{Client:} & Fable \& Lore \\
        \textbf{Date of Report:} & \today \\
        \textbf{Author:} & Cybersecurity Analyst \\
    \end{tabular}
    \vfill
    \large
    \textit{This report contains sensitive information and is intended for the exclusive use of Fable \& Lore. Distribution is strictly prohibited.}
\end{titlepage}

% --- Table of Contents ---
\tableofcontents
\newpage

% --- Section 1: Executive Summary ---
\section{Executive Summary}
This report provides a cybersecurity assessment for Fable \& Lore, based on a review of organizational security controls, an external network scan, and an analysis of known risks. The assessment was conducted to identify security gaps and provide actionable recommendations to enhance the organization's security posture.

Overall, Fable \& Lore demonstrates a strong commitment to identity and access management, with commendable enforcement of Multi-Factor Authentication (MFA) across all critical systems. The security awareness training program is also well-established.

The external network scan of the target host \texttt{192.168.1.100} revealed a positive security posture, with no open ports detected. This suggests a properly configured firewall and a minimal attack surface for the scanned asset.

The most critical finding of this assessment is an administrative gap: the absence of a formal Employee Acceptable Use Policy (AUP). This represents a high-risk issue, as it leaves the organization without a foundational document governing the secure use of its information systems.

Our recommendations prioritize the development and implementation of this core policy to mitigate insider threats and establish clear security expectations for all employees.

% --- Section 2: Organizational Information ---
\section{Organizational Information}
The following details were provided for the assessment. This information helps establish the context for the security review.

\begin{tabular}{@{}ll}
    \toprule
    \textbf{Attribute} & \textbf{Value} \\
    \midrule
    Organization Name & Fable \& Lore \\
    Email Domain & \texttt{FableLore.com} \\
    Website Domain & \url{www.FableLore.com} \\
    External IP Address & \texttt{161.187.222.176} \\
    \bottomrule
\end{tabular}

% --- Section 3: Security Control Review ---
\section{Security Control Review}
A review of administrative and technical security controls was conducted via a questionnaire. The responses indicate the current state of implemented security policies and practices. Gaps identified here often point to systemic risks.

\begin{table}[h!]
\centering
\caption{Security Controls Questionnaire Results}
\begin{tabular}{p{0.75\linewidth}c}
    \toprule
    \textbf{Control Question} & \textbf{Response} \\
    \midrule
    Do you require MFA to access email? & \yes \\
    Do you require MFA to log into computers? & \yes \\
    Do you require MFA to access sensitive data systems? & \yes \\
    Does your organization have an employee acceptable use policy? & \no \\
    Does your organization do security awareness training for new employees? & \yes \\
    Does your organization do security awareness training for all employees at least once per year? & \yes \\
    \bottomrule
\end{tabular}
\end{table}

\subsection*{Analysis}
The organization has successfully implemented MFA across key access points, significantly reducing the risk of unauthorized access from compromised credentials. The security awareness training program is robust. However, the lack of an Acceptable Use Policy is a critical administrative control gap that must be addressed.

% --- Section 4: Technical Scan Results ---
\section{Technical Scan Results}
A network scan was performed to identify open ports and exposed services on the target system. This analysis helps to evaluate the external attack surface and identify potential technical vulnerabilities.

\begin{itemize}
    \item \textbf{Target IP Address:} \texttt{192.168.1.100}
    \item \textbf{Host Status:} UP
    \item \textbf{Scan Summary:} The scan completed successfully and determined that the host was online. A comprehensive port scan was performed, and \textbf{no open ports were discovered}. All scanned ports were found to be in a 'closed' state.
\end{itemize}

\subsection*{Analysis}
This is an excellent result. A finding of no open ports indicates a strong firewall configuration and adherence to the principle of least privilege at the network level. The external attack surface for this specific host appears to be minimal, which significantly reduces the risk of network-based attacks.

% --- Section 5: Risk Assessment ---
\section{Risk Assessment}
This section synthesizes findings from the security control review, technical scans, and pre-existing risk data. The primary risk identified during this assessment is detailed below.

\begin{table}[h!]
\centering
\caption{Identified Risks}
\begin{tabular}{p{0.1\linewidth}p{0.25\linewidth}p{0.4\linewidth}p{0.1\linewidth}}
    \toprule
    \textbf{ID} & \textbf{Risk Name} & \textbf{Description} & \textbf{Severity} \\
    \midrule
    RISK-001 & Absence of Acceptable Use Policy (AUP) & The organization lacks a formal policy defining the rules and responsibilities for employees using company IT assets. This can lead to misuse of resources, data mishandling, and a lack of legal recourse in case of an internal breach. & \textbf{High} \\
    \bottomrule
\end{tabular}
\end{table}

% --- Section 6: Recommendations ---
\section{Recommendations}
Based on the findings of this assessment, the following prioritized recommendations are provided to enhance the security posture of Fable \& Lore.

\subsection{Priority 1: Develop and Implement an Acceptable Use Policy}
\begin{itemize}
    \item \textbf{Risk Mitigated:} RISK-001 (High)
    \item \textbf{Action:} Draft a comprehensive AUP that clearly outlines the rules for all employees regarding the use of company computers, networks, email, and internet access. The policy should explicitly forbid activities such as installing unauthorized software, sharing credentials, and accessing inappropriate content.
    \item \textbf{Details:} The AUP should include:
    \begin{itemize}
        \item Clear guidelines on data handling and protection.
        \item Rules for remote access and use of personal devices (BYOD), if applicable.
        \item Stated consequences for policy violations.
    \end{itemize}
    \item \textbf{Implementation:} Once drafted and approved by management, the policy must be distributed to all current employees for their review and signed acknowledgment. This process should be integrated into the onboarding procedure for all new hires.
\end{itemize}

\subsection{Priority 2: Maintain Strong Network Security Posture}
\begin{itemize}
    \item \textbf{Risk Mitigated:} Proactive measure
    \item \textbf{Action:} Continue the current practice of maintaining a minimal network attack surface. The clean network scan is a positive indicator that should be maintained.
    \item \textbf{Details:} Periodically conduct internal and external vulnerability scans to ensure that no new, unauthorized services are exposed. Ensure that firewall rules are regularly reviewed and updated, especially after any network changes.
\end{itemize}

\end{document}
```