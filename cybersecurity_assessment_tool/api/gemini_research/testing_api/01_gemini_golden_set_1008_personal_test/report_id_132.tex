```latex
\documentclass[12pt]{article}

% ----------------------------------------------------------------------
% PREAMBLE
% ----------------------------------------------------------------------
\usepackage[margin=1in]{geometry}
\usepackage{pifont} % For checkmarks and crosses
\usepackage{booktabs} % For professional tables
\usepackage{xcolor} % For colored text
\usepackage{hyperref} % For hyperlinks
\usepackage{url} % For URL formatting
\usepackage{seqsplit} % For splitting long strings to prevent overflow

% --- Color Definitions for Severity ---
\definecolor{severitycritical}{HTML}{D10000}
\definecolor{severityhigh}{HTML}{E57300}
\definecolor{severitymedium}{HTML}{FFBF00}
\definecolor{severitylow}{HTML}{0072B2}

% --- Hyperref Setup ---
\hypersetup{
    colorlinks=true,
    linkcolor=blue,
    filecolor=magenta,      
    urlcolor=cyan,
    pdftitle={Cybersecurity Posture Assessment Report},
    pdfpagemode=FullScreen,
}

% ----------------------------------------------------------------------
% DOCUMENT START
% ----------------------------------------------------------------------
\begin{document}

% --- Title Page ---
\title{Cybersecurity Posture Assessment Report \\ \large For: \textbf{Vertex Solutions}}
\author{Cybersecurity Analysis Division}
\date{\today}
\maketitle

\tableofcontents
\newpage

% ----------------------------------------------------------------------
% SECTION 1: EXECUTIVE SUMMARY
% ----------------------------------------------------------------------
\section{Executive Summary}

This report provides a cybersecurity posture assessment for \textbf{Vertex Solutions}, based on an analysis of organizational security controls, technical network scan data, and pre-existing risk information.

The analysis of the security questionnaire revealed several critical and high-risk gaps in the current security posture. The most significant findings are the absence of Multi-Factor Authentication (MFA) for computer and sensitive data system access. Furthermore, the lack of a formal Acceptable Use Policy (AUP) and mandatory security training for new hires represents a substantial weakness in security governance and employee awareness. These gaps expose the organization to significant risks, including unauthorized access, data breaches, and insider threats.

\textbf{Important Note on Data Integrity:} The technical network scan data (\texttt{Input\_1\_Network\_Scan\_JSON}) and the list of current organizational risks (\texttt{Input\_3\_Current\_Risks\_JSON}) were found to be corrupted and could not be parsed. This has limited the scope of our analysis, preventing a correlation between policy gaps and specific technical vulnerabilities. A comprehensive assessment requires a successful rescan of the network assets and a review of the complete risk register.

Immediate remediation should focus on implementing MFA across all critical systems and developing foundational security policies and training programs.

% ----------------------------------------------------------------------
% SECTION 2: ORGANIZATIONAL INFORMATION
% ----------------------------------------------------------------------
\section{Organizational Information}

The following details were provided for the assessment:
\begin{itemize}
    \item \textbf{Organization Name:} Vertex Solutions
    \item \textbf{Email Domain:} \texttt{VertexSolutions.net}
    \item \textbf{Website Domain:} \url{www.VertexSolutions.net}
    \item \textbf{External IP Address:} \texttt{108.98.255.15}
\end{itemize}

% ----------------------------------------------------------------------
% SECTION 3: SECURITY CONTROL REVIEW
% ----------------------------------------------------------------------
\section{Security Control Review (Questionnaire Analysis)}

An analysis of the security questionnaire was performed to evaluate the implementation of key administrative and technical controls. "No" answers indicate significant gaps that increase organizational risk.

\begin{table}[h!]
\centering
\caption{Security Questionnaire Analysis}
\begin{tabular}{p{0.6\linewidth} c l}
\toprule
\textbf{Control Question} & \textbf{Response} & \textbf{Assessment} \\
\midrule
Do you require MFA to access email? & \ding{51} & Implemented. \\
\addlinespace
Do you require MFA to log into computers? & \ding{55} & \textcolor{severitycritical}{\textbf{Critical Gap}} \\
\addlinespace
Do you require MFA to access sensitive data systems? & \ding{55} & \textcolor{severitycritical}{\textbf{Critical Gap}} \\
\addlinespace
Does your organization have an employee acceptable use policy? & \ding{55} & \textcolor{severityhigh}{\textbf{High Risk}} \\
\addlinespace
Does your organization do security awareness training for new employees? & \ding{55} & \textcolor{severityhigh}{\textbf{High Risk}} \\
\addlinespace
Does your organization do security awareness training for all employees at least once per year? & \ding{51} & Implemented. \\
\bottomrule
\end{tabular}
\end{table}

% ----------------------------------------------------------------------
% SECTION 4: TECHNICAL SCAN RESULTS
% ----------------------------------------------------------------------
\section{Technical Scan Results}

\subsection{Data Corruption Issue}
The provided network scan data file (\texttt{Input\_1\_Network\_Scan\_JSON}) was corrupted and could not be processed. Consequently, no analysis of open ports, running services, or potential vulnerabilities on the target external IP address (\texttt{108.98.255.15}) could be performed.

\subsection{Impact}
Without this technical data, it is impossible to identify specific software vulnerabilities, misconfigurations, or exposed services that could be exploited by an attacker. This represents a significant blind spot in the current assessment.

\subsection{Recommendation}
It is strongly recommended to conduct a new, authenticated vulnerability scan against the target IP address \texttt{108.98.255.15} and any other external-facing assets to obtain the necessary technical data for a complete risk assessment.

% ----------------------------------------------------------------------
% SECTION 5: RISK ASSESSMENT
% ----------------------------------------------------------------------
\section{Risk Assessment}

This risk assessment is based on the findings from the Security Control Review. Due to corrupted input data, it does not include risks from technical scans or pre-existing vulnerability lists.

\begin{table}[h!]
\centering
\caption{Identified Risks and Severity}
\begin{tabular}{p{0.15\linewidth} p{0.25\linewidth} p{0.4\linewidth} l}
\toprule
\textbf{Risk ID} & \textbf{Risk Name} & \textbf{Description} & \textbf{Severity} \\
\midrule
RISK-001 & Lack of MFA on Endpoints & User accounts for computer logins are protected only by passwords, making them vulnerable to brute-force attacks, credential stuffing, and phishing. & \textcolor{severitycritical}{Critical} \\
\addlinespace
RISK-002 & Lack of MFA on Sensitive Systems & Critical data systems lack a secondary authentication factor, exposing sensitive information to a high risk of unauthorized access if credentials are compromised. & \textcolor{severitycritical}{Critical} \\
\addlinespace
RISK-003 & Missing Acceptable Use Policy (AUP) & Without a formal AUP, there are no clear guidelines for employees on the acceptable use of company assets, increasing the risk of misuse and insider threats. & \textcolor{severityhigh}{High} \\
\addlinespace
RISK-004 & Inadequate New Hire Security Training & New employees are not formally trained on security best practices upon joining, leaving them unaware of policies and potential threats from day one. & \textcolor{severityhigh}{High} \\
\bottomrule
\end{tabular}
\end{table}

% ----------------------------------------------------------------------
% SECTION 6: RECOMMENDATIONS
% ----------------------------------------------------------------------
\section{Recommendations}

The following prioritized recommendations are provided to mitigate the identified risks and improve the overall security posture of \textbf{Vertex Solutions}.

\subsection{Priority 1: Critical}
\begin{enumerate}
    \item \textbf{Implement MFA for Endpoint Access:} Enforce Multi-Factor Authentication for all employee and administrator logins to workstations and servers. This is the single most effective control to prevent unauthorized access from compromised credentials.
    \item \textbf{Implement MFA for Sensitive Data Systems:} Immediately deploy MFA for all applications and systems that store, process, or transmit sensitive organizational or customer data.
\end{enumerate}

\subsection{Priority 2: High}
\begin{enumerate}
    \item \textbf{Develop and Enforce an Acceptable Use Policy (AUP):} Create a formal AUP that clearly defines the rules for using company networks, devices, and data. Require all employees to read and acknowledge the policy.
    \item \textbf{Establish a New Hire Security Training Program:} Integrate mandatory security awareness training into the employee onboarding process. This training should cover key topics such as phishing, password security, data handling, and the new AUP.
\end{enumerate}

\subsection{Priority 3: Data Remediation}
\begin{enumerate}
    \item \textbf{Conduct a New Network Vulnerability Scan:} Execute a comprehensive, authenticated vulnerability scan against the external IP \texttt{108.98.255.15} and all other internet-facing assets to identify and remediate technical vulnerabilities.
    \item \textbf{Restore Risk Register Data:} The corrupted list of existing vulnerabilities (\texttt{Input\_3\_Current\_Risks\_JSON}) should be restored from backups or regenerated to ensure a complete and accurate view of the organization's risk landscape.
\end{enumerate}

% ----------------------------------------------------------------------
% DOCUMENT END
% ----------------------------------------------------------------------
\end{document}
```