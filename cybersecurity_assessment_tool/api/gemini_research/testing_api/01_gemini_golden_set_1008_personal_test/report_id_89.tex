```latex
\documentclass[12pt]{article}

% Preamble: Required Packages
\usepackage[margin=1in]{geometry}
\usepackage{pifont} % For checkmarks and crosses
\usepackage{booktabs} % For professional tables
\usepackage{hyperref} % For clickable links
\usepackage{url} % For formatting URLs
\usepackage{seqsplit} % For splitting long strings in tt font
\usepackage{graphicx}
\usepackage[table]{xcolor}
\usepackage{fancyhdr}
\usepackage{lastpage}

% Document Metadata and Hyperref Setup
\hypersetup{
    colorlinks=true,
    linkcolor=blue,
    filecolor=magenta,      
    urlcolor=cyan,
    pdftitle={Cybersecurity Assessment Report},
    pdfauthor={Cybersecurity Analyst},
    pdfsubject={Security Assessment},
    pdfkeywords={Security, Nmap, Risk, Assessment},
    bookmarks=true
}

% Header and Footer Configuration
\pagestyle{fancy}
\fancyhf{} % Clear all header and footer fields
\fancyhead[L]{Cybersecurity Assessment Report}
\fancyhead[R]{\textbf{Great Lakes}}
\fancyfoot[C]{Page \thepage\ of \pageref{LastPage}}
\renewcommand{\headrulewidth}{0.4pt}
\renewcommand{\footrulewidth}{0.4pt}

% Define colors for table rows
\definecolor{lightgray}{gray}{0.9}

\begin{document}

% --- Title Page ---
\begin{titlepage}
    \centering
    \vspace*{2cm}
    
    \Huge
    \textbf{Cybersecurity Assessment Report}
    
    \vspace{1.5cm}
    
    \Large
    Prepared for: \\
    \vspace{0.5cm}
    \textbf{Great Lakes}
    
    \vfill
    
    \Large
    \textbf{Date:} \today
    
    \vspace{1cm}
    
    \normalsize
    This report contains a summary of findings from a recent security assessment. It includes an analysis of organizational security controls, technical network scan results, and a consolidated risk assessment with actionable recommendations.
    
\end{titlepage}

\tableofcontents
\newpage

% --- Section 1: Executive Overview ---
\section{Executive Overview}
This report details the findings of a cybersecurity assessment conducted for \textbf{Great Lakes}. The assessment combined a review of organizational security controls, an analysis of pre-existing risks, and a technical network scan to provide a holistic view of the organization's current security posture.

The assessment identified several critical and high-risk issues that require immediate attention. A key technical finding was the discovery of an open Remote Desktop Protocol (RDP) port on an internal server (\texttt{10.10.10.51}). This finding is particularly concerning as it indicates a potential pattern of insecure configuration, corroborating a pre-existing risk identified on another host (\texttt{10.10.10.50}). Exposed RDP is a primary vector for ransomware attacks and unauthorized access.

Furthermore, the security control review revealed significant gaps in foundational security practices. The lack of Multi-Factor Authentication (MFA) for email and sensitive data systems exposes the organization to severe risks of account compromise and data breaches. The absence of an acceptable use policy and mandatory annual security training for all staff further weakens the organization's defense against both internal and external threats.

Collectively, these findings place \textbf{Great Lakes} at a high risk of a significant cybersecurity incident. The recommendations provided in this report are prioritized to address the most critical vulnerabilities first.

% --- Section 2: Organizational Information ---
\section{Organizational Information}
The following details were provided for the assessment.
\begin{itemize}
    \item \textbf{Organization Name:} Great Lakes
    \item \textbf{Email Domain:} \texttt{GreatLakes.net}
    \item \textbf{Website Domain:} \seqsplit{\texttt{www.GreatLakes.net}}
    \item \textbf{External IP Address:} \texttt{192.102.104.246}
\end{itemize}

% --- Section 3: Security Control Review ---
\section{Security Control Review}
An analysis of the organization's security questionnaire responses reveals critical gaps in security controls. "No" answers indicate a deviation from security best practices and are highlighted below.

\begin{table}[h!]
\centering
\caption{Security Questionnaire Analysis}
\label{tab:questionnaire}
\rowcolors{2}{lightgray}{white}
\begin{tabular}{p{0.6\textwidth} c p{0.25\textwidth}}
\toprule
\textbf{Control Question} & \textbf{Response} & \textbf{Assessment} \\
\midrule
Do you require MFA to access email? & \color{red}{\ding{55}} & \textbf{Critical Gap}. Increases risk of Business Email Compromise (BEC). \\
Do you require MFA to log into computers? & \color{green}{\ding{51}} & Good Practice. \\
Do you require MFA to access sensitive data systems? & \color{red}{\ding{55}} & \textbf{High Risk}. Increases risk of data breach. \\
Does your organization have an employee acceptable use policy? & \color{red}{\ding{55}} & \textbf{High Risk}. Lacks clear guidelines for employees. \\
Does your organization do security awareness training for new employees? & \color{green}{\ding{51}} & Good Practice. \\
Does your organization do security awareness training for all employees at least once per year? & \color{red}{\ding{55}} & \textbf{High Risk}. Security knowledge degrades over time. \\
\bottomrule
\end{tabular}
\end{table}

% --- Section 4: Technical Scan Results ---
\section{Technical Scan Results}
A network scan was performed to identify open ports and services on the target system.
\begin{itemize}
    \item \textbf{Scan Target:} \texttt{10.10.10.51}
    \item \textbf{Scan Date:} \today
\end{itemize}

The following table details the open ports discovered on the target host.

\begin{table}[h!]
\centering
\caption{Open Port Analysis for \texttt{10.10.10.51}}
\label{tab:nmap}
\rowcolors{2}{lightgray}{white}
\begin{tabular}{l l l p{0.5\textwidth}}
\toprule
\textbf{Port} & \textbf{State} & \textbf{Service} & \textbf{Analysis} \\
\midrule
3389/tcp & open & ms-wbt-server & \textbf{Critical Finding.} This port is used for Microsoft Remote Desktop Protocol (RDP). Exposing RDP directly to a network without proper controls (e.g., firewall rules, VPN) is extremely dangerous and a common entry point for ransomware attacks. \\
\bottomrule
\end{tabular}
\end{table}

% --- Section 5: Consolidated Risk Assessment ---
\section{Consolidated Risk Assessment}
The following table synthesizes findings from the technical scan, control review, and pre-existing risk data into a prioritized list of security risks.

\begin{table}[h!]
\centering
\caption{Summary of Identified Risks}
\label{tab:risks}
\rowcolors{2}{lightgray}{white}
\begin{tabular}{p{0.2\textwidth} p{0.5\textwidth} p{0.2\textwidth}}
\toprule
\textbf{Risk Name} & \textbf{Description} & \textbf{Severity} \\
\midrule
\textbf{Systemic RDP Exposure} & RDP is exposed on \texttt{10.10.10.51} (new finding) and \texttt{10.10.10.50} (existing risk). This pattern indicates a systemic configuration weakness and a prime target for attackers. & \textbf{Critical (9.8)} \\
\addlinespace
\textbf{Lack of MFA on Critical Systems} & MFA is not enforced for accessing email or sensitive data systems. This allows an attacker with stolen credentials to gain direct access to high-value assets. & \textbf{Critical (9.1)} \\
\addlinespace
\textbf{Deficient Security Policies \& Training} & The lack of an Acceptable Use Policy and mandatory annual security training creates a high-risk environment where employees are more likely to cause an accidental breach or fall victim to social engineering. & \textbf{High (7.5)} \\
\bottomrule
\end{tabular}
\end{table}

% --- Section 6: Recommendations ---
\section{Recommendations}
The following actionable recommendations are prioritized to help \textbf{Great Lakes} mitigate the identified risks and improve its overall security posture.

\subsection{Immediate Priority (Remediate within 72 hours)}
\begin{enumerate}
    \item \textbf{Remediate RDP Exposure:} For hosts \texttt{10.10.10.51} and \texttt{10.10.10.50}, immediately implement one of the following:
    \begin{itemize}
        \item If RDP access is not required, disable the service and block port 3389.
        \item If RDP access is required, place the hosts behind a firewall and restrict access to specific, authorized IP addresses. The long-term solution is to require access via a Virtual Private Network (VPN) with MFA.
    \end{itemize}
    \item \textbf{Enable MFA for Email:} Immediately enforce MFA for all user accounts accessing the \texttt{GreatLakes.net} email system. This is the single most effective control to prevent Business Email Compromise.
\end{enumerate}

\subsection{Short-Term Priority (Remediate within 30 days)}
\begin{enumerate}
    \item \textbf{Implement MFA for Sensitive Data:} Enforce MFA for all applications and systems identified as storing or processing sensitive data.
    \item \textbf{Develop and Implement an Acceptable Use Policy (AUP):} Create a formal AUP that clearly defines the rules for using company IT assets, data, and internet access. Require all employees to read and acknowledge the policy.
\end{enumerate}

\subsection{Long-Term Priority (Implement within 90 days)}
\begin{enumerate}
    \item \textbf{Establish Annual Security Awareness Training:} Implement a mandatory, annual security awareness training program for all employees. The training should cover key topics such as phishing, password security, and the new AUP.
    \item \textbf{Implement a Vulnerability Management Program:} Establish a formal process to regularly scan internal and external assets for vulnerabilities. This will help proactively identify and remediate issues like exposed RDP before they can be exploited.
\end{enumerate}

\end{document}
```