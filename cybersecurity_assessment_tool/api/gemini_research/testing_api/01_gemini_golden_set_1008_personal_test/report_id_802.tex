```latex
\documentclass[12pt]{article}

% Preamble: Required Packages
\usepackage[margin=1in]{geometry}
\usepackage{pifont} % For checkmarks and crosses
\usepackage{booktabs} % For professional tables
\usepackage{hyperref} % For clickable links
\usepackage{url} % For URL formatting
\usepackage{seqsplit} % For splitting long strings in texttt
\usepackage[T1]{fontenc}

% Document Metadata
\title{Cybersecurity Posture Assessment Report}
\author{Cybersecurity Analysis Division}
\date{\today}

\begin{document}

\maketitle
\thispagestyle{empty}
\newpage

\tableofcontents
\thispagestyle{empty}
\newpage

\setcounter{page}{1}

% --- Executive Summary ---
\section{Executive Summary}
This report details the findings of a cybersecurity posture assessment for \textbf{Grizzly Peak}, conducted on \today. The assessment combined a review of organizational security controls, an external network vulnerability scan, and an analysis of pre-existing risks.

The key findings indicate a mixed security posture. While the organization has implemented some foundational controls, such as an acceptable use policy and annual security training, several critical gaps were identified. The most significant risks stem from the lack of Multi-Factor Authentication (MFA) for computer logins and access to sensitive data systems. Additionally, the absence of security awareness training during the new-hire onboarding process creates a significant window of vulnerability.

The external network scan of the target IP address \texttt{[Target IP]} did not identify any open ports, which suggests a potentially well-configured firewall. However, this should be verified to ensure the scan was not inadvertently blocked. No pre-existing vulnerabilities were documented for this assessment.

Immediate remediation should focus on implementing a comprehensive MFA policy and integrating security training into the employee onboarding process.

% --- Organizational Information ---
\section{Organizational Information}
The following information was provided for the assessment.
\begin{itemize}
    \item \textbf{Organization Name:} Grizzly Peak
    \item \textbf{Email Domain:} \texttt{GrizzlyPeak.com}
    \item \textbf{Website Domain:} \seqsplit{\url{www.GrizzlyPeak.com}}
    \item \textbf{Monitored External IP:} \texttt{34.120.51.42}
\end{itemize}

% --- Security Control Review ---
\section{Security Control Review}
A security questionnaire was completed to evaluate the organization's current administrative and technical controls. The responses are summarized below. Gaps identified in this review are addressed in the Risk Assessment section.

\begin{table}[h!]
\centering
\caption{Security Controls Questionnaire Results}
\begin{tabular}{p{0.8\linewidth} c}
\toprule
\textbf{Control Question} & \textbf{Response} \\
\midrule
Do you require MFA to access email? & \ding{51} \\
Do you require MFA to log into computers? & \textbf{\color{red}\ding{55}} \\
Do you require MFA to access sensitive data systems? & \textbf{\color{red}\ding{55}} \\
Does your organization have an employee acceptable use policy? & \ding{51} \\
Does your organization do security awareness training for new employees? & \textbf{\color{red}\ding{55}} \\
Does your organization do security awareness training for all employees at least once per year? & \ding{51} \\
\bottomrule
\end{tabular}
\end{table}

\noindent \textit{Note: \ding{51} indicates "Yes" (control in place), \ding{55} indicates "No" (control gap).}

% --- Technical Scan Results ---
\section{Technical Scan Results}
An external network scan was performed to identify open ports and exposed services on the public-facing infrastructure.

\subsection{Scan Details}
\begin{itemize}
    \item \textbf{Target IP Address:} \texttt{[Target IP]}
    \item \textbf{Scan Date:} 2023-10-27
\end{itemize}

\subsection{Findings}
The network scan against the target IP address \textbf{did not detect any open ports}. This outcome often indicates that a firewall is properly configured to deny unsolicited external traffic, which is a positive security practice. However, it is also possible that the scan was blocked by an Intrusion Prevention System (IPS) or other network security appliance. Further internal verification is recommended to confirm that the absence of open ports is intentional and not the result of a failed or blocked scan.

% --- Risk Assessment ---
\section{Risk Assessment}
This section synthesizes findings from the security control review and technical scan. The following table outlines the identified risks, their potential impact, and an assigned severity level.

\begin{table}[h!]
\centering
\caption{Identified Cybersecurity Risks}
\begin{tabular}{p{0.25\linewidth} p{0.5\linewidth} p{0.15\linewidth}}
\toprule
\textbf{Risk Name} & \textbf{Overview} & \textbf{Severity} \\
\midrule
\textbf{Lack of MFA for Sensitive Data Access} & Sensitive data systems can be accessed with only a username and password. A compromised credential could lead directly to a significant data breach. & \textbf{Critical} \\
\addlinespace
\textbf{Lack of MFA on Endpoints} & Employee computers do not require MFA for login. If an employee's password is stolen, an attacker could gain full access to their machine and potentially pivot to other network resources. & \textbf{High} \\
\addlinespace
\textbf{Inadequate New-Hire Security Training} & New employees are not provided with security awareness training upon joining. This makes them highly susceptible to phishing and social engineering attacks before they are familiar with company security policies. & \textbf{High} \\
\bottomrule
\end{tabular}
\end{table}

\subsection{Pre-existing Risks}
No previously documented vulnerabilities or risks were provided for this assessment.

% --- Recommendations ---
\section{Recommendations}
Based on the risk assessment, the following actions are recommended to improve the cybersecurity posture of \textbf{Grizzly Peak}. Recommendations are prioritized by severity.

\subsection{Critical Priority}
\begin{itemize}
    \item \textbf{Implement MFA for Sensitive Data Systems:} Immediately deploy a mandatory MFA policy for all user accounts (including administrative and service accounts) that have access to systems containing sensitive, confidential, or proprietary data. This is the single most effective control to prevent unauthorized access and data breaches.
\end{itemize}

\subsection{High Priority}
\begin{itemize}
    \item \textbf{Enforce MFA for Endpoint Logins:} Require MFA for all employee computer and laptop logins. This strengthens endpoint security and mitigates the risk of unauthorized access resulting from compromised credentials.
    \item \textbf{Integrate Security Training into Onboarding:} Develop and implement a mandatory security awareness training module as part of the new-hire onboarding process. This training should occur within the first week of employment and cover topics such as phishing, password security, and the acceptable use policy.
\end{itemize}

\end{document}
```