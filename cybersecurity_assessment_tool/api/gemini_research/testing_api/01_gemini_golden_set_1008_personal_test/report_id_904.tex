```latex
\documentclass[12pt]{article}

% Required Packages
\usepackage[margin=1in]{geometry} % For setting page margins
\usepackage{pifont}               % For checkmark and X symbols (\ding)
\usepackage{booktabs}             % For professional-looking tables
\usepackage{hyperref}             % For clickable links, TOC, etc.
\usepackage{url}                  % For formatting URLs
\usepackage{seqsplit}             % For splitting long strings in \texttt

% Hyperref Setup
\hypersetup{
    colorlinks=true,
    linkcolor=black,
    filecolor=magenta,      
    urlcolor=blue,
    pdftitle={Cybersecurity Posture Report},
    pdfpagemode=FullScreen,
}

% Document Title and Author Information
\title{Cybersecurity Posture Report \\ \large Prepared for: Symmetry Architecture}
\author{Cybersecurity Analysis Division}
\date{\today}

\begin{document}

\maketitle

\begin{abstract}
    This report provides a comprehensive analysis of the cybersecurity posture for Symmetry Architecture. The assessment is based on a correlation of organizational data, a review of security controls, and a technical network scan. The findings indicate a mixed security posture with strong network hygiene on the scanned asset but critical gaps in identity and access management and employee security training. This document details the identified risks and provides actionable recommendations to mitigate them and enhance the organization's overall security resilience.
\end{abstract}

\tableofcontents
\newpage

% ===================================================================
% SECTION 1: OVERVIEW
% ===================================================================
\section{Executive Overview}
The assessment reveals a security posture with notable strengths and significant weaknesses. On the positive side, the technical scan of the target host \texttt{192.168.1.100} found no open ports, indicating a securely configured device from a network perspective. The organization also has some mature security controls, such as requiring Multi-Factor Authentication (MFA) for computer and sensitive system access, and conducting annual security training for all staff.

However, two critical gaps were identified that present a high level of risk:
\begin{itemize}
    \item \textbf{Lack of MFA for Email:} The absence of MFA on email accounts is a critical vulnerability. Email is a primary target for attackers, and a compromised account can lead to Business Email Compromise (BEC), data breaches, and further network intrusion.
    \item \textbf{No Security Training for New Employees:} Failing to train new hires on security best practices from day one leaves the organization vulnerable. New employees are often targeted by phishing and social engineering attacks and may be unaware of internal security policies.
\end{itemize}

This report provides detailed findings and prioritizes recommendations to address these gaps, starting with the immediate implementation of MFA for all email accounts.

% ===================================================================
% SECTION 2: ORGANIZATIONAL INFORMATION
% ===================================================================
\section{Organizational Information}
The following information was provided for the assessment.

\begin{table}[h!]
\centering
\begin{tabular}{@{}ll@{}}
\toprule
\textbf{Attribute}        & \textbf{Value}                     \\ \midrule
Organization Name         & Symmetry Architecture              \\
Email Domain              & \texttt{SymmetryArchitecture.net}  \\
Website Domain            & \url{www.SymmetryArchitecture.net} \\
External IP Address       & \texttt{108.238.210.88}            \\ \bottomrule
\end{tabular}
\caption{Client Organizational Data.}
\label{tab:org_data}
\end{table}

% ===================================================================
% SECTION 3: SECURITY CONTROL REVIEW
% ===================================================================
\section{Security Control Review}
A review of the organization's security controls was conducted based on a standardized questionnaire. The results are summarized in Table \ref{tab:controls}. Answers marked with \ding{55} (No) indicate a potential security gap that requires attention.

\begin{table}[h!]
\centering
\begin{tabular}{@{}lc@{}}
\toprule
\textbf{Security Control Question} & \textbf{Status} \\ \midrule
Do you require MFA to access email? & \ding{55} \\
Do you require MFA to log into computers? & \ding{51} \\
Do you require MFA to access sensitive data systems? & \ding{51} \\
Does your organization have an employee acceptable use policy? & \ding{51} \\
Does your organization do security awareness training for new employees? & \ding{55} \\
Does your organization do security awareness training for all employees at least once per year? & \ding{51} \\ \bottomrule
\end{tabular}
\caption{Security Controls Questionnaire Results (\ding{51}=Yes, \ding{55}=No).}
\label{tab:controls}
\end{table}

\subsection{Analysis of Gaps}
The review identified two primary gaps in the current security control framework:
\begin{itemize}
    \item \textbf{Email MFA:} The lack of enforced MFA on email is the most critical finding from this review. This control is a foundational defense against account takeover attacks.
    \item \textbf{New Hire Training:} The absence of a mandatory security training module during employee onboarding creates an immediate risk. New staff are not formally equipped to recognize or respond to common threats like phishing.
\end{itemize}

% ===================================================================
% SECTION 4: TECHNICAL SCAN RESULTS
% ===================================================================
\section{Technical Scan Results}
A network scan was performed on the specified target to identify open ports and exposed services.

\begin{table}[h!]
\centering
\begin{tabular}{@{}ll@{}}
\toprule
\textbf{Scan Parameter} & \textbf{Value} \\ \midrule
Target IP Address & \texttt{192.168.1.100} \\
Scan Date & \today \\
Host Status & Up \\
Open Ports & None Detected \\
Filtered/Closed Ports & All scanned ports were found to be in a 'closed' state. \\ \bottomrule
\end{tabular}
\caption{Nmap Scan Summary.}
\label{tab:scan_results}
\end{table}

\subsection{Technical Findings}
The scan of host \texttt{192.168.1.100} revealed an excellent security posture from a network perspective. No open ports were discovered, which significantly reduces the attack surface of this device. This indicates proper firewall configuration and network segmentation. No vulnerabilities could be identified from this external scan.

% ===================================================================
% SECTION 5: IDENTIFIED RISKS & ASSESSMENT
% ===================================================================
\section{Identified Risks and Assessment}
This section synthesizes the findings from the security control review and technical scan to provide a consolidated list of identified risks. No pre-existing vulnerabilities were reported.

\begin{table}[h!]
\centering
\begin{tabular}{@{}p{0.1\linewidth} p{0.3\linewidth} p{0.4\linewidth} p{0.1\linewidth}@{}}
\toprule
\textbf{Risk ID} & \textbf{Risk Name} & \textbf{Overview} & \textbf{Severity} \\ \midrule
RISK-001 & Lack of MFA on Email & The absence of Multi-Factor Authentication on email accounts exposes the organization to a high likelihood of account compromise, leading to data breaches, financial fraud (BEC), and phishing campaigns launched from a trusted internal source. & \textbf{Critical} \\
\addlinespace
RISK-002 & No Security Training for New Employees & New hires are not provided with security awareness training during onboarding. This makes them highly susceptible to social engineering and phishing attacks, turning them into an unintentional insider threat from their first day. & \textbf{High} \\ \bottomrule
\end{tabular}
\caption{Consolidated Risk Register.}
\label{tab:risk_register}
\end{table}

% ===================================================================
% SECTION 6: RECOMMENDATIONS
% ===================================================================
\section{Recommendations}
The following actionable recommendations are provided to mitigate the identified risks and improve the overall security posture of Symmetry Architecture.

\subsection{RISK-001: Lack of MFA on Email (Critical)}
\begin{enumerate}
    \item \textbf{Immediate Action:} Procure and implement an MFA solution for the organization's email platform. All major providers (Microsoft 365, Google Workspace) have robust, built-in MFA capabilities.
    \item \textbf{Enforcement:} Enforce MFA for all user accounts without exception, including service accounts where technically feasible.
    \item \textbf{User Training:} Provide clear instructions and a brief training session to all employees on how to set up and use MFA to ensure a smooth rollout.
\end{enumerate}

\subsection{RISK-002: No Security Training for New Employees (High)}
\begin{enumerate}
    \item \textbf{Develop Onboarding Module:} Create a mandatory security awareness training module that is integrated into the new employee onboarding process.
    \item \textbf{Core Topics:} This training must cover, at a minimum:
    \begin{itemize}
        \item Phishing and spear-phishing recognition.
        \item The organization's acceptable use policy.
        \item Password security and hygiene.
        \item Procedures for reporting a suspected security incident.
    \end{itemize}
    \item \textbf{Acknowledgement:} Require new employees to formally acknowledge that they have completed the training and understand the policies.
\end{enumerate}

\end{document}
```