```latex
\documentclass[12pt]{article}

% Preamble: Required Packages
\usepackage[margin=1in]{geometry}
\usepackage{pifont} % For checkmarks and crosses
\usepackage{booktabs} % For professional tables
\usepackage[colorlinks=true, urlcolor=blue, linkcolor=black]{hyperref}
\usepackage{url}
\usepackage{seqsplit} % To split long strings in texttt
\usepackage{graphicx}
\usepackage{fancyhdr}
\usepackage{xcolor}

% --- Document Metadata ---
\title{Cybersecurity Posture Assessment Report}
\author{Cybersecurity Analysis Division}
\date{\today}

% --- Header and Footer ---
\pagestyle{fancy}
\fancyhf{}
\lhead{\textbf{Neon Pulse Entertainment} \\ Cybersecurity Report}
\rhead{\today}
\cfoot{\thepage}
\renewcommand{\headrulewidth}{0.4pt}
\renewcommand{\footrulewidth}{0.4pt}

\begin{document}

\maketitle
\thispagestyle{empty}
\newpage

\tableofcontents
\newpage

% --- Section 1: Executive Summary ---
\section{Executive Summary}

This report details the findings of a cybersecurity posture assessment conducted for \textbf{Neon Pulse Entertainment}. The assessment combined a review of organizational security controls, an external network scan, and an analysis of pre-existing risk data to provide a holistic view of the organization's security posture.

The primary findings indicate significant weaknesses in foundational administrative and access controls. The widespread lack of Multi-Factor Authentication (MFA) across email, computers, and sensitive data systems represents a \textbf{critical risk} of unauthorized access and potential account compromise. Furthermore, the absence of an employee acceptable use policy and mandatory annual security training for all staff creates an environment susceptible to insider threats and social engineering attacks.

A technical network scan was performed on the target host \texttt{192.168.0.5}. The scan did not identify any open ports, suggesting a limited direct attack surface on this specific system. Notably, this finding conflicts with a pre-existing documented risk concerning an "Unencrypted Web Server" on Port 80. This discrepancy suggests the risk may have been remediated or applies to a different asset not included in the scope of this scan.

Recommendations prioritize the immediate implementation of MFA, the development of key security policies, and the expansion of the security awareness training program to mitigate the most severe risks identified.

% --- Section 2: Organizational Information ---
\section{Organizational Information}

The following information was provided for the assessment.

\begin{table}[h!]
\centering
\begin{tabular}{@{}ll@{}}
\toprule
\textbf{Attribute} & \textbf{Value} \\ \midrule
Organization Name & \textbf{Neon Pulse Entertainment} \\
Email Domain & \texttt{NeonPulseEntertainment.net} \\
Website Domain & \seqsplit{\texttt{www.NeonPulseEntertainment.net}} \\
External IP Address & \texttt{218.190.4.25} \\
Scanned Target IP & \texttt{192.168.0.5} \\
\bottomrule
\end{tabular}
\caption{Client Organizational Details.}
\label{tab:org_info}
\end{table}

% --- Section 3: Security Control Review ---
\section{Security Control Review}

A security questionnaire was completed to evaluate the implementation of key administrative and technical controls. The results are summarized below. Answers marked with \ding{55} represent significant gaps in the security framework.

\begin{table}[h!]
\centering
\begin{tabular}{@{}p{0.6\textwidth}cc@{}}
\toprule
\textbf{Control Question} & \textbf{Response} & \textbf{Status} \\ \midrule
Do you require MFA to access email? & No & \ding{55} \\
Do you require MFA to log into computers? & No & \ding{55} \\
Do you require MFA to access sensitive data systems? & No & \ding{55} \\
Does your organization have an employee acceptable use policy? & No & \ding{55} \\
Does your organization do security awareness training for new employees? & Yes & \ding{51} \\
Does your organization do security awareness training for all employees at least once per year? & No & \ding{55} \\
\bottomrule
\end{tabular}
\caption{Security Controls Questionnaire Results.}
\label{tab:controls}
\end{table}

\subsection*{Analysis of Control Gaps}
The questionnaire reveals critical deficiencies in access control and security governance. The lack of MFA across all major platforms (email, endpoints, data systems) is a severe vulnerability. A single compromised password could grant an attacker broad access to the organization's digital assets. Additionally, the absence of an acceptable use policy and mandatory annual training for all staff heightens the risk of human error, policy violations, and successful phishing attacks.

% --- Section 4: Technical Scan Results ---
\section{Technical Scan Results}
A network scan was conducted on the specified target to identify open ports and exposed services.

\begin{itemize}
    \item \textbf{Target IP Address:} \texttt{192.168.0.5}
    \item \textbf{Scan Status:} Host is up.
\end{itemize}

\begin{table}[h!]
\centering
\begin{tabular}{@{}llll@{}}
\toprule
\textbf{Port} & \textbf{State} & \textbf{Service} & \textbf{Product / Version} \\ \midrule
80 & closed & http & Not Detected \\
\bottomrule
\end{tabular}
\caption{Nmap Scan Results for Target \texttt{192.168.0.5}.}
\label{tab:nmap_results}
\end{table}

\subsection*{Analysis of Technical Findings}
The scan of the target host \texttt{192.168.0.5} indicates a minimal attack surface, as no open ports were detected. The previously documented risk of an open Port 80 was not confirmed on this host at the time of the scan, suggesting it may have been closed or misidentified. While this specific host appears secure from a network perspective, this result does not preclude vulnerabilities on other network assets.

% --- Section 5: Consolidated Risk Assessment ---
\section{Consolidated Risk Assessment}
The following table synthesizes risks identified from the security control review, technical scan, and pre-existing risk data.

\begin{table}[h!]
\centering
\begin{tabular}{@{}p{0.25\textwidth}p{0.4\textwidth}ll@{}}
\toprule
\textbf{Risk Name} & \textbf{Description} & \textbf{Severity} & \textbf{Affected Elements} \\ \midrule
\textbf{Lack of Multi-Factor Authentication (MFA)} & No MFA is enforced for email, computer, or sensitive data access, exposing the organization to account takeover. & \textbf{Critical} & All Users, Data Systems, Email Infrastructure \\
\addlinespace
\textbf{Inadequate Security Governance} & The absence of an Acceptable Use Policy and mandatory annual training weakens the human firewall. & High & All Employees, Organizational Policy \\
\addlinespace
\textbf{Unencrypted Web Server (Unconfirmed)} & A pre-existing risk notes that Port 80 is open. This was not validated by the current scan on the target host. & Medium & Port 80 \\
\bottomrule
\end{tabular}
\caption{Summary of Identified Risks.}
\label{tab:risks}
\end{table}

% --- Section 6: Recommendations ---
\section{Recommendations}
Based on the consolidated risk assessment, the following actions are recommended to improve the cybersecurity posture of \textbf{Neon Pulse Entertainment}.

\subsection{Critical Priority: Implement Multi-Factor Authentication}
The most urgent action is to deploy MFA across the organization to mitigate the risk of unauthorized access.
\begin{itemize}
    \item \textbf{Phase 1 (Immediate):} Enforce MFA on all systems containing sensitive data and for all administrative accounts.
    \item \textbf{Phase 2 (Short-Term):} Roll out MFA for the corporate email system (\texttt{NeonPulseEntertainment.net}).
    \item \textbf{Phase 3 (Medium-Term):} Implement MFA for all employee computer logins.
\end{itemize}

\subsection{High Priority: Develop and Implement Security Policies}
Formalize security expectations and responsibilities by creating foundational governance documents.
\begin{itemize}
    \item \textbf{Action:} Draft and ratify an \textbf{Employee Acceptable Use Policy}. This policy should clearly define the rules for using company assets, data handling, and internet usage.
    \item \textbf{Action:} Require all employees to read and formally acknowledge the new policy.
\end{itemize}

\subsection{High Priority: Enhance Security Awareness Training}
Strengthen the organization's defense against social engineering and phishing.
\begin{itemize}
    \item \textbf{Action:} Institute a mandatory, annual security awareness training program for all employees, not just new hires.
    \item \textbf{Action:} Consider periodic phishing simulation campaigns to test and reinforce the training concepts.
\end{itemize}

\subsection{Informational: Review and Validate Risk Register}
Ensure the organization's risk register is accurate and up-to-date.
\begin{itemize}
    \item \textbf{Action:} Investigate the "Unencrypted Web Server" risk. Confirm if the risk applies to a different asset or if it has been successfully remediated. Update the risk register accordingly.
\end{itemize}

\end{document}
```