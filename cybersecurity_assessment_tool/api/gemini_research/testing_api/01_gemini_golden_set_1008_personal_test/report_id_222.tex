```latex
\documentclass[12pt]{article}

% --- PACKAGES ---
\usepackage[margin=1in]{geometry}
\usepackage{pifont} % For checkmarks and crosses
\usepackage{booktabs} % For professional tables
\usepackage{hyperref} % For hyperlinks
\usepackage{url} % For URL formatting
\usepackage{seqsplit} % To split long strings in texttt
\usepackage{graphicx}
\usepackage{fancyhdr}
\usepackage{xcolor}
\usepackage{tocloft}

% --- DOCUMENT SETUP ---
\definecolor{darkblue}{rgb}{0.0, 0.0, 0.55}
\definecolor{darkred}{rgb}{0.55, 0.0, 0.0}

\hypersetup{
    colorlinks=true,
    linkcolor=darkblue,
    filecolor=darkblue,      
    urlcolor=darkblue,
    citecolor=darkblue,
}

\pagestyle{fancy}
\fancyhf{}
\lhead{Cybersecurity Assessment Report}
\rhead{\textbf{Structure \& Form}}
\cfoot{\thepage}
\renewcommand{\headrulewidth}{0.4pt}
\renewcommand{\footrulewidth}{0.4pt}

% --- TITLE ---
\title{Cybersecurity Posture Assessment Report \\ \large For: \textbf{Structure \& Form}}
\author{Cybersecurity Analysis Division}
\date{\today}

\begin{document}

\maketitle
\thispagestyle{empty}
\newpage

\tableofcontents
\newpage

% --- SECTION 1: EXECUTIVE SUMMARY ---
\section{Executive Summary}
This report provides a comprehensive analysis of the cybersecurity posture for \textbf{Structure \& Form}, based on a synthesis of network scan data, a security controls questionnaire, and a review of pre-existing risks.

The assessment reveals significant gaps in administrative and identity management controls, which present a high level of risk to the organization. Key findings include the absence of Multi-Factor Authentication (MFA) for computer and sensitive data system access, the lack of a formal Acceptable Use Policy, and the failure to conduct annual security awareness training for all employees. These deficiencies substantially increase the risk of unauthorized access, data breaches, and insider threats.

On the technical front, a network scan of the target host \texttt{192.168.0.5} showed a minimal attack surface, with the commonly targeted Port 80 found to be closed. This finding contradicts a pre-existing risk entry that stated the port was open. This suggests either a successful remediation effort or an initial misidentification, which requires validation.

Immediate action should be focused on implementing foundational security controls, particularly MFA and employee security policies, to mitigate the most critical risks identified.

% --- SECTION 2: ORGANIZATIONAL INFORMATION ---
\section{Organizational Information}
The following details were provided for the assessment. This information is used to establish the context and scope of the review.

\begin{tabular}{@{}ll}
\toprule
\textbf{Attribute} & \textbf{Value} \\
\midrule
Organization Name & \textbf{Structure \& Form} \\
Email Domain & \texttt{StructureForm.com} \\
Website Domain & \url{www.StructureForm.com} \\
External IP Address & \texttt{34.209.123.14} \\
\bottomrule
\end{tabular}

% --- SECTION 3: SECURITY CONTROL REVIEW ---
\section{Security Control Review}
The following table summarizes the organization's responses to a security controls questionnaire. The assessment column highlights areas that deviate from established best practices and represent significant security gaps.

\begin{tabular}{p{0.6\textwidth} c l}
\toprule
\textbf{Control Question} & \textbf{Response} & \textbf{Assessment} \\
\midrule
Do you require MFA to access email? & \ding{51} & Aligned with best practice. \\
Do you require MFA to log into computers? & \color{darkred}{\ding{55}} & \color{darkred}{\textbf{High Risk Gap}} \\
Do you require MFA to access sensitive data systems? & \color{darkred}{\ding{55}} & \color{darkred}{\textbf{Critical Risk Gap}} \\
Does your organization have an employee acceptable use policy? & \color{darkred}{\ding{55}} & \color{darkred}{\textbf{High Risk Gap}} \\
Does your organization do security awareness training for new employees? & \ding{51} & Aligned with best practice. \\
Does your organization do security awareness training for all employees at least once per year? & \color{darkred}{\ding{55}} & \color{darkred}{\textbf{High Risk Gap}} \\
\bottomrule
\end{tabular}

% --- SECTION 4: TECHNICAL SCAN RESULTS ---
\section{Technical Scan Results}
An external network scan was performed to identify open ports and exposed services on the specified target system.

\subsection{Scan Details}
\begin{tabular}{@{}ll}
\toprule
\textbf{Parameter} & \textbf{Value} \\
\midrule
Target IP Address & \texttt{192.168.0.5} \\
Scan Date & \textbf{[Scan Date Not Provided]} \\
\bottomrule
\end{tabular}

\subsection{Port Scan Findings}
The scan revealed a minimal external footprint for the target host.
\begin{tabular}{@{}llll}
\toprule
\textbf{Port} & \textbf{State} & \textbf{Service} & \textbf{Product / Version} \\
\midrule
80/tcp & closed & http & N/A \\
\bottomrule
\end{tabular}

\subsection{Analysis}
The technical scan indicates a well-hardened external posture for the assessed host, with no open ports detected. Notably, the finding that Port 80 is closed directly contradicts a pre-existing risk titled "Unencrypted Web Server." This discrepancy suggests that the risk may have been successfully remediated. It is recommended that the risk register be updated to reflect this new information, pending final validation.

% --- SECTION 5: CONSOLIDATED RISK ASSESSMENT ---
\section{Consolidated Risk Assessment}
This section synthesizes findings from the security questionnaire, technical scan, and pre-existing risk data into a consolidated list of current risks.

\begin{tabular}{p{0.25\textwidth} p{0.45\textwidth} p{0.2\textwidth}}
\toprule
\textbf{Risk Name} & \textbf{Description} & \textbf{Severity} \\
\midrule
\textbf{Lack of Foundational MFA} & MFA is not enforced for computer logins or access to sensitive data systems. This dramatically increases the likelihood of a successful breach via compromised credentials. & \textbf{\color{darkred}{Critical}} \\
\addlinespace
\textbf{Inadequate Security Governance} & The organization lacks an Acceptable Use Policy and does not conduct annual security training for all staff. This creates an environment prone to human error and policy violations. & \textbf{\color{darkred}{High}} \\
\addlinespace
\textbf{Unencrypted Web Server (Contradicted)} & A pre-existing risk stated Port 80 was open. The current scan found it closed on the target host. This risk appears to be remediated or was invalid. & \textbf{Informational} \\
\bottomrule
\end{tabular}

% --- SECTION 6: RECOMMENDATIONS ---
\section{Recommendations}
The following actions are recommended to address the identified risks and improve the overall security posture of \textbf{Structure \& Form}.

\subsection{Immediate Priority (Critical)}
\begin{itemize}
    \item \textbf{Implement MFA for Critical Systems:} Immediately deploy Multi-Factor Authentication (MFA) for all user logins to computers and for any application or system that stores or processes sensitive organizational data. This is the single most effective control to mitigate unauthorized access.
\end{itemize}

\subsection{High Priority}
\begin{itemize}
    \item \textbf{Develop and Enforce an Acceptable Use Policy (AUP):} Create a formal AUP that clearly defines the rules and expectations for employees when using company technology and data. Require all employees to read and acknowledge the policy.
    \item \textbf{Establish Annual Security Awareness Training:} Institute a mandatory, annual security awareness training program for all employees. The training should cover current threats such as phishing, social engineering, and proper data handling.
\end{itemize}

\subsection{Administrative}
\begin{itemize}
    \item \textbf{Validate and Update Risk Register:} Formally confirm that the "Unencrypted Web Server" risk has been remediated across all relevant systems and update the internal risk register to reflect its closed status.
\end{itemize}

\end{document}
```