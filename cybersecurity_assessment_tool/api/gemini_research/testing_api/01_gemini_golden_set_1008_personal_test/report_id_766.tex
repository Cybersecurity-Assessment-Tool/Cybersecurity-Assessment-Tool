```latex
\documentclass[12pt]{article}

% Preamble: Required Packages
\usepackage[margin=1in]{geometry} % For setting page margins
\usepackage{pifont}                % For using dingbats like checkmarks and crosses
\usepackage{booktabs}              % For professional-looking tables
\usepackage{hyperref}              % For clickable links and references
\usepackage{url}                   % For formatting URLs
\usepackage{seqsplit}              % For splitting long strings without spaces

% Document Metadata
\hypersetup{
    colorlinks=true,
    linkcolor=black,
    urlcolor=blue,
    pdftitle={Cybersecurity Posture Report},
    pdfauthor={Cybersecurity Analyst},
    pdfsubject={Security Assessment},
    pdfkeywords={Cybersecurity, Risk, Assessment, Scan}
}

\begin{document}

% --- Title Page ---
\begin{titlepage}
    \centering
    \vspace*{\stretch{1.0}}
    \Huge\textbf{Cybersecurity Posture Report}
    \vspace{0.5cm}
    \LARGE For
    \vspace{0.5cm}
    \LARGE\textbf{Silent Spring}
    \vspace{2cm}
    \large Report Date: \today
    \vspace*{\stretch{2.0}}
    \normalsize This report is confidential and intended solely for the use of Silent Spring.
\end{titlepage}

\tableofcontents
\newpage

% --- Executive Summary ---
\section*{Executive Summary}
This report provides a comprehensive analysis of the cybersecurity posture for \textbf{Silent Spring}, based on a review of organizational security controls, an external network scan, and pre-existing risk data.

The assessment reveals a mixed security posture. The organization has implemented several positive controls, including Multi-Factor Authentication (MFA) for email and sensitive data systems, alongside a security awareness training program. These are commendable foundational elements.

However, two critical gaps were identified that present a high level of risk. The absence of mandatory MFA for logging into employee computers significantly weakens endpoint security, making the organization vulnerable to credential theft. Furthermore, the lack of a formal Employee Acceptable Use Policy (AUP) creates ambiguity regarding security responsibilities and data handling, increasing the potential for insider threats and accidental data breaches.

The external network scan of the target IP address, \texttt{[Target IP]}, did not identify any open ports. While this may indicate a securely configured firewall, it could also be the result of a scan being blocked. No previously documented vulnerabilities were provided for this assessment.

Immediate action is recommended to address the identified high-risk gaps to fortify the organization's defenses against common cyber threats.

% --- Organizational Information ---
\section*{Organizational Information}
The following details were provided for the assessment.

\begin{tabular}{@{}ll}
    \toprule
    \textbf{Attribute} & \textbf{Value} \\
    \midrule
    Organization Name & Silent Spring \\
    Email Domain      & \seqsplit{\texttt{SilentSpring.com}} \\
    Website Domain    & \seqsplit{\texttt{www.SilentSpring.com}} \\
    External IP       & \seqsplit{\texttt{209.162.87.243}} \\
    \bottomrule
\end{tabular}

% --- Security Control Review ---
\section*{Security Control Review}
A review of administrative and technical security controls was conducted via a questionnaire. The results are summarized below. Items marked with \ding{55} represent significant gaps in the current security framework.

\begin{tabular}{@{}p{0.8\linewidth}c}
    \toprule
    \textbf{Control Question} & \textbf{Status} \\
    \midrule
    Do you require MFA to access email? & \ding{51} \\
    \textbf{Do you require MFA to log into computers?} & \textbf{\color{red}\ding{55}} \\
    Do you require MFA to access sensitive data systems? & \ding{51} \\
    \textbf{Does your organization have an employee acceptable use policy?} & \textbf{\color{red}\ding{55}} \\
    Does your organization do security awareness training for new employees? & \ding{51} \\
    Does your organization do security awareness training for all employees at least once per year? & \ding{51} \\
    \bottomrule
\end{tabular}

% --- Technical Scan Results ---
\section*{Technical Scan Results}
An external network vulnerability scan was performed to identify potential exposures on the organization's perimeter.

\begin{itemize}
    \item \textbf{Scan Target:} \texttt{[Target IP]}
    \item \textbf{Scan Date:} \today
    \item \textbf{Findings:} The scan completed successfully but did not identify any open TCP or UDP ports on the target system. This indicates that the host is likely protected by a well-configured firewall that drops or rejects unsolicited incoming traffic. No vulnerabilities were discovered.
\end{itemize}

% --- Risk Assessment ---
\section*{Risk Assessment}
This section synthesizes findings from the security control review and technical scan. No pre-existing risks were provided for this assessment. The following new risks have been identified.

\begin{tabular}{@{}p{0.25\linewidth}p{0.55\linewidth}l}
    \toprule
    \textbf{Risk Name} & \textbf{Overview} & \textbf{Severity} \\
    \midrule
    \textbf{Lack of Endpoint MFA} & The absence of MFA on employee computers means that a single compromised password could grant an attacker full access to a user's workstation and any connected network resources. This is a common vector for ransomware and data exfiltration attacks. & \textbf{High} \\
    \addlinespace
    \textbf{No Acceptable Use Policy (AUP)} & Without a formal AUP, employees lack clear guidance on the secure and appropriate use of company technology and data. This increases the risk of unintentional data exposure, policy violations, and insider threats, while also creating legal and compliance challenges. & \textbf{High} \\
    \bottomrule
\end{tabular}

% --- Recommendations ---
\section*{Recommendations}
Based on the risk assessment, the following actions are recommended to mitigate the identified vulnerabilities and improve the overall security posture of \textbf{Silent Spring}.

\begin{enumerate}
    \item \textbf{Implement Mandatory MFA for All Endpoints:}
    \begin{itemize}
        \item \textbf{Action:} Deploy and enforce a robust Multi-Factor Authentication solution (e.g., authenticator app, hardware token, or biometrics) for all user logins to company-managed workstations and laptops.
        \item \textbf{Priority:} Critical. This should be the top priority to close a significant security gap.
        \item \textbf{Impact:} Significantly reduces the risk of unauthorized access resulting from compromised credentials.
    \end{itemize}
    \vspace{0.5cm}
    \item \textbf{Develop and Implement an Acceptable Use Policy (AUP):}
    \begin{itemize}
        \item \textbf{Action:} Draft a comprehensive AUP that clearly defines rules for employee use of company networks, devices, software, and data. The policy should be reviewed by legal counsel, communicated to all staff, and formally acknowledged via signature.
        \item \textbf{Priority:} High. This foundational policy is essential for establishing a culture of security.
        \item \textbf{Impact:} Establishes clear security expectations for all employees, reduces legal liability, and helps prevent both accidental and malicious insider incidents.
    \end{itemize}
\end{enumerate}

\end{document}
```