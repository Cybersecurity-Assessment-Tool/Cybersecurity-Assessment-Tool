```latex
\documentclass[12pt]{article}

% 1. Required Packages
\usepackage[margin=1in]{geometry}
\usepackage{pifont} % For checkmarks and crosses
\usepackage{booktabs} % For professional tables
\usepackage{hyperref} % For clickable links
\usepackage{url} % For URL formatting
\usepackage{seqsplit} % For splitting long strings in tt font
\usepackage{xcolor} % For colors
\usepackage{graphicx} % For logo (placeholder)

% 2. Document Metadata & Title
\title{Cybersecurity Assessment Report \\ \large For: \textbf{Maple Leaf Logistics}}
\author{Cybersecurity Analysis Division}
\date{\today}

% Hyperref Setup
\hypersetup{
    colorlinks=true,
    linkcolor=blue,
    filecolor=magenta,      
    urlcolor=cyan,
    pdftitle={Cybersecurity Assessment Report},
    pdfpagemode=FullScreen,
}

\begin{document}

\maketitle
\thispagestyle{empty}
\newpage

\tableofcontents
\newpage

% 3. Executive Summary
\section{Executive Summary}

This report details the findings of a cybersecurity assessment conducted for \textbf{Maple Leaf Logistics}. The assessment combined a review of organizational security controls, an automated network scan of internal assets, and a correlation with pre-existing risk documentation.

The analysis revealed several high-impact security gaps that require immediate attention. Key findings include:
\begin{itemize}
    \item \textbf{Critical Gaps in Access Control:} Multi-Factor Authentication (MFA) is not enforced for employee email or computer logins. This exposes the organization to significant risks, including business email compromise, ransomware, and unauthorized data access.
    \item \textbf{Significant Information Disclosure:} An internal network scan of target \texttt{10.5.5.5} discovered an open service on port 8080 with the title \textbf{"TOP SECRET DB"}. This suggests a high-value data source is exposed. This finding directly contradicts a pre-existing risk assessment entry which incorrectly labeled this port as secure.
    \item \textbf{Insufficient Security Training:} The organization does not provide mandatory, annual security awareness training for all employees, increasing susceptibility to phishing and other social engineering attacks.
\end{itemize}

Urgent remediation of these issues is recommended to reduce the organization's risk exposure and strengthen its overall security posture. Detailed recommendations are provided in Section \ref{sec:recommendations}.

% 4. Organizational Information
\section{Organizational Information}

The following details were provided for the assessment.

\begin{tabular}{@{}ll}
    \toprule
    \textbf{Attribute} & \textbf{Value} \\
    \midrule
    Organization Name & \textbf{Maple Leaf Logistics} \\
    Email Domain & \seqsplit{\texttt{MapleLeafLogistics.org}} \\
    Website Domain & \seqsplit{\url{www.MapleLeafLogistics.org}} \\
    External IP Address & \seqsplit{\texttt{44.65.75.10}} \\
    \bottomrule
\end{tabular}

% 5. Security Control Review (from Questionnaire)
\section{Security Control Review}

A review of the organization's security controls was conducted based on a standardized questionnaire. The results highlight critical gaps in identity and access management and employee training protocols.

\begin{table}[h!]
\centering
\caption{Security Control Questionnaire Analysis}
\begin{tabular}{@{}p{8cm}cc@{}}
    \toprule
    \textbf{Control Question} & \textbf{Response} & \textbf{Assessment} \\
    \midrule
    Do you require MFA to access email? & \ding{55} No & \textcolor{red}{\textbf{Critical Gap}} \\
    Do you require MFA to log into computers? & \ding{55} No & \textcolor{red}{\textbf{Critical Gap}} \\
    Do you require MFA to access sensitive data systems? & \ding{51} Yes & Control in Place \\
    Does your organization have an employee acceptable use policy? & \ding{51} Yes & Control in Place \\
    Does your organization do security awareness training for new employees? & \ding{51} Yes & Control in Place \\
    Does your organization do security awareness training for all employees at least once per year? & \ding{55} No & \textcolor{orange}{\textbf{High Risk}} \\
    \bottomrule
\end{tabular}
\end{table}

% 6. Technical Scan Results
\section{Technical Scan Results}

An Nmap scan was performed on the specified internal target to identify open ports and exposed services.

\begin{itemize}
    \item \textbf{Target IP Address:} \texttt{10.5.5.5}
    \item \textbf{Scan Date:} Assumed \today
\end{itemize}

The scan identified one open port with a highly sensitive service banner, indicating a potential database or administrative interface.

\begin{table}[h!]
\centering
\caption{Open Port Analysis for \texttt{10.5.5.5}}
\begin{tabular}{@{}llll@{}}
    \toprule
    \textbf{Port} & \textbf{State} & \textbf{Service/Banner} & \textbf{Analysis} \\
    \midrule
    8080/tcp & Open & HTTP Title: \textbf{TOP SECRET DB} & \textcolor{red}{\textbf{Critical Finding}} \\
    \bottomrule
\end{tabular}
\end{table}

\paragraph{Analysis Note:} The title "TOP SECRET DB" is a severe information disclosure. It strongly suggests that a sensitive, possibly unauthenticated, database interface is accessible on the network. This finding \textbf{invalidates} the pre-existing risk assessment entry (from Input 3) that claimed this port was a secure false positive. An attacker on the internal network could target this service to exfiltrate or manipulate critical data.

% 7. Consolidated Risk Assessment
\section{Consolidated Risk Assessment}

The following table synthesizes findings from the security control review, technical scan, and pre-existing risk data. New findings have been prioritized based on their potential impact on the organization.

\begin{table}[h!]
\centering
\caption{Summary of Identified Risks}
\begin{tabular}{@{}p{2.5cm}p{3.5cm}p{7cm}@{}}
    \toprule
    \textbf{Severity} & \textbf{Risk Name} & \textbf{Description} \\
    \midrule
    \textcolor{red}{\textbf{Critical}} & Lack of MFA on Email and Endpoints & The absence of MFA on primary access vectors like email and computers creates a high likelihood of account compromise, leading to data breaches or ransomware. \\
    \addlinespace
    \textcolor{red}{\textbf{Critical}} & Sensitive Information Disclosure on Port 8080 & The service on \texttt{10.5.5.5:8080} is titled "TOP SECRET DB", indicating a high-value target is exposed internally. This contradicts previous risk assessments. \\
    \addlinespace
    \textcolor{orange}{\textbf{High}} & Inadequate Security Awareness Training & Without mandatory annual training, employees are more likely to fall victim to phishing and social engineering, undermining other security controls. \\
    \addlinespace
    \textbf{Informational} & Outdated Risk Assessment Data & The pre-existing assessment incorrectly classified Port 8080 as a "false positive". This indicates a potential flaw in the risk management process. \\
    \bottomrule
\end{tabular}
\end{table}

% 8. Recommendations
\section{Recommendations}
\label{sec:recommendations}

The following actionable recommendations are provided to address the identified risks. They are prioritized to focus on the most critical vulnerabilities first.

\subsection{Immediate Actions (0-7 Days)}

\begin{enumerate}
    \item \textbf{Investigate and Secure Port 8080:}
        \begin{itemize}
            \item Immediately investigate the service running on \texttt{10.5.5.5:8080} to identify its function and the data it contains.
            \item If the service is critical, implement strong authentication (e.g., username/password and MFA) and apply network-level access controls (firewall rules) to restrict access to only authorized personnel.
            \item If the service is non-essential or a development remnant, it should be disabled and removed from the server.
        \end{itemize}
    \item \textbf{Update Risk Register:} The risk entry for "Port 8080 Secured" must be immediately updated to reflect the new finding. The severity should be re-classified as Critical.
\end{enumerate}

\subsection{Short-Term Actions (1-3 Months)}

\begin{enumerate}
    \item \textbf{Deploy Multi-Factor Authentication (MFA):}
        \begin{itemize}
            \item Prioritize the rollout of MFA for all user accounts on the primary email system (\texttt{MapleLeafLogistics.org}).
            \item Plan and execute the deployment of MFA for all employee computer (workstation and laptop) logins.
        \end{itemize}
    \item \textbf{Implement Annual Security Training:}
        \begin{itemize}
            \item Procure and schedule mandatory security awareness training for all current employees.
            \item Formalize a policy requiring all employees to complete this training on an annual basis.
        \end{itemize}
\end{enumerate}

\subsection{Long-Term Actions (3+ Months)}

\begin{enumerate}
    \item \textbf{Enhance Vulnerability Management:} Review and improve the internal vulnerability scanning and risk assessment process to ensure that findings are accurately verified and not prematurely dismissed as false positives.
    \item \textbf{Conduct Regular Assessments:} Schedule regular, independent security assessments (including penetration testing) to proactively identify and remediate vulnerabilities.
\end{enumerate}

\end{document}
```