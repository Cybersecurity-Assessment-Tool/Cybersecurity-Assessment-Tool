```latex
\documentclass[12pt]{article}

% --- PACKAGES ---
\usepackage[margin=1in]{geometry}
\usepackage{pifont} % For checkmarks and crosses
\usepackage{booktabs} % For professional tables
\usepackage{hyperref} % For clickable links
\usepackage{url} % For URL formatting
\usepackage{seqsplit} % For splitting long strings
\usepackage[T1]{fontenc}

% --- DOCUMENT SETUP ---
\hypersetup{
    colorlinks=true,
    linkcolor=blue,
    filecolor=magenta,      
    urlcolor=cyan,
    pdftitle={Cybersecurity Posture Report},
    pdfpagemode=FullScreen,
}

\newcommand{\yes}{\ding{51}} % Checkmark
\newcommand{\no}{\ding{55}}  % Cross

% --- DOCUMENT START ---
\begin{document}

% --- TITLE PAGE ---
\begin{titlepage}
    \centering
    \vspace*{\stretch{1.0}}
    \Huge{\textbf{Cybersecurity Posture Report}}
    \vspace{0.5cm}
    \LARGE{Prepared for: Green Sprout Organic}
    \vspace{1.5cm}
    \large{Generated: \today}
    \vspace{1.0cm}
    \large{Author: Cybersecurity Analyst}
    \vspace*{\stretch{2.0}}
\end{titlepage}

\tableofcontents
\newpage

% --- EXECUTIVE SUMMARY ---
\section{Executive Summary}
This report provides a comprehensive analysis of the cybersecurity posture for Green Sprout Organic, based on a combination of technical network scanning, a review of organizational security controls, and an assessment of pre-existing risks.

The analysis revealed critical gaps in administrative security controls. The lack of Multi-Factor Authentication (MFA) on email services represents a \textbf{Critical Risk}, exposing the organization to business email compromise and account takeover attacks. Furthermore, the absence of a formal security awareness training program for employees constitutes a \textbf{High Risk}, increasing susceptibility to phishing and other social engineering tactics.

On the technical front, a network scan of the target host \texttt{192.168.0.5} showed a secure configuration with no open ports detected, indicating that the previously identified risk of an "Unencrypted Web Server" on port 80 has likely been remediated or was misidentified.

Immediate action should be taken to implement MFA for email and to establish a security awareness training program.

% --- ORGANIZATIONAL INFORMATION ---
\section{Organizational Information}
The following details were provided for the assessment.

\begin{tabular}{@{}ll}
    \toprule
    \textbf{Attribute} & \textbf{Value} \\
    \midrule
    Organization Name & Green Sprout Organic \\
    Email Domain & \texttt{GreenSproutOrganic.net} \\
    Website Domain & \url{www.GreenSproutOrganic.net} \\
    External IP Address & \texttt{85.130.72.158} \\
    \bottomrule
\end{tabular}

% --- SECURITY CONTROL REVIEW ---
\section{Security Control Review}
A review of administrative security controls was conducted based on a standardized questionnaire. The results highlight key areas of strength and weakness in the current security policy.

\begin{tabular}{@{}p{0.8\linewidth}c}
    \toprule
    \textbf{Control Question} & \textbf{Status} \\
    \midrule
    Do you require MFA to access email? & \no \\
    Do you require MFA to log into computers? & \yes \\
    Do you require MFA to access sensitive data systems? & \yes \\
    Does your organization have an employee acceptable use policy? & \yes \\
    Does your organization do security awareness training for new employees? & \no \\
    Does your organization do security awareness training for all employees at least once per year? & \no \\
    \bottomrule
\end{tabular}

\subsection*{Analysis}
The lack of MFA for email is a critical deficiency. Email is a primary target for attackers seeking to gain an initial foothold in an organization. The absence of a security awareness training program is also a significant gap, as employees are the first line of defense against phishing and social engineering attacks.

% --- TECHNICAL SCAN RESULTS ---
\section{Technical Scan Results}
A network port scan was performed to identify accessible services on the specified target system.

\begin{itemize}
    \item \textbf{Target IP Address:} \texttt{192.168.0.5}
    \item \textbf{Scan Summary:} The scan revealed no open ports on the target host. This indicates a strong network perimeter for this specific system, minimizing its attack surface.
\end{itemize}

\begin{table}[h!]
\centering
\caption{Port Scan Details for \texttt{192.168.0.5}}
\begin{tabular}{@{}ccccc}
    \toprule
    \textbf{Port} & \textbf{State} & \textbf{Service} & \textbf{Product} & \textbf{Version} \\
    \midrule
    80 & closed & http & N/A & N/A \\
    \bottomrule
\end{tabular}
\end{table}

\subsection*{Analysis}
The scan results are positive, showing that the host is not exposing any vulnerable services to the network. The previously identified risk concerning an open port 80 is not present on this host at the time of scanning. This suggests the risk has been successfully remediated or was associated with a different asset.

% --- RISK ASSESSMENT SUMMARY ---
\section{Risk Assessment Summary}
The following table synthesizes findings from the security control review, technical scan, and pre-existing risk data.

\begin{table}[h!]
\centering
\caption{Consolidated Risk Register}
\begin{tabular}{@{}p{0.25\linewidth}p{0.45\linewidth}p{0.1\linewidth}p{0.1\linewidth}@{}}
    \toprule
    \textbf{Risk Name} & \textbf{Description} & \textbf{Severity} & \textbf{Status} \\
    \midrule
    \textbf{No MFA on Email} & Lack of multi-factor authentication on email accounts exposes the organization to account takeovers and business email compromise. & \textbf{Critical} & \textbf{Active} \\
    \addlinespace
    \textbf{No Security Awareness Training} & Employees are not trained to recognize or respond to phishing, malware, or social engineering attacks. & \textbf{High} & \textbf{Active} \\
    \addlinespace
    \textbf{Unencrypted Web Server} & The pre-existing risk register noted that port 80 (HTTP) was open. This was not confirmed by the recent scan. & Medium & Remediated \\
    \bottomrule
\end{tabular}
\end{table}

% --- RECOMMENDATIONS ---
\section{Recommendations}
Based on the analysis, the following actions are recommended to mitigate the identified risks. Recommendations are prioritized by severity.

\begin{enumerate}
    \item \textbf{Implement MFA for Email (Critical):}
    \begin{itemize}
        \item \textbf{Action:} Enforce mandatory Multi-Factor Authentication (MFA) for all user accounts on the \texttt{GreenSproutOrganic.net} email domain immediately.
        \item \textbf{Impact:} Drastically reduces the risk of unauthorized access to email accounts, even if passwords are stolen.
    \end{itemize}
    \vspace{0.5cm}

    \item \textbf{Establish Security Awareness Training Program (High):}
    \begin{itemize}
        \item \textbf{Action:} Procure and implement a security awareness training solution. This program should include initial training for all new hires and mandatory annual refresher training for all employees.
        \item \textbf{Impact:} Creates a security-conscious culture and reduces the likelihood of successful phishing and social engineering attacks.
    \end{itemize}
    \vspace{0.5cm}
    
    \item \textbf{Validate Risk Register (Informational):}
    \begin{itemize}
        \item \textbf{Action:} Review and update the internal risk register to reflect that the "Unencrypted Web Server" risk on host \texttt{192.168.0.5} has been remediated.
        \item \textbf{Impact:} Ensures that security efforts are focused on current, active threats and that risk documentation is accurate.
    \end{itemize}
\end{enumerate}

\end{document}
```