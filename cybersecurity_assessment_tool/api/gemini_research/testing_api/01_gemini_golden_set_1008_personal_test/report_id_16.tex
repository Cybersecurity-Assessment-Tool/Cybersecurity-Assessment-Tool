```latex
\documentclass[12pt]{article}

% Required Packages
\usepackage[margin=1in]{geometry}
\usepackage{pifont} % For checkmarks and crosses
\usepackage{booktabs} % For professional tables
\usepackage{hyperref} % For clickable links
\usepackage{url} % For formatting URLs
\usepackage{seqsplit} % For splitting long strings in tt font
\usepackage{graphicx}
\usepackage{xcolor}

% Document Metadata
\title{Cybersecurity Posture Assessment Report}
\author{Cybersecurity Analysis Division}
\date{\today}

% Hyperref Setup
\hypersetup{
    colorlinks=true,
    linkcolor=blue,
    filecolor=magenta,      
    urlcolor=cyan,
    pdftitle={Cybersecurity Posture Assessment Report},
    pdfpagemode=FullScreen,
}

\begin{document}

\maketitle
\thispagestyle{empty}
\newpage

\tableofcontents
\newpage

\section{Executive Summary}

This report provides a comprehensive cybersecurity assessment for \textbf{Hidden Gem}, based on a correlation of network scan data, organizational security controls, and known risks. The assessment was conducted on \today.

Overall, \textbf{Hidden Gem} demonstrates a mixed security posture. The organization exhibits strong network perimeter security, as evidenced by a network scan that revealed no open ports on the target system. This indicates a well-configured firewall and a commendable "default deny" stance for external traffic.

However, a critical gap was identified in the organization's internal security controls. The lack of mandatory Multi-Factor Authentication (MFA) for computer logins presents a high-risk vulnerability. If an employee's credentials were to be compromised, an attacker could gain direct access to an endpoint on the internal network, bypassing perimeter defenses.

This report details these findings and provides an actionable, high-priority recommendation to mitigate the identified risk by implementing MFA for all endpoint logins.

\section{Organizational Information}

The following details were provided for the assessment.

\begin{tabular}{@{}ll}
\toprule
\textbf{Attribute} & \textbf{Value} \\
\midrule
Organization Name & \textbf{Hidden Gem} \\
Email Domain & \texttt{HiddenGem.com} \\
Website Domain & \url{www.HiddenGem.com} \\
External IP Address & \texttt{60.81.52.142} \\
\bottomrule
\end{tabular}

\section{Security Control Review}

The following table summarizes the organization's responses to a security controls questionnaire. The assessment column highlights adherence to best practices or identifies potential gaps.

\begin{tabular}{@{}p{0.6\linewidth} c p{0.2\linewidth}@{}}
\toprule
\textbf{Control Question} & \textbf{Response} & \textbf{Assessment} \\
\midrule
Do you require MFA to access email? & \ding{51} & Best Practice Met \\
\addlinespace
Do you require MFA to log into computers? & \textbf{\color{red}\ding{55}} & \textbf{\color{red}Critical Gap Identified} \\
\addlinespace
Do you require MFA to access sensitive data systems? & \ding{51} & Best Practice Met \\
\addlinespace
Does your organization have an employee acceptable use policy? & \ding{51} & Best Practice Met \\
\addlinespace
Does your organization do security awareness training for new employees? & \ding{51} & Best Practice Met \\
\addlinespace
Does your organization do security awareness training for all employees at least once per year? & \ding{51} & Best Practice Met \\
\bottomrule
\end{tabular}

\section{Technical Scan Results}

An external network scan was performed to identify exposed services and potential vulnerabilities.

\begin{itemize}
    \item \textbf{Scan Target:} \texttt{192.168.1.100}
    \item \textbf{Host Status:} Up
    \item \textbf{Scan Summary:} The scan confirmed that the host is online and responsive. However, \textbf{no open TCP or UDP ports were detected}. All 1000 scanned TCP ports were in a "closed" state.
\end{itemize}

\subsection{Analysis}
This is a positive security finding. It indicates that the network perimeter firewall is correctly configured to block all unsolicited inbound traffic, significantly reducing the external attack surface. No services, products, or versions could be enumerated, which is the desired outcome for a secure perimeter.

\section{Consolidated Risk Assessment}

This section synthesizes findings from the security control review, technical scans, and pre-existing risk data. No pre-existing vulnerabilities were reported. A new risk has been identified based on the questionnaire analysis.

\begin{tabular}{@{}p{0.1\linewidth} p{0.2\linewidth} p{0.5\linewidth} p{0.1\linewidth}@{}}
\toprule
\textbf{Risk ID} & \textbf{Risk Name} & \textbf{Description} & \textbf{Severity} \\
\midrule
RISK-001 & Lack of MFA on Endpoints & User workstations and laptops do not require MFA for login. This exposes the organization to significant risk from compromised credentials. An attacker with a valid password could gain initial access to the internal network, bypassing all perimeter security controls. & \textbf{High} \\
\bottomrule
\end{tabular}

\section{Recommendations}

Based on the consolidated risk assessment, the following actions are recommended to enhance the organization's security posture.

\begin{tabular}{@{}p{0.15\linewidth} p{0.65\linewidth} p{0.1\linewidth}@{}}
\toprule
\textbf{Risk ID} & \textbf{Recommendation} & \textbf{Priority} \\
\midrule
RISK-001 & Implement a mandatory Multi-Factor Authentication (MFA) solution for all employee computer (endpoint) logins. This control is critical for preventing unauthorized access in the event of password compromise. Solutions such as Windows Hello for Business, Duo Security, or other platform-specific authenticators should be deployed across the entire organization. & \textbf{High} \\
\bottomrule
\end{tabular}

\end{document}
```