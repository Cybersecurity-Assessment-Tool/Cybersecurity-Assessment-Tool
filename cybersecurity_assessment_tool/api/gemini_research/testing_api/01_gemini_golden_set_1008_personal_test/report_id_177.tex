```latex
\documentclass[12pt, a4paper]{article}

% Preamble: Required Packages
\usepackage[margin=1in]{geometry} % Set page margins
\usepackage{pifont}               % For checkmarks and crosses (\ding)
\usepackage{booktabs}             % For professional-looking tables
\usepackage{hyperref}             % For hyperlinks and document metadata
\usepackage{url}                  % For formatting URLs
\usepackage{seqsplit}             % For splitting long strings in \texttt
\usepackage{graphicx}             % For including logos (optional but good practice)
\usepackage{xcolor}               % For custom colors

% --- Document Metadata ---
\hypersetup{
    colorlinks=true,
    linkcolor=blue,
    filecolor=magenta,      
    urlcolor=cyan,
    pdftitle={Cybersecurity Posture Assessment Report},
    pdfauthor={Cybersecurity Analyst},
    pdfsubject={Security Analysis},
    pdfkeywords={Cybersecurity, Nmap, Risk Assessment},
}

\title{Cybersecurity Posture Assessment Report}
\author{Cybersecurity Analyst}
\date{\today}

% --- Start of Document ---
\begin{document}

\maketitle
\thispagestyle{empty}
\newpage

\tableofcontents
\newpage

% ===================================================================
% 1. Executive Summary
% ===================================================================
\section{Executive Summary}

This report provides a comprehensive cybersecurity assessment for \textbf{Nomad Gear Co.}, based on an analysis of network scan data, organizational security controls, and existing risk documentation. The assessment was conducted by correlating technical findings with self-reported security practices.

While the organization demonstrates a solid foundation in several areas, including employee security training and the use of Multi-Factor Authentication (MFA) for email and computer access, two critical risks were identified that require immediate attention.

A network scan of the internal asset at \texttt{10.5.5.5} revealed an open port (\texttt{8080}) hosting a service with the title \textbf{"TOP SECRET DB"}. This finding directly contradicts the existing risk register, which incorrectly classifies this port as a secure false positive. This discrepancy points to a significant flaw in the current risk management process.

Furthermore, organizational data confirms that MFA is not required for accessing sensitive data systems. The combination of an exposed database and the lack of mandatory MFA for sensitive systems creates a high-impact attack vector that could lead to a severe data breach.

Immediate remediation of the exposed service and the swift implementation of a mandatory MFA policy for all sensitive systems are the highest priority recommendations.

% ===================================================================
% 2. Organizational Information
% ===================================================================
\section{Organizational Information}

The following details were provided for the assessment. This information helps establish the context for the technical findings.

\begin{tabular}{@{}ll}
\toprule
\textbf{Attribute} & \textbf{Value} \\
\midrule
Organization Name & \textbf{Nomad Gear Co.} \\
Email Domain & \texttt{NomadGearCo.net} \\
Website Domain & \url{www.NomadGearCo.net} \\
External IP Address & \texttt{231.109.232.69} \\
\bottomrule
\end{tabular}

% ===================================================================
% 3. Security Control Review
% ===================================================================
\section{Security Control Review}

A review of the organization's security questionnaire was performed to assess the maturity of its administrative and technical controls. The results are summarized below.

\begin{table}[h!]
\centering
\caption{Security Questionnaire Analysis}
\begin{tabular}{@{}p{0.7\linewidth}c@{}}
\toprule
\textbf{Control Question} & \textbf{Status} \\
\midrule
Do you require MFA to access email? & \textcolor{green}{\ding{51}} \\
Do you require MFA to log into computers? & \textcolor{green}{\ding{51}} \\
\textbf{Do you require MFA to access sensitive data systems?} & \textcolor{red}{\ding{55}} \\
Does your organization have an employee acceptable use policy? & \textcolor{green}{\ding{51}} \\
Does your organization do security awareness training for new employees? & \textcolor{green}{\ding{51}} \\
Does your organization do security awareness training for all employees at least once per year? & \textcolor{green}{\ding{51}} \\
\bottomrule
\end{tabular}
\end{table}

\subsection*{Analysis}
The organization has implemented several key security controls effectively. However, the absence of mandatory MFA for sensitive data systems (\textcolor{red}{\ding{55}}) represents a critical gap in the security posture. This control is essential for protecting high-value assets from unauthorized access, especially in the event of credential compromise.

% ===================================================================
% 4. Technical Scan Results
% ===================================================================
\section{Technical Scan Results}

A network scan was conducted to identify open ports and exposed services on the target system.

\begin{itemize}
    \item \textbf{Target IP Address:} \texttt{10.5.5.5}
    \item \textbf{Target Status:} Up
\end{itemize}

The following table details the significant findings from the scan.

\begin{table}[h!]
\centering
\caption{Open Port and Service Information}
\begin{tabular}{@{}llll@{}}
\toprule
\textbf{Port} & \textbf{State} & \textbf{Service/Product} & \textbf{Details} \\
\midrule
8080/tcp & Open & http-title & \textbf{Title: TOP SECRET DB} \\
\bottomrule
\end{tabular}
\end{table}

\subsection*{Analysis}
The scan identified a web service running on port \texttt{8080}. The HTTP title of this service, "TOP SECRET DB," strongly suggests it is a sensitive system containing confidential or critical data. Public or internal exposure of such a system without proper access controls is a severe security risk. This technical finding directly contradicts the information in the current risk register, which dismisses this port as a false positive.

% ===================================================================
% 5. Correlated Risk Assessment
% ===================================================================
\section{Correlated Risk Assessment}

By synthesizing the security control review, technical scan results, and existing risk documentation, the following risks have been identified and prioritized.

\begin{table}[h!]
\centering
\caption{Summary of Identified Risks}
\begin{tabular}{@{}p{0.2\linewidth}p{0.5\linewidth}p{0.2\linewidth}@{}}
\toprule
\textbf{Risk Name} & \textbf{Description} & \textbf{Severity} \\
\midrule
\textbf{Exposed Sensitive Database} & A service on port 8080 titled "TOP SECRET DB" is exposed on the internal network. This system is a prime target for unauthorized access. & \textbf{\textcolor{red}{Critical}} \\
\addlinespace
\textbf{No MFA on Sensitive Systems} & The lack of MFA on systems like the one discovered creates a significant risk of unauthorized access via stolen or weak credentials. & \textbf{\textcolor{orange}{High}} \\
\addlinespace
\textbf{Inaccurate Risk Register} & The current risk register incorrectly lists port 8080 as a "confirmed secure" false positive. This indicates a failure in the risk validation and management process, preventing visibility of a critical vulnerability. & \textbf{\textcolor{orange}{High}} \\
\bottomrule
\end{tabular}
\end{table}

% ===================================================================
% 6. Recommendations
% ===================================================================
\section{Recommendations}

Based on the correlated risk assessment, the following actions are recommended to mitigate the identified vulnerabilities and improve the overall security posture.

\subsection*{Immediate Actions (0-7 Days)}
\begin{enumerate}
    \item \textbf{Isolate Exposed Database:} Immediately investigate the service running on \texttt{10.5.5.5:8080}. Apply a firewall rule to restrict all access to the port except from explicitly authorized administrative hosts.
    \item \textbf{Identify System Owner:} Determine the business purpose, data classification, and ownership of the "TOP SECRET DB" system.
\end{enumerate}

\subsection*{High Priority Actions (1-4 Weeks)}
\begin{enumerate}
    \item \textbf{Enforce MFA on Sensitive Systems:} Procure and implement an MFA solution for all systems classified as containing sensitive or critical data. Create and enforce a policy that mandates its use.
    \item \textbf{Secure or Decommission the Database:} Based on the investigation, either properly secure the database behind an authentication layer with MFA and robust logging or, if it is not a business-critical system, decommission it entirely.
\end{enumerate}

\subsection*{Process Improvement (1-3 Months)}
\begin{enumerate}
    \item \textbf{Review Risk Management Process:} Conduct a full review of the risk management and vulnerability validation process. The failure to correctly identify the risk on port 8080 must be understood and corrected to prevent future oversight.
    \item \textbf{Update Risk Register:} Perform a comprehensive update of the risk register to accurately reflect the current state of the environment, incorporating the findings from this report.
\end{enumerate}

\end{document}
```