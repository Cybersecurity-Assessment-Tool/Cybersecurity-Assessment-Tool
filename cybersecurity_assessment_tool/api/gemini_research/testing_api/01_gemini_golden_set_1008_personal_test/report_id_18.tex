```latex
\documentclass[12pt]{article}

% Required Packages
\usepackage[margin=1in]{geometry}
\usepackage{pifont} % For \ding
\usepackage{booktabs} % For professional tables (\toprule, \midrule, \bottomrule)
\usepackage{hyperref} % For clickable links
\usepackage{url}      % For \url command
\usepackage{seqsplit} % For splitting long strings in \texttt
\usepackage{xcolor}   % For colors

% Hyperlink Setup
\hypersetup{
    colorlinks=true,
    linkcolor=blue,
    filecolor=magenta,
    urlcolor=cyan,
}

% Document Start
\begin{document}

\title{Cybersecurity Posture Assessment Report}
\author{Cybersecurity Analyst}
\date{\today}
\maketitle

\begin{abstract}
This report provides a comprehensive cybersecurity assessment for Binary Star. The analysis is based on a correlation of network scan data, a security controls questionnaire, and a review of pre-existing risks. The assessment identifies critical security gaps, including a lack of Multi-Factor Authentication (MFA) for email and insufficient security training for new employees. On a positive note, a previously identified risk related to an unencrypted web server appears to have been remediated. This document outlines the findings in detail and provides prioritized, actionable recommendations to enhance the organization's security posture.
\end{abstract}

\section*{1. Organizational Information}
This section provides the high-level organizational details used as the basis for this assessment.

\begin{center}
\begin{tabular}{@{}ll@{}}
\toprule
\textbf{Attribute} & \textbf{Value} \\
\midrule
Organization Name & \textbf{Binary Star} \\
Email Domain & \texttt{BinaryStar.org} \\
Website Domain & \url{www.BinaryStar.org} \\
External IP & \texttt{110.166.149.40} \\
\bottomrule
\end{tabular}
\end{center}

\section*{2. Security Control Review}
The following table summarizes the organization's current security controls based on the provided questionnaire. Gaps are marked with a cross (\ding{55}) and represent areas of significant risk.

\begin{center}
\begin{tabular}{p{0.7\textwidth}c}
\toprule
\textbf{Control Question} & \textbf{Status} \\
\midrule
Do you require MFA to access email? & \ding{55} \\
Do you require MFA to log into computers? & \ding{51} \\
Do you require MFA to access sensitive data systems? & \ding{51} \\
Does your organization have an employee acceptable use policy? & \ding{51} \\
Does your organization do security awareness training for new employees? & \ding{55} \\
Does your organization do security awareness training for all employees at least once per year? & \ding{51} \\
\bottomrule
\end{tabular}
\end{center}

\subsection*{Analysis of Gaps}
Two critical gaps were identified from the questionnaire:
\begin{itemize}
    \item \textbf{No MFA for Email:} This is a \textbf{Critical Risk}. Email is a primary vector for phishing and account takeover attacks. Without MFA, a single compromised password grants an attacker full access to an employee's mailbox, which can lead to data breaches, financial fraud, and further internal compromise.
    \item \textbf{No Security Training for New Employees:} This is a \textbf{High Risk}. New hires are often unfamiliar with corporate security policies and are prime targets for social engineering. Failing to provide immediate training leaves a window of vulnerability where a new employee could inadvertently cause a security incident.
\end{itemize}

\section*{3. Technical Scan Results}
A network scan was performed to identify discoverable ports and services on the target system.

\begin{itemize}
    \item \textbf{Target IP:} \texttt{192.168.0.5}
    \item \textbf{Scan Date:} \today
\end{itemize}

The scan reported the following port status:
\begin{center}
\begin{tabular}{@{}lll@{}}
\toprule
\textbf{Port} & \textbf{State} & \textbf{Service} \\
\midrule
80/tcp & closed & http \\
\bottomrule
\end{tabular}
\end{center}
\textbf{Analysis:} The scan indicates that port 80 (HTTP) is closed on the target system. This is a positive security finding, as it prevents unencrypted web traffic. This result directly contradicts a pre-existing risk documented in the next section, suggesting that the risk may have been recently remediated.

\section*{4. Consolidated Risk Assessment}
This table synthesizes findings from the security questionnaire, technical scan, and pre-existing risk data into a unified view.

\begin{center}
\begin{tabular}{p{0.3\textwidth}p{0.4\textwidth}ll}
\toprule
\textbf{Risk Name} & \textbf{Overview} & \textbf{Severity} & \textbf{Status} \\
\midrule
Lack of MFA on Email & User email accounts are protected only by passwords, making them highly vulnerable to takeover. & Critical & \textbf{Active} \\
\addlinespace
Inadequate New Employee Training & New hires are not provided with security awareness training, increasing susceptibility to social engineering. & High & \textbf{Active} \\
\addlinespace
Unencrypted Web Server & Port 80 is open, potentially exposing unencrypted traffic and sensitive information. & Medium (5.0) & \textit{Potentially Resolved} \\
\bottomrule
\end{tabular}
\end{center}

\section*{5. Recommendations}
Based on the consolidated risk assessment, the following actions are recommended to mitigate the identified risks and strengthen the security posture of Binary Star.

\subsection*{Immediate Priority (To Be Completed within 30 Days)}
\begin{itemize}
    \item \textbf{Enforce MFA on All Email Accounts:} Immediately enable and enforce MFA for all user mailboxes. This is the single most effective control to mitigate the risk of email account compromise and should be treated as the top priority.
\end{itemize}

\subsection*{Short-Term Priority (To Be Completed within 90 Days)}
\begin{itemize}
    \item \textbf{Implement Onboarding Security Training:} Integrate a mandatory security awareness training module into the new employee onboarding process. This training must cover, at a minimum, phishing identification, password security, and the company's acceptable use policy.
    \item \textbf{Validate and Close Old Risks:} Conduct a thorough verification to confirm that Port 80 is intentionally and permanently closed on all external-facing production systems. If the closure is confirmed, formally update the organization's risk register to mark the "Unencrypted Web Server" risk as resolved.
\end{itemize}

\subsection*{Long-Term Strategic Recommendations}
\begin{itemize}
    \item \textbf{Establish a Continuous Monitoring Program:} Implement a schedule for regular, automated vulnerability scanning of all internal and external assets. Reviewing these scans quarterly will help proactively identify and address new risks before they can be exploited.
\end{itemize}

\end{document}
```