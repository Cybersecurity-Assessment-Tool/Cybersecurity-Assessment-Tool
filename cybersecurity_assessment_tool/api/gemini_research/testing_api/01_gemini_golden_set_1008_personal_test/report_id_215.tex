```latex
\documentclass[12pt, a4paper]{article}

% Preamble: Required Packages
\usepackage[margin=1in]{geometry}
\usepackage{pifont} % For check and cross marks
\usepackage{booktabs} % For professional tables
\usepackage{hyperref} % For clickable links and references
\usepackage{url} % For formatting URLs
\usepackage{seqsplit} % To split long strings in tt font
\usepackage{graphicx}
\usepackage[table]{xcolor}
\usepackage{fancyhdr}

% --- Document Setup ---
% Define colors for risk levels
\definecolor{criticalred}{HTML}{D10000}
\definecolor{highorange}{HTML}{E25F00}
\definecolor{mediumyellow}{HTML}{F5C700}
\definecolor{lowgreen}{HTML}{008B00}

% Hyperref Setup
\hypersetup{
    colorlinks=true,
    linkcolor=blue,
    filecolor=magenta,      
    urlcolor=cyan,
    pdftitle={Cybersecurity Assessment Report},
    pdfauthor={Cybersecurity Analyst},
    pdfsubject={Security Assessment},
    pdfkeywords={Security, Assessment, Report},
    bookmarks=true
}

% Header and Footer
\pagestyle{fancy}
\fancyhf{}
\fancyhead[L]{Cybersecurity Assessment Report}
\fancyhead[R]{Arcane Security}
\fancyfoot[C]{\thepage}
\fancyfoot[L]{\small \textit{CONFIDENTIAL}}
\fancyfoot[R]{\small \today}

% --- Document Start ---
\begin{document}

% --- Title Page ---
\begin{titlepage}
    \centering
    \vspace*{1cm}
    \includegraphics[width=0.4\textwidth]{example-image-a} % Placeholder for company logo
    
    \vspace{1.5cm}
    
    \Huge
    \textbf{Cybersecurity Assessment Report}
    
    \vspace{1.5cm}
    
    \Large
    Prepared for: \\
    \textbf{Arcane Security}
    
    \vspace{2cm}
    
    \large
    Date of Report: \today \\
    Date of Scan: \today % Placeholder as scan_date was not in JSON
    
    \vfill
    
    \normalsize
    \textit{This document contains sensitive and confidential information. Access, distribution, and use are restricted to authorized personnel only. Handle with care.}
    
\end{titlepage}

\tableofcontents
\newpage

% --- Section 1: Executive Summary ---
\section{Executive Summary}

This report details the findings of a cybersecurity assessment conducted for \textbf{Arcane Security}. The assessment combined a review of organizational security controls, an analysis of pre-existing risks, and a technical network scan to provide a holistic view of the organization's security posture.

The analysis revealed several critical and high-risk vulnerabilities that require immediate attention. Key findings include:
\begin{itemize}
    \item \textbf{Critical Gaps in Identity and Access Management:} Multi-Factor Authentication (MFA) is not enforced for accessing email or for logging into company computers. This exposes the organization to significant risk from credential theft and unauthorized access.
    \item \textbf{Critical System Exposure:} The technical scan confirmed a pre-existing risk, identifying an exposed service on a local loopback interface (\texttt{127.0.0.1}). This configuration is highly anomalous and rated with a CVSS score of 10.0, indicating a critical vulnerability.
    \item \textbf{Deficiencies in Security Governance:} The organization lacks a formal employee Acceptable Use Policy and does not provide mandatory annual security awareness training for all staff. These gaps weaken the human element of security, increasing susceptibility to social engineering and insider threats.
\end{itemize}

The overall security posture is assessed as \textbf{High-Risk}. This report provides specific, actionable recommendations to mitigate the identified vulnerabilities and strengthen the organization's defenses against cyber threats.

\newpage

% --- Section 2: Organizational Information ---
\section{Organizational Information}

The following details were provided for the assessment.

\begin{table}[h!]
\centering
\caption{Client Organizational Details}
\label{tab:orginfo}
\begin{tabular}{@{}ll@{}}
\toprule
\textbf{Attribute} & \textbf{Value} \\ \midrule
Organization Name & \textbf{Arcane Security} \\
Email Domain      & \texttt{ArcaneSecurity.org} \\
Website Domain    & \url{www.ArcaneSecurity.org} \\
External IP Address & \seqsplit{\texttt{230.160.139.68}} \\ \bottomrule
\end{tabular}
\end{table}

% --- Section 3: Security Control Review ---
\section{Security Control Review}

A review of internal security controls was conducted via a questionnaire. The responses highlight significant gaps in foundational security practices. A "No" response indicates a deviation from best practices and introduces risk.

\begin{table}[h!]
\centering
\caption{Security Controls Questionnaire Analysis}
\label{tab:controls}
\renewcommand{\arraystretch}{1.2}
\begin{tabular}{@{}p{0.6\linewidth} c l@{}}
\toprule
\textbf{Control Question} & \textbf{Response} & \textbf{Assessment} \\ \midrule
Do you require MFA to access email? & \ding{55} & \textcolor{criticalred}{\textbf{Critical Gap}} \\
Do you require MFA to log into computers? & \ding{55} & \textcolor{highorange}{\textbf{High Risk}} \\
Do you require MFA to access sensitive data systems? & \ding{51} & Best Practice \\
Does your organization have an employee acceptable use policy? & \ding{55} & \textcolor{highorange}{\textbf{High Risk}} \\
Does your organization do security awareness training for new employees? & \ding{51} & Best Practice \\
Does your organization do security awareness training for all employees at least once per year? & \ding{55} & \textcolor{highorange}{\textbf{High Risk}} \\ \bottomrule
\end{tabular}
\end{table}

% --- Section 4: Technical Scan Results ---
\section{Technical Scan Results}

A network scan was performed to identify open ports and exposed services on the target system. The results confirm the presence of an open service that requires immediate investigation.

\subsection{Open Ports}
The following ports were found to be open on the target host.

\begin{table}[h!]
\centering
\caption{Open Port Findings}
\label{tab:ports}
\begin{tabular}{@{}lllll@{}}
\toprule
\textbf{Host} & \textbf{Port} & \textbf{State} & \textbf{Service (Inferred)} & \textbf{Product / Version} \\ \midrule
\texttt{127.0.0.1} & 22/tcp & open & SSH & N/A \\ \bottomrule
\end{tabular}
\end{table}

\subsection{Analysis}
The scan identified that port 22 (SSH - Secure Shell) is open on the loopback interface \texttt{127.0.0.1}. While often used for secure remote administration, its presence on a supposedly internal-only interface, combined with the pre-existing risk data, suggests a critical misconfiguration. Without version information, it is impossible to rule out vulnerabilities in the SSH software itself. This finding directly correlates with the "Localhost Exposed" risk identified in Section 5.

\newpage

% --- Section 5: Consolidated Risk Assessment ---
\section{Consolidated Risk Assessment}

This section synthesizes findings from the security control review, technical scan, and pre-existing risk data into a consolidated list of key risks facing the organization.

\begin{table}[h!]
\centering
\caption{Summary of Identified Risks}
\label{tab:risks}
\renewcommand{\arraystretch}{1.5}
\begin{tabular}{@{}p{0.25\linewidth} p{0.45\linewidth} p{0.15\linewidth}@{}}
\toprule
\textbf{Risk Title} & \textbf{Description} & \textbf{Severity} \\ \midrule
\rowcolor{criticalred!20}
\textbf{Exposed Localhost Service} & An SSH service is accessible on the local loopback interface, correlating with a known risk rated at CVSS 10.0. This indicates a severe misconfiguration that could lead to a full system compromise. & \textcolor{criticalred}{\textbf{Critical}} \\
\addlinespace[3pt]
\rowcolor{criticalred!20}
\textbf{Lack of MFA for Critical Systems} & Email and computer logins are protected only by passwords. This makes the organization highly vulnerable to phishing, credential stuffing, and account takeover attacks. & \textcolor{criticalred}{\textbf{Critical}} \\
\addlinespace[3pt]
\rowcolor{highorange!20}
\textbf{Inadequate Security Governance} & The absence of an Acceptable Use Policy and mandatory annual security training for all staff creates an environment where employees are more likely to engage in risky behavior, fall victim to social engineering, or cause accidental data breaches. & \textcolor{highorange}{\textbf{High}} \\ \bottomrule
\end{tabular}
\end{table}

% --- Section 6: Recommendations ---
\section{Recommendations}

The following actions are recommended to mitigate the identified risks and improve the overall security posture of \textbf{Arcane Security}. Recommendations are prioritized based on severity.

\subsection{Remediate Exposed Localhost Service (Critical)}
\begin{itemize}
    \item \textbf{Immediate Action:} Investigate the SSH service running on \texttt{127.0.0.1}. Determine its business purpose. If it is not required, the service should be disabled immediately.
    \item \textbf{Secure Configuration:} If the service is required, ensure it is properly firewalled and not accessible from any external network. Enforce strong configuration, including disabling root login, using key-based authentication instead of passwords, and keeping the software patched to the latest version.
\end{itemize}

\subsection{Implement Comprehensive Multi-Factor Authentication (Critical)}
\begin{itemize}
    \item \textbf{Immediate Action:} Enable MFA for all user accounts on the primary email system (\texttt{ArcaneSecurity.org}). Prioritize administrators, executives, and finance personnel.
    \item \textbf{Short-Term Action:} Deploy a solution to enforce MFA for all computer and laptop logins. This significantly raises the difficulty for an attacker with stolen credentials to gain access to an endpoint.
\end{itemize}

\subsection{Strengthen Security Governance and Awareness (High)}
\begin{itemize}
    \item \textbf{Policy Development:} Develop and implement a formal Acceptable Use Policy (AUP) that clearly defines the rules for using company assets, data, and networks. All employees must read and acknowledge this policy.
    \item \textbf{Training Program:} Establish a mandatory, annual security awareness training program for all employees. This program should cover key topics such as phishing, password security, social engineering, and the new AUP.
\end{itemize}

\end{document}
```