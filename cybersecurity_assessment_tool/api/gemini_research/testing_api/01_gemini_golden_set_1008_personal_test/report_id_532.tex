```latex
\documentclass[12pt]{article}

% Preamble: Required Packages and Document Setup
\usepackage[margin=1in]{geometry}
\usepackage{pifont} % Required for checkmark and X symbols (\ding)
\usepackage{booktabs} % For professional-looking tables (\toprule, \midrule, \bottomrule)
\usepackage{hyperref} % For creating hyperlinks, especially for URLs
\usepackage{url}      % For typesetting URLs
\usepackage{seqsplit} % To split long strings in \texttt environments without breaking
\usepackage[T1]{fontenc}

% Custom Commands for convenience
\newcommand{\yes}{\ding{51}} % Green checkmark
\newcommand{\no}{\ding{55}}  % Red X

% Hyperlink Setup
\hypersetup{
    colorlinks=true,
    linkcolor=black,
    filecolor=magenta,
    urlcolor=blue,
    pdftitle={Cybersecurity Posture Assessment Report},
    pdfauthor={Cybersecurity Analyst},
}

% Document Start
\begin{document}

% --- TITLE SECTION ---
\title{
    \vspace{-1.5cm} % Adjust vertical space
    \textbf{Cybersecurity Posture Assessment Report} \\
    \large For: Nomad Gear Co.
}
\author{Cybersecurity Analyst}
\date{\today}
\maketitle

% --- TABLE OF CONTENTS ---
\tableofcontents
\newpage

% --- EXECUTIVE SUMMARY ---
\section*{1. Executive Summary}

This report provides a cybersecurity posture assessment for \textbf{Nomad Gear Co.}, based on a combination of organizational data, a security controls questionnaire, and a network scan. The assessment identified significant gaps in identity and access management controls, which present a high risk to the organization despite a clean network scan on the targeted host.

The primary findings indicate a lack of Multi-Factor Authentication (MFA) for accessing critical systems such as email and employee computers. Furthermore, the absence of a formal Employee Acceptable Use Policy represents a foundational gap in security governance. While the network scan of the target host \texttt{192.168.1.100} showed no open ports—indicating a well-configured firewall—this does not mitigate the risks associated with compromised credentials.

Immediate remediation should focus on implementing MFA across all user accounts and developing essential security policies to strengthen the organization's overall defense against common cyber threats like phishing and account takeover.

% --- ORGANIZATIONAL INFORMATION ---
\section*{2. Organizational Information}

The following details were provided by the client and used as a baseline for this assessment.

\begin{tabular}{@{}ll}
\toprule
\textbf{Attribute} & \textbf{Value} \\
\midrule
Organization Name & \textbf{Nomad Gear Co.} \\
Email Domain & \texttt{NomadGearCo.org} \\
Website Domain & \url{www.NomadGearCo.org} \\
External IP Address & \texttt{38.58.48.154} \\
\bottomrule
\end{tabular}

% --- SECURITY CONTROL REVIEW ---
\section*{3. Security Control Review}

A review of the organization's security controls was conducted via a questionnaire. The responses highlight critical areas requiring immediate attention. "No" answers indicate a lack of a fundamental security control.

\begin{tabular}{@{}p{0.8\linewidth}c}
\toprule
\textbf{Control Question} & \textbf{Response} \\
\midrule
Do you require MFA to access email? & \no \\
Do you require MFA to log into computers? & \no \\
Do you require MFA to access sensitive data systems? & \yes \\
Does your organization have an employee acceptable use policy? & \no \\
Does your organization do security awareness training for new employees? & \yes \\
Does your organization do security awareness training for all employees at least once per year? & \yes \\
\bottomrule
\end{tabular}

% --- TECHNICAL SCAN RESULTS ---
\section*{4. Technical Scan Results}

A network scan was performed to identify externally or internally exposed services on the specified target system.

\subsection*{4.1. Scan Details}
\begin{itemize}
    \item \textbf{Target IP Address:} \texttt{192.168.1.100}
    \item \textbf{Scan Type:} Nmap TCP Port Scan
\end{itemize}

\subsection*{4.2. Findings}
The scan revealed that the host at \texttt{192.168.1.100} is online and responsive. However, no open TCP ports were detected. All scanned ports were in a "closed" state.

\textbf{Interpretation:} This result suggests that the target system is protected by a well-configured firewall that denies incoming connections. While this is a positive security posture for this specific host, it does not provide visibility into other systems on the network or vulnerabilities related to software, configuration, or user credentials.

% --- RISK ASSESSMENT ---
\section*{5. Risk Assessment}

The following risks were identified by correlating the findings from the security control review and the technical scan. Since no pre-existing vulnerabilities were reported, all risks listed below are newly identified.

\begin{tabular}{@{}lp{0.55\linewidth}ll}
\toprule
\textbf{ID} & \textbf{Risk Description} & \textbf{Severity} & \textbf{Affected Asset(s)} \\
\midrule
RISK-001 & \textbf{Lack of MFA for Email Access.} Without MFA, email accounts are highly susceptible to takeover via phishing or credential stuffing, leading to data breaches and further attacks. & \textbf{High} & User Accounts, Corporate Data, Email System \\
\addlinespace
RISK-002 & \textbf{Lack of MFA for Computer Logins.} Stolen or weak credentials can be used to gain direct access to employee workstations, enabling lateral movement and data exfiltration. & \textbf{High} & Endpoints, User Accounts, Internal Network \\
\addlinespace
RISK-003 & \textbf{No Employee Acceptable Use Policy (AUP).} The absence of a formal AUP creates ambiguity regarding safe technology use and leaves the organization vulnerable to insider threats and non-compliance. & \textbf{Medium} & Employees, IT Governance, Compliance \\
\bottomrule
\end{tabular}

% --- RECOMMENDATIONS ---
\section*{6. Recommendations}

The following actions are recommended to mitigate the identified risks and improve the overall security posture of \textbf{Nomad Gear Co.}.

\subsection*{6.1. Remediate RISK-001: Enforce MFA for Email (High)}
\begin{itemize}
    \item \textbf{Immediate Action:} Enable and enforce MFA for all user email accounts, starting with administrators and executives.
    \item \textbf{Implementation:} Utilize authenticator apps (e.g., Google Authenticator, Microsoft Authenticator) or hardware tokens as second factors.
    \item \textbf{Goal:} Prevent unauthorized access to email accounts even if passwords are compromised.
\end{itemize}

\subsection*{6.2. Remediate RISK-002: Deploy MFA for Endpoints (High)}
\begin{itemize}
    \item \textbf{Immediate Action:} Implement an MFA solution for all company computer logins (laptops and desktops).
    \item \textbf{Implementation:} Leverage solutions like Windows Hello for Business, Duo Security, or other identity providers that integrate with operating systems.
    \item \textbf{Goal:} Protect endpoints from unauthorized local and remote access.
\end{itemize}

\subsection*{6.3. Remediate RISK-003: Establish an Acceptable Use Policy (Medium)}
\begin{itemize}
    \item \textbf{Action:} Develop a comprehensive AUP that clearly defines the rules for using company networks, devices, and data.
    \item \textbf{Implementation:} The policy should be reviewed by management, distributed to all employees, and formally acknowledged via signature. Integrate AUP training into the new employee onboarding process.
    \item \textbf{Goal:} Establish a clear security baseline for all employees and reduce the risk of unintentional policy violations.
\end{itemize}

\end{document}
```