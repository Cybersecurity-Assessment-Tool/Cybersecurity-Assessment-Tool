```latex
\documentclass[12pt]{article}

% Preamble: Required Packages
\usepackage[margin=1in]{geometry}
\usepackage{pifont} % For checkmarks and crosses
\usepackage{booktabs} % For professional tables
\usepackage{hyperref} % For clickable links
\usepackage{url} % For URL formatting
\usepackage{seqsplit} % To split long strings like IPs
\usepackage[utf8]{inputenc}

% Hyperref Setup
\hypersetup{
    colorlinks=true,
    linkcolor=black,
    urlcolor=blue,
    pdftitle={Cybersecurity Posture Report},
    pdfauthor={Cybersecurity Analyst},
}

% Document Start
\begin{document}

% --- Title Page ---
\begin{titlepage}
    \centering
    \vspace*{\stretch{1.0}}
    \Huge{\textbf{Cybersecurity Posture Report}}
    \vspace{0.5cm}
    \LARGE{For: Terraform Global}
    \vspace{1.5cm}
    \large{Generated: \today}
    \vspace*{\stretch{2.0}}
    \normalsize{This report provides a summary of the security posture based on a review of organizational security controls and a technical network scan. It includes an analysis of identified risks and actionable recommendations for remediation.}
    \vfill
\end{titlepage}

\tableofcontents
\newpage

% --- Section 1: Overview and Executive Summary ---
\section{Overview and Executive Summary}
This report assesses the cybersecurity posture of Terraform Global. The analysis is based on a combination of a self-reported security controls questionnaire and an external network vulnerability scan.

\paragraph{Key Findings:} The organization demonstrates a foundational security awareness by implementing Multi-Factor Authentication (MFA) for computer and sensitive system access, alongside annual security training for all staff. These are commendable controls that reduce significant risk.

However, several critical gaps were identified that expose the organization to substantial threats. The most severe risks include:
\begin{itemize}
    \item \textbf{Lack of MFA on Email:} This is a critical vulnerability, making the organization highly susceptible to phishing attacks and business email compromise (BEC).
    * \textbf{Absence of an Acceptable Use Policy (AUP):} This creates ambiguity regarding the secure use of company assets and data.
    * \textbf{No Security Training for New Hires:} New employees are a primary target for attackers, and this gap leaves them unprepared to identify and respond to threats.
\end{itemize}

The external network scan of the target IP address \texttt{[Target IP]} did not identify any open ports. This indicates a strong firewall configuration, which significantly reduces the external attack surface. Recommendations in this report focus on remediating the identified policy and access control gaps to build a more resilient security posture.

% --- Section 2: Organizational Information ---
\section{Organizational Information}
The following details were provided for the assessment.

\begin{tabular}{@{}ll}
    \toprule
    \textbf{Attribute} & \textbf{Value} \\
    \midrule
    Organization Name & Terraform Global \\
    Email Domain & \seqsplit{\texttt{TerraformGlobal.net}} \\
    Website Domain & \seqsplit{\url{www.TerraformGlobal.net}} \\
    External IP Scanned & \seqsplit{\texttt{222.140.0.178}} \\
    \bottomrule
\end{tabular}

% --- Section 3: Security Control Review ---
\section{Security Control Review}
The following table summarizes the organization's responses to the security controls questionnaire. Each "No" response represents a potential security gap that requires attention.

\begin{table}[h!]
\centering
\begin{tabular}{p{0.6\textwidth} c p{0.2\textwidth}}
    \toprule
    \textbf{Control Question} & \textbf{Response} & \textbf{Assessment} \\
    \midrule
    Do you require MFA to access email? & \ding{55} & \textbf{Critical Gap} \\
    Do you require MFA to log into computers? & \ding{51} & Best Practice \\
    Do you require MFA to access sensitive data systems? & \ding{51} & Best Practice \\
    Does your organization have an employee acceptable use policy? & \ding{55} & \textbf{High Risk} \\
    Does your organization do security awareness training for new employees? & \ding{55} & \textbf{High Risk} \\
    Does your organization do security awareness training for all employees at least once per year? & \ding{51} & Best Practice \\
    \bottomrule
\end{tabular}
\caption{Security Controls Questionnaire Analysis}
\end{table}

% --- Section 4: Technical Scan Results ---
\section{Technical Scan Results}
An external network scan was performed to identify open ports and services exposed to the internet.

\begin{itemize}
    \item \textbf{Target IP Address:} \texttt{[Target IP]}
    \item \textbf{Scan Date:} \today
\end{itemize}

\paragraph{Findings:} The scan completed successfully and \textbf{found no open TCP or UDP ports} on the target system. This is a positive security finding, suggesting that a well-configured firewall is in place, enforcing a "default deny" policy for unsolicited inbound traffic. This significantly limits the external attack surface available to potential adversaries.

% --- Section 5: Consolidated Risk Assessment ---
\section{Consolidated Risk Assessment}
The following risks have been identified and prioritized based on the analysis of the security controls questionnaire and technical findings.

\begin{table}[h!]
\centering
\begin{tabular}{p{0.2\textwidth} p{0.5\textwidth} p{0.2\textwidth}}
    \toprule
    \textbf{Risk Title} & \textbf{Description} & \textbf{Severity} \\
    \midrule
    Email Account Compromise & The absence of MFA on email accounts exposes the organization to a high risk of phishing, credential theft, and subsequent business email compromise (BEC) attacks. & \textbf{Critical} \\
    \addlinespace
    Insider Threat \& Misuse & Without a formal Acceptable Use Policy (AUP), there are no clear guidelines for employees on the secure use of corporate assets, increasing the risk of unintentional data exposure or malicious activity. & \textbf{High} \\
    \addlinespace
    Targeted Social Engineering & New employees, lacking initial security awareness training, are highly vulnerable to social engineering and phishing attacks, as they are not yet familiar with company policies and security protocols. & \textbf{High} \\
    \bottomrule
\end{tabular}
\caption{Summary of Identified Risks}
\end{table}

% --- Section 6: Recommendations ---
\section{Recommendations}
The following actions are recommended to mitigate the identified risks and improve the overall security posture of Terraform Global.

\begin{enumerate}
    \item \textbf{[Critical] Implement MFA for Email Access:} Immediately enforce mandatory Multi-Factor Authentication (MFA) for all user access to the email system. This is the single most effective control to prevent account takeovers resulting from compromised credentials.

    \item \textbf{[High] Develop and Enforce an Acceptable Use Policy (AUP):} Create a formal AUP that clearly defines the rules and expectations for employees when using company networks, devices, and data. This policy should be a mandatory part of the employee handbook and require signed acknowledgment from all staff.

    \item \textbf{[High] Integrate Security Training into Onboarding:} Develop a security awareness training module specifically for new hires. This training should be a mandatory part of the onboarding process and cover key topics such as phishing identification, password hygiene, and the new AUP.

    \item \textbf{[Informational] Maintain and Audit Firewall Configuration:} Continue to maintain the current restrictive firewall policy. It is recommended to perform periodic (e.g., quarterly) reviews of firewall rules to ensure they remain aligned with business needs and the principle of least privilege.
\end{enumerate}

\end{document}
```