```latex
\documentclass[12pt, a4paper]{article}

% Preamble: Required Packages
\usepackage[margin=1in]{geometry}
\usepackage{pifont} % For checkmarks and crosses
\usepackage{booktabs} % For professional tables
\usepackage{hyperref} % For clickable links
\usepackage{url} % For URL formatting
\usepackage{seqsplit} % To split long strings in tt font
\usepackage{graphicx} % For logos, etc.
\usepackage{xcolor} % For custom colors
\usepackage{fancyhdr} % For headers and footers
\usepackage{lastpage} % To get the total number of pages

% --- Document Setup ---

% Define colors for severity
\definecolor{criticalred}{HTML}{D73027}
\definecolor{highorange}{HTML}{F46D43}
\definecolor{mediumyellow}{HTML}{FEE090}
\definecolor{lowblue}{HTML}{4575B4}
\definecolor{infogray}{HTML}{E0E0E0}
\definecolor{darkgray}{rgb}{0.25, 0.25, 0.25}

% Hyperref setup
\hypersetup{
    colorlinks=true,
    linkcolor=blue,
    filecolor=magenta,      
    urlcolor=cyan,
    pdftitle={Cybersecurity Risk Assessment Report},
    pdfpagemode=FullScreen,
}

% Header and Footer Configuration
\pagestyle{fancy}
\fancyhf{} % Clear all header and footer fields
\fancyhead[L]{Cybersecurity Risk Assessment Report}
\fancyhead[R]{Grizzly Peak}
\fancyfoot[C]{\thepage\ of \pageref{LastPage}}
\renewcommand{\headrulewidth}{0.4pt}
\renewcommand{\footrulewidth}{0.4pt}

% --- Document Start ---

\begin{document}

% --- Title Page ---
\begin{titlepage}
    \centering
    \vspace*{1cm}
    
    \Huge
    \textbf{Cybersecurity Risk Assessment Report}
    
    \vspace{1.5cm}
    
    \Large
    Prepared for: \\
    \vspace{0.5cm}
    \textbf{Grizzly Peak}
    
    \vspace{2cm}
    
    \large
    Report Generated: \today \\
    
    \vfill
    
    \normalsize
    \textit{This report contains sensitive information and should be handled with care. Access is restricted to authorized personnel only.}
    
\end{titlepage}

\tableofcontents
\newpage

% --- Section 1: Executive Summary ---
\section{Executive Summary}
This report provides a comprehensive cybersecurity risk assessment for \textbf{Grizzly Peak}, based on an analysis of network scan data, organizational security controls, and known risks. The assessment synthesizes these data points to provide a holistic view of the organization's current security posture.

The analysis revealed a \textbf{critical risk} stemming from the combination of an externally exposed database and significant gaps in access control policies. An open MySQL database port (\texttt{3306}) was identified on an internal system (\texttt{172.16.50.20}). The database software, MySQL version 5.7.33, is \textbf{End-of-Life (EOL)} and no longer receives security updates, making it highly susceptible to exploitation.

This technical vulnerability is severely exacerbated by organizational policy gaps. The absence of Multi-Factor Authentication (MFA) on both email and sensitive data systems creates a clear pathway for an attacker to compromise credentials and gain direct access to the exposed database. While the organization has positive controls, such as mandatory security training and MFA for computer logins, the identified critical risks require immediate attention to prevent a potential data breach.

% --- Section 2: Organizational Information ---
\section{Organizational Information}
The following details were provided for the assessment.
\begin{itemize}
    \item \textbf{Organization Name:} Grizzly Peak
    \item \textbf{Email Domain:} \texttt{GrizzlyPeak.com}
    \item \textbf{Website Domain:} \url{www.GrizzlyPeak.com}
    \item \textbf{External IP Address:} \texttt{13.48.229.167}
\end{itemize}

% --- Section 3: Security Control Review ---
\section{Security Control Review}
A review of the organization's security questionnaire responses highlights key strengths and critical weaknesses in the current security control landscape. "No" answers indicate significant gaps that increase organizational risk.

\begin{table}[h!]
\centering
\caption{Security Control Questionnaire Analysis}
\label{tab:controls}
\begin{tabular}{p{7cm} c p{4.5cm}}
\toprule
\textbf{Control Question} & \textbf{Response} & \textbf{Assessment} \\
\midrule
Do you require MFA to access email? & \textcolor{criticalred}{\ding{55}} & \textbf{Critical Gap.} Email is a primary target for phishing. Lack of MFA significantly increases the risk of account takeover and business email compromise. \\
\addlinespace
Do you require MFA to log into computers? & \textcolor{lowblue}{\ding{51}} & \textbf{Good Practice.} This control strengthens endpoint security against unauthorized local access. \\
\addlinespace
Do you require MFA to access sensitive data systems? & \textcolor{criticalred}{\ding{55}} & \textbf{Critical Gap.} This directly impacts the security of critical assets. Compromised credentials could lead to a direct data breach. \\
\addlinespace
Does your organization have an employee acceptable use policy? & \textcolor{lowblue}{\ding{51}} & \textbf{Good Practice.} Establishes clear guidelines for employees, forming a baseline for security culture. \\
\addlinespace
Does your organization do security awareness training for new employees? & \textcolor{lowblue}{\ding{51}} & \textbf{Good Practice.} Ensures new hires are aware of security policies from the start. \\
\addlinespace
Does your organization do security awareness training for all employees at least once per year? & \textcolor{lowblue}{\ding{51}} & \textbf{Good Practice.} Reinforces security awareness and helps defend against evolving threats like phishing. \\
\bottomrule
\end{tabular}
\end{table}

% --- Section 4: Technical Scan Results ---
\section{Technical Scan Results}
A network scan was conducted to identify open ports and exposed services on the specified target system.

\begin{itemize}
    \item \textbf{Target IP Address:} \texttt{172.16.50.20}
\end{itemize}

\begin{table}[h!]
\centering
\caption{Open Port Analysis}
\label{tab:ports}
\begin{tabular}{c c l l p{5cm}}
\toprule
\textbf{Port} & \textbf{State} & \textbf{Service} & \textbf{Version} & \textbf{Analysis} \\
\midrule
3306/tcp & Open & MySQL & 5.7.33 & \textbf{Critical Finding.} This version of MySQL reached its official End-of-Life (EOL) in October 2023. It no longer receives security patches, leaving it vulnerable to known exploits. Exposing a database port directly to the network is a severe security risk. \\
\bottomrule
\end{tabular}
\end{table}

% --- Section 5: Correlated Risk Assessment ---
\section{Correlated Risk Assessment}
The following table synthesizes findings from the security control review, technical scans, and pre-existing risk data to provide a correlated view of the primary threats.

\begin{table}[h!]
\centering
\caption{Synthesized Risk Summary}
\label{tab:risks}
\begin{tabular}{p{3.5cm} p{5.5cm} c p{3cm}}
\toprule
\textbf{Risk Name} & \textbf{Description} & \textbf{Severity} & \textbf{Contributing Factors} \\
\midrule
\textbf{Critical Database Exposure} & A MySQL database is directly accessible on the network. The software version is End-of-Life (EOL), meaning it is unpatched against new vulnerabilities. & \textcolor{criticalred}{\textbf{9.8 (Critical)}} & \begin{itemize} \itemsep0em \item Open Port 3306 \item EOL Software (MySQL 5.7) \item No MFA on sensitive systems \end{itemize} \\
\addlinespace
\textbf{High Risk of Account Compromise} & The lack of MFA on the primary email system exposes the organization to a high likelihood of successful phishing attacks, credential theft, and business email compromise (BEC). & \textcolor{highorange}{\textbf{8.1 (High)}} & \begin{itemize} \itemsep0em \item No MFA on email \item Reliance on passwords alone \end{itemize} \\
\bottomrule
\end{tabular}
\end{table}

% --- Section 6: Recommendations ---
\section{Recommendations}
Based on the correlated risk assessment, the following actions are recommended to mitigate the identified vulnerabilities. Recommendations are prioritized by urgency.

\subsection{Priority 1: Immediate Actions (Urgent)}
\begin{enumerate}
    \item \textbf{Restrict Database Access:} Immediately apply firewall rules to block all access to port \texttt{3306} on host \texttt{172.16.50.20} from any source except for explicitly authorized application servers. The principle of least privilege must be enforced.
    \item \textbf{Implement MFA for Sensitive Systems:} Enforce mandatory MFA for all access to sensitive data systems, including databases, financial applications, and administrative interfaces. This is the most effective control to mitigate the risk of compromised credentials being used to access the database.
    \item \textbf{Implement MFA for Email:} Enforce mandatory MFA across the entire \texttt{GrizzlyPeak.com} email domain. This will drastically reduce the risk of phishing-based account takeovers.
\end{enumerate}

\subsection{Priority 2: Necessary Actions (High)}
\begin{enumerate}
    \item \textbf{Upgrade MySQL Database:} Develop and execute a plan to upgrade the MySQL 5.7.33 database to a currently supported version (e.g., MySQL 8.x). This is essential to ensure the system receives critical security patches.
    \item \textbf{Conduct a Perimeter Review:} Perform a comprehensive review of all external firewall rules to identify and close any other unnecessary open ports. Ensure that all services exposed to the internet are justified and properly secured.
\end{enumerate}

\subsection{Priority 3: Best Practice Improvements (Medium)}
\begin{enumerate}
    \item \textbf{Review and Enhance Logging:} Ensure that logging is enabled on critical systems, including the database server and firewalls. Logs should be monitored for suspicious access patterns.
\end{enumerate}

\end{document}
```