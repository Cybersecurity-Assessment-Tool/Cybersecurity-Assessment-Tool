```latex
\documentclass[12pt]{article}

% Preamble: Required Packages
\usepackage[margin=1in]{geometry}
\usepackage{pifont} % For checkmarks and crosses
\usepackage{booktabs} % For professional tables
\usepackage{hyperref} % For clickable links
\usepackage{url} % For formatting URLs
\usepackage{seqsplit} % For splitting long strings to prevent overflow
\usepackage{graphicx}
\usepackage{xcolor}

% Document Metadata
\title{Cybersecurity Posture Assessment Report}
\author{Cybersecurity Analysis Division}
\date{\today}

% Hyperref Setup
\hypersetup{
    colorlinks=true,
    linkcolor=blue,
    filecolor=magenta,      
    urlcolor=cyan,
    pdftitle={Cybersecurity Posture Assessment Report},
    pdfpagemode=FullScreen,
}

\begin{document}

\maketitle
\thispagestyle{empty}
\newpage

\tableofcontents
\newpage

\section{Executive Summary}

This report details the findings of a cybersecurity posture assessment conducted for \textbf{Firebrand Media}. The assessment combined a review of organizational security controls via a questionnaire, a technical network scan of the external perimeter, and an analysis of pre-existing risk data.

The key findings indicate a significant disparity between the organization's external network security and its internal security maturity. The external network scan of the target IP address revealed no open ports, suggesting a robust firewall configuration that effectively limits the external attack surface. This is a strong positive finding.

However, the security control review identified several critical and high-risk gaps. The most severe issues are the lack of Multi-Factor Authentication (MFA) for email and sensitive data systems, and a complete absence of a formal security awareness training program and an employee acceptable use policy. These deficiencies expose \textbf{Firebrand Media} to a high risk of account compromise, data breaches, and successful social engineering attacks.

Immediate remediation should focus on implementing MFA across all critical systems and establishing a foundational security awareness and policy framework to mitigate the most pressing human-centric risks.

\section{Organizational Information}

The following information was provided for the assessment.

\begin{itemize}
    \item \textbf{Organization Name:} Firebrand Media
    \item \textbf{Email Domain:} \texttt{FirebrandMedia.net}
    \item \textbf{Website Domain:} \seqsplit{\url{www.FirebrandMedia.net}}
    \item \textbf{External IP Address:} \texttt{58.124.83.69}
\end{itemize}

\section{Security Control Review}

The following table summarizes the organization's responses to the security controls questionnaire. A green checkmark (\ding{51}) indicates a positive control is in place, while a red cross (\ding{55}) indicates a control gap that introduces risk.

\begin{table}[h!]
\centering
\begin{tabular}{p{0.7\textwidth} c}
\toprule
\textbf{Control Question} & \textbf{Response} \\
\midrule
Do you require MFA to access email? & \textcolor{red}{\ding{55}} \\
Do you require MFA to log into computers? & \textcolor{green}{\ding{51}} \\
Do you require MFA to access sensitive data systems? & \textcolor{red}{\ding{55}} \\
Does your organization have an employee acceptable use policy? & \textcolor{red}{\ding{55}} \\
Does your organization do security awareness training for new employees? & \textcolor{red}{\ding{55}} \\
Does your organization do security awareness training for all employees at least once per year? & \textcolor{red}{\ding{55}} \\
\bottomrule
\end{tabular}
\caption{Security Controls Questionnaire Results}
\label{tab:controls}
\end{table}

The review reveals critical weaknesses in authentication security for email and sensitive data, as well as a complete lack of foundational employee-facing security policies and training.

\section{Technical Scan Results}

A network port scan was conducted to assess the external attack surface of the organization's provided IP address.

\begin{itemize}
    \item \textbf{Scan Target:} \texttt{[Target IP]}
    \item \textbf{Scan Date:} \today
\end{itemize}

\subsection{Summary of Findings}
No open TCP or UDP ports were detected on the target system. All ports scanned appeared to be in a \texttt{closed} or \texttt{filtered} state.

\subsection{Interpretation}
This result is a strong indicator of a well-configured perimeter firewall. It suggests that the organization is not exposing any unnecessary services to the public internet, which significantly reduces the risk of direct network-based attacks. This is a positive security finding. However, it is recommended to confirm that the target IP was correct and the system was operational during the scan to ensure the validity of this result.

\section{Risk Assessment}

The following table synthesizes risks identified from the security control review, technical scan, and pre-existing vulnerability data. As no pre-existing risks were provided, this assessment is based solely on the data collected for this report.

\begin{table}[h!]
\centering
\begin{tabular}{p{0.25\textwidth} p{0.15\textwidth} p{0.5\textwidth}}
\toprule
\textbf{Risk Name} & \textbf{Severity} & \textbf{Overview} \\
\midrule
\textbf{Lack of Multi-Factor Authentication (MFA)} & \textbf{Critical} & The absence of MFA on email and sensitive data systems creates a high risk of account compromise through phishing or credential stuffing. A single compromised password could lead to a significant data breach. \\
\addlinespace
\textbf{Deficient Security Awareness Program} & \textbf{High} & Without security awareness training, employees are significantly more vulnerable to social engineering and phishing attacks. This represents a major gap in the organization's human firewall. \\
\addlinespace
\textbf{Insufficient Security Policies} & \textbf{High} & The lack of an Acceptable Use Policy (AUP) leads to an undefined security environment. Employees may unintentionally misuse company assets or engage in risky behavior, and the organization lacks a formal basis for enforcement. \\
\bottomrule
\end{tabular}
\caption{Synthesized Risk Summary}
\label{tab:risks}
\end{table}

\section{Recommendations}

Based on the analysis, the following actions are recommended to improve the cybersecurity posture of \textbf{Firebrand Media}. Recommendations are prioritized by severity.

\begin{enumerate}
    \item \textbf{(Critical) Implement Comprehensive MFA:}
    \begin{itemize}
        \item Immediately enforce MFA for all user accounts on the primary email platform (e.g., Microsoft 365, Google Workspace).
        \item Conduct an inventory of all systems that process or store sensitive data and prioritize the rollout of MFA for access to these systems.
    \end{itemize}

    \item \textbf{(High) Establish a Security Awareness Program:}
    \begin{itemize}
        \item Develop and implement a mandatory security awareness training module for all new employees as part of the onboarding process.
        \item Schedule and conduct annual, mandatory security training for all staff. This training should cover key topics such as phishing identification, password security, and safe browsing habits.
    \end{itemize}

    \item \textbf{(High) Develop and Implement Foundational Security Policies:}
    \begin{itemize}
        \item Draft, approve, and publish an Employee Acceptable Use Policy (AUP). This policy should clearly define the rules and responsibilities for using company information technology assets.
        \item Require all current employees and new hires to read and formally acknowledge the AUP.
    \end{itemize}
    
    \item \textbf{(Informational) Verify Network Scan Target:}
    \begin{itemize}
        \item Internally confirm that the target IP address \texttt{[Target IP]} was the correct external endpoint and that the associated system was online and operational during the scan. This will validate the positive finding of a secure network perimeter.
    \end{itemize}
\end{enumerate}

\end{document}
```