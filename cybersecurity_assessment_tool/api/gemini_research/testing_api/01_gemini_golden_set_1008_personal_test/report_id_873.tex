```latex
\documentclass[12pt]{article}

% Preamble: Required Packages
\usepackage[margin=1in]{geometry}
\usepackage{pifont} % For \ding
\usepackage{booktabs} % For professional tables (\toprule, \midrule, \bottomrule)
\usepackage{hyperref} % For clickable links
\usepackage{url} % For URL formatting
\usepackage{seqsplit} % To split long strings in \texttt
\usepackage{graphicx}
\usepackage[utf8]{inputenc}
\usepackage{xcolor}
\usepackage{fancyhdr}

% --- Document Setup ---
\hypersetup{
    colorlinks=true,
    linkcolor=blue,
    filecolor=magenta,
    urlcolor=cyan,
    pdftitle={Cybersecurity Posture Assessment Report},
    pdfauthor={Cybersecurity Analysis Division},
}

% Custom Commands for Readability
\newcommand{\yes}{\textcolor{green}{\ding{51}}}
\newcommand{\no}{\textcolor{red}{\ding{55}}}

% --- Header and Footer ---
\pagestyle{fancy}
\fancyhf{} % clear all header and footer fields
\fancyhead[L]{Cybersecurity Posture Assessment Report}
\fancyhead[R]{For: White Label}
\fancyfoot[C]{\thepage}

% --- Document Body ---
\begin{document}

% --- Title Page ---
\begin{titlepage}
    \centering
    \vspace*{1cm}
    \includegraphics[width=0.4\textwidth]{example-image-a} % Placeholder for company logo
    \vfill
    \Huge\bfseries
    Cybersecurity Posture Assessment Report
    \vspace{1cm}
    \Large
    \textbf{Prepared for:}\\
    White Label
    \vspace{2cm}
    \large
    \textbf{Date of Report:}\\
    \today
    \vfill
    \textbf{Generated By:}\\
    Cybersecurity Analysis Division
\end{titlepage}

\newpage
\tableofcontents
\newpage

% --- Section 1: Executive Overview ---
\section{Executive Overview}
This report provides a comprehensive assessment of the cybersecurity posture for \textbf{White Label}. The analysis is based on a correlation of network scan data, a review of organizational security controls, and pre-existing risk information.

The assessment has identified several critical and high-severity risks that require immediate attention. The most severe finding is a network service on internal host \texttt{10.5.5.5} broadcasting the title \textbf{"TOP SECRET DB"} on port \texttt{8080}. This indicates a high probability of sensitive data exposure on the internal network.

This technical vulnerability is compounded by significant gaps in organizational security controls. The lack of mandatory Multi-Factor Authentication (MFA) for email and computer access presents a critical risk, substantially increasing the likelihood of account compromise. Furthermore, the absence of a formal security awareness training program for employees leaves the organization highly susceptible to social engineering and phishing attacks.

It is crucial to note that the technical finding on port \texttt{8080} directly contradicts a pre-existing risk assessment which incorrectly labeled the port as secure. This report supersedes that previous assessment and provides actionable recommendations to mitigate these identified risks and strengthen the organization's overall security posture.

% --- Section 2: Organizational Information ---
\section{Organizational Information}
The following information was provided for the assessment.
\begin{table}[h!]
\centering
\begin{tabular}{@{}ll@{}}
\toprule
\textbf{Attribute} & \textbf{Value} \\ \midrule
Organization Name  & White Label \\
Email Domain       & \texttt{WhiteLabel.net} \\
Website Domain     & \url{www.WhiteLabel.net} \\
External IP Address & \texttt{106.3.119.106} \\ \bottomrule
\end{tabular}
\caption{Client Organizational Details}
\end{table}

% --- Section 3: Security Control Review ---
\section{Security Control Review}
A review of the organization's security controls was conducted via a questionnaire. The responses highlight significant areas for improvement, particularly concerning access control and employee security awareness.

\begin{table}[h!]
\centering
\begin{tabular}{@{}p{0.6\linewidth}cp{0.2\linewidth}@{}}
\toprule
\textbf{Control Question} & \textbf{Response} & \textbf{Assessment} \\ \midrule
Do you require MFA to access email? & \no & Critical Gap \\
Do you require MFA to log into computers? & \no & Critical Gap \\
Do you require MFA to access sensitive data systems? & \yes & Good Practice \\
Does your organization have an employee acceptable use policy? & \yes & Good Practice \\
Does your organization do security awareness training for new employees? & \no & High Risk \\
Does your organization do security awareness training for all employees at least once per year? & \no & High Risk \\ \bottomrule
\end{tabular}
\caption{Security Control Questionnaire Results}
\end{table}

% --- Section 4: Technical Scan Results ---
\section{Technical Scan Results}
A network scan was performed to identify open ports and exposed services on the target system.

\subsection{Host: 10.5.5.5}
The scan identified one host as active and responsive.
\begin{itemize}
    \item \textbf{Status:} Up
    \item \textbf{Open Ports Found:} 1
\end{itemize}

\begin{table}[h!]
\centering
\begin{tabular}{@{}llll@{}}
\toprule
\textbf{Port} & \textbf{State} & \textbf{Service Details} \\ \midrule
8080/tcp & Open & \textbf{HTTP Title:} \texttt{TOP SECRET DB} \\ \bottomrule
\end{tabular}
\caption{Open Ports and Services on Host 10.5.5.5}
\end{table}

\paragraph{Analysis:} The presence of an open port with a service title of \texttt{"TOP SECRET DB"} is a \textbf{critical finding}. This strongly suggests that a sensitive database or application interface is exposed on the network without adequate access controls. This finding directly invalidates the pre-existing risk data which claimed this port was secure.

% --- Section 5: Risk Assessment ---
\section{Risk Assessment}
The following table summarizes the key risks identified through the correlation of technical findings and organizational control gaps. These risks should be prioritized for immediate remediation.

\begin{table}[h!]
\centering
\begin{tabular}{@{}lp{0.5\linewidth}l@{}}
\toprule
\textbf{Risk Title} & \textbf{Description} & \textbf{Severity} \\ \midrule
\textbf{Critical Information Disclosure} & An internal service on port 8080 is advertising itself as a "TOP SECRET DB". This presents a severe risk of unauthorized access to sensitive data. & \textbf{Critical} \\
\textbf{Lack of MFA} & The absence of MFA for email and computer access makes user accounts highly vulnerable to compromise via phishing or password spraying attacks. & \textbf{Critical} \\
\textbf{Inadequate Security Training} & Without security awareness training, employees are more likely to fall victim to social engineering, inadvertently granting attackers access to the network. & \textbf{High} \\
\bottomrule
\end{tabular}
\caption{Summary of Identified Risks}
\end{table}

% --- Section 6: Recommendations ---
\section{Recommendations}
The following actionable recommendations are provided to mitigate the identified risks.

\subsection{Remediation for: Critical Information Disclosure}
\begin{itemize}
    \item \textbf{Immediate Action:} Apply a firewall rule to block all access to TCP port 8080 on host \texttt{10.5.5.5} from any non-essential source.
    \item \textbf{Short-Term:} Conduct an immediate investigation to identify the system and data associated with this service. Determine why it was exposed and if a data breach has occurred.
    \item \textbf{Long-Term:} If the service is required, reconfigure it to use strong authentication, enforce the principle of least privilege, and enable encryption (HTTPS/TLS).
\end{itemize}

\subsection{Remediation for: Lack of MFA}
\begin{itemize}
    \item \textbf{Immediate Action:} Procure and configure an MFA solution compatible with your existing identity provider (e.g., Microsoft 365, Google Workspace).
    \item \textbf{Short-Term:} Enforce MFA for all administrative accounts and for access to all externally-facing services, especially email.
    \item \textbf{Long-Term:} Develop a phased rollout plan to enforce MFA for all employee computer logins and access to all internal systems containing sensitive data.
\end{itemize}

\subsection{Remediation for: Inadequate Security Training}
\begin{itemize}
    \item \textbf{Immediate Action:} Select and procure a reputable security awareness training platform.
    \item \textbf{Short-Term:} Enroll all current employees in a foundational security awareness training module. Make this training a mandatory part of the onboarding process for all new hires.
    \item \textbf{Long-Term:} Establish a continuous training program that includes annual refresher courses and regular phishing simulations to reinforce security best practices.
\end{itemize}

\end{document}
```