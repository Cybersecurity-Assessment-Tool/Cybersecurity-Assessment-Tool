```latex
\documentclass[12pt]{article}

% Preamble: Required Packages
\usepackage[margin=1in]{geometry}
\usepackage{pifont} % For checkmarks and crosses
\usepackage{booktabs} % For professional tables
\usepackage{hyperref} % For clickable links
\usepackage{url} % For URL formatting
\usepackage{seqsplit} % To split long strings in tt font
\usepackage{datetime} % To format the date

% Document Metadata and Hyperlink Setup
\hypersetup{
    colorlinks=true,
    linkcolor=black,
    urlcolor=blue,
    pdftitle={Cybersecurity Assessment Report},
    pdfauthor={Cybersecurity Analyst},
    pdfsubject={Security Posture Analysis},
    pdfkeywords={Cybersecurity, Risk Assessment, Network Scan}
}

% Define checkmark and cross symbols for convenience
\newcommand{\cmark}{\ding{51}}
\newcommand{\xmark}{\ding{55}}

% Document Start
\begin{document}

% --- Title Page ---
\begin{titlepage}
    \centering
    \vspace*{\stretch{1.0}}
    \Huge{\textbf{Cybersecurity Assessment Report}}
    \vspace{0.5cm}
    \LARGE{\textbf{Crestview Analytics}}
    \vspace{1.5cm}
    \large{Generated: \today}
    \vspace{1.5cm}
    \large{Prepared by:} \\
    \large{Expert Cybersecurity Analyst}
    \vspace*{\stretch{2.0}}
    \small{\textit{This report contains sensitive information and should be handled with care.}}
\end{titlepage}

% --- Table of Contents ---
\tableofcontents
\newpage

% --- Section 1: Executive Summary ---
\section{Executive Summary}
This report provides a comprehensive cybersecurity assessment for Crestview Analytics, synthesizing data from network scans, organizational questionnaires, and pre-existing risk registers. The analysis reveals several critical and high-risk security gaps that require immediate attention.

The primary findings indicate a significant risk of unauthorized access due to insufficient access controls and the exposure of high-risk services. Specifically, the lack of Multi-Factor Authentication (MFA) for email and computer logins, combined with an open Remote Desktop Protocol (RDP) port on an internal server (\texttt{10.10.10.51}), creates a direct pathway for attackers to compromise systems. This technical finding expands upon a previously identified risk, suggesting a systemic issue with RDP exposure within the network.

Furthermore, foundational security practices are lacking. The absence of an employee acceptable use policy and security training for new hires weakens the organization's human firewall, making it more susceptible to social engineering and phishing attacks.

Immediate remediation should focus on securing the exposed RDP service and implementing mandatory MFA across all critical systems. Subsequently, developing and enforcing robust security policies and training programs is essential to building a resilient, long-term security posture.

\newpage

% --- Section 2: Organizational Information ---
\section{Organizational Information}
The following details were provided for the assessment. This information is used to establish the context and scope of the review.

\begin{table}[h!]
\centering
\begin{tabular}{@{}ll@{}}
\toprule
\textbf{Attribute} & \textbf{Value} \\
\midrule
Organization Name & Crestview Analytics \\
Email Domain & \seqsplit{\texttt{CrestviewAnalytics.org}} \\
Website Domain & \seqsplit{\url{www.CrestviewAnalytics.org}} \\
External IP Address & \seqsplit{\texttt{112.6.154.5}} \\
\bottomrule
\end{tabular}
\caption{Client Organizational Details}
\end{table}

% --- Section 3: Security Control Review ---
\section{Security Control Review}
The following table summarizes the organization's self-reported security controls based on the provided questionnaire. Items marked with \xmark{} represent significant gaps in the security framework and are correlated with findings in the risk assessment section.

\begin{table}[h!]
\centering
\begin{tabular}{@{}p{0.6\textwidth} c p{0.2\textwidth}@{}}
\toprule
\textbf{Control Question} & \textbf{Response} & \textbf{Assessment} \\
\midrule
Do you require MFA to access email? & \xmark{} & \textbf{Critical Gap} \\
Do you require MFA to log into computers? & \xmark{} & \textbf{Critical Gap} \\
Do you require MFA to access sensitive data systems? & \cmark{} & Good Practice \\
\addlinespace
Does your organization have an employee acceptable use policy? & \xmark{} & \textbf{High Risk} \\
Does your organization do security awareness training for new employees? & \xmark{} & \textbf{High Risk} \\
Does your organization do security awareness training for all employees at least once per year? & \cmark{} & Good Practice \\
\bottomrule
\end{tabular}
\caption{Security Questionnaire Analysis}
\end{table}

\newpage

% --- Section 4: Technical Scan Results ---
\section{Technical Scan Results}
A network scan was performed to identify active services and potential vulnerabilities on the target system.

\begin{itemize}
    \item \textbf{Target IP Address:} \texttt{10.10.10.51}
    \item \textbf{Scan Status:} Host is up.
\end{itemize}

The following table details the open ports discovered on the target host.

\begin{table}[h!]
\centering
\begin{tabular}{@{}llll@{}}
\toprule
\textbf{Port} & \textbf{State} & \textbf{Service Name} & \textbf{Analysis} \\
\midrule
3389/tcp & open & ms-wbt-server & Remote Desktop Protocol (RDP) \\
\bottomrule
\end{tabular}
\caption{Open Ports on \texttt{10.10.10.51}}
\end{table}

\subsection{Analysis of Findings}
The scan identified that port \textbf{3389 (RDP)} is open. RDP is a common vector for cyberattacks, including brute-force credential attacks and exploitation of vulnerabilities (e.g., BlueKeep). Exposing RDP directly on a network without mitigating controls like a VPN or bastion host is a critical security risk. This finding is especially severe when combined with the lack of MFA for computer logins, as a single compromised password could lead to a full system compromise.

This discovery on \texttt{10.10.10.51} corroborates a pre-existing risk identified on \texttt{10.10.10.50}, indicating a pattern of insecure RDP configuration within the organization.

\newpage

% --- Section 5: Consolidated Risk Assessment ---
\section{Consolidated Risk Assessment}
The following table synthesizes findings from the security control review, technical scan, and pre-existing risk data into a prioritized list of security risks.

\begin{table}[h!]
\centering
\begin{tabular}{@{}p{0.1\textwidth} p{0.25\textwidth} p{0.45\textwidth} p{0.1\textwidth}@{}}
\toprule
\textbf{Risk ID} & \textbf{Risk Title} & \textbf{Description} & \textbf{Severity} \\
\midrule
\textbf{RISK-001} & Systemic RDP Exposure & RDP is exposed on multiple internal servers (\texttt{10.10.10.50}, \texttt{10.10.10.51}). This service is a primary target for attackers seeking to gain unauthorized remote access to the network. & \textbf{Critical} \\
\addlinespace
\textbf{RISK-002} & Insufficient Multi-Factor Authentication & Lack of MFA on email and computer logins significantly increases the risk of account compromise via phishing or password spraying. This directly exacerbates RISK-001. & \textbf{Critical} \\
\addlinespace
\textbf{RISK-003} & Inadequate Security Policies and Training & The absence of an Acceptable Use Policy and security training for new hires leaves the organization vulnerable to insider threats and social engineering attacks. & \textbf{High} \\
\bottomrule
\end{tabular}
\caption{Summary of Key Security Risks}
\end{table}

% --- Section 6: Recommendations ---
\section{Recommendations}
Based on the consolidated risk assessment, the following prioritized actions are recommended to mitigate the identified vulnerabilities and improve the overall security posture of Crestview Analytics.

\subsection{Immediate Priority (Critical Risks)}
\begin{enumerate}
    \item \textbf{Remediate RDP Exposure (RISK-001):}
    \begin{itemize}
        \item Immediately implement firewall rules to block all external access to TCP port 3389 on all systems, including \texttt{10.10.10.50} and \texttt{10.10.10.51}.
        \item For necessary administrative access, implement a secure solution such as a Virtual Private Network (VPN) or a bastion host with strong access controls and logging.
    \end{itemize}

    \item \textbf{Implement Mandatory MFA (RISK-002):}
    \begin{itemize}
        \item Prioritize the deployment of MFA for all users on the email system. This is the most critical step to defend against phishing and business email compromise.
        \item Enforce MFA for all computer logins, especially for accounts with administrative privileges.
    \end{itemize}
\end{enumerate}

\subsection{Short-Term Priority (High Risks)}
\begin{enumerate}
    \setcounter{enumi}{2} % Continue numbering from the previous list
    \item \textbf{Develop and Implement Security Policies (RISK-003):}
    \begin{itemize}
        \item Draft and formally approve an employee Acceptable Use Policy (AUP) that clearly defines rules for using company assets, data handling, and internet usage.
        \item Ensure all current employees read and acknowledge the new policy.
    \end{itemize}

    \item \textbf{Enhance Security Training Program (RISK-003):}
    \begin{itemize}
        \item Institute a mandatory security awareness training module as part of the onboarding process for all new employees.
        \item This training should cover key topics such as phishing identification, password hygiene, and the new AUP.
    \end{itemize}
\end{enumerate}

\end{document}
```