```latex
\documentclass[12pt]{article}

% Preamble: Required Packages
\usepackage[margin=1in]{geometry}
\usepackage{pifont} % For checkmarks and crosses
\usepackage{booktabs} % For professional tables
\usepackage{hyperref} % For hyperlinks
\usepackage{url} % For URL formatting
\usepackage{seqsplit} % For splitting long strings
\usepackage{graphicx}
\usepackage{xcolor}
\usepackage{fancyhdr}
\usepackage{lastpage}

% --- Document Setup ---
\hypersetup{
    colorlinks=true,
    linkcolor=blue,
    filecolor=magenta,      
    urlcolor=cyan,
    pdftitle={Cybersecurity Posture Report},
    pdfauthor={Cybersecurity Analysis Cell},
    pdfsubject={Security Assessment},
    pdfkeywords={Security, Report, Analysis},
}

% --- Header and Footer ---
\pagestyle{fancy}
\fancyhf{} % Clear all header and footer fields
\fancyhead[L]{Cybersecurity Posture Report}
\fancyhead[R]{Sterling Silver}
\fancyfoot[C]{\thepage\ of \pageref{LastPage}}
\renewcommand{\headrulewidth}{0.4pt}
\renewcommand{\footrulewidth}{0.4pt}

% --- Document Start ---
\begin{document}

% --- Title Page ---
\begin{titlepage}
    \centering
    \vspace*{1cm}
    \Huge{\textbf{Cybersecurity Posture Report}}
    \vspace{1.5cm}
    \Large{\textbf{Prepared for:}} \\
    \vspace{0.5cm}
    \Large{Sterling Silver}
    \vspace{2cm}
    \large{\textbf{Date of Report:}} \\
    \vspace{0.5cm}
    \large{\today}
    \vfill
    \large{This report contains a comprehensive analysis of the organization's security posture based on technical scans, organizational data, and pre-existing risk assessments. The findings and recommendations herein are confidential.}
\end{titlepage}

\tableofcontents
\newpage

% --- Section 1: Executive Summary ---
\section{Executive Summary}
This report provides a detailed assessment of the cybersecurity posture for Sterling Silver. The analysis is based on a correlation of network scan data, a review of organizational security controls, and an evaluation of known risks.

The overall security posture is determined to be \textbf{CRITICAL}. Several significant, high-impact vulnerabilities and security gaps were identified that expose the organization to immediate and severe threats.

Key findings include:
\begin{itemize}
    \item \textbf{Critically Vulnerable External Service:} A public-facing FTP server is running a version of \texttt{vsftpd} (2.3.4) with a known, easily exploitable remote code execution vulnerability (CVE-2011-2523).
    \item \textbf{Systemic Lack of Multi-Factor Authentication (MFA):} MFA is not enforced for email, computer logins, or access to sensitive data systems. This represents a critical failure in access control and dramatically increases the risk of account compromise.
    \item \textbf{Inadequate Security Awareness:} The organization does not provide security awareness training to new or existing employees, making it highly susceptible to social engineering attacks such as phishing.
    \item \textbf{Insecure Server Configuration:} The identified FTP server permits anonymous logins, allowing unauthorized users to access the system, which could lead to data exfiltration or malware implantation.
\end{itemize}

Immediate and decisive action is required to remediate these issues. The recommendations outlined in this report provide a clear path forward to significantly improve the organization's defensive capabilities and reduce its attack surface.

\newpage

% --- Section 2: Organizational Information ---
\section{Organizational Information}
The following details were provided for the assessment. This information forms the basis of the analysis and helps define the scope of the review.

\begin{tabular}{@{}ll}
    \toprule
    \textbf{Attribute} & \textbf{Value} \\
    \midrule
    Organization Name & \textbf{Sterling Silver} \\
    Email Domain & \texttt{SterlingSilver.com} \\
    Website Domain & \url{www.SterlingSilver.com} \\
    External IP Address & \texttt{204.58.56.47} \\
    \bottomrule
\end{tabular}

% --- Section 3: Security Control Review ---
\section{Security Control Review}
A review of organizational security controls was conducted via a questionnaire. The responses highlight critical gaps in foundational security practices. A "No" response indicates a missing control that elevates organizational risk.

\begin{tabular}{@{}p{0.75\linewidth}c}
    \toprule
    \textbf{Control Question} & \textbf{Response} \\
    \midrule
    Do you require MFA to access email? & \textcolor{red}{\ding{55}} \\
    Do you require MFA to log into computers? & \textcolor{red}{\ding{55}} \\
    Do you require MFA to access sensitive data systems? & \textcolor{red}{\ding{55}} \\
    Does your organization have an employee acceptable use policy? & \textcolor{green}{\ding{51}} \\
    Does your organization do security awareness training for new employees? & \textcolor{red}{\ding{55}} \\
    Does your organization do security awareness training for all employees at least once per year? & \textcolor{red}{\ding{55}} \\
    \bottomrule
\end{tabular}

\subsection*{Analysis}
The complete absence of Multi-Factor Authentication (MFA) is a critical deficiency. It leaves all accounts vulnerable to takeover via stolen or weak credentials. Furthermore, the lack of a security awareness training program negates the value of the acceptable use policy, as employees are not equipped to recognize or respond to common cyber threats like phishing.

\newpage

% --- Section 4: Technical Scan Results ---
\section{Technical Scan Results}
An external network scan was performed on the target IP address \texttt{10.0.0.15}. The scan identified one host with an open port exposing a vulnerable service.

\subsection*{Host: 10.0.0.15}
\begin{itemize}
    \item \textbf{Status:} Up
    \item \textbf{Open Ports Found:} 1
\end{itemize}

\begin{tabular}{@{}lllll}
    \toprule
    \textbf{Port} & \textbf{State} & \textbf{Service} & \textbf{Version} & \textbf{Details} \\
    \midrule
    21/tcp & Open & FTP & vsftpd 2.3.4 & \textbf{Anonymous FTP login allowed.} \\
    \bottomrule
\end{tabular}

\subsection*{Analysis}
The scan revealed a highly critical finding. The FTP service is running \texttt{vsftpd version 2.3.4}, which is known to be vulnerable to a backdoor command execution vulnerability (CVE-2011-2523). This allows an attacker to gain a remote shell on the server with minimal effort.

Compounding this issue, the server is configured to allow anonymous FTP logins. This misconfiguration permits any external entity to connect to the server, which could be used to exfiltrate data or upload malicious files. The combination of a vulnerable version and insecure configuration presents an immediate and severe threat to the organization.

\newpage

% --- Section 5: Risk Assessment ---
\section{Risk Assessment}
The following table synthesizes findings from the security control review, technical scan, and pre-existing risk data into a prioritized list.

\begin{tabular}{@{}p{0.1\linewidth}p{0.2\linewidth}p{0.45\linewidth}p{0.15\linewidth}}
    \toprule
    \textbf{Risk ID} & \textbf{Risk Title} & \textbf{Description} & \textbf{Severity} \\
    \midrule
    R-01 & Vulnerable FTP Service & The public-facing FTP server is running vsftpd 2.3.4, which contains a critical backdoor vulnerability (CVE-2011-2523). & \textbf{Critical} \\
    \addlinespace
    R-02 & Lack of Multi-Factor Authentication & MFA is not enforced for email, computers, or sensitive data systems, exposing all accounts to takeover. & \textbf{Critical} \\
    \addlinespace
    R-03 & Anonymous FTP Access Enabled & The FTP server is configured to allow anonymous logins, permitting unauthorized access to the file system. & \textbf{High} \\
    \addlinespace
    R-04 & No Security Awareness Training & Employees do not receive training, making them vulnerable to phishing and other social engineering attacks. & \textbf{High} \\
    \addlinespace
    R-05 & Outdated Windows Policy & Pre-existing risk: Workstations are running Windows 7, which is an unsupported and insecure operating system. & \textbf{Medium} \\
    \bottomrule
\end{tabular}

% --- Section 6: Recommendations ---
\section{Recommendations}
The following actions are recommended to mitigate the identified risks. They are prioritized based on severity and impact.

\subsection*{R-01: Vulnerable FTP Service (Critical)}
\begin{itemize}
    \item \textbf{Immediate Action:} Take the FTP server offline immediately to prevent exploitation.
    \item \textbf{Short-Term:} If the FTP service is business-critical, upgrade the \texttt{vsftpd} software to the latest stable version. If it is not required, the service should be permanently disabled and the port closed on the firewall.
    \item \textbf{Long-Term:} Migrate all file transfer workflows to a secure protocol such as SFTP (SSH File Transfer Protocol) or FTPS (FTP over SSL/TLS).
\end{itemize}

\subsection*{R-02: Lack of Multi-Factor Authentication (Critical)}
\begin{itemize}
    \item \textbf{Immediate Action:} Begin a phased rollout of MFA across the organization.
    \item \textbf{Priority 1:} Enforce MFA for all external access, including email (e.g., Office 365, G Suite) and VPN, as well as for all administrative accounts.
    \item \textbf{Priority 2:} Extend MFA enforcement to all sensitive data systems and, subsequently, to all employee computer logins.
\end{itemize}

\subsection*{R-03: Anonymous FTP Access Enabled (High)}
\begin{itemize}
    \item \textbf{Immediate Action:} As part of the remediation for R-01, ensure that anonymous login is explicitly disabled in the FTP server configuration.
    \item \textbf{Follow-Up:} Audit the FTP server's file system for any unauthorized or suspicious files that may have been uploaded.
\end{itemize}

\subsection*{R-04: No Security Awareness Training (High)}
\begin{itemize}
    \item \textbf{Short-Term:} Procure and implement a security awareness training program for all employees. This should be mandatory for all new hires.
    \item \textbf{Long-Term:} Conduct annual refresher training and periodic phishing simulations to build a culture of security and keep employee skills sharp.
\end{itemize}

\subsection*{R-05: Outdated Windows Policy (Medium)}
\begin{itemize}
    \item \textbf{Action:} Execute the existing plan to upgrade all Windows 7 workstations to a modern, supported operating system like Windows 10 or 11. Prioritize systems used by employees who handle sensitive data.
\end{itemize}

\end{document}
```