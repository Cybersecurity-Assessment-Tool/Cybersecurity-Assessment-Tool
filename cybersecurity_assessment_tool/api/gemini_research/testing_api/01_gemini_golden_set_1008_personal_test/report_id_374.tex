```latex
\documentclass[12pt, a4paper]{article}

% Preamble: Required Packages
\usepackage[margin=1in]{geometry}
\usepackage{pifont} % For checkmarks and crosses
\usepackage{booktabs} % For professional tables
\usepackage{hyperref} % For clickable links
\usepackage{url} % For URL formatting
\usepackage{seqsplit} % For splitting long strings
\usepackage{xcolor} % For colors
\usepackage{graphicx} % For potential logos/images
\usepackage{lastpage} % To get total page number
\usepackage{fancyhdr} % For header/footer

% --- Document Metadata and Hyperref Setup ---
\hypersetup{
    colorlinks=true,
    linkcolor=blue,
    filecolor=magenta,      
    urlcolor=cyan,
    pdftitle={Cybersecurity Assessment Report},
    pdfauthor={Cybersecurity Analyst},
    pdfsubject={Security Analysis},
    pdfkeywords={Cybersecurity, Pentest, Nmap, Risk Assessment},
    pdftoolbar=true,
}

% --- Custom Colors for Severity ---
\definecolor{sev_critical}{HTML}{940000}
\definecolor{sev_high}{HTML}{D14000}
\definecolor{sev_medium}{HTML}{E8A100}
\definecolor{sev_low}{HTML}{3584E4}

% --- Header and Footer Configuration ---
\pagestyle{fancy}
\fancyhf{} % Clear all header and footer fields
\fancyhead[L]{Cybersecurity Assessment Report}
\fancyhead[R]{Pioneer Pulse}
\fancyfoot[C]{\thepage\ of \pageref{LastPage}}
\renewcommand{\headrulewidth}{0.4pt}
\renewcommand{\footrulewidth}{0.4pt}

% --- Document Start ---
\begin{document}

% --- Title Page ---
\begin{titlepage}
    \centering
    \vspace*{1cm}
    
    \includegraphics[width=0.3\textwidth]{example-image-a} % Placeholder for company logo
    
    \vspace{1.5cm}
    
    {\Huge\bfseries Cybersecurity Assessment Report\par}
    
    \vspace{1.5cm}
    
    {\Large Prepared for:\par}
    {\Large\bfseries Pioneer Pulse\par}
    
    \vspace{2cm}
    
    {\large\today\par}
    
    \vfill
    
    {\large Generated by:\par}
    {\large\bfseries Cybersecurity Analyst\par}
    
    \vspace{0.5cm}
    
    \textit{This document contains sensitive information. Distribution is restricted.}
    
\end{titlepage}

\newpage
\tableofcontents
\newpage

% --- Section 1: Executive Summary ---
\section{Executive Summary}

This report details the findings of a cybersecurity assessment conducted for Pioneer Pulse. The assessment combined an automated network scan, a review of organizational security controls via a questionnaire, and an analysis of pre-existing risks.

The analysis revealed several critical and high-risk vulnerabilities that require immediate attention. The most severe findings include:

\begin{itemize}
    \item \textbf{Critical Vulnerable Service:} An externally facing FTP server (\texttt{10.0.0.15}) was identified running a dangerously outdated version of \texttt{vsftpd} (2.3.4). This version is known to contain a critical backdoor vulnerability (CVE-2011-2523). Furthermore, the server is configured to allow anonymous logins, posing a severe risk of data breach and unauthorized system access.
    \item \textbf{Critical Control Gap:} Multi-Factor Authentication (MFA) is not enforced for accessing sensitive data systems. This significantly increases the risk of unauthorized access to critical company and customer data through credential theft or compromise.
    \item \textbf{High-Risk Process Gap:} The organization lacks a formal security awareness training program for new employees during their onboarding process. This exposes the organization to a higher risk of social engineering, phishing, and policy violations.
\end{itemize}

This report provides a consolidated view of these risks and offers prioritized, actionable recommendations to mitigate them effectively. We strongly advise that the recommendations in Section \ref{sec:recommendations} are addressed promptly to improve the organization's security posture.

% --- Section 2: Organizational Information ---
\section{Organizational Information}

The following information was provided for the assessment.

\begin{table}[h!]
\centering
\begin{tabular}{@{}ll@{}}
\toprule
\textbf{Attribute} & \textbf{Value} \\ \midrule
Organization Name & Pioneer Pulse \\
Email Domain & \texttt{PioneerPulse.com} \\
Website Domain & \url{www.PioneerPulse.com} \\
External IP Address & \texttt{144.89.146.174} \\ \bottomrule
\end{tabular}
\caption{Client Organizational Data}
\label{tab:org_data}
\end{table}

% --- Section 3: Security Control Review ---
\section{Security Control Review (Questionnaire Analysis)}

A review of the organization's administrative and technical security controls was conducted based on a standardized questionnaire. The results indicate significant gaps in critical areas. A "No" answer (\ding{55}) highlights a deviation from security best practices and represents a potential risk.

\begin{table}[h!]
\centering
\begin{tabular}{@{}p{0.7\linewidth}c@{}}
\toprule
\textbf{Security Control Question} & \textbf{Status} \\ \midrule
Do you require MFA to access email? & \textcolor{green}{\ding{51}} \\
Do you require MFA to log into computers? & \textcolor{green}{\ding{51}} \\
Do you require MFA to access sensitive data systems? & \textcolor{red}{\ding{55}} \\
Does your organization have an employee acceptable use policy? & \textcolor{green}{\ding{51}} \\
Does your organization do security awareness training for new employees? & \textcolor{red}{\ding{55}} \\
Does your organization do security awareness training for all employees at least once per year? & \textcolor{green}{\ding{51}} \\ \bottomrule
\end{tabular}
\caption{Security Controls Questionnaire Results}
\label{tab:controls}
\end{table}

\subsection*{Analysis of Gaps}
\begin{itemize}
    \item \textbf{MFA on Sensitive Systems:} The absence of MFA for sensitive data systems is a critical weakness. Should an employee's credentials be compromised, an attacker could gain direct access to the organization's most valuable data.
    \item \textbf{New Employee Training:} Failing to provide security training during onboarding leaves new hires, who are often prime targets for social engineering, unprepared to identify and respond to threats. This creates an immediate and ongoing vulnerability.
\end{itemize}

% --- Section 4: Technical Scan Results ---
\section{Technical Scan Results}

An Nmap scan was performed to identify open ports and running services on the target system.

\subsection*{Target: \texttt{10.0.0.15}}
The scan identified one open port with a critically vulnerable service.

\begin{table}[h!]
\centering
\begin{tabular}{@{}lllll@{}}
\toprule
\textbf{Port} & \textbf{State} & \textbf{Service} & \textbf{Version} & \textbf{Finding} \\ \midrule
21/tcp & Open & ftp & vsftpd 2.3.4 & \begin{tabular}[c]{@{}l@{}}Anonymous FTP login allowed.\\ \textbf{Critically vulnerable version.}\end{tabular} \\ \bottomrule
\end{tabular}
\caption{Open Ports and Services on \texttt{10.0.0.15}}
\label{tab:nmap_results}
\end{table}

\subsection*{Analysis of Technical Findings}
The FTP service running on port 21 presents two major risks:
\begin{enumerate}
    \item \textbf{Known Backdoor (CVE-2011-2523):} The identified version, \texttt{vsftpd 2.3.4}, contains a well-documented backdoor that was inserted into the source code. An attacker can exploit this vulnerability to gain a command shell on the underlying server, leading to a full system compromise.
    \item \textbf{Anonymous FTP Access:} The server is configured to allow anonymous logins. This permits any external user to connect and potentially access, upload, or download files. This could lead to sensitive data exposure or be used by an attacker to stage malicious files.
\end{enumerate}

% --- Section 5: Consolidated Risk Assessment ---
\section{Consolidated Risk Assessment}
This section synthesizes findings from the questionnaire, technical scan, and pre-existing risk data into a prioritized list.

\begin{table}[h!]
\centering
\begin{tabular}{@{}lp{0.45\linewidth}p{0.2\linewidth}l@{}}
\toprule
\textbf{ID} & \textbf{Risk Description} & \textbf{Affected Systems} & \textbf{Severity} \\ \midrule
R-01 & Vulnerable FTP server (\texttt{vsftpd 2.3.4}) with anonymous access allows remote code execution. & Server at \texttt{10.0.0.15} & \textcolor{sev_critical}{\textbf{Critical}} \\
\addlinespace
R-02 & Lack of MFA on sensitive data systems exposes critical data to credential compromise. & All sensitive systems & \textcolor{sev_critical}{\textbf{Critical}} \\
\addlinespace
R-03 & No security awareness training for new employees increases susceptibility to social engineering. & All new employees & \textcolor{sev_high}{\textbf{High}} \\
\addlinespace
R-04 & Workstations are running an unsupported OS (Windows 7), which no longer receives security updates. & Workstations & \textcolor{sev_medium}{\textbf{Medium}} \\ \bottomrule
\end{tabular}
\caption{Summary of Identified Risks}
\label{tab:risk_summary}
\end{table}

% --- Section 6: Recommendations ---
\section{Recommendations}
\label{sec:recommendations}
The following actions are recommended to mitigate the identified risks. They are prioritized based on severity and potential impact.

\subsection*{Priority 1: Immediate Actions (Address within 72 hours)}
\begin{enumerate}
    \item \textbf{Remediate Vulnerable FTP Server (R-01):}
    \begin{itemize}
        \item Immediately take the FTP server at \texttt{10.0.0.15} offline.
        \item Investigate the server for signs of compromise.
        \item If the FTP service is business-critical, replace it with a secure alternative like SFTP (SSH File Transfer Protocol).
        \item If FTP must be used, upgrade \texttt{vsftpd} to the latest stable version and disable anonymous access.
    \end{itemize}
    \item \textbf{Implement MFA on Sensitive Systems (R-02):}
    \begin{itemize}
        \item Immediately enable MFA on all systems classified as containing sensitive data.
        \item Develop a policy requiring MFA for all new sensitive systems deployed in the future.
    \end{itemize}
\end{enumerate}

\subsection*{Priority 2: Near-Term Actions (Address within 30 days)}
\begin{enumerate}
    \item \textbf{Establish Onboarding Security Training (R-03):}
    \begin{itemize}
        \item Develop a mandatory security awareness training module for all new hires.
        \item This training should be part of the formal onboarding process and cover topics such as phishing, acceptable use, and data handling policies.
    \end{itemize}
    \item \textbf{Plan Operating System Upgrades (R-04):}
    \begin{itemize}
        \item Create a project plan to upgrade all workstations from Windows 7 to a modern, supported operating system (e.g., Windows 11).
        \item Prioritize the upgrade of systems used by employees with access to sensitive data.
    \end{itemize}
\end{enumerate}

\end{document}
```