```latex
\documentclass[12pt]{article}

% Preamble: Required Packages and Document Setup
\usepackage[margin=1in]{geometry}
\usepackage{pifont} % For checkmarks and crosses
\usepackage{booktabs} % For professional tables
\usepackage{hyperref} % For clickable links and ToC
\usepackage{url} % For formatting URLs
\usepackage{seqsplit} % For splitting long strings like IPs/hashes
\usepackage{graphicx} % For potential logos in a real report
\usepackage{xcolor} % For colors

% Hyperref Setup
\hypersetup{
    colorlinks=true,
    linkcolor=blue,
    filecolor=magenta,      
    urlcolor=cyan,
    pdftitle={Cybersecurity Risk Assessment Report},
    pdfpagemode=FullScreen,
}

% Custom Commands
\newcommand{\yes}{\ding{51}}
\newcommand{\no}{\ding{55}}
\newcommand{\orgname}{\textbf{Modern Myth}}
\newcommand{\orgdomain}{\texttt{ModernMyth.com}}
\newcommand{\orgip}{\texttt{165.250.60.101}}

\begin{document}

% --- Title Page ---
\begin{titlepage}
    \centering
    \vspace*{1cm}
    \Huge \textbf{Cybersecurity Risk Assessment Report}
    \vspace{1.5cm}
    \Large \textbf{Prepared for:} \\
    \vspace{0.5cm}
    \orgname
    \vfill
    \large \textbf{Date of Report:} \\
    \today
    \vspace{1cm}
    \large \textbf{Analysis Conducted By:} \\
    Expert Cybersecurity Analyst
\end{titlepage}

% --- Table of Contents ---
\tableofcontents
\newpage

% --- Section 1: Executive Summary ---
\section{Executive Summary}
This report provides a comprehensive cybersecurity risk assessment for \orgname, based on an analysis of network scan data, organizational security controls, and pre-existing risk information. The assessment was conducted on \today.

The analysis revealed several critical and high-severity risks that require immediate attention. Key findings include:
\begin{itemize}
    \item \textbf{Systemic Remote Access Vulnerability:} The technical scan identified a server with an exposed Remote Desktop Protocol (RDP) port (\texttt{3389}). This finding, correlated with pre-existing risk data, indicates a systemic issue of insecure remote access, which is a primary vector for ransomware attacks.
    \item \textbf{Critical Gaps in Access Control:} The organization does not enforce Multi-Factor Authentication (MFA) for accessing sensitive data systems. This significantly increases the risk of a data breach resulting from compromised credentials.
    \item \textbf{Deficient Security Policies and Training:} There is a complete absence of an employee acceptable use policy and a formal security awareness training program. This lack of a security-conscious culture elevates the risk of human error, such as falling victim to phishing attacks, which could serve as the initial entry point for a major incident.
\end{itemize}

The overall security posture of \orgname\ is considered high-risk. This report outlines specific, actionable recommendations to mitigate these findings, prioritizing the most critical vulnerabilities to enhance the organization's resilience against cyber threats.

% --- Section 2: Organizational Information ---
\section{Organizational Information}
The following information was provided for the assessment:
\begin{itemize}
    \item \textbf{Organization Name:} \orgname
    \item \textbf{Email Domain:} \orgdomain
    \item \textbf{Website Domain:} \texttt{www.ModernMyth.com}
    \item \textbf{External IP Address:} \orgip
\end{itemize}

% --- Section 3: Security Control Review ---
\section{Security Control Review}
A review of the organization's security controls was conducted via a questionnaire. The responses highlight significant gaps in foundational security practices. A "No" response indicates a missing control and a potential area of high risk.

\begin{table}[h!]
\centering
\caption{Security Controls Questionnaire Analysis}
\begin{tabular}{p{0.75\textwidth} c}
\toprule
\textbf{Control Question} & \textbf{Response} \\
\midrule
Do you require MFA to access email? & \yes \\
Do you require MFA to log into computers? & \yes \\
\textcolor{red}{Do you require MFA to access sensitive data systems?} & \textcolor{red}{\no} \\
\textcolor{red}{Does your organization have an employee acceptable use policy?} & \textcolor{red}{\no} \\
\textcolor{red}{Does your organization do security awareness training for new employees?} & \textcolor{red}{\no} \\
\textcolor{red}{Does your organization do security awareness training for all employees at least once per year?} & \textcolor{red}{\no} \\
\bottomrule
\end{tabular}
\end{table}

\subsection*{Analysis of Control Gaps}
The lack of MFA on sensitive systems, coupled with the absence of security policies and training, creates a high-risk environment. An attacker who obtains a single user credential could potentially gain direct access to critical data. Furthermore, without security training, employees are more susceptible to social engineering and phishing attacks, which are the most common methods for stealing credentials.

% --- Section 4: Technical Scan Results ---
\section{Technical Scan Results}
A network scan was performed to identify externally or internally exposed services. The scan targeted the host at \seqsplit{\texttt{10.10.10.51}}.

\subsection*{Open Ports and Services}
The following open port was discovered on the target system:

\begin{table}[h!]
\centering
\caption{Open Ports on \seqsplit{\texttt{10.10.10.51}}}
\begin{tabular}{l l l l}
\toprule
\textbf{Port} & \textbf{State} & \textbf{Service Name} & \textbf{Description} \\
\midrule
3389/tcp & open & \texttt{ms-wbt-server} & Microsoft Remote Desktop Protocol (RDP) \\
\bottomrule
\end{tabular}
\end{table}

\subsection*{Technical Analysis}
The scan confirms that the host \seqsplit{\texttt{10.10.10.51}} has RDP exposed. RDP is a common target for brute-force attacks and exploitation of vulnerabilities (e.g., BlueKeep). Exposing this service without mitigating controls like a VPN or strict firewall rules is a critical security risk. This finding corroborates a pre-existing risk related to RDP exposure on another host (\seqsplit{\texttt{10.10.10.50}}), indicating a pattern of insecure configuration within the network.

% --- Section 5: Consolidated Risk Assessment ---
\section{Consolidated Risk Assessment}
This section synthesizes findings from the security control review, technical scan, and pre-existing risk data into a consolidated list of identified risks.

\begin{table}[h!]
\centering
\caption{Summary of Identified Risks}
\begin{tabular}{p{0.2\textwidth} p{0.5\textwidth} p{0.2\textwidth}}
\toprule
\textbf{Risk Name} & \textbf{Description} & \textbf{Severity} \\
\midrule
\textbf{Systemic RDP Exposure} & The Remote Desktop Protocol is exposed on multiple internal servers (\seqsplit{\texttt{10.10.10.50}}, \seqsplit{\texttt{10.10.10.51}}). This allows attackers to attempt brute-force logins or exploit RDP vulnerabilities to gain unauthorized access. & \textbf{Critical} \\
\addlinespace
\textbf{No MFA on Sensitive Data Systems} & Lack of Multi-Factor Authentication on critical systems means that a single compromised password provides an attacker with direct access to the organization's most valuable data. & \textbf{Critical} \\
\addlinespace
\textbf{Absence of Security Policies \& Training} & The organization lacks an Acceptable Use Policy and a security awareness training program. This increases the likelihood of human error leading to a security breach via phishing or other social engineering tactics. & \textbf{High} \\
\bottomrule
\end{tabular}
\end{table}

% --- Section 6: Recommendations ---
\section{Recommendations}
The following recommendations are provided to mitigate the identified risks. They are prioritized based on severity and potential impact.

\subsection{Immediate Actions (To Be Completed within 7 Days)}
\begin{enumerate}
    \item \textbf{Mitigate RDP Exposure:} Immediately close or firewall port \texttt{3389} on host \seqsplit{\texttt{10.10.10.51}} and any other systems where it is not essential. Access should be restricted to only authorized IP addresses. Conduct a network-wide scan to identify and remediate all other instances of exposed RDP.
    \item \textbf{Deploy MFA on Sensitive Systems:} Prioritize the deployment of MFA on all systems classified as containing sensitive or critical data. This is the single most effective control to prevent unauthorized access from compromised credentials.
\end{enumerate}

\subsection{Strategic Actions (To Be Completed within 90 Days)}
\begin{enumerate}
    \item \textbf{Implement Secure Remote Access:} For necessary administrative access, deploy a Virtual Private Network (VPN) or a Zero Trust Network Access (ZTNA) solution. All RDP access should be routed through this secure channel, which should also be protected by MFA.
    \item \textbf{Develop an Acceptable Use Policy (AUP):} Create and enforce an AUP that clearly defines the rules for using company IT assets, data handling, and internet usage. All employees must read and acknowledge this policy.
    \item \textbf{Establish a Security Awareness Program:} Implement a mandatory security awareness training program for all employees. This should be conducted for all new hires upon onboarding and annually for all staff. The training must cover key topics such as phishing identification, password security, and social engineering.
\end{enumerate}

\end{document}
```