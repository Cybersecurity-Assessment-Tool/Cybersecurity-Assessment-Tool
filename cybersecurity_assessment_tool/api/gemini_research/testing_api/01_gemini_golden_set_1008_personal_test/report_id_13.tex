```latex
\documentclass[12pt]{article}

% Preamble: Required Packages
\usepackage[margin=1in]{geometry}
\usepackage{pifont} % For checkmarks and crosses (\ding)
\usepackage{booktabs} % For professional-looking tables
\usepackage{graphicx}
\usepackage{fancyhdr}
\usepackage[hidelinks]{hyperref}
\usepackage{url}
\usepackage{seqsplit} % For breaking long strings in \texttt

% Document Information
\title{Cybersecurity Posture Assessment Report \\ \large For: \textbf{Crestview Analytics}}
\author{Cybersecurity Analysis Division}
\date{November 22, 2025}

\begin{document}

\maketitle
\thispagestyle{empty}
\newpage

\tableofcontents
\newpage

% --- 1. EXECUTIVE SUMMARY ---
\section{Executive Summary}

This report provides a comprehensive cybersecurity posture assessment for \textbf{Crestview Analytics}, conducted on November 22, 2025. The analysis is based on a network vulnerability scan, a review of organizational security controls, and an evaluation of pre-existing risks.

The assessment identified several critical and high-risk findings that require immediate attention. Key areas of concern include:
\begin{itemize}
    \item \textbf{Critical Gaps in Access Control:} Multi-Factor Authentication (MFA) is not enforced for accessing email or other sensitive data systems. This significantly increases the risk of account compromise and unauthorized data access.
    \item \textbf{Lack of Security Awareness:} The organization does not provide security awareness training for new or existing employees. This exposes the organization to a high likelihood of successful phishing attacks, social engineering, and other human-centered threats.
    \item \textbf{Outdated Web Server Software:} The external-facing web server is running an outdated version of Nginx (1.18.0), which is no longer supported and may contain known, unpatched vulnerabilities.
\end{itemize}

The combination of these findings places the organization at a \textbf{High Risk} of a significant security incident. This report details these risks and provides actionable recommendations to mitigate them and improve the overall security posture.

% --- 2. ORGANIZATIONAL INFORMATION ---
\section{Organizational Information}

The following information was provided for the assessment.
\begin{center}
\begin{tabular}{ll}
\toprule
\textbf{Attribute} & \textbf{Value} \\
\midrule
Organization Name & \textbf{Crestview Analytics} \\
Email Domain & \texttt{CrestviewAnalytics.com} \\
Website Domain & \texttt{www.CrestviewAnalytics.com} \\
External IP Address & \texttt{174.63.124.27} \\
\bottomrule
\end{tabular}
\end{center}

% --- 3. SECURITY CONTROL REVIEW ---
\section{Security Control Review}

A review of internal security controls was conducted based on a standardized questionnaire. The responses reveal significant gaps in fundamental security practices, particularly concerning access control and employee training.

\begin{table}[h!]
\centering
\caption{Security Controls Questionnaire Results}
\begin{tabular}{p{0.6\textwidth} c l}
\toprule
\textbf{Control Question} & \textbf{Response} & \textbf{Assessment} \\
\midrule
Do you require MFA to access email? & \ding{55} & \textbf{Critical Gap} \\
Do you require MFA to log into computers? & \ding{51} & Implemented \\
Do you require MFA to access sensitive data systems? & \ding{55} & \textbf{Critical Gap} \\
Does your organization have an employee acceptable use policy? & \ding{51} & Implemented \\
Does your organization do security awareness training for new employees? & \ding{55} & \textbf{High Risk} \\
Does your organization do security awareness training for all employees at least once per year? & \ding{55} & \textbf{High Risk} \\
\bottomrule
\end{tabular}
\end{table}

% --- 4. TECHNICAL SCAN RESULTS ---
\section{Technical Scan Results}

An external network scan was performed to identify open ports and exposed services on the organization's infrastructure.

\subsection{Scan Details}
\begin{itemize}
    \item \textbf{Target IP Address:} \texttt{192.168.10.5}
    \item \textbf{Scan Date:} 2025-11-22T10:00:00Z
\end{itemize}

\subsection{Open Ports and Services}
The scan identified one open port on the target system.

\begin{table}[h!]
\centering
\caption{Nmap Scan Findings}
\begin{tabular}{c c c l l}
\toprule
\textbf{Port} & \textbf{State} & \textbf{Service} & \textbf{Product} & \textbf{Version} \\
\midrule
443/tcp & open & https & nginx & 1.18.0 \\
\bottomrule
\end{tabular}
\end{table}

\subsection{Technical Analysis}
The scan confirms an Nginx web server is exposed on port 443 (HTTPS). The identified version, \textbf{1.18.0}, was released in April 2020. This version is outdated and no longer receives security updates from the developer. Running unsupported software exposes the organization to numerous publicly known vulnerabilities that could be exploited by attackers to compromise the server.

% --- 5. RISK ASSESSMENT SUMMARY ---
\section{Risk Assessment Summary}

This section synthesizes findings from the security control review, technical scan, and pre-existing risk data. No pre-existing vulnerabilities were documented in the provided data. The following new risks have been identified.

\begin{table}[h!]
\centering
\caption{Identified Cybersecurity Risks}
\begin{tabular}{p{0.15\textwidth} p{0.55\textwidth} p{0.15\textwidth}}
\toprule
\textbf{Risk ID} & \textbf{Description} & \textbf{Severity} \\
\midrule
\textbf{RISK-001} & \textbf{Lack of Multi-Factor Authentication:} Email and sensitive data systems are protected only by username and password. A single credential leak could lead to a full system compromise. & \textbf{Critical} \\
\addlinespace
\textbf{RISK-002} & \textbf{Inadequate Security Awareness Program:} The absence of employee security training makes the organization highly susceptible to phishing, malware, and social engineering attacks. & \textbf{High} \\
\addlinespace
\textbf{RISK-003} & \textbf{Outdated Web Server Software:} The public-facing Nginx server is running an unsupported version (1.18.0), likely containing unpatched vulnerabilities that can be easily exploited. & \textbf{High} \\
\bottomrule
\end{tabular}
\end{table}

% --- 6. RECOMMENDATIONS ---
\section{Recommendations}

The following actions are recommended to mitigate the identified risks and strengthen the organization's security posture. Recommendations are prioritized based on risk severity.

\begin{enumerate}
    \item \textbf{Implement Multi-Factor Authentication (RISK-001):}
    \begin{itemize}
        \item \textbf{Priority: Immediate.}
        \item Enforce MFA on all user accounts for accessing company email (e.g., Office 365, Google Workspace) and any systems identified as containing sensitive data. This is the single most effective control to prevent unauthorized access.
    \end{itemize}
    \vspace{1em}
    \item \textbf{Upgrade Nginx Web Server (RISK-003):}
    \begin{itemize}
        \item \textbf{Priority: High.}
        \item Plan and execute an upgrade of the Nginx server from version 1.18.0 to the latest stable version. Before deployment, test the new version in a staging environment to ensure application compatibility.
    \end{itemize}
    \vspace{1em}
    \item \textbf{Establish a Security Awareness Training Program (RISK-002):}
    \begin{itemize}
        \item \textbf{Priority: High.}
        \item Procure and implement a security awareness training solution. All employees must complete a foundational training module upon hiring and participate in annual refresher training. Conduct regular phishing simulations to measure effectiveness.
    \end{itemize}
    \vspace{1em}
    \item \textbf{Develop a Vulnerability Management Program:}
    \begin{itemize}
        \item \textbf{Priority: Medium.}
        \item To prevent future issues like the outdated Nginx server, establish a formal program to regularly scan for, prioritize, and remediate vulnerabilities across all organizational assets.
    \end{itemize}
\end{enumerate}

\end{document}
```