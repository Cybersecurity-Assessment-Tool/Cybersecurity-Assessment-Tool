```latex
\documentclass[12pt]{article}

% Preamble: Required Packages
\usepackage[margin=1in]{geometry}
\usepackage{pifont} % For checkmarks and crosses
\usepackage{booktabs} % For professional tables
\usepackage{hyperref} % For clickable links
\usepackage{url} % For URL formatting
\usepackage{seqsplit} % For splitting long strings in tt font
\usepackage{graphicx}
\usepackage{xcolor}

% Hyperref Setup
\hypersetup{
    colorlinks=true,
    linkcolor=blue,
    filecolor=magenta,      
    urlcolor=cyan,
    pdftitle={Cybersecurity Posture Report},
    pdfpagemode=FullScreen,
}

% Document Information
\title{Cybersecurity Posture Report \\ \large For Firebrand Media}
\author{Cybersecurity Analyst}
\date{\today}

\begin{document}

\maketitle
\thispagestyle{empty}
\newpage

\tableofcontents
\newpage

% --- 1. Executive Overview ---
\section{Executive Overview}
This report provides a comprehensive analysis of the cybersecurity posture for \textbf{Firebrand Media}. The assessment is based on a correlation of data from an internal network scan, a security controls questionnaire, and a review of pre-existing risks.

The analysis reveals a mixed security posture. While foundational controls such as an acceptable use policy and Multi-Factor Authentication (MFA) for email are in place, several critical and high-risk gaps were identified. The most significant concerns are the lack of MFA for computer and sensitive data system access, and the absence of annual security awareness training for all employees.

Furthermore, a technical scan of the internal network identified an unencrypted web service running on port 80. This, combined with the administrative gaps, exposes the organization to significant risks, including unauthorized access, data breaches, and increased susceptibility to social engineering attacks.

This report outlines these findings in detail and provides prioritized, actionable recommendations to mitigate the identified risks and strengthen the overall security framework of \textbf{Firebrand Media}.

% --- 2. Organizational Information ---
\section{Organizational Information}
The following details were provided for the assessment:
\begin{itemize}
    \item \textbf{Organization Name:} Firebrand Media
    \item \textbf{Email Domain:} \texttt{FirebrandMedia.net}
    \item \textbf{Website Domain:} \url{www.FirebrandMedia.net}
    \item \textbf{External IP Address:} \texttt{211.168.12.113}
\end{itemize}

% --- 3. Security Control Review ---
\section{Security Control Review (Questionnaire Analysis)}
A review of the security controls questionnaire was conducted to evaluate administrative and policy-based safeguards. The responses indicate several areas requiring immediate attention. "No" answers represent significant gaps in the organization's defense-in-depth strategy.

\begin{table}[h!]
\centering
\caption{Security Controls Questionnaire Results}
\begin{tabular}{p{0.7\linewidth} c c}
\toprule
\textbf{Control Question} & \textbf{Response} & \textbf{Status} \\
\midrule
Do you require MFA to access email? & Yes & \ding{51} \\
Do you require MFA to log into computers? & No & \textcolor{red}{\ding{55}} \\
Do you require MFA to access sensitive data systems? & No & \textcolor{red}{\ding{55}} \\
Does your organization have an employee acceptable use policy? & Yes & \ding{51} \\
Does your organization do security awareness training for new employees? & Yes & \ding{51} \\
Does your organization do security awareness training for all employees at least once per year? & No & \textcolor{red}{\ding{55}} \\
\bottomrule
\end{tabular}
\end{table}

\subsection*{Analysis of Gaps}
\begin{itemize}
    \item \textbf{MFA for Computers \& Sensitive Systems:} The absence of MFA on endpoints (computers) and sensitive data systems is a critical vulnerability. If an employee's credentials are compromised, an attacker could gain direct access to their workstation and potentially escalate privileges to access sensitive data without needing a second authentication factor.
    \item \textbf{Annual Security Training:} Security threats evolve constantly. Failing to provide annual refresher training for all employees leaves the organization highly vulnerable to phishing, ransomware, and other social engineering attacks, as security knowledge becomes outdated.
\end{itemize}

% --- 4. Technical Scan Results ---
\section{Technical Scan Results}
An Nmap scan was performed on the internal network to identify active services and potential vulnerabilities.

\begin{itemize}
    \item \textbf{Target IP Address:} \seqsplit{\texttt{172.16.0.1}}
    \item \textbf{Host Status:} Up
\end{itemize}

The following open ports were discovered on the target system:

\begin{table}[h!]
\centering
\caption{Open Port Analysis for 172.16.0.1}
\begin{tabular}{c c c p{0.5\linewidth}}
\toprule
\textbf{Port} & \textbf{State} & \textbf{Service} & \textbf{Notes} \\
\midrule
80/tcp & open & HTTP & Hypertext Transfer Protocol. This is an unencrypted web service. All data, including potential login credentials or sensitive information, is transmitted in cleartext. This poses a significant risk of data interception on the internal network. \\
\bottomrule
\end{tabular}
\end{table}

% --- 5. Consolidated Risk Assessment ---
\section{Consolidated Risk Assessment}
By correlating the security control gaps with the technical findings, we have identified the following key risks to the organization. The malicious/irrelevant risk entry from the input data was disregarded during this analysis.

\begin{table}[h!]
\centering
\caption{Summary of Identified Risks}
\begin{tabular}{p{0.15\linewidth} p{0.55\linewidth} c}
\toprule
\textbf{Risk ID} & \textbf{Risk Description} & \textbf{Severity} \\
\midrule
RISK-001 & Lack of MFA on sensitive data systems allows for single-factor authentication to critical assets. & \textbf{Critical} \\
\addlinespace
RISK-002 & Lack of MFA on computer logins exposes endpoints to unauthorized access if credentials are compromised. & \textbf{High} \\
\addlinespace
RISK-003 & Absence of annual security awareness training increases susceptibility to phishing and social engineering. & \textbf{High} \\
\addlinespace
RISK-004 & An internal service is using unencrypted HTTP, exposing internal network traffic to eavesdropping. & \textbf{Medium} \\
\bottomrule
\end{tabular}
\end{table}

% --- 6. Recommendations ---
\section{Recommendations}
The following actions are recommended to mitigate the identified risks and improve the overall security posture of \textbf{Firebrand Media}.

\subsection*{RISK-001 \& RISK-002: Implement Comprehensive MFA (Critical)}
\begin{itemize}
    \item \textbf{Immediate Action:} Prioritize the deployment of a robust MFA solution (e.g., TOTP authenticator apps, FIDO2 security keys) for all systems classified as containing sensitive data.
    \item \textbf{Short-Term Action:} Enforce MFA for all user logins to company computers and laptops. This can be integrated with existing identity providers like Azure AD or Okta, or implemented via third-party endpoint security tools.
\end{itemize}

\subsection*{RISK-003: Establish Annual Security Training (High)}
\begin{itemize}
    \item \textbf{Action:} Procure and implement a mandatory security awareness training program for all employees, to be completed annually. This program should cover modern threats such as phishing, ransomware, business email compromise, and safe browsing habits. Track completion to ensure 100\% compliance.
\end{itemize}

\subsection*{RISK-004: Remediate Insecure HTTP Service (Medium)}
\begin{itemize}
    \item \textbf{Action:} Investigate the service running on port 80 of \texttt{172.16.0.1}.
    \begin{enumerate}
        \item If the service is a legitimate and necessary web application, reconfigure it to use HTTPS with a valid internal TLS/SSL certificate.
        \item If the service is not required, it should be disabled and the port closed to reduce the attack surface.
    \end{enumerate}
\end{itemize}

\end{document}
```