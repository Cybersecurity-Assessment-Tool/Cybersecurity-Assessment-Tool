```latex
\documentclass[12pt]{article}

% Preamble: Required Packages
\usepackage[margin=1in]{geometry}
\usepackage{pifont} % For \ding
\usepackage{booktabs} % For professional tables
\usepackage{hyperref} % For hyperlinks
\usepackage{url} % For URL formatting
\usepackage{seqsplit} % For splitting long strings
\usepackage{xcolor} % For colors

% Document Information
\title{Cybersecurity Posture Assessment Report}
\author{Cybersecurity Analysis Division}
\date{\today}

% Hyperref Setup
\hypersetup{
    colorlinks=true,
    linkcolor=blue,
    filecolor=magenta,      
    urlcolor=cyan,
    pdftitle={Cybersecurity Posture Assessment Report},
    pdfpagemode=FullScreen,
}

\begin{document}

\maketitle
\thispagestyle{empty}
\newpage
\tableofcontents
\newpage

\section*{Executive Summary}

This report provides a comprehensive cybersecurity assessment for \textbf{Aventine Research}, synthesizing data from a network vulnerability scan, an organizational security questionnaire, and a review of pre-existing risks.

The assessment reveals a notable contrast in the organization's security posture. On the technical front, the external network scan of the target host \texttt{192.168.1.100} showed a strong defensive configuration, with no open ports detected. This indicates effective firewalling and a minimal attack surface on the scanned asset.

However, significant administrative and policy-based vulnerabilities were identified through the security control review. Critical gaps exist in access control for sensitive data, which is not protected by Multi-Factor Authentication (MFA). Furthermore, the absence of a formal Acceptable Use Policy and a lack of security awareness training for new employees create a high-risk environment susceptible to insider threats, social engineering, and policy violations.

Immediate action is required to address these procedural and policy-related deficiencies to build a more resilient and holistic security program. Key recommendations focus on implementing MFA for sensitive systems, developing foundational security policies, and integrating security training into the employee onboarding process.

\section{Organizational Information}

The following details were provided for the assessment.

\begin{itemize}
    \item \textbf{Organization Name:} Aventine Research
    \item \textbf{Primary Email Domain:} \texttt{AventineResearch.net}
    \item \textbf{Primary Website Domain:} \url{www.AventineResearch.net}
    \item \textbf{Known External IP Address:} \texttt{16.90.48.189}
\end{itemize}

\section{Security Control Review}

The following table summarizes the organization's responses to the security controls questionnaire. Items marked with \textcolor{red}{\ding{55}} indicate significant gaps in security posture and are addressed in the Risk Assessment section.

\begin{table}[h!]
\centering
\begin{tabular}{p{0.7\linewidth} c}
\toprule
\textbf{Control Question} & \textbf{Status} \\
\midrule
Do you require MFA to access email? & \textcolor{green}{\ding{51}} \\
Do you require MFA to log into computers? & \textcolor{green}{\ding{51}} \\
Do you require MFA to access sensitive data systems? & \textcolor{red}{\ding{55}} \\
Does your organization have an employee acceptable use policy? & \textcolor{red}{\ding{55}} \\
Does your organization do security awareness training for new employees? & \textcolor{red}{\ding{55}} \\
Does your organization do security awareness training for all employees at least once per year? & \textcolor{green}{\ding{51}} \\
\bottomrule
\end{tabular}
\caption{Security Controls Questionnaire Results}
\end{table}

\section{Technical Scan Results}

An Nmap scan was conducted on the specified target to identify open ports and running services.

\begin{itemize}
    \item \textbf{Target IP Address:} \texttt{192.168.1.100}
    \item \textbf{Host Status:} UP
\end{itemize}

\textbf{Findings:} The scan concluded that the host is online but has \textbf{no open ports}. All 1000 scanned ports were found to be in a 'closed' state. This is a positive security finding, suggesting that the host is either not running any network services or is protected by a well-configured firewall that blocks incoming connections. No vulnerabilities could be identified at the network level due to this strong defensive posture.

\section{Consolidated Risk Assessment}

This section correlates findings from the security control review and technical scan. While no technical vulnerabilities were found, the administrative gaps present significant risks to the organization. No pre-existing vulnerabilities were reported.

\begin{table}[h!]
\centering
\begin{tabular}{p{0.25\linewidth} p{0.55\linewidth} l}
\toprule
\textbf{Risk Name} & \textbf{Overview} & \textbf{Severity} \\
\midrule
\textbf{Lack of MFA on Sensitive Data Systems} & Failure to enforce MFA on systems containing sensitive data exposes critical assets to unauthorized access via credential theft or compromise. This is a primary target for attackers. & \textbf{Critical} \\
\addlinespace
\textbf{Absence of Acceptable Use Policy (AUP)} & Without a formal AUP, employees lack clear guidelines on the proper use of company assets. This increases the risk of insider threats, data leakage, and legal liability. & \textbf{High} \\
\addlinespace
\textbf{No Security Training for New Hires} & New employees are not receiving security awareness training upon joining. This makes them highly susceptible to phishing and social engineering attacks from their first day, creating a significant weak point in the organization's human firewall. & \textbf{High} \\
\bottomrule
\end{tabular}
\caption{Identified Risks and Severity}
\end{table}

\section{Recommendations}

The following actions are recommended to mitigate the identified risks and improve the overall security posture of \textbf{Aventine Research}.

\begin{enumerate}
    \item \textbf{[Critical] Implement MFA for Sensitive Data Access:}
    Immediately prioritize the deployment and enforcement of Multi-Factor Authentication (MFA) across all applications, databases, and systems that store or process sensitive organizational or client data. This is the single most effective control to mitigate the risk of unauthorized access.

    \item \textbf{[High] Develop and Enforce an Acceptable Use Policy (AUP):}
    Create a comprehensive AUP that clearly defines the rules and expectations for using company networks, devices, and data. This policy should be a mandatory part of the employee handbook and all current and new employees must formally acknowledge it.

    \item \textbf{[High] Integrate Security Training into Onboarding:}
    Establish a mandatory security awareness training module as part of the new employee onboarding process. This training should cover key topics such as phishing identification, password hygiene, data handling, and the new Acceptable Use Policy. This ensures a baseline of security knowledge from day one.
\end{enumerate}

\end{document}
```