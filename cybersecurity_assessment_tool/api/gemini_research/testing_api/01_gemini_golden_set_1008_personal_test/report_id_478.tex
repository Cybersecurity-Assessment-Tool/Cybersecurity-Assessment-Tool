```latex
\documentclass[12pt]{article}

% --- PACKAGES ---
\usepackage[margin=1in]{geometry}
\usepackage{pifont} % For checkmark and cross symbols
\usepackage{booktabs} % For professional tables
\usepackage{hyperref} % For clickable links
\usepackage{url} % For URL formatting
\usepackage{seqsplit} % To split long strings without breaking
\usepackage{graphicx}
\usepackage{xcolor}
\usepackage{array}

% --- DOCUMENT SETUP ---
\hypersetup{
    colorlinks=true,
    linkcolor=blue,
    filecolor=magenta,      
    urlcolor=cyan,
    pdftitle={Cybersecurity Posture Assessment Report},
    pdfpagemode=FullScreen,
}

% --- COMMANDS ---
\newcommand{\yes}{\ding{51}} % Checkmark
\newcommand{\no}{\ding{55}}  % Cross

% --- METADATA ---
\title{Cybersecurity Posture Assessment Report \\ \large For: \textbf{Prism Logic}}
\author{Cybersecurity Analysis Division}
\date{\today}

% --- DOCUMENT START ---
\begin{document}

\maketitle
\tableofcontents
\newpage

% ===================================================================
\section{Executive Summary}
% ===================================================================

This report details the findings of a cybersecurity posture assessment conducted for \textbf{Prism Logic}. The assessment incorporated an analysis of organizational security controls via a questionnaire, a technical network scan of the designated external asset, and a review of pre-existing risks.

The primary and most critical finding is the systemic lack of Multi-Factor Authentication (MFA) across the organization. MFA is not enforced for email, computer logins, or access to sensitive data systems. This absence represents a significant security gap, as it leaves the organization highly vulnerable to account takeover attacks resulting from credential compromise. A single compromised password could grant an attacker broad access to critical systems.

On a positive note, the organization has established foundational security practices, including an employee acceptable use policy and a consistent security awareness training program for all staff. These controls are commendable and reduce the risk of insider threats and social engineering.

The external network scan performed on the target IP address \texttt{[Target IP]} did not identify any open ports or exposed services. While this may indicate a strong firewall posture, it also warrants verification to ensure the target was online and accessible during the scan.

Our recommendations prioritize the immediate implementation of MFA across all critical services to mitigate the most severe identified risk. Further details on all findings and actionable remediation steps are provided in the subsequent sections of this report.

% ===================================================================
\section{Organizational Information}
% ===================================================================

The following information was provided for the assessment.

\begin{tabular}{@{}ll}
\toprule
\textbf{Attribute} & \textbf{Value} \\
\midrule
Organization Name & \textbf{Prism Logic} \\
Email Domain & \texttt{PrismLogic.com} \\
Website Domain & \url{www.PrismLogic.com} \\
External IP Address & \seqsplit{\texttt{37.79.56.233}} \\
\bottomrule
\end{tabular}

% ===================================================================
\section{Security Control Review}
% ===================================================================

The following table summarizes the organization's responses to the security controls questionnaire. Responses marked with a \no\ represent potential security gaps that require attention.

\begin{table}[h!]
\centering
\begin{tabular}{>{\raggedright\arraybackslash}p{10cm} c}
\toprule
\textbf{Control Question} & \textbf{Response} \\
\midrule
Does your organization have an employee acceptable use policy? & \yes \\
Does your organization do security awareness training for new employees? & \yes \\
Does your organization do security awareness training for all employees at least once per year? & \yes \\
\addlinespace[0.5em]
\textbf{Do you require MFA to access email?} & \textcolor{red}{\no} \\
\textbf{Do you require MFA to log into computers?} & \textcolor{red}{\no} \\
\textbf{Do you require MFA to access sensitive data systems?} & \textcolor{red}{\no} \\
\bottomrule
\end{tabular}
\caption{Organizational Security Controls Questionnaire Results.}
\end{table}

\paragraph{Analysis:} The questionnaire reveals a critical weakness in the organization's identity and access management strategy. The consistent "No" answers regarding Multi-Factor Authentication (MFA) indicate that user accounts are protected solely by passwords. This is a significant risk, as password-based threats (e.g., phishing, credential stuffing, password spraying) are among the most common and effective attack vectors.

% ===================================================================
\section{Technical Scan Results}
% ===================================================================

A network scan was conducted to identify exposed services and potential vulnerabilities on the organization's external infrastructure.

\begin{itemize}
    \item \textbf{Target IP Address:} \texttt{[Target IP]}
    \item \textbf{Scan Date:} \today
\end{itemize}

\paragraph{Findings:} The network scan conducted on the target IP address did not identify any open TCP or UDP ports. This outcome suggests one of the following possibilities:
\begin{itemize}
    \item The host was offline or unreachable at the time of the scan.
    \item A robust firewall is in place that drops or rejects all unsolicited incoming traffic, which is a positive security practice.
    \item There are no services publicly exposed from this specific IP address.
\end{itemize}

No vulnerabilities were identified from this scan.

% ===================================================================
\section{Risk Assessment}
% ===================================================================

The following table synthesizes findings from the security control review, technical scan, and pre-existing risk data. New risks identified during this assessment are listed below.

\begin{table}[h!]
\centering
\begin{tabular}{@{}lp{5cm}cp{6cm}@{}}
\toprule
\textbf{ID} & \textbf{Risk Name} & \textbf{Severity} & \textbf{Description} \\
\midrule
\addlinespace[0.5em]
RISK-001 & Lack of Multi-Factor Authentication (MFA) & \textbf{\textcolor{red}{Critical}} & The absence of MFA for email, computer, and sensitive system access exposes the organization to a high likelihood of account compromise via stolen or weak credentials. A single compromised password could lead to a significant data breach. \\
\addlinespace[0.5em]
\bottomrule
\end{tabular}
\caption{Identified Cybersecurity Risks.}
\end{table}

\paragraph{Note:} The pre-existing risk register (\texttt{Input\_3\_Current\_Risks\_JSON}) was empty, and no technical vulnerabilities were discovered during the network scan. Therefore, the risk listed above is the sole critical finding of this assessment.

% ===================================================================
\section{Recommendations}
% ===================================================================

Based on the analysis, the following actions are recommended to improve the cybersecurity posture of \textbf{Prism Logic}. Recommendations are prioritized by severity.

\subsection*{Priority 1: Implement Multi-Factor Authentication (Critical)}
This is the most urgent action required to protect the organization from common cyberattacks.
\begin{enumerate}
    \item \textbf{Email Systems:} Immediately enforce MFA for all user access to the email system (e.g., Office 365, Google Workspace). Email accounts are a primary target for attackers seeking to gain an initial foothold.
    \item \textbf{Remote Access:} Mandate MFA for all remote access solutions, including VPNs and Remote Desktop Protocol (RDP) gateways.
    \item \textbf{Sensitive Systems:} Develop and execute a phased rollout plan to require MFA for all applications and systems that store or process sensitive data.
    \item \textbf{Computer Logins:} Evaluate and deploy an MFA solution for logging into company workstations and servers to protect against unauthorized physical or remote access.
\end{enumerate}

\subsection*{Priority 2: Network Verification and Monitoring (High)}
\begin{enumerate}
    \item \textbf{Confirm Scan Results:} Internally verify that the external firewall configuration for \texttt{[Target IP]} is correctly configured to block all non-essential traffic. Ensure that the asset was online during the assessment scan.
    \item \textbf{Implement Continuous Scanning:} Establish a regular, automated vulnerability scanning schedule for all external-facing assets to proactively identify and remediate new vulnerabilities.
\end{enumerate}

\subsection*{Priority 3: Policy Enhancement (Medium)}
\begin{enumerate}
    \item \textbf{Update Acceptable Use Policy:} Review and update the existing Acceptable Use Policy to include clauses mandating the use of MFA where it is implemented. Clearly define employee responsibilities for protecting their accounts.
\end{enumerate}

\end{document}
```