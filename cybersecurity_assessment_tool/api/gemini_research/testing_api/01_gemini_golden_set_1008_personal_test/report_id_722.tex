```latex
\documentclass[12pt]{article}

% Preamble: Required Packages
\usepackage[a4paper, margin=1in]{geometry}
\usepackage{pifont} % For checkmarks and crosses
\usepackage{booktabs} % For professional tables
\usepackage{hyperref} % For hyperlinks and document metadata
\usepackage{url} % For formatting URLs
\usepackage{seqsplit} % For splitting long strings without spaces
\usepackage{graphicx} % For potential logos
\usepackage{xcolor} % For colors

% Document Metadata
\hypersetup{
    colorlinks=true,
    linkcolor=blue,
    filecolor=magenta,
    urlcolor=cyan,
    pdftitle={Cybersecurity Posture Report},
    pdfauthor={Cybersecurity Analysis Division},
    pdfsubject={Security Assessment},
    pdfkeywords={Security, Analysis, Report},
    pdftoolbar=true,
}

% Title
\title{Cybersecurity Posture Report \\ \large For: \textbf{Velocity Ventures}}
\author{Cybersecurity Analysis Division}
\date{\today}

\begin{document}

\maketitle
\thispagestyle{empty}
\newpage

\tableofcontents
\thispagestyle{empty}
\newpage

\setcounter{page}{1}

% --- Section 1: Executive Summary ---
\section{Executive Summary}

This report provides a comprehensive analysis of the cybersecurity posture for \textbf{Velocity Ventures}, based on a synthesis of network scan data, organizational security controls, and pre-existing risk documentation. The assessment was conducted to identify vulnerabilities, security gaps, and misconfigurations that could expose the organization to cyber threats.

While the organization has implemented several strong security controls, such as multi-factor authentication (MFA) across key systems, this assessment has identified several high-priority risks requiring immediate attention.

Key findings include:
\begin{itemize}
    \item \textbf{Critical Vulnerability:} A public-facing FTP server on the internal network (\texttt{10.0.0.15}) is running a critically outdated version of \texttt{vsftpd} (2.3.4), which is known to contain a backdoor vulnerability (CVE-2011-2523). The server is further misconfigured to allow anonymous logins, significantly increasing the risk of unauthorized access and system compromise.
    \item \textbf{High-Risk Process Gap:} New employees do not receive security awareness training as part of their onboarding process. This represents a significant gap, as new hires are often prime targets for social engineering and phishing attacks.
    \item \textbf{Medium-Risk Infrastructure Issue:} The organization has a known issue of running outdated Windows 7 workstations, which are no longer supported and do not receive security updates, leaving them vulnerable to exploitation.
\end{itemize}

Immediate remediation of the FTP server vulnerability is strongly recommended, followed by the implementation of a mandatory security training program for new hires. Addressing these findings will substantially improve the organization's defensive capabilities.

% --- Section 2: Organizational Information ---
\section{Organizational Information}

The following details were provided for the assessment. This information helps establish the context and scope of the review.

\begin{tabular}{@{}ll}
    \toprule
    \textbf{Attribute} & \textbf{Value} \\
    \midrule
    Organization Name & \textbf{Velocity Ventures} \\
    Email Domain & \texttt{VelocityVentures.com} \\
    Website Domain & \url{www.VelocityVentures.com} \\
    External IP Address & \texttt{186.104.19.206} \\
    \bottomrule
\end{tabular}

% --- Section 3: Security Control Review ---
\section{Security Control Review}

A review of organizational security controls was conducted based on a standardized questionnaire. The responses indicate the current state of implemented policies and procedures. A "No" response highlights a potential gap in the security framework.

\begin{table}[h!]
\centering
\caption{Security Controls Questionnaire Results}
\begin{tabular}{@{}lc}
    \toprule
    \textbf{Control Question} & \textbf{Response} \\
    \midrule
    Do you require MFA to access email? & \ding{51} \\ % Yes
    Do you require MFA to log into computers? & \ding{51} \\ % Yes
    Do you require MFA to access sensitive data systems? & \ding{51} \\ % Yes
    Does your organization have an employee acceptable use policy? & \ding{51} \\ % Yes
    Does your organization do security awareness training for new employees? & \textcolor{red}{\ding{55}} \\ % No
    Does your organization do security awareness training for all employees annually? & \ding{51} \\ % Yes
    \bottomrule
\end{tabular}
\end{table}

\subsection*{Analysis of Controls}
The organization demonstrates a strong commitment to identity and access management through the enforcement of MFA across email, computers, and sensitive systems. However, the lack of \textbf{mandatory security awareness training for new employees} is a critical oversight. This gap leaves the organization vulnerable, as untrained personnel are more likely to fall victim to phishing, malware, and other social engineering tactics.

% --- Section 4: Technical Scan Results ---
\section{Technical Scan Results}

An Nmap scan was performed on the specified target to identify open ports, running services, and potential vulnerabilities.

\subsection*{Target: \texttt{10.0.0.15}}
The scan revealed one host as "up" with the following open port:

\begin{table}[h!]
\centering
\caption{Open Ports and Services on \texttt{10.0.0.15}}
\begin{tabular}{@{}llllll}
    \toprule
    \textbf{Port} & \textbf{State} & \textbf{Service} & \textbf{Product} & \textbf{Version} & \textbf{Notes} \\
    \midrule
    21/tcp & Open & ftp & vsftpd & 2.3.4 & \begin{tabular}[t]{@{}l@{}}Anonymous FTP login allowed.\\ \textbf{CRITICAL:} Version is vulnerable \\ to a backdoor (CVE-2011-2523).\end{tabular} \\
    \bottomrule
\end{tabular}
\end{table}

\subsection*{Analysis of Technical Findings}
The finding on host \texttt{10.0.0.15} is of critical importance. The combination of two factors creates a severe risk:
\begin{enumerate}
    \item \textbf{Vulnerable Software:} \texttt{vsftpd} version 2.3.4 contains a well-documented backdoor that was inserted into the source code. If exploited, it can grant an attacker a command shell on the system.
    \item \textbf{Insecure Configuration:} The server permits anonymous FTP logins, meaning any attacker on the network can connect without credentials, making it trivial to attempt exploitation or access any publicly shared files.
\end{enumerate}

% --- Section 5: Risk Assessment ---
\section{Risk Assessment}

The following table synthesizes findings from the security control review, technical scan, and pre-existing risk data into a prioritized list.

\begin{table}[h!]
\centering
\caption{Summary of Identified Risks}
\begin{tabular}{@{}llll}
    \toprule
    \textbf{ID} & \textbf{Risk Name} & \textbf{Severity} & \textbf{Description} \\
    \midrule
    \textbf{R-01} & FTP Backdoor Vulnerability & \textbf{Critical} & \begin{tabular}[t]{@{}l}An internal server (\texttt{10.0.0.15}) is running \\ \texttt{vsftpd 2.3.4}, which has a known backdoor. \\ It is also configured for anonymous access.\end{tabular} \\
    \addlinespace
    \textbf{R-02} & Inadequate Employee Onboarding & \textbf{High} & \begin{tabular}[t]{@{}l}New employees do not receive security \\ awareness training, increasing susceptibility \\ to social engineering and phishing.\end{tabular} \\
    \addlinespace
    \textbf{R-03} & Outdated Operating Systems & \textbf{Medium} & \begin{tabular}[t]{@{}l}Workstations are running Windows 7, an \\ unsupported OS that no longer receives \\ security updates.\end{tabular} \\
    \bottomrule
\end{tabular}
\end{table}

% --- Section 6: Recommendations ---
\section{Recommendations}

The following actions are recommended to mitigate the identified risks and strengthen the overall security posture of \textbf{Velocity Ventures}.

\subsection*{R-01: FTP Backdoor Vulnerability (Critical)}
\begin{itemize}
    \item \textbf{Immediate Action:} Disconnect the affected server (\texttt{10.0.0.15}) from the network immediately to prevent exploitation. If the service is business-critical, block all access to port 21 at the network firewall until it can be patched.
    \item \textbf{Short-Term Fix:} Upgrade the \texttt{vsftpd} service to the latest stable version. Disable anonymous FTP access unless there is an explicit and documented business requirement.
    \item \textbf{Long-Term Strategy:} Evaluate the business need for FTP. Migrate to a more secure file transfer protocol such as SFTP (SSH File Transfer Protocol) or FTPS (FTP over SSL/TLS).
\end{itemize}

\subsection*{R-02: Inadequate Employee Onboarding (High)}
\begin{itemize}
    \item \textbf{Immediate Action:} Develop and implement a mandatory security awareness training module to be completed by all new employees during their first week of onboarding.
    \item \textbf{Short-Term Fix:} The training should cover, at a minimum: phishing identification, strong password policies, acceptable use of company assets, and procedures for reporting security incidents.
    \item \textbf{Long-Term Strategy:} Foster a culture of security by incorporating security topics into regular all-hands meetings and communications, reinforcing the annual training program.
\end{itemize}

\subsection*{R-03: Outdated Operating Systems (Medium)}
\begin{itemize}
    \item \textbf{Immediate Action:} Prioritize the existing plan to upgrade all remaining Windows 7 workstations to a supported operating system, such as Windows 10 or Windows 11.
    \item \textbf{Short-Term Fix:} Ensure that any systems awaiting upgrade are isolated on the network as much as possible to limit their exposure.
    \item \textbf{Long-Term Strategy:} Implement a formal asset and patch management policy that includes end-of-life (EOL) tracking for all hardware and software to prevent future occurrences of unsupported systems.
\end{itemize}

\end{document}
```