```latex
\documentclass[12pt]{article}

% --- PACKAGES ---
\usepackage[margin=1in]{geometry}
\usepackage{pifont} % For checkmarks and crosses
\usepackage{booktabs} % For professional tables
\usepackage{hyperref} % For clickable links
\usepackage{url} % For URL formatting
\usepackage{seqsplit} % For splitting long strings like IPs

% --- DOCUMENT SETUP ---
\hypersetup{
    colorlinks=true,
    linkcolor=blue,
    filecolor=magenta,      
    urlcolor=cyan,
    pdftitle={Cybersecurity Posture Report},
    pdfauthor={Cybersecurity Analyst},
    pdfsubject={Security Assessment},
    pdfkeywords={Cybersecurity, Risk, Assessment},
    bookmarks=true
}

\newcommand{\yes}{\ding{51}}
\newcommand{\no}{\ding{55}}

% --- DOCUMENT START ---
\begin{document}

% --- TITLE PAGE ---
\title{Cybersecurity Posture Report \\ \large For: Green Sprout Organic}
\author{Cybersecurity Analyst}
\date{\today}
\maketitle

\newpage

% --- TABLE OF CONTENTS ---
\tableofcontents
\newpage

% --- EXECUTIVE SUMMARY ---
\section{Executive Summary}
This report provides a comprehensive analysis of the cybersecurity posture for Green Sprout Organic, based on a combination of network scanning, organizational data review, and an assessment of existing risks. The analysis reveals a mixed security landscape. While foundational controls like an acceptable use policy are in place, several critical and high-risk gaps were identified that require immediate attention.

Key findings include a critical lack of Multi-Factor Authentication (MFA) on sensitive data systems and for computer logins. Furthermore, new employees do not receive security awareness training, creating a significant vulnerability from their first day. Technical analysis identified an externally accessible Secure Shell (SSH) service, which, if not properly configured, presents a direct vector for attack.

This report outlines these risks in detail and provides a prioritized list of actionable recommendations. Addressing these findings, particularly the implementation of MFA and bolstering security training, is crucial to significantly improve the organization's resilience against common cyber threats.

% --- ORGANIZATIONAL INFORMATION ---
\section{Organizational Information}
The following details were provided for the assessment. This information is used to establish the context and scope of the review.

\begin{itemize}
    \item \textbf{Organization Name:} Green Sprout Organic
    \item \textbf{Email Domain:} \texttt{GreenSproutOrganic.com}
    \item \textbf{Website Domain:} \url{www.GreenSproutOrganic.com}
    \item \textbf{Primary External IP:} \texttt{215.67.29.51}
\end{itemize}

% --- SECURITY CONTROL REVIEW ---
\section{Security Control Review}
A review of the organization's security controls was conducted via a questionnaire. The responses indicate areas of both strength and weakness in the current security policies and procedures. Gaps identified here often represent significant organizational risk.

\begin{table}[h!]
\centering
\caption{Security Controls Questionnaire Results}
\begin{tabular}{p{0.75\linewidth} c}
\toprule
\textbf{Control Question} & \textbf{Response} \\
\midrule
Do you require MFA to access email? & \yes \\
Do you require MFA to log into computers? & \no \\
Do you require MFA to access sensitive data systems? & \no \\
Does your organization have an employee acceptable use policy? & \yes \\
Does your organization do security awareness training for new employees? & \no \\
Does your organization do security awareness training for all employees at least once per year? & \yes \\
\bottomrule
\end{tabular}
\end{table}

\subsection*{Analysis of Control Gaps}
The responses marked with a \no\ highlight critical deficiencies:
\begin{itemize}
    \item \textbf{No MFA for Computers \& Sensitive Systems:} This is the most critical finding. The absence of MFA on these systems means that a single compromised password could lead to unauthorized network access and a potential data breach.
    \item \textbf{No Security Training for New Employees:} New hires are a primary target for social engineering and phishing attacks. Failing to provide immediate security training leaves the organization vulnerable, as these employees are unaware of internal policies and common threats.
\end{itemize}

% --- TECHNICAL SCAN RESULTS ---
\section{Technical Scan Results}
An external network scan was performed to identify accessible services on the organization's public-facing infrastructure.

\begin{itemize}
    \item \textbf{Target IP Address:} \seqsplit{\texttt{2001:db8::1}}
\end{itemize}

The following table details the open ports discovered during the scan.

\begin{table}[h!]
\centering
\caption{Open Ports Detected on Target IP}
\begin{tabular}{l l l p{0.5\linewidth}}
\toprule
\textbf{Port} & \textbf{State} & \textbf{Service (Inferred)} & \textbf{Notes} \\
\midrule
22/tcp & Open & SSH (Secure Shell) & The scan confirmed this port is open but did not provide service version details. SSH is a common vector for brute-force attacks and requires robust configuration to be secure. \\
\bottomrule
\end{tabular}
\end{table}

\subsection*{Analysis of Technical Findings}
The primary finding is the exposed SSH service. While necessary for remote administration, an internet-facing SSH port is a high-value target for attackers. Without information on its configuration (e.g., password policy, key-based authentication), it must be treated as a significant risk. This finding, correlated with the lack of MFA, elevates the risk of unauthorized system access.

% --- RISK ASSESSMENT ---
\section{Risk Assessment}
This section synthesizes the findings from the control review and technical scan into a consolidated list of identified risks. The pre-existing risk register was empty.

\begin{table}[h!]
\centering
\caption{Summary of Identified Risks}
\begin{tabular}{p{0.1\linewidth} p{0.4\linewidth} p{0.15\linewidth} p{0.25\linewidth}}
\toprule
\textbf{Risk ID} & \textbf{Description} & \textbf{Severity} & \textbf{Affected Asset(s)} \\
\midrule
RISK-001 & Lack of MFA on sensitive data systems allows for unauthorized access via compromised credentials. & \textbf{Critical} & Sensitive Data, Core Systems, Databases \\
\addlinespace
RISK-002 & New employees are not given security training, making them highly susceptible to phishing and social engineering. & \textbf{High} & User Credentials, Endpoints, Company Reputation \\
\addlinespace
RISK-003 & The SSH management port is exposed to the internet, creating a vector for brute-force or exploit attacks. & \textbf{Medium} & Server at \seqsplit{\texttt{2001:db8::1}} \\
\addlinespace
RISK-004 & Absence of MFA on computer logins facilitates lateral movement within the network if credentials are stolen. & \textbf{Medium} & Employee Workstations, Internal Network Access \\
\bottomrule
\end{tabular}
\end{table}

% --- RECOMMENDATIONS ---
\section{Recommendations}
The following actions are recommended to mitigate the identified risks and strengthen the overall security posture of Green Sprout Organic. Recommendations are prioritized based on risk severity.

\begin{enumerate}
    \item \textbf{[Critical] Implement MFA on Sensitive Systems (RISK-001):}
    Immediately enforce mandatory MFA for all access to systems containing sensitive or critical data. Prioritize strong authentication methods such as authenticator apps (TOTP) or hardware security keys over less secure methods like SMS.

    \item \textbf{[High] Establish New Employee Security Training (RISK-002):}
    Integrate a mandatory security awareness training module into the new employee onboarding process. This training should be completed before a new hire is granted access to company systems and should cover phishing, password hygiene, and acceptable use policies.

    \item \textbf{[High] Harden Exposed SSH Service (RISK-003):}
    Review the business need for the publicly accessible SSH port on \seqsplit{\texttt{2001:db8::1}}.
    \begin{itemize}
        \item If not required, disable it.
        \item If required, implement a firewall rule to restrict access to only trusted IP addresses.
        \item Disable password-based authentication and enforce the use of strong SSH keys.
        \item Implement an intrusion detection/prevention tool like \texttt{fail2ban} to block brute-force attempts.
    \end{itemize}

    \item \textbf{[Medium] Deploy MFA for Workstation Logins (RISK-004):}
    Begin a phased rollout of MFA for all employee computer logins. This adds a critical layer of security that protects against credential theft and hinders an attacker's ability to move laterally within the network.

    \item \textbf{[Informational] Conduct Authenticated Vulnerability Scanning:}
    The initial network scan was unauthenticated and provided limited detail. A comprehensive, authenticated vulnerability scan should be conducted across all external and internal assets to identify specific software versions and missing security patches.
\end{enumerate}

% --- CONCLUSION ---
\section{Conclusion}
Green Sprout Organic has established some important security foundations, but this assessment has identified critical gaps that expose the organization to significant risk. The lack of robust identity and access management controls, particularly MFA, combined with deficiencies in employee security training, creates a high-risk environment.

By implementing the prioritized recommendations in this report, Green Sprout Organic can substantially reduce its attack surface, mitigate the most severe risks, and build a more resilient and secure operational environment.

\end{document}
```