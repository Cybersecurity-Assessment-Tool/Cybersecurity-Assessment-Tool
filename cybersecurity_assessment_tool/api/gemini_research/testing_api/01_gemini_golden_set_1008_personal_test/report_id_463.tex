```latex
\documentclass[12pt, a4paper]{article}

% Preamble: Required Packages
\usepackage[margin=1in]{geometry}
\usepackage{pifont} % For checkmarks and crosses
\usepackage{booktabs} % For professional tables
\usepackage{hyperref} % For clickable links
\usepackage{url} % For URL formatting
\usepackage{seqsplit} % To split long monospaced strings
\usepackage{graphicx}
\usepackage{xcolor}
\usepackage{datetime}

% Document Metadata
\title{Cybersecurity Posture Assessment Report}
\author{Cybersecurity Analysis Division}
\date{\today}

% Hyperref Setup
\hypersetup{
    colorlinks=true,
    linkcolor=blue,
    filecolor=magenta,      
    urlcolor=cyan,
    pdftitle={Cybersecurity Posture Assessment Report},
    pdfpagemode=FullScreen,
}

% Define colors for severity
\definecolor{criticalred}{HTML}{D7263D}
\definecolor{highorange}{HTML}{F49D40}
\definecolor{mediumyellow}{HTML}{F4D44D}
\definecolor{lowblue}{HTML}{5DADE2}
\definecolor{infogray}{HTML}{ABB2B9}

\begin{document}

\maketitle
\thispagestyle{empty}
\newpage

\tableofcontents
\thispagestyle{empty}
\newpage

\setcounter{page}{1}

% --- Executive Summary ---
\section{Executive Summary}

This report details the findings of a cybersecurity posture assessment conducted for \textbf{Digital Drift}. The assessment combined a review of organizational security controls, an external network scan, and an analysis of pre-existing risks.

The overall security posture is considered weak due to several critical and high-risk deficiencies. The most significant concerns are the absence of Multi-Factor Authentication (MFA) for email and sensitive data systems. This gap exposes the organization to a high likelihood of account compromise and subsequent data breaches.

Furthermore, the lack of a formal security awareness training program for both new and existing employees creates a significant vulnerability to social engineering and phishing attacks. These human-factor risks are often the initial vector for major security incidents.

From a technical standpoint, an exposed Secure Shell (SSH) service was identified on the organization's IPv6 network. While necessary for remote administration, its public accessibility makes it a prime target for automated brute-force attacks.

Immediate remediation of these identified risks is strongly recommended to reduce the organization's attack surface and improve its resilience against common cyber threats.

% --- Organizational Information ---
\section{Organizational Information}
The following details were provided for the assessment.

\begin{tabular}{@{}ll}
\toprule
\textbf{Attribute} & \textbf{Value} \\
\midrule
Organization Name & \textbf{Digital Drift} \\
Email Domain & \texttt{DigitalDrift.org} \\
Website Domain & \url{www.DigitalDrift.org} \\
Primary External IP & \seqsplit{\texttt{189.147.107.207}} \\
\bottomrule
\end{tabular}

% --- Security Control Review ---
\section{Security Control Review}
A review of self-reported security controls was conducted via a standardized questionnaire. The responses indicate significant gaps in foundational security practices. A summary of the responses is provided in Table \ref{tab:controls}.

\begin{table}[h!]
\centering
\caption{Organizational Security Control Responses}
\label{tab:controls}
\begin{tabular}{@{}p{0.7\linewidth}c@{}}
\toprule
\textbf{Control Question} & \textbf{Response} \\
\midrule
Do you require MFA to access email? & \textcolor{red}{\ding{55}} \\
Do you require MFA to log into computers? & \ding{51} \\
Do you require MFA to access sensitive data systems? & \textcolor{red}{\ding{55}} \\
Does your organization have an employee acceptable use policy? & \ding{51} \\
Does your organization do security awareness training for new employees? & \textcolor{red}{\ding{55}} \\
Does your organization do security awareness training for all employees at least once per year? & \textcolor{red}{\ding{55}} \\
\bottomrule
\end{tabular}
\end{table}

The items marked with \textcolor{red}{\ding{55}} represent critical control failures that require immediate attention.

% --- Technical Scan Results ---
\section{Technical Scan Results}
An external network scan was performed against the target IP address \seqsplit{\texttt{2001:db8::1}}. The scan identified one open port, which is detailed in Table \ref{tab:scan}.

\begin{table}[h!]
\centering
\caption{Open Port Scan Findings}
\label{tab:scan}
\begin{tabular}{@{}lllll@{}}
\toprule
\textbf{IP Address} & \textbf{Port} & \textbf{State} & \textbf{Service} & \textbf{Notes} \\
\midrule
\seqsplit{\texttt{2001:db8::1}} & 22/tcp & open & SSH (Inferred) & Publicly exposed. No version \\
& & & & information was retrieved. \\
\bottomrule
\end{tabular}
\end{table}

\subsection{Analysis of Findings}
The presence of an open SSH port (22) on the public internet is a notable finding. This service is commonly used for remote server administration. However, its exposure makes it a target for:
\begin{itemize}
    \item \textbf{Brute-force attacks:} Automated tools constantly scan the internet for open SSH ports and attempt to guess credentials.
    \item \textbf{Exploitation of vulnerabilities:} If the SSH server software is outdated, it may be vulnerable to known exploits that could lead to unauthorized access.
\end{itemize}
The risk associated with this finding is amplified by the organizational control gaps, particularly the lack of comprehensive MFA.

% --- Risk Assessment ---
\section{Risk Assessment}
Based on the correlation of organizational and technical findings, the following risks have been identified. The list of pre-existing vulnerabilities was empty, so this assessment is based entirely on new discoveries.

\begin{table}[h!]
\centering
\caption{Summary of Identified Risks}
\label{tab:risks}
\begin{tabular}{@{}p{0.1\linewidth}p{0.25\linewidth}p{0.45\linewidth}p{0.1\linewidth}@{}}
\toprule
\textbf{Risk ID} & \textbf{Risk Name} & \textbf{Description} & \textbf{Severity} \\
\midrule
RISK-001 & Inadequate Access Control & The lack of MFA on email and sensitive data systems severely increases the risk of unauthorized access through credential theft or phishing. & \textcolor{criticalred}{Critical} \\
\addlinespace
RISK-002 & Lack of Security Awareness Program & Without regular training, employees are more likely to fall victim to phishing and social engineering, making them an unintentional insider threat. & \textcolor{highorange}{High} \\
\addlinespace
RISK-003 & Exposed Management Service (SSH) & The publicly accessible SSH service is a constant target for automated attacks. A successful compromise would grant an attacker direct access to the server. & \textcolor{mediumyellow}{Medium} \\
\bottomrule
\end{tabular}
\end{table}

% --- Recommendations ---
\section{Recommendations}
The following actions are recommended to mitigate the identified risks and improve the overall security posture of \textbf{Digital Drift}.

\subsection{Immediate Actions (0-30 Days)}
\begin{enumerate}
    \item \textbf{Implement MFA (RISK-001):} Enforce mandatory MFA for all users on all external-facing systems, prioritizing email (e.g., Office 365, Google Workspace) and any systems containing sensitive or critical data.
    
    \item \textbf{Restrict SSH Access (RISK-003):} If SSH access is required from the internet, restrict it to known, trusted IP addresses using a firewall. If it is not required, disable the service or block the port entirely at the network perimeter. The preferred solution is to place the service behind a Virtual Private Network (VPN).
\end{enumerate}

\subsection{Strategic Actions (30-90 Days)}
\begin{enumerate}
    \setcounter{enumi}{2}
    \item \textbf{Develop Security Awareness Training (RISK-002):}
    \begin{itemize}
        \item Implement a mandatory security awareness training module for all new employees as part of the onboarding process.
        \item Schedule and conduct annual security awareness training for all staff, covering topics such as phishing, password security, and acceptable use.
        \item Consider conducting periodic phishing simulation campaigns to test and reinforce employee training.
    \end{itemize}
    
    \item \textbf{Strengthen SSH Configuration (RISK-003):}
    \begin{itemize}
        \item Ensure the SSH server software is updated to the latest stable version to patch known vulnerabilities.
        \item Disable password-based authentication and enforce the use of public-key cryptography for all SSH access.
        \item Implement an intrusion detection tool like \texttt{fail2ban} to automatically block IPs that exhibit brute-force behavior.
    \end{itemize}
\end{enumerate}

\end{document}
```