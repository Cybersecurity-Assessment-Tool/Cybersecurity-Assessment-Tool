```latex
\documentclass[12pt]{article}

% --- PACKAGE IMPORTS ---
\usepackage[margin=1in]{geometry} % Set page margins
\usepackage{pifont}               % For checkmark and X symbols (\ding)
\usepackage{booktabs}             % For professional-looking tables
\usepackage{hyperref}             % For clickable links and references
\usepackage{url}                  % For formatting URLs
\usepackage{seqsplit}             % For splitting long strings in texttt
\usepackage[utf8]{inputenc}       % For UTF-8 input encoding

% --- DOCUMENT METADATA ---
\title{Cybersecurity Posture Assessment Report}
\author{Cybersecurity Analysis Division}
\date{\today}

% --- HYPERREF SETUP ---
\hypersetup{
    colorlinks=true,
    linkcolor=blue,
    filecolor=magenta,      
    urlcolor=cyan,
    pdftitle={Cybersecurity Posture Assessment Report},
    pdfpagemode=FullScreen,
}

% --- DOCUMENT START ---
\begin{document}

\maketitle
\thispagestyle{empty}
\newpage
\tableofcontents
\newpage

% --- EXECUTIVE SUMMARY ---
\section*{Executive Summary}
This report provides a comprehensive cybersecurity assessment for \textbf{North Star Education}. The analysis is based on a synthesis of technical network scans, a review of organizational security controls, and an evaluation of pre-existing risks.

The assessment has identified several critical and high-risk vulnerabilities that require immediate attention. A key technical finding is an externally exposed FTP server running a critically outdated and vulnerable version of \texttt{vsftpd} (2.3.4), which is susceptible to a known remote code execution backdoor (CVE-2011-2523). This service is also misconfigured to allow anonymous access, posing a severe and immediate threat to data integrity and confidentiality.

Furthermore, significant gaps were identified in organizational security controls. The lack of mandatory Multi-Factor Authentication (MFA) on email accounts significantly increases the risk of account compromise via phishing or credential stuffing. This is compounded by the absence of a formal employee acceptable use policy and a structured security awareness training program. These policy-level deficiencies cultivate a high-risk environment where employees are more likely to fall victim to social engineering attacks.

This report outlines these findings in detail and provides a set of prioritized, actionable recommendations to mitigate the identified risks and strengthen the overall security posture of the organization.

% --- ORGANIZATIONAL INFORMATION ---
\section*{Organizational Information}
The following details were provided for the assessment scope.
\begin{itemize}
    \item \textbf{Organization Name:} North Star Education
    \item \textbf{Email Domain:} \texttt{NorthStarEducation.org}
    \item \textbf{Website Domain:} \url{www.NorthStarEducation.org}
    \item \textbf{External IP Address:} \texttt{210.160.180.246}
\end{itemize}

% --- SECURITY CONTROL REVIEW ---
\section*{Security Control Review}
A review of the organization's security controls was conducted via a questionnaire. The results below highlight critical gaps in administrative and access control policies. Answers marked with \ding{55} represent a deviation from security best practices and introduce significant risk.

\begin{table}[h!]
\centering
\caption{Security Controls Questionnaire Results}
\begin{tabular}{p{0.8\linewidth} c}
\toprule
\textbf{Question} & \textbf{Response} \\
\midrule
Do you require MFA to access email? & \ding{55} \\
Do you require MFA to log into computers? & \ding{51} \\
Do you require MFA to access sensitive data systems? & \ding{51} \\
Does your organization have an employee acceptable use policy? & \ding{55} \\
Does your organization do security awareness training for new employees? & \ding{55} \\
Does your organization do security awareness training for all employees at least once per year? & \ding{55} \\
\bottomrule
\end{tabular}
\end{table}

\subsection*{Analysis of Control Gaps}
\begin{itemize}
    \item \textbf{No MFA for Email (Critical Risk):} Email is the primary target for phishing and business email compromise (BEC) attacks. The absence of MFA on email accounts makes them highly vulnerable to unauthorized access if user credentials are stolen.
    \item \textbf{Lack of Security Policies and Training (High Risk):} The absence of an Acceptable Use Policy (AUP) and a recurring security awareness training program creates an environment where employees are unaware of security best practices and organizational expectations. This significantly increases the likelihood of human error leading to a security incident.
\end{itemize}

% --- TECHNICAL SCAN RESULTS ---
\section*{Technical Scan Results}
An Nmap scan was performed on the target system \texttt{10.0.0.15}. The scan identified an open port with a critically vulnerable service.

\begin{table}[h!]
\centering
\caption{Open Port and Service Details}
\begin{tabular}{l l l l}
\toprule
\textbf{Port} & \textbf{Service} & \textbf{Version} & \textbf{Notes} \\
\midrule
21/tcp & ftp & vsftpd 2.3.4 & \textbf{Critical:} Anonymous FTP login allowed. \\
\bottomrule
\end{tabular}
\end{table}

\subsection*{Analysis of Technical Findings}
The scan revealed that port 21 (FTP) is open and running \textbf{vsftpd version 2.3.4}. This specific version contains a critical backdoor vulnerability, cataloged as \textbf{CVE-2011-2523}. An attacker can exploit this vulnerability to gain a command shell on the underlying server, leading to a full system compromise.

The risk is severely amplified by the finding that \textbf{anonymous FTP login is allowed}. This misconfiguration permits any unauthenticated user on the internet to connect to the server and potentially access, upload, or modify files, depending on the directory permissions. This represents an immediate and severe threat to the organization's data.

% --- CONSOLIDATED RISK ASSESSMENT ---
\section*{Consolidated Risk Assessment}
The following table synthesizes findings from the technical scan, control review, and pre-existing risk data into a consolidated list, prioritized by severity.

\begin{table}[h!]
\centering
\caption{Summary of Identified Risks}
\begin{tabular}{p{0.25\linewidth} p{0.45\linewidth} l}
\toprule
\textbf{Risk Name} & \textbf{Description} & \textbf{Severity} \\
\midrule
\textbf{Exposed Vulnerable FTP Service} & An outdated FTP server (vsftpd 2.3.4) with a known RCE backdoor (CVE-2011-2523) is exposed. Anonymous login is enabled. & \textbf{Critical} \\
\addlinespace
\textbf{Lack of MFA on Email} & Email accounts are protected only by passwords, making them highly susceptible to phishing and account takeover. & \textbf{Critical} \\
\addlinespace
\textbf{Inadequate Security Policies \& Training} & No AUP or security awareness training program exists, increasing the risk of security incidents caused by human error. & \textbf{High} \\
\addlinespace
\textbf{Outdated Windows Policy} & Workstations are running Windows 7, an unsupported operating system that no longer receives security updates. & \textbf{Medium} \\
\bottomrule
\end{tabular}
\end{table}

% --- PRIORITIZED RECOMMENDATIONS ---
\section*{Prioritized Recommendations}
The following actions are recommended to mitigate the identified risks. They are prioritized based on severity and potential impact.

\subsection*{Immediate Actions (To Be Completed Within 72 Hours)}
\begin{enumerate}
    \item \textbf{Remediate Vulnerable FTP Server:}
    \begin{itemize}
        \item Immediately take the FTP server offline by shutting down the vsftpd service on host \texttt{10.0.0.15}.
        \item If FTP access is business-critical, migrate to a secure file transfer protocol like SFTP (SSH File Transfer Protocol) or FTPS (FTP over SSL/TLS).
        \item If vsftpd must be used, upgrade it to the latest stable version and ensure anonymous access is explicitly disabled.
    \end{itemize}
    \item \textbf{Enforce MFA on All Email Accounts:}
    \begin{itemize}
        \item Immediately enable and enforce MFA for all user accounts within the \texttt{NorthStarEducation.org} email system.
    \end{itemize}
\end{enumerate}

\subsection*{High-Priority Actions (To Be Completed Within 30 Days)}
\begin{enumerate}
    \item \textbf{Develop and Implement Security Policies:}
    \begin{itemize}
        \item Draft and formally adopt an Employee Acceptable Use Policy (AUP) that clearly defines rules for using company assets, data handling, and internet usage.
    \end{itemize}
    \item \textbf{Establish Security Awareness Training:}
    \begin{itemize}
        \item Procure and implement a security awareness training program for all employees.
        \item Ensure all new hires complete the training as part of their onboarding process.
        \item Mandate that all staff complete refresher training at least annually.
    \end{itemize}
\end{enumerate}

\subsection*{Ongoing / Planned Actions}
\begin{enumerate}
    \item \textbf{Continue Operating System Upgrades:}
    \begin{itemize}
        \item Continue with the existing plan to upgrade all workstations from Windows 7 to a modern, supported operating system like Windows 10 or 11 to mitigate risks associated with unsupported software.
    \end{itemize}
\end{enumerate}

\end{document}
```