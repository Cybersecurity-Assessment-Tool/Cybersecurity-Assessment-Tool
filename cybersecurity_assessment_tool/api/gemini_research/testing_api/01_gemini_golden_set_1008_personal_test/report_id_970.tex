```latex
\documentclass[12pt]{article}

% Preamble: Required Packages
\usepackage[margin=1in]{geometry}
\usepackage{pifont} % For checkmarks and crosses
\usepackage{booktabs} % For professional tables
\usepackage{hyperref} % For clickable links
\usepackage{url}      % For formatting URLs
\usepackage{seqsplit} % For splitting long strings in tt font
\usepackage{graphicx} % For potential logos
\usepackage{xcolor}   % For colors in tables

% Document Metadata
\title{Cybersecurity Posture Assessment Report}
\author{Cybersecurity Analyst}
\date{\today}

% Hyperref Setup
\hypersetup{
    colorlinks=true,
    linkcolor=blue,
    filecolor=magenta,      
    urlcolor=cyan,
    pdftitle={Cybersecurity Posture Assessment Report},
    pdfpagemode=FullScreen,
}

\begin{document}

\maketitle
\thispagestyle{empty}
\newpage

\tableofcontents
\newpage

\section{Executive Summary}

This report details the findings of a cybersecurity posture assessment for \textbf{Tidal Wave Sports}. The assessment combined a review of organizational security controls via a questionnaire, an external network vulnerability scan, and an analysis of pre-existing risks.

The overall security posture presents a mixed landscape. The organization demonstrates a strong commitment to identity and access management, with Multi-Factor Authentication (MFA) consistently enforced across email, computers, and sensitive data systems. From an external network perspective, the target host at \texttt{[Target IP]} showed no exposed services, indicating a robust firewall configuration.

However, significant procedural and governance gaps were identified. The absence of a formal Employee Acceptable Use Policy represents a \textbf{Critical} risk, leaving the organization vulnerable to insider threats and misuse of assets. Furthermore, the lack of mandatory annual security awareness training for all employees constitutes a \textbf{High} risk, as it increases susceptibility to evolving social engineering tactics like phishing.

Immediate remediation should focus on establishing these foundational governance controls to complement the existing technical strengths and build a more resilient security program.

\section{Organizational Information}

The following information was provided for the assessment.

\begin{tabular}{@{}ll}
\toprule
\textbf{Attribute} & \textbf{Value} \\
\midrule
Organization Name & Tidal Wave Sports \\
Email Domain & \seqsplit{\texttt{TidalWaveSports.net}} \\
Website Domain & \seqsplit{\url{www.TidalWaveSports.net}} \\
External IP Scanned & \seqsplit{\texttt{1.92.194.91}} \\
\bottomrule
\end{tabular}

\section{Security Control Review}

A review of internal security controls was conducted based on a standardized questionnaire. The responses highlight key areas of strength and weakness in the organization's policies and procedures. "No" answers indicate significant gaps that require attention.

\begin{table}[h!]
\centering
\begin{tabular}{p{0.8\linewidth} c}
\toprule
\textbf{Control Question} & \textbf{Response} \\
\midrule
Do you require MFA to access email? & \ding{51} \\ % Yes
Do you require MFA to log into computers? & \ding{51} \\ % Yes
Do you require MFA to access sensitive data systems? & \ding{51} \\ % Yes
Does your organization have an employee acceptable use policy? & \textcolor{red}{\ding{55}} \\ % No
Does your organization do security awareness training for new employees? & \ding{51} \\ % Yes
Does your organization do security awareness training for all employees at least once per year? & \textcolor{red}{\ding{55}} \\ % No
\bottomrule
\end{tabular}
\caption{Organizational Security Control Questionnaire Results.}
\end{table}

\section{Technical Scan Results}

An external network scan was performed to identify exposed services and potential vulnerabilities on the perimeter.

\begin{itemize}
    \item \textbf{Target IP Address:} \texttt{[Target IP]}
    \item \textbf{Scan Date:} \today
\end{itemize}

\textbf{Findings:} The scan completed successfully and did not identify any open TCP or UDP ports on the target system. All probes were dropped or rejected, suggesting a well-configured firewall or network access control list (ACL) is in place. This is a positive security finding, as it significantly reduces the external attack surface.

\section{Risk Assessment}

This section synthesizes findings from the security control review, technical scan, and pre-existing risk data. The risks are prioritized by severity to guide remediation efforts. No pre-existing vulnerabilities were reported.

\begin{table}[h!]
\centering
\begin{tabular}{p{0.25\linewidth} p{0.5\linewidth} p{0.15\linewidth}}
\toprule
\textbf{Risk Name} & \textbf{Overview} & \textbf{Severity} \\
\midrule
\textbf{Lack of Acceptable Use Policy (AUP)} & The absence of a formal AUP means there are no clear, enforceable rules for employees regarding the use of company IT assets. This exposes the organization to insider threats, data leakage, and legal liability. & \textbf{Critical} \\
\addlinespace
\textbf{Inadequate Security Awareness Training} & While new hires receive training, the lack of a mandatory annual refresher program for all staff increases susceptibility to phishing, social engineering, and other human-centric attacks. Threat landscapes evolve, and employee knowledge must be kept current. & \textbf{High} \\
\bottomrule
\end{tabular}
\caption{Identified Risks and Severity.}
\end{table}

\section{Recommendations}

The following actions are recommended to mitigate the identified risks and improve the overall security posture of \textbf{Tidal Wave Sports}.

\begin{description}
    \item[\textbf{1. Develop and Implement an Acceptable Use Policy (AUP):}]
        \begin{itemize}
            \item \textbf{Action:} Draft a comprehensive AUP that clearly defines the rules for using company networks, computers, email, and data. The policy should cover topics such as data privacy, password security, prohibited activities, and consequences for violation.
            \item \textbf{Priority:} Critical
            \item \textbf{Impact:} Establishes a clear security baseline for all employees, reduces the risk of misuse of assets, and provides a legal framework for enforcement.
        \end{itemize}

    \item[\textbf{2. Establish a Mandatory Annual Security Training Program:}]
        \begin{itemize}
            \item \textbf{Action:} Implement a mandatory security awareness training program for all employees, to be completed annually. The training should be updated each year to cover the latest threats, such as new phishing techniques and ransomware trends.
            \item \textbf{Priority:} High
            \item \textbf{Impact:} Creates a security-conscious culture, reduces the likelihood of successful social engineering attacks, and ensures the entire organization remains vigilant against evolving cyber threats.
        \end{itemize}
\end{description}

\section{Conclusion}

\textbf{Tidal Wave Sports} has established a strong technical foundation with excellent MFA implementation and a secure network perimeter. To mature its cybersecurity program, the organization must now focus on the identified procedural and governance weaknesses. By developing an Acceptable Use Policy and implementing annual security awareness training, \textbf{Tidal Wave Sports} can significantly reduce its risk exposure and build a more resilient, defense-in-depth security posture.

\end{document}
```