```latex
\documentclass[12pt]{article}

% Preamble: Required Packages
\usepackage[margin=1in]{geometry}
\usepackage{pifont} % For checkmarks and crosses
\usepackage{booktabs} % For professional tables
\usepackage{hyperref} % For clickable links
\usepackage{url} % For formatting URLs
\usepackage{seqsplit} % For splitting long strings to prevent overflow

% Document Metadata
\title{Cybersecurity Posture Assessment Report}
\author{Cybersecurity Analyst}
\date{\today}

\begin{document}

\maketitle
\thispagestyle{empty}
\newpage
\tableofcontents
\newpage

%======================================================================
\section{Executive Summary}
%======================================================================

This report details the findings of a cybersecurity assessment conducted for \textbf{Harbor Light Foundation}. The assessment combined a review of organizational security controls via a questionnaire, an external network scan, and an analysis of pre-existing risks.

The primary finding of this assessment is the presence of \textbf{critical gaps in identity and access management controls}, specifically the lack of Multi-Factor Authentication (MFA) across all critical systems. Additionally, significant administrative weaknesses were identified, including the absence of a formal employee acceptable use policy and a security awareness program for new hires.

From a technical network perspective, the scanned target at \texttt{192.168.1.100} demonstrated a strong security posture, with no open ports detected. While this is a positive finding, it does not mitigate the severe risks posed by the administrative and identity-related control deficiencies. An attacker could bypass network defenses entirely by compromising user credentials.

Immediate remediation efforts should focus on implementing MFA, developing foundational security policies, and enhancing the security awareness training program.

%======================================================================
\section{Organizational Information}
%======================================================================

The following information was provided for the assessment scope.

\begin{itemize}
    \item \textbf{Organization Name:} Harbor Light Foundation
    \item \textbf{Email Domain:} \texttt{HarborLightFoundation.net}
    \item \textbf{External IP Address:} \texttt{216.13.232.176}
\end{itemize}

%======================================================================
\section{Security Control Review}
%======================================================================

A review of administrative and technical security controls was conducted based on a questionnaire. The results are summarized in the table below. Answers marked with \ding{55} (No) indicate a potential security gap that requires attention.

\begin{table}[h!]
\centering
\caption{Security Controls Questionnaire Results}
\begin{tabular}{@{}lc@{}}
\toprule
\textbf{Control Question} & \textbf{Response} \\ \midrule
Do you require MFA to access email? & \ding{55} \\
Do you require MFA to log into computers? & \ding{55} \\
Do you require MFA to access sensitive data systems? & \ding{55} \\
Does your organization have an employee acceptable use policy? & \ding{55} \\
Does your organization do security awareness training for new employees? & \ding{55} \\
Does your organization do security awareness training for all employees at least once per year? & \ding{51} \\ \bottomrule
\end{tabular}
\end{table}

\subsection*{Analysis of Control Gaps}
The questionnaire revealed five significant control gaps:
\begin{itemize}
    \item \textbf{Lack of MFA:} The absence of MFA for email, computer logins, and sensitive data access is a critical vulnerability. This significantly increases the risk of account takeover via phishing or credential stuffing attacks.
    \item \textbf{Missing Acceptable Use Policy (AUP):} Without an AUP, there are no formal guidelines for employees regarding the secure use of company assets. This can lead to inconsistent security practices and unintentional data exposure.
    \item \textbf{No Onboarding Security Training:} New employees are often a primary target for social engineering. Failing to provide security training during onboarding leaves the organization vulnerable from day one of an employee's tenure.
\end{itemize}

%======================================================================
\section{Technical Scan Results}
%======================================================================

A network scan was performed to identify open ports and services on the specified target system.

\begin{itemize}
    \item \textbf{Target IP:} \texttt{192.168.1.100}
    \item \textbf{Scan Date:} \today
    \item \textbf{Status:} Host is up.
\end{itemize}

\subsection*{Findings}
The scan determined that the host was responsive but reported \textbf{zero open TCP ports}. All 1000 scanned ports were in a 'closed' state.

\paragraph{Interpretation:} This is a positive security finding. It indicates that the target system is either not running any network services or is protected by a well-configured firewall that blocks all unsolicited inbound connections. This configuration significantly reduces the external attack surface of the device.

%======================================================================
\section{Risk Assessment Summary}
%======================================================================

This section correlates the findings from the security control review and technical scan to provide a summary of the key risks facing the organization. No pre-existing vulnerabilities were provided for this assessment.

\begin{table}[h!]
\centering
\caption{Identified Risks}
\begin{tabular}{@{}p{0.25\linewidth}p{0.55\linewidth}p{0.1\linewidth}@{}}
\toprule
\textbf{Risk Name} & \textbf{Overview} & \textbf{Severity} \\ \midrule
\textbf{Lack of Multi-Factor Authentication (MFA)} & The absence of MFA on email, endpoints, and sensitive systems makes user accounts highly susceptible to takeover from stolen or weak credentials. & \textbf{Critical} \\
\addlinespace
\textbf{Missing Employee Acceptable Use Policy} & Without a formal policy, employees lack clear guidance on security expectations, increasing the likelihood of risky behavior and insider threat (unintentional or otherwise). & High \\
\addlinespace
\textbf{Inadequate Security Awareness Training} & Failing to train new hires on security best practices leaves a critical window of vulnerability. While annual training exists, the lack of onboarding training is a major gap. & High \\
\bottomrule
\end{tabular}
\end{table}

%======================================================================
\section{Recommendations}
%======================================================================

Based on the identified risks, the following prioritized actions are recommended to improve the cybersecurity posture of \textbf{Harbor Light Foundation}.

\subsection*{Immediate Actions (Critical Risk)}
\begin{enumerate}
    \item \textbf{Implement Multi-Factor Authentication (MFA):}
    \begin{itemize}
        \item \textbf{Action:} Immediately enable and enforce MFA for all users across all systems, prioritizing email, remote access solutions (VPN), and access to sensitive data.
        \item \textbf{Justification:} This is the single most effective control to prevent unauthorized access resulting from compromised credentials.
    \end{itemize}
\end{enumerate}

\subsection*{Near-Term Actions (High Risks)}
\begin{enumerate}
    \setcounter{enumi}{1} % Continue numbering from previous list
    \item \textbf{Develop and Implement an Acceptable Use Policy (AUP):}
    \begin{itemize}
        \item \textbf{Action:} Draft a formal AUP that clearly defines the rules and expectations for employees when using company-owned IT assets, networks, and data. Require all employees to read and acknowledge the policy.
        \item \textbf{Justification:} An AUP establishes a baseline for secure behavior, reduces ambiguity, and provides a framework for enforcing security standards.
    \end{itemize}
    \item \textbf{Enhance the Security Awareness Training Program:}
    \begin{itemize}
        \item \textbf{Action:} Develop a mandatory security awareness training module for all new employees to be completed as part of their onboarding process. This should cover key topics like phishing, password hygiene, and the new AUP.
        \item \textbf{Justification:} Training new hires immediately reduces their susceptibility to common attacks and instills a security-conscious mindset from the beginning of their employment.
    \end{itemize}
\end{enumerate}

\end{document}
```