```latex
\documentclass[12pt]{article}

% Preamble: Required Packages
\usepackage[margin=1in]{geometry}
\usepackage{pifont} % For checkmarks and crosses
\usepackage{booktabs} % For professional tables
\usepackage{hyperref} % For hyperlinks
\usepackage{url} % For URL formatting
\usepackage{seqsplit} % To split long strings in texttt
\usepackage{graphicx}
\usepackage{xcolor}
\usepackage{titling}

% Document Metadata
\title{Cybersecurity Posture Assessment Report}
\author{Cybersecurity Analysis Division}
\date{\today}

\begin{document}

\maketitle
\thispagestyle{empty}
\newpage

\tableofcontents
\newpage

\section*{Executive Summary}

This report provides a cybersecurity posture assessment for \textbf{Foresight Strategies}, based on a review of organizational security controls, an external network scan, and pre-existing risk data.

The assessment identified several critical and high-risk gaps in the organization's administrative and policy-based security controls. The most severe finding is the lack of Multi-Factor Authentication (MFA) for email access, which exposes the organization to significant risk of account compromise and subsequent data breaches. Additionally, the absence of an employee acceptable use policy and a mandatory annual security awareness training program for all staff members represents a substantial weakness in the overall security culture and governance.

On a technical level, the external network scan of the designated target IP address revealed no open ports. This is a positive finding, suggesting a strong firewall configuration and a minimal external attack surface for the scanned asset.

Recommendations prioritize the immediate implementation of MFA for email, the development of key security policies, and the establishment of a recurring security training program to mitigate the identified risks effectively.

\section{Organizational Information}

The following details were provided for the assessment.

\begin{tabular}{@{}ll}
\toprule
\textbf{Attribute} & \textbf{Value} \\
\midrule
Organization Name & \textbf{Foresight Strategies} \\
Email Domain & \texttt{ForesightStrategies.org} \\
Website Domain & \url{www.ForesightStrategies.org} \\
External IP Scanned & \texttt{132.180.107.97} \\
\bottomrule
\end{tabular}

\section{Security Control Review (Questionnaire Analysis)}

An analysis of the security questionnaire reveals a mixed implementation of fundamental security controls. While MFA is enforced for some systems, critical gaps exist in protecting email and establishing foundational security policies.

\begin{tabular}{@{}p{0.6\linewidth} c p{0.2\linewidth}@{}}
\toprule
\textbf{Control Question} & \textbf{Response} & \textbf{Assessment} \\
\midrule
Do you require MFA to access email? & \ding{55} No & \textcolor{red}{\textbf{Critical Gap}} \\
Do you require MFA to log into computers? & \ding{51} Yes & Good Practice \\
Do you require MFA to access sensitive data systems? & \ding{51} Yes & Good Practice \\
Does your organization have an employee acceptable use policy? & \ding{55} No & \textcolor{orange}{High Risk} \\
Does your organization do security awareness training for new employees? & \ding{51} Yes & Good Practice \\
Does your organization do security awareness training for all employees at least once per year? & \ding{55} No & \textcolor{orange}{High Risk} \\
\bottomrule
\end{tabular}

\section{Technical Scan Results}

An external network vulnerability scan was conducted to identify open ports and exposed services on the public-facing infrastructure.

\begin{itemize}
    \item \textbf{Target IP Address:} \texttt{[Target IP]}
    \item \textbf{Scan Summary:} The scan completed successfully and found \textbf{no open ports}.
    \item \textbf{Analyst Notes:} This result is positive, indicating that the target system has a strong firewall configuration that does not expose any services to the public internet. This significantly reduces the external attack surface for this specific asset.
\end{itemize}

\section{Risk Assessment}

The following table synthesizes findings from the security control review and technical scan. No pre-existing vulnerabilities were provided for this assessment. The risks below are derived directly from this analysis.

\begin{tabular}{@{}p{0.1\linewidth} p{0.2\linewidth} p{0.5\linewidth} p{0.12\linewidth}@{}}
\toprule
\textbf{Risk ID} & \textbf{Risk Name} & \textbf{Description} & \textbf{Severity} \\
\midrule
RISK-001 & Lack of MFA on Email & The absence of MFA on email accounts makes them highly susceptible to compromise via phishing or credential stuffing. A compromised email account is a primary vector for business email compromise (BEC), data exfiltration, and further network intrusion. & \textcolor{red}{Critical} \\
\addlinespace
RISK-002 & No Annual Security Training & Without regular, recurring security awareness training, employees' ability to recognize and respond to threats like phishing and social engineering diminishes over time, increasing the likelihood of a security incident. & \textcolor{orange}{High} \\
\addlinespace
RISK-003 & No Acceptable Use Policy (AUP) & The lack of a formal AUP creates ambiguity regarding the proper use of company assets. This can lead to unintentional insider threats, legal liabilities, and difficulty in enforcing security standards. & \textcolor{orange}{High} \\
\bottomrule
\end{tabular}

\section{Recommendations}

Based on the findings of this assessment, we recommend the following actions, prioritized by severity, to enhance the cybersecurity posture of \textbf{Foresight Strategies}.

\begin{enumerate}
    \item \textbf{[Critical] Enforce MFA on All Email Accounts:} Immediately implement and enforce the use of Multi-Factor Authentication for all user email accounts. This is the single most effective control to mitigate the risk of account takeovers.
    
    \item \textbf{[High] Develop and Implement an Acceptable Use Policy (AUP):} Create a formal AUP that clearly defines the rules and expectations for employees when using company networks, systems, and data. Require all employees to read and acknowledge the policy.
    
    \item \textbf{[High] Establish a Mandatory Annual Security Training Program:} Institute a recurring security awareness training program for all employees, to be completed at least once per year. This program should cover current threats such as phishing, ransomware, and social engineering.
    
    \item \textbf{[Informational] Maintain Strong Network Perimeter Security:} Continue to maintain the current firewall configuration that limits exposed services on external-facing assets. Regularly review and audit firewall rules to ensure they remain effective.
\end{enumerate}

\end{document}
```