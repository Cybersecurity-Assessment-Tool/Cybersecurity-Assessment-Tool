```latex
\documentclass[12pt]{article}

% Preamble: Required Packages
\usepackage[margin=1in]{geometry}
\usepackage{pifont} % For checkmarks and crosses
\usepackage{booktabs} % For professional tables
\usepackage{hyperref} % For clickable links
\usepackage{url} % For formatting URLs
\usepackage{seqsplit} % For splitting long strings in texttt
\usepackage{graphicx}
\usepackage{xcolor}

% Document Metadata
\title{Cybersecurity Posture Assessment Report \\ \large For: Green Sprout Organic}
\author{Cybersecurity Analyst Group}
\date{November 22, 2025}

% Hyperref Setup
\hypersetup{
    colorlinks=true,
    linkcolor=blue,
    filecolor=magenta,      
    urlcolor=cyan,
    pdftitle={Cybersecurity Posture Assessment Report},
    pdfpagemode=FullScreen,
}

\begin{document}

\maketitle
\thispagestyle{empty}
\newpage

\tableofcontents
\newpage

% --- 1. Executive Summary ---
\section{Executive Summary}

This report details the cybersecurity posture of Green Sprout Organic, based on a technical network scan performed on November 22, 2025, and a review of organizational security controls. The assessment identified several high-risk areas requiring immediate attention.

Critical gaps were discovered in identity and access management, most notably the \textbf{lack of Multi-Factor Authentication (MFA) for email access}. This exposes the organization to a significant risk of business email compromise and unauthorized data access. Furthermore, the assessment revealed a complete absence of foundational security policies and employee training programs, including an acceptable use policy and security awareness training.

From a technical standpoint, the external-facing web server was found to be running an \textbf{outdated version of Nginx (1.18.0)}, which is no longer receiving security updates and may contain known vulnerabilities.

These findings, when correlated, create a high-risk environment. The combination of weak access controls, lack of user awareness, and a technically vulnerable perimeter presents a clear and present danger to the confidentiality, integrity, and availability of Green Sprout Organic's data and systems. Remediation of these issues should be prioritized.

% --- 2. Organizational Information ---
\section{Organizational Information}

The following information was provided for the assessment.

\begin{table}[h!]
\centering
\begin{tabular}{@{}ll@{}}
\toprule
\textbf{Attribute} & \textbf{Value} \\
\midrule
Organization Name & Green Sprout Organic \\
Email Domain & \texttt{GreenSproutOrganic.net} \\
Website Domain & \seqsplit{\texttt{www.GreenSproutOrganic.net}} \\
External IP Address & \texttt{43.206.119.56} \\
\bottomrule
\end{tabular}
\caption{Client Organizational Data.}
\end{table}

% --- 3. Security Control Review ---
\section{Security Control Review}

A review of administrative and organizational security controls was conducted via a questionnaire. The results highlight significant gaps in security governance and user access policies. Answers marked with \textcolor{red}{\ding{55}} indicate a deviation from security best practices and represent an area of risk.

\begin{table}[h!]
\centering
\begin{tabular}{@{}p{0.8\linewidth}c@{}}
\toprule
\textbf{Control Question} & \textbf{Status} \\
\midrule
Do you require MFA to access email? & \textcolor{red}{\ding{55}} \\
Do you require MFA to log into computers? & \textcolor{green}{\ding{51}} \\
Do you require MFA to access sensitive data systems? & \textcolor{green}{\ding{51}} \\
Does your organization have an employee acceptable use policy? & \textcolor{red}{\ding{55}} \\
Does your organization do security awareness training for new employees? & \textcolor{red}{\ding{55}} \\
Does your organization do security awareness training for all employees at least once per year? & \textcolor{red}{\ding{55}} \\
\bottomrule
\end{tabular}
\caption{Security Controls Questionnaire Results.}
\end{table}

% --- 4. Technical Scan Results ---
\section{Technical Scan Results}

An external network scan was performed against the target host \texttt{192.168.10.5} on the specified date. The scan identified one open port running a web server.

\begin{itemize}
    \item \textbf{Scan Date:} 2025-11-22T10:00:00Z
    \item \textbf{Target IP:} \texttt{192.168.10.5}
\end{itemize}

\begin{table}[h!]
\centering
\begin{tabular}{@{}lllll@{}}
\toprule
\textbf{Port} & \textbf{State} & \textbf{Service} & \textbf{Product} & \textbf{Version} \\
\midrule
443/tcp & open & https & nginx & 1.18.0 \\
\bottomrule
\end{tabular}
\caption{Open Ports and Services Detected on \texttt{192.168.10.5}.}
\end{table}

\subsection*{Analysis of Technical Findings}
The scan revealed that the web server is running \textbf{Nginx version 1.18.0}. This version was released in April 2020. As of the date of this report, it is considered outdated and is no longer part of the actively supported development branches. Running outdated software exposes the server to publicly known vulnerabilities that have been patched in newer versions.

% --- 5. Risk Assessment ---
\section{Risk Assessment}

The following risks were identified by correlating the security control gaps with the technical scan results. The pre-existing risk register was empty.

\begin{table}[h!]
\centering
\begin{tabular}{@{}p{0.1\linewidth}p{0.4\linewidth}p{0.3\linewidth}l@{}}
\toprule
\textbf{ID} & \textbf{Risk Description} & \textbf{Affected Asset(s)} & \textbf{Severity} \\
\midrule
\textbf{RISK-001} & Lack of MFA on email accounts allows for account takeover via credential theft (e.g., phishing, password reuse). & Email System, User Credentials, Sensitive Data & \textbf{Critical} \\
\addlinespace
\textbf{RISK-002} & The public-facing web server is running an outdated version of Nginx, potentially exposing it to known remote code execution or denial-of-service vulnerabilities. & Web Server (\texttt{192.168.10.5}), Website Data & \textbf{High} \\
\addlinespace
\textbf{RISK-003} & Absence of security policies (AUP) and awareness training increases the likelihood of human error leading to security incidents. & All Employees, Organizational Data, IT Systems & \textbf{High} \\
\bottomrule
\end{tabular}
\caption{Summary of Identified Risks.}
\end{table}

% --- 6. Recommendations ---
\section{Recommendations}

The following actions are recommended to mitigate the identified risks. Recommendations are prioritized based on severity.

\subsection*{RISK-001: Lack of MFA on Email (Critical)}
\begin{itemize}
    \item \textbf{Immediate Action:} Procure and implement an MFA solution for the organization's email platform.
    \item \textbf{Policy:} Enforce mandatory MFA enrollment for all users, including administrative and service accounts, without exception.
    \item \textbf{Timeline:} Within 14 days.
\end{itemize}

\subsection*{RISK-002: Outdated Nginx Web Server (High)}
\begin{itemize}
    \item \textbf{Immediate Action:} Develop a patch management plan. Schedule a maintenance window to upgrade the Nginx server on \texttt{192.168.10.5} from version 1.18.0 to the latest stable version.
    \item \textbf{Ongoing Action:} Implement a regular vulnerability scanning and patch management cycle for all internet-facing systems to prevent software from becoming outdated in the future.
    \item \textbf{Timeline:} Within 30 days.
\end{itemize}

\subsection*{RISK-003: Absence of Security Policies and Training (High)}
\begin{itemize}
    \item \textbf{Immediate Action:} Develop and ratify a formal Acceptable Use Policy (AUP) that all employees must read and sign.
    \item \textbf{Mid-Term Action:} Institute a mandatory security awareness training program for all employees. This program should be completed by new hires during onboarding and by all staff at least annually. Training should cover topics such as phishing, password security, and data handling.
    \item \textbf{Timeline:} Within 60 days.
\end{itemize}

% --- 7. Disclaimer ---
\section{Disclaimer}
This report is based on the information provided and the results of a point-in-time network scan. The security landscape is dynamic, and new vulnerabilities can emerge at any time. The findings and recommendations in this document are intended to help improve the organization's security posture but do not guarantee immunity from security incidents.

\end{document}
```