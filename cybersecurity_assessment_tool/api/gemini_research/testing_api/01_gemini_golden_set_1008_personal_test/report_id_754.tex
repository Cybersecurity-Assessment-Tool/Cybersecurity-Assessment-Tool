```latex
\documentclass[12pt]{article}

% Preamble: Required Packages
\usepackage[margin=1in]{geometry}
\usepackage{pifont} % For checkmarks and crosses
\usepackage{booktabs} % For professional tables
\usepackage{hyperref} % For clickable links
\usepackage{url}      % For proper URL formatting
\usepackage{seqsplit} % For splitting long text strings
\usepackage{xcolor}   % For colors in the report

% Document Information
\title{Cybersecurity Posture Assessment Report}
\author{Cybersecurity Analysis Division}
\date{\today}

% Hyperref Setup
\hypersetup{
    colorlinks=true,
    linkcolor=blue,
    filecolor=magenta,      
    urlcolor=cyan,
    pdftitle={Cybersecurity Posture Assessment Report},
    pdfpagemode=FullScreen,
}

\begin{document}

\maketitle
\hrule
\vspace{1em}

% --- 1. Executive Summary ---
\section*{Executive Summary}

This report provides a comprehensive analysis of the cybersecurity posture for \textbf{New Era}. The assessment is based on a correlation of data from a technical network scan, a security controls questionnaire, and a review of pre-existing risks.

The external network scan of the target IP address, \seqsplit{\texttt{[Target IP]}}, revealed no open ports. This indicates a strong perimeter firewall configuration, which is a significant security strength. However, the security controls review identified several critical internal weaknesses. The most severe findings include the absence of Multi-Factor Authentication (MFA) for accessing email and sensitive data systems, and a lack of mandatory security awareness training for new employees.

These gaps expose the organization to significant risks, including business email compromise, data breaches, and social engineering attacks. This report outlines these risks in detail and provides actionable recommendations to mitigate them and enhance the overall security posture.

% --- 2. Organizational Information ---
\section{Organizational Information}

The following details were provided for the assessment.

\begin{tabular}{@{}ll}
    \toprule
    \textbf{Attribute} & \textbf{Value} \\
    \midrule
    Organization Name & \textbf{New Era} \\
    Primary Email Domain & \seqsplit{\texttt{NewEra.net}} \\
    Primary Website & \url{www.NewEra.net} \\
    External IP Address & \seqsplit{\texttt{213.83.10.7}} \\
    \bottomrule
\end{tabular}

% --- 3. Security Control Review ---
\section{Security Control Review}

The following table summarizes the organization's responses to a security controls questionnaire. "No" answers represent significant gaps in the security framework and are highlighted for immediate attention.

\begin{table}[h!]
\centering
\begin{tabular}{@{}p{8cm} c p{4cm}@{}}
    \toprule
    \textbf{Control Question} & \textbf{Response} & \textbf{Assessment} \\
    \midrule
    Do you require MFA to access email? & \textcolor{red}{\ding{55}} & \textbf{Critical Gap} \\
    \addlinespace
    Do you require MFA to log into computers? & \textcolor{green}{\ding{51}} & Best Practice Met \\
    \addlinespace
    Do you require MFA to access sensitive data systems? & \textcolor{red}{\ding{55}} & \textbf{Critical Gap} \\
    \addlinespace
    Does your organization have an employee acceptable use policy? & \textcolor{green}{\ding{51}} & Best Practice Met \\
    \addlinespace
    Does your organization do security awareness training for new employees? & \textcolor{red}{\ding{55}} & \textbf{High Risk} \\
    \addlinespace
    Does your organization do security awareness training for all employees at least once per year? & \textcolor{green}{\ding{51}} & Best Practice Met \\
    \bottomrule
\end{tabular}
\caption{Security Controls Questionnaire Analysis}
\end{table}

% --- 4. Technical Scan Results ---
\section{Technical Scan Results}

A network scan was performed to identify open ports and exposed services on the organization's external infrastructure.

\begin{itemize}
    \item \textbf{Target IP Address:} \seqsplit{\texttt{[Target IP]}}
    \item \textbf{Scan Date:} [Scan Date Not Provided]
    \item \textbf{Findings:} The scan completed successfully and found \textbf{no open ports}. This suggests that the perimeter firewall is properly configured to deny unsolicited inbound traffic, which is an excellent security practice. No vulnerabilities could be identified from this external perspective.
\end{itemize}

% --- 5. Risk Assessment ---
\section{Risk Assessment}

This section synthesizes findings from the security control review and technical scan. While the external posture is strong, critical internal control deficiencies present a high level of risk. No pre-existing vulnerabilities were reported.

\begin{table}[h!]
\centering
\begin{tabular}{@{}p{4cm} p{7cm} l@{}}
    \toprule
    \textbf{Risk Name} & \textbf{Overview} & \textbf{Severity} \\
    \midrule
    \textbf{Lack of MFA on Email} & The absence of MFA on email accounts makes them highly susceptible to phishing and credential stuffing attacks. A compromised email account can lead to business email compromise (BEC), data exfiltration, and further internal network compromise. & \textbf{Critical} \\
    \addlinespace
    \textbf{Lack of MFA on Sensitive Data} & Sensitive data systems are not protected by MFA, leaving them vulnerable to unauthorized access if credentials are stolen. This directly risks a major data breach of confidential customer or company information. & \textbf{Critical} \\
    \addlinespace
    \textbf{Inadequate New Hire Training} & New employees are not provided with security awareness training upon joining. This makes them more likely to fall victim to social engineering or phishing attacks, as they are unfamiliar with organizational security policies and common threats. & \textbf{High} \\
    \bottomrule
\end{tabular}
\caption{Summary of Identified Risks}
\end{table}

% --- 6. Recommendations ---
\section{Recommendations}

To address the identified risks and strengthen the organization's security posture, the following actions are recommended with high priority:

\begin{enumerate}
    \item \textbf{Implement MFA for Email:} Immediately enforce MFA for all user access to the email system (e.g., Office 365, Google Workspace). This is the single most effective control to prevent account takeovers.

    \item \textbf{Enforce MFA on Sensitive Systems:} Identify all systems containing sensitive or critical data and enforce MFA for all user and administrative access. Prioritize systems with financial, personal (PII), or proprietary information.

    \item \textbf{Integrate Security Training into Onboarding:} Develop and implement a mandatory security awareness training module for all new employees as part of the standard onboarding process. This training should cover topics such as phishing, acceptable use, and password security.
\end{enumerate}

\end{document}
```