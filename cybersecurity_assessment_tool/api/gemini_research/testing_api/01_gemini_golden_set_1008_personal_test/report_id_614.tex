```latex
\documentclass[12pt]{article}

% --- PACKAGES ---
\usepackage[margin=1in]{geometry} % Set page margins
\usepackage{pifont}               % For checkmarks and crosses
\usepackage{booktabs}             % For professional-looking tables
\usepackage{hyperref}             % For hyperlinks and document metadata
\usepackage{url}                  % For formatting URLs
\usepackage{seqsplit}             % To split long strings in texttt
\usepackage[T1]{fontenc}          % Font encoding

% --- DOCUMENT METADATA ---
\hypersetup{
    colorlinks=true,
    linkcolor=black,
    urlcolor=blue,
    pdftitle={Cybersecurity Posture Assessment Report},
    pdfauthor={Cybersecurity Analyst AI},
    pdfsubject={Security Analysis},
    pdfkeywords={Cybersecurity, Nmap, Risk Assessment}
}

% --- TITLE ---
\title{Cybersecurity Posture Assessment Report \\ \large For: Quantum Reach}
\author{Cybersecurity Analyst AI}
\date{\today}

\begin{document}

\maketitle
\tableofcontents
\newpage

% --- EXECUTIVE SUMMARY ---
\section{Executive Summary}
This report provides a comprehensive cybersecurity posture assessment for \textbf{Quantum Reach}. The analysis is based on a combination of technical network scanning, a review of organizational security controls, and an evaluation of pre-existing risks.

The assessment reveals a mixed security posture. The organization demonstrates strong identity and access management controls, with Multi-Factor Authentication (MFA) widely implemented. However, several critical and high-risk vulnerabilities were identified that require immediate attention.

Key findings include a critically vulnerable public-facing FTP service running an outdated version with anonymous access enabled. Furthermore, significant procedural gaps exist, including the absence of an employee Acceptable Use Policy and a lack of recurring, annual security awareness training for all staff. These issues, combined with the known risk of outdated operating systems, expose the organization to significant threats, including data breaches, ransomware, and unauthorized access.

This report details these findings and provides actionable recommendations to mitigate the identified risks and strengthen the overall security posture of the organization.

% --- ORGANIZATIONAL INFORMATION ---
\section{Organizational Information}
The following details were provided for the assessment. This information helps to establish the context and scope of the review.

\begin{tabular}{@{}ll}
    \toprule
    \textbf{Attribute} & \textbf{Value} \\
    \midrule
    Organization Name & \textbf{Quantum Reach} \\
    Email Domain & \texttt{QuantumReach.com} \\
    Website Domain & \url{www.QuantumReach.com} \\
    External IP Address & \seqsplit{\texttt{178.169.112.180}} \\
    \bottomrule
\end{tabular}

% --- SECURITY CONTROL REVIEW ---
\section{Security Control Review}
A review of organizational security controls was conducted via a questionnaire. The responses indicate foundational gaps in security policy and training procedures. A "No" response highlights an area requiring remediation.

\begin{table}[h!]
\centering
\begin{tabular}{@{}p{0.8\linewidth}c@{}}
    \toprule
    \textbf{Control Question} & \textbf{Response} \\
    \midrule
    Do you require MFA to access email? & \ding{51} \\
    Do you require MFA to log into computers? & \ding{51} \\
    Do you require MFA to access sensitive data systems? & \ding{51} \\
    Does your organization have an employee acceptable use policy? & \textbf{\color{red}\ding{55}} \\
    Does your organization do security awareness training for new employees? & \ding{51} \\
    Does your organization do security awareness training for all employees at least once per year? & \textbf{\color{red}\ding{55}} \\
    \bottomrule
\end{tabular}
\caption{Organizational Security Control Questionnaire Results.}
\end{table}

% --- TECHNICAL SCAN RESULTS ---
\section{Technical Scan Results}
An external network scan was performed to identify open ports and exposed services. The scan targeted the host at \texttt{10.0.0.15}.

\begin{table}[h!]
\centering
\begin{tabular}{@{}lllll@{}}
    \toprule
    \textbf{Port} & \textbf{State} & \textbf{Service} & \textbf{Product / Version} & \textbf{Notes} \\
    \midrule
    21/tcp & Open & ftp & vsftpd 2.3.4 & \textbf{Anonymous FTP login allowed.} \\
    \bottomrule
\end{tabular}
\caption{Open Ports and Services Detected on \texttt{10.0.0.15}.}
\end{table}

\subsection*{Analysis of Technical Findings}
The scan identified a single open port, 21/tcp, running \texttt{vsftpd version 2.3.4}. This is a significant finding for two primary reasons:
\begin{enumerate}
    \item \textbf{Outdated and Vulnerable Software:} \texttt{vsftpd 2.3.4} was released in 2011 and contains a well-documented critical backdoor vulnerability (CVE-2011-2523), which allows for remote command execution.
    \item \textbf{Insecure Configuration:} The service is configured to allow anonymous FTP logins. This permits any unauthenticated user on the internet to connect to the server and potentially access, upload, or download files, posing a severe data breach risk.
\end{enumerate}

% --- RISK ASSESSMENT SUMMARY ---
\section{Risk Assessment Summary}
The following table synthesizes findings from the technical scan, the security control review, and pre-existing risk data. Risks are categorized by severity to guide prioritization of remediation efforts.

\begin{table}[h!]
\centering
\begin{tabular}{@{}p{0.3\linewidth}p{0.15\linewidth}p{0.45\linewidth}@{}}
    \toprule
    \textbf{Risk Name} & \textbf{Severity} & \textbf{Overview} \\
    \midrule
    \textbf{Vulnerable FTP Service} & \textbf{Critical} & An outdated version of vsftpd (2.3.4) is exposed, which is known to have a remote code execution vulnerability. Anonymous login is also enabled. \\
    \addlinespace
    \textbf{Lack of Annual Security Training} & \textbf{High} & Employees do not receive recurring security awareness training, increasing susceptibility to phishing, social engineering, and other human-targeted attacks. \\
    \addlinespace
    \textbf{Missing Acceptable Use Policy} & \textbf{High} & The absence of a formal policy creates ambiguity regarding the proper use of company assets and data, weakening the organization's legal and security posture. \\
    \addlinespace
    \textbf{Outdated Windows Policy} & \textbf{Medium} & Pre-existing risk: Workstations are running Windows 7, an unsupported operating system that no longer receives security updates from Microsoft. \\
    \bottomrule
\end{tabular}
\caption{Consolidated Risk Register.}
\end{table}

% --- RECOMMENDATIONS ---
\section{Recommendations}
Based on the analysis, the following actions are recommended to mitigate the identified risks. Recommendations are prioritized from Critical to Medium.

\subsection*{Immediate Actions (Critical Priority)}
\begin{enumerate}
    \item \textbf{Remediate Vulnerable FTP Service:}
    \begin{itemize}
        \item Immediately disable anonymous access on the FTP server.
        \item If the FTP service is not essential for business operations, disable it entirely.
        \item If the service is required, upgrade \texttt{vsftpd} to the latest stable version and implement a secure configuration (e.g., require authenticated access, use FTPS). Consider migrating to a more secure file transfer protocol like SFTP (SSH File Transfer Protocol).
    \end{itemize}
\end{enumerate}

\subsection*{Short-Term Actions (High Priority)}
\begin{enumerate}
    \item \textbf{Implement Annual Security Awareness Training:}
    \begin{itemize}
        \item Procure or develop a security awareness training program covering key topics such as phishing, password security, and data handling.
        \item Mandate that all employees complete this training annually and track completion.
    \end{itemize}
    \item \textbf{Develop and Implement an Acceptable Use Policy (AUP):}
    \begin{itemize}
        \item Draft a formal AUP that clearly defines the rules for using company networks, devices, and data.
        \item Require all employees to read and acknowledge the policy as a condition of their employment.
    \end{itemize}
\end{enumerate}

\subsection*{Mid-Term Actions (Medium Priority)}
\begin{enumerate}
    \item \textbf{Address Outdated Windows Policy:}
    \begin{itemize}
        \item Continue and prioritize the project to upgrade or replace all workstations running Windows 7 with a modern, supported operating system such as Windows 10 or Windows 11.
    \end{itemize}
\end{enumerate}

\end{document}
```