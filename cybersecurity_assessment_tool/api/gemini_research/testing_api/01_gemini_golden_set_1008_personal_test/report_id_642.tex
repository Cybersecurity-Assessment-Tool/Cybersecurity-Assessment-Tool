```latex
\documentclass[12pt, a4paper]{article}

% Preamble: Required Packages
\usepackage[margin=1in]{geometry}
\usepackage{pifont} % For checkmarks and crosses
\usepackage{booktabs} % For professional tables
\usepackage{hyperref} % For clickable links
\usepackage{url} % For URL formatting
\usepackage{seqsplit} % For splitting long strings
\usepackage[table]{xcolor} % For table cell colors and text color

% Document Metadata
\title{Cybersecurity Assessment Report \\ \large Prepared for Maple Leaf Logistics}
\author{Cybersecurity Analysis Division}
\date{\today}

% Hyperref Setup
\hypersetup{
    colorlinks=true,
    linkcolor=blue,
    filecolor=magenta,      
    urlcolor=cyan,
    pdftitle={Cybersecurity Assessment Report for Maple Leaf Logistics},
    pdfpagemode=FullScreen,
}

\begin{document}

\maketitle
\thispagestyle{empty}
\newpage

\tableofcontents
\newpage

% --- 1. Executive Summary ---
\section*{1. Executive Summary}

This report details the findings of a cybersecurity assessment conducted for \textbf{Maple Leaf Logistics}. The assessment involved a synthesis of an external network scan, a review of organizational security controls via a questionnaire, and an analysis of pre-existing risk data.

The primary findings indicate significant risks stemming from procedural and policy gaps within the organization. The absence of Multi-Factor Authentication (MFA) for email and sensitive data systems, coupled with a lack of fundamental security policies and employee training, exposes the organization to a high likelihood of account compromise, data breaches, and social engineering attacks.

On a technical level, the network scan of the target host \texttt{192.168.0.5} showed a secure posture, with no open ports detected. This finding suggests that a previously identified risk concerning an unencrypted web server has been successfully remediated.

Immediate and decisive action is required to address the identified policy and control deficiencies. Recommendations focus on the rapid implementation of MFA, the development of core security policies, and the establishment of a comprehensive security awareness training program.

% --- 2. Organizational Information ---
\section*{2. Organizational Information}

The following information was provided for the assessment.

\begin{table}[h!]
\centering
\begin{tabular}{@{}ll@{}}
\toprule
\textbf{Attribute} & \textbf{Value} \\
\midrule
Organization Name & \textbf{Maple Leaf Logistics} \\
Email Domain & \texttt{MapleLeafLogistics.org} \\
Website Domain & \url{www.MapleLeafLogistics.org} \\
External IP Address & \texttt{217.113.249.21} \\
\bottomrule
\end{tabular}
\caption{Client Organizational Details}
\end{table}

% --- 3. Security Control Review ---
\section*{3. Security Control Review}

A review of internal security controls was conducted based on a questionnaire. The responses reveal critical gaps in the organization's security posture. A summary of the findings is presented below. The status symbol \textcolor{red}{\ding{55}} indicates a significant control gap that increases risk.

\begin{table}[h!]
\centering
\begin{tabular}{@{}p{0.7\linewidth} c c@{}}
\toprule
\textbf{Control Question} & \textbf{Response} & \textbf{Status} \\
\midrule
Do you require MFA to access email? & No & \textcolor{red}{\ding{55}} \\
Do you require MFA to log into computers? & Yes & \textcolor{green}{\ding{51}} \\
Do you require MFA to access sensitive data systems? & No & \textcolor{red}{\ding{55}} \\
Does your organization have an employee acceptable use policy? & No & \textcolor{red}{\ding{55}} \\
Does your organization do security awareness training for new employees? & No & \textcolor{red}{\ding{55}} \\
Does your organization do security awareness training for all employees at least once per year? & No & \textcolor{red}{\ding{55}} \\
\bottomrule
\end{tabular}
\caption{Organizational Security Controls Questionnaire}
\end{table}

% --- 4. Technical Scan Results ---
\section*{4. Technical Scan Results}

An Nmap scan was performed on the specified target to identify open ports and exposed services. The scan results are summarized below.

\begin{table}[h!]
\centering
\begin{tabular}{@{}lllll@{}}
\toprule
\textbf{Target IP} & \textbf{Port} & \textbf{Protocol} & \textbf{State} & \textbf{Service} \\
\midrule
\texttt{192.168.0.5} & 80 & TCP & \textbf{Closed} & http \\
\bottomrule
\end{tabular}
\caption{Network Scan Findings}
\end{table}

\paragraph{Analysis:} The scan of host \texttt{192.168.0.5} revealed that port 80 (HTTP) is closed. No other open ports were detected on this host during the scan. This is a positive security finding. It is noted that this result contradicts a pre-existing risk entry (\textit{Unencrypted Web Server}), suggesting that the vulnerability has been remediated since it was last recorded.

% --- 5. Risk Assessment Summary ---
\section*{5. Risk Assessment Summary}

This section correlates the findings from the security control review, technical scan, and pre-existing risk data to provide a consolidated view of the current risk posture.

\begin{table}[h!]
\centering
\begin{tabular}{@{}p{0.2\linewidth} p{0.55\linewidth} l@{}}
\toprule
\textbf{Risk Name} & \textbf{Description} & \textbf{Severity} \\
\midrule
\rowcolor{red!20}
\textbf{No MFA on Critical Systems} & Email and sensitive data systems lack MFA, making them highly vulnerable to credential theft and unauthorized access. & \textbf{Critical} \\
\rowcolor{red!20}
\textbf{Lack of Security Policies} & The absence of an Acceptable Use Policy means there are no formal guidelines for employees on the secure use of company assets, leading to inconsistent and risky behavior. & \textbf{High} \\
\rowcolor{red!20}
\textbf{No Security Awareness Training} & Employees are not trained to recognize or respond to security threats like phishing, significantly increasing the likelihood of a successful social engineering attack. & \textbf{High} \\
\rowcolor{green!15}
Unencrypted Web Server & \textit{(Previously identified risk)} A prior assessment noted port 80 was open. The current scan confirms this port is now closed. & \textbf{Remediated} \\
\bottomrule
\end{tabular}
\caption{Consolidated Risk Register}
\end{table}

% --- 6. Recommendations ---
\section*{6. Recommendations}

Based on the risk assessment, the following actions are recommended to improve the cybersecurity posture of \textbf{Maple Leaf Logistics}. These recommendations are prioritized based on risk severity and impact.

\begin{enumerate}
    \item \textbf{Implement Multi-Factor Authentication (MFA) Immediately (CRITICAL):}
    \begin{itemize}
        \item \textbf{Action:} Enforce MFA for all user access to email services (e.g., Microsoft 365, Google Workspace) and all systems identified as containing sensitive data.
        \item \textbf{Justification:} This is the single most effective control to prevent account compromise resulting from stolen credentials.
    \end{itemize}
    
    \item \textbf{Develop and Implement a Security Training Program (HIGH):}
    \begin{itemize}
        \item \textbf{Action:} Establish a mandatory security awareness training program for all new hires. Subsequently, conduct annual refresher training for all employees. Topics should include phishing, password security, and safe browsing.
        \item \textbf{Justification:} A well-trained workforce is the first line of defense against social engineering and common cyber threats.
    \end{itemize}
    
    \item \textbf{Establish Foundational Security Policies (HIGH):}
    \begin{itemize}
        \item \textbf{Action:} Draft, approve, and disseminate an Employee Acceptable Use Policy (AUP). This policy should clearly define the rules for using company networks, devices, and data.
        \item \textbf{Justification:} Formal policies establish a baseline for secure behavior, reduce insider threats, and provide a basis for disciplinary action if violated.
    \end{itemize}
    
    \item \textbf{Conduct Regular Vulnerability Scanning (MEDIUM):}
    \begin{itemize}
        \item \textbf{Action:} Implement a schedule for regular, authenticated vulnerability scans of all internal and external network assets.
        \item \textbf{Justification:} The remediation of the Port 80 risk is positive, but continuous scanning is necessary to validate configurations and identify new vulnerabilities as they emerge.
    \end{itemize}
\end{enumerate}

\end{document}
```