```latex
\documentclass[12pt]{article}

% --- PACKAGE IMPORTS ---
\usepackage[a4paper, margin=1in]{geometry}
\usepackage{pifont} % For checkmarks and crosses
\usepackage{booktabs} % For professional tables
\usepackage{hyperref} % For hyperlinks
\usepackage{url}      % For URL formatting
\usepackage{seqsplit} % For splitting long strings to prevent overflow
\usepackage{graphicx} % For logo (optional placeholder)
\usepackage{xcolor}   % For colors

% --- DOCUMENT METADATA & STYLING ---
\hypersetup{
    colorlinks=true,
    linkcolor=blue,
    filecolor=magenta,      
    urlcolor=cyan,
    pdftitle={Cybersecurity Posture Report},
    pdfpagemode=FullScreen,
}

\newcommand{\yes}{\ding{51}} % Checkmark
\newcommand{\no}{\ding{55}}  % X-mark

\begin{document}

% --- TITLE PAGE ---
\begin{titlepage}
    \centering
    \vspace*{1cm}
    \Huge\textbf{Cybersecurity Posture Report}
    \vspace{1.5cm}
    \Large
    \textbf{Prepared for:}\\
    \vspace{0.5cm}
    \textbf{Wildfire Communications}
    \vspace{2cm}
    \large
    \textbf{Date of Report:}\\
    \today
    \vfill
    \textit{This report contains sensitive information and should be handled with care.}
\end{titlepage}

\tableofcontents
\newpage

% --- EXECUTIVE SUMMARY ---
\section*{1. Executive Summary}

This report provides a comprehensive analysis of the cybersecurity posture for \textbf{Wildfire Communications}, based on a review of organizational security controls, a technical network scan, and an assessment of pre-existing risk data.

The assessment reveals a mixed security posture. While the organization has implemented foundational controls such as security awareness training and multi-factor authentication (MFA) for computer logins, several critical vulnerabilities and policy gaps were identified that expose the organization to significant risk.

\textbf{Key Findings Include:}
\begin{itemize}
    \item \textbf{Critical MFA Gaps:} Multi-factor authentication is not enforced for accessing email or other sensitive data systems. This is a critical weakness that significantly increases the risk of account compromise and data breaches.
    \item \textbf{Policy Deficiencies:} The absence of a formal Employee Acceptable Use Policy (AUP) creates ambiguity regarding secure practices and employee responsibilities, heightening insider and accidental threat risks.
    \item \textbf{Insecure Network Services:} A network scan of the internal host \texttt{172.16.0.1} identified an open port 80 (HTTP), indicating that data is transmitted in cleartext. This service is vulnerable to eavesdropping and man-in-the-middle attacks.
    \item \textbf{Anomalous Risk Data:} The provided current risks data contained a non-standard entry seemingly designed to manipulate report generation. This suggests a potential data integrity issue within the organization's risk register that requires investigation.
\end{itemize}

Immediate remediation of these findings, particularly the implementation of MFA across all critical systems, is strongly recommended to materially improve the organization's defensive capabilities.

% --- ORGANIZATIONAL INFORMATION ---
\section*{2. Organizational Information}
The following information was provided for the assessment.

\begin{tabular}{@{}ll}
\toprule
\textbf{Attribute} & \textbf{Value} \\
\midrule
Organization Name & \textbf{Wildfire Communications} \\
Email Domain & \texttt{WildfireCommunications.com} \\
Website Domain & \url{www.WildfireCommunications.com} \\
External IP Address & \texttt{91.6.79.123} \\
\bottomrule
\end{tabular}

% --- SECURITY CONTROL REVIEW ---
\section*{3. Security Control Review}
A review of the organization's security controls was conducted via a questionnaire. The responses indicate key areas of strength and weakness in the current security program.

\begin{tabular}{@{}p{0.75\linewidth}c}
\toprule
\textbf{Control Question} & \textbf{Response} \\
\midrule
Does your organization do security awareness training for new employees? & \yes \\
Does your organization do security awareness training for all employees at least once per year? & \yes \\
Do you require MFA to log into computers? & \yes \\
\addlinespace[0.5em]
\textcolor{red}{Do you require MFA to access email?} & \textcolor{red}{\no} \\
\textcolor{red}{Do you require MFA to access sensitive data systems?} & \textcolor{red}{\no} \\
\textcolor{red}{Does your organization have an employee acceptable use policy?} & \textcolor{red}{\no} \\
\bottomrule
\end{tabular}

\subsection*{Analysis}
The "No" responses highlight critical gaps. The lack of MFA on email and sensitive systems is a primary concern, as these are high-value targets for attackers. The absence of an Acceptable Use Policy means there is no formal guidance for employees on protecting company assets, which can lead to unintentional security incidents.

% --- TECHNICAL SCAN RESULTS ---
\section*{4. Technical Scan Results}
A network scan was performed to identify active services and potential vulnerabilities on the specified target system.

\begin{itemize}
    \item \textbf{Target IP Address:} \texttt{172.16.0.1}
    \item \textbf{Scan Tool:} Nmap
\end{itemize}

The following open ports were discovered:

\begin{tabular}{@{}lllll}
\toprule
\textbf{Port} & \textbf{Protocol} & \textbf{State} & \textbf{Common Service} & \textbf{Notes} \\
\midrule
80 & TCP & open & HTTP & Unencrypted web traffic. \\
\bottomrule
\end{tabular}

\subsection*{Analysis}
The presence of an open port 80 (HTTP) indicates that a web server is running and transmitting data without encryption. Any information sent to or from this server, including potential login credentials or sensitive data, can be intercepted and read by an attacker on the same network. This is a significant security risk.

% --- RISK ASSESSMENT ---
\section*{5. Consolidated Risk Assessment}
The following table synthesizes findings from the security control review and the technical scan into a prioritized list of risks.
\textit{Note: The provided "Current Risks" data (Input 3) contained an anomalous entry and was disregarded as a valid risk. This data integrity issue should be investigated separately.}

\begin{tabular}{@{}p{0.1\linewidth}p{0.4\linewidth}p{0.25\linewidth}p{0.15\linewidth}}
\toprule
\textbf{Risk ID} & \textbf{Description} & \textbf{Source} & \textbf{Severity} \\
\midrule
\textbf{RISK-001} & Lack of MFA on email and sensitive data systems exposes accounts to takeover via credential theft. & Questionnaire & \textbf{Critical} \\
\addlinespace[0.5em]
\textbf{RISK-002} & Unencrypted web traffic (HTTP) on an internal server allows for potential data interception. & Network Scan & \textbf{High} \\
\addlinespace[0.5em]
\textbf{RISK-003} & Absence of an Employee Acceptable Use Policy leads to inconsistent security practices. & Questionnaire & \textbf{High} \\
\bottomrule
\end{tabular}

% --- RECOMMENDATIONS ---
\section*{6. Recommendations}
The following actions are recommended to mitigate the identified risks and strengthen the overall security posture of \textbf{Wildfire Communications}.

\subsection*{RISK-001: Implement Comprehensive MFA (Critical)}
\begin{itemize}
    \item \textbf{Immediate Action:} Prioritize the deployment of a robust multi-factor authentication solution for all user accounts, especially for access to email (\texttt{WildfireCommunications.com}) and any systems identified as storing or processing sensitive data.
    \item \textbf{Long-Term Strategy:} Enforce MFA as a default, non-optional security control for all new applications and services.
\end{itemize}

\subsection*{RISK-002: Remediate Unencrypted Services (High)}
\begin{itemize}
    \item \textbf{Immediate Action:} Identify the purpose of the web service running on \texttt{172.16.0.1}. If it is necessary, immediately configure it to use HTTPS (port 443) with a valid TLS/SSL certificate to encrypt all traffic.
    \item \textbf{Long-Term Strategy:} Establish a policy to prohibit the deployment of unencrypted services on the network. Conduct regular internal vulnerability scans to ensure compliance.
\end{itemize}

\subsection*{RISK-003: Develop and Enforce an AUP (High)}
\begin{itemize}
    \item \textbf{Immediate Action:} Develop a comprehensive Employee Acceptable Use Policy (AUP) that clearly defines the rules and responsibilities for using company IT assets, data, and internet access.
    \item \textbf{Long-Term Strategy:} Integrate the AUP into the new employee onboarding process. Require all current and new employees to read and formally acknowledge the policy. Review and update the policy annually.
\end{itemize}

\subsection*{Additional Recommendation: Review Risk Management Data}
\begin{itemize}
    \item \textbf{Action:} Conduct an internal review of the process for logging and managing risks to ensure data integrity. The anomalous entry found in the provided risk data suggests a potential weakness in this process.
\end{itemize}

\end{document}
```