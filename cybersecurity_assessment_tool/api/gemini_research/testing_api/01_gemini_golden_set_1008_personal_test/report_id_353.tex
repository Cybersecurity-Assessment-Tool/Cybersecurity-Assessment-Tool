```latex
\documentclass[12pt]{article}

% Preamble: Required Packages
\usepackage[margin=1in]{geometry}
\usepackage{pifont} % For checkmarks and crosses
\usepackage{booktabs} % For professional tables
\usepackage{hyperref} % For clickable links
\usepackage{url} % For formatting URLs
\usepackage{seqsplit} % For splitting long strings to prevent overflow
\usepackage{graphicx}
\usepackage{xcolor}
\usepackage{datetime}

% Document Metadata
\title{Cybersecurity Assessment Report \\ \large For: Urban Jungle Planning}
\author{Cybersecurity Analyst Group}
\date{\today}

% Hyperref Setup
\hypersetup{
    colorlinks=true,
    linkcolor=blue,
    filecolor=magenta,      
    urlcolor=cyan,
    pdftitle={Cybersecurity Assessment Report},
    pdfpagemode=FullScreen,
}

\begin{document}

\maketitle
\thispagestyle{empty}
\newpage

\tableofcontents
\newpage

% --- 1. Executive Summary ---
\section{Executive Summary}

This report provides a comprehensive cybersecurity assessment for \textbf{Urban Jungle Planning}, synthesizing data from technical network scans, an organizational security questionnaire, and a review of pre-existing risks.

The analysis reveals several critical-risk findings that require immediate attention. The primary concern is the systemic exposure of the Remote Desktop Protocol (RDP) on internal systems, including the newly identified host at \texttt{10.10.10.51}. This technical vulnerability is severely compounded by organizational policy gaps, most notably the lack of Multi-Factor Authentication (MFA) for computer and sensitive data system access.

Furthermore, the absence of an employee acceptable use policy and mandatory security training for new hires creates a permissive environment for human error, which could easily lead to the compromise of an exposed service like RDP. The combination of these factors places the organization at a high risk of unauthorized access, data breach, and potential ransomware attacks.

Immediate remediation should focus on securing all RDP instances, mandating MFA across all critical systems, and formalizing employee security policies and training.

% --- 2. Organizational Information ---
\section{Organizational Information}

The following details were provided for the assessment. This information is used to establish the context and scope of the review.

\begin{tabular}{@{}ll}
\toprule
\textbf{Attribute} & \textbf{Value} \\
\midrule
Organization Name & \textbf{Urban Jungle Planning} \\
Email Domain & \texttt{UrbanJunglePlanning.net} \\
Website Domain & \url{www.UrbanJunglePlanning.net} \\
External IP Address & \texttt{38.140.194.27} \\
\bottomrule
\end{tabular}

% --- 3. Security Control Review ---
\section{Security Control Review}

A security questionnaire was completed to evaluate the organization's current security policies and controls. The responses are summarized below. Items marked with \ding{55} represent significant gaps in the security posture.

\begin{table}[h!]
\centering
\begin{tabular}{@{}lc}
\toprule
\textbf{Control Question} & \textbf{Response} \\
\midrule
Do you require MFA to access email? & \textcolor{green}{\ding{51}} \\
Do you require MFA to log into computers? & \textcolor{red}{\ding{55}} \\
Do you require MFA to access sensitive data systems? & \textcolor{red}{\ding{55}} \\
Does your organization have an employee acceptable use policy? & \textcolor{red}{\ding{55}} \\
Does your organization do security awareness training for new employees? & \textcolor{red}{\ding{55}} \\
Does your organization do security awareness training for all employees annually? & \textcolor{green}{\ding{51}} \\
\bottomrule
\end{tabular}
\caption{Organizational Security Control Questionnaire Results}
\end{table}

\subsection{Analysis of Control Gaps}
The questionnaire reveals critical deficiencies in fundamental security controls:
\begin{itemize}
    \item \textbf{Lack of MFA:} The absence of MFA on computer logins and sensitive data systems is a critical vulnerability. This significantly increases the risk of a successful breach via stolen or brute-forced credentials.
    \item \textbf{Policy and Training Gaps:} The lack of an acceptable use policy and security training for new hires are high-risk findings. New employees are often targeted by attackers, and without clear policy and initial training, they are more susceptible to social engineering and other common threats.
\end{itemize}

% --- 4. Technical Scan Results ---
\section{Technical Scan Results}

A network scan was performed to identify open ports and exposed services on the target system.

\begin{itemize}
    \item \textbf{Target IP Address:} \texttt{10.10.10.51}
\end{itemize}

\begin{table}[h!]
\centering
\begin{tabular}{@{}llll@{}}
\toprule
\textbf{Port} & \textbf{State} & \textbf{Service Name} & \textbf{Description} \\
\midrule
3389/tcp & open & \texttt{ms-wbt-server} & Microsoft Remote Desktop Protocol (RDP) \\
\bottomrule
\end{tabular}
\caption{Open Ports Detected on \texttt{10.10.10.51}}
\end{table}

\subsection{Analysis of Technical Findings}
The scan identified that port \textbf{3389 (RDP)} is open on the host \texttt{10.10.10.51}. RDP is a common vector for network intrusion and ransomware deployment. When exposed without compensating controls such as a VPN, Network Level Authentication (NLA), and MFA, it presents a critical risk. This finding, correlated with the lack of MFA for computer logins, indicates that a compromised user password could lead directly to unauthorized remote control of this system.

% --- 5. Consolidated Risk Assessment ---
\section{Consolidated Risk Assessment}

This section correlates findings from the security questionnaire, the technical scan, and pre-existing risk data to provide a holistic view of the organization's risk posture.

\begin{table}[h!]
\centering
\begin{tabular}{@{}p{0.25\linewidth}p{0.4\linewidth}p{0.15\linewidth}p{0.1\linewidth}@{}}
\toprule
\textbf{Risk Name} & \textbf{Description} & \textbf{Affected Systems} & \textbf{Severity} \\
\midrule
\textbf{Systemic RDP Exposure} & The RDP service is exposed on multiple internal systems. This is exacerbated by the lack of MFA on computer logins, making systems vulnerable to credential-based attacks. & \texttt{10.10.10.51} (new), \texttt{10.10.10.50} (existing) & \textbf{Critical} \\
\addlinespace
\textbf{Insufficient MFA Implementation} & MFA is not enforced for computer logins or access to sensitive data systems, failing to protect against credential theft or brute-force attacks. & All workstations and sensitive data repositories & \textbf{Critical} \\
\addlinespace
\textbf{Inadequate Employee Onboarding \& Policy} & The absence of an Acceptable Use Policy and security training for new hires increases the likelihood of security incidents caused by human error. & All employees, especially new hires & \textbf{High} \\
\bottomrule
\end{tabular}
\caption{Summary of Identified Risks}
\end{table}

% --- 6. Recommendations ---
\section{Recommendations}

The following prioritized recommendations are provided to mitigate the identified risks and improve the overall security posture of \textbf{Urban Jungle Planning}.

\subsection{Immediate Actions (Critical Priority)}
\begin{enumerate}
    \item \textbf{Remediate RDP Exposure:}
    \begin{itemize}
        \item Immediately close port 3389 on \texttt{10.10.10.51} and any other systems where it is not strictly required for business operations.
        \item For all systems requiring remote access, implement a Virtual Private Network (VPN) solution with mandatory MFA. RDP access should only be permitted through the secure VPN tunnel.
    \end{itemize}
    \item \textbf{Enforce MFA Across All Critical Systems:}
    \begin{itemize}
        \item Deploy and mandate the use of MFA for all employee computer/workstation logins.
        \item Enforce MFA for all applications and systems that store or process sensitive data.
    \end{itemize}
\end{enumerate}

\subsection{Short-Term Actions (High Priority)}
\begin{enumerate}
    \item \textbf{Develop and Implement an Acceptable Use Policy (AUP):}
    \begin{itemize}
        \item Draft a formal AUP that clearly defines the rules for using company technology and data.
        \item Require all employees to read and formally acknowledge the policy.
    \end{itemize}
    \item \textbf{Establish a New Hire Security Training Program:}
    \begin{itemize}
        \item Create a mandatory security awareness training module as part of the new employee onboarding process. This should cover topics such as phishing, password hygiene, and the new AUP.
    \end{itemize}
\end{enumerate}

\subsection{Strategic Recommendations}
\begin{enumerate}
    \item \textbf{Conduct Comprehensive Vulnerability Scanning:} Perform authenticated and unauthenticated vulnerability scans across the entire internal network to identify other misconfigurations, outdated software, and security weaknesses.
    \item \textbf{Establish a Formal Risk Management Program:} Develop a continuous process for identifying, assessing, and mitigating cybersecurity risks to ensure a proactive defense posture.
\end{enumerate}

\end{document}
```