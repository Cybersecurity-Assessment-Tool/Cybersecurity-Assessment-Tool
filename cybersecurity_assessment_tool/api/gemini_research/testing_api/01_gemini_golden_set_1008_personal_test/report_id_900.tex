```latex
\documentclass[12pt]{article}

% Preamble: Required Packages
\usepackage[margin=1in]{geometry}
\usepackage{pifont} % For checkmarks and crosses
\usepackage{booktabs} % For professional tables
\usepackage{hyperref} % For clickable links and references
\usepackage{url} % For formatting URLs
\usepackage{seqsplit} % To split long strings in tt font
\usepackage{graphicx}
\usepackage[table]{xcolor}
\usepackage{fancyhdr}
\usepackage{lastpage}

% --- Document Setup ---
\hypersetup{
    colorlinks=true,
    linkcolor=blue,
    filecolor=magenta,      
    urlcolor=cyan,
    pdftitle={Cybersecurity Posture Report},
    pdfpagemode=FullScreen,
}

% Define colors for severity
\definecolor{severityhigh}{HTML}{D9534F}
\definecolor{severitymedium}{HTML}{F0AD4E}
\definecolor{severitylow}{HTML}{5CB85C}
\definecolor{tablehead}{gray}{0.9}

% Header and Footer
\pagestyle{fancy}
\fancyhf{} % Clear all header and footer fields
\fancyhead[L]{Cybersecurity Posture Report}
\fancyhead[R]{Falcon Heavy}
\fancyfoot[C]{\thepage\ of \pageref{LastPage}}
\renewcommand{\headrulewidth}{0.4pt}
\renewcommand{\footrulewidth}{0.4pt}

% --- Document Start ---
\begin{document}

% --- Title Page ---
\begin{titlepage}
    \centering
    \vfill
    {\Huge \textbf{Cybersecurity Posture Report}\par}
    \vspace{1.5cm}
    {\Large \textbf{Prepared for:}\par}
    \vspace{0.5cm}
    {\huge Falcon Heavy\par}
    \vspace{2cm}
    {\large \textbf{Date of Analysis:}\par}
    \vspace{0.5cm}
    {\Large \today\par}
    \vfill
    \textit{This report contains sensitive information and should be handled with care.}
\end{titlepage}

\tableofcontents
\newpage

% --- Section 1: Executive Summary ---
\section{Executive Summary}

This report provides a cybersecurity assessment for Falcon Heavy, based on an analysis of organizational data, a security controls questionnaire, and a review of technical scan data.

The organization demonstrates a strong commitment to identity and access management, with Multi-Factor Authentication (MFA) consistently enforced across email, computer logins, and sensitive data systems. This significantly reduces the risk of unauthorized access through compromised credentials.

However, critical gaps were identified in foundational security policies and employee training. The absence of an employee acceptable use policy and the lack of security awareness training for new hires represent high-risk vulnerabilities. These gaps can lead to inconsistent security practices, accidental data breaches, and a workforce unprepared to identify and report social engineering attacks.

Furthermore, the analysis was hampered by corrupted input data. Both the network vulnerability scan results (\texttt{Input\_1}) and the list of current known risks (\texttt{Input\_3}) were unreadable. This creates a significant blind spot regarding the organization's external attack surface and its existing risk landscape.

Immediate action is required to address the policy and training deficiencies. It is also imperative to conduct a new network scan and compile a list of known risks to achieve a complete and accurate understanding of the organization's security posture.

% --- Section 2: Organizational Information ---
\section{Organizational Information}
The following details were provided for the assessment.

\begin{itemize}
    \item \textbf{Organization Name:} Falcon Heavy
    \item \textbf{Email Domain:} \texttt{FalconHeavy.net}
    \item \textbf{Website Domain:} \url{www.FalconHeavy.net}
    \item \textbf{External IP Address:} \texttt{94.139.253.99}
\end{itemize}

% --- Section 3: Security Control Review ---
\section{Security Control Review}
A review of the security controls questionnaire was conducted to evaluate existing administrative and procedural safeguards. The results are summarized below. Items marked with \ding{55} indicate significant gaps that increase organizational risk.

\rowcolors{2}{gray!10}{white}
\begin{table}[h!]
\centering
\caption{Security Controls Questionnaire Analysis}
\label{tab:controls}
\begin{tabular}{p{0.7\textwidth} c c}
\toprule
\rowcolor{tablehead}
\textbf{Control Question} & \textbf{Response} & \textbf{Status} \\
\midrule
Do you require MFA to access email? & Yes & \ding{51} \\
Do you require MFA to log into computers? & Yes & \ding{51} \\
Do you require MFA to access sensitive data systems? & Yes & \ding{51} \\
Does your organization do security awareness training for all employees at least once per year? & Yes & \ding{51} \\
\midrule
\rowcolor{severityhigh!20}
\textbf{Does your organization have an employee acceptable use policy?} & \textbf{No} & \textbf{\ding{55}} \\
\rowcolor{severityhigh!20}
\textbf{Does your organization do security awareness training for new employees?} & \textbf{No} & \textbf{\ding{55}} \\
\bottomrule
\end{tabular}
\end{table}

The findings highlight a strong implementation of MFA. However, the absence of an Acceptable Use Policy (AUP) and a formal security training program for new hires are critical deficiencies that must be addressed.

% --- Section 4: Technical Scan Results ---
\section{Technical Scan Results}
An analysis of the external network perimeter was attempted using the provided scan data.

\subsection{Data Integrity Issue}
The provided network scan data (\texttt{Input\_1\_Network\_Scan\_JSON}) was found to be corrupted and could not be parsed. As a result, no analysis of open ports, running services, or potential vulnerabilities on the external IP address (\texttt{94.139.253.99}) could be performed.

\textbf{Conclusion:} The organization's external technical posture is currently unknown. A full, authenticated network scan is required to identify potential vulnerabilities that could be exploited by external attackers.

% --- Section 5: Risk Assessment ---
\section{Risk Assessment}
This risk assessment is based on the findings from the Security Control Review. The assessment is incomplete due to the corrupted network scan and current risks data.

\begin{table}[h!]
\centering
\caption{Identified Risks and Severity}
\label{tab:risks}
\begin{tabular}{p{0.25\textwidth} p{0.5\textwidth} p{0.15\textwidth}}
\toprule
\rowcolor{tablehead}
\textbf{Risk Name} & \textbf{Overview} & \textbf{Severity} \\
\midrule
\rowcolor{severityhigh!20}
Lack of Acceptable Use Policy (AUP) & Without a formal AUP, employees lack clear guidelines on the acceptable use of company assets, data handling, and security responsibilities. This increases the risk of insider threats, data leakage, and non-compliance. & \cellcolor{severityhigh} \textbf{High} \\
\addlinespace
\rowcolor{severityhigh!20}
No Onboarding Security Training & New employees are not receiving security awareness training upon joining. This makes them highly susceptible to phishing, social engineering, and other common attacks from their first day, creating an immediate vulnerability. & \cellcolor{severityhigh} \textbf{High} \\
\addlinespace
\rowcolor{severitymedium!20}
Incomplete Risk Picture & The inability to analyze technical scan data or review a list of pre-existing vulnerabilities means the organization is operating with significant blind spots. Unknown critical vulnerabilities may exist on external systems. & \cellcolor{severitymedium} \textbf{Medium} \\
\bottomrule
\end{tabular}
\end{table}

% --- Section 6: Recommendations ---
\section{Recommendations}
Based on the analysis, the following actions are recommended to improve the cybersecurity posture of Falcon Heavy.

\subsection{Immediate Actions (Next 7 Days)}
\begin{enumerate}
    \item \textbf{Rescan External Infrastructure:} Commission a new, comprehensive vulnerability scan against the external IP address \texttt{94.139.253.99}. This is critical to identify and remediate technical vulnerabilities.
    \item \textbf{Compile Current Risk Register:} The provided list of current risks (\texttt{Input\_3}) was corrupted. A definitive list of all known and tracked vulnerabilities must be compiled and provided for a complete assessment.
\end{enumerate}

\subsection{High-Priority Actions (Next 30 Days)}
\begin{enumerate}
    \item \textbf{Develop and Implement an Acceptable Use Policy (AUP):}
    \begin{itemize}
        \item Draft a clear and concise AUP that covers topics such as data handling, internet usage, password requirements, and incident reporting.
        \item Have the policy reviewed by legal and HR departments.
        \item Distribute the policy to all employees and require a signed acknowledgment of receipt and understanding.
    \end{itemize}
    \item \textbf{Establish a New Hire Security Training Program:}
    \begin{itemize}
        \item Create a mandatory security awareness training module as part of the standard employee onboarding process.
        \item The training should cover phishing, password security, safe browsing, and the new AUP.
        \item This ensures a baseline of security knowledge for all personnel from day one.
    \end{itemize}
\end{enumerate}

\subsection{Medium-Priority Actions (Next 90 Days)}
\begin{enumerate}
    \item \textbf{Audit MFA Implementation:} While MFA is reportedly in place, conduct a periodic audit to ensure it is enforced on all applicable accounts and systems without exception.
\end{enumerate}

% --- Section 7: Conclusion ---
\section{Conclusion}
Falcon Heavy has established a commendable foundation in access control through the widespread use of MFA. However, this strength is undermined by critical weaknesses in administrative controls, specifically the lack of a guiding Acceptable Use Policy and security training for new employees.

The most pressing issue is the lack of visibility into the technical attack surface and current risks due to corrupted data. Addressing the recommendations in this report—starting with a new network scan and the development of foundational policies—will create a more resilient and defensible security posture.

\end{document}
```