```latex
\documentclass[12pt, a4paper]{article}

% Preamble: Required Packages
\usepackage[margin=1in]{geometry}
\usepackage{pifont} % For checkmarks and crosses
\usepackage{booktabs} % For professional tables
\usepackage{graphicx} % For potential logos/images
\usepackage[table]{xcolor} % For colors in tables
\usepackage{hyperref} % For hyperlinks
\usepackage{url} % For formatting URLs
\usepackage{seqsplit} % For splitting long strings without spaces
\usepackage{lastpage} % To get the total number of pages
\usepackage{fancyhdr} % For custom headers and footers

% --- Document Setup ---

% Color Definitions
\definecolor{tablehead}{gray}{0.9}
\definecolor{critical}{HTML}{990000}
\definecolor{high}{HTML}{D14900}
\definecolor{medium}{HTML}{E0C000}
\definecolor{low}{HTML}{339900}

% Hyperref Configuration
\hypersetup{
    colorlinks=true,
    linkcolor=blue,
    filecolor=magenta,      
    urlcolor=cyan,
    pdftitle={Cybersecurity Posture Report},
    pdfauthor={Cybersecurity Analyst},
    pdfsubject={Security Assessment},
    pdfkeywords={Security, Report, Analysis},
    bookmarks=true
}

% Header and Footer Configuration
\pagestyle{fancy}
\fancyhf{} % Clear all header and footer fields
\fancyhead[L]{Cybersecurity Posture Report}
\fancyhead[R]{Vertex Solutions}
\fancyfoot[C]{\thepage\ of \pageref{LastPage}}
\renewcommand{\headrulewidth}{0.4pt}
\renewcommand{\footrulewidth}{0.4pt}

% --- Document Start ---

\begin{document}

% --- Title Page ---
\begin{titlepage}
    \centering
    \vspace*{1cm}
    \Huge\textbf{Cybersecurity Posture Report}
    \vspace{1.5cm}
    \Large\textbf{Prepared for:} \\
    \vspace{0.5cm}
    \Large Vertex Solutions
    \vfill
    \large\textbf{Date of Report:} \today \\
    \vspace{0.5cm}
    \large\textbf{Report ID:} CSR-2023-451
    \vspace{1cm}
    \rule{\textwidth}{0.4pt}
    \par
    \small \textit{This document contains sensitive information and is intended for the exclusive use of the recipient. Unauthorized distribution is strictly prohibited.}
\end{titlepage}

\tableofcontents
\newpage

% --- Section 1: Executive Overview ---
\section{Executive Overview}
This report provides a comprehensive analysis of the cybersecurity posture for Vertex Solutions, based on a combination of technical network scanning, a review of existing risks, and an organizational security controls questionnaire.

The assessment identified several critical and high-risk findings that require immediate attention. The primary areas of concern are:
\begin{itemize}
    \item \textbf{Systemic Remote Access Exposure:} The technical scan confirmed an open Remote Desktop Protocol (RDP) port on a new host (\texttt{10.10.10.51}). This finding, correlated with a pre-existing risk for a different host (\texttt{10.10.10.50}), indicates a systemic and dangerous pattern of exposing remote management services directly to the network.
    \item \textbf{Critical Authentication Gaps:} The organization does not enforce Multi-Factor Authentication (MFA) for computer logins. This significantly increases the risk of unauthorized access from compromised credentials, a risk that is amplified by the exposed RDP services.
    \item \textbf{Inadequate Security Onboarding:} New employees do not receive security awareness training upon being hired. This creates a window of vulnerability where new staff are more susceptible to phishing and social engineering attacks before they are familiarized with organizational security policies.
\end{itemize}

The combination of these findings places the organization at a high risk of a security breach, particularly from ransomware or other network intrusion attacks. This report provides detailed findings and actionable recommendations to mitigate these risks and improve the overall security posture.

% --- Section 2: Organizational Information ---
\section{Organizational Information}
The following details were provided for the assessment.
\begin{itemize}
    \item \textbf{Organization Name:} Vertex Solutions
    \item \textbf{Email Domain:} \texttt{VertexSolutions.org}
    \item \textbf{Monitored External IP:} \texttt{99.37.55.26}
\end{itemize}

% --- Section 3: Security Control Review ---
\section{Security Control Review}
The following table summarizes the organization's responses to the security controls questionnaire. Gaps in security best practices are highlighted for review.
\vspace{0.5cm}

\begin{table}[h!]
\centering
\caption{Security Controls Questionnaire Analysis}
\begin{tabular}{p{0.6\textwidth} c p{0.2\textwidth}}
\toprule
\rowcolor{tablehead}
\textbf{Control Question} & \textbf{Response} & \textbf{Assessment} \\
\midrule
Do you require MFA to access email? & \textcolor{green}{\ding{51}} & Good Practice \\
\addlinespace
Do you require MFA to log into computers? & \textcolor{red}{\ding{55}} & \textbf{Critical Gap} \\
\addlinespace
Do you require MFA to access sensitive data systems? & \textcolor{green}{\ding{51}} & Good Practice \\
\addlinespace
Does your organization have an employee acceptable use policy? & \textcolor{green}{\ding{51}} & Good Practice \\
\addlinespace
Does your organization do security awareness training for new employees? & \textcolor{red}{\ding{55}} & \textbf{High Risk} \\
\addlinespace
Does your organization do security awareness training for all employees at least once per year? & \textcolor{green}{\ding{51}} & Good Practice \\
\bottomrule
\end{tabular}
\end{table}

% --- Section 4: Technical Scan Results ---
\section{Technical Scan Results}
A network scan was performed to identify open ports and services on the target system.
\begin{itemize}
    \item \textbf{Target IP Address:} \texttt{10.10.10.51}
    \item \textbf{Scan Status:} Host is up.
\end{itemize}

\begin{table}[h!]
\centering
\caption{Open Port Analysis for \texttt{10.10.10.51}}
\begin{tabular}{l l l p{0.5\textwidth}}
\toprule
\rowcolor{tablehead}
\textbf{Port} & \textbf{State} & \textbf{Service} & \textbf{Notes} \\
\midrule
3389/tcp & Open & ms-wbt-server & This port is used for Microsoft Remote Desktop Protocol (RDP). Exposing RDP directly to a network without compensating controls is a critical security risk, as it is a primary target for brute-force and ransomware attacks. \\
\bottomrule
\end{tabular}
\end{table}

% --- Section 5: Consolidated Risk Assessment ---
\section{Consolidated Risk Assessment}
The following table synthesizes findings from the questionnaire, technical scan, and pre-existing risk data into a prioritized list.

\begin{table}[h!]
\centering
\caption{Summary of Identified Risks}
\begin{tabular}{p{0.25\textwidth} p{0.55\textwidth} c}
\toprule
\rowcolor{tablehead}
\textbf{Risk Name} & \textbf{Description} & \textbf{Severity} \\
\midrule
\textbf{Systemic RDP Exposure} & Multiple systems (\texttt{10.10.10.50}, \texttt{10.10.10.51}) have RDP (port 3389) open. This service is a frequent target for attackers seeking to gain initial access to a network. & \cellcolor{critical!25}Critical \\
\addlinespace
\textbf{Lack of Endpoint MFA} & User workstations and servers can be accessed with only a password. If credentials are stolen, there is no second factor to prevent an attacker from logging in. This risk is severely amplified by the exposed RDP services. & \cellcolor{critical!25}Critical \\
\addlinespace
\textbf{Inadequate New Employee Onboarding} & New hires are not provided with security awareness training. This makes them highly vulnerable to phishing or social engineering attacks, which could lead to the credential compromise needed to exploit the other identified risks. & \cellcolor{high!25}High \\
\bottomrule
\end{tabular}
\end{table}

% --- Section 6: Recommendations ---
\section{Recommendations}
The following actions are recommended to mitigate the identified risks and strengthen the organization's security posture.

\subsection{Risk: Systemic RDP Exposure (Critical)}
\begin{itemize}
    \item \textbf{Immediate Action:} Implement firewall rules to deny all access to TCP port 3389 on affected systems (\texttt{10.10.10.50}, \texttt{10.10.10.51}) from any untrusted network zone. If remote access is required, create explicit allow-rules for specific, trusted source IP addresses only.
    \item \textbf{Long-Term Solution:} Decommission direct RDP access. Implement a secure remote access solution such as a Virtual Private Network (VPN) or a Zero Trust Network Access (ZTNA) gateway. All remote administration should occur through this secure channel.
\end{itemize}

\subsection{Risk: Lack of Endpoint MFA (Critical)}
\begin{itemize}
    \item \textbf{Immediate Action:} Prioritize the deployment of an MFA solution (e.g., DUO, Microsoft Authenticator) for all computer logins, starting with administrators and users who require remote access. This is the single most effective control to mitigate the risk of compromised credentials.
    \item \textbf{Long-Term Solution:} Integrate MFA into the standard Identity and Access Management (IAM) policy. Ensure that all new user accounts and systems are configured with MFA by default.
\end{itemize}

\subsection{Risk: Inadequate New Employee Onboarding (High)}
\begin{itemize}
    \item \textbf{Immediate Action:} Develop a foundational security awareness training module and mandate its completion for all new employees \textit{before} they are granted access to sensitive systems or data. This training should cover phishing, acceptable use, and password security.
    \item \textbf{Long-Term Solution:} Formally integrate the security training module into the Human Resources onboarding checklist to ensure it is consistently delivered to all new hires.
\end{itemize}

\end{document}
```