```latex
\documentclass[12pt]{article}

% --- PACKAGES ---
\usepackage[margin=1in]{geometry}
\usepackage{pifont} % For checkmarks and crosses
\usepackage{booktabs} % For professional tables
\usepackage{hyperref} % For clickable links
\usepackage{url} % For URL formatting
\usepackage{seqsplit} % For splitting long strings in tt font
\usepackage{graphicx}
\usepackage{xcolor}

% --- DOCUMENT SETUP ---
\hypersetup{
    colorlinks=true,
    linkcolor=blue,
    filecolor=magenta,      
    urlcolor=cyan,
    pdftitle={Cybersecurity Assessment Report},
    pdfpagemode=FullScreen,
}

\newcommand{\yes}{\ding{51}}
\newcommand{\no}{\ding{55}}

% --- TITLE ---
\title{Cybersecurity Assessment Report \\ \large For: \textbf{Signal Flare}}
\author{Cybersecurity Analyst}
\date{\today}

% --- BEGIN DOCUMENT ---
\begin{document}

\maketitle
\thispagestyle{empty}
\newpage

\tableofcontents
\newpage

% ===================================================================
% 1. EXECUTIVE SUMMARY
% ===================================================================
\section{Executive Summary}

This report provides a comprehensive cybersecurity assessment for \textbf{Signal Flare}, based on an analysis of network scan data, organizational security controls, and pre-existing risk information. The assessment was conducted on \today.

The analysis reveals several critical and high-risk security gaps that require immediate attention. A key technical finding was an openly accessible network service on port 8080 with the title \textbf{"TOP SECRET DB"}. This directly contradicts a pre-existing risk assessment that marked this port as secure, indicating a significant and potentially dangerous discrepancy.

Furthermore, critical procedural weaknesses were identified, most notably the lack of Multi-Factor Authentication (MFA) for email and computer access. The absence of a formal Acceptable Use Policy and security training for new employees exacerbates these risks, creating an environment susceptible to credential theft, unauthorized access, and insider threats.

Immediate remediation is recommended to address the exposed service and implement foundational security controls like MFA. Addressing these findings will significantly improve the organization's security posture and reduce its attack surface.

% ===================================================================
% 2. ORGANIZATIONAL INFORMATION
% ===================================================================
\section{Organizational Information}

The following information was provided by the client and used as a baseline for this assessment.

\begin{tabular}{@{}ll}
\toprule
\textbf{Attribute} & \textbf{Value} \\
\midrule
Organization Name & \textbf{Signal Flare} \\
Email Domain & \texttt{SignalFlare.net} \\
Website Domain & \seqsplit{\url{www.SignalFlare.net}} \\
External IP Address & \texttt{139.215.113.232} \\
\bottomrule
\end{tabular}

% ===================================================================
% 3. SECURITY CONTROL REVIEW
% ===================================================================
\section{Security Control Review}

A review of the organization's security controls was conducted based on a questionnaire. The responses highlight significant gaps in fundamental security practices.

\begin{tabular}{@{}p{0.6\linewidth}cp{0.25\linewidth}@{}}
\toprule
\textbf{Control Question} & \textbf{Status} & \textbf{Analyst Note} \\
\midrule
Do you require MFA to access email? & \textcolor{red}{\no} & \textbf{Critical Gap.} Compromised credentials can lead to email account takeover. \\
\addlinespace
Do you require MFA to log into computers? & \textcolor{red}{\no} & \textbf{Critical Gap.} Increases risk of unauthorized workstation access. \\
\addlinespace
Do you require MFA to access sensitive data systems? & \textcolor{green}{\yes} & Good practice. \\
\addlinespace
Does your organization have an employee acceptable use policy? & \textcolor{red}{\no} & \textbf{High Risk.} Lack of clear guidelines for employees. \\
\addlinespace
Does your organization do security awareness training for new employees? & \textcolor{red}{\no} & \textbf{High Risk.} New hires are a common target for social engineering. \\
\addlinespace
Does your organization do security awareness training for all employees at least once per year? & \textcolor{green}{\yes} & Good practice. \\
\bottomrule
\end{tabular}

% ===================================================================
% 4. TECHNICAL SCAN RESULTS
% ===================================================================
\section{Technical Scan Results}

A network scan was performed to identify open ports and exposed services on the target system.

\begin{itemize}
    \item \textbf{Target IP Address:} \texttt{10.5.5.5}
    \item \textbf{Scan Date:} Scan data processed on \today
\end{itemize}

\subsection{Open Ports Discovered}
The scan identified the following open port.

\begin{tabular}{@{}lllll@{}}
\toprule
\textbf{Port} & \textbf{State} & \textbf{Service} & \textbf{Product/Version} & \textbf{Details} \\
\midrule
8080/tcp & open & http & Unknown & \textbf{HTTP Title: TOP SECRET DB} \\
\bottomrule
\end{tabular}

\subsection{Analysis of Technical Findings}
The most critical finding is the service on port \texttt{8080}. The HTTP title, \textbf{"TOP SECRET DB"}, strongly suggests that a sensitive, possibly internal, database or application is exposed. This finding is of the highest concern because it directly contradicts the information from the existing risk register (\textit{Input\_3\_Current\_Risks\_JSON}), which stated this port was "confirmed secure and false positive." This discrepancy implies either a recent, insecure change in the environment or a flawed previous assessment.

% ===================================================================
% 5. CONSOLIDATED RISK ASSESSMENT
% ===================================================================
\section{Consolidated Risk Assessment}

This section synthesizes findings from the security control review, technical scan, and pre-existing risk data into a consolidated list of identified risks.

\begin{tabular}{@{}p{0.25\linewidth}p{0.5\linewidth}p{0.15\linewidth}@{}}
\toprule
\textbf{Risk Title} & \textbf{Description} & \textbf{Severity} \\
\midrule
\textbf{Exposed Sensitive Internal Service} & A service on \texttt{10.5.5.5:8080} is publicly accessible and identified as "TOP SECRET DB". This contradicts a prior assessment and poses a severe risk of data breach. & \textbf{CRITICAL} \\
\addlinespace
\textbf{Lack of Multi-Factor Authentication (MFA)} & Email and computer logins are secured only by passwords. This makes the organization highly vulnerable to phishing, credential stuffing, and brute-force attacks. & \textbf{CRITICAL} \\
\addlinespace
\textbf{Insufficient Security Policies \& Onboarding Training} & The absence of an Acceptable Use Policy and security training for new employees creates a weak security culture and increases the likelihood of insider threats and human error. & \textbf{HIGH} \\
\addlinespace
\textbf{Outdated Risk Register} & The existing risk register incorrectly identifies port 8080 as a secured false positive. This indicates that risk management processes may be unreliable or not consistently followed. & \textbf{HIGH} \\
\bottomrule
\end{tabular}

% ===================================================================
% 6. RECOMMENDATIONS
% ===================================================================
\section{Recommendations}

The following actions are recommended to mitigate the identified risks. Recommendations are prioritized based on severity.

\begin{description}
    \item[\textbf{Priority 1 (Immediate): Investigate and Secure Exposed Service}]
        \begin{itemize}
            \item \textbf{Action:} Immediately investigate the service running on \texttt{10.5.5.5:8080}. Determine its purpose and the sensitivity of the data it contains.
            \item \textbf{Action:} If the service is not intended for public access, implement firewall rules to restrict access to authorized personnel only.
            \item \textbf{Action:} Update the risk register with the accurate findings and review the process that led to the incorrect "false positive" conclusion.
        \end{itemize}

    \item[\textbf{Priority 2 (Urgent): Implement Multi-Factor Authentication}]
        \begin{itemize}
            \item \textbf{Action:} Enforce MFA across all user accounts for email access (e.g., via Office 365 or Google Workspace security settings).
            \item \textbf{Action:} Deploy and enforce MFA for all workstation and laptop logins (e.g., using Windows Hello for Business, Duo, or a similar solution).
        \end{itemize}

    \item[\textbf{Priority 3 (Necessary): Develop Foundational Security Policies}]
        \begin{itemize}
            \item \textbf{Action:} Draft and implement a formal Acceptable Use Policy (AUP) that all employees must read and sign.
            \item \textbf{Action:} Integrate mandatory security awareness training into the new employee onboarding process. This should cover topics like phishing, password security, and the new AUP.
        \end{itemize}
\end{description}

\end{document}
```