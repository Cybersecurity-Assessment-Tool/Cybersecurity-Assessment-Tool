```latex
\documentclass[12pt, a4paper]{article}

% Preamble: Required Packages
\usepackage[margin=1in]{geometry}
\usepackage{pifont} % For checkmarks and crosses
\usepackage{booktabs} % For professional tables
\usepackage{hyperref} % For clickable links
\usepackage{url} % For formatting URLs
\usepackage{seqsplit} % For splitting long strings in tt font
\usepackage{graphicx}
\usepackage[table]{xcolor} % For coloring table cells
\usepackage{tocloft} % For table of contents customization

% --- Document Metadata and Hyperref Setup ---
\hypersetup{
    colorlinks=true,
    linkcolor=blue,
    filecolor=magenta,      
    urlcolor=cyan,
    pdftitle={Cybersecurity Posture Assessment Report},
    pdfauthor={Cybersecurity Analyst},
    pdfsubject={Security Assessment},
    pdfkeywords={Security, Report, Analysis},
    bookmarks=true
}

% --- Custom Commands and Color Definitions ---
\newcommand{\yes}{\ding{51}} % Green checkmark
\newcommand{\no}{\ding{55}}  % Red cross
\definecolor{sev_critical}{HTML}{940000}
\definecolor{sev_high}{HTML}{D14000}
\definecolor{sev_medium}{HTML}{E49B00}
\definecolor{sev_low}{HTML}{3A7D44}
\definecolor{tablegray}{gray}{0.9}

% --- Document Start ---
\begin{document}

% --- Title Page ---
\begin{titlepage}
    \centering
    \vspace*{1cm}
    \Huge\textbf{Cybersecurity Posture Assessment Report}
    \vspace{1.5cm}
    \large
    \begin{tabular}{ll}
        \textbf{Organization:} & Nova Terra \\
        \textbf{Report Date:} & \today \\
        \textbf{Report ID:} & SEC-2023-042 \\
    \end{tabular}
    \vfill
    \large
    \textbf{CONFIDENTIAL}
    \vspace{0.8cm}
    \small
    This document contains sensitive information. Access is restricted to authorized personnel only. Do not distribute without explicit permission.
\end{titlepage}

% --- Table of Contents ---
\newpage
\tableofcontents
\newpage

% --- Section 1: Executive Summary ---
\section{Executive Summary}
This report details the findings of a cybersecurity posture assessment for Nova Terra. The assessment combined a review of organizational security controls, an analysis of pre-existing risks, and a technical network scan.

The analysis revealed two critical-risk findings that require immediate attention. Firstly, a systemic exposure of the Remote Desktop Protocol (RDP) was identified on internal systems, including a new discovery on host \texttt{10.10.10.51}. This corroborates a previously known risk and indicates a wider network configuration issue, posing a significant threat of unauthorized access and ransomware.

Secondly, the organizational review highlighted a critical gap in access control: Multi-Factor Authentication (MFA) is not enforced for accessing sensitive data systems. This weakness dramatically increases the risk of data breach should an attacker compromise user credentials.

While the organization has implemented several positive security controls, such as MFA for email and regular security training, the identified critical risks substantially elevate the overall threat profile. Immediate remediation of the RDP exposure and implementation of MFA for sensitive systems are paramount to mitigating these threats.

% --- Section 2: Organizational Information ---
\section{Organizational Information}
The following information was provided for the assessment.

\begin{table}[h!]
\centering
\begin{tabular}{@{}ll@{}}
\toprule
\textbf{Attribute} & \textbf{Value} \\ \midrule
Organization Name & Nova Terra \\
Email Domain & \seqsplit{\texttt{NovaTerra.net}} \\
Website Domain & \seqsplit{\texttt{www.NovaTerra.net}} \\
External IP Address & \seqsplit{\texttt{48.174.214.60}} \\ \bottomrule
\end{tabular}
\caption{Client Organizational Data.}
\label{tab:org_info}
\end{table}

% --- Section 3: Security Control Review ---
\section{Security Control Review}
A review of administrative and technical security controls was conducted based on a standardized questionnaire. The results indicate a solid foundation in user awareness and endpoint security, but a critical gap exists in access control for sensitive data.

\begin{table}[h!]
\centering
\rowcolors{2}{white}{tablegray}
\begin{tabular}{@{}p{0.75\linewidth}c@{}}
\toprule
\textbf{Control Question} & \textbf{Response} \\ \midrule
Do you require MFA to access email? & \yes \\
Do you require MFA to log into computers? & \yes \\
\textbf{Do you require MFA to access sensitive data systems?} & \textbf{\no} \\
Does your organization have an employee acceptable use policy? & \yes \\
Does your organization do security awareness training for new employees? & \yes \\
Does your organization do security awareness training for all employees at least once per year? & \yes \\ \bottomrule
\end{tabular}
\caption{Security Controls Questionnaire Results.}
\label{tab:controls}
\end{table}

\paragraph{Analysis:} The failure to enforce MFA for sensitive data systems (marked with \no) is a critical security gap. While other controls are in place, compromised credentials could grant an attacker direct, unimpeded access to the organization's most valuable information assets. This finding is classified as a high-priority risk.

% --- Section 4: Technical Scan Results ---
\section{Technical Scan Results}
An Nmap scan was performed on the specified target to identify open ports and exposed services.

\begin{itemize}
    \item \textbf{Target IP:} \texttt{10.10.10.51}
    \item \textbf{Scan Date:} Not specified in scan data.
\end{itemize}

The scan revealed the following open port:

\begin{table}[h!]
\centering
\rowcolors{2}{white}{tablegray}
\begin{tabular}{@{}llll@{}}
\toprule
\textbf{Port} & \textbf{State} & \textbf{Service Name} & \textbf{Product / Version} \\ \midrule
3389/tcp & open & ms-wbt-server & (Not specified) \\ \bottomrule
\end{tabular}
\caption{Open Ports on \texttt{10.10.10.51}.}
\label{tab:nmap_results}
\end{table}

\paragraph{Analysis:} The scan confirms that port \textbf{3389/tcp}, used for Microsoft Remote Desktop Protocol (RDP), is open on \texttt{10.10.10.51}. RDP is a primary target for attackers seeking to gain initial access to a network, often through brute-force attacks or exploitation of vulnerabilities. Exposing this service, even on an internal network, significantly increases the attack surface. This finding, combined with pre-existing risk data, points to a systemic issue.

% --- Section 5: Correlated Risk Assessment ---
\section{Correlated Risk Assessment}
This section synthesizes findings from the security control review, technical scan, and pre-existing risk register into a consolidated list of key risks.

\begin{table}[h!]
\centering
\begin{tabular}{@{}p{0.1\linewidth}p{0.3\linewidth}p{0.15\linewidth}p{0.35\linewidth}@{}}
\toprule
\textbf{Risk ID} & \textbf{Risk Title} & \textbf{Severity} & \textbf{Description \& Affected Systems} \\ \midrule
\rowcolor{white}
RISK-001 & \textbf{Systemic RDP Exposure} & \cellcolor{sev_critical}\color{white} \textbf{Critical} & The Remote Desktop Protocol (RDP) is exposed on multiple internal systems. This is a common vector for ransomware and unauthorized access. \newline \textbf{Affected:} \texttt{10.10.10.50}, \texttt{10.10.10.51} \\
\addlinespace[3pt]
\rowcolor{tablegray}
RISK-002 & \textbf{Lack of MFA for Sensitive Systems} & \cellcolor{sev_critical}\color{white} \textbf{Critical} & Multi-Factor Authentication is not required to access systems containing sensitive data. A single compromised password could lead to a major data breach. \newline \textbf{Affected:} All sensitive data systems. \\
\bottomrule
\end{tabular}
\caption{Summary of Identified Risks.}
\label{tab:risks}
\end{table}

% --- Section 6: Recommendations ---
\section{Recommendations}
The following actions are recommended to mitigate the identified risks. Recommendations are prioritized based on severity and potential impact.

\subsection{Priority 1: Critical Risks (Remediate Immediately)}
\begin{enumerate}
    \item \textbf{Remediate RDP Exposure (RISK-001):}
        \begin{itemize}
            \item \textbf{Immediate Action:} Implement host-based or network firewall rules to block all access to TCP port 3389 on \texttt{10.10.10.50}, \texttt{10.10.10.51}, and any other systems where it is not strictly required for business operations.
            \item \textbf{Long-Term Solution:} For necessary remote administration, implement a secure access solution such as a Virtual Private Network (VPN) or a bastion host (jump box) with strong access controls and monitoring. Public or broad internal exposure of RDP should be strictly prohibited by policy.
        \end{itemize}

    \item \textbf{Implement MFA for Sensitive Systems (RISK-002):}
        \begin{itemize}
            \item \textbf{Immediate Action:} Identify all systems classified as containing sensitive data. Develop a phased plan to deploy and mandate the use of MFA for all user accounts (including administrative and service accounts) that can access these systems.
            \item \textbf{Long-Term Solution:} Integrate MFA into the identity and access management (IAM) lifecycle. Ensure that any new system deployed with sensitive data automatically enforces MFA as a baseline security requirement.
        \end{itemize}
\end{enumerate}

\subsection{Priority 2: Proactive Measures}
\begin{enumerate}
    \setcounter{enumi}{2} % Continue numbering
    \item \textbf{Conduct Comprehensive Internal Scans:}
        \begin{itemize}
            \item Perform authenticated and unauthenticated vulnerability scans across the entire internal network to identify other instances of exposed services, misconfigurations, or outdated software.
        \end{itemize}
    \item \textbf{Review Network Segmentation and Firewall Rules:}
        \begin{itemize}
            \item Review the current network architecture to ensure proper segmentation is in place, limiting lateral movement for potential attackers. Firewall rules should be audited to enforce a principle of least privilege.
        \end{itemize}
\end{enumerate}

\end{document}
```