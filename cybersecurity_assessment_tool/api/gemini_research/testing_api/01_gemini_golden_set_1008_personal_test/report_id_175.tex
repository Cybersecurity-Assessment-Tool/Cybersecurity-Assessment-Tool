```latex
\documentclass[12pt]{article}

% Preamble: Required Packages
\usepackage[margin=1in]{geometry}
\usepackage{pifont}
\usepackage{booktabs}
\usepackage{hyperref}
\usepackage{url}
\usepackage{seqsplit}
\usepackage{graphicx}
\usepackage{xcolor}
\usepackage{datetime}

% --- Document Metadata ---
\title{Cybersecurity Posture Assessment Report}
\author{Cybersecurity Analysis Division}
\date{\today}

% --- Hyperref Setup ---
\hypersetup{
    colorlinks=true,
    linkcolor=blue,
    filecolor=magenta,      
    urlcolor=cyan,
    pdftitle={Cybersecurity Posture Assessment Report},
    pdfpagemode=FullScreen,
}

% --- Custom Commands ---
\newcommand{\yes}{\textcolor{green}{\ding{51}}}
\newcommand{\no}{\textcolor{red}{\ding{55}}}

\begin{document}

\maketitle
\thispagestyle{empty}
\newpage

\tableofcontents
\newpage

% --- Section 1: Executive Summary ---
\section{Executive Summary}

This report provides a cybersecurity posture assessment for \textbf{Nova Terra}, conducted on \today. The analysis is based on a synthesis of network scan data, an organizational security questionnaire, and a review of pre-existing risks.

The assessment reveals several critical and high-risk gaps in the organization's security controls. While foundational controls like Multi-Factor Authentication (MFA) are implemented for email and computer access, they are critically absent for sensitive data systems. Furthermore, the complete lack of a security awareness training program for both new and existing employees presents a significant risk, leaving the organization vulnerable to social engineering and phishing attacks.

Technical analysis identified an exposed Secure Shell (SSH) service on the organization's IPv6 network. While not an immediate vulnerability, an exposed management port is a common target for attackers and must be rigorously secured and monitored.

Immediate remediation should focus on implementing MFA across all sensitive systems and establishing a comprehensive security awareness training program. These actions will substantially mitigate the most severe risks identified in this report.

% --- Section 2: Organizational Information ---
\section{Organizational Information}

The following details were provided for the assessment. This information is used to establish the context and scope of the review.

\begin{tabular}{@{}ll}
\toprule
\textbf{Attribute} & \textbf{Value} \\
\midrule
Organization Name & \textbf{Nova Terra} \\
Primary Email Domain & \texttt{NovaTerra.org} \\
Primary Website & \url{www.NovaTerra.org} \\
External IPv4 Address & \seqsplit{\texttt{208.204.223.196}} \\
Scanned IPv6 Target & \seqsplit{\texttt{2001:db8::1}} \\
\bottomrule
\end{tabular}

% --- Section 3: Security Control Review ---
\section{Security Control Review}

An administrative security control review was conducted based on a questionnaire. The responses are summarized below. Items marked with a \no\ represent significant gaps in the security program and are discussed further in the Risk Assessment section.

\begin{table}[h!]
\centering
\caption{Security Questionnaire Responses}
\begin{tabular}{p{0.7\textwidth}c}
\toprule
\textbf{Control Question} & \textbf{Response} \\
\midrule
Do you require MFA to access email? & \yes \\
Do you require MFA to log into computers? & \yes \\
Do you require MFA to access sensitive data systems? & \no \\
Does your organization have an employee acceptable use policy? & \yes \\
Does your organization do security awareness training for new employees? & \no \\
Does your organization do security awareness training for all employees at least once per year? & \no \\
\bottomrule
\end{tabular}
\end{table}

\subsection*{Analysis of Gaps}
The review identified three major control gaps:
\begin{itemize}
    \item \textbf{No MFA for Sensitive Data:} The absence of MFA on systems housing sensitive data is a critical vulnerability. Should an attacker compromise a user's credentials, they would have direct access to the organization's most valuable information assets.
    \item \textbf{No New Employee Training:} Failing to train new employees on security best practices from day one introduces immediate risk. New hires are often prime targets for social engineering attacks.
    \item \textbf{No Annual Refresher Training:} The security landscape is constantly evolving. Without regular, ongoing training, employees' security awareness diminishes, making them more susceptible to modern threats like sophisticated phishing campaigns.
\end{itemize}

% --- Section 4: Technical Scan Results ---
\section{Technical Scan Results}

An external network scan was performed against the target IP address \seqsplit{\texttt{2001:db8::1}}. The scan identified the following open ports and services.

\begin{table}[h!]
\centering
\caption{Open Port Scan Results for \seqsplit{\texttt{2001:db8::1}}}
\begin{tabular}{llll}
\toprule
\textbf{Port} & \textbf{State} & \textbf{Service (Inferred)} & \textbf{Notes} \\
\midrule
22/tcp & open & SSH & Secure Shell is a remote management protocol. \\
& & & No version information was available from the scan. \\
& & & Exposed management services are a common attack vector. \\
\bottomrule
\end{tabular}
\end{table}

\subsection*{Analysis of Findings}
The scan confirms that Port 22 (SSH) is accessible from the internet. While SSH is an encrypted protocol, its exposure requires strict security measures. Unauthorized access could grant an attacker administrative control over the underlying system. The risk is heightened by the identified gaps in security training, as a phished credential could potentially be used to access this service.

% --- Section 5: Risk Assessment ---
\section{Risk Assessment}

This section correlates the findings from the security control review and the technical scan. No pre-existing vulnerabilities were reported in the input data. The following new risks have been identified and prioritized.

\begin{table}[h!]
\centering
\caption{Identified Risks and Severity}
\begin{tabular}{p{0.1\textwidth}p{0.45\textwidth}p{0.2\textwidth}l}
\toprule
\textbf{ID} & \textbf{Risk Description} & \textbf{Affected Asset(s)} & \textbf{Severity} \\
\midrule
RISK-001 & Lack of MFA on sensitive systems allows for single-factor credential compromise, leading to data breach. & Sensitive Data, Core Systems & \textbf{Critical} \\
\addlinespace
RISK-002 & Inadequate security awareness training program increases susceptibility to phishing and social engineering attacks. & All Employees, All Systems & \textbf{High} \\
\addlinespace
RISK-003 & Exposed SSH management port on an external-facing system could be exploited via brute-force or stolen credentials. & Server at \seqsplit{\texttt{2001:db8::1}} & \textbf{Medium} \\
\bottomrule
\end{tabular}
\end{table}

% --- Section 6: Recommendations ---
\section{Recommendations}

Based on the identified risks, the following prioritized actions are recommended to improve the cybersecurity posture of \textbf{Nova Terra}.

\subsection*{RISK-001: Lack of MFA on Sensitive Systems (Critical)}
\begin{itemize}
    \item \textbf{Action:} Immediately develop and execute a plan to enforce MFA on all systems and applications classified as containing sensitive data.
    \item \textbf{Justification:} This is the most effective control to prevent unauthorized access resulting from credential theft. It adds a critical layer of defense for the organization's most important assets.
\end{itemize}

\subsection*{RISK-002: Inadequate Security Awareness Training (High)}
\begin{itemize}
    \item \textbf{Action:} Establish a formal security awareness training program. This must include mandatory training for all new hires during their onboarding process and an annual refresher course for all existing staff.
    \item \textbf{Justification:} A well-trained workforce is the first line of defense against phishing and social engineering. This program will reduce human error and strengthen the overall security culture.
\end{itemize}

\subsection*{RISK-003: Exposed SSH Management Port (Medium)}
\begin{itemize}
    \item \textbf{Action:} Review the business necessity for the publicly accessible SSH port on \seqsplit{\texttt{2001:db8::1}}.
    \begin{itemize}
        \item If not required, block access at the firewall.
        \item If required, implement compensating controls: restrict access to a whitelist of trusted IP addresses, enforce the use of strong cryptographic keys for authentication, and disable password-based logins entirely.
    \end{itemize}
    \item \textbf{Justification:} This follows the principle of least privilege, minimizing the attack surface and making it significantly harder for attackers to gain unauthorized administrative access.
\end{itemize}

\end{document}
```