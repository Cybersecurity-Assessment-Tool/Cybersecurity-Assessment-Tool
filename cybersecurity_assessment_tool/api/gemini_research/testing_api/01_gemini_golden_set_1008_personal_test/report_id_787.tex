```latex
\documentclass[12pt]{article}

% Preamble: Required Packages
\usepackage[a4paper, margin=1in]{geometry}
\usepackage{pifont} % For checkmarks and crosses
\usepackage{booktabs} % For professional tables
\usepackage{hyperref} % For clickable links
\usepackage{url} % For URL formatting
\usepackage{seqsplit} % For splitting long strings
\usepackage{graphicx}
\usepackage{fancyhdr}
\usepackage{xcolor}

% Define colors
\definecolor{darkblue}{rgb}{0.0, 0.0, 0.55}
\definecolor{gray}{rgb}{0.5, 0.5, 0.5}

% Hyperref setup
\hypersetup{
    colorlinks=true,
    linkcolor=darkblue,
    filecolor=darkblue,      
    urlcolor=darkblue,
    citecolor=darkblue,
}

% Header and Footer
\pagestyle{fancy}
\fancyhf{}
\fancyhead[L]{Cybersecurity Posture Report}
\fancyhead[R]{Solaris Energy}
\fancyfoot[C]{\thepage}
\renewcommand{\headrulewidth}{0.4pt}
\renewcommand{\footrulewidth}{0.4pt}

% Check and Cross symbols
\newcommand{\cmark}{\ding{51}}
\newcommand{\xmark}{\ding{55}}

\begin{document}

% --- Title Page ---
\begin{titlepage}
    \centering
    \vspace*{2cm}
    
    {\Huge \textbf{Cybersecurity Posture Report}\par}
    \vspace{1.5cm}
    
    {\Large \textbf{Prepared for:}\par}
    \vspace{0.5cm}
    {\Large Solaris Energy\par}
    
    \vfill
    
    {\large \textbf{Date of Report:}} \today\par
    \vspace{0.5cm}
    {\large \textbf{Generated By:}} Cybersecurity Analyst\par
    
    \vspace{2cm}
    
    \begin{center}
        \rule{0.8\textwidth}{0.4pt}
    \end{center}
    \vspace{0.5cm}
    
    \textit{This report contains sensitive information and should be handled with care. Distribution is restricted to authorized personnel only.}
    
\end{titlepage}

\tableofcontents
\newpage

% --- Executive Summary ---
\section{Executive Summary}

This report provides a comprehensive analysis of the cybersecurity posture for Solaris Energy, based on a review of organizational security controls, an external network scan, and pre-existing risk data.

The assessment reveals several critical and high-risk gaps that require immediate attention. The most significant concerns are the lack of Multi-Factor Authentication (MFA) for email and sensitive data systems. These gaps expose the organization to substantial risks, including business email compromise, data breaches, and unauthorized access. Furthermore, foundational security practices, such as an Acceptable Use Policy and security training for new employees, are not in place, weakening the overall security culture.

A technical scan identified an exposed Secure Shell (SSH) service on the IPv6 address \texttt{2001:db8::1}. While not immediately vulnerable, this service presents a potential entry point for attackers if not properly configured and monitored.

Immediate remediation should focus on implementing MFA across all critical platforms, developing core security policies, and securing the exposed network service. Addressing these findings will significantly improve the organization's resilience against common cyber threats.

% --- Organizational Information ---
\section{Organizational Information}

The following details were provided for the assessment. This information establishes the context and scope of the review.

\begin{tabular}{@{}ll}
    \toprule
    \textbf{Attribute} & \textbf{Value} \\
    \midrule
    Organization Name & Solaris Energy \\
    Email Domain & \texttt{SolarisEnergy.com} \\
    Website Domain & \url{www.SolarisEnergy.com} \\
    External IP Address & \texttt{169.144.21.174} \\
    \bottomrule
\end{tabular}

% --- Security Control Review ---
\section{Security Control Review}

A review of administrative security controls was conducted via a questionnaire. The results highlight critical deficiencies in access control and employee security awareness policies.

\begin{table}[h!]
\centering
\caption{Organizational Security Controls Questionnaire}
\begin{tabular}{@{}p{0.7\linewidth}c@{}}
    \toprule
    \textbf{Control Question} & \textbf{Status} \\
    \midrule
    Do you require MFA to access email? & \textcolor{red}{\xmark} \\
    Do you require MFA to log into computers? & \textcolor{green}{\cmark} \\
    Do you require MFA to access sensitive data systems? & \textcolor{red}{\xmark} \\
    Does your organization have an employee acceptable use policy? & \textcolor{red}{\xmark} \\
    Does your organization do security awareness training for new employees? & \textcolor{red}{\xmark} \\
    Does your organization do security awareness training for all employees at least once per year? & \textcolor{green}{\cmark} \\
    \bottomrule
\end{tabular}
\end{table}

\subsection*{Analysis of Control Gaps}
The responses marked with an \textcolor{red}{\xmark} represent significant security gaps:
\begin{itemize}
    \item \textbf{No MFA for Email/Sensitive Data:} This is a critical vulnerability. The lack of MFA on these systems means a single compromised password could grant an attacker full access, leading to potential data exfiltration or financial fraud.
    \item \textbf{No Acceptable Use Policy (AUP):} An AUP is a foundational document that sets clear expectations for employee behavior when using company assets. Its absence creates ambiguity and legal risk.
    \item \textbf{No New Employee Training:} New hires are often targeted by attackers. Failing to provide security training during onboarding leaves a critical window of vulnerability open.
\end{itemize}

% --- Technical Scan Results ---
\section{Technical Scan Results}

An external network scan was performed to identify exposed services and potential vulnerabilities.

\subsection*{Scan Target: \texttt{2001:db8::1}}
The scan revealed the following open port on the target system.

\begin{table}[h!]
\centering
\caption{Open Ports on \texttt{2001:db8::1}}
\begin{tabular}{@{}llll@{}}
    \toprule
    \textbf{Port} & \textbf{State} & \textbf{Inferred Service} & \textbf{Notes} \\
    \midrule
    22/tcp & open & SSH (Secure Shell) & Administrative access protocol. \\
    \bottomrule
\end{tabular}
\end{table}

\subsection*{Analysis of Technical Findings}
The presence of an open SSH port indicates that remote administrative access is enabled on this system. While SSH is a secure protocol, its exposure to the public internet makes it a prime target for automated brute-force and credential stuffing attacks. Without further information on its configuration (e.g., password vs. key-based authentication, firewall rules), this service is considered a medium-level risk.

% --- Risk Assessment ---
\section{Risk Assessment}

The following table synthesizes findings from the security control review and the technical scan into a prioritized list of identified risks. No pre-existing risks were reported.

\begin{table}[h!]
\centering
\caption{Summary of Identified Risks}
\begin{tabular}{@{}p{0.1\linewidth}p{0.25\linewidth}p{0.45\linewidth}l@{}}
    \toprule
    \textbf{ID} & \textbf{Risk Name} & \textbf{Description} & \textbf{Severity} \\
    \midrule
    RISK-001 & Lack of MFA on Critical Systems & Email and sensitive data systems are protected only by passwords, creating a high risk of account takeover and data breach. & \textbf{Critical} \\
    \addlinespace
    RISK-002 & Inadequate Security Policies \& Training & The absence of an AUP and new hire training leads to inconsistent security practices and increased susceptibility to social engineering. & \textbf{High} \\
    \addlinespace
    RISK-003 & Exposed SSH Management Port & An open SSH port on \texttt{2001:db8::1} is exposed to the internet, creating a target for unauthorized access attempts. & \textbf{Medium} \\
    \bottomrule
\end{tabular}
\end{table}

% --- Recommendations ---
\section{Recommendations}

The following actionable recommendations are provided to mitigate the identified risks. They are prioritized based on severity and potential impact.

\subsection*{Priority 1: Critical}
\begin{enumerate}
    \item \textbf{Implement MFA on All Critical Systems:}
    \begin{itemize}
        \item \textbf{Action:} Immediately deploy and enforce MFA for all user accounts with access to email (e.g., Office 365, Google Workspace) and any systems identified as storing or processing sensitive data.
        \item \textbf{Impact:} Drastically reduces the risk of account compromise and unauthorized data access.
    \end{itemize}
\end{enumerate}

\subsection*{Priority 2: High}
\begin{enumerate}
    \setcounter{enumi}{1}
    \item \textbf{Develop and Implement Foundational Policies:}
    \begin{itemize}
        \item \textbf{Action:} Draft and formally adopt an Acceptable Use Policy (AUP) that all employees must read and acknowledge. This policy should define the rules for using company networks, devices, and data.
        \item \textbf{Impact:} Establishes a clear security baseline for all employees and reduces legal liability.
    \end{itemize}
    \item \textbf{Establish a Security Onboarding Program:}
    \begin{itemize}
        \item \textbf{Action:} Create a mandatory security awareness training module for all new employees to be completed during their first week of onboarding.
        \item \textbf{Impact:} Ensures new staff are aware of key security risks and organizational policies from day one.
    \end{itemize}
\end{enumerate}

\subsection*{Priority 3: Medium}
\begin{enumerate}
    \setcounter{enumi}{3}
    \item \textbf{Secure the Exposed SSH Service:}
    \begin{itemize}
        \item \textbf{Action:} Review the business need for exposing SSH on \texttt{2001:db8::1}. If required, implement the following controls:
        \begin{itemize}
            \item Restrict access using a firewall to only trusted source IP addresses.
            \item Disable password-based authentication and enforce the use of strong cryptographic keys (e.g., ED25519).
            \item Implement an intrusion detection tool like Fail2ban to automatically block brute-force attempts.
        \end{itemize}
        \item \textbf{Impact:} Hardens a potential entry point and reduces the external attack surface.
    \end{itemize}
\end{enumerate}

\end{document}
```