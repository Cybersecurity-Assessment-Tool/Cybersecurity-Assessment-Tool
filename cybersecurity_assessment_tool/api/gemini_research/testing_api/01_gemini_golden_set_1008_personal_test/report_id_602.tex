```latex
\documentclass[12pt, a4paper]{article}

%========================================================================================
%   PACKAGES
%========================================================================================
\usepackage[margin=1in]{geometry} % Set page margins
\usepackage{pifont}               % For checkmarks and crosses (\ding)
\usepackage{booktabs}             % For professional-looking tables
\usepackage{xcolor}               % For custom colors
\usepackage{graphicx}             % To include images (e.g., logo)
\usepackage{hyperref}             % For clickable links and table of contents
\usepackage{url}                  % For formatting URLs
\usepackage{seqsplit}             % For splitting long strings in \texttt

%========================================================================================
%   DOCUMENT CONFIGURATION
%========================================================================================
% Hyperref setup
\hypersetup{
    colorlinks=true,
    linkcolor=blue,
    filecolor=magenta,      
    urlcolor=cyan,
    pdftitle={Cybersecurity Posture Report},
    pdfpagemode=FullScreen,
}

% Define custom colors for severity levels
\definecolor{criticalred}{HTML}{D7263D}
\definecolor{highorange}{HTML}{F49D42}
\definecolor{mediumyellow}{HTML}{F4D442}
\definecolor{lowblue}{HTML}{88A2AA}

% Define commands for checkmarks and crosses for consistency
\newcommand{\yes}{\textcolor{green}{\ding{51}}}
\newcommand{\no}{\textcolor{red}{\ding{55}}}

%========================================================================================
%   TITLE PAGE
%========================================================================================
\title{
    \vspace{2cm}
    \textbf{Cybersecurity Posture Report} \\
    \large \textit{Analysis and Recommendations} \\
    \vspace{1cm}
    \textbf{Prepared for: True Grit} \\
    \vspace{4cm}
}
\author{Cybersecurity Analyst Group}
\date{\today}

%========================================================================================
%   BEGIN DOCUMENT
%========================================================================================
\begin{document}

\maketitle
\thispagestyle{empty}
\newpage

\tableofcontents
\newpage

%========================================================================================
%   1. EXECUTIVE SUMMARY
%========================================================================================
\section{Executive Summary}

This report provides a comprehensive analysis of the cybersecurity posture for \textbf{True Grit}, based on a combination of technical network scanning, a review of organizational security controls, and an assessment of pre-existing risks.

The analysis revealed several critical and high-severity risks that require immediate attention. The most significant finding is a publicly accessible FTP server running a dangerously outdated version of \texttt{vsftpd} (2.3.4). This specific version is known to contain a critical backdoor vulnerability (CVE-2011-2523) and is currently configured to allow anonymous logins, posing a direct and immediate threat of system compromise and data exfiltration.

Furthermore, significant gaps were identified in the organization's access control policies. The absence of Multi-Factor Authentication (MFA) for accessing email and other sensitive data systems exposes the organization to severe risks, including business email compromise, phishing attacks, and unauthorized data access. The lack of a formal employee acceptable use policy exacerbates these risks by failing to establish clear security guidelines.

This report outlines these findings in detail and provides a prioritized list of actionable recommendations to mitigate the identified risks and strengthen the overall security posture of \textbf{True Grit}.

%========================================================================================
%   2. ORGANIZATIONAL INFORMATION
%========================================================================================
\section{Organizational Information}

The following details were provided for the assessment. This information helps contextualize the findings within the organization's environment.

\begin{table}[h!]
\centering
\begin{tabular}{@{}ll@{}}
\toprule
\textbf{Attribute} & \textbf{Value} \\ \midrule
Organization Name & True Grit \\
Email Domain & \texttt{TrueGrit.net} \\
Website Domain & \url{www.TrueGrit.net} \\
External IP Address & \texttt{120.127.236.21} \\ \bottomrule
\end{tabular}
\caption{Client Organizational Details}
\label{tab:org_info}
\end{table}

%========================================================================================
%   3. SECURITY CONTROL REVIEW
%========================================================================================
\section{Security Control Review}

A review of internal security controls was conducted via a questionnaire. The responses highlight critical gaps in administrative and access control policies. A summary of the questions and their corresponding analysis is presented below.

\begin{table}[h!]
\centering
\begin{tabular}{@{}p{0.6\textwidth} c p{0.25\textwidth}@{}}
\toprule
\textbf{Control Question} & \textbf{Status} & \textbf{Analyst Note} \\ \midrule
Do you require MFA to access email? & \no & \textbf{Critical Gap.} Lack of MFA is a primary vector for account compromise. \\
Do you require MFA to log into computers? & \yes & Good Practice. Protects against unauthorized physical access. \\
Do you require MFA to access sensitive data systems? & \no & \textbf{Critical Gap.} Exposes crown jewel data assets to high risk. \\
Does your organization have an employee acceptable use policy? & \no & \textbf{High Risk.} Creates ambiguity and lacks a legal baseline for user actions. \\
Does your organization do security awareness training for new employees? & \yes & Good Practice. Establishes a security baseline from day one. \\
Does your organization do security awareness training for all employees at least once per year? & \yes & Good Practice. Reinforces security-conscious behavior. \\
\bottomrule
\end{tabular}
\caption{Security Control Questionnaire Analysis}
\label{tab:controls}
\end{table}

%========================================================================================
%   4. TECHNICAL SCAN RESULTS
%========================================================================================
\section{Technical Scan Results}

A network scan was performed on the target system to identify open ports, running services, and potential vulnerabilities.

\subsection{Nmap Scan Findings}
\begin{itemize}
    \item \textbf{Target IP Address:} \texttt{10.0.0.15}
    \item \textbf{Scan Date:} \today
\end{itemize}

The scan identified one open port with a critically vulnerable service.

\begin{table}[h!]
\centering
\begin{tabular}{@{}lllll@{}}
\toprule
\textbf{Port} & \textbf{State} & \textbf{Service} & \textbf{Version} & \textbf{Finding} \\ \midrule
21/tcp & Open & FTP & vsftpd 2.3.4 & \begin{tabular}[t]{@{}l@{}}\textbf{Critical Vulnerability:}\\ - Anonymous FTP login allowed. \\ - Version is vulnerable to a known \\ \hspace{2mm} backdoor (CVE-2011-2523).\end{tabular} \\ \bottomrule
\end{tabular}
\caption{Open Ports and Services Analysis}
\label{tab:nmap}
\end{table}

\paragraph{Analysis:} The presence of \texttt{vsftpd 2.3.4} is a severe security risk. This version, released in 2011, contains a well-documented backdoor that allows an attacker to gain a command shell on the server by sending a specific string as the username. Combined with the "Anonymous FTP login allowed" configuration, this system is trivial to exploit and should be considered compromised.

%========================================================================================
%   5. CONSOLIDATED RISK ASSESSMENT
%========================================================================================
\section{Consolidated Risk Assessment}

The following table synthesizes all findings from the technical scan, control review, and pre-existing risk data into a prioritized list.

\begin{table}[h!]
\centering
\begin{tabular}{@{}lp{0.4\textwidth}p{0.15\textwidth}p{0.2\textwidth}@{}}
\toprule
\textbf{ID} & \textbf{Risk Title \& Description} & \textbf{Severity} & \textbf{Affected Elements} \\ \midrule
\textbf{TR-001} & \textbf{Exploitable FTP Server:} The server at \texttt{10.0.0.15} is running vsftpd 2.3.4, which has a known backdoor (CVE-2011-2523) and allows anonymous login. & \textcolor{criticalred}{\textbf{Critical}} & Network Server, Internal Data \\
\addlinespace
\textbf{PR-001} & \textbf{No MFA on Email/Sensitive Data:} Lack of MFA on critical systems creates a high risk of account takeover and subsequent data breach. & \textcolor{criticalred}{\textbf{Critical}} & Email System, Data Repositories, User Accounts \\
\addlinespace
\textbf{PR-002} & \textbf{No Acceptable Use Policy:} Absence of a formal AUP leads to inconsistent security practices and exposes the organization to insider threats and legal liabilities. & \textcolor{highorange}{\textbf{High}} & All Employees, Organizational Governance \\
\addlinespace
\textbf{ER-001} & \textbf{Outdated Windows Policy:} (Existing Risk) Computers are running Windows 7, which is end-of-life and no longer receives security updates. & \textcolor{mediumyellow}{\textbf{Medium}} & Workstations \\
\bottomrule
\end{tabular}
\caption{Summary of Identified Risks}
\label{tab:risks}
\end{table}

%========================================================================================
%   6. RECOMMENDATIONS
%========================================================================================
\section{Recommendations}

The following actions are recommended to mitigate the identified risks. They are prioritized based on severity and potential impact.

\subsection{Immediate Actions (TR-001)}
\begin{itemize}
    \item \textbf{Decommission FTP Server:} Immediately take the server at \texttt{10.0.0.15} offline. Disconnect it from the network to prevent exploitation.
    \item \textbf{Investigate for Compromise:} Assume the system has been compromised. Conduct a forensic analysis to determine the extent of any breach.
    \item \textbf{Replace File Transfer Solution:} If FTP is a business requirement, replace it with a modern, secure protocol like SFTP (SSH File Transfer Protocol) or a managed file transfer solution. Ensure anonymous access is disabled.
\end{itemize}

\subsection{High-Priority Actions (PR-001)}
\begin{itemize}
    \item \textbf{Implement MFA:} Enforce MFA for all users on the email system (\texttt{TrueGrit.net}) and all systems identified as containing sensitive data. This is the single most effective control to prevent account compromise.
    \item \textbf{Provide User Training:} Accompany the MFA rollout with clear instructions and support for all employees to ensure a smooth transition.
\end{itemize}

\subsection{Medium-Priority Actions (PR-002 \& ER-001)}
\begin{itemize}
    \item \textbf{Develop an Acceptable Use Policy (AUP):} Create a formal AUP that clearly defines the rules for using company assets, data handling, and internet usage. Require all employees to read and acknowledge the policy.
    \item \textbf{Accelerate OS Upgrades:} Prioritize the plan to upgrade all remaining Windows 7 workstations to a supported operating system, such as Windows 10 or 11, to ensure they receive critical security patches.
\end{itemize}

\end{document}
```