```latex
\documentclass[12pt]{article}

% --- PACKAGES ---
\usepackage[margin=1in]{geometry}
\usepackage{pifont} % For checkmarks and crosses
\usepackage{booktabs} % For professional tables
\usepackage{hyperref} % For clickable links
\usepackage{url} % For URL formatting
\usepackage{seqsplit} % To split long strings in tt font
\usepackage{graphicx}
\usepackage{fancyhdr}
\usepackage{xcolor}
\usepackage{lastpage}

% --- DOCUMENT SETUP ---
\hypersetup{
    colorlinks=true,
    linkcolor=black,
    urlcolor=blue,
}

% --- HEADER & FOOTER ---
\pagestyle{fancy}
\fancyhf{} % Clear all header and footer fields
\fancyhead[L]{Cybersecurity Risk Assessment Report}
\fancyhead[R]{\textbf{Modern Myth}}
\fancyfoot[C]{\thepage\ of \pageref{LastPage}}
\renewcommand{\headrulewidth}{0.4pt}
\renewcommand{\footrulewidth}{0.4pt}

% --- CUSTOM COMMANDS ---
\newcommand{\yes}{\ding{51}}
\newcommand{\no}{\ding{55}}
\definecolor{severitycritical}{HTML}{990000}
\definecolor{severityhigh}{HTML}{D14302}
\definecolor{severitymedium}{HTML}{EFAF00}
\newcommand{\sev_critical}[1]{\textcolor{severitycritical}{\textbf{#1}}}
\newcommand{\sev_high}[1]{\textcolor{severityhigh}{\textbf{#1}}}
\newcommand{\sev_medium}[1]{\textcolor{severitymedium}{\textbf{#1}}}

% --- DOCUMENT START ---
\begin{document}

\title{
    \vspace{2cm}
    \textbf{Cybersecurity Risk Assessment Report} \\
    \large Prepared for: \textbf{Modern Myth}
    \vspace{1cm}
}
\author{Cybersecurity Analysis Division}
\date{\today}
\maketitle
\thispagestyle{empty}

\newpage

\tableofcontents

\newpage

% ==============================================================================
\section{Executive Summary}
% ==============================================================================

This report details the findings of a cybersecurity assessment for \textbf{Modern Myth}. The analysis is based on a combination of network scanning, a review of organizational security controls, and an evaluation of pre-existing risk data.

The assessment identified several areas of significant concern that require immediate attention. Key findings include a \sev_critical{critically vulnerable FTP server} configured with anonymous access, posing a direct and immediate threat of system compromise. Furthermore, significant procedural gaps were identified, including a \sev_high{lack of multi-factor authentication (MFA)} for computer and sensitive data access, and a \sev_high{complete absence of a security awareness training program}.

These vulnerabilities, combined with the known risk of outdated Windows 7 workstations, create a high-risk environment susceptible to unauthorized access, data breach, and malware infection. This report provides a consolidated view of these risks and outlines prioritized, actionable recommendations to mitigate them and improve the organization's overall security posture.

% ==============================================================================
\section{Organizational Information}
% ==============================================================================

The following information was provided by the client and used as a baseline for this assessment.

\begin{table}[h!]
\centering
\begin{tabular}{@{}ll@{}}
\toprule
\textbf{Attribute} & \textbf{Value} \\ \midrule
Organization Name & \textbf{Modern Myth} \\
Primary Email Domain & \seqsplit{\texttt{ModernMyth.net}} \\
Primary Website Domain & \seqsplit{\texttt{www.ModernMyth.net}} \\
External IP Address & \seqsplit{\texttt{48.112.84.90}} \\ \bottomrule
\end{tabular}
\caption{Client Profile}
\label{tab:client_profile}
\end{table}

% ==============================================================================
\section{Security Control Review}
% ==============================================================================

A review of internal security controls was conducted based on a standardized questionnaire. The responses highlight critical gaps in user access controls and employee security training. "No" answers indicate a deviation from security best practices and represent significant areas of risk.

\begin{table}[h!]
\centering
\begin{tabular}{@{}p{0.7\textwidth}cc@{}}
\toprule
\textbf{Control Question} & \textbf{Response} & \textbf{Status} \\ \midrule
Do you require MFA to access email? & Yes & \yes \\
\textbf{Do you require MFA to log into computers?} & \textbf{No} & \sev_high{\no} \\
\textbf{Do you require MFA to access sensitive data systems?} & \textbf{No} & \sev_high{\no} \\
Does your organization have an employee acceptable use policy? & Yes & \yes \\
\textbf{Does your organization do security awareness training for new employees?} & \textbf{No} & \sev_high{\no} \\
\textbf{Does your organization do security awareness training for all employees at least once per year?} & \textbf{No} & \sev_high{\no} \\ \bottomrule
\end{tabular}
\caption{Security Controls Questionnaire Results}
\label{tab:controls_review}
\end{table}

\subsection{Analysis of Control Gaps}
The lack of MFA for computer and sensitive data access drastically increases the risk of unauthorized access from compromised credentials. The absence of a security awareness training program leaves the organization highly vulnerable to phishing and social engineering attacks, which are primary vectors for credential theft.

% ==============================================================================
\section{Technical Scan Results}
% ==============================================================================

An external network scan was performed against the target IP address \seqsplit{\texttt{10.0.0.15}}. The scan revealed one open port with a service that presents a critical vulnerability.

\begin{table}[h!]
\centering
\begin{tabular}{@{}lllll@{}}
\toprule
\textbf{Port} & \textbf{State} & \textbf{Service} & \textbf{Version} & \textbf{Details} \\ \midrule
21/tcp & open & ftp & vsftpd 2.3.4 & \sev_critical{Anonymous FTP login allowed} \\ \bottomrule
\end{tabular}
\caption{Open Port Analysis}
\label{tab:nmap_results}
\end{table}

\subsection{Analysis of Technical Findings}
The FTP server running \textbf{vsftpd version 2.3.4} is a major security risk. This specific version is widely known to be vulnerable to a critical backdoor (CVE-2011-2523), which allows an unauthenticated attacker to execute arbitrary commands on the server. The configuration allowing \textbf{anonymous FTP login} makes this vulnerability trivial to exploit. This finding requires immediate remediation.

% ==============================================================================
\section{Consolidated Risk Assessment}
% ==============================================================================

The following table synthesizes findings from the security control review, technical scan, and pre-existing risk data into a prioritized list.

\begin{table}[h!]
\centering
\begin{tabular}{@{}p{0.1\textwidth}p{0.5\textwidth}p{0.15\textwidth}p{0.15\textwidth}@{}}
\toprule
\textbf{ID} & \textbf{Risk Description} & \textbf{Source(s)} & \textbf{Severity} \\ \midrule
RISK-001 & A publicly accessible FTP server (vsftpd 2.3.4) is vulnerable to remote code execution and allows anonymous login. & Technical Scan & \sev_critical{Critical} \\
\addlinespace
RISK-002 & Lack of Multi-Factor Authentication (MFA) on employee computers and sensitive data systems. & Questionnaire & \sev_high{High} \\
\addlinespace
RISK-003 & No security awareness training program exists for new or current employees, increasing susceptibility to phishing. & Questionnaire & \sev_high{High} \\
\addlinespace
RISK-004 & Workstations are running on an outdated and unsupported operating system (Windows 7). & Existing Risks & \sev_medium{Medium} \\ \bottomrule
\end{tabular}
\caption{Consolidated Risk Register}
\label{tab:risk_register}
\end{table}

% ==============================================================================
\section{Recommendations}
% ==============================================================================

Based on the consolidated risk assessment, the following actions are recommended to mitigate the identified vulnerabilities and strengthen the organization's security posture.

\subsection{Immediate Priority (Remediate within 24-48 hours)}
\begin{itemize}
    \item \textbf{RISK-001: Remediate Vulnerable FTP Server}
    \begin{enumerate}
        \item Immediately take the FTP server at \seqsplit{\texttt{10.0.0.15}} offline or firewall it from external access.
        \item If the FTP service is business-critical, upgrade the \texttt{vsftpd} software to the latest stable version.
        \item Disable anonymous FTP access permanently. Access should only be granted to authenticated, authorized users.
        \item Evaluate the need for FTP and consider more secure alternatives like SFTP (SSH File Transfer Protocol).
    \end{enumerate}
\end{itemize}

\subsection{High Priority (Remediate within 30 days)}
\begin{itemize}
    \item \textbf{RISK-002: Implement Multi-Factor Authentication (MFA)}
    \begin{enumerate}
        \item Procure and deploy an MFA solution for all remote access, administrator access, and access to sensitive data systems.
        \item Develop a policy and rollout plan to require MFA for all employee computer logins.
    \end{enumerate}
    \item \textbf{RISK-003: Establish Security Awareness Training}
    \begin{enumerate}
        \item Implement a mandatory security awareness training module for all new employees as part of the onboarding process.
        \item Schedule and require annual security awareness training for all existing employees, covering topics like phishing, password security, and acceptable use.
    \end{enumerate}
\end{itemize}

\subsection{Medium Priority (Remediate within 90 days)}
\begin{itemize}
    \item \textbf{RISK-004: Upgrade Outdated Operating Systems}
    \begin{enumerate}
        \item Continue with the existing plan to upgrade all Windows 7 workstations to a modern, supported operating system (e.g., Windows 10/11).
        \item Prioritize the upgrade for users with access to sensitive information.
    \end{enumerate}
\end{itemize}

\end{document}
```