```latex
\documentclass[12pt]{article}

% --- PACKAGES ---
\usepackage[margin=1in]{geometry} % Set page margins
\usepackage{pifont}               % For checkmarks and crosses (dingbats)
\usepackage{booktabs}             % For professional-looking tables
\usepackage[hidelinks]{hyperref}  % For hyperlinks without ugly boxes
\usepackage{url}                  % For formatting URLs
\usepackage{seqsplit}             % For splitting long strings in texttt
\usepackage{graphicx}             % For logos, etc.
\usepackage{fancyhdr}             % For headers and footers

% --- DOCUMENT METADATA ---
\title{Cybersecurity Posture Assessment Report \\ \large For: \textbf{Blackwood Industries}}
\author{Cybersecurity Analysis Division}
\date{\today}

% --- HEADER & FOOTER ---
\pagestyle{fancy}
\fancyhf{} % Clear all header and footer fields
\fancyhead[L]{Cybersecurity Assessment Report}
\fancyhead[R]{Blackwood Industries}
\fancyfoot[C]{\thepage}
\renewcommand{\headrulewidth}{0.4pt}
\renewcommand{\footrulewidth}{0.4pt}

\begin{document}

\maketitle
\thispagestyle{empty}
\newpage

\tableofcontents
\newpage

% --- EXECUTIVE SUMMARY ---
\section*{1.0 Executive Summary}

This report provides a cybersecurity posture assessment for \textbf{Blackwood Industries}, based on an analysis of network scan data, organizational security controls, and a review of pre-existing risk information. The assessment was conducted on \today.

The analysis revealed several critical and high-risk findings that require immediate attention. A significant discrepancy was identified between the current technical findings and the organization's existing risk register, indicating a potential flaw in the risk management process.

\textbf{Key Critical Findings:}
\begin{itemize}
    \item \textbf{Exposed Sensitive System:} A network scan identified an open service on internal IP \texttt{10.5.5.5} at port \texttt{8080}. The service's title is publicly advertised as \textbf{"TOP SECRET DB"}, which represents a critical information disclosure vulnerability.
    \item \textbf{Lack of Multi-Factor Authentication (MFA):} The organization does not enforce MFA for accessing email or sensitive data systems. This gap, combined with the exposed database, creates a high-impact scenario for potential data breaches.
    \item \textbf{Inaccurate Risk Assessment:} The existing risk register incorrectly classifies port \texttt{8080} as a "confirmed secure" false positive. This new finding directly contradicts that assessment, suggesting the risk management process is not accurately reflecting the current security posture.
\end{itemize}

This report outlines these findings in detail and provides prioritized, actionable recommendations to mitigate the identified risks and strengthen the overall security posture of \textbf{Blackwood Industries}.

% --- ORGANIZATIONAL INFORMATION ---
\section*{2.0 Organizational Information}

This section details the organizational data provided for the assessment. This information is used to establish the context for the technical and procedural analysis.

\begin{tabular}{@{}ll}
\toprule
\textbf{Attribute} & \textbf{Value} \\
\midrule
Organization Name & \textbf{Blackwood Industries} \\
Email Domain & \texttt{BlackwoodIndustries.net} \\
Website Domain & \texttt{www.BlackwoodIndustries.net} \\
External IP Address & \texttt{93.70.249.236} \\
\bottomrule
\end{tabular}

% --- SECURITY CONTROL REVIEW ---
\section*{3.0 Security Control Review}

The following table summarizes the organization's responses to a security controls questionnaire. "No" answers indicate significant gaps in the security framework and are correlated with other findings in this report.

\begin{tabular}{@{}p{0.7\linewidth}cc}
\toprule
\textbf{Control Question} & \textbf{Response} & \textbf{Status} \\
\midrule
Do you require MFA to access email? & No & \ding{55} \\
Do you require MFA to log into computers? & Yes & \ding{51} \\
Do you require MFA to access sensitive data systems? & No & \ding{55} \\
Does your organization have an employee acceptable use policy? & Yes & \ding{51} \\
Does your organization do security awareness training for new employees? & No & \ding{55} \\
Does your organization do security awareness training for all employees at least once per year? & Yes & \ding{51} \\
\bottomrule
\end{tabular}

\subsection*{Analysis of Control Gaps}
\begin{itemize}
    \item \textbf{No MFA for Email/Sensitive Systems:} This is a critical deficiency. Email is a primary vector for phishing and account compromise. Lack of MFA on sensitive systems removes a crucial layer of defense against unauthorized access, especially relevant given the findings in Section 4.0.
    \item \textbf{No Security Training for New Employees:} New hires are often more susceptible to social engineering. Failing to provide security training upon onboarding creates a persistent window of vulnerability.
\end{itemize}

% --- TECHNICAL SCAN RESULTS ---
\section*{4.0 Technical Scan Results}

A network scan was performed to identify active services and potential vulnerabilities on the specified target system.

\subsection*{Scan Target}
\begin{itemize}
    \item \textbf{Target IP Address:} \texttt{10.5.5.5}
\end{itemize}

\subsection*{Open Ports and Service Information}
The scan revealed the following open port.

\begin{tabular}{@{}llll}
\toprule
\textbf{Port} & \textbf{State} & \textbf{Service/Script} & \textbf{Output / Notes} \\
\midrule
8080/tcp & open & http-title & \textbf{TOP SECRET DB} \\
\bottomrule
\end{tabular}

\subsection*{Technical Findings Analysis}
The discovery of port \texttt{8080} with the HTTP title \textbf{"TOP SECRET DB"} is a \textbf{CRITICAL} finding.
\begin{enumerate}
    \item \textbf{Information Disclosure:} The title explicitly advertises the system as containing highly sensitive information ("TOP SECRET"). This makes it a high-value target for any malicious actor with internal network access.
    \item \textbf{Poor Configuration:} Such a descriptive title is a severe operational security failure. System banners and titles should be generic and non-descriptive.
    \item \textbf{Contradiction of Existing Data:} This finding directly invalidates the pre-existing risk entry which stated, "Port 8080 is confirmed secure and false positive." The system is neither secure nor a false positive.
\end{enumerate}

% --- RISK ASSESSMENT ---
\section*{5.0 Consolidated Risk Assessment}

This section synthesizes the findings from the security control review and the technical scan to provide a consolidated view of the primary risks facing the organization.

\begin{tabular}{@{}p{0.2\linewidth}p{0.6\linewidth}l}
\toprule
\textbf{Risk Name} & \textbf{Overview} & \textbf{Severity} \\
\midrule
\textbf{Exposed Sensitive Data System} & A system on \texttt{10.5.5.5:8080} is titled "TOP SECRET DB", indicating high sensitivity. It is exposed on the internal network without adequate access controls or obfuscation, creating a direct path to a potential data breach. & \textbf{Critical} \\
\addlinespace
\textbf{Lack of MFA on Critical Systems} & MFA is not enforced on email or sensitive data systems. This weakness dramatically increases the risk of account takeover and unauthorized access to critical assets, including the exposed database. & \textbf{High} \\
\addlinespace
\textbf{Inadequate Employee Onboarding} & New employees do not receive security awareness training, leaving a gap in the human firewall. This makes the organization more susceptible to phishing and social engineering attacks. & \textbf{Medium} \\
\addlinespace
\textbf{Flawed Risk Management Process} & The existing risk register contains critically inaccurate information, providing a false sense of security. This indicates the process for identifying, assessing, and tracking risks is unreliable. & \textbf{Medium} \\
\bottomrule
\end{tabular}

% --- RECOMMENDATIONS ---
\section*{6.0 Recommendations}

The following actions are recommended to mitigate the identified risks. They are prioritized based on severity and potential impact.

\subsection*{Priority 1: Immediate Actions}
\begin{enumerate}
    \item \textbf{Isolate and Investigate Exposed System:} Immediately investigate the system at \texttt{10.5.5.5:8080}.
    \begin{itemize}
        \item Restrict all network access to the system except for essential administrative personnel.
        \item Identify the owner, purpose, and sensitivity of the data it contains.
        \item Change the HTTP title to a generic, non-descriptive value.
    \end{itemize}
    \item \textbf{Implement Access Controls:} Place the system behind a firewall and enforce strict Access Control Lists (ACLs), ensuring only authorized users and systems can connect to it.
\end{enumerate}

\subsection*{Priority 2: High-Impact Remediations}
\begin{enumerate}
    \item \textbf{Enforce Mandatory MFA:} Deploy and enforce MFA across all email accounts and any system classified as containing sensitive data. This is the single most effective control to prevent unauthorized access.
    \item \textbf{Implement Onboarding Security Training:} Integrate mandatory security awareness training into the new employee onboarding process. This training should occur within the first week of employment.
\end{enumerate}

\subsection*{Priority 3: Process Improvement}
\begin{enumerate}
    \item \textbf{Overhaul Risk Management Process:} Conduct a full review of the risk assessment and management lifecycle.
    \begin{itemize}
        \item Ensure that technical scan results are regularly used to validate and update the risk register.
        \item Decommission any risk tracking data that is proven to be inaccurate.
    \end{itemize}
\end{enumerate}

\end{document}
```