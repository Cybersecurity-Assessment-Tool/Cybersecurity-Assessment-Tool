```latex
\documentclass[12pt]{article}

% ----------------------------------------------------------------------
% PREAMBLE
% ----------------------------------------------------------------------
\usepackage[margin=1in]{geometry}
\usepackage{pifont} % For checkmarks and crosses
\usepackage{booktabs} % For professional tables
\usepackage{hyperref} % For clickable links
\usepackage{url} % For URL formatting
\usepackage{seqsplit} % For splitting long strings in tt font
\usepackage[utf8]{inputenc}

% --- Hyperref Setup ---
\hypersetup{
    colorlinks=true,
    linkcolor=black,
    citecolor=black,
    urlcolor=blue,
    pdftitle={Cybersecurity Posture Assessment Report},
    pdfauthor={Cybersecurity Analysis Division},
    pdfsubject={Security Assessment},
    pdfkeywords={Cybersecurity, Risk, Assessment, Nmap, Security Controls}
}

% --- Custom Commands for Tables ---
\newcommand{\yes}{\ding{51}} % Green checkmark
\newcommand{\no}{\ding{55}}  % Red cross

% ----------------------------------------------------------------------
% DOCUMENT START
% ----------------------------------------------------------------------
\begin{document}

% --- Title Page ---
\title{Cybersecurity Posture Assessment Report \\ \large For: White Label}
\author{Cybersecurity Analysis Division}
\date{November 22, 2025}
\maketitle

\tableofcontents
\newpage

% ----------------------------------------------------------------------
% SECTION 1: EXECUTIVE OVERVIEW
% ----------------------------------------------------------------------
\section{Executive Overview}

This report details the findings of a cybersecurity posture assessment conducted on November 22, 2025, for White Label. The assessment combined a review of organizational security controls via a questionnaire, an external network scan, and a review of pre-existing risks.

Overall, White Label has implemented several foundational security controls, including Multi-Factor Authentication (MFA) for email and computer access, and maintains a security awareness training program. These are commendable practices that reduce the organization's attack surface.

However, this assessment has identified two high-severity risks that require immediate attention:
\begin{enumerate}
    \item \textbf{Lack of MFA on Sensitive Data Systems:} The absence of mandatory MFA for accessing critical data introduces a significant risk of unauthorized access and potential data breach.
    \item \textbf{Outdated Web Server Software:} The public-facing web server is running an outdated version of Nginx (1.18.0), which is missing critical security patches and is likely vulnerable to known exploits.
\end{enumerate}

These findings indicate critical gaps in the organization's defense-in-depth strategy. This report provides a detailed analysis of these risks and offers actionable recommendations to mitigate them effectively.

% ----------------------------------------------------------------------
% SECTION 2: ORGANIZATIONAL INFORMATION
% ----------------------------------------------------------------------
\section{Organizational Information}

The following information was provided for the assessment.

\begin{tabular}{@{}ll}
    \toprule
    \textbf{Attribute} & \textbf{Value} \\
    \midrule
    Organization Name & \textbf{White Label} \\
    Email Domain & \texttt{WhiteLabel.org} \\
    Website Domain & \url{www.WhiteLabel.org} \\
    External IP Address & \texttt{118.147.23.206} \\
    \bottomrule
\end{tabular}

% ----------------------------------------------------------------------
% SECTION 3: SECURITY CONTROL REVIEW
% ----------------------------------------------------------------------
\section{Security Control Review}

A security questionnaire was completed to evaluate the implementation of key administrative and technical controls. The results are summarized below. A checkmark (\yes) indicates a positive response (control in place), while a cross (\no) indicates a negative response, representing a potential security gap.

\subsection{Questionnaire Results}

\begin{tabular}{@{}p{0.8\linewidth}c@{}}
    \toprule
    \textbf{Control Question} & \textbf{Response} \\
    \midrule
    Do you require MFA to access email? & \yes \\
    Do you require MFA to log into computers? & \yes \\
    \textbf{Do you require MFA to access sensitive data systems?} & \textbf{\no} \\
    Does your organization have an employee acceptable use policy? & \yes \\
    Does your organization do security awareness training for new employees? & \yes \\
    Does your organization do security awareness training for all employees at least once per year? & \yes \\
    \bottomrule
\end{tabular}

\subsection{Analysis}
The organization has successfully implemented MFA for standard access controls like email and computer logins. However, the failure to extend this critical control to sensitive data systems represents a significant security oversight. An attacker who compromises a user's credentials would have direct access to the organization's most valuable data. This gap is classified as a high-severity risk.

% ----------------------------------------------------------------------
% SECTION 4: TECHNICAL SCAN RESULTS
% ----------------------------------------------------------------------
\section{Technical Scan Results}

A network scan was performed on November 22, 2025, to identify open ports and exposed services on the target system.

\subsection{Host: 192.168.10.5}

The scan identified one host as active, with the following open port and service.

\begin{tabular}{@{}llll@{}}
    \toprule
    \textbf{Port} & \textbf{State} & \textbf{Service} & \textbf{Product \& Version} \\
    \midrule
    443/tcp & open & https & Nginx 1.18.0 \\
    \bottomrule
\end{tabular}

\subsection{Analysis}
The scan revealed a web server running Nginx version 1.18.0. This version was released in April 2020 and has since been superseded by numerous stable releases containing important security fixes. Running outdated, internet-facing software exposes the organization to a wide range of publicly known vulnerabilities (CVEs), which could be exploited by attackers to achieve remote code execution, denial of service, or information disclosure. This finding is classified as a high-severity risk.

% ----------------------------------------------------------------------
% SECTION 5: RISK ASSESSMENT SUMMARY
% ----------------------------------------------------------------------
\section{Risk Assessment Summary}

This section correlates the findings from the security control review and the technical scan. No pre-existing vulnerabilities were reported. The following new risks have been identified.

\begin{tabular}{@{}p{0.1\linewidth}p{0.25\linewidth}p{0.4\linewidth}p{0.1\linewidth}@{}}
    \toprule
    \textbf{Risk ID} & \textbf{Risk Name} & \textbf{Description} & \textbf{Severity} \\
    \midrule
    \textbf{RISK-001} & Lack of MFA on Sensitive Systems & The absence of multi-factor authentication for systems storing or processing sensitive data allows for unauthorized access if user credentials are compromised. & \textbf{High} \\
    \addlinespace
    \textbf{RISK-002} & Outdated Web Server Software & The public-facing web server is running Nginx 1.18.0, a version with known vulnerabilities. This could allow for a system compromise. & \textbf{High} \\
    \bottomrule
\end{tabular}

% ----------------------------------------------------------------------
% SECTION 6: RECOMMENDATIONS
% ----------------------------------------------------------------------
\section{Recommendations}

The following actions are recommended to mitigate the identified risks and improve the overall security posture of White Label.

\subsection{RISK-001: Remediate MFA Gap}
\begin{itemize}
    \item \textbf{Immediate Action:} Prioritize and enforce mandatory MFA for all user accounts (including administrative and service accounts) that have access to systems containing sensitive or critical data.
    \item \textbf{Strategic Action:} Review the data classification policy to ensure all systems are correctly categorized. Implement role-based access control (RBAC) in conjunction with MFA to enforce the principle of least privilege.
\end{itemize}

\subsection{RISK-002: Remediate Outdated Software}
\begin{itemize}
    \item \textbf{Immediate Action:} Plan and execute an upgrade of the Nginx web server from version 1.18.0 to the latest stable version. Before deployment, test the new version in a staging environment to ensure compatibility with existing applications.
    \item \textbf{Strategic Action:} Establish a formal patch and vulnerability management program. This program should include regular automated scanning for outdated software, a defined patching schedule for critical systems, and a process for tracking and remediating identified vulnerabilities.
\end{itemize}

\end{document}
```