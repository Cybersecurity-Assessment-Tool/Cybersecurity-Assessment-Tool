```latex
\documentclass[12pt]{article}

% Preamble: Required Packages
\usepackage[margin=1in]{geometry}
\usepackage{pifont} % For \ding symbols
\usepackage{booktabs} % For professional tables
\usepackage{hyperref} % For hyperlinks and metadata
\usepackage{url} % For URL formatting
\usepackage{seqsplit} % For splitting long strings like IPs
\usepackage{graphicx} % For potential logos (not used here, but good practice)
\usepackage{xcolor} % For colors

% Hyperref Setup for PDF Metadata
\hypersetup{
    colorlinks=true,
    linkcolor=black,
    urlcolor=blue,
    pdftitle={Cybersecurity Posture Assessment Report},
    pdfauthor={Cybersecurity Analyst},
    pdfsubject={Security Analysis},
    pdfkeywords={Cybersecurity, Risk Assessment, Network Scan}
}

% Document Title and Author
\title{Cybersecurity Posture Assessment Report \\ \large For: \textbf{Iron Bridge Legal}}
\author{Cybersecurity Analyst}
\date{\today}

\begin{document}

\maketitle
\tableofcontents
\newpage

% --- 1. Executive Summary ---
\section*{1. Executive Summary}

This report provides a comprehensive cybersecurity posture assessment for \textbf{Iron Bridge Legal}. The analysis is based on a correlation of network scan data, a security controls questionnaire, and a review of pre-existing risks.

The assessment reveals a mixed security posture with several critical and high-risk gaps. While the organization has implemented foundational controls such as security awareness training and Multi-Factor Authentication (MFA) for email, significant weaknesses exist in access control and administrative policies. The lack of MFA for computer and sensitive data system access represents a critical vulnerability. Furthermore, the absence of a formal Acceptable Use Policy (AUP) indicates a gap in governance.

Technical scans identified an externally exposed Secure Shell (SSH) management port on an IPv6 address, which increases the attack surface. Immediate remediation of these identified risks is strongly recommended to reduce the likelihood of a security incident.

% --- 2. Organizational Information ---
\section*{2. Organizational Information}

The following details were provided for the assessment scope.

\begin{itemize}
    \item \textbf{Organization Name:} \textbf{Iron Bridge Legal}
    \item \textbf{Email Domain:} \texttt{IronBridgeLegal.org}
    \item \textbf{Website Domain:} \url{www.IronBridgeLegal.org}
    \item \textbf{Primary External IP:} \texttt{202.3.185.228}
\end{itemize}

% --- 3. Security Control Review ---
\section*{3. Security Control Review}

A review of administrative and technical security controls was conducted via a questionnaire. The responses indicate critical gaps in the organization's access control policies. A summary of the findings is presented in Table 1.

\begin{table}[h!]
\centering
\caption{Security Controls Questionnaire Results}
\begin{tabular}{p{0.8\linewidth} c}
\toprule
\textbf{Control Question} & \textbf{Response} \\
\midrule
Do you require MFA to access email? & \textcolor{green}{\ding{51}} \\
Do you require MFA to log into computers? & \textcolor{red}{\ding{55}} \\
Do you require MFA to access sensitive data systems? & \textcolor{red}{\ding{55}} \\
Does your organization have an employee acceptable use policy? & \textcolor{red}{\ding{55}} \\
Does your organization do security awareness training for new employees? & \textcolor{green}{\ding{51}} \\
Does your organization do security awareness training for all employees at least once per year? & \textcolor{green}{\ding{51}} \\
\bottomrule
\end{tabular}
\end{table}

% --- 4. Technical Scan Results ---
\section*{4. Technical Scan Results}

An external network scan was performed to identify open ports and exposed services on the organization's perimeter. The scan identified one open port on the specified target.

\begin{itemize}
    \item \textbf{Scan Target:} \seqsplit{\texttt{2001:db8::1}}
    \item \textbf{Host Status:} Up
\end{itemize}

\begin{table}[h!]
\centering
\caption{Open Port Analysis}
\begin{tabular}{l l l l}
\toprule
\textbf{Port} & \textbf{State} & \textbf{Service} & \textbf{Notes} \\
\midrule
22/tcp & open & ssh & Secure Shell (SSH) is a common remote management \\
       &      &     & protocol. Exposing this service directly to the \\
       &      &     & internet is a significant security risk. \\
\bottomrule
\end{tabular}
\end{table}

\textit{Note: The scan did not retrieve service version information. A more in-depth, authenticated scan is recommended to identify potential vulnerabilities associated with outdated software.}

% --- 5. Risk Assessment Summary ---
\section*{5. Risk Assessment Summary}

This section synthesizes findings from the security control review and technical scans. No pre-existing vulnerabilities were reported. The following new risks have been identified and prioritized based on their potential impact.

\begin{table}[h!]
\centering
\caption{Identified Risks and Severity}
\begin{tabular}{p{0.1\linewidth} p{0.6\linewidth} l}
\toprule
\textbf{Risk ID} & \textbf{Description} & \textbf{Severity} \\
\midrule
\textbf{RISK-001} & \textbf{Lack of MFA on Endpoints and Sensitive Systems:} The absence of MFA on computers and sensitive data systems exposes the organization to credential theft and unauthorized access. An attacker with valid credentials could gain deep access to the network. & \textbf{Critical} \\
\addlinespace
\textbf{RISK-002} & \textbf{No Employee Acceptable Use Policy (AUP):} Without a formal AUP, there is no documented standard for employee behavior regarding company assets and data. This creates legal and security ambiguity and weakens the organization's ability to enforce security policies. & \textbf{High} \\
\addlinespace
\textbf{RISK-003} & \textbf{Exposed SSH Management Port:} Port 22 (SSH) is open to the public internet. This exposes a critical management interface to brute-force attacks, credential stuffing, and exploitation of potential zero-day vulnerabilities in the SSH service. & \textbf{High} \\
\bottomrule
\end{tabular}
\end{table}

% --- 6. Recommendations ---
\section*{6. Recommendations}

Based on the findings in this report, the following actions are recommended to mitigate the identified risks and improve the overall security posture of \textbf{Iron Bridge Legal}.

\begin{enumerate}
    \item \textbf{Implement Comprehensive MFA (RISK-001):} Prioritize the deployment of Multi-Factor Authentication across all endpoints (computer logins) and for all applications or systems that store or process sensitive data. This is the single most effective control to mitigate unauthorized access.

    \item \textbf{Develop and Enforce an AUP (RISK-002):} Create a formal Acceptable Use Policy that clearly defines the rules and responsibilities for all employees when using company technology and data. Ensure all employees read and acknowledge the policy.

    \item \textbf{Secure the SSH Port (RISK-003):} Immediately restrict access to the open SSH port (22). Access should be limited to specific, trusted IP addresses via firewall rules. The preferred solution is to place this management interface behind a Virtual Private Network (VPN).

    \item \textbf{Conduct In-depth Vulnerability Scanning:} Schedule regular, authenticated vulnerability scans for all internal and external assets. This will provide deeper insight into software versions and patch levels, allowing for proactive vulnerability management.
\end{enumerate}

\end{document}
```