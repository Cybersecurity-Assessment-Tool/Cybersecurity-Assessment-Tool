```latex
\documentclass[12pt]{article}

% --- PACKAGE IMPORTS ---
\usepackage[margin=1in]{geometry}
\usepackage{pifont} % For checkmarks and crosses
\usepackage{booktabs} % For professional tables
\usepackage[hidelinks]{hyperref} % For clickable links
\usepackage{url} % For formatting URLs
\usepackage{seqsplit} % For splitting long strings without spaces
\usepackage{graphicx}
\usepackage{xcolor}

% --- DOCUMENT METADATA ---
\title{Cybersecurity Assessment Report \\ \large For: Iron River Finance}
\author{Cybersecurity Analysis Division}
\date{\today}

% --- DOCUMENT START ---
\begin{document}

\maketitle
\thispagestyle{empty}
\newpage
\tableofcontents
\newpage

% =============================================================================
% SECTION 1: EXECUTIVE SUMMARY
% =============================================================================
\section{Executive Summary}

This report details the findings of a cybersecurity assessment for \textbf{Iron River Finance}. The analysis combines a review of organizational security controls, a technical network scan, and pre-existing risk data to provide a consolidated view of the organization's security posture.

The assessment identified several critical and high-risk vulnerabilities that require immediate attention. Key findings include:
\begin{itemize}
    \item \textbf{Critical Access Control Gap:} Multi-Factor Authentication (MFA) is not enforced for email access. This exposes the organization to significant risk from phishing, credential stuffing, and subsequent business email compromise (BEC) attacks.
    \item \textbf{Critical Policy Deficiencies:} The organization lacks a formal Acceptable Use Policy (AUP) and does not conduct security awareness training for employees. This indicates a substantial weakness in the human element of security, leaving the organization vulnerable to social engineering and insider threats.
    \item \textbf{Corroborated Technical Risk:} The network scan confirmed a service running on port 22 (typically SSH) on the localhost interface (\texttt{127.0.0.1}). This finding directly correlates with a pre-existing, maximum-severity risk entry ("Localhost Exposed," CVSS 10.0), validating a critical technical exposure that must be investigated and remediated.
\end{itemize}

Immediate remediation should focus on implementing MFA for email, developing foundational security policies, and investigating the exposed internal service. Addressing these issues will substantially improve the security posture of \textbf{Iron River Finance}.

% =============================================================================
% SECTION 2: ORGANIZATIONAL INFORMATION
% =============================================================================
\section{Organizational Information}

The following details were provided for the assessment. This information is used to establish the context and scope of the review.

\begin{table}[h!]
\centering
\begin{tabular}{@{}ll@{}}
\toprule
\textbf{Attribute} & \textbf{Value} \\ \midrule
Organization Name    & Iron River Finance \\
Email Domain         & \texttt{IronRiverFinance.com} \\
Website Domain       & \href{http://www.IronRiverFinance.com}{\texttt{www.IronRiverFinance.com}} \\
External IP Address  & \texttt{156.56.176.115} \\ \bottomrule
\end{tabular}
\caption{Client Organizational Details.}
\label{tab:org_info}
\end{table}

% =============================================================================
% SECTION 3: SECURITY CONTROL REVIEW
% =============================================================================
\section{Security Control Review}

A review of organizational security controls was conducted via a standardized questionnaire. The responses highlight significant gaps in administrative and policy-based controls. A "No" response (\ding{55}) indicates a deviation from security best practices and represents a potential risk.

\begin{table}[h!]
\centering
\begin{tabular}{@{}lc@{}}
\toprule
\textbf{Security Control Question} & \textbf{Response} \\ \midrule
Do you require MFA to access email? & \ding{55} \\
Do you require MFA to log into computers? & \ding{51} \\
Do you require MFA to access sensitive data systems? & \ding{51} \\
Does your organization have an employee acceptable use policy? & \ding{55} \\
Does your organization do security awareness training for new employees? & \ding{55} \\
Does your organization do security awareness training for all employees at least once per year? & \ding{55} \\ \bottomrule
\end{tabular}
\caption{Organizational Security Control Questionnaire Results.}
\label{tab:controls}
\end{table}

\paragraph{Analysis:} The lack of MFA on email is a critical weakness. Email is a primary target for attackers seeking to gain an initial foothold. Furthermore, the complete absence of an Acceptable Use Policy and any form of security awareness training creates a high-risk environment where employees are likely unaware of security best practices and organizational expectations, making them susceptible to social engineering attacks.

% =============================================================================
% SECTION 4: TECHNICAL SCAN RESULTS
% =============================================================================
\section{Technical Scan Results}

A network scan was performed to identify open ports and services on the specified target.

\subsection{Scan Summary}
\begin{itemize}
    \item \textbf{Target IP Address:} \texttt{127.0.0.1}
    \item \textbf{Host Status:} Up
\end{itemize}

\subsection{Open Ports and Services}
The scan identified the following open port. The absence of detailed service and version information indicates that further, more intrusive scanning may be necessary for a complete vulnerability assessment.

\begin{table}[h!]
\centering
\begin{tabular}{@{}lllll@{}}
\toprule
\textbf{Port} & \textbf{State} & \textbf{Service} & \textbf{Product} & \textbf{Version} \\ \midrule
22/tcp        & open           & ssh (inferred)   & Unknown          & Unknown          \\ \bottomrule
\end{tabular}
\caption{Open Ports Detected on \texttt{127.0.0.1}.}
\label{tab:scan_results}
\end{table}

\paragraph{Analysis:} The presence of an open port on the localhost interface (\texttt{127.0.0.1}) is a significant finding. This directly corroborates the pre-existing risk documented in the \textit{Current Risks} data, which flags an exposed service on this host as a critical (10.0) vulnerability. This suggests a potential misconfiguration or an unnecessary service running, which could be exploited by local processes or in chained attacks.

% =============================================================================
% SECTION 5: CONSOLIDATED RISK ASSESSMENT
% =============================================================================
\section{Consolidated Risk Assessment}

This section synthesizes findings from the security control review, technical scan, and pre-existing risk data into a unified list of identified risks.

\begin{table}[h!]
\centering
\begin{tabular}{@{}lll@{}}
\toprule
\textbf{Risk / Vulnerability} & \textbf{Severity} & \textbf{Description} \\ \midrule
\textbf{Localhost Exposed (Port 22)} & \textbf{Critical} & An open port on \texttt{127.0.0.1} corroborates a known \\
& (CVSS 10.0) & 10.0 severity risk. Potentially abusable by local malware. \\
\addlinespace
\textbf{No MFA on Email} & \textbf{Critical} & Lack of MFA on a primary communication channel exposes \\
& & the organization to account takeover and BEC. \\
\addlinespace
\textbf{No Security Awareness Training} & \textbf{High} & Employees are not trained to recognize or report \\
& & phishing, malware, or other security threats. \\
\addlinespace
\textbf{No Acceptable Use Policy} & \textbf{High} & Lack of a formal policy creates ambiguity regarding \\
& & secure and acceptable use of company assets. \\
\bottomrule
\end{tabular}
\caption{Summary of Identified Risks.}
\label{tab:risk_summary}
\end{table}

% =============================================================================
% SECTION 6: RECOMMENDATIONS
% =============================================================================
\section{Recommendations}

The following actions are recommended to mitigate the identified risks. Recommendations are prioritized based on severity and potential impact.

\subsection{Critical Priority}
\begin{enumerate}
    \item \textbf{Enforce MFA for Email Access:} Immediately implement and enforce multi-factor authentication (MFA) for all user accounts accessing the \texttt{IronRiverFinance.com} email system. Prioritize phishing-resistant methods such as authenticator apps (TOTP) or hardware security keys.
    \item \textbf{Investigate and Remediate Exposed Localhost Service:} Conduct an immediate investigation into the service running on port 22 of \texttt{127.0.0.1}.
    \begin{itemize}
        \item Determine the purpose of the service and why it is running.
        \item If the service is not essential for business operations, disable it.
        \item If the service is required, ensure it is properly configured and restrict access using host-based firewall rules to only the necessary applications or users.
    \end{itemize}
\end{enumerate}

\subsection{High Priority}
\begin{enumerate}
    \setcounter{enumi}{2} % Continue numbering from previous list
    \item \textbf{Implement a Security Awareness Training Program:}
    \begin{itemize}
        \item Institute mandatory security awareness training for all new hires as part of the onboarding process.
        \item Conduct annual, mandatory security training for all employees covering topics such as phishing, password security, and data handling.
        \item Perform regular phishing simulations to test and reinforce training.
    \end{itemize}
    \item \textbf{Develop and Implement an Acceptable Use Policy (AUP):} Create a formal AUP that clearly defines the rules and expectations for using company technology and data. This policy should be reviewed and signed by all employees.
\end{enumerate}

\end{document}
```