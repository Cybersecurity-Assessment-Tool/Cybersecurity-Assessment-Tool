```latex
\documentclass[12pt]{article}

% Preamble: Required Packages
\usepackage[margin=1in]{geometry}
\usepackage{pifont} % For checkmarks and crosses
\usepackage{booktabs} % For professional tables
\usepackage{hyperref} % For clickable links and references
\usepackage{url} % For formatting URLs
\usepackage{seqsplit} % To split long strings in tt font
\usepackage{graphicx} % For potential logos/images
\usepackage[T1]{fontenc}

% Document Metadata
\title{Cybersecurity Posture Assessment Report}
\author{Cybersecurity Analysis Division}
\date{\today}

% Hyperref Setup
\hypersetup{
    colorlinks=true,
    linkcolor=black,
    urlcolor=blue,
    pdftitle={Cybersecurity Posture Assessment Report},
    pdfauthor={Cybersecurity Analysis Division},
    pdfsubject={Security Assessment},
    pdfkeywords={Security, Nmap, Risk, Analysis}
}

\begin{document}

\maketitle
\thispagestyle{empty}
\newpage

\tableofcontents
\thispagestyle{empty}
\newpage

\setcounter{page}{1}

% ==============================================================================
% 1. Executive Summary
% ==============================================================================
\section{Executive Summary}

This report details the findings of a cybersecurity assessment conducted for \textbf{Astraeus Aerospace}. The analysis correlates data from a network vulnerability scan, an organizational security questionnaire, and a list of pre-existing risks.

The assessment identified several critical and high-risk issues requiring immediate attention. A key technical finding is the public exposure of a MySQL database service (\texttt{Port 3306}) running an End-of-Life (EOL) version (\texttt{MySQL 5.7.33}). EOL software no longer receives security updates, posing a significant and unpatchable threat to data confidentiality and integrity.

This technical vulnerability is compounded by significant gaps in organizational security controls. The organization currently lacks a formal Acceptable Use Policy and does not conduct security awareness training for new or existing employees. This absence of foundational security practices dramatically increases the organization's susceptibility to social engineering, phishing, and insider threats.

On a positive note, the organization has implemented Multi-Factor Authentication (MFA) across email, computer logins, and sensitive data systems, which is a commendable security strength.

Immediate remediation should focus on restricting network access to the exposed database and creating a migration plan for the EOL software. Concurrently, developing and implementing core security policies and a training program is crucial for long-term risk reduction.

% ==============================================================================
% 2. Organizational Information
% ==============================================================================
\section{Organizational Information}

The following details were provided for the assessment. This information is used to establish the context and scope of the review.

\begin{table}[h!]
\centering
\begin{tabular}{@{}ll@{}}
\toprule
\textbf{Attribute} & \textbf{Value} \\ \midrule
Organization Name & Astraeus Aerospace \\
Email Domain & \texttt{AstraeusAerospace.com} \\
Website Domain & \url{www.AstraeusAerospace.com} \\
External IP Address & \texttt{145.113.106.122} \\ \bottomrule
\end{tabular}
\caption{Client Organizational Details}
\label{tab:org_info}
\end{table}

% ==============================================================================
% 3. Security Control Review (Questionnaire)
% ==============================================================================
\section{Security Control Review (Questionnaire)}

A review of the organization's security controls was conducted via a questionnaire. The responses highlight both strengths and critical weaknesses in the current security posture.

\begin{table}[h!]
\centering
\begin{tabular}{@{}p{0.8\textwidth}c@{}}
\toprule
\textbf{Control Question} & \textbf{Response} \\ \midrule
Do you require MFA to access email? & \ding{51} \\
Do you require MFA to log into computers? & \ding{51} \\
Do you require MFA to access sensitive data systems? & \ding{51} \\
Does your organization have an employee acceptable use policy? & \ding{55} \\
Does your organization do security awareness training for new employees? & \ding{55} \\
Does your organization do security awareness training for all employees at least once per year? & \ding{55} \\ \bottomrule
\end{tabular}
\caption{Security Control Questionnaire Responses (\ding{51}=Yes, \ding{55}=No)}
\label{tab:controls}
\end{table}

\subsection*{Analysis of Gaps}
The responses indicate a robust implementation of Multi-Factor Authentication (MFA), which is a significant strength. However, the "No" responses identify three critical administrative control gaps:
\begin{itemize}
    \item \textbf{No Acceptable Use Policy (AUP):} Without an AUP, employees lack clear guidance on the proper use of company assets, data handling, and security responsibilities. This increases the risk of accidental data breaches and insider threats.
    \item \textbf{No Security Awareness Training:} The complete absence of a security awareness training program for both new and existing employees is a major vulnerability. Employees are the first line of defense, and without training, they are highly susceptible to phishing, social engineering, and other common attack vectors.
\end{itemize}

% ==============================================================================
% 4. Technical Scan Results
% ==============================================================================
\section{Technical Scan Results}

An Nmap scan was performed on the target system \texttt{172.16.50.20} to identify open ports and exposed services.

\begin{table}[h!]
\centering
\begin{tabular}{@{}llll@{}}
\toprule
\textbf{Port} & \textbf{State} & \textbf{Service} & \textbf{Product \& Version} \\ \midrule
3306/tcp & open & mysql & MySQL 5.7.33 \\ \bottomrule
\end{tabular}
\caption{Open Ports Detected on \texttt{172.16.50.20}}
\label{tab:nmap_results}
\end{table}

\subsection*{Analysis of Findings}
The scan revealed one open port, which presents two distinct, high-impact risks:
\begin{itemize}
    \item \textbf{Database Service Exposure:} Port 3306 is the default port for the MySQL database service. Exposing a database directly to the network is a critical security misconfiguration. It allows attackers to directly target the database with brute-force attacks, credential stuffing, and exploits, bypassing other layers of security.
    \item \textbf{End-of-Life (EOL) Software:} The identified version, \textbf{MySQL 5.7.33}, reached its official End of Life in October 2023. This means it no longer receives security patches from the vendor. Any vulnerabilities discovered after this date will remain unpatched, making the service a permanent and easily exploitable target.
\end{itemize}

% ==============================================================================
% 5. Consolidated Risk Assessment
% ==============================================================================
\section{Consolidated Risk Assessment}

The following table synthesizes findings from the questionnaire, technical scan, and pre-existing risk data into a consolidated list of security risks.

\begin{table}[h!]
\centering
\begin{tabular}{@{}p{0.25\textwidth}p{0.5\textwidth}l@{}}
\toprule
\textbf{Risk Name} & \textbf{Description} & \textbf{Severity} \\ \midrule
\textbf{End-of-Life Software in Use} & The exposed MySQL database is running version 5.7.33, which is past its End of Life and no longer receives security updates. & \textbf{Critical} \\
\addlinespace
\textbf{Database Exposure} & The MySQL database service on port 3306 is open to the network, allowing attackers to directly target a critical data asset. & \textbf{High (7.5)} \\
\addlinespace
\textbf{Lack of Security Policies \& Training} & The absence of an Acceptable Use Policy and security awareness training program leaves the organization vulnerable to human error and social engineering. & \textbf{High} \\
\bottomrule
\end{tabular}
\caption{Summary of Identified Risks}
\label{tab:risk_summary}
\end{table}

% ==============================================================================
% 6. Recommendations
% ==============================================================================
\section{Recommendations}

Based on the consolidated risk assessment, the following actions are recommended to mitigate the identified vulnerabilities and improve the overall security posture of \textbf{Astraeus Aerospace}.

\subsection{Immediate Actions (Critical Priority)}
\begin{enumerate}
    \item \textbf{Restrict Access to Database Port:} Immediately implement firewall rules to block all public access to TCP port 3306. Access should be restricted to only trusted application servers or specific administrative IP addresses.
    \item \textbf{Develop EOL Migration Plan:} Initiate a project to migrate the database from MySQL 5.7 to a currently supported version (e.g., MySQL 8.0 or later). This is the only way to resolve the underlying EOL vulnerability.
\end{enumerate}

\subsection{High Priority Actions}
\begin{enumerate}
    \item \textbf{Develop and Implement an Acceptable Use Policy (AUP):} Draft and enforce an AUP that clearly defines rules for employees regarding the use of company networks, systems, and data.
    \item \textbf{Establish a Security Awareness Training Program:} Procure and roll out a mandatory security awareness training program. This program must be included in the onboarding process for all new employees and conducted at least annually for all staff.
\end{enumerate}

\subsection{Long-Term Strategic Improvements}
\begin{enumerate}
    \item \textbf{Implement Network Segmentation:} As a long-term solution, ensure that critical systems like databases are placed in a separate, isolated network segment. Access should be brokered through a secure gateway, such as a VPN or a bastion host, rather than direct exposure.
    \item \textbf{Conduct Regular Vulnerability Scanning:} Implement a recurring, automated vulnerability scanning process for all external and internal assets to proactively identify and remediate security issues.
\end{enumerate}

\end{document}
```