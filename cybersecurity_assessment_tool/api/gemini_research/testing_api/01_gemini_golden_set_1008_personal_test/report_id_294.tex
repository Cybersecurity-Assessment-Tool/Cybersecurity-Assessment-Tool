```latex
\documentclass[12pt]{article}

% Preamble: Required Packages
\usepackage[margin=1in]{geometry}
\usepackage{pifont} % For checkmarks and crosses
\usepackage{booktabs} % For professional tables
\usepackage[hidelinks]{hyperref} % For clickable links
\usepackage{url} % For formatting URLs
\usepackage{seqsplit} % For breaking long strings
\usepackage{graphicx}
\usepackage{xcolor}

% Define colors for severity
\definecolor{criticalred}{HTML}{D73B3E}
\definecolor{highorange}{HTML}{F5A623}
\definecolor{mediumyellow}{HTML}{F8E71C}
\definecolor{lowblue}{HTML}{4A90E2}
\definecolor{infogray}{HTML}{9B9B9B}
\definecolor{mitigatedgreen}{HTML}{7ED321}

% --- Document Start ---
\begin{document}

% --- Title Page ---
\title{
    Cybersecurity Posture Assessment Report \\
    \large For: \textbf{Vivid Vision}
}
\author{Cybersecurity Analysis Division}
\date{\today}
\maketitle
\thispagestyle{empty}
\newpage

% --- Table of Contents ---
\tableofcontents
\newpage

% --- Section 1: Executive Summary ---
\section{Executive Summary}

This report provides a comprehensive cybersecurity posture assessment for \textbf{Vivid Vision}, conducted on \today. The analysis is based on a synthesis of network scan data, a security controls questionnaire, and a review of pre-existing risk documentation.

\paragraph{Key Findings:} The assessment reveals a mixed security posture. \textbf{Vivid Vision} has implemented foundational controls, such as Multi-Factor Authentication (MFA) for email and computer access. However, several critical and high-risk gaps were identified that significantly increase the organization's risk exposure. These include the absence of MFA for sensitive data systems, the lack of a formal Acceptable Use Policy, and no recurring annual security awareness training for all employees.

\paragraph{Positive Developments:} A technical network scan of the target host \texttt{192.168.0.5} found no open ports. This finding indicates that a previously documented risk, "Unencrypted Web Server" on port 80, has been effectively mitigated or was a false positive. This is a positive security development.

\paragraph{Primary Recommendations:} Immediate focus should be placed on implementing MFA for all sensitive systems. Concurrently, developing and enforcing an Acceptable Use Policy and establishing a mandatory annual security training program are crucial next steps to bolster the organization's human firewall and overall defensive capabilities.

% --- Section 2: Organizational Information ---
\section{Organizational Information}

The following details were provided for the assessment.

\begin{tabular}{@{}ll}
    \toprule
    \textbf{Attribute} & \textbf{Value} \\
    \midrule
    Organization Name & \textbf{Vivid Vision} \\
    Email Domain & \texttt{VividVision.com} \\
    Website Domain & \url{www.VividVision.com} \\
    External IP Address & \texttt{23.91.110.240} \\
    \bottomrule
\end{tabular}

% --- Section 3: Security Control Review ---
\section{Security Control Review}

The following table summarizes the organization's responses to a security controls questionnaire. Items marked with \ding{55} represent significant gaps in the current security framework and are discussed in the Risk Assessment section.

\begin{table}[h!]
\centering
\begin{tabular}{@{}p{0.7\linewidth} c c@{}}
    \toprule
    \textbf{Control Question} & \textbf{Response} & \textbf{Status} \\
    \midrule
    Do you require MFA to access email? & Yes & \ding{51} \\
    Do you require MFA to log into computers? & Yes & \ding{51} \\
    Do you require MFA to access sensitive data systems? & No & \textcolor{criticalred}{\ding{55}} \\
    \addlinespace
    Does your organization have an employee acceptable use policy? & No & \textcolor{highorange}{\ding{55}} \\
    \addlinespace
    Does your organization do security awareness training for new employees? & Yes & \ding{51} \\
    Does your organization do security awareness training for all employees at least once per year? & No & \textcolor{highorange}{\ding{55}} \\
    \bottomrule
\end{tabular}
\caption{Security Controls Questionnaire Results.}
\end{table}

% --- Section 4: Technical Scan Results ---
\section{Technical Scan Results}

A network scan was performed to identify externally exposed services and potential vulnerabilities.

\begin{itemize}
    \item \textbf{Scan Target:} \texttt{192.168.0.5}
    \item \textbf{Scan Date:} \today
\end{itemize}

\subsection{Port Scan Analysis}
The scan results indicate a very limited attack surface on the target host. No open ports were discovered.

\begin{table}[h!]
\centering
\begin{tabular}{@{}llll@{}}
    \toprule
    \textbf{Port} & \textbf{Protocol} & \textbf{State} & \textbf{Service/Version} \\
    \midrule
    80 & tcp & \textbf{closed} & http \\
    \bottomrule
\end{tabular}
\caption{Nmap Scan Results for \texttt{192.168.0.5}.}
\end{table}

\subsection{Correlation with Existing Risks}
A pre-existing risk entry, "Unencrypted Web Server," indicated that port 80 was open and serving unencrypted traffic. The current scan results directly contradict this finding, showing port 80 as \textbf{closed}. This suggests the vulnerability has been successfully remediated. It is recommended to formally update the internal risk register to reflect this mitigation.

% --- Section 5: Risk Assessment ---
\section{Risk Assessment}

This section correlates findings from the security control review, technical scans, and pre-existing risk data to provide a unified view of the current risk landscape.

\begin{table}[h!]
\centering
\begin{tabular}{@{}p{0.25\linewidth} p{0.5\linewidth} l@{}}
    \toprule
    \textbf{Risk Name} & \textbf{Description} & \textbf{Severity} \\
    \midrule
    \textbf{No MFA on Sensitive Systems} & The absence of MFA on systems storing or processing sensitive data exposes critical assets to unauthorized access via credential compromise. & \textcolor{criticalred}{\textbf{Critical}} \\
    \addlinespace
    \textbf{No Acceptable Use Policy (AUP)} & Without a formal AUP, employees lack clear guidelines on the secure use of company assets, increasing the risk of insider threat and unintentional data exposure. & \textcolor{highorange}{\textbf{High}} \\
    \addlinespace
    \textbf{No Annual Security Training} & Security knowledge degrades over time. The lack of annual refresher training weakens the organization's "human firewall," making employees more susceptible to phishing and social engineering. & \textcolor{highorange}{\textbf{High}} \\
    \addlinespace
    \textbf{Unencrypted Web Server} & \textit{(From Input 3)} Port 80 was believed to be open, exposing the organization to unencrypted communication. The recent scan confirms this port is closed. & \textcolor{mitigatedgreen}{\textbf{Mitigated}} \\
    \bottomrule
\end{tabular}
\caption{Summary of Identified and Reviewed Risks.}
\end{table}

% --- Section 6: Recommendations ---
\section{Recommendations}

The following actionable recommendations are prioritized based on the risk assessment to help \textbf{Vivid Vision} improve its cybersecurity posture.

\subsection{Priority 1: Critical}
\begin{itemize}
    \item \textbf{Implement MFA for Sensitive Data Access:}
    \begin{itemize}
        \item \textbf{Action:} Immediately deploy and enforce MFA for all user accounts (including administrative and service accounts) that have access to systems containing sensitive, proprietary, or confidential data.
        \item \textbf{Impact:} Drastically reduces the risk of unauthorized access and data breaches resulting from stolen credentials.
    \end{itemize}
\end{itemize}

\subsection{Priority 2: High}
\begin{itemize}
    \item \textbf{Develop and Implement an Acceptable Use Policy (AUP):}
    \begin{itemize}
        \item \textbf{Action:} Draft a comprehensive AUP that clearly defines the rules for using company networks, devices, and data. This policy should be formally communicated to all employees and acknowledged via signature.
        \item \textbf{Impact:} Establishes a baseline for secure employee behavior, reduces legal liability, and provides a framework for enforcing security standards.
    \end{itemize}
    \item \textbf{Establish Annual Security Awareness Training:}
    \begin{itemize}
        \item \textbf{Action:} Institute a mandatory security awareness training program for all employees to be completed annually. The training should cover current threats such as phishing, ransomware, and social engineering.
        \item \textbf{Impact:} Reinforces security best practices, keeps employees vigilant against evolving threats, and strengthens the overall security culture.
    \end{itemize}
\end{itemize}

\subsection{Priority 3: Informational}
\begin{itemize}
    \item \textbf{Update Risk Register:}
    \begin{itemize}
        \item \textbf{Action:} Formally document that the risk associated with the "Unencrypted Web Server" on port 80 has been investigated and confirmed as mitigated based on current network scan data.
        \item \textbf{Impact:} Ensures the risk register is accurate and reflects the current security posture, allowing resources to be focused on active threats.
    \end{itemize}
\end{itemize}

% --- Document End ---
\end{document}
```