```latex
\documentclass[12pt]{article}

% Preamble: Required Packages
\usepackage[margin=1in]{geometry}
\usepackage{pifont} % For checkmarks and crosses
\usepackage{booktabs} % For professional tables
\usepackage{hyperref} % For clickable links
\usepackage{url} % For formatting URLs
\usepackage{seqsplit} % For splitting long strings without spaces
\usepackage{graphicx}
\usepackage{xcolor}
\usepackage{fancyhdr}

% Document Metadata and Styling
\hypersetup{
    colorlinks=true,
    linkcolor=blue,
    filecolor=magenta,      
    urlcolor=cyan,
    pdftitle={Cybersecurity Posture Report},
    pdfpagemode=FullScreen,
}

\pagestyle{fancy}
\fancyhf{}
\lhead{Cybersecurity Posture Report}
\rhead{\textbf{Radiant Life}}
\cfoot{\thepage}

\title{Cybersecurity Posture Report \\ \large For: \textbf{Radiant Life}}
\author{Cybersecurity Analysis Division}
\date{\today}

\begin{document}

\maketitle
\thispagestyle{empty}
\newpage

\tableofcontents
\newpage

\section{Executive Overview}

This report provides a comprehensive analysis of the cybersecurity posture for \textbf{Radiant Life}, based on a review of organizational security controls, an external network scan, and a summary of known risks. The assessment was conducted on \today.

Overall, \textbf{Radiant Life} demonstrates a solid foundation in security practices, including the implementation of Multi-Factor Authentication (MFA) for email and computer access, and a consistent security awareness training program. The external network scan of the provided target IP address revealed no open ports, indicating a well-hardened perimeter at that specific point.

However, a critical security gap was identified: the absence of MFA for accessing sensitive data systems. This represents a significant risk, as it exposes the organization's most valuable data to potential compromise through stolen or weak credentials. This report details this finding and provides actionable recommendations to mitigate the identified risk and further strengthen the organization's security posture.

\section{Organizational Information}

The following information was provided for the assessment.

\begin{itemize}
    \item \textbf{Organization Name:} Radiant Life
    \item \textbf{Email Domain:} \texttt{RadiantLife.org}
    \item \textbf{Website Domain:} \seqsplit{\texttt{www.RadiantLife.org}}
    \item \textbf{External IP Address:} \texttt{13.18.247.146}
\end{itemize}

\section{Security Control Review}

A review of administrative and policy-based security controls was conducted via a questionnaire. The results are summarized below. A green checkmark (\textcolor{green}{\ding{51}}) indicates a positive control is in place, while a red 'X' (\textcolor{red}{\ding{55}}) highlights a potential security gap.

\begin{table}[h!]
\centering
\caption{Security Controls Questionnaire Results}
\begin{tabular}{p{0.7\linewidth} c}
\toprule
\textbf{Control Question} & \textbf{Response} \\
\midrule
Do you require MFA to access email? & \textcolor{green}{\ding{51}} \\
Do you require MFA to log into computers? & \textcolor{green}{\ding{51}} \\
Do you require MFA to access sensitive data systems? & \textcolor{red}{\ding{55}} \\
Does your organization have an employee acceptable use policy? & \textcolor{green}{\ding{51}} \\
Does your organization do security awareness training for new employees? & \textcolor{green}{\ding{51}} \\
Does your organization do security awareness training for all employees at least once per year? & \textcolor{green}{\ding{51}} \\
\bottomrule
\end{tabular}
\end{table}

\subsection*{Analysis}
The questionnaire reveals a strong commitment to security fundamentals. However, the lack of MFA on sensitive data systems is a critical weakness. While email and endpoint security are robust, the organization's "crown jewels" are not protected with the same level of assurance. This gap must be addressed with high priority.

\section{Technical Scan Results}

A network scan was performed using Nmap to identify open ports and services on the designated target.

\begin{itemize}
    \item \textbf{Target IP Address:} \texttt{192.168.1.100}
    \item \textbf{Scan Status:} Host is up.
\end{itemize}

\subsection*{Findings}
The scan concluded that there were \textbf{no open ports} on the target host. All 1000 scanned ports were reported as `closed`. This is a positive security finding, as it indicates that the host has a minimal attack surface from the scanner's perspective. This could be due to a properly configured firewall or a lack of network-facing services on the machine.

\section{Risk Assessment}

This section correlates findings from the security control review, technical scans, and pre-existing risk data. Based on the available inputs, one high-severity risk has been identified.

\begin{table}[h!]
\centering
\caption{Identified Risk Summary}
\begin{tabular}{p{0.1\linewidth} p{0.25\linewidth} p{0.45\linewidth} p{0.1\linewidth}}
\toprule
\textbf{ID} & \textbf{Risk Name} & \textbf{Description} & \textbf{Severity} \\
\midrule
RISK-001 & Lack of MFA on Sensitive Data Systems & Access to critical systems containing sensitive organizational or customer data is authenticated using only a single factor (e.g., username and password). This exposes the data to a high risk of unauthorized access via compromised credentials. & \textbf{High} \\
\bottomrule
\end{tabular}
\end{table}

\section{Recommendations}

The following recommendations are provided to mitigate the identified risks and improve the overall security posture of \textbf{Radiant Life}.

\subsection*{RISK-001: Implement MFA for Sensitive Systems (High Priority)}
It is strongly recommended that the organization immediately begin planning and implementing Multi-Factor Authentication (MFA) across all systems that store, process, or transmit sensitive data.

\begin{itemize}
    \item \textbf{Action:} Identify all sensitive data repositories and the applications used to access them.
    \item \textbf{Action:} Evaluate and deploy an enterprise-grade MFA solution that is compatible with these systems. Solutions can include authenticator apps (TOTP), hardware tokens, or biometrics.
    \item \textbf{Justification:} This action directly mitigates the risk of unauthorized access through credential theft, phishing, or password reuse. It is one of the most effective controls for protecting critical assets.
\end{itemize}

\subsection*{Maintain Network Hardening (Good Practice)}
The network scan results for the target IP were positive. This practice should be continued and expanded.

\begin{itemize}
    \item \textbf{Action:} Continue to enforce the principle of least privilege for network services, ensuring that only necessary ports are open to required networks.
    \item \textbf{Action:} Implement a regular, automated scanning schedule for all external-facing and critical internal IP addresses to promptly identify and remediate any unauthorized changes or newly opened ports.
    \item \textbf{Justification:} Proactive network monitoring ensures that the organization's attack surface remains minimal and that misconfigurations are detected quickly.
\end{itemize}

\end{document}
```