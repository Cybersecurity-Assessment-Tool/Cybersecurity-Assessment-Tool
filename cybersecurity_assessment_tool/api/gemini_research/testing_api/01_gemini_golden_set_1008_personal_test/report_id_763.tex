```latex
\documentclass[12pt]{article}

% Preamble: Required Packages
\usepackage[margin=1in]{geometry}
\usepackage{pifont} % For checkmarks and crosses
\usepackage{booktabs} % For professional tables
\usepackage{hyperref} % For clickable links
\usepackage{url}      % For URL formatting
\usepackage{seqsplit} % To split long strings in tt font
\usepackage{graphicx} % For potential logos
\usepackage{array}    % For table column formatting

% Document Metadata
\title{Cybersecurity Posture Assessment Report}
\author{Cybersecurity Analysis Division}
\date{\today}

% Hyperref Setup
\hypersetup{
    colorlinks=true,
    linkcolor=black,
    urlcolor=blue,
    pdftitle={Cybersecurity Posture Assessment Report},
    pdfauthor={Cybersecurity Analysis Division},
}

% Custom command for table cells
\newcolumntype{L}[1]{>{\raggedright\let\newline\\\arraybackslash\hspace{0pt}}m{#1}}
\newcolumntype{C}[1]{>{\centering\let\newline\\\arraybackslash\hspace{0pt}}m{#1}}

\begin{document}

\maketitle
\thispagestyle{empty}
\newpage

\tableofcontents
\thispagestyle{empty}
\newpage

\setcounter{page}{1}

% --- SECTION 1: EXECUTIVE SUMMARY ---
\section{Executive Summary}

This report provides a cybersecurity posture assessment for \textbf{Solaris Energy}, based on a combination of organizational data, a security controls questionnaire, and a network vulnerability scan.

The overall analysis indicates a mixed security posture. The organization demonstrates a commitment to security through established policies and training programs, and has implemented Multi-Factor Authentication (MFA) for key assets like computer logins and sensitive data systems. These are commendable foundational controls.

However, two significant risks were identified that require immediate attention:
\begin{itemize}
    \item \textbf{Critical Risk:} The lack of mandatory MFA for email access presents a substantial vulnerability. Email is a primary target for phishing and business email compromise (BEC) attacks, and its exposure could lead to account takeovers, data breaches, and financial loss.
    \item \textbf{Medium Risk:} The external network scan identified an exposed Secure Shell (SSH) service on a public-facing IPv6 address. While necessary for remote administration, an improperly configured or unpatched SSH service can be a gateway for unauthorized access.
\end{itemize}

This report details these findings and provides actionable recommendations to mitigate the identified risks and strengthen the organization's overall security posture.

% --- SECTION 2: ORGANIZATIONAL INFORMATION ---
\section{Organizational Information}

The following information was provided by the client and used as a baseline for this assessment.

\begin{table}[h!]
\centering
\begin{tabular}{@{}ll@{}}
\toprule
\textbf{Attribute} & \textbf{Value} \\
\midrule
Organization Name & \textbf{Solaris Energy} \\
Email Domain      & \texttt{SolarisEnergy.net} \\
Website Domain    & \url{www.SolarisEnergy.net} \\
External IP (IPv4) & \seqsplit{\texttt{203.216.15.87}} \\
Scanned IP (IPv6)  & \seqsplit{\texttt{2001:db8::1}} \\
\bottomrule
\end{tabular}
\caption{Client Organizational Details}
\label{tab:org_info}
\end{table}

% --- SECTION 3: SECURITY CONTROL REVIEW ---
\section{Security Control Review}

A review of the organization's security controls was conducted via a questionnaire. The responses indicate a good foundation but highlight a critical gap in email security. The green checkmark (\ding{51}) indicates a positive control, while the red X (\ding{55}) indicates a gap.

\begin{table}[h!]
\centering
\begin{tabular}{@{}L{9cm}C{1.5cm}L{3.5cm}@{}}
\toprule
\textbf{Control Question} & \textbf{Response} & \textbf{Analyst Note} \\
\midrule
Do you require MFA to access email? & \ding{55} & \textbf{Critical Gap.} A primary vector for account compromise. \\
\addlinespace
Do you require MFA to log into computers? & \ding{51} & Best practice implemented. \\
\addlinespace
Do you require MFA to access sensitive data systems? & \ding{51} & Best practice implemented. \\
\addlinespace
Does your organization have an employee acceptable use policy? & \ding{51} & Foundational policy in place. \\
\addlinespace
Does your organization do security awareness training for new employees? & \ding{51} & Good security onboarding. \\
\addlinespace
Does your organization do security awareness training for all employees at least once per year? & \ding{51} & Good security hygiene. \\
\bottomrule
\end{tabular}
\caption{Security Controls Questionnaire Analysis}
\label{tab:controls}
\end{table}

% --- SECTION 4: TECHNICAL SCAN RESULTS ---
\section{Technical Scan Results}

An external network scan was performed on the specified target to identify open ports and exposed services.

\begin{itemize}
    \item \textbf{Target IP Address:} \seqsplit{\texttt{2001:db8::1}}
    \item \textbf{Scan Tool:} Nmap
\end{itemize}

The scan revealed the following open port:

\begin{table}[h!]
\centering
\begin{tabular}{@{}llll@{}}
\toprule
\textbf{Port} & \textbf{State} & \textbf{Inferred Service} & \textbf{Notes} \\
\midrule
22/tcp & open & SSH (Secure Shell) & Remote administration port. Version information was not available in the provided scan data. Exposure should be reviewed. \\
\bottomrule
\end{tabular}
\caption{Open Ports Detected on \seqsplit{\texttt{2001:db8::1}}}
\label{tab:scan_results}
\end{table}

% --- SECTION 5: RISK ASSESSMENT ---
\section{Risk Assessment}

Based on the correlation of the security control review and technical scan results, the following risks have been identified.

\begin{table}[h!]
\centering
\begin{tabular}{@{}L{3cm}L{6cm}C{2cm}L{3cm}@{}}
\toprule
\textbf{Risk ID} & \textbf{Description} & \textbf{Severity} & \textbf{Affected Asset(s)} \\
\midrule
\textbf{RISK-001} & \textbf{Lack of MFA on Email Accounts.} User email accounts can be compromised via credential stuffing or phishing, leading to data breaches, financial fraud (BEC), and further internal compromise. & \textbf{Critical} & Corporate Email System \\
\addlinespace
\textbf{RISK-002} & \textbf{Exposed SSH Service.} The SSH port (22) is open on a public-facing server. This exposes the system to brute-force attacks and exploitation of potential vulnerabilities in the SSH service itself. & \textbf{Medium} & Server at \seqsplit{\texttt{2001:db8::1}} \\
\bottomrule
\end{tabular}
\caption{Identified Risk Summary}
\label{tab:risk_summary}
\end{table}

% --- SECTION 6: RECOMMENDATIONS ---
\section{Recommendations}

The following actions are recommended to mitigate the identified risks and improve the overall security posture of \textbf{Solaris Energy}.

\subsection{RISK-001: Lack of MFA on Email (Critical)}
\begin{itemize}
    \item \textbf{Immediate Action:} Procure and enforce MFA for all user access to the corporate email system (e.g., Microsoft 365, Google Workspace). This is the single most effective control to prevent unauthorized account access.
    \item \textbf{Policy Update:} Update the information security policy to mandate MFA for all externally accessible services that handle sensitive corporate data.
\end{itemize}

\subsection{RISK-002: Exposed SSH Service (Medium)}
\begin{itemize}
    \item \textbf{Review Business Need:} Validate the requirement for external SSH access to the server at \seqsplit{\texttt{2001:db8::1}}. If it is not required, the port should be blocked at the network firewall.
    \item \textbf{Implement Access Controls:} If external access is required, restrict SSH access to a whitelist of trusted source IP addresses (e.g., corporate offices, administrator VPNs).
    \item \textbf{Harden Configuration:} Ensure the SSH service is configured according to security best practices:
        \begin{itemize}
            \item Disable password-based authentication and require public key authentication.
            \item Disable root user login over SSH.
            \item Ensure the SSH service is kept up-to-date with the latest security patches.
        \end{itemize}
\end{itemize}

\end{document}
```