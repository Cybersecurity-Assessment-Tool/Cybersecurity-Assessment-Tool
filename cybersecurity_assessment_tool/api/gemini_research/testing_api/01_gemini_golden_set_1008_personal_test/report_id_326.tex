```latex
\documentclass[12pt]{article}

% Preamble: Required Packages
\usepackage[margin=1in]{geometry}
\usepackage{pifont} % For \ding
\usepackage{booktabs} % For professional tables (\toprule, \midrule, \bottomrule)
\usepackage[hidelinks]{hyperref} % For clickable links without boxes
\usepackage{url} % For URL formatting
\usepackage{seqsplit} % For splitting long strings in texttt
\usepackage{graphicx}
\usepackage{fancyhdr}
\usepackage[utf8]{inputenc}

% Document Information
\title{Cybersecurity Posture Assessment Report}
\author{Cybersecurity Analyst}
\date{\today}

% Header and Footer
\pagestyle{fancy}
\fancyhf{}
\fancyhead[L]{Silent Spring // Confidential}
\fancyfoot[C]{\thepage}

\begin{document}

\maketitle
\thispagestyle{empty}
\newpage

\tableofcontents
\newpage

% --- 1. Executive Summary ---
\section{Executive Summary}

This report provides a comprehensive analysis of the cybersecurity posture for \textbf{Silent Spring}, based on network scans, a security controls questionnaire, and a review of known risks. The assessment was conducted to identify vulnerabilities, policy gaps, and technical misconfigurations that could expose the organization to significant cyber threats.

The overall security posture is assessed as \textbf{CRITICAL}. This assessment is driven by several key findings:
\begin{itemize}
    \item \textbf{Critical FTP Vulnerability:} A public-facing FTP server (\texttt{10.0.0.15}) is running a dangerously outdated version of \texttt{vsftpd} (2.3.4), which contains a known remote code execution vulnerability (CVE-2011-2523). The service also permits anonymous login, allowing unauthorized access to files. This represents an immediate and severe threat to the network.
    \item \textbf{Significant Policy and Control Gaps:} The organization lacks fundamental security controls, including Multi-Factor Authentication (MFA) for sensitive systems, an employee acceptable use policy, and a consistent security awareness training program. These gaps dramatically increase the risk of security incidents originating from both internal and external threats.
    \item \textbf{Pre-existing Unmediated Risks:} The organization is aware of existing risks, such as outdated Windows 7 workstations, which compound the vulnerabilities discovered during this assessment.
\end{itemize}

Immediate remediation of the FTP vulnerability is strongly advised, followed by a systematic effort to implement the foundational security controls outlined in the recommendations section of this report.

% --- 2. Organizational Information ---
\section{Organizational Information}

The following information was provided for the assessment scope.

\begin{itemize}
    \item \textbf{Organization Name:} Silent Spring
    \item \textbf{Email Domain:} \texttt{SilentSpring.org}
    \item \textbf{Website Domain:} \url{www.SilentSpring.org}
    \item \textbf{External IP Address:} \texttt{188.144.60.76}
\end{itemize}

% --- 3. Security Control Review ---
\section{Security Control Review}

A review of organizational security controls was conducted via a questionnaire. The results indicate significant gaps in foundational security policies and procedures. "No" answers represent areas of high risk that require immediate attention.

\begin{table}[h!]
\centering
\caption{Security Controls Questionnaire Results}
\begin{tabular}{p{0.8\linewidth} c}
\toprule
\textbf{Control Question} & \textbf{Status} \\
\midrule
Do you require MFA to access email? & \ding{51} \\
Do you require MFA to log into computers? & \ding{51} \\
\textbf{Do you require MFA to access sensitive data systems?} & \textbf{\ding{55}} \\
\textbf{Does your organization have an employee acceptable use policy?} & \textbf{\ding{55}} \\
\textbf{Does your organization do security awareness training for new employees?} & \textbf{\ding{55}} \\
\textbf{Does your organization do security awareness training for all employees at least once per year?} & \textbf{\ding{55}} \\
\bottomrule
\end{tabular}
\end{table}

% --- 4. Technical Scan Results ---
\section{Technical Scan Results}

An external network scan was performed on the target system to identify open ports and exposed services.

\subsection{Host: \texttt{10.0.0.15}}
The scan revealed the following open port and service, which presents a critical vulnerability.

\begin{table}[h!]
\centering
\caption{Open Ports and Services for \texttt{10.0.0.15}}
\begin{tabular}{l l l l l}
\toprule
\textbf{Port} & \textbf{State} & \textbf{Service} & \textbf{Version} & \textbf{Notes} \\
\midrule
21/tcp & open & ftp & vsftpd 2.3.4 & \begin{tabular}[t]{@{}l@{}}\textbf{CRITICAL VULNERABILITY}\\ Anonymous FTP login allowed. \\ Version is vulnerable to a backdoor \\ (CVE-2011-2523).\end{tabular} \\
\bottomrule
\end{tabular}
\end{table}

\paragraph{Finding Detail:} The FTP service is running \texttt{vsftpd version 2.3.4}. This specific version was compromised in 2011, and a malicious backdoor was inserted into the source code. An attacker can gain a command shell on the server by sending a specific sequence of characters as the username. Combined with the allowance of anonymous logins, this vulnerability is trivial to exploit and could lead to a full system compromise.

% --- 5. Consolidated Risk Assessment ---
\section{Consolidated Risk Assessment}

The following table synthesizes findings from the technical scan, security control review, and pre-existing risk data into a prioritized list.

\begin{table}[h!]
\centering
\caption{Summary of Identified Risks}
\begin{tabular}{p{0.1\linewidth} p{0.4\linewidth} p{0.15\linewidth} p{0.25\linewidth}}
\toprule
\textbf{Risk ID} & \textbf{Description} & \textbf{Severity} & \textbf{Affected Elements} \\
\midrule
\textbf{TEC-001} & \textbf{Vulnerable FTP Service (\texttt{vsftpd 2.3.4}) with Anonymous Login.} & \textbf{Critical} & Network Server (\texttt{10.0.0.15}), Data Integrity \\
\addlinespace
POL-001 & Lack of MFA on sensitive data systems. & High & Sensitive Data, User Accounts \\
\addlinespace
POL-002 & No formal Employee Acceptable Use Policy (AUP). & High & All Employees, Legal/Compliance \\
\addlinespace
POL-003 & No security awareness training program for employees. & High & All Employees, Organizational Resilience \\
\addlinespace
RISK-001 & Outdated Windows 7 workstations are in use. & Medium & Workstations, Internal Network \\
\bottomrule
\end{tabular}
\end{table}

% --- 6. Recommendations ---
\section{Recommendations}

The following actions are recommended to mitigate the identified risks. They are prioritized based on severity and potential impact.

\subsection{Immediate Actions (Critical Priority)}
\begin{enumerate}
    \item \textbf{Remediate Vulnerable FTP Server (TEC-001):}
    \begin{itemize}
        \item \textbf{Option A (Preferred):} If the FTP service is not essential, disable and uninstall it from server \texttt{10.0.0.15} immediately.
        \item \textbf{Option B:} If the service is required, immediately upgrade \texttt{vsftpd} to the latest stable version. Disable anonymous FTP access unless there is an explicit and documented business requirement.
    \end{itemize}
\end{enumerate}

\subsection{High Priority Actions}
\begin{enumerate}
    \setcounter{enumi}{1}
    \item \textbf{Implement MFA for Sensitive Systems (POL-001):} Enforce Multi-Factor Authentication for all access to systems containing sensitive, confidential, or critical business data.
    \item \textbf{Develop and Implement an Acceptable Use Policy (POL-002):} Create a formal AUP that clearly defines the rules for using company technology and data. Ensure all employees read and acknowledge the policy.
    \item \textbf{Establish a Security Awareness Training Program (POL-003):}
    \begin{itemize}
        \item Implement mandatory security training for all new hires during their onboarding process.
        \item Conduct annual security awareness training for all staff, covering topics such as phishing, password security, and data handling.
    \end{itemize}
\end{enumerate}

\subsection{Medium Priority Actions}
\begin{enumerate}
    \setcounter{enumi}{4}
    \item \textbf{Plan Windows 7 Deprecation (RISK-001):} Develop a phased plan to upgrade or replace all workstations running Windows 7. This operating system is no longer supported by Microsoft and does not receive security updates, posing a significant risk to the internal network.
\end{enumerate}

\end{document}
```