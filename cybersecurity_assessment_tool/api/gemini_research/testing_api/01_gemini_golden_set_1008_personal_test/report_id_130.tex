```latex
\documentclass[12pt]{article}

% Preamble: Required Packages
\usepackage[margin=1in]{geometry}
\usepackage{pifont} % For checkmarks and crosses
\usepackage{booktabs} % For professional tables
\usepackage{hyperref} % For clickable links
\usepackage{url} % For formatting URLs
\usepackage{seqsplit} % For splitting long strings in tt font
\usepackage{graphicx}
\usepackage{xcolor}

% Document Metadata
\title{Cybersecurity Assessment Report}
\author{Cybersecurity Analyst}
\date{\today}

% Hyperref Setup
\hypersetup{
    colorlinks=true,
    linkcolor=blue,
    filecolor=magenta,      
    urlcolor=cyan,
    pdftitle={Cybersecurity Assessment Report},
    pdfpagemode=FullScreen,
}

\begin{document}

\maketitle
\thispagestyle{empty}
\newpage

\tableofcontents
\thispagestyle{empty}
\newpage

% --- 1. Executive Summary ---
\section{Executive Summary}
This report provides a comprehensive cybersecurity assessment for \textbf{Modern Myth}, based on a review of organizational security controls, an external network scan, and an analysis of pre-existing risks. The assessment was conducted to identify security gaps, evaluate the current risk posture, and provide actionable recommendations to enhance security.

The analysis revealed several critical and high-risk gaps in the organization's foundational security controls. Notably, the absence of mandatory Multi-Factor Authentication (MFA) for email and computer access represents a critical vulnerability, significantly increasing the risk of unauthorized access and account compromise. Furthermore, the lack of a formal Acceptable Use Policy (AUP) and missing security awareness training for new employees indicate deficiencies in security governance and culture.

The external network scan conducted on the target IP address \texttt{[Target IP]} did not detect any open ports or services. While this may suggest a well-configured perimeter firewall, it could also indicate that the target was unresponsive at the time of the scan. No pre-existing vulnerabilities were provided for this assessment.

The overall security posture is considered high-risk due to the identified policy and authentication control gaps. We strongly recommend prioritizing the implementation of MFA, developing key security policies, and strengthening the security awareness program to mitigate these risks effectively.

% --- 2. Organizational Information ---
\section{Organizational Information}
The following information was provided by the client and used as a baseline for this assessment.

\begin{tabular}{@{}ll}
\toprule
\textbf{Attribute} & \textbf{Value} \\
\midrule
Organization Name & \textbf{Modern Myth} \\
Email Domain & \texttt{ModernMyth.net} \\
Website Domain & \seqsplit{\url{www.ModernMyth.net}} \\
External IP Address & \texttt{174.75.83.64} \\
\bottomrule
\end{tabular}

% --- 3. Security Control Review ---
\section{Security Control Review}
A security questionnaire was completed to evaluate the implementation of essential administrative and technical controls. The responses are summarized below, highlighting areas that meet best practices and those that represent security gaps.

\begin{tabular}{@{}p{0.6\linewidth} p{0.15\linewidth} p{0.2\linewidth}@{}}
\toprule
\textbf{Control Question} & \textbf{Response} & \textbf{Assessment} \\
\midrule
Do you require MFA to access email? & \centering{\ding{55}} & \textcolor{red}{Critical Gap} \\
Do you require MFA to log into computers? & \centering{\ding{55}} & \textcolor{red}{Critical Gap} \\
Do you require MFA to access sensitive data systems? & \centering{\ding{51}} & Good Practice \\
Does your organization have an employee acceptable use policy? & \centering{\ding{55}} & \textcolor{orange}{High Risk} \\
Does your organization do security awareness training for new employees? & \centering{\ding{55}} & \textcolor{orange}{High Risk} \\
Does your organization do security awareness training for all employees at least once per year? & \centering{\ding{51}} & Good Practice \\
\bottomrule
\end{tabular}

% --- 4. Technical Scan Results ---
\section{Technical Scan Results}
An external network vulnerability scan was performed to identify open ports, running services, and potential exposures on the organization's public-facing infrastructure.

\begin{itemize}
    \item \textbf{Target IP Address:} \texttt{[Target IP]}
    \item \textbf{Scan Date:} No scan date was provided in the scan metadata.
\end{itemize}

\subsection{Scan Summary}
The network scan did not identify any open TCP or UDP ports on the target system. This result indicates one of the following possibilities:
\begin{itemize}
    \item The host was offline or not responsive to network probes at the time of the scan.
    \item A perimeter firewall is effectively configured to drop or reject all unsolicited incoming traffic, which is a strong security practice.
    \item The scan was inadvertently blocked by an Intrusion Prevention System (IPS).
\end{itemize}
No vulnerabilities could be identified based on these results. Further internal and authenticated scanning is recommended to gain a complete view of the technical security posture.

% --- 5. Consolidated Risk Assessment ---
\section{Consolidated Risk Assessment}
This section synthesizes findings from the security control review and technical scan. No pre-existing vulnerabilities were provided for this assessment. The following new risks have been identified:

\begin{tabular}{@{}p{0.1\linewidth} p{0.25\linewidth} p{0.5\linewidth} p{0.1\linewidth}@{}}
\toprule
\textbf{ID} & \textbf{Risk Name} & \textbf{Overview} & \textbf{Severity} \\
\midrule
RISK-001 & Lack of MFA for Email Access & Without MFA, email accounts are vulnerable to takeover via phishing or credential stuffing, exposing sensitive communications and providing a launchpad for internal attacks. & \textcolor{red}{Critical} \\
\addlinespace
RISK-002 & Lack of MFA for Endpoint Login & The absence of MFA on computers allows an attacker with valid credentials to gain direct access to the endpoint and potentially the internal network. & \textcolor{red}{Critical} \\
\addlinespace
RISK-003 & No Employee Acceptable Use Policy (AUP) & Without a formal AUP, there are no clear guidelines for employees on the acceptable use of company assets, data handling, and security responsibilities, increasing the risk of insider threat and policy violations. & \textcolor{orange}{High} \\
\addlinespace
RISK-004 & No Security Training for New Hires & New employees are not equipped with fundamental security knowledge from day one, making them more susceptible to social engineering attacks and unintentional security mistakes. & \textcolor{orange}{High} \\
\bottomrule
\end{tabular}

% --- 6. Recommendations ---
\section{Recommendations}
The following actions are recommended to mitigate the identified risks and strengthen the overall security posture of \textbf{Modern Myth}.

\begin{enumerate}
    \item \textbf{Implement Mandatory Multi-Factor Authentication (Critical):}
    \begin{itemize}
        \item \textbf{Action:} Enforce MFA for all user accounts across all critical systems, prioritizing email (Office 365 / Google Workspace) and endpoint logins (Windows / macOS).
        \item \textbf{Justification:} This is the single most effective control to prevent unauthorized access resulting from compromised credentials. It directly mitigates RISK-001 and RISK-002.
    \end{itemize}
    \vspace{0.5cm}
    \item \textbf{Develop and Implement an Acceptable Use Policy (High):}
    \begin{itemize}
        \item \textbf{Action:} Create a formal AUP that clearly defines the rules for using company networks, devices, and data. Require all current and new employees to read and formally acknowledge the policy.
        \item \textbf{Justification:} An AUP establishes a baseline for secure employee behavior, reduces legal liability, and mitigates RISK-003.
    \end{itemize}
    \vspace{0.5cm}
    \item \textbf{Integrate Security Training into Employee Onboarding (High):}
    \begin{itemize}
        \item \textbf{Action:} Develop a mandatory security awareness training module for all new employees to be completed during their first week of onboarding. This should cover topics like phishing, password security, and the new AUP.
        \item \textbf{Justification:} This ensures that security is a day-one priority and provides a foundational layer of defense against common cyber threats, mitigating RISK-004.
    \end{itemize}
    \vspace{0.5cm}
    \item \textbf{Conduct Comprehensive Internal Scans (Recommended):}
    \begin{itemize}
        \item \textbf{Action:} Perform authenticated vulnerability scans on internal network segments and key servers.
        \item \textbf{Justification:} Since the external scan was inconclusive, an internal scan will provide a much more accurate and detailed view of technical vulnerabilities that exist behind the firewall.
    \end{itemize}
\end{enumerate}

\end{document}
```