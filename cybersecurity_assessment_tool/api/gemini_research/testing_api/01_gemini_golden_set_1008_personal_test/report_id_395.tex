```latex
\documentclass[12pt]{article}

% Required Packages
\usepackage[margin=1in]{geometry}
\usepackage{pifont} % For checkmarks and crosses
\usepackage{booktabs} % For professional tables
\usepackage{hyperref} % For clickable links
\usepackage{url} % For URL formatting
\usepackage{seqsplit} % For splitting long strings
\usepackage{graphicx}
\usepackage[table]{xcolor}
\usepackage{lastpage}
\usepackage{fancyhdr}

% --- Document Setup ---
\hypersetup{
    colorlinks=true,
    linkcolor=blue,
    filecolor=magenta,      
    urlcolor=cyan,
    pdftitle={Cybersecurity Assessment Report},
    pdfpagemode=FullScreen,
}

% Define colors for severity levels
\definecolor{criticalred}{HTML}{D10000}
\definecolor{highorange}{HTML}{E25F00}
\definecolor{mediumyellow}{HTML}{F2C000}
\definecolor{lowblue}{HTML}{0073E6}

% Header and Footer
\pagestyle{fancy}
\fancyhf{}
\fancyhead[L]{Cybersecurity Assessment Report}
\fancyhead[R]{Terraform Global}
\fancyfoot[C]{\thepage\ of \pageref{LastPage}}
\renewcommand{\headrulewidth}{0.4pt}
\renewcommand{\footrulewidth}{0.4pt}

% --- Document Start ---
\begin{document}

% --- Title Page ---
\begin{titlepage}
    \centering
    \vspace*{1cm}
    
    \Huge
    \textbf{Cybersecurity Assessment Report}
    
    \vspace{1.5cm}
    
    \Large
    Prepared for:
    
    \vspace{0.5cm}
    
    \textbf{Terraform Global}
    
    \vspace{2cm}
    
    \large
    Report Date: \today
    
    \vfill
    
    \large
    \textit{This report contains sensitive and confidential information. Distribution is restricted to authorized personnel only.}
    
\end{titlepage}

\tableofcontents
\newpage

% --- Section 1: Executive Summary ---
\section{Executive Summary}
This report details the findings of a cybersecurity assessment conducted for \textbf{Terraform Global}. The assessment combined a review of organizational security controls, an external network scan, and an analysis of pre-existing risk data.

The analysis revealed several \textbf{critical} security gaps that significantly increase the organization's risk of compromise. The most pressing issues are the lack of Multi-Factor Authentication (MFA) for email and sensitive data systems, and the complete absence of a security awareness training program. These deficiencies create a high susceptibility to phishing, business email compromise, and unauthorized data access.

Furthermore, a technical scan confirmed a pre-existing critical risk related to a service exposed on the localhost interface (\texttt{127.0.0.1}). While not directly accessible from the internet, this misconfiguration could be exploited by an attacker who has already gained an initial foothold on the system.

Immediate remediation is required to address these findings. Recommendations focus on implementing foundational security controls, such as MFA and employee training, to build a more resilient security posture.

% --- Section 2: Organizational Information ---
\section{Organizational Information}
The following information was provided for the assessment.
\begin{center}
\begin{tabular}{ll}
\toprule
\textbf{Attribute} & \textbf{Value} \\
\midrule
Organization Name & Terraform Global \\
Email Domain & \texttt{TerraformGlobal.com} \\
Website Domain & \url{www.TerraformGlobal.com} \\
External IP Address & \texttt{13.234.152.132} \\
\bottomrule
\end{tabular}
\end{center}

% --- Section 3: Security Control Review ---
\section{Security Control Review}
A review of administrative and organizational security controls was conducted via a questionnaire. The results highlight significant gaps in the current security posture. A "No" answer indicates a deviation from security best practices and a potential risk.

\begin{center}
\rowcolors{2}{gray!10}{white}
\begin{tabular}{p{0.55\linewidth} c p{0.25\linewidth}}
\toprule
\textbf{Control Question} & \textbf{Response} & \textbf{Analyst Note} \\
\midrule
Do you require MFA to access email? & \ding{55} & \textcolor{criticalred}{\textbf{Critical Gap.}} Email is a primary target. Lack of MFA exposes the organization to account takeover and phishing. \\
\addlinespace
Do you require MFA to log into computers? & \ding{51} & Good practice for endpoint security. \\
\addlinespace
Do you require MFA to access sensitive data systems? & \ding{55} & \textcolor{criticalred}{\textbf{Critical Gap.}} Failure to protect sensitive data with MFA presents a severe risk of data breach. \\
\addlinespace
Does your organization have an employee acceptable use policy? & \ding{55} & \textcolor{highorange}{\textbf{High Risk.}} Lack of a formal policy creates ambiguity regarding the proper use of company assets. \\
\addlinespace
Does your organization do security awareness training for new employees? & \ding{55} & \textcolor{criticalred}{\textbf{Critical Gap.}} Employees are the first line of defense. This gap leaves the organization vulnerable to social engineering. \\
\addlinespace
Does your organization do security awareness training for all employees at least once per year? & \ding{55} & \textcolor{criticalred}{\textbf{Critical Gap.}} Ongoing training is essential to maintain security consciousness. \\
\bottomrule
\end{tabular}
\end{center}

% --- Section 4: Technical Scan Results ---
\section{Technical Scan Results}
A network scan was performed to identify open ports and exposed services on the specified target.

\subsection{Scan Details}
\begin{itemize}
    \item \textbf{Target IP:} \texttt{127.0.0.1}
    \item \textbf{Scan Date:} Scan data provided on \today
    \item \textbf{Scanner Used:} Nmap
\end{itemize}

\subsection{Open Ports}
The scan identified the following open port.
\begin{center}
\begin{tabular}{llll}
\toprule
\textbf{Port} & \textbf{State} & \textbf{Service (Inferred)} & \textbf{Notes} \\
\midrule
22/tcp & open & SSH & The scan confirmed the port is open. No detailed service version information was available in the provided data. \\
\bottomrule
\end{tabular}
\end{center}

\subsection{Technical Analysis}
The scan confirms that a service, presumed to be SSH, is listening on port 22 on the localhost interface (\texttt{127.0.0.1}). This finding directly correlates with and validates the pre-existing risk "Localhost Exposed" identified in \textbf{Input\_3\_Current\_Risks\_JSON}. While not directly exposed to the internet, this service could be leveraged by an attacker for privilege escalation or lateral movement after gaining initial access to the machine.

% --- Section 5: Consolidated Risk Assessment ---
\section{Consolidated Risk Assessment}
The following table synthesizes findings from the security control review, technical scan, and pre-existing risk data into a consolidated list of identified risks.

\begin{center}
\rowcolors{2}{gray!10}{white}
\begin{tabular}{p{0.2\linewidth} p{0.5\linewidth} l}
\toprule
\textbf{Risk Name} & \textbf{Description} & \textbf{Severity} \\
\midrule
\textbf{Localhost Exposed} & The network scan confirmed a service (SSH on port 22) is exposed on the localhost interface, validating a known critical risk. & \colorbox{criticalred}{\color{white}\textbf{Critical (10.0)}} \\
\addlinespace
\textbf{Lack of MFA on Critical Systems} & Multi-Factor Authentication is not enforced for accessing email or sensitive data systems, exposing them to unauthorized access and account takeover. & \colorbox{criticalred}{\color{white}\textbf{Critical}} \\
\addlinespace
\textbf{No Security Awareness Program} & The organization does not conduct security awareness training for new or existing employees, making them highly vulnerable to social engineering attacks like phishing. & \colorbox{criticalred}{\color{white}\textbf{Critical}} \\
\addlinespace
\textbf{No Acceptable Use Policy (AUP)} & The absence of an AUP leads to a lack of clear guidelines for employees on the secure and acceptable use of corporate resources. & \colorbox{highorange}{\color{white}\textbf{High}} \\
\bottomrule
\end{tabular}
\end{center}

% --- Section 6: Recommendations ---
\section{Recommendations}
The following actions are recommended to mitigate the identified risks and improve the overall security posture of \textbf{Terraform Global}. Recommendations are prioritized based on severity.

\subsection{Critical Priority}
\begin{enumerate}
    \item \textbf{Implement and Enforce MFA:}
    \begin{itemize}
        \item \textbf{Action:} Immediately enable and mandate the use of MFA for all user accounts on email systems (e.g., Office 365, G Suite) and all systems identified as containing sensitive data.
        \item \textbf{Justification:} This is the single most effective control to prevent account takeover and unauthorized access resulting from compromised credentials.
    \end{itemize}

    \item \textbf{Establish a Security Awareness Program:}
    \begin{itemize}
        \item \textbf{Action:} Procure and deploy a security awareness training platform. Enroll all current employees immediately and integrate the training into the new-hire onboarding process. Conduct annual refresher training and periodic phishing simulations.
        \item \textbf{Justification:} An educated workforce is a critical layer of defense against phishing and other social engineering attacks.
    \end{itemize}
    
    \item \textbf{Remediate Localhost Exposure:}
    \begin{itemize}
        \item \textbf{Action:} Investigate the service running on \texttt{127.0.0.1:22}. Determine its business purpose. If the service is not required, disable it. If it is required, ensure it is securely configured according to best practices and accessible only by authorized processes.
        \item \textbf{Justification:} Eliminates a potential pivot point for an attacker who has already gained a foothold on the system.
    \end{itemize}
\end{enumerate}

\subsection{High Priority}
\begin{enumerate}
    \setcounter{enumi}{3} % Continue numbering
    \item \textbf{Develop and Implement an Acceptable Use Policy (AUP):}
    \begin{itemize}
        \item \textbf{Action:} Draft a formal AUP that clearly defines the rules for using company IT assets, data, and internet access. The policy should be reviewed by management, communicated to all employees, and acknowledged via signature.
        \item \textbf{Justification:} An AUP establishes a baseline for secure behavior, reduces organizational liability, and provides a framework for enforcing security standards.
    \end{itemize}
\end{enumerate}

\end{document}
```