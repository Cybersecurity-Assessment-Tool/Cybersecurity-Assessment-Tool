```latex
\documentclass[12pt]{article}

% Preamble: Required Packages
\usepackage[margin=1in]{geometry}
\usepackage{pifont} % For checkmarks and crosses
\usepackage{booktabs} % For professional tables
\usepackage{hyperref} % For clickable links
\usepackage{url}      % For formatting URLs
\usepackage{seqsplit} % For splitting long strings in texttt
\usepackage{graphicx} % For potential logos (placeholder)
\usepackage{xcolor}   % For colors

% Document Information
\title{Cybersecurity Posture Assessment Report \\ \large For: Silent Spring}
\author{Cybersecurity Analysis Division}
\date{\today}

\begin{document}

\maketitle
\thispagestyle{empty}
\newpage

\tableofcontents
\newpage

% --- 1. Executive Overview ---
\section{Executive Overview}
This report provides a comprehensive cybersecurity assessment for \textbf{Silent Spring}, conducted on \today. The analysis synthesizes data from an external network scan, a review of organizational security controls, and a list of pre-existing risks.

The assessment reveals several critical and high-risk vulnerabilities that require immediate attention. The most significant findings include:
\begin{itemize}
    \item \textbf{Systemic Remote Desktop Protocol (RDP) Exposure:} The network scan identified an open RDP port on a new asset (\texttt{10.10.10.51}). When correlated with pre-existing risks, this points to a systemic pattern of exposing critical administrative services, significantly increasing the risk of unauthorized access and ransomware attacks.
    \item \textbf{Lack of Multi-Factor Authentication (MFA) on Email:} The organization does not require MFA for email access. This represents a critical security gap, as email accounts are a primary target for phishing and business email compromise (BEC) attacks. A single compromised password could lead to a significant data breach.
    \item \textbf{Inadequate Security Awareness Training:} While new employees receive training, there is no mandatory annual security training for all staff. This allows security knowledge to become outdated, making the organization more susceptible to evolving social engineering tactics.
\end{itemize}

The combination of these vulnerabilities places the organization at a \textbf{High} risk of a significant security incident. This report outlines detailed findings and provides actionable recommendations to mitigate these risks and improve the overall security posture.

% --- 2. Organizational Information ---
\section{Organizational Information}
The following information was provided for the assessment.

\begin{tabular}{@{}ll}
    \toprule
    \textbf{Attribute} & \textbf{Value} \\
    \midrule
    Organization Name & \textbf{Silent Spring} \\
    Email Domain      & \texttt{SilentSpring.com} \\
    Website Domain    & \url{www.SilentSpring.com} \\
    External IP Address & \texttt{189.67.99.249} \\
    \bottomrule
\end{tabular}

% --- 3. Security Control Review ---
\section{Security Control Review}
A review of the organization's security controls was conducted via a questionnaire. The responses highlight critical gaps in the current security framework. A "Yes" response (\ding{51}) indicates a control is in place, while a "No" response (\ding{55}) indicates a control is absent and represents a potential risk.

\begin{table}[h!]
\centering
\begin{tabular}{@{}lc}
    \toprule
    \textbf{Control Question} & \textbf{Response} \\
    \midrule
    Do you require MFA to access email? & \ding{55} \\
    Do you require MFA to log into computers? & \ding{51} \\
    Do you require MFA to access sensitive data systems? & \ding{51} \\
    Does your organization have an employee acceptable use policy? & \ding{51} \\
    Does your organization do security awareness training for new employees? & \ding{51} \\
    Does your organization do security awareness training for all employees at least once per year? & \ding{55} \\
    \bottomrule
\end{tabular}
\caption{Organizational Security Controls Questionnaire.}
\end{table}

\subsection*{Analysis of Control Gaps}
\begin{itemize}
    \item \textbf{No MFA for Email:} This is a critical deficiency. Email is the gateway to an organization's data and is frequently targeted by attackers. Without MFA, a compromised password is all an attacker needs to gain access, read sensitive communications, and launch further attacks.
    \item \textbf{No Annual Security Training:} The threat landscape evolves rapidly. Failing to provide regular, updated training for all employees leaves the organization vulnerable to modern phishing and social engineering tactics.
\end{itemize}

% --- 4. Technical Scan Results ---
\section{Technical Scan Results}
An external network scan was performed to identify exposed services on the organization's perimeter.

\begin{itemize}
    \item \textbf{Target IP Address:} \texttt{10.10.10.51}
    \item \textbf{Scan Date:} Scan data processed on \today.
\end{itemize}

The following open port was discovered:

\begin{table}[h!]
\centering
\begin{tabular}{@{}llll@{}}
    \toprule
    \textbf{Port} & \textbf{State} & \textbf{Service Name} & \textbf{Description} \\
    \midrule
    3389/tcp & open & \texttt{ms-wbt-server} & Microsoft Remote Desktop Protocol (RDP) \\
    \bottomrule
\end{tabular}
\caption{Open Ports Detected on \texttt{10.10.10.51}.}
\end{table}

\subsection*{Analysis of Technical Findings}
The discovery of an open RDP port is a significant finding. RDP is a common vector for network intrusion, used by attackers for brute-force password attacks and exploitation of vulnerabilities (e.g., BlueKeep). This finding, when correlated with the pre-existing risk of RDP exposure on another host (\texttt{10.10.10.50}), indicates a systemic lack of network hardening and access control.

% --- 5. Consolidated Risk Assessment ---
\section{Consolidated Risk Assessment}
The following table summarizes and prioritizes the identified risks by combining the security control gaps, technical findings, and pre-existing vulnerabilities.

\begin{table}[h!]
\centering
\resizebox{\textwidth}{!}{%
\begin{tabular}{@{}lllll@{}}
    \toprule
    \textbf{ID} & \textbf{Risk Name} & \textbf{Description} & \textbf{Affected Asset(s)} & \textbf{Severity} \\
    \midrule
    RISK-001 & Systemic RDP Exposure & RDP is exposed on multiple internal servers. & \texttt{10.10.10.50}, \texttt{10.10.10.51} & \textbf{Critical} \\
    RISK-002 & No MFA on Email & Lack of MFA on email accounts allows for easy takeover. & All email accounts & \textbf{Critical} \\
    RISK-003 & Inadequate Security Training & No annual security training for all employees. & All employees & \textbf{High} \\
    \bottomrule
\end{tabular}
}
\caption{Summary of Identified Cybersecurity Risks.}
\end{table}

% --- 6. Recommendations ---
\section{Recommendations}
The following actions are recommended to mitigate the identified risks and strengthen the organization's security posture.

\subsection{RISK-001: Systemic RDP Exposure (Critical)}
\begin{itemize}
    \item \textbf{Immediate Action:} Implement firewall rules to block all inbound traffic to TCP port 3389 on all public-facing assets, including \texttt{10.10.10.50} and \texttt{10.10.10.51}. This port should not be accessible from the public internet.
    \item \textbf{Long-Term Strategy:} For necessary remote administration, implement a secure access solution such as a Virtual Private Network (VPN) or a Zero Trust Network Access (ZTNA) gateway. This ensures that all remote access is authenticated and encrypted.
\end{itemize}

\subsection{RISK-002: No MFA on Email (Critical)}
\begin{itemize}
    \item \textbf{Immediate Action:} Enforce MFA for all user accounts on the \texttt{SilentSpring.com} email domain. Prioritize enabling it for administrative, executive, and finance-related accounts within 24-48 hours.
    \item \textbf{Long-Term Strategy:} Integrate MFA enrollment into the standard employee onboarding process. Ensure that all new services and applications procured by the organization support and enforce MFA.
\end{itemize}

\subsection{RISK-003: Inadequate Security Training (High)}
\begin{itemize}
    \item \textbf{Immediate Action:} Schedule and conduct a mandatory security awareness training session for all current employees within the next quarter. The training should cover modern threats such as phishing, ransomware, and business email compromise.
    \item \textbf{Long-Term Strategy:} Establish a formal, ongoing security awareness program that includes annual mandatory training, regular phishing simulations, and periodic security newsletters or updates.
\end{itemize}

\end{document}
```