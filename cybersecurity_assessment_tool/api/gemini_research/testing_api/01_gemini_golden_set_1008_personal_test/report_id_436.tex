```latex
\documentclass[12pt, a4paper]{article}

% Preamble: Required Packages
\usepackage[margin=1in]{geometry}
\usepackage{pifont} % For checkmarks and crosses
\usepackage{booktabs} % For professional tables
\usepackage{hyperref} % For clickable links
\usepackage{url} % For URL formatting
\usepackage{seqsplit} % For splitting long strings in tt font
\usepackage{graphicx}
\usepackage[table]{xcolor}
\usepackage{fancyhdr}

% --- Document Setup ---
\definecolor{tablehead}{gray}{0.9}
\definecolor{critical}{HTML}{990000}
\definecolor{high}{HTML}{D14900}
\definecolor{medium}{HTML}{E09100}

\hypersetup{
    colorlinks=true,
    linkcolor=blue,
    filecolor=magenta,      
    urlcolor=cyan,
    pdftitle={Cybersecurity Posture Report},
    pdfpagemode=FullScreen,
}

\pagestyle{fancy}
\fancyhf{}
\fancyhead[L]{Cybersecurity Posture Report}
\fancyhead[R]{Skyward Bound}
\fancyfoot[C]{\thepage}

% --- Document Body ---
\begin{document}

\title{
    \vspace{2cm}
    \textbf{Cybersecurity Posture Report} \\
    \large \textit{Analysis of Organizational and Technical Controls} \\
    \vspace{1.5cm}
    \includegraphics[width=0.4\textwidth]{example-image-a} \\ % Placeholder for company logo
    \vspace{1cm}
    \textbf{Prepared for: Skyward Bound}
}

\author{Cybersecurity Analyst Group}
\date{\today}

\maketitle
\thispagestyle{empty}
\newpage

\tableofcontents
\newpage

% --- Executive Summary ---
\section{Executive Summary}

This report provides a comprehensive analysis of the cybersecurity posture for \textbf{Skyward Bound}, based on a review of organizational security controls, a network scan of a key internal asset, and a review of pre-existing risks.

The assessment revealed a mixed security posture. On the technical front, the network scan of the target host (\texttt{192.168.1.100}) showed a strong configuration, with no open ports detected. This indicates effective firewalling or host hardening for this specific asset.

However, significant and critical gaps were identified in the organizational security controls. The most severe finding is the lack of Multi-Factor Authentication (MFA) for email access, which exposes the organization to a high risk of Business Email Compromise (BEC), phishing, and account takeover. Furthermore, the absence of a formal security awareness training program for both new and existing employees constitutes a high-risk vulnerability, leaving the organization susceptible to social engineering and human error.

Immediate remediation should focus on implementing MFA for email and establishing a comprehensive security awareness training program to mitigate these critical human-factor risks.

% --- Organizational Information ---
\section{Organizational Information}

The following information was provided for the assessment.

\begin{table}[h!]
\centering
\rowcolors{2}{gray!10}{white}
\begin{tabular}{ll}
\toprule
\rowcolor{tablehead}
\textbf{Attribute} & \textbf{Value} \\
\midrule
Organization Name & \textbf{Skyward Bound} \\
Email Domain & \texttt{SkywardBound.net} \\
Website Domain & \seqsplit{\url{www.SkywardBound.net}} \\
External IP Address & \texttt{102.165.183.121} \\
\bottomrule
\end{tabular}
\caption{Client Organizational Data}
\label{tab:org_data}
\end{table}

% --- Security Control Review ---
\section{Security Control Review}

A security questionnaire was completed to evaluate the implementation of fundamental security controls. The results are summarized below. Answers marked with a cross (\ding{55}) indicate significant gaps in the security framework.

\begin{table}[h!]
\centering
\rowcolors{2}{gray!10}{white}
\begin{tabular}{p{0.8\linewidth} c}
\toprule
\rowcolor{tablehead}
\textbf{Control Question} & \textbf{Status} \\
\midrule
Do you require MFA to access email? & \textcolor{red}{\ding{55}} \\
Do you require MFA to log into computers? & \textcolor{green}{\ding{51}} \\
Do you require MFA to access sensitive data systems? & \textcolor{green}{\ding{51}} \\
Does your organization have an employee acceptable use policy? & \textcolor{green}{\ding{51}} \\
Does your organization do security awareness training for new employees? & \textcolor{red}{\ding{55}} \\
Does your organization do security awareness training for all employees at least once per year? & \textcolor{red}{\ding{55}} \\
\bottomrule
\end{tabular}
\caption{Security Controls Questionnaire Results}
\label{tab:controls}
\end{table}

\subsection*{Analysis of Control Gaps}
\begin{itemize}
    \item \textbf{No MFA for Email (Critical Risk):} Email is a primary target for attackers. Without MFA, a compromised password is all that is needed for an attacker to gain access to sensitive communications, launch internal phishing campaigns, or conduct Business Email Compromise (BEC) attacks.
    \item \textbf{No Security Awareness Training (High Risk):} The lack of a formal training program for new and existing employees significantly increases the organization's vulnerability to phishing, malware, and social engineering attacks. Employees are the first line of defense, and without proper training, they are unprepared to identify and respond to threats.
\end{itemize}

% --- Technical Scan Results ---
\section{Technical Scan Results}

A network port scan was conducted to identify exposed services on the specified target system.

\begin{table}[h!]
\centering
\rowcolors{2}{gray!10}{white}
\begin{tabular}{ll}
\toprule
\rowcolor{tablehead}
\textbf{Scan Parameter} & \textbf{Value} \\
\midrule
Target IP Address & \texttt{192.168.1.100} \\
Scan Date & \today \\
Host Status & Up \\
\midrule
\multicolumn{2}{c}{\textbf{Finding}} \\
\midrule
\multicolumn{2}{p{0.8\linewidth}}{The scan completed successfully and found \textbf{zero open ports}. All 1000 scanned ports were reported as 'closed'. This is a positive security finding, indicating that the host is not exposing any network services and is likely well-protected by a firewall.} \\
\bottomrule
\end{tabular}
\caption{Nmap Scan Summary}
\label{tab:nmap_results}
\end{table}

% --- Consolidated Risk Assessment ---
\section{Consolidated Risk Assessment}

This section synthesizes findings from the security control review, technical scans, and pre-existing risk data into a prioritized list.
\vspace{0.5cm}

\begin{tabular}{p{0.2\linewidth} p{0.6\linewidth} p{0.15\linewidth}}
\toprule
\rowcolor{tablehead}
\textbf{Risk Name} & \textbf{Overview} & \textbf{Severity} \\
\midrule
\rowcolor{critical!20}
No MFA on Email & The absence of Multi-Factor Authentication on the primary email system (\texttt{SkywardBound.net}) leaves all accounts vulnerable to takeover via password compromise. This can lead to data breaches, financial loss, and reputational damage. & \textcolor{critical}{\textbf{Critical}} \\
\addlinespace[5pt]
\rowcolor{high!20}
Lack of Security Awareness Training & Without a formal and recurring training program, employees are ill-equipped to recognize and defend against common cyber threats like phishing and social engineering, making them a vulnerable entry point for attackers. & \textcolor{high}{\textbf{High}} \\
\bottomrule
\end{tabular}

% --- Recommendations ---
\section{Recommendations}

Based on the consolidated risk assessment, the following actions are recommended to improve the cybersecurity posture of \textbf{Skyward Bound}. Recommendations are prioritized by severity.

\subsection{Priority 1: Remediate Critical Risks}
\begin{description}
    \item[Implement MFA for Email:]
    \begin{itemize}
        \item \textbf{Action:} Immediately enable and enforce MFA for all user accounts on the \texttt{SkywardBound.net} email platform.
        \item \textbf{Justification:} This is the single most effective control to prevent unauthorized account access and mitigate the risk of Business Email Compromise.
        \item \textbf{Timeline:} Within 7 days.
    \end{itemize}
\end{description}

\subsection{Priority 2: Remediate High Risks}
\begin{description}
    \item[Establish Security Awareness Training Program:]
    \begin{itemize}
        \item \textbf{Action:} Procure and implement a security awareness training solution. The program must include:
        \begin{enumerate}
            \item Mandatory training for all new employees during onboarding.
            \item Mandatory annual refresher training for all staff.
            \item Regular simulated phishing campaigns to test and reinforce learning.
        \end{enumerate}
        \item \textbf{Justification:} A well-trained workforce provides a resilient human firewall, drastically reducing the likelihood of successful phishing and social engineering attacks.
        \item \textbf{Timeline:} Within 60 days.
    \end{itemize}
\end{description}

\end{document}
```