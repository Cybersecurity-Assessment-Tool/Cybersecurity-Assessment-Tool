```latex
\documentclass[12pt]{article}

% Preamble: Required Packages
\usepackage[a4paper, margin=1in]{geometry}
\usepackage{pifont} % For checkmarks and crosses
\usepackage{booktabs} % For professional tables
\usepackage{hyperref} % For clickable links
\usepackage{url}      % For URL formatting
\usepackage{seqsplit} % For splitting long strings like URLs or IPs
\usepackage[utf8]{inputenc}
\usepackage{graphicx}

% Document Metadata
\title{Cybersecurity Assessment Report \\ \large For: Vanguard Heritage}
\author{Cybersecurity Analyst}
\date{\today}

\begin{document}

\maketitle
\thispagestyle{empty}
\newpage

\tableofcontents
\newpage

% --- 1. Executive Summary ---
\section{Executive Summary}

This report provides a comprehensive cybersecurity assessment for Vanguard Heritage, based on an analysis of network scan data, organizational security controls, and pre-existing risk information. The assessment was conducted on \today.

The analysis reveals several critical security deficiencies that expose the organization to significant risk. The most pressing issues are the complete absence of Multi-Factor Authentication (MFA) across all key systems—including email, user computers, and sensitive data repositories. This gap dramatically increases the risk of unauthorized access and account compromise.

Furthermore, while foundational security practices like an acceptable use policy and new-hire training are in place, the lack of mandatory annual security awareness training for all staff leaves the organization vulnerable to evolving social engineering threats.

Technical analysis confirmed a pre-existing critical risk, "Localhost Exposed," by identifying an open Secure Shell (SSH) port on the local loopback interface (\texttt{127.0.0.1}). This configuration, rated with the highest possible severity score (CVSS 10.0), requires immediate remediation to prevent potential exploitation.

In summary, the current security posture of Vanguard Heritage is considered poor. Immediate and decisive action is required to address the identified critical vulnerabilities to protect organizational assets and data.

% --- 2. Organizational Information ---
\section{Organizational Information}

The following details were provided for the assessment. This information helps establish the context for the technical and procedural findings.

\begin{tabular}{@{}ll}
\toprule
\textbf{Attribute} & \textbf{Value} \\
\midrule
Organization Name & Vanguard Heritage \\
Email Domain & \seqsplit{\texttt{VanguardHeritage.org}} \\
Website Domain & \url{www.VanguardHeritage.org} \\
External IP Address & \seqsplit{\texttt{86.116.11.173}} \\
\bottomrule
\end{tabular}

% --- 3. Security Control Review ---
\section{Security Control Review}

A review of the organization's security controls was conducted via a questionnaire. The results below highlight the current state of procedural and administrative safeguards. Answers marked with a cross (\ding{55}) indicate significant gaps in the security framework.

\begin{table}[h!]
\centering
\begin{tabular}{@{}p{0.7\linewidth}c}
\toprule
\textbf{Security Control Question} & \textbf{Response} \\
\midrule
Does your organization have an employee acceptable use policy? & \ding{51} \\
Does your organization do security awareness training for new employees? & \ding{51} \\
\addlinespace
Do you require MFA to access email? & \textbf{\color{red}\ding{55}} \\
Do you require MFA to log into computers? & \textbf{\color{red}\ding{55}} \\
Do you require MFA to access sensitive data systems? & \textbf{\color{red}\ding{55}} \\
\addlinespace
Does your organization do security awareness training for all employees at least once per year? & \textbf{\color{red}\ding{55}} \\
\bottomrule
\end{tabular}
\caption{Organizational Security Controls Questionnaire Results.}
\end{table}

\subsection*{Analysis of Controls}
The questionnaire reveals critical deficiencies in access control management. The lack of Multi-Factor Authentication (MFA) for email, computer logins, and sensitive data systems is a severe weakness. A compromised password would be sufficient for an attacker to gain access to these critical assets. Additionally, the absence of annual security awareness training for all employees heightens the risk of successful phishing and other social engineering attacks, as employee knowledge is not kept current with emerging threats.

% --- 4. Technical Scan Results ---
\section{Technical Scan Results}

A network scan was performed to identify open ports and services on the target system. This scan provides a technical snapshot of the system's external-facing posture.

\begin{itemize}
    \item \textbf{Target IP Address:} \texttt{127.0.0.1}
    \item \textbf{Scan Date:} Scan data provided on \today.
\end{itemize}

\begin{table}[h!]
\centering
\begin{tabular}{@{}llll}
\toprule
\textbf{Port} & \textbf{State} & \textbf{Service (Presumed)} & \textbf{Product / Version} \\
\midrule
22/tcp & open & SSH & Not Determined \\
\bottomrule
\end{tabular}
\caption{Open Ports Detected on \texttt{127.0.0.1}.}
\end{table}

\subsection*{Analysis of Scan Results}
The scan identified that port 22, commonly used for the Secure Shell (SSH) protocol, is open on the localhost interface (\texttt{127.0.0.1}). While this port is not directly exposed to the internet, it presents a significant risk if an attacker gains an initial foothold on the system through other means (e.g., malware, web application vulnerability). This finding directly correlates with and validates the pre-existing risk documented as "Localhost Exposed."

% --- 5. Consolidated Risk Assessment ---
\section{Consolidated Risk Assessment}

The following table synthesizes findings from the security questionnaire, technical scan, and pre-existing risk data into a consolidated list of identified risks.

\begin{table}[h!]
\centering
\begin{tabular}{@{}lp{0.5\linewidth}ll}
\toprule
\textbf{Risk ID} & \textbf{Risk Description} & \textbf{Source} & \textbf{Severity} \\
\midrule
RISK-001 & \textbf{No MFA on Email:} User email accounts are protected only by passwords. & Questionnaire & \textbf{Critical} \\
\addlinespace
RISK-002 & \textbf{No MFA on Endpoints:} User computers/laptops are protected only by passwords. & Questionnaire & \textbf{Critical} \\
\addlinespace
RISK-003 & \textbf{No MFA on Sensitive Systems:} Access to critical data systems lacks MFA. & Questionnaire & \textbf{Critical} \\
\addlinespace
RISK-004 & \textbf{Localhost Exposed:} SSH service is running and accessible on the local interface. & Existing Risk, Scan & \textbf{Critical} \\
\addlinespace
RISK-005 & \textbf{Inadequate Security Training:} Employees do not receive recurring annual security training. & Questionnaire & \textbf{High} \\
\bottomrule
\end{tabular}
\caption{Summary of Identified Risks.}
\end{table}

% --- 6. Recommendations ---
\section{Recommendations}

Based on the consolidated risk assessment, the following actions are recommended to mitigate the identified vulnerabilities and improve the overall security posture of Vanguard Heritage. Recommendations are prioritized based on severity.

\subsection*{Immediate Priority (Critical Risks)}
\begin{enumerate}
    \item \textbf{Implement Comprehensive MFA:} Immediately begin a project to deploy Multi-Factor Authentication across all critical systems. The rollout should be prioritized as follows:
        \begin{itemize}
            \item Email accounts (e.g., Office 365, Google Workspace).
            \item Logins to all employee computers (endpoints).
            \item Access to all systems storing or processing sensitive data.
        \end{itemize}
    \item \textbf{Remediate "Localhost Exposed" Vulnerability:} Investigate the SSH service running on \texttt{127.0.0.1}. If this service is not essential for a specific application's function, it should be disabled immediately. If it is required, access should be restricted to only the necessary users and processes, and the service should be configured according to security best practices.
\end{enumerate}

\subsection*{High Priority}
\begin{enumerate}
    \setcounter{enumi}{2} % Continue numbering
    \item \textbf{Establish Annual Security Awareness Training:} Develop and implement a mandatory annual security awareness training program for all employees. This program should cover current threats such as phishing, ransomware, and proper data handling to foster a security-conscious culture.
\end{enumerate}

\subsection*{General Recommendations}
\begin{enumerate}
    \setcounter{enumi}{3}
    \item \textbf{Conduct In-Depth Vulnerability Scanning:} Schedule regular, authenticated vulnerability scans for all internal and external assets. The basic scan performed for this report only provides a limited view; a more comprehensive assessment is necessary to uncover other potential weaknesses.
\end{enumerate}

\end{document}
```