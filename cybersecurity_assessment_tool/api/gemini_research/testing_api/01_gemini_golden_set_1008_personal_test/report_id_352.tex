```latex
\documentclass[12pt]{article}

% Preamble: Required Packages
\usepackage[margin=1in]{geometry}
\usepackage{pifont} % For checkmarks and crosses
\usepackage{booktabs} % For professional tables
\usepackage{hyperref} % For clickable links
\usepackage{url} % For URL formatting
\usepackage{seqsplit} % To split long monospaced strings
\usepackage{graphicx}
\usepackage{xcolor}
\usepackage{array}

% Document Information
\title{Cybersecurity Posture Assessment Report \\ \large For: \textbf{Orchid Isle}}
\author{Cybersecurity Analysis Division}
\date{\today}

% Hyperref Setup
\hypersetup{
    colorlinks=true,
    linkcolor=blue,
    filecolor=magenta,      
    urlcolor=cyan,
    pdftitle={Cybersecurity Posture Assessment Report},
    pdfpagemode=FullScreen,
}

% Define a new column type for better table formatting
\newcolumntype{L}[1]{>{\raggedright\let\newline\\\arraybackslash\hspace{0pt}}m{#1}}
\newcolumntype{C}[1]{>{\centering\let\newline\\\arraybackslash\hspace{0pt}}m{#1}}

\begin{document}

\maketitle
\thispagestyle{empty}
\newpage

\tableofcontents
\newpage

% --- 1. Executive Summary ---
\section{Executive Summary}
This report provides a comprehensive cybersecurity posture assessment for \textbf{Orchid Isle}, conducted on \today. The analysis synthesizes data from an external network scan, a review of organizational security controls, and a list of pre-existing risks.

\paragraph{Key Findings:} The assessment reveals a mixed security posture. On a positive note, the external network scan of the target host (\texttt{192.168.1.100}) showed no exposed services, indicating a well-hardened network perimeter for that specific asset. However, significant and critical gaps were identified in the organization's internal security policies and controls.

\paragraph{Primary Risks:} The most critical risks stem from procedural and policy-based deficiencies rather than technical vulnerabilities. These include:
\begin{itemize}
    \item \textbf{Lack of Multi-Factor Authentication (MFA) for Sensitive Data Systems:} This is a critical vulnerability that significantly increases the risk of unauthorized access and data breaches.
    \item \textbf{Absence of an Acceptable Use Policy (AUP):} Without a formal AUP, there is no clear guidance for employees on the secure use of company assets, leading to inconsistent security practices.
    \item \textbf{No Security Awareness Training Program:} The complete lack of security training for both new and existing employees leaves the organization highly susceptible to social engineering attacks, such as phishing.
\end{itemize}

\paragraph{Recommendations:} Immediate action is required to address these policy gaps. We recommend prioritizing the implementation of MFA across all sensitive systems, followed by the development of a comprehensive AUP and the establishment of a mandatory, ongoing security awareness training program.

% --- 2. Organizational Information ---
\section{Organizational Information}
The following details were provided for the assessment.

\begin{tabular}{@{}ll}
    \toprule
    \textbf{Attribute} & \textbf{Value} \\
    \midrule
    Organization Name & \textbf{Orchid Isle} \\
    Email Domain & \texttt{OrchidIsle.org} \\
    Website Domain & \href{http://www.OrchidIsle.org}{\texttt{www.OrchidIsle.org}} \\
    External IP Address & \texttt{55.21.189.51} \\
    \bottomrule
\end{tabular}

% --- 3. Security Control Review ---
\section{Security Control Review}
The following table summarizes the organization's responses to a security controls questionnaire. "No" answers indicate significant gaps in the security framework and are flagged as risks.

\begin{table}[h!]
\centering
\begin{tabular}{L{9cm} C{2cm} L{3cm}}
    \toprule
    \textbf{Control Question} & \textbf{Response} & \textbf{Assessment} \\
    \midrule
    Do you require MFA to access email? & \ding{51} & Compliant \\
    Do you require MFA to log into computers? & \ding{51} & Compliant \\
    Do you require MFA to access sensitive data systems? & \textcolor{red}{\ding{55}} & \textbf{Critical Gap} \\
    Does your organization have an employee acceptable use policy? & \textcolor{red}{\ding{55}} & \textbf{High Risk} \\
    Does your organization do security awareness training for new employees? & \textcolor{red}{\ding{55}} & \textbf{High Risk} \\
    Does your organization do security awareness training for all employees at least once per year? & \textcolor{red}{\ding{55}} & \textbf{High Risk} \\
    \bottomrule
\end{tabular}
\caption{Organizational Security Controls Questionnaire Results.}
\end{table}

% --- 4. Technical Scan Results ---
\section{Technical Scan Results}
An external network vulnerability scan was performed to identify exposed services and potential entry points.

\begin{itemize}
    \item \textbf{Scan Target:} \texttt{192.168.1.100}
    \item \textbf{Scan Date:} \today
\end{itemize}

\subsection{Summary of Findings}
The scan results were positive. The target host was responsive (status: up), but no open TCP or UDP ports were discovered. All 65,535 ports were reported as being in a "closed" state.

\paragraph{Analysis:} A host with no open ports presents a minimal attack surface from the network perspective. This indicates effective firewall configuration and network segmentation for this specific asset, which is a strong security practice. No vulnerabilities were identified based on this scan.

% --- 5. Risk Assessment Summary ---
\section{Risk Assessment Summary}
This section consolidates risks identified from the security control review, technical scans, and pre-existing vulnerability data. As no pre-existing or technical risks were found, the summary focuses on the organizational gaps identified.

\begin{table}[h!]
\centering
\begin{tabular}{L{7cm} l l}
    \toprule
    \textbf{Risk Description} & \textbf{Severity} & \textbf{Source} \\
    \midrule
    \textbf{No MFA on Sensitive Systems:} Lack of MFA on systems storing or processing sensitive data exposes the organization to high-impact breaches from compromised credentials. & \textbf{Critical} & Questionnaire \\
    \addlinespace
    \textbf{No Acceptable Use Policy:} The absence of a formal policy creates ambiguity regarding secure practices, increasing the likelihood of insider threat (unintentional) and policy violations. & \textbf{High} & Questionnaire \\
    \addlinespace
    \textbf{No Security Awareness Training:} Employees are not trained to recognize or respond to cyber threats like phishing, making them the primary vector for potential intrusions. & \textbf{High} & Questionnaire \\
    \bottomrule
\end{tabular}
\caption{Consolidated Risk Register.}
\end{table}

% --- 6. Recommendations ---
\section{Recommendations}
The following actionable recommendations are provided to mitigate the identified risks and improve the overall security posture of \textbf{Orchid Isle}.

\subsection{Priority 1: Remediate Critical Risks}
\begin{description}
    \item[R-01: Implement MFA for Sensitive Systems]
    \textbf{Action:} Immediately deploy and enforce MFA for all user accounts (including administrative and service accounts) that have access to systems containing sensitive, confidential, or critical data.
    \textbf{Impact:} Drastically reduces the risk of unauthorized access via stolen or weak credentials.
\end{description}

\subsection{Priority 2: Remediate High Risks}
\begin{description}
    \item[R-02: Develop and Implement an Acceptable Use Policy (AUP)]
    \textbf{Action:} Create a formal AUP document that clearly defines the rules and expectations for employees when using company-owned IT assets, networks, and data. This policy should be reviewed by legal counsel, communicated to all staff, and formally acknowledged via signature.
    \textbf{Impact:} Establishes a baseline for secure user behavior and provides a framework for enforcement.

    \item[R-03: Establish a Security Awareness Training Program]
    \textbf{Action:} Procure and implement a security awareness training solution. The program must include mandatory initial training for all new hires and annual refresher training for all employees. Topics should cover phishing, password security, data handling, and threat reporting.
    \textbf{Impact:} Reduces the organization's susceptibility to social engineering and human error, strengthening the "human firewall."
\end{description}

\end{document}
```