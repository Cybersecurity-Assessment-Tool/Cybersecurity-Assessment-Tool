```latex
\documentclass[12pt]{article}

% Preamble: Required Packages
\usepackage[margin=1in]{geometry}
\usepackage{pifont} % For check marks and X's (\ding)
\usepackage{booktabs} % For professional-looking tables
\usepackage{hyperref} % For clickable links
\usepackage{url} % For formatting URLs
\usepackage{seqsplit} % For splitting long strings without spaces
\usepackage{array} % For table column formatting

% Document Metadata
\title{Cybersecurity Posture Assessment Report}
\author{Cybersecurity Analysis Division}
\date{\today}

\begin{document}

\maketitle
\thispagestyle{empty}
\newpage
\tableofcontents
\newpage
\pagestyle{plain}

% --- Executive Summary ---
\section*{Executive Summary}

This report provides a cybersecurity posture assessment for \textbf{Aeon Pharmaceuticals}, conducted on \today. The analysis synthesizes data from an external network scan, a security controls questionnaire, and a review of pre-existing risks.

The assessment reveals a mixed security posture. From a network perimeter perspective, the organization presents a strong defensive stance, as the external scan of the target IP address \texttt{[Client IP]} revealed no open ports. This significantly reduces the external attack surface.

However, the internal security controls and employee awareness programs exhibit critical deficiencies. The lack of Multi-Factor Authentication (MFA) on email and sensitive data systems exposes the organization to a high risk of account compromise and subsequent data breaches. Furthermore, the complete absence of an acceptable use policy and security awareness training leaves the organization highly vulnerable to phishing, social engineering, and insider threats.

Immediate remediation should focus on implementing MFA across all critical platforms and establishing a foundational security awareness program.

% --- Organizational Information ---
\section*{Organizational Information}

The following details were provided for the assessment.

\begin{tabular}{@{}ll}
\toprule
\textbf{Attribute} & \textbf{Value} \\
\midrule
Organization Name & \textbf{Aeon Pharmaceuticals} \\
Primary Email Domain & \texttt{AeonPharmaceuticals.com} \\
Primary Website & \seqsplit{\url{http://www.AeonPharmaceuticals.com}} \\
External IP Address & \texttt{235.101.225.254} \\
\bottomrule
\end{tabular}

% --- Security Control Review ---
\section*{Security Control Review}

A review of the organization's security controls was conducted via a questionnaire. The results highlight significant gaps in identity and access management and security governance. Answers marked with \ding{55} represent a deviation from security best practices and introduce substantial risk.

\begin{tabular}{@{} >{\raggedright\arraybackslash}p{0.7\textwidth} c c @{}}
\toprule
\textbf{Control Question} & \textbf{Response} & \textbf{Status} \\
\midrule
Do you require MFA to access email? & No & \ding{55} \\
Do you require MFA to log into computers? & Yes & \ding{51} \\
Do you require MFA to access sensitive data systems? & No & \ding{55} \\
Does your organization have an employee acceptable use policy? & No & \ding{55} \\
Does your organization do security awareness training for new employees? & No & \ding{55} \\
Does your organization do security awareness training for all employees at least once per year? & No & \ding{55} \\
\bottomrule
\end{tabular}

\subsection*{Analysis of Control Gaps}
\begin{itemize}
    \item \textbf{Lack of MFA:} The absence of MFA for email and sensitive data access is a critical vulnerability. Compromised credentials could grant an attacker direct access to confidential communications and proprietary data.
    \item \textbf{Policy and Training Deficiencies:} Without an acceptable use policy or security awareness training, employees are likely unaware of their security responsibilities. This makes them prime targets for phishing and other social engineering attacks, turning the workforce into an unintentional attack vector.
\end{itemize}

% --- Technical Scan Results ---
\section*{Technical Scan Results}

An external network vulnerability scan was performed on the designated target IP address.

\begin{itemize}
    \item \textbf{Target IP:} \texttt{[Target IP]}
    \item \textbf{Scan Date:} Data not available in scan results.
\end{itemize}

\subsection*{Findings}
The network scan completed successfully but did not identify any open TCP ports on the target host. This indicates a well-configured firewall or network access control list (ACL) that denies all unsolicited inbound traffic from the internet. While this is a strong security posture for the network perimeter, it does not provide insight into potential vulnerabilities in web applications or risks from internal threats.

% --- Identified Risks and Vulnerabilities ---
\section*{Identified Risks and Vulnerabilities}

This section correlates findings from the security control review and technical scans. No pre-existing vulnerabilities were provided for this assessment. The following risks have been identified based on the current analysis.

\begin{tabular}{@{} p{0.25\textwidth} p{0.55\textwidth} p{0.1\textwidth} @{}}
\toprule
\textbf{Risk Name} & \textbf{Overview} & \textbf{Severity} \\
\midrule
\textbf{Email Account Compromise} & The lack of MFA on email accounts makes them highly susceptible to takeover via credential stuffing or phishing. A compromised account can be used for business email compromise (BEC), further phishing attacks, and data exfiltration. & \textbf{Critical} \\
\addlinespace
\textbf{Sensitive Data Exposure} & Systems containing sensitive data are not protected by MFA. A single compromised password could lead to a significant breach of proprietary research, patient, or financial data, resulting in severe regulatory and reputational damage. & \textbf{Critical} \\
\addlinespace
\textbf{Untrained Workforce} & With no security awareness training, employees are unable to recognize or properly respond to cyber threats. This exponentially increases the likelihood of a successful phishing or social engineering attack against the organization. & \textbf{High} \\
\addlinespace
\textbf{Lack of Security Governance} & The absence of a formal Acceptable Use Policy means there are no defined rules for employee use of corporate assets. This ambiguity increases the risk of data misuse, policy violations, and insider threats. & \textbf{High} \\
\bottomrule
\end{tabular}

% --- Recommendations ---
\section*{Recommendations}

Based on the identified risks, the following remediation actions are recommended. They are prioritized to address the most critical vulnerabilities first.

\subsection*{Immediate Priority (Critical Risks)}
\begin{enumerate}
    \item \textbf{Implement MFA for Email:} Immediately enforce MFA for all user access to the corporate email system (e.g., Office 365, Google Workspace). This is the single most effective control to prevent email account takeovers.
    \item \textbf{Implement MFA for Sensitive Systems:} Identify all systems containing sensitive or critical data and enforce MFA for all user access, especially for administrative accounts.
\end{enumerate}

\subsection*{High Priority}
\begin{enumerate}
    \setcounter{enumi}{2}
    \item \textbf{Develop and Implement Security Policies:} Draft and formally adopt a baseline set of security policies, starting with an Employee Acceptable Use Policy. Ensure all employees read and acknowledge the policy.
    \item \textbf{Establish a Security Awareness Program:} Procure and deploy a security awareness training solution. All employees must complete a baseline training module upon hire and an annual refresher course. The training must cover, at a minimum, phishing, password security, and safe data handling practices.
\end{enumerate}

\end{document}
```