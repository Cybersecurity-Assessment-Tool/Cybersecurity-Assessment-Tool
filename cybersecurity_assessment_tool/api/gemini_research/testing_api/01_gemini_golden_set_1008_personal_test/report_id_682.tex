```latex
\documentclass[12pt]{article}

% Preamble: Required Packages
\usepackage[margin=1in]{geometry}
\usepackage{pifont} % For checkmarks and crosses
\usepackage{booktabs} % For professional tables
\usepackage{hyperref} % For clickable links
\usepackage{url} % For formatting URLs
\usepackage{seqsplit} % For splitting long strings in texttt
\usepackage{xcolor} % For colors in tables

% Document Metadata
\title{Cybersecurity Posture Assessment Report}
\author{Cybersecurity Analysis Division}
\date{\today}

% Hyperref Setup
\hypersetup{
    colorlinks=true,
    linkcolor=blue,
    filecolor=magenta,      
    urlcolor=cyan,
    pdftitle={Cybersecurity Posture Assessment Report},
    pdfpagemode=FullScreen,
}

\begin{document}

\maketitle
\thispagestyle{empty}
\newpage

\tableofcontents
\newpage

% --- 1. Executive Summary ---
\section{Executive Summary}
This report provides a comprehensive cybersecurity posture assessment for \textbf{Vanguard Heritage}. The analysis is based on a correlation of organizational data, a review of existing security controls, and an external network scan.

The assessment identified two significant areas of concern stemming from gaps in security controls. A critical risk was identified due to the absence of Multi-Factor Authentication (MFA) for computer logins, which exposes the organization to increased risk from compromised credentials. Additionally, a high risk was noted regarding the lack of mandatory security awareness training for new employees, making them susceptible to social engineering and phishing attacks.

The external network scan of the target IP address, \texttt{[Target IP]}, did not reveal any open ports. While this may indicate a strong firewall configuration, it could also suggest the host was offline or the scan was blocked.

Immediate remediation should focus on implementing endpoint MFA and integrating security training into the employee onboarding process to mitigate these identified risks and strengthen the overall security posture.

% --- 2. Organizational Information ---
\section{Organizational Information}
The following details were provided for the assessment.

\begin{tabular}{@{}ll}
\toprule
\textbf{Attribute} & \textbf{Value} \\
\midrule
Organization Name & \textbf{Vanguard Heritage} \\
Email Domain & \texttt{VanguardHeritage.net} \\
Website Domain & \url{www.VanguardHeritage.net} \\
External IP Address & \texttt{103.111.175.128} \\
\bottomrule
\end{tabular}

% --- 3. Security Control Review ---
\section{Security Control Review}
A review of the organization's security controls was conducted via a questionnaire. The responses are summarized below. Gaps in controls, indicated by a "No" response, represent significant risks that require attention.

\begin{tabular}{@{}p{0.6\linewidth} c p{0.2\linewidth}@{}}
\toprule
\textbf{Control Question} & \textbf{Response} & \textbf{Assessment} \\
\midrule
Do you require MFA to access email? & \ding{51} & Control in place \\
Do you require MFA to log into computers? & \textcolor{red}{\ding{55}} & \textbf{Critical Gap} \\
Do you require MFA to access sensitive data systems? & \ding{51} & Control in place \\
Does your organization have an employee acceptable use policy? & \ding{51} & Control in place \\
Does your organization do security awareness training for new employees? & \textcolor{red}{\ding{55}} & \textbf{High-Risk Gap} \\
Does your organization do security awareness training for all employees at least once per year? & \ding{51} & Control in place \\
\bottomrule
\end{tabular}

% --- 4. Technical Scan Results ---
\section{Technical Scan Results}
An external network vulnerability scan was performed to identify open ports and exposed services.

\begin{itemize}
    \item \textbf{Target IP Address:} \texttt{[Target IP]}
    \item \textbf{Scan Date:} \today
    \item \textbf{Summary:} The scan completed successfully but did not identify any open TCP or UDP ports on the target system.
\end{itemize}

\textbf{Analysis:} The absence of open ports is a positive security finding, suggesting that a well-configured firewall may be in place, restricting external access. However, this result could also be due to the host being offline at the time of the scan or an Intrusion Prevention System (IPS) blocking the scan traffic. Further internal verification is recommended to confirm the firewall's effectiveness.

% --- 5. Risk Assessment ---
\section{Risk Assessment}
This section synthesizes findings from the security control review and technical scan. The following new risks have been identified. No pre-existing vulnerabilities were provided for this assessment.

\begin{tabular}{@{}p{0.1\linewidth} p{0.25\linewidth} p{0.4\linewidth} p{0.15\linewidth}@{}}
\toprule
\textbf{Risk ID} & \textbf{Risk Name} & \textbf{Description} & \textbf{Severity} \\
\midrule
RISK-001 & Lack of Endpoint MFA & The absence of MFA for computer logins means that a single compromised password could grant an attacker full access to an employee's workstation and potentially the internal network. & \textbf{Critical} \\
\addlinespace
RISK-002 & No Security Training for New Hires & New employees are not provided with security awareness training during onboarding. This makes them highly vulnerable to phishing, social engineering, and other common attack vectors, as they are unfamiliar with corporate security policies. & \textbf{High} \\
\bottomrule
\end{tabular}

% --- 6. Recommendations ---
\section{Recommendations}
Based on the identified risks, the following prioritized actions are recommended to enhance the security posture of \textbf{Vanguard Heritage}.

\subsection{RISK-001: Lack of Endpoint MFA (Critical)}
\begin{itemize}
    \item \textbf{Action:} Procure and deploy a Multi-Factor Authentication solution for all employee computer and laptop logins (both local and remote).
    \item \textbf{Justification:} This control adds a critical layer of security that protects against credential theft and unauthorized access. Even if an attacker obtains a user's password, they will be unable to log in without the second factor.
    \item \textbf{Priority:} Immediate
\end{itemize}

\subsection{RISK-002: No Security Training for New Hires (High)}
\begin{itemize}
    \item \textbf{Action:} Develop a mandatory security awareness training module and integrate it into the formal new employee onboarding process.
    \item \textbf{Justification:} Educating new hires on topics such as phishing identification, strong password creation, and acceptable use policies from day one significantly reduces the organization's human-based risk surface.
    \item \textbf{Priority:} High
\end{itemize}

\end{document}
```