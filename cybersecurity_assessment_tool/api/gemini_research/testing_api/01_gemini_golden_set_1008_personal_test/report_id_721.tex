```latex
\documentclass[12pt]{article}

% Preamble: Required Packages
\usepackage[margin=1in]{geometry}
\usepackage{pifont}
\usepackage{booktabs}
\usepackage{hyperref}
\usepackage{url}
\usepackage{seqsplit}

% Document Metadata
\title{Cybersecurity Posture Assessment Report}
\author{Cybersecurity Analysis Division}
\date{November 22, 2025}

% Hyperref Setup
\hypersetup{
    colorlinks=true,
    linkcolor=black,
    urlcolor=blue,
    pdftitle={Cybersecurity Posture Assessment Report},
    pdfauthor={Cybersecurity Analysis Division},
    pdfsubject={Security Assessment},
    pdfkeywords={Security, Assessment, Report}
}

\begin{document}

\maketitle
\thispagestyle{empty}
\newpage

\tableofcontents
\thispagestyle{empty}
\newpage

\setcounter{page}{1}

% ==============================================================================
% Section 1: Executive Overview
% ==============================================================================
\section{Executive Overview}

This report details the findings of a cybersecurity posture assessment conducted for \textbf{Hearth \& Home}. The assessment combined a review of organizational security controls, an external network scan, and an analysis of pre-existing risks.

The overall security posture is assessed as \textbf{High-Risk}. This is primarily driven by critical deficiencies in identity and access management, specifically a systemic lack of Multi-Factor Authentication (MFA) across all key systems including email, endpoints, and sensitive data repositories. This exposes the organization to a high likelihood of account compromise and subsequent data breaches.

Furthermore, the external-facing web server was found to be running significantly outdated software (\texttt{nginx 1.18.0}), which is known to contain multiple security vulnerabilities. This, combined with gaps in the security awareness training program for new employees, creates a high-risk environment.

Immediate remediation of the identified critical and high-severity risks is strongly recommended to reduce the organization's attack surface and improve its defensive capabilities.

% ==============================================================================
% Section 2: Organizational Information
% ==============================================================================
\section{Organizational Information}

The following information was provided for the assessment.

\begin{itemize}
    \item \textbf{Organization Name:} Hearth \& Home
    \item \textbf{Primary Email Domain:} \texttt{HearthHome.com}
    \item \textbf{Primary Website Domain:} \url{www.HearthHome.com}
    \item \textbf{External IP Scanned:} \texttt{75.228.134.68}
\end{itemize}

% ==============================================================================
% Section 3: Security Control Review
% ==============================================================================
\section{Security Control Review}

A review of administrative and organizational security controls was conducted via a standardized questionnaire. The responses indicate significant gaps in foundational security practices. A checkmark (\ding{51}) indicates a positive control, while an X (\ding{55}) indicates a control gap.

\begin{table}[h!]
\centering
\caption{Organizational Security Control Status}
\label{tab:controls}
\begin{tabular}{@{}lc@{}}
\toprule
\textbf{Control Question} & \textbf{Response} \\ \midrule
Do you require MFA to access email? & \ding{55} \\
Do you require MFA to log into computers? & \ding{55} \\
Do you require MFA to access sensitive data systems? & \ding{55} \\
Does your organization have an employee acceptable use policy? & \ding{51} \\
Does your organization do security awareness training for new employees? & \ding{55} \\
Does your organization do security awareness training for all employees at least once per year? & \ding{51} \\ \bottomrule
\end{tabular}
\end{table}

% ==============================================================================
% Section 4: Technical Scan Results
% ==============================================================================
\section{Technical Scan Results}

An external network scan was performed to identify open ports and exposed services on the organization's public-facing infrastructure.

\subsection{Scan Metadata}
\begin{itemize}
    \item \textbf{Target IP Address:} \texttt{192.168.10.5}
    \item \textbf{Scan Date:} 2025-11-22T10:00:00Z
\end{itemize}

\subsection{Open Ports and Services}
The following table details the open ports and services discovered during the scan.

\begin{table}[h!]
\centering
\caption{Discovered Open Ports}
\label{tab:ports}
\begin{tabular}{@{}lllll@{}}
\toprule
\textbf{Port} & \textbf{State} & \textbf{Service} & \textbf{Product} & \textbf{Version} \\ \midrule
443/tcp & open & https & nginx & 1.18.0 \\ \bottomrule
\end{tabular}
\end{table}

\subsection{Technical Observations}
\begin{enumerate}
    \item \textbf{Outdated Web Server Software:} The web server is running \textbf{nginx version 1.18.0}, which was released in April 2020. This version is significantly outdated and no longer receives security support. It is susceptible to numerous publicly disclosed vulnerabilities that could be exploited by an attacker to compromise the server.
    
    \item \textbf{SSL Certificate Mismatch:} The SSL certificate presented by the server has a Common Name of \texttt{www.acme-corp.com}, which does not match the organization's domain (\texttt{www.HearthHome.com}). This will cause browser trust errors for visitors and may indicate a server misconfiguration or a temporary, non-production certificate being used in a live environment.
\end{enumerate}

% ==============================================================================
% Section 5: Consolidated Risk Assessment
% ==============================================================================
\section{Consolidated Risk Assessment}

The following table synthesizes findings from the security control review and technical scan into a prioritized list of risks. No pre-existing vulnerabilities were reported.

\begin{table}[h!]
\centering
\caption{Identified Security Risks}
\label{tab:risks}
\begin{tabular}{@{}lp{5cm}p{3.5cm}l@{}}
\toprule
\textbf{ID} & \textbf{Risk Description} & \textbf{Affected Asset(s)} & \textbf{Severity} \\ \midrule
RISK-001 & Complete lack of Multi-Factor Authentication (MFA) enforcement for employees. & Email System, Endpoints, Sensitive Data Systems & \textbf{Critical} \\
\addlinespace
RISK-002 & External web server is running a vulnerable, end-of-life version of nginx. & Web Server (\texttt{75.228.134.68}) & \textbf{High} \\
\addlinespace
RISK-003 & New employees do not receive security awareness training upon being hired. & All New Employees, Corporate Network & \textbf{High} \\
\addlinespace
RISK-004 & The public-facing website has a misconfigured SSL certificate. & Website (\url{www.HearthHome.com}) & \textbf{Medium} \\ \bottomrule
\end{tabular}
\end{table}

% ==============================================================================
% Section 6: Recommendations
% ==============================================================================
\section{Recommendations}

The following actions are recommended to mitigate the identified risks and improve the overall security posture of \textbf{Hearth \& Home}.

\subsection{RISK-001: Implement Multi-Factor Authentication (Critical)}
\begin{itemize}
    \item \textbf{Immediate Action:} Enforce MFA for all user accounts, starting with privileged administrators and executives.
    \item \textbf{Short-Term Action:} Roll out MFA enforcement across all critical systems, prioritizing email (e.g., Office 365, Google Workspace), VPN access, and access to any system containing sensitive or financial data.
    \item \textbf{Long-Term Goal:} Ensure MFA is a mandatory component for accessing any corporate resource.
\end{itemize}

\subsection{RISK-002: Remediate Outdated Web Server (High)}
\begin{itemize}
    \item \textbf{Immediate Action:} Develop a patch plan to upgrade the nginx server from version 1.18.0 to the latest stable version.
    \item \textbf{Short-Term Action:} Execute the patch plan in a staging environment before deploying to production.
    \item \textbf{Long-Term Goal:} Implement a formal patch and vulnerability management policy for all internet-facing systems, ensuring software is updated on a regular, proactive basis.
\end{itemize}

\subsection{RISK-003: Enhance Security Awareness Program (High)}
\begin{itemize}
    \item \textbf{Immediate Action:} Integrate a mandatory security awareness training module into the new employee onboarding process. This training should cover phishing, acceptable use, password hygiene, and incident reporting.
    \item \textbf{Long-Term Goal:} Continue the annual security training for all staff and supplement it with periodic phishing simulation campaigns to measure and improve employee resilience.
\end{itemize}

\subsection{RISK-004: Correct SSL Certificate Configuration (Medium)}
\begin{itemize}
    \item \textbf{Immediate Action:} Investigate the SSL certificate mismatch on the web server.
    \item \textbf{Short-Term Action:} Procure and install a valid SSL/TLS certificate issued for the correct domain, \texttt{www.HearthHome.com}. Ensure the entire certificate chain is trusted.
\end{itemize}

\end{document}
```