```latex
\documentclass[12pt]{article}

% -----------------------------------------------------------------------------
% PREAMBLE
% -----------------------------------------------------------------------------
\usepackage[margin=1in]{geometry}
\usepackage{pifont} % For checkmarks and crosses
\usepackage{booktabs} % For professional tables
\usepackage[table]{xcolor} % For coloring table cells
\usepackage{hyperref} % For hyperlinks
\usepackage{url} % For URL formatting
\usepackage{seqsplit} % To split long strings in texttt
\usepackage{graphicx} % For logo (placeholder)

% Define colors for risk levels
\definecolor{criticalred}{HTML}{D12B2B}
\definecolor{highorange}{HTML}{E57C2F}
\definecolor{mediumyellow}{HTML}{F5C344}
\definecolor{lowblue}{HTML}{4A90E2}
\definecolor{tableheadgray}{HTML}{EAEAEA}

% Hyperref setup
\hypersetup{
    colorlinks=true,
    linkcolor=blue,
    filecolor=magenta,      
    urlcolor=cyan,
    pdftitle={Cybersecurity Assessment Report},
    pdfpagemode=FullScreen,
}

% Define checkmark and cross symbols
\newcommand{\cmark}{\ding{51}}
\newcommand{\xmark}{\ding{55}}

% -----------------------------------------------------------------------------
% DOCUMENT START
% -----------------------------------------------------------------------------
\begin{document}

% -----------------------------------------------------------------------------
% TITLE PAGE
% -----------------------------------------------------------------------------
\begin{titlepage}
    \centering
    \vspace*{1cm}
    
    % Placeholder for a logo
    % \includegraphics[width=0.4\textwidth]{logo.png} 
    
    \vspace{1.5cm}
    
    \Huge
    \textbf{Cybersecurity Assessment Report}
    
    \vspace{1.5cm}
    
    \Large
    Prepared for: \\
    \vspace{0.5cm}
    \textbf{Nomad Gear Co.}
    
    \vfill
    
    \Large
    \today
    
\end{titlepage}

\tableofcontents
\newpage

% -----------------------------------------------------------------------------
% 1. EXECUTIVE SUMMARY
% -----------------------------------------------------------------------------
\section{Executive Summary}

This report details the findings of a cybersecurity assessment conducted for \textbf{Nomad Gear Co.}. The evaluation combined a review of organizational security controls, an external network scan, and an analysis of pre-existing risks.

The assessment identified several areas of significant concern that require immediate attention. A critical vulnerability was discovered on an internal network host (\texttt{10.0.0.15}), which is running a dangerously outdated and misconfigured FTP service (\textbf{vsftpd 2.3.4}) with a known public backdoor. This vulnerability presents a direct and immediate path for an attacker to compromise the internal network.

Furthermore, a critical gap was identified in the organization's security culture: the complete absence of security awareness training for employees. This deficiency significantly increases the risk of successful phishing attacks, social engineering, and other human-vector threats.

While the organization has implemented strong Multi-Factor Authentication (MFA) controls for key systems, the combination of the technical vulnerability and the lack of employee training creates a high-risk environment. This report provides a detailed breakdown of these findings and offers actionable recommendations to mitigate the identified risks and strengthen the overall security posture of \textbf{Nomad Gear Co.}.

% -----------------------------------------------------------------------------
% 2. ORGANIZATIONAL INFORMATION
% -----------------------------------------------------------------------------
\section{Organizational Information}

The following information was provided for the assessment.

\begin{description}
    \item[Organization Name:] \textbf{Nomad Gear Co.}
    \item[Email Domain:] \texttt{NomadGearCo.net}
    \item[Website Domain:] \texttt{www.NomadGearCo.net}
    \item[External IP Address:] \texttt{118.210.136.105}
\end{description}

% -----------------------------------------------------------------------------
% 3. SECURITY CONTROL REVIEW
% -----------------------------------------------------------------------------
\section{Security Control Review}

A review of administrative and policy-based security controls was conducted via a questionnaire. The results are summarized below. "Yes" responses (\cmark) indicate a control is in place, while "No" responses (\xmark) indicate a control gap.

\begin{table}[h!]
\centering
\rowcolors{2}{gray!10}{white}
\begin{tabular}{p{0.8\linewidth} c}
\toprule
\rowcolor{tableheadgray}
\textbf{Control Question} & \textbf{Response} \\
\midrule
Do you require MFA to access email? & \cmark \\
Do you require MFA to log into computers? & \cmark \\
Do you require MFA to access sensitive data systems? & \cmark \\
Does your organization have an employee acceptable use policy? & \cmark \\
Does your organization do security awareness training for new employees? & \xmark \\
Does your organization do security awareness training for all employees at least once per year? & \xmark \\
\bottomrule
\end{tabular}
\caption{Organizational Security Controls Questionnaire Results}
\end{table}

\subsection*{Analysis of Control Gaps}
The questionnaire reveals a critical weakness in the "human firewall." The lack of security awareness training for both new and existing employees is a major concern. Without regular training, employees are significantly more susceptible to phishing, malware, and social engineering attacks, which are the most common initial access vectors for threat actors. This gap undermines other technical controls that are in place.

% -----------------------------------------------------------------------------
% 4. TECHNICAL SCAN RESULTS
% -----------------------------------------------------------------------------
\section{Technical Scan Results}

An Nmap scan was performed on the specified target to identify open ports and exposed services.

\subsection*{Host: \texttt{10.0.0.15}}
\begin{description}
    \item[Status:] UP
\end{description}

\begin{table}[h!]
\centering
\begin{tabular}{lllll}
\toprule
\rowcolor{tableheadgray}
\textbf{Port} & \textbf{State} & \textbf{Service} & \textbf{Version} & \textbf{Notes} \\
\midrule
21/tcp & open & ftp & \seqsplit{\texttt{vsftpd 2.3.4}} & Anonymous FTP login allowed \\
\bottomrule
\end{tabular}
\caption{Open Ports and Services on \texttt{10.0.0.15}}
\end{table}

\subsection*{Analysis of Technical Findings}
The scan identified a \textbf{critical vulnerability}. The FTP service is running \textbf{vsftpd version 2.3.4}. This specific version is widely known to contain a critical backdoor vulnerability (CVE-2011-2523), which was intentionally added to the source code. An attacker can exploit this backdoor to gain a command shell on the underlying operating system with minimal effort.

Compounding this issue, the service is configured to allow \textbf{anonymous FTP login}. This removes the need for an attacker to have credentials, making exploitation trivial. This server should be considered compromised and must be addressed immediately.

% -----------------------------------------------------------------------------
% 5. CONSOLIDATED RISK ASSESSMENT
% -----------------------------------------------------------------------------
\section{Consolidated Risk Assessment}

The following table synthesizes findings from the security control review, technical scan, and pre-existing risk data into a prioritized list.

\begin{table}[h!]
\centering
\begin{tabular}{p{0.1\linewidth} p{0.4\linewidth} p{0.2\linewidth} p{0.2\linewidth}}
\toprule
\rowcolor{tableheadgray}
\textbf{Risk ID} & \textbf{Description} & \textbf{Severity} & \textbf{Affected Elements} \\
\midrule
RISK-001 & A publicly known backdoor vulnerability (CVE-2011-2523) exists in the vsftpd 2.3.4 service, exacerbated by anonymous login being enabled. & \cellcolor{criticalred!80}\textbf{Critical} & Server at \texttt{10.0.0.15}, Internal Network Integrity \\
\addlinespace
RISK-002 & The organization does not conduct security awareness training for any employees. This elevates the risk of human-centric attacks like phishing. & \cellcolor{highorange!80}\textbf{High} & All Employees, Organizational Data \\
\addlinespace
RISK-003 & Workstations are running the outdated and unsupported Windows 7 operating system, which no longer receives security updates. & \cellcolor{mediumyellow!80}\textbf{Medium} & Workstations \\
\bottomrule
\end{tabular}
\caption{Summary of Identified Risks}
\end{table}

% -----------------------------------------------------------------------------
% 6. RECOMMENDATIONS
% -----------------------------------------------------------------------------
\section{Recommendations}

The following actions are recommended to mitigate the identified risks, prioritized by severity.

\subsection*{Priority: Immediate (Critical Risks)}
\begin{enumerate}
    \item \textbf{Remediate Vulnerable FTP Server (RISK-001):}
    \begin{itemize}
        \item \textbf{Action:} Immediately disconnect the server at \texttt{10.0.0.15} from the network.
        \item \textbf{Action:} Perform a forensic analysis to determine if the backdoor has already been exploited.
        \item \textbf{Action:} Rebuild the server with a modern, supported operating system. If FTP is required, install the latest stable version of an FTP server (e.g., vsftpd 3.0.3+) and disable anonymous access unless there is an explicit business requirement.
        \item \textbf{Action:} If FTP is not required, ensure the service is not installed on the new server.
    \end{itemize}
\end{enumerate}

\subsection*{Priority: High}
\begin{enumerate}
    \setcounter{enumi}{1}
    \item \textbf{Implement Security Awareness Training (RISK-002):}
    \begin{itemize}
        \item \textbf{Action:} Procure and deploy a security awareness training platform for all employees.
        \item \textbf{Action:} Make training mandatory for all new hires during their onboarding process.
        \item \textbf{Action:} Require all existing employees to complete the training annually.
        \item \textbf{Action:} Conduct regular phishing simulation campaigns to test and reinforce the training.
    \end{itemize}
\end{enumerate}

\subsection*{Priority: Medium}
\begin{enumerate}
    \setcounter{enumi}{2}
    \item \textbf{Upgrade Outdated Workstations (RISK-003):}
    \begin{itemize}
        \item \textbf{Action:} Continue and accelerate the project to upgrade or replace all workstations running Windows 7 with a modern, supported operating system like Windows 10 or 11.
        \item \textbf{Action:} Implement a patch management policy to ensure all systems receive security updates in a timely manner.
    \end{itemize}
\end{enumerate}

% -----------------------------------------------------------------------------
% DOCUMENT END
% -----------------------------------------------------------------------------
\end{document}
```