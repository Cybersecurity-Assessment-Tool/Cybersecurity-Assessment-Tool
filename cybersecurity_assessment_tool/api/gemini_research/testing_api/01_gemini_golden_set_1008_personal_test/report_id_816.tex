Here is the complete and professional LaTeX report, generated based on the provided data and instructions.

```latex
\documentclass[12pt]{article}

% ----------------------------------------------------------------
% PREAMBLE
% ----------------------------------------------------------------
\usepackage[margin=1in]{geometry}
\usepackage{pifont} % For checkmarks and crosses
\usepackage{booktabs} % For professional tables
\usepackage{hyperref} % For hyperlinks and metadata
\usepackage{url}      % For proper URL formatting
\usepackage{seqsplit} % For splitting long strings in tt font

% Hyperref setup for a clean look
\hypersetup{
    colorlinks=true,
    linkcolor=black,
    filecolor=black,      
    urlcolor=blue,
    pdftitle={Cybersecurity Posture Report},
    pdfpagemode=FullScreen,
}

% ----------------------------------------------------------------
% DOCUMENT START
% ----------------------------------------------------------------
\begin{document}

% ----------------------------------------------------------------
% TITLE PAGE
% ----------------------------------------------------------------
\begin{titlepage}
    \centering
    \vspace*{\fill}
    \Huge\bfseries
    Cybersecurity Posture Report
    \vspace{0.5cm}
    
    \Large\bfseries
    Prepared for: Apex Legends Group
    \vspace{1.5cm}
    
    \normalsize
    \textbf{Date of Report:} \today \\
    \textbf{Author:} Cybersecurity Analyst
    \vspace*{\fill}
\end{titlepage}

\tableofcontents
\newpage

% ----------------------------------------------------------------
% SECTION 1: EXECUTIVE OVERVIEW
% ----------------------------------------------------------------
\section{Executive Overview}
This report provides an assessment of the cybersecurity posture for Apex Legends Group. The analysis is based on a review of organizational security controls provided via a questionnaire. 

It is critical to note that the technical network scan data (\texttt{Input\_1\_Network\_Scan\_JSON}) and the list of pre-existing risks (\texttt{Input\_3\_Current\_Risks\_JSON}) were found to be corrupted and could not be processed for this report. Consequently, this assessment is focused exclusively on the organizational data provided.

The review identified two high-risk security gaps:
\begin{itemize}
    \item \textbf{Lack of Multi-Factor Authentication (MFA) on Endpoints:} The absence of mandatory MFA for computer logins presents a significant risk. A compromised password could grant an attacker direct access to an employee's workstation and, potentially, the internal network.
    \item \textbf{No Security Training for New Employees:} New hires are not receiving security awareness training upon joining the organization. This makes them highly susceptible to social engineering and phishing attacks, which are common initial access vectors for threat actors.
\end{itemize}

These findings indicate critical weaknesses in the organization's defense-in-depth strategy. Immediate remediation is recommended to mitigate the risk of unauthorized access and strengthen the overall security posture.

% ----------------------------------------------------------------
% SECTION 2: ORGANIZATIONAL INFORMATION
% ----------------------------------------------------------------
\section{Organizational Information}
The following details were provided by the organization for the scope of this assessment.

\begin{table}[h!]
\centering
\begin{tabular}{@{}ll@{}}
\toprule
\textbf{Attribute} & \textbf{Value} \\ \midrule
Organization Name  & Apex Legends Group \\
Email Domain       & \seqsplit{\texttt{ApexLegendsGroup.com}} \\
Website Domain     & \url{www.ApexLegendsGroup.com} \\
External IP Address & \seqsplit{\texttt{193.73.125.24}} \\ \bottomrule
\end{tabular}
\caption{Client Organizational Details.}
\end{table}

% ----------------------------------------------------------------
% SECTION 3: SECURITY CONTROL REVIEW
% ----------------------------------------------------------------
\section{Security Control Review}
The following table summarizes the organization's responses to the security controls questionnaire. A checkmark (\ding{51}) indicates a positive control is in place, while a cross (\ding{55}) indicates a control gap that requires attention.

\begin{table}[h!]
\centering
\begin{tabular}{@{}lc@{}}
\toprule
\textbf{Control Question} & \textbf{Response} \\ \midrule
Do you require MFA to access email? & \ding{51} \\
Do you require MFA to log into computers? & \ding{55} \\
Do you require MFA to access sensitive data systems? & \ding{51} \\
Does your organization have an employee acceptable use policy? & \ding{51} \\
Does your organization do security awareness training for new employees? & \ding{55} \\
Does your organization do security training for all employees annually? & \ding{51} \\ \bottomrule
\end{tabular}
\caption{Security Controls Questionnaire Results.}
\end{table}

% ----------------------------------------------------------------
% SECTION 4: TECHNICAL SCAN RESULTS
% ----------------------------------------------------------------
\section{Technical Scan Results}
The external network scan against the target IP address, \texttt{[Target IP]}, could not be completed. The provided input data file containing the scan results was corrupted and unparsable.

\textbf{Implication:} Without this data, there is no visibility into the external attack surface of the specified asset. This includes open ports, running services, software versions, and potential configuration weaknesses that could be exploited by an external attacker. It is crucial to resolve the data integrity issue with the scanning tool to enable a comprehensive technical assessment in the future.

% ----------------------------------------------------------------
% SECTION 5: RISK ASSESSMENT
% ----------------------------------------------------------------
\section{Risk Assessment}
This risk assessment is based on the gaps identified in the Security Control Review. Due to corrupted input data, it does not include findings from the technical network scan or a review of pre-existing vulnerabilities.

\begin{table}[h!]
\centering
\begin{tabular}{@{}p{0.3\linewidth}p{0.5\linewidth}l@{}}
\toprule
\textbf{Identified Risk} & \textbf{Description} & \textbf{Severity} \\ \midrule
\textbf{Lack of MFA on Endpoints} & The absence of MFA on computer logins means that a single compromised password is sufficient for an attacker to gain access to an employee's workstation. This significantly increases the risk of lateral movement and unauthorized data access. & High \\
\addlinespace
\textbf{No Security Training for New Hires} & New employees are a prime target for phishing and social engineering attacks. Without immediate training, they are unaware of company policies and common threats, making them a vulnerable entry point into the organization. & High \\ \bottomrule
\end{tabular}
\caption{Summary of Identified Risks.}
\end{table}

% ----------------------------------------------------------------
% SECTION 6: RECOMMENDATIONS
% ----------------------------------------------------------------
\section{Recommendations}
Based on the analysis, the following actions are recommended to mitigate the identified risks and improve the overall cybersecurity posture of Apex Legends Group.

\begin{enumerate}
    \item \textbf{High Priority: Implement Endpoint MFA.}
    \begin{itemize}
        \item \textbf{Action:} Deploy a mandatory Multi-Factor Authentication solution for all employee computer and laptop logins.
        \item \textbf{Justification:} This creates a critical barrier against credential theft. Even if an attacker obtains a user's password, they cannot log in without the second factor (e.g., a mobile push notification, authenticator code, or hardware token).
        \item \textbf{Examples:} Microsoft Windows Hello for Business, Duo Security, Okta.
    \end{itemize}
    \vspace{0.5cm}
    
    \item \textbf{High Priority: Institute Onboarding Security Training.}
    \begin{itemize}
        \item \textbf{Action:} Develop and integrate a mandatory security awareness training module into the new employee onboarding process. This training must be completed before a new hire is granted full access to network resources.
        \item \textbf{Justification:} This ensures that all employees, from their first day, have a baseline understanding of cyber threats, safe computing practices, and their responsibilities in protecting company data.
        \item \textbf{Topics to Cover:} Phishing and spear-phishing identification, password security, acceptable use of company assets, and procedures for reporting security incidents.
    \end{itemize}
    \vspace{0.5cm}
    
    \item \textbf{Procedural: Remediate Data Corruption Issues.}
    \begin{itemize}
        \item \textbf{Action:} Investigate the root cause of the data corruption affecting the network scanning and risk management systems.
        \item \textbf{Justification:} To conduct comprehensive and effective security assessments, all data sources must be reliable. Ensuring the integrity of these systems is essential for future risk management and visibility into the technical attack surface.
    \end{itemize}
\end{enumerate}

% ----------------------------------------------------------------
% DOCUMENT END
% ----------------------------------------------------------------
\end{document}
```