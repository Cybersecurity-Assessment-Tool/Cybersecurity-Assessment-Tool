```latex
\documentclass[12pt]{article}

% Preamble: Required Packages and Document Setup
\usepackage[margin=1in]{geometry}
\usepackage{pifont} % For checkmarks and crosses
\usepackage{booktabs} % For professional tables
\usepackage{hyperref} % For clickable links
\usepackage{url} % For formatting URLs
\usepackage{seqsplit} % To split long strings in tt font
\usepackage{graphicx}
\usepackage{xcolor}
\usepackage{fancyhdr}
\usepackage{lastpage}

% --- Document Metadata ---
\hypersetup{
    colorlinks=true,
    linkcolor=blue,
    filecolor=magenta,      
    urlcolor=cyan,
    pdftitle={Cybersecurity Posture Assessment Report},
    pdfauthor={Cybersecurity Analyst AI},
    pdfsubject={Security Assessment},
    pdfkeywords={Cybersecurity, Risk, Assessment},
    pdffitwindow=true
}

% --- Header and Footer Configuration ---
\pagestyle{fancy}
\fancyhf{} % Clear all header and footer fields
\fancyhead[L]{Cybersecurity Posture Assessment Report}
\fancyhead[R]{Signal Flare}
\fancyfoot[C]{\thepage\ of \pageref{LastPage}}
\renewcommand{\headrulewidth}{0.4pt}
\renewcommand{\footrulewidth}{0.4pt}

% --- Custom Commands for Severity ---
\newcommand{\sevCRITICAL}{\textcolor{red}{\textbf{Critical}}}
\newcommand{\sevHIGH}{\textcolor{orange}{\textbf{High}}}
\newcommand{\sevMEDIUM}{\textcolor{yellow!80!black}{\textbf{Medium}}}
\newcommand{\sevLOW}{\textcolor{green}{\textbf{Low}}}
\newcommand{\sevINFO}{\textcolor{blue}{\textbf{Informational}}}

\begin{document}

% --- Title Page ---
\begin{titlepage}
    \centering
    \vspace*{1cm}
    
    \Huge
    \textbf{Cybersecurity Posture Assessment Report}
    
    \vspace{1.5cm}
    
    \Large
    Prepared For:
    
    \vspace{0.5cm}
    
    \textbf{Signal Flare}
    
    \vfill
    
    \large
    Date of Report: \today
    
    \vspace{1.5cm}
    
    \includegraphics[width=0.2\textwidth]{example-image-a} % Placeholder for a logo
    
    \vspace{1cm}
    
    \small
    This report is confidential and intended solely for the use of the individual or entity to whom it is addressed.
    
\end{titlepage}

\fancypagestyle{plain}{
  \fancyhf{}
  \fancyfoot[C]{\thepage\ of \pageref{LastPage}}
  \renewcommand{\headrulewidth}{0pt}
  \renewcommand{\footrulewidth}{0.4pt}
}

\tableofcontents
\newpage

% --- Section 1: Executive Summary ---
\section{Executive Summary}
This report details the findings of a cybersecurity posture assessment for Signal Flare. The assessment combined a review of organizational security controls, an external network scan, and an analysis of pre-existing risks.

The overall security posture reveals a mix of significant strengths and critical weaknesses. On the positive side, the organization has demonstrated a strong commitment to identity security through the consistent implementation of Multi-Factor Authentication (MFA) across email, computer logins, and sensitive systems. Furthermore, the technical network scan of the target host \seqsplit{\texttt{192.168.1.100}} revealed no open ports, indicating a robust firewall configuration and a minimal external attack surface for that specific asset.

However, two critical administrative gaps were identified that introduce significant risk. The organization lacks a formal Employee Acceptable Use Policy (AUP), creating ambiguity and potential for misuse of company assets. Additionally, while new hires receive security training, there is no program for annual refresher training for all employees. This oversight allows security knowledge to become outdated, increasing susceptibility to evolving threats like phishing and social engineering.

This report provides a detailed breakdown of these findings and offers actionable recommendations to mitigate the identified risks and strengthen the organization's overall security posture.

\newpage

% --- Section 2: Organizational Information ---
\section{Organizational Information}
The following details were provided for the assessment.

\begin{tabular}{@{}ll}
\toprule
\textbf{Attribute} & \textbf{Value} \\
\midrule
Organization Name & Signal Flare \\
Email Domain & \texttt{SignalFlare.org} \\
Website Domain & \url{www.SignalFlare.org} \\
External IP Address & \seqsplit{\texttt{89.220.247.99}} \\
\bottomrule
\end{tabular}

% --- Section 3: Security Control Review ---
\section{Security Control Review}
A review of administrative and procedural security controls was conducted based on a questionnaire. The results below highlight areas of compliance and identify key policy gaps. A checkmark (\ding{51}) indicates a positive control is in place, while a cross (\ding{55}) indicates a gap.

\begin{table}[h!]
\centering
\begin{tabular}{@{}p{0.6\linewidth} c l@{}}
\toprule
\textbf{Control Question} & \textbf{Response} & \textbf{Assessment} \\
\midrule
Do you require MFA to access email? & \ding{51} & Compliant \\
Do you require MFA to log into computers? & \ding{51} & Compliant \\
Do you require MFA to access sensitive data systems? & \ding{51} & Compliant \\
Does your organization have an employee acceptable use policy? & \ding{55} & \sevCRITICAL{} Gap \\
Does your organization do security awareness training for new employees? & \ding{51} & Compliant \\
Does your organization do security awareness training for all employees at least once per year? & \ding{55} & \sevHIGH{} Risk \\
\bottomrule
\end{tabular}
\caption{Organizational Security Control Status}
\end{label{tab:controls}
\end{table}

% --- Section 4: Technical Scan Results ---
\section{Technical Scan Results}
An external network scan was performed to identify exposed services and potential vulnerabilities on the perimeter.

\subsection{Scan Summary}
\begin{itemize}
    \item \textbf{Target IP Address:} \seqsplit{\texttt{192.168.1.100}}
    \item \textbf{Scan Result:} \textbf{No open ports were detected.}
\end{itemize}

\subsection{Analysis}
The scan indicates that all 1000 of the most common TCP ports on the target host are in a `closed` state. This is an excellent security posture for a network host, as it presents no listening services to an external attacker. This finding suggests that the host is either offline, not running any network services, or is protected by a very effective firewall that blocks all unsolicited inbound connections. This significantly reduces the attack surface of the scanned asset.

% --- Section 5: Risk Assessment Summary ---
\section{Risk Assessment Summary}
The following table synthesizes findings from the security control review and technical analysis. No pre-existing vulnerabilities were reported. The primary risks identified are administrative and procedural in nature.

\begin{table}[h!]
\centering
\begin{tabular}{@{}p{0.1\linewidth} p{0.3\linewidth} p{0.15\linewidth} p{0.35\linewidth}@{}}
\toprule
\textbf{Risk ID} & \textbf{Risk Name} & \textbf{Severity} & \textbf{Description} \\
\midrule
R-01 & Lack of Employee Acceptable Use Policy (AUP) & \sevCRITICAL & The absence of a formal AUP creates legal and security ambiguity regarding the use of company assets. This can lead to insider threats, data misuse, and difficulty in enforcing security standards. \\
\addlinespace
R-02 & No Annual Security Awareness Training & \sevHIGH & Without regular, recurring security training, employees are more likely to fall victim to evolving threats like phishing and social engineering, potentially leading to credential theft or malware infection. \\
\bottomrule
\end{tabular}
\caption{Summary of Identified Risks}
\label{tab:risks}
\end{table}

% --- Section 6: Recommendations ---
\section{Recommendations}
The following actionable recommendations are provided to address the risks identified in this report.

\subsection{R-01: Implement an Acceptable Use Policy}
\begin{itemize}
    \item \textbf{Severity:} \sevCRITICAL
    \item \textbf{Action:} Develop and formally implement a comprehensive Acceptable Use Policy (AUP) for all employees and contractors.
    \item \textbf{Details:} The policy should clearly define the rules for using company equipment, accessing the network, handling data, and using email and the internet. All current employees must review and acknowledge the policy, and it should be a mandatory part of the onboarding process for all new hires.
\end{itemize}

\subsection{R-02: Establish Annual Security Training Program}
\begin{itemize}
    \item \textbf{Severity:} \sevHIGH
    \item \textbf{Action:} Establish a mandatory annual security awareness training program for all employees.
    \item \textbf{Details:} This program should go beyond onboarding training and cover current and evolving cyber threats, including advanced phishing tactics, social engineering, and ransomware prevention. The effectiveness of the training should be measured, for example, through periodic, unannounced phishing simulations.
\end{itemize}

\end{document}
```