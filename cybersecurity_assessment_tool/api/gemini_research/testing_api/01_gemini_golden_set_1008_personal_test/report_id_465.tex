```latex
\documentclass[12pt]{article}

% Preamble: Required Packages
\usepackage[margin=1in]{geometry}
\usepackage{pifont} % For checkmarks and crosses
\usepackage{booktabs} % For professional tables
\usepackage[hidelinks]{hyperref} % For clickable links
\usepackage{url} % For URL formatting
\usepackage{seqsplit} % To split long strings in tt font
\usepackage{graphicx}
\usepackage{xcolor}

% Define colors for severity
\definecolor{criticalred}{HTML}{D7263D}
\definecolor{highorange}{HTML}{F49D0C}
\definecolor{mediumyellow}{HTML}{F4D03F}

% Document Information
\title{Cybersecurity Posture and Risk Assessment Report}
\author{Cybersecurity Analysis Division}
\date{\today}

\begin{document}

\maketitle
\thispagestyle{empty}
\newpage

\tableofcontents
\newpage

% --- 1. Executive Summary ---
\section{Executive Summary}

This report provides a comprehensive cybersecurity assessment for \textbf{Ember Glow Hospitality}. The analysis is based on a correlation of network scan data, organizational security control questionnaires, and a review of pre-existing risk documentation.

The assessment reveals several critical and high-risk security gaps. A network scan identified an openly accessible web interface on an internal system (\texttt{10.5.5.5:8080}) with the title \textbf{"TOP SECRET DB"}. This finding directly contradicts a previous risk assessment which incorrectly labeled this port as secure. This discrepancy points to a severe and immediate risk of sensitive data exposure.

Furthermore, significant procedural and policy-based weaknesses were identified. The organization lacks mandatory multi-factor authentication (MFA) for computer logins, an employee acceptable use policy, and a security awareness training program. These gaps substantially increase the organization's susceptibility to credential theft, insider threats, and social engineering attacks.

Immediate remediation is required to secure the exposed database interface. Following this, a strategic effort must be undertaken to implement foundational security controls, including MFA, policy development, and employee training, to build a more resilient security posture.

% --- 2. Organizational Information ---
\section{Organizational Information}

This section details the organizational data provided for this assessment.

\begin{itemize}
    \item \textbf{Organization Name:} Ember Glow Hospitality
    \item \textbf{Email Domain:} \seqsplit{\texttt{EmberGlowHospitality.org}}
    \item \textbf{Website Domain:} \seqsplit{\texttt{www.EmberGlowHospitality.org}}
    \item \textbf{Known External IP:} \texttt{183.30.24.122}
\end{itemize}

% --- 3. Security Control Review ---
\section{Security Control Review}

The following table summarizes the organization's responses to a security controls questionnaire. Each "No" response, marked with \ding{55}, represents a significant gap in the organization's defensive posture.

\begin{table}[h!]
\centering
\caption{Security Controls Questionnaire Analysis}
\begin{tabular}{p{0.7\linewidth} c}
\toprule
\textbf{Control Question} & \textbf{Status} \\
\midrule
Do you require MFA to access email? & \ding{51} \\
\textbf{Do you require MFA to log into computers?} & \textcolor{criticalred}{\ding{55}} \\
Do you require MFA to access sensitive data systems? & \ding{51} \\
\textbf{Does your organization have an employee acceptable use policy?} & \textcolor{criticalred}{\ding{55}} \\
\textbf{Does your organization do security awareness training for new employees?} & \textcolor{criticalred}{\ding{55}} \\
\textbf{Does your organization do security awareness training for all employees at least once per year?} & \textcolor{criticalred}{\ding{55}} \\
\bottomrule
\end{tabular}
\end{table}

\subsection*{Analysis of Control Gaps}
\begin{itemize}
    \item \textbf{No MFA for Computer Logins:} The absence of MFA on endpoints is a critical vulnerability. If an employee's password is stolen, an attacker can gain direct access to their workstation and potentially the internal network.
    \item \textbf{Lack of Acceptable Use Policy (AUP):} Without a formal AUP, there are no clear guidelines for employees on the proper use of company assets. This increases the risk of misuse, data leakage, and legal liability.
    \item \textbf{No Security Awareness Training:} The complete absence of a security training program leaves employees uninformed about modern threats like phishing and social engineering, making them highly susceptible to attacks. This is a foundational failure in building a security-conscious culture.
\end{itemize}

% --- 4. Technical Scan Results ---
\section{Technical Scan Results}

A network scan was performed to identify active services and potential vulnerabilities on the specified target system.

\begin{itemize}
    \item \textbf{Scan Date:} \today
    \item \textbf{Target IP Address:} \texttt{10.5.5.5}
\end{itemize}

\begin{table}[h!]
\centering
\caption{Open Ports and Services on \texttt{10.5.5.5}}
\begin{tabular}{l l p{0.5\linewidth}}
\toprule
\textbf{Port} & \textbf{State} & \textbf{Service Information / Banner} \\
\midrule
8080/tcp & Open & HTTP service with the title: \textbf{"TOP SECRET DB"} \\
\bottomrule
\end{tabular}
\end{table}

\subsection*{Analysis of Technical Findings}
The scan revealed a highly alarming finding. Port 8080 is open and hosts a web service explicitly titled \textbf{"TOP SECRET DB"}. This strongly suggests an exposed database or administration interface containing highly sensitive information. This finding is made more severe by its direct contradiction with the existing risk documentation (\textit{Input\_3\_Current\_Risks\_JSON}), which stated this port was a "confirmed secure" false positive. This indicates a failure in the previous risk assessment process and an active, unmitigated critical threat.

% --- 5. Correlated Risk Assessment ---
\section{Correlated Risk Assessment}

This section synthesizes findings from the security questionnaire, technical scans, and existing risk data to present a holistic view of the current risk landscape.

\begin{table}[h!]
\centering
\caption{Summary of Identified Risks}
\begin{tabular}{p{0.25\linewidth} p{0.55\linewidth} l}
\toprule
\textbf{Risk Name} & \textbf{Description} & \textbf{Severity} \\
\midrule
\textbf{Exposed Sensitive Database Interface} & An open port (8080) on an internal server (\texttt{10.5.5.5}) is broadcasting a web page titled "TOP SECRET DB". This contradicts previous assessments and presents an immediate threat of a major data breach. & \textcolor{criticalred}{\textbf{Critical}} \\
\addlinespace
\textbf{Lack of Endpoint Access Control} & The absence of MFA for computer logins means that a single compromised password could grant an attacker full access to an employee's workstation and the internal network. & \textcolor{highorange}{\textbf{High}} \\
\addlinespace
\textbf{Untrained Workforce} & With no security awareness training, employees are highly vulnerable to phishing and social engineering, which are the primary vectors for initial network compromise. & \textcolor{highorange}{\textbf{High}} \\
\addlinespace
\textbf{Absence of Governance Policies} & The lack of a formal Acceptable Use Policy creates ambiguity for employees and exposes the organization to insider threats and potential legal non-compliance. & \textcolor{highorange}{\textbf{High}} \\
\bottomrule
\end{tabular}
\end{table}

% --- 6. Recommendations ---
\section{Recommendations}

The following actions are recommended to mitigate the identified risks. They are prioritized based on severity and potential impact.

\subsection*{Immediate Actions (Within 24 Hours)}
\begin{enumerate}
    \item \textbf{Isolate and Investigate Exposed Database:} Immediately apply a firewall rule to block all access to \texttt{10.5.5.5:8080}. Launch an urgent investigation to identify the system's owner, the nature of the data it contains, and review its access logs for signs of compromise.
    \item \textbf{Review Risk Assessment Process:} The failure to correctly identify the risk on port 8080 indicates a flawed validation process. A full review of how risks are assessed and validated must be conducted.
\end{enumerate}

\subsection*{High-Priority Actions (Next 30-90 Days)}
\begin{enumerate}
    \item \textbf{Deploy Endpoint MFA:} Implement and enforce Multi-Factor Authentication for all employee computer and laptop logins.
    \item \textbf{Establish Security Awareness Training Program:} Procure and deploy a security awareness training solution. All employees must complete an initial training module covering phishing, password hygiene, and data handling, with mandatory annual refreshers.
    \item \textbf{Develop and Implement an AUP:} Draft a formal Acceptable Use Policy that clearly defines the rules for using company technology and data. Require all employees to read and formally acknowledge the policy.
\end{enumerate}

\subsection*{Long-Term Recommendations}
\begin{enumerate}
    \item \textbf{Conduct Comprehensive Vulnerability Scanning:} Implement a regular, authenticated vulnerability scanning program for the entire internal network to proactively identify misconfigurations and outdated software.
    \item \textbf{Develop an Incident Response Plan:} Formalize a plan that details the steps to be taken in the event of a security breach to ensure a swift and effective response.
\end{enumerate}

\end{document}
```