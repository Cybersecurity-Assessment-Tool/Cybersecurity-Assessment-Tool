```latex
\documentclass[12pt]{article}

% Preamble: Required Packages
\usepackage[margin=1in]{geometry}
\usepackage{pifont} % For checkmarks and crosses
\usepackage{booktabs} % For professional tables
\usepackage{hyperref} % For clickable links and ToC
\usepackage{url} % For formatting URLs
\usepackage{seqsplit} % For splitting long text strings like versions
\usepackage{graphicx} % For potential logos (not used here but good practice)
\usepackage{xcolor} % For colors

% Hyperlink Setup
\hypersetup{
    colorlinks=true,
    linkcolor=black,
    urlcolor=blue,
    pdftitle={Cybersecurity Posture Assessment Report},
    pdfauthor={Cybersecurity Analyst},
}

% Document Title Information
\title{Cybersecurity Posture Assessment Report \\ \large For: \textbf{Modern Myth}}
\author{Cybersecurity Analyst}
\date{November 22, 2025}

\begin{document}

\maketitle
\thispagestyle{empty}
\newpage

\tableofcontents
\newpage

% --- Section 1: Executive Overview ---
\section{Executive Overview}

This report details the findings of a cybersecurity posture assessment conducted for \textbf{Modern Myth}. The assessment synthesizes data from an external network scan, a self-reported security controls questionnaire, and a review of pre-existing risks. The objective is to identify security gaps, assess their potential impact, and provide actionable recommendations to enhance the organization's security posture.

The assessment identified several areas of significant concern. The most critical findings relate to deficiencies in access control, specifically the absence of Multi-Factor Authentication (MFA) for computer logins and access to sensitive data systems. These gaps expose the organization to a high risk of unauthorized access and potential data breaches.

Furthermore, a technical scan of the internal network revealed a web server running an outdated and vulnerable version of Nginx software. Procedural gaps were also noted, including the lack of mandatory security awareness training for new employees during their onboarding process.

While the organization has implemented some foundational security controls, such as MFA for email and an acceptable use policy, the identified vulnerabilities require immediate attention to mitigate substantial risks. This report provides a detailed breakdown of these findings and a prioritized list of recommendations.

% --- Section 2: Organizational Information ---
\section{Organizational Information}

The following details were provided for the assessment. This information establishes the context and scope of the review.

\begin{itemize}
    \item \textbf{Organization Name:} Modern Myth
    \item \textbf{Email Domain:} \texttt{ModernMyth.com}
    \item \textbf{Primary Website:} \url{www.ModernMyth.com}
    \item \textbf{Known External IP:} \texttt{105.253.155.235}
\end{itemize}

% --- Section 3: Security Control Review ---
\section{Security Control Review}

A security controls questionnaire was completed by the organization to provide insight into its current policies and procedures. The responses are summarized below. Answers marked with a cross (\ding{55}) indicate significant control gaps that increase organizational risk.

\begin{table}[h!]
\centering
\caption{Security Controls Questionnaire Responses}
\begin{tabular}{p{0.8\linewidth} c}
\toprule
\textbf{Control Question} & \textbf{Response} \\
\midrule
Do you require MFA to access email? & \textcolor{green}{\ding{51}} \\
Do you require MFA to log into computers? & \textcolor{red}{\ding{55}} \\
Do you require MFA to access sensitive data systems? & \textcolor{red}{\ding{55}} \\
Does your organization have an employee acceptable use policy? & \textcolor{green}{\ding{51}} \\
Does your organization do security awareness training for new employees? & \textcolor{red}{\ding{55}} \\
Does your organization do security awareness training for all employees at least once per year? & \textcolor{green}{\ding{51}} \\
\bottomrule
\end{tabular}
\end{table}

\subsection*{Analysis of Control Gaps}
The questionnaire reveals critical weaknesses in the organization's identity and access management (IAM) framework. The lack of MFA on computer logins and, more importantly, on sensitive data systems, dramatically lowers the barrier for an attacker with compromised credentials to gain widespread access. Additionally, the failure to provide security training to new employees creates a window of vulnerability, as new hires are often targeted by social engineering attacks.

% --- Section 4: Technical Scan Results ---
\section{Technical Scan Results}

A network scan was performed to identify active services and potential vulnerabilities on the organization's network infrastructure.

\begin{itemize}
    \item \textbf{Target IP Address:} \texttt{192.168.10.5}
    \item \textbf{Scan Date:} November 22, 2025
\end{itemize}

\begin{table}[h!]
\centering
\caption{Open Ports and Services on \texttt{192.168.10.5}}
\begin{tabular}{llll}
\toprule
\textbf{Port} & \textbf{State} & \textbf{Service} & \textbf{Product \& Version} \\
\midrule
443/tcp & open & https & \seqsplit{\texttt{nginx 1.18.0}} \\
\bottomrule
\end{tabular}
\end{table}

\subsection*{Technical Analysis}
The scan identified one open port, 443 (HTTPS), which is being served by Nginx version 1.18.0.

\begin{itemize}
    \item \textbf{Outdated Software:} Nginx version 1.18.0 was released in April 2020. This version is outdated and no longer receives security patches. It is known to be vulnerable to several security exploits (e.g., CVE-2021-23017), which could allow an attacker to gain control of the web server.
    \item \textbf{SSL Certificate Mismatch:} The SSL certificate presented by the server has a Common Name of \texttt{www.acme-corp.com}. This does not match the organization's domain (\texttt{ModernMyth.com}). This misconfiguration can cause browser trust errors and may indicate that an incorrect or temporary certificate is in use.
\end{itemize}

% --- Section 5: Risk Assessment ---
\section{Risk Assessment}

The following table summarizes the key risks identified by correlating the security control gaps, technical findings, and pre-existing vulnerability data. Each risk has been assigned a severity level based on its potential impact and likelihood of exploitation.

\begin{table}[h!]
\centering
\caption{Summary of Identified Risks}
\begin{tabular}{p{0.1\linewidth} p{0.25\linewidth} p{0.15\linewidth} p{0.4\linewidth}}
\toprule
\textbf{Risk ID} & \textbf{Risk Name} & \textbf{Severity} & \textbf{Description} \\
\midrule
RISK-001 & No MFA on Sensitive Data Systems & \textbf{Critical} & Lack of MFA on critical systems allows an attacker with stolen credentials to directly access and exfiltrate sensitive data. \\
\addlinespace
RISK-002 & Outdated Web Server Software & \textbf{High} & The Nginx server is running a version with known vulnerabilities, exposing it to remote code execution or denial-of-service attacks. \\
\addlinespace
RISK-003 & No MFA on Endpoint Logins & \textbf{High} & The absence of MFA on computer logins allows for lateral movement within the network if a user's credentials are compromised. \\
\addlinespace
RISK-004 & No Security Training for New Hires & \textbf{Medium} & New employees are not trained on security policies, making them more susceptible to phishing and social engineering attacks. \\
\addlinespace
RISK-005 & SSL Certificate Misconfiguration & \textbf{Low} & The certificate mismatch on the web server can erode user trust and may indicate other underlying configuration issues. \\
\bottomrule
\end{tabular}
\end{table}

% --- Section 6: Recommendations ---
\section{Recommendations}

The following actions are recommended to mitigate the identified risks and strengthen the overall security posture of \textbf{Modern Myth}. Recommendations are prioritized based on risk severity.

\begin{description}
    \item[\textbf{RISK-001 (Critical):}] \textbf{Implement MFA for Sensitive Systems.} Immediately deploy a mandatory MFA solution for all user and administrative access to systems containing sensitive or critical data. This is the highest priority action.

    \item[\textbf{RISK-002 (High):}] \textbf{Upgrade Nginx Server.} Plan and execute the upgrade of the Nginx server at \texttt{192.168.10.5} from version 1.18.0 to the latest stable version. Ensure a backup is performed before the upgrade and that all configurations are tested post-upgrade.

    \item[\textbf{RISK-003 (High):}] \textbf{Enforce MFA for Endpoint Logins.} Roll out mandatory MFA for all employee and contractor logins to company-managed computers (desktops and laptops).

    \item[\textbf{RISK-004 (Medium):}] \textbf{Establish Onboarding Security Training.} Develop and integrate a mandatory security awareness training module into the new employee onboarding process. This training should cover acceptable use, phishing identification, and incident reporting.

    \item[\textbf{RISK-005 (Low):}] \textbf{Correct SSL Certificate.} Investigate the SSL certificate mismatch on the web server at \texttt{192.168.10.5}. Procure and install a valid certificate that matches the organization's domain name.
\end{description}

\end{document}
```