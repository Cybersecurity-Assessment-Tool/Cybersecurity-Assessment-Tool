```latex
\documentclass[12pt]{article}

% Preamble: Required Packages
\usepackage[margin=1in]{geometry}
\usepackage{pifont} % For checkmarks and crosses
\usepackage{booktabs} % For professional tables
\usepackage{hyperref} % For clickable links
\usepackage{url} % For URL formatting
\usepackage{seqsplit} % To split long strings in texttt
\usepackage{graphicx}
\usepackage{xcolor}
\usepackage{fancyhdr}

% Define colors for risk levels
\definecolor{critical}{HTML}{990000}
\definecolor{high}{HTML}{D14302}
\definecolor{medium}{HTML}{E5A800}
\definecolor{low}{HTML}{3A7D44}

% Setup Hyperref
\hypersetup{
    colorlinks=true,
    linkcolor=blue,
    filecolor=magenta,      
    urlcolor=cyan,
    pdftitle={Cybersecurity Posture Report},
    pdfpagemode=FullScreen,
}

% Header and Footer
\pagestyle{fancy}
\fancyhf{}
\fancyhead[L]{Cybersecurity Posture Report}
\fancyhead[R]{Gilded Cage Design}
\fancyfoot[C]{\thepage}

\begin{document}

% --- Title Page ---
\begin{titlepage}
    \centering
    \vspace*{1cm}
    
    \includegraphics[width=0.4\textwidth]{example-image-a} % Placeholder for company logo
    
    \vspace{1.5cm}
    
    {\Huge\bfseries Cybersecurity Posture Report\par}
    
    \vspace{1cm}
    
    {\Large Prepared for: Gilded Cage Design\par}
    
    \vspace{2cm}
    
    {\large \today\par}
    
    \vfill
    
    \textit{This report contains sensitive information and should be handled with care. Access is restricted to authorized personnel only.}
    
\end{titlepage}

\tableofcontents
\newpage

% --- 1. Executive Summary ---
\section{Executive Summary}

This report provides a comprehensive analysis of the cybersecurity posture for Gilded Cage Design, based on a combination of technical network scanning, a review of organizational security controls, and an assessment of known risks.

The external network scan of the target host (\texttt{192.168.1.100}) revealed a strong perimeter defense, with no open ports detected. This indicates a well-configured firewall and a minimal attack surface from a network perspective, which is a significant strength.

However, the review of organizational security controls identified several critical and high-risk gaps in policy and procedure. The most pressing issues are the lack of multi-factor authentication (MFA) for sensitive data systems, the absence of an employee acceptable use policy, and the omission of security awareness training for new hires. These gaps expose the organization to significant risks, including unauthorized data access, insider threats, and susceptibility to social engineering attacks.

Immediate action is recommended to address these procedural and policy-based vulnerabilities to complement the existing technical strengths and establish a more resilient security posture.

% --- 2. Organizational Information ---
\section{Organizational Information}
The following details were provided for the assessment.

\begin{tabular}{@{}ll}
    \toprule
    \textbf{Attribute} & \textbf{Value} \\
    \midrule
    Organization Name & Gilded Cage Design \\
    Email Domain & \texttt{GildedCageDesign.net} \\
    Website Domain & \url{www.GildedCageDesign.net} \\
    External IP Address & \texttt{129.7.84.58} \\
    \bottomrule
\end{tabular}

% --- 3. Security Control Review ---
\section{Security Control Review}
A review of key administrative and technical security controls was conducted via a questionnaire. The results are summarized below. Items marked with \ding{55} represent significant gaps in the current security framework.

\begin{tabular}{@{}p{0.7\linewidth}c}
    \toprule
    \textbf{Control Question} & \textbf{Status} \\
    \midrule
    Do you require MFA to access email? & \ding{51} \\
    Do you require MFA to log into computers? & \ding{51} \\
    \textbf{Do you require MFA to access sensitive data systems?} & \textcolor{critical}{\ding{55}} \\
    \textbf{Does your organization have an employee acceptable use policy?} & \textcolor{high}{\ding{55}} \\
    \textbf{Does your organization do security awareness training for new employees?} & \textcolor{high}{\ding{55}} \\
    Does your organization do security awareness training for all employees at least once per year? & \ding{51} \\
    \bottomrule
\end{tabular}

\subsection{Analysis of Gaps}
\begin{itemize}
    \item \textbf{MFA for Sensitive Systems:} The absence of MFA on systems containing sensitive data is a critical vulnerability. This control is essential for preventing unauthorized access in the event of credential compromise.
    \item \textbf{Acceptable Use Policy (AUP):} Lacking a formal AUP means there are no clear, enforceable rules for employees regarding the use of company systems and data. This increases the risk of misuse, data leakage, and legal liability.
    \item \textbf{New Employee Training:} While annual training is in place, failing to train new employees upon hiring leaves a window of vulnerability. New staff may be unaware of company policies and are often prime targets for social engineering attacks.
\end{itemize}

% --- 4. Technical Scan Results ---
\section{Technical Scan Results}
A network port scan was performed to identify accessible services on the specified target system.

\begin{tabular}{@{}ll}
    \toprule
    \textbf{Scan Parameter} & \textbf{Value} \\
    \midrule
    Target IP Address & \texttt{192.168.1.100} \\
    Host Status & Up \\
    Open Ports Found & 0 \\
    All Other Ports & Closed \\
    \bottomrule
\end{tabular}

\subsection{Findings}
The scan confirmed that the host at \texttt{192.168.1.100} is online but has no open TCP ports. This is a positive security finding, suggesting that the host is either not running any externally-facing services or is protected by a properly configured firewall that denies all inbound traffic. This significantly reduces the external attack surface of the scanned asset.

% --- 5. Risk Assessment Summary ---
\section{Risk Assessment Summary}
The following table synthesizes findings from the security control review, technical scan, and pre-existing risk data. The risks are prioritized based on their potential impact on the organization.

\begin{tabular}{@{}p{0.3\linewidth}p{0.5\linewidth}l}
    \toprule
    \textbf{Risk Name} & \textbf{Overview} & \textbf{Severity} \\
    \midrule
    No MFA on Sensitive Systems & The lack of mandatory MFA for accessing sensitive data systems exposes critical assets to unauthorized access if user credentials are stolen or compromised. & \textcolor{critical}{\textbf{Critical}} \\
    \addlinespace
    No Employee Acceptable Use Policy & Without a formal AUP, the organization cannot enforce secure behavior, leading to a higher risk of data misuse, policy violations, and insider threats. & \textcolor{high}{\textbf{High}} \\
    \addlinespace
    No Security Training for New Hires & New employees are not trained on security best practices during onboarding, making them highly susceptible to phishing, social engineering, and unintentional policy violations. & \textcolor{high}{\textbf{High}} \\
    \bottomrule
\end{tabular}

\textit{Note: The pre-existing risk register (Input 3) reported no known vulnerabilities at the time of this assessment.}

% --- 6. Recommendations ---
\section{Recommendations}
Based on the analysis, the following actions are recommended to mitigate the identified risks and improve the overall security posture of Gilded Cage Design.

\subsection{Critical Priority}
\begin{enumerate}
    \item \textbf{Implement MFA for All Sensitive Data Systems:}
    \begin{itemize}
        \item \textbf{Action:} Immediately deploy and enforce multi-factor authentication (MFA) on all applications, databases, and administrative interfaces that process or store sensitive company or customer data.
        \item \textbf{Impact:} Drastically reduces the risk of unauthorized access due to compromised credentials. This is the single most effective control to implement.
    \end{itemize}
\end{enumerate}

\subsection{High Priority}
\begin{enumerate}
    \setcounter{enumi}{1}
    \item \textbf{Develop and Implement an Acceptable Use Policy (AUP):}
    \begin{itemize}
        \item \textbf{Action:} Draft a formal AUP that clearly defines the rules for using company IT assets, data handling, internet usage, and email. This policy should be reviewed by legal counsel, communicated to all employees, and formally acknowledged via signature.
        \item \textbf{Impact:} Establishes a clear baseline for secure behavior, reduces insider risk, and provides a framework for disciplinary action in case of violations.
    \end{itemize}
    \item \textbf{Integrate Security Training into Employee Onboarding:}
    \begin{itemize}
        \item \textbf{Action:} Create a mandatory security awareness training module for all new hires to be completed within their first week of employment. This training should cover key topics such as phishing, password security, data handling, and the new AUP.
        \item \textbf{Impact:} Ensures that all employees, from day one, have a foundational understanding of security risks and their responsibilities in protecting the organization.
    \end{itemize}
\end{enumerate}

\end{document}
```