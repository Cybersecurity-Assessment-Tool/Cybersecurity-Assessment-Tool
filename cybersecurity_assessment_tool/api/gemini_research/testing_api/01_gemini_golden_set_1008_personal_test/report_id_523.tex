```latex
\documentclass[12pt]{article}

% Preamble: Required Packages
\usepackage[margin=1in]{geometry}
\usepackage{pifont} % For \ding
\usepackage{booktabs} % For professional tables (\toprule, \midrule, \bottomrule)
\usepackage{hyperref} % For clickable links and metadata
\usepackage{url} % For formatting URLs
\usepackage{seqsplit} % To split long strings without breaking words
\usepackage{graphicx} % For potential logos
\usepackage{xcolor} % For colors

% Document Metadata
\hypersetup{
    colorlinks=true,
    linkcolor=blue,
    filecolor=magenta,      
    urlcolor=cyan,
    pdftitle={Cybersecurity Posture Assessment Report},
    pdfauthor={Cybersecurity Analyst},
    pdfsubject={Security Analysis},
    pdfkeywords={Cybersecurity, Risk, Assessment},
    pdftoolbar=true,
}

% Define custom colors for severity
\definecolor{criticalred}{HTML}{D7263D}
\definecolor{highorange}{HTML}{F49D40}
\definecolor{mediumyellow}{HTML}{F4E409}
\definecolor{lowblue}{HTML}{2A75B3}

% Check and Cross symbols
\newcommand{\cmark}{\ding{51}}
\newcommand{\xmark}{\ding{55}}

% --- DOCUMENT START ---
\begin{document}

% --- TITLE PAGE ---
\begin{titlepage}
    \centering
    \vspace*{1cm}
    \Huge\textbf{Cybersecurity Posture Assessment Report}
    \vspace{1.5cm}
    \Large
    \textbf{Prepared for:}\\
    Hidden Gem
    \vspace{2cm}
    \rule{\linewidth}{0.5mm}
    \vspace{0.5cm}
    \large
    \textbf{Date of Report:} \today \\
    \textbf{Report ID:} CYBER-2023-001
    \rule{\linewidth}{0.5mm}
    \vfill
    \large
    \textbf{CONFIDENTIAL} \\
    \textit{This document contains sensitive information and is intended for the exclusive use of the recipient organization. Unauthorized distribution is prohibited.}
\end{titlepage}

\tableofcontents
\newpage

% --- EXECUTIVE SUMMARY ---
\section{Executive Summary}

This report details the findings of a cybersecurity posture assessment for \textbf{Hidden Gem}. The assessment combined a review of organizational security controls, an external network scan, and an analysis of known risks.

The overall security posture requires significant improvement. Several critical and high-risk gaps were identified, primarily in administrative and policy-based controls. Key findings include:

\begin{itemize}
    \item \textbf{Critical Multi-Factor Authentication (MFA) Gaps:} MFA is not enforced for computer logins or access to sensitive data systems. This exposes the organization to significant risk from credential compromise.
    \item \textbf{Inadequate Policy and Training Framework:} The organization lacks a formal Acceptable Use Policy and does not conduct annual security awareness training for all employees, increasing the likelihood of human error leading to a security incident.
    \item \textbf{Exposed Administrative Service:} An external scan identified an open SSH port (\texttt{22/tcp}) on the IPv6 address \seqsplit{\texttt{2001:db8::1}}. While necessary for remote administration, public exposure requires stringent security configurations to prevent unauthorized access.
\end{itemize}

Immediate action is recommended to address the identified MFA gaps. Further recommendations are provided to remediate all findings and strengthen the organization's defense-in-depth strategy.

% --- ORGANIZATIONAL INFORMATION ---
\section{Organizational Information}

The following details were provided for the assessment.

\begin{table}[h!]
\centering
\begin{tabular}{@{}ll@{}}
\toprule
\textbf{Attribute} & \textbf{Value} \\ \midrule
Organization Name & Hidden Gem \\
Email Domain      & \texttt{HiddenGem.org} \\
Website Domain    & \url{www.HiddenGem.org} \\
External IP (IPv4) & \texttt{133.14.188.15} \\ 
Scanned IP (IPv6) & \seqsplit{\texttt{2001:db8::1}} \\
\bottomrule
\end{tabular}
\caption{Client Organizational Details.}
\end{table}

% --- SECURITY CONTROL REVIEW ---
\section{Security Control Review}

A review of administrative security controls was conducted based on a standardized questionnaire. The responses highlight several areas of concern where security best practices are not being met.

\begin{table}[h!]
\centering
\begin{tabular}{@{}p{0.6\linewidth} c l@{}}
\toprule
\textbf{Control Question} & \textbf{Response} & \textbf{Assessment} \\ \midrule
Do you require MFA to access email? & \cmark & Good Practice \\
Do you require MFA to log into computers? & \xmark & \textbf{Critical Gap} \\
Do you require MFA to access sensitive data systems? & \xmark & \textbf{Critical Gap} \\
\addlinespace
Does your organization have an employee acceptable use policy? & \xmark & \textbf{High Risk} \\
\addlinespace
Does your organization do security awareness training for new employees? & \cmark & Good Practice \\
Does your organization do security awareness training for all employees at least once per year? & \xmark & \textbf{High Risk} \\
\bottomrule
\end{tabular}
\caption{Security Controls Questionnaire Analysis.}
\end{table}

% --- TECHNICAL SCAN RESULTS ---
\section{Technical Scan Results}

An external network reconnaissance scan was performed against the target IP address. The scan identified the following open port.

\begin{itemize}
    \item \textbf{Target IP Address:} \seqsplit{\texttt{2001:db8::1}}
    \item \textbf{Scan Date:} Not specified in scan data.
\end{itemize}

\begin{table}[h!]
\centering
\begin{tabular}{@{}llll@{}}
\toprule
\textbf{Port} & \textbf{State} & \textbf{Service (Presumed)} & \textbf{Notes} \\ \midrule
22/tcp & Open & SSH & Exposed administrative service. \\
\bottomrule
\end{tabular}
\caption{Open Ports Identified on Target Host.}
\end{table}

\paragraph{Analysis:} The presence of an open SSH port is common for remote system administration. However, when exposed to the public internet, it becomes a primary target for brute-force attacks and exploitation of potential vulnerabilities. Secure configuration (e.g., disabling password authentication, using fail2ban, and IP whitelisting) is essential.

% --- RISK ASSESSMENT ---
\section{Risk Assessment Summary}

The following table synthesizes findings from the security control review and technical scan into a prioritized list of risks. No pre-existing vulnerabilities were provided for this assessment.

\begin{table}[h!]
\centering
\begin{tabular}{@{}lp{0.3\linewidth}p{0.4\linewidth}l@{}}
\toprule
\textbf{ID} & \textbf{Risk Title} & \textbf{Description} & \textbf{Severity} \\ \midrule
\textbf{RISK-001} & Lack of MFA on Critical Systems & The absence of MFA on computer and sensitive data system access allows an attacker with stolen credentials to gain unauthorized entry. & \textcolor{criticalred}{\textbf{Critical}} \\
\addlinespace
\textbf{RISK-002} & Inadequate Security Policies and Training & The lack of an Acceptable Use Policy and annual training leaves employees without clear guidelines and vulnerable to social engineering. & \textcolor{highorange}{\textbf{High}} \\
\addlinespace
\textbf{RISK-003} & Exposed Administrative Service & The SSH service on \seqsplit{\texttt{2001:db8::1}} is open to the internet, increasing its exposure to automated attacks and unauthorized access attempts. & \textcolor{mediumyellow}{\textbf{Medium}} \\
\bottomrule
\end{tabular}
\caption{Identified Risk Summary.}
\end{table}

% --- RECOMMENDATIONS ---
\section{Recommendations}

To mitigate the identified risks and improve the overall security posture, the following actions are recommended, prioritized by severity.

\subsection{Immediate Priority (Critical Risks)}
\begin{enumerate}
    \item \textbf{Implement Comprehensive MFA (RISK-001):}
    \begin{itemize}
        \item Enforce mandatory MFA for all user logins to company computers (desktops and laptops).
        \item Enforce mandatory MFA for access to all systems classified as containing sensitive or critical data (e.g., databases, file servers, financial applications).
    \end{itemize}
\end{enumerate}

\subsection{High Priority}
\begin{enumerate}
    \setcounter{enumi}{1}
    \item \textbf{Develop and Implement Security Policies (RISK-002):}
    \begin{itemize}
        \item Create a formal Acceptable Use Policy (AUP) that clearly defines the rules for using company IT assets.
        \item Require all employees to read and acknowledge the AUP upon hiring and annually thereafter.
    \end{itemize}
    \item \textbf{Establish a Security Training Program (RISK-002):}
    \begin{itemize}
        \item Institute a mandatory, annual security awareness training program for all employees.
        \item Training should cover modern threats such as phishing, ransomware, and social engineering.
    \end{itemize}
    \item \textbf{Secure the Exposed SSH Service (RISK-003):}
    \begin{itemize}
        \item Review the business need for public-facing SSH on \seqsplit{\texttt{2001:db8::1}}. If possible, place it behind a VPN or bastion host.
        \item If it must remain public, enforce key-based authentication and disable password-based logins.
        \item Implement an intrusion prevention tool like \texttt{fail2ban} to block IPs after multiple failed login attempts.
        \item Restrict access to a whitelist of trusted source IP addresses.
    \end{itemize}
\end{enumerate}

% --- DOCUMENT END ---
\end{document}
```