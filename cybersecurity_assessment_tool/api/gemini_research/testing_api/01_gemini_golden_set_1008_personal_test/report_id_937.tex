```latex
\documentclass[12pt]{article}

% Preamble: Required Packages
\usepackage[margin=1in]{geometry}
\usepackage{pifont} % For \ding
\usepackage{booktabs} % For professional tables (\toprule, \midrule, \bottomrule)
\usepackage{hyperref} % For clickable links and metadata
\usepackage{url} % For formatting URLs
\usepackage{seqsplit} % To split long monospaced strings
\usepackage{xcolor} % For colors

% Hyperref Setup
\hypersetup{
    colorlinks=true,
    linkcolor=blue,
    filecolor=magenta,      
    urlcolor=cyan,
    pdftitle={Cybersecurity Posture Assessment Report},
    pdfauthor={Cybersecurity Analysis Division},
    pdfsubject={Security Assessment},
    pdfkeywords={Cybersecurity, Nmap, Risk Assessment},
}

% Document Title
\title{Cybersecurity Posture Assessment Report}
\author{Cybersecurity Analysis Division}
\date{November 22, 2025}

\begin{document}

\maketitle
\thispagestyle{empty}
\newpage
\tableofcontents
\newpage

\section*{Executive Summary}

This report provides a comprehensive cybersecurity posture assessment for \textbf{Midnight Oil Studios}, conducted on November 22, 2025. The analysis is based on a synthesis of network scan data, an organizational security questionnaire, and a review of pre-existing risks.

The assessment reveals several critical and high-risk security gaps that require immediate attention. The most significant findings include a complete lack of Multi-Factor Authentication (MFA) across all key systems (email, endpoints, and sensitive data access), and the presence of an externally-facing web server running outdated and vulnerable software (Nginx 1.18.0). Furthermore, the absence of a formal Acceptable Use Policy indicates a gap in foundational security governance.

While the organization has implemented security awareness training, the identified technical and policy-based vulnerabilities create a high-risk environment susceptible to unauthorized access, data breaches, and service disruption. This report outlines these risks in detail and provides actionable recommendations for remediation.

\section*{Organizational Information}

The following information was provided for the assessment. This data establishes the context and scope of the review.

\begin{tabular}{@{}ll}
\toprule
\textbf{Attribute} & \textbf{Value} \\
\midrule
Organization Name & \textbf{Midnight Oil Studios} \\
Email Domain & \texttt{MidnightOilStudios.com} \\
Website Domain & \url{www.MidnightOilStudios.com} \\
External IP Address & \texttt{138.223.69.238} \\
\bottomrule
\end{tabular}

\section*{Security Control Review}

A review of the organization's security controls was conducted via a questionnaire. The responses highlight significant gaps in access control measures. A green checkmark (\textcolor{green}{\ding{51}}) indicates a positive control, while a red cross (\textcolor{red}{\ding{55}}) indicates a control gap.

\begin{tabular}{@{}p{0.8\textwidth}c}
\toprule
\textbf{Control Question} & \textbf{Status} \\
\midrule
Do you require MFA to access email? & \textcolor{red}{\ding{55}} \\
Do you require MFA to log into computers? & \textcolor{red}{\ding{55}} \\
Do you require MFA to access sensitive data systems? & \textcolor{red}{\ding{55}} \\
Does your organization have an employee acceptable use policy? & \textcolor{red}{\ding{55}} \\
Does your organization do security awareness training for new employees? & \textcolor{green}{\ding{51}} \\
Does your organization do security awareness training for all employees at least once per year? & \textcolor{green}{\ding{51}} \\
\bottomrule
\end{tabular}

\section*{Technical Scan Results}

An external network scan was performed to identify exposed services and potential technical vulnerabilities.

\subsection*{Scan Metadata}
\begin{tabular}{@{}ll}
\toprule
\textbf{Attribute} & \textbf{Value} \\
\midrule
Scan Target IP & \texttt{192.168.10.5} \\
Scan Date & \texttt{2025-11-22T10:00:00Z} \\
\bottomrule
\end{tabular}

\subsection*{Open Ports and Services}
The following table details the services discovered on the target system.

\begin{tabular}{@{}lllll}
\toprule
\textbf{Port} & \textbf{State} & \textbf{Service} & \textbf{Product} & \textbf{Version} \\
\midrule
443/tcp & open & https & nginx & 1.18.0 \\
\bottomrule
\end{tabular}

\subsection*{Technical Analysis}
The scan identified a single open port, 443 (HTTPS), running an \textbf{Nginx 1.18.0} web server. This version was released in April 2020 and is now considered outdated and end-of-life. It contains multiple publicly known vulnerabilities, including but not limited to CVE-2021-23017, which can lead to request smuggling and security bypass. Running outdated software on an internet-facing system presents a high risk of compromise.

\section*{Consolidated Risk Assessment}

This section synthesizes findings from the security control review, technical scan, and pre-existing risk data. Since no pre-existing risks were provided, the following table is based entirely on new findings from this assessment.

\begin{tabular}{@{}p{0.1\textwidth} p{0.3\textwidth} p{0.15\textwidth} p{0.35\textwidth}}
\toprule
\textbf{ID} & \textbf{Risk Name} & \textbf{Severity} & \textbf{Description} \\
\midrule
\textbf{RISK-001} & Widespread Lack of Multi-Factor Authentication (MFA) & \textbf{Critical} & The absence of MFA for email, computer logins, and sensitive systems drastically increases the risk of unauthorized access via credential theft or phishing. A single compromised password could lead to a full-scale breach. \\
\addlinespace
\textbf{RISK-002} & Outdated and Vulnerable Web Server Software & \textbf{High} & The public-facing web server runs Nginx 1.18.0, which is end-of-life and has known, exploitable vulnerabilities. This exposes the organization to remote code execution, denial-of-service, and other attacks. \\
\addlinespace
\textbf{RISK-003} & Missing Acceptable Use Policy (AUP) & \textbf{Medium} & The lack of a formal AUP creates ambiguity for employees regarding the proper use of company assets. This increases the risk of insider threat (both malicious and accidental) and poses compliance challenges. \\
\bottomrule
\end{tabular}

\section*{Recommendations}

The following actions are recommended to mitigate the identified risks and improve the overall security posture of \textbf{Midnight Oil Studios}.

\subsection*{RISK-001: Implement Multi-Factor Authentication (Critical)}
\begin{itemize}
    \item \textbf{Immediate Action:} Enable MFA for all user accounts on the primary email system (e.g., Microsoft 365, Google Workspace).
    \item \textbf{Short-Term Action:} Deploy MFA for all remote access solutions (e.g., VPN) and endpoint logins (e.g., Windows Hello for Business, Duo).
    \item \textbf{Mid-Term Action:} Enforce MFA for all applications and systems that process or store sensitive data.
\end{itemize}

\subsection*{RISK-002: Upgrade Web Server Software (High)}
\begin{itemize}
    \item \textbf{Immediate Action:} Develop a patch management plan to upgrade the Nginx server at \texttt{192.168.10.5}.
    \item \textbf{Short-Term Action:} Upgrade Nginx from version 1.18.0 to a current, stable, and supported version (e.g., 1.26.x or newer). Before deployment, test the new version in a staging environment to ensure application compatibility.
    \item \textbf{Ongoing Action:} Implement a continuous vulnerability scanning and patch management program for all external-facing systems.
\end{itemize}

\subsection*{RISK-003: Develop Acceptable Use Policy (Medium)}
\begin{itemize}
    \item \textbf{Immediate Action:} Draft a formal Acceptable Use Policy (AUP) that clearly defines the rules for using company networks, devices, and data.
    \item \textbf{Short-Term Action:} Have the AUP reviewed by legal or HR, approved by management, and distributed to all employees. Require all employees to read and formally acknowledge the policy.
    \item \textbf{Ongoing Action:} Incorporate the AUP into the new-hire onboarding process and review it annually for necessary updates.
\end{itemize}

\end{document}
```