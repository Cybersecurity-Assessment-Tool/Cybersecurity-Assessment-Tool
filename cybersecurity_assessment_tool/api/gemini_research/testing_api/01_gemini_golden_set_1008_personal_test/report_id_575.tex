```latex
\documentclass[12pt]{article}

% Required Packages
\usepackage[margin=1in]{geometry}
\usepackage{pifont} % For checkmarks and crosses
\usepackage{booktabs} % For professional tables
\usepackage{hyperref} % For clickable links
\usepackage{url} % For formatting URLs
\usepackage{seqsplit} % For splitting long strings
\usepackage{graphicx} % For logo (placeholder)
\usepackage{fancyhdr} % For header/footer

% --- Document Metadata ---
\title{Cybersecurity Posture Assessment Report}
\author{Cybersecurity Analysis Division}
\date{\today}

% --- Header and Footer ---
\pagestyle{fancy}
\fancyhf{}
\lhead{Confidential Report for Oasis Wellness}
\rhead{\thepage}
\rfoot{\copyright~\the\year~Cybersecurity Analysis Division}

\begin{document}

\maketitle
\thispagestyle{empty}
\newpage

\tableofcontents
\newpage

% --- Section 1: Executive Summary ---
\section{Executive Summary}

This report provides a comprehensive cybersecurity posture assessment for \textbf{Oasis Wellness}, conducted on \today. The analysis synthesizes data from an external network scan, a review of internal security controls via a questionnaire, and a list of pre-existing risks.

The assessment reveals a mixed security posture. While the organization has implemented foundational controls such as security awareness training and multi-factor authentication (MFA) for email access, several critical and high-risk gaps were identified.

\textbf{Key Findings:}
\begin{itemize}
    \item \textbf{Critical Gaps in Access Control:} The lack of mandatory MFA for computer logins and, most importantly, for access to sensitive data systems, represents a critical vulnerability. This significantly increases the risk of unauthorized access and data breaches resulting from compromised credentials.
    \item \textbf{Policy Deficiencies:} The absence of an employee Acceptable Use Policy (AUP) creates ambiguity regarding secure practices and exposes the organization to insider threats, whether malicious or accidental.
    \item \textbf{Technical Exposures:} The network scan identified an open SSH port (22) on host \texttt{127.0.0.1}. This finding directly correlates with a pre-existing high-severity risk, "Localhost Exposed," indicating a potentially misconfigured or vulnerable service that requires immediate investigation.
\end{itemize}

Immediate remediation efforts should focus on implementing comprehensive MFA across all critical systems and workstations. Concurrently, the exposed SSH service must be investigated and secured, and a formal Acceptable Use Policy should be developed and enforced.

% --- Section 2: Organizational Information ---
\section{Organizational Information}

The following details were provided for the assessment. This information is used to establish the context and scope of the review.

\begin{tabular}{@{}ll}
    \toprule
    \textbf{Attribute} & \textbf{Value} \\
    \midrule
    Organization Name & \textbf{Oasis Wellness} \\
    Primary Email Domain & \texttt{OasisWellness.org} \\
    Primary Website & \url{www.OasisWellness.org} \\
    External IP Address & \texttt{21.220.212.179} \\
    \bottomrule
\end{tabular}

% --- Section 3: Security Control Review ---
\section{Security Control Review}

The following table summarizes the organization's responses to a security controls questionnaire. A \textcolor{green}{\ding{51}} indicates a positive control is in place, while a \textcolor{red}{\ding{55}} indicates a potential security gap.

\begin{table}[h!]
\centering
\begin{tabular}{@{}lc}
    \toprule
    \textbf{Security Control Question} & \textbf{Response} \\
    \midrule
    Do you require MFA to access email? & \textcolor{green}{\ding{51}} \\
    Do you require MFA to log into computers? & \textcolor{red}{\ding{55}} \\
    Do you require MFA to access sensitive data systems? & \textcolor{red}{\ding{55}} \\
    Does your organization have an employee acceptable use policy? & \textcolor{red}{\ding{55}} \\
    Does your organization do security awareness training for new employees? & \textcolor{green}{\ding{51}} \\
    Does your organization do security awareness training for all employees annually? & \textcolor{green}{\ding{51}} \\
    \bottomrule
\end{tabular}
\caption{Security Controls Questionnaire Results}
\end{table}

\subsection*{Analysis of Controls}
The questionnaire reveals significant weaknesses in access control and policy governance. The absence of MFA on workstations and sensitive systems dramatically lowers the barrier for an attacker with stolen credentials to move laterally within the network and access crown jewel data. The lack of an Acceptable Use Policy can lead to inconsistent security practices and a weakened defense against insider threats.

% --- Section 4: Technical Scan Results ---
\section{Technical Scan Results}

A network scan was performed to identify open ports and services visible on the target host. The results are detailed below.

\begin{table}[h!]
\centering
\begin{tabular}{@{}llll}
    \toprule
    \textbf{Target IP} & \textbf{Port} & \textbf{State} & \textbf{Inferred Service} \\
    \midrule
    \texttt{127.0.0.1} & 22 & open & SSH (Secure Shell) \\
    \bottomrule
\end{tabular}
\caption{Nmap Scan Findings}
\end{table}

\subsection*{Analysis of Scan Results}
The scan identified that port 22, commonly used for SSH, is open on the host \texttt{127.0.0.1}. This finding aligns with the pre-existing risk "Localhost Exposed" (CVSS 10.0), suggesting this is a known critical issue. An exposed SSH service, if not securely configured, can be a primary vector for unauthorized remote access. Further investigation is required to determine the service version and configuration to rule out specific vulnerabilities.

% --- Section 5: Consolidated Risk Assessment ---
\section{Consolidated Risk Assessment}

This section correlates findings from the security control review, technical scan, and pre-existing risk data to provide a unified view of the organization's risk profile.

\begin{table}[h!]
\centering
\begin{tabular}{@{}p{0.3\textwidth}p{0.5\textwidth}l}
    \toprule
    \textbf{Risk Title} & \textbf{Description} & \textbf{Severity} \\
    \midrule
    \textbf{No MFA on Sensitive Systems} & Lack of multi-factor authentication on systems storing sensitive data allows for single-factor credential compromise to lead directly to a major data breach. & \textbf{Critical} \\
    \addlinespace
    \textbf{Exposed SSH Service} & An open SSH port was detected on \texttt{127.0.0.1}, correlating with a known risk rated CVSS 10.0. This could allow for unauthorized remote system administration. & \textbf{Critical} \\
    \addlinespace
    \textbf{No MFA on Workstations} & Lack of MFA for computer logins allows an attacker with a user's password to easily gain a foothold on the internal network and escalate privileges. & \textbf{High} \\
    \addlinespace
    \textbf{No Acceptable Use Policy} & The absence of a formal AUP means employees lack clear guidelines for protecting company assets, increasing the risk of unintentional data exposure or system misuse. & \textbf{High} \\
    \bottomrule
\end{tabular}
\caption{Summary of Identified Risks}
\end{table}

% --- Section 6: Recommendations ---
\section{Recommendations}

Based on the consolidated risk assessment, the following actions are recommended to mitigate the identified vulnerabilities and improve the overall security posture of \textbf{Oasis Wellness}.

\subsection{Immediate Priority (Within 30 Days)}
\begin{enumerate}
    \item \textbf{Implement MFA on All Sensitive Systems:} Deploy a robust MFA solution for all applications and systems that contain sensitive or critical data. This is the single most effective control to prevent unauthorized access to key assets.
    \item \textbf{Investigate and Secure Exposed SSH Service:} Immediately investigate the SSH service running on \texttt{127.0.0.1}. Determine why it is exposed and, if not essential, disable it. If required, ensure it is hardened by disabling root login, using key-based authentication, and restricting access to authorized IP addresses.
\end{enumerate}

\subsection{High Priority (Within 90 Days)}
\begin{enumerate}
    \item \textbf{Enforce MFA for All Workstation Logins:} Extend MFA to all employee computer and laptop logins. This measure hardens endpoints and prevents unauthorized use of compromised credentials for initial network access.
    \item \textbf{Develop and Implement an Acceptable Use Policy (AUP):} Create a formal AUP that clearly defines the rules for using company technology and data. This policy should be communicated to all employees, and acknowledgement should be tracked.
\end{enumerate}

\subsection{Ongoing Recommendations}
\begin{enumerate}
    \item \textbf{Continue Security Awareness Training:} Maintain the existing security awareness training program to ensure that security remains a top-of-mind concern for all employees.
    \item \textbf{Conduct Regular Vulnerability Scanning:} Implement a program for regular, authenticated vulnerability scanning of all internal and external assets. This will help proactively identify and remediate technical vulnerabilities before they can be exploited.
\end{enumerate}

\end{document}
```