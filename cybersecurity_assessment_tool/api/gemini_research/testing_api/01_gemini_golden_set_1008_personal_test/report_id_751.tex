```latex
\documentclass[12pt]{article}

% Preamble: Required Packages
\usepackage[margin=1in]{geometry} % Set page margins
\usepackage{pifont}                 % For checkmarks and crosses (\ding)
\usepackage{booktabs}               % For professional-looking tables
\usepackage{hyperref}               % For clickable links and metadata
\usepackage{url}                    % For formatting URLs
\usepackage{seqsplit}               % For splitting long strings without spaces

% Document Metadata
\hypersetup{
    colorlinks=true,
    linkcolor=black,
    filecolor=magenta,      
    urlcolor=blue,
    pdftitle={Cybersecurity Posture Report},
    pdfauthor={Cybersecurity Analyst},
    pdfsubject={Security Assessment},
    pdfkeywords={Cybersecurity, Risk, Assessment},
    bookmarks=true
}

% --- Document Start ---
\begin{document}

% Title Block
\title{
    \textbf{Cybersecurity Posture Report} \\
    \large For: True North Travel
}
\author{Cybersecurity Analyst}
\date{\today}
\maketitle

\hrule
\vspace{1em}

% --- Table of Contents ---
\tableofcontents
\newpage

% --- Section 1: Executive Overview ---
\section{Executive Overview}
This report provides a comprehensive analysis of the cybersecurity posture for True North Travel, based on a combination of organizational data, a security controls questionnaire, and an external network scan.

The assessment reveals several critical and high-risk security gaps that require immediate attention. The most significant concerns are the absence of Multi-Factor Authentication (MFA) for email and computer access, and a complete lack of a formal security awareness training program. These deficiencies create a high probability of successful phishing attacks, credential compromise, and subsequent unauthorized access to corporate systems.

Furthermore, an external network scan identified an open Secure Shell (SSH) port. When combined with the lack of MFA and potentially weak user credentials, this exposed service presents a direct and significant vector for a network breach.

This report outlines these findings in detail and provides actionable recommendations to mitigate the identified risks and strengthen the organization's overall security posture.

% --- Section 2: Organizational Information ---
\section{Organizational Information}
The following information was provided for the assessment.

\begin{tabular}{@{}ll}
    \toprule
    \textbf{Attribute} & \textbf{Value} \\
    \midrule
    Organization Name & \textbf{True North Travel} \\
    Email Domain & \texttt{TrueNorthTravel.com} \\
    Website Domain & \texttt{www.TrueNorthTravel.com} \\
    External IP Address & \texttt{72.123.154.230} \\
    \bottomrule
\end{tabular}

% --- Section 3: Security Control Review ---
\section{Security Control Review}
A review of the organization's security controls was conducted via a questionnaire. The results below highlight significant gaps in foundational security practices. A checkmark (\ding{51}) indicates a positive control is in place, while a cross (\ding{55}) indicates a control gap.

\begin{table}[h!]
\centering
\begin{tabular}{@{}lc@{}}
\toprule
\textbf{Security Control Question} & \textbf{Status} \\
\midrule
Do you require MFA to access email? & \ding{55} \\
Do you require MFA to log into computers? & \ding{55} \\
Do you require MFA to access sensitive data systems? & \ding{51} \\
Does your organization have an employee acceptable use policy? & \ding{51} \\
Does your organization do security awareness training for new employees? & \ding{55} \\
Does your organization do security awareness training for all employees at least once per year? & \ding{55} \\
\bottomrule
\end{tabular}
\caption{Security Controls Questionnaire Results}
\end{label{tab:controls}
\end{table}

\paragraph{Analysis:}
The lack of MFA for email and computer logins represents a \textbf{critical risk}. Email is the primary target for phishing attacks, and compromised credentials can lead to widespread system access. The absence of a security awareness training program for both new and existing employees is a \textbf{high risk}, as it leaves the organization vulnerable to social engineering and other human-targeted attacks.

% --- Section 4: Technical Scan Results ---
\section{Technical Scan Results}
An external network scan was performed to identify exposed services. The scan was conducted against the target IP address provided.

\begin{itemize}
    \item \textbf{Target IP Address:} \seqsplit{\texttt{2001:db8::1}}
    \item \textbf{Scan Date:} Not Specified
\end{itemize}

The following open port was identified:

\begin{table}[h!]
\centering
\begin{tabular}{@{}llll@{}}
\toprule
\textbf{Port} & \textbf{State} & \textbf{Service} & \textbf{Notes} \\
\midrule
22/tcp & open & ssh & Secure Shell (SSH) access is enabled. \\
\bottomrule
\end{tabular}
\caption{Open Ports Detected on Target IP}
\label{tab:scanresults}
\end{table}

\paragraph{Analysis:}
The SSH service on port 22 is commonly used for remote administration. While necessary for system management, its exposure to the public internet makes it a prime target for brute-force and credential-stuffing attacks. This finding is especially concerning given the identified gaps in MFA and security training, which increase the likelihood of weak or compromised user credentials.

% --- Section 5: Correlated Risk Assessment ---
\section{Correlated Risk Assessment}
This section synthesizes the findings from the security control review and the technical scan to provide a consolidated view of the primary risks facing the organization.

\begin{table}[h!]
\centering
\begin{tabular}{@{}p{0.3\textwidth}p{0.55\textwidth}l@{}}
\toprule
\textbf{Risk Name} & \textbf{Overview} & \textbf{Severity} \\
\midrule
\textbf{Lack of MFA on Critical Systems} & The absence of MFA on email and endpoints allows an attacker with valid credentials (e.g., from a phishing attack) to gain immediate and unfettered access. & \textbf{Critical} \\
\addlinespace
\textbf{Inadequate Security Awareness Program} & Employees are not trained to recognize or respond to phishing, social engineering, or other common cyber threats, making them the weakest link in the organization's defense. & \textbf{High} \\
\addlinespace
\textbf{Exposed SSH Management Service} & The publicly accessible SSH port, combined with the lack of MFA and poor security awareness, creates a high-risk entry point for attackers to compromise the network. & \textbf{High} \\
\bottomrule
\end{tabular}
\caption{Summary of Identified Risks}
\label{tab:risks}
\end{table}

% --- Section 6: Recommendations ---
\section{Recommendations}
The following actions are recommended to mitigate the identified risks and improve the overall security posture of True North Travel. Recommendations are prioritized based on severity.

\begin{enumerate}
    \item \textbf{[Critical] Implement Multi-Factor Authentication (MFA):}
    \begin{itemize}
        \item \textbf{Action:} Immediately enforce MFA for all user accounts across all systems, prioritizing email (e.g., Office 365, Google Workspace) and endpoint logins (e.g., Windows, macOS).
        \item \textbf{Impact:} Drastically reduces the risk of account compromise from stolen credentials.
    \end{itemize}

    \item \textbf{[High] Establish a Security Awareness Training Program:}
    \begin{itemize}
        \item \textbf{Action:} Procure and implement a security awareness training solution. Mandate foundational training for all new hires and annual refresher training for all staff. Conduct regular phishing simulations to reinforce learning.
        \item \textbf{Impact:} Creates a human firewall, reducing susceptibility to phishing and social engineering attacks.
    \end{itemize}
    
    \item \textbf{[High] Secure the Exposed SSH Service:}
    \begin{itemize}
        \item \textbf{Action:} Review the business need for the public-facing SSH port.
        \begin{itemize}
            \item If not needed, disable the firewall rule allowing access.
            \item If required, restrict access to known, trusted IP addresses (IP whitelisting). Enforce the use of key-based authentication instead of passwords and disable root login over SSH.
        \end{itemize}
        \item \textbf{Impact:} Minimizes the external attack surface and hardens a critical administrative entry point against brute-force attacks.
    \end{itemize}
\end{enumerate}

% --- Document End ---
\end{document}
```