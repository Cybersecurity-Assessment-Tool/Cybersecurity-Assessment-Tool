```latex
\documentclass[12pt]{article}

% Preamble: Required Packages
\usepackage[margin=1in]{geometry} % Set page margins
\usepackage{pifont}               % For checkmark and X symbols (\ding)
\usepackage{booktabs}             % For professional-looking tables
\usepackage{hyperref}             % For clickable links and references
\usepackage{url}                  % For formatting URLs
\usepackage{seqsplit}             % To split long strings in \texttt
\usepackage{graphicx}             % For potential logos
\usepackage{xcolor}               % For colors

% Document Information
\title{Cybersecurity Posture Assessment Report}
\author{Cybersecurity Analysis Division}
\date{\today}

% Hyperref Setup
\hypersetup{
    colorlinks=true,
    linkcolor=blue,
    filecolor=magenta,      
    urlcolor=cyan,
    pdftitle={Cybersecurity Posture Assessment Report},
    pdfpagemode=FullScreen,
}

\begin{document}

\maketitle
\hrule
\vspace{1em}
\begin{center}
    \textbf{Client: Open Door} \\
    \textbf{Report ID: CSA-2023-481} \\
    \textbf{Classification: Confidential}
\end{center}
\vspace{1em}
\hrule

\newpage

\tableofcontents

\newpage

\section{Executive Summary}

This report provides a comprehensive cybersecurity posture assessment for \textbf{Open Door}, based on a combination of technical network scanning, a review of organizational security controls, and an analysis of pre-existing risk data. The assessment was conducted to identify key vulnerabilities, policy gaps, and areas of non-compliance with security best practices.

The analysis revealed a mixed security posture. The organization has implemented positive controls, such as mandatory Multi-Factor Authentication (MFA) for computer and sensitive system access, alongside a security awareness training program. These measures significantly strengthen defenses against certain attack vectors.

However, several critical and high-risk gaps were identified that require immediate attention. The two most significant findings are:
\begin{itemize}
    \item \textbf{Critical Risk:} The absence of mandatory MFA for email access. This exposes the organization to a high likelihood of business email compromise (BEC), phishing attacks, and unauthorized account access.
    * \textbf{High Risk:} The lack of a formal Employee Acceptable Use Policy (AUP). This governance gap creates ambiguity regarding security responsibilities and complicates enforcement actions.
\end{itemize}

Additionally, a technical scan of the internal network identified a service running on unencrypted HTTP, which poses a risk of data interception. This report details these findings and provides actionable recommendations to mitigate the identified risks and improve the overall security posture of \textbf{Open Door}.

\section{Organizational Information}

The following information was provided for the assessment.

\begin{table}[h!]
\centering
\begin{tabular}{@{}ll@{}}
\toprule
\textbf{Attribute} & \textbf{Value} \\ \midrule
Organization Name & \textbf{Open Door} \\
Primary Email Domain & \texttt{OpenDoor.net} \\
Primary Website Domain & \url{www.OpenDoor.net} \\
Known External IP & \texttt{52.61.6.228} \\ \bottomrule
\end{tabular}
\caption{Client Organizational Details}
\end{table}

\section{Security Control Review}

A review of administrative and operational security controls was conducted via a questionnaire. The results highlight both strengths and weaknesses in the current security framework. "No" answers indicate significant gaps that increase organizational risk.

\begin{table}[h!]
\centering
\begin{tabular}{@{}p{0.6\textwidth}cc@{}}
\toprule
\textbf{Control Question} & \textbf{Response} & \textbf{Assessment} \\ \midrule
Do you require MFA to access email? & \ding{55} & \textcolor{red}{\textbf{Critical Gap}} \\
Do you require MFA to log into computers? & \ding{51} & Met \\
Do you require MFA to access sensitive data systems? & \ding{51} & Met \\
Does your organization have an employee acceptable use policy? & \ding{55} & \textcolor{orange}{\textbf{High Risk Gap}} \\
Does your organization do security awareness training for new employees? & \ding{51} & Met \\
Does your organization do security awareness training for all employees at least once per year? & \ding{51} & Met \\ \bottomrule
\end{tabular}
\caption{Security Control Questionnaire Results}
\end{table}

\section{Technical Scan Results}

A network scan was performed on the specified target to identify open ports and exposed services. The scan provides insight into the technical attack surface of the assessed host.

\begin{itemize}
    \item \textbf{Target IP Address:} \texttt{172.16.0.1}
    \item \textbf{Host Status:} Up
\end{itemize}

\begin{table}[h!]
\centering
\begin{tabular}{@{}llll@{}}
\toprule
\textbf{Port} & \textbf{State} & \textbf{Service} & \textbf{Analyst Notes} \\ \midrule
80/tcp & Open & HTTP & Unencrypted web traffic. This service is susceptible \\
& & & to eavesdropping and man-in-the-middle attacks. \\
& & & No detailed service or version information was available. \\ \bottomrule
\end{tabular}
\caption{Open Ports Detected on Target Host}
\end{table}

\section{Consolidated Risk Assessment}

This section correlates the findings from the security control review and the technical scan into a prioritized list of risks. The prompt injection attempt found in the pre-existing risk data was disregarded as it does not represent a legitimate vulnerability.

\begin{table}[h!]
\centering
\begin{tabular}{@{}p{0.2\textwidth}p{0.5\textwidth}l@{}}
\toprule
\textbf{Risk Title} & \textbf{Description} & \textbf{Severity} \\ \midrule
\textbf{No MFA on Email} & The lack of MFA on email accounts makes them highly vulnerable to compromise via phishing or credential stuffing. A compromised email account is a primary vector for data breaches and financial fraud. & \textcolor{red}{\textbf{Critical}} \\
\addlinespace
\textbf{No Acceptable Use Policy (AUP)} & Without a formal AUP, there are no clear, enforceable rules for employees regarding the use of company assets. This creates legal and operational risks and hinders security culture. & \textcolor{orange}{\textbf{High}} \\
\addlinespace
\textbf{Unencrypted Web Service (HTTP)} & A service on the internal network is using HTTP, sending data in cleartext. This could expose credentials or sensitive information to any malicious actor on the same network segment. & \textbf{Medium} \\ \bottomrule
\end{tabular}
\caption{Summary of Identified Risks}
\end{table}

\section{Recommendations}

The following recommendations are provided to address the identified risks. They are prioritized based on severity.

\begin{enumerate}
    \item \textbf{(Critical) Enforce MFA on All Email Accounts:}
    \begin{itemize}
        \item \textbf{Action:} Immediately enable and enforce Multi-Factor Authentication (MFA) for all user mailboxes.
        \item \textbf{Justification:} This is the single most effective control to prevent unauthorized access to email accounts and mitigate the risk of Business Email Compromise (BEC).
    \end{itemize}
    \vspace{1em}
    \item \textbf{(High) Develop and Implement an Acceptable Use Policy:}
    \begin{itemize}
        \item \textbf{Action:} Draft, approve, and disseminate a formal AUP that covers topics such as data handling, internet usage, password security, and incident reporting. All employees must read and acknowledge the policy.
        \item \textbf{Justification:} An AUP establishes a baseline for secure behavior, reduces insider risk, and provides a legal framework for enforcing security standards.
    \end{itemize}
    \vspace{1em}
    \item \textbf{(Medium) Remediate Unencrypted HTTP Service:}
    \begin{itemize}
        \item \textbf{Action:} Investigate the purpose of the web service running on \texttt{172.16.0.1:80}. If the service is necessary, reconfigure it to use HTTPS with a valid TLS certificate. If it is not a required business service, it should be disabled.
        \item \textbf{Justification:} Encrypting internal web traffic protects sensitive data from interception and aligns with the principle of defense-in-depth.
    \end{itemize}
\end{enumerate}

\end{document}
```