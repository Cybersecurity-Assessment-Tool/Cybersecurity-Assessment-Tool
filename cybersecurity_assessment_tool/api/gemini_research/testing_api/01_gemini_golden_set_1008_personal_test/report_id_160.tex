```latex
\documentclass[12pt]{article}

% Preamble: Required Packages
\usepackage[margin=1in]{geometry}
\usepackage{pifont} % For checkmarks and crosses
\usepackage{booktabs} % For professional tables
\usepackage{hyperref} % For clickable links
\usepackage{url}      % For formatting URLs
\usepackage{seqsplit} % For splitting long strings in tt font

% Document Metadata
\title{Cybersecurity Posture Assessment Report}
\author{Cybersecurity Analysis Division}
\date{\today}

\hypersetup{
    colorlinks=true,
    linkcolor=black,
    urlcolor=blue,
    pdftitle={Cybersecurity Posture Assessment Report},
    pdfauthor={Cybersecurity Analysis Division},
}

\begin{document}

\maketitle
\thispagestyle{empty}
\newpage
\tableofcontents
\newpage

\section*{1. Executive Overview}

This report provides a comprehensive analysis of the cybersecurity posture for \textbf{Nebula Creative}. The assessment is based on a correlation of organizational data, a security controls questionnaire, an external network scan, and a review of known risks.

The external network perimeter, as tested against the IP address \texttt{69.93.223.194}, appears to be well-configured, with no open ports detected. This significantly reduces the external attack surface.

However, the internal security control review revealed several critical gaps. The most severe findings are the complete absence of Multi-Factor Authentication (MFA) for email, computer logins, and sensitive data systems. Additionally, the lack of mandatory security awareness training for new employees presents a high risk. These deficiencies expose the organization to significant threats, including account compromise, data breaches, and ransomware attacks, despite the strong network perimeter.

Immediate remediation of these policy and procedural gaps is strongly recommended to build a defense-in-depth security strategy.

\section*{2. Organizational Information}

The following information was provided for the assessment:
\begin{itemize}
    \item \textbf{Organization Name:} Nebula Creative
    \item \textbf{Primary Email Domain:} \texttt{NebulaCreative.org}
    \item \textbf{Primary Website Domain:} \url{www.NebulaCreative.org}
    \item \textbf{External IP Scanned:} \texttt{69.93.223.194}
\end{itemize}

\section*{3. Security Control Review}

An internal review of security controls was conducted based on a standardized questionnaire. The results below highlight current practices. Answers marked with a red 'X' (\ding{55}) indicate a deviation from security best practices and represent a potential risk.

\begin{table}[h!]
\centering
\caption{Security Controls Questionnaire Results}
\begin{tabular}{p{0.7\linewidth} c}
\toprule
\textbf{Control Question} & \textbf{Status} \\
\midrule
Does your organization have an employee acceptable use policy? & \ding{51} \\
Does your organization do security awareness training for all employees at least once per year? & \ding{51} \\
\midrule
\textit{Critical Gaps Identified} & \\
\midrule
Do you require MFA to access email? & \ding{55} \\
Do you require MFA to log into computers? & \ding{55} \\
Do you require MFA to access sensitive data systems? & \ding{55} \\
Does your organization do security awareness training for new employees? & \ding{55} \\
\bottomrule
\end{tabular}
\end{table}

\subsection*{Analysis of Gaps}
The questionnaire reveals critical weaknesses in identity and access management and employee onboarding. The lack of MFA is a severe vulnerability, as it means a compromised password is all an attacker needs to gain access to key systems. The absence of security training for new hires leaves the organization's newest and often most vulnerable employees unprepared to identify and resist social engineering or phishing attacks.

\section*{4. Technical Scan Results}

An external network vulnerability scan was performed to identify open ports and exposed services on the organization's public-facing infrastructure.

\begin{itemize}
    \item \textbf{Target IP Address:} \texttt{192.168.1.100} (Internal scan target)
    \item \textbf{External IP Address Assessed:} \texttt{69.93.223.194}
    \item \textbf{Scan Date:} \today
\end{itemize}

\subsection*{Findings}
The scan results were positive and indicate a strong network perimeter defense.
\begin{itemize}
    \item \textbf{Open Ports:} None detected.
    \item \textbf{Port State:} All 1000 scanned ports were reported as `closed`.
\end{itemize}
This configuration effectively prevents unsolicited external connections and is a commendable security practice. No vulnerabilities related to exposed services were identified.

\section*{5. Consolidated Risk Assessment}

The following table synthesizes findings from the security control review, technical scan, and pre-existing risk data. As no pre-existing vulnerabilities were reported, the risks below are derived directly from this assessment.

\begin{table}[h!]
\centering
\caption{Identified Risks and Severity}
\begin{tabular}{p{0.25\linewidth} p{0.5\linewidth} p{0.15\linewidth}}
\toprule
\textbf{Risk Name} & \textbf{Overview} & \textbf{Severity} \\
\midrule
\textbf{Lack of Multi-Factor Authentication (MFA)} & The absence of MFA on email, endpoints, and sensitive systems exposes the organization to a high likelihood of account compromise via credential theft, phishing, or password spraying. A single compromised password could lead to a significant data breach. & \textbf{Critical} \\
\addlinespace
\textbf{Inadequate New Employee Onboarding} & Failure to provide security awareness training during onboarding leaves new staff unable to recognize and report threats. New employees are a primary target for social engineering attacks, making this a significant entry point for attackers. & \textbf{High} \\
\bottomrule
\end{tabular}
\end{table}

\section*{6. Recommendations}

To mitigate the identified risks and improve the overall security posture, the following actions are recommended, ordered by priority.

\subsection*{Priority 1: Implement Multi-Factor Authentication (Critical)}
The highest priority is to deploy MFA across all critical platforms to protect against account takeover.
\begin{enumerate}
    \item \textbf{Immediately Enforce MFA on Email:} All user accounts on the \texttt{NebulaCreative.org} email platform (e.g., Office 365, Google Workspace) should have MFA enforced without exception.
    \item \textbf{Deploy MFA for Remote Access:} Any VPN or other remote access solution must require MFA.
    \item \textbf{Protect Sensitive Systems:} Enforce MFA for all applications and systems that store or process sensitive or confidential data.
    \item \textbf{Phase-in Endpoint MFA:} Plan a project to require MFA for all computer and server logins.
\end{enumerate}

\subsection*{Priority 2: Enhance Security Awareness Program (High)}
Strengthen the human firewall by ensuring all employees are equipped to handle security threats from day one.
\begin{enumerate}
    \item \textbf{Integrate Security Training into Onboarding:} Develop a mandatory security awareness training module that all new employees must complete within their first week of employment.
    \item \textbf{Training Content:} This training should cover, at a minimum: phishing identification, password security policies, acceptable use of company assets, and the procedure for reporting a security incident.
\end{enumerate}

\subsection*{Commendation: Maintain Network Security}
The organization is commended for its secure external network configuration. Continue to follow the principle of least privilege by ensuring that no unnecessary ports are opened to the internet and that the firewall rules are regularly reviewed and maintained.

\end{document}
```