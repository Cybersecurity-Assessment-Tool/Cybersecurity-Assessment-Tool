```latex
\documentclass[12pt]{article}

% Preamble: Required Packages
\usepackage[a4paper, margin=1in]{geometry}
\usepackage{pifont} % For \ding
\usepackage{booktabs} % For professional tables
\usepackage{hyperref} % For clickable links and references
\usepackage{url} % For URL formatting
\usepackage{seqsplit} % To split long strings in \texttt
\usepackage{graphicx}
\usepackage{fancyhdr}
\usepackage{lastpage}
\usepackage{xcolor}

% --- Document Setup ---

% Define custom colors
\definecolor{TitleBlue}{RGB}{0, 82, 155}
\definecolor{GrayBG}{gray}{0.95}

% Hyperlink setup
\hypersetup{
    colorlinks=true,
    linkcolor=TitleBlue,
    urlcolor=blue,
    citecolor=TitleBlue,
}

% Header and Footer
\pagestyle{fancy}
\fancyhf{} % Clear all header and footer fields
\fancyhead[L]{Cybersecurity Posture Report}
\fancyhead[R]{Fable \& Lore}
\fancyfoot[C]{\thepage\ of \pageref{LastPage}}
\renewcommand{\headrulewidth}{0.4pt}
\renewcommand{\footrulewidth}{0.4pt}

% --- Document Start ---

\begin{document}

% --- Title Page ---
\begin{titlepage}
    \centering
    \vspace*{2cm}
    
    {\Huge \bfseries \color{TitleBlue} Cybersecurity Posture Report}
    
    \vspace{1.5cm}
    
    {\Large \bfseries Prepared For:}
    
    {\Large Fable \& Lore}
    
    \vspace{2cm}
    
    {\large \today}
    
    \vfill
    
    {\itshape This report is confidential and intended solely for the use of Fable \& Lore. It contains a summary of findings based on the data provided. Distribution without prior consent is prohibited.}
    
\end{titlepage}

\newpage

% --- Table of Contents ---
\tableofcontents

\newpage

% --- Section 1: Executive Summary ---
\section{Executive Summary}

This report provides a cybersecurity assessment for Fable \& Lore, based on an analysis of organizational security controls, technical scan data, and pre-existing risk information. The objective is to identify security gaps, assess their potential impact, and provide actionable recommendations for remediation.

\paragraph{Key Findings:}
The organization demonstrates a strong commitment to identity and access management, with Multi-Factor Authentication (MFA) consistently enforced across email, computer logins, and sensitive data systems. This significantly reduces the risk of unauthorized access via compromised credentials.

However, a critical gap was identified in the employee onboarding process: \textbf{new hires do not receive mandatory security awareness training}. This oversight exposes the organization to a high risk of social engineering, phishing attacks, and unintentional security breaches, as new employees are often prime targets for malicious actors.

\paragraph{Data Limitations:}
It is crucial to note that the technical network scan data (\texttt{Input\_1\_Network\_Scan\_JSON}) and the list of current organizational risks (\texttt{Input\_3\_Current\_Risks\_JSON}) were found to be corrupted and could not be processed. Consequently, this assessment is primarily based on the security questionnaire and cannot provide insights into potential vulnerabilities in external-facing services or correlate findings with previously identified risks.

\paragraph{Primary Recommendation:}
The most urgent recommendation is to implement a mandatory security awareness training program for all new employees as part of their standard onboarding procedure. This single measure will substantially strengthen the organization's human firewall and mitigate a significant vector of attack.

% --- Section 2: Organizational Information ---
\section{Organizational Information}

The following details were provided for the assessment.

\begin{table}[h!]
\centering
\begin{tabular}{@{}ll@{}}
\toprule
\textbf{Attribute} & \textbf{Value} \\
\midrule
Organization Name & Fable \& Lore \\
Email Domain & \texttt{FableLore.com} \\
Website Domain & \url{www.FableLore.com} \\
External IP Address & \texttt{186.247.131.87} \\
\bottomrule
\end{tabular}
\caption{Client Organizational Details}
\end{table}

% --- Section 3: Security Control Review ---
\section{Security Control Review (Questionnaire Analysis)}

The following table summarizes the organization's responses to the security controls questionnaire. The analysis highlights a strong foundation in access control but reveals a critical weakness in security training.

\begin{table}[h!]
\centering
\begin{tabular}{@{}lc@{}}
\toprule
\textbf{Control Question} & \textbf{Response} \\
\midrule
Do you require MFA to access email? & \color{green}\ding{51} \\ % Yes
Do you require MFA to log into computers? & \color{green}\ding{51} \\ % Yes
Do you require MFA to access sensitive data systems? & \color{green}\ding{51} \\ % Yes
Does your organization have an employee acceptable use policy? & \color{green}\ding{51} \\ % Yes
Does your organization do security awareness training for new employees? & \color{red}\ding{55} \\ % No
Does your organization do security awareness training for all employees at least once per year? & \color{green}\ding{51} \\ % Yes
\bottomrule
\end{tabular}
\caption{Security Controls Questionnaire Results}
\end{table}

\paragraph{Analysis:}
The consistent implementation of MFA is commendable and serves as a robust defense against account takeover attacks. The existence of an acceptable use policy and annual training for existing staff are also positive indicators of a maturing security program.

The single "No" response is a significant finding. The absence of security training during the onboarding phase leaves the organization vulnerable. New employees are unfamiliar with corporate policies and are more susceptible to social engineering tactics. This gap undermines the effectiveness of other security controls.

% --- Section 4: Technical Scan Results ---
\section{Technical Scan Results}

The data file intended to contain the results of the network scan against the target IP address (\texttt{[Target IP]}) was found to be corrupted or incomplete. 

\begin{center}
\fcolorbox{red}{GrayBG}{%
\begin{minipage}{0.8\textwidth}
\textbf{Scan Data Unavailable:} No analysis of open ports, running services, or software versions could be performed. This represents a significant blind spot in the current assessment, as vulnerabilities in external-facing systems remain unevaluated. It is strongly recommended to conduct a new scan to identify and address potential technical weaknesses.
\end{minipage}%
}
\end{center}

% --- Section 5: Risk Assessment ---
\section{Risk Assessment}

This section details the risks identified during this assessment. Due to corrupted input data, pre-existing risks could not be included or correlated. The primary risk identified stems directly from the security control review.

\begin{table}[h!]
\centering
\begin{tabular}{@{}lp{6cm}l@{}}
\toprule
\textbf{Risk ID} & \textbf{Risk Name \& Overview} & \textbf{Severity} \\
\midrule
R-001 & \textbf{Lack of Onboarding Security Awareness Training} \newline \small New employees are not trained on security best practices, acceptable use, or threat identification upon joining. This makes them highly susceptible to phishing, social engineering, and unintentional policy violations, creating a weak link in the organization's defense. & \textbf{High} \\
\bottomrule
\end{tabular}
\caption{Identified Risks}
\end{table}

% --- Section 6: Recommendations ---
\section{Recommendations}

The following prioritized recommendations are provided to address the findings of this report.

\subsection{Priority 1 (High): Implement Mandatory Onboarding Security Training}
\begin{itemize}
    \item \textbf{Action:} Develop or procure a security awareness training module that is a mandatory component of the new-hire onboarding process. This training should be completed before a new employee is granted access to sensitive data or systems.
    \item \textbf{Justification:} This directly mitigates the highest-rated risk (R-001) by equipping new staff with the knowledge to recognize and report threats. It is a cost-effective measure that significantly reduces the likelihood of a breach caused by human error.
\end{itemize}

\subsection{Priority 2 (Medium): Re-run Technical Infrastructure Scans}
\begin{itemize}
    \item \textbf{Action:} Schedule and execute a new, authenticated and unauthenticated network vulnerability scan against the organization's external IP (\texttt{186.247.131.87}) and other critical assets.
    \item \textbf{Justification:} The corrupted scan data creates a critical visibility gap. A successful scan is essential to identify outdated software, insecure configurations, and other technical vulnerabilities that could be exploited by attackers.
\end{itemize}

\subsection{Priority 3 (Medium): Restore and Review Risk Register}
\begin{itemize}
    \item \textbf{Action:} Investigate the cause of the data corruption for the existing risk register (\texttt{Input\_3\_Current\_Risks\_JSON}) and restore the information from a valid backup.
    \item \textbf{Justification:} A complete and up-to-date risk register is a cornerstone of an effective risk management program. It is necessary for tracking vulnerabilities, prioritizing remediation efforts, and demonstrating due diligence.
\end{itemize}

\end{document}
```