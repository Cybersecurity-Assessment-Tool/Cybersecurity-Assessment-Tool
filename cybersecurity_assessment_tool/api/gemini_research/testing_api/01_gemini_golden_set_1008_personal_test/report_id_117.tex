```latex
\documentclass[12pt]{article}

% Required Packages
\usepackage[margin=1in]{geometry}
\usepackage{pifont} % For \ding
\usepackage{booktabs} % For professional tables
\usepackage{hyperref} % For clickable links
\usepackage{url} % For URL formatting
\usepackage{seqsplit} % For splitting long text strings

% Document Metadata
\title{Cybersecurity Assessment Report}
\author{Cybersecurity Analysis Division}
\date{\today}

\begin{document}

\maketitle

\begin{abstract}
This report provides a comprehensive cybersecurity assessment for Symmetry Architecture. The analysis is based on a synthesis of network scan data, organizational security control questionnaires, and a review of pre-existing risk documentation. The assessment identified several critical and high-risk findings, including the exposure of a sensitive internal system, a systemic lack of Multi-Factor Authentication (MFA), and gaps in the security awareness training program. This report outlines these risks and provides actionable recommendations to mitigate them and improve the organization's overall security posture.
\end{abstract}

\section*{1. Overview and Executive Summary}

The analysis reveals significant security risks that require immediate attention. A network scan identified an open port (8080) on an internal host (\texttt{10.5.5.5}) with an HTTP title of \textbf{"TOP SECRET DB"}. This finding indicates a potentially sensitive database or application interface is exposed without adequate protection. 

This technical vulnerability is severely compounded by organizational policy gaps identified in the security questionnaire. The complete absence of Multi-Factor Authentication (MFA) for email, computer logins, and sensitive data systems creates a critical risk of unauthorized access. Furthermore, the lack of annual security training for all employees suggests that the workforce may not be equipped to recognize and respond to modern cyber threats.

A notable discrepancy was found between our technical scan results and the provided current risk documentation, which listed port 8080 as a secure false positive. This indicates that the existing risk register is outdated and does not reflect the current state of the network, representing a procedural failure in risk management.

\section*{2. Organizational Information}

The following information was provided by the client and used as a baseline for this assessment.

\begin{table}[h!]
\centering
\begin{tabular}{@{}ll@{}}
\toprule
\textbf{Attribute} & \textbf{Value} \\
\midrule
Organization Name & \textbf{Symmetry Architecture} \\
Email Domain & \texttt{SymmetryArchitecture.com} \\
Website Domain & \seqsplit{\url{www.SymmetryArchitecture.com}} \\
External IP Address & \texttt{128.246.113.49} \\
\bottomrule
\end{tabular}
\caption{Client Organizational Data.}
\label{tab:org_data}
\end{table}

\section*{3. Security Control Review}

A review of the organization's security controls was conducted via a questionnaire. The responses are summarized below. Answers marked with \ding{55} represent significant gaps in the security framework and are correlated with findings in the Risk Assessment section.

\begin{table}[h!]
\centering
\begin{tabular}{@{}lc@{}}
\toprule
\textbf{Security Control Question} & \textbf{Response} \\
\midrule
Do you require MFA to access email? & \ding{55} \\
Do you require MFA to log into computers? & \ding{55} \\
Do you require MFA to access sensitive data systems? & \ding{55} \\
Does your organization have an employee acceptable use policy? & \ding{51} \\
Does your organization do security awareness training for new employees? & \ding{51} \\
Does your organization do security awareness training for all employees at least once per year? & \ding{55} \\
\bottomrule
\end{tabular}
\caption{Security Controls Questionnaire Results (\ding{51}=Yes, \ding{55}=No).}
\label{tab:controls}
\end{table}

\section*{4. Technical Scan Results}

An internal network scan was performed to identify active services and potential vulnerabilities. The scan targeted the internal IP address \texttt{10.5.5.5}.

\subsection*{Host: \texttt{10.5.5.5}}
\begin{itemize}
    \item \textbf{Status:} Host is up.
    \item \textbf{Open Ports:}
        \begin{itemize}
            \item \textbf{Port 8080/tcp (Open):} An HTTP service was identified. A script interrogating the service returned the title: \textbf{"TOP SECRET DB"}. This is a critical information disclosure finding. The title strongly suggests that a sensitive, possibly unauthenticated, database management interface is accessible on the network. This finding directly contradicts the information provided in the existing risk documentation.
        \end{itemize}
\end{itemize}

\section*{5. Risk Assessment}

The following table synthesizes the findings from the security control review and the technical scan. The risks are prioritized based on their potential impact on the organization.

\begin{table}[h!]
\centering
\begin{tabular}{@{}p{0.2\linewidth}p{0.2\linewidth}p{0.5\linewidth}@{}}
\toprule
\textbf{Risk Name} & \textbf{Severity} & \textbf{Overview} \\
\midrule
\textbf{Exposed Sensitive Database Interface} & \textbf{Critical} & The service on \texttt{10.5.5.5:8080} is titled "TOP SECRET DB", indicating a high-value target is exposed. This risk is amplified by the lack of MFA for sensitive systems, making it vulnerable to credential-based attacks. \\
\addlinespace
\textbf{Systemic Lack of MFA} & \textbf{Critical} & The absence of MFA for email, computer access, and sensitive systems exposes the organization to account takeovers, business email compromise, and data breaches. A single compromised password could lead to a major incident. \\
\addlinespace
\textbf{Inadequate Security Awareness Program} & \textbf{High} & While new employees receive training, the lack of an annual refresher for all staff means that awareness of evolving threats like phishing and social engineering degrades over time, increasing susceptibility. \\
\addlinespace
\textbf{Outdated Risk Register} & \textbf{Medium} & The existing risk documentation incorrectly lists Port 8080 as a secure false positive. This demonstrates a failure in the risk management lifecycle, preventing the organization from having an accurate view of its threat landscape. \\
\bottomrule
\end{tabular}
\caption{Synthesized Risk Summary.}
\label{tab:risks}
\end{table}

\section*{6. Recommendations}

The following actions are recommended to mitigate the identified risks.

\subsection*{Immediate Actions (0-7 Days)}
\begin{enumerate}
    \item \textbf{Contain Exposed Database:} Immediately investigate the service on \texttt{10.5.5.5:8080}. If it is not business-critical, shut it down. If it is required, place it behind a firewall and restrict access to only authorized personnel and systems.
    \item \textbf{Enforce MFA on Critical Systems:} Begin emergency rollout of MFA for all administrative accounts and access to sensitive data systems, including the one identified in the scan.
\end{enumerate}

\subsection*{Short-Term Actions (1-3 Months)}
\begin{enumerate}
    \item \textbf{Complete MFA Rollout:} Implement and enforce MFA for all employees across all key systems, including email (e.g., Office 365, G Suite) and VPN access.
    \item \textbf{Update Risk Register:} Conduct a formal review of the risk register. Invalidate the incorrect finding for Port 8080 and add the new risks identified in this report. Establish a process for regular verification of technical risks.
\end{enumerate}

\subsection*{Long-Term Actions (3-6 Months)}
\begin{enumerate}
    \item \textbf{Implement Annual Security Training:} Develop and deploy a mandatory annual security awareness training program for all employees. The training should cover phishing, password hygiene, acceptable use, and incident reporting.
    \item \textbf{Conduct Vulnerability Assessment:} Perform a comprehensive internal and external vulnerability assessment to identify other potential exposures and outdated systems.
\end{enumerate}

\end{document}
```