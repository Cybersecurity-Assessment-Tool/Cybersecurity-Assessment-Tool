```latex
\documentclass[12pt]{article}

% Preamble: Required Packages
\usepackage[margin=1in]{geometry}
\usepackage{pifont} % For checkmarks and crosses (\ding)
\usepackage{booktabs} % For professional tables
\usepackage{hyperref} % For clickable links
\usepackage{url} % For URL formatting
\usepackage{seqsplit} % To split long strings in \texttt
\usepackage{graphicx}
\usepackage{xcolor}
\usepackage{fancyhdr}
\usepackage{lastpage}

% --- Document Setup ---
\definecolor{darkblue}{rgb}{0.0, 0.0, 0.55}
\definecolor{darkred}{rgb}{0.55, 0.0, 0.0}

\hypersetup{
    colorlinks=true,
    linkcolor=darkblue,
    filecolor=magenta,      
    urlcolor=darkblue,
    citecolor=darkblue,
}

% --- Header & Footer ---
\pagestyle{fancy}
\fancyhf{} % Clear all header and footer fields
\fancyhead[L]{\textit{Cybersecurity Posture Assessment}}
\fancyhead[R]{Pacific Rim Exports}
\fancyfoot[C]{\thepage\ of \pageref{LastPage}}
\renewcommand{\headrulewidth}{0.4pt}
\renewcommand{\footrulewidth}{0.4pt}

% --- Document Body ---
\begin{document}

% --- Title Page ---
\begin{titlepage}
    \centering
    \vspace*{1cm}
    
    \Huge
    \textbf{Cybersecurity Posture Assessment Report}
    
    \vspace{1.5cm}
    
    \Large
    Prepared for:
    
    \vspace{0.5cm}
    
    \textbf{Pacific Rim Exports}
    
    \vspace{2cm}
    
    \large
    Report Date: \today
    
    \vfill
    
    \normalsize
    \textit{This report contains sensitive information and should be handled with care. Distribution is restricted to authorized personnel only.}
    
\end{titlepage}

\tableofcontents
\newpage

% --- Section 1: Executive Summary ---
\section{Executive Summary}
This report provides a cybersecurity posture assessment for Pacific Rim Exports, based on a combination of organizational data, a security controls questionnaire, and a network vulnerability scan.

The assessment reveals a mixed security posture. On the one hand, the external network scan of the target host showed a strong defensive configuration, with no open ports detected. This indicates a well-configured firewall and a minimal attack surface from a network perspective.

However, significant procedural and policy gaps were identified through the security controls review. Two critical findings require immediate attention:
\begin{itemize}
    \item \textbf{Lack of MFA on Sensitive Systems:} The absence of multi-factor authentication (MFA) on systems containing sensitive data represents a critical risk. A single compromised password could lead to a significant data breach.
    \item \textbf{No Onboarding Security Training:} New employees are not provided with security awareness training upon being hired. This creates a window of vulnerability where new staff are highly susceptible to phishing and social engineering attacks.
\end{itemize}

While some security controls are effectively implemented, such as MFA for email and annual training, the identified gaps could be exploited by malicious actors. Prioritized recommendations are provided in Section \ref{sec:recommendations} to address these risks and strengthen the overall security posture of Pacific Rim Exports.

% --- Section 2: Organizational Information ---
\section{Organizational Information}
The following details were provided for the assessment.

\begin{table}[h!]
\centering
\begin{tabular}{@{}ll@{}}
\toprule
\textbf{Attribute} & \textbf{Value} \\ \midrule
Organization Name & Pacific Rim Exports \\
Email Domain      & \texttt{PacificRimExports.com} \\
Website Domain    & \url{www.PacificRimExports.com} \\
External IP Address & \texttt{174.203.112.108} \\ \bottomrule
\end{tabular}
\caption{Client Organizational Details}
\end{table}

% --- Section 3: Security Control Review ---
\section{Security Control Review}
A review of organizational security controls was conducted via a questionnaire. The results below highlight implemented controls and identify significant gaps. A green checkmark (\textcolor{green}{\ding{51}}) indicates a positive response, while a red cross (\textcolor{darkred}{\ding{55}}) indicates a negative response that constitutes a security gap.

\begin{table}[h!]
\centering
\begin{tabular}{@{}p{0.3\linewidth} p{0.5\linewidth} c@{}}
\toprule
\textbf{Control Area} & \textbf{Question} & \textbf{Status} \\ \midrule
Access Control & Do you require MFA to access email? & \textcolor{green}{\ding{51}} \\
Access Control & Do you require MFA to log into computers? & \textcolor{green}{\ding{51}} \\
\textbf{Access Control} & \textbf{Do you require MFA to access sensitive data systems?} & \textcolor{darkred}{\ding{55}} \\
\addlinespace
Policy \& Training & Does your organization have an employee acceptable use policy? & \textcolor{green}{\ding{51}} \\
\textbf{Policy \& Training} & \textbf{Does your organization do security awareness training for new employees?} & \textcolor{darkred}{\ding{55}} \\
Policy \& Training & Does your organization do security awareness training for all employees at least once per year? & \textcolor{green}{\ding{51}} \\ \bottomrule
\end{tabular}
\caption{Security Controls Questionnaire Results}
\end{table}

\subsection*{Analysis of Gaps}
The two negative responses represent the most significant risks identified during this assessment. The lack of MFA on sensitive systems is a critical failure of the principle of least privilege and defense-in-depth. The absence of security training during employee onboarding ignores a crucial opportunity to instill a security-conscious mindset from day one, leaving the organization vulnerable.

% --- Section 4: Technical Scan Results ---
\section{Technical Scan Results}
A network scan was performed to identify potential vulnerabilities on the specified target system.

\subsection*{Nmap Scan of \texttt{192.168.1.100}}
\begin{itemize}
    \item \textbf{Scan Date:} \today
    \item \textbf{Target IP:} \texttt{192.168.1.100}
    \item \textbf{Status:} Host is Up
\end{itemize}

\textbf{Findings:} The scan concluded that \textbf{no open ports} were detected on the target host. All 1000 scanned ports were reported as `closed`.

\textbf{Interpretation:} This is a positive security finding. It suggests that the host is either not running any network services or is protected by a well-configured firewall that is correctly blocking all unsolicited inbound connections. This significantly reduces the external network attack surface for this specific system.

% --- Section 5: Consolidated Risk Assessment ---
\section{Consolidated Risk Assessment}
The following table synthesizes findings from the security control review and technical scans into a prioritized list of risks.

\begin{table}[h!]
\centering
\begin{tabular}{@{}p{0.05\linewidth} p{0.25\linewidth} p{0.45\linewidth} p{0.1\linewidth}@{}}
\toprule
\textbf{ID} & \textbf{Risk Name} & \textbf{Overview} & \textbf{Severity} \\ \midrule
\textbf{R-01} & \textbf{Lack of MFA on Sensitive Systems} & Failure to enforce MFA on systems containing sensitive data exposes the organization to significant risk from credential theft. A single compromised password could grant an attacker access to critical assets. & \textbf{Critical} \\
\addlinespace
\textbf{R-02} & \textbf{Inadequate New Employee Onboarding} & New employees are not provided with security awareness training upon hiring. This makes them highly susceptible to phishing, social engineering, and unintentional policy violations during their initial, most vulnerable period. & \textbf{High} \\
\addlinespace
\textbf{I-01} & \textbf{Positive Network Posture} & The network scan of the target host showed no open ports. This is an informational finding representing a security strength. & Low \\
\bottomrule
\end{tabular}
\caption{Summary of Identified Risks}
\end{table}

% --- Section 6: Recommendations ---
\section{Recommendations}
\label{sec:recommendations}
Based on the risk assessment, the following prioritized actions are recommended to improve the security posture of Pacific Rim Exports.

\subsection*{Priority 1: Critical}
\begin{description}
    \item[Recommendation for R-01:] \textbf{Implement MFA on All Sensitive Systems.}
    \begin{itemize}
        \item \textbf{Action:} Immediately begin a project to identify all systems, applications, and databases that store or process sensitive information. Procure and deploy an MFA solution that integrates with these systems.
        \item \textbf{Justification:} This is the single most effective control to prevent unauthorized access resulting from compromised credentials. It mitigates the risk of a catastrophic data breach.
    \end{itemize}
\end{description}

\subsection*{Priority 2: High}
\begin{description}
    \item[Recommendation for R-02:] \textbf{Integrate Security Training into Employee Onboarding.}
    \begin{itemize}
        \item \textbf{Action:} Develop or procure a foundational security awareness training module and make it a mandatory part of the new employee onboarding checklist, to be completed within the first week of employment.
        \item \textbf{Justification:} Training new hires immediately reduces the likelihood of them falling victim to early-stage social engineering or phishing attacks, closing a significant window of vulnerability.
    \end{itemize}
\end{description}

\subsection*{Priority 3: Informational}
\begin{description}
    \item[Recommendation for I-01:] \textbf{Maintain and Expand Network Scanning.}
    \begin{itemize}
        \item \textbf{Action:} Continue the practice of maintaining a strong firewall posture. Implement a schedule for regular, authenticated vulnerability scanning across all critical network segments to ensure this posture is maintained and to identify any new vulnerabilities.
        \item \textbf{Justification:} The current state is positive, but security is a continuous process. Regular scanning ensures that new services or misconfigurations are detected and remediated promptly.
    \end{itemize}
\end{description}

\end{document}
```