```latex
\documentclass[12pt]{article}

% --- PACKAGE IMPORTS ---
\usepackage[margin=1in]{geometry}
\usepackage{pifont} % For checkmarks and crosses
\usepackage{booktabs} % For professional tables
\usepackage{hyperref} % For clickable links
\usepackage{url} % For formatting URLs
\usepackage{seqsplit} % For splitting long strings in tt font
\usepackage{graphicx} % For logo (placeholder)
\usepackage{xcolor} % For colors

% --- DOCUMENT METADATA ---
\title{Cybersecurity Posture Assessment Report}
\author{Cybersecurity Analyst}
\date{\today}

% --- HYPERREF SETUP ---
\hypersetup{
    colorlinks=true,
    linkcolor=blue,
    filecolor=magenta,      
    urlcolor=cyan,
    pdftitle={Cybersecurity Posture Assessment Report},
    pdfpagemode=FullScreen,
}

\begin{document}

\maketitle

\begin{abstract}
This report details the findings of a cybersecurity posture assessment for Ember Glow Hospitality. The analysis correlates data from an external network scan, a security controls questionnaire, and a review of pre-existing risks. The assessment identified several critical and high-risk security gaps, including an exposed service with a highly sensitive banner, a lack of multi-factor authentication (MFA) on critical systems, and the absence of a security awareness training program. These findings indicate a significant risk of unauthorized access and data compromise. This report provides a detailed risk assessment and actionable recommendations to mitigate the identified vulnerabilities.
\end{abstract}

\newpage

% ===================================================================
% SECTION 1: ORGANIZATIONAL INFORMATION
% ===================================================================
\section{Organizational Information}

This section provides the organizational details as provided for this assessment.

\begin{table}[h!]
\centering
\begin{tabular}{@{}ll@{}}
\toprule
\textbf{Attribute} & \textbf{Value} \\ \midrule
Organization Name & Ember Glow Hospitality \\
Email Domain & \texttt{EmberGlowHospitality.org} \\
Website Domain & \url{www.EmberGlowHospitality.org} \\
External IP Address & \texttt{30.13.63.43} \\ \bottomrule
\end{tabular}
\caption{Client Organizational Data.}
\end{table}

% ===================================================================
% SECTION 2: SECURITY CONTROL REVIEW
% ===================================================================
\section{Security Control Review}

A review of the organization's administrative and procedural security controls was conducted via a questionnaire. The responses reveal critical gaps in user access controls and employee security training.

\begin{table}[h!]
\centering
\begin{tabular}{@{}lc@{}}
\toprule
\textbf{Control Question} & \textbf{Response} \\ \midrule
Do you require MFA to access email? & \ding{51} \\
Do you require MFA to log into computers? & \textbf{\color{red}\ding{55}} \\
Do you require MFA to access sensitive data systems? & \textbf{\color{red}\ding{55}} \\
Does your organization have an employee acceptable use policy? & \ding{51} \\
Does your organization do security awareness training for new employees? & \textbf{\color{red}\ding{55}} \\
Does your organization do security awareness training for all employees annually? & \textbf{\color{red}\ding{55}} \\ \bottomrule
\end{tabular}
\caption{Security Controls Questionnaire Results. (\ding{51} = Yes, \ding{55} = No)}
\end{label{tab:controls}
\end{table}

\subsection*{Analysis of Control Gaps}
The "No" responses in Table \ref{tab:controls} highlight significant weaknesses:
\begin{itemize}
    \item \textbf{Lack of MFA:} The absence of MFA for computer and sensitive data system access is a critical vulnerability. If an employee's credentials are stolen (e.g., through phishing), an attacker could gain direct access to company workstations and sensitive information without needing a second authentication factor.
    \item \textbf{No Security Awareness Training:} Without a formal training program, employees are significantly more likely to fall victim to social engineering and phishing attacks. This "human firewall" weakness directly increases the risk of credential compromise and malware infection.
\end{itemize}

% ===================================================================
% SECTION 3: TECHNICAL SCAN RESULTS
% ===================================================================
\section{Technical Scan Results}

A network scan was performed to identify open ports and exposed services on the target system.

\begin{itemize}
    \item \textbf{Target IP Address:} \texttt{10.5.5.5}
    \item \textbf{Scan Date:} \today
\end{itemize}

\begin{table}[h!]
\centering
\begin{tabular}{@{}llll@{}}
\toprule
\textbf{Port} & \textbf{State} & \textbf{Service/Banner Information} \\ \midrule
8080/tcp & OPEN & HTTP Title: \textbf{\color{red}TOP SECRET DB} \\ \bottomrule
\end{tabular}
\caption{Open Ports Detected on \texttt{10.5.5.5}.}
\label{tab:scanresults}
\end{table}

\subsection*{Analysis of Technical Findings}
The scan revealed a critically severe finding. Port 8080 is open and hosts a web service with the title \textbf{"TOP SECRET DB"}. 
\begin{itemize}
    \item \textbf{Information Disclosure:} The service title itself is a major information leak, explicitly labeling the system as sensitive and making it a high-value target for attackers.
    \item \textbf{Exposure of Sensitive System:} This finding strongly suggests that a database management interface or a related sensitive application is directly exposed to the network. This contradicts the pre-existing risk assessment data (\textit{Input\_3\_Current\_Risks\_JSON}), which incorrectly classified this port as a "false positive." That assessment is now considered outdated and inaccurate.
    \item \textbf{Correlated Risk:} When combined with the lack of MFA for sensitive systems, this exposed service represents an immediate and severe threat to the confidentiality of the organization's data.
\end{itemize}

% ===================================================================
% SECTION 4: CONSOLIDATED RISK ASSESSMENT
% ===================================================================
\section{Consolidated Risk Assessment}

The following table summarizes the most critical risks identified by correlating the security control gaps, technical findings, and pre-existing risk data.

\begin{table}[h!]
\centering
\resizebox{\textwidth}{!}{%
\begin{tabular}{@{}lll@{}}
\toprule
\textbf{Risk Title} & \textbf{Severity} & \textbf{Description} \\ \midrule
\begin{tabular}[t]{@{}l@{}}Exposed Sensitive \\ Database Interface\end{tabular} & \textbf{CRITICAL} & \begin{tabular}[t]{@{}l@{}}Port 8080 on \texttt{10.5.5.5} exposes a service titled "TOP SECRET DB," \\ indicating a direct, unsecured interface to a highly sensitive system. \\ This invalidates the previous risk assessment of this port.\end{tabular} \\ \\
\begin{tabular}[t]{@{}l@{}}Lack of Multi-Factor \\ Authentication (MFA)\end{tabular} & \textbf{HIGH} & \begin{tabular}[t]{@{}l@{}}MFA is not enforced for computer logins or access to sensitive data \\ systems, drastically increasing the risk of unauthorized access via \\ compromised credentials.\end{tabular} \\ \\
\begin{tabular}[t]{@{}l@{}}Insufficient Security \\ Awareness Training\end{tabular} & \textbf{HIGH} & \begin{tabular}[t]{@{}l@{}}The absence of a security training program for new and existing \\ employees makes the organization highly vulnerable to phishing and \\ social engineering attacks.\end{tabular} \\ \bottomrule
\end{tabular}%
}
\caption{Summary of Identified Risks.}
\label{tab:risks}
\end{table}

% ===================================================================
% SECTION 5: RECOMMENDATIONS
% ===================================================================
\section{Recommendations}

The following actions are recommended to mitigate the identified risks. They are prioritized based on severity and potential impact.

\subsection*{Immediate Priority (Critical)}
\begin{enumerate}
    \item \textbf{Isolate Exposed Service on Port 8080:} Immediately investigate the service running on \texttt{10.5.5.5:8080}. Access to this port should be blocked from the general network using a firewall. If this system is required for business operations, access should be restricted to only authorized IP addresses or moved behind a VPN with MFA.
    \item \textbf{Remediate Information Disclosure:} Change the HTTP title of the service on port 8080 to a generic, non-descriptive name to prevent it from being easily identified as a sensitive system.
\end{enumerate}

\subsection*{High Priority}
\begin{enumerate}
    \item \textbf{Implement Mandatory MFA:} Deploy a robust MFA solution across the organization. Prioritize its enforcement for:
        \begin{itemize}
            \item All systems identified as containing sensitive data.
            \item All remote access solutions (VPN, RDP).
            \item All user computer logins.
        \end{itemize}
    \item \textbf{Establish Security Awareness Training Program:} Develop and implement a mandatory security awareness training program. This program should be required for all new employees during onboarding and for all staff on an annual basis. Topics should include phishing identification, password hygiene, and acceptable use policies.
\end{enumerate}

\subsection*{Medium Priority}
\begin{enumerate}
    \item \textbf{Update and Review Risk Register:} The organization's internal risk register must be updated to reflect these findings. The previous entry for Port 8080 being a "false positive" must be removed and replaced with the new critical risk assessment.
\end{enumerate}

\end{document}
```