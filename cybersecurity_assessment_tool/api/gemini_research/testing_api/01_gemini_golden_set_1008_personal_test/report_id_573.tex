```latex
\documentclass[12pt]{article}

% Preamble: Required Packages
\usepackage[margin=1in]{geometry}
\usepackage{pifont} % For checkmarks and crosses
\usepackage{booktabs} % For professional tables
\usepackage{hyperref} % For clickable links and metadata
\usepackage{url} % For formatting URLs
\usepackage{seqsplit} % To split long strings without breaking
\usepackage[utf8]{inputenc} % For UTF-8 input
\usepackage{graphicx}
\usepackage{xcolor}

% Document Metadata
\hypersetup{
    colorlinks=true,
    linkcolor=blue,
    filecolor=magenta,      
    urlcolor=cyan,
    pdftitle={Cybersecurity Posture Assessment Report},
    pdfauthor={Cybersecurity Analyst},
    pdfsubject={Security Analysis},
    pdfkeywords={Cybersecurity, Nmap, Risk Assessment},
}

% Title Information
\title{Cybersecurity Posture Assessment Report}
\author{Cybersecurity Analysis Division}
\date{\today}

\begin{document}

\maketitle
\thispagestyle{empty}
\newpage

\tableofcontents
\thispagestyle{empty}
\newpage

\setcounter{page}{1}

% --- 1. Executive Summary ---
\section{Executive Summary}

This report provides a comprehensive cybersecurity assessment for \textbf{Solaris Energy}, based on a correlation of network scan data, organizational security controls, and pre-existing risk information. The analysis reveals several critical and high-risk findings that require immediate attention to mitigate potential threats to the organization's data and infrastructure.

The most critical finding is the discovery of an open service on port 8080 with the title \textbf{"TOP SECRET DB"} on an internal system (\texttt{10.5.5.5}). This finding directly contradicts the existing risk register, which incorrectly lists this port as a secure false positive. This discrepancy suggests a significant, unmitigated exposure of a potentially sensitive database.

This technical vulnerability is compounded by critical gaps in organizational security controls. The lack of multi-factor authentication (MFA) for sensitive data systems and the absence of annual security awareness training for all employees create a high-risk environment. An attacker who compromises a single set of credentials could potentially gain direct access to sensitive data without facing additional security checks.

Immediate remediation is required to address the exposed service and implement mandatory MFA on all sensitive systems.

% --- 2. Organizational Information ---
\section{Organizational Information}

The following details were provided for the assessment.

\begin{itemize}
    \item \textbf{Organization Name:} Solaris Energy
    \item \textbf{Primary Email Domain:} \texttt{SolarisEnergy.net}
    \item \textbf{External IP Address:} \texttt{107.4.154.251}
\end{itemize}

% --- 3. Security Control Review ---
\section{Security Control Review}

A review of the organization's security controls was conducted via a standardized questionnaire. The responses highlight critical gaps in the current security posture. A summary is provided in Table \ref{tab:controls}.

\begin{table}[h!]
\centering
\caption{Security Control Questionnaire Analysis}
\label{tab:controls}
\begin{tabular}{p{0.6\linewidth} c c}
\toprule
\textbf{Control Question} & \textbf{Response} & \textbf{Status} \\
\midrule
Do you require MFA to access email? & Yes & \ding{51} \\
Do you require MFA to log into computers? & Yes & \ding{51} \\
\textbf{Do you require MFA to access sensitive data systems?} & \textbf{No} & \textbf{\color{red}\ding{55}} \\
Does your organization have an employee acceptable use policy? & Yes & \ding{51} \\
Does your organization do security awareness training for new employees? & Yes & \ding{51} \\
\textbf{Does your organization do security awareness training for all employees at least once per year?} & \textbf{No} & \textbf{\color{red}\ding{55}} \\
\bottomrule
\end{tabular}
\end{table}

The two "No" responses represent significant weaknesses:
\begin{itemize}
    \item \textbf{MFA for Sensitive Systems:} The absence of MFA on sensitive systems is a critical vulnerability. It removes a crucial layer of defense against credential theft, allowing an attacker with valid credentials to access high-value data unimpeded.
    \item \textbf{Annual Security Training:} Lack of recurring training for all employees increases susceptibility to phishing, social engineering, and other human-centric attacks.
\end{itemize}

% --- 4. Technical Scan Results ---
\section{Technical Scan Results}

A network scan was performed on the specified target to identify open ports and exposed services.

\begin{itemize}
    \item \textbf{Target IP Address:} \texttt{10.5.5.5}
\end{itemize}

\begin{table}[h!]
\centering
\caption{Open Port Analysis for \texttt{10.5.5.5}}
\label{tab:scan}
\begin{tabular}{c c l}
\toprule
\textbf{Port} & \textbf{State} & \textbf{Service / Information} \\
\midrule
8080 & Open & HTTP Title: \textbf{TOP SECRET DB} \\
\bottomrule
\end{tabular}
\end{table}

\paragraph{Analysis:} The scan identified port 8080 as open. The HTTP title discovered on this port, "TOP SECRET DB," is highly alarming. This strongly suggests that a database, potentially containing highly sensitive or classified information, is directly accessible over the network. This finding is especially concerning as it directly contradicts information from the current risk register (\textit{Input\_3\_Current\_Risks\_JSON}), which states this port is a secure false positive. This indicates the risk register is outdated and inaccurate, leading to a false sense of security.

% --- 5. Consolidated Risk Assessment ---
\section{Consolidated Risk Assessment}

The following table synthesizes findings from the security control review, technical scan, and pre-existing risk data to provide a holistic view of the primary risks facing the organization.

\begin{table}[h!]
\centering
\caption{Summary of Identified Risks}
\label{tab:risks}
\begin{tabular}{p{0.1\linewidth} p{0.4\linewidth} p{0.2\linewidth} p{0.2\linewidth}}
\toprule
\textbf{ID} & \textbf{Risk Description} & \textbf{Severity} & \textbf{Related Findings} \\
\midrule
\textbf{R-01} & \textbf{Potential Sensitive Database Exposure.} An open service on port 8080 is titled "TOP SECRET DB". This contradicts the existing risk register, which incorrectly labels it a false positive. & \textbf{Critical} & Technical Scan, Outdated Risk Data \\
\addlinespace
\textbf{R-02} & \textbf{Lack of MFA for Sensitive Systems.} No MFA is required to access sensitive data, creating a single point of failure if user credentials are compromised. & \textbf{Critical} & Security Control Review \\
\addlinespace
\textbf{R-03} & \textbf{Inadequate Security Awareness Program.} The absence of annual security training for all employees increases the likelihood of successful phishing and social engineering attacks. & \textbf{High} & Security Control Review \\
\bottomrule
\end{tabular}
\end{table}

% --- 6. Recommendations ---
\section{Recommendations}

The following actions are recommended to mitigate the identified risks. Recommendations are prioritized based on severity.

\subsection{R-01: Remediate Exposed Database (Critical)}
\begin{itemize}
    \item \textbf{Immediate Action:} Immediately investigate the service running on \texttt{10.5.5.5:8080}. Identify the data it contains and its business purpose. Restrict all access to this port via host-based or network firewalls until the investigation is complete.
    \item \textbf{Short-Term Fix:} If the service is required, ensure it is protected by strong authentication controls (including MFA), encrypted with TLS/SSL, and placed behind a web application firewall (WAF).
    \item \textbf{Process Improvement:} Immediately update the organization's risk register to reflect the true status of this service. Implement a formal process for regularly validating and updating risk assessments.
\end{itemize}

\subsection{R-02: Implement MFA on Sensitive Systems (Critical)}
\begin{itemize}
    \item \textbf{Immediate Action:} Prioritize and deploy MFA on all systems identified as containing sensitive data, with the highest priority given to database administration interfaces and other critical systems.
    \item \textbf{Long-Term Fix:} Develop and enforce a company-wide policy that mandates the use of MFA for accessing all internal and external services containing sensitive or critical information.
\end{itemize}

\subsection{R-03: Establish Annual Security Training (High)}
\begin{itemize}
    \item \textbf{Short-Term Fix:} Schedule and conduct mandatory security awareness training for all current employees within the next quarter. The training should cover phishing, password security, and acceptable use policies.
    \item \textbf{Long-Term Fix:} Establish a formal security awareness program that includes mandatory annual training and regular phishing simulations to ensure continuous employee education and vigilance.
\end{itemize}

\end{document}
```