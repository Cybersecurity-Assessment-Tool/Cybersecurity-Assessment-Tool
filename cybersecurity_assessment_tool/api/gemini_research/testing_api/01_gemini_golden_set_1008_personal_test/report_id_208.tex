```latex
\documentclass[12pt, a4paper]{article}

% Preamble: Required Packages
\usepackage[margin=1in]{geometry}
\usepackage{pifont} % For checkmarks and crosses
\usepackage{booktabs} % For professional tables
\usepackage{hyperref} % For clickable links and metadata
\usepackage{url} % For formatting URLs
\usepackage{seqsplit} % For splitting long strings in tt font
\usepackage{graphicx} % For potential logos
\usepackage[table]{xcolor} % For coloring table rows
\usepackage{fancyhdr} % For custom headers/footers
\usepackage{lastpage} % To get the total number of pages

% --- Document Metadata ---
\hypersetup{
    colorlinks=true,
    linkcolor=blue,
    filecolor=magenta,      
    urlcolor=cyan,
    pdftitle={Cybersecurity Posture Assessment},
    pdfauthor={Automated Security Report Generator},
    pdfsubject={Security Analysis},
    pdfkeywords={Cybersecurity, Risk, Assessment},
}

% --- Header and Footer Configuration ---
\pagestyle{fancy}
\fancyhf{} % Clear all header and footer fields
\fancyhead[L]{Cybersecurity Posture Assessment}
\fancyhead[R]{\textbf{Modern Myth}}
\fancyfoot[C]{\thepage\ of \pageref{LastPage}}
\renewcommand{\headrulewidth}{0.4pt}
\renewcommand{\footrulewidth}{0.4pt}

% --- Custom Commands ---
\newcommand{\yes}{\textcolor{green!70!black}{\ding{51}}}
\newcommand{\no}{\textcolor{red!80!black}{\ding{55}}}

% --- Document Start ---
\begin{document}

% --- Title Page ---
\begin{titlepage}
    \centering
    \vspace*{1cm}
    
    \includegraphics[width=0.3\textwidth]{example-image-a} % Placeholder for a logo
    
    \vspace{1.5cm}
    
    \Huge
    \textbf{Cybersecurity Posture Assessment Report}
    
    \vspace{1.5cm}
    
    \Large
    Prepared for: \\
    \vspace{0.5cm}
    \textbf{Modern Myth}
    
    \vspace{2cm}
    
    \large
    Report Date: \today
    
    \vfill
    
    \normalsize
    \textit{This report contains sensitive information and should be handled with care. Distribution is restricted to authorized personnel only.}
    
\end{titlepage}

\tableofcontents
\newpage

% --- Section 1: Executive Summary ---
\section{Executive Summary}
This report provides a cybersecurity posture assessment for \textbf{Modern Myth}, based on a technical network scan, a review of organizational security controls, and an analysis of pre-existing risks. The assessment was conducted on \today.

The key findings indicate a mixed security posture. On a positive note, the technical scan of the target host (\texttt{192.168.1.100}) revealed an excellent security configuration, with no open ports detected. This significantly reduces the external attack surface of the scanned asset.

However, critical gaps were identified in the organization's procedural and administrative controls. The two most significant risks stem from policy shortcomings rather than technical vulnerabilities:
\begin{itemize}
    \item \textbf{Critical Risk:} Multi-Factor Authentication (MFA) is not required for accessing sensitive data systems. This gap exposes the organization's most valuable assets to potential unauthorized access and compromise.
    \item \textbf{High Risk:} The lack of mandatory, annual security awareness training for all employees increases susceptibility to phishing, social engineering, and other human-centric attacks.
\end{itemize}

While the technical perimeter of the assessed system is strong, the identified policy gaps present a significant threat to the overall security of \textbf{Modern Myth}. Recommendations in this report prioritize addressing these procedural weaknesses to build a more resilient, defense-in-depth security strategy.

\newpage

% --- Section 2: Organizational Information ---
\section{Organizational Information}
The following details were provided for the assessment.

\begin{tabular}{@{}ll}
    \toprule
    \textbf{Attribute} & \textbf{Value} \\
    \midrule
    Organization Name & \textbf{Modern Myth} \\
    Email Domain & \texttt{ModernMyth.com} \\
    Website Domain & \url{www.ModernMyth.com} \\
    External IP Address & \texttt{80.146.188.163} \\
    \bottomrule
\end{tabular}

% --- Section 3: Security Control Review ---
\section{Security Control Review}
A review of administrative and procedural security controls was conducted via a questionnaire. The responses highlight areas of both strength and weakness in the current security policy framework. Gaps identified here directly contribute to the organization's risk profile.

\begin{table}[h!]
\centering
\begin{tabular}{p{0.6\linewidth} c c}
    \toprule
    \textbf{Control Question} & \textbf{Response} & \textbf{Status} \\
    \midrule
    Do you require MFA to access email? & Yes & \yes \\
    Do you require MFA to log into computers? & Yes & \yes \\
    \rowcolor{red!15}
    Do you require MFA to access sensitive data systems? & No & \no \\
    Does your organization have an employee acceptable use policy? & Yes & \yes \\
    Does your organization do security awareness training for new employees? & Yes & \yes \\
    \rowcolor{orange!20}
    Does your organization do security awareness training for all employees at least once per year? & No & \no \\
    \bottomrule
\end{tabular}
\caption{Organizational Security Control Status.}
\label{tab:controls}
\end{table}

\newpage

% --- Section 4: Technical Scan Results ---
\section{Technical Scan Results}
An external network scan was performed to identify open ports and exposed services on the specified target system.

\subsection{Scan Summary}
\begin{itemize}
    \item \textbf{Target IP Address:} \texttt{192.168.1.100}
    \item \textbf{Scan Date:} \today
    \item \textbf{Key Finding:} \textbf{No open ports were detected.} All scanned ports were found to be in a 'closed' state.
\end{itemize}

\subsection{Analysis}
The absence of open ports on the scanned host is an excellent security finding. This indicates a strong firewall configuration and adherence to the principle of least privilege, effectively minimizing the system's attack surface. No network-based vulnerabilities could be identified on this host.

% --- Section 5: Risk Assessment ---
\section{Risk Assessment}
This section synthesizes findings from the security control review, technical scan, and pre-existing risk data. The primary risks identified are procedural and administrative in nature.

\begin{table}[h!]
\centering
\begin{tabular}{p{0.1\linewidth} p{0.3\linewidth} p{0.15\linewidth} p{0.35\linewidth}}
    \toprule
    \textbf{Risk ID} & \textbf{Risk Name} & \textbf{Severity} & \textbf{Description} \\
    \midrule
    \rowcolor{red!15}
    RISK-001 & Lack of MFA on Sensitive Systems & \textbf{Critical} & The absence of MFA for systems containing sensitive data creates a single point of failure (passwords) for protecting critical assets. A compromised password could lead to a significant data breach. \\
    \addlinespace[3pt]
    \rowcolor{orange!20}
    RISK-002 & Inadequate Annual Security Training & \textbf{High} & Without regular, mandatory security training, employees are more likely to fall victim to phishing, malware, and social engineering attacks, potentially providing an initial access vector for attackers. \\
    \bottomrule
\end{tabular}
\caption{Summary of Identified Risks.}
\label{tab:risks}
\end{table}

\newpage

% --- Section 6: Recommendations ---
\section{Recommendations}
The following actionable recommendations are provided to mitigate the identified risks and improve the overall security posture of \textbf{Modern Myth}.

\subsection{RISK-001 (Critical): Implement MFA for Sensitive Systems}
\begin{itemize}
    \item \textbf{Action:} Mandate the use of Multi-Factor Authentication (MFA) for all user accounts, especially privileged accounts, that have access to systems classified as containing sensitive or critical data.
    \item \textbf{Justification:} This action directly mitigates the risk of unauthorized access due to compromised credentials. Even if an attacker obtains a valid password, they will be unable to log in without the second factor.
    \item \textbf{Implementation Guidance:} Deploy a robust MFA solution (e.g., TOTP apps like Google Authenticator, hardware tokens, or push notifications) across all critical applications, databases, and administrative interfaces.
\end{itemize}

\subsection{RISK-002 (High): Establish Annual Security Training Program}
\begin{itemize}
    \item \textbf{Action:} Develop and implement a mandatory annual security awareness training program for all employees, including management.
    \item \textbf{Justification:} Regular training reinforces security best practices, keeps employees aware of evolving threats (like new phishing techniques), and fosters a security-conscious culture, reducing the likelihood of human error leading to a compromise.
    \item \textbf{Implementation Guidance:} The program should cover key topics such as phishing identification, password hygiene, acceptable use, and incident reporting. Consider using a third-party training platform to track completion and conduct phishing simulations.
\end{itemize}

% --- Section 7: Conclusion ---
\section{Conclusion}
The security assessment of \textbf{Modern Myth} reveals a commendable technical security posture on the scanned asset, characterized by a well-hardened network perimeter. However, this strength is undermined by significant procedural weaknesses. The lack of mandatory MFA on sensitive systems and the absence of annual security awareness training for all staff represent the most pressing threats.

By implementing the critical and high-priority recommendations outlined in this report, \textbf{Modern Myth} can substantially reduce its risk exposure and build a more comprehensive and resilient security program that protects both its technical and human elements.

\end{document}
```