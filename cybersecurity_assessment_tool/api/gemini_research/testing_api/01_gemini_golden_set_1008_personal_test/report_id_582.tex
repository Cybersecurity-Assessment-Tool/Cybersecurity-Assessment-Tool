```latex
\documentclass[12pt]{article}

% Required Packages
\usepackage[margin=1in]{geometry}
\usepackage{pifont} % For checkmarks and crosses
\usepackage{booktabs} % For professional tables
\usepackage{hyperref} % For clickable links and TOC
\usepackage{url} % For formatting URLs
\usepackage{seqsplit} % For breaking long strings
\usepackage{xcolor} % For custom colors

% Hyperref Configuration
\hypersetup{
    colorlinks=true,
    linkcolor=black,
    filecolor=magenta,      
    urlcolor=blue,
    pdftitle={Cybersecurity Assessment Report},
    pdfpagemode=FullScreen,
}

% Custom Commands for Readability
\newcommand{\yes}{\textcolor{green}{\ding{51}}}
\newcommand{\no}{\textcolor{red}{\ding{55}}}
\newcommand{\orgname}{Catalyst Consulting}
\newcommand{\orgip}{\texttt{50.153.239.15}}
\newcommand{\orgdomain}{\texttt{CatalystConsulting.com}}
\newcommand{\targetip}{\texttt{192.168.0.5}}

\begin{document}

% --- TITLE PAGE ---
\begin{titlepage}
    \centering
    \vspace*{1cm}
    \Huge\textbf{Cybersecurity Assessment Report}
    \vspace{1.5cm}
    \Large\textbf{Prepared For:} \\
    \vspace{0.5cm}
    \huge{\orgname}
    \vfill
    \large
    \textbf{Date:} \today \\
    \textbf{Report ID:} CYBER-2023-001
    \vspace{1cm}
\end{gantt}
\end{titlepage}

% --- TABLE OF CONTENTS ---
\tableofcontents
\newpage

% --- EXECUTIVE SUMMARY ---
\section{Executive Summary}
This report provides a comprehensive cybersecurity assessment for \textbf{\orgname}, based on an analysis of organizational data, technical network scans, and a review of pre-existing risks.

The assessment reveals a mixed security posture. The organization has implemented several foundational security controls, including mandatory Multi-Factor Authentication (MFA) for email and computer access, and maintains a consistent security awareness training program. These are commendable practices that significantly reduce common cyber threats.

However, a \textbf{critical security gap} was identified: the absence of mandatory MFA for accessing sensitive data systems. This oversight exposes critical assets to significant risk from compromised credentials.

On the technical front, the external network scan of host \targetip{} revealed a strong security configuration, with no open ports discovered. This finding indicates that a previously identified risk, "Unencrypted Web Server" on Port 80, has been successfully remediated.

Our primary recommendation is the immediate implementation of MFA across all systems housing sensitive data. Further details on all findings and actionable recommendations are provided in the subsequent sections of this report.

% --- ORGANIZATIONAL INFORMATION ---
\section{Organizational Information}
The following details were provided for the assessment. This information helps establish the context and scope of the review.

\begin{tabular}{@{}ll}
\toprule
\textbf{Attribute} & \textbf{Value} \\
\midrule
Organization Name & \orgname \\
Primary Email Domain & \seqsplit{\orgdomain} \\
External IP Address & \orgip \\
\bottomrule
\end{tabular}

% --- SECURITY CONTROL REVIEW ---
\section{Security Control Review}
A review of the organization's security controls was conducted based on a standardized questionnaire. The results highlight areas of both strength and weakness in the current security policy framework.

\subsection{Questionnaire Results}
\begin{tabular}{@{}p{0.8\linewidth}c}
\toprule
\textbf{Control Question} & \textbf{Status} \\
\midrule
Do you require MFA to access email? & \yes \\
Do you require MFA to log into computers? & \yes \\
\textbf{Do you require MFA to access sensitive data systems?} & \no \\
Does your organization have an employee acceptable use policy? & \yes \\
Does your organization do security awareness training for new employees? & \yes \\
Does your organization do security awareness training for all employees at least once per year? & \yes \\
\bottomrule
\end{tabular}

\subsection{Analysis}
The organization demonstrates a strong commitment to endpoint and communication security by enforcing MFA for computer and email access. However, the lack of MFA for sensitive data systems is a significant vulnerability. An attacker with valid credentials could gain direct, unrestricted access to the organization's most valuable information. This gap undermines the security provided by other controls and must be addressed as a high priority.

% --- TECHNICAL SCAN RESULTS ---
\section{Technical Scan Results}
A network scan was performed to identify discoverable services and potential vulnerabilities on the organization's infrastructure.

\begin{itemize}
    \item \textbf{Target IP Address:} \targetip{}
    \item \textbf{Scan Type:} Nmap Port Scan
\end{itemize}

\subsection{Port Scan Findings}
The scan of the target host revealed no open ports. The state of all scanned ports was either `closed` or `filtered`. The table below summarizes the findings for a key port of interest.

\begin{tabular}{@{}ccccc}
\toprule
\textbf{Port} & \textbf{State} & \textbf{Service} & \textbf{Product} & \textbf{Version} \\
\midrule
80/tcp & closed & http & N/A & N/A \\
\bottomrule
\end{tabular}

\subsection{Analysis}
The technical scan indicates a hardened external posture for the target host. The fact that Port 80 (HTTP) is `closed` is a positive security finding. This directly contradicts a pre-existing risk entry ("Unencrypted Web Server"), suggesting that this vulnerability has been successfully remediated since it was last documented. This is a positive development and demonstrates proactive security management.

% --- CONSOLIDATED RISK ASSESSMENT ---
\section{Consolidated Risk Assessment}
This section synthesizes findings from the security control review, technical scan, and pre-existing risk data into a consolidated list.

\begin{tabular}{@{}p{0.5\linewidth}lll}
\toprule
\textbf{Risk Description} & \textbf{Severity} & \textbf{Status} & \textbf{Source} \\
\midrule
\textbf{Lack of MFA on Sensitive Data Systems} & \textbf{High} & \textbf{Active} & Questionnaire \\
Unencrypted Web Server (Port 80) & Medium & Remediated & Input 3 / Scan \\
\bottomrule
\end{tabular}

% --- RECOMMENDATIONS ---
\section{Recommendations}
Based on the consolidated risk assessment, the following actions are recommended to enhance the security posture of \orgname{}.

\subsection{High Priority}
\begin{itemize}
    \item \textbf{Implement MFA for All Sensitive Systems:}
    \begin{itemize}
        \item \textbf{Action:} Enforce MFA for all user accounts, including administrative and service accounts, that have access to databases, file shares, or applications containing sensitive or critical data.
        \item \textbf{Justification:} This is the single most effective control to prevent unauthorized access resulting from credential theft (e.g., phishing, password spraying). It adds a critical layer of defense for the organization's most valuable assets.
    \end{itemize}
\end{itemize}

\subsection{Informational}
\begin{itemize}
    \item \textbf{Validate Remediation Efforts:}
    \begin{itemize}
        \item \textbf{Action:} Acknowledge the successful remediation of the open Port 80 vulnerability. Continue to conduct regular, authenticated vulnerability scans to confirm that security configurations are maintained and that new vulnerabilities are identified promptly.
        \item \textbf{Justification:} Continuous monitoring ensures that the security posture does not degrade over time and that fixes are effective.
    \end{itemize}
\end{itemize}

\end{document}
```