```latex
\documentclass[12pt]{article}

% Preamble: Required Packages
\usepackage[margin=1in]{geometry}
\usepackage{pifont} % For checkmarks and crosses
\usepackage{booktabs} % For professional tables
\usepackage{hyperref} % For clickable links
\usepackage{url} % For formatting URLs
\usepackage{seqsplit} % For splitting long strings to prevent overflow

% Document Metadata
\title{Cybersecurity Posture Assessment Report}
\author{Cybersecurity Analysis Division}
\date{\today}

\begin{document}

\maketitle
\thispagestyle{empty}
\newpage
\tableofcontents
\newpage

% --- 1. Executive Summary ---
\section{Executive Summary}
This report provides a comprehensive analysis of the cybersecurity posture for \textbf{Apex Legends Group}. The assessment is based on a correlation of network scan data, a security controls questionnaire, and a review of previously identified risks.

The analysis has uncovered several \textbf{critical-risk} findings that require immediate attention. A publicly accessible FTP server was identified running a notoriously vulnerable version of \texttt{vsftpd (2.3.4)}, which is known to contain a backdoor. This is compounded by the allowance of anonymous logins, presenting a severe and immediate threat of unauthorized access and system compromise.

Furthermore, significant gaps were identified in the organization's identity and access management controls. The absence of Multi-Factor Authentication (MFA) for email and computer logins drastically increases the risk of account takeover. This, combined with a lack of annual security awareness training, leaves the organization highly susceptible to phishing and other social engineering attacks.

Immediate remediation of the vulnerable FTP server and the phased implementation of MFA are the highest priority recommendations outlined in this report.

% --- 2. Organizational Information ---
\section{Organizational Information}
This section details the information provided by the client for the scope of this assessment.

\begin{itemize}
    \item \textbf{Organization Name:} Apex Legends Group
    \item \textbf{Email Domain:} \texttt{ApexLegendsGroup.com}
    \item \textbf{Website Domain:} \url{www.ApexLegendsGroup.com}
    \item \textbf{External IP Address:} \texttt{15.56.179.29}
\end{itemize}

% --- 3. Security Control Review (Questionnaire Analysis) ---
\section{Security Control Review}
The following table summarizes the organization's responses to a security controls questionnaire. Items marked with a red 'X' (\ding{55}) indicate a deviation from security best practices and represent a significant gap in the defensive posture.

\begin{table}[h!]
\centering
\caption{Security Controls Questionnaire Results}
\begin{tabular}{p{0.7\linewidth} c c}
\toprule
\textbf{Control Question} & \textbf{Response} & \textbf{Status} \\
\midrule
Do you require MFA to access email? & No & \ding{55} \\
Do you require MFA to log into computers? & No & \ding{55} \\
Do you require MFA to access sensitive data systems? & Yes & \ding{51} \\
Does your organization have an employee acceptable use policy? & Yes & \ding{51} \\
Does your organization do security awareness training for new employees? & Yes & \ding{51} \\
Does your organization do security awareness training for all employees at least once per year? & No & \ding{55} \\
\bottomrule
\end{tabular}
\end{table}

\subsection*{Analysis of Gaps}
\begin{itemize}
    \item \textbf{Lack of MFA:} The absence of MFA for email and general computer access is a critical vulnerability. Stolen credentials could directly lead to unauthorized access to sensitive communications and internal network resources.
    \item \textbf{Lack of Annual Training:} Security threats evolve constantly. Without annual refresher training, employees are more likely to fall victim to modern phishing and social engineering tactics, undermining other security controls.
\end{itemize}

% --- 4. Technical Scan Results ---
\section{Technical Scan Results}
An external network scan was performed to identify exposed services and potential vulnerabilities.

\begin{itemize}
    \item \textbf{Scan Target:} \texttt{10.0.0.15}
    \item \textbf{Scan Tool:} Nmap
\end{itemize}

The following table details the open ports and services discovered on the target system.

\begin{table}[h!]
\centering
\caption{Discovered Open Ports and Services}
\begin{tabular}{l l l l p{0.3\linewidth}}
\toprule
\textbf{Port} & \textbf{Service} & \textbf{Product} & \textbf{Version} & \textbf{Notes} \\
\midrule
21/tcp & ftp & vsftpd & 2.3.4 & \textbf{CRITICAL:} This version is known to be vulnerable to a backdoor (CVE-2011-2523). Anonymous FTP login is also permitted. \\
\bottomrule
\end{tabular}
\end{table}

\subsection*{Analysis of Findings}
The discovery of \textbf{\texttt{vsftpd version 2.3.4}} is a finding of the highest severity. This specific version was compromised, and a backdoor was inserted into the source code, which could allow an attacker to gain a command shell on the server. The configuration also allows for anonymous FTP logins, making the service trivial for an attacker to access and exploit.

% --- 5. Overall Risk Assessment ---
\section{Overall Risk Assessment}
This section synthesizes all findings from the questionnaire, technical scan, and pre-existing risk register into a consolidated list of organizational risks.

\begin{table}[h!]
\centering
\caption{Consolidated Risk Summary}
\begin{tabular}{p{0.2\linewidth} p{0.5\linewidth} l}
\toprule
\textbf{Risk Title} & \textbf{Description} & \textbf{Severity} \\
\midrule
\textbf{Vulnerable FTP Server} & An internet-facing FTP server is running \texttt{vsftpd 2.3.4}, which contains a critical backdoor vulnerability. Anonymous login is enabled. & \textbf{Critical} \\
\addlinespace
\textbf{Lack of MFA} & No MFA is enforced for access to email or workstations, exposing the organization to account takeover via credential theft. & \textbf{Critical} \\
\addlinespace
\textbf{Inadequate Security Training} & Security awareness training is not conducted annually, increasing the risk of employees falling victim to phishing and social engineering. & \textbf{High} \\
\addlinespace
\textbf{Outdated Windows Policy} & Pre-existing risk: Workstations are running Windows 7, an unsupported OS that no longer receives security updates. & \textbf{Medium} \\
\bottomrule
\end{tabular}
\end{table}

% --- 6. Recommendations ---
\section{Recommendations}
The following actionable recommendations are provided to mitigate the identified risks. They are prioritized based on severity and exploitability.

\subsection*{Priority 1: Immediate Actions}
\begin{enumerate}
    \item \textbf{Remediate Vulnerable FTP Server:}
    \begin{itemize}
        \item Take the server at \texttt{10.0.0.15} offline \textbf{immediately}.
        \item If the FTP service is business-critical, rebuild the server using a modern, patched version of an FTP daemon (e.g., latest vsftpd, ProFTPD).
        \item \textbf{Disable anonymous access} and enforce strong, unique passwords for all FTP accounts.
        \item Consider using more secure file transfer protocols like SFTP (SSH File Transfer Protocol).
    \end{itemize}
\end{enumerate}

\subsection*{Priority 2: High-Impact Actions}
\begin{enumerate}
    \setcounter{enumi}{1} % Continue numbering
    \item \textbf{Implement Multi-Factor Authentication (MFA):}
    \begin{itemize}
        \item Begin a project to roll out MFA for all employees.
        \item Prioritize enforcement on email systems (e.g., Office 365, Google Workspace) and any remote access solutions (e.g., VPN).
        \item Plan for subsequent rollout to all workstation logins.
    \end{itemize}
\end{enumerate}

\subsection*{Priority 3: Foundational Improvements}
\begin{enumerate}
    \setcounter{enumi}{2} % Continue numbering
    \item \textbf{Establish Annual Security Awareness Training:}
    \begin{itemize}
        \item Procure or develop a security awareness training program.
        \item Make the training mandatory for all employees on an annual basis.
        \item Include modules on phishing identification, password hygiene, and acceptable use.
    \end{itemize}
    \item \textbf{Upgrade End-of-Life Systems:}
    \begin{itemize}
        \item Accelerate the existing plan to upgrade all Windows 7 workstations to a supported operating system, such as Windows 10 or 11.
    \end{itemize}
\end{enumerate}

\end{document}
```