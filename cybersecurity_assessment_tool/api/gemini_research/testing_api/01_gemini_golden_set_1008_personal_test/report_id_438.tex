```latex
\documentclass[12pt]{article}

% Preamble: Required Packages
\usepackage[margin=1in]{geometry}
\usepackage{pifont} % For checkmarks and crosses
\usepackage{booktabs} % For professional tables
\usepackage{hyperref} % For clickable links
\usepackage{url} % For URL formatting
\usepackage{seqsplit} % To split long strings without breaking
\usepackage{graphicx}
\usepackage{xcolor}
\usepackage{lastpage}
\usepackage{fancyhdr}

% --- Document Setup ---
\hypersetup{
    colorlinks=true,
    linkcolor=blue,
    filecolor=magenta,      
    urlcolor=cyan,
    pdftitle={Cybersecurity Posture Report},
    pdfpagemode=FullScreen,
}

% Define colors for severity
\definecolor{criticalred}{HTML}{D10000}
\definecolor{highorange}{HTML}{E25F00}
\definecolor{mediumyellow}{HTML}{F5A623}
\definecolor{lowblue}{HTML}{4A90E2}
\definecolor{infogray}{HTML}{808080}

% Header and Footer
\pagestyle{fancy}
\fancyhf{} % clear all header and footer fields
\fancyhead[L]{Cybersecurity Posture Report}
\fancyhead[R]{Digital Drift}
\fancyfoot[C]{\thepage\ of \pageref{LastPage}}
\renewcommand{\headrulewidth}{0.4pt}
\renewcommand{\footrulewidth}{0.4pt}

% --- Document Start ---
\begin{document}

% --- Title Page ---
\begin{titlepage}
    \centering
    \vfill
    {\Huge\bfseries Cybersecurity Posture Report\par}
    \vspace{1.5cm}
    {\Large Prepared for:\par}
    \vspace{0.5cm}
    {\huge\bfseries Digital Drift\par}
    \vspace{2cm}
    {\large Report Date: \today\par}
    \vfill
    \textit{This report contains sensitive information and should be handled with care.}
\end{titlepage}

\tableofcontents
\newpage

% --- Section 1: Executive Summary ---
\section{Executive Summary}

This report provides a comprehensive analysis of the cybersecurity posture for Digital Drift, based on a combination of technical network scanning, a review of existing risk documentation, and an organizational security controls questionnaire.

The assessment identified several critical and high-risk gaps in the organization's security controls. The most pressing concerns are the lack of Multi-Factor Authentication (MFA) for accessing email and logging into computers. These deficiencies expose the organization to significant risks, including business email compromise (BEC), ransomware, and unauthorized access to the internal network.

Furthermore, a gap was identified in the security training program, where new employees do not receive mandatory security awareness training upon hiring. This increases the organization's susceptibility to social engineering and phishing attacks.

On a positive note, the technical scan of the internal asset \texttt{192.168.0.5} did not validate a pre-existing risk related to an unencrypted web server, as the relevant port (80/tcp) was found to be closed. This may indicate that the risk has been remediated or applies to a different asset.

Recommendations are provided to address these findings in a prioritized manner, focusing on the immediate implementation of MFA and the integration of security training into the employee onboarding process.

\newpage

% --- Section 2: Organizational Information ---
\section{Organizational Information}
The following information was provided for the assessment.

\begin{tabular}{@{}ll}
    \toprule
    \textbf{Attribute} & \textbf{Value} \\
    \midrule
    Organization Name & Digital Drift \\
    Email Domain & \texttt{DigitalDrift.com} \\
    Website Domain & \url{www.DigitalDrift.com} \\
    External IP Address & \texttt{64.190.55.15} \\
    \bottomrule
\end{tabular}

% --- Section 3: Security Control Review ---
\section{Security Control Review}
The following table summarizes the responses from the organizational security questionnaire. Items marked with \ding{55} represent significant gaps in security controls and are discussed in the Risk Assessment section.

\begin{table}[h!]
\centering
\begin{tabular}{@{}lc}
    \toprule
    \textbf{Control Question} & \textbf{Response} \\
    \midrule
    Do you require MFA to access email? & {\color{criticalred}\ding{55}} \\
    Do you require MFA to log into computers? & {\color{criticalred}\ding{55}} \\
    Do you require MFA to access sensitive data systems? & {\color{green}\ding{51}} \\
    Does your organization have an employee acceptable use policy? & {\color{green}\ding{51}} \\
    Does your organization do security awareness training for new employees? & {\color{highorange}\ding{55}} \\
    Does your organization do security awareness training for all employees at least once per year? & {\color{green}\ding{51}} \\
    \bottomrule
\end{tabular}
\caption{Security Controls Questionnaire Results}
\end{table}

% --- Section 4: Technical Scan Results ---
\section{Technical Scan Results}
A network port scan was conducted to identify accessible services on the specified target system.

\subsection{Target: \texttt{192.168.0.5}}
The scan confirmed that the host was online and responsive. The following table details the state of network ports discovered during the scan.

\begin{table}[h!]
\centering
\begin{tabular}{@{}lllll}
    \toprule
    \textbf{Port} & \textbf{State} & \textbf{Service} & \textbf{Product} & \textbf{Version} \\
    \midrule
    80/tcp & closed & http & N/A & N/A \\
    \bottomrule
\end{tabular}
\caption{Port Scan Results for \texttt{192.168.0.5}}
\end{table}

\textbf{Analysis:} The scan of the internal host \texttt{192.168.0.5} revealed no open ports. Port 80, commonly used for unencrypted web traffic, was confirmed to be closed. This is a positive security finding for this specific asset. This result contradicts a pre-existing risk entry, suggesting that risk may be outdated or related to a different system.

\newpage

% --- Section 5: Risk Assessment ---
\section{Risk Assessment}
This section correlates findings from the security control review, technical scans, and pre-existing risk documentation to provide a unified view of the current risk landscape.

\begin{table}[h!]
\centering
\resizebox{\textwidth}{!}{%
\begin{tabular}{@{}llll}
    \toprule
    \textbf{Risk Name} & \textbf{Severity} & \textbf{Affected Elements} & \textbf{Description} \\
    \midrule
    \textbf{Lack of MFA for Email Access} & \colorbox{criticalred}{\color{white}\textbf{CRITICAL}} & All Email Accounts & The absence of MFA on email exposes the organization to account takeover \\
    & & & via phishing or credential stuffing, leading to data breaches or BEC. \\
    \addlinespace
    \textbf{Lack of MFA for Computer Logins} & \colorbox{criticalred}{\color{white}\textbf{CRITICAL}} & Employee Endpoints & A compromised password could grant an attacker direct access to an \\
    & & & employee's computer, bypassing perimeter defenses to access the internal network. \\
    \addlinespace
    \textbf{Inadequate Onboarding Security} & \colorbox{highorange}{\color{white}\textbf{HIGH}} & New Hire Process & New employees do not receive security training upon hiring, making them \\
    \textbf{Training} & & & more susceptible to social engineering and unaware of security policies. \\
    \addlinespace
    \textbf{Unencrypted Web Server} & \colorbox{mediumyellow}{\color{black}\textbf{MEDIUM}} & Port 80 & A pre-existing risk stated Port 80 was open. The scan of \texttt{192.168.0.5} \\
    \textbf{(Not Validated)} & & & found this port closed. The risk may be remediated or apply to another asset. \\
    \bottomrule
\end{tabular}%
}
\caption{Summary of Identified Risks}
\end{table}

% --- Section 6: Recommendations ---
\section{Recommendations}
The following actionable recommendations are provided to mitigate the identified risks, prioritized by severity.

\subsection{Priority 1: Implement Comprehensive MFA (Critical)}
\begin{itemize}
    \item \textbf{Action:} Enforce mandatory Multi-Factor Authentication (MFA) for all user accounts across all critical systems.
    \item \textbf{Details:}
        \begin{itemize}
            \item \textbf{Email:} Immediately enable and enforce MFA for the \texttt{DigitalDrift.com} email domain (e.g., via Microsoft 365 Conditional Access, Google Workspace 2-Step Verification).
            \item \textbf{Endpoints:} Deploy an MFA solution for all Windows, macOS, and Linux computer logins (e.g., Duo, Windows Hello for Business).
        \end{itemize}
    \item \textbf{Justification:} This is the single most effective control to prevent unauthorized access resulting from compromised credentials and will significantly reduce the risk of account takeovers and ransomware attacks.
\end{itemize}

\subsection{Priority 2: Enhance Security Training Program (High)}
\begin{itemize}
    \item \textbf{Action:} Integrate mandatory security awareness training into the new employee onboarding process.
    \item \textbf{Details:} This initial training should be completed within the first week of employment and cover, at a minimum: phishing identification, acceptable use of company assets, password security policies, and how to report a security incident.
    \item \textbf{Justification:} A strong security culture starts on day one. Training new hires immediately reduces the likelihood of them falling victim to early social engineering attempts.
\end{itemize}

\subsection{Priority 3: Validate and Update Risk Register (Medium)}
\begin{itemize}
    \item \textbf{Action:} Conduct a comprehensive vulnerability scan of all organizational assets to validate the status of pre-existing risks.
    \item \textbf{Details:} Specifically, perform an external scan on the IP address \texttt{64.190.55.15} to determine if the "Unencrypted Web Server" risk is present there. All findings should be used to update the central risk register.
    \item \textbf{Justification:} An accurate and up-to-date risk register is essential for effective risk management. This action will clarify whether the Port 80 risk is resolved or simply located on a different asset.
\end{itemize}

% --- Document End ---
\end{document}
```