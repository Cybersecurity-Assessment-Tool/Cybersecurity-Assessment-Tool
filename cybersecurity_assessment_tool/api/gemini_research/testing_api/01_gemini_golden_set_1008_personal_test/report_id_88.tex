```latex
\documentclass[12pt]{article}

% --- PACKAGES ---
\usepackage[margin=1in]{geometry}
\usepackage{pifont} % For checkmarks and crosses
\usepackage{booktabs} % For professional tables
\usepackage{hyperref} % For clickable links and references
\usepackage{url} % For formatting URLs
\usepackage{seqsplit} % To split long strings without spaces
\usepackage{graphicx}
\usepackage{fancyhdr}
\usepackage{lastpage}
\usepackage[table]{xcolor}

% --- DOCUMENT & HYPERREF SETUP ---
\hypersetup{
    colorlinks=true,
    linkcolor=blue,
    filecolor=magenta,      
    urlcolor=cyan,
    pdftitle={Cybersecurity Assessment Report},
    pdfpagemode=FullScreen,
}

% --- HEADER & FOOTER ---
\pagestyle{fancy}
\fancyhf{} % Clear all header and footer fields
\fancyhead[L]{Cybersecurity Assessment Report}
\fancyhead[R]{Confidential}
\fancyfoot[C]{\thepage\ of \pageref{LastPage}}
\fancyfoot[R]{Neon Pulse Entertainment}

% --- COMMANDS ---
\newcommand{\yes}{\ding{51}}
\newcommand{\no}{\ding{55}}
\definecolor{lightgray}{gray}{0.9}

% --- DOCUMENT START ---
\begin{document}

% --- TITLE PAGE ---
\begin{titlepage}
    \centering
    \vspace*{1cm}
    \includegraphics[width=0.4\textwidth]{example-image-a} % Placeholder for company logo
    \vfill
    \huge\textbf{Cybersecurity Assessment Report}
    \vspace{1.5cm}
    \Large
    \textbf{Prepared for:}\\
    Neon Pulse Entertainment
    \vspace{2cm}
    \large
    \textbf{Date of Report:}\\
    \today
    \vfill
    \textit{This document contains sensitive and confidential information. Distribution is restricted.}
\end{titlepage}

\tableofcontents
\newpage

% --- EXECUTIVE SUMMARY ---
\section{Executive Summary}
This report provides a cybersecurity assessment for Neon Pulse Entertainment, conducted on \today. The analysis combines a review of organizational security controls, a technical network scan, and an evaluation of known risks.

The assessment reveals a mixed security posture. On one hand, the organization demonstrates strong foundational controls, particularly in the mandatory use of Multi-Factor Authentication (MFA) across key systems. The technical scan of the target host (\texttt{192.168.1.100}) was positive, showing no open ports and indicating a minimal attack surface.

However, a \textbf{critical gap} was identified in the organization's security awareness program. The lack of mandatory security training for both new and existing employees presents a high risk. This gap exposes the organization to human-centric threats such as phishing, social engineering, and malware infection, which can bypass even strong technical controls.

Immediate action is recommended to implement a comprehensive security awareness training program to mitigate this significant human-factor risk.

% --- ORGANIZATIONAL INFORMATION ---
\section{Organizational Information}
The following details were provided for the assessment.

\begin{tabular}{@{}ll}
    \toprule
    \textbf{Attribute} & \textbf{Value} \\
    \midrule
    Organization Name & Neon Pulse Entertainment \\
    Email Domain & \seqsplit{\texttt{NeonPulseEntertainment.net}} \\
    Website Domain & \seqsplit{\url{www.NeonPulseEntertainment.net}} \\
    External IP Address & \texttt{142.91.81.112} \\
    \bottomrule
\end{tabular}

% --- SECURITY CONTROL REVIEW ---
\section{Security Control Review}
A review of administrative and procedural security controls was conducted based on a standardized questionnaire. The results are summarized below. "No" answers indicate potential gaps in the security framework.

\begin{table}[h!]
\centering
\caption{Security Controls Questionnaire Results}
\begin{tabular}{p{0.6\linewidth} c p{0.2\linewidth}}
    \toprule
    \textbf{Control Question} & \textbf{Response} & \textbf{Assessment} \\
    \midrule
    Do you require MFA to access email? & \yes & Compliant \\
    Do you require MFA to log into computers? & \yes & Compliant \\
    Do you require MFA to access sensitive data systems? & \yes & Compliant \\
    Does your organization have an employee acceptable use policy? & \yes & Compliant \\
    \rowcolor{lightgray}
    Does your organization do security awareness training for new employees? & \no & \textbf{High Risk Gap} \\
    \rowcolor{lightgray}
    Does your organization do security awareness training for all employees at least once per year? & \no & \textbf{High Risk Gap} \\
    \bottomrule
\end{tabular}
\end{table}

The absence of a formal security awareness training program is the most significant finding from this review. Employees are the first line of defense, and without proper training, they are more susceptible to attacks that can compromise organizational data and systems.

% --- TECHNICAL SCAN RESULTS ---
\section{Technical Scan Results}
A network scan was performed to identify open ports and services on the specified target system.

\begin{itemize}
    \item \textbf{Target IP Address:} \texttt{192.168.1.100}
    \item \textbf{Scan Date:} \today
    \item \textbf{Status:} Host is Up
\end{itemize}

\textbf{Findings:}
The scan completed successfully and determined the host was online. However, \textbf{no open TCP ports were discovered}. All 1000 scanned ports were in a 'closed' state.

\textbf{Analysis:}
This is a positive security finding. A host with no open ports presents a minimal network attack surface, making it significantly more difficult for an external or internal attacker to compromise. This suggests a well-configured host-based firewall or that the device serves no network-facing functions.

% --- RISK ASSESSMENT ---
\section{Risk Assessment}
This section synthesizes findings from the security control review, technical scans, and pre-existing risk data. The primary risk identified during this assessment is detailed below. No pre-existing vulnerabilities were reported.

\begin{table}[h!]
\centering
\caption{Identified Risks}
\begin{tabular}{p{0.25\linewidth} p{0.5\linewidth} l}
    \toprule
    \textbf{Risk Name} & \textbf{Overview} & \textbf{Severity} \\
    \midrule
    \rowcolor{lightgray}
    Lack of Security Awareness Training & The absence of a formal training program for new and existing employees leaves the organization vulnerable to phishing, malware, and social engineering attacks. Untrained staff are more likely to click malicious links, open infected attachments, or divulge sensitive information. & \textbf{High} \\
    \bottomrule
\end{tabular}
\end{table}

% --- RECOMMENDATIONS ---
\section{Recommendations}
Based on the assessment findings, the following actions are recommended to enhance the cybersecurity posture of Neon Pulse Entertainment.

\begin{table}[h!]
\centering
\caption{Actionable Recommendations}
\begin{tabular}{p{0.25\linewidth} p{0.65\linewidth}}
    \toprule
    \textbf{Priority \& Item} & \textbf{Description} \\
    \midrule
    \textbf{High Priority:} \newline Implement Security Awareness Training & Develop and deploy a mandatory security awareness training program. This program should include:
    \begin{itemize}
        \item \textbf{Onboarding Training:} A required module for all new hires before they are granted access to systems.
        \item \textbf{Annual Refresher Training:} A yearly course for all employees covering current threats, company policies, and best practices.
        \item \textbf{Phishing Simulations:} Regular, simulated phishing campaigns to test and reinforce employee vigilance.
    \end{itemize} \\
    \midrule
    \textbf{Good Practice:} \newline Maintain Strong Network Hardening & The scanned host (\texttt{192.168.1.100}) demonstrates excellent network hardening by exposing no services. This practice should be maintained and adopted as a standard for all servers and endpoints where feasible. Conduct regular internal vulnerability scans to ensure this posture is maintained across the network. \\
    \bottomrule
\end{tabular}
\end{table}

\end{document}
```