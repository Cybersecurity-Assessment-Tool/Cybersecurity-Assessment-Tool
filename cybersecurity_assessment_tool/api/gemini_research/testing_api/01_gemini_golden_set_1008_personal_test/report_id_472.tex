```latex
\documentclass[12pt]{article}

% Preamble: Required Packages
\usepackage[margin=1in]{geometry}
\usepackage{pifont} % For checkmarks and crosses
\usepackage{booktabs} % For professional tables
\usepackage{hyperref} % For clickable links
\usepackage{url} % For URL formatting
\usepackage{seqsplit} % For splitting long strings
\usepackage{graphicx}
\usepackage[utf8]{inputenc}

% Document Metadata
\title{Cybersecurity Assessment Report for Tidal Wave Sports}
\author{Cybersecurity Analyst Group}
\date{\today}

\begin{document}

\maketitle
\thispagestyle{empty}
\newpage

\tableofcontents
\newpage

% --- 1. Executive Summary ---
\section{Executive Summary}
This report details the findings of a cybersecurity assessment conducted for \textbf{Tidal Wave Sports}. The assessment incorporated a review of organizational security controls, a technical network scan of a key internal asset, and an analysis of pre-existing risks.

The primary finding of this assessment is the presence of \textbf{critical gaps in identity and access management controls}. The organization does not enforce Multi-Factor Authentication (MFA) for email, computer logins, or access to sensitive data systems. This represents a significant and immediate risk, as it greatly increases the likelihood of a successful account compromise via phishing or credential theft. A secondary high-risk finding is the lack of security awareness training for new employees, creating a window of vulnerability during their initial tenure.

On a positive note, the technical network scan of the target host \texttt{192.168.1.100} revealed no open ports. This indicates a strong network-level configuration for this specific asset, reducing its external attack surface.

Immediate remediation efforts should focus on implementing a comprehensive MFA policy across all critical systems and integrating security training into the employee onboarding process.

% --- 2. Organizational Information ---
\section{Organizational Information}
The following details were provided for the assessment.

\begin{itemize}
    \item \textbf{Organization Name:} Tidal Wave Sports
    \item \textbf{Email Domain:} \texttt{TidalWaveSports.org}
    \item \textbf{Website Domain:} \url{www.TidalWaveSports.org}
    \item \textbf{External IP Address:} \texttt{20.212.110.158}
\end{itemize}

% --- 3. Security Control Review ---
\section{Security Control Review}
A questionnaire was completed to evaluate the organization's current security policies and procedures. The responses are summarized below. Answers marked with \ding{55} indicate a deviation from security best practices and represent a potential control gap.

\begin{table}[h!]
\centering
\caption{Organizational Security Controls Questionnaire}
\begin{tabular}{@{}lc@{}}
\toprule
\textbf{Control Question} & \textbf{Response} \\
\midrule
Does your organization have an employee acceptable use policy? & \ding{51} \\
Does your organization do security awareness training for all employees at least once per year? & \ding{51} \\
Do you require MFA to access email? & \ding{55} \\
Do you require MFA to log into computers? & \ding{55} \\
Do you require MFA to access sensitive data systems? & \ding{55} \\
Does your organization do security awareness training for new employees? & \ding{55} \\
\bottomrule
\end{tabular}
\end{table}

\subsection*{Analysis of Control Gaps}
The review identified critical weaknesses in access control and employee training:
\begin{itemize}
    \item \textbf{Lack of MFA:} The absence of MFA for email, computer, and sensitive data access is a critical vulnerability. This single point of failure allows an attacker with valid credentials (e.g., from a phishing attack) to gain unauthorized access to key organizational resources.
    \item \textbf{Onboarding Training Gap:} While annual training is in place, the lack of security training for new hires means that the newest, and often most vulnerable, employees are not equipped to identify and resist social engineering or phishing attacks from day one.
\end{itemize}

% --- 4. Technical Scan Results ---
\section{Technical Scan Results}
A network scan was performed to identify open ports and services on a designated target system.

\begin{itemize}
    \item \textbf{Target IP Address:} \texttt{192.168.1.100}
    \item \textbf{Scan Date:} \today
\end{itemize}

\subsection*{Findings}
The scan confirmed that the target host was online and responsive. However, the scan reported that all 1000 scanned TCP ports were in a \textbf{`closed`} state.

\begin{verbatim}
Host is up.
All 1000 scanned ports on 192.168.1.100 are in state: closed
\end{verbatim}

\subsection*{Conclusion}
No open ports or exposed services were discovered on the target host. From a network perspective, this system is well-hardened and presents a minimal attack surface. No network-based vulnerabilities were identified on this host during the assessment.

% --- 5. Consolidated Risk Assessment ---
\section{Consolidated Risk Assessment}
The following table synthesizes risks identified from the security control review. No pre-existing vulnerabilities were provided for this assessment.

\begin{table}[h!]
\centering
\caption{Identified Risks and Severity}
\begin{tabular}{@{}p{0.1\textwidth} p{0.3\textwidth} p{0.4\textwidth} p{0.1\textwidth}@{}}
\toprule
\textbf{Risk ID} & \textbf{Risk Name} & \textbf{Overview} & \textbf{Severity} \\
\midrule
RISK-001 & Lack of Multi-Factor Authentication (MFA) & The absence of a secondary authentication factor for email, endpoints, and sensitive systems exposes the organization to account takeovers, data breaches, and ransomware deployment via compromised credentials. & \textbf{Critical} \\
\addlinespace
RISK-002 & Inadequate Employee Onboarding Security Training & New employees are not provided with security awareness training upon being hired. This makes them highly susceptible to phishing and social engineering attacks before they receive their first annual training. & \textbf{High} \\
\bottomrule
\end{tabular}
\end{table}

% --- 6. Recommendations ---
\section{Recommendations}
The following actions are recommended to mitigate the identified risks and improve the overall security posture of \textbf{Tidal Wave Sports}.

\subsection*{RISK-001: Lack of Multi-Factor Authentication (MFA)}
\begin{itemize}
    \item \textbf{Immediate Action:} Procure and deploy an MFA solution across the organization.
    \item \textbf{Tactical Steps:}
    \begin{enumerate}
        \item Enable and enforce MFA for all user accounts on the primary email platform (e.g., Microsoft 365, Google Workspace).
        \item Implement MFA for all remote access solutions (e.g., VPN).
        \item Deploy MFA for logging into company-managed computers (e.g., via Windows Hello for Business, Duo, etc.).
        \item Enforce MFA for all applications containing sensitive or critical business data.
    \end{enumerate}
    \item \textbf{Priority:} \textbf{Critical}. This should be the top remediation priority.
\end{itemize}

\subsection*{RISK-002: Inadequate Employee Onboarding Security Training}
\begin{itemize}
    \item \textbf{Immediate Action:} Develop a security awareness training module specifically for new hires.
    \item \textbf{Tactical Steps:}
    \begin{enumerate}
        \item Integrate a mandatory 30-60 minute security training session into the formal onboarding process for all new employees, to be completed within their first week.
        \item The training should cover, at a minimum: phishing identification, password security policies, acceptable use of company assets, and how to report a security incident.
        \item Ensure the employee acceptable use policy is reviewed and signed as part of this onboarding training.
    \end{enumerate}
    \item \textbf{Priority:} \textbf{High}.
\end{itemize}

% --- 7. Conclusion ---
\section{Conclusion}
The security posture of \textbf{Tidal Wave Sports} is characterized by a mix of strengths and significant weaknesses. While network-level controls on the assessed asset appear robust, the organization's procedural and identity controls are critically underdeveloped. The lack of Multi-Factor Authentication is the most severe risk and exposes the organization to a high probability of a security breach.

By implementing the recommendations outlined in this report, particularly the rapid deployment of MFA and the enhancement of the employee training program, \textbf{Tidal Wave Sports} can substantially reduce its risk exposure and build a more resilient security foundation.

\end{document}
```