```latex
\documentclass[12pt]{article}

% Preamble: Required Packages
\usepackage[margin=1in]{geometry}
\usepackage{pifont} % For checkmarks and crosses
\usepackage{booktabs} % For professional tables
\usepackage{hyperref} % For clickable links
\usepackage{url} % For URL formatting
\usepackage{seqsplit} % For splitting long strings like IPs
\usepackage{xcolor} % For colors
\usepackage{graphicx} % For potential logos/images
\usepackage{datetime} % For report date

% Document Metadata
\title{Cybersecurity Posture Assessment Report}
\author{Cybersecurity Analysis Division}
\date{\today}

% Hyperref Setup
\hypersetup{
    colorlinks=true,
    linkcolor=blue,
    filecolor=magenta,      
    urlcolor=cyan,
    pdftitle={Cybersecurity Posture Assessment Report},
    pdfpagemode=FullScreen,
}

\begin{document}

\maketitle
\thispagestyle{empty}
\clearpage

\tableofcontents
\clearpage

% --- 1. Executive Summary ---
\section{Executive Summary}

This report provides a comprehensive cybersecurity posture assessment for \textbf{Mainframe Managed}, based on a combination of self-reported organizational controls, external network scanning, and a review of known risks. The analysis was conducted to identify security gaps, assess their potential impact, and provide actionable recommendations for remediation.

Overall, the organization demonstrates a strong commitment to identity and access management, with Multi-Factor Authentication (MFA) consistently enforced across email, workstations, and sensitive systems. This significantly reduces the risk of unauthorized access via compromised credentials.

However, two key areas of risk were identified:

\begin{itemize}
    \item \textbf{High Risk - Security Training Gap:} The organization does not conduct mandatory annual security awareness training for all employees. This represents a critical gap, as the human element is often the weakest link in security. Without regular training, staff may be more susceptible to evolving threats like phishing and social engineering.
    \item \textbf{Medium Risk - Exposed Network Service:} An external network scan identified an open SSH port (22) on the IPv6 address \seqsplit{\texttt{2001:db8::1}}. While necessary for remote administration, public exposure of this service increases the attack surface and exposes the organization to brute-force attacks and potential exploitation.
\end{itemize}

The recommendations in this report focus on closing the identified training gap and implementing compensating controls to secure the exposed network service, thereby enhancing the organization's overall resilience against common cyber threats.

% --- 2. Organizational Information ---
\section{Organizational Information}

The following details were provided for the assessment. This information helps establish the context and scope of the review.

\begin{tabular}{@{}ll}
\toprule
\textbf{Attribute} & \textbf{Value} \\
\midrule
Organization Name & \textbf{Mainframe Managed} \\
Primary Email Domain & \texttt{MainframeManaged.org} \\
Primary Website & \url{www.MainframeManaged.org} \\
Known External IP & \texttt{90.30.240.22} \\
\bottomrule
\end{tabular}

% --- 3. Security Control Review ---
\section{Security Control Review}

This section details the organization's self-reported security controls based on a standardized questionnaire. A green checkmark (\textcolor{green}{\ding{51}}) indicates a positive control is in place, while a red cross (\textcolor{red}{\ding{55}}) highlights a potential security gap.

\begin{table}[h!]
\centering
\caption{Security Controls Questionnaire Results}
\begin{tabular}{@{}p{0.8\linewidth}c@{}}
\toprule
\textbf{Control Question} & \textbf{Response} \\
\midrule
Do you require MFA to access email? & \textcolor{green}{\ding{51}} \\
Do you require MFA to log into computers? & \textcolor{green}{\ding{51}} \\
Do you require MFA to access sensitive data systems? & \textcolor{green}{\ding{51}} \\
Does your organization have an employee acceptable use policy? & \textcolor{green}{\ding{51}} \\
Does your organization do security awareness training for new employees? & \textcolor{green}{\ding{51}} \\
Does your organization do security awareness training for all employees at least once per year? & \textcolor{red}{\ding{55}} \\
\bottomrule
\end{tabular}
\end{table}

The review indicates a significant gap in the security training lifecycle. While new employees are trained, the lack of an annual refresher for all staff is a critical deficiency that needs to be addressed.

% --- 4. Technical Scan Results ---
\section{Technical Scan Results}

An external, non-intrusive network scan was performed against the target IP address provided. The following table summarizes the findings.

\begin{table}[h!]
\centering
\caption{Open Port Scan Results}
\begin{tabular}{@{}lllll@{}}
\toprule
\textbf{Target IP Address} & \textbf{Port} & \textbf{State} & \textbf{Service} & \textbf{Details} \\
\midrule
\seqsplit{\texttt{2001:db8::1}} & 22 & open & ssh & Service exposed externally. No version \\
& & & (inferred) & information was retrieved. \\
\bottomrule
\end{tabular}
\end{table}

The scan confirms that the Secure Shell (SSH) service on port 22 is accessible from the scanning location. Exposing administrative services like SSH directly to the internet is a common risk, as they are frequent targets for automated brute-force attacks and vulnerability scanning by malicious actors.

% --- 5. Risk Assessment ---
\section{Risk Assessment}

This section synthesizes the findings from the security control review and technical scans into a prioritized list of risks. The organization reported no pre-existing vulnerabilities.

\begin{table}[h!]
\centering
\caption{Identified Risk Summary}
\begin{tabular}{@{}p{0.1\linewidth}p{0.25\linewidth}p{0.45\linewidth}l@{}}
\toprule
\textbf{Risk ID} & \textbf{Risk Name} & \textbf{Description} & \textbf{Severity} \\
\midrule
\textbf{ORG-001} & Lack of Annual Security Training & The absence of a mandatory, annual security training program for all employees increases the organizational susceptibility to phishing, social engineering, and other human-targeted attacks. & \textbf{\textcolor{red}{High}} \\
\addlinespace
\textbf{NET-001} & Exposed SSH Service & The SSH service on \seqsplit{\texttt{2001:db8::1}} is publicly accessible. This creates an unnecessary attack surface, exposing the system to brute-force login attempts and potential exploitation of known or unknown vulnerabilities. & \textbf{\textcolor{orange}{Medium}} \\
\bottomrule
\end{tabular}
\end{table}

% --- 6. Recommendations ---
\section{Recommendations}

The following actionable recommendations are provided to mitigate the identified risks and improve the overall security posture of \textbf{Mainframe Managed}.

\subsection{ORG-001: Implement Annual Security Training (High)}

\begin{itemize}
    \item \textbf{Action:} Institute a mandatory annual security awareness training program for all employees, contractors, and relevant stakeholders. The training should cover current threats such as phishing, ransomware, password hygiene, and social engineering tactics.
    \item \textbf{Justification:} The threat landscape evolves continuously. Regular training ensures that the "human firewall" remains effective, reinforcing security best practices and reducing the likelihood of a security incident caused by human error. This is one of the most cost-effective security investments an organization can make.
\end{itemize}

\subsection{NET-001: Secure Exposed SSH Service (Medium)}

\begin{itemize}
    \item \textbf{Action:} Review the business need for external SSH access to \seqsplit{\texttt{2001:db8::1}}.
    \begin{enumerate}
        \item \textbf{If required:} Restrict access to a whitelist of trusted IP addresses using a firewall. Do not allow access from any source (\texttt{0.0.0.0/0} or \texttt{::/0}). Additionally, consider implementing a tool like \texttt{fail2ban} to automatically block IPs that exhibit malicious behavior (e.g., multiple failed login attempts).
        \item \textbf{If not required:} Disable the SSH service or configure the firewall to block all external access to port 22 on this host.
    \end{enumerate}
    \item \textbf{Justification:} Applying the principle of least privilege by restricting access significantly reduces the attack surface. This action mitigates the risk of brute-force attacks and prevents unauthorized actors from attempting to exploit the service.
\end{itemize}

\end{document}
```