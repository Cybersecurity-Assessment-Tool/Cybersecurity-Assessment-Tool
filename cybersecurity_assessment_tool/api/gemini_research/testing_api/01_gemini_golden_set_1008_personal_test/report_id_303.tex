```latex
\documentclass[12pt]{article}

% Required Packages
\usepackage[margin=1in]{geometry}
\usepackage{pifont} % For checkmarks and crosses
\usepackage{booktabs} % For professional tables
\usepackage{hyperref} % For clickable links and better PDF navigation
\usepackage{url} % For formatting URLs
\usepackage{seqsplit} % To split long strings without breaking
\usepackage{xcolor} % For colors

% Hyperref Setup
\hypersetup{
    colorlinks=true,
    linkcolor=blue,
    filecolor=magenta,      
    urlcolor=cyan,
    pdftitle={Cybersecurity Posture Assessment Report},
    pdfpagemode=FullScreen,
}

% --- Document Start ---
\begin{document}

% --- Title Page ---
\title{
    Cybersecurity Posture Assessment Report \\
    \large Prepared for: \textbf{Solaris Energy}
}
\author{Cybersecurity Analysis Division}
\date{\today}
\maketitle

\hrule
\vspace{1em}
\begin{abstract}
\noindent This report provides a comprehensive analysis of the cybersecurity posture of Solaris Energy. The assessment is based on the correlation of network scan data, a review of organizational security controls, and an evaluation of pre-existing risk information. The analysis reveals a critical risk related to a publicly exposed and outdated database service, alongside significant gaps in endpoint security and administrative controls. Immediate remediation is required to mitigate the high probability of a security breach.
\end{abstract}
\vspace{1em}
\hrule

\tableofcontents
\newpage

% --- Section 1: Overview ---
\section{Executive Summary}
This assessment synthesizes technical scan data, a security controls questionnaire, and known risks to provide a holistic view of the organization's security posture.

The key findings indicate a critical external exposure. A MySQL database server is directly accessible from the scanned network segment. Furthermore, this database is running version \texttt{5.7.33}, which is an unsupported, end-of-life (EOL) version that no longer receives security patches. This finding confirms and elevates a pre-existing identified risk, "Database Exposure," from High to Critical.

Additionally, the security control review identified significant gaps. The lack of Multi-Factor Authentication (MFA) on employee computers and the absence of a formal Acceptable Use Policy (AUP) represent high-impact risks that weaken the organization's defense against unauthorized access and insider threats.

This report outlines these risks in detail and provides prioritized, actionable recommendations to strengthen the security posture of Solaris Energy.

% --- Section 2: Organizational Information ---
\section{Organizational Information}
The following details were provided for the assessment.
\begin{itemize}
    \item \textbf{Organization Name:} Solaris Energy
    \item \textbf{Email Domain:} \texttt{SolarisEnergy.net}
    \item \textbf{Website Domain:} \seqsplit{\url{www.SolarisEnergy.net}}
    \item \textbf{External IP Address:} \texttt{76.70.131.145}
\end{itemize}

% --- Section 3: Security Control Review ---
\section{Security Control Review}
An internal review of administrative and technical security controls was conducted based on a standardized questionnaire. The results highlight key areas of strength and weakness. Gaps identified with a \ding{55} represent a significant increase in organizational risk.

\begin{table}[h!]
\centering
\caption{Security Controls Questionnaire Results}
\begin{tabular}{p{0.8\linewidth} c}
\toprule
\textbf{Control Question} & \textbf{Response} \\
\midrule
Do you require MFA to access email? & \textcolor{green}{\ding{51}} \\
Do you require MFA to log into computers? & \textcolor{red}{\ding{55}} \\
Do you require MFA to access sensitive data systems? & \textcolor{green}{\ding{51}} \\
Does your organization have an employee acceptable use policy? & \textcolor{red}{\ding{55}} \\
Does your organization do security awareness training for new employees? & \textcolor{green}{\ding{51}} \\
Does your organization do security awareness training for all employees at least once per year? & \textcolor{green}{\ding{51}} \\
\bottomrule
\end{tabular}
\end{table}

\subsection*{Analysis of Control Gaps}
\begin{itemize}
    \item \textbf{No MFA on Computers:} This is a critical weakness. If an employee's credentials are stolen (e.g., through phishing), an attacker can gain direct access to their computer and the internal network without any further barriers.
    \item \textbf{No Acceptable Use Policy (AUP):} The lack of a formal AUP creates ambiguity regarding safe computing practices and the consequences of misuse. This administrative gap weakens the overall security culture and legal standing in case of an internal incident.
\end{itemize}

% --- Section 4: Technical Scan Results ---
\section{Technical Scan Results}
A network scan was performed to identify active services and potential vulnerabilities on the target host.

\begin{itemize}
    \item \textbf{Target IP Address:} \texttt{172.16.50.20}
\end{itemize}

\subsection*{Open Ports and Services}
The following table details the open ports and services discovered on the target system.

\begin{table}[h!]
\centering
\caption{Nmap Scan Findings for \texttt{172.16.50.20}}
\begin{tabular}{l l l l}
\toprule
\textbf{Port} & \textbf{State} & \textbf{Service} & \textbf{Product \& Version} \\
\midrule
3306/tcp & open & mysql & MySQL 5.7.33 \\
\bottomrule
\end{tabular}
\end{table}

\subsection*{Analysis of Technical Findings}
The scan identified one open port, \textbf{3306/tcp}, which is the default port for the MySQL database service. 
\begin{itemize}
    \item \textbf{Service Exposure:} Exposing a database directly to the network is a significant security risk. It allows attackers to directly target the database for brute-force attacks, credential stuffing, or exploitation of known vulnerabilities. This finding validates the pre-existing risk "Database Exposure."
    \item \textbf{Outdated Software:} The identified version, \textbf{MySQL 5.7.33}, is critically outdated. The MySQL 5.7 series reached its official End of Life (EOL) in October 2023. This means it no longer receives security updates from the vendor, and numerous vulnerabilities discovered since its release in early 2021 remain unpatched.
\end{itemize}

% --- Section 5: Risk Assessment ---
\section{Risk Assessment}
By correlating the security control gaps, technical findings, and pre-existing risk data, we have compiled the following summary of key risks facing the organization.

\begin{table}[h!]
\centering
\caption{Consolidated Risk Summary}
\begin{tabular}{p{0.25\linewidth} p{0.55\linewidth} l}
\toprule
\textbf{Risk Name} & \textbf{Description} & \textbf{Severity} \\
\midrule
\textbf{Exposed \& Outdated Database} & The MySQL database on port 3306 is open to the network and is running an unsupported, end-of-life version with known vulnerabilities. & \textbf{Critical} \\
\addlinespace
\textbf{Lack of Endpoint MFA} & The absence of MFA for computer logins exposes the organization to account takeover and facilitates unauthorized lateral movement within the network. & \textbf{High} \\
\addlinespace
\textbf{Missing Acceptable Use Policy} & The lack of a formal AUP introduces administrative and legal risk and fails to establish a clear baseline for secure employee behavior. & \textbf{High} \\
\bottomrule
\end{tabular}
\end{table}

% --- Section 6: Recommendations ---
\section{Recommendations}
The following prioritized recommendations are provided to address the identified risks and improve the overall security posture of Solaris Energy.

\subsection*{Immediate Actions (Critical Priority)}
\begin{enumerate}
    \item \textbf{Restrict Database Access:} Immediately implement strict firewall rules to block all public access to TCP port 3306 on host \texttt{172.16.50.20}. Access should be restricted to a whitelist of specific, trusted IP addresses only. If remote access is required, it should be facilitated through a secure VPN.
    \item \textbf{Plan Database Upgrade:} Develop an urgent plan to upgrade the MySQL 5.7.33 instance to a currently supported version (e.g., MySQL 8.x). The current version is EOL and presents an unacceptable level of risk due to unpatched vulnerabilities.
\end{enumerate}

\subsection*{High Priority Actions}
\begin{enumerate}
    \item \textbf{Deploy MFA on Endpoints:} Implement a mandatory MFA solution for all employee computer logins. This is a critical defense against credential theft and unauthorized access.
    \item \textbf{Develop and Implement an AUP:} Draft a formal Acceptable Use Policy that clearly defines rules for using company assets, data handling, and internet usage. Ensure all employees read and acknowledge the policy as a condition of employment.
\end{enumerate}

\subsection*{General Recommendations}
\begin{enumerate}
    \item \textbf{Conduct Comprehensive Vulnerability Scanning:} Perform authenticated and unauthenticated vulnerability scans across the entire network infrastructure to identify other outdated systems or misconfigurations.
\end{enumerate}

% --- Document End ---
\end{document}
```