```latex
\documentclass[12pt]{article}

% Preamble: Required Packages
\usepackage[margin=1in]{geometry}
\usepackage{pifont} % For checkmarks and crosses
\usepackage{booktabs} % For professional tables
\usepackage{hyperref} % For clickable links
\usepackage{url} % For formatting URLs
\usepackage{seqsplit} % For splitting long strings in tt font
\usepackage[utf8]{inputenc}

% Hyperref Setup
\hypersetup{
    colorlinks=true,
    linkcolor=blue,
    filecolor=magenta,      
    urlcolor=cyan,
    pdftitle={Cybersecurity Assessment Report},
    pdfauthor={Cybersecurity Analyst},
    pdfsubject={Security Analysis},
    pdfkeywords={Cybersecurity, Nmap, Risk Assessment},
}

% Document Title
\title{Cybersecurity Assessment Report}
\author{Cybersecurity Analyst}
\date{\today}

\begin{document}

\maketitle
\tableofcontents
\newpage

% --- Section 1: Executive Summary ---
\section{Executive Summary}
This report provides a comprehensive cybersecurity assessment for Modern Myth, based on network scan data, organizational security control questionnaires, and a review of pre-existing risks. The analysis reveals several critical and high-risk findings that require immediate attention.

The primary concerns identified are systemic exposures of the Remote Desktop Protocol (RDP) on internal systems, a critical gap in multi-factor authentication (MFA) for sensitive data systems, and a complete absence of foundational security governance controls, including an acceptable use policy and employee security awareness training.

While the organization has implemented MFA for email and computer logins, the identified vulnerabilities create significant pathways for potential attackers. If exploited, these weaknesses could lead to unauthorized access to sensitive data, system compromise, and business disruption. This report outlines these risks in detail and provides a prioritized list of actionable recommendations to mitigate them effectively.

% --- Section 2: Organizational Information ---
\section{Organizational Information}
The following information was provided for the assessment.

\begin{tabular}{@{}ll}
\toprule
\textbf{Attribute} & \textbf{Value} \\
\midrule
Organization Name & \textbf{Modern Myth} \\
Email Domain & \texttt{ModernMyth.net} \\
Website Domain & \url{www.ModernMyth.net} \\
External IP Address & \texttt{191.164.172.91} \\
\bottomrule
\end{tabular}

% --- Section 3: Security Control Review ---
\section{Security Control Review}
A review of the organization's security controls was conducted via a questionnaire. The responses indicate foundational strengths in endpoint and email security but reveal critical gaps in data protection and security governance.

\begin{tabular}{@{}p{0.8\linewidth}c}
\toprule
\textbf{Control Question} & \textbf{Response} \\
\midrule
Do you require MFA to access email? & \ding{51} \\
Do you require MFA to log into computers? & \ding{51} \\
Do you require MFA to access sensitive data systems? & \textcolor{red}{\ding{55}} \\
Does your organization have an employee acceptable use policy? & \textcolor{red}{\ding{55}} \\
Does your organization do security awareness training for new employees? & \textcolor{red}{\ding{55}} \\
Does your organization do security awareness training for all employees at least once per year? & \textcolor{red}{\ding{55}} \\
\bottomrule
\end{tabular}

\subsection*{Analysis of Control Gaps}
The responses marked with a \textcolor{red}{\ding{55}} (No) represent significant security deficiencies:
\begin{itemize}
    \item \textbf{No MFA for Sensitive Data Systems:} This is a critical vulnerability. Without MFA, these systems are susceptible to compromise via stolen or weak credentials, which is a common attack vector.
    \item \textbf{Lack of Acceptable Use Policy:} This governance gap means there are no formal rules for employees regarding the use of company assets, which can lead to unintentional security incidents.
    \item \textbf{No Security Awareness Training:} The absence of a training program leaves the organization highly vulnerable to social engineering attacks, such as phishing, which are a primary method for initial network compromise.
\end{itemize}

% --- Section 4: Technical Scan Results ---
\section{Technical Scan Results}
A network scan was performed to identify open ports and exposed services on the target system.

\subsection*{Target: \texttt{10.10.10.51}}
The scan identified the following open port on the target host:

\begin{tabular}{@{}llll}
\toprule
\textbf{Port} & \textbf{State} & \textbf{Service Name} & \textbf{Description} \\
\midrule
3389/tcp & open & \texttt{ms-wbt-server} & Microsoft Remote Desktop Protocol (RDP) \\
\bottomrule
\end{tabular}

\subsection*{Analysis of Technical Findings}
The scan confirms that port \textbf{3389 (RDP)} is open on the host \texttt{10.10.10.51}. RDP is a common target for attackers who use brute-force techniques or exploit vulnerabilities to gain remote control of systems. This finding, combined with the pre-existing risk on another host (\texttt{10.10.10.50}), suggests a systemic issue with RDP management within the network.

% --- Section 5: Correlated Risk Assessment ---
\section{Correlated Risk Assessment}
This section synthesizes findings from the security questionnaire, technical scan, and pre-existing risk data into a consolidated list of security risks.

\begin{tabular}{@{}p{0.2\linewidth}p{0.45\linewidth}p{0.15\linewidth}p{0.1\linewidth}}
\toprule
\textbf{Risk Name} & \textbf{Description} & \textbf{Affected Systems} & \textbf{Severity} \\
\midrule
\textbf{Systemic RDP Exposure} & Remote Desktop Protocol is exposed on multiple internal servers, increasing the risk of unauthorized remote access and lateral movement. & \texttt{10.10.10.51} \texttt{10.10.10.50} & \textbf{Critical} \\
\addlinespace
\textbf{Missing MFA on Sensitive Systems} & Lack of multi-factor authentication on critical data systems allows for potential compromise through a single factor (password). & Sensitive Data Systems & \textbf{Critical} \\
\addlinespace
\textbf{Lack of Security Governance} & The absence of an acceptable use policy and security awareness training program creates a weak human firewall, making the organization highly susceptible to phishing and insider threats. & All Employees & \textbf{High} \\
\bottomrule
\end{tabular}

% --- Section 6: Recommendations ---
\section{Recommendations}
The following prioritized recommendations are provided to address the identified risks and improve the overall security posture of Modern Myth.

\subsection{Immediate Priority (Critical Risks)}
\begin{enumerate}
    \item \textbf{Remediate RDP Exposure:}
    \begin{itemize}
        \item Immediately assess the business need for RDP on hosts \texttt{10.10.10.51} and \texttt{10.10.10.50}.
        \item If RDP is not required, disable the service and close port 3389 on these systems.
        \item If RDP is required, restrict access to authorized personnel only via a firewall or place the RDP service behind a Virtual Private Network (VPN) with MFA enabled.
    \end{itemize}
    \item \textbf{Implement MFA for Sensitive Systems:}
    \begin{itemize}
        \item Immediately deploy a mandatory MFA solution for all access to systems identified as containing sensitive data. This is the single most effective control to prevent unauthorized access due to compromised credentials.
    \end{itemize}
\end{enumerate}

\subsection{High Priority (High Risks)}
\begin{enumerate}
    \setcounter{enumi}{2} % Continue numbering from previous list
    \item \textbf{Develop and Implement Security Policies:}
    \begin{itemize}
        \item Create and enforce a formal Employee Acceptable Use Policy (AUP). This policy should clearly define the rules for using company networks, devices, and data.
    \end{itemize}
    \item \textbf{Establish Security Awareness Training:}
    \begin{itemize}
        \item Implement a mandatory security awareness training program for all new hires upon onboarding.
        \item Conduct annual security awareness training for all employees to keep them informed about current threats, such as phishing, and reinforce secure behaviors.
    \end{itemize}
\end{enumerate}

\end{document}
```