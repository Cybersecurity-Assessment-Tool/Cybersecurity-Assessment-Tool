```latex
\documentclass[12pt, a4paper]{article}

% Preamble: Required Packages
\usepackage[margin=1in]{geometry}
\usepackage{pifont} % For checkmarks and crosses (\ding{51}, \ding{55})
\usepackage{booktabs} % For professional-looking tables
\usepackage{hyperref} % For clickable links and better PDF navigation
\usepackage{url}      % For formatting URLs
\usepackage{seqsplit} % For splitting long strings without spaces
\usepackage{graphicx}
\usepackage{xcolor}

% Hyperref Setup
\hypersetup{
    colorlinks=true,
    linkcolor=blue,
    filecolor=magenta,      
    urlcolor=cyan,
    pdftitle={Cybersecurity Posture Assessment Report},
    pdfpagemode=FullScreen,
}

% Define custom colors
\definecolor{darkred}{rgb}{0.55, 0.0, 0.0}
\definecolor{darkorange}{rgb}{0.8, 0.33, 0.0}
\definecolor{darkyellow}{rgb}{0.75, 0.55, 0.0}

% Document Start
\begin{document}

% --- Title Page ---
\begin{titlepage}
    \centering
    \vspace*{1cm}
    \Huge\textbf{Cybersecurity Posture Assessment Report}
    \vspace{1.5cm}
    \Large
    \textbf{Prepared for:}\\
    Clear Path
    \vspace{3cm}
    \large
    \textbf{Date of Report:}\\
    \today
    \vfill
    \textit{This report contains sensitive information and should be handled with care.}
\end{titlepage}

\tableofcontents
\newpage

% --- 1. Executive Summary ---
\section*{1. Executive Summary}

This report provides a comprehensive assessment of the cybersecurity posture for Clear Path, based on a combination of technical network scanning, a review of existing risks, and an analysis of organizational security controls.

The assessment identified a \textbf{critical-risk vulnerability} on the internal network. A server at \texttt{10.0.0.15} is running an outdated and notoriously vulnerable FTP service (\texttt{vsftpd 2.3.4}) with anonymous login enabled. This configuration exposes the organization to immediate risk of unauthorized access, data theft, and potential network compromise.

Furthermore, significant gaps were identified in the organization's security controls. The absence of multi-factor authentication (MFA) for computer logins and the lack of security awareness training for new employees represent high-risk deficiencies. These gaps, combined with the pre-existing risk of outdated Windows 7 workstations, significantly increase the likelihood of a successful cyberattack through credential theft or social engineering.

Immediate remediation of the vulnerable FTP server is paramount. Following this, the implementation of mandatory MFA and a comprehensive security training program is strongly recommended to strengthen the organization's overall defense-in-depth strategy.

% --- 2. Organizational Information ---
\section*{2. Organizational Information}

The following information was provided for the assessment.

\begin{table}[h!]
\centering
\begin{tabular}{@{}ll@{}}
\toprule
\textbf{Attribute}       & \textbf{Value} \\ \midrule
Organization Name        & Clear Path \\
Email Domain             & \texttt{ClearPath.org} \\
Website Domain           & \url{www.ClearPath.org} \\
External IP Address      & \texttt{72.180.184.125} \\ \bottomrule
\end{tabular}
\caption{Client Organizational Data}
\end{table}

% --- 3. Security Control Review ---
\section*{3. Security Control Review}

A review of the organization's security controls was conducted via a questionnaire. The responses highlight key areas of strength and weakness in the current security posture. "No" answers indicate significant gaps that require attention.

\begin{table}[h!]
\centering
\begin{tabular}{@{}p{0.7\textwidth}c@{}}
\toprule
\textbf{Control Question} & \textbf{Status} \\ \midrule
Do you require MFA to access email? & \ding{51} \\
Do you require MFA to log into computers? & \textcolor{darkred}{\ding{55}} \\
Do you require MFA to access sensitive data systems? & \ding{51} \\
Does your organization have an employee acceptable use policy? & \ding{51} \\
Does your organization do security awareness training for new employees? & \textcolor{darkred}{\ding{55}} \\
Does your organization do security awareness training for all employees at least once per year? & \ding{51} \\ \bottomrule
\end{tabular}
\caption{Security Controls Questionnaire Results}
\end{table}

\subsection*{Analysis}
The lack of MFA for computer logins is a high-risk issue. If an employee's password is compromised, an attacker could gain direct access to their workstation and, potentially, the internal network. Similarly, the absence of security training for new hires leaves the organization vulnerable, as new employees are often prime targets for phishing and social engineering attacks.

% --- 4. Technical Scan Results ---
\section*{4. Technical Scan Results}

A network scan was performed on the internal target \texttt{10.0.0.15}. The scan revealed an open port with a critically vulnerable service.

\begin{table}[h!]
\centering
\begin{tabular}{@{}lllll@{}}
\toprule
\textbf{Port} & \textbf{State} & \textbf{Service} & \textbf{Version} & \textbf{Finding} \\ \midrule
21/tcp & Open & ftp & vsftpd 2.3.4 & \begin{tabular}[t]{@{}l@{}}Anonymous FTP Login Allowed \\ \textbf{CVE-2011-2523 Vulnerability}\end{tabular} \\ \bottomrule
\end{tabular}
\caption{Open Ports and Services on \texttt{10.0.0.15}}
\end{table}

\subsection*{Analysis of Findings}
\begin{itemize}
    \item \textbf{Vulnerable FTP Service:} The version \texttt{vsftpd 2.3.4} is known to be vulnerable to a critical backdoor command execution flaw (CVE-2011-2523). An attacker can exploit this vulnerability to gain a root shell on the server, effectively taking complete control of the system.
    \item \textbf{Anonymous FTP Login:} The configuration allows any user on the network to log into the FTP server without credentials. This permits unauthorized access to files stored on the server and could be used to exfiltrate data or upload malicious files.
\end{itemize}

% --- 5. Consolidated Risk Assessment ---
\section*{5. Consolidated Risk Assessment}

The following table synthesizes findings from the technical scan, security control review, and pre-existing risk data to provide a prioritized list of security risks.

\begin{table}[h!]
\centering
\begin{tabular}{@{}p{0.3\textwidth}p{0.15\textwidth}p{0.45\textwidth}@{}}
\toprule
\textbf{Risk Name} & \textbf{Severity} & \textbf{Overview} \\ \midrule
\textbf{Vulnerable FTP Server} & \textcolor{darkred}{\textbf{Critical}} & A server is running \texttt{vsftpd 2.3.4} with a known remote code execution backdoor (CVE-2011-2523) and allows anonymous login. \\
\addlinespace
\textbf{Lack of MFA on Workstations} & \textcolor{darkorange}{\textbf{High}} & The absence of MFA for computer logins exposes the organization to unauthorized access via compromised credentials. \\
\addlinespace
\textbf{No Security Training for New Hires} & \textcolor{darkorange}{\textbf{High}} & New employees are not trained on security best practices, making them highly susceptible to phishing and social engineering attacks. \\
\addlinespace
\textbf{Outdated Windows Policy} & \textcolor{darkyellow}{\textbf{Medium}} & Workstations are running Windows 7, an unsupported operating system that no longer receives security updates. (CVSS 5.0) \\ \bottomrule
\end{tabular}
\caption{Prioritized Security Risks}
\end{table}

% --- 6. Recommendations ---
\section*{6. Recommendations}

The following actions are recommended to mitigate the identified risks, prioritized by severity.

\subsection*{Immediate (Critical Risk)}
\begin{enumerate}
    \item \textbf{Remediate Vulnerable FTP Server (\texttt{10.0.0.15}):}
    \begin{itemize}
        \item Immediately take the FTP service offline.
        \item If FTP is required, upgrade \texttt{vsftpd} to the latest stable version.
        \item Disable anonymous login and enforce strong, unique passwords for all FTP accounts.
        \item If FTP is not a business requirement, disable and uninstall the service entirely.
        \item Segment the server from the general corporate network using firewall rules.
    \end{itemize}
\end{enumerate}

\subsection*{High Priority}
\begin{enumerate}
    \setcounter{enumi}{1}
    \item \textbf{Implement MFA for Workstation Logins:}
    \begin{itemize}
        \item Deploy a multi-factor authentication solution (e.g., authenticator app, hardware token, biometrics) for all employee computer logins.
        \item Prioritize deployment for privileged users (administrators) and remote workers.
    \end{itemize}
    \item \textbf{Establish New Hire Security Training:}
    \begin{itemize}
        \item Develop a mandatory security awareness training module as part of the new employee onboarding process.
        \item Training should cover phishing identification, password hygiene, acceptable use, and incident reporting procedures.
    \end{itemize}
\end{enumerate}

\subsection*{Medium Priority}
\begin{enumerate}
    \setcounter{enumi}{3}
    \item \textbf{Accelerate Operating System Upgrades:}
    \begin{itemize}
        \item Continue with the existing plan to upgrade all Windows 7 workstations to a modern, supported operating system like Windows 10 or 11.
    \end{itemize}
\end{enumerate}

\end{document}
```