```latex
\documentclass[12pt, a4paper]{article}

% Preamble: Required Packages
\usepackage[margin=1in]{geometry}
\usepackage{pifont} % For checkmarks and crosses
\usepackage{booktabs} % For professional tables
\usepackage{hyperref} % For clickable links
\usepackage{url} % For URL formatting
\usepackage{seqsplit} % To split long strings in tt font
\usepackage{graphicx}
\usepackage{xcolor}
\usepackage{fancyhdr}
\usepackage{lastpage}

% --- Document Setup ---
\definecolor{darkblue}{rgb}{0.0, 0.0, 0.55}
\hypersetup{
    colorlinks=true,
    linkcolor=darkblue,
    filecolor=darkblue,      
    urlcolor=darkblue,
    citecolor=darkblue,
}

% --- Header and Footer ---
\pagestyle{fancy}
\fancyhf{} % Clear all header and footer fields
\fancyhead[L]{Cybersecurity Assessment Report}
\fancyhead[R]{\textbf{Quantum Reach}}
\fancyfoot[C]{\thepage\ of \pageref{LastPage}}
\renewcommand{\headrulewidth}{0.4pt}
\renewcommand{\footrulewidth}{0.4pt}

% --- Document Start ---
\begin{document}

% --- Title Page ---
\begin{titlepage}
    \centering
    \vspace*{2cm}
    
    {\Huge \textbf{Cybersecurity Posture Assessment Report}\par}
    \vspace{1.5cm}
    
    {\Large Prepared for:\par}
    \vspace{0.5cm}
    {\Huge \textbf{Quantum Reach}\par}
    
    \vfill
    
    {\large \today\par}
    
    \vspace{1cm}
    \rule{\linewidth}{0.5pt}
    \vspace{0.2cm}
    \textit{This report contains sensitive information and is intended solely for the use of the recipient organization. Distribution is strictly prohibited.}
    
\end{titlepage}

\newpage
\tableofcontents
\newpage

% --- Section 1: Executive Summary ---
\section{Executive Summary}
This report provides a comprehensive analysis of the cybersecurity posture for \textbf{Quantum Reach}, based on a review of organizational security controls, an external network scan, and pre-existing risk data.

The assessment reveals a mixed security posture. The organization demonstrates a strong commitment to identity and access management, with mandatory Multi-Factor Authentication (MFA) for email, computer logins, and sensitive systems. This significantly reduces the risk of unauthorized access through compromised credentials.

However, critical gaps were identified in foundational security policies and procedures. The absence of an employee Acceptable Use Policy (AUP) and the lack of security awareness training for new hires represent high-risk vulnerabilities. These gaps expose the organization to significant insider threats, both malicious and accidental, and increase susceptibility to social engineering attacks targeting new employees.

The external network scan of the target IP address \texttt{[Target IP]} did not identify any open ports. While this can indicate a strong firewall configuration, it is essential to ensure the scan was not blocked entirely and that the target was responsive during the assessment period.

Immediate action is recommended to address the identified policy and training deficiencies to establish a robust security baseline and mitigate the associated risks.

% --- Section 2: Organizational Information ---
\section{Organizational Information}
The following table summarizes the key organizational details provided for this assessment.

\begin{table}[h!]
\centering
\caption{Client Organizational Data}
\label{tab:org_data}
\begin{tabular}{@{}ll@{}}
\toprule
\textbf{Attribute} & \textbf{Value} \\ \midrule
Organization Name    & Quantum Reach \\
Email Domain         & \seqsplit{\texttt{QuantumReach.net}} \\
Website Domain       & \seqsplit{\url{www.QuantumReach.net}} \\
External IP Address  & \texttt{179.8.42.253} \\ \bottomrule
\end{tabular}
\end{table}

% --- Section 3: Security Control Review ---
\section{Security Control Review}
A review of the organization's security controls was conducted via a questionnaire. The responses are detailed below, with a checkmark (\ding{51}) indicating a positive control ("Yes") and a cross mark (\ding{55}) indicating a control gap ("No").

\begin{table}[h!]
\centering
\caption{Security Controls Questionnaire Results}
\label{tab:controls}
\begin{tabular}{@{}lc@{}}
\toprule
\textbf{Control Question} & \textbf{Status} \\ \midrule
Do you require MFA to access email? & \textcolor{green}{\ding{51}} \\
Do you require MFA to log into computers? & \textcolor{green}{\ding{51}} \\
Do you require MFA to access sensitive data systems? & \textcolor{green}{\ding{51}} \\
Does your organization have an employee acceptable use policy? & \textcolor{red}{\ding{55}} \\
Does your organization do security awareness training for new employees? & \textcolor{red}{\ding{55}} \\
Does your organization do security training for all employees annually? & \textcolor{green}{\ding{51}} \\ \bottomrule
\end{tabular}
\end{table}

\subsection*{Analysis}
The organization has successfully implemented MFA across critical access points, which is a commendable security practice. However, two significant control gaps were identified:
\begin{itemize}
    \item \textbf{No Acceptable Use Policy (AUP):} The lack of a formal AUP means there are no clear, documented rules for employees regarding the use of company systems and data. This can lead to unintentional misuse, data leakage, and legal ambiguity.
    \item \textbf{No New Employee Security Training:} Failing to train new hires on security best practices from day one leaves a critical window of vulnerability. New employees are often prime targets for phishing and other social engineering attacks.
\end{itemize}

% --- Section 4: Technical Scan Results ---
\section{Technical Scan Results}
An external network vulnerability scan was conducted to identify open ports and services exposed to the internet.

\begin{itemize}
    \item \textbf{Target IP Address:} \texttt{[Target IP]}
    \item \textbf{Scan Date:} Not provided in scan data.
\end{itemize}

\subsection*{Findings}
The scan results indicated that \textbf{no open ports were detected} on the specified target IP address. This is a positive security finding, suggesting that a well-configured firewall or security group is in place, effectively blocking unsolicited inbound traffic and reducing the external attack surface. It is recommended to verify that the target system was online and accessible from the scanning source during the assessment to confirm these results.

% --- Section 5: Risk Assessment ---
\section{Risk Assessment}
This section synthesizes findings from the security control review, technical scan, and any pre-existing risk data. Based on the available information, the following risks have been identified.

\begin{table}[h!]
\centering
\caption{Identified Risks}
\label{tab:risks}
\begin{tabular}{@{}p{0.1\linewidth} p{0.25\linewidth} p{0.45\linewidth} p{0.1\linewidth}@{}}
\toprule
\textbf{Risk ID} & \textbf{Risk Name} & \textbf{Overview} & \textbf{Severity} \\ \midrule
ORG-001 & Absence of Acceptable Use Policy (AUP) & Without a formal AUP, employees lack clear guidance on the proper use of IT assets. This increases the risk of data breaches, non-compliance, and insider threats due to unintentional misuse of systems. & \textbf{High} \\
\addlinespace
ORG-002 & Lack of Onboarding Security Training & New employees are not receiving security awareness training upon being hired. This makes them highly susceptible to social engineering attacks like phishing, potentially leading to credential compromise and initial network access for attackers. & \textbf{High} \\
\bottomrule
\end{tabular}
\end{table}
\textit{Note: No pre-existing vulnerabilities or technical findings were reported in the input data.}

% --- Section 6: Recommendations ---
\section{Recommendations}
The following actionable recommendations are provided to address the identified risks and improve the overall security posture of \textbf{Quantum Reach}.

\begin{table}[h!]
\centering
\caption{Remediation Recommendations}
\label{tab:recommendations}
\begin{tabular}{@{}p{0.15\linewidth} p{0.65\linewidth} p{0.1\linewidth}@{}}
\toprule
\textbf{Risk ID} & \textbf{Recommendation} & \textbf{Priority} \\ \midrule
ORG-001 & Develop and implement a comprehensive Acceptable Use Policy (AUP). This policy should be formally documented, communicated to all employees, and acknowledged via signature. It should clearly define rules for email, internet, software, and data handling. & \textbf{High} \\
\addlinespace
ORG-002 & Establish a mandatory security awareness training program for all new employees as part of the standard onboarding process. This training should cover key topics such as phishing, password security, data handling, and the new AUP. & \textbf{High} \\
\bottomrule
\end{tabular}
\end{table}

% --- Section 7: Conclusion ---
\section{Conclusion}
\textbf{Quantum Reach} has established a strong foundation in access control through the robust implementation of MFA. To build upon this strength, the organization must prioritize the development and enforcement of foundational security policies and training programs.

By implementing the high-priority recommendations to create an Acceptable Use Policy and introduce security training for new hires, \textbf{Quantum Reach} can significantly reduce its risk exposure from human factors and cultivate a more security-conscious culture. Continuous review and improvement of these controls will be essential for maintaining a resilient security posture against evolving threats.

\end{document}
```