```latex
\documentclass[12pt]{article}

% Required Packages
\usepackage[margin=1in]{geometry}
\usepackage{pifont} % For checkmark and X symbols
\usepackage{booktabs} % For professional tables
\usepackage{hyperref} % For clickable links
\usepackage{url} % For formatting URLs
\usepackage{seqsplit} % To split long text sequences like IPs
\usepackage{graphicx} % For logo, if needed
\usepackage{xcolor} % For colors in text

% Document Metadata
\title{Cybersecurity Posture Assessment Report}
\author{Cybersecurity Analysis Division}
\date{November 22, 2025}

% Hyperref Setup
\hypersetup{
    colorlinks=true,
    linkcolor=blue,
    filecolor=magenta,      
    urlcolor=cyan,
    pdftitle={Cybersecurity Posture Assessment Report},
    pdfpagemode=FullScreen,
}

\begin{document}

\maketitle
\thispagestyle{empty}
\newpage

\tableofcontents
\newpage

% --- 1. Executive Summary ---
\section{Executive Summary}

This report details the findings of a cybersecurity posture assessment conducted for \textbf{Clear Path} on November 22, 2025. The assessment combined a review of organizational security controls, an external network scan, and an analysis of known risks.

The analysis revealed several critical and high-risk gaps that require immediate attention. The most significant concerns are the widespread lack of Multi-Factor Authentication (MFA) for email and computer access, and the absence of a formal security awareness training program for employees. These deficiencies create a high probability of successful phishing attacks and subsequent account compromise.

Furthermore, the external network scan identified an internet-facing web server running an outdated version of nginx (\texttt{1.18.0}). This software version is several years old and is susceptible to multiple publicly known vulnerabilities, presenting a direct and exploitable threat to the organization's perimeter.

Immediate remediation efforts should focus on implementing MFA across all critical systems, establishing a comprehensive security awareness training program, and upgrading the vulnerable web server software. Addressing these key areas will substantially improve the organization's resilience against common cyber threats.

% --- 2. Organizational Information ---
\section{Organizational Information}

The following information was provided for the assessment.

\begin{tabular}{@{}ll}
\toprule
\textbf{Attribute} & \textbf{Value} \\
\midrule
Organization Name & \textbf{Clear Path} \\
Email Domain & \texttt{ClearPath.org} \\
Website Domain & \url{www.ClearPath.org} \\
External IP Address & \texttt{72.118.146.158} \\
\bottomrule
\end{tabular}

% --- 3. Security Control Review ---
\section{Security Control Review}

A review of administrative and organizational security controls was conducted based on a standardized questionnaire. The responses indicate significant gaps in identity and access management and employee security training. A summary of the findings is presented in Table \ref{tab:controls}.

\begin{table}[h!]
\centering
\caption{Organizational Security Control Status}
\label{tab:controls}
\begin{tabular}{@{}lc}
\toprule
\textbf{Control Question} & \textbf{Response} \\
\midrule
Do you require MFA to access email? & \textcolor{red}{\ding{55}} \\
Do you require MFA to log into computers? & \textcolor{red}{\ding{55}} \\
Do you require MFA to access sensitive data systems? & \textcolor{green}{\ding{51}} \\
Does your organization have an employee acceptable use policy? & \textcolor{green}{\ding{51}} \\
Does your organization do security awareness training for new employees? & \textcolor{red}{\ding{55}} \\
Does your organization do security awareness training for all employees annually? & \textcolor{red}{\ding{55}} \\
\bottomrule
\end{tabular}
\end{table}

The lack of MFA on email and endpoints, combined with the absence of a security training program, represents a critical vulnerability to social engineering and credential theft attacks.

% --- 4. Technical Scan Results ---
\section{Technical Scan Results}

An external network scan was performed on \texttt{192.168.10.5} to identify open ports and exposed services.

\begin{itemize}
    \item \textbf{Scan Date:} 2025-11-22T10:00:00Z
    \item \textbf{Target IP:} \texttt{192.168.10.5}
\end{itemize}

The scan identified one open port, as detailed in Table \ref{tab:scanresults}.

\begin{table}[h!]
\centering
\caption{Open Port Scan Results}
\label{tab:scanresults}
\begin{tabular}{@{}lllll}
\toprule
\textbf{Port} & \textbf{State} & \textbf{Service} & \textbf{Product} & \textbf{Version} \\
\midrule
443/tcp & open & https & nginx & 1.18.0 \\
\bottomrule
\end{tabular}
\end{table}

\subsection{Analysis of Technical Findings}
The scan confirmed that an \textbf{nginx} web server is exposed to the internet on port 443 (HTTPS). The identified version, \texttt{1.18.0}, was released in April 2020. This version is significantly outdated and lacks security patches for numerous vulnerabilities discovered since its release. Running outdated software on perimeter systems is a high-risk practice, as it provides a clear and often easily exploitable entry point for attackers.

% --- 5. Risk Assessment Summary ---
\section{Risk Assessment Summary}

The following table synthesizes the findings from the security control review and technical scan into a prioritized list of identified risks. No pre-existing vulnerabilities were reported.

\begin{table}[h!]
\centering
\caption{Identified Risks}
\label{tab:risks}
\begin{tabular}{@{}p{0.1\linewidth} p{0.25\linewidth} p{0.4\linewidth} p{0.15\linewidth}@{}}
\toprule
\textbf{Risk ID} & \textbf{Risk Name} & \textbf{Description} & \textbf{Severity} \\
\midrule
\textbf{RISK-001} & Lack of Multi-Factor Authentication (MFA) & MFA is not enforced for accessing corporate email or for logging into employee computers. This severely increases the risk of account compromise via stolen credentials. & \textbf{\textcolor{red}{Critical}} \\
\addlinespace
\textbf{RISK-002} & Inadequate Security Awareness Training & The organization does not provide security awareness training to new or existing employees. This leaves the workforce vulnerable to phishing, social engineering, and other common attacks. & \textbf{\textcolor{orange}{High}} \\
\addlinespace
\textbf{RISK-003} & Outdated Web Server Software & The public-facing web server runs nginx 1.18.0, an outdated version with multiple known vulnerabilities. This exposes the organization to potential remote code execution or denial-of-service attacks. & \textbf{\textcolor{orange}{High}} \\
\bottomrule
\end{tabular}
\end{table}

% --- 6. Recommendations ---
\section{Recommendations}

To mitigate the identified risks and improve the overall security posture, the following actions are recommended with urgency.

\subsection{Remediation for RISK-001: Lack of MFA}
\begin{itemize}
    \item \textbf{Immediate Action:} Enforce MFA for all user accounts, starting with email access. This is the single most effective control to prevent account takeovers.
    \item \textbf{Short-Term Action:} Expand the MFA requirement to include all remote access solutions (e.g., VPN) and endpoint logins (computer access).
    \item \textbf{Policy:} Update the access control policy to mandate MFA for all systems handling sensitive or critical information.
\end{itemize}

\subsection{Remediation for RISK-002: Inadequate Training}
\begin{itemize}
    \item \textbf{Immediate Action:} Procure and deploy a security awareness training platform. Enroll all current employees in a foundational course covering phishing, password security, and acceptable use.
    \item \textbf{Ongoing Action:} Integrate security awareness training into the new employee onboarding process.
    \item \textbf{Policy:} Establish a formal policy requiring all employees to complete security awareness training at least once per year.
\end{itemize}

\subsection{Remediation for RISK-003: Outdated Software}
\begin{itemize}
    \item \textbf{Immediate Action:} Develop a patch management plan to upgrade the nginx server from version \texttt{1.18.0} to the latest stable version. The plan should include pre-production testing to ensure compatibility.
    \item \textbf{Ongoing Action:} Implement a formal vulnerability and patch management program. Regularly scan external and internal assets for outdated software and apply security patches in a timely manner.
\end{itemize}

\end{document}
```