```latex
\documentclass[12pt]{article}

% --- PACKAGES ---
\usepackage[margin=1in]{geometry}
\usepackage{pifont} % For checkmarks and crosses
\usepackage{booktabs} % For professional tables
\usepackage{hyperref} % For clickable links
\usepackage{url} % For URL formatting
\usepackage{seqsplit} % For splitting long strings in texttt
\usepackage[T1]{fontenc}

% --- DOCUMENT METADATA ---
\hypersetup{
    colorlinks=true,
    linkcolor=black,
    urlcolor=blue,
    pdftitle={Cybersecurity Posture Assessment Report},
    pdfauthor={Cybersecurity Analyst},
    pdfsubject={Security Analysis},
    pdfkeywords={Cybersecurity, Nmap, Risk Assessment}
}

% --- DOCUMENT START ---
\begin{document}

% --- TITLE PAGE ---
\begin{titlepage}
    \centering
    \vspace*{\stretch{1.0}}
    \Huge{\textbf{Cybersecurity Posture Assessment Report}}
    \vspace{0.5cm}
    \LARGE{\textbf{for}}
    \vspace{0.5cm}
    \LARGE{Modern Myth}
    \vspace*{\stretch{2.0}}
    \large{
        \textbf{Date of Report:} \today \\
        \textbf{Scan Date:} 2025-11-22 \\
        \textbf{Author:} Cybersecurity Analyst
    }
    \vspace*{\stretch{1.0}}
\end{titlepage}

% --- TABLE OF CONTENTS ---
\tableofcontents
\newpage

% --- EXECUTIVE SUMMARY ---
\section{Executive Summary}
This report provides a comprehensive cybersecurity posture assessment for Modern Myth, based on a combination of network scanning, organizational data review, and an analysis of current risks. The assessment was conducted on November 22, 2025.

The analysis revealed several high-priority security gaps that require immediate attention. A critical vulnerability was identified on the public-facing web server, which is running a significantly outdated and unsupported version of nginx. This exposes the organization to numerous publicly known exploits.

Furthermore, critical procedural gaps were identified, most notably the lack of Multi-Factor Authentication (MFA) for email access, which is a primary vector for account compromise and business email compromise (BEC) attacks. Another significant finding is the absence of security awareness training for new employees, leaving the organization vulnerable to social engineering attacks during the critical onboarding period.

This report details these findings and provides actionable recommendations to mitigate the identified risks and strengthen the overall security posture of the organization.

% --- ORGANIZATIONAL INFORMATION ---
\section{Organizational Information}
The following information was provided for the assessment.
\begin{itemize}
    \item \textbf{Organization Name:} Modern Myth
    \item \textbf{Email Domain:} \texttt{ModernMyth.net}
    \item \textbf{Website Domain:} \url{www.ModernMyth.net}
    \item \textbf{External IP Address:} \texttt{152.124.16.9}
\end{itemize}

% --- SECURITY CONTROL REVIEW ---
\section{Security Control Review}
A review of the organization's security controls was conducted via a questionnaire. The responses indicate key areas where security practices are strong and where they are critically lacking. A "No" response indicates a potential security gap that increases organizational risk.

\begin{table}[h!]
\centering
\caption{Security Controls Questionnaire Results}
\begin{tabular}{p{0.75\textwidth} c}
\toprule
\textbf{Control Question} & \textbf{Response} \\
\midrule
Do you require MFA to access email? & \ding{55} \\
Do you require MFA to log into computers? & \ding{51} \\
Do you require MFA to access sensitive data systems? & \ding{51} \\
Does your organization have an employee acceptable use policy? & \ding{51} \\
Does your organization do security awareness training for new employees? & \ding{55} \\
Does your organization do security awareness training for all employees at least once per year? & \ding{51} \\
\bottomrule
\end{tabular}
\\
\vspace{0.2cm}
\small{\textit{Key: \ding{51} = Yes, \ding{55} = No}}
\end{table}

\subsection*{Analysis of Control Gaps}
\begin{itemize}
    \item \textbf{No MFA for Email:} This is a \textbf{Critical} gap. Email is a primary target for attackers seeking to gain initial access, conduct phishing campaigns, or execute Business Email Compromise (BEC) attacks. The absence of MFA significantly lowers the barrier for an attacker to compromise an account.
    \item \textbf{No Security Training for New Employees:} This is a \textbf{High} risk. New hires are often targeted by social engineering attacks as they are less familiar with company policies and security procedures. Failing to provide training during onboarding creates a recurring window of vulnerability.
\end{itemize}

% --- TECHNICAL SCAN RESULTS ---
\section{Technical Scan Results}
An external network scan was performed to identify open ports and exposed services on the target system.

\begin{itemize}
    \item \textbf{Target IP Address:} \texttt{192.168.10.5}
    \item \textbf{Scan Date:} 2025-11-22T10:00:00Z
\end{itemize}

\begin{table}[h!]
\centering
\caption{Open Ports and Services Detected}
\begin{tabular}{l l l l l}
\toprule
\textbf{Port} & \textbf{State} & \textbf{Service} & \textbf{Product} & \textbf{Version} \\
\midrule
443/tcp & open & https & nginx & 1.18.0 \\
\bottomrule
\end{tabular}
\end{table}

\subsection*{Analysis of Technical Findings}
The scan identified a web server running \textbf{nginx version 1.18.0}. This version was released in April 2020 and is now considered outdated and unsupported. It is known to be vulnerable to multiple Common Vulnerabilities and Exposures (CVEs), including but not limited to request smuggling and memory disclosure issues (e.g., CVE-2021-23017). Running outdated software on a public-facing service presents a \textbf{Critical} risk of compromise, as automated attack tools constantly scan for and exploit such vulnerabilities.

% --- RISK ASSESSMENT SUMMARY ---
\section{Risk Assessment Summary}
This section synthesizes the findings from the security control review and the technical scan into a prioritized list of risks. No pre-existing vulnerabilities were reported.

\begin{table}[h!]
\centering
\caption{Consolidated Risk Register}
\begin{tabular}{p{0.1\textwidth} p{0.3\textwidth} p{0.15\textwidth} p{0.35\textwidth}}
\toprule
\textbf{Risk ID} & \textbf{Risk Name} & \textbf{Severity} & \textbf{Description} \\
\midrule
RISK-001 & Lack of MFA on Email & \textbf{Critical} & The absence of MFA on email accounts exposes the organization to a high likelihood of account takeover and BEC attacks. \\
\addlinespace
RISK-002 & Outdated Web Server Software & \textbf{Critical} & The public-facing nginx server is running a vulnerable version (1.18.0), making it a prime target for automated exploitation. \\
\addlinespace
RISK-003 & Inadequate New Employee Onboarding & \textbf{High} & New employees are not receiving security awareness training, making them highly susceptible to phishing and social engineering attacks. \\
\bottomrule
\end{tabular}
\end{table}

% --- RECOMMENDATIONS ---
\section{Recommendations}
Based on the risk assessment, the following actions are recommended to mitigate the identified vulnerabilities and improve the overall security posture. Recommendations are prioritized by severity.

\begin{enumerate}
    \item \textbf{[Critical] Enforce MFA for All Email Accounts:}
    \begin{itemize}
        \item \textbf{Action:} Immediately enable and enforce MFA for all user accounts across the \texttt{ModernMyth.net} email domain.
        \item \textbf{Impact:} Drastically reduces the risk of email account compromise, even if user credentials are stolen.
    \end{itemize}
    \vspace{0.5cm}
    \item \textbf{[Critical] Upgrade Web Server Software:}
    \begin{itemize}
        \item \textbf{Action:} Plan and execute an upgrade of the nginx server from version 1.18.0 to the latest stable version. A patch management process should be implemented to ensure software is kept up-to-date.
        \item \textbf{Impact:} Mitigates numerous publicly known vulnerabilities, protecting the server and its data from compromise.
    \end{itemize}
    \vspace{0.5cm}
    \item \textbf{[High] Implement New Employee Security Training:}
    \begin{itemize}
        \item \textbf{Action:} Integrate a mandatory security awareness training module into the new employee onboarding process. This training should cover topics such as phishing, acceptable use, and password security.
        \item \textbf{Impact:} Reduces the human-factor risk by equipping new staff with the knowledge to identify and report security threats from their first day.
    \end{itemize}
\end{enumerate}

% --- CONCLUSION ---
\section{Conclusion}
The assessment for Modern Myth has identified critical and high-severity risks that require immediate remediation. The combination of a vulnerable public-facing server and weak access controls for email creates a significant attack surface. By implementing the recommendations outlined in this report—specifically enforcing MFA, upgrading outdated software, and formalizing security training—Modern Myth can substantially reduce its risk exposure and build a more resilient security foundation.

\end{document}
```