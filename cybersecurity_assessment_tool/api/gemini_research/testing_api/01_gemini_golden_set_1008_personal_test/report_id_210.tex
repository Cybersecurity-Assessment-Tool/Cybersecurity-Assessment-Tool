```latex
\documentclass[12pt]{article}

% --- PACKAGE IMPORTS ---
\usepackage[margin=1in]{geometry}
\usepackage{pifont} % For checkmarks and crosses
\usepackage{booktabs} % For professional tables
\usepackage{xcolor} % For colors
\usepackage{hyperref} % For hyperlinks
\usepackage{url} % For URL formatting
\usepackage{seqsplit} % To split long strings in tt font

% --- DOCUMENT SETUP ---
\hypersetup{
    colorlinks=true,
    linkcolor=blue,
    filecolor=magenta,      
    urlcolor=cyan,
    pdftitle={Cybersecurity Posture Report},
    pdfauthor={Cybersecurity Analysis Division},
    pdfsubject={Security Assessment},
    pdfkeywords={Cybersecurity, Risk, Assessment},
}

% --- CUSTOM COMMANDS ---
\newcommand{\yes}{\ding{51}} % Green checkmark
\newcommand{\no}{\ding{55}}  % Red cross

% --- TITLE ---
\title{Cybersecurity Posture Report \\ \large For Modern Myth}
\author{Cybersecurity Analysis Division}
\date{\today}

% --- DOCUMENT START ---
\begin{document}

\maketitle
\tableofcontents
\newpage

% ===================================================================
% SECTION 1: EXECUTIVE SUMMARY
% ===================================================================
\section{Executive Summary}

This report provides a comprehensive analysis of the cybersecurity posture for \textbf{Modern Myth}, based on a synthesis of organizational data, technical network scans, and a review of pre-existing risk documentation. The assessment was conducted to identify security gaps, validate existing controls, and provide actionable recommendations to enhance the organization's overall security resilience.

\paragraph{Key Findings:} The analysis reveals a mixed security posture. The organization has implemented several positive security controls, including mandatory Multi-Factor Authentication (MFA) for email and sensitive systems, as well as a robust security awareness training program. Furthermore, a technical scan of the host at \texttt{192.168.0.5} indicates that a previously identified risk—an open and unencrypted web server on Port 80—appears to have been remediated on that specific system, as the port was found to be closed.

However, two significant areas of concern were identified through the security controls review:
\begin{itemize}
    \item \textbf{Critical Risk:} Lack of mandatory MFA for logging into employee computers. This exposes the organization to significant risk from credential compromise and unauthorized endpoint access.
    \item \textbf{High Risk:} Absence of a formal employee Acceptable Use Policy (AUP). This creates ambiguity regarding the secure and appropriate use of company assets and data.
\end{itemize}

This report details these findings and concludes with prioritized, actionable recommendations designed to mitigate the identified risks and strengthen the organization's defensive capabilities.

% ===================================================================
% SECTION 2: ORGANIZATIONAL INFORMATION
% ===================================================================
\section{Organizational Information}

The following details were provided for the assessment. This information is used to establish the context and scope of the review.

\begin{description}
    \item[Organization Name:] \textbf{Modern Myth}
    \item[Email Domain:] \texttt{ModernMyth.net}
    \item[Website Domain:] \url{www.ModernMyth.net}
    \item[External IP Address:] \texttt{210.236.186.193}
\end{description}

% ===================================================================
% SECTION 3: SECURITY CONTROL REVIEW
% ===================================================================
\section{Security Control Review}

A review of organizational security controls was conducted based on a standardized questionnaire. The responses are summarized below and indicate the current state of implemented policies and procedures. Gaps identified here represent significant organizational risks.

\begin{table}[h!]
\centering
\caption{Organizational Security Control Status}
\begin{tabular}{p{0.8\linewidth} c}
\toprule
\textbf{Control Question} & \textbf{Response} \\
\midrule
Do you require MFA to access email? & \yes \\
Do you require MFA to log into computers? & \no \\
Do you require MFA to access sensitive data systems? & \yes \\
Does your organization have an employee acceptable use policy? & \no \\
Does your organization do security awareness training for new employees? & \yes \\
Does your organization do security awareness training for all employees at least once per year? & \yes \\
\bottomrule
\end{tabular}
\end{table}

\paragraph{Analysis of Gaps:} The responses marked with a \no\ highlight critical deficiencies in the organization's security framework. The lack of MFA on computer logins is a primary concern, as it removes a vital layer of defense against stolen credentials. Additionally, the absence of an Acceptable Use Policy can lead to inconsistent security practices and a lack of legal and administrative recourse for technology misuse.

% ===================================================================
% SECTION 4: TECHNICAL SCAN RESULTS
% ===================================================================
\section{Technical Scan Results}

A network scan was performed on the specified target to identify open ports and exposed services.

\begin{description}
    \item[Target IP Address:] \texttt{192.168.0.5}
    \item[Scan Type:] Nmap Port Scan
\end{description}

The scan results for the target host are detailed in the table below.

\begin{table}[h!]
\centering
\caption{Port Scan Results for \texttt{192.168.0.5}}
\begin{tabular}{l l l l}
\toprule
\textbf{Port} & \textbf{State} & \textbf{Service} & \textbf{Version} \\
\midrule
80/tcp & closed & http & N/A \\
\bottomrule
\end{tabular}
\end{table}

\paragraph{Analysis:} The scan revealed that Port 80 (HTTP) is closed on the target system. This is a positive security finding, as it prevents unencrypted web traffic. This result directly contradicts a pre-existing risk entry which stated this port was open. This suggests that the previously identified vulnerability has been successfully remediated on this host, or the risk documentation was inaccurate.

% ===================================================================
% SECTION 5: CONSOLIDATED RISK ASSESSMENT
% ===================================================================
\section{Consolidated Risk Assessment}

This section synthesizes findings from the security control review, technical scans, and pre-existing risk data into a consolidated list.

\begin{table}[h!]
\centering
\caption{Summary of Identified Risks}
\begin{tabular}{p{0.25\linewidth} p{0.45\linewidth} l l}
\toprule
\textbf{Risk Name} & \textbf{Description} & \textbf{Severity} & \textbf{Status} \\
\midrule
\textbf{Lack of MFA on Endpoints} & Employees are not required to use MFA to log into their computers, increasing risk from compromised credentials. & \textcolor{red}{Critical} & Active \\
\addlinespace
\textbf{Missing Acceptable Use Policy} & The organization lacks a formal policy governing the use of IT assets, leading to potential misuse and insider threats. & \textcolor{orange}{High} & Active \\
\addlinespace
\textbf{Unencrypted Web Server} & Pre-existing risk stated Port 80 was open. The technical scan confirmed this port is closed on the scanned host. & Medium & \textbf{Remediated} \\
\bottomrule
\end{tabular}
\end{table}

% ===================================================================
% SECTION 6: RECOMMENDATIONS
% ===================================================================
\section{Recommendations}

The following actions are recommended to address the active risks identified during this assessment. Recommendations are prioritized based on severity and potential impact.

\subsection{Implement MFA for Endpoint Access (Priority: Critical)}
\begin{description}
    \item[Action:] Deploy a mandatory Multi-Factor Authentication solution for all employee computer and laptop logins. This should apply to both on-premise and remote access.
    \item[Justification:] Stolen or weak passwords are a leading cause of security breaches. Enforcing MFA at the endpoint level provides a critical layer of defense that protects against unauthorized access even if an employee's credentials are compromised.
\end{description}

\subsection{Develop and Implement an Acceptable Use Policy (Priority: High)}
\begin{description}
    \item[Action:] Create a formal Acceptable Use Policy (AUP) that clearly defines the rules and expectations for all users of the organization's network and IT resources. This policy should be reviewed by legal counsel, communicated to all employees, and formally acknowledged via signature.
    \item[Justification:] An AUP is a foundational security document that reduces organizational risk by establishing clear guidelines for behavior, protecting against data loss, and providing a framework for enforcing security standards.
\end{description}

\subsection{Validate Risk Register Accuracy (Priority: Informational)}
\begin{description}
    \item[Action:] Conduct a review of the existing risk register to validate the status of other documented vulnerabilities. The finding that the "Unencrypted Web Server" risk was remediated on host \texttt{192.168.0.5} should be formally documented, and the risk should be marked as closed or updated.
    \item[Justification:] An accurate and up-to-date risk register is essential for effective security management, resource allocation, and strategic planning.
\end{description}

\end{document}
```