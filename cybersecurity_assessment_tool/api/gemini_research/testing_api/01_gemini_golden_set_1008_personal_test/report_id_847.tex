```latex
\documentclass[12pt]{article}

% Required Packages
\usepackage[margin=1in]{geometry}
\usepackage{pifont} % For checkmarks and crosses
\usepackage{booktabs} % For professional tables
\usepackage{hyperref} % For clickable links and references
\usepackage{url} % For URL formatting
\usepackage{seqsplit} % To split long monospaced strings
\usepackage{graphicx} % For potential logos
\usepackage{xcolor} % For colors in tables

% Document Metadata
\title{Cybersecurity Posture Assessment Report}
\author{Cybersecurity Analysis Division}
\date{\today}

% Hyperref Setup
\hypersetup{
    colorlinks=true,
    linkcolor=blue,
    filecolor=magenta,      
    urlcolor=cyan,
    pdftitle={Cybersecurity Posture Assessment Report},
    pdfpagemode=FullScreen,
}

% Custom Commands
\newcommand{\yes}{\ding{51}}
\newcommand{\no}{\ding{55}}
\definecolor{critical}{HTML}{990000}
\definecolor{high}{HTML}{D1451D}
\definecolor{medium}{HTML}{E8B514}
\definecolor{low}{HTML}{339900}

\begin{document}

\maketitle
\thispagestyle{empty}
\newpage

\tableofcontents
\newpage

% --- 1. Executive Summary ---
\section{Executive Summary}
This report provides a comprehensive analysis of the cybersecurity posture for \textbf{Opal Sky Media}. The assessment is based on a correlation of organizational data, a security controls questionnaire, and a network vulnerability scan conducted on the specified external assets.

The overall security posture has several foundational strengths, including the enforcement of Multi-Factor Authentication (MFA) for computer and sensitive system access. However, two significant gaps were identified that present an immediate and high level of risk to the organization:
\begin{itemize}
    \item \textbf{Critical Risk:} The absence of mandatory MFA for accessing email exposes the organization to a high likelihood of business email compromise (BEC), phishing attacks, and unauthorized account access.
    \item \textbf{High Risk:} The lack of annual, recurring security awareness training for all employees diminishes the organization's resilience against evolving social engineering tactics.
\end{itemize}

Additionally, a technical scan revealed an exposed SSH management port on an IPv6 address. While necessary for remote administration, if not properly secured, this service can be a target for brute-force attacks.

Immediate remediation of the identified critical and high-risk findings is strongly recommended to mitigate potential threats and strengthen the organization's defensive capabilities.

% --- 2. Organizational Information ---
\section{Organizational Information}
The following details were provided for the assessment scope.

\begin{tabular}{@{}ll}
\toprule
\textbf{Attribute} & \textbf{Value} \\
\midrule
Organization Name & \textbf{Opal Sky Media} \\
Primary Email Domain & \texttt{OpalSkyMedia.com} \\
Primary Website & \url{www.OpalSkyMedia.com} \\
Scoped External IP (IPv4) & \texttt{106.165.47.177} \\
Scoped External IP (IPv6) & \texttt{2001:db8::1} \\
\bottomrule
\end{tabular}

% --- 3. Security Control Review ---
\section{Security Control Review}
A review of self-reported security controls was conducted via a questionnaire. The responses are summarized below, highlighting areas that deviate from established best practices.

\begin{tabular}{@{}p{0.6\linewidth}cp{0.25\linewidth}@{}}
\toprule
\textbf{Control Question} & \textbf{Response} & \textbf{Assessment} \\
\midrule
Do you require MFA to access email? & \no & \textcolor{critical}{\textbf{Critical Gap.}} Lack of MFA on email is a primary vector for account takeover. \\
\addlinespace
Do you require MFA to log into computers? & \yes & Meets best practice. \\
\addlinespace
Do you require MFA to access sensitive data systems? & \yes & Meets best practice. \\
\addlinespace
Does your organization have an employee acceptable use policy? & \yes & Foundational policy is in place. \\
\addlinespace
Does your organization do security awareness training for new employees? & \yes & Good practice for onboarding. \\
\addlinespace
Does your organization do security awareness training for all employees at least once per year? & \no & \textcolor{high}{\textbf{High Risk.}} Security skills degrade and threats evolve. Annual training is essential. \\
\bottomrule
\end{tabular}

% --- 4. Technical Scan Results ---
\section{Technical Scan Results}
An external network scan was performed to identify open ports and exposed services on the target infrastructure.

\subsection{Target: \texttt{2001:db8::1}}
The scan identified the following open port on the specified IPv6 address.

\begin{tabular}{@{}llll@{}}
\toprule
\textbf{Port} & \textbf{State} & \textbf{Likely Service} & \textbf{Notes} \\
\midrule
22/tcp & Open & SSH (Secure Shell) & The service version was not enumerated in this scan. Exposed SSH ports are common targets for automated brute-force attacks. Access should be strictly controlled. \\
\bottomrule
\end{tabular}

% --- 5. Consolidated Risk Assessment ---
\section{Consolidated Risk Assessment}
The following table synthesizes findings from the security control review and technical scans into a prioritized list of risks.

\begin{tabular}{@{}p{0.1\linewidth}p{0.4\linewidth}p{0.15\linewidth}p{0.25\linewidth}@{}}
\toprule
\textbf{Risk ID} & \textbf{Description} & \textbf{Severity} & \textbf{Affected Asset(s)} \\
\midrule
RISK-001 & Lack of MFA on email systems allows for straightforward account compromise via stolen or weak credentials. & \textcolor{critical}{\textbf{Critical}} & Email System, User Accounts, Sensitive Data \\
\addlinespace
RISK-002 & Inadequate security training (not performed annually for all staff) increases susceptibility to phishing and other social engineering attacks. & \textcolor{high}{\textbf{High}} & All Employees, Overall Security Culture \\
\addlinespace
RISK-003 & The SSH management port (22) is exposed to the internet, creating a target for brute-force login attempts and exploitation of potential vulnerabilities. & \textcolor{medium}{\textbf{Medium}} & Server at \seqsplit{\texttt{2001:db8::1}} \\
\bottomrule
\end{tabular}

% --- 6. Recommendations ---
\section{Recommendations}
Based on the analysis, the following actions are recommended to mitigate the identified risks. Recommendations are prioritized by severity.

\subsection{RISK-001: Enforce MFA on Email (Critical)}
\begin{itemize}
    \item \textbf{Immediate Action:} Procure and implement a Multi-Factor Authentication solution for the organization's email platform.
    \item \textbf{Policy:} Update IT and security policies to mandate the use of MFA for all email access, including mobile and webmail clients.
    \item \textbf{Rollout:} Develop a phased rollout plan, starting with administrative and executive accounts, to ensure 100\% adoption across the organization.
\end{itemize}

\subsection{RISK-002: Implement Annual Security Training (High)}
\begin{itemize}
    \item \textbf{Program Development:} Establish a formal, mandatory security awareness training program for all employees to be completed annually.
    \item \textbf{Content:} Ensure the training content is updated yearly to reflect current threats, such as new phishing techniques, ransomware trends, and AI-driven social engineering.
    \item \textbf{Phishing Simulation:} Supplement the training with periodic, unannounced phishing simulation campaigns to measure effectiveness and provide targeted remediation for employees who are repeatedly susceptible.
\end{itemize}

\subsection{RISK-003: Secure Exposed SSH Service (Medium)}
\begin{itemize}
    \item \textbf{Access Control:} If remote SSH access is required, implement a firewall policy or Access Control List (ACL) to restrict source IP addresses to only trusted administrative networks (e.g., corporate VPN, specific office IPs).
    \item \textbf{Authentication Hardening:} Disable password-based authentication for SSH and enforce the use of public-key cryptography (SSH keys) exclusively.
    \item \textbf{Monitoring:} Implement a log monitoring solution (such as fail2ban) to detect and automatically block IPs that exhibit brute-force behavior against the SSH service.
\end{itemize}

\end{document}
```