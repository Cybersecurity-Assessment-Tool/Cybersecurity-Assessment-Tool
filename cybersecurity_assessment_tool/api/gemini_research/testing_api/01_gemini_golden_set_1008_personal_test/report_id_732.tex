```latex
\documentclass[12pt]{article}

% Preamble: Required Packages
\usepackage[margin=1in]{geometry}
\usepackage{pifont} % For checkmarks and crosses
\usepackage{booktabs} % For professional tables
\usepackage{hyperref} % For clickable links and ToC
\usepackage{url} % For formatting URLs
\usepackage{seqsplit} % For splitting long strings in tt font
\usepackage{graphicx}
\usepackage{xcolor}

% Hyperref Setup
\hypersetup{
    colorlinks=true,
    linkcolor=blue,
    filecolor=magenta,      
    urlcolor=cyan,
    pdftitle={Cybersecurity Posture Report},
    pdfpagemode=FullScreen,
}

% Document Information
\title{Cybersecurity Posture Report \\ \large For: Granite Shield}
\author{Cybersecurity Analyst}
\date{\today}

\begin{document}

\maketitle
\thispagestyle{empty}
\newpage

\tableofcontents
\newpage

% --- 1. Executive Overview ---
\section{Executive Overview}
This report provides a cybersecurity posture assessment for Granite Shield, based on organizational data provided, a security controls questionnaire, and an attempted external network scan.

The initial analysis reveals a \textbf{critical risk posture}. This conclusion is primarily driven by the complete absence of fundamental security controls, as indicated by the security questionnaire. Every control questioned, including Multi-Factor Authentication (MFA), employee security policies, and security awareness training, is reportedly not in place. These gaps expose the organization to significant threats, including account compromise, unauthorized access, and social engineering attacks.

Furthermore, the technical assessment was inconclusive. The provided network scan data (\texttt{Input\_1\_Network\_Scan\_JSON}) and existing risk data (\texttt{Input\_3\_Current\_Risks\_JSON}) were corrupted and could not be parsed. This prevents a full evaluation of the external attack surface and existing vulnerabilities.

Immediate and decisive action is required to implement foundational security measures to mitigate these severe risks. Recommendations are detailed in Section \ref{sec:recommendations}.

% --- 2. Organizational Information ---
\section{Organizational Information}
The following information was provided by the client and used as the basis for this assessment.

\begin{table}[h!]
\centering
\caption{Client Organizational Data}
\label{tab:org_data}
\begin{tabular}{@{}ll@{}}
\toprule
\textbf{Attribute} & \textbf{Value} \\ \midrule
Organization Name & Granite Shield \\
Email Domain & \texttt{GraniteShield.net} \\
Website Domain & \url{www.GraniteShield.net} \\
External IP Address & \texttt{29.211.190.13} \\ \bottomrule
\end{tabular}
\end{table}

% --- 3. Security Control Review ---
\section{Security Control Review}
A security questionnaire was completed to evaluate the implementation of essential administrative and technical controls. The results, detailed in Table \ref{tab:controls}, indicate critical deficiencies across all surveyed areas. The symbol \ding{55} represents a "No" answer, signifying a missing control.

\begin{table}[h!]
\centering
\caption{Security Controls Questionnaire Results}
\label{tab:controls}
\begin{tabular}{@{}lc@{}}
\toprule
\textbf{Control Question} & \textbf{Implemented?} \\ \midrule
Do you require MFA to access email? & \ding{55} \\
Do you require MFA to log into computers? & \ding{55} \\
Do you require MFA to access sensitive data systems? & \ding{55} \\
Does your organization have an employee acceptable use policy? & \ding{55} \\
Does your organization do security awareness training for new employees? & \ding{55} \\
Does your organization do security awareness training for all employees annually? & \ding{55} \\ \bottomrule
\end{tabular}
\end{table}

\subsection{Analysis of Control Gaps}
The lack of these fundamental controls presents an immediate and severe threat:
\begin{itemize}
    \item \textbf{No Multi-Factor Authentication (MFA):} The absence of MFA for email, computer logins, and sensitive systems means that a single compromised password could grant an attacker full access to critical assets. This is the most significant weakness identified.
    \item \textbf{No Acceptable Use Policy (AUP):} Without a formal AUP, there are no established rules for employees regarding the use of company assets. This can lead to risky behavior and creates ambiguity in enforcing security standards.
    \item \textbf{No Security Awareness Training:} Employees are the first line of defense. Without training, they are highly susceptible to phishing, social engineering, and other common attack vectors, rendering technical controls less effective.
\end{itemize}

% --- 4. Technical Scan Results ---
\section{Technical Scan Results}
An external network scan was intended to be performed against the organization's public-facing IP address to identify open ports, running services, and potential vulnerabilities.

\subsection{Scan Status}
\textbf{Analysis Failure:} The provided network scan data file (\texttt{Input\_1\_Network\_Scan\_JSON}) was found to be corrupted or incomplete. As a result, no technical analysis of the external perimeter could be performed. It is crucial to conduct a new, successful scan to understand the external attack surface.

\subsection{Findings}
No findings can be reported due to the data corruption issue.

% --- 5. Risk Assessment ---
\section{Risk Assessment}
This assessment is based on the conclusive results from the Security Control Review. The inability to analyze technical scan data or pre-existing risks means the overall risk posture could be significantly worse than what is documented here.

\begin{table}[h!]
\centering
\caption{Identified Risks and Severity}
\label{tab:risks}
\begin{tabular}{@{}p{0.3\linewidth}p{0.5\linewidth}l@{}}
\toprule
\textbf{Risk Name} & \textbf{Overview} & \textbf{Severity} \\ \midrule
\textbf{Widespread Lack of MFA} & User accounts for email, endpoints, and sensitive data are protected only by passwords, making them highly vulnerable to compromise via phishing, credential stuffing, or password spraying attacks. & \textcolor{red}{\textbf{Critical}} \\
\addlinespace
\textbf{No Security Awareness Program} & Employees are not trained to recognize or respond to security threats like phishing. This significantly increases the likelihood of a successful social engineering attack leading to a breach. & \textcolor{orange}{\textbf{High}} \\
\addlinespace
\textbf{Absence of Security Policies} & The lack of a formal Acceptable Use Policy means there is no governance defining secure employee behavior, creating an undisciplined security environment and potential insider threats. & \textcolor{orange}{\textbf{High}} \\
\addlinespace
\textbf{Incomplete Assessment Data} & The technical scan and existing risk data were unavailable. The true external attack surface and known vulnerabilities are currently unknown, preventing a comprehensive assessment. & Informational \\
\bottomrule
\end{tabular}
\end{table}

% --- 6. Recommendations ---
\section{Recommendations}
\label{sec:recommendations}
Based on the findings, the following actions are recommended to be implemented with urgency to reduce the organization's risk exposure.

\begin{enumerate}
    \item \textbf{Implement Multi-Factor Authentication (Critical):}
    \begin{itemize}
        \item \textbf{Priority 1:} Immediately enforce MFA on all email accounts (e.g., Office 365, Google Workspace).
        \item \textbf{Priority 2:} Roll out MFA for all remote access systems (VPNs) and administrative accounts.
        \item \textbf{Priority 3:} Expand MFA to protect all employee computer logins and access to systems containing sensitive data.
    \end{itemize}
    
    \item \textbf{Establish a Security Awareness Training Program (High):}
    \begin{itemize}
        \item Develop or procure a security awareness training module and ensure it is a mandatory part of the onboarding process for all new employees.
        \item Conduct annual security awareness training for all staff, including phishing simulations, to keep security top-of-mind.
    \end{itemize}
    
    \item \textbf{Develop and Enforce an Acceptable Use Policy (AUP) (High):}
    \begin{itemize}
        \item Draft a clear AUP that outlines the rules for using company networks, devices, and data.
        \item Require all employees to read and formally acknowledge the policy.
    \end{itemize}
    
    \item \textbf{Remediate Assessment Data Gaps (Informational):}
    \begin{itemize}
        \item Conduct a new, authenticated external network vulnerability scan against the IP address \texttt{29.211.190.13} to identify technical vulnerabilities.
        \item Provide the uncorrupted existing risk register data (\texttt{Input\_3\_Current\_Risks\_JSON}) for a more complete analysis.
    \end{itemize}
\end{enumerate}

\end{document}
```