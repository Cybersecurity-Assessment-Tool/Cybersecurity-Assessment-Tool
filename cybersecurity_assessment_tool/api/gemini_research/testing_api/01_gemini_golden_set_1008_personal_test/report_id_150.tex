```latex
\documentclass[12pt]{article}

% ----------------------------------------------------------------------
% PREAMBLE
% ----------------------------------------------------------------------
\usepackage[margin=1in]{geometry} % Set page margins
\usepackage{pifont}               % For checkmark and X symbols (\ding)
\usepackage{booktabs}             % For professional-looking tables
\usepackage{hyperref}             % For clickable links and better PDF navigation
\usepackage{url}                  % For formatting URLs
\usepackage{seqsplit}             % To split long strings in \texttt
\usepackage{graphicx}             % For logos, etc. (optional but good practice)
\usepackage{xcolor}               % For colored text

% --- Hyperref Setup ---
\hypersetup{
    colorlinks=true,
    linkcolor=blue,
    filecolor=magenta,      
    urlcolor=cyan,
    pdftitle={Cybersecurity Posture Report},
    pdfpagemode=FullScreen,
}

% --- Custom Commands ---
\newcommand{\yes}{\ding{51}} % Checkmark
\newcommand{\no}{\ding{55}}  % X mark

% ----------------------------------------------------------------------
% DOCUMENT START
% ----------------------------------------------------------------------
\begin{document}

% --- TITLE PAGE ---
\begin{titlepage}
    \centering
    \vspace*{\fill}
    \Huge \textbf{Cybersecurity Posture Report} \\
    \vspace{1.5cm}
    \Large \textbf{Prepared for:} \\
    \Large \textbf{Grizzly Peak} \\
    \vspace{2cm}
    \large \textbf{Date of Report:} \today \\
    \vspace*{\fill}
\end{titlepage}

\tableofcontents
\newpage

% ----------------------------------------------------------------------
% 1. EXECUTIVE SUMMARY
% ----------------------------------------------------------------------
\section{Executive Summary}

This report provides a consolidated analysis of the cybersecurity posture for \textbf{Grizzly Peak}, based on a review of organizational security controls, technical network scanning, and pre-existing risk data.

The assessment reveals a mixed security posture. The organization demonstrates strong maturity in identity and access management through the consistent enforcement of Multi-Factor Authentication (MFA) across key systems. Additionally, a robust security awareness training program is in place for all employees.

However, a critical administrative gap was identified: the absence of a formal Employee Acceptable Use Policy (AUP). This exposes the organization to insider threats and legal liabilities by failing to establish clear guidelines for technology use.

On the technical front, a network scan of the target host \texttt{192.168.0.5} showed no open ports or exposed services, indicating a secure configuration for that specific asset. This finding contradicts a pre-existing documented risk concerning an unencrypted web server on Port 80. This suggests the risk may be remediated on the scanned host or applies to other network assets not included in this scan's scope.

Recommendations are prioritized to first address the policy gap and then to validate the status of the web server risk across the entire network.

% ----------------------------------------------------------------------
% 2. ORGANIZATIONAL INFORMATION
% ----------------------------------------------------------------------
\section{Organizational Information}

The following details were provided for the assessment:

\begin{itemize}
    \item \textbf{Organization Name:} Grizzly Peak
    \item \textbf{Email Domain:} \texttt{GrizzlyPeak.org}
    \item \textbf{Website Domain:} \texttt{www.GrizzlyPeak.org}
    \item \textbf{External IP Address:} \texttt{202.227.191.138}
\end{itemize}

% ----------------------------------------------------------------------
% 3. SECURITY CONTROL REVIEW (QUESTIONNAIRE)
% ----------------------------------------------------------------------
\section{Security Control Review}

An analysis of the organization's security questionnaire responses highlights the current state of administrative and preventative controls. "No" answers indicate significant gaps that increase organizational risk.

\begin{table}[h!]
\centering
\caption{Security Control Questionnaire Analysis}
\begin{tabular}{p{0.7\linewidth} c c}
\toprule
\textbf{Control Question} & \textbf{Response} & \textbf{Status} \\
\midrule
Do you require MFA to access email? & Yes & \textcolor{green}{\yes} \\
Do you require MFA to log into computers? & Yes & \textcolor{green}{\yes} \\
Do you require MFA to access sensitive data systems? & Yes & \textcolor{green}{\yes} \\
\addlinespace[0.5em]
Does your organization have an employee acceptable use policy? & No & \textcolor{red}{\no} \\
\addlinespace[0.5em]
Does your organization do security awareness training for new employees? & Yes & \textcolor{green}{\yes} \\
Does your organization do security awareness training for all employees at least once per year? & Yes & \textcolor{green}{\yes} \\
\bottomrule
\end{tabular}
\end{table}

\paragraph{Finding:} The lack of an Employee Acceptable Use Policy is a critical administrative control gap. This policy is foundational for setting security expectations, defining prohibited activities, and providing a basis for disciplinary action in case of violations.

% ----------------------------------------------------------------------
% 4. TECHNICAL SCAN RESULTS
% ----------------------------------------------------------------------
\section{Technical Scan Results}

A network scan was performed on the specified target to identify open ports and exposed services.

\begin{itemize}
    \item \textbf{Target IP Address:} \texttt{192.168.0.5}
    \item \textbf{Scanner Used:} Nmap
\end{itemize}

\begin{table}[h!]
\centering
\caption{Network Scan Findings for \texttt{192.168.0.5}}
\begin{tabular}{l l l l l}
\toprule
\textbf{Port} & \textbf{State} & \textbf{Service} & \textbf{Product} & \textbf{Version} \\
\midrule
80/tcp & closed & http & N/A & N/A \\
\bottomrule
\end{tabular}
\end{table}

\paragraph{Finding:} The scan of the target host revealed no open ports or identifiable services. This indicates a strong network perimeter for this specific device at the time of the scan. This result conflicts with the pre-existing risk documented in the following section.

% ----------------------------------------------------------------------
% 5. CONSOLIDATED RISK ASSESSMENT
% ----------------------------------------------------------------------
\section{Consolidated Risk Assessment}

The following table synthesizes risks identified from the security control review, technical scans, and pre-existing risk documentation.

\begin{table}[h!]
\centering
\caption{Summary of Identified Risks}
\begin{tabular}{p{0.25\linewidth} p{0.1\linewidth} p{0.55\linewidth}}
\toprule
\textbf{Risk Name} & \textbf{Severity} & \textbf{Overview \& Affected Elements} \\
\midrule
\textbf{Lack of Acceptable Use Policy} & \textbf{High} & The organization does not have a formal AUP. This creates ambiguity in security responsibilities and acceptable user behavior, increasing the risk of insider threat and policy violations. \newline \textit{Affected: All Employees, Organizational Governance} \\
\addlinespace[1em]
\textbf{Unencrypted Web Server} & \textbf{Medium} & A pre-existing risk indicates a web server is operating over unencrypted HTTP on Port 80. \textbf{Note:} The current scan of \texttt{192.168.0.5} found this port to be closed, suggesting this risk may be remediated on this host or applies to other assets. \newline \textit{Affected: Potentially Public-Facing Web Assets} \\
\bottomrule
\end{tabular}
\end{table}

% ----------------------------------------------------------------------
% 6. RECOMMENDATIONS
% ----------------------------------------------------------------------
\section{Recommendations}

The following actionable recommendations are provided to mitigate the identified risks and improve the overall security posture of \textbf{Grizzly Peak}.

\subsection{Priority 1 (High): Develop and Implement an Acceptable Use Policy}
\begin{itemize}
    \item \textbf{Action:} Draft a comprehensive AUP that clearly defines the rules and expectations for the use of company technology, networks, and data. This policy should cover topics such as data handling, internet usage, password security, and incident reporting.
    \item \textbf{Justification:} An AUP is a critical administrative control that reduces organizational risk by establishing a clear security baseline for all employees. It provides legal protection and a formal framework for enforcing security standards.
    \item \textbf{Resources:} SANS provides excellent, free templates for creating security policies, including AUPs.
\end{itemize}

\subsection{Priority 2 (Medium): Validate and Remediate Web Server Risk}
\begin{itemize}
    \item \textbf{Action:} Conduct a comprehensive vulnerability scan across all internal and external network assets to confirm if any hosts are running services on Port 80 (HTTP).
    \item \textbf{Justification:} The discrepancy between the pre-existing risk data and the recent scan requires clarification. If an unencrypted web server is found, it must be remediated immediately to prevent data interception and protect user credentials.
    \item \textbf{Long-Term Fix:} For any required web services, implement Transport Layer Security (TLS/SSL) to enforce HTTPS (Port 443). Configure web servers to redirect all HTTP traffic to HTTPS and disable Port 80 where possible.
\end{itemize}

\end{document}
```