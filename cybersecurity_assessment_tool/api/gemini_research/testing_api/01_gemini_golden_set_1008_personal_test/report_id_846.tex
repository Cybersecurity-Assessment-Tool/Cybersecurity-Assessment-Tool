```latex
\documentclass[12pt]{article}

% === PACKAGES ===
\usepackage[margin=1in]{geometry}
\usepackage{pifont} % For checkmarks and crosses
\usepackage{booktabs} % For professional tables
\usepackage{hyperref} % For clickable links and metadata
\usepackage{url} % For URL formatting
\usepackage{seqsplit} % To split long strings in tt font

% === DOCUMENT METADATA ===
\hypersetup{
    colorlinks=true,
    linkcolor=blue,
    filecolor=magenta,      
    urlcolor=cyan,
    pdftitle={Cybersecurity Assessment Report},
    pdfauthor={Cybersecurity Analysis Division},
    pdfsubject={Security Posture Analysis},
    pdfkeywords={Cybersecurity, Risk, Assessment, Scan},
    bookmarks=true
}

% === CUSTOM COMMANDS ===
\newcommand{\yes}{\ding{51}} % Green checkmark
\newcommand{\no}{\ding{55}}  % Red X

% === DOCUMENT START ===
\begin{document}

\title{Cybersecurity Assessment Report \\ \large For: \textbf{Nova Terra}}
\author{Cybersecurity Analysis Division}
\date{\today}
\maketitle

\section*{Executive Summary}

This report provides a comprehensive analysis of the cybersecurity posture of \textbf{Nova Terra}, based on a network scan, a security controls questionnaire, and a review of pre-existing risks.

The assessment reveals a mixed security posture. The organization demonstrates strong foundational controls in user authentication, with Multi-Factor Authentication (MFA) widely implemented across key systems. Furthermore, a previously identified risk concerning an unencrypted web server on port 80 has been successfully remediated, as our technical scan confirmed the port is now closed.

However, significant gaps were identified in security governance and employee training. The absence of a formal Acceptable Use Policy (AUP) and the lack of mandatory annual security awareness training for all employees represent high-risk areas. These policy-level deficiencies expose the organization to substantial human-factor risks, such as insider threats and susceptibility to social engineering attacks like phishing.

Immediate action is recommended to develop and implement these foundational security policies to mitigate the identified risks and strengthen the overall security framework.

\section{Organizational Information}

The following information was provided for the assessment:
\begin{center}
\begin{tabular}{ll}
\toprule
\textbf{Item} & \textbf{Detail} \\
\midrule
Organization Name & \textbf{Nova Terra} \\
Email Domain & \texttt{NovaTerra.net} \\
Website Domain & \url{www.NovaTerra.net} \\
External IP Address & \seqsplit{\texttt{56.129.156.26}} \\
\bottomrule
\end{tabular}
\end{center}

\section{Security Control Review}

A review of the organization's security controls was conducted via a questionnaire. The results indicate strong technical controls for authentication but highlight critical gaps in administrative policies.

\begin{center}
\begin{tabular}{p{0.6\textwidth} c l}
\toprule
\textbf{Control Question} & \textbf{Response} & \textbf{Assessment} \\
\midrule
Do you require MFA to access email? & \yes & Best Practice Met \\
Do you require MFA to log into computers? & \yes & Best Practice Met \\
Do you require MFA to access sensitive data systems? & \yes & Best Practice Met \\
\addlinespace
Does your organization have an employee acceptable use policy? & \no & \textbf{Critical Gap} \\
\addlinespace
Does your organization do security awareness training for new employees? & \yes & Good Practice \\
Does your organization do security awareness training for all employees at least once per year? & \no & \textbf{High Risk} \\
\bottomrule
\end{tabular}
\end{center}

\section{Technical Scan Results}

A network scan was performed on the specified target to identify open ports and exposed services. The scan results are detailed below.

\begin{itemize}
    \item \textbf{Target IP Address:} \seqsplit{\texttt{192.168.0.5}}
    \item \textbf{Scan Date:} As per scan metadata.
\end{itemize}

\begin{center}
\begin{tabular}{llll}
\toprule
\textbf{Port} & \textbf{State} & \textbf{Service} & \textbf{Product / Version} \\
\midrule
80/tcp & closed & http & N/A \\
\bottomrule
\end{tabular}
\end{center}
\vspace{1em}

\textbf{Analysis:} The scan indicates that port 80 (HTTP) is closed on the target system. This finding confirms that the previously identified risk, "Unencrypted Web Server," has been successfully remediated. No other open ports or active services were discovered during this assessment, indicating a minimal attack surface on the scanned host.

\section{Consolidated Risk Assessment}

The following table synthesizes findings from the security control review and technical analysis. Risks are prioritized based on their potential impact on the organization.

\begin{center}
\begin{tabular}{p{0.25\textwidth} p{0.5\textwidth} l}
\toprule
\textbf{Risk Name} & \textbf{Description} & \textbf{Severity} \\
\midrule
\textbf{Lack of Acceptable Use Policy (AUP)} & The absence of a formal AUP creates ambiguity regarding the proper use of company assets and data. This increases the risk of insider threat, data leakage, and legal liability. & \textbf{High} \\
\addlinespace
\textbf{Inadequate Security Awareness Training} & While new hires receive training, the lack of annual training for all staff means existing employees may not be aware of evolving threats (e.g., new phishing techniques), increasing the likelihood of a security breach. & \textbf{High} \\
\addlinespace
\textit{Unencrypted Web Server (Remediated)} & \textit{A previously identified risk of an open port 80 has been confirmed as closed. This risk is now considered remediated and is listed for tracking purposes only.} & \textit{Informational} \\
\bottomrule
\end{tabular}
\end{center}

\section{Recommendations}

Based on the consolidated risk assessment, the following actions are recommended to enhance the security posture of \textbf{Nova Terra}.

\begin{enumerate}
    \item \textbf{High Priority: Develop and Implement an Acceptable Use Policy (AUP).}
    \begin{itemize}
        \item Draft a comprehensive AUP that clearly defines rules for employee use of networks, computers, email, and data.
        \item The policy should explicitly state prohibited activities and outline consequences for violations.
        \item Require all current and future employees to read and formally acknowledge the policy as a condition of employment.
    \end{itemize}
    
    \item \textbf{High Priority: Establish a Mandatory Annual Security Awareness Training Program.}
    \begin{itemize}
        \item Procure or develop a training program that covers current cybersecurity threats, including phishing, ransomware, social engineering, and password hygiene.
        \item Make this training mandatory for all employees, regardless of role or tenure, on an annual basis.
        \item Implement a system to track completion and ensure 100\% compliance.
    \end{itemize}

    \item \textbf{Informational: Continue Proactive Vulnerability Management.}
    \begin{itemize}
        \item The organization is commended for remediating the previously identified open port.
        \item Continue to perform regular, automated vulnerability scans on all external and internal assets to proactively identify and address new security weaknesses.
    \end{itemize}
\end{enumerate}

\end{document}
```