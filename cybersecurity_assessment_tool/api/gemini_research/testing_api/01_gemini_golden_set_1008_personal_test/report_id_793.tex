```latex
\documentclass[12pt]{article}

% Preamble: Required Packages
\usepackage[margin=1in]{geometry}
\usepackage{pifont} % For checkmarks and crosses
\usepackage{booktabs} % For professional tables
\usepackage{hyperref} % For clickable links and references
\usepackage{url} % For formatting URLs
\usepackage{seqsplit} % For splitting long strings in texttt
\usepackage{graphicx} % For potential logos
\usepackage{fancyhdr} % For headers/footers

% Document Metadata
\title{Cybersecurity Posture Assessment Report}
\author{Cybersecurity Analysis Division}
\date{November 22, 2025}

% Header and Footer Configuration
\pagestyle{fancy}
\fancyhf{} % Clear all header and footer fields
\fancyhead[L]{Catalyst Consulting // Confidential}
\fancyfoot[C]{\thepage}
\renewcommand{\headrulewidth}{0.4pt}
\renewcommand{\footrulewidth}{0.4pt}

\begin{document}

\maketitle
\thispagestyle{empty}
\newpage

\tableofcontents
\newpage

% --- Section 1: Executive Overview ---
\section{Executive Overview}

This report provides a comprehensive cybersecurity assessment for Catalyst Consulting, based on data collected on November 22, 2025. The analysis correlates information from organizational security questionnaires, external network scans, and a review of pre-existing risks.

The assessment reveals several critical and high-risk gaps in the current security posture. The most significant concerns are the absence of Multi-Factor Authentication (MFA) on email accounts and the complete lack of a security awareness training program. These policy and procedural gaps expose the organization to significant threats, including business email compromise (BEC), phishing, and ransomware.

From a technical standpoint, the external-facing web server is running an outdated version of Nginx (1.18.0), which is known to have multiple published vulnerabilities. This presents a direct, exploitable pathway for attackers to compromise the server.

Immediate remediation efforts should focus on implementing MFA for email, establishing a security awareness training program, and upgrading the vulnerable web server software. Addressing these key areas will substantially improve the organization's resilience against common cyber threats.

% --- Section 2: Organizational Information ---
\section{Organizational Information}

The following details were provided for the assessment. This information forms the basis for understanding the organization's digital footprint and internal context.

\begin{tabular}{@{}ll}
\toprule
\textbf{Attribute} & \textbf{Value} \\
\midrule
Organization Name & Catalyst Consulting \\
Email Domain & \texttt{CatalystConsulting.com} \\
Website Domain & \url{www.CatalystConsulting.com} \\
External IP Address & \texttt{218.129.156.21} \\
\bottomrule
\end{tabular}

% --- Section 3: Security Control Review ---
\section{Security Control Review}

A review of administrative and policy-based security controls was conducted via a questionnaire. The results below highlight the current state of implementation. A checkmark (\ding{51}) indicates a positive control is in place, while a cross (\ding{55}) indicates a gap.

\begin{tabular}{@{}p{0.8\textwidth}c}
\toprule
\textbf{Control Question} & \textbf{Status} \\
\midrule
Do you require MFA to access email? & \ding{55} \\
Do you require MFA to log into computers? & \ding{51} \\
Do you require MFA to access sensitive data systems? & \ding{51} \\
Does your organization have an employee acceptable use policy? & \ding{55} \\
Does your organization do security awareness training for new employees? & \ding{55} \\
Does your organization do security awareness training for all employees at least once per year? & \ding{55} \\
\bottomrule
\end{tabular}

\subsection*{Analysis of Gaps}
The review identified three critical control gaps:
\begin{itemize}
    \item \textbf{No MFA for Email:} This is a critical vulnerability. Email is a primary target for attackers, and a compromised account can lead to data breaches, financial fraud, and further network intrusion.
    \item \textbf{No Acceptable Use Policy (AUP):} The absence of an AUP means there are no formal, enforceable rules for how employees should use company technology and data, increasing the risk of insider threats and accidental data loss.
    \item \textbf{No Security Awareness Training:} Without training, employees are significantly more likely to fall victim to phishing and other social engineering attacks, rendering technical controls less effective.
\end{itemize}

% --- Section 4: Technical Scan Results ---
\section{Technical Scan Results}

An external network scan was performed to identify open ports and exposed services on the organization's public-facing infrastructure.

\begin{itemize}
    \item \textbf{Scan Target:} \texttt{192.168.10.5}
    \item \textbf{Scan Date:} 2025-11-22
\end{itemize}

The following table details the findings from the scan.

\begin{tabular}{@{}lllll}
\toprule
\textbf{Port} & \textbf{State} & \textbf{Service} & \textbf{Product} & \textbf{Version} \\
\midrule
443/tcp & Open & https & nginx & 1.18.0 \\
\bottomrule
\end{tabular}

\subsection*{Analysis of Findings}
The scan identified one open port running a web server.
\begin{itemize}
    \item \textbf{Outdated Nginx Server:} The web server is running Nginx version 1.18.0, which was released in April 2020. This version is outdated and no longer receives security patches. It is susceptible to numerous publicly disclosed vulnerabilities (CVEs) that could allow an attacker to compromise the server, deface the website, or gain access to the internal network.
\end{itemize}

% --- Section 5: Consolidated Risk Assessment ---
\section{Consolidated Risk Assessment}

This section synthesizes the findings from the security control review and the technical scan into a consolidated list of identified risks.

\begin{tabular}{@{}p{0.25\textwidth}p{0.55\textwidth}l}
\toprule
\textbf{Risk Name} & \textbf{Overview} & \textbf{Severity} \\
\midrule
\textbf{Lack of MFA on Email} & The absence of MFA on the primary communication platform allows for credential-based account takeovers, leading to potential business email compromise, data exfiltration, and phishing attacks against partners. & \textbf{Critical} \\
\addlinespace
\textbf{No Security Awareness Training Program} & Employees are not trained to identify or respond to social engineering attacks like phishing. This makes them the weakest link and dramatically increases the likelihood of a security breach originating from human error. & \textbf{Critical} \\
\addlinespace
\textbf{Outdated Web Server Software (Nginx)} & The public-facing web server is running an old version of Nginx with known, exploitable vulnerabilities. This provides a direct attack vector for external threats to compromise the system. & \textbf{High} \\
\addlinespace
\textbf{No Employee Acceptable Use Policy} & Without a formal policy, there is no governance framework to guide employee behavior regarding data handling and system usage. This increases the risk of both malicious and accidental insider threats. & \textbf{High} \\
\bottomrule
\end{tabular}

% --- Section 6: Recommendations ---
\section{Recommendations}

The following actionable recommendations are provided to mitigate the identified risks. They are prioritized based on severity and potential impact.

\begin{enumerate}
    \item \textbf{Implement MFA for Email (Priority: Immediate):}
    \begin{itemize}
        \item Enforce mandatory MFA for all user access to the email system (\texttt{CatalystConsulting.com}).
        \item Provide clear instructions and support to all employees during the rollout process.
        \item This is the single most effective control to prevent email account compromise.
    \end{itemize}

    \item \textbf{Establish a Security Awareness Training Program (Priority: Immediate):}
    \begin{itemize}
        \item Procure and implement a security awareness training solution.
        \item Mandate training for all new hires as part of their onboarding process.
        \item Require all current employees to complete the training annually.
        \item Conduct regular phishing simulations to test and reinforce the training.
    \end{itemize}

    \item \textbf{Upgrade Nginx Web Server (Priority: High):}
    \begin{itemize}
        \item Develop a patch management plan to upgrade the Nginx server running on \texttt{192.168.10.5} to a current, stable, and fully patched version.
        \item Test the upgrade in a staging environment before deploying to production to ensure no disruption to the website (\url{www.CatalystConsulting.com}).
        \item Implement a regular patch management cycle for all external-facing systems.
    \end{itemize}

    \item \textbf{Develop and Implement an Acceptable Use Policy (Priority: High):}
    \begin{itemize}
        \item Draft a clear and concise AUP that defines the rules for using company assets, including computers, networks, and data.
        \item Have the policy reviewed by legal and HR departments.
        \item Require all employees to read and formally acknowledge the policy.
    \end{itemize}
\end{enumerate}

\end{document}
```