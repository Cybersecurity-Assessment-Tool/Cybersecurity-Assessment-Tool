```latex
\documentclass[12pt]{article}

% Preamble: Required Packages
\usepackage[margin=1in]{geometry}
\usepackage{pifont} % For checkmarks and crosses
\usepackage{booktabs} % For professional tables
\usepackage{hyperref} % For clickable links and references
\usepackage{url} % For formatting URLs
\usepackage{seqsplit} % To split long strings without breaking
\usepackage{graphicx}
\usepackage{xcolor}
\usepackage{fancyhdr}

% --- Document Setup ---
\hypersetup{
    colorlinks=true,
    linkcolor=blue,
    filecolor=magenta,      
    urlcolor=cyan,
    pdftitle={Cybersecurity Posture Report},
    pdfpagemode=FullScreen,
}

% Define colors for severity
\definecolor{criticalred}{HTML}{990000}
\definecolor{highorange}{HTML}{E69138}
\definecolor{mediumyellow}{HTML}{F1C232}
\definecolor{lowblue}{HTML}{3D85C6}
\definecolor{infogray}{HTML}{999999}

% Header and Footer
\pagestyle{fancy}
\fancyhf{}
\fancyhead[L]{Cybersecurity Posture Report}
\fancyhead[R]{Sovereign Trust}
\fancyfoot[C]{\thepage}

% --- Document Body ---
\begin{document}

% --- Title Page ---
\begin{titlepage}
    \centering
    \vspace*{1cm}
    \Huge{\textbf{Cybersecurity Posture Report}}
    \vspace{1.5cm}
    \Large{\textbf{Prepared for:}} \\
    \vspace{0.5cm}
    \Large{Sovereign Trust}
    \vspace{2cm}
    \large{\textbf{Date of Report:}} \\
    \vspace{0.5cm}
    \large{\today}
    \vfill
    \large{This report contains a consolidated analysis of organizational security controls, technical network scan results, and pre-existing risk data. It is intended for management and technical teams to understand the current security posture and prioritize remediation efforts.}
\end{titlepage}

\tableofcontents
\newpage

% --- Section 1: Executive Summary ---
\section{Executive Summary}
This report provides a comprehensive security assessment for Sovereign Trust, synthesizing information from a security questionnaire, a network vulnerability scan, and a list of current risks.

The assessment identified two \textbf{critical} security gaps related to the lack of Multi-Factor Authentication (MFA) for email access and computer logins. These represent a significant risk of unauthorized access and potential account compromise. Immediate action is required to mitigate these vulnerabilities.

On a positive note, the technical network scan of the target host \texttt{192.168.0.5} showed no open ports. This finding directly contradicts a pre-existing risk item concerning an "Unencrypted Web Server" on port 80. This suggests that the previously identified risk has been successfully remediated or was a false positive. Further validation is recommended to formally close this risk.

Overall, while the organization demonstrates a solid foundation in security policies and training, the identified MFA gaps must be the highest priority for remediation to significantly improve the security posture.

% --- Section 2: Organizational Information ---
\section{Organizational Information}
The following details were provided for the assessment.
\begin{itemize}
    \item \textbf{Organization Name:} Sovereign Trust
    \item \textbf{Email Domain:} \texttt{SovereignTrust.org}
    \item \textbf{Website Domain:} \url{www.SovereignTrust.org}
    \item \textbf{External IP Address:} \texttt{140.213.174.53}
\end{itemize}

% --- Section 3: Security Control Review ---
\section{Security Control Review (Questionnaire)}
The following table summarizes the organization's self-reported security controls. "No" answers indicate potential gaps that increase risk.

\begin{table}[h!]
\centering
\caption{Security Controls Questionnaire Analysis}
\label{tab:controls}
\begin{tabular}{@{}p{0.6\linewidth} c p{0.25\linewidth}@{}}
\toprule
\textbf{Control Question} & \textbf{Status} & \textbf{Analyst Note} \\
\midrule
Do you require MFA to access email? & \ding{55} & \textcolor{criticalred}{\textbf{Critical Risk.}} Lack of MFA on email is a primary vector for account takeover. \\
\addlinespace
Do you require MFA to log into computers? & \ding{55} & \textcolor{highorange}{\textbf{High Risk.}} Increases risk of lateral movement if credentials are stolen. \\
\addlinespace
Do you require MFA to access sensitive data systems? & \ding{51} & Good. Protects critical assets. \\
\addlinespace
Does your organization have an employee acceptable use policy? & \ding{51} & Good. Establishes a baseline for security. \\
\addlinespace
Does your organization do security awareness training for new employees? & \ding{51} & Good. Foundational for new hires. \\
\addlinespace
Does your organization do security awareness training for all employees at least once per year? & \ding{51} & Good. Maintains security awareness. \\
\bottomrule
\end{tabular}
\end{table}

% --- Section 4: Technical Scan Results ---
\section{Technical Scan Results}
An Nmap scan was performed to identify open ports and services on the specified target.

\begin{itemize}
    \item \textbf{Target IP:} \texttt{192.168.0.5}
    \item \textbf{Scan Date:} \today
\end{itemize}

\subsection{Port Scan Findings}
The scan results for the target host are detailed below.
\begin{table}[h!]
\centering
\caption{Nmap Port Scan Results for \texttt{192.168.0.5}}
\label{tab:nmap}
\begin{tabular}{@{}llll@{}}
\toprule
\textbf{Port} & \textbf{State} & \textbf{Service} & \textbf{Version} \\
\midrule
80/tcp & closed & http & N/A \\
\bottomrule
\end{tabular}
\end{table}

\subsection{Technical Analysis}
The scan of host \texttt{192.168.0.5} revealed a strong security posture, with no open ports detected. The scan specifically confirmed that port 80 (HTTP) is \textbf{closed}. This finding is significant as it contradicts a previously documented risk (see Section 5). This indicates either a successful remediation of a past vulnerability or that the initial finding was a false positive.

% --- Section 5: Consolidated Risk Assessment ---
\section{Consolidated Risk Assessment}
This section synthesizes findings from the security questionnaire, technical scan, and pre-existing risk data into a prioritized list.

\begin{table}[h!]
\centering
\caption{Consolidated Risk Register}
\label{tab:risks}
\begin{tabular}{@{}p{0.4\linewidth} p{0.15\linewidth} p{0.35\linewidth}@{}}
\toprule
\textbf{Risk Name} & \textbf{Severity} & \textbf{Status \& Analyst Notes} \\
\midrule
\textbf{Lack of MFA for Email Access} & \textcolor{criticalred}{\textbf{Critical}} & \textbf{New Finding.} Identified from questionnaire. This is the highest priority risk. Failure to protect email exposes the organization to phishing, data breaches, and further system compromise. \\
\addlinespace
\textbf{Lack of MFA for Computer Login} & \textcolor{highorange}{\textbf{High}} & \textbf{New Finding.} Identified from questionnaire. Stolen credentials could lead to direct endpoint compromise and lateral movement within the network. \\
\addlinespace
\textbf{Unencrypted Web Server} & \textcolor{mediumyellow}{Medium} & \textbf{Contradicted by Scan.} The pre-existing risk stated port 80 was open. Our scan on \today\ confirmed port 80 is \textbf{closed} on the target. This risk should be validated and likely closed. \\
\bottomrule
\end{tabular}
\end{table}

% --- Section 6: Recommendations ---
\section{Recommendations}
Based on the consolidated risk assessment, the following actions are recommended, ordered by priority.

\subsection{Priority 1 (Critical): Implement MFA for Email}
\begin{itemize}
    \item \textbf{Action:} Enforce MFA for all user accounts accessing the email system (e.g., Microsoft 365, Google Workspace).
    \item \textbf{Justification:} Email is a primary target for attackers. MFA provides a critical layer of defense against credential theft and phishing attacks, preventing unauthorized access even if a password is compromised.
    \item \textbf{Suggested Solutions:} Use authenticator apps (e.g., Google Authenticator, Microsoft Authenticator), hardware tokens, or biometric authentication.
\end{itemize}

\subsection{Priority 2 (High): Implement MFA for Endpoint Login}
\begin{itemize}
    \item \textbf{Action:} Deploy MFA for all employee logins to company-managed computers (desktops and laptops).
    \item \textbf{Justification:} This measure protects against unauthorized physical access and the use of stolen credentials to access the internal network, thereby preventing lateral movement and deeper system compromise.
    \item \textbf{Suggested Solutions:} Windows Hello for Business, Duo Security, or similar endpoint MFA solutions.
\end{itemize}

\subsection{Priority 3 (Informational): Validate and Close Old Risk}
\begin{itemize}
    \item \textbf{Action:} Formally review the "Unencrypted Web Server" risk. Confirm with system administrators that port 80 is intentionally closed on relevant systems.
    \item \textbf{Justification:} The recent scan data indicates this risk is remediated. Maintaining an accurate risk register is crucial for focusing resources on current, active threats. If the remediation is confirmed, the risk ticket should be formally closed.
\end{itemize}

\end{document}
```