```latex
\documentclass[12pt]{article}

% Required Packages
\usepackage[margin=1in]{geometry}
\usepackage{pifont} % For checkmarks and crosses
\usepackage{booktabs} % For professional tables
\usepackage{hyperref} % For clickable links
\usepackage{url} % For URL formatting
\usepackage{seqsplit} % To split long strings in texttt
\usepackage{graphicx}
\usepackage{xcolor}

% Hyperref Setup
\hypersetup{
    colorlinks=true,
    linkcolor=blue,
    filecolor=magenta,      
    urlcolor=cyan,
    pdftitle={Cybersecurity Assessment Report},
    pdfpagemode=FullScreen,
}

% Define checkmark and cross symbols for clarity
\newcommand{\cmark}{\ding{51}}%
\newcommand{\xmark}{\ding{55}}%

% --- Document Start ---
\begin{document}

% --- Title Page ---
\begin{titlepage}
    \centering
    \vspace*{1cm}
    \Huge\textbf{Cybersecurity Assessment Report}
    \vspace{1.5cm}
    \Large
    \textbf{Prepared for:}\\
    Stone Arch Masonry
    \vspace{2cm}
    \large
    \textbf{Date of Report:}\\
    \today
    \vfill
    \large
    \textit{This report contains sensitive information and should be handled with care. Distribution is restricted to authorized personnel only.}
\end{titlepage}

\tableofcontents
\newpage

% --- Section 1: Executive Summary ---
\section{Executive Summary}

This report provides a comprehensive cybersecurity assessment for Stone Arch Masonry, based on an analysis of network scan data, organizational security controls, and pre-existing risk information. The assessment has identified several areas of significant concern that require immediate attention.

A \textbf{critical vulnerability} was discovered on an internal network host (\texttt{10.0.0.15}). The host is running an outdated and vulnerable version of vsftpd (2.3.4) with anonymous FTP login enabled. This configuration is susceptible to a known remote code execution backdoor (CVE-2011-2523) and poses an immediate and severe threat to the internal network.

Furthermore, significant gaps in administrative and policy controls were identified. The lack of mandatory Multi-Factor Authentication (MFA) for computer logins, the absence of an employee Acceptable Use Policy (AUP), and the failure to provide security awareness training to new hires represent high-risk deficiencies. These gaps weaken the organization's defense against common attack vectors like phishing and unauthorized access.

This report outlines these findings in detail and provides a prioritized list of actionable recommendations to mitigate the identified risks and strengthen the overall security posture of Stone Arch Masonry.

% --- Section 2: Organizational Information ---
\section{Organizational Information}

The following details were provided for the assessment. This information is used to establish the context and scope of the review.

\begin{tabular}{@{}ll}
    \toprule
    \textbf{Attribute} & \textbf{Value} \\
    \midrule
    Organization Name & Stone Arch Masonry \\
    Email Domain & \texttt{StoneArchMasonry.org} \\
    Website Domain & \url{www.StoneArchMasonry.org} \\
    External IP Address & \seqsplit{\texttt{162.191.202.115}} \\
    \bottomrule
\end{tabular}

% --- Section 3: Security Control Review ---
\section{Security Control Review}

A review of the organization's security controls was conducted via a questionnaire. The responses highlight critical gaps in policy and procedure. "No" answers indicate a deviation from security best practices and are flagged as risks.

\begin{tabular}{@{}p{0.6\linewidth}p{0.2\linewidth}c}
    \toprule
    \textbf{Control Question} & \textbf{Best Practice} & \textbf{Status} \\
    \midrule
    Do you require MFA to access email? & Yes & \textcolor{green}{\cmark} \\
    Do you require MFA to log into computers? & Yes & \textcolor{red}{\xmark} \\
    Do you require MFA to access sensitive data systems? & Yes & \textcolor{green}{\cmark} \\
    Does your organization have an employee acceptable use policy? & Yes & \textcolor{red}{\xmark} \\
    Does your organization do security awareness training for new employees? & Yes & \textcolor{red}{\xmark} \\
    Does your organization do security awareness training for all employees at least once per year? & Yes & \textcolor{green}{\cmark} \\
    \bottomrule
\end{tabular}

% --- Section 4: Technical Scan Results ---
\section{Technical Scan Results}

An Nmap scan was performed on the target host \texttt{10.0.0.15}. The scan revealed a critical vulnerability that requires immediate remediation.

\subsection{Host Status}
The target host \texttt{10.0.0.15} was found to be online and responsive.

\subsection{Open Ports and Services}
The following open port was identified:

\begin{tabular}{@{}lllll}
    \toprule
    \textbf{Port} & \textbf{State} & \textbf{Service} & \textbf{Version} & \textbf{Notes} \\
    \midrule
    21/tcp & Open & ftp & vsftpd 2.3.4 & \textbf{Critical:} Anonymous login allowed. \\
    \bottomrule
\end{tabular}

\subsection{Vulnerability Analysis}
The FTP service running on port 21 presents a severe security risk:
\begin{itemize}
    \item \textbf{Vulnerable Version:} \texttt{vsftpd 2.3.4} is known to contain a critical backdoor vulnerability (CVE-2011-2523). An attacker can gain a command shell on the server by sending a specific string as the username, leading to a full system compromise.
    \item \textbf{Insecure Configuration:} The service is configured to allow anonymous FTP logins. This allows any user on the network to connect, list directories, and potentially upload or download files without authentication, which could be used to exfiltrate data or stage malware.
\end{itemize}

% --- Section 5: Consolidated Risk Assessment ---
\section{Consolidated Risk Assessment}

The following table synthesizes findings from the technical scan, control review, and pre-existing risk data into a prioritized list.

\begin{tabular}{@{}p{0.3\linewidth}p{0.5\linewidth}l}
    \toprule
    \textbf{Risk Name} & \textbf{Overview} & \textbf{Severity} \\
    \midrule
    \textbf{Vulnerable FTP Server} & A server is running vsftpd 2.3.4 with anonymous login enabled, which is vulnerable to remote code execution (CVE-2011-2523). & \textbf{Critical} \\
    \addlinespace
    \textbf{No MFA on Endpoints} & Employees can log into their computers using only a password, making them susceptible to stolen credential attacks. & High \\
    \addlinespace
    \textbf{No Acceptable Use Policy} & The absence of a formal AUP creates ambiguity regarding proper use of company assets and a lack of legal recourse for policy violations. & High \\
    \addlinespace
    \textbf{No New Hire Training} & New employees are not provided with security awareness training, leaving a critical window of vulnerability during their initial tenure. & High \\
    \addlinespace
    \textbf{Outdated Windows Policy} & Workstations are running Windows 7, an unsupported operating system that no longer receives security updates from Microsoft. & Medium \\
    \bottomrule
\end{tabular}

% --- Section 6: Recommendations ---
\section{Recommendations}

Based on the consolidated risk assessment, the following actions are recommended to improve the security posture of Stone Arch Masonry. Recommendations are prioritized by severity.

\begin{enumerate}
    \item \textbf{[Immediate] Remediate Vulnerable FTP Server:}
    \begin{itemize}
        \item Immediately take the FTP server at \texttt{10.0.0.15} offline.
        \item If the service is required, upgrade \texttt{vsftpd} to the latest stable version.
        \item If the service is not required, disable and uninstall it completely.
        \item In all cases, disable anonymous FTP login and enforce strong, authenticated access controls.
    \end{itemize}

    \item \textbf{[High Priority] Implement MFA for Computer Logins:}
    \begin{itemize}
        \item Deploy a Multi-Factor Authentication solution for all employee workstations and servers.
        \item This significantly reduces the risk of unauthorized access from compromised credentials.
    \end{itemize}
    
    \item \textbf{[High Priority] Develop and Implement an Acceptable Use Policy (AUP):}
    \begin{itemize}
        \item Create a formal AUP that clearly defines the rules for using company IT assets, data, and internet access.
        \item Require all employees to read and sign the policy as a condition of use.
    \end{itemize}
    
    \item \textbf{[High Priority] Integrate Security Training into Onboarding:}
    \begin{itemize}
        \item Develop a mandatory security awareness training module for all new hires.
        \item This module should cover key topics such as phishing, password security, and the new AUP.
    \end{itemize}
    
    \item \textbf{[Medium Priority] Accelerate Windows 7 Upgrade Plan:}
    \begin{itemize}
        \item Continue with existing plans to upgrade all Windows 7 workstations to a modern, supported operating system like Windows 10 or 11.
        \item Prioritize systems used by employees with access to sensitive data.
    \end{itemize}
\end{enumerate}

% --- Document End ---
\end{document}
```