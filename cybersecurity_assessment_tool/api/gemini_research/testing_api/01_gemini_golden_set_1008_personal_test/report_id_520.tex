```latex
\documentclass[12pt, a4paper]{article}

% Preamble: Required Packages
\usepackage[margin=1in]{geometry} % Set page margins
\usepackage{pifont}               % For checkmarks and crosses (\ding)
\usepackage{booktabs}             % For professional-looking tables
\usepackage{graphicx}             % For including logos, etc.
\usepackage[table]{xcolor}        % For coloring table rows
\usepackage{hyperref}             % For hyperlinks
\usepackage{url}                  % For formatting URLs
\usepackage{seqsplit}             % For splitting long strings in texttt

% Document Metadata and Hyperref Configuration
\hypersetup{
    colorlinks=true,
    linkcolor=black,
    filecolor=magenta,      
    urlcolor=blue,
    pdftitle={Cybersecurity Posture Assessment Report},
    pdfauthor={Cybersecurity Analyst AI},
    pdfsubject={Security Analysis},
    pdfkeywords={Security, Report, Analysis},
    bookmarks=true
}

% Define colors for table rows
\definecolor{tablehead}{gray}{0.9}
\definecolor{tablerow}{gray}{0.97}

% --- DOCUMENT START ---
\begin{document}

% Title Page
\begin{titlepage}
    \centering
    \vspace*{1cm}
    \Huge\textbf{Cybersecurity Posture Assessment Report}
    \vspace{1.5cm}
    \large
    \begin{tabular}{ll}
        \textbf{Organization:} & Cloud Nine Software \\
        \textbf{Report Date:} & \today \\
        \textbf{Analysis Period:} & \today \\
        \textbf{Classification:} & Confidential \\
    \end{tabular}
    \vfill
    \large
    \textit{This report contains a detailed analysis of the organization's security posture based on network scans, organizational data, and pre-existing risk assessments. The findings and recommendations are intended for management and technical teams to improve overall security.}
    \vspace{1cm}
\end{titlepage}

\tableofcontents
\newpage

% --- 1. Executive Summary ---
\section{Executive Summary}
This report provides a comprehensive cybersecurity assessment for \textbf{Cloud Nine Software}, synthesizing information from a technical network scan, a security controls questionnaire, and a review of known risks.

The overall security posture is generally strong, with excellent security awareness and policy controls in place. The technical network scan of the target host (\texttt{192.168.1.100}) revealed no open ports, indicating a robust firewall configuration and a minimal attack surface for that specific asset.

However, a critical gap was identified in the access control policies. The organization does not currently mandate Multi-Factor Authentication (MFA) for accessing sensitive data systems. This represents the most significant risk, as a single compromised credential could potentially lead to a major data breach.

Recommendations in this report focus on mitigating this critical risk by implementing MFA on sensitive systems and maintaining the current high standards of network security.

% --- 2. Organizational Information ---
\section{Organizational Information}
The following details were provided for the assessment. This information is used to establish the context and scope of the analysis.

\begin{tabular}{@{}ll}
    \toprule
    \textbf{Attribute} & \textbf{Value} \\
    \midrule
    Organization Name & \textbf{Cloud Nine Software} \\
    Email Domain & \texttt{CloudNineSoftware.org} \\
    Website Domain & \url{www.CloudNineSoftware.org} \\
    External IP Address & \texttt{15.202.240.36} \\
    \bottomrule
\end{tabular}

% --- 3. Security Control Review ---
\section{Security Control Review}
The following table details the responses from the security controls questionnaire. This review assesses the organization's policies and procedures against common cybersecurity best practices.

\rowcolors{2}{tablerow}{white}
\begin{tabular}{p{0.6\textwidth} c p{0.2\textwidth}}
    \toprule
    \rowcolor{tablehead}
    \textbf{Control Question} & \textbf{Response} & \textbf{Assessment} \\
    \midrule
    Do you require MFA to access email? & \ding{51} Yes & Strong Control \\
    Do you require MFA to log into computers? & \ding{51} Yes & Strong Control \\
    \textbf{Do you require MFA to access sensitive data systems?} & \textbf{\color{red}\ding{55} No} & \textbf{\color{red}Critical Gap} \\
    Does your organization have an employee acceptable use policy? & \ding{51} Yes & Good Practice \\
    Does your organization do security awareness training for new employees? & \ding{51} Yes & Good Practice \\
    Does your organization do security awareness training for all employees at least once per year? & \ding{51} Yes & Strong Control \\
    \bottomrule
\end{tabular}

\subsection*{Analysis}
The organization demonstrates a mature security program regarding user awareness, policy, and endpoint/email security. However, the absence of MFA for sensitive data systems is a major weakness. This gap significantly increases the risk of unauthorized access to the organization's most valuable information assets.

% --- 4. Technical Scan Results ---
\section{Technical Scan Results}
A network scan was performed to identify open ports and exposed services on the target system.

\begin{tabular}{@{}ll}
    \toprule
    \textbf{Scan Parameter} & \textbf{Value} \\
    \midrule
    Target IP Address & \texttt{192.168.1.100} \\
    Scan Date & \today \\
    Scanner Used & Nmap \\
    \bottomrule
\end{tabular}

\subsection*{Findings}
The scan results were positive, indicating a strong network security posture for the assessed host.
\begin{itemize}
    \item \textbf{Open Ports:} None detected.
    \item \textbf{Host Status:} Up.
    \item \textbf{Port State:} All 1000 scanned ports were reported as `closed`.
\end{itemize}
This result suggests that the host is protected by a well-configured firewall that denies unsolicited inbound traffic, effectively minimizing its external attack surface.

% --- 5. Consolidated Risk Assessment ---
\section{Consolidated Risk Assessment}
This section synthesizes findings from the security control review, technical scans, and pre-existing vulnerability data. Based on the analysis, one new high-severity risk has been identified.

\rowcolors{2}{tablerow}{white}
\begin{tabular}{p{0.1\textwidth} p{0.25\textwidth} p{0.45\textwidth} l}
    \toprule
    \rowcolor{tablehead}
    \textbf{Risk ID} & \textbf{Risk Name} & \textbf{Description} & \textbf{Severity} \\
    \midrule
    RISK-001 & Lack of MFA on Sensitive Data Systems & User accounts with access to sensitive data are protected only by a password. If an authorized user's credentials are stolen, an attacker can gain direct access to critical information assets without needing a second authentication factor. & \textbf{High} \\
    \bottomrule
\end{tabular}
\vspace{0.5cm}

\textit{Note: The pre-existing risk register (Input 3) contained no vulnerabilities.}

% --- 6. Recommendations ---
\section{Recommendations}
The following actions are recommended to address the identified risks and improve the overall security posture of \textbf{Cloud Nine Software}.

\begin{itemize}
    \item[\textbf{1.}] \textbf{Implement MFA for Sensitive Systems (High Priority)}
    \begin{itemize}
        \item \textbf{Action:} Mandate and enforce the use of Multi-Factor Authentication (MFA) for all accounts (including administrative, service, and user accounts) that can access systems designated as containing sensitive or critical data.
        \item \textbf{Justification:} This is the single most effective control to prevent unauthorized access resulting from compromised credentials. It adds a critical layer of defense, ensuring that a stolen password alone is not enough to cause a data breach.
    \end{itemize}
    \vspace{0.3cm}
    \item[\textbf{2.}] \textbf{Maintain Strong Network Security Posture (Informational)}
    \begin{itemize}
        \item \textbf{Action:} Continue to enforce and regularly audit the firewall policies that resulted in a clean network scan. Apply the principle of least privilege to all network access rules.
        \item \textbf{Justification:} The current network configuration for the scanned host is excellent. Maintaining this standard across the infrastructure is crucial for preventing external threats.
    \end{itemize}
\end{itemize}

\end{document}
```