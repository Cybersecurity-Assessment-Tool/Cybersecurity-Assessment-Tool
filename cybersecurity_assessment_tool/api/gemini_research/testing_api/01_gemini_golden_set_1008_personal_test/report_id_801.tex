```latex
\documentclass[12pt]{article}

% Preamble: Required Packages
\usepackage[margin=1in]{geometry}
\usepackage{pifont} % For checkmarks and crosses
\usepackage{booktabs} % For professional tables
\usepackage{hyperref} % For clickable links
\usepackage{url} % For formatting URLs
\usepackage{seqsplit} % For splitting long strings
\usepackage{xcolor} % For colors
\usepackage{graphicx} % For potential logos/images

% Hyperref Setup
\hypersetup{
    colorlinks=true,
    linkcolor=blue,
    filecolor=magenta,      
    urlcolor=cyan,
    pdftitle={Cybersecurity Assessment Report},
    pdfpagemode=FullScreen,
}

% Define custom colors for severity
\definecolor{criticalred}{HTML}{D7263D}
\definecolor{highorange}{HTML}{F49D42}
\definecolor{mediumyellow}{HTML}{F4D442}
\definecolor{lowblue}{HTML}{4287F4}
\definecolor{infogray}{HTML}{808080}

% Document Start
\begin{document}

% --- Title Page ---
\begin{titlepage}
    \centering
    \vspace*{1cm}
    \Huge\textbf{Cybersecurity Assessment Report}
    \vspace{1.5cm}
    \large
    \begin{tabular}{ll}
        \textbf{Organization:} & Borealis Tech \\
        \textbf{Date of Report:} & \today \\
        \textbf{Author:} & Cybersecurity Analyst \\
    \end{tabular}
    \vfill
    \textit{This report contains sensitive information and should be handled with the utmost confidentiality. Distribution is restricted to authorized personnel only.}
\end{titlepage}

\tableofcontents
\newpage

% --- 1. Executive Summary ---
\section*{1. Executive Summary}
This report details the findings of a cybersecurity assessment conducted for Borealis Tech. The analysis combines a technical network scan, a review of organizational security controls, and a correlation with pre-existing risk data.

The assessment identified several critical and high-risk vulnerabilities that require immediate attention. The most severe finding is a potentially exposed database interface on the internal network, labeled \textbf{"TOP SECRET DB"}, which directly contradicts a pre-existing risk assessment that marked the port as secure.

Furthermore, significant gaps were identified in the organization's access control policies, specifically the lack of Multi-Factor Authentication (MFA) for computer and sensitive data system access. The absence of mandatory annual security awareness training for all employees presents another high-risk gap, increasing the likelihood of human error leading to a security incident.

Immediate remediation of the exposed service and implementation of robust MFA controls are the highest priorities.

% --- 2. Organizational Information ---
\section*{2. Organizational Information}
The following information was provided for the assessment.

\begin{tabular}{@{}ll}
    \toprule
    \textbf{Attribute} & \textbf{Value} \\
    \midrule
    Organization Name & \textbf{Borealis Tech} \\
    Email Domain & \texttt{BorealisTech.com} \\
    Website Domain & \url{www.BorealisTech.com} \\
    External IP Address & \texttt{102.39.156.82} \\
    \bottomrule
\end{tabular}

% --- 3. Security Control Review ---
\section*{3. Security Control Review}
A review of administrative security controls was conducted based on a standardized questionnaire. The results highlight critical gaps in the organization's identity and access management framework. A "No" response indicates a missing control and a significant increase in risk.

\begin{table}[h!]
\centering
\begin{tabular}{@{}p{8cm}ccp{3.5cm}@{}}
    \toprule
    \textbf{Control Question} & \textbf{Response} & \textbf{Status} & \textbf{Assessment} \\
    \midrule
    Do you require MFA to access email? & Yes & \ding{51} & Control in Place \\
    Do you require MFA to log into computers? & No & \ding{55} & \textcolor{criticalred}{\textbf{Critical Gap}} \\
    Do you require MFA to access sensitive data systems? & No & \ding{55} & \textcolor{criticalred}{\textbf{Critical Gap}} \\
    Does your organization have an employee acceptable use policy? & Yes & \ding{51} & Control in Place \\
    Does your organization do security awareness training for new employees? & Yes & \ding{51} & Control in Place \\
    Does your organization do security awareness training for all employees at least once per year? & No & \ding{55} & \textcolor{highorange}{\textbf{High Risk}} \\
    \bottomrule
\end{tabular}
\caption{Organizational Security Control Status}
\end{table}

% --- 4. Technical Scan Results ---
\section*{4. Technical Scan Results}
An Nmap scan was performed to identify open ports and exposed services on the specified target system.

\begin{itemize}
    \item \textbf{Target IP:} \texttt{10.5.5.5}
    \item \textbf{Scan Date:} \today
\end{itemize}

The scan revealed the following open port:

\begin{table}[h!]
\centering
\begin{tabular}{@{}llll@{}}
    \toprule
    \textbf{Port} & \textbf{State} & \textbf{Service} & \textbf{Banner / Details} \\
    \midrule
    8080/tcp & OPEN & http & \textbf{http-title: TOP SECRET DB} \\
    \bottomrule
\end{tabular}
\caption{Open Port Findings for \texttt{10.5.5.5}}
\end{table}

\subsection*{Analysis of Technical Findings}
The finding on port 8080 is of \textbf{critical concern}. The HTTP title "TOP SECRET DB" strongly suggests that a sensitive, possibly production, database management interface is exposed on the network. This service could be an administrative panel or a direct data interface. Such a revealing title constitutes a significant information disclosure vulnerability. 

\textbf{Crucially, this live scan data invalidates the pre-existing risk entry (from Input 3) which incorrectly labeled this port as "confirmed secure and false positive."} The service is active, exposed, and improperly configured.

% --- 5. Correlated Risk Assessment ---
\section*{5. Correlated Risk Assessment}
This section synthesizes findings from the security control review and the technical scan to provide a holistic view of the primary risks facing the organization.

\begin{table}[h!]
\centering
\begin{tabular}{@{}p{2.5cm}p{8cm}l@{}}
    \toprule
    \textbf{Risk ID} & \textbf{Description} & \textbf{Severity} \\
    \midrule
    RISK-001 & An exposed web service on \texttt{10.5.5.5:8080} is titled "TOP SECRET DB". This, combined with the lack of MFA on sensitive systems, creates a high probability of unauthorized access to critical data. & \textcolor{criticalred}{\textbf{Critical}} \\
    \addlinespace
    RISK-002 & Lack of MFA for computer and sensitive data system logins significantly weakens access controls, making the organization highly vulnerable to credential theft and lateral movement attacks. & \textcolor{criticalred}{\textbf{Critical}} \\
    \addlinespace
    RISK-003 & The absence of mandatory annual security awareness training for all staff increases the risk of successful phishing, social engineering, and other attacks that rely on human error. & \textcolor{highorange}{\textbf{High}} \\
    \bottomrule
\end{tabular}
\caption{Summary of Identified Risks}
\end{table}

% --- 6. Recommendations ---
\section*{6. Recommendations}
The following prioritized recommendations are provided to mitigate the identified risks.

\subsection*{Priority 1: Immediate Actions}
\begin{enumerate}
    \item \textbf{Secure Exposed Database Interface (RISK-001):}
    \begin{itemize}
        \item Immediately investigate the service running on \texttt{10.5.5.5:8080}.
        \item If the service is not essential, disable it.
        \item If essential, apply a firewall rule to restrict access to only authorized personnel and systems.
        \item Implement strong authentication and encryption (HTTPS) for the interface.
    \end{itemize}
    
    \item \textbf{Implement Multi-Factor Authentication (RISK-002):}
    \begin{itemize}
        \item Enable MFA for all user accounts for logging into company computers (endpoints).
        \item Enforce MFA for access to all systems classified as containing sensitive or critical data, including the system identified on port 8080.
    \end{itemize}
\end{enumerate}

\subsection*{Priority 2: Strategic Actions}
\begin{enumerate}
    \setcounter{enumi}{2}
    \item \textbf{Establish Annual Security Training (RISK-003):}
    \begin{itemize}
        \item Develop or procure a mandatory security awareness training program for all employees.
        \item Conduct this training annually and track completion to ensure 100\% compliance.
        \item Include modules on phishing, password security, and acceptable use.
    \end{itemize}
\end{enumerate}

\end{document}
```