```latex
\documentclass[12pt]{article}

% 1. DOCUMENT SETUP & PACKAGES
\usepackage[margin=1in]{geometry}
\usepackage{pifont} % For checkmarks and crosses
\usepackage{booktabs} % For professional tables
\usepackage{xcolor} % For colors
\usepackage{graphicx}
\usepackage{hyperref} % For hyperlinks
\usepackage{url} % For URL formatting
\usepackage{seqsplit} % To split long strings in tt font

% Hyperref setup
\hypersetup{
    colorlinks=true,
    linkcolor=blue,
    filecolor=magenta,      
    urlcolor=cyan,
    pdftitle={Cybersecurity Assessment Report},
    pdfpagemode=FullScreen,
}

% Define custom colors for severity
\definecolor{sev_critical}{HTML}{990000}
\definecolor{sev_high}{HTML}{DD4B39}
\definecolor{sev_medium}{HTML}{F4B400}

% 2. DOCUMENT METADATA
\title{Cybersecurity Assessment Report \\ \large For: \textbf{Mainframe Managed}}
\author{Cybersecurity Analysis Division}
\date{\today}

\begin{document}

\maketitle
\thispagestyle{empty}
\newpage

\tableofcontents
\newpage

% 3. EXECUTIVE SUMMARY
\section{Executive Summary}
This report details the findings of a cybersecurity assessment conducted for \textbf{Mainframe Managed}. The assessment combined an analysis of organizational security controls, a technical network scan, and a review of pre-existing risks to provide a holistic view of the organization's security posture.

The analysis revealed several critical and high-risk gaps that require immediate attention. Key findings include:
\begin{itemize}
    \item \textbf{Critical Control Gaps:} The lack of Multi-Factor Authentication (MFA) on email systems represents a significant exposure to account compromise and Business Email Compromise (BEC) attacks. Furthermore, the complete absence of a security awareness training program and an acceptable use policy leaves the organization vulnerable to human error and insider threats.
    \item \textbf{Critical Technical Vulnerability:} A pre-existing critical risk, "Localhost Exposed" (CVSS 10.0), was correlated with a technical finding of an open SSH port (22/TCP) on the scanned host (\texttt{127.0.0.1}). This indicates a severe misconfiguration that could be exploited for unauthorized access.
\end{itemize}

The overall security posture is considered high-risk. The recommendations provided in this report are prioritized to address the most critical vulnerabilities first, with the goal of tangibly improving the organization's resilience against common cyber threats.

% 4. ORGANIZATIONAL INFORMATION
\section{Organizational Information}
The following details were provided for the assessment. This information is used to establish the context and scope of the review.

\begin{tabular}{@{}ll}
    \toprule
    \textbf{Attribute} & \textbf{Value} \\
    \midrule
    Organization Name & \textbf{Mainframe Managed} \\
    Email Domain & \seqsplit{\texttt{MainframeManaged.org}} \\
    Website Domain & \seqsplit{\texttt{www.MainframeManaged.org}} \\
    External IP Address & \seqsplit{\texttt{239.84.60.106}} \\
    \bottomrule
\end{tabular}

% 5. SECURITY CONTROL REVIEW (QUESTIONNAIRE)
\section{Security Control Review}
An analysis of the organization's administrative and procedural security controls was performed based on a supplied questionnaire. The findings highlight significant gaps in foundational security practices.

\begin{tabular}{@{}p{0.6\linewidth}ccp{0.2\linewidth}@{}}
    \toprule
    \textbf{Control Question} & \multicolumn{2}{c}{\textbf{Status}} & \textbf{Assessment} \\
    \midrule
    Do you require MFA to access email? & \textcolor{red}{\ding{55}} & (No) & \textbf{Critical Gap.} Increases risk of account compromise. \\
    Do you require MFA to log into computers? & \textcolor{green}{\ding{51}} & (Yes) & Meets best practice. \\
    Do you require MFA to access sensitive data systems? & \textcolor{green}{\ding{51}} & (Yes) & Meets best practice. \\
    Does your organization have an employee acceptable use policy? & \textcolor{red}{\ding{55}} & (No) & \textbf{High Risk.} Lack of formal policy creates legal and operational risks. \\
    Does your organization do security awareness training for new employees? & \textcolor{red}{\ding{55}} & (No) & \textbf{High Risk.} New staff are not trained on security threats. \\
    Does your organization do security awareness training for all employees at least once per year? & \textcolor{red}{\ding{55}} & (No) & \textbf{High Risk.} Exacerbates susceptibility to phishing and social engineering. \\
    \bottomrule
\end{tabular}

% 6. TECHNICAL SCAN RESULTS
\section{Technical Scan Results}
A network scan was conducted to identify open ports and services on the target system. The results confirm a pre-existing identified risk.

\begin{itemize}
    \item \textbf{Target IP:} \texttt{127.0.0.1}
    \item \textbf{Scan Date:} \today
\end{itemize}

\begin{tabular}{@{}lllll@{}}
    \toprule
    \textbf{Port} & \textbf{Protocol} & \textbf{State} & \textbf{Service (Inferred)} & \textbf{Notes} \\
    \midrule
    22 & TCP & Open & SSH & Service version was not identified. This finding \\
    & & & & directly correlates with the "Localhost Exposed" risk. \\
    \bottomrule
\end{tabular}

% 7. CONSOLIDATED RISK ASSESSMENT
\section{Consolidated Risk Assessment}
This section synthesizes findings from the security control review, technical scan, and pre-existing risk data into a unified list of prioritized risks.

\begin{tabular}{@{}p{0.2\linewidth}p{0.15\linewidth}p{0.4\linewidth}p{0.2\linewidth}@{}}
    \toprule
    \textbf{Risk Name} & \textbf{Severity} & \textbf{Description} & \textbf{Affected Asset(s)} \\
    \midrule
    \textbf{Localhost Exposed} & \textcolor{sev_critical}{\textbf{Critical (10.0)}} & A service is exposed on the localhost interface, confirmed by an open SSH port (22/TCP). This could indicate a backdoor or severe misconfiguration. & \seqsplit{\texttt{127.0.0.1}} \\
    \addlinespace
    \textbf{No MFA for Email Access} & \textcolor{sev_critical}{\textbf{Critical}} & Lack of Multi-Factor Authentication for email significantly increases the risk of Business Email Compromise (BEC) and unauthorized data access. & \seqsplit{\texttt{MainframeManaged.org}} \\
    \addlinespace
    \textbf{Lack of Security Awareness Training} & \textcolor{sev_high}{\textbf{High}} & Employees are not trained on security best practices, making them highly susceptible to phishing and social engineering attacks. & All Employees \\
    \addlinespace
    \textbf{No Acceptable Use Policy (AUP)} & \textcolor{sev_high}{\textbf{High}} & Absence of a formal AUP creates ambiguity regarding the acceptable use of company assets, posing legal and operational risks. & Organizational Policy \\
    \bottomrule
\end{tabular}

% 8. RECOMMENDATIONS
\section{Recommendations}
The following actionable recommendations are provided to mitigate the identified risks. They are prioritized based on severity and potential impact.

\subsection{Immediate Priority (Critical Risks)}
\begin{enumerate}
    \item \textbf{Remediate "Localhost Exposed" Vulnerability:}
    \begin{itemize}
        \item Immediately investigate the purpose of the SSH service running on \texttt{127.0.0.1:22}.
        \item If the service is not essential for business operations, it must be disabled and removed.
        \item If the service is required, ensure it is properly configured, firewalled, and bound only to the necessary interface. Access should be restricted to authorized personnel only.
    \end{itemize}
    
    \item \textbf{Implement MFA for Email:}
    \begin{itemize}
        \item Enforce MFA across the entire email system (\seqsplit{\texttt{MainframeManaged.org}}) for all users without exception.
        \item This is the single most effective control to prevent unauthorized account access and should be treated as the highest administrative priority.
    \end{itemize}
\end{enumerate}

\subsection{Near-Term Priority (High Risks)}
\begin{enumerate}
    \setcounter{enumi}{2} % Continue numbering
    \item \textbf{Develop and Implement a Security Awareness Program:}
    \begin{itemize}
        \item Procure or develop a security awareness training module that covers, at a minimum, phishing, password hygiene, and data handling.
        \item Mandate completion of this training for all new hires during their onboarding process.
        \item Require all existing employees to complete the training annually.
    \end{itemize}

    \item \textbf{Establish and Enforce an Acceptable Use Policy (AUP):}
    \begin{itemize}
        \item Draft a formal AUP that clearly defines the rules and responsibilities for employees using company IT assets, network resources, and data.
        \item Have the policy reviewed by legal counsel and senior management.
        \item Communicate the policy to all employees and require them to formally acknowledge it.
    \end{itemize}
\end{enumerate}

\end{document}
```