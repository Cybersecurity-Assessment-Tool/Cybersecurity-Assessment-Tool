```latex
\documentclass[12pt]{article}

% Required Packages
\usepackage[margin=1in]{geometry}
\usepackage{pifont} % For checkmarks and crosses
\usepackage{booktabs} % For professional tables
\usepackage{hyperref} % For clickable links
\usepackage{url} % For URL formatting
\usepackage{seqsplit} % To split long strings in tt font
\usepackage{xcolor} % For colors

% Document Information
\title{Cybersecurity Assessment Report \\ \large Prepared for: \textbf{Maple Leaf Logistics}}
\author{Cybersecurity Analyst}
\date{\today}

% Hyperref Setup
\hypersetup{
    colorlinks=true,
    linkcolor=blue,
    filecolor=magenta,      
    urlcolor=cyan,
    pdftitle={Cybersecurity Assessment Report},
    pdfpagemode=FullScreen,
}

\begin{document}

\maketitle
\thispagestyle{empty}
\newpage

\tableofcontents
\newpage

% --- Section 1: Executive Overview ---
\section{Executive Overview}

This report provides a comprehensive cybersecurity assessment for \textbf{Maple Leaf Logistics}, conducted on \today. The analysis synthesizes data from a network vulnerability scan, a review of organizational security controls, and pre-existing risk information.

The assessment identified several critical and high-risk security gaps that require immediate attention. The most significant findings are related to the lack of Multi-Factor Authentication (MFA) on critical systems, including email and sensitive data repositories. These gaps expose the organization to a high risk of account compromise, data breaches, and subsequent business disruption. Additionally, the absence of a formal Acceptable Use Policy (AUP) represents a significant governance and compliance risk.

On a positive note, the technical network scan of the target host \seqsplit{\texttt{192.168.0.5}} did not reveal any open ports or immediate vulnerabilities. This finding contradicts a pre-existing risk item concerning an unencrypted web server, suggesting that the risk may be outdated or associated with a different asset.

This report outlines these findings in detail and provides a prioritized list of actionable recommendations to mitigate the identified risks and strengthen the overall security posture of \textbf{Maple Leaf Logistics}.

% --- Section 2: Organizational Information ---
\section{Organizational Information}

The following details were provided for the assessment and used as the basis for this report.

\begin{tabular}{@{}ll}
\toprule
\textbf{Attribute} & \textbf{Value} \\
\midrule
Organization Name & \textbf{Maple Leaf Logistics} \\
Email Domain & \seqsplit{\texttt{MapleLeafLogistics.net}} \\
Website Domain & \seqsplit{\url{www.MapleLeafLogistics.net}} \\
External IP Address & \seqsplit{\texttt{69.169.167.225}} \\
\bottomrule
\end{tabular}

% --- Section 3: Security Control Review ---
\section{Security Control Review}

A review of administrative and technical security controls was conducted based on a questionnaire. The responses reveal critical gaps in the organization's identity and access management framework. The results are summarized below.

\begin{table}[h!]
\centering
\caption{Security Controls Questionnaire Analysis}
\begin{tabular}{@{}p{0.6\linewidth} c p{0.2\linewidth}@{}}
\toprule
\textbf{Control Question} & \textbf{Response} & \textbf{Assessment} \\
\midrule
Do you require MFA to access email? & \textcolor{red}{\ding{55}} & Critical Gap \\
Do you require MFA to log into computers? & \textcolor{green}{\ding{51}} & Best Practice Met \\
Do you require MFA to access sensitive data systems? & \textcolor{red}{\ding{55}} & Critical Gap \\
Does your organization have an employee acceptable use policy? & \textcolor{red}{\ding{55}} & High Risk \\
Does your organization do security awareness training for new employees? & \textcolor{green}{\ding{51}} & Best Practice Met \\
Does your organization do security awareness training for all employees at least once per year? & \textcolor{green}{\ding{51}} & Best Practice Met \\
\bottomrule
\end{tabular}
\end{table}

% --- Section 4: Technical Scan Results ---
\section{Technical Scan Results}

A network scan was performed to identify open ports and services on the specified target system.

\begin{itemize}
    \item \textbf{Target IP Address:} \seqsplit{\texttt{192.168.0.5}}
    \item \textbf{Scan Date:} \today
\end{itemize}

The scan results indicate a secure configuration for the target host, with no open ports detected. This is a positive finding.

\begin{table}[h!]
\centering
\caption{Nmap Scan Results for \seqsplit{\texttt{192.168.0.5}}}
\begin{tabular}{@{}llll@{}}
\toprule
\textbf{Port} & \textbf{State} & \textbf{Service} & \textbf{Product / Version} \\
\midrule
80/tcp & Closed & http & N/A \\
\bottomrule
\end{tabular}
\end{table}

\subsection*{Analysis of Technical Findings}
The scan confirmed that port 80 (HTTP) was closed on the target system. This finding directly contradicts the pre-existing risk titled "Unencrypted Web Server," which stated that port 80 was open. This suggests the pre-existing risk may be outdated, resolved, or associated with a different network asset. No immediate technical vulnerabilities were identified on this specific host.

% --- Section 5: Consolidated Risk Assessment ---
\section{Consolidated Risk Assessment}

This section correlates findings from the security control review, technical scan, and pre-existing risk data. The primary risks identified are administrative and procedural, stemming from the security control gaps.

\begin{table}[h!]
\centering
\caption{Summary of Identified Risks}
\begin{tabular}{@{}p{0.25\linewidth} p{0.5\linewidth} p{0.15\linewidth}@{}}
\toprule
\textbf{Risk Name} & \textbf{Description} & \textbf{Severity} \\
\midrule
\textbf{No MFA on Email} & The lack of MFA on email accounts exposes the organization to significant risk of business email compromise (BEC), phishing, and account takeover attacks. & \textbf{Critical} \\
\addlinespace
\textbf{No MFA on Sensitive Data Systems} & Sensitive corporate and client data is protected only by username and password, making it highly vulnerable to unauthorized access if credentials are compromised. & \textbf{Critical} \\
\addlinespace
\textbf{Missing Acceptable Use Policy (AUP)} & Without a formal AUP, there is no clear guidance for employees on the proper use of company assets, increasing the risk of insider threats and non-compliance. & \textbf{High} \\
\addlinespace
\textbf{Unencrypted Web Server (Not Validated)} & A pre-existing risk noted an open port 80. The current scan of \seqsplit{\texttt{192.168.0.5}} found this port to be closed. This risk should be re-evaluated. & Informational \\
\bottomrule
\end{tabular}
\end{table}

% --- Section 6: Recommendations ---
\section{Recommendations}

The following prioritized recommendations are provided to address the identified risks and improve the overall security posture of \textbf{Maple Leaf Logistics}.

\subsection*{Priority 1: Remediate Critical Risks}
\begin{enumerate}
    \item \textbf{Implement MFA for Email Access:} Immediately enforce MFA for all user mailboxes. This is the single most effective control to prevent business email compromise. Solutions like Microsoft Authenticator, Google Authenticator, or hardware tokens should be deployed.
    \item \textbf{Enforce MFA for Sensitive Systems:} Identify all systems containing sensitive or critical data and enforce MFA for all user access, especially for administrative accounts.
\end{enumerate}

\subsection*{Priority 2: Remediate High Risks}
\begin{enumerate}
    \setcounter{enumi}{2}
    \item \textbf{Develop and Implement an Acceptable Use Policy (AUP):} Create a formal AUP that clearly defines the rules and expectations for employees using company networks, systems, and data. This policy should be integrated into the new employee onboarding process and reviewed annually by all staff.
\end{enumerate}

\subsection*{Priority 3: Administrative and Follow-up Actions}
\begin{enumerate}
    \setcounter{enumi}{3}
    \item \textbf{Review and Update the Risk Register:} The pre-existing risk "Unencrypted Web Server" was not validated by this assessment's technical scan. A full review of the risk register should be conducted to ensure all entries are accurate, current, and relevant. This may require a comprehensive asset inventory and vulnerability scan across the entire network.
\end{enumerate}

\end{document}
```