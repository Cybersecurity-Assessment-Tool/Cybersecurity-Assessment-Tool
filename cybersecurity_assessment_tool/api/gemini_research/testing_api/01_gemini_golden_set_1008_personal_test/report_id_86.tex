```latex
\documentclass[12pt]{article}

% Preamble: Required Packages
\usepackage[margin=1in]{geometry}
\usepackage{pifont} % For checkmarks and crosses
\usepackage{booktabs} % For professional tables
\usepackage{hyperref} % For clickable links
\usepackage{url} % For formatting URLs
\usepackage{seqsplit} % For splitting long strings without spaces
\usepackage{xcolor} % For colors
\usepackage{graphicx} % For potential logos/images
\usepackage{datetime} % For report date

% --- Document Metadata ---
\title{Cybersecurity Risk Assessment Report}
\author{Cybersecurity Analysis Division}
\date{\today}

% --- Hyperref Setup ---
\hypersetup{
    colorlinks=true,
    linkcolor=blue,
    filecolor=magenta,      
    urlcolor=cyan,
    pdftitle={Cybersecurity Risk Assessment Report},
    pdfpagemode=FullScreen,
}

\begin{document}

\maketitle
\thispagestyle{empty}
\newpage

\tableofcontents
\newpage

% --- Section 1: Executive Summary ---
\section{Executive Summary}

This report provides a comprehensive cybersecurity risk assessment for \textbf{Digital Drift}, based on an analysis of network scan data, organizational security controls, and pre-existing risk documentation. The assessment was conducted to identify vulnerabilities, evaluate the current security posture, and provide actionable recommendations for risk mitigation.

The analysis revealed several high-impact risks that require immediate attention. Key findings include:
\begin{itemize}
    \item \textbf{Critical Network Vulnerability:} A highly vulnerable FTP server (\texttt{vsftpd 2.3.4}) was identified on the internal network. This version contains a known, publicly documented backdoor allowing for remote code execution. The server is also misconfigured to allow anonymous access.
    \item \textbf{Insufficient Access Controls:} Multi-Factor Authentication (MFA) is not enforced for employee email or computer logins, creating a significant risk of account compromise via stolen credentials.
    \item \textbf{Gaps in Security Training:} New employees do not receive security awareness training, making them particularly susceptible to phishing and social engineering attacks.
    \item \textbf{Outdated Operating Systems:} The organization continues to use Windows 7, an end-of-life operating system that no longer receives security updates, on its workstations.
\end{itemize}

The confluence of these issues indicates a fragile security posture. A single compromised credential could lead to a significant breach due to the lack of MFA and the presence of vulnerable internal services. We strongly recommend prioritizing the remediation steps outlined in Section 6 of this report.

% --- Section 2: Organizational Information ---
\section{Organizational Information}

The following information was provided for the assessment.

\begin{tabular}{@{}ll}
    \toprule
    \textbf{Attribute} & \textbf{Value} \\
    \midrule
    Organization Name & \textbf{Digital Drift} \\
    Primary Email Domain & \texttt{DigitalDrift.org} \\
    Primary Website & \url{www.DigitalDrift.org} \\
    Known External IP & \texttt{134.64.199.60} \\
    \bottomrule
\end{tabular}

% --- Section 3: Security Control Review ---
\section{Security Control Review}

A review of the organization's security controls was conducted via a questionnaire. The responses highlight critical gaps in identity and access management and employee security training. A checkmark (\ding{51}) indicates an affirmative response (control in place), while a cross (\ding{55}) indicates a negative response (control gap).

\begin{table}[h!]
\centering
\begin{tabular}{@{}p{0.6\textwidth}cc@{}}
\toprule
\textbf{Control Question} & \textbf{Response} & \textbf{Assessment} \\
\midrule
Does your organization require MFA to access sensitive data systems? & \ding{51} & Good \\
Does your organization have an employee acceptable use policy? & \ding{51} & Good \\
Does your organization do security awareness training for all employees at least once per year? & \ding{51} & Good \\
\midrule
\textcolor{red}{Do you require MFA to access email?} & \textcolor{red}{\ding{55}} & \textcolor{red}{High Risk} \\
\textcolor{red}{Do you require MFA to log into computers?} & \textcolor{red}{\ding{55}} & \textcolor{red}{High Risk} \\
\textcolor{red}{Does your organization do security awareness training for new employees?} & \textcolor{red}{\ding{55}} & \textcolor{red}{Critical Gap} \\
\bottomrule
\end{tabular}
\caption{Security Control Questionnaire Analysis}
\end{table}

% --- Section 4: Technical Scan Results ---
\section{Technical Scan Results}

A network scan was performed on the target host \texttt{10.0.0.15} to identify open ports and exposed services. The scan revealed a critical vulnerability.

\begin{table}[h!]
\centering
\begin{tabular}{@{}llll@{}}
\toprule
\textbf{Port} & \textbf{Service} & \textbf{Version} & \textbf{Finding} \\
\midrule
21/tcp & FTP & vsftpd 2.3.4 & \parbox[t]{0.5\textwidth}{
    \textbf{Anonymous FTP login is allowed.} This version is critically outdated (c. 2011) and contains a well-known backdoor vulnerability (\textbf{CVE-2011-2523}) that allows for remote command execution.
} \\
\bottomrule
\end{tabular}
\caption{Open Ports and Services on \texttt{10.0.0.15}}
\end{table}

The presence of this specific version of vsftpd, combined with the anonymous login configuration, represents a severe and immediate threat to the internal network. An attacker could easily gain a foothold on this server and use it as a pivot point to attack other systems.

% --- Section 5: Consolidated Risk Assessment ---
\section{Consolidated Risk Assessment}

This section synthesizes findings from the technical scan, control review, and existing risk documentation into a prioritized list.

\begin{table}[h!]
\centering
\begin{tabular}{@{}p{0.25\textwidth}p{0.5\textwidth}l@{}}
\toprule
\textbf{Risk Title} & \textbf{Description} & \textbf{Severity} \\
\midrule
\textbf{Vulnerable FTP Server} & An outdated vsftpd server (v2.3.4) with anonymous login is exposed internally. This version has a known remote code execution backdoor. & \textbf{\textcolor{red}{Critical}} \\
\addlinespace
\textbf{Lack of MFA for Core Systems} & MFA is not enforced for email or computer logins, exposing the organization to account takeover attacks from a single compromised password. & \textbf{\textcolor{orange}{High}} \\
\addlinespace
\textbf{Inadequate Employee Onboarding} & New employees do not receive security awareness training, making them prime targets for social engineering and phishing attacks. & \textbf{\textcolor{orange}{High}} \\
\addlinespace
\textbf{End-of-Life Operating Systems} & Workstations are running Windows 7, which no longer receives security updates, leaving them vulnerable to a wide range of known exploits. & \textbf{\textcolor{yellow!80!black}{Medium}} \\
\bottomrule
\end{tabular}
\caption{Summary of Identified Risks}
\end{table}

% --- Section 6: Recommendations ---
\section{Recommendations}

Based on the consolidated risk assessment, the following actions are recommended to improve the security posture of \textbf{Digital Drift}. Recommendations are prioritized by severity.

\begin{enumerate}
    \item \textbf{Remediate Vulnerable FTP Server (Immediate Priority)}
    \begin{itemize}
        \item Immediately take the server at \texttt{10.0.0.15} offline.
        \item If the FTP service is business-critical, migrate the functionality to a modern, secure file transfer solution (e.g., SFTP) on a fully patched server.
        \item If FTP must be used, upgrade to the latest stable version of the software and disable anonymous access. Access should be restricted by a firewall to only trusted sources.
    \end{itemize}
    
    \item \textbf{Implement Multi-Factor Authentication (High Priority)}
    \begin{itemize}
        \item Procure and deploy an MFA solution for all user accounts.
        \item Enforce MFA for all access to email (e.g., Office 365, G Suite) and VPN services.
        \item Enforce MFA for all computer logins, especially for administrative accounts.
    \end{itemize}
    
    \item \textbf{Enhance Security Awareness Program (High Priority)}
    \begin{itemize}
        \item Develop and mandate a security awareness training module for all new employees as a required part of the onboarding process.
        \item This training should cover phishing, password hygiene, acceptable use, and incident reporting.
    \end{itemize}
    
    \item \textbf{Upgrade End-of-Life Systems (Medium Priority)}
    \begin{itemize}
        \item Prioritize and execute the plan to upgrade all remaining Windows 7 workstations to a supported operating system (e.g., Windows 10/11).
        \item Implement a patch management policy to ensure all operating systems and software are kept up to date.
    \end{itemize}
\end{enumerate}

\end{document}
```