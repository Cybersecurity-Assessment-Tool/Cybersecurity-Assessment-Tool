```latex
\documentclass[12pt]{article}

% --- PACKAGES ---
\usepackage[margin=1in]{geometry} % Set page margins
\usepackage{pifont}               % For checkmark and x-mark symbols (\ding)
\usepackage{booktabs}             % For professional-looking tables
\usepackage{hyperref}             % For hyperlinks and document metadata
\usepackage{url}                  % For formatting URLs
\usepackage{seqsplit}             % For splitting long strings in texttt
\usepackage[utf8]{inputenc}       % For UTF-8 input encoding

% --- DOCUMENT METADATA ---
\hypersetup{
    colorlinks=true,
    linkcolor=black,
    urlcolor=blue,
    pdftitle={Cybersecurity Posture Assessment Report},
    pdfauthor={Cybersecurity Analyst},
    pdfsubject={Security Assessment}
}

\title{Cybersecurity Posture Assessment Report \\ \large For: \textbf{Sovereign Trust}}
\author{Cybersecurity Analyst}
\date{\today}

% --- DOCUMENT START ---
\begin{document}

\maketitle
\tableofcontents
\newpage

% ==============================================================================
% SECTION 1: EXECUTIVE OVERVIEW
% ==============================================================================
\section{Executive Overview}

This report details the findings of a cybersecurity posture assessment for \textbf{Sovereign Trust}. The analysis is based on a combination of organizational data, security control questionnaires, and technical network scans.

The overall security posture reveals a solid foundation in employee security awareness and endpoint/email access controls. The organization has successfully implemented Multi-Factor Authentication (MFA) for email and computer access, and maintains policies and training programs that align with industry best practices.

However, a critical security gap was identified: the absence of mandatory MFA for accessing sensitive data systems. This represents a significant risk, as it exposes the organization's most valuable data to potential compromise via stolen or weak credentials.

Furthermore, the technical network scan data and the list of current organizational risks were provided in a corrupted format. This prevented a complete technical vulnerability analysis of the external perimeter and a review of known, pre-existing risks.

Key recommendations focus on immediately addressing the MFA gap for sensitive systems, re-initiating the technical network scan, and providing a valid list of current risks for a more comprehensive future assessment.

% ==============================================================================
% SECTION 2: ORGANIZATIONAL INFORMATION
% ==============================================================================
\section{Organizational Information}

The following details were provided by the client and used as the basis for this assessment.

\begin{itemize}
    \item \textbf{Organization Name:} Sovereign Trust
    \item \textbf{Email Domain:} \texttt{SovereignTrust.org}
    \item \textbf{Website Domain:} \url{www.SovereignTrust.org}
    \item \textbf{Primary External IP:} \texttt{154.1.235.88}
\end{itemize}


% ==============================================================================
% SECTION 3: SECURITY CONTROL REVIEW
% ==============================================================================
\section{Security Control Review}

A review of the organization's security controls was conducted via a questionnaire. The responses are summarized below. A checkmark (\ding{51}) indicates an affirmative response (control in place), while an X (\ding{55}) indicates a negative response, highlighting a potential security gap.

\begin{table}[h!]
\centering
\caption{Security Control Questionnaire Responses}
\begin{tabular}{p{0.75\linewidth} c}
\toprule
\textbf{Control Question} & \textbf{Response} \\
\midrule
Do you require MFA to access email? & \ding{51} \\
Do you require MFA to log into computers? & \ding{51} \\
\textbf{Do you require MFA to access sensitive data systems?} & \textbf{\ding{55}} \\
Does your organization have an employee acceptable use policy? & \ding{51} \\
Does your organization do security awareness training for new employees? & \ding{51} \\
Does your organization do security awareness training for all employees at least once per year? & \ding{51} \\
\bottomrule
\end{tabular}
\end{table}

\subsection*{Analysis}
The questionnaire results are largely positive, indicating a mature approach to endpoint security and user awareness. However, the lack of MFA for sensitive data systems is a critical oversight. This gap could allow an attacker with compromised credentials to directly access and exfiltrate the organization's most critical assets, such as client data, financial records, or intellectual property. This finding is classified as a \textbf{Critical Risk}.


% ==============================================================================
% SECTION 4: TECHNICAL SCAN RESULTS
% ==============================================================================
\section{Technical Scan Results}

An external network scan was intended to be performed against the organization's public-facing infrastructure to identify open ports, running services, and potential vulnerabilities.

\begin{itemize}
    \item \textbf{Target IP Address:} \texttt{[Target IP]}
    \item \textbf{Scan Date:} Not Available
    \item \textbf{Status:} \textbf{Failed - Corrupted Data}
\end{itemize}

\subsection*{Analysis}
The provided network scan data (Input\_1\_Network\_Scan\_JSON) was corrupted and could not be parsed. As a result, no analysis of the external technical posture could be performed. It is not possible at this time to determine if outdated services, insecure configurations, or known vulnerabilities exist on the network perimeter. A new scan is required to complete this portion of the assessment.


% ==============================================================================
% SECTION 5: RISK ASSESSMENT
% ==============================================================================
\section{Risk Assessment}

This section synthesizes findings from all available data sources. Due to corrupted inputs for the technical scan and pre-existing risks, this assessment is primarily based on the Security Control Review.

\begin{table}[h!]
\centering
\caption{Identified Risks}
\begin{tabular}{p{0.1\linewidth} p{0.5\linewidth} p{0.15\linewidth} p{0.15\linewidth}}
\toprule
\textbf{Risk ID} & \textbf{Description} & \textbf{Severity} & \textbf{Source} \\
\midrule
RISK-001 & Lack of Multi-Factor Authentication (MFA) on sensitive data systems. This significantly increases the risk of unauthorized access to critical data through compromised credentials. & \textbf{Critical} & Security Questionnaire \\
\midrule
\multicolumn{4}{p{0.9\linewidth}}{\textit{Note: The list of pre-existing organizational risks (Input\_3\_Current\_Risks\_JSON) was unavailable due to data corruption. This table may be incomplete.}} \\
\bottomrule
\end{tabular}
\end{table}


% ==============================================================================
% SECTION 6: RECOMMENDATIONS
% ==============================================================================
\section{Recommendations}

Based on the analysis, the following actions are recommended to improve the cybersecurity posture of \textbf{Sovereign Trust}. Recommendations are prioritized by severity.

\begin{enumerate}
    \item \textbf{Implement MFA for Sensitive Systems (Critical):}
    Immediately prioritize and deploy a mandatory MFA solution for all user accounts (including administrative and service accounts) that have access to sensitive data systems. This is the single most effective control to mitigate the risk of unauthorized data access from compromised credentials.

    \item \textbf{Conduct a New External Network Scan (High):}
    A new, comprehensive vulnerability scan of the external IP address (\texttt{154.1.235.88}) and any other public-facing assets must be conducted. The previous scan data was unusable. This action is essential to identify and remediate technical vulnerabilities that could be exploited by external attackers.

    \item \textbf{Provide Current Risk Register (Medium):}
    For future assessments, ensure that a valid and up-to-date list of known organizational risks is provided. A holistic security review requires understanding both newly discovered and previously accepted risks.
\end{enumerate}

\end{document}
```