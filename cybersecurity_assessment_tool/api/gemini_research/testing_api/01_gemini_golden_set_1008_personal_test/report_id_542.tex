```latex
\documentclass[12pt]{article}

% Preamble: Required Packages
\usepackage[margin=1in]{geometry}
\usepackage{pifont} % For checkmarks and crosses
\usepackage{booktabs} % For professional tables
\usepackage{hyperref} % For clickable links
\usepackage{url} % For URL formatting
\usepackage{seqsplit} % To split long strings in tt font
\usepackage{graphicx}
\usepackage{fancyhdr}
\usepackage{xcolor}

% --- Document Metadata ---
\title{Cybersecurity Posture Assessment Report}
\author{Cybersecurity Analysis Division}
\date{\today}

% --- Hyperref Setup ---
\hypersetup{
    colorlinks=true,
    linkcolor=blue,
    filecolor=magenta,      
    urlcolor=cyan,
    pdftitle={Cybersecurity Posture Assessment Report},
    pdfpagemode=FullScreen,
}

% --- Header and Footer ---
\pagestyle{fancy}
\fancyhf{}
\fancyhead[L]{Cybersecurity Report: Nova Terra}
\fancyfoot[C]{\thepage}
\renewcommand{\headrulewidth}{0.4pt}
\renewcommand{\footrulewidth}{0.4pt}

\begin{document}

\maketitle
\thispagestyle{empty}
\newpage

\tableofcontents
\newpage

% --- Section 1: Executive Summary ---
\section{Executive Summary}
This report provides a comprehensive analysis of the cybersecurity posture for \textbf{Nova Terra}. The assessment combines a review of organizational security controls, an external network scan, and an evaluation of pre-existing risks.

The overall security posture is considered \textbf{weak} and requires immediate attention. Several critical-risk findings were identified that expose the organization to significant threats, including potential data breaches and unauthorized system access.

Key findings include:
\begin{itemize}
    \item \textbf{Critical FTP Vulnerability:} An internal host (\texttt{10.0.0.15}) is running a dangerously outdated version of vsftpd (2.3.4), which is known to have a critical backdoor vulnerability. Furthermore, it is configured to allow anonymous logins, posing an immediate and severe risk.
    \item \textbf{Insufficient Access Controls:} Multi-Factor Authentication (MFA) is not enforced for logging into computers or accessing sensitive data systems. This represents a major gap in identity and access management.
    \item \textbf{Foundational Policy Gaps:} The organization lacks a formal Acceptable Use Policy and does not provide security awareness training to new employees during onboarding. These gaps increase the risk of insider threats and human error.
    \item \textbf{End-of-Life Systems:} The continued use of Windows 7 workstations, a known pre-existing risk, leaves the organization vulnerable to exploits for which no security patches are available.
\end{itemize}

Immediate remediation of the identified technical vulnerabilities and policy gaps is strongly recommended to mitigate these risks and improve the organization's defensive capabilities.

% --- Section 2: Organizational Information ---
\section{Organizational Information}
The following details were provided for the assessment.

\begin{table}[h!]
\centering
\begin{tabular}{@{}ll@{}}
\toprule
\textbf{Attribute} & \textbf{Value} \\ \midrule
Organization Name & \textbf{Nova Terra} \\
Email Domain & \texttt{NovaTerra.com} \\
Website Domain & \url{www.NovaTerra.com} \\
External IP Address & \texttt{161.150.27.240} \\ \bottomrule
\end{tabular}
\caption{Client Organizational Data}
\end{table}

% --- Section 3: Security Control Review ---
\section{Security Control Review}
A review of organizational security controls was conducted via a questionnaire. The responses highlight significant gaps in foundational security practices. A red \ding{55} indicates a negative response that elevates organizational risk.

\begin{table}[h!]
\centering
\begin{tabular}{@{}lc@{}}
\toprule
\textbf{Control Question} & \textbf{Response} \\ \midrule
Do you require MFA to access email? & \ding{51} \\
Do you require MFA to log into computers? & \textcolor{red}{\ding{55}} \\
Do you require MFA to access sensitive data systems? & \textcolor{red}{\ding{55}} \\
Does your organization have an employee acceptable use policy? & \textcolor{red}{\ding{55}} \\
Does your organization do security awareness training for new employees? & \textcolor{red}{\ding{55}} \\
Does your organization do security training for all employees annually? & \ding{51} \\ \bottomrule
\end{tabular}
\caption{Security Controls Questionnaire Results}
\end{table}

% --- Section 4: Technical Scan Results ---
\section{Technical Scan Results}
A network scan was performed to identify open ports and exposed services on the specified target system.

\subsection{Target: \texttt{10.0.0.15}}
The scan revealed one host to be active and exposed a critical service vulnerability.

\begin{table}[h!]
\centering
\begin{tabular}{@{}lllll@{}}
\toprule
\textbf{Port} & \textbf{State} & \textbf{Service} & \textbf{Version} & \textbf{Finding} \\ \midrule
21/tcp & Open & FTP & vsftpd 2.3.4 & Anonymous FTP login allowed. \\ \bottomrule
\end{tabular}
\caption{Open Ports and Services on \texttt{10.0.0.15}}
\end{table}

\paragraph{Analysis:} The FTP service is running \textbf{vsftpd version 2.3.4}. This specific version was compromised in 2011, and a malicious backdoor was inserted into the source code. If this version is in use, it is trivial for an attacker to gain a command shell on the system. The allowance of anonymous FTP logins further lowers the barrier to exploitation and potential data exfiltration. This finding is classified as \textbf{Critical}.

% --- Section 5: Risk Assessment Summary ---
\section{Risk Assessment Summary}
The following table synthesizes findings from the security control review, technical scan, and pre-existing risk data into a consolidated list of identified risks.

\begin{table}[h!]
\centering
\resizebox{\textwidth}{!}{%
\begin{tabular}{@{}llll@{}}
\toprule
\textbf{Risk Name} & \textbf{Description} & \textbf{Source} & \textbf{Severity} \\ \midrule
Vulnerable FTP Server & Host is running a backdoored version of vsftpd (2.3.4) & Network Scan & \textbf{Critical} \\
& with anonymous login enabled. & & \\
\addlinespace
Lack of Endpoint MFA & No MFA is required for computer logins, allowing for & Questionnaire & High \\
& easier account compromise. & & \\
\addlinespace
Lack of Data Access MFA & Sensitive data systems are not protected by MFA, & Questionnaire & High \\
& increasing the risk of a data breach. & & \\
\addlinespace
Missing Acceptable Use & No formal policy exists to govern the use of company & Questionnaire & High \\
Policy & assets, leading to potential misuse. & & \\
\addlinespace
No Onboarding Training & New employees are not trained on security best & Questionnaire & High \\
& practices, creating immediate insider risk. & & \\
\addlinespace
End-of-Life OS & Workstations are running Windows 7, which is no longer & Existing Risks & Medium \\
& supported with security patches. & & \\ \bottomrule
\end{tabular}%
}
\caption{Consolidated Risk Register}
\end{table}

% --- Section 6: Recommendations ---
\section{Recommendations}
Based on the analysis, the following actions are recommended to mitigate the identified risks and strengthen the security posture of \textbf{Nova Terra}.

\subsection{Immediate Actions (Critical Priority)}
\begin{enumerate}
    \item \textbf{Remediate Vulnerable FTP Server:} Immediately take the FTP server on \texttt{10.0.0.15} offline. If the service is business-critical, it must be upgraded to a modern, patched version, and anonymous access must be disabled. A more secure file transfer protocol like SFTP should be considered as a replacement.
\end{enumerate}

\subsection{High Priority Actions}
\begin{enumerate}
    \setcounter{enumi}{1}
    \item \textbf{Implement Multi-Factor Authentication (MFA):} Deploy a robust MFA solution across the organization. Prioritize its enforcement for:
    \begin{itemize}
        \item All workstation and server logins (local and remote).
        \item Access to all systems storing or processing sensitive data.
    \end{itemize}
    \item \textbf{Develop and Implement Core Security Policies:}
    \begin{itemize}
        \item Create a formal \textbf{Acceptable Use Policy (AUP)} that all employees must read and acknowledge.
        \item Institute a mandatory \textbf{Security Awareness Training} module as part of the new employee onboarding process.
    \end{itemize}
\end{enumerate}

\subsection{Medium Priority Actions}
\begin{enumerate}
    \setcounter{enumi}{3}
    \item \textbf{Upgrade End-of-Life Systems:} Execute the plan to upgrade all Windows 7 workstations to a currently supported operating system (e.g., Windows 10/11) to ensure they receive critical security patches.
\end{enumerate}

\end{document}
```