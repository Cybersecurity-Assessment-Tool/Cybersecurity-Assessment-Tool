```latex
\documentclass[12pt]{article}

% Preamble: Required Packages
\usepackage[margin=1in]{geometry}
\usepackage{pifont} % For checkmarks and crosses
\usepackage{booktabs} % For professional tables
\usepackage{hyperref} % For clickable links
\usepackage{url}      % For formatting URLs
\usepackage{seqsplit} % For splitting long strings in tt font

% Document Metadata
\title{Cybersecurity Posture Assessment Report}
\author{Cybersecurity Analysis Division}
\date{\today}

\begin{document}

\maketitle
\thispagestyle{empty}
\newpage
\tableofcontents
\newpage

% ===================================================================
\section{Executive Summary}
% ===================================================================

This report provides a cybersecurity assessment for \textbf{Borealis Tech}, synthesizing data from network scans, organizational questionnaires, and a review of pre-existing risks.

The analysis reveals a mixed security posture. The organization has implemented strong multi-factor authentication (MFA) controls across key services, which is a commendable best practice. However, this strong foundation is undermined by two critical areas of concern:

\begin{enumerate}
    \item \textbf{Systemic Service Exposure:} A network scan identified a server (\texttt{10.10.10.51}) with an exposed Remote Desktop Protocol (RDP) service. When correlated with existing risk data, which identifies a similar issue on another host (\texttt{10.10.10.50}), this points to a systemic configuration weakness. Exposed RDP is a primary vector for ransomware and unauthorized access.
    \item \textbf{Gaps in Security Training:} The security questionnaire revealed that while annual training is conducted, new employees do not receive mandatory security awareness training during their onboarding. This creates a window of vulnerability where new hires are more susceptible to social engineering and phishing attacks.
\end{enumerate}

Immediate remediation is required to address the exposed RDP services. Furthermore, implementing a mandatory security training module for all new hires is strongly recommended to mitigate human-related risks and bolster the organization's overall security culture.

% ===================================================================
\section{Organizational Information}
% ===================================================================

The following information was provided by the client and used as a baseline for this assessment.

\begin{tabular}{@{}ll}
    \toprule
    \textbf{Attribute} & \textbf{Value} \\
    \midrule
    Organization Name & \textbf{Borealis Tech} \\
    Email Domain & \texttt{BorealisTech.com} \\
    Website Domain & \url{http://www.BorealisTech.com} \\
    Known External IP & \texttt{168.103.93.236} \\
    \bottomrule
\end{tabular}

% ===================================================================
\section{Security Control Review}
% ===================================================================

A review of the organization's security controls was conducted based on a standardized questionnaire. The results indicate a strong adoption of MFA but highlight a significant gap in the employee onboarding process.

\begin{table}[h!]
\centering
\caption{Security Controls Questionnaire Results}
\begin{tabular}{@{}p{0.8\linewidth}c@{}}
    \toprule
    \textbf{Security Control Question} & \textbf{Response} \\
    \midrule
    Do you require MFA to access email? & \ding{51} \\ % Yes
    Do you require MFA to log into computers? & \ding{51} \\ % Yes
    Do you require MFA to access sensitive data systems? & \ding{51} \\ % Yes
    Does your organization have an employee acceptable use policy? & \ding{51} \\ % Yes
    Does your organization do security awareness training for all employees at least once per year? & \ding{51} \\ % Yes
    \addlinespace
    \textbf{Does your organization do security awareness training for new employees?} & \textbf{\ding{55}} \\ % No
    \bottomrule
\end{tabular}
\end{table}

The failure to provide security awareness training to new employees represents a high-risk gap. New hires are often unfamiliar with corporate policies and are prime targets for phishing and social engineering attacks. This gap should be addressed to ensure all personnel, from their first day, are equipped to recognize and report potential threats.

% ===================================================================
\section{Technical Scan Results}
% ===================================================================

An internal network scan was performed to identify active services and potential vulnerabilities.

\subsection{Scan Target: \texttt{10.10.10.51}}
The scan identified the following open port on the target host:

\begin{table}[h!]
\centering
\caption{Open Ports on \texttt{10.10.10.51}}
\begin{tabular}{@{}llll@{}}
    \toprule
    \textbf{Port} & \textbf{State} & \textbf{Service Name} & \textbf{Description} \\
    \midrule
    3389/tcp & open & \texttt{ms-wbt-server} & Microsoft Remote Desktop Protocol (RDP) \\
    \bottomrule
\end{tabular}
\end{table}

\subsection{Analysis of Findings}
The discovery of an open RDP port is a critical finding. RDP is a powerful remote administration tool, but when exposed insecurely, it becomes a primary target for attackers. Threat actors actively scan for open RDP ports to exploit them for:
\begin{itemize}
    \item Brute-force password attacks to gain unauthorized access.
    \item Deployment of ransomware.
    \item Exploitation of known vulnerabilities (e.g., BlueKeep).
\end{itemize}

This finding, combined with pre-existing risk data showing RDP exposure on another host (\texttt{10.10.10.50}), indicates a pattern of insecure configuration that significantly elevates the organization's risk profile.

% ===================================================================
\section{Consolidated Risk Assessment}
% ===================================================================

The following table synthesizes findings from the security questionnaire, technical scans, and pre-existing risk data into a prioritized list of current risks.

\begin{table}[h!]
\centering
\caption{Summary of Identified Risks}
\begin{tabular}{@{}p{0.15\linewidth}p{0.55\linewidth}p{0.2\linewidth}@{}}
    \toprule
    \textbf{Risk Name} & \textbf{Description} & \textbf{Severity} \\
    \midrule
    \textbf{Systemic RDP Exposure} & Remote Desktop Protocol (port 3389) is exposed on multiple internal servers, including \texttt{10.10.10.50} and \texttt{10.10.10.51}. This provides a direct path for attackers to compromise key systems. & \textbf{Critical} \\
    \addlinespace
    \textbf{Lack of Onboarding Security Training} & New employees do not receive security awareness training upon being hired. This leaves them vulnerable to social engineering and phishing, increasing the risk of initial compromise. & \textbf{High} \\
    \bottomrule
\end{tabular}
\end{table}

% ===================================================================
\section{Recommendations}
% ===================================================================

The following actions are recommended to mitigate the identified risks. Recommendations are prioritized based on severity.

\subsection{Priority 1: Remediate RDP Exposure (Critical)}
\begin{itemize}
    \item \textbf{Immediate Action:} Apply network firewall rules to block all access to TCP port 3389 on \texttt{10.10.10.50}, \texttt{10.10.10.51}, and any other systems from untrusted networks. Access should be restricted to only authorized administrative subnets.
    \item \textbf{Long-Term Strategy:} Implement a secure remote access solution, such as a Virtual Private Network (VPN) with MFA, for all administrative access. Direct RDP access over the network should be disabled as a matter of policy.
\end{itemize}

\subsection{Priority 2: Implement New Hire Security Training (High)}
\begin{itemize}
    \item \textbf{Immediate Action:} Develop and mandate a security awareness training module as a required step in the employee onboarding process.
    \item \textbf{Content Strategy:} The training should cover, at a minimum:
    \begin{itemize}
        \item Phishing and social engineering awareness.
        \item Strong password and credential management.
        \item The organization's acceptable use policy.
        \item Procedures for reporting security incidents.
    \end{itemize}
\end{itemize}

% ===================================================================
\section{Conclusion}
% ===================================================================

\textbf{Borealis Tech} has established a solid security foundation with its widespread use of MFA. However, significant risks stemming from insecure service configurations and gaps in the security training program threaten to undermine these strengths.

By taking immediate action to secure exposed RDP services and formalizing security training within the onboarding process, the organization can substantially reduce its attack surface and improve its overall resilience against common cyber threats. Continued vigilance and proactive risk management will be essential to maintaining a robust security posture.

\end{document}
```