```latex
\documentclass[12pt, a4paper]{article}

% Preamble: Required Packages
\usepackage[margin=1in]{geometry}
\usepackage{pifont} % For check and cross marks
\usepackage{booktabs} % For professional tables
\usepackage{hyperref} % For clickable links
\usepackage{url} % For formatting URLs
\usepackage{seqsplit} % For splitting long text strings to prevent overflow
\usepackage{graphicx}
\usepackage{xcolor}

% Hyperref Setup
\hypersetup{
    colorlinks=true,
    linkcolor=blue,
    filecolor=magenta,      
    urlcolor=cyan,
    pdftitle={Cybersecurity Assessment Report},
    pdfpagemode=FullScreen,
}

% Define custom colors
\definecolor{darkred}{rgb}{0.55, 0.0, 0.0}
\definecolor{darkorange}{rgb}{1.0, 0.55, 0.0}
\definecolor{darkyellow}{rgb}{0.8, 0.8, 0.0}

% Custom command for severity
\newcommand{\severity}[2]{
    \ifstrequal{#1}{Critical}{\textcolor{darkred}{\textbf{#2}}}{}
    \ifstrequal{#1}{High}{\textcolor{darkorange}{\textbf{#2}}}{}
    \ifstrequal{#1}{Medium}{\textcolor{darkyellow}{\textbf{#2}}}{}
    \ifstrequal{#1}{Low}{\textbf{#2}}{}
}

% Document Start
\begin{document}

% --- Title Page ---
\begin{titlepage}
    \centering
    \vspace*{1cm}
    \Huge\textbf{Cybersecurity Assessment Report}
    \vspace{1.5cm}
    \Large\textbf{Prepared for:} \\
    \vspace{0.5cm}
    \huge{Pioneer Pulse}
    \vspace{2cm}
    \large\textbf{Date of Report:} \\
    \vspace{0.5cm}
    \Large{\today}
    \vfill
    \large\textit{This report contains sensitive information and should be handled with care.}
\end{titlepage}

\tableofcontents
\newpage

% --- Section 1: Executive Overview ---
\section{Executive Overview}
This report provides a comprehensive cybersecurity assessment for \textbf{Pioneer Pulse}, based on an analysis of network scan data, organizational security controls, and pre-existing risk information. The assessment was conducted to identify vulnerabilities, evaluate the current security posture, and provide actionable recommendations for risk mitigation.

The analysis revealed several critical and high-severity risks that require immediate attention. Key findings include a publicly accessible and highly vulnerable FTP service (\texttt{vsftpd 2.3.4}), a complete absence of Multi-Factor Authentication (MFA) across all critical systems, and significant gaps in foundational security policies and employee training.

Collectively, these findings indicate a security posture with a high level of exposure to common cyber threats such as unauthorized access, data breach, and malware infection. Prioritized recommendations are provided in Section \ref{sec:recommendations} to address these issues systematically.

% --- Section 2: Organizational Information ---
\section{Organizational Information}
The following details were provided for the assessment scope.
\begin{itemize}
    \item \textbf{Organization Name:} Pioneer Pulse
    \item \textbf{Email Domain:} \texttt{PioneerPulse.org}
    \item \textbf{Website Domain:} \url{www.PioneerPulse.org}
    \item \textbf{External IP Address:} \texttt{231.175.128.160}
\end{itemize}

% --- Section 3: Security Control Review ---
\section{Security Control Review}
A review of organizational security controls was conducted via a questionnaire. The responses highlight significant gaps in access control and employee governance, which are foundational elements of a robust security program.

\begin{table}[h!]
\centering
\caption{Organizational Security Control Questionnaire}
\label{tab:controls}
\begin{tabular}{p{0.75\textwidth} c}
\toprule
\textbf{Control Question} & \textbf{Response} \\
\midrule
Do you require MFA to access email? & \ding{55} \\
Do you require MFA to log into computers? & \ding{55} \\
Do you require MFA to access sensitive data systems? & \ding{55} \\
Does your organization have an employee acceptable use policy? & \ding{55} \\
Does your organization do security awareness training for new employees? & \ding{55} \\
Does your organization do security awareness training for all employees at least once per year? & \ding{51} \\
\bottomrule
\end{tabular}
\end{table}

\noindent \textbf{Analysis:} The lack of MFA for email, computers, and sensitive data systems represents a \textbf{critical risk}. The absence of an acceptable use policy and security training for new hires constitutes a \textbf{high risk}, as it fails to establish a baseline for secure employee behavior.

% --- Section 4: Technical Scan Results ---
\section{Technical Scan Results}
A network scan was performed to identify open ports and exposed services on the target system.

\begin{itemize}
    \item \textbf{Target IP Address:} \texttt{10.0.0.15}
\end{itemize}

\begin{table}[h!]
\centering
\caption{Open Port Analysis}
\label{tab:nmap}
\begin{tabular}{l l l l p{0.3\textwidth}}
\toprule
\textbf{Port} & \textbf{State} & \textbf{Service} & \textbf{Version} & \textbf{Notes} \\
\midrule
21/tcp & Open & ftp & vsftpd 2.3.4 & Anonymous FTP login is allowed. This version is known to be vulnerable to a critical backdoor (CVE-2011-2523). \\
\bottomrule
\end{tabular}
\end{table}

\noindent \textbf{Analysis:} The presence of an open FTP port with version \texttt{vsftpd 2.3.4} is a \textbf{critical vulnerability}. This specific version contains a well-documented backdoor that allows an attacker to gain remote command execution. Furthermore, allowing anonymous FTP login creates a high-risk channel for unauthorized data access or for attackers to stage malicious files.

% --- Section 5: Synthesized Risk Assessment ---
\section{Synthesized Risk Assessment}
The following table correlates findings from the security control review, technical scan, and pre-existing risk data to provide a unified view of the organization's risk profile.

\begin{table}[h!]
\centering
\caption{Summary of Identified Risks}
\label{tab:risks}
\begin{tabular}{p{0.3\textwidth} p{0.5\textwidth} l}
\toprule
\textbf{Risk Name} & \textbf{Description} & \textbf{Severity} \\
\midrule
Vulnerable FTP Service (CVE-2011-2523) & The FTP server is running a version with a known remote code execution backdoor. & \severity{Critical}{Critical} \\
\addlinespace
Absence of Multi-Factor Authentication & Lack of MFA on email, computers, and sensitive systems makes accounts highly susceptible to compromise via stolen credentials. & \severity{Critical}{Critical} \\
\addlinespace
Insecure Protocol and Configuration & Use of unencrypted FTP protocol and allowing anonymous login exposes data and credentials to interception and abuse. & \severity{High}{High} \\
\addlinespace
Weak Policy and Training Framework & No acceptable use policy or new hire training program leads to inconsistent and insecure employee practices. & \severity{High}{High} \\
\addlinespace
Outdated Windows Policy & Workstations are running Windows 7, an unsupported operating system that no longer receives security updates. & \severity{Medium}{Medium} \\
\bottomrule
\end{tabular}
\end{table}

% --- Section 6: Recommendations ---
\section{Recommendations}
\label{sec:recommendations}
The following actions are recommended to mitigate the identified risks. They are prioritized based on severity and potential impact.

\subsection{Immediate Priority (Critical Risks)}
\begin{enumerate}
    \item \textbf{Remediate Vulnerable FTP Service:} Immediately take the server at \texttt{10.0.0.15} offline. Patch or upgrade the \texttt{vsftpd} service to the latest stable version. If FTP is not a business requirement, the service should be disabled entirely.
    \item \textbf{Replace FTP with Secure Alternative:} Decommission the use of FTP and replace it with a secure file transfer protocol such as SFTP (SSH File Transfer Protocol) or FTPS (FTP over SSL/TLS).
    \item \textbf{Implement Multi-Factor Authentication (MFA):} Begin a phased rollout of MFA across the organization.
        \begin{itemize}
            \item \textbf{Phase 1:} Enforce MFA for email access and all administrative accounts.
            \item \textbf{Phase 2:} Enforce MFA for access to all systems containing sensitive data.
            \item \textbf{Phase 3:} Enforce MFA for all standard user computer logins.
        \end{itemize}
\end{enumerate}

\subsection{High Priority (High Risks)}
\begin{enumerate}
    \setcounter{enumi}{3}
    \item \textbf{Develop Foundational Security Policies:} Draft and implement an official Acceptable Use Policy (AUP) that all employees must read and sign. This policy should govern the use of company assets, data handling, and internet usage.
    \item \textbf{Establish a Security Training Program:} Institute a mandatory security awareness training module for all new employees as part of their onboarding process. Continue the existing annual training for all staff.
\end{enumerate}

\subsection{Medium Priority (Medium Risks)}
\begin{enumerate}
    \setcounter{enumi}{5}
    \item \textbf{Plan Operating System Upgrades:} Develop a project plan to upgrade all workstations from Windows 7 to a modern, supported operating system such as Windows 10 or Windows 11 to ensure the continued receipt of security patches.
\end{enumerate}

\end{document}
```