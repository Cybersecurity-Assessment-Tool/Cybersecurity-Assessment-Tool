```latex
\documentclass[12pt]{article}

% Preamble: Required Packages
\usepackage[margin=1in]{geometry}
\usepackage{pifont} % For checkmarks and crosses
\usepackage{booktabs} % For professional tables
\usepackage{hyperref} % For clickable links
\usepackage{url} % For URL formatting
\usepackage{seqsplit} % For splitting long strings
\usepackage{graphicx} % For logo (optional, but good practice)
\usepackage{xcolor} % For colors in text

% Document Metadata
\title{Cybersecurity Posture Assessment Report}
\author{Cybersecurity Analysis Division}
\date{November 22, 2025}

% Hyperref Setup
\hypersetup{
    colorlinks=true,
    linkcolor=blue,
    filecolor=magenta,      
    urlcolor=cyan,
    pdftitle={Cybersecurity Posture Assessment Report},
    pdfpagemode=FullScreen,
}

\begin{document}

\maketitle
\thispagestyle{empty}
\newpage

\tableofcontents
\newpage

% --- Section 1: Executive Overview ---
\section{Executive Overview}

This report provides a comprehensive cybersecurity posture assessment for \textbf{Nexus Dynamics}, conducted on November 22, 2025. The assessment combines a review of organizational security controls, an external network scan, and an analysis of pre-existing risks.

Overall, \textbf{Nexus Dynamics} demonstrates a solid foundation in identity and access management, with strong Multi-Factor Authentication (MFA) controls implemented across key systems. However, two high-risk vulnerabilities were identified that require immediate attention to mitigate potential threats.

\begin{itemize}
    \item \textbf{High-Risk Finding 1:} The organization's public-facing web server is running an outdated version of Nginx (1.18.0), which is susceptible to multiple known vulnerabilities.
    \item \textbf{High-Risk Finding 2:} There is a significant gap in the security awareness program, as ongoing annual training is not provided to all employees. This increases the organization's susceptibility to social engineering and phishing attacks.
\end{itemize}

While existing controls are commendable, these findings present a tangible risk to the organization's security. The recommendations outlined in this report are designed to address these gaps and enhance the overall defensive posture.

% --- Section 2: Organizational Information ---
\section{Organizational Information}

The following details were provided for the assessment.

\begin{table}[h!]
\centering
\begin{tabular}{@{}ll@{}}
\toprule
\textbf{Attribute} & \textbf{Value} \\ \midrule
Organization Name    & \textbf{Nexus Dynamics} \\
Email Domain         & \texttt{NexusDynamics.com} \\
External IP Address  & \texttt{218.150.156.36} \\
Scan Target IP       & \texttt{192.168.10.5} \\
Scan Date            & 2025-11-22 \\ \bottomrule
\end{tabular}
\caption{Client Organizational Data}
\label{tab:org_data}
\end{table}

% --- Section 3: Security Control Review ---
\section{Security Control Review}

A review of the organization's security controls was conducted via a standardized questionnaire. The results indicate strong preventative measures in access control but highlight a critical weakness in ongoing employee security education.

\begin{table}[h!]
\centering
\begin{tabular}{@{}p{0.8\linewidth}c@{}}
\toprule
\textbf{Control Question} & \textbf{Response} \\ \midrule
Do you require MFA to access email? & \ding{51} \\
Do you require MFA to log into computers? & \ding{51} \\
Do you require MFA to access sensitive data systems? & \ding{51} \\
Does your organization have an employee acceptable use policy? & \ding{51} \\
Does your organization do security awareness training for new employees? & \ding{51} \\
\textbf{Does your organization do security awareness training for all employees at least once per year?} & \textcolor{red}{\ding{55}} \\ \bottomrule
\end{tabular}
\caption{Security Controls Questionnaire Results (\ding{51}=Yes, \ding{55}=No)}
\label{tab:controls}
\end{table}

\subsection*{Analysis of Findings}
The failure to provide annual security awareness training for all staff is a critical gap. The threat landscape evolves continuously, and without regular reinforcement, employees are more likely to fall victim to sophisticated phishing, ransomware, and other social engineering attacks. This oversight undermines the effectiveness of technical controls and exposes the organization to significant risk from human error.

% --- Section 4: Technical Scan Results ---
\section{Technical Scan Results}

An Nmap scan was performed on the target IP address \texttt{192.168.10.5} to identify open ports and exposed services.

\begin{table}[h!]
\centering
\begin{tabular}{@{}lllll@{}}
\toprule
\textbf{Port} & \textbf{State} & \textbf{Service} & \textbf{Product} & \textbf{Version} \\ \midrule
443/tcp & open & https & nginx & \textbf{\textcolor{red}{1.18.0}} \\ \bottomrule
\end{tabular}
\caption{Open Ports and Services on \texttt{192.168.10.5}}
\label{tab:scan_results}
\end{table}

\subsection*{Analysis of Findings}
The scan identified a single open port, 443 (HTTPS), running an Nginx web server. The detected version, \textbf{1.18.0}, was released in April 2020 and is now significantly outdated. This version is known to be vulnerable to several Common Vulnerabilities and Exposures (CVEs), including but not limited to issues related to request smuggling and DNS resolver vulnerabilities (e.g., CVE-2021-23017). Running unsupported and unpatched software on a public-facing service presents a high risk of compromise.

% --- Section 5: Risk Assessment ---
\section{Risk Assessment}

The following table synthesizes findings from the security control review and technical scan into a prioritized list of risks. No pre-existing vulnerabilities were reported.

\begin{table}[h!]
\centering
\begin{tabular}{@{}lp{0.25\linewidth}llp{0.4\linewidth}@{}}
\toprule
\textbf{ID} & \textbf{Risk Name} & \textbf{Severity} & \textbf{Description} \\ \midrule
R-001 & Outdated Web Server Software (Nginx 1.18.0) & \textbf{High} & The public-facing web server is running a vulnerable version of Nginx. Attackers can exploit known vulnerabilities to gain unauthorized access, exfiltrate data, or cause a denial of service. \\
\addlinespace
R-002 & Lack of Annual Security Awareness Training & \textbf{High} & Employees do not receive ongoing security training. This increases the likelihood of successful phishing and social engineering attacks, which can lead to credential theft, malware infection, and data breaches. \\ \bottomrule
\end{tabular}
\caption{Identified Risks and Severity}
\label{tab:risk_assessment}
\end{table}

% --- Section 6: Recommendations ---
\section{Recommendations}

The following actions are recommended to mitigate the identified risks and improve the overall security posture of \textbf{Nexus Dynamics}.

\subsection*{R-001: Remediate Outdated Web Server Software}
\begin{itemize}
    \item \textbf{Immediate Action:} Develop a patch and upgrade plan for the Nginx server located at \texttt{192.168.10.5}. Upgrade the software to the latest stable version recommended by the vendor immediately.
    \item \textbf{Strategic Action:} Implement a formal patch management policy and program. This program should ensure that all critical infrastructure is scanned for vulnerabilities regularly and that patches are applied within a defined timeframe based on risk.
\end{itemize}

\subsection*{R-002: Implement Annual Security Training Program}
\begin{itemize}
    \item \textbf{Immediate Action:} Procure and schedule a mandatory security awareness training program for all current employees to be completed within the next quarter.
    \item \textbf{Strategic Action:} Establish a formal, ongoing security awareness program that includes, at a minimum, annual training for all staff. This program should cover topics such as phishing identification, password security, acceptable use, and incident reporting. Consider incorporating periodic phishing simulations to test and reinforce learning.
\end{itemize}

\end{document}
```