Here is the complete and professional LaTeX report generated from the provided data inputs.

```latex
\documentclass[12pt]{article}

% --- PACKAGES ---
\usepackage[margin=1in]{geometry}
\usepackage{pifont} % For checkmarks and crosses
\usepackage{booktabs} % For professional tables
\usepackage{hyperref} % For clickable links and metadata
\usepackage{url} % For formatting URLs
\usepackage{seqsplit} % To split long strings without breaking words

% --- DOCUMENT METADATA ---
\hypersetup{
    colorlinks=true,
    linkcolor=black,
    urlcolor=blue,
    pdftitle={Cybersecurity Posture Assessment Report},
    pdfauthor={Cybersecurity Analyst},
    pdfsubject={Security Analysis}
}

% --- TITLE ---
\title{Cybersecurity Posture Assessment Report}
\author{Cybersecurity Analyst}
\date{\today}

\begin{document}

\maketitle
\tableofcontents
\newpage

% ==============================================================================
\section{Executive Overview}
% ==============================================================================

This report provides a cybersecurity posture assessment for \textbf{Blackwood Industries}. The analysis is based on a combination of self-reported organizational data, a technical network scan, and a review of pre-existing risks.

A critical issue was identified during data collection: both the technical network scan data (\texttt{Input\_1}) and the current risks data (\texttt{Input\_3}) were corrupted and could not be parsed. This creates a significant visibility gap into the organization's external attack surface and known vulnerabilities.

The analysis of the available organizational data (\texttt{Input\_2}) reveals several high-priority security gaps. The most critical findings are the lack of Multi-Factor Authentication (MFA) for accessing email and sensitive data systems. These gaps expose the organization to a high risk of account compromise, business email compromise (BEC), and data breaches. Additionally, the absence of a formal Employee Acceptable Use Policy indicates a weakness in security governance.

While the organization has implemented security awareness training and requires MFA for computer logins, the identified control failures require immediate attention. The recommendations in this report are prioritized to address the most severe risks first. A comprehensive re-assessment, including a successful network scan, is strongly advised.

% ==============================================================================
\section{Organizational Information}
% ==============================================================================

The following information was provided by the client and used as a baseline for this assessment.

\begin{itemize}
    \item \textbf{Organization Name:} Blackwood Industries
    \item \textbf{Email Domain:} \texttt{BlackwoodIndustries.com}
    \item \textbf{Website Domain:} \seqsplit{\url{www.BlackwoodIndustries.com}}
    \item \textbf{Primary External IP:} \texttt{226.212.187.57}
\end{itemize}

% ==============================================================================
\section{Security Control Review (Questionnaire)}
% ==============================================================================

A security questionnaire was completed to assess the implementation of key administrative and technical controls. The responses are summarized below. Items marked with a red cross (\ding{55}) represent significant control gaps that increase organizational risk.

\begin{table}[h!]
\centering
\caption{Security Control Questionnaire Responses}
\begin{tabular}{p{0.8\linewidth} c}
\toprule
\textbf{Control Question} & \textbf{Response} \\
\midrule
Do you require MFA to access email? & \ding{55} \\
Do you require MFA to log into computers? & \ding{51} \\
Do you require MFA to access sensitive data systems? & \ding{55} \\
Does your organization have an employee acceptable use policy? & \ding{55} \\
Does your organization do security awareness training for new employees? & \ding{51} \\
Does your organization do security awareness training for all employees at least once per year? & \ding{51} \\
\bottomrule
\end{tabular}
\end{table}

\subsection*{Analysis}
The questionnaire reveals critical weaknesses in access control policies. The failure to enforce MFA on email and sensitive data systems are high-severity findings. Email is a primary vector for phishing and account takeover, while sensitive data systems are the ultimate target for attackers seeking to exfiltrate valuable information. The lack of an Acceptable Use Policy (AUP) represents a foundational governance gap, leaving employees without clear guidelines on security responsibilities.

% ==============================================================================
\section{Technical Scan Results}
% ==============================================================================

An external network scan was initiated against the target IP address \texttt{[Target IP]}.

\subsection*{Scan Status: FAILED}
The raw data file received from the network scanner (\texttt{Input\_1\_Network\_Scan\_JSON}) was found to be corrupted and could not be processed. 

\subsection*{Implications}
Due to the corrupted data, there is currently no visibility into the external network perimeter of \textbf{Blackwood Industries}. Key information that could not be obtained includes:
\begin{itemize}
    \item Open TCP/UDP ports
    \item Services, products, and versions running on open ports
    \item Potential vulnerabilities associated with outdated or misconfigured services
\end{itemize}
This lack of visibility is a significant finding in itself. Without this data, it is impossible to assess the technical attack surface exposed to the internet. It is imperative that a new scan is conducted successfully.

% ==============================================================================
\section{Risk Assessment}
% ==============================================================================

This risk assessment is based exclusively on the findings from the Security Control Review due to the unavailability of technical scan data and pre-existing risk information (\texttt{Input\_3\_Current\_Risks\_JSON} was corrupted). The following new risks have been identified.

\begin{table}[h!]
\centering
\caption{Identified Risks and Severity}
\begin{tabular}{p{0.1\linewidth} p{0.3\linewidth} p{0.4\linewidth} p{0.1\linewidth}}
\toprule
\textbf{Risk ID} & \textbf{Risk Name} & \textbf{Overview} & \textbf{Severity} \\
\midrule
R-01 & No MFA on Sensitive Data Systems & Lack of a second authentication factor for critical systems makes them highly susceptible to compromise via stolen credentials, leading to a potential data breach. & \textbf{Critical} \\
\addlinespace
R-02 & No MFA on Email & Exposes the organization to Business Email Compromise (BEC), phishing, and account takeovers. Compromised email can be used to pivot to other internal systems. & High \\
\addlinespace
R-03 & No Acceptable Use Policy (AUP) & Lack of a formal policy weakens security governance and creates ambiguity around employee responsibilities for protecting company assets and data. & High \\
\addlinespace
R-04 & Incomplete External Visibility & The failed network scan means the external attack surface is unknown, preventing the identification of potentially exploitable vulnerabilities. & High \\
\bottomrule
\end{tabular}
\end{table}

% ==============================================================================
\section{Recommendations}
% ==============================================================================

The following actionable recommendations are prioritized based on the severity of the identified risks.

\begin{enumerate}
    \item \textbf{[Critical] Implement MFA on Sensitive Data Systems:}
    Immediately enforce MFA for all user accounts (especially privileged accounts) that can access systems containing sensitive or critical business data. This is the single most effective control to prevent unauthorized access and data exfiltration.

    \item \textbf{[High] Enforce MFA for Email Access:}
    Deploy mandatory MFA for all users accessing the corporate email system, whether via webmail or client applications. This will drastically reduce the likelihood of a successful phishing or account takeover attack.

    \item \textbf{[High] Develop and Implement an Acceptable Use Policy (AUP):}
    Create a formal AUP that clearly defines the rules and responsibilities for all employees when using company technology and data. Ensure all current and new employees read and formally acknowledge the policy.

    \item \textbf{[Informational] Conduct a New External Network Scan:}
    A new, complete network scan of the external IP range must be performed to identify open ports and vulnerable services. The results of this scan are essential for understanding and mitigating technical risks on the network perimeter.
    
    \item \textbf{[Informational] Restore Pre-existing Risk Register:}
    Investigate the cause of the data corruption for the existing risk list (\texttt{Input\_3}) and restore it from backups. A historical view of risks is crucial for a mature risk management program.
\end{enumerate}

\end{document}
```