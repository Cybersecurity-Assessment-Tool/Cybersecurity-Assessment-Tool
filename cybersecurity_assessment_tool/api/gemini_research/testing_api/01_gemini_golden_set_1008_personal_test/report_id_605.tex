```latex
\documentclass[12pt]{article}

% 1. Required Packages
\usepackage[margin=1in]{geometry}
\usepackage{pifont} % For checkmarks and crosses
\usepackage{booktabs} % For professional tables
\usepackage{hyperref} % For clickable links and metadata
\usepackage{url} % For URL formatting
\usepackage{seqsplit} % To split long monospaced text
\usepackage{xcolor} % For colors
\usepackage{graphicx} % For potential logos in the future
\usepackage{fancyhdr} % For headers/footers

% 2. Document Setup and Metadata
\hypersetup{
    colorlinks=true,
    linkcolor=blue,
    filecolor=magenta,      
    urlcolor=cyan,
    pdftitle={Cybersecurity Risk Assessment Report},
    pdfauthor={Cybersecurity Analyst},
    pdfsubject={Security Analysis},
    pdfkeywords={Security, Risk, Assessment},
}

% 3. Custom Commands and Colors
\newcommand{\yes}{\ding{51}}
\newcommand{\no}{\ding{55}}
\definecolor{sev_critical}{HTML}{990000}
\definecolor{sev_high}{HTML}{D2691E}
\definecolor{sev_medium}{HTML}{DAA520}

% Header and Footer
\pagestyle{fancy}
\fancyhf{} % clear all header and footer fields
\fancyhead[L]{Cybersecurity Risk Assessment Report}
\fancyhead[R]{Confidential}
\fancyfoot[C]{\thepage}

\begin{document}

% --- Title Page ---
\begin{titlepage}
    \centering
    \vspace*{1cm}
    \Huge\textbf{Cybersecurity Risk Assessment Report}
    \vspace{1.5cm}
    \Large
    \textbf{Prepared for:} \\
    \vspace{0.5cm}
    \textbf{Great Lakes} % From Input 2
    \vspace{2cm}
    \large
    \textbf{Date of Report:} \\
    \vspace{0.5cm}
    \today
    \vfill
    \textit{This document is confidential and intended solely for the use of the recipient.}
\end{titlepage}

\tableofcontents
\newpage

% --- Section 1: Executive Summary ---
\section{Executive Summary}
This report provides a comprehensive analysis of the current cybersecurity posture of \textbf{Great Lakes}, based on a synthesis of network scan data, organizational security controls, and pre-existing risk information. The assessment reveals several critical and high-risk vulnerabilities that require immediate attention to mitigate potential threats.

Key findings indicate significant gaps in fundamental security controls. The absence of Multi-Factor Authentication (MFA) for computer and sensitive data access, coupled with a lack of a formal Acceptable Use Policy and recurring security training, creates a high-risk environment susceptible to credential compromise and insider threats.

Furthermore, technical scans have confirmed the presence of an exposed Remote Desktop Protocol (RDP) service on the internal network. This finding, correlated with pre-existing risks, points to a systemic issue of insecure service configuration. When combined with the identified MFA gaps, this vulnerability presents a direct and severe threat, as it is a common vector for ransomware attacks and unauthorized access.

Immediate remediation of the identified RDP exposures and the implementation of a comprehensive MFA policy are the highest priorities.

% --- Section 2: Organizational Information ---
\section{Organizational Information}
The following details were provided for the assessment.
\begin{itemize}
    \item \textbf{Organization Name:} Great Lakes
    \item \textbf{Email Domain:} \texttt{GreatLakes.com}
    \item \textbf{Website Domain:} \texttt{www.GreatLakes.com}
    \item \textbf{Known External IP:} \texttt{34.114.94.226}
\end{itemize}

% --- Section 3: Security Control Review ---
\section{Security Control Review}
A review of the organization's security controls was conducted via a questionnaire. The responses highlight critical gaps in administrative and technical safeguards. A "No" response indicates a missing control and a significant increase in risk.

\begin{table}[h!]
\centering
\caption{Security Controls Questionnaire Analysis}
\label{tab:controls}
\begin{tabular}{p{8cm} c p{4cm}}
\toprule
\textbf{Control Question} & \textbf{Response} & \textbf{Assessment} \\
\midrule
Do you require MFA to access email? & \yes & \textbf{Good Practice:} Control is in place. \\
\addlinespace
Do you require MFA to log into computers? & \no & \textcolor{sev_critical}{\textbf{Critical Gap:} Lack of endpoint MFA allows for trivial lateral movement if credentials are stolen.} \\
\addlinespace
Do you require MFA to access sensitive data systems? & \no & \textcolor{sev_critical}{\textbf{Critical Gap:} The organization's most valuable data is not protected by a critical access control.} \\
\addlinespace
Does your organization have an employee acceptable use policy? & \no & \textcolor{sev_high}{\textbf{High Risk:} No formal guidelines for employees, leading to inconsistent practices and potential misuse.} \\
\addlinespace
Does your organization do security awareness training for new employees? & \yes & \textbf{Good Practice:} Foundational training is provided. \\
\addlinespace
Does your organization do security awareness training for all employees at least once per year? & \no & \textcolor{sev_high}{\textbf{High Risk:} Security knowledge degrades over time. Lack of recurring training increases susceptibility to phishing and social engineering.} \\
\bottomrule
\end{tabular}
\end{table}

% --- Section 4: Technical Scan Results ---
\section{Technical Scan Results}
An Nmap scan was performed to identify open ports and services on the target system. The results indicate the presence of an exposed remote access service.

\begin{itemize}
    \item \textbf{Target IP Address:} \texttt{10.10.10.51}
\end{itemize}

\begin{table}[h!]
\centering
\caption{Open Ports Detected on \texttt{10.10.10.51}}
\label{tab:nmap}
\begin{tabular}{l l l p{7cm}}
\toprule
\textbf{Port} & \textbf{State} & \textbf{Service} & \textbf{Analysis} \\
\midrule
3389/tcp & Open & \texttt{ms-wbt-server} & This is the Microsoft Remote Desktop Protocol (RDP). Exposing RDP without compensating controls (e.g., VPN, MFA, IP whitelisting) is a critical security risk and a primary target for attackers. \\
\bottomrule
\end{tabular}
\end{table}

% --- Section 5: Consolidated Risk Assessment ---
\section{Consolidated Risk Assessment}
The following table synthesizes findings from the security control review, technical scan, and pre-existing risk data into a prioritized list of risks.

\begin{table}[h!]
\centering
\caption{Prioritized Risk Summary}
\label{tab:risks}
\begin{tabular}{p{1.5cm} p{3.5cm} p{6cm} l}
\toprule
\textbf{Risk ID} & \textbf{Risk Name} & \textbf{Description} & \textbf{Severity} \\
\midrule
RISK-001 & Systemic RDP Exposure & The technical scan identified RDP open on \texttt{10.10.10.51}. This correlates with a known risk on \texttt{10.10.10.50}, indicating a pattern of insecure configuration. RDP is a primary vector for ransomware. & \textcolor{sev_critical}{\textbf{Critical}} \\
\addlinespace
RISK-002 & Insufficient Identity and Access Management & The lack of MFA on computer logins and sensitive data systems means a single compromised password provides an attacker with significant access to the internal network and critical data. & \textcolor{sev_critical}{\textbf{Critical}} \\
\addlinespace
RISK-003 & Deficient Security Governance & The absence of an Acceptable Use Policy and mandatory annual security training for all staff significantly increases the human risk factor, making technical vulnerabilities more likely to be exploited. & \textcolor{sev_high}{\textbf{High}} \\
\bottomrule
\end{tabular}
\end{table}

% --- Section 6: Recommendations ---
\section{Recommendations}
Based on the consolidated risk assessment, the following actions are recommended to improve the security posture of \textbf{Great Lakes}. Recommendations are prioritized to address the most critical risks first.

\begin{enumerate}
    \item \textbf{[Immediate] Remediate RDP Exposure (RISK-001):}
    \begin{itemize}
        \item Immediately close or firewall port 3389 on all affected systems, including \texttt{10.10.10.51} and \texttt{10.10.10.50}, from all non-essential sources.
        \item For necessary remote access, implement a Virtual Private Network (VPN) with MFA as the sole method for connecting to internal resources like RDP.
        \item Conduct a network-wide audit to identify and remediate any other instances of exposed management services.
    \end{itemize}
    \vspace{0.5cm}
    \item \textbf{[High Priority] Implement Comprehensive MFA (RISK-002):}
    \begin{itemize}
        \item Procure and deploy an MFA solution for all employee computer logins (Windows, macOS, etc.).
        \item Enforce MFA for access to all applications and repositories containing sensitive or critical business data.
        \item This action directly mitigates the risk of credential theft leading to a breach.
    \end{itemize}
    \vspace{0.5cm}
    \item \textbf{[Medium Priority] Establish Security Governance (RISK-003):}
    \begin{itemize}
        \item Develop and ratify a formal Employee Acceptable Use Policy (AUP). This policy should be signed by all new and existing employees.
        \item Institute a mandatory annual security awareness training program for all employees. The training should cover current threats such as phishing, password hygiene, and social engineering.
    \end{itemize}
\end{enumerate}

\end{document}
```