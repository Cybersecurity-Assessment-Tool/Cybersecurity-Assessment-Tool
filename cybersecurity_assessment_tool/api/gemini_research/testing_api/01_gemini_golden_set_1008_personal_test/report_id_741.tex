```latex
\documentclass[12pt]{article}

% Preamble: Required Packages
\usepackage[margin=1in]{geometry}
\usepackage{pifont}
\usepackage{booktabs}
\usepackage{hyperref}
\usepackage{url}
\usepackage{seqsplit}

% Document Metadata
\title{Cybersecurity Posture Assessment Report}
\author{Cybersecurity Analyst}
\date{\today}

% Hyperref Setup
\hypersetup{
    colorlinks=true,
    linkcolor=black,
    urlcolor=blue,
    pdftitle={Cybersecurity Posture Assessment Report},
    pdfauthor={Cybersecurity Analyst},
}

\begin{document}

\maketitle
\thispagestyle{empty}
\newpage
\tableofcontents
\newpage

% --- 1. Executive Summary ---
\section{Executive Summary}
This report provides a comprehensive cybersecurity assessment for \textbf{Hidden Gem}, synthesizing data from an external network scan, a security controls questionnaire, and a review of pre-existing risk documentation.

The assessment reveals a mixed security posture. While the organization has implemented foundational controls such as Multi-Factor Authentication (MFA) for email and computer access, critical deficiencies were identified that present a significant risk.

The most severe finding is the discovery of an openly accessible web service on port 8080 of an internal system (\texttt{10.5.5.5}) with the title \textbf{"TOP SECRET DB"}. This directly contradicts existing risk documentation which incorrectly labeled this port as a secure false positive. This finding, combined with the policy gap of not requiring MFA for sensitive data systems, creates a critical vulnerability that could lead to a major data breach.

Furthermore, significant gaps exist in employee security governance, including the absence of an acceptable use policy and a lack of security awareness training for new hires. These issues weaken the organization's human firewall and increase susceptibility to social engineering and insider threats.

Immediate remediation is required to secure the exposed database and to address the identified policy and training gaps.

% --- 2. Organizational Information ---
\section{Organizational Information}
The following details were provided for the assessment.

\begin{tabular}{@{}ll}
\toprule
\textbf{Attribute} & \textbf{Value} \\
\midrule
Organization Name & \textbf{Hidden Gem} \\
Email Domain & \texttt{HiddenGem.org} \\
Website Domain & \url{www.HiddenGem.org} \\
External IP Address & \texttt{89.102.232.176} \\
\bottomrule
\end{tabular}

% --- 3. Security Control Review ---
\section{Security Control Review}
A review of the organization's security controls was conducted via a questionnaire. The responses indicate key areas of weakness, particularly in access control for sensitive systems and employee security governance. "No" answers represent significant control gaps.

\begin{table}[h!]
\centering
\caption{Security Controls Questionnaire Results}
\begin{tabular}{@{}p{0.8\linewidth}c@{}}
\toprule
\textbf{Control Question} & \textbf{Response} \\
\midrule
Do you require MFA to access email? & \ding{51} \\ % Yes
Do you require MFA to log into computers? & \ding{51} \\ % Yes
\textbf{Do you require MFA to access sensitive data systems?} & \textbf{\ding{55}} \\ % No - CRITICAL
\textbf{Does your organization have an employee acceptable use policy?} & \textbf{\ding{55}} \\ % No - HIGH RISK
\textbf{Does your organization do security awareness training for new employees?} & \textbf{\ding{55}} \\ % No - HIGH RISK
Does your organization do security awareness training for all employees at least once per year? & \ding{51} \\ % Yes
\bottomrule
\end{tabular}
\end{table}

% --- 4. Technical Scan Results ---
\section{Technical Scan Results}
An Nmap scan was performed on the internal network to identify active services. A critical finding was identified on the target system.

\begin{itemize}
    \item \textbf{Target IP Address:} \texttt{10.5.5.5}
    \item \textbf{Scan Status:} Host is up.
\end{itemize}

The following open port was discovered:

\begin{table}[h!]
\centering
\caption{Open Port Analysis}
\begin{tabular}{@{}llll@{}}
\toprule
\textbf{Port} & \textbf{State} & \textbf{Service/Product} & \textbf{Details} \\
\midrule
8080/tcp & open & http & HTTP Title: \textbf{TOP SECRET DB} \\
\bottomrule
\end{tabular}
\end{table}

\subsection{Analysis of Technical Findings}
The service on port 8080 presents a severe information disclosure risk. The title "TOP SECRET DB" strongly suggests that a sensitive, possibly confidential, database is accessible via this port. This finding is particularly alarming because it contradicts the existing risk documentation (from Input 3), which stated this port was a "confirmed secure" false positive. This indicates a failure in the previous risk assessment process.

% --- 5. Correlated Risk Assessment ---
\section{Correlated Risk Assessment}
By correlating the technical scan results with the security control gaps and existing risk data, we have identified the following high-priority risks.

\begin{table}[h!]
\centering
\caption{Summary of Identified Risks}
\begin{tabular}{@{}p{0.25\linewidth}p{0.5\linewidth}l@{}}
\toprule
\textbf{Risk Name} & \textbf{Overview} & \textbf{Severity} \\
\midrule
\textbf{Exposed Sensitive Database} & An open port (8080) on an internal server reveals a service titled "TOP SECRET DB". This is compounded by the lack of mandatory MFA for sensitive systems, creating a direct path for unauthorized data access. & \textbf{Critical} \\
\textbf{Inadequate Employee Governance} & The absence of an acceptable use policy and security training for new hires creates an environment where employees are unaware of security expectations and are more vulnerable to attacks. & \textbf{High} \\
\textbf{Inaccurate Risk Management Process} & The current risk register incorrectly identified port 8080 as a secure false positive. This indicates a flawed validation process, suggesting other critical risks may be undocumented or misclassified. & \textbf{High} \\
\bottomrule
\end{tabular}
\end{table}

% --- 6. Recommendations ---
\section{Recommendations}
The following actions are recommended to mitigate the identified risks and improve the overall security posture of \textbf{Hidden Gem}.

\subsection{Immediate Actions (0-7 Days)}
\begin{enumerate}
    \item \textbf{Secure Exposed Database:} Immediately investigate the service running on \texttt{10.5.5.5:8080}.
    \begin{itemize}
        \item If the service is not essential, disable it.
        \item If essential, restrict access to authorized personnel only via firewall rules.
        \item Implement strong authentication and enforce MFA for access.
        \item Remove or change the revealing HTTP title "TOP SECRET DB".
    \end{itemize}
\end{enumerate}

\subsection{High-Priority Actions (1-3 Months)}
\begin{enumerate}
    \item \textbf{Enforce MFA on Sensitive Systems:} Expand the MFA policy to cover all systems that store, process, or transmit sensitive organizational data. This is the most effective control to mitigate the risk of unauthorized access.
    \item \textbf{Develop Acceptable Use Policy (AUP):} Create and implement a formal AUP that clearly defines the rules and expectations for employee use of company technology and data. All employees must read and acknowledge this policy.
    \item \textbf{Implement Onboarding Security Training:} Develop a mandatory security awareness training module for all new employees as part of the onboarding process. This should cover key topics like phishing, password security, and the new AUP.
\end{enumerate}

\subsection{Strategic Actions (3-6 Months)}
\begin{enumerate}
    \item \textbf{Review and Validate Risk Register:} Conduct a full review of the organization's risk register. All documented risks, especially those marked as low-risk or false positives, should be re-validated with technical testing to ensure accuracy.
\end{enumerate}

\end{document}
```