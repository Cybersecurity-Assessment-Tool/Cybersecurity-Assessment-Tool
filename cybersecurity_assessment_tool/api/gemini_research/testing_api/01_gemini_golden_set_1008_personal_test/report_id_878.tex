```latex
\documentclass[12pt]{article}

% --- PACKAGES ---
\usepackage[margin=1in]{geometry}
\usepackage{pifont} % For checkmarks and crosses
\usepackage{booktabs} % For professional tables
\usepackage{hyperref} % For clickable links
\usepackage{url}      % For URL formatting
\usepackage{seqsplit} % To split long strings in tt font
\usepackage{graphicx} % For logo (placeholder)

% --- DOCUMENT METADATA ---
\hypersetup{
    colorlinks=true,
    linkcolor=black,
    urlcolor=blue,
    pdftitle={Cybersecurity Posture Assessment Report},
    pdfauthor={Cybersecurity Analysis Division},
    pdfsubject={Security Assessment},
    pdfkeywords={Security, Analysis, Report}
}

% --- CUSTOM COMMANDS ---
\newcommand{\yes}{\ding{51}} % Checkmark
\newcommand{\no}{\ding{55}}  % X-mark

% --- DOCUMENT START ---
\begin{document}

% --- TITLE PAGE ---
\begin{titlepage}
    \centering
    \vspace*{1cm}
    
    \Huge
    \textbf{Cybersecurity Posture Assessment Report}
    
    \vspace{1.5cm}
    
    \Large
    Prepared for: \\
    \vspace{0.5cm}
    \textbf{Aetheric Systems}
    
    \vspace{2cm}
    
    \large
    \textbf{Date of Report:} \today \\
    \textbf{Date of Scan:} 2023-10-27 % Assuming a date for the scan
    
    \vfill
    
    \large
    \textbf{CONFIDENTIAL} \\
    \textit{This document contains sensitive information. Distribution is restricted.}
    
\end{titlepage}

\tableofcontents
\newpage

% --- EXECUTIVE SUMMARY ---
\section*{Executive Summary}

This report details the findings of a cybersecurity assessment conducted for \textbf{Aetheric Systems}. The assessment combined a technical network scan, a review of existing risks, and an analysis of organizational security controls based on a questionnaire.

The analysis revealed several high-risk and critical vulnerabilities that require immediate attention. A key technical finding is a publicly accessible FTP server running a critically outdated and vulnerable version of \texttt{vsftpd} (2.3.4), which is known to have a backdoor (CVE-2011-2523). This service is also misconfigured to allow anonymous logins, posing a severe and immediate threat to data integrity and network security.

Furthermore, the organizational controls review identified significant gaps in access management. The lack of multi-factor authentication (MFA) for email and computer access drastically increases the risk of account compromise and unauthorized access. These control gaps, combined with the technical vulnerabilities and pre-existing risks like outdated operating systems, create a high-risk environment.

Immediate remediation should focus on securing the vulnerable FTP server and implementing MFA across critical systems. A detailed breakdown of all findings and prioritized recommendations is provided in the subsequent sections.

% --- ORGANIZATIONAL INFORMATION ---
\section*{1. Organizational Information}

This section provides the organizational details as provided for this assessment.

\begin{tabular}{@{}ll}
    \toprule
    \textbf{Attribute} & \textbf{Value} \\
    \midrule
    Organization Name & \textbf{Aetheric Systems} \\
    Email Domain & \texttt{AethericSystems.org} \\
    Website Domain & \url{www.AethericSystems.org} \\
    External IP Address & \texttt{97.120.250.126} \\
    \bottomrule
\end{tabular}

% --- SECURITY CONTROL REVIEW ---
\section*{2. Security Control Review}

The following table summarizes the organization's responses to the security controls questionnaire. Items marked with a \no\ represent significant gaps in the security posture.

\begin{table}[h!]
\centering
\begin{tabular}{@{}p{0.8\linewidth}c@{}}
    \toprule
    \textbf{Security Control Question} & \textbf{Status} \\
    \midrule
    Do you require MFA to access email? & \no \\
    Do you require MFA to log into computers? & \no \\
    Do you require MFA to access sensitive data systems? & \yes \\
    Does your organization have an employee acceptable use policy? & \yes \\
    Does your organization do security awareness training for new employees? & \yes \\
    Does your organization do security awareness training for all employees at least once per year? & \yes \\
    \bottomrule
\end{tabular}
\caption{Security Controls Questionnaire Results}
\end{label{tab:controls}
\end{table}

\subsection*{Analysis of Control Gaps}
The review identified two critical control gaps:
\begin{itemize}
    \item \textbf{No MFA for Email Access:} Email is a primary target for phishing attacks and account takeovers. Without MFA, a compromised password is all an attacker needs to gain access to sensitive communications and potentially pivot to other systems.
    \item \textbf{No MFA for Computer Logins:} Lack of MFA on workstations and servers allows an attacker with stolen credentials to easily gain a foothold within the internal network, escalate privileges, and deploy malware such as ransomware.
\end{itemize}
While it is positive that MFA is used for sensitive data systems, the lack of it on primary entry points like email and computers undermines this control.

% --- TECHNICAL SCAN RESULTS ---
\section*{3. Technical Scan Results}

A network scan was performed on the target system to identify open ports and exposed services.

\begin{itemize}
    \item \textbf{Target IP Address:} \texttt{10.0.0.15}
\end{itemize}

\begin{table}[h!]
\centering
\begin{tabular}{@{}lllll@{}}
    \toprule
    \textbf{Port} & \textbf{State} & \textbf{Service} & \textbf{Version} & \textbf{Notes} \\
    \midrule
    21/tcp & Open & FTP & vsftpd 2.3.4 & \begin{tabular}[t]{@{}l@{}}\textbf{Critical Vulnerability}\\ Anonymous FTP login allowed \\ Version is vulnerable to a backdoor\\ (CVE-2011-2523)\end{tabular} \\
    \bottomrule
\end{tabular}
\caption{Open Port and Service Analysis}
\label{tab:scan}
\end{table}

\subsection*{Analysis of Technical Findings}
The scan identified a critically vulnerable File Transfer Protocol (FTP) service.
\begin{itemize}
    \item \textbf{Vulnerable Software (CVE-2011-2523):} The version \texttt{vsftpd 2.3.4} contains a well-documented backdoor that was inserted into the source code. If exploited, this vulnerability allows an attacker to execute arbitrary commands on the server with root-level privileges, leading to a complete system compromise.
    \item \textbf{Anonymous Login Enabled:} The FTP server is configured to allow anonymous logins. This allows any user on the network to connect to the server without authentication, posing a significant risk of unauthorized data access, modification, or upload of malicious files.
\end{itemize}

% --- RISK ASSESSMENT SUMMARY ---
\section*{4. Risk Assessment Summary}

This section correlates the findings from the security control review, technical scan, and pre-existing risk register.

\begin{table}[h!]
\centering
\begin{tabular}{@{}p{0.3\linewidth}p{0.5\linewidth}l@{}}
    \toprule
    \textbf{Risk Name} & \textbf{Overview} & \textbf{Severity} \\
    \midrule
    \textbf{Vulnerable FTP Server} & An FTP server is running \texttt{vsftpd 2.3.4}, which is vulnerable to a remote code execution backdoor (CVE-2011-2523) and allows anonymous login. & \textbf{Critical} \\
    \addlinespace
    \textbf{Lack of MFA on Core Systems} & Multi-factor authentication is not enforced for email or computer logins, making credential theft and account takeover attacks highly likely to succeed. & \textbf{High} \\
    \addlinespace
    \textbf{Outdated Windows Policy} & (Pre-existing risk) Computers are running Windows 7, which is an end-of-life operating system no longer receiving security updates. & \textbf{Medium} \\
    \bottomrule
\end{tabular}
\caption{Consolidated Risk Register}
\label{tab:risks}
\end{table}

% --- RECOMMENDATIONS ---
\section*{5. Recommendations}

Based on the identified risks, the following prioritized recommendations are provided to improve the security posture of \textbf{Aetheric Systems}.

\subsection*{Immediate Priority (0-7 Days)}
\begin{enumerate}
    \item \textbf{Remediate Vulnerable FTP Server:}
        \begin{itemize}
            \item Immediately take the FTP server offline (\texttt{10.0.0.15:21}).
            \item If the service is business-critical, upgrade \texttt{vsftpd} to the latest stable version.
            \item Disable anonymous FTP login. Access should be restricted to authenticated users only.
            \item Consider replacing FTP with a more secure protocol like SFTP (SSH File Transfer Protocol).
        \end{itemize}
\end{enumerate}

\subsection*{High Priority (1-4 Weeks)}
\begin{enumerate}
    \item \textbf{Implement Multi-Factor Authentication (MFA):}
        \begin{itemize}
            \item Enforce MFA for all users on the email system (\texttt{AethericSystems.org}).
            \item Deploy and enforce MFA for all workstation and server logins, for both standard and administrative users.
        \end{itemize}
\end{enumerate}

\subsection*{Medium Priority (1-3 Months)}
\begin{enumerate}
    \item \textbf{Address Outdated Operating Systems:}
        \begin{itemize}
            \item Develop and execute a plan to upgrade all workstations running Windows 7 to a modern, supported operating system like Windows 10 or 11.
            \item Isolate any systems that cannot be immediately upgraded from the main network to limit their exposure.
        \end{itemize}
\end{enumerate}

% --- DOCUMENT END ---
\end{document}
```