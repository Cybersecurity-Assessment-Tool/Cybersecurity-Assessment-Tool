```latex
\documentclass[12pt]{article}

% Preamble: Required Packages
\usepackage[margin=1in]{geometry}
\usepackage{pifont} % For checkmarks and crosses
\usepackage{booktabs} % For professional tables
\usepackage{hyperref} % For clickable links
\usepackage{url} % For URL formatting
\usepackage{seqsplit} % For splitting long strings like IPs

% Document Metadata
\title{Cybersecurity Posture Assessment Report}
\author{Cybersecurity Analyst}
\date{\today}

\begin{document}

\maketitle
\thispagestyle{empty}
\newpage
\tableofcontents
\newpage

% --- 1. Executive Summary ---
\section{Executive Summary}

This report provides a cybersecurity assessment for Paper Plane Publishing, synthesizing data from a network scan, an organizational security questionnaire, and a review of pre-existing risks.

The assessment reveals a mixed security posture. The organization demonstrates strong foundational controls in identity and access management, with consistent enforcement of Multi-Factor Authentication (MFA) across key systems. Security awareness training programs are also well-established for both new and existing employees.

However, two key areas of risk were identified. A critical administrative gap exists due to the lack of an employee Acceptable Use Policy (AUP). This exposes the organization to insider threats and inconsistent security practices. Additionally, a technical scan revealed an exposed Secure Shell (SSH) service on an external-facing IPv6 address, which presents a potential vector for unauthorized access if not properly secured.

Immediate recommendations focus on developing and implementing a formal AUP and restricting access to the exposed SSH service to only trusted sources.

% --- 2. Organizational Information ---
\section{Organizational Information}

The following information was provided for the assessment.

\begin{itemize}
    \item \textbf{Organization Name:} Paper Plane Publishing
    \item \textbf{Email Domain:} \texttt{PaperPlanePublishing.com}
    \item \textbf{Website Domain:} \url{www.PaperPlanePublishing.com}
    \item \textbf{Primary External IP:} \texttt{32.245.127.188}
\end{itemize}

% --- 3. Security Control Review ---
\section{Security Control Review}

A review of the organization's administrative and technical security controls was conducted via a questionnaire. The results indicate a strong commitment to MFA and employee training, but highlight a critical gap in policy documentation.

\begin{table}[h!]
\centering
\caption{Security Questionnaire Responses}
\begin{tabular}{@{}lc@{}}
\toprule
\textbf{Control Question} & \textbf{Response} \\
\midrule
Do you require MFA to access email? & \ding{51} \\ % Yes
Do you require MFA to log into computers? & \ding{51} \\ % Yes
Do you require MFA to access sensitive data systems? & \ding{51} \\ % Yes
Does your organization have an employee acceptable use policy? & \textbf{\color{red}\ding{55}} \\ % No
Does your organization do security awareness training for new employees? & \ding{51} \\ % Yes
Does your organization do security awareness training for all employees at least once per year? & \ding{51} \\ % Yes
\bottomrule
\end{tabular}
\end{table}

\subsection*{Analysis}
The absence of an \textbf{Acceptable Use Policy (AUP)} is a significant finding. An AUP is a foundational document that defines how employees may use company IT assets, protecting the organization from legal and security risks. Without it, there is no formal standard for user behavior, data handling, or consequences for misuse. This has been classified as a high-severity risk.

% --- 4. Technical Scan Results ---
\section{Technical Scan Results}

An external network scan was performed to identify open ports and exposed services on the organization's perimeter.

\begin{itemize}
    \item \textbf{Target IP Address:} \seqsplit{\texttt{2001:db8::1}}
    \item \textbf{Host Status:} Up
\end{itemize}

The following open ports were discovered on the target system:

\begin{table}[h!]
\centering
\caption{Open Port Analysis}
\begin{tabular}{@{}llll@{}}
\toprule
\textbf{Port} & \textbf{State} & \textbf{Service} & \textbf{Notes} \\
\midrule
22/tcp & open & SSH (Secure Shell) & Standard port for remote administration. \\
\bottomrule
\end{tabular}
\end{table}

\subsection*{Analysis}
The scan identified that port 22, commonly used for the Secure Shell (SSH) protocol, is open to the public internet. While SSH is essential for remote system administration, its exposure is a security risk. It becomes a primary target for automated brute-force attacks attempting to guess credentials. Detailed service and version information was not available in the scan data, preventing an analysis for specific known vulnerabilities. However, the exposure itself constitutes a medium-severity risk.

% --- 5. Risk Assessment Summary ---
\section{Risk Assessment Summary}

This section correlates findings from the security control review and the technical scan. No pre-existing vulnerabilities were reported.

\begin{table}[h!]
\centering
\caption{Identified Risks}
\begin{tabular}{@{}lp{6cm}l@{}}
\toprule
\textbf{Risk ID} & \textbf{Risk Name \& Description} & \textbf{Severity} \\
\midrule
\textbf{RISK-001} & \textbf{Lack of Acceptable Use Policy:} The organization does not have a formal policy governing the use of corporate IT assets. This increases the risk of insider threat, data misuse, and inconsistent security practices. & \textbf{High} \\
\noalign{\smallskip}
\textbf{RISK-002} & \textbf{Exposed SSH Service:} Port 22 (SSH) is accessible from the public internet, making it a target for brute-force login attempts and potential exploitation if a vulnerable version is in use. & \textbf{Medium} \\
\bottomrule
\end{tabular}
\end{table}

% --- 6. Recommendations ---
\section{Recommendations}

Based on the identified risks, the following actions are recommended to improve the organization's security posture.

\subsection*{RISK-001: Lack of Acceptable Use Policy (High)}
\begin{enumerate}
    \item \textbf{Develop and Implement an AUP:} Draft a comprehensive Acceptable Use Policy that clearly outlines rules for computer, network, email, and internet usage. The policy should specify prohibited activities and the consequences for violation.
    \item \textbf{Employee Training and Acknowledgment:} Distribute the new policy to all employees and require them to formally acknowledge that they have read and understood it. Incorporate AUP review into the new-hire onboarding process.
\end{enumerate}

\subsection*{RISK-002: Exposed SSH Service (Medium)}
\begin{enumerate}
    \item \textbf{Restrict Access with Firewall Rules:} If remote access is required, configure perimeter firewall rules to allow SSH connections (port 22) only from specific, trusted IP addresses (e.g., administrator locations or a corporate VPN).
    \item \textbf{Enforce Secure Configuration:} Ensure the SSH service is securely configured by:
    \begin{itemize}
        \item Disabling direct root login.
        \item Enforcing the use of strong, key-based authentication instead of passwords.
        \item Keeping the SSH server software up-to-date to patch known vulnerabilities.
    \end{itemize}
\end{enumerate}

\end{document}
```