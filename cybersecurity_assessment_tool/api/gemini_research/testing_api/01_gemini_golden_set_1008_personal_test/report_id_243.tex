```latex
\documentclass[12pt, a4paper]{article}

% Preamble: Required Packages
\usepackage[margin=1in]{geometry}
\usepackage{pifont} % For checkmarks and crosses
\usepackage{booktabs} % For professional tables
\usepackage{hyperref} % For clickable links
\usepackage{url} % For URL formatting
\usepackage{seqsplit} % To split long strings in texttt
\usepackage{graphicx} % For logos, etc.
\usepackage{xcolor} % For custom colors
\usepackage{fancyhdr} % For headers and footers
\usepackage{lastpage} % To get the total number of pages

% --- Document Setup ---

% Define colors for severity
\definecolor{critical}{HTML}{990000}
\definecolor{high}{HTML}{D14302}
\definecolor{medium}{HTML}{E5A50A}
\definecolor{low}{HTML}{3A7D34}

% Hyperref setup
\hypersetup{
    colorlinks=true,
    linkcolor=blue,
    filecolor=magenta,      
    urlcolor=cyan,
    pdftitle={Cybersecurity Risk Assessment Report},
    pdfpagemode=FullScreen,
}

% Header and Footer
\pagestyle{fancy}
\fancyhf{} % Clear all header and footer fields
\fancyhead[L]{Cybersecurity Risk Assessment Report}
\fancyhead[R]{Falcon Heavy}
\fancyfoot[C]{\thepage\ of \pageref{LastPage}}
\renewcommand{\headrulewidth}{0.4pt}
\renewcommand{\footrulewidth}{0.4pt}

% --- Document Start ---

\begin{document}

\title{
    \vspace{2cm}
    \textbf{Cybersecurity Risk Assessment Report}\\
    \large Prepared for: Falcon Heavy
    \vspace{1.5cm}
}
\author{Cybersecurity Analysis Division}
\date{\today}

\maketitle
\thispagestyle{empty}
\newpage

\tableofcontents
\newpage

% --- 1. Executive Summary ---
\section*{Executive Summary}
This report provides a comprehensive cybersecurity risk assessment for Falcon Heavy, based on an analysis of network scan data, organizational security controls, and pre-existing risk information. The assessment was conducted on \today.

Overall, Falcon Heavy has implemented several important foundational security controls, including Multi-Factor Authentication (MFA) for email and computer access, as well as a consistent security awareness training program. These are commendable measures that reduce significant areas of risk.

However, the assessment identified several critical and high-risk vulnerabilities that require immediate attention. The most severe finding is an externally exposed MySQL database service. This service is not only accessible from the network but is also running an End-of-Life (EOL) version (5.7.33) that no longer receives security updates. This exposure is significantly amplified by a corresponding policy gap: the lack of mandatory MFA for accessing sensitive data systems.

Furthermore, the absence of a formal Acceptable Use Policy (AUP) represents a significant administrative gap that can lead to inconsistent security practices and insider threats.

This report details these findings and provides actionable recommendations to mitigate the identified risks, strengthen the organization's security posture, and protect critical data assets.

% --- 2. Organizational Information ---
\section{Organizational Information}
This section provides a summary of the organizational details used as a basis for this assessment.

\begin{tabular}{@{}ll}
\toprule
\textbf{Attribute} & \textbf{Value} \\
\midrule
Organization Name & Falcon Heavy \\
Email Domain & \texttt{FalconHeavy.net} \\
Website Domain & \url{www.FalconHeavy.net} \\
External IP Address & \seqsplit{\texttt{12.99.114.115}} \\
\bottomrule
\end{tabular}

% --- 3. Security Control Review ---
\section{Security Control Review}
The following table summarizes the organization's responses to a security controls questionnaire. This review helps identify gaps in administrative and policy-based security measures. A checkmark (\ding{51}) indicates a positive control is in place, while a cross (\ding{55}) indicates a potential gap.

\begin{table}[h!]
\centering
\begin{tabular}{@{}p{8cm}ccp{3cm}@{}}
\toprule
\textbf{Control Question} & \textbf{Response} & \textbf{Assessment} \\
\midrule
Do you require MFA to access email? & Yes & \ding{51} \\
Do you require MFA to log into computers? & Yes & \ding{51} \\
Do you require MFA to access sensitive data systems? & No & \textcolor{critical}{\ding{55}} \textbf{Critical Gap} \\
Does your organization have an employee acceptable use policy? & No & \textcolor{high}{\ding{55}} \textbf{High Risk} \\
Does your organization do security awareness training for new employees? & Yes & \ding{51} \\
Does your organization do security awareness training for all employees at least once per year? & Yes & \ding{51} \\
\bottomrule
\end{tabular}
\caption{Security Controls Questionnaire Analysis}
\end{table}

% --- 4. Technical Scan Results ---
\section{Technical Scan Results}
A network scan was performed on the specified target to identify open ports and exposed services.

\begin{itemize}
    \item \textbf{Target IP Address:} \texttt{172.16.50.20}
    \item \textbf{Scan Date:} Scan data processed on \today
\end{itemize}

\subsection{Open Ports}
The following table details the services discovered during the network scan.

\begin{table}[h!]
\centering
\begin{tabular}{@{}lllll@{}}
\toprule
\textbf{Port} & \textbf{State} & \textbf{Service} & \textbf{Product} & \textbf{Version} \\
\midrule
3306/tcp & open & mysql & MySQL & 5.7.33 \\
\bottomrule
\end{tabular}
\caption{Discovered Network Services}
\end{table}

\subsection{Technical Analysis}
The scan identified a critical exposure:
\begin{itemize}
    \item \textbf{Exposed MySQL Database (Port 3306):} Database services should not be directly exposed to untrusted networks. This configuration allows attackers to directly target the database for brute-force attacks, credential stuffing, or exploitation of vulnerabilities.
    \item \textbf{End-of-Life Software:} The detected version, \textbf{MySQL 5.7.33}, reached its official End of Life (EOL) in October 2023. This means it no longer receives security patches from the vendor, and any newly discovered vulnerabilities will remain unpatched, posing a severe and unmitigable risk to the data it contains.
\end{itemize}

% --- 5. Consolidated Risk Assessment ---
\section{Consolidated Risk Assessment}
This section synthesizes findings from the security control review, technical scan, and pre-existing risk data into a consolidated list of key risks.

\begin{table}[h!]
\centering
\begin{tabular}{@{}p{1.5cm}p{3.5cm}p{6.5cm}c@{}}
\toprule
\textbf{Risk ID} & \textbf{Risk Name} & \textbf{Description} & \textbf{Severity} \\
\midrule
\textbf{RISK-001} & Exposed End-of-Life Database Service & MySQL port 3306 is open to the network, and the service version (5.7.33) is past its End of Life, leaving it vulnerable to unpatched exploits. & \textcolor{critical}{\textbf{Critical}} \\
\addlinespace
\textbf{RISK-002} & Lack of MFA for Sensitive Systems & The absence of MFA on sensitive systems, including the exposed database, dramatically increases the risk of unauthorized access via compromised credentials. & \textcolor{high}{\textbf{High}} \\
\addlinespace
\textbf{RISK-003} & Missing Acceptable Use Policy & The lack of a formal AUP creates ambiguity regarding the proper use of company assets and data, increasing the risk of insider threat and non-compliance. & \textcolor{medium}{\textbf{Medium}} \\
\bottomrule
\end{tabular}
\caption{Summary of Identified Risks}
\end{table}

% --- 6. Recommendations ---
\section{Recommendations}
The following actions are recommended to mitigate the identified risks and improve the overall security posture of Falcon Heavy.

\subsection{RISK-001: Exposed End-of-Life Database Service}
\begin{itemize}
    \item \textbf{Immediate (Containment):} Implement strict firewall rules to block all access to TCP port 3306 from the internet. Access should be restricted to a minimal set of trusted internal IP addresses only.
    \item \textbf{Short-Term (Remediation):} Develop and execute a plan to upgrade the MySQL 5.7 database to a currently supported version (e.g., MySQL 8.x). This is critical for receiving future security patches.
    \item \textbf{Long-Term (Architectural):} Implement a secure remote access solution, such as a VPN or a bastion host (jump box), for all administrative access to database infrastructure. This removes the need for any direct public exposure.
\end{itemize}

\subsection{RISK-002: Lack of MFA for Sensitive Systems}
\begin{itemize}
    \item \textbf{Immediate (Scoping):} Conduct an inventory of all systems that store or process sensitive data to identify all assets that require MFA protection.
    \item \textbf{Short-Term (Implementation):} Procure and deploy an MFA solution for all identified sensitive systems, prioritizing the database infrastructure. Enforce its use for all privileged and administrative accounts.
\end{itemize}

\subsection{RISK-003: Missing Acceptable Use Policy}
\begin{itemize}
    \item \textbf{Short-Term (Development):} Draft a comprehensive Acceptable Use Policy (AUP) that clearly defines the rules and responsibilities for all employees and contractors when using company IT assets.
    \item \textbf{Ongoing (Enforcement):} Integrate the AUP into the new employee onboarding process. Require all current employees to read and acknowledge the policy, and conduct annual reviews to ensure it remains current.
\end{itemize}

\end{document}
```