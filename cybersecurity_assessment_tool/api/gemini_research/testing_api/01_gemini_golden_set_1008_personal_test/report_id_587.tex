```latex
\documentclass[12pt, a4paper]{article}

% Preamble: Required Packages
\usepackage[margin=1in]{geometry}
\usepackage{pifont} % For checkmarks and crosses
\usepackage{booktabs} % For professional tables
\usepackage{hyperref} % For hyperlinks
\usepackage{url} % For URL formatting
\usepackage{seqsplit} % For splitting long strings
\usepackage{graphicx}
\usepackage{xcolor}

% Document Information
\title{Cybersecurity Assessment Report}
\author{Cybersecurity Analysis Division}
\date{\today}

% Hyperref Setup
\hypersetup{
    colorlinks=true,
    linkcolor=blue,
    filecolor=magenta,      
    urlcolor=cyan,
    pdftitle={Cybersecurity Assessment Report},
    pdfpagemode=FullScreen,
}

\begin{document}

\maketitle
\thispagestyle{empty}
\newpage
\tableofcontents
\newpage

% --- 1. Executive Overview ---
\section{Executive Overview}
This report provides a comprehensive cybersecurity assessment for \textbf{Cloud Nine Software}. The analysis is based on a technical network scan, a review of organizational security controls, and an evaluation of pre-existing risk data.

The assessment identified several critical and high-risk deficiencies. The most significant concerns are policy-related, including the absence of Multi-Factor Authentication (MFA) for computer and sensitive data access, the lack of an employee acceptable use policy, and no security training for new hires. These gaps represent a substantial risk of unauthorized access and insider threat.

Technical analysis confirmed a pre-existing critical vulnerability, "Localhost Exposed," with an open SSH port (22) detected on the local loopback interface. While this is an internal-facing service, its exposure in certain contexts can be a significant security risk.

Immediate remediation should focus on implementing MFA across all critical assets and developing foundational security policies to establish a stronger security baseline.

% --- 2. Organizational Information ---
\section{Organizational Information}
The following details were provided for the assessment.

\begin{itemize}
    \item \textbf{Organization Name:} \textbf{Cloud Nine Software}
    \item \textbf{Email Domain:} \texttt{CloudNineSoftware.com}
    \item \textbf{Website Domain:} \texttt{www.CloudNineSoftware.com}
    \item \textbf{External IP Address:} \texttt{225.250.58.10}
\end{itemize}

% --- 3. Security Control Review ---
\section{Security Control Review}
A review of administrative and policy-based security controls was conducted based on a questionnaire. The responses indicate significant gaps in fundamental security practices. A checkmark (\ding{51}) indicates a positive control is in place, while a cross (\ding{55}) highlights a control gap.

\begin{table}[h!]
\centering
\caption{Organizational Security Control Status}
\label{tab:controls}
\begin{tabular}{p{0.7\textwidth}c}
\toprule
\textbf{Control Question} & \textbf{Response} \\
\midrule
Do you require MFA to access email? & \ding{51} \\
Do you require MFA to log into computers? & \textcolor{red}{\ding{55}} \\
Do you require MFA to access sensitive data systems? & \textcolor{red}{\ding{55}} \\
Does your organization have an employee acceptable use policy? & \textcolor{red}{\ding{55}} \\
Does your organization do security awareness training for new employees? & \textcolor{red}{\ding{55}} \\
Does your organization do security awareness training for all employees at least once per year? & \ding{51} \\
\bottomrule
\end{tabular}
\end{table}

% --- 4. Technical Scan Results ---
\section{Technical Scan Results}
A network scan was performed to identify open ports and services on the target system. The scan confirmed the presence of an open service which correlates with a known high-risk finding.

\begin{itemize}
    \item \textbf{Target IP:} \texttt{127.0.0.1}
    \item \textbf{Scan Date:} [Scan Date Not Provided]
\end{itemize}

\begin{table}[h!]
\centering
\caption{Open Ports Detected on \texttt{127.0.0.1}}
\label{tab:ports}
\begin{tabular}{llll}
\toprule
\textbf{Port} & \textbf{State} & \textbf{Service (Inferred)} & \textbf{Notes} \\
\midrule
22/tcp & open & SSH & Secure Shell access. Correlates with the \\
& & & "Localhost Exposed" pre-existing risk. \\
\bottomrule
\end{tabular}
\end{table}

\noindent No detailed service, product, or version information was available from the provided scan data.

% --- 5. Synthesized Risk Assessment ---
\section{Synthesized Risk Assessment}
The following table synthesizes findings from the security control review, technical scan, and pre-existing risk data into a consolidated list of identified risks.

\begin{table}[h!]
\centering
\caption{Consolidated Risk Register}
\label{tab:risks}
\begin{tabular}{p{0.25\textwidth}p{0.5\textwidth}l}
\toprule
\textbf{Risk Title} & \textbf{Description} & \textbf{Severity} \\
\midrule
\textbf{Localhost Exposed} & Port 22 (SSH) is open on the localhost interface (\texttt{127.0.0.1}). This was identified as a pre-existing risk and confirmed by the network scan. & \textbf{Critical (10.0)} \\
\addlinespace
\textbf{Lack of MFA on Endpoints} & MFA is not required for computer logins. This significantly increases the risk of unauthorized access from compromised credentials. & \textbf{Critical} \\
\addlinespace
\textbf{Lack of MFA on Sensitive Systems} & MFA is not enforced for accessing sensitive data systems, leaving critical information vulnerable to credential-based attacks. & \textbf{Critical} \\
\addlinespace
\textbf{Missing Acceptable Use Policy (AUP)} & The organization lacks a formal AUP, leading to ambiguity in employee responsibilities and acceptable behavior regarding company assets. & \textbf{High} \\
\addlinespace
\textbf{Inadequate New Hire Security Training} & New employees do not receive security awareness training, creating a window of vulnerability before they are included in the annual training cycle. & \textbf{High} \\
\bottomrule
\end{tabular}
\end{table}

% --- 6. Recommendations ---
\section{Recommendations}
Based on the synthesized risk assessment, the following actions are recommended to improve the organization's cybersecurity posture. Recommendations are prioritized by severity.

\subsection{Priority 1: Critical Risks}
\begin{enumerate}
    \item \textbf{Implement Comprehensive MFA:} Immediately deploy and enforce Multi-Factor Authentication (MFA) for all computer logins and for all systems that store or process sensitive data. This is the single most effective control to mitigate credential-based attacks.
    \item \textbf{Remediate Exposed Service:} Investigate the "Localhost Exposed" finding. Determine why the SSH service is running on the loopback interface and if it is necessary. If not, disable the service. If it is required, ensure it is properly configured and firewalled to prevent potential abuse or unauthorized access.
\end{enumerate}

\subsection{Priority 2: High Risks}
\begin{enumerate}
    \item \textbf{Develop and Implement an AUP:} Draft a formal Acceptable Use Policy (AUP) that clearly defines the rules and responsibilities for all employees when using company IT assets. This policy should be distributed to all current employees for acknowledgment and integrated into the new hire onboarding process.
    \item \textbf{Establish Onboarding Security Training:} Create a mandatory security awareness training module for all new employees. This training should be a required part of the onboarding process to ensure a baseline of security knowledge from day one.
\end{enumerate}

\end{document}
```