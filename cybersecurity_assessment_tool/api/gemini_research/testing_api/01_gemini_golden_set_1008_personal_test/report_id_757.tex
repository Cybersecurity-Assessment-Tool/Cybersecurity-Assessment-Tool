```latex
\documentclass[12pt]{article}

% Preamble: Required Packages
\usepackage[margin=1in]{geometry}
\usepackage{pifont} % For checkmarks and crosses
\usepackage{booktabs} % For professional tables
\usepackage{hyperref} % For clickable links
\usepackage{url} % For formatting URLs
\usepackage{seqsplit} % For splitting long strings in texttt
\usepackage[T1]{fontenc}

% Document Metadata
\title{Cybersecurity Assessment Report}
\author{Cybersecurity Analyst}
\date{November 22, 2025}

\begin{document}

\maketitle
\thispagestyle{empty}
\newpage
\tableofcontents
\thispagestyle{empty}
\newpage
\setcounter{page}{1}

% --- 1. Executive Summary ---
\section{Executive Summary}
This report details the findings of a cybersecurity assessment conducted on \textbf{Pacific Rim Exports}. The assessment combined a review of organizational security controls, an external network scan, and an analysis of pre-existing risks.

The analysis revealed a mixed security posture. The organization demonstrates strong identity and access management practices with consistent enforcement of Multi-Factor Authentication (MFA) and a solid security awareness training program.

However, two significant risks were identified that require immediate attention:
\begin{itemize}
    \item \textbf{Critical Governance Gap:} The absence of an employee Acceptable Use Policy (AUP) creates ambiguity regarding the proper use of company assets and data, exposing the organization to insider threats and potential legal liabilities.
    \item \textbf{High-Severity Technical Vulnerability:} The external-facing web server is running an outdated version of Nginx (1.18.0), which is known to have multiple publicly disclosed vulnerabilities. This exposes the server to potential compromise.
\end{itemize}

This report provides a detailed breakdown of these findings and offers actionable recommendations to mitigate the identified risks and improve the overall security posture of \textbf{Pacific Rim Exports}.

% --- 2. Organizational Information ---
\section{Organizational Information}
The following information was provided for the assessment.

\begin{tabular}{@{}ll}
    \toprule
    \textbf{Attribute} & \textbf{Value} \\
    \midrule
    Organization Name & \textbf{Pacific Rim Exports} \\
    Email Domain & \texttt{PacificRimExports.com} \\
    Website Domain & \url{www.PacificRimExports.com} \\
    External IP Address & \texttt{131.211.175.100} \\
    \bottomrule
\end{tabular}

% --- 3. Security Control Review ---
\section{Security Control Review}
A review of administrative and organizational security controls was conducted based on a standardized questionnaire. The results are summarized below. A checkmark (\ding{51}) indicates a positive control is in place, while a cross (\ding{55}) indicates a control gap.

\begin{table}[h!]
\centering
\begin{tabular}{@{}lc}
    \toprule
    \textbf{Control Question} & \textbf{Response} \\
    \midrule
    Do you require MFA to access email? & \ding{51} \\
    Do you require MFA to log into computers? & \ding{51} \\
    Do you require MFA to access sensitive data systems? & \ding{51} \\
    Does your organization have an employee acceptable use policy? & \color{red}\ding{55} \\
    Does your organization do security awareness training for new employees? & \ding{51} \\
    Does your organization do security awareness training for all employees annually? & \ding{51} \\
    \bottomrule
\end{tabular}
\caption{Organizational Security Control Status}
\end{table}

\subsection*{Analysis}
The organization has effectively implemented MFA across key systems and maintains a robust security awareness training schedule. However, the lack of an \textbf{Acceptable Use Policy (AUP)} is a critical administrative control gap. An AUP is essential for setting clear expectations for employees regarding the use of company systems and data, reducing the risk of misuse and providing a basis for disciplinary action if policies are violated.

% --- 4. Technical Scan Results ---
\section{Technical Scan Results}
An external network scan was performed to identify open ports and exposed services on the organization's infrastructure.

\begin{itemize}
    \item \textbf{Target IP Address:} \texttt{192.168.10.5}
    \item \textbf{Scan Date:} 2025-11-22
\end{itemize}

\begin{table}[h!]
\centering
\begin{tabular}{@{}lllll@{}}
    \toprule
    \textbf{Port} & \textbf{State} & \textbf{Service} & \textbf{Product} & \textbf{Version} \\
    \midrule
    443/tcp & open & https & nginx & 1.18.0 \\
    \bottomrule
\end{tabular}
\caption{Open Ports and Services Detected}
\end{table}

\subsection*{Analysis}
The scan identified a web server running \textbf{Nginx version 1.18.0}. This version was released in April 2020 and is now significantly outdated. The current stable version of Nginx has received numerous security patches and feature updates since 2020. Running this outdated version exposes the organization to a wide range of publicly known vulnerabilities (CVEs), which could be exploited by attackers to compromise the web server, deface the website, or gain unauthorized access to the internal network.

% --- 5. Risk Assessment ---
\section{Risk Assessment}
The following table synthesizes findings from the security control review and the technical scan. No pre-existing risks were reported.

\begin{table}[h!]
\centering
\begin{tabular}{@{}lp{5cm}l@{}}
    \toprule
    \textbf{Risk ID} & \textbf{Risk Name \& Description} & \textbf{Severity} \\
    \midrule
    RISK-001 & \textbf{Lack of Acceptable Use Policy} \newline An essential administrative control is missing, leading to undefined rules for employee use of IT assets. This increases the risk of data misuse, insider threats, and legal complications. & \textbf{High} \\
    \addlinespace
    RISK-002 & \textbf{Outdated Web Server Software} \newline The public-facing web server runs Nginx 1.18.0, a version with known vulnerabilities. This could allow an attacker to compromise the server, leading to data breach or service disruption. & \textbf{High} \\
    \bottomrule
\end{tabular}
\caption{Summary of Identified Risks}
\end{table}

% --- 6. Recommendations ---
\section{Recommendations}
To address the identified risks and strengthen the organization's security posture, the following actions are recommended:

\begin{enumerate}
    \item \textbf{Develop and Implement an Acceptable Use Policy (AUP):}
    \begin{itemize}
        \item \textbf{Action:} Draft a formal AUP that clearly defines the rules for using company networks, computers, email, and internet access.
        \item \textbf{Details:} The policy should be reviewed by HR and legal counsel, communicated to all employees, and formally acknowledged via signature. It should be integrated into the new employee onboarding process.
        \item \textbf{Priority:} \textbf{High}
    \end{itemize}
    \vspace{1em}
    \item \textbf{Upgrade Nginx and Implement Patch Management:}
    \begin{itemize}
        \item \textbf{Action:} Plan and execute an upgrade of the Nginx server from version 1.18.0 to the latest stable version recommended by the vendor.
        \item \textbf{Details:} Before upgrading the production server, test the new version in a staging environment to ensure compatibility with the website. Establish a formal patch management policy to ensure all external-facing software is updated in a timely manner.
        \item \textbf{Priority:} \textbf{High}
    \end{itemize}
\end{enumerate}

% --- 7. Conclusion ---
\section{Conclusion}
\textbf{Pacific Rim Exports} has established a strong foundation in identity management and security training. By addressing the critical policy gap and remediating the technical vulnerability identified in this report, the organization can significantly reduce its risk exposure and enhance its overall resilience against cyber threats.

\end{document}
```