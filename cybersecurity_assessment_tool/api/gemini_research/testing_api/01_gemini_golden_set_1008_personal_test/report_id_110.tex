```latex
\documentclass[12pt]{article}

% Required Packages
\usepackage[margin=1in]{geometry}
\usepackage{pifont} % For \ding{51} (checkmark) and \ding{55} (cross)
\usepackage{booktabs} % For professional-looking tables (\toprule, \midrule, \bottomrule)
\usepackage{hyperref} % For clickable links and PDF metadata
\usepackage{url}      % For formatting URLs
\usepackage{seqsplit} % For splitting long strings in \texttt
\usepackage{xcolor}   % For custom colors

% Document Metadata
\hypersetup{
    colorlinks=true,
    linkcolor=blue,
    filecolor=magenta,
    urlcolor=cyan,
    pdftitle={Cybersecurity Posture Report},
    pdfauthor={Cybersecurity Analyst},
    pdfsubject={Security Assessment},
    pdfkeywords={security, report, analysis, nmap, risk},
}

% Define custom colors for severity
\definecolor{criticalred}{HTML}{D7263D}
\definecolor{highorange}{HTML}{F46036}
\definecolor{mediumyellow}{HTML}{F2A541}

\begin{document}

\title{Cybersecurity Posture Report \\ \large For: \textbf{Sovereign Trust}}
\author{Cybersecurity Analyst}
\date{\today}
\maketitle

\section*{1. Executive Summary}
This report provides a comprehensive analysis of the cybersecurity posture for \textbf{Sovereign Trust}. The assessment is based on a synthesis of network scan data, an organizational security questionnaire, and a review of pre-existing risks.

The analysis reveals several critical and high-risk vulnerabilities that require immediate attention. A key technical finding is an exposed, outdated FTP server (\texttt{vsftpd 2.3.4}) which is vulnerable to a known remote code execution backdoor (CVE-2011-2523) and is dangerously misconfigured to allow anonymous logins.

Furthermore, significant gaps in security controls were identified. The lack of Multi-Factor Authentication (MFA) for computer and sensitive data system access, combined with the absence of a formal employee acceptable use policy, presents a high risk to the organization's data integrity and confidentiality. These policy-level weaknesses are compounded by an existing issue of outdated Windows 7 workstations.

Immediate remediation of the exposed FTP service and the implementation of robust identity and access management controls are strongly recommended to mitigate the risk of a significant security breach.

\section*{2. Organizational Information}
The following details were provided by the client and used as a baseline for this assessment.

\begin{table}[h!]
\centering
\begin{tabular}{ll}
\toprule
\textbf{Attribute} & \textbf{Value} \\
\midrule
Organization Name & \textbf{Sovereign Trust} \\
Email Domain & \texttt{SovereignTrust.com} \\
Website Domain & \texttt{www.SovereignTrust.com} \\
External IP & \texttt{115.109.168.62} \\
\bottomrule
\end{tabular}
\caption{Client Profile}
\end{table}

\section*{3. Security Control Review}
The following table summarizes the organization's responses to a security controls questionnaire. Items marked with \ding{55} represent significant gaps in the current security framework and are correlated with findings in the Risk Assessment section.

\begin{table}[h!]
\centering
\begin{tabular}{p{0.8\linewidth}c}
\toprule
\textbf{Control Question} & \textbf{Response} \\
\midrule
Do you require MFA to access email? & \ding{51} \\
Do you require MFA to log into computers? & \textbf{\color{criticalred}\ding{55}} \\
Do you require MFA to access sensitive data systems? & \textbf{\color{criticalred}\ding{55}} \\
Does your organization have an employee acceptable use policy? & \textbf{\color{highorange}\ding{55}} \\
Does your organization do security awareness training for new employees? & \ding{51} \\
Does your organization do security awareness training for all employees at least once per year? & \ding{51} \\
\bottomrule
\end{tabular}
\caption{Security Controls Questionnaire Results (\ding{51}=Yes, \ding{55}=No)}
\end{table}

\section*{4. Technical Scan Results}
An external network scan was performed to identify exposed services and potential vulnerabilities.

\subsection*{4.1 Network Scan (Nmap)}
The scan was conducted against the target IP address \texttt{10.0.0.15}. The results indicate one open port with a critically outdated and misconfigured service.

\begin{table}[h!]
\centering
\begin{tabular}{lllll}
\toprule
\textbf{Port} & \textbf{State} & \textbf{Service} & \textbf{Version} & \textbf{Notes} \\
\midrule
21/tcp & open & ftp & vsftpd 2.3.4 & \textbf{Anonymous FTP login allowed} \\
\bottomrule
\end{tabular}
\caption{Open Port Findings}
\end{table}

\paragraph{Analysis:} The presence of an open FTP port is concerning, but the specific version, \texttt{vsftpd 2.3.4}, is known to contain a critical backdoor vulnerability (CVE-2011-2523). This vulnerability was intentionally added to the source code and allows an attacker to execute arbitrary commands with root privileges by sending a specific sequence of characters as the username. The misconfiguration allowing anonymous login dramatically lowers the barrier for an attacker to exploit this flaw.

\section*{5. Risk Assessment}
The following table consolidates risks identified from the technical scan, security control review, and pre-existing vulnerability data. Each risk is assigned a severity level based on its potential impact and likelihood of exploitation.

\begin{table}[h!]
\centering
\begin{tabular}{p{0.25\linewidth}p{0.55\linewidth}l}
\toprule
\textbf{Risk Name} & \textbf{Overview} & \textbf{Severity} \\
\midrule
\textbf{Exposed FTP with Known Backdoor} & An outdated FTP server (\texttt{vsftpd 2.3.4}) is exposed, allowing for potential remote code execution. The risk is amplified by an anonymous login configuration. & \textbf{\color{criticalred}Critical} \\
\addlinespace
\textbf{Insufficient Identity and Access Control} & Lack of Multi-Factor Authentication (MFA) on employee computers and sensitive data systems creates a high risk of unauthorized access via compromised credentials. & \textbf{\color{highorange}High} \\
\addlinespace
\textbf{Missing Acceptable Use Policy (AUP)} & The absence of a formal AUP means there are no clear guidelines for employees on the acceptable use of company assets, increasing the risk of insider threat and misuse. & \textbf{\color{highorange}High} \\
\addlinespace
\textbf{Outdated Windows Policy} & Workstations are running Windows 7, an unsupported operating system that no longer receives security updates, leaving them vulnerable to a wide range of exploits. & \textbf{\color{mediumyellow}Medium} \\
\bottomrule
\end{tabular}
\caption{Consolidated Risk Summary}
\end{table}

\section*{6. Recommendations}
Based on the analysis, the following actions are recommended to mitigate the identified risks. Recommendations are prioritized by severity.

\begin{enumerate}
    \item \textbf{Remediate Exposed FTP Service (Critical):}
    \begin{itemize}
        \item \textbf{Immediate Action:} Take the FTP service on \texttt{10.0.0.15} offline immediately to prevent exploitation.
        \item \textbf{Short-Term:} If the service is business-critical, upgrade the \texttt{vsftpd} software to the latest stable version and place the server behind a firewall with strict access control lists.
        \item \textbf{Configuration:} Disable anonymous FTP login. Enforce strong password policies for all FTP accounts.
    \end{itemize}

    \item \textbf{Implement Multi-Factor Authentication (MFA) (High):}
    \begin{itemize}
        \item \textbf{Immediate Action:} Develop a project plan for a phased rollout of MFA.
        \item \textbf{Short-Term:} Prioritize the implementation of MFA for all access to sensitive data systems, followed by all employee computer logins.
        \item \textbf{Long-Term:} Ensure MFA is a default requirement for any new system or application deployed within the organization.
    \end{itemize}

    \item \textbf{Develop and Implement an Acceptable Use Policy (AUP) (High):}
    \begin{itemize}
        \item \textbf{Short-Term:} Draft a formal AUP that clearly defines rules for the use of company networks, systems, and data.
        \item \textbf{Implementation:} Require all employees to read and formally acknowledge the policy. Integrate this step into the onboarding process for new hires.
    \end{itemize}

    \item \textbf{Address Outdated Operating Systems (Medium):}
    \begin{itemize}
        \item \textbf{Short-Term:} Accelerate the existing plan to upgrade all Windows 7 workstations to a supported operating system, such as Windows 10 or 11.
        \item \textbf{Long-Term:} Establish a formal patch and lifecycle management policy to ensure all systems remain supported and up-to-date.
    \end{itemize}
\end{enumerate}

\end{document}
```