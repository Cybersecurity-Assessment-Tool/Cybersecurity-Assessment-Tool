```latex
\documentclass[12pt]{article}

% Preamble: Required Packages
\usepackage[margin=1in]{geometry}
\usepackage{pifont} % For checkmarks and crosses
\usepackage{booktabs} % For professional tables
\usepackage{hyperref} % For clickable links
\usepackage{url} % For formatting URLs
\usepackage{seqsplit} % For splitting long strings like IPs
\usepackage{graphicx} % For potential logo inclusion

% Document Metadata
\title{Cybersecurity Posture Assessment Report}
\author{Cybersecurity Analysis Division}
\date{\today}

\begin{document}

\maketitle
\thispagestyle{empty}
\newpage

\tableofcontents
\newpage

% --- 1. Executive Summary ---
\section{Executive Summary}
This report provides a cybersecurity posture assessment for \textbf{Blue Marble}, conducted on \today. The analysis is based on a synthesis of network scan data, an organizational security questionnaire, and a review of known outstanding risks.

The assessment identified several areas of significant concern that require immediate attention. The most critical findings include a lack of multi-factor authentication (MFA) on sensitive data systems and a complete absence of a security awareness training program for employees. Additionally, an externally accessible Secure Shell (SSH) service was discovered, which presents a notable attack vector.

While the organization has implemented some foundational controls, such as MFA for email and computer access, the identified gaps substantially increase the risk of unauthorized access, data breach, and compromise through social engineering. Recommendations have been prioritized to address the most critical vulnerabilities first.

% --- 2. Organizational Information ---
\section{Organizational Information}
The following details were provided for the assessment. This information is used to establish the context and scope of the review.

\begin{tabular}{@{}ll}
\toprule
\textbf{Attribute} & \textbf{Value} \\
\midrule
Organization Name & \textbf{Blue Marble} \\
Primary Email Domain & \texttt{BlueMarble.net} \\
Primary Website & \url{www.BlueMarble.net} \\
External IP Address & \texttt{156.155.118.93} \\
\bottomrule
\end{tabular}

% --- 3. Security Control Review ---
\section{Security Control Review}
The following table summarizes the organization's responses to a security controls questionnaire. A checkmark (\ding{51}) indicates a positive control is in place, while a cross (\ding{55}) indicates a control gap.

\begin{table}[h!]
\centering
\caption{Security Controls Questionnaire Results}
\begin{tabular}{@{}lc}
\toprule
\textbf{Control Question} & \textbf{Response} \\
\midrule
Do you require MFA to access email? & \ding{51} \\
Do you require MFA to log into computers? & \ding{51} \\
Do you require MFA to access sensitive data systems? & \textbf{\color{red}\ding{55}} \\
Does your organization have an employee acceptable use policy? & \ding{51} \\
Does your organization do security awareness training for new employees? & \textbf{\color{red}\ding{55}} \\
Does your organization do security awareness training for all employees at least once per year? & \textbf{\color{red}\ding{55}} \\
\bottomrule
\end{tabular}
\end{table}

\subsection*{Analysis of Control Gaps}
The questionnaire revealed two major control deficiencies:
\begin{itemize}
    \item \textbf{MFA on Sensitive Systems:} The absence of MFA for accessing sensitive data is a critical vulnerability. This allows a single point of failure (a compromised password) to potentially lead to a major data breach.
    \item \textbf{Security Awareness Training:} The lack of any security awareness training program for new or existing employees is a high-risk gap. This leaves the organization highly susceptible to phishing, social engineering, and other human-centric attacks.
\end{itemize}

% --- 4. Technical Scan Results ---
\section{Technical Scan Results}
An external network scan was performed to identify exposed services and potential vulnerabilities.

\begin{itemize}
    \item \textbf{Target IP Address:} \seqsplit{\texttt{2001:db8::1}}
\end{itemize}

\begin{table}[h!]
\centering
\caption{Open Ports Detected}
\begin{tabular}{@{}llll}
\toprule
\textbf{Port} & \textbf{State} & \textbf{Service/Product/Version} \\
\midrule
22/tcp & open & SSH (Service details not available) \\
\bottomrule
\end{tabular}
\end{table}

\subsection*{Analysis of Technical Findings}
The scan identified an open SSH port (22) on the target system. SSH is a common protocol for remote administration, but its exposure to the public internet creates a significant risk. Without proper hardening, this service can be targeted by:
\begin{itemize}
    \item \textbf{Brute-force attacks:} Automated tools attempting to guess usernames and passwords.
    \item \textbf{Credential stuffing:} Using credentials stolen from other breaches to gain access.
    \item \textbf{Exploitation of vulnerabilities:} If the SSH server software is outdated, it may be susceptible to known exploits.
\end{itemize}
This finding, combined with the lack of security awareness training, elevates the risk of a successful compromise.

% --- 5. Consolidated Risk Assessment ---
\section{Consolidated Risk Assessment}
The following table synthesizes findings from the security control review, technical scan, and pre-existing risk data. Risks are prioritized by severity to guide remediation efforts.

\begin{table}[h!]
\centering
\caption{Summary of Identified Risks}
\begin{tabular}{@{}p{0.1\textwidth}p{0.6\textwidth}p{0.2\textwidth}}
\toprule
\textbf{Risk ID} & \textbf{Description} & \textbf{Severity} \\
\midrule
\textbf{RISK-001} & Lack of Multi-Factor Authentication (MFA) on systems containing sensitive data. A compromised password could lead directly to a data breach. & \textbf{Critical} \\
\addlinespace
\textbf{RISK-002} & No security awareness training program for new or existing employees. This significantly increases susceptibility to phishing and social engineering attacks. & \textbf{High} \\
\addlinespace
\textbf{RISK-003} & Exposed SSH service on an external-facing system (\seqsplit{\texttt{2001:db8::1}}). This provides a direct vector for attackers to attempt unauthorized remote access. & \textbf{High} \\
\bottomrule
\end{tabular}
\end{table}

% --- 6. Recommendations ---
\section{Recommendations}
Based on the consolidated risk assessment, the following actions are recommended to improve the security posture of \textbf{Blue Marble}.

\begin{enumerate}
    \item \textbf{[Critical] Implement MFA on Sensitive Systems:}
    \begin{itemize}
        \item \textbf{Action:} Immediately deploy mandatory MFA for all user accounts (including administrative and service accounts) that have access to sensitive data repositories, applications, and infrastructure.
        \item \textbf{Justification:} This is the single most effective control to prevent unauthorized access resulting from compromised credentials. It directly mitigates RISK-001.
    \end{itemize}
    \vspace{0.5cm}
    \item \textbf{[High] Establish a Security Awareness Training Program:}
    \begin{itemize}
        \item \textbf{Action:} Develop and implement a formal security awareness training program. This must include mandatory training for all new employees during onboarding and annual refresher training for all staff.
        \item \textbf{Justification:} An educated workforce is the first line of defense. Training on topics like phishing identification, password security, and acceptable use will reduce the likelihood of human error leading to a compromise, mitigating RISK-002.
    \end{itemize}
    \vspace{0.5cm}
    \item \textbf{[High] Harden the Exposed SSH Service:}
    \begin{itemize}
        \item \textbf{Action:} Apply hardening best practices to the SSH service on \seqsplit{\texttt{2001:db8::1}}. This should include:
        \begin{itemize}
            \item Restricting access to a whitelist of trusted source IP addresses.
            \item Disabling password-based authentication and requiring public key cryptography.
            \item Implementing an automated lockout tool (e.g., Fail2Ban) to block brute-force attempts.
            \item Ensuring the SSH server software is fully patched and up-to-date.
        \end{itemize}
        \item \textbf{Justification:} These measures will significantly reduce the attack surface of the exposed service and protect against common automated and targeted attacks, mitigating RISK-003.
    \end{itemize}
\end{enumerate}

\end{document}
```