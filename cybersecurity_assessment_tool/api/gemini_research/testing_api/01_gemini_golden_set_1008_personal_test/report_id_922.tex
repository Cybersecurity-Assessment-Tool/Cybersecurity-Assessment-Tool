```latex
\documentclass[12pt]{article}

% Preamble: Required Packages
\usepackage[margin=1in]{geometry}
\usepackage{pifont} % For checkmarks and crosses
\usepackage{booktabs} % For professional tables
\usepackage{hyperref} % For clickable links
\usepackage{url}      % For URL formatting
\usepackage{seqsplit} % For splitting long strings like hashes or IPs
\usepackage{graphicx} % For potential logos/images
\usepackage{xcolor}   % For colors in text

% Document Metadata
\title{Cybersecurity Posture Assessment Report}
\author{Cybersecurity Analysis Division}
\date{\today}

% Hyperref Setup
\hypersetup{
    colorlinks=true,
    linkcolor=blue,
    filecolor=magenta,      
    urlcolor=cyan,
    pdftitle={Cybersecurity Posture Assessment Report},
    pdfpagemode=FullScreen,
}

\begin{document}

\maketitle
\thispagestyle{empty}
\newpage

\tableofcontents
\newpage

% --- Section 1: Executive Summary ---
\section{Executive Summary}

This report provides a comprehensive cybersecurity assessment for \textbf{Open Door}. The analysis is based on a correlation of organizational data, a security controls questionnaire, an external network scan, and a review of pre-existing risks.

The assessment revealed a mixed security posture. On a positive note, the external network scan of the target IP address \texttt{[Target IP]} showed no open ports, which indicates a strong firewall configuration and a hardened external perimeter. This significantly reduces the attack surface available to external threat actors.

However, the security controls review identified two critical administrative and technical gaps. The absence of an employee \textbf{Acceptable Use Policy (AUP)} is a high-risk policy gap that can lead to inconsistent security practices and insider threats. More critically, the lack of \textbf{Multi-Factor Authentication (MFA) on sensitive data systems} represents a critical vulnerability. Should an attacker compromise a user's credentials, they would have direct access to the organization's most valuable data.

Immediate remediation should focus on implementing MFA across all sensitive systems and developing a formal Acceptable Use Policy to be acknowledged by all employees.

% --- Section 2: Organizational Information ---
\section{Organizational Information}

The following details were provided for the assessment. This information is used to establish the context and scope of the review.

\begin{table}[h!]
\centering
\begin{tabular}{@{}ll@{}}
\toprule
\textbf{Attribute} & \textbf{Value} \\ \midrule
Organization Name & \textbf{Open Door} \\
Primary Email Domain & \seqsplit{\texttt{OpenDoor.net}} \\
Primary Website Domain & \seqsplit{\url{www.OpenDoor.net}} \\
External IP Address & \seqsplit{\texttt{119.209.248.224}} \\ \bottomrule
\end{tabular}
\caption{Client Organizational Details.}
\label{tab:org_info}
\end{table}

% --- Section 3: Security Control Review ---
\section{Security Control Review}

A security questionnaire was completed to evaluate the implementation of key administrative and technical security controls. The results are summarized below. "No" answers indicate significant gaps that increase organizational risk.

\begin{table}[h!]
\centering
\begin{tabular}{@{}p{0.7\textwidth}c@{}}
\toprule
\textbf{Control Question} & \textbf{Status} \\ \midrule
Do you require MFA to access email? & \textcolor{green}{\ding{51}} \\
Do you require MFA to log into computers? & \textcolor{green}{\ding{51}} \\
\textbf{Do you require MFA to access sensitive data systems?} & \textcolor{red}{\ding{55}} \\
\textbf{Does your organization have an employee acceptable use policy?} & \textcolor{red}{\ding{55}} \\
Does your organization do security awareness training for new employees? & \textcolor{green}{\ding{51}} \\
Does your organization do security awareness training for all employees at least once per year? & \textcolor{green}{\ding{51}} \\ \bottomrule
\end{tabular}
\caption{Security Controls Questionnaire Results (\ding{51}=Yes, \ding{55}=No).}
\label{tab:controls_review}
\end{table}

\subsection*{Analysis of Gaps}
\begin{itemize}
    \item \textbf{MFA for Sensitive Data Systems:} The lack of MFA on systems containing sensitive data is a critical weakness. Credential theft, a common attack vector, could lead to an immediate and severe data breach without the protective layer of a second authentication factor.
    \item \textbf{Acceptable Use Policy (AUP):} An AUP is a foundational policy that defines the rules for using company IT assets. Without it, there is no formal standard for employee behavior, data handling, or consequences for misuse, creating legal and operational risks.
\end{itemize}

% --- Section 4: Technical Scan Results ---
\section{Technical Scan Results}

An external network vulnerability scan was performed to identify open ports, running services, and potential vulnerabilities visible from the public internet.

\begin{itemize}
    \item \textbf{Target IP Address:} \texttt{[Target IP]}
    \item \textbf{Scan Date:} \today
\end{itemize}

\subsection*{Findings}
The scan completed successfully but found \textbf{no open TCP or UDP ports} on the target system.

\subsection*{Interpretation}
This is a positive security finding. It indicates that the external firewall is configured with a "default deny" policy and is effectively blocking unsolicited inbound traffic. This significantly reduces the external attack surface and protects internal systems from direct network-based attacks from the internet.

% --- Section 5: Risk Assessment ---
\section{Risk Assessment}

This section synthesizes findings from the security control review and technical scan. The pre-existing risk register was empty. The following new risks have been identified and prioritized based on their potential impact on the organization.

\begin{table}[h!]
\centering
\begin{tabular}{@{}p{0.25\textwidth}p{0.55\textwidth}l@{}}
\toprule
\textbf{Risk Name} & \textbf{Overview} & \textbf{Severity} \\ \midrule
\textbf{Lack of MFA on Sensitive Systems} & Compromised credentials could grant an attacker direct, unimpeded access to critical business data, leading to a major data breach. & \textbf{\textcolor{red}{Critical}} \\
\addlinespace
\textbf{Absence of Acceptable Use Policy} & Without a formal AUP, employees may misuse IT resources or mishandle data, increasing the risk of insider threats, data loss, and legal liability. & \textbf{\textcolor{orange}{High}} \\
\bottomrule
\end{tabular}
\caption{Summary of Identified Risks.}
\label{tab:risk_summary}
\end{table}

% --- Section 6: Recommendations ---
\section{Recommendations}

Based on the risk assessment, the following actions are recommended to mitigate the identified vulnerabilities and improve the overall security posture of \textbf{Open Door}.

\begin{description}
    \item[\textbf{Recommendation 1: Implement MFA for Sensitive Systems (Priority: Critical)}]
    \begin{itemize}
        \item \textbf{Action:} Deploy mandatory Multi-Factor Authentication (MFA) for all user accounts (including administrative and service accounts) that have access to systems storing or processing sensitive or critical data.
        \item \textbf{Justification:} This is the single most effective control to prevent unauthorized access resulting from compromised credentials. It directly mitigates the risk of a severe data breach.
    \end{itemize}
    \vspace{1em}
    \item[\textbf{Recommendation 2: Develop and Enforce an Acceptable Use Policy (Priority: High)}]
    \begin{itemize}
        \item \textbf{Action:} Create a formal Acceptable Use Policy (AUP) that clearly outlines the rules for using company networks, computers, and data. The policy should be reviewed by legal counsel, communicated to all employees, and formally acknowledged via signature.
        \item \textbf{Justification:} An AUP establishes a baseline for secure behavior, reduces the risk of insider threat (both malicious and accidental), and provides a framework for enforcing security standards.
    \end{itemize}
\end{description}

\end{document}
```