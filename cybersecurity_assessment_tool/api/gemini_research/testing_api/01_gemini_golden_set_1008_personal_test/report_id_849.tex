```latex
\documentclass[12pt]{article}

% --- PACKAGES ---
\usepackage[margin=1in]{geometry}
\usepackage{pifont} % For checkmarks and crosses
\usepackage{booktabs} % For professional tables
\usepackage{hyperref} % For clickable links
\usepackage{url} % For URL formatting
\usepackage{seqsplit} % To split long monospaced strings
\usepackage{graphicx}
\usepackage{xcolor}

% --- DOCUMENT SETUP ---
\hypersetup{
    colorlinks=true,
    linkcolor=blue,
    filecolor=magenta,      
    urlcolor=cyan,
    pdftitle={Cybersecurity Posture Report},
    pdfpagemode=FullScreen,
}

\newcommand{\yes}{\ding{51}}
\newcommand{\no}{\ding{55}}

% --- DOCUMENT START ---
\begin{document}

% --- TITLE PAGE ---
\begin{titlepage}
    \centering
    \vspace*{1cm}
    \Huge\textbf{Cybersecurity Posture Report}
    \vspace{0.5cm}
    \Large For
    \vspace{0.5cm}
    \Large\textbf{Firebrand Media}
    
    \vspace{1.5cm}
    
    \textbf{Generated By:} \\
    Cybersecurity Analyst
    
    \vfill
    
    \textbf{Date:} \\
    \today
    
\end{titlepage}

\tableofcontents
\newpage

% --- EXECUTIVE OVERVIEW ---
\section*{Executive Overview}
This report provides a comprehensive analysis of the cybersecurity posture for \textbf{Firebrand Media}, based on a synthesis of network scan data, organizational security controls, and pre-existing risk information.

The assessment has identified a \textbf{critical-risk finding}. An externally accessible service on port 8080 of host \texttt{10.5.5.5} presents a web page titled \texttt{"TOP SECRET DB"}. This suggests a high probability of sensitive data exposure and requires immediate investigation and remediation. This active finding directly contradicts a previous risk assessment which incorrectly labeled this port as secure.

Furthermore, significant gaps were identified in the organization's security controls. The lack of mandatory Multi-Factor Authentication (MFA) for email access and the absence of annual security awareness training for all employees represent high-risk vulnerabilities. These policy gaps substantially increase the organization's susceptibility to phishing, business email compromise, and other social engineering attacks.

Immediate action is required to address the exposed service. Subsequently, a focused effort should be made to implement the recommended security controls to mitigate the identified policy-based risks and improve the overall security posture.

% --- ORGANIZATIONAL INFORMATION ---
\section*{Organizational Information}
The following details were provided for the assessment.

\begin{tabular}{@{}ll}
    \toprule
    \textbf{Attribute} & \textbf{Value} \\
    \midrule
    Organization Name & \textbf{Firebrand Media} \\
    Email Domain & \texttt{FirebrandMedia.org} \\
    Website Domain & \href{http://www.FirebrandMedia.org}{\texttt{www.FirebrandMedia.org}} \\
    External IP Address & \seqsplit{\texttt{73.217.168.248}} \\
    \bottomrule
\end{tabular}

% --- SECURITY CONTROL REVIEW ---
\section*{Security Control Review (Questionnaire Analysis)}
A review of the organization's security controls was conducted via a questionnaire. The responses highlight several areas of concern where current practices do not align with security best practices. "No" answers indicate significant gaps in the security framework.

\begin{tabular}{@{}p{0.7\textwidth}c}
    \toprule
    \textbf{Control Question} & \textbf{Response} \\
    \midrule
    Do you require MFA to access email? & \textcolor{red}{\no} \\
    Do you require MFA to log into computers? & \textcolor{green}{\yes} \\
    Do you require MFA to access sensitive data systems? & \textcolor{green}{\yes} \\
    Does your organization have an employee acceptable use policy? & \textcolor{green}{\yes} \\
    Does your organization do security awareness training for new employees? & \textcolor{green}{\yes} \\
    Does your organization do security awareness training for all employees at least once per year? & \textcolor{red}{\no} \\
    \bottomrule
\end{tabular}

% --- TECHNICAL SCAN RESULTS ---
\section*{Technical Scan Results}
A network scan was performed to identify open ports and exposed services on the target system.

\begin{itemize}
    \item \textbf{Target IP Address:} \texttt{10.5.5.5}
\end{itemize}

The scan revealed the following open port:

\begin{tabular}{@{}llll}
    \toprule
    \textbf{Port} & \textbf{State} & \textbf{Service/Banner} \\
    \midrule
    8080/tcp & Open & HTTP (Title: \texttt{TOP SECRET DB}) \\
    \bottomrule
\end{tabular}

\subsection*{Analysis of Technical Findings}
The finding on port 8080 is of \textbf{critical concern}. The service's title, \texttt{"TOP SECRET DB"}, strongly implies that a database or a system containing highly sensitive information is exposed. This finding directly contradicts the provided risk data (\textit{Input\_3\_Current\_Risks\_JSON}), which incorrectly states that this port is secure and the finding was a false positive. The current, active scan confirms this is a live and dangerous exposure.

% --- RISK ASSESSMENT ---
\section*{Risk Assessment}
Based on the correlation of all data inputs, the following risks have been identified and prioritized.

\begin{tabular}{@{}p{0.1\textwidth}p{0.4\textwidth}p{0.15\textwidth}p{0.25\textwidth}}
    \toprule
    \textbf{Risk ID} & \textbf{Description} & \textbf{Severity} & \textbf{Affected Systems} \\
    \midrule
    \textbf{R-01} & A publicly accessible web service on port 8080 is titled "TOP SECRET DB," indicating a high likelihood of a critical data exposure. & \textbf{Critical} & \texttt{10.5.5.5} \\
    \addlinespace
    \textbf{R-02} & Lack of MFA on email accounts exposes the organization to account takeovers, phishing, and business email compromise. & \textbf{High} & All employee email accounts \\
    \addlinespace
    \textbf{R-03} & The absence of mandatory annual security training for all staff increases susceptibility to social engineering and human error. & \textbf{High} & All employees \\
    \addlinespace
    \textbf{R-04} & The risk management process is flawed, as demonstrated by the incorrect closure of the Port 8080 risk, which was proven to be an active threat. & \textbf{Medium} & Organizational Risk Management Process \\
    \bottomrule
\end{tabular}

% --- RECOMMENDATIONS ---
\section*{Recommendations}
The following actions are recommended to mitigate the identified risks and strengthen the security posture of \textbf{Firebrand Media}.

\subsection*{Immediate Priority (Next 24 Hours)}
\begin{itemize}
    \item \textbf{Remediate R-01 (Exposed Service):}
    \begin{enumerate}
        \item Immediately restrict all external access to port 8080 on host \texttt{10.5.5.5} using a firewall.
        \item Conduct an urgent investigation to identify the nature of the "TOP SECRET DB" service, the data it contains, and whether it has been compromised.
        \item If the service is required for business, ensure it is placed behind a secure authentication mechanism (e.g., VPN, SSO with MFA) and is not exposed directly to the internet.
    \end{enumerate}
\end{itemize}

\subsection*{High Priority (Next 30 Days)}
\begin{itemize}
    \item \textbf{Remediate R-02 (Email MFA):}
    \begin{enumerate}
        \item Procure and implement a Multi-Factor Authentication solution for the organization's email system (\texttt{FirebrandMedia.org}).
        \item Develop a rollout plan and enforce mandatory MFA for all users, including administrative and executive accounts.
    \end{enumerate}
    \item \textbf{Remediate R-03 (Security Training):}
    \begin{enumerate}
        \item Establish a mandatory annual security awareness training program for all employees.
        \item Topics should include phishing identification, password hygiene, and acceptable use policies. Track completion to ensure 100\% compliance.
    \end{enumerate}
\end{itemize}

\subsection*{Process Improvement (Next 90 Days)}
\begin{itemize}
    \item \textbf{Remediate R-04 (Risk Management Process):}
    \begin{enumerate}
        \item Review and update the organization's risk assessment and vulnerability management procedures.
        \item Ensure that technical validation (e.g., a confirmation scan) is a mandatory step before any technically-oriented risk can be closed or accepted as a false positive.
    \end{enumerate}
\end{itemize}

\end{document}
```