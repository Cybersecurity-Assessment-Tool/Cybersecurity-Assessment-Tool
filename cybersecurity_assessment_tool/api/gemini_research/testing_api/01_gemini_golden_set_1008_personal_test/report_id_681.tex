```latex
\documentclass[12pt]{article}

% --- PACKAGES ---
\usepackage[margin=1in]{geometry}
\usepackage{pifont} % For checkmarks and crosses
\usepackage{booktabs} % For professional tables
\usepackage{hyperref} % For hyperlinks and metadata
\usepackage{url}      % For formatting URLs
\usepackage{seqsplit} % For splitting long strings without spaces
\usepackage{graphicx} % For logo (placeholder)
\usepackage{xcolor}   % For colors

% --- DOCUMENT METADATA ---
\hypersetup{
    colorlinks=true,
    linkcolor=blue,
    filecolor=magenta,      
    urlcolor=cyan,
    pdftitle={Cybersecurity Posture Assessment Report},
    pdfauthor={Cybersecurity Analyst},
    pdfsubject={Security Analysis},
    pdfkeywords={Cybersecurity, Risk Assessment, Network Scan},
}

% --- TITLE ---
\title{Cybersecurity Posture Assessment Report \\ \large For: \textbf{Stellar Pathways}}
\author{Cybersecurity Analyst}
\date{\today}

% --- BEGIN DOCUMENT ---
\begin{document}

\maketitle
\thispagestyle{empty}
\newpage

\tableofcontents
\newpage

% ==============================================================================
% 1. EXECUTIVE SUMMARY
% ==============================================================================
\section{Executive Summary}

This report provides a comprehensive cybersecurity posture assessment for \textbf{Stellar Pathways}, conducted on \today. The analysis is based on a correlation of technical network scan data, a review of organizational security controls, and an evaluation of pre-existing risk documentation.

The assessment reveals a mixed security posture. The organization demonstrates maturity in implementing Multi-Factor Authentication (MFA) across key systems, which is a commendable foundational control. However, two significant risks were identified that require immediate attention:

\begin{enumerate}
    \item \textbf{Critical Information Disclosure:} A network scan of the internal host \texttt{10.5.5.5} discovered an open service on port 8080 with a highly sensitive and revealing title: ``TOP SECRET DB''. This finding directly contradicts a previous risk assessment that marked this port as secure, indicating a potential failure in the vulnerability management lifecycle.
    
    \item \textbf{High-Risk Onboarding Gap:} A critical gap exists in the security awareness program. New employees do not receive mandatory security training upon joining the organization. This exposes the organization to significant risk from social engineering and policy violations during the crucial initial employment period.
\end{enumerate}

While strong preventative controls are in place, the identified vulnerabilities could be exploited by malicious actors. This report outlines these risks in detail and provides actionable recommendations to mitigate them, thereby strengthening the overall security posture of \textbf{Stellar Pathways}.

% ==============================================================================
% 2. ORGANIZATIONAL INFORMATION
% ==============================================================================
\section{Organizational Information}

The following details were provided for the assessment.

\begin{table}[h!]
\centering
\begin{tabular}{@{}ll@{}}
\toprule
\textbf{Attribute} & \textbf{Value} \\ \midrule
Organization Name & \textbf{Stellar Pathways} \\
Email Domain      & \texttt{StellarPathways.com} \\
External IP Address & \texttt{11.109.19.68} \\ \bottomrule
\end{tabular}
\caption{Client Organizational Details}
\end{table}

% ==============================================================================
% 3. SECURITY CONTROL REVIEW
% ==============================================================================
\section{Security Control Review}

A review of the organization's security controls was conducted via a questionnaire. The responses indicate the status of key administrative and technical safeguards. Findings are summarized below.

\begin{table}[h!]
\centering
\begin{tabular}{@{}p{0.6\textwidth}cc@{}}
\toprule
\textbf{Control Question} & \textbf{Response} & \textbf{Assessment} \\ \midrule
Do you require MFA to access email? & \ding{51} & Implemented \\
Do you require MFA to log into computers? & \ding{51} & Implemented \\
Do you require MFA to access sensitive data systems? & \ding{51} & Implemented \\
Does your organization have an employee acceptable use policy? & \ding{51} & Implemented \\
\textbf{Does your organization do security awareness training for new employees?} & \textbf{\ding{55}} & \textbf{Critical Gap} \\
Does your organization do security awareness training for all employees at least once per year? & \ding{51} & Implemented \\ \bottomrule
\end{tabular}
\caption{Security Controls Questionnaire Analysis. (\ding{51}=Yes, \ding{55}=No)}
\end{table}

The primary concern identified is the lack of security awareness training for new employees. This gap means that new staff may not be aware of the acceptable use policy or how to identify common threats like phishing, creating a significant vulnerability from day one of their employment.

% ==============================================================================
% 4. TECHNICAL SCAN RESULTS
% ==============================================================================
\section{Technical Scan Results}

An Nmap scan was performed on the specified target to identify open ports and exposed services.

\subsection{Target Information}
\begin{itemize}
    \item \textbf{Target IP:} \texttt{10.5.5.5}
    \item \textbf{Status:} Host is Up
\end{itemize}

\subsection{Open Ports and Services}
The scan revealed the following open port.

\begin{table}[h!]
\centering
\begin{tabular}{@{}llll@{}}
\toprule
\textbf{Port} & \textbf{State} & \textbf{Service} & \textbf{Details} \\ \midrule
8080/tcp & Open & http-proxy & \textbf{HTTP Title: TOP SECRET DB} \\ \bottomrule
\end{tabular}
\caption{Scan Results for Host \texttt{10.5.5.5}}
\end{table}

\subsection{Technical Analysis}
The discovery of an open service on port 8080 is a significant finding. The HTTP title "TOP SECRET DB" represents a critical information disclosure vulnerability. This title strongly suggests the service is connected to a highly sensitive database or application. Exposing such a descriptive name to the network allows an attacker to immediately identify a high-value target for further exploitation.

This finding is particularly alarming as it contradicts the information from the existing risk register (\textit{Input\_3\_Current\_Risks\_JSON}), which incorrectly classified this port as a secure false positive.

% ==============================================================================
% 5. CORRELATED RISK ASSESSMENT
% ==============================================================================
\section{Correlated Risk Assessment}

By synthesizing the security control review, technical scan results, and existing risk data, we have identified the following key risks.

\begin{table}[h!]
\centering
\begin{tabular}{@{}p{0.1\textwidth}p{0.3\textwidth}p{0.4\textwidth}p{0.1\textwidth}@{}}
\toprule
\textbf{Risk ID} & \textbf{Risk Title} & \textbf{Description} & \textbf{Severity} \\ \midrule
R-001 & Critical Information Disclosure on Internal Server & The service on \texttt{10.5.5.5:8080} exposes the title "TOP SECRET DB", identifying it as a high-value target. This contradicts a previous risk assessment, indicating a flawed validation process. & \textbf{Critical} \\
\addlinespace
R-002 & Lack of Onboarding Security Training & New employees are not provided with security training upon hiring, leaving them unaware of policies and susceptible to social engineering attacks from their first day. & \textbf{High} \\ \bottomrule
\end{tabular}
\caption{Summary of Identified Risks}
\end{table}

% ==============================================================================
% 6. RECOMMENDATIONS
% ==============================================================================
\section{Recommendations}

The following actions are recommended to mitigate the identified risks and improve the overall security posture.

\subsection{R-001: Critical Information Disclosure on Internal Server}
\begin{itemize}
    \item \textbf{Immediate Action (0-24 hours):}
    \begin{itemize}
        \item Immediately investigate the service running on \texttt{10.5.5.5:8080}.
        \item Determine the nature of the "TOP SECRET DB" system and the data it contains.
        \item If the system is sensitive and not intended for broad access, apply firewall rules to restrict access to only authorized personnel or shut down the service if it is non-essential.
    \end{itemize}
    \item \textbf{Short-Term Action (1-2 weeks):}
    \begin{itemize}
        \item If the service is required, ensure it is placed behind a robust authentication and authorization mechanism.
        \item Modify the service configuration to remove or obfuscate the revealing HTTP title.
    \end{itemize}
    \item \textbf{Long-Term Action (1-3 months):}
    \begin{itemize}
        \item Review and enhance the vulnerability management and risk assessment process. Implement a more rigorous validation procedure to ensure that findings are accurately assessed and not prematurely closed as false positives.
    \end{itemize}
\end{itemize}

\subsection{R-002: Lack of Onboarding Security Training}
\begin{itemize}
    \item \textbf{Immediate Action (0-24 hours):}
    \begin{itemize}
        \item Disseminate a security bulletin to all managers of new hires, reminding them of key policies like the Acceptable Use Policy.
    \end{itemize}
    \item \textbf{Short-Term Action (1-4 weeks):}
    \begin{itemize}
        \item Develop and mandate a security awareness training module as a required step in the new employee onboarding process.
        \item This training must cover phishing identification, password security, acceptable use, and procedures for reporting security incidents.
    \end{itemize}
    \item \textbf{Long-Term Action (1-3 months):}
    \begin{itemize}
        \item Integrate the onboarding training content with the annual security refresher course to ensure a consistent and evolving security culture within the organization.
    \end{itemize}
\end{itemize}

\end{document}
```