```latex
\documentclass[12pt]{article}

% Preamble: Required Packages
\usepackage[margin=1in]{geometry}
\usepackage{pifont} % For checkmarks and crosses
\usepackage{booktabs} % For professional tables
\usepackage{hyperref} % For clickable links
\usepackage{url} % For URL formatting
\usepackage{seqsplit} % To split long strings like hashes or tokens
\usepackage{graphicx} % For potential logos/images
\usepackage{xcolor} % For colors in tables

% Document Metadata
\title{Cybersecurity Posture Assessment Report}
\author{Cybersecurity Analysis Division}
\date{\today}

% Hyperref Setup
\hypersetup{
    colorlinks=true,
    linkcolor=blue,
    filecolor=magenta,      
    urlcolor=cyan,
    pdftitle={Cybersecurity Posture Assessment Report},
    pdfpagemode=FullScreen,
}

\begin{document}

\maketitle
\thispagestyle{empty}
\newpage

\tableofcontents
\newpage

% --- Section 1: Executive Overview ---
\section{Executive Overview}
This report provides a comprehensive analysis of the cybersecurity posture for \textbf{Tidal Wave Sports}. The assessment is based on a synthesis of organizational data, a technical network scan, and a review of pre-existing risks.

The analysis revealed several critical and high-risk security gaps that require immediate attention. Key findings include the absence of Multi-Factor Authentication (MFA) on critical systems like email and sensitive data repositories, a complete lack of a security awareness training program, and the exposure of a Secure Shell (SSH) service on an IPv6 address to the public internet. These deficiencies create significant vulnerabilities to account compromise, data breaches, and targeted cyberattacks.

This document details these findings and provides actionable recommendations to mitigate the identified risks and strengthen the overall security posture.

% --- Section 2: Organizational Information ---
\section{Organizational Information}
The following information was provided for the assessment. This data forms the baseline for understanding the organization's context.

\begin{tabular}{@{}ll}
\toprule
\textbf{Attribute} & \textbf{Value} \\
\midrule
Organization Name & \textbf{Tidal Wave Sports} \\
Email Domain & \texttt{TidalWaveSports.net} \\
Website Domain & \url{www.TidalWaveSports.net} \\
External IP (IPv4) & \texttt{42.146.155.86} \\
External IP (IPv6 Target) & \seqsplit{\texttt{2001:db8::1}} \\
\bottomrule
\end{tabular}

% --- Section 3: Security Control Review ---
\section{Security Control Review}
A security questionnaire was completed to evaluate existing administrative and technical controls. The results highlight significant gaps in foundational security practices. A checkmark (\ding{51}) indicates a positive control, while a cross (\ding{55}) indicates a gap.

\begin{tabular}{@{}p{0.8\linewidth}c}
\toprule
\textbf{Control Question} & \textbf{Status} \\
\midrule
Do you require MFA to access email? & \ding{55} \\
Do you require MFA to log into computers? & \ding{51} \\
Do you require MFA to access sensitive data systems? & \ding{55} \\
Does your organization have an employee acceptable use policy? & \ding{55} \\
Does your organization do security awareness training for new employees? & \ding{55} \\
Does your organization do security awareness training for all employees at least once per year? & \ding{55} \\
\bottomrule
\end{tabular}

% --- Section 4: Technical Scan Results ---
\section{Technical Scan Results}
An external network scan was performed on the target IP address to identify exposed services.

\begin{itemize}
    \item \textbf{Target IP Address:} \seqsplit{\texttt{2001:db8::1}}
    \item \textbf{Scan Status:} Host is up.
\end{itemize}

The scan identified the following open port, indicating a service accessible from the public internet.

\begin{tabular}{@{}llll}
\toprule
\textbf{Port} & \textbf{State} & \textbf{Service} & \textbf{Notes} \\
\midrule
22/tcp & open & SSH & Secure Shell. Used for remote administration. \\
\bottomrule
\end{tabular}

\subsection*{Analysis}
The presence of an open SSH port (22) on an external-facing IPv6 address is a significant finding. While necessary for remote administration, exposing SSH directly to the internet makes it a prime target for automated brute-force attacks and exploitation of potential vulnerabilities. The scan did not retrieve version information, so the patch level of the service is unknown.

% --- Section 5: Risk Assessment ---
\section{Risk Assessment}
The following table synthesizes findings from the security control review and the technical scan. No pre-existing vulnerabilities were documented. The risks are prioritized based on their potential impact and likelihood of exploitation.

\begin{tabular}{@{}p{0.2\linewidth}p{0.5\linewidth}p{0.2\linewidth}}
\toprule
\textbf{Risk Name} & \textbf{Overview} & \textbf{Severity} \\
\midrule
\textbf{Lack of MFA on Critical Systems} & Email and sensitive data systems are protected only by passwords. This creates a high risk of account takeover via phishing or credential stuffing, leading to potential data breaches. & \textbf{Critical} \\
\textbf{Exposed SSH Service} & The SSH management port is open to the entire internet, making it a target for brute-force login attempts and exploits. A compromise could lead to full server control. & \textbf{Critical} \\
\textbf{No Security Awareness Training} & Employees are not trained to recognize or report security threats like phishing. This significantly increases the likelihood of a successful social engineering attack. & \textbf{High} \\
\textbf{No Acceptable Use Policy} & The absence of a formal policy creates ambiguity regarding the secure and appropriate use of company assets, increasing the risk of insider threats and unintentional data exposure. & \textbf{High} \\
\bottomrule
\end{tabular}

% --- Section 6: Recommendations ---
\section{Recommendations}
Based on the analysis, the following actions are recommended to mitigate the identified risks and improve the security posture of \textbf{Tidal Wave Sports}.

\subsection*{Immediate Actions (Critical Priority)}
\begin{enumerate}
    \item \textbf{Implement MFA on All Critical Systems:} Immediately enforce MFA for all users on email platforms (e.g., Office 365, Google Workspace) and any systems containing sensitive data. This is the single most effective control to prevent account compromise.
    \item \textbf{Restrict SSH Access:} Immediately restrict access to the SSH service on \seqsplit{\texttt{2001:db8::1}}. Access should be limited to trusted IP addresses via a firewall or, preferably, placed behind a Virtual Private Network (VPN). Public-facing SSH should be disabled.
\end{enumerate}

\subsection*{High Priority Actions}
\begin{enumerate}
    \item \textbf{Establish a Security Awareness Training Program:} Develop and deploy a mandatory security awareness training program for all employees. The program should begin with onboarding for new hires and be conducted annually for all staff. It must cover key topics such as phishing, password security, and data handling.
    \item \textbf{Develop and Enforce an Acceptable Use Policy (AUP):} Create a formal AUP that clearly defines the rules for using company technology and data. This policy should be communicated to all employees and acknowledged via signature.
\end{enumerate}

\subsection*{General Recommendations}
\begin{enumerate}
    \item \textbf{Conduct Regular Vulnerability Scanning:} Implement a regular, automated vulnerability scanning process for all external-facing systems to proactively identify outdated software, misconfigurations, and other security flaws.
    \item \textbf{Enforce Strong Password Policies:} Ensure that a strong password policy is in place and enforced for all systems, complementing the deployment of MFA.
\end{enumerate}

\end{document}
```