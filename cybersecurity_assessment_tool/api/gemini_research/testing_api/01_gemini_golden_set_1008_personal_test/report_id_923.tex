```latex
\documentclass[12pt]{article}

% Preamble: Required Packages
\usepackage[margin=1in]{geometry}
\usepackage{pifont} % For checkmarks and crosses
\usepackage{booktabs} % For professional tables
\usepackage{hyperref} % For clickable links
\usepackage{url} % For URL formatting
\usepackage{seqsplit} % For splitting long strings
\usepackage{graphicx} % For potential logos
\usepackage{xcolor} % For colors

% Document Information
\title{Cybersecurity Posture Assessment Report}
\author{Cybersecurity Analyst}
\date{\today}

% Hyperref Setup
\hypersetup{
    colorlinks=true,
    linkcolor=blue,
    filecolor=magenta,      
    urlcolor=cyan,
    pdftitle={Cybersecurity Posture Assessment Report},
    pdfpagemode=FullScreen,
}

\begin{document}

\maketitle
\thispagestyle{empty}
\newpage

\tableofcontents
\newpage

% ------------------------------------------------------------------
% Section 1: Executive Summary
% ------------------------------------------------------------------
\section{Executive Summary}

This report provides a comprehensive cybersecurity assessment for \textbf{Atlas Mapping}, based on an analysis of network scan data, organizational security controls, and known pre-existing risks. The assessment synthesizes technical findings with procedural and policy-based controls to provide a holistic view of the organization's security posture.

The analysis revealed several areas of concern requiring immediate attention. A critical technical vulnerability, \textbf{Localhost Exposed}, was identified and correlated with an open SSH port on the loopback interface. Furthermore, significant gaps in security policy were discovered, including the lack of mandatory Multi-Factor Authentication (MFA) for computer logins and the absence of annual security awareness training for all staff.

These findings indicate a high risk of unauthorized access and an increased susceptibility to social engineering attacks. This report details these risks and provides actionable recommendations to mitigate them and strengthen the overall security posture of the organization.

% ------------------------------------------------------------------
% Section 2: Organizational Information
% ------------------------------------------------------------------
\section{Organizational Information}

The following details were provided for the assessment.

\begin{tabular}{@{}ll}
\toprule
\textbf{Attribute} & \textbf{Value} \\
\midrule
Organization Name & \textbf{Atlas Mapping} \\
Email Domain & \texttt{AtlasMapping.net} \\
Website Domain & \url{www.AtlasMapping.net} \\
External IP Address & \texttt{188.66.252.204} \\
\bottomrule
\end{tabular}

% ------------------------------------------------------------------
% Section 3: Security Control Review
% ------------------------------------------------------------------
\section{Security Control Review}

A review of the organization's security controls was conducted via a questionnaire. The responses are summarized below. Items marked with a red cross (\ding{55}) indicate a deviation from security best practices and represent a significant gap in the defensive posture.

\begin{table}[h!]
\centering
\begin{tabular}{@{}p{0.8\textwidth}c@{}}
\toprule
\textbf{Control Question} & \textbf{Response} \\
\midrule
Do you require MFA to access email? & \textcolor{green}{\ding{51}} \\
Do you require MFA to log into computers? & \textcolor{red}{\ding{55}} \\
Do you require MFA to access sensitive data systems? & \textcolor{green}{\ding{51}} \\
Does your organization have an employee acceptable use policy? & \textcolor{green}{\ding{51}} \\
Does your organization do security awareness training for new employees? & \textcolor{green}{\ding{51}} \\
Does your organization do security awareness training for all employees at least once per year? & \textcolor{red}{\ding{55}} \\
\bottomrule
\end{tabular}
\caption{Organizational Security Control Questionnaire Results}
\end{table}

\subsection*{Analysis of Control Gaps}
\begin{itemize}
    \item \textbf{Lack of Endpoint MFA:} The absence of MFA for computer logins is a critical weakness. If an employee's credentials are stolen (e.g., through phishing), an attacker could gain direct access to their workstation and potentially the internal network.
    \item \textbf{Lack of Annual Security Training:} While new employees receive training, the lack of an annual refresher course for all staff is a high-risk gap. Threat landscapes evolve, and without continuous education, employees are more likely to fall victim to modern phishing and social engineering tactics.
\end{itemize}

% ------------------------------------------------------------------
% Section 4: Technical Scan Results
% ------------------------------------------------------------------
\section{Technical Scan Results}

A network scan was performed to identify open ports and exposed services. The findings below are based on the provided scan data.

\begin{itemize}
    \item \textbf{Target IP Address:} \texttt{127.0.0.1}
    \item \textbf{Scan Date:} Not specified in data.
\end{itemize}

\begin{table}[h!]
\centering
\begin{tabular}{@{}lllll@{}}
\toprule
\textbf{Port} & \textbf{State} & \textbf{Service} & \textbf{Product} & \textbf{Version} \\
\midrule
22 & open & ssh & \textit{Not Provided} & \textit{Not Provided} \\
\bottomrule
\end{tabular}
\caption{Open Ports Detected on \texttt{127.0.0.1}}
\end{table}

\subsection*{Analysis of Technical Findings}
The scan identified that port 22 (SSH - Secure Shell) is open on the loopback interface (\texttt{127.0.0.1}). This finding directly correlates with the pre-existing risk "Localhost Exposed". While often used for local development or tunneling, an exposed SSH service on any interface must be carefully managed. Without version information, it is not possible to determine if the running SSH server is vulnerable to known exploits. This configuration is considered a critical risk.

% ------------------------------------------------------------------
% Section 5: Consolidated Risk Assessment
% ------------------------------------------------------------------
\section{Consolidated Risk Assessment}

The following table synthesizes findings from the security control review, technical scans, and pre-existing risk data to provide a consolidated view of the primary risks facing the organization.

\begin{table}[h!]
\centering
\begin{tabular}{@{}p{0.25\textwidth}p{0.5\textwidth}p{0.15\textwidth}@{}}
\toprule
\textbf{Risk Name} & \textbf{Description} & \textbf{Severity} \\
\midrule
\textbf{Localhost Exposed} & The SSH service (port 22) is open on the loopback interface, confirming a pre-existing identified risk. This could be exploited by local processes or misconfigurations to gain unauthorized access. & \textbf{Critical (10.0)} \\
\addlinespace
\textbf{Lack of Endpoint MFA} & The absence of MFA on computer logins exposes the organization to significant risk from compromised credentials, potentially leading to full device and network compromise. & \textbf{Critical} \\
\addlinespace
\textbf{Inadequate Security Training Program} & The lack of mandatory annual security training for all employees leads to a decay in security awareness, increasing the likelihood of successful phishing and social engineering attacks. & \textbf{High} \\
\bottomrule
\end{tabular}
\caption{Summary of Identified Risks}
\end{table}

% ------------------------------------------------------------------
% Section 6: Recommendations
% ------------------------------------------------------------------
\section{Recommendations}

Based on the analysis, the following actions are recommended to mitigate the identified risks and improve the overall security posture.

\begin{enumerate}
    \item \textbf{Remediate Localhost Exposure (Critical):}
    \begin{itemize}
        \item Immediately investigate the purpose of the open SSH port on \texttt{127.0.0.1}.
        \item If the service is not required for business operations, it should be disabled.
        \item If the service is required, ensure it is properly configured, updated to the latest stable version, and firewalled to prevent any unintended access.
    \end{itemize}
    
    \item \textbf{Implement Mandatory Endpoint MFA (Critical):}
    \begin{itemize}
        \item Procure and deploy an MFA solution for all employee computer and workstation logins (e.g., Windows Hello for Business, Duo, etc.).
        \item Develop a policy that mandates the use of MFA for all endpoint access. This will serve as a critical compensating control against credential theft.
    \end{itemize}
    
    \item \textbf{Establish Annual Security Awareness Training (High):}
    \begin{itemize}
        \item Develop or procure a security awareness training program to be completed by all employees on an annual basis.
        \item The training should cover modern threats such as phishing, ransomware, social engineering, and acceptable use of company assets.
        \item Track completion to ensure 100\% compliance across the organization.
    \end{itemize}
\end{enumerate}

\end{document}
```