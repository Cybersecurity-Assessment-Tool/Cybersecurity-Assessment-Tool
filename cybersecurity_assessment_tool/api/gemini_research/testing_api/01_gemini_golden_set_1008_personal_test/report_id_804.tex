Of course. As a Cybersecurity Analyst and LaTeX expert, I will analyze the provided data, including the corrupted inputs, and generate a complete and professional LaTeX report. The report will reflect the available information and provide actionable recommendations based on the significant gaps identified.

Here is the complete LaTeX report:

```latex
\documentclass[12pt]{article}

% Required Packages
\usepackage[margin=1in]{geometry}
\usepackage{pifont} % For checkmarks and crosses
\usepackage{booktabs} % For professional tables
\usepackage{hyperref} % For clickable links
\usepackage{url} % For URL formatting
\usepackage{seqsplit} % To split long strings in tt font
\usepackage{graphicx}
\usepackage{fancyhdr}
\usepackage{lastpage}
\usepackage{xcolor}

% --- Document Setup ---
\hypersetup{
    colorlinks=true,
    linkcolor=blue,
    filecolor=magenta,      
    urlcolor=cyan,
    pdftitle={Cybersecurity Posture Assessment Report},
    pdfpagemode=FullScreen,
}

% Define colors for severity
\definecolor{sev_critical}{HTML}{990000}
\definecolor{sev_high}{HTML}{DD4B39}
\definecolor{sev_medium}{HTML}{F4B400}
\definecolor{sev_low}{HTML}{4285F4}

% --- Header and Footer ---
\pagestyle{fancy}
\fancyhf{} % Clear all header and footer fields
\fancyhead[L]{Cybersecurity Posture Assessment}
\fancyhead[R]{For: Nebula Creative}
\fancyfoot[C]{\thepage\ of \pageref{LastPage}}
\renewcommand{\headrulewidth}{0.4pt}
\renewcommand{\footrulewidth}{0.4pt}

% --- Document Start ---
\begin{document}

% --- Title Page ---
\begin{titlepage}
    \centering
    \vspace*{2cm}
    
    {\Huge \textbf{Cybersecurity Posture Assessment Report}}
    \vspace{1.5cm}
    
    {\Large Prepared For:} \\
    \vspace{0.5cm}
    {\huge \textbf{Nebula Creative}}
    
    \vfill % Pushes content to the bottom
    
    {\large \today}
    
\end{titlepage}

\newpage
\tableofcontents
\newpage

% --- Section 1: Executive Summary ---
\section{Executive Summary}

This report provides an assessment of the cybersecurity posture for Nebula Creative. The analysis is primarily based on a self-reported security controls questionnaire. It is critical to note that the provided technical network scan data (\texttt{Input\_1\_Network\_Scan\_JSON}) and the list of current risks (\texttt{Input\_3\_Current\_Risks\_JSON}) were found to be corrupted and could not be processed.

The assessment reveals several critical and high-risk security gaps based on the available organizational data. The most severe findings are a complete lack of Multi-Factor Authentication (MFA) across all key systems—including email, computer logins, and sensitive data repositories. This exposes the organization to a significant risk of account compromise, which is a primary vector for major security incidents like data breaches and ransomware attacks.

Furthermore, a high-risk gap was identified in the employee onboarding process, where new hires do not receive mandatory security awareness training. This leaves the organization vulnerable to social engineering and phishing attacks.

Due to these fundamental control deficiencies, the overall security posture of Nebula Creative is assessed as \textbf{High-Risk}. Immediate remediation of the identified issues is strongly recommended. A new technical vulnerability scan is also a top priority to identify external-facing weaknesses that could not be assessed at this time.

% --- Section 2: Organizational Information ---
\section{Organizational Information}

The following details were provided for the assessment:

\begin{table}[h!]
\centering
\begin{tabular}{@{}ll@{}}
\toprule
\textbf{Item} & \textbf{Detail} \\
\midrule
Organization Name & Nebula Creative \\
Email Domain & \texttt{NebulaCreative.org} \\
Website Domain & \url{www.NebulaCreative.org} \\
Primary External IP & \seqsplit{\texttt{51.252.69.243}} \\
\bottomrule
\end{tabular}
\caption{Client Organizational Details}
\end{table}

% --- Section 3: Security Control Review ---
\section{Security Control Review}

The following table summarizes the organization's responses to a security controls questionnaire. A green checkmark (\ding{51}) indicates a positive control is in place, while a red cross (\ding{55}) indicates a control gap that introduces risk.

\begin{table}[h!]
\centering
\begin{tabular}{@{}lcc@{}}
\toprule
\textbf{Control Question} & \textbf{Response} & \textbf{Status} \\
\midrule
Do you require MFA to access email? & No & \textcolor{red}{\ding{55}} \\
Do you require MFA to log into computers? & No & \textcolor{red}{\ding{55}} \\
Do you require MFA to access sensitive data systems? & No & \textcolor{red}{\ding{55}} \\
Does your organization have an employee acceptable use policy? & Yes & \textcolor{green}{\ding{51}} \\
Does your organization do security awareness training for new employees? & No & \textcolor{red}{\ding{55}} \\
Does your organization do security awareness training for all employees annually? & Yes & \textcolor{green}{\ding{51}} \\
\bottomrule
\end{tabular}
\caption{Security Controls Questionnaire Results}
\end{table}

\subsection*{Analysis of Control Gaps}
The questionnaire reveals critical deficiencies in access control management. The absence of MFA across all platforms is a severe weakness. While it is positive that an acceptable use policy exists and annual training is conducted, the lack of training for new hires creates a window of vulnerability for every new employee.

% --- Section 4: Technical Scan Results ---
\section{Technical Scan Results}

The data file intended to contain the results of the external network scan (\texttt{Input\_1\_Network\_Scan\_JSON}) was found to be corrupted and unreadable. 

\textbf{Consequently, no analysis of open ports, running services, or potential software vulnerabilities on the external IP address (\seqsplit{\texttt{51.252.69.243}}) could be performed.}

It is imperative to conduct a new, comprehensive vulnerability scan of the organization's external-facing infrastructure to identify and remediate any technical vulnerabilities that could be exploited by attackers.

% --- Section 5: Risk Assessment ---
\section{Risk Assessment}

The following risks have been identified based on the security control gaps. The severity level is assigned based on the potential impact and likelihood of exploitation. Note that the pre-existing risk register data (\texttt{Input\_3\_Current\_Risks\_JSON}) was also unavailable due to data corruption.

\begin{table}[h!]
\centering
\begin{tabular}{@{}p{0.1\linewidth} p{0.25\linewidth} p{0.4\linewidth} p{0.15\linewidth}@{}}
\toprule
\textbf{Risk ID} & \textbf{Risk Name} & \textbf{Overview} & \textbf{Severity} \\
\midrule
R-001 & \textbf{Systemic Lack of Multi-Factor Authentication (MFA)} & The absence of MFA for email, endpoints, and sensitive data systems drastically increases the risk of unauthorized access via compromised credentials. This is a primary vector for data breaches and ransomware. & \textbf{\textcolor{sev_critical}{Critical}} \\
\addlinespace
R-002 & \textbf{Inadequate Security Onboarding for New Employees} & New employees are not receiving security awareness training upon being hired. This makes them highly susceptible to phishing and social engineering attacks, creating an immediate and persistent vulnerability. & \textbf{\textcolor{sev_high}{High}} \\
\addlinespace
R-003 & \textbf{Unknown External Attack Surface} & Due to the corrupted network scan data, the organization has no current visibility into its external technical vulnerabilities. Exploitable services or outdated software may be exposed to the internet, inviting automated or targeted attacks. & \textbf{\textcolor{sev_high}{High}} \\
\bottomrule
\end{tabular}
\caption{Identified Risks and Severity}
\end{table}

% --- Section 6: Recommendations ---
\section{Recommendations}

The following prioritized recommendations are provided to address the identified risks and improve the overall security posture of Nebula Creative.

\subsection*{Priority 1: Remediate Critical Risks}
\begin{enumerate}
    \item \textbf{Implement Multi-Factor Authentication (MFA) Immediately:}
    \begin{itemize}
        \item \textbf{Email:} Enable MFA for all user mailboxes within your email provider (e.g., Microsoft 365, Google Workspace) as the top priority.
        \item \textbf{Sensitive Systems:} Enforce MFA for access to all systems containing financial, customer, or proprietary data.
        \item \textbf{Endpoints:} Develop a roadmap to deploy MFA for all employee computer logins.
    \end{itemize}
\end{enumerate}

\subsection*{Priority 2: Remediate High Risks}
\begin{enumerate}
    \setcounter{enumi}{1}
    \item \textbf{Enhance Security Awareness Program:}
    \begin{itemize}
        \item Develop and mandate a security awareness training module for all new employees as a required part of the onboarding process. This should cover, at a minimum: phishing identification, password security, and the acceptable use policy.
    \end{itemize}
    \item \textbf{Conduct an External Vulnerability Assessment:}
    \begin{itemize}
        \item Commission a new, authenticated external network vulnerability scan against the public IP address \seqsplit{\texttt{51.252.69.243}} and any other external assets.
        \item Remediate any critical or high-severity vulnerabilities discovered during the scan, starting with the most severe.
    \end{itemize}
\end{enumerate}

\subsection*{Priority 3: Improve Security Governance}
\begin{enumerate}
    \setcounter{enumi}{3}
    \item \textbf{Establish a Data Management and Integrity Process:}
    \begin{itemize}
        \item Investigate the root cause of the data corruption for the scan and risk files to ensure the reliability of future security assessment data.
        \item Implement regular, automated vulnerability scanning (e.g., quarterly) to maintain continuous visibility of the external attack surface.
    \end{itemize}
\end{enumerate}

\end{document}
```