```latex
\documentclass[12pt]{article}

% ----------------------------------------------------------------------
% PREAMBLE
% ----------------------------------------------------------------------
\usepackage[margin=1in]{geometry}
\usepackage{pifont} % For checkmarks and crosses (\ding)
\usepackage{booktabs} % For professional-looking tables
\usepackage{hyperref} % For clickable links and references
\usepackage{url} % For formatting URLs
\usepackage{seqsplit} % To split long strings in tt font

% Hyperref setup
\hypersetup{
    colorlinks=true,
    linkcolor=black,
    filecolor=magenta,
    urlcolor=blue,
    pdftitle={Cybersecurity Posture Assessment},
    pdfpagemode=FullScreen,
}

% Define checkmark and cross symbols for clarity
\newcommand{\yes}{\ding{51}}
\newcommand{\no}{\ding{55}}

% ----------------------------------------------------------------------
% DOCUMENT START
% ----------------------------------------------------------------------
\begin{document}

% --- Title Page ---
\title{
    Cybersecurity Posture Assessment Report \\
    \large For: \textbf{Aventine Research}
}
\author{Cybersecurity Analyst}
\date{November 22, 2025}
\maketitle

\hrule
\vspace{1em}
\begin{center}
    \textbf{CONFIDENTIAL} \\
    \textit{This document contains sensitive information and is intended for the exclusive use of Aventine Research. Distribution without prior written consent is prohibited.}
\end{center}
\vspace{1em}
\hrule

\newpage

\tableofcontents

\newpage

% ----------------------------------------------------------------------
% SECTION 1: EXECUTIVE OVERVIEW
% ----------------------------------------------------------------------
\section{Executive Overview}

This report details the findings of a cybersecurity posture assessment conducted on November 22, 2025. The assessment combined an analysis of organizational security controls, a technical network scan of a key asset, and a review of pre-existing risks.

\textbf{Aventine Research} demonstrates a strong commitment to identity and access management, with mandatory Multi-Factor Authentication (MFA) across all critical systems. However, this assessment has identified significant and high-risk gaps in two key areas: employee security training and web server patch management.

The most critical findings include:
\begin{itemize}
    \item \textbf{High-Risk Vulnerability:} The external-facing web server at \texttt{192.168.10.5} is running an outdated and vulnerable version of Nginx (1.18.0). This exposes the organization to numerous publicly known exploits that could lead to a system compromise.
    \item \textbf{High-Risk Control Gap:} The organization does not provide security awareness training to new or existing employees. This represents a critical deficiency, as a well-trained workforce is the first line of defense against phishing, social engineering, and other common cyberattacks.
    \item \textbf{Medium-Risk Misconfiguration:} The SSL certificate on the web server is misconfigured, presenting a certificate for a different domain. This erodes user trust and may indicate other underlying configuration issues.
\end{itemize}

Immediate action is recommended to address these findings, focusing on software upgrades and the implementation of a comprehensive security awareness program. Detailed analysis and actionable recommendations are provided in the subsequent sections of this report.

% ----------------------------------------------------------------------
% SECTION 2: ORGANIZATIONAL INFORMATION
% ----------------------------------------------------------------------
\section{Organizational Information}

The following details were provided for the assessment. This information is used to establish the context for the technical and administrative control review.

\begin{tabular}{@{}ll}
    \toprule
    \textbf{Attribute} & \textbf{Value} \\
    \midrule
    Organization Name & \textbf{Aventine Research} \\
    Email Domain & \texttt{AventineResearch.org} \\
    Website Domain & \texttt{www.AventineResearch.org} \\
    External IP Address & \texttt{28.86.194.48} \\
    \bottomrule
\end{tabular}

% ----------------------------------------------------------------------
% SECTION 3: SECURITY CONTROL REVIEW
% ----------------------------------------------------------------------
\section{Security Control Review (Questionnaire Analysis)}

An administrative control review was conducted based on a security questionnaire. The responses indicate the current state of implemented security policies and procedures.

\subsection{Response Summary}

\begin{tabular}{@{}p{0.8\linewidth}c@{}}
    \toprule
    \textbf{Control Question} & \textbf{Response} \\
    \midrule
    Do you require MFA to access email? & \yes \\
    Do you require MFA to log into computers? & \yes \\
    Do you require MFA to access sensitive data systems? & \yes \\
    Does your organization have an employee acceptable use policy? & \yes \\
    Does your organization do security awareness training for new employees? & \no \\
    Does your organization do security awareness training for all employees at least once per year? & \no \\
    \bottomrule
\end{tabular}

\subsection{Analysis of Control Gaps}

The questionnaire reveals a critical gap in the organization's security posture. The two ``No'' responses highlight a complete lack of a security awareness training program.
\begin{itemize}
    \item \textbf{Lack of Onboarding Training:} New employees are not trained on security best practices, acceptable use, or how to identify and report threats. This leaves the organization vulnerable from the moment a new employee gains system access.
    \item \textbf{Lack of Annual Refresher Training:} Without ongoing training, employee awareness of evolving threats (like new phishing techniques) diminishes over time. This significantly increases the risk of a security incident caused by human error.
\end{itemize}
While the implementation of MFA is commendable, it does not protect against all forms of social engineering or policy violation. This gap is classified as a \textbf{High Risk}.

% ----------------------------------------------------------------------
% SECTION 4: TECHNICAL SCAN RESULTS
% ----------------------------------------------------------------------
\section{Technical Scan Results}

A network scan was performed to identify open ports, running services, and potential vulnerabilities on a target system.

\begin{itemize}
    \item \textbf{Target IP:} \texttt{192.168.10.5}
    \item \textbf{Scan Date:} 2025-11-22T10:00:00Z
\end{itemize}

\subsection{Open Ports and Services}

\begin{tabular}{@{}llll@{}}
    \toprule
    \textbf{Port} & \textbf{State} & \textbf{Service} & \textbf{Product \& Version} \\
    \midrule
    443/tcp & open & https & nginx 1.18.0 \\
    \bottomrule
\end{tabular}

\subsection{Analysis of Technical Findings}

The scan identified one open port, which is hosting a public-facing web service. Analysis of this service revealed two notable security risks.

\subsubsection{Outdated Web Server Software}
The web server is running \textbf{Nginx version 1.18.0}, which was released in April 2020 and reached its end-of-life in April 2022. This version is no longer supported with security patches and is known to be vulnerable to multiple publicly disclosed exploits (e.g., CVE-2021-23017). Running outdated, unpatched software on an internet-facing server presents a \textbf{High Risk} of system compromise.

\subsubsection{SSL/TLS Certificate Mismatch}
Further inspection of the service on port 443 revealed that the SSL/TLS certificate presented by the server has a Common Name of \texttt{www.acme-corp.com}. This does not match the organization's domain, \texttt{www.AventineResearch.org}. This misconfiguration will cause browser trust warnings for visitors, can be an indicator of a phishing attempt, and suggests a lack of proper configuration management. This finding is classified as a \textbf{Medium Risk}.

% ----------------------------------------------------------------------
% SECTION 5: CONSOLIDATED RISK ASSESSMENT
% ----------------------------------------------------------------------
\section{Consolidated Risk Assessment}
This section synthesizes all findings from the administrative and technical reviews into a consolidated list of identified risks. No pre-existing vulnerabilities were reported.

\begin{tabular}{@{}lp{0.3\linewidth}p{0.4\linewidth}l@{}}
    \toprule
    \textbf{ID} & \textbf{Risk Name} & \textbf{Description} & \textbf{Severity} \\
    \midrule
    R-01 & Outdated Web Server Software & The public-facing web server runs Nginx 1.18.0, an end-of-life version with known critical vulnerabilities. & \textbf{High} \\
    \addlinespace
    R-02 & Lack of Security Awareness Training & The organization has no security training program for new or existing employees, increasing susceptibility to phishing and social engineering. & \textbf{High} \\
    \addlinespace
    R-03 & SSL Certificate Mismatch & The server at \texttt{192.168.10.5} presents an SSL certificate that does not match the organization's domain, eroding trust and indicating a misconfiguration. & \textbf{Medium} \\
    \bottomrule
\end{tabular}

% ----------------------------------------------------------------------
% SECTION 6: RECOMMENDATIONS
% ----------------------------------------------------------------------
\section{Recommendations}
The following actions are recommended to mitigate the identified risks and improve the overall security posture of \textbf{Aventine Research}. Recommendations are prioritized based on risk severity.

\begin{enumerate}
    \item \textbf{[High] Upgrade Nginx Web Server:}
    Immediately plan and execute the upgrade of the Nginx server at \texttt{192.168.10.5} from version 1.18.0 to the latest stable version. Establish a patch management policy to ensure all internet-facing systems are updated in a timely manner.

    \item \textbf{[High] Implement Security Awareness Training Program:}
    Develop and deploy a mandatory security awareness training program. This program should include:
    \begin{itemize}
        \item An initial training module for all new employees during onboarding.
        \item Annual refresher training for all staff covering topics like phishing, password security, and acceptable use.
        \item Periodic simulated phishing campaigns to test and reinforce employee learning.
    \end{itemize}

    \item \textbf{[Medium] Correct SSL/TLS Certificate Configuration:}
    Replace the incorrect SSL/TLS certificate on the web server at \texttt{192.168.10.5}. The new certificate must list \texttt{www.AventineResearch.org} as either the Common Name (CN) or a Subject Alternative Name (SAN) to resolve browser trust errors and align with security best practices.
\end{enumerate}

% ----------------------------------------------------------------------
% DOCUMENT END
% ----------------------------------------------------------------------
\end{document}
```