```latex
\documentclass[12pt]{article}

% Preamble: Required Packages and Document Setup
\usepackage[margin=1in]{geometry}
\usepackage{pifont} % For checkmarks and crosses
\usepackage{booktabs} % For professional tables
\usepackage{hyperref} % For clickable links
\usepackage{url} % For URL formatting
\usepackage{seqsplit} % For splitting long strings to prevent overflow

\hypersetup{
    colorlinks=true,
    linkcolor=blue,
    filecolor=magenta,      
    urlcolor=cyan,
    pdftitle={Cybersecurity Posture Report},
    pdfpagemode=FullScreen,
}

\newcommand{\yes}{\ding{51}} % Green checkmark
\newcommand{\no}{\ding{55}}  % Red X

\title{Cybersecurity Posture Report for Crestview Analytics}
\author{Cybersecurity Analysis Division}
\date{November 22, 2025}

\begin{document}

\maketitle

\section*{Executive Summary}
This report provides a comprehensive analysis of the cybersecurity posture for Crestview Analytics, based on a combination of technical network scanning, a review of organizational security controls, and an assessment of known risks. The evaluation was conducted on November 22, 2025.

The analysis identified several critical and high-risk areas requiring immediate attention. While foundational controls like Multi-Factor Authentication (MFA) for email and computer access are in place, significant gaps exist in protecting sensitive data systems. Furthermore, the organization's external-facing web server is running outdated and potentially vulnerable software. Key administrative controls, such as an Acceptable Use Policy and mandatory annual security training, are also absent.

Immediate remediation is recommended to address these findings, focusing on implementing MFA for sensitive systems, upgrading server software, and establishing formal security policies and training programs to mitigate significant cyber threats.

\section{Organizational Information}
The following details were provided for the assessment.

\begin{tabular}{@{}ll}
\toprule
\textbf{Attribute} & \textbf{Value} \\
\midrule
Organization Name & \textbf{Crestview Analytics} \\
Email Domain & \texttt{CrestviewAnalytics.com} \\
External IP Address & \texttt{19.247.28.60} \\
\bottomrule
\end{tabular}

\section{Security Control Review}
A review of administrative and technical security controls was conducted via a questionnaire. The responses indicate a mixed level of maturity. While some essential controls are implemented, critical gaps were identified.

\begin{table}[h!]
\centering
\caption{Security Controls Questionnaire Results}
\begin{tabular}{@{}p{0.8\linewidth}c@{}}
\toprule
\textbf{Control Question} & \textbf{Response} \\
\midrule
Do you require MFA to access email? & \yes \\
Do you require MFA to log into computers? & \yes \\
Do you require MFA to access sensitive data systems? & \no \\
\addlinespace
Does your organization have an employee acceptable use policy? & \no \\
\addlinespace
Does your organization do security awareness training for new employees? & \yes \\
Does your organization do security awareness training for all employees at least once per year? & \no \\
\bottomrule
\end{tabular}
\end{table}

\subsection*{Analysis of Control Gaps}
The "No" responses highlight significant weaknesses:
\begin{itemize}
    \item \textbf{No MFA for Sensitive Data:} This is a critical vulnerability. Without MFA, sensitive systems are susceptible to compromise via stolen credentials, which is a common attack vector.
    \item \textbf{No Acceptable Use Policy (AUP):} The absence of an AUP means there are no formal guidelines for employees regarding the use of company assets, which can lead to unintentional security incidents and insider threats.
    \item \textbf{No Annual Security Training:} Security awareness is not a one-time event. Without annual reinforcement, employee knowledge of current threats (like phishing and social engineering) diminishes, increasing the organization's risk exposure.
\end{itemize}

\section{Technical Scan Results}
An external network scan was performed against the target IP address \texttt{192.168.10.5} to identify open ports and exposed services.

\begin{table}[h!]
\centering
\caption{Open Port and Service Information}
\begin{tabular}{@{}lllll@{}}
\toprule
\textbf{Port} & \textbf{State} & \textbf{Service} & \textbf{Product} & \textbf{Version} \\
\midrule
443/tcp & open & https & nginx & 1.18.0 \\
\bottomrule
\end{tabular}
\end{table}

\subsection*{Analysis of Technical Findings}
The scan revealed a web server running \textbf{nginx version 1.18.0}. This version was released in April 2020 and is now significantly outdated. Legacy software is a primary target for attackers, as it often contains numerous publicly known and unpatched vulnerabilities (CVEs). Running this version poses a high risk of compromise to the web server and any data it processes or hosts.

\section{Consolidated Risk Assessment}
The following table synthesizes findings from the security control review and technical scan into a prioritized list of risks. No pre-existing vulnerabilities were reported.

\begin{table}[h!]
\centering
\caption{Summary of Identified Risks}
\begin{tabular}{@{}lp{0.5\linewidth}l@{}}
\toprule
\textbf{Risk Name} & \textbf{Overview} & \textbf{Severity} \\
\midrule
\textbf{Lack of MFA on Sensitive Systems} & The absence of MFA on critical data systems makes them highly vulnerable to unauthorized access via compromised credentials. & \textbf{Critical} \\
\addlinespace
\textbf{Outdated Web Server Software} & The public-facing nginx server (v1.18.0) is outdated and likely contains unpatched vulnerabilities, exposing the system to remote exploitation. & \textbf{High} \\
\addlinespace
\textbf{No Annual Security Training} & Without recurring training, employees are more likely to fall victim to phishing, social engineering, and other common attacks. & \textbf{High} \\
\addlinespace
\textbf{No Employee Acceptable Use Policy} & Lack of a formal policy creates ambiguity and increases the risk of misuse of company resources and data, whether intentional or accidental. & \textbf{High} \\
\bottomrule
\end{tabular}
\end{table}

\section{Recommendations}
Based on the analysis, the following actions are recommended to strengthen the security posture of Crestview Analytics. These recommendations are prioritized by severity.

\begin{enumerate}
    \item \textbf{Implement MFA for Sensitive Systems (Critical):} Immediately deploy a robust MFA solution for all systems classified as containing sensitive or critical data. This is the most effective single control to prevent unauthorized access.
    
    \item \textbf{Upgrade Web Server (High):} Plan and execute an upgrade of the nginx server from version 1.18.0 to the latest stable version. Before upgrading, test for compatibility with the existing web application. Regularly schedule patch management for all external-facing systems.
    
    \item \textbf{Establish Annual Security Training (High):} Develop and mandate an annual security awareness training program for all employees. The training should cover current threats, such as phishing, ransomware, and proper data handling.
    
    \item \textbf{Develop an Acceptable Use Policy (High):} Create a formal Acceptable Use Policy (AUP) that clearly defines the rules and responsibilities for all employees when using company technology and data. Ensure all employees read and acknowledge the policy.
\end{enumerate}

\end{document}
```