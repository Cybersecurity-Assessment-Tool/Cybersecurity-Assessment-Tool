```latex
\documentclass[12pt]{article}

% Preamble: Required Packages
\usepackage[margin=1in]{geometry}
\usepackage{pifont} % For checkmarks and crosses
\usepackage{booktabs} % For professional-looking tables
\usepackage{hyperref} % For clickable links and metadata
\usepackage{url} % For formatting URLs
\usepackage{seqsplit} % For splitting long strings in texttt
\usepackage{graphicx} % For logo (optional, placeholder)
\usepackage{xcolor} % For colors in text

% Document Metadata
\hypersetup{
    colorlinks=true,
    linkcolor=blue,
    filecolor=magenta,      
    urlcolor=cyan,
    pdftitle={Cybersecurity Posture Report},
    pdfauthor={Cybersecurity Analysis Team},
    pdfsubject={Security Assessment},
    pdfkeywords={Cybersecurity, Nmap, Risk Assessment},
}

% Custom Commands
\newcommand{\yes}{\ding{51}}
\newcommand{\no}{\ding{55}}

% --- Document Start ---
\begin{document}

% --- Title Page ---
\begin{titlepage}
    \centering
    \vspace*{1cm}
    
    \Huge
    \textbf{Cybersecurity Posture Report}
    
    \vspace{1.5cm}
    
    \Large
    Prepared for: \\
    \vspace{0.5cm}
    \textbf{Obsidian Operatives}
    
    \vspace{2cm}
    
    \large
    \textbf{Date of Report:} \today \\
    \textbf{Date of Assessment:} 2025-11-22
    
    \vfill
    
    \large
    \textbf{Generated by:} \\
    Cybersecurity Analysis Team
    
\end{titlepage}

\tableofcontents
\newpage

% --- Section 1: Executive Overview ---
\section{Executive Overview}
This report details the findings of a cybersecurity assessment conducted on \textbf{Obsidian Operatives}. The analysis combines a review of organizational security controls, an external network scan, and an evaluation of known risks. The assessment was performed on \textbf{2025-11-22}.

The organization's overall security posture is moderate, with several effective controls in place. However, we have identified \textbf{two critical gaps} and \textbf{one high-risk technical vulnerability} that require immediate attention to prevent potential security incidents.

Key findings include:
\begin{itemize}
    \item \textbf{Critical Control Gap:} Multi-Factor Authentication (MFA) is not enforced for email access, exposing the organization to a high risk of business email compromise and phishing attacks.
    \item \textbf{Critical Control Gap:} New employees do not receive security awareness training, creating a significant vulnerability from their first day.
    \item \textbf{High-Risk Technical Finding:} The public-facing web server is running an outdated version of Nginx (1.18.0), which is known to have multiple security vulnerabilities.
\end{itemize}

This report provides a detailed breakdown of these findings and offers actionable recommendations to mitigate the identified risks and strengthen the organization's security posture.

% --- Section 2: Organizational Information ---
\section{Organizational Information}
The following details were provided for the assessment. This information is used to establish the context and scope of the analysis.

\begin{tabular}{@{}ll}
\toprule
\textbf{Attribute} & \textbf{Value} \\
\midrule
Organization Name & \textbf{Obsidian Operatives} \\
Email Domain & \texttt{ObsidianOperatives.net} \\
Website Domain & \seqsplit{\texttt{www.ObsidianOperatives.net}} \\
External IP Address & \texttt{80.239.6.199} \\
\bottomrule
\end{tabular}

% --- Section 3: Security Control Review ---
\section{Security Control Review}
A review of organizational security controls was conducted via a standardized questionnaire. The responses highlight the current state of administrative and policy-based security measures. Gaps identified here often represent significant organizational risk.

\begin{table}[h!]
\centering
\caption{Security Controls Questionnaire Results}
\begin{tabular}{@{}p{0.75\textwidth}c@{}}
\toprule
\textbf{Control Question} & \textbf{Response} \\
\midrule
Do you require MFA to access email? & \textcolor{red}{\no} \\
Do you require MFA to log into computers? & \textcolor{green}{\yes} \\
Do you require MFA to access sensitive data systems? & \textcolor{green}{\yes} \\
Does your organization have an employee acceptable use policy? & \textcolor{green}{\yes} \\
Does your organization do security awareness training for new employees? & \textcolor{red}{\no} \\
Does your organization do security awareness training for all employees at least once per year? & \textcolor{green}{\yes} \\
\bottomrule
\end{tabular}
\end{table}

\subsection*{Analysis of Control Gaps}
\begin{itemize}
    \item \textbf{Lack of MFA on Email:} This is a critical vulnerability. Email accounts are primary targets for attackers seeking to gain an initial foothold, conduct phishing campaigns, or perform financial fraud.
    \item \textbf{No Security Training for New Hires:} New employees are often prime targets for social engineering. Without immediate training, they are unaware of company policies and common threats, posing a risk to the entire organization.
\end{itemize}

% --- Section 4: Technical Scan Results ---
\section{Technical Scan Results}
A network scan was performed on the target system to identify open ports, running services, and potential vulnerabilities.

\begin{itemize}
    \item \textbf{Target IP:} \texttt{192.168.10.5}
    \item \textbf{Scan Date:} 2025-11-22T10:00:00Z
\end{itemize}

\begin{table}[h!]
\centering
\caption{Open Port and Service Information}
\begin{tabular}{@{}lllll@{}}
\toprule
\textbf{Port} & \textbf{State} & \textbf{Service} & \textbf{Product} & \textbf{Version} \\
\midrule
443/tcp & open & https & nginx & \textbf{1.18.0} \\
\bottomrule
\end{tabular}
\end{table}

\subsection*{Analysis of Technical Findings}
The scan identified one open port running a web server.
\begin{itemize}
    \item \textbf{Outdated Nginx Version:} The identified Nginx version \texttt{1.18.0} was released in April 2020. It is significantly outdated and is affected by numerous publicly disclosed vulnerabilities (CVEs). Attackers can exploit these known weaknesses to compromise the server, potentially leading to data breaches or further network intrusion.
    \item \textbf{SSL Certificate Mismatch:} The SSL certificate's Common Name is listed as \texttt{www.acme-corp.com}, which does not match the organization's domain (\texttt{www.ObsidianOperatives.net}). This is a misconfiguration that can cause browser trust errors for visitors and may indicate a lack of proper system management.
\end{itemize}

% --- Section 5: Risk Assessment Summary ---
\section{Risk Assessment Summary}
The following table synthesizes findings from the security control review and the technical scan into a prioritized list of risks.

\begin{table}[h!]
\centering
\caption{Consolidated Risk Register}
\begin{tabular}{@{}lp{0.6\textwidth}l@{}}
\toprule
\textbf{Risk ID} & \textbf{Description} & \textbf{Severity} \\
\midrule
RISK-001 & Lack of MFA on email exposes the organization to account takeovers and business email compromise. & \textbf{Critical} \\
\addlinespace
RISK-002 & The public-facing web server runs an outdated version of Nginx (\texttt{1.18.0}) with known vulnerabilities. & \textbf{High} \\
\addlinespace
RISK-003 & New employees are not provided with security awareness training, making them susceptible to social engineering. & \textbf{High} \\
\addlinespace
RISK-004 & The SSL certificate on the web server is misconfigured, which can erode user trust and indicates a management gap. & \textbf{Medium} \\
\bottomrule
\end{tabular}
\end{table}

% --- Section 6: Recommendations ---
\section{Recommendations}
The following actions are recommended to mitigate the identified risks and improve the overall security posture of \textbf{Obsidian Operatives}.

\subsection*{RISK-001: Lack of MFA on Email (Critical)}
\begin{itemize}
    \item \textbf{Immediate Action:} Enforce mandatory Multi-Factor Authentication (MFA) for all user accounts on the email platform.
    \item \textbf{Policy:} Update the organization's access control policy to require MFA for all externally accessible services, especially email.
\end{itemize}

\subsection*{RISK-002: Outdated Nginx Web Server (High)}
\begin{itemize}
    \item \textbf{Immediate Action:} Plan and execute an upgrade of the Nginx server at \texttt{192.168.10.5} to the latest stable version. Test the application for compatibility before deploying to production.
    \item \textbf{Policy:} Implement a formal patch management policy that includes regular vulnerability scanning and a defined schedule for applying security patches to all systems.
\end{itemize}

\subsection*{RISK-003: No Security Training for New Hires (High)}
\begin{itemize}
    \item \textbf{Immediate Action:} Develop a security awareness training module and integrate it into the mandatory onboarding process for all new employees.
    \item \textbf{Content:} The training should cover, at a minimum, phishing identification, acceptable use of company assets, and password security policies.
\end{itemize}

\subsection*{RISK-004: SSL Certificate Misconfiguration (Medium)}
\begin{itemize}
    \item \textbf{Action:} Replace the current SSL certificate on the web server with one that is correctly issued for the domain \texttt{www.ObsidianOperatives.net}.
    \item \textbf{Process:} Implement a certificate lifecycle management process to track expiration dates and ensure correct configuration for all public-facing services.
\end{itemize}

\end{document}
```