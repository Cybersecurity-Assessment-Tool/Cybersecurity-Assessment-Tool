```latex
\documentclass[12pt]{article}

% Preamble: Required Packages
\usepackage[margin=1in]{geometry} % Set margins
\usepackage{pifont}               % For checkmarks and crosses (\ding)
\usepackage{booktabs}             % For professional tables (\toprule, \midrule, \bottomrule)
\usepackage{hyperref}             % For clickable links and better PDF metadata
\usepackage{url}                  % For formatting URLs
\usepackage{seqsplit}             % For splitting long strings without spaces
\usepackage{graphicx}             % For potential logos (not used here, but good practice)
\usepackage{xcolor}               % For colors in text

% Document Metadata
\hypersetup{
    colorlinks=true,
    linkcolor=blue,
    filecolor=magenta,      
    urlcolor=cyan,
    pdftitle={Cybersecurity Posture Assessment Report},
    pdfauthor={Cybersecurity Analyst},
    pdfsubject={Security Analysis},
    pdfkeywords={Security, Report, Analysis},
}

% Custom Commands
\newcommand{\yes}{\ding{51}} % Green checkmark
\newcommand{\no}{\ding{55}}  % Red X

\begin{document}

% --- Title Page ---
\begin{titlepage}
    \centering
    \vspace*{1cm}
    \Huge\textbf{Cybersecurity Posture Assessment Report}
    \vspace{1.5cm}
    \Large
    \textbf{Prepared for:}\\
    \vspace{0.5cm}
    \textbf{Nebula Creative}
    \vspace{2cm}
    \large
    \textbf{Date of Report:}\\
    \today
    \vfill
    \large
    \textbf{Generated By:}\\
    Expert Cybersecurity Analyst
\end{titlepage}

\tableofcontents
\newpage

% --- Section 1: Executive Summary ---
\section{Executive Summary}
This report provides a comprehensive cybersecurity assessment for \textbf{Nebula Creative}, synthesizing findings from technical network scans, a review of organizational security controls, and an analysis of pre-existing risks. The assessment was conducted to identify vulnerabilities, security gaps, and misconfigurations that could expose the organization to cyber threats.

The analysis revealed several critical and high-risk findings that require immediate attention. A key vulnerability was discovered on an internal server (\texttt{10.0.0.15}), which is running a critically outdated and backdoored version of the vsftpd FTP service. This is compounded by an insecure configuration allowing anonymous user access.

Furthermore, significant gaps were identified in organizational security policies. The absence of Multi-Factor Authentication (MFA) on email accounts presents a high risk of business email compromise. The lack of a formal Acceptable Use Policy for employees also introduces unnecessary risk.

This report details these findings and provides prioritized, actionable recommendations to mitigate the identified risks and strengthen the overall security posture of \textbf{Nebula Creative}.

% --- Section 2: Organizational Information ---
\section{Organizational Information}
The following information was provided for the assessment. This data provides context for the organization's digital footprint and internal policies.

\begin{tabular}{@{}ll}
    \toprule
    \textbf{Attribute} & \textbf{Value} \\
    \midrule
    Organization Name & \textbf{Nebula Creative} \\
    Email Domain & \texttt{NebulaCreative.net} \\
    Website Domain & \url{www.NebulaCreative.net} \\
    External IP Address & \texttt{141.228.167.162} \\
    \bottomrule
\end{tabular}

% --- Section 3: Security Control Review ---
\section{Security Control Review}
A review of administrative and policy-based security controls was conducted via a questionnaire. The results highlight areas of strength and identify significant gaps in the current security framework. "No" answers indicate a deviation from security best practices and represent areas for improvement.

\begin{tabular}{@{}p{0.75\linewidth}c@{}}
    \toprule
    \textbf{Security Control Question} & \textbf{Status} \\
    \midrule
    Do you require MFA to access email? & \no \\
    Do you require MFA to log into computers? & \yes \\
    Do you require MFA to access sensitive data systems? & \yes \\
    Does your organization have an employee acceptable use policy? & \no \\
    Does your organization do security awareness training for new employees? & \yes \\
    Does your organization do security awareness training for all employees at least once per year? & \yes \\
    \bottomrule
\end{tabular}

% --- Section 4: Technical Scan Results ---
\section{Technical Scan Results}
A network scan was performed to identify open ports, running services, and potential vulnerabilities on target systems.

\subsection{Scan Target}
\begin{itemize}
    \item \textbf{Target IP Address:} \texttt{10.0.0.15}
    \item \textbf{Scan Date:} Not specified in scan data
\end{itemize}

\subsection{Open Ports and Services}
The following services were identified as accessible on the target system.

\begin{tabular}{@{}llllll@{}}
    \toprule
    \textbf{Port} & \textbf{State} & \textbf{Service} & \textbf{Product} & \textbf{Version} & \textbf{Analyst Notes} \\
    \midrule
    21/tcp & Open & ftp & vsftpd & 2.3.4 & \textbf{CRITICAL:} Vulnerable to backdoor (CVE-2011-2523). \\
           &      &     &        &       & \textbf{HIGH:} Anonymous FTP login is allowed. \\
    \bottomrule
\end{tabular}

% --- Section 5: Consolidated Risk Assessment ---
\section{Consolidated Risk Assessment}
This section correlates findings from the security control review, technical scans, and pre-existing risk data to provide a unified view of the organization's risk landscape. Risks are prioritized based on their potential impact and likelihood of exploitation.

\begin{tabular}{@{}p{0.1\linewidth}p{0.4\linewidth}p{0.2\linewidth}p{0.2\linewidth}@{}}
    \toprule
    \textbf{Risk ID} & \textbf{Description} & \textbf{Severity} & \textbf{Affected Systems} \\
    \midrule
    \textbf{RISK-001} & A public-facing FTP server is running vsftpd 2.3.4, which contains a known critical backdoor (CVE-2011-2523), allowing an attacker to gain remote command execution. & \textbf{Critical} & Server at \texttt{10.0.0.15} \\
    \addlinespace
    \textbf{RISK-002} & The FTP server is configured to allow anonymous logins, permitting unauthorized users to access, upload, or download files, potentially leading to data leakage or malware distribution. & \textbf{High} & Server at \texttt{10.0.0.15} \\
    \addlinespace
    \textbf{RISK-003} & Email accounts are not protected by Multi-Factor Authentication (MFA). This significantly increases the risk of account takeover via phishing or credential stuffing attacks. & \textbf{High} & All email accounts \\
    \addlinespace
    \textbf{RISK-004} & The organization lacks a formal Acceptable Use Policy (AUP). This creates ambiguity for employees regarding secure behavior and limits the organization's ability to enforce security standards. & \textbf{Medium} & All employees \\
    \addlinespace
    \textbf{RISK-005} & Existing risk: Workstations are running the outdated and unsupported Windows 7 operating system, which no longer receives security updates. (CVSS 5.0) & \textbf{Medium} & Workstations \\
    \bottomrule
\end{tabular}

% --- Section 6: Recommendations ---
\section{Recommendations}
The following actions are recommended to mitigate the identified risks. They are prioritized to address the most severe threats first.

\subsection{Critical Priority}
\begin{itemize}
    \item \textbf{Remediate Vulnerable FTP Service (RISK-001):} Immediately take the server at \texttt{10.0.0.15} offline. The vsftpd 2.3.4 software must be upgraded to the latest stable version or replaced with a secure file transfer solution (e.g., SFTP). If FTP is not a business requirement, the service should be disabled entirely.
\end{itemize}

\subsection{High Priority}
\begin{itemize}
    \item \textbf{Disable Anonymous FTP (RISK-002):} Once the FTP service is patched, reconfigure it to disable anonymous access. All access should require authentication with strong, unique credentials.
    \item \textbf{Implement MFA for Email (RISK-003):} Enforce mandatory MFA for all user accounts on the \texttt{NebulaCreative.net} email domain. This is the single most effective control to prevent account compromise.
\end{itemize}

\subsection{Medium Priority}
\begin{itemize}
    \item \textbf{Develop an Acceptable Use Policy (RISK-004):} Create and implement a formal AUP that clearly defines rules for the use of company technology, data handling, and internet access. Ensure all employees read and acknowledge the policy.
    \item \textbf{Upgrade Outdated Operating Systems (RISK-005):} Continue with the plan to upgrade or replace all workstations running Windows 7. Prioritize systems used by employees who handle sensitive data.
\end{itemize}

\end{document}
```