```latex
\documentclass[12pt]{article}

% --- PACKAGES ---
\usepackage[margin=1in]{geometry}
\usepackage{pifont} % For checkmarks and crosses
\usepackage{booktabs} % For professional tables
\usepackage{hyperref} % For clickable links
\usepackage{url}      % For URL formatting
\usepackage{seqsplit} % For splitting long strings
\usepackage{xcolor}   % For colors in text

% --- DOCUMENT SETUP ---
\hypersetup{
    colorlinks=true,
    linkcolor=blue,
    filecolor=magenta,      
    urlcolor=cyan,
}

% --- TITLE ---
\title{Cybersecurity Assessment Report \\ \large For: Hidden Gem}
\author{Cybersecurity Analyst}
\date{\today}

% --- DOCUMENT START ---
\begin{document}

\maketitle
\thispagestyle{empty}
\newpage
\tableofcontents
\newpage

% ===================================================================
\section{Executive Summary}
% ===================================================================

This report details the findings of a cybersecurity assessment for Hidden Gem, conducted on \today. The assessment combined a review of organizational security controls, an external network scan, and an analysis of pre-existing risks.

The analysis uncovered a \textbf{critical vulnerability} on an internal network host (\texttt{10.0.0.15}). An outdated and misconfigured FTP server (\texttt{vsftpd 2.3.4}) was identified, which is known to contain a backdoor vulnerability (CVE-2011-2523) and allows for anonymous, unauthenticated access. This finding presents an immediate and severe risk of unauthorized access, data breach, and potential network compromise.

Furthermore, significant gaps were identified in the organization's security policies. The absence of an Acceptable Use Policy and the lack of mandatory annual security awareness training for all staff create a high-risk environment susceptible to insider threats and social engineering attacks.

This report provides a detailed breakdown of these findings and offers prioritized, actionable recommendations to mitigate the identified risks and strengthen the overall security posture of Hidden Gem. Immediate action is required to address the critical FTP server vulnerability.

% ===================================================================
\section{Organizational Information}
% ===================================================================

The following information was provided for the assessment.

\begin{itemize}
    \item \textbf{Organization Name:} Hidden Gem
    \item \textbf{Email Domain:} \texttt{HiddenGem.net}
    \item \textbf{Website Domain:} \url{www.HiddenGem.net}
    \item \textbf{External IP Address:} \texttt{20.53.59.255}
\end{itemize}

% ===================================================================
\section{Security Control Review}
% ===================================================================

A review of the organization's security controls was conducted via a questionnaire. The results are summarized below. "Yes" answers (\ding{51}) indicate a control is in place, while "No" answers (\ding{55}) represent a security gap.

\subsection{Questionnaire Results}

\begin{table}[h!]
\centering
\begin{tabular}{p{0.8\linewidth} c}
\toprule
\textbf{Control Question} & \textbf{Status} \\
\midrule
Do you require MFA to access email? & \ding{51} \\
Do you require MFA to log into computers? & \ding{51} \\
Do you require MFA to access sensitive data systems? & \ding{51} \\
Does your organization have an employee acceptable use policy? & \textcolor{red}{\ding{55}} \\
Does your organization do security awareness training for new employees? & \ding{51} \\
Does your organization do security awareness training for all employees at least once per year? & \textcolor{red}{\ding{55}} \\
\bottomrule
\end{tabular}
\caption{Security Control Questionnaire Summary}
\end{table}

\subsection{Analysis of Gaps}
The questionnaire reveals two high-risk procedural gaps:
\begin{enumerate}
    \item \textbf{No Acceptable Use Policy (AUP):} The lack of an AUP creates ambiguity regarding the proper use of company assets. This increases the risk of misuse, whether accidental or malicious, and complicates disciplinary action in the event of a policy violation.
    \item \textbf{No Annual Security Awareness Training:} While new employees receive training, the absence of an annual refresher for all staff is a significant weakness. The threat landscape evolves constantly, and without ongoing education, employees are more likely to fall victim to phishing, malware, and other social engineering tactics.
\end{enumerate}

% ===================================================================
\section{Technical Scan Results}
% ===================================================================

An Nmap scan was performed on the internal network target \texttt{10.0.0.15}. The scan identified one open port with a critically vulnerable service.

\subsection{Host: \texttt{10.0.0.15}}
\begin{itemize}
    \item \textbf{Status:} Up
    \item \textbf{Open Ports Summary:}
\end{itemize}

\begin{table}[h!]
\centering
\begin{tabular}{l l l l}
\toprule
\textbf{Port} & \textbf{Service} & \textbf{Product / Version} & \textbf{Finding} \\
\midrule
21/tcp & ftp & vsftpd 2.3.4 & \textbf{Critical Vulnerability} \\
\bottomrule
\end{tabular}
\caption{Open Ports and Services on \texttt{10.0.0.15}}
\end{table}

\subsection{Critical Findings Detail}
The FTP service running on port 21 presents two immediate and severe risks:
\begin{enumerate}
    \item \textbf{Known Backdoor (CVE-2011-2523):} The identified version, \texttt{vsftpd 2.3.4}, contains a well-documented backdoor that was inserted into the source code. An attacker can gain a command shell on the server by sending a specific string as the username, leading to a full system compromise.
    \item \textbf{Anonymous FTP Login Allowed:} The scan confirmed that anonymous login is enabled. This allows any user on the network to access, download, or potentially upload files without authentication. This could lead to sensitive data exposure or allow an attacker to place malicious files on the server.
\end{enumerate}

% ===================================================================
\section{Consolidated Risk Assessment}
% ===================================================================

The following table synthesizes findings from the technical scan, the control review, and pre-existing risk data into a prioritized list.

\begin{table}[h!]
\centering
\begin{tabular}{p{0.25\linewidth} p{0.45\linewidth} l l}
\toprule
\textbf{Risk Name} & \textbf{Overview} & \textbf{Severity} & \textbf{Affected Systems} \\
\midrule
\textbf{FTP Server Backdoor} & The FTP server (\texttt{vsftpd 2.3.4}) has a known remote code execution vulnerability (CVE-2011-2523). & \textbf{Critical} & Server at \texttt{10.0.0.15} \\
\addlinespace
\textbf{Anonymous FTP Access} & Unauthenticated users can access files via FTP, risking data leakage or malware upload. & \textbf{Critical} & Server at \texttt{10.0.0.15} \\
\addlinespace
\textbf{No Annual Security Training} & Lack of ongoing training increases susceptibility to phishing and social engineering attacks. & High & All Employees \\
\addlinespace
\textbf{No Acceptable Use Policy} & Absence of a formal policy creates risk of intentional or accidental misuse of company assets. & High & All Employees \\
\addlinespace
\textbf{Outdated Windows Policy} & Workstations are running Windows 7, which is an unsupported operating system. & Medium & Workstations \\
\bottomrule
\end{tabular}
\caption{Summary of Identified Risks}
\end{table}

% ===================================================================
\section{Recommendations}
% ===================================================================

The following actions are recommended to mitigate the identified risks, prioritized by severity.

\subsection{Immediate Actions (Critical Priority)}
\begin{enumerate}
    \item \textbf{Isolate the FTP Server:} Immediately disconnect the server at \texttt{10.0.0.15} from the network to prevent exploitation.
    \item \textbf{Investigate for Compromise:} Conduct a forensic analysis of the server to determine if the backdoor has already been exploited.
    \item \textbf{Decommission or Remediate:}
        \begin{itemize}
            \item If the FTP service is not essential, it should be permanently disabled and removed.
            \item If it is required, the server must be rebuilt with a patched, secure version of the FTP software (or preferably, replaced with a secure alternative like SFTP). Anonymous access must be disabled.
        \end{itemize}
\end{enumerate}

\subsection{High Priority Actions}
\begin{enumerate}
    \item \textbf{Develop and Implement an AUP:} Draft a comprehensive Acceptable Use Policy that clearly defines rules for all employees regarding the use of company networks, systems, and data. Require all employees to read and acknowledge the policy.
    \item \textbf{Establish Annual Security Training:} Implement a mandatory security awareness training program for all employees to be completed annually. The training should cover current threats such as phishing, ransomware, and secure data handling.
\end{enumerate}

\subsection{Medium Priority Actions}
\begin{enumerate}
    \item \textbf{Upgrade Outdated Workstations:} Continue with the existing plan to upgrade all Windows 7 workstations to a supported operating system like Windows 10 or 11 to ensure they receive critical security patches.
\end{enumerate}

\end{document}
```