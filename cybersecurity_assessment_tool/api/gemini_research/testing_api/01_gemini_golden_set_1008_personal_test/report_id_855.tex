```latex
\documentclass[12pt]{article}

% --- PACKAGES ---
\usepackage[margin=1in]{geometry} % Set page margins
\usepackage{pifont}                % For checkmark and cross symbols
\usepackage{booktabs}              % For professional-looking tables
\usepackage{hyperref}              % For hyperlinks and document metadata
\usepackage{url}                   % For formatting URLs
\usepackage{seqsplit}              % For splitting long strings in tt font
\usepackage{graphicx}              % For logos (optional)
\usepackage{xcolor}                % For colors

% --- DOCUMENT METADATA ---
\hypersetup{
    colorlinks=true,
    linkcolor=blue,
    filecolor=magenta,      
    urlcolor=cyan,
    pdftitle={Cybersecurity Assessment Report},
    pdfpagemode=FullScreen,
}

% --- CUSTOM COMMANDS ---
\newcommand{\yes}{\ding{51}} % Checkmark
\newcommand{\no}{\ding{55}}  % Cross

% --- TITLE ---
\title{Cybersecurity Assessment Report \\ \large For: \textbf{Borealis Tech}}
\author{Cybersecurity Analyst}
\date{\today}

% --- DOCUMENT START ---
\begin{document}

\maketitle
\thispagestyle{empty}
\newpage

\tableofcontents
\newpage

% ===================================================================
% SECTION 1: EXECUTIVE SUMMARY
% ===================================================================
\section{Executive Summary}

This report details the findings of a cybersecurity assessment conducted for \textbf{Borealis Tech}. The analysis is based on a combination of technical network scanning, a review of organizational security controls, and an evaluation of pre-existing risk documentation.

The assessment identified several high-priority risks that require immediate attention. A key technical finding confirmed a pre-existing risk: a MySQL database on an internal host (\texttt{172.16.50.20}) is exposed with port 3306 open to the network. Critically, this database is running MySQL version 5.7.33, which is past its official End-of-Life (EOL) and no longer receives security patches.

Furthermore, significant gaps were identified in the organization's security policies. The lack of Multi-Factor Authentication (MFA) on the primary email system (\texttt{BorealisTech.net}) presents a critical vulnerability to account takeover and phishing attacks. This is compounded by a complete absence of a security awareness training program for employees, who represent the first line of defense.

Collectively, these findings indicate a significant risk of data breach, ransomware infection, and business email compromise. This report provides specific, actionable recommendations to mitigate these risks and improve the overall security posture of the organization.

% ===================================================================
% SECTION 2: ORGANIZATIONAL INFORMATION
% ===================================================================
\section{Organizational Information}

The following information was provided for the assessment.

\begin{tabular}{@{}ll}
    \toprule
    \textbf{Attribute} & \textbf{Value} \\
    \midrule
    Organization Name & \textbf{Borealis Tech} \\
    Email Domain & \seqsplit{\texttt{BorealisTech.net}} \\
    Website Domain & \seqsplit{\url{www.BorealisTech.net}} \\
    External IP Address & \seqsplit{\texttt{141.128.85.167}} \\
    \bottomrule
\end{tabular}

% ===================================================================
% SECTION 3: SECURITY CONTROL REVIEW
% ===================================================================
\section{Security Control Review}

A review of the organization's security controls was conducted via a questionnaire. The responses highlight critical gaps in foundational security practices.

\begin{table}[h!]
\centering
\begin{tabular}{@{}p{8.5cm}cc@{}}
    \toprule
    \textbf{Control Question} & \textbf{Response} & \textbf{Status} \\
    \midrule
    Do you require MFA to access email? & \no & \textcolor{red}{\textbf{Critical Gap}} \\
    Do you require MFA to log into computers? & \yes & Best Practice \\
    Do you require MFA to access sensitive data systems? & \yes & Best Practice \\
    Does your organization have an employee acceptable use policy? & \yes & Best Practice \\
    Does your organization do security awareness training for new employees? & \no & \textcolor{red}{\textbf{High Risk}} \\
    Does your organization do security awareness training for all employees at least once per year? & \no & \textcolor{red}{\textbf{High Risk}} \\
    \bottomrule
\end{tabular}
\caption{Organizational Security Control Status}
\end{table}

% ===================================================================
% SECTION 4: TECHNICAL SCAN RESULTS
% ===================================================================
\section{Technical Scan Results}

A network scan was performed on the specified target to identify open ports and running services.

\begin{itemize}
    \item \textbf{Target IP Address:} \texttt{172.16.50.20}
\end{itemize}

\begin{table}[h!]
\centering
\begin{tabular}{@{}lllll@{}}
    \toprule
    \textbf{Port} & \textbf{State} & \textbf{Service} & \textbf{Product} & \textbf{Version} \\
    \midrule
    3306/tcp & open & mysql & MySQL & 5.7.33 \\
    \bottomrule
\end{tabular}
\caption{Open Ports and Services Detected on \texttt{172.16.50.20}}
\end{table}

\subsection*{Analysis}
The scan confirms that port 3306 is open, exposing a MySQL database service directly to the network. The detected version, \textbf{MySQL 5.7.33}, reached its official End-of-Life (EOL) in October 2023. EOL software no longer receives security updates from the vendor, leaving it perpetually vulnerable to newly discovered exploits. This finding corroborates the pre-existing risk documented in \texttt{Input\_3\_Current\_Risks\_JSON} and elevates its severity due to the EOL status.

% ===================================================================
% SECTION 5: CONSOLIDATED RISK ASSESSMENT
% ===================================================================
\section{Consolidated Risk Assessment}

The following table synthesizes findings from the security control review, technical scan, and pre-existing risk data into a consolidated list of key risks.

\begin{table}[h!]
\centering
\begin{tabular}{@{}p{3.5cm}p{2cm}p{8cm}@{}}
    \toprule
    \textbf{Risk Name} & \textbf{Severity} & \textbf{Description} \\
    \midrule
    \textbf{Exposed End-of-Life Database} & \textcolor{red}{\textbf{Critical}} & A MySQL 5.7.33 database (EOL) is directly exposed to the network on port 3306. This presents a severe risk of data breach from unpatched vulnerabilities. (CVSS 7.5) \\
    \addlinespace
    \textbf{No MFA on Email System} & \textcolor{red}{\textbf{Critical}} & The lack of MFA on the primary email system makes the organization highly susceptible to account compromise via phishing or credential stuffing, which can lead to widespread system access and data exfiltration. \\
    \addlinespace
    \textbf{No Security Awareness Training} & \textcolor{orange}{\textbf{High}} & The absence of an employee security training program significantly increases the organization's vulnerability to social engineering attacks, as personnel are not equipped to identify or report threats. \\
    \bottomrule
\end{tabular}
\caption{Summary of Identified Risks}
\end{table}

% ===================================================================
% SECTION 6: RECOMMENDATIONS
% ===================================================================
\section{Recommendations}

The following actions are recommended to mitigate the identified risks. They are prioritized based on severity and potential impact.

\subsection*{1. Remediate Exposed Database (Priority: Critical)}
\begin{itemize}
    \item \textbf{Immediate Action:} Implement strict firewall rules to block all external and non-essential internal access to port 3306 on host \texttt{172.16.50.20}. Access should be restricted to only the application servers that explicitly require it.
    \item \textbf{Short-Term Plan:} Initiate a project to migrate the database from the EOL MySQL 5.7 to a fully supported version (e.g., MySQL 8.x or a managed cloud equivalent).
    \item \textbf{Long-Term Strategy:} For any future database deployments, adopt a policy of default denial and require access to be brokered through a secure channel, such as a VPN or bastion host, rather than direct network exposure.
\end{itemize}

\subsection*{2. Implement MFA for Email (Priority: Critical)}
\begin{itemize}
    \item \textbf{Immediate Action:} Enable and enforce MFA for all user accounts on the \texttt{BorealisTech.net} email platform. This is the single most effective control to prevent unauthorized account access.
\end{itemize}

\subsection*{3. Establish Security Awareness Program (Priority: High)}
\begin{itemize}
    \item \textbf{Short-Term Plan:} Procure and deploy a security awareness training solution. All current employees must complete foundational training covering phishing, password security, and acceptable use policies within the next quarter. This training should be mandatory for all new hires.
    \item \textbf{Long-Term Strategy:} Schedule annual refresher training for all staff and conduct periodic, unannounced phishing simulations to measure effectiveness and reinforce learning.
\end{itemize}

\end{document}
```