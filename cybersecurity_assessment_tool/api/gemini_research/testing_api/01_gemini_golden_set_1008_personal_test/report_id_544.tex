```latex
\documentclass[12pt]{article}

% Preamble: Required Packages
\usepackage[margin=1in]{geometry}
\usepackage{pifont} % For checkmarks and crosses
\usepackage{booktabs} % For professional tables
\usepackage{hyperref} % For clickable links
\usepackage{url} % For URL formatting
\usepackage{seqsplit} % For splitting long strings
\usepackage{graphicx}
\usepackage{xcolor}
\usepackage{fancyhdr}

% --- Document Setup ---
\hypersetup{
    colorlinks=true,
    linkcolor=blue,
    filecolor=magenta,      
    urlcolor=cyan,
    pdftitle={Cybersecurity Posture Assessment Report},
    pdfpagemode=FullScreen,
}

% Define colors for severity
\definecolor{criticalred}{HTML}{D10000}
\definecolor{highorange}{HTML}{E25F00}
\definecolor{mediumyellow}{HTML}{F8D210}
\definecolor{lowblue}{HTML}{009DDC}
\definecolor{infogray}{HTML}{808080}

% Header and Footer
\pagestyle{fancy}
\fancyhf{}
\fancyhead[L]{Cybersecurity Posture Assessment}
\fancyhead[R]{Radiant Life}
\fancyfoot[C]{\thepage}
\renewcommand{\headrulewidth}{0.4pt}
\renewcommand{\footrulewidth}{0.4pt}

% --- Document Body ---
\begin{document}

% --- Title Page ---
\begin{titlepage}
    \centering
    \vspace*{1cm}
    \includegraphics[width=0.4\textwidth]{example-image-a} % Placeholder for client logo
    \vfill
    \huge\bfseries
    Cybersecurity Posture Assessment Report
    \vspace{1cm}
    \Large\bfseries
    Prepared for: Radiant Life
    \vspace{2cm}
    \normalsize
    \begin{tabular}{ll}
        \textbf{Date of Report:} & \today \\
        \textbf{Scan Date:} & 2023-10-27 \\ % Extracted from Input 1 Metadata
        \textbf{Report Version:} & 1.0 \\
    \end{tabular}
    \vfill
    \textit{This report contains sensitive information and should be handled with care.}
\end{titlepage}

\tableofcontents
\newpage

% --- Section 1: Executive Summary ---
\section{Executive Summary}
This report details the findings of a cybersecurity posture assessment conducted for Radiant Life. The assessment combined a technical network scan, a review of existing risks, and an analysis of organizational security controls via a questionnaire.

\textbf{Key Findings:}
The overall security posture presents a mixed landscape. On a positive note, the technical network scan of the target host (\texttt{192.168.1.100}) revealed a strong defensive configuration, with no open ports detected. This indicates a well-maintained firewall and a minimized attack surface for that specific asset.

However, significant and critical gaps were identified in the organization's procedural and access control policies. The lack of Multi-Factor Authentication (MFA) for email and computer access represents a critical vulnerability. These gaps substantially increase the risk of unauthorized access, credential compromise, and ransomware attacks. Furthermore, the absence of an employee Acceptable Use Policy and mandatory annual security awareness training for all staff creates a high-risk environment susceptible to human error and insider threats.

Immediate remediation should focus on implementing MFA across all critical systems, followed by the development and enforcement of foundational security policies and training programs.

% --- Section 2: Organizational Information ---
\section{Organizational Information}
The following information was provided for the assessment.
\begin{center}
\begin{tabular}{ll}
\toprule
\textbf{Item} & \textbf{Detail} \\
\midrule
Organization Name & Radiant Life \\
Email Domain & \texttt{RadiantLife.com} \\
Website Domain & \url{www.RadiantLife.com} \\
External IP Address & \texttt{14.10.26.25} \\
\bottomrule
\end{tabular}
\end{center}

% --- Section 3: Security Control Review ---
\section{Security Control Review}
The following table summarizes the organization's responses to the security controls questionnaire. Items marked with \textcolor{red}{\ding{55}} indicate significant gaps in security posture and are addressed in the Risk Assessment section.

\begin{center}
\begin{tabular}{p{0.6\linewidth} c c}
\toprule
\textbf{Control Question} & \textbf{Response} & \textbf{Status} \\
\midrule
Do you require MFA to access email? & \textcolor{red}{\ding{55}} & \textcolor{criticalred}{\textbf{Critical Gap}} \\
Do you require MFA to log into computers? & \textcolor{red}{\ding{55}} & \textcolor{criticalred}{\textbf{Critical Gap}} \\
Do you require MFA to access sensitive data systems? & \textcolor{green}{\ding{51}} & Met \\
Does your organization have an employee acceptable use policy? & \textcolor{red}{\ding{55}} & \textcolor{highorange}{\textbf{High Risk}} \\
Does your organization do security awareness training for new employees? & \textcolor{green}{\ding{51}} & Met \\
Does your organization do security awareness training for all employees at least once per year? & \textcolor{red}{\ding{55}} & \textcolor{highorange}{\textbf{High Risk}} \\
\bottomrule
\end{tabular}
\end{center}

% --- Section 4: Technical Scan Results ---
\section{Technical Scan Results}
An external network scan was performed to identify accessible services and potential vulnerabilities on the specified target.

\subsection{Scan Summary}
\begin{itemize}
    \item \textbf{Target IP:} \texttt{192.168.1.100}
    \item \textbf{Host Status:} Up
    \item \textbf{Key Finding:} The scan confirmed the host is online and responsive. However, \textbf{no open ports were detected}. All other scanned ports were found to be in a "closed" state.
\end{itemize}

\subsection{Analysis}
The absence of open ports is a strong positive security finding. It suggests that the host is protected by a well-configured firewall that correctly implements a "default deny" policy, only allowing traffic that is explicitly permitted. This significantly reduces the external attack surface of the asset. No vulnerabilities related to exposed services could be identified.

% --- Section 5: Risk Assessment ---
\section{Risk Assessment}
This section synthesizes findings from the security control review, technical scans, and pre-existing risk data. Based on the assessment, four new risks have been identified. No pre-existing vulnerabilities were reported.

\begin{center}
\begin{tabular}{p{0.15\linewidth} p{0.65\linewidth} p{0.1\linewidth}}
\toprule
\textbf{Risk ID} & \textbf{Risk Name \& Overview} & \textbf{Severity} \\
\midrule
\textbf{RISK-001} & \textbf{Lack of MFA on Email Systems} \newline \textit{The absence of MFA on email accounts allows an attacker with compromised credentials (e.g., from a phishing attack) to gain full access to an employee's mailbox, leading to data breaches, business email compromise, and further internal attacks.} & \textcolor{criticalred}{\textbf{Critical}} \\
\addlinespace
\textbf{RISK-002} & \textbf{Lack of MFA on Workstation Logins} \newline \textit{Without MFA for computer access, compromised credentials can be used to log into company workstations, granting an attacker network access and a foothold to move laterally, deploy ransomware, or exfiltrate data.} & \textcolor{criticalred}{\textbf{Critical}} \\
\addlinespace
\textbf{RISK-003} & \textbf{Missing Acceptable Use Policy (AUP)} \newline \textit{The lack of a formal AUP means employees are not given clear guidelines on the secure and acceptable use of company assets. This can lead to unintentional security incidents, data mishandling, and legal liabilities.} & \textcolor{highorange}{\textbf{High}} \\
\addlinespace
\textbf{RISK-004} & \textbf{Inadequate Annual Security Training} \newline \textit{Failing to provide annual security awareness training for all staff means that knowledge of current threats (like phishing and social engineering) degrades over time, making the organization more susceptible to human-centric attacks.} & \textcolor{highorange}{\textbf{High}} \\
\bottomrule
\end{tabular}
\end{center}

% --- Section 6: Recommendations ---
\section{Recommendations}
The following actionable recommendations are provided to mitigate the identified risks and improve the overall security posture of Radiant Life.

\subsection{Immediate Priorities (Critical Risks)}
\begin{description}
    \item[For RISK-001:] \textbf{Enforce MFA on All Email Accounts.}
    \begin{itemize}
        \item Immediately enable and enforce MFA for all user accounts within your email platform (e.g., Microsoft 365, Google Workspace).
        \item Prioritize deployment for privileged accounts (administrators, executives) and then roll out to all users within 30 days.
    \end{itemize}
    
    \item[For RISK-002:] \textbf{Deploy MFA for Endpoint/Computer Logins.}
    \begin{itemize}
        \item Implement an MFA solution for logging into all company-managed computers (desktops and laptops).
        \item Solutions like Windows Hello for Business, Duo, or other identity providers can be integrated with Active Directory or Azure AD.
    \end{itemize}
\end{description}

\subsection{Secondary Priorities (High Risks)}
\begin{description}
    \item[For RISK-003:] \textbf{Develop and Implement an Acceptable Use Policy.}
    \begin{itemize}
        \item Draft a formal AUP that clearly defines rules for using company networks, devices, software, and data.
        \item Require all current employees to read and acknowledge the policy. Incorporate this step into the onboarding process for new hires.
    \end{itemize}
    
    \item[For RISK-004:] \textbf{Establish a Mandatory Annual Security Training Program.}
    \begin{itemize}
        \item Procure a security awareness training platform or develop in-house materials covering key topics like phishing, password security, and data handling.
        \item Mandate that all employees complete this training annually and track completion to ensure compliance.
    \end{itemize}
\end{description}

\end{document}
```