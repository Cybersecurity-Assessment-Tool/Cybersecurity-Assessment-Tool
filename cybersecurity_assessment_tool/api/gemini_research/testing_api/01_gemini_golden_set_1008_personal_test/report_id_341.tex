```latex
\documentclass[12pt]{article}

% Preamble: Required Packages
\usepackage[margin=1in]{geometry}
\usepackage{pifont} % For check and cross marks
\usepackage{booktabs} % For professional tables
\usepackage{hyperref} % For clickable links
\usepackage{url} % For URL formatting
\usepackage{seqsplit} % To split long strings in texttt
\usepackage{xcolor} % For colors
\usepackage{graphicx} % For potential logos/images
\usepackage{fancyhdr} % For headers/footers

% --- Document Setup ---
\hypersetup{
    colorlinks=true,
    linkcolor=blue,
    filecolor=magenta,      
    urlcolor=cyan,
    pdftitle={Cybersecurity Assessment Report},
    pdfpagemode=FullScreen,
}

% Define colors for severity
\definecolor{criticalred}{HTML}{D10000}
\definecolor{highorange}{HTML}{E57300}
\definecolor{mediumyellow}{HTML}{FFBF00}
\definecolor{lowgreen}{HTML}{00A550}

% --- Header and Footer ---
\pagestyle{fancy}
\fancyhf{}
\fancyhead[L]{Cybersecurity Assessment Report}
\fancyhead[R]{Maple Leaf Logistics}
\fancyfoot[C]{\thepage}

% --- Document Start ---
\begin{document}

% --- Title Page ---
\begin{titlepage}
    \centering
    \vspace*{1cm}
    \Huge\textbf{Cybersecurity Assessment Report}
    \vspace{1.5cm}
    \Large
    \textbf{Prepared for:}\\
    \vspace{0.5cm}
    Maple Leaf Logistics
    \vfill
    \large
    \textbf{Date of Report:}\\
    \today
    \vspace{1.5cm}
    \textit{This report contains sensitive information and should be handled with care.}
\end{titlepage}

\tableofcontents
\newpage

% --- Section 1: Executive Summary ---
\section{Executive Summary}
This report details the findings of a cybersecurity assessment conducted for Maple Leaf Logistics. The assessment combined a review of organizational security controls, an automated network scan, and an analysis of pre-existing risk data.

The overall security posture is determined to be at a \textbf{high risk level}. Several critical and high-severity vulnerabilities were identified that require immediate attention.

Key findings include:
\begin{itemize}
    \item \textbf{Systemic Remote Desktop Protocol (RDP) Exposure:} The network scan identified an open RDP port on a new host (\seqsplit{\texttt{10.10.10.51}}), which correlates with a previously identified risk on another host. This indicates a pattern of insecure remote access configurations, posing a critical threat of unauthorized access and ransomware.
    \item \textbf{Lack of Multi-Factor Authentication (MFA) for Email:} The organization does not enforce MFA for email access. This is a critical security gap, as email is a primary target for phishing attacks and account takeovers, which can serve as an entry point for broader network compromise.
    \item \textbf{Insufficient Security Awareness Training:} The organization does not provide security awareness training for new or existing employees. This high-risk gap makes personnel more susceptible to social engineering attacks, such as phishing and business email compromise.
\end{itemize}

Immediate remediation of these issues is strongly recommended to reduce the organization's attack surface and mitigate the risk of a significant security incident. Detailed recommendations are provided in Section \ref{sec:recommendations}.

\newpage

% --- Section 2: Organizational Information ---
\section{Organizational Information}
The following information was provided for the assessment.

\begin{tabular}{@{}ll}
    \toprule
    \textbf{Attribute} & \textbf{Value} \\
    \midrule
    Organization Name & Maple Leaf Logistics \\
    Email Domain & \seqsplit{\texttt{MapleLeafLogistics.org}} \\
    Website Domain & \url{www.MapleLeafLogistics.org} \\
    External IP Address & \seqsplit{\texttt{14.62.173.94}} \\
    \bottomrule
\end{tabular}

% --- Section 3: Security Control Review ---
\section{Security Control Review (Questionnaire Analysis)}
A review of the organization's security controls was conducted via a questionnaire. The responses highlight significant gaps in the security framework, particularly concerning user authentication and security awareness.

\begin{table}[h!]
\centering
\caption{Security Control Questionnaire Responses}
\begin{tabular}{@{}lc@{}}
    \toprule
    \textbf{Control Question} & \textbf{Response} \\
    \midrule
    Do you require MFA to access email? & \textcolor{criticalred}{\ding{55}} \\
    Do you require MFA to log into computers? & \textcolor{lowgreen}{\ding{51}} \\
    Do you require MFA to access sensitive data systems? & \textcolor{lowgreen}{\ding{51}} \\
    Does your organization have an employee acceptable use policy? & \textcolor{lowgreen}{\ding{51}} \\
    Does your organization do security awareness training for new employees? & \textcolor{highorange}{\ding{55}} \\
    Does your organization do security awareness training for all employees annually? & \textcolor{highorange}{\ding{55}} \\
    \bottomrule
\end{tabular}
\end{table}

\subsection*{Analysis of Gaps}
\begin{itemize}
    \item \textbf{No MFA for Email:} The absence of MFA on email accounts is a critical vulnerability. Compromised email credentials could lead to data breaches, financial fraud, and further system compromise.
    \item \textbf{No Security Awareness Training:} The lack of a formal training program for new and existing employees creates a high-risk environment. Employees are the first line of defense, and without training, they are significantly more vulnerable to phishing, malware, and other social engineering tactics.
\end{itemize}

% --- Section 4: Technical Scan Results ---
\section{Technical Scan Results}
An Nmap scan was performed to identify open ports and services on the target system.

\begin{itemize}
    \item \textbf{Target IP Address:} \seqsplit{\texttt{10.10.10.51}}
    \item \textbf{Scan Status:} Host is up.
\end{itemize}

\begin{table}[h!]
\centering
\caption{Open Ports Detected on \seqsplit{\texttt{10.10.10.51}}}
\begin{tabular}{@{}llll@{}}
    \toprule
    \textbf{Port} & \textbf{State} & \textbf{Service Name} & \textbf{Analysis} \\
    \midrule
    3389/tcp & open & \texttt{ms-wbt-server} & Microsoft Remote Desktop Protocol (RDP) \\
    \bottomrule
\end{tabular}
\end{table}

\subsection*{Analysis of Findings}
The scan confirmed that port 3389, used for RDP, is open. RDP is a frequent target for attackers who use brute-force methods or exploit vulnerabilities to gain unauthorized remote access to systems. This finding, correlated with a pre-existing risk of the same nature, suggests a systemic issue in managing remote access controls.

\newpage

% --- Section 5: Consolidated Risk Assessment ---
\section{Consolidated Risk Assessment}
The following table synthesizes findings from the security questionnaire, the technical scan, and pre-existing risk data to provide a consolidated view of the current risk posture.

\begin{table}[h!]
\centering
\caption{Summary of Identified Risks}
\begin{tabular}{@{}p{0.1\linewidth} p{0.2\linewidth} p{0.35\linewidth} p{0.15\linewidth} p{0.1\linewidth}@{}}
    \toprule
    \textbf{Risk ID} & \textbf{Risk Name} & \textbf{Description} & \textbf{Affected Asset(s)} & \textbf{Severity} \\
    \midrule
    RISK-001 & Systemic RDP Exposure & RDP is exposed on multiple internal systems, creating a high risk of unauthorized access and ransomware. & \seqsplit{\texttt{10.10.10.50}}, \seqsplit{\texttt{10.10.10.51}} & \textcolor{criticalred}{\textbf{Critical}} \\
    \addlinespace
    RISK-002 & Lack of MFA for Email & Email accounts are protected only by passwords, making them highly susceptible to phishing and credential theft. & All Email Accounts & \textcolor{criticalred}{\textbf{Critical}} \\
    \addlinespace
    RISK-003 & Insufficient Security Training & Employees are not trained to recognize or respond to security threats, increasing susceptibility to social engineering. & All Employees & \textcolor{highorange}{\textbf{High}} \\
    \bottomrule
\end{tabular}
\end{table}

% --- Section 6: Recommendations ---
\section{Recommendations}
\label{sec:recommendations}
The following actions are recommended to mitigate the identified risks. Recommendations are prioritized based on severity.

\subsection{RISK-001: Remediate Systemic RDP Exposure (Critical)}
\begin{itemize}
    \item \textbf{Immediate Action:} If remote access is not required, close port 3389 on all affected systems immediately.
    \item \textbf{Short-Term Action:} If remote access is necessary, restrict access to port 3389 to specific, trusted IP addresses using firewall rules.
    \item \textbf{Long-Term Solution:} Implement a Virtual Private Network (VPN) with Multi-Factor Authentication (MFA) for all remote access to the internal network. Disable direct RDP access from the internet entirely.
\end{itemize}

\subsection{RISK-002: Implement MFA for Email (Critical)}
\begin{itemize}
    \item \textbf{Immediate Action:} Enable MFA for all email accounts, starting with administrative and executive accounts.
    \item \textbf{Short-Term Action:} Develop a plan and timeline to roll out MFA enforcement for all employees within the next 30 days. Provide clear instructions and support to users during the transition.
\end{itemize}

\subsection{RISK-003: Establish a Security Awareness Program (High)}
\begin{itemize}
    \item \textbf{Short-Term Action:} Source and implement a security awareness training program. Ensure all new hires complete this training as part of their onboarding process.
    \item \textbf{Long-Term Solution:} Conduct mandatory, annual security awareness training for all employees. Supplement this with periodic phishing simulations to test and reinforce learning.
\end{itemize}

\end{document}
```