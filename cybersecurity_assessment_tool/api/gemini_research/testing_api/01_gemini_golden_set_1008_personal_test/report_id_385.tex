```latex
\documentclass[12pt]{article}

% Preamble: Required Packages
\usepackage[margin=1in]{geometry}
\usepackage{pifont} % For checkmarks and crosses
\usepackage{booktabs} % For professional tables
\usepackage[hidelinks]{hyperref} % For clickable links
\usepackage{url} % For URL formatting
\usepackage{seqsplit} % To split long strings in tt font
\usepackage{graphicx}
\usepackage{fancyhdr}

% Document Information
\title{Cybersecurity Posture Assessment Report}
\author{Cybersecurity Analysis Division}
\date{November 22, 2025}

% Header/Footer
\pagestyle{fancy}
\fancyhf{}
\lhead{Skyward Bound Security Report}
\rfoot{Page \thepage}

\begin{document}

\maketitle
\thispagestyle{empty}
\newpage

\tableofcontents
\newpage

% --- 1. Executive Overview ---
\section{Executive Overview}
This report provides a comprehensive cybersecurity posture assessment for \textbf{Skyward Bound}, conducted on November 22, 2025. The analysis is based on a combination of network scanning, a review of organizational security controls, and an evaluation of pre-existing risks.

Overall, \textbf{Skyward Bound} demonstrates a strong foundation in identity and access management, with mandatory Multi-Factor Authentication (MFA) across email, computers, and sensitive data systems. However, two high-severity risks were identified that require immediate attention.

\begin{itemize}
    \item \textbf{High Risk - Inadequate Employee Onboarding:} A critical gap was identified in the security training process. New employees do not receive security awareness training upon being hired, exposing the organization to a heightened risk of social engineering and policy violations from day one.
    \item \textbf{High Risk - Vulnerable External Service:} The external network scan of host \texttt{192.168.10.5} revealed an outdated version of the Nginx web server (1.18.0). This version is several years old and has multiple known vulnerabilities, presenting a significant risk of compromise.
\end{itemize}

This report details these findings and provides actionable recommendations to mitigate the identified risks and strengthen the organization's overall security posture.

% --- 2. Organizational Information ---
\section{Organizational Information}
The following details were provided for the assessment.

\begin{tabular}{@{}ll}
    \toprule
    \textbf{Attribute} & \textbf{Value} \\
    \midrule
    Organization Name & \textbf{Skyward Bound} \\
    Email Domain & \texttt{SkywardBound.org} \\
    Website Domain & \url{www.SkywardBound.org} \\
    External IP Address & \texttt{152.0.76.186} \\
    \bottomrule
\end{tabular}

% --- 3. Security Control Review ---
\section{Security Control Review}
A review of administrative and procedural security controls was conducted via a questionnaire. The results indicate a strong policy on MFA but a critical weakness in employee security training at the point of onboarding.

\begin{table}[h!]
\centering
\begin{tabular}{@{}p{0.8\linewidth}c@{}}
    \toprule
    \textbf{Control Question} & \textbf{Response} \\
    \midrule
    Do you require MFA to access email? & \ding{51} \\
    Do you require MFA to log into computers? & \ding{51} \\
    Do you require MFA to access sensitive data systems? & \ding{51} \\
    Does your organization have an employee acceptable use policy? & \ding{51} \\
    \addlinespace
    \textbf{Does your organization do security awareness training for new employees?} & \textbf{\ding{55}} \\
    \addlinespace
    Does your organization do security awareness training for all employees at least once per year? & \ding{51} \\
    \bottomrule
\end{tabular}
\caption{Organizational Security Control Questionnaire Results. (\ding{51}=Yes, \ding{55}=No)}
\end{table}

% --- 4. Technical Scan Results ---
\section{Technical Scan Results}
An external network scan was performed to identify open ports and exposed services.

\begin{itemize}
    \item \textbf{Target IP Address:} \texttt{192.168.10.5}
    \item \textbf{Scan Date:} 2025-11-22T10:00:00Z
\end{itemize}

\subsection{Open Ports and Services}
The scan identified one open port on the target system.

\begin{table}[h!]
\centering
\begin{tabular}{@{}lllll@{}}
    \toprule
    \textbf{Port} & \textbf{State} & \textbf{Service} & \textbf{Product} & \textbf{Version} \\
    \midrule
    443/tcp & open & https & nginx & 1.18.0 \\
    \bottomrule
\end{tabular}
\caption{Open Ports Detected on \texttt{192.168.10.5}.}
\end{table}

\subsection{Analysis}
The scan revealed that the web server is running \textbf{Nginx version 1.18.0}. This version was released in April 2020 and is now significantly outdated. It is known to be vulnerable to several Common Vulnerabilities and Exposures (CVEs), including but not limited to CVE-2021-23017. Running outdated software on internet-facing systems is a critical security risk, as it provides a direct vector for attackers to exploit known flaws.

% --- 5. Consolidated Risk Assessment ---
\section{Consolidated Risk Assessment}
The following table synthesizes findings from the security control review and the technical scan. No pre-existing risks were reported.

\begin{table}[h!]
\centering
\begin{tabular}{@{}p{0.1\linewidth}p{0.45\linewidth}p{0.15\linewidth}p{0.2\linewidth}@{}}
    \toprule
    \textbf{ID} & \textbf{Risk Description} & \textbf{Severity} & \textbf{Source} \\
    \midrule
    R-01 & Lack of mandatory security awareness training for new employees during onboarding. & \textbf{High} & Questionnaire \\
    \addlinespace
    R-02 & The internet-facing web server at \texttt{192.168.10.5} is running an outdated and vulnerable version of Nginx (1.18.0). & \textbf{High} & Network Scan \\
    \bottomrule
\end{tabular}
\caption{Summary of Identified Risks.}
\end{table}

% --- 6. Recommendations ---
\section{Recommendations}
Based on the consolidated risk assessment, the following actions are recommended to mitigate the identified vulnerabilities and improve the overall security posture of \textbf{Skyward Bound}.

\subsection{R-01: Implement Onboarding Security Training}
\begin{itemize}
    \item \textbf{Action:} Develop and implement a mandatory security awareness training module that is a required part of the onboarding process for all new employees and contractors.
    \item \textbf{Justification:} New hires are often targeted by attackers. This training will ensure that they understand company security policies, can identify common threats like phishing, and know how to report security incidents from their first day. This closes a critical window of vulnerability.
\end{itemize}

\subsection{R-02: Upgrade Nginx Web Server}
\begin{itemize}
    \item \textbf{Action:} Immediately plan and execute an upgrade of the Nginx server on host \texttt{192.168.10.5} from version 1.18.0 to the latest stable version recommended by the vendor.
    \item \textbf{Justification:} Upgrading the software will patch known vulnerabilities, significantly reducing the risk of a system compromise. This should be part of a broader patch management policy where all internet-facing systems are regularly monitored and updated.
\end{itemize}

\end{document}
```