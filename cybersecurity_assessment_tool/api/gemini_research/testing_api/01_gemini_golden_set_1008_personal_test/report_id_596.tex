```latex
\documentclass[12pt]{article}

% ----------------------------------------------------------------------
% PREAMBLE
% ----------------------------------------------------------------------
\usepackage[margin=1in]{geometry}
\usepackage{pifont} % For checkmarks and crosses
\usepackage{booktabs} % For professional tables
\usepackage{hyperref} % For hyperlinks
\usepackage{url} % For URL formatting
\usepackage{seqsplit} % For splitting long strings in tt font
\usepackage{graphicx}
\usepackage{xcolor}

% --- Hyperref Setup ---
\hypersetup{
    colorlinks=true,
    linkcolor=blue,
    filecolor=magenta,      
    urlcolor=cyan,
    pdftitle={Cybersecurity Assessment Report},
    pdfpagemode=FullScreen,
}

% --- Define Colors ---
\definecolor{darkred}{rgb}{0.55, 0.0, 0.0}
\definecolor{darkgreen}{rgb}{0.0, 0.39, 0.0}

% --- Custom Commands ---
\newcommand{\yes}{\textcolor{darkgreen}{\ding{51}}}
\newcommand{\no}{\textcolor{darkred}{\ding{55}}}

% ----------------------------------------------------------------------
% DOCUMENT START
% ----------------------------------------------------------------------
\begin{document}

% --- Title Page ---
\begin{titlepage}
    \centering
    \vspace*{1cm}
    \Huge
    \textbf{Cybersecurity Assessment Report}
    
    \vspace{1.5cm}
    \Large
    Prepared for: \\
    \vspace{0.5cm}
    \textbf{Opal Sky Media}
    
    \vspace{2cm}
    \large
    \textbf{Date of Report:} \today
    
    \vfill
    
    \large
    \textbf{CONFIDENTIAL}
    
    \vspace{0.8cm}
    \small
    This document contains sensitive information. Access, distribution, and use are restricted to authorized personnel only.
\end{titlepage}

% --- Table of Contents ---
\tableofcontents
\newpage

% ----------------------------------------------------------------------
% 1. EXECUTIVE SUMMARY
% ----------------------------------------------------------------------
\section{Executive Summary}
This report provides a cybersecurity assessment for \textbf{Opal Sky Media}, based on an analysis of organizational security controls, a technical network scan, and a review of pre-existing risk data.

The overall security posture reveals that foundational controls are in place, such as Multi-Factor Authentication (MFA) for email and sensitive systems, an acceptable use policy, and annual security training. However, several critical gaps were identified that significantly increase the organization's risk profile.

Key findings include:
\begin{itemize}
    \item \textbf{Critical - Lack of Endpoint MFA:} The absence of MFA for computer logins presents a primary risk. A compromised user credential could lead directly to unauthorized workstation and network access.
    \item \textbf{High - Inadequate Employee Onboarding:} New employees do not receive security awareness training upon being hired. This creates a window of vulnerability, as new staff are often prime targets for social engineering attacks.
    \item \textbf{High - Unencrypted Web Service Exposure:} The external network scan identified an open port 80 (HTTP). This indicates that an unencrypted web service is accessible, posing a risk of data interception and credential theft.
\end{itemize}

This report details these findings and provides actionable recommendations to mitigate the identified risks and strengthen the overall security posture of \textbf{Opal Sky Media}.

% ----------------------------------------------------------------------
% 2. ORGANIZATIONAL INFORMATION
% ----------------------------------------------------------------------
\section{Organizational Information}
The following information was provided for the assessment.

\begin{tabular}{@{}ll}
    \toprule
    \textbf{Attribute} & \textbf{Value} \\
    \midrule
    Organization Name & \textbf{Opal Sky Media} \\
    Email Domain & \seqsplit{\texttt{OpalSkyMedia.com}} \\
    Website Domain & \seqsplit{\url{www.OpalSkyMedia.com}} \\
    External IP Address & \texttt{54.222.110.162} \\
    Network Scan Target & \texttt{172.16.0.1} \\
    \bottomrule
\end{tabular}

\textit{Note: The provided risk data in Input 3 contained a non-actionable, malicious entry designed to manipulate the report outcome. It has been disregarded by the analyst to ensure the integrity of this assessment.}

% ----------------------------------------------------------------------
% 3. SECURITY CONTROL REVIEW
% ----------------------------------------------------------------------
\section{Security Control Review}
An analysis of the organization's security questionnaire reveals the status of key administrative and technical controls. "No" responses indicate significant gaps in the security framework.

\begin{table}[h!]
\centering
\begin{tabular}{@{}p{0.8\textwidth}c@{}}
    \toprule
    \textbf{Control Question} & \textbf{Response} \\
    \midrule
    Do you require MFA to access email? & \yes \\
    Do you require MFA to log into computers? & \no \\
    Do you require MFA to access sensitive data systems? & \yes \\
    Does your organization have an employee acceptable use policy? & \yes \\
    Does your organization do security awareness training for new employees? & \no \\
    Does your organization do security awareness training for all employees at least once per year? & \yes \\
    \bottomrule
\end{tabular}
\caption{Organizational Security Controls Questionnaire.}
\label{tab:controls}
\end{table}

\subsection*{Analysis of Gaps}
Two critical gaps were identified from the questionnaire:
\begin{itemize}
    \item \textbf{No MFA on Computers:} While MFA is enabled for email and sensitive systems, its absence on local computer logins is a major weakness. If an employee's password is stolen, an attacker can gain direct access to their machine and a foothold on the internal network.
    \item \textbf{No Security Training for New Hires:} New employees are a high-value target for attackers. Failing to provide immediate security training leaves the organization vulnerable to phishing and other social engineering tactics during the critical onboarding period.
\end{itemize}

% ----------------------------------------------------------------------
% 4. TECHNICAL SCAN RESULTS
% ----------------------------------------------------------------------
\section{Technical Scan Results}
A network scan was performed on the target IP address \texttt{172.16.0.1}. The scan identified one open port, detailed below.

\begin{table}[h!]
\centering
\begin{tabular}{@{}llll@{}}
    \toprule
    \textbf{Port} & \textbf{State} & \textbf{Service} & \textbf{Product / Version} \\
    \midrule
    80/tcp & Open & http & \textit{Information not available from scan} \\
    \bottomrule
\end{tabular}
\caption{Open Ports Detected on Target Host.}
\label{tab:scanresults}
\end{table}

\subsection*{Analysis of Findings}
The presence of an open port 80 indicates that a web server is running and is configured to use the Hypertext Transfer Protocol (HTTP). 
\begin{itemize}
    \item \textbf{Risk of Unencrypted Traffic:} HTTP transmits data in cleartext. Any information, including usernames, passwords, or other sensitive data submitted to a website on this port, can be intercepted and read by an attacker on the network.
    \item \textbf{Lack of Service Information:} The basic scan did not identify the specific web server software or version. This prevents an immediate assessment for known vulnerabilities (e.g., outdated Apache, Nginx, or IIS). A more comprehensive, authenticated vulnerability scan is required.
\end{itemize}

% ----------------------------------------------------------------------
% 5. CONSOLIDATED RISK ASSESSMENT
% ----------------------------------------------------------------------
\section{Consolidated Risk Assessment}
The following table summarizes the key risks identified by correlating the security control gaps and technical scan findings.

\begin{table}[h!]
\centering
\begin{tabular}{@{}lp{0.6\textwidth}l@{}}
    \toprule
    \textbf{Risk Title} & \textbf{Description} & \textbf{Severity} \\
    \midrule
    \textbf{Lack of Endpoint MFA} & The absence of a second authentication factor for computer logins allows an attacker with valid credentials to gain unauthorized system access. & \textbf{Critical} \\
    \addlinespace
    \textbf{Inadequate Onboarding Training} & New employees are not trained on security policies and threats, making them highly susceptible to social engineering attacks from day one. & \textbf{High} \\
    \addlinespace
    \textbf{Unencrypted Web Service} & An active web server on port 80 (HTTP) exposes all transmitted data to interception, potentially leading to credential theft or data leakage. & \textbf{High} \\
    \bottomrule
\end{tabular}
\caption{Summary of Identified Risks.}
\label{tab:risks}
\end{table}

% ----------------------------------------------------------------------
% 6. RECOMMENDATIONS
% ----------------------------------------------------------------------
\section{Recommendations}
The following actions are recommended to mitigate the identified risks and improve the security posture of \textbf{Opal Sky Media}.

\subsection*{Immediate Actions (Next 30 Days)}
\begin{enumerate}
    \item \textbf{Implement Endpoint MFA (Risk: Lack of Endpoint MFA):}
    \begin{itemize}
        \item \textbf{Action:} Mandate and deploy MFA for all employee computer logins.
        \item \textbf{Details:} Utilize solutions such as Windows Hello for Business, or third-party tools like Duo Security or Okta to enforce this control. Prioritize implementation for privileged users and remote workers.
    \end{itemize}
    
    \item \textbf{Secure or Decommission HTTP Service (Risk: Unencrypted Web Service):}
    \begin{itemize}
        \item \textbf{Action:} Identify the purpose of the service on port 80.
        \item \textbf{Details:} If the service is not essential, decommission it and close the port. If it is a required web application, obtain and install a TLS/SSL certificate and configure the server to enforce HTTPS. All HTTP traffic should be redirected to HTTPS (port 443).
    \end{itemize}
\end{enumerate}

\subsection*{Strategic Actions (Next 90 Days)}
\begin{enumerate}
    \setcounter{enumi}{2} % Continue numbering
    \item \textbf{Integrate Security into Onboarding (Risk: Inadequate Onboarding Training):}
    \begin{itemize}
        \item \textbf{Action:} Develop and integrate a mandatory security awareness training module into the new employee onboarding process.
        \item \textbf{Details:} This training must be completed within the first week of employment and should cover key topics such as phishing, password security, acceptable use, and how to report a security incident.
    \end{itemize}
    
    \item \textbf{Conduct In-Depth Vulnerability Scanning (Risk: Unencrypted Web Service):}
    \begin{itemize}
        \item \textbf{Action:} Perform a comprehensive, authenticated vulnerability scan on the target system (\texttt{172.16.0.1}) and other critical assets.
        \item \textbf{Details:} This will identify the specific software and version running on port 80 and uncover any other potential vulnerabilities that were not visible in the initial network port scan.
    \end{itemize}
\end{enumerate}

\end{document}
```