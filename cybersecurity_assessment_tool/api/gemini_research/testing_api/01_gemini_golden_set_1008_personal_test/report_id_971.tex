```latex
\documentclass[12pt]{article}

% Preamble: Required Packages
\usepackage[margin=1in]{geometry}
\usepackage{pifont} % For checkmarks and crosses
\usepackage{booktabs} % For professional tables
\usepackage{hyperref} % For clickable links
\usepackage{url} % For URL formatting
\usepackage{seqsplit} % To split long text strings

% Document Metadata
\title{Cybersecurity Posture Assessment Report}
\author{Cybersecurity Analysis Division}
\date{\today}

\hypersetup{
    colorlinks=true,
    linkcolor=black,
    urlcolor=blue,
    pdftitle={Cybersecurity Posture Assessment Report},
    pdfauthor={Cybersecurity Analysis Division},
}

\begin{document}

\maketitle
\thispagestyle{empty}
\newpage
\tableofcontents
\newpage

% --- Executive Summary ---
\section*{Executive Summary}
This report provides a comprehensive cybersecurity assessment for \textbf{Sovereign Trust}, based on an analysis of organizational data, technical network scans, and pre-existing risk information. The assessment reveals a mixed security posture with several critical areas requiring immediate attention.

While foundational controls such as Multi-Factor Authentication (MFA) for email and computer access are in place, significant gaps exist in policy and access control for sensitive systems. Specifically, the lack of MFA for sensitive data systems, the absence of an employee Acceptable Use Policy (AUP), and no security awareness training for new hires represent critical vulnerabilities.

Technical analysis confirmed an open SSH port (22/TCP) on a local interface, which correlates with a pre-existing high-severity risk ("Localhost Exposed"). This configuration could be exploited if not properly secured.

Immediate remediation should focus on implementing MFA for all sensitive data systems, developing and enforcing an AUP, and integrating mandatory security training into the employee onboarding process.

% --- Organizational Information ---
\section*{Organizational Information}
The following details were provided for the assessment. This information is used to establish the context and scope of the review.

\begin{tabular}{@{}ll}
\toprule
\textbf{Attribute} & \textbf{Value} \\
\midrule
Organization Name & Sovereign Trust \\
Email Domain & \texttt{SovereignTrust.net} \\
Website Domain & \href{http://www.SovereignTrust.net}{\texttt{www.SovereignTrust.net}} \\
External IP Address & \texttt{178.73.50.196} \\
\bottomrule
\end{tabular}

% --- Security Control Review ---
\section*{Security Control Review}
A review of the organization's security controls was conducted via a questionnaire. The results below highlight key strengths and weaknesses in the current security framework. "No" answers indicate significant gaps that increase organizational risk.

\begin{tabular}{@{}p{0.7\textwidth}cc}
\toprule
\textbf{Control Question} & \textbf{Response} & \textbf{Status} \\
\midrule
Do you require MFA to access email? & Yes & \ding{51} \\
Do you require MFA to log into computers? & Yes & \ding{51} \\
\textbf{Do you require MFA to access sensitive data systems?} & \textbf{No} & \textbf{\ding{55}} \\
\textbf{Does your organization have an employee acceptable use policy?} & \textbf{No} & \textbf{\ding{55}} \\
\textbf{Does your organization do security awareness training for new employees?} & \textbf{No} & \textbf{\ding{55}} \\
Does your organization do security awareness training for all employees at least once per year? & Yes & \ding{51} \\
\bottomrule
\end{tabular}

\subsection*{Analysis of Control Gaps}
\begin{itemize}
    \item \textbf{No MFA for Sensitive Data Systems:} This is a critical weakness. In the event of a credential compromise, sensitive data is left highly vulnerable to unauthorized access and exfiltration.
    \item \textbf{No Acceptable Use Policy (AUP):} The absence of a formal AUP creates ambiguity for employees regarding the proper use of company assets, potentially leading to unintentional security incidents and insider threats.
    \item \textbf{No Security Training for New Hires:} New employees are a primary target for social engineering attacks. Without immediate training, they are more likely to fall victim to phishing or other scams, compromising the organization from their first day.
\end{itemize}

% --- Technical Scan Results ---
\section*{Technical Scan Results}
A network scan was performed to identify open ports and services on the target system. The findings provide insight into the external and internal attack surface.

\begin{itemize}
    \item \textbf{Target IP Address:} \texttt{127.0.0.1}
    \item \textbf{Scan Date:} Data from latest available scan.
\end{itemize}

\begin{tabular}{@{}lllll}
\toprule
\textbf{Port} & \textbf{State} & \textbf{Service} & \textbf{Product / Version} \\
\midrule
22/tcp & open & ssh & (Not specified in scan) \\
\bottomrule
\end{tabular}

\subsection*{Analysis of Technical Findings}
The scan identified an open SSH port (22/TCP) on the localhost interface (\texttt{127.0.0.1}). While often used for legitimate administrative purposes, an exposed SSH service can be a vector for attack if it is not securely configured (e.g., allows password-based or root logins). This finding directly correlates with the pre-existing "Localhost Exposed" risk identified in Input 3.

% --- Consolidated Risk Assessment ---
\section*{Consolidated Risk Assessment}
The following table synthesizes findings from the security control review, technical scan, and pre-existing risk data to provide a consolidated view of the primary risks facing the organization.

\begin{tabular}{@{}p{0.3\textwidth}p{0.15\textwidth}p{0.5\textwidth}}
\toprule
\textbf{Risk Name} & \textbf{Severity} & \textbf{Description} \\
\midrule
\textbf{No MFA on Sensitive Systems} & \textbf{Critical} & Lack of MFA on critical data systems exposes the organization to severe risk of data breach from compromised credentials. \\
\addlinespace
\textbf{Exposed Localhost Service (SSH)} & \textbf{Critical} & An open SSH port on localhost (127.0.0.1) was confirmed, aligning with a known CVSS 10.0 risk. Misconfiguration could lead to local privilege escalation or unauthorized access. \\
\addlinespace
\textbf{No New Hire Security Training} & \textbf{High} & New employees are not trained on security best practices, making them highly susceptible to phishing and social engineering attacks. \\
\addlinespace
\textbf{No Acceptable Use Policy} & \textbf{High} & The absence of a formal policy leads to inconsistent security practices and a higher likelihood of misuse of corporate assets. \\
\bottomrule
\end{tabular}

% --- Recommendations ---
\section*{Recommendations}
Based on the consolidated risk assessment, the following actions are recommended to mitigate the identified vulnerabilities and improve the overall security posture of \textbf{Sovereign Trust}.

\begin{enumerate}
    \item \textbf{Implement MFA for Sensitive Systems (Priority: Immediate):}
    \begin{itemize}
        \item \textbf{Action:} Deploy a robust Multi-Factor Authentication solution for all applications, databases, and systems identified as containing sensitive or critical data.
        \item \textbf{Impact:} Drastically reduces the risk of unauthorized access from stolen credentials.
    \end{itemize}
    
    \item \textbf{Secure or Disable Exposed SSH Service (Priority: Immediate):}
    \begin{itemize}
        \item \textbf{Action:} Review the purpose of the SSH service running on \texttt{127.0.0.1}. If it is not essential, disable the service. If required, harden its configuration by disabling root login, enforcing key-based authentication, and limiting access.
        \item \textbf{Impact:} Eliminates a potential vector for local privilege escalation or unauthorized system access.
    \end{itemize}

    \item \textbf{Develop and Implement an Acceptable Use Policy (Priority: High):}
    \begin{itemize}
        \item \textbf{Action:} Draft a comprehensive AUP that clearly defines rules for the use of company networks, devices, and data. Require all current and new employees to read and formally acknowledge the policy.
        \item \textbf{Impact:} Establishes a clear security baseline for employee behavior and reduces insider risk.
    \end{itemize}

    \item \textbf{Institute Onboarding Security Training (Priority: High):}
    \begin{itemize}
        \item \textbf{Action:} Create a mandatory security awareness training module as part of the new employee onboarding process. This module should cover phishing, password security, data handling, and the new AUP.
        \item \textbf{Impact:} Reduces the vulnerability of new employees to common cyberattacks and fosters a security-conscious culture from day one.
    \end{itemize}
\end{enumerate}

\end{document}
```