```latex
\documentclass[12pt]{article}

% 1. REQUIRED PACKAGES
\usepackage[margin=1in]{geometry}
\usepackage{pifont} % For checkmarks and crosses (\ding{51}, \ding{55})
\usepackage{booktabs} % For professional tables
\usepackage[hidelinks]{hyperref} % For clickable links
\usepackage{url} % For formatting URLs
\usepackage{seqsplit} % To split long strings in tt font
\usepackage{graphicx}
\usepackage{fancyhdr}
\usepackage{xcolor}

% 2. DOCUMENT METADATA & SETUP
\hypersetup{
    colorlinks=true,
    linkcolor=blue,
    filecolor=magenta,      
    urlcolor=cyan,
    pdftitle={Cybersecurity Posture Report},
    pdfauthor={Cybersecurity Analysis Division},
    pdfsubject={Security Assessment},
    pdfkeywords={Cybersecurity, Risk, Analysis},
}

\pagestyle{fancy}
\fancyhf{}
\fancyhead[L]{\textbf{Cybersecurity Posture Report}}
\fancyhead[R]{\textbf{Apex Legends Group}}
\fancyfoot[C]{\thepage}
\renewcommand{\headrulewidth}{0.4pt}
\renewcommand{\footrulewidth}{0.4pt}

\newcommand{\yes}{\ding{51}}
\newcommand{\no}{\ding{55}}

\begin{document}

% 3. TITLE PAGE
\begin{titlepage}
    \centering
    \vspace*{1cm}
    \Huge{\textbf{Cybersecurity Posture Report}}
    \vspace{1.5cm}
    \Large{\textbf{Prepared for:}}
    \vspace{0.5cm}
    \Large{Apex Legends Group}
    \vspace{2cm}
    \large{\textbf{Date of Report:}}
    \vspace{0.5cm}
    \large{\today}
    \vfill
    \large{\textit{This report contains sensitive information and is intended solely for the recipient. Distribution is strictly prohibited.}}
\end{titlepage}

\tableofcontents
\newpage

% 4. EXECUTIVE SUMMARY
\section{Executive Summary}
This report provides a comprehensive analysis of the cybersecurity posture for \textbf{Apex Legends Group}. The assessment is based on a correlation of organizational data, a technical network scan, and a review of pre-existing risks.

The overall security posture is considered \textbf{weak} due to several critical and high-risk findings. While foundational controls like Multi-Factor Authentication (MFA) are in place for email and computer access, significant gaps exist in policy, employee training, and protection of sensitive data systems.

Key findings include:
\begin{itemize}
    \item \textbf{Critical Pre-existing Risk:} A critical vulnerability, "Localhost Exposed," was identified, corresponding to an open SSH port (22) on an internal loopback address (\texttt{127.0.0.1}). This indicates a severe misconfiguration that could allow unauthorized access.
    \item \textbf{Lack of MFA on Sensitive Systems:} The absence of mandatory MFA for accessing sensitive data systems presents a direct and significant risk of data breach.
    \item \textbf{Deficient Security Policies and Training:} The organization lacks a formal Acceptable Use Policy and does not conduct security awareness training. This elevates the risk of human error, such as falling victim to phishing attacks, which could lead to credential compromise.
\end{itemize}

Immediate remediation is required to address the exposed service and enforce MFA on critical systems. Subsequently, establishing a robust security awareness program and formalizing security policies are essential steps to improve the organization's defensive capabilities.

% 5. ORGANIZATIONAL INFORMATION
\section{Organizational Information}
The following details were provided by the organization and used as a baseline for this assessment.

\begin{tabular}{@{}ll}
\toprule
\textbf{Attribute} & \textbf{Value} \\
\midrule
Organization Name & Apex Legends Group \\
Email Domain & \seqsplit{\texttt{ApexLegendsGroup.net}} \\
Website Domain & \seqsplit{\url{www.ApexLegendsGroup.net}} \\
External IP Address & \seqsplit{\texttt{31.54.191.1}} \\
\bottomrule
\end{tabular}

% 6. SECURITY CONTROL REVIEW (QUESTIONNAIRE)
\section{Security Control Review}
A review of the organization's security controls was conducted via a standardized questionnaire. The results highlight critical gaps in administrative and technical controls. A "No" answer indicates a deviation from security best practices and represents a significant risk.

\begin{table}[h!]
\centering
\caption{Security Questionnaire Results}
\begin{tabular}{@{}p{0.7\textwidth}c@{}}
\toprule
\textbf{Control Question} & \textbf{Status} \\
\midrule
Do you require MFA to access email? & \yes \\
Do you require MFA to log into computers? & \yes \\
\textbf{Do you require MFA to access sensitive data systems?} & \textcolor{red}{\no} \\
\textbf{Does your organization have an employee acceptable use policy?} & \textcolor{red}{\no} \\
\textbf{Does your organization do security awareness training for new employees?} & \textcolor{red}{\no} \\
\textbf{Does your organization do security awareness training for all employees at least once per year?} & \textcolor{red}{\no} \\
\bottomrule
\end{tabular}
\end{table}

\subsection*{Analysis of Control Gaps}
The questionnaire reveals four major areas of concern:
\begin{itemize}
    \item \textbf{MFA on Sensitive Systems:} Failure to enforce MFA on systems containing sensitive data is a critical oversight. Stolen credentials alone could be sufficient for an attacker to access and exfiltrate the organization's most valuable information.
    \item \textbf{Acceptable Use Policy (AUP):} Without an AUP, there are no clear, enforceable rules for employees regarding the use of company assets. This can lead to unintentional data exposure or misuse of systems.
    \item \textbf{Security Awareness Training:} The complete absence of a training program leaves employees vulnerable to common cyberattacks like phishing, social engineering, and malware. Employees are the first line of defense, and without training, they are an unfortified entry point for attackers.
\end{itemize}

% 7. TECHNICAL SCAN RESULTS
\section{Technical Scan Results}
A network scan was performed to identify open ports and exposed services on the target system.

\begin{itemize}
    \item \textbf{Target IP:} \seqsplit{\texttt{127.0.0.1}}
    \item \textbf{Scan Date:} \today
\end{itemize}

\begin{table}[h!]
\centering
\caption{Open Port Findings}
\begin{tabular}{@{}llll@{}}
\toprule
\textbf{Port} & \textbf{State} & \textbf{Service (Inferred)} & \textbf{Product / Version} \\
\midrule
22/tcp & open & SSH (Secure Shell) & Not detected \\
\bottomrule
\end{tabular}
\end{table}

\subsection*{Analysis of Technical Findings}
The scan identified that port 22 (SSH) is open on the localhost interface (\texttt{127.0.0.1}). This finding directly correlates with the pre-existing risk "Localhost Exposed" from Input 3. Exposing a service on the loopback address is a highly unusual and dangerous configuration, suggesting a potential tunnel or misconfigured service that could be exploited for unauthorized remote access. The lack of version information prevents a specific vulnerability check, but any exposed SSH service must be assumed to be a target.

% 8. CONSOLIDATED RISK ASSESSMENT
\section{Consolidated Risk Assessment}
The following table synthesizes findings from the security control review, technical scan, and pre-existing risk data into a consolidated list of identified risks.

\begin{table}[h!]
\centering
\caption{Summary of Identified Risks}
\begin{tabular}{@{}p{0.45\textwidth}p{0.2\textwidth}p{0.25\textwidth}@{}}
\toprule
\textbf{Risk Description} & \textbf{Severity} & \textbf{Affected Asset(s)} \\
\midrule
\textbf{Localhost Exposed via Open SSH Port} & \textbf{Critical} & Network Infrastructure, Server hosting the service \\
\addlinespace
Lack of MFA on Sensitive Data Systems & Critical & Sensitive Data, Core Business Systems \\
\addlinespace
Absence of Security Awareness Training Program & High & All Employees, Endpoints \\
\addlinespace
No Formal Acceptable Use Policy & High & All Employees, Corporate Systems \\
\bottomrule
\end{tabular}
\end{table}

% 9. RECOMMENDATIONS
\section{Recommendations}
Based on the consolidated risk assessment, the following actions are recommended to mitigate the identified risks and improve the overall security posture of \textbf{Apex Legends Group}.

\subsection{Immediate Priority (Critical Risks)}
\subsubsection{Remediate Exposed SSH Service}
\begin{itemize}
    \item \textbf{Action:} Immediately investigate the service running on port 22 on \texttt{127.0.0.1}. This port should be closed to all external access. Implement strict firewall rules to ensure that internal-only services are not accessible from the internet.
    \item \textbf{Justification:} This is a critical misconfiguration that provides a potential vector for direct, unauthorized access to the network.
\end{itemize}

\subsubsection{Implement MFA on All Sensitive Systems}
\begin{itemize}
    \item \textbf{Action:} Procure and deploy an MFA solution for all applications and systems that store, process, or transmit sensitive data. This policy should be enforced for all users, including administrators and executives.
    \item \textbf{Justification:} MFA is one of the most effective controls to prevent account takeover attacks resulting from stolen credentials.
\end{itemize}

\subsection{High Priority (Strengthening Defenses)}
\subsubsection{Establish a Security Awareness Training Program}
\begin{itemize}
    \item \textbf{Action:} Implement a mandatory security awareness training program. This should include initial training for all new hires and annual refresher courses for all staff. Topics must cover phishing, password security, and safe internet use.
    \item \textbf{Justification:} A well-trained workforce is a critical layer of defense against common cyber threats, reducing the likelihood of human error leading to a breach.
\end{itemize}

\subsubsection{Develop and Enforce an Acceptable Use Policy (AUP)}
\begin{itemize}
    \item \textbf{Action:} Draft a formal AUP that clearly defines the rules for using company technology and data. This policy should be distributed to all employees, who must formally acknowledge their understanding and agreement.
    \item \textbf{Justification:} An AUP establishes a baseline for secure behavior, reduces legal liability, and provides a framework for enforcing security standards.
\end{itemize}

\end{document}
```