```latex
\documentclass[12pt]{article}

% --- PACKAGES ---
\usepackage[margin=1in]{geometry}
\usepackage{pifont} % For checkmarks and crosses
\usepackage{booktabs} % For professional tables
\usepackage{hyperref} % For clickable links
\usepackage{url} % For URL formatting
\usepackage{seqsplit} % To split long strings in tt font
\usepackage{graphicx}
\usepackage{xcolor}

% --- DOCUMENT METADATA ---
\title{Cybersecurity Assessment Report \\ \large Green Sprout Organic}
\author{Cybersecurity Analyst Group}
\date{\today}

% --- HYPERREF SETUP ---
\hypersetup{
    colorlinks=true,
    linkcolor=black,
    urlcolor=blue,
    pdftitle={Cybersecurity Assessment Report},
    pdfauthor={Cybersecurity Analyst Group},
}

% --- DOCUMENT START ---
\begin{document}

\maketitle
\hrule
\vspace{1em}

% ===================================================================
% SECTION: EXECUTIVE SUMMARY
% ===================================================================
\section*{1.0 Executive Summary}

This report details the findings of a cybersecurity assessment for Green Sprout Organic. The analysis is based on a network scan, a review of organizational security controls, and pre-existing risk data.

The assessment identified several critical and high-risk vulnerabilities that require immediate attention. The most significant findings include:
\begin{itemize}
    \item \textbf{Systemic RDP Exposure:} The technical scan identified a system with Remote Desktop Protocol (RDP) open to the internal network. This finding, correlated with pre-existing risk data, indicates a pattern of insecure remote access configurations, which are a primary vector for ransomware attacks.
    \item \textbf{Critical Gaps in Multi-Factor Authentication (MFA):} MFA is not enforced for accessing email or sensitive data systems. This significantly increases the risk of account takeover via phishing or credential stuffing attacks.
    \item \textbf{Lack of Security Awareness Training:} The organization does not provide security awareness training to new or existing employees. This absence of a "human firewall" makes the organization highly susceptible to social engineering and phishing attacks.
\end{itemize}

These vulnerabilities, particularly when combined, create a high likelihood of a significant security incident. Immediate remediation of the identified risks is strongly recommended to improve the organization's security posture.

% ===================================================================
% SECTION: ORGANIZATIONAL INFORMATION
% ===================================================================
\section*{2.0 Organizational Information}

The following information was provided for the assessment.

\begin{tabular}{@{}ll}
    \toprule
    \textbf{Attribute} & \textbf{Value} \\
    \midrule
    Organization Name & Green Sprout Organic \\
    Email Domain & \texttt{GreenSproutOrganic.net} \\
    Website Domain & \url{www.GreenSproutOrganic.net} \\
    External IP Address & \texttt{28.11.16.254} \\
    \bottomrule
\end{tabular}

% ===================================================================
% SECTION: SECURITY CONTROL REVIEW
% ===================================================================
\section*{3.0 Security Control Review}

A review of administrative security controls was conducted based on a questionnaire. The results below highlight significant gaps in the organization's security policies and procedures. Answers marked with \ding{55} indicate a deviation from security best practices and represent a potential risk.

\begin{table}[h!]
\centering
\begin{tabular}{@{}lc@{}}
    \toprule
    \textbf{Control Question} & \textbf{Response} \\
    \midrule
    Do you require MFA to access email? & \ding{55} \\
    Do you require MFA to log into computers? & \ding{51} \\
    Do you require MFA to access sensitive data systems? & \ding{55} \\
    Does your organization have an employee acceptable use policy? & \ding{51} \\
    Does your organization do security awareness training for new employees? & \ding{55} \\
    Does your organization do security awareness training for all employees annually? & \ding{55} \\
    \bottomrule
\end{tabular}
\caption{Security Controls Questionnaire Results}
\end{table}

\subsection*{Analysis of Control Gaps}
\begin{itemize}
    \item \textbf{MFA Deficiencies:} The lack of MFA on email and sensitive data systems is a critical vulnerability. Email is a primary target for attackers seeking to gain an initial foothold for further attacks, such as business email compromise (BEC) or internal phishing.
    \item \textbf{Training Deficiencies:} Without security awareness training, employees are significantly more likely to click on malicious links, open infected attachments, or divulge sensitive information. This undermines all other technical security controls in place.
\end{itemize}

% ===================================================================
% SECTION: TECHNICAL SCAN RESULTS
% ===================================================================
\section*{4.0 Technical Scan Results}

An Nmap scan was performed to identify open ports and services on the target system.

\begin{itemize}
    \item \textbf{Target IP:} \texttt{10.10.10.51}
    \item \textbf{Scan Date:} \today
\end{itemize}

The following open ports were discovered:

\begin{table}[h!]
\centering
\begin{tabular}{@{}llll@{}}
    \toprule
    \textbf{Port} & \textbf{State} & \textbf{Service Name} & \textbf{Analysis} \\
    \midrule
    3389/tcp & open & ms-wbt-server & Microsoft Remote Desktop Protocol (RDP) \\
    \bottomrule
\end{tabular}
\caption{Open Ports on \texttt{10.10.10.51}}
\end{table}

\subsection*{Analysis of Technical Findings}
The scan confirms that RDP is exposed on host \texttt{10.10.10.51}. When correlated with the pre-existing risk data, which identified RDP exposure on \texttt{10.10.10.50}, this reveals a systemic issue. RDP is a frequent target for brute-force attacks and exploitation, often leading to ransomware deployment.

% ===================================================================
% SECTION: RISK ASSESSMENT
% ===================================================================
\section*{5.0 Risk Assessment}

The following table summarizes the key risks identified by correlating the security control review, technical scan results, and pre-existing vulnerability data.

\begin{table}[h!]
\centering
\begin{tabular}{@{}p{0.15\linewidth} p{0.55\linewidth} p{0.2\linewidth}@{}}
    \toprule
    \textbf{Risk Name} & \textbf{Description} & \textbf{Severity} \\
    \midrule
    \textbf{Systemic RDP Exposure} & Multiple systems (\texttt{10.10.10.50}, \texttt{10.10.10.51}) have RDP exposed. This service is a primary target for ransomware and unauthorized access. & \textbf{Critical} \\
    \addlinespace
    \textbf{Lack of MFA on Critical Systems} & Email and sensitive data systems are protected only by passwords, making them highly vulnerable to account takeover from phishing or credential stuffing attacks. & \textbf{Critical} \\
    \addlinespace
    \textbf{Inadequate Security Awareness Program} & A complete lack of employee security training significantly increases the organization's susceptibility to social engineering, phishing, and other human-targeted attacks. & \textbf{High} \\
    \bottomrule
\end{tabular}
\caption{Summary of Identified Risks}
\end{table}

% ===================================================================
% SECTION: RECOMMENDATIONS
% ===================================================================
\section*{6.0 Recommendations}

The following actions are recommended to mitigate the identified risks. Recommendations are prioritized based on severity and potential impact.

\subsection*{Immediate Actions (Critical Priority)}
\begin{enumerate}
    \item \textbf{Restrict RDP Access:}
    \begin{itemize}
        \item Immediately implement firewall rules to block all access to TCP port 3389 on \texttt{10.10.10.50} and \texttt{10.10.10.51} from untrusted networks.
        \item If remote access is required, it must be facilitated through a secure Virtual Private Network (VPN) that requires Multi-Factor Authentication.
    \end{itemize}
    \item \textbf{Enforce MFA on All External-Facing Services:}
    \begin{itemize}
        \item Immediately enable MFA for all user accounts on the email platform (\texttt{GreenSproutOrganic.net}).
        \item Immediately enforce MFA for access to all systems categorized as containing sensitive data.
    \end{itemize}
\end{enumerate}

\subsection*{Short-Term Actions (High Priority)}
\begin{enumerate}
    \item \textbf{Implement Security Awareness Training:}
    \begin{itemize}
        \item Procure and implement a security awareness training program.
        \item Enroll all current employees in a foundational training module covering phishing, password hygiene, and acceptable use.
        \item Integrate this training into the onboarding process for all new hires.
    \end{itemize}
    \item \textbf{Conduct Annual Refresher Training:}
    \begin{itemize}
        \item Establish a formal policy requiring all employees to complete security awareness training at least once per year.
    \end{itemize}
\end{enumerate}

\subsection*{Long-Term Strategic Actions}
\begin{enumerate}
    \item \textbf{Comprehensive Vulnerability Scanning:} Conduct a credentialed vulnerability scan across the entire internal network to identify other misconfigurations, outdated software, and exposed services.
    \item \textbf{Develop a Formal Remote Access Policy:} Create and enforce a policy that defines the requirements for all remote access to the internal network, mandating the use of a company-approved VPN with MFA.
\end{enumerate}

\end{document}
```