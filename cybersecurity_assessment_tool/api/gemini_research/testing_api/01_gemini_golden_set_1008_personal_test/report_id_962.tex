```latex
\documentclass[12pt, a4paper]{article}

% Preamble: Required Packages
\usepackage[margin=1in]{geometry}
\usepackage{pifont} % For checkmarks and crosses (\ding)
\usepackage{booktabs} % For professional tables
\usepackage{hyperref} % For clickable links
\usepackage{url} % For URL formatting
\usepackage{seqsplit} % To split long strings without breaking
\usepackage[table]{xcolor} % For coloring table cells

% Document Metadata and Hyperref Setup
\hypersetup{
    colorlinks=true,
    linkcolor=blue,
    filecolor=magenta,      
    urlcolor=cyan,
    pdftitle={Cybersecurity Posture Report},
    pdfauthor={Cybersecurity Analyst},
    pdfsubject={Security Assessment},
    pdfkeywords={Cybersecurity, Risk, Assessment},
    bookmarks=true
}

% Custom Commands and Color Definitions
\newcommand{\yes}{\ding{51}} % Green checkmark
\newcommand{\no}{\ding{55}}  % Red X
\definecolor{sev_critical}{HTML}{D10000}
\definecolor{sev_high}{HTML}{E97409}
\definecolor{sev_medium}{HTML}{F5C30D}
\definecolor{sev_low}{HTML}{83B521}
\definecolor{tablehead}{gray}{0.9}

\begin{document}

% --- Title Page ---
\begin{titlepage}
    \centering
    \vspace*{1cm}
    \Huge{\textbf{Cybersecurity Posture Report}}
    \vspace{1.5cm}
    \Large{\textbf{Prepared for:}} \\
    \vspace{0.5cm}
    \Large{Vertex Solutions}
    \vspace{2cm}
    \large{\textbf{Date of Report:}} \\
    \vspace{0.5cm}
    \large{\today}
    \vfill
    \large{Generated by: \\ \textbf{Cybersecurity Analyst}}
\end{titlepage}

\tableofcontents
\newpage

% --- Executive Summary ---
\section*{Executive Summary}
This report provides a comprehensive analysis of the cybersecurity posture for \textbf{Vertex Solutions}. The assessment is based on a correlation of network scan data, organizational security control questionnaires, and a review of pre-existing risks.

The overall security posture is determined to be \textbf{critically weak}. Several high-impact vulnerabilities and security gaps were identified that expose the organization to significant risk of data breach, unauthorized access, and system compromise. 

Key findings include a publicly accessible FTP server running a critically outdated and vulnerable version of \texttt{vsftpd} which allows for anonymous access, a complete lack of Multi-Factor Authentication (MFA) across all critical services, and the absence of foundational security policies. These issues are compounded by the continued use of an outdated operating system (Windows 7) on workstations.

Immediate and decisive action is required to remediate these findings. This report outlines specific, prioritized recommendations to mitigate the identified risks and strengthen the organization's defensive capabilities.

% --- Organizational Information ---
\section*{1. Organizational Information}
The following details were provided for the assessment.
\begin{itemize}
    \item \textbf{Organization Name:} Vertex Solutions
    \item \textbf{Email Domain:} \texttt{VertexSolutions.org}
    \item \textbf{External IP Address:} \texttt{27.204.211.97}
\end{itemize}

% --- Security Control Review ---
\section*{2. Security Control Review}
An internal review of security controls was conducted via a questionnaire. The responses indicate significant gaps in fundamental security practices. A "No" response highlights a missing control that increases organizational risk.

\begin{table}[h!]
\centering
\caption{Security Controls Questionnaire Results}
\begin{tabular}{p{0.8\linewidth} c}
\toprule
\rowcolor{tablehead}
\textbf{Control Question} & \textbf{Response} \\
\midrule
Do you require MFA to access email? & \no \\
Do you require MFA to log into computers? & \no \\
Do you require MFA to access sensitive data systems? & \no \\
Does your organization have an employee acceptable use policy? & \no \\
Does your organization do security awareness training for new employees? & \no \\
Does your organization do security awareness training for all employees at least once per year? & \yes \\
\bottomrule
\end{tabular}
\end{table}

\subsection*{Analysis of Control Gaps}
The lack of Multi-Factor Authentication (MFA) for email, computer logins, and sensitive data access is a critical deficiency. This significantly lowers the barrier for an attacker to gain access to key systems using compromised credentials alone. Furthermore, the absence of an acceptable use policy and security training for new hires creates an environment where employees may be unaware of security best practices, making them more susceptible to social engineering attacks.

% --- Technical Scan Results ---
\section*{3. Technical Scan Results}
An external network scan was performed to identify open ports and exposed services.

\begin{itemize}
    \item \textbf{Target IP Address:} \texttt{10.0.0.15}
    \item \textbf{Scan Utility:} Nmap
\end{itemize}

\begin{table}[h!]
\centering
\caption{Open Port Analysis}
\begin{tabular}{l l l l p{0.3\linewidth}}
\toprule
\rowcolor{tablehead}
\textbf{Port} & \textbf{State} & \textbf{Service} & \textbf{Version} & \textbf{Details} \\
\midrule
21/tcp & open & ftp & vsftpd 2.3.4 & \textbf{Critical Finding:} Anonymous FTP login is allowed. This version is known to be vulnerable to a backdoor (CVE-2011-2523). \\
\bottomrule
\end{tabular}
\end{table}

\subsection*{Analysis of Technical Findings}
The scan identified a single, yet extremely critical, finding. An FTP server is exposed running \texttt{vsftpd version 2.3.4}. This specific version contains a well-documented remote command execution backdoor (CVE-2011-2523), allowing an attacker to gain complete control of the server. This is exacerbated by the configuration allowing anonymous, unauthenticated access, which makes exploitation trivial. The use of FTP is also inherently insecure as it transmits data, including credentials, in cleartext.

% --- Risk Assessment Summary ---
\section*{4. Risk Assessment Summary}
The following table synthesizes findings from the security control review, technical scan, and pre-existing risk data into a prioritized list of risks.

\begin{table}[h!]
\centering
\caption{Consolidated Risk Register}
\begin{tabular}{p{0.6\linewidth} l}
\toprule
\rowcolor{tablehead}
\textbf{Risk Title \& Description} & \textbf{Severity} \\
\midrule
\textbf{RCE via vsftpd 2.3.4 Backdoor (CVE-2011-2523)} \newline An attacker can gain full remote control of the server by exploiting a known backdoor in the exposed FTP service. & \cellcolor{sev_critical!80}\textbf{Critical} \\
\midrule
\textbf{Lack of Multi-Factor Authentication (MFA)} \newline The absence of MFA on email, endpoints, and sensitive systems makes account takeover trivial with compromised credentials. & \cellcolor{sev_critical!80}\textbf{Critical} \\
\midrule
\textbf{Anonymous FTP Access} \newline Unauthenticated users can access the FTP server, potentially leading to sensitive data exfiltration or the upload of malicious files. & \cellcolor{sev_high!80}\textbf{High} \\
\midrule
\textbf{Missing Foundational Security Policies} \newline The lack of an Acceptable Use Policy and new hire security training results in an uninformed user base, increasing susceptibility to phishing and insider threats. & \cellcolor{sev_high!80}\textbf{High} \\
\midrule
\textbf{Outdated Windows 7 Operating System} \newline Workstations are running an unsupported OS (Windows 7) that no longer receives security updates, leaving them vulnerable to known exploits. & \cellcolor{sev_medium!80}\textbf{Medium} \\
\bottomrule
\end{tabular}
\end{table}

% --- Recommendations ---
\section*{5. Recommendations}
The following actions are recommended to mitigate the identified risks. They are prioritized based on severity and ease of implementation.

\subsection*{Immediate Actions (Next 24 Hours)}
\begin{enumerate}
    \item \textbf{Remediate FTP Server Vulnerability:} Immediately take the FTP server offline. If the service is business-critical, it must be upgraded to a secure, patched version of FTP software (e.g., a modern SFTP server).
    \item \textbf{Disable Anonymous FTP:} If the FTP server must remain online temporarily before being replaced, anonymous access must be disabled immediately.
\end{enumerate}

\subsection*{High Priority Actions (Next 30 Days)}
\begin{enumerate}
    \item \textbf{Implement Multi-Factor Authentication (MFA):} Procure and deploy an MFA solution. Prioritize enforcement on the following systems in order:
    \begin{itemize}
        \item Email (e.g., Office 365, Google Workspace)
        \item VPN and remote access solutions
        \item Access to all systems containing sensitive data
    \end{itemize}
    \item \textbf{Develop and Implement Foundational Policies:} Draft and formally approve an \textit{Acceptable Use Policy} for all employees. Create and mandate a security awareness training module for all new hires as part of their onboarding process.
\end{enumerate}

\subsection*{Scheduled Projects (Next 6-12 Months)}
\begin{enumerate}
    \item \textbf{Operating System Upgrade:} Initiate a project to upgrade all workstations from Windows 7 to a modern, supported operating system such as Windows 11. This will likely require a hardware refresh plan for incompatible devices.
\end{enumerate}

\end{document}
```