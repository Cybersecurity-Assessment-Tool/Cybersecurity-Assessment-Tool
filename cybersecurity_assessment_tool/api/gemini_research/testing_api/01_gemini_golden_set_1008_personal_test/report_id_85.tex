```latex
\documentclass[12pt]{article}

% Preamble: Required Packages
\usepackage[margin=1in]{geometry}
\usepackage{pifont}
\usepackage{booktabs}
\usepackage{graphicx}
\usepackage[hidelinks]{hyperref}
\usepackage{url}
\usepackage{seqsplit}
\usepackage{xcolor}
\usepackage{array}

% Define custom colors
\definecolor{darkblue}{rgb}{0.0, 0.0, 0.55}
\definecolor{darkred}{rgb}{0.55, 0.0, 0.0}

% Hyperref Setup
\hypersetup{
    colorlinks=true,
    linkcolor=darkblue,
    filecolor=darkblue,      
    urlcolor=darkblue,
    citecolor=darkblue,
}

% Document Title
\title{Cybersecurity Posture Assessment Report}
\author{Cybersecurity Analysis Division}
\date{November 22, 2025}

\begin{document}

\maketitle
\thispagestyle{empty}
\newpage

\tableofcontents
\thispagestyle{empty}
\newpage
\setcounter{page}{1}

% --- 1. Executive Summary ---
\section{Executive Summary}
This report provides a comprehensive cybersecurity assessment for \textbf{Cinder \& Ash}, based on data collected on November 22, 2025. The analysis correlates organizational security controls, external network scan results, and pre-existing risk data.

The assessment identified several high-impact security gaps that require immediate attention. The most critical finding is the absence of Multi-Factor Authentication (MFA) for accessing sensitive data systems, which exposes the organization's most valuable assets to significant risk from compromised credentials.

Furthermore, a technical scan of the external infrastructure revealed a web server running an outdated version of \textbf{Nginx (1.18.0)}. This version is vulnerable to multiple publicly disclosed exploits and presents a direct vector for an external attack.

Key administrative and procedural weaknesses were also noted, including the lack of a formal Acceptable Use Policy and the absence of mandatory security awareness training for new employees during their onboarding process. While some security controls are in place, these identified deficiencies collectively weaken the organization's defense-in-depth strategy.

This report outlines these findings in detail and provides actionable recommendations to mitigate the identified risks and strengthen the overall security posture of \textbf{Cinder \& Ash}.

% --- 2. Organizational Information ---
\section{Organizational Information}
The following information was provided for the assessment. This data forms the baseline for understanding the organization's digital footprint and internal context.

\begin{table}[h!]
\centering
\begin{tabular}{@{}ll@{}}
\toprule
\textbf{Attribute} & \textbf{Value} \\
\midrule
Organization Name & \textbf{Cinder \& Ash} \\
Email Domain & \texttt{CinderAsh.com} \\
Website Domain & \url{www.CinderAsh.com} \\
External IP Address & \texttt{92.36.199.233} \\
Assessment Date & November 22, 2025 \\
\bottomrule
\end{tabular}
\caption{Client Organizational Details}
\end{table}

% --- 3. Security Control Review ---
\section{Security Control Review (Questionnaire Analysis)}
A review of the organization's self-reported security controls was conducted. The following table summarizes the responses and highlights critical gaps where controls are not implemented. A green checkmark (\textcolor{green}{\ding{51}}) indicates a positive control, while a red 'X' (\textcolor{red}{\ding{55}}) indicates a control gap.

\begin{table}[h!]
\centering
\begin{tabular}{>{\raggedright\arraybackslash}p{9cm} c}
\toprule
\textbf{Security Control Question} & \textbf{Status} \\
\midrule
Do you require MFA to access email? & \textcolor{green}{\ding{51}} \\
Do you require MFA to log into computers? & \textcolor{green}{\ding{51}} \\
\textbf{Do you require MFA to access sensitive data systems?} & \textcolor{red}{\ding{55}} \\
\textbf{Does your organization have an employee acceptable use policy?} & \textcolor{red}{\ding{55}} \\
\textbf{Does your organization do security awareness training for new employees?} & \textcolor{red}{\ding{55}} \\
Does your organization do security awareness training for all employees at least once per year? & \textcolor{green}{\ding{51}} \\
\bottomrule
\end{tabular}
\caption{Security Controls Questionnaire Results}
\end{table}

The "No" responses represent significant weaknesses in the organization's security posture. The lack of MFA on sensitive systems is particularly alarming, as it removes a critical layer of defense against credential theft. Similarly, the absence of an acceptable use policy and new-hire training creates unmanaged risk from insider threats, whether malicious or unintentional.

% --- 4. Technical Scan Results ---
\section{Technical Scan Results}
An external network scan was performed on the target IP address \texttt{192.168.10.5} to identify open ports and exposed services.

\subsection{Nmap Scan Findings}
The scan identified one open port, which is hosting a public-facing web service. The details are outlined below.

\begin{table}[h!]
\centering
\begin{tabular}{@{}lllll@{}}
\toprule
\textbf{Port} & \textbf{State} & \textbf{Service} & \textbf{Product} & \textbf{Version} \\
\midrule
443/tcp & Open & https & nginx & \textbf{1.18.0} \\
\bottomrule
\end{tabular}
\caption{Open Ports and Services on \texttt{192.168.10.5}}
\end{table}

\paragraph{Analysis:} The scan confirms that a web server is running \textbf{Nginx version 1.18.0}. This version was released in April 2020 and is now significantly outdated. It is known to be affected by multiple Common Vulnerabilities and Exposures (CVEs), which could allow an attacker to cause a denial of service, bypass security restrictions, or potentially execute arbitrary code. This finding represents a high-severity technical vulnerability.

% --- 5. Consolidated Risk Assessment ---
\section{Consolidated Risk Assessment}
The following table synthesizes the findings from the security control review and the technical scan into a prioritized list of risks. No pre-existing vulnerabilities were reported.

\begin{table}[h!]
\centering
\begin{tabular}{@{}p{4.5cm} p{7.5cm} l@{}}
\toprule
\textbf{Risk Title} & \textbf{Description} & \textbf{Severity} \\
\midrule
\textbf{No MFA on Sensitive Systems} & The absence of MFA on systems containing sensitive data allows an attacker with stolen credentials to gain direct access to critical assets. & \textcolor{red}{\textbf{Critical}} \\
\addlinespace
\textbf{Outdated Nginx Web Server} & The public-facing web server is running Nginx 1.18.0, which is vulnerable to multiple known exploits. This could lead to a system compromise. & \textcolor{orange}{\textbf{High}} \\
\addlinespace
\textbf{No Acceptable Use Policy (AUP)} & Without a formal AUP, there are no clear guidelines for employees on the proper use of company assets, increasing the risk of misuse and insider threats. & \textcolor{orange}{\textbf{High}} \\
\addlinespace
\textbf{No New-Hire Security Training} & New employees are not receiving security awareness training upon joining, leaving them vulnerable to phishing and social engineering attacks. & \textcolor{orange}{\textbf{High}} \\
\bottomrule
\end{tabular}
\caption{Summary of Identified Risks}
\end{table}

% --- 6. Recommendations ---
\section{Recommendations}
Based on the consolidated risk assessment, the following actions are recommended to mitigate the identified vulnerabilities and improve the overall security posture of \textbf{Cinder \& Ash}.

\begin{enumerate}
    \item \textbf{Implement MFA on All Sensitive Systems (Critical):}
    \begin{itemize}
        \item \textbf{Action:} Immediately prioritize the deployment of a robust MFA solution across all applications, databases, and administrative interfaces that process or store sensitive data.
        \item \textbf{Impact:} This will drastically reduce the risk of unauthorized access via compromised credentials.
    \end{itemize}
    \vspace{1em}
    
    \item \textbf{Upgrade Nginx and Implement Patch Management (High):}
    \begin{itemize}
        \item \textbf{Action:} Plan and execute an upgrade of the Nginx server from version 1.18.0 to the latest stable version. Establish a formal patch management policy to ensure all public-facing systems are updated in a timely manner.
        \item \textbf{Impact:} This mitigates the risk of exploitation from known vulnerabilities.
    \end{itemize}
    \vspace{1em}
    
    \item \textbf{Develop and Enforce an Acceptable Use Policy (High):}
    \begin{itemize}
        \item \textbf{Action:} Create a comprehensive AUP that clearly defines the rules for using company networks, devices, and data. Ensure all employees read and formally acknowledge the policy.
        \item \textbf{Impact:} This establishes a baseline for secure user behavior and provides a framework for enforcement.
    \end{itemize}
    \vspace{1em}
    
    \item \textbf{Institute Mandatory Onboarding Security Training (High):}
    \begin{itemize}
        \item \textbf{Action:} Integrate a mandatory security awareness training module into the new-hire onboarding process. This should cover key topics such as phishing, password security, and data handling.
        \item \textbf{Impact:} This ensures that new employees are equipped to identify and resist common cyber threats from their first day.
    \end{itemize}
\end{enumerate}

% --- 7. Conclusion ---
\section{Conclusion}
The assessment for \textbf{Cinder \& Ash} reveals a mixed security posture. While foundational controls like MFA for email are in place, critical and high-severity gaps exist in both technical and administrative domains. The combination of an externally vulnerable web server and weak internal controls creates a significant risk of a security breach.

By implementing the recommendations outlined in this report, particularly the enforcement of MFA on sensitive systems and the patching of the outdated web server, \textbf{Cinder \& Ash} can substantially reduce its attack surface and build a more resilient and defensible security environment.

\end{document}
```