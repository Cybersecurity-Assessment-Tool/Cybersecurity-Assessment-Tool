```latex
\documentclass[12pt]{article}

% Required Packages
\usepackage[margin=1in]{geometry}
\usepackage{pifont} % For checkmarks and crosses
\usepackage{booktabs} % For professional tables
\usepackage{hyperref} % For clickable links
\usepackage{url}      % For URL formatting
\usepackage{seqsplit} % For splitting long strings in tt font

% Document Metadata
\title{Cybersecurity Assessment Report}
\author{Cybersecurity Analyst}
\date{\today}

\begin{document}

\maketitle
\thispagestyle{empty}
\newpage
\tableofcontents
\newpage

% --- Executive Summary ---
\section{Executive Summary}
This report details the findings of a cybersecurity assessment conducted for Aeon Pharmaceuticals. The assessment included a review of organizational security controls, an external network scan, and a correlation with pre-existing risk data.

The analysis revealed several critical and high-risk vulnerabilities that expose the organization to significant threats, including remote code execution, business email compromise, and data breaches. Key findings include:
\begin{itemize}
    \item \textbf{Critical FTP Vulnerability:} A publicly accessible FTP server is running a dangerously outdated version of \texttt{vsftpd} (2.3.4), which is known to contain a critical backdoor vulnerability (CVE-2011-2523).
    \item \textbf{Critical FTP Misconfiguration:} The same FTP server is configured to allow anonymous logins, permitting unauthorized access to stored files.
    \item \textbf{Critical Control Gap:} Multi-Factor Authentication (MFA) is not enforced for email access, leaving the primary communication platform vulnerable to account takeovers.
    \item \textbf{High-Risk Policy Gaps:} The organization lacks a formal Acceptable Use Policy and does not conduct mandatory annual security awareness training for all employees, increasing the risk of insider threats and successful social engineering attacks.
\end{itemize}

Immediate remediation of the identified critical vulnerabilities is strongly recommended to mitigate the risk of a severe security incident.

% --- Organizational Information ---
\section{Organizational Information}
The following information was provided for the assessment.

\begin{tabular}{@{}ll}
    \toprule
    \textbf{Attribute} & \textbf{Value} \\
    \midrule
    Organization Name & Aeon Pharmaceuticals \\
    Email Domain & \seqsplit{\texttt{AeonPharmaceuticals.com}} \\
    Website Domain & \seqsplit{\url{www.AeonPharmaceuticals.com}} \\
    External IP Address & \texttt{65.60.14.167} \\
    \bottomrule
\end{tabular}

% --- Security Control Review ---
\section{Security Control Review}
A review of organizational security controls was conducted based on a standard questionnaire. The responses indicate significant gaps in foundational security policies and practices. A "No" response (\ding{55}) highlights a weakness that should be addressed.

\begin{tabular}{@{}p{0.75\linewidth}c}
    \toprule
    \textbf{Control Question} & \textbf{Response} \\
    \midrule
    Do you require MFA to access email? & \ding{55} \\
    Do you require MFA to log into computers? & \ding{51} \\
    Do you require MFA to access sensitive data systems? & \ding{51} \\
    Does your organization have an employee acceptable use policy? & \ding{55} \\
    Does your organization do security awareness training for new employees? & \ding{51} \\
    Does your organization do security awareness training for all employees at least once per year? & \ding{55} \\
    \bottomrule
\end{tabular}

% --- Technical Scan Results ---
\section{Technical Scan Results}
An Nmap scan was performed on the target system to identify open ports and exposed services. The scan identified one host as active and responsive.

\subsection{Host: \texttt{10.0.0.15}}
The following services were found to be exposed:

\begin{tabular}{@{}lllll}
    \toprule
    \textbf{Port} & \textbf{State} & \textbf{Service} & \textbf{Product \& Version} & \textbf{Notes} \\
    \midrule
    21/tcp & open & ftp & vsftpd 2.3.4 & \begin{tabular}[t]{@{}l@{}}Anonymous FTP login allowed.\\ \textbf{CRITICAL:} Version is vulnerable \\ to remote code execution \\ (CVE-2011-2523).\end{tabular} \\
    \bottomrule
\end{tabular}

% --- Risk Assessment ---
\section{Risk Assessment}
The following table synthesizes findings from the security control review, technical scan, and pre-existing risk data. Risks are prioritized by severity to guide remediation efforts.

\begin{tabular}{@{}p{0.3\linewidth}p{0.5\linewidth}l}
    \toprule
    \textbf{Risk / Vulnerability} & \textbf{Description} & \textbf{Severity} \\
    \midrule
    \textbf{FTP Server Backdoor} & The version of \texttt{vsftpd} (2.3.4) in use contains a well-known backdoor that allows an attacker to gain command-line access to the server. & \textbf{Critical} \\
    \addlinespace
    \textbf{Anonymous FTP Access} & The FTP server is misconfigured to allow anonymous logins, enabling unauthorized users to access, download, or upload files, leading to a potential data breach. & \textbf{Critical} \\
    \addlinespace
    \textbf{No MFA for Email} & Lack of MFA on email accounts makes them highly susceptible to phishing and credential stuffing attacks, which can lead to Business Email Compromise (BEC). & \textbf{Critical} \\
    \addlinespace
    \textbf{No Acceptable Use Policy} & Without a formal AUP, employees may be unaware of security responsibilities, increasing the likelihood of unintentional policy violations and insider threats. & High \\
    \addlinespace
    \textbf{No Annual Security Training} & Security knowledge degrades over time. Without annual training, employees are less prepared to identify and resist modern phishing and social engineering attacks. & High \\
    \addlinespace
    \textbf{Outdated Windows Policy} & Workstations are running Windows 7, an end-of-life operating system that no longer receives security updates, leaving them vulnerable to exploitation. & Medium \\
    \bottomrule
\end{tabular}

% --- Recommendations ---
\section{Recommendations}
Based on the assessment, the following actions are recommended, prioritized by severity.

\subsection{Immediate Actions (Critical Priority)}
\begin{enumerate}
    \item \textbf{Remediate FTP Server:} Immediately take the FTP server offline. If the service is business-critical, upgrade \texttt{vsftpd} to the latest stable version or replace it with a secure alternative like SFTP (SSH File Transfer Protocol). If it is not critical, decommission it permanently.
    \item \textbf{Disable Anonymous FTP:} Regardless of the remediation path, immediately disable anonymous FTP access to prevent unauthorized file access.
    \item \textbf{Enforce MFA on Email:} Implement and enforce mandatory Multi-Factor Authentication for all user email accounts without delay.
\end{enumerate}

\subsection{High Priority Actions}
\begin{enumerate}
    \item \textbf{Develop an Acceptable Use Policy (AUP):} Create a formal AUP that clearly defines rules for the use of company technology and data. Ensure all employees read and acknowledge the policy.
    \item \textbf{Implement Annual Security Training:} Establish a mandatory security awareness training program for all employees to be completed on an annual basis. The training should cover phishing, password security, and social engineering.
\end{enumerate}

\subsection{Medium Priority Actions}
\begin{enumerate}
    \item \textbf{Upgrade End-of-Life Systems:} Continue with the planned project to upgrade all workstations from Windows 7 to a modern, supported operating system like Windows 10 or 11.
\end{enumerate}

\end{document}
```