```latex
\documentclass[12pt]{article}

% --- PACKAGES ---
\usepackage[margin=1in]{geometry}
\usepackage{pifont} % For checkmarks and crosses
\usepackage{booktabs} % For professional tables
\usepackage{hyperref} % For clickable links
\usepackage{url} % For URL formatting
\usepackage{seqsplit} % To split long strings in tt font
\usepackage{xcolor} % For colors
\usepackage{fancyhdr} % For headers and footers

% --- DOCUMENT SETUP ---
\hypersetup{
    colorlinks=true,
    linkcolor=blue,
    filecolor=magenta,      
    urlcolor=cyan,
    pdftitle={Cybersecurity Posture Report},
    pdfpagemode=FullScreen,
}

% --- HEADER & FOOTER ---
\pagestyle{fancy}
\fancyhf{} % Clear all header and footer fields
\fancyhead[L]{Cybersecurity Posture Report}
\fancyhead[R]{For: Aeon Pharmaceuticals}
\fancyfoot[C]{\thepage}
\renewcommand{\headrulewidth}{0.4pt}
\renewcommand{\footrulewidth}{0.4pt}

% --- DOCUMENT START ---
\begin{document}

% --- TITLE PAGE ---
\begin{titlepage}
    \centering
    \vspace*{\stretch{1.0}}
    {\Huge\bfseries Cybersecurity Posture Report\par}
    \vspace{1.5cm}
    {\Large\bfseries Prepared for:\par}
    \vspace{0.5cm}
    {\LARGE Aeon Pharmaceuticals\par}
    \vspace{2cm}
    {\large Report Date: \today\par}
    \vspace*{\stretch{2.0}}
    \vfill
    {\small Confidential \textbar\ Internal Use Only}
\end{titlepage}

\tableofcontents
\newpage

% --- SECTION 1: EXECUTIVE OVERVIEW ---
\section*{Executive Overview}
This report provides a comprehensive analysis of the cybersecurity posture for Aeon Pharmaceuticals, based on a review of organizational security controls, a technical network scan, and pre-existing risk data. The assessment synthesizes these data points to identify key vulnerabilities, control gaps, and provide actionable recommendations.

The analysis revealed several high-priority areas requiring immediate attention. \textbf{Critically, Multi-Factor Authentication (MFA) is not enforced for accessing email or other sensitive data systems.} This represents a significant vulnerability, leaving the organization susceptible to account compromise and subsequent data breaches. Furthermore, the lack of mandatory security awareness training for new employees creates an ongoing risk from phishing and social engineering attacks.

On a positive note, a technical scan of the target host \texttt{192.168.0.5} did not validate a previously identified risk concerning an open, unencrypted web server port (Port 80). The scan found this port to be closed, suggesting the risk has been remediated.

Overall, while some security controls are in place, the identified gaps in authentication and employee onboarding present a high level of risk to the organization. The recommendations in this report are prioritized to address these critical weaknesses first.

% --- SECTION 2: ORGANIZATIONAL INFORMATION ---
\section*{Organizational Information}
The following details were provided for the assessment scope.
\begin{itemize}
    \item \textbf{Organization Name:} Aeon Pharmaceuticals
    \item \textbf{Email Domain:} \texttt{AeonPharmaceuticals.com}
    \item \textbf{Website Domain:} \texttt{www.AeonPharmaceuticals.com}
    \item \textbf{External IP Address:} \texttt{115.99.215.195}
\end{itemize}

% --- SECTION 3: SECURITY CONTROL REVIEW ---
\section*{Security Control Review}
An assessment of organizational security controls was conducted via a questionnaire. The responses are summarized below. Gaps identified by a "No" answer are significant and directly contribute to the organization's risk profile.

\begin{table}[h!]
\centering
\caption{Security Controls Questionnaire Analysis}
\begin{tabular}{@{}p{0.6\linewidth} c l@{}}
\toprule
\textbf{Control Question} & \textbf{Response} & \textbf{Assessment} \\
\midrule
Do you require MFA to access email? & \ding{55} & \textcolor{red}{\textbf{Critical Gap}} \\
Do you require MFA to log into computers? & \ding{51} & Meets Best Practice \\
Do you require MFA to access sensitive data systems? & \ding{55} & \textcolor{red}{\textbf{Critical Gap}} \\
Does your organization have an employee acceptable use policy? & \ding{51} & Meets Best Practice \\
Does your organization do security awareness training for new employees? & \ding{55} & \textcolor{orange}{\textbf{High Risk}} \\
Does your organization do security awareness training for all employees at least once per year? & \ding{51} & Meets Best Practice \\
\bottomrule
\end{tabular}
\end{table}

% --- SECTION 4: TECHNICAL SCAN RESULTS ---
\section*{Technical Scan Results}
A network scan was performed to identify open ports and exposed services on the specified target system.

\begin{itemize}
    \item \textbf{Target IP Address:} \texttt{192.168.0.5}
    \item \textbf{Scan Date:} \today
\end{itemize}

The scan results are detailed in the table below. No open ports were discovered on the target host at the time of the scan.

\begin{table}[h!]
\centering
\caption{Nmap Scan Results for \texttt{192.168.0.5}}
\begin{tabular}{@{}llll@{}}
\toprule
\textbf{Port} & \textbf{State} & \textbf{Service} & \textbf{Version} \\
\midrule
80/tcp & closed & http & N/A \\
\bottomrule
\end{tabular}
\end{table}

\subsection*{Analysis of Technical Findings}
The technical scan indicates a strong network perimeter for the scanned host, with no services exposed to the network. This directly contradicts a pre-existing risk entry (\textit{Unencrypted Web Server}) which stated Port 80 was open. This finding suggests that the previously identified risk has been successfully remediated.

% --- SECTION 5: RISK ASSESSMENT SUMMARY ---
\section*{Risk Assessment Summary}
The following table synthesizes findings from the security control review, technical scan, and pre-existing risk data into a unified risk summary.

\begin{table}[h!]
\centering
\caption{Synthesized Risk Register}
\begin{tabular}{@{}p{0.4\linewidth} p{0.2\linewidth} p{0.3\linewidth}@{}}
\toprule
\textbf{Risk Description} & \textbf{Severity} & \textbf{Source / Status} \\
\midrule
\textbf{Lack of MFA on Email \& Sensitive Systems:} User accounts are vulnerable to takeover via credential theft, leading to potential data breaches. & \textcolor{red}{\textbf{Critical}} & Questionnaire Finding \\
\addlinespace
\textbf{No Security Training for New Hires:} New employees are not equipped to identify or respond to phishing and social engineering attacks. & \textcolor{orange}{\textbf{High}} & Questionnaire Finding \\
\addlinespace
\textbf{Unencrypted Web Server (Port 80):} A previously identified risk that an unencrypted web server was exposed. & \textcolor{green}{Medium} & \textbf{Status: Not Validated.} The technical scan found Port 80 to be closed. \\
\bottomrule
\end{tabular}
\end{table}

% --- SECTION 6: RECOMMENDATIONS ---
\section*{Recommendations}
Based on the analysis, the following actions are recommended to mitigate the identified risks and improve the overall security posture of Aeon Pharmaceuticals. Recommendations are prioritized by severity.

\subsection*{1. Enforce Multi-Factor Authentication (Priority: Critical)}
\begin{itemize}
    \item \textbf{Action:} Immediately begin a phased rollout of mandatory MFA for all user accounts when accessing company email and any system classified as containing sensitive data.
    \item \textbf{Justification:} MFA is one of the most effective controls to prevent unauthorized access. It mitigates the risk of compromised credentials, which is a primary vector in modern cyberattacks. This action directly addresses the most critical vulnerability identified.
\end{itemize}

\subsection*{2. Implement Onboarding Security Training (Priority: High)}
\begin{itemize}
    \item \textbf{Action:} Develop and integrate a mandatory security awareness training module into the new employee onboarding process. This training should cover, at a minimum, phishing identification, acceptable use policies, and incident reporting procedures.
    \item \textbf{Justification:} New employees are often targeted by attackers. Equipping them with security knowledge from day one reduces the organization's susceptibility to human-centric attacks and fosters a culture of security.
\end{itemize}

\subsection*{3. Update and Validate Risk Register (Priority: Informational)}
\begin{itemize}
    \item \textbf{Action:} Formally verify that the server previously associated with the "Unencrypted Web Server" risk no longer requires Port 80. Update the internal risk register to mark this finding as remediated.
    \item \textbf{Justification:} Maintaining an accurate risk register is crucial for effective security management. This ensures that resources are allocated to current, valid risks rather than outdated information.
\end{itemize}

\end{document}
```