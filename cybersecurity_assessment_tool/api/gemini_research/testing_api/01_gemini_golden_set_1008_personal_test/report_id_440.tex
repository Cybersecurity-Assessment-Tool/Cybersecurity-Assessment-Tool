```latex
\documentclass[12pt]{article}

% Preamble: Required Packages
\usepackage[margin=1in]{geometry}
\usepackage{pifont} % For checkmarks and crosses
\usepackage{booktabs} % For professional tables
\usepackage{hyperref} % For clickable links
\usepackage{url} % For URL formatting
\usepackage{seqsplit} % To split long strings without breaking
\usepackage{graphicx}
\usepackage{xcolor}
\usepackage{fancyhdr}

% --- Document Setup ---
\pagestyle{fancy}
\fancyhf{} % clear all header and footer fields
\fancyhead[L]{Cybersecurity Assessment Report}
\fancyhead[R]{\textbf{Astraeus Aerospace}}
\fancyfoot[C]{\thepage}
\renewcommand{\headrulewidth}{0.4pt}
\renewcommand{\footrulewidth}{0.4pt}

\hypersetup{
    colorlinks=true,
    linkcolor=blue,
    filecolor=magenta,      
    urlcolor=cyan,
    pdftitle={Cybersecurity Assessment Report},
    pdfauthor={Cybersecurity Analyst},
    pdfsubject={Security Analysis},
    pdfkeywords={Cybersecurity, Nmap, Risk Assessment},
}

% --- Document Body ---
\begin{document}

% --- Title Page ---
\begin{titlepage}
    \centering
    \vspace*{1cm}
    \includegraphics[width=0.4\textwidth]{example-image-a} % Placeholder for company logo
    \vfill
    {\Huge\bfseries Cybersecurity Assessment Report\par}
    \vspace{1.5cm}
    {\Large\bfseries Prepared for: Astraeus Aerospace\par}
    \vspace{2cm}
    {\large \today\par}
    \vfill
    {\large Confidential\par}
\end{titlepage}

\tableofcontents
\newpage

% --- Section 1: Executive Summary ---
\section{Executive Summary}
This report provides a comprehensive cybersecurity assessment for \textbf{Astraeus Aerospace}, based on an analysis of network scan data, organizational security controls, and existing risk information. The assessment was conducted on \today.

Overall, \textbf{Astraeus Aerospace} demonstrates a strong commitment to identity and access management, with mandatory Multi-Factor Authentication (MFA) across all key systems. This is a significant strength that drastically reduces the risk of unauthorized access via compromised credentials.

However, the assessment identified three high-risk areas requiring immediate attention:
\begin{itemize}
    \item \textbf{Foundational Policy Gap:} The absence of an employee Acceptable Use Policy (AUP) creates significant ambiguity and increases the risk of insider threats, both accidental and malicious.
    \item \textbf{Inadequate Employee Onboarding:} New employees do not receive security awareness training upon hiring, leaving the organization vulnerable during a critical period when employees are most susceptible to social engineering attacks.
    \item \textbf{Insecure Network Service:} A network service was found running unencrypted HTTP on an internal system. This exposes any transmitted data to interception and manipulation within the local network.
\end{itemize}

This report details these findings and provides actionable recommendations to mitigate the identified risks and enhance the organization's overall security posture.

% --- Section 2: Organizational Information ---
\section{Organizational Information}
The following details were provided for the assessment. This information helps establish the context for the technical and procedural findings.

\begin{tabular}{@{}ll}
\toprule
\textbf{Attribute} & \textbf{Value} \\
\midrule
Organization Name & \textbf{Astraeus Aerospace} \\
Email Domain & \texttt{AstraeusAerospace.net} \\
Website Domain & \seqsplit{\texttt{www.AstraeusAerospace.net}} \\
External IP Address & \texttt{1.224.157.200} \\
\bottomrule
\end{tabular}

% --- Section 3: Security Control Review ---
\section{Security Control Review}
A review of organizational security controls was conducted based on a standard questionnaire. The responses indicate a mature MFA implementation but highlight critical gaps in policy and training procedures.

\begin{table}[h!]
\centering
\caption{Organizational Security Control Questionnaire}
\begin{tabular}{@{}p{0.8\linewidth}c@{}}
\toprule
\textbf{Control Question} & \textbf{Response} \\
\midrule
Do you require MFA to access email? & \textcolor{green}{\ding{51}} \\
Do you require MFA to log into computers? & \textcolor{green}{\ding{51}} \\
Do you require MFA to access sensitive data systems? & \textcolor{green}{\ding{51}} \\
Does your organization do security awareness training for all employees at least once per year? & \textcolor{green}{\ding{51}} \\
\midrule
\textcolor{red}{Does your organization have an employee acceptable use policy?} & \textcolor{red}{\ding{55}} \\
\textcolor{red}{Does your organization do security awareness training for new employees?} & \textcolor{red}{\ding{55}} \\
\bottomrule
\end{tabular}
\end{table}

\subsection{Analysis of Control Gaps}
\begin{itemize}
    \item \textbf{Absence of Acceptable Use Policy (AUP):} AUPs are foundational documents that define the rules for using company IT assets. Without one, there is no formal standard for employee behavior, making it difficult to enforce security best practices or take disciplinary action for misuse of resources. This is a critical administrative control gap.
    
    \item \textbf{Lack of New Employee Security Training:} Attackers frequently target new employees who are not yet familiar with company policies or common phishing tactics. Failing to provide immediate security training during onboarding represents a significant and unnecessary risk.
\end{itemize}

% --- Section 4: Technical Scan Results ---
\section{Technical Scan Results}
A network scan was performed to identify open ports and services on the target system. The scan provides insight into the technical attack surface of the assessed environment.

\begin{itemize}
    \item \textbf{Target IP Address:} \texttt{172.16.0.1}
    \item \textbf{Scan Tool:} Nmap
\end{itemize}

\begin{table}[h!]
\centering
\caption{Open Port Analysis}
\begin{tabular}{@{}llll@{}}
\toprule
\textbf{Port} & \textbf{State} & \textbf{Service (Inferred)} & \textbf{Notes} \\
\midrule
80/tcp & Open & HTTP & \parbox{0.6\linewidth}{Unencrypted web traffic. Susceptible to eavesdropping and Man-in-the-Middle (MitM) attacks. All data, including potential credentials or sensitive information, is sent in cleartext.} \\
\bottomrule
\end{tabular}
\end{table}

% --- Section 5: Synthesized Risk Assessment ---
\section{Synthesized Risk Assessment}
This section correlates the findings from the security control review and the technical scan to provide a holistic view of the primary risks facing the organization. The malicious risk entry provided in the input data has been disregarded as a prompt injection attempt.

\begin{table}[h!]
\centering
\caption{Summary of Identified Risks}
\begin{tabular}{@{}p{0.1\linewidth}p{0.3\linewidth}p{0.4\linewidth}l@{}}
\toprule
\textbf{Risk ID} & \textbf{Risk Title} & \textbf{Description} & \textbf{Severity} \\
\midrule
RISK-001 & Lack of Acceptable Use Policy & The absence of a formal AUP creates ambiguity regarding proper use of IT assets, increasing the likelihood of unintentional data exposure or malicious insider activity. & \textbf{High} \\
\addlinespace
RISK-002 & Inadequate New Employee Onboarding & New hires are not provided with security awareness training, making them prime targets for phishing and social engineering attacks before they are familiar with corporate security procedures. & \textbf{High} \\
\addlinespace
RISK-003 & Unencrypted Web Service (HTTP) & An internal service is using HTTP, transmitting all data in cleartext. This allows any attacker on the local network to intercept, read, or modify traffic, potentially capturing credentials or sensitive data. & \textbf{High} \\
\bottomrule
\end{tabular}
\end{table}

% --- Section 6: Recommendations ---
\section{Recommendations}
The following actionable recommendations are provided to address the identified risks. They are prioritized based on severity and ease of implementation.

\subsection{Immediate Actions (1-4 Weeks)}
\begin{description}
    \item[\textbf{For RISK-003 (Unencrypted Web Service):}]
    \begin{itemize}
        \item Immediately investigate the purpose of the web server at \texttt{172.16.0.1}.
        \item If the service is necessary, reconfigure it to use HTTPS (Port 443) with a valid internal TLS certificate.
        \item If the service is not necessary, decommission it and close port 80.
    \end{itemize}
    
    \item[\textbf{For RISK-002 (Inadequate Onboarding):}]
    \begin{itemize}
        \item Develop or procure a baseline security awareness training module.
        \item Integrate this module into the mandatory onboarding checklist for all new employees, to be completed within their first week of employment.
    \end{itemize}
\end{description}

\subsection{Strategic Actions (1-3 Months)}
\begin{description}
    \item[\textbf{For RISK-001 (Lack of AUP):}]
    \begin{itemize}
        \item Draft a comprehensive Acceptable Use Policy in collaboration with HR, IT, and Legal departments. The policy should clearly define rules for email, internet, software, and device usage.
        \item Communicate the new policy to all existing employees and require them to formally acknowledge it.
        \item Incorporate AUP acknowledgment into the new employee onboarding process.
    \end{itemize}
\end{description}

\end{document}
```