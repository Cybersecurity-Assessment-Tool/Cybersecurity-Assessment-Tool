```latex
\documentclass[12pt]{article}

% Preamble: Required Packages
\usepackage[margin=1in]{geometry}
\usepackage{pifont} % For checkmarks and crosses
\usepackage{booktabs} % For professional tables
\usepackage{hyperref} % For clickable links and TOC
\usepackage{url} % For formatting URLs
\usepackage{seqsplit} % For splitting long strings
\usepackage{graphicx}
\usepackage{xcolor}

% Document Metadata
\title{Cybersecurity Posture Assessment Report}
\author{Cybersecurity Analysis Division}
\date{\today}

% Hyperref Setup
\hypersetup{
    colorlinks=true,
    linkcolor=blue,
    filecolor=magenta,      
    urlcolor=cyan,
    pdftitle={Cybersecurity Posture Assessment Report},
    pdfpagemode=FullScreen,
}

\begin{document}

\maketitle
\thispagestyle{empty}
\newpage

\tableofcontents
\thispagestyle{empty}
\newpage

\setcounter{page}{1}

% ===================================================================
% SECTION 1: EXECUTIVE OVERVIEW
% ===================================================================
\section{Executive Overview}

This report provides a comprehensive analysis of the cybersecurity posture for \textbf{Titanium Core}. The assessment is based on a correlation of organizational data, a review of security controls, and an external network scan conducted on the designated public-facing IP address.

The external network scan of \texttt{173.129.78.193} revealed no open ports or exposed services. This indicates a strong network perimeter security posture, likely enforced by a well-configured firewall, which significantly reduces the external attack surface.

However, the security control review identified several critical internal weaknesses. The absence of mandatory Multi-Factor Authentication (MFA) for computer and sensitive data system access presents a \textbf{Critical} risk. This gap could allow an attacker with stolen credentials to gain unauthorized access to endpoints and critical data. Furthermore, the lack of a formal security awareness training program for new and existing employees constitutes a \textbf{High} risk, leaving the organization vulnerable to phishing, social engineering, and other human-targeted attacks.

Immediate remediation should focus on implementing a comprehensive MFA policy and establishing a robust security awareness training program to mitigate these significant risks.

% ===================================================================
% SECTION 2: ORGANIZATIONAL INFORMATION
% ===================================================================
\section{Organizational Information}

The following details were provided for the assessment scope.

\begin{itemize}
    \item \textbf{Organization Name:} Titanium Core
    \item \textbf{Email Domain:} \texttt{TitaniumCore.org}
    \item \textbf{Website Domain:} \url{www.TitaniumCore.org}
    \item \textbf{External IP Scanned:} \texttt{173.129.78.193}
\end{itemize}

% ===================================================================
% SECTION 3: SECURITY CONTROL REVIEW
% ===================================================================
\section{Security Control Review}

A review of administrative and technical security controls was conducted based on the provided questionnaire. The results highlight key areas of strength and weakness in the current security framework. Answers of ``No'' indicate significant control gaps that require immediate attention.

\begin{table}[h!]
\centering
\caption{Security Control Questionnaire Analysis}
\label{tab:controls}
\begin{tabular}{@{}p{8cm}cc@{}}
\toprule
\textbf{Control Question} & \textbf{Response} & \textbf{Assessment} \\
\midrule
Do you require MFA to access email? & \ding{51} Yes & Control Met \\
Do you require MFA to log into computers? & \ding{55} No & \textbf{Critical Gap} \\
Do you require MFA to access sensitive data systems? & \ding{55} No & \textbf{Critical Gap} \\
Does your organization have an employee acceptable use policy? & \ding{51} Yes & Control Met \\
Does your organization do security awareness training for new employees? & \ding{55} No & \textbf{High Risk Gap} \\
Does your organization do security awareness training for all employees at least once per year? & \ding{55} No & \textbf{High Risk Gap} \\
\bottomrule
\end{tabular}
\end{table}

% ===================================================================
% SECTION 4: TECHNICAL SCAN RESULTS
% ===================================================================
\section{Technical Scan Results}

An external network vulnerability scan was performed to identify exposed services and potential vulnerabilities on the organization's public-facing infrastructure.

\begin{itemize}
    \item \textbf{Scan Target:} \texttt{[Target IP]} (Note: Target was empty in scan data; using IP from organizational data: \texttt{173.129.78.193})
    \item \textbf{Scan Date:} \today
\end{itemize}

\subsection{Findings}
The scan completed successfully but did not detect any open TCP or UDP ports on the target system.

\textbf{Conclusion:} This result strongly suggests that the target host is protected by a firewall that is properly configured to block unsolicited inbound traffic. While this is a positive security finding, it also limits the scope of a purely external technical assessment. The primary risks identified in this report are therefore based on the internal control gaps identified in Section 3.

% ===================================================================
% SECTION 5: RISK ASSESSMENT SUMMARY
% ===================================================================
\section{Risk Assessment Summary}

The following table summarizes the key risks identified during this assessment, combining findings from the security control review and technical analysis. These risks should be prioritized for remediation.

\begin{table}[h!]
\centering
\caption{Identified Risks and Severity}
\label{tab:risks}
\begin{tabular}{@{}p{1.5cm}p{4cm}p{6cm}l@{}}
\toprule
\textbf{Risk ID} & \textbf{Risk Name} & \textbf{Description} & \textbf{Severity} \\
\midrule
RISK-001 & Lack of MFA on Endpoints & The absence of MFA for computer logins allows an attacker with compromised credentials to gain full access to an employee's workstation, facilitating lateral movement. & \textbf{Critical} \\
\addlinespace
RISK-002 & Lack of MFA on Sensitive Systems & Critical data systems are accessible without a secondary authentication factor. Stolen credentials could directly lead to a significant data breach. & \textbf{Critical} \\
\addlinespace
RISK-003 & Inadequate Security Awareness Program & Without initial and ongoing training, employees are more likely to fall victim to phishing and social engineering attacks, which are primary vectors for initial compromise. & \textbf{High} \\
\bottomrule
\end{tabular}
\end{table}

% ===================================================================
% SECTION 6: RECOMMENDATIONS
% ===================================================================
\section{Recommendations}

The following actions are recommended to mitigate the identified risks and improve the overall security posture of \textbf{Titanium Core}.

\subsection{Remediation for RISK-001 \& RISK-002 (MFA Implementation)}
\begin{itemize}
    \item \textbf{Action:} Implement and enforce mandatory Multi-Factor Authentication (MFA) across all employee endpoints (laptops, desktops) and for all systems that store, process, or access sensitive organizational data.
    \item \textbf{Details:} Prioritize systems such as VPNs, administrative portals, databases, and financial applications. Utilize solutions that support modern authentication methods like authenticator apps (TOTP), hardware tokens (FIDO2/U2F), or biometrics.
    \item \textbf{Impact:} Drastically reduces the risk of unauthorized access via compromised credentials.
\end{itemize}

\subsection{Remediation for RISK-003 (Security Awareness)}
\begin{itemize}
    \item \textbf{Action:} Establish a formal, mandatory security awareness training program.
    \item \textbf{Details:} The program should include:
    \begin{enumerate}
        \item \textbf{Onboarding Training:} A required module for all new employees covering core topics like acceptable use, phishing identification, password hygiene, and incident reporting.
        \item \textbf{Annual Refresher Training:} A yearly mandatory course for all staff to reinforce security concepts and provide updates on new threats.
        \item \textbf{Phishing Simulations:} Conduct regular, simulated phishing campaigns to test employee awareness and provide targeted, just-in-time training to those who click malicious links.
    \end{enumerate}
    \item \textbf{Impact:} Strengthens the "human firewall" and reduces the likelihood of a successful social engineering attack.
\end{itemize}

\end{document}
```