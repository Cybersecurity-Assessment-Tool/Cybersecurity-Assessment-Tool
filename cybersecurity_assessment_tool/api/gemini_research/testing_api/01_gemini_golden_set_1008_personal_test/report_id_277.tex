```latex
\documentclass[12pt]{article}

% Preamble: Required Packages and Document Setup
\usepackage[margin=1in]{geometry}
\usepackage{pifont} % For checkmarks (\ding{51}) and crosses (\ding{55})
\usepackage{booktabs} % For professional-looking tables
\usepackage{hyperref} % For clickable links and references
\usepackage{url} % For formatting URLs
\usepackage{seqsplit} % For splitting long, unbreakable strings like hashes or tokens
\usepackage{graphicx}
\usepackage{fancyhdr}
\usepackage{lastpage}
\usepackage{xcolor}

% Define custom colors
\definecolor{darkblue}{rgb}{0.0, 0.0, 0.55}
\definecolor{darkred}{rgb}{0.55, 0.0, 0.0}

% Hyperref setup
\hypersetup{
    colorlinks=true,
    linkcolor=darkblue,
    filecolor=darkblue,      
    urlcolor=darkblue,
    citecolor=darkblue,
}

% Header and Footer Configuration
\pagestyle{fancy}
\fancyhf{} % Clear all header and footer fields
\fancyhead[L]{\textbf{Cybersecurity Posture Assessment}}
\fancyhead[R]{\textbf{Pacific Rim Exports}}
\fancyfoot[C]{\thepage\ of \pageref{LastPage}}
\renewcommand{\headrulewidth}{0.4pt}
\renewcommand{\footrulewidth}{0.4pt}

% Document Title Block
\title{
    \vspace{-2cm}
    \includegraphics[width=0.3\textwidth]{example-image-a} \\ % Placeholder for a logo
    \vspace{1cm}
    \textbf{Cybersecurity Posture Assessment Report} \\
    \large For: \textbf{Pacific Rim Exports}
}
\author{Cybersecurity Analysis Division}
\date{\today}

\begin{document}

\maketitle
\thispagestyle{empty}
\newpage

\tableofcontents
\newpage

% --- Executive Summary ---
\section{Executive Summary}
This report provides a comprehensive analysis of the cybersecurity posture for \textbf{Pacific Rim Exports}, based on a review of organizational security controls, an external network scan, and pre-existing risk data. The assessment was conducted on November 22, 2025.

The organization demonstrates a solid foundation in security awareness, with established policies and training programs for employees. However, this assessment has identified several high-impact vulnerabilities that require immediate attention to mitigate significant risks.

\textbf{Key findings include:}
\begin{itemize}
    \item \textbf{Critical Control Gaps:} Multi-Factor Authentication (MFA) is not enforced for accessing email or other sensitive data systems. This exposes the organization to a high risk of account compromise and subsequent data breaches.
    \item \textbf{High-Risk Technical Vulnerability:} The public-facing web server is running an outdated version of Nginx (1.18.0), which is known to have multiple security vulnerabilities. This could allow an attacker to compromise the server and gain access to the internal network.
\end{itemize}

Immediate remediation of these issues is strongly recommended to reduce the organization's attack surface and protect critical assets. Detailed findings and actionable recommendations are provided in the subsequent sections of this report.

% --- Organizational Information ---
\section{Organizational Information}
The following details were provided for the assessment. This information helps to establish the context and scope of the review.

\begin{tabular}{@{}ll@{}}
\toprule
\textbf{Attribute} & \textbf{Value} \\ \midrule
Organization Name & \textbf{Pacific Rim Exports} \\
Email Domain & \texttt{PacificRimExports.com} \\
External IP Address & \texttt{33.249.255.167} \\
\bottomrule
\end{tabular}

% --- Security Control Review ---
\section{Security Control Review}
A review of self-reported security controls was conducted via a standardized questionnaire. The results highlight the current state of administrative and technical safeguards.

\begin{table}[h!]
\centering
\caption{Security Controls Questionnaire Results}
\begin{tabular}{@{}p{0.7\linewidth}c@{}}
\toprule
\textbf{Security Control Question} & \textbf{Status} \\ \midrule
Do you require MFA to access email? & \ding{55} \\
Do you require MFA to log into computers? & \ding{51} \\
Do you require MFA to access sensitive data systems? & \ding{55} \\
Does your organization have an employee acceptable use policy? & \ding{51} \\
Does your organization do security awareness training for new employees? & \ding{51} \\
Does your organization do security awareness training for all employees at least once per year? & \ding{51} \\
\bottomrule
\end{tabular}
\end{table}

\paragraph{Analysis:} The questionnaire reveals critical gaps in the implementation of Multi-Factor Authentication (MFA). While MFA is commendably used for computer logins, its absence on email and sensitive data systems constitutes a major security risk. Email is a primary target for phishing attacks, and a compromised account can serve as a gateway for further network intrusion. Lack of MFA on sensitive systems directly exposes the organization's most valuable data.

% --- Technical Scan Results ---
\section{Technical Scan Results}
A network scan was performed to identify open ports, running services, and potential vulnerabilities on the organization's external infrastructure.

\subsection{External Network Scan (Nmap)}
\begin{itemize}
    \item \textbf{Scan Date:} 2025-11-22T10:00:00Z
    \item \textbf{Target IP:} \texttt{192.168.10.5}
\end{itemize}

The following table details the open ports and services discovered on the target system.

\begin{table}[h!]
\centering
\caption{Open Port and Service Details}
\begin{tabular}{@{}lllll@{}}
\toprule
\textbf{Port} & \textbf{State} & \textbf{Service} & \textbf{Product} & \textbf{Version} \\ \midrule
443/tcp & open & https & nginx & 1.18.0 \\
\bottomrule
\end{tabular}
\end{table}

\paragraph{Analysis:} The scan identified an Nginx web server, version \textbf{1.18.0}, accessible on port 443 (HTTPS). This version was released in April 2020 and is now considered outdated and end-of-life. It is susceptible to numerous publicly disclosed vulnerabilities, including but not limited to request smuggling and denial-of-service attacks. Running outdated software on an internet-facing server presents a high risk of compromise.

% --- Risk Assessment ---
\section{Risk Assessment}
This section synthesizes the findings from the security control review and the technical scan into a prioritized list of identified risks. No pre-existing risks were reported.

\begin{table}[h!]
\centering
\caption{Summary of Identified Risks}
\begin{tabular}{@{}p{0.15\linewidth}p{0.5\linewidth}l@{}}
\toprule
\textbf{Risk ID} & \textbf{Risk Title and Description} & \textbf{Severity} \\ \midrule
\textbf{RISK-001} & \textbf{Lack of MFA for Email Access} \newline A primary vector for account takeover, leading to data breaches, phishing, and further network compromise. & \textcolor{red}{\textbf{Critical}} \\
\addlinespace
\textbf{RISK-002} & \textbf{Lack of MFA for Sensitive Data Systems} \newline Direct exposure of critical business data to unauthorized access if credentials are stolen. & \textcolor{red}{\textbf{Critical}} \\
\addlinespace
\textbf{RISK-003} & \textbf{Outdated Nginx Web Server} \newline The public-facing web server is running a version with known vulnerabilities, making it a prime target for automated attacks. & \textcolor{orange}{\textbf{High}} \\
\bottomrule
\end{tabular}
\end{table}

% --- Recommendations ---
\section{Recommendations}
Based on the identified risks, the following prioritized actions are recommended to improve the security posture of \textbf{Pacific Rim Exports}.

\begin{enumerate}
    \item \textbf{[Critical] Implement MFA for All Email Accounts:}
    \begin{itemize}
        \item \textbf{Action:} Enforce MFA for all user accounts accessing the email system (e.g., Microsoft 365, Google Workspace).
        \item \textbf{Impact:} Drastically reduces the risk of business email compromise (BEC) and phishing-related account takeovers.
    \end{itemize}
    \vspace{0.5cm}
    \item \textbf{[Critical] Enforce MFA for Access to Sensitive Systems:}
    \begin{itemize}
        \item \textbf{Action:} Identify all systems containing sensitive data (e.g., financial, customer, proprietary) and immediately enforce MFA for all user access, including administrative accounts.
        \item \textbf{Impact:} Protects the organization's most critical data assets from unauthorized access, even if user credentials are compromised.
    \end{itemize}
    \vspace{0.5cm}
    \item \textbf{[High] Upgrade the Nginx Web Server:}
    \begin{itemize}
        \item \textbf{Action:} Plan and execute an upgrade of the Nginx server from version 1.18.0 to the latest stable version. This should be done in a test environment first to ensure compatibility.
        \item \textbf{Impact:} Patches numerous known security vulnerabilities, significantly hardening the external perimeter against automated and targeted attacks.
    \end{itemize}
    \vspace{0.5cm}
    \item \textbf{[Medium] Establish a Formal Patch Management Program:}
    \begin{itemize}
        \item \textbf{Action:} Develop and implement a formal policy for regularly scanning, testing, and applying security patches to all systems and software.
        \item \textbf{Impact:} Proactively reduces the window of opportunity for attackers to exploit newly discovered vulnerabilities.
    \end{itemize}
\end{enumerate}

\end{document}
```