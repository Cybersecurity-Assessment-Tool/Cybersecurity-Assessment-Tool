```latex
\documentclass[12pt]{article}

% --- PACKAGES ---
\usepackage[margin=1in]{geometry}
\usepackage{pifont} % For checkmarks and crosses
\usepackage{booktabs} % For professional tables
\usepackage{hyperref} % For clickable links
\usepackage{url} % For URL formatting
\usepackage{seqsplit} % For splitting long strings in texttt
\usepackage{graphicx} % For logo (placeholder)
\usepackage{xcolor} % For colors

% --- DOCUMENT METADATA ---
\title{Cybersecurity Assessment Report}
\author{Cybersecurity Analysis Division}
\date{\today}

% --- HYPERREF SETUP ---
\hypersetup{
    colorlinks=true,
    linkcolor=blue,
    filecolor=magenta,      
    urlcolor=cyan,
    pdftitle={Cybersecurity Assessment Report},
    pdfpagemode=FullScreen,
}

% --- CUSTOM COMMANDS ---
\newcommand{\yes}{\ding{51}}
\newcommand{\no}{\ding{55}}

\begin{document}

\maketitle
\thispagestyle{empty}
\newpage

\tableofcontents
\newpage

% ===================================================================
% SECTION 1: EXECUTIVE SUMMARY
% ===================================================================
\section{Executive Summary}

This report provides a comprehensive cybersecurity assessment for \textbf{Hearth \& Home}, conducted on \today. The analysis synthesizes data from a network infrastructure scan, a review of existing risks, and an organizational security controls questionnaire.

The assessment reveals several critical and high-risk security gaps that require immediate attention. Key findings include the absence of Multi-Factor Authentication (MFA) for computer and sensitive data access, a lack of fundamental governance policies such as an Acceptable Use Policy, and insufficient ongoing security training for employees.

Furthermore, technical analysis confirmed an existing high-severity risk related to an exposed service on the local loopback interface (\texttt{127.0.0.1}), which corresponds to a pre-identified critical vulnerability. These combined findings indicate a security posture that is highly susceptible to compromise. This report outlines actionable recommendations to mitigate these identified risks and strengthen the organization's overall defensive capabilities.

% ===================================================================
% SECTION 2: ORGANIZATIONAL INFORMATION
% ===================================================================
\section{Organizational Information}

The following details were provided for the assessment scope. This information is used to correlate findings and understand the organizational context.

\begin{table}[h!]
\centering
\begin{tabular}{@{}ll@{}}
\toprule
\textbf{Attribute} & \textbf{Value} \\ \midrule
Organization Name  & Hearth \& Home \\
Email Domain       & \texttt{HearthHome.net} \\
Website Domain     & \url{www.HearthHome.net} \\
External IP Address & \texttt{205.150.145.118} \\ \bottomrule
\end{tabular}
\caption{Client Organizational Details}
\end{table}

% ===================================================================
% SECTION 3: SECURITY CONTROL REVIEW
% ===================================================================
\section{Security Control Review}

A security questionnaire was completed to evaluate the current state of administrative and procedural controls. The responses are summarized below. Answers marked with \no\ represent significant gaps in the security framework.

\begin{table}[h!]
\centering
\begin{tabular}{@{}lp{0.7\textwidth}c@{}}
\toprule
\textbf{ID} & \textbf{Control Question} & \textbf{Status} \\ \midrule
C1 & Do you require MFA to access email? & \yes \\
C2 & Do you require MFA to log into computers? & \no \\
C3 & Do you require MFA to access sensitive data systems? & \no \\
C4 & Does your organization have an employee acceptable use policy? & \no \\
C5 & Does your organization do security awareness training for new employees? & \yes \\
C6 & Does your organization do security awareness training for all employees at least once per year? & \no \\ \bottomrule
\end{tabular}
\caption{Security Controls Questionnaire Results}
\end{table}

\subsection{Analysis of Control Gaps}
The questionnaire reveals several critical deficiencies:
\begin{itemize}
    \item \textbf{Lack of Endpoint and Data MFA (C2, C3):} While MFA is enabled for email, its absence on computer logins and sensitive data systems exposes the organization to significant risk from credential theft and unauthorized access.
    \item \textbf{Missing Acceptable Use Policy (C4):} The lack of a formal AUP creates ambiguity for employees regarding the proper use of company assets and data, increasing the risk of insider threats and non-compliance.
    \item \textbf{Insufficient Security Training (C6):} Security training for new hires is a good first step, but the lack of mandatory annual training for all employees means that the workforce's ability to recognize and respond to evolving threats (like phishing) will degrade over time.
\end{itemize}

% ===================================================================
% SECTION 4: TECHNICAL SCAN RESULTS
% ===================================================================
\section{Technical Scan Results}

A network scan was performed to identify open ports and services on the target system. This scan provides insight into the technical attack surface.

\subsection{Nmap Scan: \texttt{127.0.0.1}}
The scan on the local loopback interface revealed the following:

\begin{table}[h!]
\centering
\begin{tabular}{@{}llll@{}}
\toprule
\textbf{Port} & \textbf{State} & \textbf{Service (Inferred)} & \textbf{Product/Version} \\ \midrule
22/tcp & open & SSH (Secure Shell) & Not Detected \\ \bottomrule
\end{tabular}
\caption{Open Ports on Target: \texttt{127.0.0.1}}
\end{table}

\paragraph{Analysis:} The presence of an open SSH port on the localhost interface is notable. This finding directly correlates with the pre-existing risk documented in Input 3, "Localhost Exposed". While typically used for local administration, an improperly configured SSH service can be a security risk. The inability to fingerprint the service version prevents automated checks for known vulnerabilities (CVEs) affecting the SSH daemon.

% ===================================================================
% SECTION 5: CONSOLIDATED RISK ASSESSMENT
% ===================================================================
\section{Consolidated Risk Assessment}

This section consolidates all identified risks from the security control review, technical scan, and pre-existing risk data. Each risk is assigned a severity level to guide prioritization.

\begin{table}[h!]
\centering
\begin{tabular}{@{}lp{0.5\textwidth}l@{}}
\toprule
\textbf{Risk Name} & \textbf{Description} & \textbf{Severity} \\ \midrule
Localhost Exposed & An SSH service is running and accessible on the local interface, posing a severe risk if misconfigured. & \textbf{Critical} \\
No MFA on Endpoints & Lack of MFA for computer logins allows for trivial lateral movement if credentials are compromised. & \textbf{Critical} \\
No MFA on Sensitive Data & Sensitive data systems are not protected by MFA, creating a high risk of data breach. & \textbf{Critical} \\
No Acceptable Use Policy & Absence of a formal AUP leads to inconsistent security practices and potential compliance violations. & High \\
Insufficient Annual Training & Lack of ongoing security awareness training increases susceptibility to social engineering attacks. & High \\ \bottomrule
\end{tabular}
\caption{Summary of Identified Risks}
\end{table}

% ===================================================================
% SECTION 6: RECOMMENDATIONS
% ===================================================================
\section{Recommendations}

The following actionable recommendations are provided to address the identified risks. They are prioritized based on severity.

\subsection{Critical Priority}
\begin{enumerate}
    \item \textbf{Remediate Exposed Service:} Immediately investigate the SSH service running on \texttt{127.0.0.1}.
    \begin{itemize}
        \item Determine the business justification for this service.
        \item If it is not required, disable and remove the service.
        \item If it is required, ensure it is hardened according to security best practices (e.g., disable root login, use key-based authentication) and firewalled to prevent any unintended access.
    \end{itemize}
    \item \textbf{Implement Comprehensive MFA:} Deploy a mandatory MFA solution across the organization.
    \begin{itemize}
        \item Prioritize enforcement for all computer/endpoint logins.
        \item Extend MFA enforcement to all systems containing sensitive or critical data, including databases, file shares, and administrative portals.
    \end{itemize}
\end{enumerate}

\subsection{High Priority}
\begin{enumerate}
    \setcounter{enumi}{2} % Continue numbering from previous list
    \item \textbf{Develop and Implement an Acceptable Use Policy (AUP):} Create a formal AUP that clearly defines the rules and expectations for employees when using company technology and data. This policy should be reviewed by legal counsel, communicated to all staff, and formally acknowledged by each employee.
    \item \textbf{Establish an Annual Security Training Program:} Institute a mandatory security awareness training program for all employees to be completed annually. The training should cover current threats such as phishing, ransomware, and social engineering.
\end{enumerate}

\end{document}
```