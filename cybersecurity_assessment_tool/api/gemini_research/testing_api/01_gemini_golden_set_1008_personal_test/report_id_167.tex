```latex
\documentclass[12pt, a4paper]{article}

% Preamble: Required Packages
\usepackage[margin=1in]{geometry}
\usepackage{pifont} % For checkmarks and crosses
\usepackage{booktabs} % For professional tables
\usepackage{hyperref} % For clickable links
\usepackage{url} % For URL formatting
\usepackage{seqsplit} % To split long text strings
\usepackage{graphicx}
\usepackage[table]{xcolor}
\usepackage{fancyhdr}

% --- Document Setup ---
\hypersetup{
    colorlinks=true,
    linkcolor=blue,
    filecolor=magenta,      
    urlcolor=cyan,
    pdftitle={Cybersecurity Assessment Report},
    pdfpagemode=FullScreen,
}

\pagestyle{fancy}
\fancyhf{}
\fancyhead[L]{Cybersecurity Assessment Report}
\fancyhead[R]{\textbf{Digital Drift}}
\fancyfoot[C]{\thepage}
\renewcommand{\headrulewidth}{0.4pt}
\renewcommand{\footrulewidth}{0.4pt}

% Define colors for severity
\definecolor{sev_critical}{HTML}{990000}
\definecolor{sev_high}{HTML}{DD2222}
\definecolor{sev_medium}{HTML}{FF8C00}
\definecolor{sev_low}{HTML}{F0E68C}

% Helper command for severity labels
\newcommand{\severitylabel}[2]{\colorbox{#1}{\textcolor{white}{\textbf{\strut #2}}}}

\begin{document}

% --- Title Page ---
\begin{titlepage}
    \centering
    \vspace*{1cm}
    \includegraphics[width=0.4\textwidth]{example-image-a} % Placeholder logo
    \vfill
    \huge\bfseries
    Cybersecurity Posture \& Risk Assessment Report
    \vspace{1cm}
    \Large\bfseries
    Prepared for: Digital Drift
    \vspace{2cm}
    \normalsize
    \begin{tabular}{ll}
        \textbf{Date of Report:} & \today \\
        \textbf{Scan Date:} & Not Specified in Scan Data \\
        \textbf{Report ID:} & CYBER-2023-001 \\
    \end{tabular}
    \vfill
    \small
    \textit{This report contains sensitive information and is intended solely for the recipient. Unauthorized distribution is prohibited.}
\end{titlepage}

\tableofcontents
\newpage

% --- Section 1: Executive Summary ---
\section{Executive Summary}
This report provides a comprehensive analysis of the cybersecurity posture for \textbf{Digital Drift}, based on a combination of network scanning, a security controls questionnaire, and a review of pre-existing risks.

The assessment reveals a mixed security posture. While the organization has implemented some key controls, such as Multi-Factor Authentication (MFA) for email and sensitive systems, several critical and high-risk gaps were identified that significantly increase the organization's exposure to cyber threats.

\textbf{Key Findings:}
\begin{itemize}
    \item \textbf{Critical Policy Gap:} The absence of a formal Employee Acceptable Use Policy (AUP) represents a fundamental gap in security governance.
    \item \textbf{High-Risk Endpoint Weakness:} The lack of mandatory MFA for computer logins leaves endpoints vulnerable to unauthorized access if user credentials are compromised.
    \item \textbf{Inadequate Training Cadence:} Security awareness training is not conducted annually for all employees, leading to a potential degradation of security consciousness over time.
    \item \textbf{Validated Technical Risk:} The network scan confirmed an open port (22/SSH) on a local interface, directly correlating with a pre-existing critical risk named "Localhost Exposed". This configuration is highly dangerous and requires immediate investigation.
\end{itemize}

The overall risk profile is elevated due to the combination of these policy, training, and technical vulnerabilities. Immediate remediation of the identified issues is strongly recommended to reduce the likelihood of a security incident.

\newpage

% --- Section 2: Organizational Information ---
\section{Organizational Information}
The following details were provided for the assessment.

\begin{tabular}{@{}ll}
    \toprule
    \textbf{Attribute} & \textbf{Value} \\
    \midrule
    Organization Name & \textbf{Digital Drift} \\
    Primary Email Domain & \seqsplit{\texttt{DigitalDrift.net}} \\
    Primary Website & \url{www.DigitalDrift.net} \\
    External IP Address & \seqsplit{\texttt{134.189.75.173}} \\
    \bottomrule
\end{tabular}

% --- Section 3: Security Control Review ---
\section{Security Control Review}
A review of the organization's security controls was conducted via a questionnaire. The results below highlight implemented controls and identify significant gaps. A "No" response indicates a missing control and a potential risk.

\begin{tabular}{@{}p{0.5\linewidth}ccc}
    \toprule
    \textbf{Control Question} & \textbf{Response} & \textbf{Analyst Note} \\
    \midrule
    \multicolumn{3}{l}{\textit{\textbf{Access Control}}} \\
    \cmidrule(r){1-3}
    Do you require MFA to access email? & \ding{51} Yes & Good Practice \\
    Do you require MFA to log into computers? & \colorbox{sev_high!20}{\ding{55} No} & \textbf{High Risk} \\
    Do you require MFA to access sensitive data systems? & \ding{51} Yes & Good Practice \\
    \midrule
    \multicolumn{3}{l}{\textit{\textbf{Policy \& Governance}}} \\
    \cmidrule(r){1-3}
    Does your organization have an employee acceptable use policy? & \colorbox{sev_critical!20}{\ding{55} No} & \textbf{Critical Gap} \\
    \midrule
    \multicolumn{3}{l}{\textit{\textbf{Security Awareness \& Training}}} \\
    \cmidrule(r){1-3}
    Does your organization do security awareness training for new employees? & \ding{51} Yes & Good Practice \\
    Does your organization do security awareness training for all employees at least once per year? & \colorbox{sev_high!20}{\ding{55} No} & \textbf{High Risk} \\
    \bottomrule
\end{tabular}

\newpage

% --- Section 4: Technical Scan Results ---
\section{Technical Scan Results}
An external network scan was performed to identify open ports and exposed services on the target system.

\subsection{Scan Summary}
\begin{itemize}
    \item \textbf{Target IP:} \seqsplit{\texttt{127.0.0.1}}
    \item \textbf{Host Status:} Up
\end{itemize}

\subsection{Open Ports Discovered}
The following table details the ports found to be open and accessible.

\begin{tabular}{@{}lllll}
    \toprule
    \textbf{Port} & \textbf{State} & \textbf{Service} & \textbf{Product/Version} & \textbf{Analyst Note} \\
    \midrule
    22/tcp & open & ssh (inferred) & Not available & Correlates with "Localhost Exposed" risk. \\
    \bottomrule
\end{tabular}

\subsection{Technical Analysis}
The scan identified port 22 (SSH - Secure Shell) as open on the localhost interface (\texttt{127.0.0.1}). While typically used for secure remote administration, its exposure on this interface is highly unusual and directly validates the pre-existing critical risk. This could indicate a misconfigured service that is unintentionally exposed. An attacker who gains a foothold on the system could potentially exploit this service to escalate privileges or pivot to other systems. The lack of version information from the scan prevents a detailed vulnerability assessment, but the exposure itself is a significant finding.

% --- Section 5: Consolidated Risk Assessment ---
\section{Consolidated Risk Assessment}
The following table synthesizes findings from the security control review, technical scan, and pre-existing risk data into a consolidated list of identified risks.

\begin{tabular}{@{}p{0.2\linewidth}p{0.15\linewidth}p{0.6\linewidth}}
    \toprule
    \textbf{Risk Name} & \textbf{Severity} & \textbf{Description \& Affected Elements} \\
    \midrule
    \textbf{Localhost Exposed} & \severitylabel{sev_critical}{Critical} & A network service (SSH on port 22) is exposed on the localhost interface, validating a known critical risk. This could be exploited for privilege escalation. \newline \textit{Affected: \texttt{127.0.0.1}} \\
    \addlinespace
    \textbf{Missing Acceptable Use Policy} & \severitylabel{sev_critical}{Critical} & The absence of a formal AUP creates significant legal and operational risk. There are no clear, enforceable guidelines for employees regarding the secure use of company assets. \newline \textit{Affected: Organization-wide} \\
    \addlinespace
    \textbf{Lack of Endpoint MFA} & \severitylabel{sev_high}{High} & The absence of MFA on computer logins means that a single compromised password could grant an attacker full access to an employee's workstation and any data it can access. \newline \textit{Affected: All employee endpoints} \\
    \addlinespace
    \textbf{Inadequate Security Training Cadence} & \severitylabel{sev_high}{High} & Without mandatory annual training, employees' security awareness will diminish, making them more susceptible to phishing, social engineering, and other common attack vectors. \newline \textit{Affected: All employees} \\
    \bottomrule
\end{tabular}

\newpage

% --- Section 6: Recommendations ---
\section{Recommendations}
The following actionable recommendations are provided to mitigate the identified risks and improve the overall security posture of \textbf{Digital Drift}.

\subsection{Immediate Actions (To Be Completed in < 7 Days)}
\begin{description}
    \item[Remediate Exposed Service:] \hfill \\
    Investigate the SSH service running on port 22 of \texttt{127.0.0.1}. Determine why it is exposed and if it is necessary. If not, disable the service. If required, ensure it is properly configured and firewalled to prevent unauthorized access.
\end{description}

\subsection{Strategic Recommendations (To Be Completed in 1-3 Months)}
\begin{description}
    \item[Develop and Implement an Acceptable Use Policy (AUP):] \hfill \\
    \textbf{Risk Mitigated:} Missing Acceptable Use Policy (Critical) \\
    Draft, approve, and implement a formal AUP that clearly defines the rules and responsibilities for all employees when using company IT assets. Require all employees to read and acknowledge the policy.

    \item[Deploy Multi-Factor Authentication (MFA) for Endpoints:] \hfill \\
    \textbf{Risk Mitigated:} Lack of Endpoint MFA (High) \\
    Implement an MFA solution for all employee computer and laptop logins. This adds a critical layer of security that protects against credential theft.

    \item[Establish Annual Security Awareness Training:] \hfill \\
    \textbf{Risk Mitigated:} Inadequate Security Training Cadence (High) \\
    Institute a mandatory security awareness training program to be completed by all employees on an annual basis. This program should cover current threats such as phishing, malware, and secure data handling.
\end{description}

\end{document}
```