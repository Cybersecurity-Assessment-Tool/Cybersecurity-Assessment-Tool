```latex
\documentclass[12pt]{article}

% Preamble: Required Packages
\usepackage[margin=1in]{geometry}
\usepackage{pifont} % For checkmarks and crosses
\usepackage{booktabs} % For professional tables
\usepackage{hyperref} % For clickable links
\usepackage{url}      % For formatting URLs
\usepackage{seqsplit} % For splitting long strings in texttt

% Document Metadata
\title{Cybersecurity Assessment Report}
\author{Cybersecurity Analysis Division}
\date{\today}

\begin{document}

\maketitle
\tableofcontents
\newpage

% ===================================================================
% SECTION 1: EXECUTIVE OVERVIEW
% ===================================================================
\section{Executive Overview}

This report details the findings of a cybersecurity assessment conducted for \textbf{Sterling Silver}. The assessment combined a technical network scan, a review of organizational security controls via a questionnaire, and an analysis of pre-existing risks.

The analysis revealed several critical and high-severity risks that expose the organization to significant threats. The most critical finding is an externally facing FTP server (\texttt{10.0.0.15}) running a dangerously outdated and vulnerable version of \texttt{vsftpd} (2.3.4), which is known to contain a backdoor. This service is further misconfigured to allow anonymous access, presenting an immediate and severe risk of system compromise and data breach.

Furthermore, significant gaps were identified in administrative controls. The lack of Multi-Factor Authentication (MFA) for computer and sensitive data access, coupled with the absence of an employee Acceptable Use Policy and security training for new hires, indicates a weak foundational security posture. These deficiencies dramatically increase the likelihood of a successful cyberattack through credential theft or insider threat.

Immediate remediation of the vulnerable FTP server is paramount. Concurrently, the organization must prioritize the implementation of MFA and the development of core security policies and training programs to mitigate these risks effectively.

% ===================================================================
% SECTION 2: ORGANIZATIONAL INFORMATION
% ===================================================================
\section{Organizational Information}

The following information was provided for the assessment.

\begin{tabular}{@{}ll}
\toprule
\textbf{Attribute} & \textbf{Value} \\
\midrule
Organization Name & \textbf{Sterling Silver} \\
Email Domain      & \texttt{SterlingSilver.org} \\
Website Domain    & \url{www.SterlingSilver.org} \\
External IP Address & \texttt{191.73.13.235} \\
\bottomrule
\end{tabular}

% ===================================================================
% SECTION 3: SECURITY CONTROL REVIEW
% ===================================================================
\section{Security Control Review}

The following table summarizes the organization's responses to the security controls questionnaire. Items marked with \ding{55} represent significant gaps in the security posture and are discussed in the risk assessment section.

\begin{table}[h!]
\centering
\begin{tabular}{@{}p{0.5\linewidth}cp{0.3\linewidth}@{}}
\toprule
\textbf{Control Question} & \textbf{Response} & \textbf{Analyst Notes} \\
\midrule
Do you require MFA to access email? & \ding{51} & Good control. Protects a primary communication channel. \\
\addlinespace
Do you require MFA to log into computers? & \ding{55} & \textbf{Critical Gap.} Lack of MFA on endpoints allows for easy lateral movement if credentials are stolen. \\
\addlinespace
Do you require MFA to access sensitive data systems? & \ding{55} & \textbf{Critical Gap.} High-value data is not adequately protected from unauthorized access. \\
\addlinespace
Does your organization have an employee acceptable use policy? & \ding{55} & \textbf{High Risk.} No formal guidelines for employees on the acceptable use of company assets. \\
\addlinespace
Does your organization do security awareness training for new employees? & \ding{55} & \textbf{High Risk.} New hires are a common target and are not being equipped with necessary security knowledge. \\
\addlinespace
Does your organization do security awareness training for all employees at least once per year? & \ding{51} & Good practice for maintaining security awareness. \\
\bottomrule
\end{tabular}
\caption{Security Controls Questionnaire Analysis}
\end{table}

% ===================================================================
% SECTION 4: TECHNICAL SCAN RESULTS
% ===================================================================
\section{Technical Scan Results}

A network scan was performed to identify open ports and exposed services.

\subsection{Target: \texttt{10.0.0.15}}
The scan on this host revealed the following critical vulnerability:

\begin{table}[h!]
\centering
\begin{tabular}{@{}lllll@{}}
\toprule
\textbf{Port} & \textbf{State} & \textbf{Service} & \textbf{Version} & \textbf{Details} \\
\midrule
21/tcp & Open & ftp & vsftpd 2.3.4 & \textbf{Critical Finding:} Anonymous FTP login is allowed. \\
 & & & & This version is known to be vulnerable to a backdoor \\
 & & & & exploit (\textbf{CVE-2011-2523}). \\
\bottomrule
\end{tabular}
\caption{Open Ports and Services on \texttt{10.0.0.15}}
\end{table}

The combination of anonymous access and a known vulnerable service version on port 21 presents an immediate and severe threat. An attacker could leverage this to gain unauthorized access to the server, exfiltrate data, or use it as a pivot point to attack the internal network.

% ===================================================================
% SECTION 5: CONSOLIDATED RISK ASSESSMENT
% ===================================================================
\section{Consolidated Risk Assessment}

The following table synthesizes findings from the technical scan, control review, and pre-existing risk register into a prioritized list.

\begin{table}[h!]
\centering
\begin{tabular}{@{}lp{0.5\linewidth}ll@{}}
\toprule
\textbf{Risk ID} & \textbf{Risk Description} & \textbf{Severity} & \textbf{Source} \\
\midrule
RISK-001 & A publicly accessible FTP server is running a vulnerable version (\texttt{vsftpd 2.3.4}) and allows anonymous login. & \textbf{Critical} & Technical Scan \\
\addlinespace
RISK-002 & Lack of Multi-Factor Authentication (MFA) on employee computers and sensitive data systems. & \textbf{Critical} & Questionnaire \\
\addlinespace
RISK-003 & Absence of foundational security policies (Acceptable Use) and mandatory training for new employees. & \textbf{High} & Questionnaire \\
\addlinespace
RISK-004 & Workstations are running an outdated and unsupported operating system (Windows 7). & Medium & Pre-existing Risk \\
\bottomrule
\end{tabular}
\caption{Summary of Identified Risks}
\end{table}

% ===================================================================
% SECTION 6: RECOMMENDATIONS
% ===================================================================
\section{Recommendations}

The following actions are recommended to mitigate the identified risks.

\subsection{RISK-001: Remediate Vulnerable FTP Server (Immediate)}
\begin{itemize}
    \item \textbf{Immediate Action:} Disable the FTP service on \texttt{10.0.0.15} immediately to remove the threat.
    \item \textbf{Short-Term:} Investigate the business need for this service. If it is not required, decommission the server permanently.
    \item \textbf{Long-Term:} If a file transfer service is required, replace FTP with a secure alternative such as SFTP (SSH File Transfer Protocol) or FTPS (FTP over SSL/TLS). Ensure any new service is configured to require strong authentication.
\end{itemize}

\subsection{RISK-002: Implement Multi-Factor Authentication (High Priority)}
\begin{itemize}
    \item \textbf{Action:} Procure and deploy an MFA solution across the organization.
    \item \textbf{Priority 1:} Enforce MFA for all logins to employee computers and workstations.
    \item \textbf{Priority 2:} Enforce MFA for all access to systems containing sensitive or critical data.
\end{itemize}

\subsection{RISK-003: Establish Foundational Policies \& Training (High Priority)}
\begin{itemize}
    \item \textbf{Action 1:} Develop and implement a formal Acceptable Use Policy (AUP) that all employees must read and acknowledge.
    \item \textbf{Action 2:} Integrate mandatory cybersecurity awareness training into the new employee onboarding process. This training should cover topics such as phishing, password security, and the new AUP.
\end{itemize}

\subsection{RISK-004: Upgrade Outdated Workstations (Medium Priority)}
\begin{itemize}
    \item \textbf{Action:} Continue with the existing plan to upgrade all workstations from Windows 7 to a modern, supported operating system (e.g., Windows 10/11). This mitigates risks associated with unpatched vulnerabilities inherent in end-of-life software.
\end{itemize}

\end{document}
```