```latex
\documentclass[12pt]{article}

% Required Packages
\usepackage[margin=1in]{geometry}
\usepackage{pifont} % For checkmarks and crosses
\usepackage{booktabs} % For professional tables
\usepackage{hyperref} % For hyperlinks
\usepackage{url} % For URL formatting
\usepackage{seqsplit} % For splitting long text strings

% Document Metadata
\title{Cybersecurity Posture Assessment Report}
\author{Cybersecurity Analysis Division}
\date{\today}

\begin{document}

\maketitle

\begin{abstract}
    This report provides a comprehensive analysis of the cybersecurity posture for Solaris Energy. The assessment is based on a synthesis of network scan data, organizational security control questionnaires, and a review of pre-existing risks. The findings highlight critical gaps in security controls and provide actionable recommendations to mitigate identified vulnerabilities and enhance the overall security framework.
\end{abstract}

\section*{1. Executive Overview}

The assessment of Solaris Energy reveals a mixed security posture with notable strengths and critical weaknesses. The technical network scan of the internal host \texttt{192.168.1.100} indicated a secure configuration, with no open ports detected. This suggests effective network segmentation or host-level firewalling.

However, significant risks were identified in the organizational security controls. The absence of Multi-Factor Authentication (MFA) for email and computer access represents a critical vulnerability, exposing the organization to account compromise and unauthorized access. Furthermore, the lack of a formal Acceptable Use Policy and mandatory annual security awareness training for all staff increases susceptibility to insider threats and social engineering attacks.

Immediate remediation should focus on implementing MFA across all critical systems, developing foundational security policies, and establishing a continuous security training program.

\section*{2. Organizational Information}

The following details were provided for the assessment. This information is used to establish the context for the technical and procedural analysis.

\begin{itemize}
    \item \textbf{Organization Name:} Solaris Energy
    \item \textbf{Email Domain:} \texttt{SolarisEnergy.net}
    \item \textbf{External IP Address:} \texttt{194.126.129.235}
\end{itemize}

\section*{3. Security Control Review}

A review of the organization's security controls was conducted via a questionnaire. The responses are compared against industry best practices to identify procedural gaps. Answers marked with \ding{55} (No) indicate a deviation from best practices and represent a significant risk.

\begin{table}[h!]
\centering
\caption{Organizational Security Control Assessment}
\begin{tabular}{@{}lccc@{}}
\toprule
\textbf{Control Question} & \textbf{Response} & \textbf{Assessment} \\
\midrule
Do you require MFA to access email? & \ding{55} & \textbf{Critical Gap} \\
Do you require MFA to log into computers? & \ding{55} & \textbf{High Risk} \\
Do you require MFA to access sensitive data systems? & \ding{51} & Meets Best Practice \\
Does your organization have an employee acceptable use policy? & \ding{55} & \textbf{High Risk} \\
Does your organization do security awareness training for new employees? & \ding{51} & Meets Best Practice \\
Does your organization do security awareness training for all employees annually? & \ding{55} & \textbf{High Risk} \\
\bottomrule
\end{tabular}
\end{table}

\section*{4. Technical Scan Results}

A network scan was performed to identify active services and potential vulnerabilities on the specified target system.

\begin{itemize}
    \item \textbf{Target IP Address:} \texttt{192.168.1.100}
    \item \textbf{Scan Date:} \today
    \item \textbf{Host Status:} Up
\end{itemize}

\subsection*{Scan Summary}
The scan results indicate that the host at \texttt{192.168.1.100} is online but has no detectable open ports. All 1000 scanned TCP ports were found to be in a `closed` state. This is a positive security finding, as it drastically reduces the attack surface of the target machine. A closed port responds to probes but has no application listening on it, indicating it is not available for communication.

\section*{5. Consolidated Risk Assessment}

This section correlates findings from the security control review and technical scans. As no pre-existing risks were documented and the technical scan found no open ports, all identified risks are derived from the procedural gaps identified in Section 3.

\begin{table}[h!]
\centering
\caption{Identified Risks and Severity}
\begin{tabular}{@{}p{0.3\linewidth}p{0.5\linewidth}l@{}}
\toprule
\textbf{Risk Name} & \textbf{Overview} & \textbf{Severity} \\
\midrule
\textbf{Email Account Compromise} & The lack of MFA on email accounts makes them highly vulnerable to phishing and credential stuffing attacks, which can lead to data breaches and further internal compromise. & \textbf{Critical} \\
\addlinespace
\textbf{Unauthorized Workstation Access} & The absence of MFA for computer logins increases the risk of unauthorized access if user credentials are stolen or guessed. This could lead to data theft or malware deployment. & \textbf{High} \\
\addlinespace
\textbf{Lack of Acceptable Use Policy (AUP)} & Without a formal AUP, employees may be unaware of security responsibilities and prohibited activities, increasing the likelihood of insider threats and unintentional policy violations. & \textbf{High} \\
\addlinespace
\textbf{Degraded Security Awareness} & Without annual refresher training, employees' ability to recognize and respond to evolving threats like phishing and social engineering diminishes over time. & \textbf{High} \\
\bottomrule
\end{tabular}
\end{table}

\section*{6. Recommendations}

Based on the consolidated risk assessment, the following actions are recommended to strengthen the security posture of Solaris Energy. Recommendations are prioritized based on risk severity.

\begin{description}
    \item[\textbf{1. (Critical) Implement MFA for Email:}] Immediately enforce MFA for all user access to the email system (\texttt{SolarisEnergy.net}). This is the single most effective control to prevent unauthorized account access.
    
    \item[\textbf{2. (High) Deploy MFA for Endpoint Access:}] Roll out MFA for all employee computer and laptop logins. This adds a critical layer of defense against credential theft and protects data stored on local machines.
    
    \item[\textbf{3. (High) Develop and Implement an AUP:}] Create a comprehensive Acceptable Use Policy that clearly defines rules for using company assets, data handling, and internet usage. Require all employees to read and acknowledge the policy.
    
    \item[\textbf{4. (High) Establish Annual Security Training:}] Institute a mandatory security awareness training program for all employees to be completed annually. The training should cover current threats, such as phishing, malware, and social engineering, as well as company security policies.
\end{description}

\end{document}
```