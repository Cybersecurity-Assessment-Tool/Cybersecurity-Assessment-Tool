```latex
\documentclass[12pt]{article}

% Preamble: Required Packages
\usepackage[a4paper, margin=1in]{geometry}
\usepackage{pifont} % For checkmarks and crosses (\ding)
\usepackage{booktabs} % For professional-looking tables (\toprule, \midrule, \bottomrule)
\usepackage{hyperref} % For hyperlinks and PDF metadata
\usepackage{url} % For formatting URLs
\usepackage{seqsplit} % To break long strings like hashes or URLs

% --- Document Metadata ---
\hypersetup{
    colorlinks=true,
    linkcolor=black,
    urlcolor=blue,
    pdftitle={Cybersecurity Posture Assessment Report},
    pdfauthor={Cybersecurity Analyst},
    pdfsubject={Security Analysis for Midnight Oil Studios},
    pdfkeywords={Cybersecurity, Risk, Assessment, Nmap, Policy}
}

% --- Document Start ---
\begin{document}

% --- Title Page ---
\title{Cybersecurity Posture Assessment Report\\ \large For: Midnight Oil Studios}
\author{Generated by Expert Cybersecurity Analyst}
\date{\today}
\maketitle
\thispagestyle{empty}
\newpage

% --- Table of Contents ---
\tableofcontents
\newpage

% --- Section 1: Executive Summary ---
\section*{Executive Summary}

This report provides a comprehensive cybersecurity posture assessment for \textbf{Midnight Oil Studios}, based on a synthesis of network scan data, organizational security controls, and known risks. The analysis was conducted on \today.

The assessment reveals several critical and high-risk security gaps originating from policy and procedural deficiencies rather than technical vulnerabilities on the external perimeter. The most pressing concerns are the complete absence of Multi-Factor Authentication (MFA) for email, computer logins, and sensitive data access. This exposes the organization to a significant risk of account compromise, which is a primary vector for ransomware and data breach incidents.

Furthermore, the lack of a formal security awareness training program for employees presents a high risk, as staff are likely unprepared to identify and resist phishing or social engineering attacks.

The external network scan of the designated target IP address did not identify any open ports. While this may indicate a well-configured firewall, it should be verified, as the lack of findings can sometimes result from scan configuration or network filtering issues.

Immediate remediation should focus on implementing MFA across all critical systems and establishing a mandatory security awareness training program.

% --- Section 2: Organizational Information ---
\section*{Organizational Information}

The following details were provided for the assessment. This information is used to establish the context and scope of the review.

\begin{table}[h!]
\centering
\begin{tabular}{@{}ll@{}}
\toprule
\textbf{Attribute} & \textbf{Value} \\ \midrule
Organization Name & Midnight Oil Studios \\
Email Domain & \texttt{MidnightOilStudios.com} \\
Website Domain & \url{www.MidnightOilStudios.com} \\
External IP Address & \texttt{60.45.0.201} \\ \bottomrule
\end{tabular}
\caption{Client Organizational Details.}
\end{table}

% --- Section 3: Security Control Review ---
\section*{Security Control Review}

A review of the organization's security controls was conducted via a questionnaire. The responses highlight significant gaps in fundamental security practices. A "No" response indicates a missing control and a potential security risk.

\begin{table}[h!]
\centering
\begin{tabular}{@{}p{0.5\textwidth} c p{0.3\textwidth}@{}}
\toprule
\textbf{Control Question} & \textbf{Response} & \textbf{Analyst's Note} \\ \midrule
Do you require MFA to access email? & \ding{55} & \textbf{Critical Risk.} Email is a primary target for attackers. Lack of MFA makes account takeover trivial. \\
\addlinespace
Do you require MFA to log into computers? & \ding{55} & \textbf{Critical Risk.} Compromised credentials could lead to direct endpoint and network access. \\
\addlinespace
Do you require MFA to access sensitive data systems? & \ding{55} & \textbf{Critical Risk.} The organization's most valuable data is not adequately protected from unauthorized access. \\
\addlinespace
Does your organization have an employee acceptable use policy? & \ding{51} & A foundational policy is in place, which is a positive control. \\
\addlinespace
Does your organization do security awareness training for new employees? & \ding{55} & \textbf{High Risk.} New staff are not equipped to handle security threats, making them prime targets for attackers. \\
\addlinespace
Does your organization do security awareness training for all employees at least once per year? & \ding{55} & \textbf{High Risk.} Without regular training, employees are more susceptible to phishing and social engineering. \\ \bottomrule
\end{tabular}
\caption{Security Controls Questionnaire Analysis. (\ding{51} = Yes, \ding{55} = No)}
\end{label{tab:controls}
\end{table}

% --- Section 4: Technical Scan Results ---
\section*{Technical Scan Results}

An external network vulnerability scan was performed to identify open ports and services exposed to the internet.

\begin{itemize}
    \item \textbf{Scan Target:} \texttt{[Target IP]}
    \item \textbf{Scan Date:} \today
\end{itemize}

\subsection*{Findings}
The scan completed successfully but did not detect any open TCP or UDP ports on the target host.

\subsection*{Analysis}
No exposed services were identified. This is often a positive sign, suggesting that a firewall is in place and properly configured to block unsolicited inbound traffic (a "default deny" posture). However, this result could also be due to ICMP (ping) blocking or other network filtering that prevented the scanner from accurately assessing the target. Further internal and authenticated scans are recommended to validate this finding and assess the internal network posture.

% --- Section 5: Consolidated Risk Assessment ---
\section*{Consolidated Risk Assessment}

The following table synthesizes findings from the security control review and technical analysis into a prioritized list of risks. No pre-existing vulnerabilities were reported.

\begin{table}[h!]
\centering
\begin{tabular}{@{}lp{0.55\textwidth}l@{}}
\toprule
\textbf{Risk Name} & \textbf{Description} & \textbf{Severity} \\ \midrule
\textbf{Lack of Multi-Factor Authentication (MFA)} & The absence of MFA for email, endpoints, and sensitive systems creates a severe risk of unauthorized access via stolen or weak credentials. This is a common precursor to major data breaches and ransomware attacks. & \textbf{Critical} \\
\addlinespace
\textbf{Inadequate Security Awareness Training} & Employees are not trained to recognize or respond to cyber threats like phishing, malware, or social engineering. This makes the organization highly vulnerable to human-targeted attacks. & \textbf{High} \\
\bottomrule
\end{tabular}
\caption{Summary of Identified Risks.}
\end{label{tab:risks}
\end{table}

% --- Section 6: Recommendations ---
\section*{Recommendations}

Based on the identified risks, the following prioritized actions are recommended to improve the cybersecurity posture of \textbf{Midnight Oil Studios}.

\begin{enumerate}
    \item \textbf{[Critical] Implement Multi-Factor Authentication (MFA):}
    \begin{itemize}
        \item \textbf{Priority 1a:} Immediately enable MFA for all user accounts on the email platform (\texttt{MidnightOilStudios.com}).
        \item \textbf{Priority 1b:} Deploy MFA for all remote access solutions (e.g., VPN) and administrative access to servers and network devices.
        \item \textbf{Priority 1c:} Enforce MFA for logging into all company computers and accessing systems that store sensitive or critical data.
    \end{itemize}
    \vspace{0.5cm}
    \item \textbf{[High] Establish a Security Awareness Training Program:}
    \begin{itemize}
        \item \textbf{Priority 2a:} Procure and deploy a security awareness training solution.
        \item \textbf{Priority 2b:} Make training mandatory for all new employees as part of their onboarding process.
        \item \textbf{Priority 2c:} Ensure all current employees complete the training annually and conduct periodic phishing simulation campaigns to reinforce learning.
    \end{itemize}
    \vspace{0.5cm}
    \item \textbf{[Medium] Validate Network Security Posture:}
    \begin{itemize}
        \item \textbf{Priority 3a:} Conduct a follow-up external scan with different parameters to confirm that no ports are exposed.
        \item \textbf{Priority 3b:} Schedule an internal vulnerability scan to identify potential risks within the network perimeter that are not visible from the internet.
    \end{itemize}
\end{enumerate}

% --- Document End ---
\end{document}
```