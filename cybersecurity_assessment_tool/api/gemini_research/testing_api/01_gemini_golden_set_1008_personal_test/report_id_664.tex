```latex
\documentclass[12pt, a4paper]{article}

% Preamble: Required Packages
\usepackage[margin=1in]{geometry}
\usepackage{pifont} % For checkmarks and crosses
\usepackage{booktabs} % For professional tables
\usepackage{hyperref} % For clickable links
\usepackage{url} % For URL formatting
\usepackage{seqsplit} % To split long strings in texttt
\usepackage{graphicx}
\usepackage[table]{xcolor}
\usepackage{datetime}

% --- Document Metadata ---
\hypersetup{
    colorlinks=true,
    linkcolor=blue,
    filecolor=magenta,      
    urlcolor=cyan,
    pdftitle={Cybersecurity Posture Report},
    pdfauthor={Cybersecurity Analyst},
    pdfsubject={Security Assessment},
    pdfkeywords={Cybersecurity, Risk, Nmap, Assessment},
    pdftoolbar=true,
}

% --- Custom Commands & Settings ---
\newcommand{\yes}{\ding{51}}
\newcommand{\no}{\ding{55}}
\definecolor{lightgray}{gray}{0.9}
\rowcolors{2}{lightgray}{white}

% ==============================================================================
% --- BEGIN DOCUMENT ---
% ==============================================================================
\begin{document}

% --- Title Page ---
\begin{titlepage}
    \centering
    \vspace*{1cm}
    \Huge
    \textbf{Cybersecurity Posture Report}
    
    \vspace{1.5cm}
    \Large
    Prepared for: \\
    \vspace{0.5cm}
    \textbf{Borealis Tech}
    
    \vfill
    
    \large
    Report Date: \today \\
    Analysis Period: \today
    
    \vspace{1.5cm}
    \textit{This document contains sensitive information and is intended for the exclusive use of the recipient.}
    
\end{titlepage}

\tableofcontents
\newpage

% ==============================================================================
% --- 1. Executive Summary ---
% ==============================================================================
\section{Executive Summary}

This report provides a comprehensive analysis of the cybersecurity posture for \textbf{Borealis Tech}. The assessment is based on a correlation of organizational data, a review of existing security controls, and a technical network scan.

The overall security posture presents a mixed landscape. On a positive note, the technical scan of the target host \texttt{192.168.1.100} revealed a strong network configuration, with no open ports detected. This indicates effective firewalling and a minimal attack surface for that specific asset.

However, significant and critical gaps were identified in the organization's security controls. The two most pressing issues are:
\begin{itemize}
    \item \textbf{Lack of Multi-Factor Authentication (MFA) for Email:} This is a critical vulnerability. Email is a primary target for attackers, and the absence of MFA greatly increases the risk of Business Email Compromise (BEC), account takeovers, and subsequent data breaches.
    \item \textbf{No Security Awareness Training for New Employees:} New hires are a high-value target for social engineering attacks. Failing to provide immediate security training during onboarding leaves the organization vulnerable to phishing, malware, and unintentional policy violations.
\end{itemize}

While other controls, such as MFA for computer and sensitive system access, are in place, the identified gaps require immediate attention. This report outlines specific, actionable recommendations to mitigate these risks and strengthen the organization's overall defense-in-depth strategy.

% ==============================================================================
% --- 2. Organizational Information ---
% ==============================================================================
\section{Organizational Information}

The following details were provided for the assessment. This information is used to establish the context and scope of the analysis.

\begin{table}[h!]
\centering
\caption{Client Organizational Data}
\label{tab:org_data}
\begin{tabular}{@{}ll@{}}
\toprule
\textbf{Attribute} & \textbf{Value} \\ \midrule
Organization Name    & \textbf{Borealis Tech} \\
Email Domain         & \texttt{BorealisTech.org} \\
Website Domain       & \url{www.BorealisTech.org} \\
External IP Address  & \texttt{175.233.140.214} \\ \bottomrule
\end{tabular}
\end{table}

% ==============================================================================
% --- 3. Security Control Review ---
% ==============================================================================
\section{Security Control Review}

A review of organizational security controls was conducted based on a standardized questionnaire. The results highlight key areas of strength and weakness in current security policies and practices. "No" answers indicate significant gaps that increase organizational risk.

\begin{table}[h!]
\centering
\caption{Security Controls Questionnaire Analysis}
\label{tab:controls}
\renewcommand{\arraystretch}{1.3}
\begin{tabular}{@{}p{0.6\textwidth}ccp{0.2\textwidth}@{}}
\toprule
\textbf{Control Question} & \textbf{Status} & \textbf{Risk Level} \\ \midrule
Does your organization have an employee acceptable use policy? & \yes & Low \\
Do you require MFA to log into computers? & \yes & Low \\
Do you require MFA to access sensitive data systems? & \yes & Low \\
Does your organization do security awareness training for all employees at least once per year? & \yes & Low \\
\rowcolor{red!20} Do you require MFA to access email? & \no & \textbf{Critical} \\
\rowcolor{orange!20} Does your organization do security awareness training for new employees? & \no & \textbf{High} \\
\bottomrule
\end{tabular}
\end{table}

% ==============================================================================
% --- 4. Technical Scan Results ---
% ==============================================================================
\section{Technical Scan Results}

A network scan was performed to identify open ports and exposed services on the specified target system.

\subsection{Scan Summary}
The scan against the target IP address revealed no open ports. All 1000 scanned ports were found to be in a 'closed' state. This is a positive security finding, as it indicates that the host has a minimal network attack surface and is likely protected by a well-configured firewall.

\begin{table}[h!]
\centering
\caption{Nmap Scan Metadata}
\label{tab:nmap_meta}
\begin{tabular}{@{}ll@{}}
\toprule
\textbf{Attribute} & \textbf{Value} \\ \midrule
Target IP Address    & \texttt{192.168.1.100} \\
Scan Date            & \today \\
Host Status          & Up \\
Open Ports Found     & 0 \\
Filtered Ports Found & 0 \\
Other Ports State    & Closed \\ \bottomrule
\end{tabular}
\end{table}

\subsection{Detailed Findings}
No vulnerabilities or misconfigurations were identified from the network scan, as no services were exposed.

% ==============================================================================
% --- 5. Risk Assessment Summary ---
% ==============================================================================
\section{Risk Assessment Summary}

This section synthesizes findings from the security control review and technical scans. The following table details the most significant risks identified during this assessment.

\begin{table}[h!]
\centering
\caption{Identified Risks and Severity}
\label{tab:risks}
\renewcommand{\arraystretch}{1.4}
\begin{tabular}{@{}p{0.05\textwidth}p{0.3\textwidth}p{0.15\textwidth}p{0.4\textwidth}@{}}
\toprule
\textbf{ID} & \textbf{Risk Name} & \textbf{Severity} & \textbf{Overview} \\ \midrule
\rowcolor{red!20}
RISK-001 & Lack of MFA for Email Access & \textbf{Critical} & The absence of MFA on email accounts exposes the organization to a high likelihood of account compromise through phishing or credential stuffing. A compromised email account can lead to data exfiltration, financial fraud, and further internal network compromise. \\
\rowcolor{orange!20}
RISK-002 & Inadequate New Employee Security Onboarding & \textbf{High} & New employees are not receiving security awareness training upon being hired. This makes them highly susceptible to social engineering and phishing attacks, as they are unfamiliar with corporate security policies and common threats targeting the organization. \\
\bottomrule
\end{tabular}
\end{table}

% ==============================================================================
% --- 6. Recommendations ---
% ==============================================================================
\section{Recommendations}

The following actions are recommended to mitigate the identified risks and improve the overall security posture of \textbf{Borealis Tech}. Recommendations are prioritized based on the severity of the associated risk.

\subsection{RISK-001: Implement MFA for Email (Critical)}
\begin{itemize}
    \item \textbf{Action:} Immediately enable and enforce Multi-Factor Authentication (MFA) for all user accounts with access to the corporate email system (\texttt{BorealisTech.org}).
    \item \textbf{Justification:} MFA is one of the most effective controls for preventing unauthorized account access. It provides a critical layer of defense against password-based attacks, significantly reducing the risk of Business Email Compromise (BEC) and related data breaches.
    \item \textbf{Implementation:} Prioritize deployment for privileged accounts (administrators, executives) and then roll out to all employees. Acceptable MFA factors include authenticator apps (e.g., Google Authenticator, Microsoft Authenticator) or hardware tokens (e.g., YubiKey).
\end{itemize}

\subsection{RISK-002: Integrate Security Training into Onboarding (High)}
\begin{itemize}
    \item \textbf{Action:} Develop and integrate a mandatory security awareness training module into the new employee onboarding process.
    \item \textbf{Justification:} Educating employees on security best practices from day one is essential for building a strong security culture. This training should equip them to recognize and report threats like phishing, understand the acceptable use policy, and handle sensitive data correctly.
    \item \textbf{Implementation:} The training module should cover, at a minimum: phishing awareness, password security, acceptable use of company assets, and procedures for reporting security incidents. This should be a required step before a new employee is granted full access to corporate systems.
\end{itemize}

\end{document}
% ==============================================================================
% --- END OF DOCUMENT ---
% ==============================================================================
```