```latex
\documentclass[12pt]{article}

% Preamble: Required Packages
\usepackage[margin=1in]{geometry}
\usepackage{pifont} % For checkmarks and crosses
\usepackage{booktabs} % For professional tables
\usepackage{hyperref} % For clickable links
\usepackage{url} % For URL formatting
\usepackage{seqsplit} % For splitting long strings like IPs
\usepackage{graphicx} % For potential logos
\usepackage{fancyhdr} % For headers/footers
\usepackage{xcolor} % For color definitions

% --- Document Setup ---
\hypersetup{
    colorlinks=true,
    linkcolor=blue,
    filecolor=magenta,      
    urlcolor=cyan,
    pdftitle={Cybersecurity Posture Assessment},
    pdfpagemode=FullScreen,
}

\pagestyle{fancy}
\fancyhf{}
\lhead{Cybersecurity Posture Assessment}
\rhead{\textbf{Hidden Gem}}
\cfoot{\thepage}

% --- Custom Commands ---
\newcommand{\yes}{\ding{51}}
\newcommand{\no}{\ding{55}}

\begin{document}

% --- Title Page ---
\begin{titlepage}
    \centering
    \vspace*{1cm}
    \Huge{\textbf{Cybersecurity Posture Assessment}}
    \vspace{1.5cm}
    \Large{\textbf{Prepared for:}}
    \vspace{0.5cm}
    \Large{Hidden Gem}
    \vspace{2cm}
    \large{\textbf{Date of Report:}}
    \vspace{0.5cm}
    \large{\today}
    \vfill
    \large{\textit{This report contains sensitive information and should be handled with care.}}
\end{titlepage}

\tableofcontents
\newpage

% --- Section 1: Executive Summary ---
\section{Executive Summary}
This report provides a cybersecurity posture assessment for \textbf{Hidden Gem}, synthesized from organizational data, a network scan, and a review of current risks. The assessment identified several critical and high-risk gaps in the organization's security controls that require immediate attention.

The primary areas of concern are the lack of multi-factor authentication (MFA) on email accounts and the absence of a formal security awareness training program for employees. These gaps significantly increase the risk of account compromise through phishing and other social engineering attacks.

Additionally, a technical scan identified an externally exposed Secure Shell (SSH) service. While necessary for remote administration, its configuration and software version could not be determined by the initial scan, representing a potential unquantified risk.

Recommendations focus on implementing foundational security controls, including mandatory MFA, establishing a comprehensive training program, and conducting in-depth vulnerability scanning on external services. Addressing these findings will substantially improve the organization's resilience against common cyber threats.

% --- Section 2: Organizational Information ---
\section{Organizational Information}
The following details were provided for the assessment. This information is used to establish the context and scope of the review.

\begin{tabular}{@{}ll}
    \toprule
    \textbf{Attribute} & \textbf{Value} \\
    \midrule
    Organization Name & Hidden Gem \\
    Email Domain & \texttt{HiddenGem.com} \\
    Website Domain & \url{www.HiddenGem.com} \\
    External IP Address & \texttt{172.90.136.238} \\
    \bottomrule
\end{tabular}

% --- Section 3: Security Control Review ---
\section{Security Control Review}
A review of administrative and technical security controls was conducted based on a standardized questionnaire. The results highlight both strengths and critical weaknesses in the current security posture. "No" answers indicate a gap in controls and are considered high-risk findings.

\begin{tabular}{@{}p{0.8\linewidth}c@{}}
    \toprule
    \textbf{Security Control Question} & \textbf{Status} \\
    \midrule
    Do you require MFA to access email? & \no \\
    Do you require MFA to log into computers? & \yes \\
    Do you require MFA to access sensitive data systems? & \yes \\
    Does your organization have an employee acceptable use policy? & \yes \\
    Does your organization do security awareness training for new employees? & \no \\
    Does your organization do security awareness training for all employees at least once per year? & \no \\
    \bottomrule
\end{tabular}

\subsection*{Analysis of Control Gaps}
\begin{itemize}
    \item \textbf{MFA on Email (Critical Gap):} The absence of MFA on email is a critical vulnerability. Email is a primary target for attackers seeking to gain an initial foothold, conduct business email compromise (BEC), or pivot to other systems.
    \item \textbf{Security Awareness Training (Critical Gap):} The lack of both initial and recurring security awareness training leaves employees, the "human firewall," unprepared to identify and report phishing attempts, malware, and other social engineering tactics.
\end{itemize}

% --- Section 4: Technical Scan Results ---
\section{Technical Scan Results}
An external network scan was performed to identify open ports and services visible on the public internet.

\begin{itemize}
    \item \textbf{Target IP Address:} \seqsplit{\texttt{2001:db8::1}}
    \item \textbf{Scan Tool:} Nmap
\end{itemize}

\subsection*{Open Ports}
The following ports were found to be open on the target system.

\begin{tabular}{@{}llll@{}}
    \toprule
    \textbf{Port/Proto} & \textbf{State} & \textbf{Probable Service} & \textbf{Notes} \\
    \midrule
    22/tcp & open & SSH (Secure Shell) & Version information not available from scan. \\
    \bottomrule
\end{tabular}

\subsection*{Technical Findings}
The primary finding is the exposure of the SSH service on port 22. This service is commonly used for remote system administration. While its exposure is often necessary, it presents a significant risk if not configured securely. Potential weaknesses include:
\begin{itemize}
    \item Weak or reused passwords susceptible to brute-force attacks.
    \item Outdated software versions with known vulnerabilities (e.g., Heartbleed, Shellshock).
    \item Permitting root user login directly over SSH.
\end{itemize}
The basic nature of the performed scan did not retrieve service version information, which is essential for identifying specific vulnerabilities.

% --- Section 5: Risk Assessment Summary ---
\section{Risk Assessment Summary}
The following table correlates the findings from the security control review and the technical scan into a prioritized list of risks. No pre-existing vulnerabilities were reported.

\begin{tabular}{@{}p{0.1\linewidth}p{0.25\linewidth}p{0.4\linewidth}l@{}}
    \toprule
    \textbf{Risk ID} & \textbf{Risk Name} & \textbf{Overview} & \textbf{Severity} \\
    \midrule
    RISK-001 & Lack of MFA on Email & The absence of MFA on email accounts allows for account takeover with only a compromised password, enabling phishing, data theft, and further network intrusion. & \textbf{Critical} \\
    \addlinespace
    RISK-002 & Inadequate Security Awareness Training & Employees are not trained to recognize or respond to cyber threats. This makes the organization highly susceptible to phishing and social engineering attacks. & \textbf{High} \\
    \addlinespace
    RISK-003 & Exposed SSH Service with Unknown Version & An externally accessible SSH service is a target for attackers. Without version details, it is impossible to rule out known, exploitable vulnerabilities. & \textbf{Medium} \\
    \bottomrule
\end{tabular}

% --- Section 6: Recommendations ---
\section{Recommendations}
To mitigate the identified risks and improve the overall security posture, the following actions are recommended with high priority.

\begin{enumerate}
    \item \textbf{[RISK-001] Implement MFA for Email:} Immediately enforce mandatory multi-factor authentication (MFA) for all user and administrative email accounts. This is the single most effective control to prevent account compromise.

    \item \textbf{[RISK-002] Establish a Security Awareness Program:}
    \begin{itemize}
        \item Develop and implement a mandatory security awareness training module for all new employees as part of their onboarding process.
        \item Institute a mandatory annual security awareness refresher course for all staff.
        \item Conduct periodic, unannounced phishing simulations to test and reinforce employee training.
    \end{itemize}

    \item \textbf{[RISK-003] Harden and Assess Exposed Services:}
    \begin{itemize}
        \item Conduct an authenticated, in-depth vulnerability scan on the host at \seqsplit{\texttt{2001:db8::1}} to identify the specific SSH server version and any associated vulnerabilities.
        \item Ensure the SSH service is configured according to best practices: disable password-based authentication in favor of cryptographic keys, disable direct root login, and implement an intrusion detection tool like Fail2Ban.
        \item If SSH access is not required from the public internet, restrict access to known, trusted IP addresses using a firewall.
    \end{itemize}
\end{enumerate}

\end{document}
```