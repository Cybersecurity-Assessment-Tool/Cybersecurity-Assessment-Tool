```latex
\documentclass[12pt]{article}

% --- PACKAGE IMPORTS ---
\usepackage[margin=1in]{geometry}
\usepackage{pifont} % For checkmarks and crosses
\usepackage{booktabs} % For professional tables
\usepackage{hyperref} % For clickable links
\usepackage{url} % For URL formatting
\usepackage{seqsplit} % To split long monospaced strings
\usepackage{graphicx}
\usepackage{xcolor}

% --- DOCUMENT METADATA ---
\hypersetup{
    colorlinks=true,
    linkcolor=blue,
    filecolor=magenta,      
    urlcolor=cyan,
    pdftitle={Cybersecurity Posture Assessment Report},
    pdfauthor={Cybersecurity Analyst},
    pdfsubject={Security Analysis},
    pdfkeywords={Cybersecurity, Risk Assessment, Nmap, LaTeX},
}

% --- CUSTOM COMMANDS ---
\newcommand{\yes}{\ding{51}}
\newcommand{\no}{\ding{55}}
\newcommand{\orgname}{Digital Drift}
\newcommand{\orgdomain}{\texttt{DigitalDrift.org}}
\newcommand{\orgip}{\seqsplit{\texttt{80.251.51.32}}}
\newcommand{\targetip}{\seqsplit{\texttt{127.0.0.1}}}

\begin{document}

% --- TITLE PAGE ---
\begin{titlepage}
    \centering
    \vspace*{1cm}
    \Huge\textbf{Cybersecurity Posture Assessment Report}
    \vspace{1.5cm}
    \large
    \textbf{Prepared for:}\\
    \vspace{0.5cm}
    \orgname
    \vfill
    \large
    \textbf{Date of Report:}\\
    \vspace{0.5cm}
    \today
    \vspace{1cm}
    \large
    \textbf{Generated By:}\\
    \vspace{0.5cm}
    Expert Cybersecurity Analyst
\end{titlepage}

\tableofcontents
\newpage

% --- 1. EXECUTIVE SUMMARY ---
\section{Executive Summary}
This report provides a comprehensive cybersecurity assessment for \orgname, based on a combination of technical network scanning, a review of organizational security controls, and an analysis of pre-existing risk data. The assessment was conducted on \today.

The analysis reveals several \textbf{critical security gaps} that expose the organization to significant risk. The most pressing issues are the complete absence of Multi-Factor Authentication (MFA) for email, computer logins, and access to sensitive systems. This deficiency, coupled with a lack of foundational security policies and training for new employees, creates a high-risk environment susceptible to credential theft, unauthorized access, and social engineering attacks.

Technical scanning confirmed the presence of an open SSH port on \targetip, which correlates directly with a pre-identified critical risk named "Localhost Exposed". While an open SSH port on a local interface is not inherently a vulnerability, its presence must be managed carefully, especially in an environment with weak authentication controls.

Immediate and decisive action is required to remediate these findings. The highest priority is the organization-wide implementation of MFA. This single control will drastically reduce the risk of account compromise. Following this, the development of key security policies and training programs is essential to building a resilient and security-aware culture.

% --- 2. ORGANIZATIONAL INFORMATION ---
\section{Organizational Information}
The following details were provided by the client and used as a baseline for this assessment.

\begin{tabular}{@{}ll}
    \toprule
    \textbf{Attribute} & \textbf{Value} \\
    \midrule
    Organization Name & \orgname \\
    Email Domain & \orgdomain \\
    Website Domain & \texttt{www.DigitalDrift.org} \\
    External IP Address & \orgip \\
    \bottomrule
\end{tabular}

% --- 3. SECURITY CONTROL REVIEW ---
\section{Security Control Review}
A review of the organization's security controls was conducted via a questionnaire. The results highlight significant deficiencies in access control and employee security awareness programs.

\begin{table}[h!]
\centering
\caption{Security Controls Questionnaire Analysis}
\begin{tabular}{p{0.6\linewidth} c p{0.25\linewidth}}
    \toprule
    \textbf{Control Question} & \textbf{Status} & \textbf{Analyst Note} \\
    \midrule
    Do you require MFA to access email? & \no & \textcolor{red}{\textbf{Critical Gap.}} Email is a primary target for attackers. Lack of MFA significantly increases the risk of business email compromise. \\
    \addlinespace
    Do you require MFA to log into computers? & \no & \textcolor{red}{\textbf{Critical Gap.}} Compromised credentials could lead directly to endpoint access and lateral movement within the network. \\
    \addlinespace
    Do you require MFA to access sensitive data systems? & \no & \textcolor{red}{\textbf{Critical Gap.}} The organization's most valuable data is not protected by modern authentication standards. \\
    \addlinespace
    Does your organization have an employee acceptable use policy? & \no & \textbf{High Risk.} Without a formal policy, employees lack clear guidance on safe and acceptable use of company assets, increasing insider risk. \\
    \addlinespace
    Does your organization do security awareness training for new employees? & \no & \textbf{High Risk.} New hires are often targeted by social engineering. A lack of initial training leaves them vulnerable. \\
    \addlinespace
    Does your organization do security awareness training for all employees at least once per year? & \yes & Good practice. However, its effectiveness is reduced without foundational policies and new-hire training. \\
    \bottomrule
\end{tabular}
\end{table}

% --- 4. TECHNICAL SCAN RESULTS ---
\section{Technical Scan Results}
A network scan was performed to identify open ports and services on the target system. The results confirm the findings from the pre-existing risk data.

\begin{itemize}
    \item \textbf{Target IP Address:} \targetip
    \item \textbf{Scan Date:} Scan data provided on \today
\end{itemize}

\begin{table}[h!]
\centering
\caption{Open Ports Detected on \targetip}
\begin{tabular}{llll}
    \toprule
    \textbf{Port} & \textbf{State} & \textbf{Service} & \textbf{Notes} \\
    \midrule
    22/tcp & open & ssh & Secure Shell (SSH) access is enabled. No version information was detected. \\
    \bottomrule
\end{tabular}
\end{table}

The scan confirms that port 22 (SSH) is accessible on the local loopback interface. This aligns with the "Localhost Exposed" risk identified in Input 3. While access is limited to the local machine, any service compromise on this host could potentially leverage this access point.

% --- 5. CONSOLIDATED RISK ASSESSMENT ---
\section{Consolidated Risk Assessment}
The following table synthesizes findings from the security control review, technical scan, and pre-existing risk data into a prioritized list of identified risks.

\begin{table}[h!]
\centering
\caption{Summary of Identified Risks}
\begin{tabular}{p{0.25\linewidth} p{0.45\linewidth} l}
    \toprule
    \textbf{Risk Name} & \textbf{Description} & \textbf{Severity} \\
    \midrule
    \textbf{No Multi-Factor Authentication (MFA)} & The absence of MFA for email, endpoints, and sensitive data systems exposes the organization to a high likelihood of account compromise and subsequent data breach. & \textcolor{red}{\textbf{Critical}} \\
    \addlinespace
    \textbf{Localhost Exposed} & The SSH service is running and accessible on \targetip. This was identified as a pre-existing risk with a CVSS score of 10.0. & \textcolor{red}{\textbf{Critical}} \\
    \addlinespace
    \textbf{Lack of Security Policies and Training} & The absence of an Acceptable Use Policy and security training for new hires creates a weak security culture and increases susceptibility to human error and social engineering. & \textbf{High} \\
    \bottomrule
\end{tabular}
\end{table}

% --- 6. RECOMMENDATIONS ---
\section{Recommendations}
Based on the consolidated risk assessment, the following actions are recommended to improve the security posture of \orgname. Recommendations are prioritized based on severity and impact.

\begin{enumerate}
    \item \textbf{[Critical Priority] Implement Multi-Factor Authentication (MFA):}
    \begin{itemize}
        \item Immediately enable MFA for all user accounts across all critical systems, including:
            \begin{itemize}
                \item Email (e.g., Office 365, Google Workspace)
                \item VPN and remote access solutions
                \item Workstation and server logins
                \item All applications handling sensitive data
            \end{itemize}
        \item Provide users with clear instructions on how to enroll and use MFA.
    \end{itemize}

    \item \textbf{[High Priority] Develop and Implement Foundational Security Policies:}
    \begin{itemize}
        \item Create and enforce an \textbf{Acceptable Use Policy (AUP)} that clearly defines rules for using company IT assets.
        \item Develop a \textbf{Password Policy} that mandates strong, unique passwords.
    \end{itemize}

    \item \textbf{[High Priority] Establish a Comprehensive Security Awareness Program:}
    \begin{itemize}
        \item Institute a mandatory security awareness training module for all new employees as part of the onboarding process.
        \item Continue the annual refresher training but enhance it with topics relevant to current threats.
    \end{itemize}

    \item \textbf{[Medium Priority] Review and Harden Exposed Services:}
    \begin{itemize}
        \item Investigate the business need for the SSH service running on \targetip.
        \item If required, ensure it is securely configured: disable password-based authentication in favor of cryptographic keys, restrict access to authorized users/groups, and ensure it is patched and up-to-date.
        \item If not required, disable the service to reduce the attack surface.
    \end{itemize}
\end{enumerate}

% --- 7. CONCLUSION ---
\section{Conclusion}
The current security posture of \orgname is considered weak due to critical deficiencies in fundamental security controls. While the organization has initiated annual security training, this measure alone is insufficient. The lack of MFA, formal policies, and onboarding training presents an unacceptable level of risk.

By implementing the prioritized recommendations outlined in this report, \orgname can significantly enhance its resilience against common cyber threats and build a strong foundation for a mature cybersecurity program.

\end{document}
```