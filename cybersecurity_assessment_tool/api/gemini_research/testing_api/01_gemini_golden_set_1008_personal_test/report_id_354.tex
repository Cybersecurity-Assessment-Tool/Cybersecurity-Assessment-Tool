```latex
\documentclass[12pt, a4paper]{article}

% Preamble: Required Packages
\usepackage[margin=1in]{geometry}
\usepackage{pifont} % For checkmarks and crosses
\usepackage{booktabs} % For professional tables
\usepackage{hyperref} % For clickable links
\usepackage{url} % For formatting URLs
\usepackage{seqsplit} % For splitting long text strings
\usepackage{graphicx} % For potential logos
\usepackage{xcolor} % For colors in text

% Document Metadata
\title{Cybersecurity Posture Assessment Report}
\author{Cybersecurity Analysis Division}
\date{\today}

% Hyperref Setup
\hypersetup{
    colorlinks=true,
    linkcolor=blue,
    filecolor=magenta,      
    urlcolor=cyan,
    pdftitle={Cybersecurity Posture Assessment Report},
    pdfpagemode=FullScreen,
}

\begin{document}

\maketitle
\thispagestyle{empty}
\newpage

\tableofcontents
\newpage

% --- 1. Executive Summary ---
\section{Executive Summary}

This report provides a comprehensive cybersecurity assessment for \textbf{True North Travel}, conducted on \today. The analysis is based on a correlation of network scan data, an organizational security controls questionnaire, and a review of pre-existing risk documentation.

The assessment reveals a mixed security posture. While foundational technical controls appear to be in place, as evidenced by a limited external attack surface, there are \textbf{critical gaps in administrative and access controls}. The lack of Multi-Factor Authentication (MFA) for email and sensitive data systems represents a significant and immediate risk of account compromise and potential data breach.

Furthermore, the absence of an employee Acceptable Use Policy and mandatory annual security awareness training for all staff weakens the organization's human firewall, leaving it more susceptible to social engineering and insider threats.

On a positive note, a previously identified risk concerning an unencrypted web server on Port 80 appears to have been remediated, as our technical scan found the port to be closed.

Immediate remediation efforts should be prioritized on the implementation of MFA and the development of core security policies and training programs.

% --- 2. Organizational Information ---
\section{Organizational Information}

The following details were provided for the assessment. This information forms the basis for defining the scope and context of the security review.

\begin{itemize}
    \item \textbf{Organization Name:} True North Travel
    \item \textbf{Primary Email Domain:} \texttt{TrueNorthTravel.com}
    \item \textbf{Primary Website:} \url{www.TrueNorthTravel.com}
    \item \textbf{Known External IP:} \texttt{81.53.117.232}
\end{itemize}

% --- 3. Security Control Review ---
\section{Security Control Review}

A security questionnaire was completed to evaluate the current state of administrative and policy-based controls. The responses are summarized below. Items marked with \textcolor{red}{\ding{55}} indicate significant gaps in the security framework.

\begin{table}[h!]
\centering
\caption{Security Controls Questionnaire Results}
\begin{tabular}{p{0.75\linewidth} c}
\toprule
\textbf{Control Question} & \textbf{Response} \\
\midrule
Do you require MFA to access email? & \textcolor{red}{\ding{55}} \\
Do you require MFA to log into computers? & \textcolor{green}{\ding{51}} \\
Do you require MFA to access sensitive data systems? & \textcolor{red}{\ding{55}} \\
Does your organization have an employee acceptable use policy? & \textcolor{red}{\ding{55}} \\
Does your organization do security awareness training for new employees? & \textcolor{green}{\ding{51}} \\
Does your organization do security awareness training for all employees at least once per year? & \textcolor{red}{\ding{55}} \\
\bottomrule
\end{tabular}
\end{table}

% --- 4. Technical Scan Results ---
\section{Technical Scan Results}

An external network scan was performed to identify open ports and exposed services on the target system.

\begin{itemize}
    \item \textbf{Target IP Address:} \texttt{192.168.0.5}
    \item \textbf{Scan Date:} \today
\end{itemize}

\subsection{Port Scan Analysis}
The scan revealed a very limited attack surface, which is a positive security finding. The status of scanned ports is detailed in the table below.

\begin{table}[h!]
\centering
\caption{Nmap Port Scan Findings for \texttt{192.168.0.5}}
\begin{tabular}{ccccc}
\toprule
\textbf{Port} & \textbf{State} & \textbf{Service} & \textbf{Product} & \textbf{Version} \\
\midrule
80/tcp & closed & http & N/A & N/A \\
\bottomrule
\end{tabular}
\end{table}

\subsection{Interpretation}
The scan indicates that port 80 (HTTP) is \textbf{closed}. This finding is significant because it directly contradicts a pre-existing risk documented in the organization's risk register. The closure of this port effectively mitigates the risk of unencrypted web communications on this system. No other open ports were discovered during this scan.

% --- 5. Correlated Risk Assessment ---
\section{Correlated Risk Assessment}

This section synthesizes findings from the security control review, technical scans, and pre-existing risk data to provide a holistic view of the current risk landscape.

\begin{table}[h!]
\centering
\caption{Summary of Identified Risks}
\begin{tabular}{p{0.25\linewidth} p{0.5\linewidth} p{0.15\linewidth}}
\toprule
\textbf{Risk Name} & \textbf{Description} & \textbf{Severity} \\
\midrule
\textbf{Inadequate Access Control} & The absence of MFA on email and sensitive data systems exposes the organization to a high likelihood of account takeover, credential theft, and subsequent data breaches. & \textbf{Critical} \\
\addlinespace
\textbf{Lack of Security Governance} & The absence of a formal Acceptable Use Policy and mandatory annual security training creates an environment where employees are unaware of their security responsibilities, increasing susceptibility to phishing and insider threats. & \textbf{High} \\
\addlinespace
\textbf{Unencrypted Web Server} & A pre-existing risk stated that Port 80 was open. Our technical scan confirmed this port is now \textbf{closed}. This risk appears to be remediated. & \textbf{Remediated} \\
\bottomrule
\end{tabular}
\end{table}

% --- 6. Recommendations ---
\section{Recommendations}

Based on the correlated risk assessment, the following prioritized actions are recommended to improve the cybersecurity posture of \textbf{True North Travel}.

\subsection{Priority 1: Implement Comprehensive MFA (Critical)}
\begin{itemize}
    \item \textbf{Action:} Immediately enable and enforce MFA for all user accounts across all email platforms (e.g., Office 365, Google Workspace).
    \item \textbf{Action:} Identify all systems containing sensitive data and enforce MFA for access. This is crucial for protecting client information and proprietary data.
    \item \textbf{Justification:} This is the single most effective control to prevent unauthorized access and mitigate the risk of credential compromise.
\end{itemize}

\subsection{Priority 2: Establish Foundational Security Policies (High)}
\begin{itemize}
    \item \textbf{Action:} Develop, approve, and disseminate a formal Acceptable Use Policy (AUP). This policy should clearly define the rules for using company networks, devices, and data.
    \item \textbf{Action:} Institute a mandatory annual security awareness training program for all employees. The training should cover current threats such as phishing, social engineering, and password hygiene.
    \item \textbf{Justification:} Policies and training build a strong security culture and reduce human error, which is a factor in the majority of security incidents.
\end{itemize}

\subsection{Priority 3: Validate Risk Register (Informational)}
\begin{itemize}
    \item \textbf{Action:} Formally update the internal risk register to mark the "Unencrypted Web Server" risk as remediated.
    \item \textbf{Action:} Confirm that the closure of Port 80 was an intentional and documented change, ensuring it aligns with business requirements.
    \item \textbf{Justification:} Maintaining an accurate and up-to-date risk register is essential for effective risk management and strategic planning.
\end{itemize}

\end{document}
```