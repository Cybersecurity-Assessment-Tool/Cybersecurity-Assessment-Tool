```latex
\documentclass[12pt]{article}

% Preamble: Required Packages
\usepackage[margin=1in]{geometry}
\usepackage{pifont} % For checkmarks and crosses
\usepackage{booktabs} % For professional tables
\usepackage{hyperref} % For clickable links
\usepackage{url} % For URL formatting
\usepackage{seqsplit} % To split long strings in texttt
\usepackage[T1]{fontenc}

% Document Metadata
\title{Cybersecurity Assessment Report \\ \large For: \textbf{True North Travel}}
\author{Cybersecurity Analyst}
\date{November 22, 2025}

\begin{document}

\maketitle
\thispagestyle{empty}
\newpage
\tableofcontents
\newpage

% --- 1. Executive Summary ---
\section{Executive Summary}
This report provides a cybersecurity assessment for \textbf{True North Travel}, based on data collected on November 22, 2025. The analysis combines a review of organizational security controls, a technical network scan of a key internal asset, and a review of pre-existing risks.

The assessment identified two high-impact security gaps in organizational policy and one moderate-risk technical vulnerability. The most critical findings are the absence of Multi-Factor Authentication (MFA) for computer logins and the lack of mandatory security awareness training for new employees. These gaps significantly increase the risk of unauthorized access and successful social engineering attacks.

Additionally, a network scan revealed an outdated version of the Nginx web server software on an internal system, which could be exploited if an attacker gains internal network access.

Immediate remediation of the identified policy gaps is strongly recommended to strengthen the organization's security posture. A detailed breakdown of findings and actionable recommendations is provided in the subsequent sections.

% --- 2. Organizational Information ---
\section{Organizational Information}
The following information was provided for the assessment.

\begin{itemize}
    \item \textbf{Organization Name:} True North Travel
    \item \textbf{Email Domain:} \texttt{TrueNorthTravel.com}
    \item \textbf{Website Domain:} \url{www.TrueNorthTravel.com}
    \item \textbf{External IP Address:} \texttt{196.98.149.24}
\end{itemize}

% --- 3. Security Control Review ---
\section{Security Control Review}
A review of the organization's security controls was conducted via a questionnaire. The responses highlight key areas of strength and weakness in the current security posture. Gaps identified with a \ding{55} represent a significant increase in risk.

\begin{table}[h!]
\centering
\caption{Security Controls Questionnaire Results}
\begin{tabular}{p{0.75\linewidth} c}
\toprule
\textbf{Control Question} & \textbf{Response} \\
\midrule
Do you require MFA to access email? & \ding{51} \\
\textbf{Do you require MFA to log into computers?} & \textbf{\color{red}\ding{55}} \\
Do you require MFA to access sensitive data systems? & \ding{51} \\
Does your organization have an employee acceptable use policy? & \ding{51} \\
\textbf{Does your organization do security awareness training for new employees?} & \textbf{\color{red}\ding{55}} \\
Does your organization do security awareness training for all employees at least once per year? & \ding{51} \\
\bottomrule
\end{tabular}
\end{table}

\subsection*{Analysis of Control Gaps}
\begin{itemize}
    \item \textbf{MFA for Computer Logins:} The absence of MFA on endpoints is a critical vulnerability. If an employee's password is compromised, an attacker can gain direct access to their workstation and, potentially, the entire internal network.
    \item \textbf{Security Training for New Employees:} New hires are often targeted by phishing and social engineering attacks. Failing to provide immediate security training upon hiring leaves a significant window of vulnerability for the organization.
\end{itemize}

% --- 4. Technical Scan Results ---
\section{Technical Scan Results}
A network scan was performed to identify open ports and running services on a target system.

\begin{itemize}
    \item \textbf{Target IP Address:} \texttt{192.168.10.5}
    \item \textbf{Scan Date:} 2025-11-22T10:00:00Z
\end{itemize}

The following table details the open ports and services discovered on the target host.

\begin{table}[h!]
\centering
\caption{Open Ports and Services on \texttt{192.168.10.5}}
\begin{tabular}{l l l l l}
\toprule
\textbf{Port} & \textbf{State} & \textbf{Service} & \textbf{Product} & \textbf{Version} \\
\midrule
443/tcp & Open & https & nginx & 1.18.0 \\
\bottomrule
\end{tabular}
\end{table}

\subsection*{Analysis of Technical Findings}
The scan identified an Nginx web server, version \textbf{1.18.0}. This version was released in April 2020 and is now considered outdated. Outdated software often contains publicly known vulnerabilities that could be exploited by an attacker to gain unauthorized access or cause a denial of service.

% --- 5. Consolidated Risk Assessment ---
\section{Consolidated Risk Assessment}
The following table synthesizes findings from the security control review and the technical scan into a prioritized list of risks. No pre-existing vulnerabilities were reported.

\begin{table}[h!]
\centering
\caption{Summary of Identified Risks}
\begin{tabular}{p{0.1\linewidth} p{0.3\linewidth} p{0.15\linewidth} p{0.35\linewidth}}
\toprule
\textbf{ID} & \textbf{Risk Name} & \textbf{Severity} & \textbf{Overview} \\
\midrule
RISK-001 & Lack of Endpoint MFA & \textbf{Critical} & The absence of MFA for computer logins exposes the organization to unauthorized access if user credentials are stolen. \\
\addlinespace
RISK-002 & Inadequate New Hire Security Training & \textbf{High} & New employees are not trained on security policies upon hiring, making them susceptible to phishing and social engineering attacks. \\
\addlinespace
RISK-003 & Outdated Web Server Software & \textbf{Medium} & The Nginx server (v1.18.0) is outdated and likely contains known vulnerabilities, posing a risk to the internal network. \\
\bottomrule
\end{tabular}
\end{table}

% --- 6. Recommendations ---
\section{Recommendations}
The following actions are recommended to mitigate the identified risks and improve the overall security posture of \textbf{True North Travel}.

\begin{enumerate}
    \item \textbf{[Critical] Implement MFA for All Endpoint Logins (RISK-001):}
    \begin{itemize}
        \item \textbf{Action:} Enforce a mandatory policy requiring Multi-Factor Authentication for all employee computer logins.
        \item \textbf{Guidance:} Utilize solutions such as Windows Hello for Business, Duo Security, or other FIDO2-compliant authenticators to secure endpoint access.
    \end{itemize}
    \vspace{1em}
    
    \item \textbf{[High] Establish a Security Training Program for New Hires (RISK-002):}
    \begin{itemize}
        \item \textbf{Action:} Develop and integrate a mandatory security awareness training module into the new employee onboarding process.
        \item \textbf{Guidance:} This training should be completed within the first week of employment and cover topics such as phishing identification, password hygiene, and the acceptable use policy.
    \end{itemize}
    \vspace{1em}

    \item \textbf{[Medium] Upgrade Nginx Web Server (RISK-003):}
    \begin{itemize}
        \item \textbf{Action:} Plan and execute an upgrade of the Nginx server running on \texttt{192.168.10.5} from version 1.18.0 to the latest stable release.
        \item \textbf{Guidance:} Before deploying to production, test the new version in a staging environment to ensure compatibility with existing applications. Regularly schedule patching for all critical software.
    \end{itemize}
\end{enumerate}

\end{document}
```