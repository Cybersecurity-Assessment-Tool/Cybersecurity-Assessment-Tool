```latex
\documentclass[12pt]{article}

% Preamble: Required Packages
\usepackage[margin=1in]{geometry}
\usepackage{pifont} % For checkmarks and crosses
\usepackage{booktabs} % For professional tables
\usepackage{hyperref} % For hyperlinks
\usepackage{url} % For formatting URLs
\usepackage{seqsplit} % To split long strings without breaking
\usepackage{graphicx}
\usepackage{xcolor}
\usepackage{fancyhdr}

% Document Information
\title{Cybersecurity Posture Assessment Report}
\author{Cybersecurity Analysis Division}
\date{\today}

% Hyperref Setup
\hypersetup{
    colorlinks=true,
    linkcolor=blue,
    filecolor=magenta,      
    urlcolor=cyan,
    pdftitle={Cybersecurity Posture Assessment Report},
    pdfpagemode=FullScreen,
}

% Header and Footer
\pagestyle{fancy}
\fancyhf{}
\fancyhead[L]{Vivid Vision - Confidential}
\fancyfoot[C]{\thepage}

\begin{document}

\maketitle
\thispagestyle{empty}
\newpage

\tableofcontents
\newpage

% --- 1. Executive Summary ---
\section{Executive Summary}
This report provides a comprehensive cybersecurity assessment for \textbf{Vivid Vision}, conducted on \today. The analysis synthesizes data from a network infrastructure scan, a review of organizational security controls, and a list of pre-existing risks.

The assessment reveals several critical and high-risk gaps in the organization's security posture, primarily related to administrative and policy controls rather than technical vulnerabilities on the scanned asset. The most critical finding is the lack of Multi-Factor Authentication (MFA) for email access, which exposes the organization to significant risk of business email compromise and account takeover.

Furthermore, significant policy gaps were identified, including the absence of an employee acceptable use policy and a lack of security awareness training for new hires. These deficiencies increase the risk of insider threats and successful social engineering attacks.

A notable discrepancy was found during the technical scan. The pre-existing risk register listed an "Unencrypted Web Server" on port 80 as an active vulnerability. However, our network scan confirmed that port 80 on the target system (\texttt{192.168.0.5}) is \textbf{closed}. This indicates that the risk has either been remediated or was a false positive, and highlights a potential need to improve risk register validation processes.

Immediate remediation should focus on implementing MFA for email and establishing foundational security policies.

% --- 2. Organizational Information ---
\section{Organizational Information}
The following details were provided for the assessment. This information is used to establish the context and scope of the review.

\begin{table}[h!]
\centering
\begin{tabular}{@{}ll@{}}
\toprule
\textbf{Attribute} & \textbf{Value} \\ \midrule
Organization Name & \textbf{Vivid Vision} \\
Email Domain & \texttt{VividVision.com} \\
Website Domain & \seqsplit{\url{www.VividVision.com}} \\
External IP Address & \texttt{225.107.237.68} \\ \bottomrule
\end{tabular}
\caption{Client Organizational Details}
\end{table}

% --- 3. Security Control Review ---
\section{Security Control Review (Questionnaire)}
A review of administrative security controls was conducted based on a standardized questionnaire. The responses indicate the current state of security policies and procedures. "No" answers represent significant gaps that increase organizational risk.

\begin{table}[h!]
\centering
\begin{tabular}{@{}p{0.6\linewidth}cc@{}}
\toprule
\textbf{Control Question} & \textbf{Response} & \textbf{Assessment} \\ \midrule
Do you require MFA to access email? & \ding{55} & \textcolor{red}{\textbf{Critical Gap}} \\
Do you require MFA to log into computers? & \ding{51} & Met \\
Do you require MFA to access sensitive data systems? & \ding{51} & Met \\
Does your organization have an employee acceptable use policy? & \ding{55} & \textcolor{orange}{High Risk} \\
Does your organization do security awareness training for new employees? & \ding{55} & \textcolor{orange}{High Risk} \\
Does your organization do security awareness training for all employees at least once per year? & \ding{51} & Met \\ \bottomrule
\end{tabular}
\caption{Security Control Questionnaire Analysis}
\end{table}

% --- 4. Technical Scan Results ---
\section{Technical Scan Results}
A network scan was performed to identify open ports, running services, and potential technical vulnerabilities on the specified target system.

\begin{itemize}
    \item \textbf{Target IP:} \texttt{192.168.0.5}
    \item \textbf{Scan Date:} \today
\end{itemize}

The scan results are summarized in the table below.

\begin{table}[h!]
\centering
\begin{tabular}{@{}llll@{}}
\toprule
\textbf{Port} & \textbf{State} & \textbf{Service} & \textbf{Product / Version} \\ \midrule
80/tcp & Closed & http & N/A \\ \bottomrule
\end{tabular}
\caption{Nmap Port Scan Results for \texttt{192.168.0.5}}
\end{table}

\subsection*{Analysis of Technical Findings}
The scan of the target host revealed no open ports. Port 80, which is commonly used for unencrypted web traffic (HTTP), was specifically checked and found to be \textbf{closed}.

This finding directly contradicts the pre-existing risk documented in `Input_3_Current_Risks_JSON`, which stated "Port 80 is open." This suggests the existing risk register is outdated. The vulnerability has likely been remediated, or the initial finding was inaccurate. It is recommended to formally close the "Unencrypted Web Server" risk and review the risk validation process.

% --- 5. Consolidated Risk Assessment ---
\section{Consolidated Risk Assessment}
This section synthesizes findings from the security control review, technical scan, and pre-existing risk data into a consolidated list of current risks facing the organization.

\begin{table}[h!]
\centering
\begin{tabular}{@{}p{0.1\linewidth}p{0.3\linewidth}p{0.4\linewidth}l@{}}
\toprule
\textbf{Risk ID} & \textbf{Risk Name} & \textbf{Description} & \textbf{Severity} \\ \midrule
ORG-001 & Lack of MFA on Email & The absence of MFA on email accounts allows for account takeover with only a compromised password, enabling phishing and data breaches. & \textcolor{red}{Critical} \\
\addlinespace
ORG-002 & No Employee Acceptable Use Policy & Without a formal policy, employees may misuse company assets or access inappropriate content, increasing insider and legal risks. & \textcolor{orange}{High} \\
\addlinespace
ORG-003 & No Security Training for New Hires & New employees are not trained on security best practices upon joining, creating a window of vulnerability until the annual training. & \textcolor{orange}{High} \\
\addlinespace
RISK-001 & Outdated Risk Register & The technical scan invalidated a previously documented risk (Open Port 80), indicating the risk register is not being actively verified or updated. & \textcolor{yellow!80!black}{Medium} \\
\bottomrule
\end{tabular}
\caption{Summary of Identified Risks}
\end{table}

% --- 6. Recommendations ---
\section{Recommendations}
The following prioritized recommendations are provided to mitigate the identified risks and strengthen the overall security posture of \textbf{Vivid Vision}.

\subsection*{Priority 1: Critical}
\begin{itemize}
    \item \textbf{Implement MFA for Email Access:} Immediately enforce MFA for all user accounts accessing the email system (\texttt{VividVision.com}). This is the single most effective control to prevent business email compromise.
\end{itemize}

\subsection*{Priority 2: High}
\begin{itemize}
    \item \textbf{Develop and Implement an Acceptable Use Policy (AUP):} Create a formal AUP that clearly defines the rules and expectations for employees using company IT assets. Ensure all employees read and acknowledge the policy.
    \item \textbf{Integrate Security Training into Onboarding:} Develop a security awareness training module specifically for new hires. This training should be a mandatory part of the employee onboarding process.
\end{itemize}

\subsection*{Priority 3: Medium}
\begin{itemize}
    \item \textbf{Establish a Risk Register Validation Process:} Implement a recurring process (e.g., quarterly) to review and validate all entries in the risk register. This includes re-scanning for technical vulnerabilities to confirm they are still present before dedicating remediation resources. Formally close the "Unencrypted Web Server" risk based on current scan data.
\end{itemize}

\end{document}
```