```latex
\documentclass[12pt]{article}

% --- PACKAGES ---
\usepackage[a4paper, margin=1in]{geometry}
\usepackage{pifont} % For \ding
\usepackage{booktabs} % For professional tables
\usepackage{hyperref} % For clickable links
\usepackage{url} % For URL formatting
\usepackage{seqsplit} % For splitting long strings
\usepackage{graphicx}
\usepackage{xcolor}
\usepackage{fancyhdr}
\usepackage{lastpage}

% --- DOCUMENT & HYPERREF SETUP ---
\hypersetup{
    colorlinks=true,
    linkcolor=blue,
    filecolor=magenta,      
    urlcolor=cyan,
    pdftitle={Cybersecurity Posture Report},
    pdfauthor={Cybersecurity Analyst},
    pdfsubject={Security Assessment},
    pdfkeywords={Security, Report, Analysis},
    bookmarks=true
}

% --- HEADER & FOOTER ---
\pagestyle{fancy}
\fancyhf{} % Clear all header and footer fields
\fancyhead[L]{\textbf{Cybersecurity Posture Report}}
\fancyhead[R]{\textbf{Nexus Dynamics}}
\fancyfoot[C]{\thepage\ of \pageref{LastPage}}
\renewcommand{\headrulewidth}{0.4pt}
\renewcommand{\footrulewidth}{0.4pt}

% --- TITLE PAGE ---
\title{
    \vspace{2cm}
    \textbf{Cybersecurity Posture Report}\\
    \large \textit{Confidential Security Assessment}
    \vspace{1.5cm}
    \hrule
    \vspace{0.5cm}
    Prepared for: \textbf{Nexus Dynamics} \\
    \vspace{0.5cm}
    \hrule
    \vspace{3cm}
}
\author{Cybersecurity Analyst}
\date{\today}

% --- DOCUMENT START ---
\begin{document}

\maketitle
\thispagestyle{empty}
\newpage

\tableofcontents
\newpage

% --- EXECUTIVE SUMMARY ---
\section*{1.0 Executive Summary}
This report provides a comprehensive analysis of the cybersecurity posture for \textbf{Nexus Dynamics}. The assessment is based on a correlation of network scan data, a security controls questionnaire, and a review of pre-existing risks.

The analysis reveals several critical and high-risk vulnerabilities that require immediate attention. Key findings include an exposed and highly vulnerable FTP service (\texttt{vsftpd 2.3.4}) on the internal network, significant gaps in Multi-Factor Authentication (MFA) for workstations and sensitive systems, and a lack of security training for new employees. These issues, combined with the known risk of outdated Windows 7 workstations, create a significant attack surface that could be exploited by malicious actors to gain unauthorized access, compromise sensitive data, and disrupt business operations.

Urgent remediation is recommended, focusing on securing network services, implementing comprehensive MFA, and strengthening the employee security training program.

% --- ORGANIZATIONAL INFORMATION ---
\section*{2.0 Organizational Information}
The following information was provided for the assessment. This data forms the baseline for understanding the organization's digital footprint.

\begin{table}[h!]
\centering
\caption{Client Organizational Data}
\begin{tabular}{@{}ll@{}}
\toprule
\textbf{Attribute} & \textbf{Value} \\ \midrule
Organization Name & \textbf{Nexus Dynamics} \\
Email Domain & \texttt{NexusDynamics.com} \\
Website Domain & \url{www.NexusDynamics.com} \\
External IP Address & \texttt{9.15.146.40} \\ \bottomrule
\end{tabular}
\end{table}

% --- SECURITY CONTROL REVIEW ---
\section*{3.0 Security Control Review}
A review of the organization's security controls was conducted via a questionnaire. The results highlight foundational gaps in access control and employee security awareness. Answers marked with \ding{55} represent significant risks that weaken the overall security posture.

\begin{table}[h!]
\centering
\caption{Security Controls Questionnaire Results}
\begin{tabular}{@{}lc@{}}
\toprule
\textbf{Control Question} & \textbf{Status} \\ \midrule
Do you require MFA to access email? & \textcolor{green}{\ding{51}} \\
Do you require MFA to log into computers? & \textcolor{red}{\ding{55}} \\
Do you require MFA to access sensitive data systems? & \textcolor{red}{\ding{55}} \\
Does your organization have an employee acceptable use policy? & \textcolor{green}{\ding{51}} \\
Does your organization do security awareness training for new employees? & \textcolor{red}{\ding{55}} \\
Does your organization do security awareness training annually? & \textcolor{green}{\ding{51}} \\ \bottomrule
\end{tabular}
\end{table}

\paragraph{Analysis:} While MFA is enabled for email (a positive control), its absence on computer logins and sensitive data systems is a \textbf{critical vulnerability}. This allows an attacker with stolen credentials to move laterally within the network with minimal resistance. Furthermore, the lack of security training for new hires creates an immediate risk, as new employees are often prime targets for phishing and social engineering attacks.

% --- TECHNICAL SCAN RESULTS ---
\section*{4.0 Technical Scan Results}
An Nmap scan was performed on the internal network to identify open ports and exposed services. A critical vulnerability was discovered.

\begin{table}[h!]
\centering
\caption{Network Scan Findings for Target: \texttt{10.0.0.15}}
\begin{tabular}{@{}lllll@{}}
\toprule
\textbf{Port} & \textbf{State} & \textbf{Service} & \textbf{Version} & \textbf{Notes} \\ \midrule
21/tcp & Open & FTP & vsftpd 2.3.4 & \begin{tabular}[c]{@{}l@{}}Anonymous FTP login allowed.\\ \textbf{CRITICAL:} Version is vulnerable\\ to a backdoor (CVE-2011-2523).\end{tabular} \\ \bottomrule
\end{tabular}
\end{table}

\paragraph{Analysis:} The presence of \texttt{vsftpd version 2.3.4} is a severe and immediate threat. This specific version contains a well-known, critical backdoor vulnerability (\textbf{CVE-2011-2523}) that allows an unauthenticated attacker to execute arbitrary commands with root privileges on the server. The configuration also permits anonymous FTP logins, further lowering the barrier for an attacker to exploit this vulnerability. This service must be disabled or updated immediately.

% --- RISK ASSESSMENT SUMMARY ---
\section*{5.0 Risk Assessment Summary}
This section synthesizes findings from the security questionnaire, technical scan, and pre-existing risk data into a consolidated list of security risks.

\begin{table}[h!]
\centering
\caption{Consolidated Risk Register}
\begin{tabular}{@{}p{0.3\linewidth}p{0.5\linewidth}l@{}}
\toprule
\textbf{Risk Name} & \textbf{Overview} & \textbf{Severity} \\ \midrule
\textbf{Insecure FTP Service} & A server is running a version of vsftpd (2.3.4) with a known remote code execution backdoor (CVE-2011-2523). Anonymous login is enabled. & \textbf{Critical} \\
\addlinespace
\textbf{No MFA on Sensitive Systems} & Lack of multi-factor authentication on critical data systems allows for unauthorized access with a single set of compromised credentials. & \textbf{Critical} \\
\addlinespace
\textbf{No MFA on Workstations} & Lack of MFA on computer logins exposes the internal network to lateral movement and privilege escalation if credentials are stolen. & \textbf{High} \\
\addlinespace
\textbf{No Onboarding Security Training} & New employees are not provided with security awareness training, making them highly susceptible to phishing and social engineering attacks. & \textbf{High} \\
\addlinespace
\textbf{Outdated Windows Policy} & Workstations are running Windows 7, an End-of-Life operating system that no longer receives security updates from Microsoft. & \textbf{Medium} \\ \bottomrule
\end{tabular}
\end{table}

% --- RECOMMENDATIONS ---
\section*{6.0 Recommendations}
The following actionable recommendations are provided to mitigate the identified risks. They are prioritized based on severity and potential impact.

\subsection*{Priority 1: Immediate Actions (0-7 Days)}
\begin{itemize}
    \item \textbf{Remediate Insecure FTP Service:} Immediately disconnect the server at \texttt{10.0.0.15} from the network. If the FTP service is business-critical, upgrade it to the latest patched version of vsftpd and disable anonymous access. If it is not critical, decommission the service entirely. The preferred solution is to replace FTP with a secure alternative like SFTP (SSH File Transfer Protocol).
    \item \textbf{Implement MFA on Sensitive Systems:} Deploy a mandatory MFA solution for all access to systems identified as containing sensitive or critical data. This is the single most effective control to prevent unauthorized access to key assets.
\end{itemize}

\subsection*{Priority 2: Short-Term Actions (1-3 Months)}
\begin{itemize}
    \item \textbf{Deploy MFA for All Logins:} Enforce MFA for all user logins to company workstations and laptops. This will significantly reduce the risk of lateral movement from a compromised account.
    \item \textbf{Develop New Hire Security Training:} Create and implement a mandatory security awareness training module as part of the new employee onboarding process. This training should cover phishing, password hygiene, acceptable use, and incident reporting.
\end{itemize}

\subsection*{Priority 3: Long-Term Actions (3-6 Months)}
\begin{itemize}
    \item \textbf{Upgrade End-of-Life Systems:} Execute the plan to upgrade all remaining Windows 7 workstations to a currently supported operating system (e.g., Windows 10/11) to ensure they receive critical security patches.
\end{itemize}

\end{document}
```