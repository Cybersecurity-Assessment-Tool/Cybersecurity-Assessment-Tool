```latex
\documentclass[12pt]{article}

% Preamble: Required Packages
\usepackage[margin=1in]{geometry} % For setting page margins
\usepackage{pifont}             % For symbols like checkmarks and crosses
\usepackage{booktabs}           % For professional-looking tables
\usepackage{hyperref}           % For creating hyperlinks
\usepackage{url}                % For formatting URLs
\usepackage{seqsplit}           % For splitting long, unbreakable strings
\usepackage{graphicx}
\usepackage{xcolor}
\usepackage{parskip}            % Adds space between paragraphs

% Define some colors for the report
\definecolor{darkblue}{rgb}{0.0, 0.0, 0.55}
\definecolor{darkred}{rgb}{0.55, 0.0, 0.0}

% Hyperlink Setup
\hypersetup{
    colorlinks=true,
    linkcolor=darkblue,
    filecolor=magenta,      
    urlcolor=darkblue,
    pdftitle={Cybersecurity Posture Assessment Report},
    pdfpagemode=FullScreen,
}

% Custom commands for consistency
\newcommand{\yes}{\textcolor{green}{\ding{51}}}
\newcommand{\no}{\textcolor{red}{\ding{55}}}
\newcommand{\riskcritical}{\textcolor{red}{Critical}}
\newcommand{\riskhigh}{\textcolor{orange}{High}}
\newcommand{\riskmedium}{\textcolor{yellow}{Medium}}

\begin{document}

% --- Title Page ---
\begin{titlepage}
    \centering
    \vspace*{1cm}
    \Huge\textbf{Cybersecurity Posture Assessment Report}
    \vspace{1.5cm}
    \Large
    \textbf{Prepared for:}\\
    Nexus Dynamics
    \vspace{2cm}
    \large
    \textbf{Date of Report:}\\
    \today
    \vfill
    \large
    \textbf{Generated By:}\\
    Expert Cybersecurity Analyst
\end{titlepage}

\tableofcontents
\newpage

% --- 1. Executive Summary ---
\section{Executive Summary}

This report provides a comprehensive assessment of the cybersecurity posture for \textbf{Nexus Dynamics}, based on a correlation of network scan data, organizational security controls, and known risks. The analysis reveals several critical and high-risk vulnerabilities that require immediate attention to mitigate the threat of a potential data breach.

The most critical finding is a publicly accessible MySQL database (version 5.7.33) on host \texttt{172.16.50.20}. This version is outdated and contains numerous publicly disclosed vulnerabilities. This technical exposure is severely compounded by significant gaps in organizational security controls. Specifically, the lack of Multi-Factor Authentication (MFA) for computer and sensitive data system access, combined with the absence of any security awareness training program for employees, creates a high-risk environment.

An adversary could exploit the outdated database software directly or leverage weak or stolen employee credentials—obtained through phishing—to gain unauthorized access to sensitive data. Immediate remediation should focus on restricting network access to the database, implementing mandatory MFA, and establishing a baseline security awareness program.

% --- 2. Organizational Information ---
\section{Organizational Information}

The following details were provided for the assessment. This information is used to establish the context and scope of the review.

\begin{itemize}
    \item \textbf{Organization Name:} Nexus Dynamics
    \item \textbf{Primary Email Domain:} \texttt{NexusDynamics.org}
    \item \textbf{Primary Website Domain:} \url{www.NexusDynamics.org}
    \item \textbf{Known External IP:} \texttt{88.137.178.102}
\end{itemize}

% --- 3. Security Control Review ---
\section{Security Control Review}

A review of the organization's self-reported security controls was conducted. The following table summarizes the responses and provides an analyst's assessment of gaps based on cybersecurity best practices. "No" answers indicate significant weaknesses in the defense-in-depth strategy.

\begin{table}[h!]
\centering
\caption{Security Controls Questionnaire Analysis}
\begin{tabular}{p{0.6\textwidth} c l}
\toprule
\textbf{Control Question} & \textbf{Response} & \textbf{Analyst Note} \\
\midrule
Do you require MFA to access email? & \yes & Good Practice \\
Do you require MFA to log into computers? & \no & \riskcritical{} Gap \\
Do you require MFA to access sensitive data systems? & \no & \riskcritical{} Gap \\
Does your organization have an employee acceptable use policy? & \yes & Good Practice \\
Does your organization do security awareness training for new employees? & \no & \riskhigh{} Risk \\
Does your organization do security awareness training for all employees at least once per year? & \no & \riskhigh{} Risk \\
\bottomrule
\end{tabular}
\end{table}

The lack of MFA on computers and sensitive systems is a critical vulnerability. It significantly lowers the barrier for an attacker with compromised credentials to gain deep network access. Furthermore, the absence of a security awareness program makes employees highly susceptible to phishing and social engineering attacks, which are the primary vectors for credential theft.

% --- 4. Technical Scan Results ---
\section{Technical Scan Results}

A network scan was performed to identify open ports and exposed services on the target system.

\subsection{Nmap Scan Findings}
\begin{itemize}
    \item \textbf{Target IP:} \texttt{172.16.50.20}
    \item \textbf{Host Status:} Up
\end{itemize}

The scan identified the following open port:

\begin{table}[h!]
\centering
\caption{Open Ports Detected on \texttt{172.16.50.20}}
\begin{tabular}{l l l l}
\toprule
\textbf{Port} & \textbf{Service} & \textbf{Product} & \textbf{Version} \\
\midrule
3306/tcp & mysql & MySQL & 5.7.33 \\
\bottomrule
\end{tabular}
\end{table}

\subsection{Technical Analysis}
The scan confirms that a MySQL database server is directly exposed to the network on port 3306. The detected version, \textbf{MySQL 5.7.33}, reached its official End-of-Life (EOL) in October 2023. EOL software no longer receives security updates from the vendor, leaving it perpetually vulnerable to newly discovered exploits. This finding corroborates the pre-existing risk "Database Exposure" and elevates its urgency.

% --- 5. Correlated Risk Assessment ---
\section{Correlated Risk Assessment}

The following table synthesizes findings from the technical scan, control review, and pre-existing risk data into a prioritized list of security risks.

\begin{table}[h!]
\centering
\caption{Summary of Identified Security Risks}
\begin{tabular}{p{0.2\textwidth} p{0.5\textwidth} p{0.15\textwidth}}
\toprule
\textbf{Risk Title} & \textbf{Description} & \textbf{Severity} \\
\midrule
\textbf{Exposed \& Outdated Database} & A MySQL 5.7.33 database is directly accessible on port 3306. The software is End-of-Life and vulnerable to known exploits. This confirms the pre-existing "Database Exposure" risk. & \riskcritical{} \\
\addlinespace
\textbf{Lack of Multi-Factor Authentication} & MFA is not enforced for computer logins or access to sensitive systems. A single compromised password could lead to a full system breach, including the exposed database. & \riskcritical{} \\
\addlinespace
\textbf{Insufficient Security Awareness Program} & The organization provides no security training to new or existing employees. This makes the organization highly vulnerable to phishing, which could provide attackers with the credentials needed to access the exposed database. & \riskhigh{} \\
\bottomrule
\end{tabular}
\end{table}

% --- 6. Recommendations ---
\section{Recommendations}

Based on the correlated risk assessment, the following actions are recommended to improve the security posture of \textbf{Nexus Dynamics}. Recommendations are prioritized by urgency.

\subsection{Immediate Priority (Remediate within 72 hours)}
\begin{enumerate}
    \item \textbf{Restrict Access to Database:} Immediately apply firewall rules to block all public access to TCP port 3306 on \texttt{172.16.50.20}. Access should only be permitted from trusted internal IP addresses or via a secure VPN connection.
\end{enumerate}

\subsection{High Priority (Remediate within 30 days)}
\begin{enumerate}
    \item \textbf{Implement MFA Everywhere:} Enforce MFA for all employees for:
    \begin{itemize}
        \item Logging into company computers (laptops/desktops).
        \item Accessing all sensitive data systems, including databases, financial applications, and administrative portals.
    \end{itemize}
    \item \textbf{Plan Database Upgrade:} Develop and schedule a plan to upgrade the MySQL 5.7.33 instance to a currently supported version (e.g., MySQL 8.x). This is critical for receiving future security patches.
\end{enumerate}

\subsection{Medium Priority (Remediate within 90 days)}
\begin{enumerate}
    \item \textbf{Establish Security Awareness Training:} Implement a mandatory security awareness training program for all employees. This should include:
    \begin{itemize}
        \item An initial training module for all new hires.
        \item Annual refresher training for all staff.
        \item Topics should cover phishing identification, password hygiene, and the acceptable use policy.
    \end{itemize}
\end{enumerate}

% --- 7. Conclusion ---
\section{Conclusion}

The combination of an exposed, outdated database and weak internal security controls places \textbf{Nexus Dynamics} at a significant and immediate risk of a data breach. The findings in this report are highly correlated: a technical vulnerability is made far more dangerous by gaps in security policies.

By implementing the prioritized recommendations—starting with restricting database access and enforcing MFA—the organization can drastically reduce its attack surface and build a more resilient security foundation. Continuous monitoring and regular security assessments are advised to maintain and improve this posture over time.

\end{document}
```