```latex
\documentclass[12pt]{article}

% --- PACKAGES ---
\usepackage[margin=1in]{geometry}
\usepackage{pifont} % For checkmarks and crosses (\ding{51}, \ding{55})
\usepackage{booktabs} % For professional tables (\toprule, \midrule, \bottomrule)
\usepackage{hyperref} % For clickable links and metadata
\usepackage{url}      % For typesetting URLs
\usepackage{seqsplit} % For splitting long strings to prevent overflow
\usepackage{xcolor}   % For custom colors

% --- DOCUMENT METADATA ---
\hypersetup{
    colorlinks=true,
    linkcolor=blue,
    urlcolor=cyan,
    pdftitle={Cybersecurity Posture Assessment Report},
    pdfauthor={Cybersecurity Analyst},
    pdfsubject={Security Analysis}
}

% --- CUSTOM COMMANDS ---
\newcommand{\yes}{\textcolor{green}{\ding{51}}}
\newcommand{\no}{\textcolor{red}{\ding{55}}}

% --- DOCUMENT START ---
\begin{document}

% --- TITLE PAGE ---
\title{Cybersecurity Posture Assessment Report \\ \large For: Gilded Cage Design}
\author{Cybersecurity Analyst}
\date{\today}
\maketitle
\thispagestyle{empty}
\newpage

% --- TABLE OF CONTENTS ---
\tableofcontents
\newpage

% --- EXECUTIVE SUMMARY ---
\section*{Executive Summary}
This report details the cybersecurity posture of Gilded Cage Design based on a synthesis of network scan data, organizational security controls, and pre-existing risk information. The assessment identified several critical and high-risk vulnerabilities that require immediate attention.

A key technical finding is an exposed FTP server on the internal network (\texttt{10.0.0.15}) running a severely outdated and vulnerable version of \texttt{vsftpd} (2.3.4), which is known to have a critical backdoor vulnerability (CVE-2011-2523). This service is dangerously configured to allow anonymous logins, posing a direct and immediate threat of unauthorized access and data compromise.

Furthermore, significant policy and procedural gaps were noted. These include the lack of mandatory multi-factor authentication (MFA) on employee computers, the absence of an employee Acceptable Use Policy (AUP), and inadequate annual security awareness training for all staff. These issues, combined with pre-existing risks such as outdated Windows 7 workstations, create a high-risk environment.

Immediate remediation of the FTP server and the implementation of stronger access controls and security policies are strongly recommended to mitigate these threats.

% --- ORGANIZATIONAL INFORMATION ---
\section{Organizational Information}
The following information was provided for the assessment.

\begin{tabular}{@{}ll}
\toprule
\textbf{Attribute} & \textbf{Value} \\
\midrule
Organization Name & Gilded Cage Design \\
Email Domain & \texttt{GildedCageDesign.com} \\
Website Domain & \url{www.GildedCageDesign.com} \\
External IP Address & \texttt{155.80.124.235} \\
\bottomrule
\end{tabular}

% --- SECURITY CONTROL REVIEW ---
\section{Security Control Review}
A review of organizational security controls was conducted via a questionnaire. The responses indicate several significant gaps in the current security posture. "No" answers represent a deviation from security best practices and are flagged as risks.

\begin{table}[h!]
\centering
\caption{Security Controls Questionnaire Results}
\begin{tabular}{@{}p{0.5\linewidth}p{0.3\linewidth}c@{}}
\toprule
\textbf{Control Question} & \textbf{Control Area} & \textbf{Response} \\
\midrule
Do you require MFA to access email? & Identity \& Access Management & \yes \\
Do you require MFA to log into computers? & Identity \& Access Management & \no \\
Do you require MFA to access sensitive data systems? & Identity \& Access Management & \yes \\
\addlinespace
Does your organization have an employee acceptable use policy? & Governance \& Policy & \no \\
\addlinespace
Does your organization do security awareness training for new employees? & Security Awareness & \yes \\
Does your organization do security awareness training for all employees at least once per year? & Security Awareness & \no \\
\bottomrule
\end{tabular}
\end{table}

% --- TECHNICAL SCAN RESULTS ---
\section{Technical Scan Results}
An Nmap scan was performed on the specified target to identify open ports and exposed services.

\subsection{Target: \texttt{10.0.0.15}}
The scan revealed one open port with a critically vulnerable service.

\begin{table}[h!]
\centering
\caption{Open Port Analysis for \texttt{10.0.0.15}}
\begin{tabular}{@{}lllll@{}}
\toprule
\textbf{Port} & \textbf{State} & \textbf{Service} & \textbf{Version} & \textbf{Notes} \\
\midrule
21/tcp & Open & ftp & vsftpd 2.3.4 & Anonymous FTP login allowed \\
\bottomrule
\end{tabular}
\end{table}

\paragraph{Analysis:} The presence of \textbf{vsftpd version 2.3.4} is a \textbf{critical risk}. This specific version was compromised in 2011, and a malicious backdoor was inserted into the source code. An attacker can gain a command shell on the server by simply entering a smiley face `:)` as the username. Compounding this issue, the server is configured to allow \textbf{anonymous FTP logins}, which removes the first layer of defense (authentication) and makes the system trivial to exploit. This vulnerability is tracked as CVE-2011-2523.

% --- CONSOLIDATED RISK ASSESSMENT ---
\section{Consolidated Risk Assessment}
The following table synthesizes findings from the technical scan, control review, and pre-existing risk data into a prioritized list.

\begin{table}[h!]
\centering
\caption{Prioritized Risk Register}
\begin{tabular}{@{}lp{0.5\linewidth}ll@{}}
\toprule
\textbf{Risk ID} & \textbf{Description} & \textbf{Source} & \textbf{Severity} \\
\midrule
RISK-001 & Exposed FTP server with a known backdoor vulnerability (vsftpd 2.3.4) and anonymous access enabled. & Network Scan & \textbf{Critical} \\
\addlinespace
RISK-002 & Lack of mandatory MFA for workstation logins, increasing the risk of unauthorized access via compromised credentials. & Questionnaire & High \\
\addlinespace
RISK-003 & Absence of an Acceptable Use Policy, leading to inconsistent security practices and lack of employee accountability. & Questionnaire & High \\
\addlinespace
RISK-004 & Inadequate security awareness training (not performed annually), increasing susceptibility to phishing and social engineering. & Questionnaire & High \\
\addlinespace
RISK-005 & Workstations are running an outdated and unsupported operating system (Windows 7), which no longer receives security updates. & Existing Risks & Medium \\
\bottomrule
\end{tabular}
\end{table}

% --- RECOMMENDATIONS ---
\section{Recommendations}
Based on the consolidated risk assessment, the following actions are recommended to improve the security posture of Gilded Cage Design. Recommendations are prioritized by severity.

\begin{enumerate}
    \item \textbf{[Critical] Remediate Vulnerable FTP Server (RISK-001):}
    \begin{itemize}
        \item \textbf{Immediate:} Take the server at \texttt{10.0.0.15} offline immediately to prevent exploitation.
        \item \textbf{Short-Term:} If FTP is required, upgrade the \texttt{vsftpd} service to the latest stable version and disable anonymous access.
        \item \textbf{Long-Term:} Evaluate the business need for FTP. If possible, replace it with a more secure file transfer protocol like SFTP (SSH File Transfer Protocol).
    \end{itemize}
    
    \item \textbf{[High] Implement MFA on Workstations (RISK-002):}
    \begin{itemize}
        \item Deploy a multi-factor authentication solution for all employee computer logins to protect against credential theft and unauthorized access.
    \end{itemize}
    
    \item \textbf{[High] Develop and Enforce Security Policies (RISK-003):}
    \begin{itemize}
        \item Create a formal Acceptable Use Policy (AUP) that defines rules for the use of company assets, data handling, and internet usage.
        \item Require all employees to read and acknowledge the policy.
    \end{itemize}
    
    \item \textbf{[High] Establish a Recurring Training Program (RISK-004):}
    \begin{itemize}
        \item Institute a mandatory security awareness training program for all employees to be completed on an annual basis. This should cover topics like phishing, password security, and data protection.
    \end{itemize}
    
    \item \textbf{[Medium] Upgrade Outdated Operating Systems (RISK-005):}
    \begin{itemize}
        \item Execute the plan to upgrade all remaining Windows 7 workstations to a modern, supported operating system (e.g., Windows 10/11) to ensure they receive critical security patches.
    \end{itemize}
\end{enumerate}

\end{document}
```