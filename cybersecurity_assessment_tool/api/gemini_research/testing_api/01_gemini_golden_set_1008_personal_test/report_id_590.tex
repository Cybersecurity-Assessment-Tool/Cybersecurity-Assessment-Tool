```latex
\documentclass[12pt]{article}

% --- PACKAGE IMPORTS ---
\usepackage[margin=1in]{geometry}
\usepackage{pifont} % Required for \ding
\usepackage{booktabs} % For professional-looking tables
\usepackage{hyperref} % For clickable links and table of contents
\usepackage{url} % For formatting URLs
\usepackage{seqsplit} % To split long strings in texttt
\usepackage{xcolor} % For custom colors
\usepackage{graphicx}
\usepackage{fancyhdr}

% --- DOCUMENT SETUP ---
\hypersetup{
    colorlinks=true,
    linkcolor=blue,
    filecolor=magenta,
    urlcolor=cyan,
    pdftitle={Cybersecurity Risk Assessment Report},
    pdfauthor={Cybersecurity Analyst},
}

% --- CUSTOM COMMANDS ---
\newcommand{\yes}{\textcolor{green}{\ding{51}}} % Green checkmark
\newcommand{\no}{\textcolor{red}{\ding{55}}}   % Red X

% --- HEADER & FOOTER ---
\pagestyle{fancy}
\fancyhf{}
\fancyhead[L]{Cybersecurity Risk Assessment Report}
\fancyhead[R]{For: Calyx Botany}
\fancyfoot[C]{\thepage}

% --- DOCUMENT START ---
\begin{document}

% --- TITLE PAGE ---
\begin{titlepage}
    \centering
    \vspace*{1cm}
    \Huge\textbf{Cybersecurity Risk Assessment Report}
    \vspace{1.5cm}
    \Large
    \textbf{Prepared For:} \\
    Calyx Botany
    \vspace{2cm}
    \large
    \textbf{Date of Report:} \\
    \today
    \vfill
    \large
    \textbf{Generated By:} \\
    Expert Cybersecurity Analyst
\end{titlepage}

\tableofcontents
\newpage

% ===================================================================
\section{Executive Summary}
% ===================================================================
This report provides a comprehensive cybersecurity assessment for Calyx Botany, based on network scan data, an organizational security questionnaire, and a review of pre-existing risks.

The analysis revealed several high-impact vulnerabilities that require immediate attention. Key findings include critical gaps in access control, specifically the lack of Multi-Factor Authentication (MFA) for email and sensitive data systems. This significantly increases the risk of account compromise and unauthorized data access.

Furthermore, a technical network scan identified an internal FTP server (\texttt{10.0.0.15}) running a dangerously outdated and vulnerable version of \texttt{vsftpd} (2.3.4), which is susceptible to remote code execution. The server is also misconfigured to allow anonymous access, posing a direct threat to internal network security.

These new findings, combined with the existing risk of outdated Windows 7 workstations, paint a picture of a security posture with multiple exploitable weaknesses. This report outlines these risks in detail and provides actionable recommendations to mitigate them and improve the overall security resilience of the organization.

% ===================================================================
\section{Organizational Information}
% ===================================================================
The following information was provided for the assessment.

\begin{tabular}{@{}ll}
    \toprule
    \textbf{Attribute} & \textbf{Value} \\
    \midrule
    Organization Name & Calyx Botany \\
    Email Domain & \texttt{CalyxBotany.net} \\
    Website Domain & \url{www.CalyxBotany.net} \\
    External IP Address & \texttt{2.154.41.64} \\
    \bottomrule
\end{tabular}

% ===================================================================
\section{Security Control Review (Questionnaire Analysis)}
% ===================================================================
An analysis of the security questionnaire reveals the current state of implemented administrative controls. Answers marked with \no{} indicate significant gaps in the organization's security posture that directly contribute to increased risk.

\begin{table}[h!]
\centering
\caption{Security Questionnaire Results}
\begin{tabular}{@{}p{8cm}lc@{}}
    \toprule
    \textbf{Control Question} & \textbf{Control Area} & \textbf{Implemented} \\
    \midrule
    Do you require MFA to access email? & Access Control & \no \\
    Do you require MFA to log into computers? & Access Control & \yes \\
    Do you require MFA to access sensitive data systems? & Access Control & \no \\
    \addlinespace
    Does your organization have an employee acceptable use policy? & Policy \& Governance & \yes \\
    Does your organization do security awareness training for new employees? & Training & \yes \\
    Does your organization do security awareness training for all employees at least once per year? & Training & \yes \\
    \bottomrule
\end{tabular}
\end{table}

\paragraph{Analysis:} The lack of enforced MFA on email and sensitive data systems are critical weaknesses. Email is a primary target for phishing and account takeover attacks, which can serve as a pivot point into other systems. Failure to protect sensitive data systems with MFA exposes critical assets to unauthorized access.

% ===================================================================
\section{Technical Scan Results}
% ===================================================================
A network scan was performed to identify active services and potential technical vulnerabilities on the target host.

\subsection{Host: \texttt{10.0.0.15}}
The scan identified the following open port and service on an internal network host.

\begin{table}[h!]
\centering
\caption{Open Ports and Services for \texttt{10.0.0.15}}
\begin{tabular}{@{}lllll@{}}
    \toprule
    \textbf{Port} & \textbf{State} & \textbf{Service} & \textbf{Product & Version} \\
    \midrule
    21/tcp & open & ftp & \texttt{vsftpd 2.3.4} \\
    \bottomrule
\end{tabular}
\end{table}

\paragraph{Critical Findings:}
\begin{itemize}
    \item \textbf{Vulnerable Software:} The identified version, \textbf{\texttt{vsftpd 2.3.4}}, is extremely old and contains a critical backdoor vulnerability (\textbf{CVE-2011-2523}). A malicious actor with network access can exploit this vulnerability to gain a command shell on the server, leading to a full system compromise.
    \item \textbf{Insecure Configuration:} The scan confirmed that \textbf{Anonymous FTP login is allowed}. This misconfiguration permits any user on the network to access the FTP server without authentication, potentially exposing sensitive files or providing a platform for attackers to stage malicious tools.
\end{itemize}

% ===================================================================
\section{Synthesized Risk Assessment}
% ===================================================================
The following table synthesizes findings from the questionnaire, technical scans, and pre-existing risk data into a prioritized list.

\begin{table}[h!]
\centering
\caption{Consolidated Risk Register}
\begin{tabular}{@{}p{1.5cm}p{3.5cm}p{7.5cm}l@{}}
    \toprule
    \textbf{Risk ID} & \textbf{Risk Name} & \textbf{Description} & \textbf{Severity} \\
    \midrule
    RISK-001 & Vulnerable FTP Server & An internal server is running \texttt{vsftpd 2.3.4}, which is vulnerable to remote code execution (CVE-2011-2523). Anonymous login is also enabled. & \textcolor{red}{\textbf{Critical}} \\
    \addlinespace
    RISK-002 & Inadequate MFA Implementation & MFA is not enforced on critical systems including email and sensitive data stores, exposing them to account takeover attacks. & \textcolor{red}{\textbf{Critical}} \\
    \addlinespace
    RISK-003 & Outdated Operating Systems & (Existing Risk) Workstations are running Windows 7, which is end-of-life and no longer receives security updates. & \textcolor{orange}{\textbf{Medium}} \\
    \bottomrule
\end{tabular}
\end{table}

% ===================================================================
\section{Recommendations}
% ===================================================================
The following actions are recommended to mitigate the identified risks.

\subsection{RISK-001: Vulnerable FTP Server (Critical)}
\begin{itemize}
    \item \textbf{Immediate Action:} Investigate the business purpose of the FTP server on \texttt{10.0.0.15}. If it is not essential, decommission it immediately by shutting down the service and uninstalling the software.
    \item \textbf{Remediation:} If the service is required, it must be secured:
    \begin{enumerate}
        \item Upgrade \texttt{vsftpd} to the latest stable version provided by the operating system's package manager.
        \item Disable anonymous login in the configuration file (\texttt{anonymous\_enable=NO}).
        \item Enforce strong, unique passwords for all FTP users.
    \end{enumerate}
    \item \textbf{Long-Term Strategy:} Replace the FTP protocol with a secure alternative such as SFTP (SSH File Transfer Protocol) for all file transfer needs.
\end{itemize}

\subsection{RISK-002: Inadequate MFA Implementation (Critical)}
\begin{itemize}
    \item \textbf{Immediate Action:} Prioritize the deployment of Multi-Factor Authentication for all user accounts.
    \item \textbf{Remediation:}
    \begin{enumerate}
        \item Enforce MFA on the email system (e.g., Office 365, Google Workspace) for all users within the next 30 days.
        \item Identify all systems classified as containing sensitive data and enforce MFA for access within the next 60 days.
    \end{enumerate}
    \item \textbf{Long-Term Strategy:} Develop a formal access control policy that mandates MFA for all remote access and all systems that process, store, or transmit sensitive information.
\end{itemize}

\subsection{RISK-003: Outdated Operating Systems (Medium)}
\begin{itemize}
    \item \textbf{Immediate Action:} Ensure all Windows 7 machines are isolated from critical network segments if they cannot be immediately upgraded.
    \item \textbf{Remediation:} Continue and, if possible, accelerate the existing project to upgrade all Windows 7 workstations to a supported version, such as Windows 10 or Windows 11.
    \item \textbf{Long-Term Strategy:} Implement a patch and vulnerability management policy that includes tracking operating system life cycles to prevent future use of end-of-life software.
\end{itemize}

\end{document}
```