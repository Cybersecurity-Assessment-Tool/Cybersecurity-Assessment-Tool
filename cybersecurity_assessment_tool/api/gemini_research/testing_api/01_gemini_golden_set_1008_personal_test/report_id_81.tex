```latex
\documentclass[12pt]{article}

% Preamble: Required Packages
\usepackage[margin=1in]{geometry}
\usepackage{pifont} % For checkmarks and crosses
\usepackage{booktabs} % For professional tables
\usepackage{hyperref} % For clickable links
\usepackage{url} % For URL formatting
\usepackage{seqsplit} % For splitting long strings in tt font
\usepackage[T1]{fontenc}

% Document Metadata
\title{Cybersecurity Posture Assessment Report}
\author{Cybersecurity Analysis Division}
\date{\today}

\hypersetup{
    colorlinks=true,
    linkcolor=black,
    urlcolor=blue,
    pdftitle={Cybersecurity Posture Assessment Report},
    pdfauthor={Cybersecurity Analysis Division},
}

\begin{document}

\maketitle
\tableofcontents
\newpage

% --- 1. Executive Overview ---
\section{Executive Overview}
This report provides a comprehensive cybersecurity assessment for \textbf{Iron Bridge Legal}, synthesizing data from technical network scans, an organizational security questionnaire, and a review of pre-existing risk documentation.

The analysis has uncovered a \textbf{Critical} risk finding. A network service on an internal system, \texttt{10.5.5.5}, is exposed on port 8080 with an HTTP title of ``TOP SECRET DB''. This finding directly contradicts a pre-existing risk assessment which incorrectly labeled this port as secure. This suggests a potentially exposed sensitive database that requires immediate investigation and remediation.

Furthermore, significant procedural gaps were identified, including the lack of mandatory Multi-Factor Authentication (MFA) for computer logins and the absence of security awareness training for new employees. These gaps represent a \textbf{High} level of risk, as they increase the organization's vulnerability to credential theft and social engineering attacks.

Immediate action is required to address the exposed network service. Recommendations are provided to mitigate all identified risks and strengthen the overall security posture of the organization.

% --- 2. Organizational Information ---
\section{Organizational Information}
The following details were provided for the assessment. This information is used to establish the context for the technical and procedural findings.

\begin{tabular}{@{}ll}
\toprule
\textbf{Attribute} & \textbf{Value} \\
\midrule
Organization Name & \textbf{Iron Bridge Legal} \\
Email Domain & \texttt{IronBridgeLegal.org} \\
Website Domain & \url{www.IronBridgeLegal.org} \\
External IP Address & \texttt{47.213.202.35} \\
\bottomrule
\end{tabular}

% --- 3. Security Control Review ---
\section{Security Control Review}
A security questionnaire was completed to assess the current state of administrative and procedural controls. The responses are summarized below. Items marked with \ding{55} indicate significant gaps in the security framework.

\begin{tabular}{@{}p{0.8\linewidth}c}
\toprule
\textbf{Control Question} & \textbf{Status} \\
\midrule
Do you require MFA to access email? & \ding{51} \\
Do you require MFA to log into computers? & \textbf{\color{red}\ding{55}} \\
Do you require MFA to access sensitive data systems? & \ding{51} \\
Does your organization have an employee acceptable use policy? & \ding{51} \\
Does your organization do security awareness training for new employees? & \textbf{\color{red}\ding{55}} \\
Does your organization do security awareness training for all employees at least once per year? & \ding{51} \\
\bottomrule
\end{tabular}

\subsection*{Analysis of Controls}
\begin{itemize}
    \item \textbf{High Risk - Lack of Endpoint MFA:} The absence of MFA for computer logins is a critical weakness. If an employee's password is stolen, an attacker could gain direct access to their workstation and potentially the internal network.
    \item \textbf{High Risk - No New Employee Training:} New hires are often prime targets for phishing and social engineering. Failing to provide security training during onboarding leaves a recurring window of vulnerability as the organization grows.
\end{itemize}

% --- 4. Technical Scan Results ---
\section{Technical Scan Results}
A network scan was performed to identify open ports and exposed services on the specified target.

\begin{itemize}
    \item \textbf{Target IP Address:} \texttt{10.5.5.5}
    \item \textbf{Scan Tool:} Nmap
\end{itemize}

\begin{tabular}{@{}llll}
\toprule
\textbf{Port} & \textbf{State} & \textbf{Service} & \textbf{Details / Banner} \\
\midrule
8080/tcp & Open & http-proxy & \textbf{\color{red}HTTP Title: TOP SECRET DB} \\
\bottomrule
\end{tabular}

\subsection*{Analysis of Technical Findings}
The scan identified a single open port, 8080, on the internal host \texttt{10.5.5.5}. The HTTP title discovered on this port, ``TOP SECRET DB'', is a major cause for concern. This strongly indicates that a database, potentially containing highly sensitive information, is accessible over the network via a web interface. 

This finding is particularly alarming because it directly contradicts the information provided in the \textit{Current Risks} documentation, which stated this port was a ``confirmed secure'' false positive. The previous assessment is demonstrably incorrect and must be updated.

% --- 5. Consolidated Risk Assessment ---
\section{Consolidated Risk Assessment}
The following table synthesizes findings from the security control review and the technical scan into a prioritized list of risks.

\begin{tabular}{@{}p{0.25\linewidth}p{0.5\linewidth}p{0.15\linewidth}}
\toprule
\textbf{Risk Name} & \textbf{Description} & \textbf{Severity} \\
\midrule
\textbf{Exposed Sensitive Database Interface} & An open port (8080) on an internal server (\texttt{10.5.5.5}) has an HTTP title of ``TOP SECRET DB''. This suggests an unprotected interface to sensitive data. This invalidates a previous risk assessment that marked this as a false positive. & \textbf{Critical} \\
\addlinespace
\textbf{Lack of MFA on Endpoints} & The absence of MFA on computer logins exposes the organization to unauthorized access if an employee's credentials are compromised through phishing or other means. & \textbf{High} \\
\addlinespace
\textbf{Inadequate New Employee Onboarding} & New hires are not provided with security awareness training, making them highly susceptible to social engineering and phishing attacks from their first day. & \textbf{High} \\
\bottomrule
\end{tabular}

% --- 6. Recommendations ---
\section{Recommendations}
Based on the consolidated risk assessment, the following actions are recommended to mitigate the identified vulnerabilities and improve the overall security posture.

\begin{enumerate}
    \item \textbf{[Immediate] Investigate and Secure Exposed Service:}
    \begin{itemize}
        \item Immediately investigate the service running on \texttt{10.5.5.5:8080}.
        \item Determine the nature of the data in the "TOP SECRET DB".
        \item If the service is not required for business operations, disable it.
        \item If required, apply strict access controls (e.g., firewall rules, authentication, MFA) to ensure it is only accessible to authorized personnel from trusted network locations.
    \end{itemize}

    \item \textbf{[High Priority] Implement Endpoint MFA:}
    \begin{itemize}
        \item Procure and deploy a Multi-Factor Authentication solution for all employee computer logins (e.g., Windows Hello, Duo, etc.).
        \item Enforce this policy for all users, including administrative and executive staff.
    \end{itemize}

    \item \textbf{[High Priority] Establish New Hire Security Training:}
    \begin{itemize}
        \item Develop a mandatory security awareness training module to be completed by all new employees as part of their standard onboarding process.
        \item The training should cover, at a minimum: phishing identification, password hygiene, and the company's acceptable use policy.
    \end{itemize}
    
    \item \textbf{[Process Improvement] Review Risk Assessment Procedures:}
    \begin{itemize}
        \item Conduct a post-mortem to understand why the risk on port 8080 was previously misclassified as a secure false positive.
        \item Update risk assessment and validation procedures to include more rigorous technical verification to prevent future inaccuracies.
    \end{itemize}
\end{enumerate}

\end{document}
```