```latex
\documentclass[12pt]{article}

% Preamble: Required Packages
\usepackage[margin=1in]{geometry}
\usepackage{pifont} % For checkmarks and crosses
\usepackage{booktabs} % For professional tables
\usepackage{hyperref} % For clickable links
\usepackage{url} % For formatting URLs
\usepackage{seqsplit} % To split long strings in tt font
\usepackage{graphicx}
\usepackage{xcolor}

% Document Information
\title{Cybersecurity Assessment Report}
\author{Cybersecurity Analysis Division}
\date{\today}

% Hyperref Setup
\hypersetup{
    colorlinks=true,
    linkcolor=blue,
    filecolor=magenta,      
    urlcolor=cyan,
    pdftitle={Cybersecurity Assessment Report},
    pdfpagemode=FullScreen,
}

\begin{document}

\maketitle
\hrule
\vspace{1em}

% ------------------------------------------------------------------
% 1. Executive Summary
% ------------------------------------------------------------------
\section*{Executive Summary}

This report provides a comprehensive cybersecurity assessment for \textbf{Granite Shield}, correlating data from technical network scans, organizational security control questionnaires, and a review of pre-existing risks.

The analysis has identified several critical-risk findings that require immediate attention. The most severe issue is the public exposure of an End-of-Life (EOL) MySQL database server (version 5.7.33) on the network. This version no longer receives security updates, making it a prime target for exploitation.

Furthermore, significant gaps were identified in the organization's identity and access management controls. The absence of Multi-Factor Authentication (MFA) for email and computer logins represents a critical vulnerability to account takeover and business email compromise attacks.

Immediate remediation of the exposed database and the enforcement of MFA are paramount to mitigating the substantial risk of a security breach. This report details these findings and provides actionable recommendations for remediation.

% ------------------------------------------------------------------
% 2. Organizational Information
% ------------------------------------------------------------------
\section{Organizational Information}

The following details were provided for the assessment scope.

\begin{tabular}{@{}ll}
\toprule
\textbf{Attribute} & \textbf{Value} \\
\midrule
Organization Name & \textbf{Granite Shield} \\
Email Domain & \texttt{GraniteShield.com} \\
Website Domain & \url{www.GraniteShield.com} \\
External IP Address & \texttt{162.226.235.134} \\
\bottomrule
\end{tabular}

% ------------------------------------------------------------------
% 3. Security Control Review (Questionnaire)
% ------------------------------------------------------------------
\section{Security Control Review}

An assessment of organizational security controls was conducted based on a supplied questionnaire. The results are summarized below. Answers marked with a red 'X' (\textcolor{red}{\ding{55}}) indicate a deviation from security best practices and represent a significant risk.

\begin{table}[h!]
\centering
\begin{tabular}{@{}lc}
\toprule
\textbf{Control Question} & \textbf{Response} \\
\midrule
Does your organization have an employee acceptable use policy? & \textcolor{green}{\ding{51}} \\
Does your organization do security awareness training for new employees? & \textcolor{green}{\ding{51}} \\
Does your organization do security awareness training for all employees annually? & \textcolor{green}{\ding{51}} \\
Do you require MFA to access sensitive data systems? & \textcolor{green}{\ding{51}} \\
\textbf{Do you require MFA to access email?} & \textcolor{red}{\ding{55}} \\
\textbf{Do you require MFA to log into computers?} & \textcolor{red}{\ding{55}} \\
\bottomrule
\end{tabular}
\caption{Organizational Security Control Responses}
\end{table}

\subsection*{Analysis}
The lack of mandatory MFA for email and computer logins is a critical security gap. Email is a primary vector for phishing and business email compromise (BEC). Unprotected computer logins significantly increase the risk of lateral movement within the network should an attacker compromise a user's credentials. While MFA on sensitive systems is a positive control, the entry points (email and computers) remain dangerously exposed.

% ------------------------------------------------------------------
% 4. Technical Scan Results
% ------------------------------------------------------------------
\section{Technical Scan Results}

A network scan was performed on the target system to identify open ports and exposed services.

\begin{itemize}
    \item \textbf{Target IP Address:} \texttt{172.16.50.20}
\end{itemize}

\begin{table}[h!]
\centering
\begin{tabular}{@{}lllll}
\toprule
\textbf{Port} & \textbf{State} & \textbf{Service} & \textbf{Product} & \textbf{Version} \\
\midrule
3306/tcp & open & mysql & MySQL & 5.7.33 \\
\bottomrule
\end{tabular}
\caption{Open Ports and Services}
\end{table}

\subsection*{Analysis}
The scan confirms that port \textbf{3306} is open, exposing a MySQL database service directly to the network. This configuration is highly discouraged as it makes the database vulnerable to brute-force attacks, credential stuffing, and direct exploitation.

Critically, the detected MySQL version \textbf{5.7.33} reached its official End-of-Life (EOL) in October 2023. This means it no longer receives security patches from the vendor, and any newly discovered vulnerabilities will remain unpatched. Running EOL software, especially for a critical database service, poses an extreme and unacceptable level of risk.

% ------------------------------------------------------------------
% 5. Correlated Risk Assessment
% ------------------------------------------------------------------
\section{Correlated Risk Assessment}

The following table synthesizes findings from the technical scan, control review, and pre-existing risk data into a prioritized list of security risks.

\begin{table}[h!]
\centering
\begin{tabular}{@{}p{0.3\linewidth} p{0.5\linewidth} p{0.15\linewidth}}
\toprule
\textbf{Risk Title} & \textbf{Description} & \textbf{Severity} \\
\midrule
\textbf{Exposed End-of-Life Database Server} & A MySQL 5.7.33 database server is exposed on port 3306. This version is past its End-of-Life and no longer receives security updates, making it highly susceptible to exploitation. & \textbf{Critical} \\
\addlinespace
\textbf{Insufficient Multi-Factor Authentication (MFA)} & MFA is not enforced for email or computer logins. This exposes the organization to a high risk of account compromise, phishing, and unauthorized access to endpoints. & \textbf{Critical} \\
\bottomrule
\end{tabular}
\caption{Summary of Identified Risks}
\end{table}

% ------------------------------------------------------------------
% 6. Recommendations
% ------------------------------------------------------------------
\section{Recommendations}

The following actions are recommended to mitigate the identified risks. They are prioritized based on severity and potential impact.

\subsection*{Immediate Priority (Critical)}
\begin{enumerate}
    \item \textbf{Restrict Database Access:} Immediately apply firewall rules to block all public access to port 3306 on host \texttt{172.16.50.20}. Access should be restricted to only trusted, internal application servers or through a secure VPN connection.
    \item \textbf{Enforce MFA for All Users:} Immediately enable and enforce MFA for all user accounts across all critical systems, especially for email access (e.g., Office 365, Google Workspace) and computer logins (e.g., via Windows Hello for Business, Duo).
\end{enumerate}

\subsection*{Short-Term Priority (High)}
\begin{enumerate}
    \item \textbf{Upgrade Database Server:} Plan and execute the migration of the MySQL 5.7 database to a currently supported version (e.g., MySQL 8.0 or higher). This is essential to ensure the system receives ongoing security patches.
    \item \textbf{Vulnerability Scanning:} Implement a regular, authenticated vulnerability scanning program for all internal and external assets to proactively identify outdated software and misconfigurations.
\end{enumerate}

\subsection*{Long-Term Priority (Medium)}
\begin{enumerate}
    \item \textbf{Network Segmentation:} Review and enhance the network architecture to implement proper segmentation. Database servers should be isolated in a secure network zone, separate from user workstations and other less-sensitive assets.
    \item \textbf{Policy Update:} Update the organization's Acceptable Use Policy to explicitly mandate the use of MFA for all company-provided services where it is available.
\end{enumerate}

\end{document}
```