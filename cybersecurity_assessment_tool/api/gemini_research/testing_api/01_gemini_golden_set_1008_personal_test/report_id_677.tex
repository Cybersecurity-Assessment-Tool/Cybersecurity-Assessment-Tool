```latex
\documentclass[12pt]{article}

% --- PACKAGES ---
\usepackage[margin=1in]{geometry}
\usepackage{pifont} % For checkmarks and crosses
\usepackage{booktabs} % For professional tables
\usepackage{hyperref} % For clickable links
\usepackage{url} % For URL formatting
\usepackage{seqsplit} % To split long monospaced text
\usepackage{graphicx}
\usepackage{xcolor}
\usepackage{fancyhdr}

% --- DOCUMENT SETUP ---
\hypersetup{
    colorlinks=true,
    linkcolor=blue,
    filecolor=magenta,      
    urlcolor=cyan,
    pdftitle={Cybersecurity Posture Assessment Report},
    pdfpagemode=FullScreen,
}

\pagestyle{fancy}
\fancyhf{}
\lhead{Cybersecurity Posture Assessment}
\rhead{Terraform Global}
\cfoot{\thepage}

% --- DOCUMENT START ---
\begin{document}

% --- TITLE PAGE ---
\begin{titlepage}
    \centering
    \vspace*{1cm}
    \Huge\textbf{Cybersecurity Posture Assessment Report}
    \vspace{1.5cm}
    \Large
    \textbf{Prepared for:} \\
    Terraform Global
    \vspace{2cm}
    \includegraphics[width=0.4\textwidth]{example-image-a} % Placeholder for company logo
    \vfill
    \large
    \textbf{Date of Report:} \today
\end{titlepage}

\tableofcontents
\newpage

% --- EXECUTIVE SUMMARY ---
\section{Executive Summary}
This report provides a comprehensive cybersecurity assessment for Terraform Global, based on an analysis of network scan data, organizational security controls, and pre-existing risk information. The assessment was conducted to identify vulnerabilities, evaluate current security practices, and provide actionable recommendations to enhance the organization's security posture.

\paragraph{Key Findings:}
The analysis reveals a mixed security posture. On a positive note, the organization has implemented strong Multi-Factor Authentication (MFA) controls across email, computer logins, and sensitive data systems. This significantly reduces the risk of unauthorized access via compromised credentials.

However, several critical and high-risk issues were identified that require immediate attention:
\begin{itemize}
    \item \textbf{Systemic RDP Exposure:} A network scan discovered a new server (\texttt{10.10.10.51}) with an exposed Remote Desktop Protocol (RDP) port. When correlated with existing risk data, which notes a similar issue on another host, this points to a systemic control failure rather than an isolated incident. Exposed RDP is a primary vector for ransomware attacks.
    \item \textbf{Governance Gaps:} The organization lacks a formal employee Acceptable Use Policy (AUP). This foundational document is essential for setting clear security expectations for all staff.
    \item \textbf{Inadequate Onboarding:} New employees do not receive security awareness training. This oversight leaves the organization vulnerable, as new hires are often targeted by social engineering attacks and may be unaware of internal security procedures.
\end{itemize}

\paragraph{Conclusion:}
While Terraform Global has strong authentication controls, the identified gaps in network security and internal governance present a significant risk. We strongly recommend prioritizing the remediation of the RDP exposures and implementing the proposed policy and training enhancements.

\newpage

% --- ORGANIZATIONAL INFORMATION ---
\section{Organizational Information}
This section details the organizational data provided for the assessment.

\begin{tabular}{@{}ll}
\toprule
\textbf{Attribute} & \textbf{Value} \\
\midrule
Organization Name & Terraform Global \\
Email Domain & \texttt{TerraformGlobal.org} \\
Website Domain & \seqsplit{\url{www.TerraformGlobal.org}} \\
External IP Address & \texttt{170.146.194.115} \\
\bottomrule
\end{tabular}

% --- SECURITY CONTROL REVIEW ---
\section{Security Control Review}
The following table summarizes the organization's responses to a security controls questionnaire. "No" answers indicate significant gaps in the security framework.

\begin{table}[h!]
\centering
\caption{Security Controls Questionnaire Analysis}
\begin{tabular}{@{}p{8cm}ccp{3cm}@{}}
\toprule
\textbf{Control Question} & \textbf{Response} & \textbf{Status} & \textbf{Assessment} \\
\midrule
Do you require MFA to access email? & Yes & \ding{51} & Strong Control \\
Do you require MFA to log into computers? & Yes & \ding{51} & Strong Control \\
Do you require MFA to access sensitive data systems? & Yes & \ding{51} & Strong Control \\
Does your organization have an employee acceptable use policy? & No & \ding{55} & \textbf{High Risk} \\
Does your organization do security awareness training for new employees? & No & \ding{55} & \textbf{High Risk} \\
Does your organization do security awareness training for all employees at least once per year? & Yes & \ding{51} & Good Practice \\
\bottomrule
\end{tabular}
\end{table}

\paragraph{Analysis:} The consistent implementation of MFA is commendable. However, the absence of an Acceptable Use Policy and security training for new hires are critical administrative control failures. These gaps can lead to unintentional policy violations and increase susceptibility to phishing and other social engineering attacks.

% --- TECHNICAL SCAN RESULTS ---
\section{Technical Scan Results}
An Nmap scan was performed on the target system to identify open ports and exposed services.

\begin{itemize}
    \item \textbf{Target IP:} \texttt{10.10.10.51}
    \item \textbf{Scan Date:} Data provided on \today
\end{itemize}

\begin{table}[h!]
\centering
\caption{Open Ports on Target: 10.10.10.51}
\begin{tabular}{@{}lllll@{}}
\toprule
\textbf{Port} & \textbf{Protocol} & \textbf{State} & \textbf{Service} & \textbf{Details} \\
\midrule
3389 & TCP & open & ms-wbt-server & Microsoft Remote Desktop Protocol (RDP) \\
\bottomrule
\end{tabular}
\end{table}

\paragraph{Analysis:} The scan identified that port 3389 (RDP) is open. RDP is a common target for brute-force attacks and exploitation of vulnerabilities (e.g., BlueKeep). Exposing RDP directly to a network without mitigating controls like a VPN or IP whitelisting is a critical security risk.

\newpage

% --- RISK ASSESSMENT ---
\section{Risk Assessment and Correlation}
This section synthesizes findings from the questionnaire, technical scan, and pre-existing risk data into a consolidated list of identified risks.

\begin{table}[h!]
\centering
\caption{Consolidated Risk Register}
\begin{tabular}{@{}p{1.5cm}p{3cm}p{5cm}p{1.5cm}p{2.5cm}@{}}
\toprule
\textbf{Risk ID} & \textbf{Risk Name} & \textbf{Description} & \textbf{Severity} & \textbf{Affected Systems} \\
\midrule
\textbf{RISK-001} & RDP Exposure (Existing) & RDP service exposed on an internal server, as per prior risk data. & Critical (9.0) & \texttt{10.10.10.50} \\
\addlinespace
\textbf{RISK-002} & \textbf{RDP Exposure (New)} & \textbf{New finding of an exposed RDP service. This indicates a systemic problem.} & \textbf{Critical} & \texttt{10.10.10.51} \\
\addlinespace
\textbf{RISK-003} & \textbf{Lack of Acceptable Use Policy} & No formal policy exists to guide employees on secure use of company assets. & \textbf{High} & Organization-wide \\
\addlinespace
\textbf{RISK-004} & \textbf{Inadequate Onboarding Training} & New employees do not receive security awareness training, increasing risk. & \textbf{High} & Organization-wide \\
\bottomrule
\end{tabular}
\end{table}

% --- RECOMMENDATIONS ---
\section{Recommendations}
The following actions are recommended to mitigate the identified risks and improve the overall security posture of Terraform Global.

\subsection{RISK-001 \& RISK-002: Systemic RDP Exposure (Critical)}
\begin{description}
    \item[Immediate (0-7 days):] Immediately block public access to TCP port 3389 on hosts \texttt{10.10.10.50} and \texttt{10.10.10.51} at the network firewall. Conduct an emergency network-wide scan to identify any other instances of exposed RDP.
    \item[Short-Term (1-3 months):] Implement a secure remote access solution, such as a VPN with MFA, for all administrative access. Access should be granted on a least-privilege basis.
    \item[Long-Term (3-6 months):] Develop and enforce a network security policy that explicitly forbids the exposure of management protocols (RDP, SSH, etc.) to the internet. Implement automated configuration checks to prevent recurrence.
\end{description}

\subsection{RISK-003: Lack of Acceptable Use Policy (High)}
\begin{description}
    \item[Immediate (0-30 days):] Draft a formal Acceptable Use Policy (AUP) covering topics such as data handling, internet usage, password security, and incident reporting.
    \item[Short-Term (1-3 months):] Ratify the AUP with management and legal teams. Distribute the policy to all current employees and require them to read and provide digital acknowledgement.
\end{description}

\subsection{RISK-004: Inadequate Onboarding Training (High)}
\begin{description}
    \item[Immediate (0-30 days):] Procure or develop a foundational security awareness training module for new hires. This should cover key topics like phishing, password hygiene, and the new AUP.
    \item[Short-Term (1-3 months):] Integrate this training into the mandatory onboarding process for all new employees, to be completed before they are granted access to sensitive systems.
\end{description}

\end{document}
```