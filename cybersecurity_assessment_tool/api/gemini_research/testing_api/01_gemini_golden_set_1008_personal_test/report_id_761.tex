```latex
\documentclass[12pt]{article}

% Preamble: Required Packages
\usepackage[margin=1in]{geometry}
\usepackage{pifont} % For checkmarks and crosses
\usepackage{booktabs} % For professional tables
\usepackage{hyperref} % For clickable links
\usepackage{url} % For URL formatting
\usepackage{seqsplit} % To split long text strings
\usepackage{graphicx}
\usepackage[table]{xcolor}

% --- Document Setup ---
\hypersetup{
    colorlinks=true,
    linkcolor=blue,
    filecolor=magenta,      
    urlcolor=cyan,
    pdftitle={Cybersecurity Posture Assessment Report},
    pdfpagemode=FullScreen,
}

\newcommand{\yes}{\ding{51}}
\newcommand{\no}{\ding{55}}
\definecolor{lightgray}{gray}{0.9}

% --- Document Start ---
\begin{document}

% --- Title Page ---
\begin{titlepage}
    \centering
    \vspace*{1cm}
    \Huge\textbf{Cybersecurity Posture Assessment Report}
    \vspace{1.5cm}
    \Large
    \textbf{Prepared for:}\\
    Zenith Point
    \vspace{2cm}
    \rule{\linewidth}{0.5mm}
    \vspace{0.5cm}
    \begin{center}
        \textbf{CONFIDENTIAL}
    \end{center}
    \rule{\linewidth}{0.5mm}
    \vfill
    \large
    \textbf{Date of Report:}\\
    \today
\end{titlepage}

\tableofcontents
\newpage

% --- Section 1: Executive Summary ---
\section{Executive Summary}
This report provides a comprehensive analysis of the cybersecurity posture for \textbf{Zenith Point}, based on a combination of technical network scanning, a review of existing risks, and an organizational security controls questionnaire.

The assessment identified several areas of significant concern that elevate the organization's risk profile. The primary findings include:

\begin{itemize}
    \item \textbf{Systemic Remote Access Exposure:} A network scan identified an open Remote Desktop Protocol (RDP) port on an internal system (\texttt{10.10.10.51}). This finding, correlated with a pre-existing risk on another host (\texttt{10.10.10.50}), indicates a pattern of insecure remote access configurations. Exposed RDP is a primary vector for ransomware attacks.
    \item \textbf{Critical Endpoint Security Gap:} The organization does not enforce Multi-Factor Authentication (MFA) for computer logins. This gap critically undermines credential security, as a compromised password would grant an attacker direct access to an endpoint, facilitating lateral movement within the network.
    \item \textbf{Inadequate Security Training:} While new employees receive security training, there is no mandatory annual refresher for all staff. This lapse allows security knowledge to become outdated, increasing susceptibility to social engineering and phishing attacks.
\end{itemize}

The combination of exposed services and weak endpoint authentication creates a high-impact attack path. It is strongly recommended that \textbf{Zenith Point} prioritize the remediation actions outlined in Section 6 of this report to mitigate these critical risks.

% --- Section 2: Organizational Information ---
\section{Organizational Information}
The following details were provided for the assessment.

\begin{table}[h!]
\centering
\begin{tabular}{@{}ll@{}}
\toprule
\textbf{Attribute} & \textbf{Value} \\ \midrule
Organization Name    & Zenith Point \\
Email Domain         & \texttt{ZenithPoint.net} \\
Website Domain       & \url{www.ZenithPoint.net} \\
External IP Address  & \texttt{214.139.166.175} \\ \bottomrule
\end{tabular}
\caption{Client Organizational Details}
\end{label{tab:org_info}
\end{table}

% --- Section 3: Security Control Review ---
\section{Security Control Review}
A questionnaire was completed to evaluate the implementation of key administrative and technical security controls. The results are summarized below.

\begin{table}[h!]
\centering
\rowcolors{2}{lightgray}{}
\begin{tabular}{@{}p{0.8\linewidth}c@{}}
\toprule
\textbf{Control Question} & \textbf{Status} \\ \midrule
Do you require MFA to access email? & \yes \\
\rowcolor{red!20} Do you require MFA to log into computers? & \no \\
Do you require MFA to access sensitive data systems? & \yes \\
Does your organization have an employee acceptable use policy? & \yes \\
Does your organization do security awareness training for new employees? & \yes \\
\rowcolor{red!20} Does your organization do security awareness training for all employees at least once per year? & \no \\ \bottomrule
\end{tabular}
\caption{Security Controls Questionnaire Results}
\label{tab:controls}
\end{table}

\subsection*{Analysis of Control Gaps}
The review identified two significant control gaps, highlighted in red above.
\begin{itemize}
    \item \textbf{No MFA for Computer Logins:} This is a critical vulnerability. If an employee's credentials are stolen (e.g., via phishing), an attacker can log into their computer without a second authentication factor, gaining a foothold on the internal network.
    \item \textbf{No Annual Security Awareness Training:} The threat landscape evolves continuously. Without regular, recurring training, employees are less likely to recognize and appropriately respond to modern phishing, social engineering, and malware threats.
\end{itemize}

% --- Section 4: Technical Scan Results ---
\section{Technical Scan Results}
An Nmap scan was performed to identify open ports and services on the specified target system.

\subsection*{Target: \texttt{10.10.10.51}}
The scan revealed the following open port:

\begin{table}[h!]
\centering
\begin{tabular}{@{}llll@{}}
\toprule
\textbf{Port} & \textbf{State} & \textbf{Service Name} & \textbf{Product / Version} \\ \midrule
3389/tcp & open & ms-wbt-server & Not Determined \\ \bottomrule
\end{tabular}
\caption{Open Ports on \texttt{10.10.10.51}}
\label{tab:scan_results}
\end{table}

\subsection*{Analysis of Technical Findings}
The scan confirmed that port \textbf{3389/tcp}, the standard port for Microsoft's Remote Desktop Protocol (RDP), is open on the host \texttt{10.10.10.51}. RDP is a powerful remote administration tool, but when exposed without proper security controls, it becomes a high-value target for attackers. Threat actors frequently scan for open RDP ports to launch brute-force password attacks or exploit vulnerabilities to gain unauthorized access, often as a precursor to ransomware deployment.

% --- Section 5: Correlated Risk Assessment ---
\section{Correlated Risk Assessment}
This section synthesizes findings from the security questionnaire, the technical scan, and pre-existing risk data to provide a holistic view of the current risk posture.

\begin{table}[h!]
\centering
\begin{tabular}{@{}p{0.2\linewidth}p{0.5\linewidth}p{0.1\linewidth}p{0.15\linewidth}@{}}
\toprule
\textbf{Risk Name} & \textbf{Description} & \textbf{Severity} & \textbf{Affected Assets} \\ \midrule
\rowcolor{red!25} \textbf{Systemic RDP Exposure} & RDP is exposed on multiple internal systems (\texttt{10.10.10.50}, \texttt{10.10.10.51}). This pattern suggests a lack of a secure remote access policy and presents a direct path for network compromise. & Critical & \texttt{10.10.10.50} \texttt{10.10.10.51} \\
\rowcolor{orange!25} \textbf{Lack of Endpoint MFA} & The absence of MFA on computer logins drastically lowers the effort required for an attacker with stolen credentials to compromise an endpoint and move laterally. & High & All Workstations \& Servers \\
\rowcolor{yellow!25} \textbf{Inadequate Security Training Program} & The lack of annual refresher training for all staff increases the likelihood of successful phishing and social engineering attacks, which are primary sources of credential theft. & Medium & All Employees \\ \bottomrule
\end{tabular}
\caption{Summary of Identified Risks}
\label{tab:risk_summary}
\end{table}

% --- Section 6: Recommendations ---
\section{Recommendations}
The following prioritized recommendations are provided to address the identified risks and improve the overall security posture of \textbf{Zenith Point}.

\subsection*{Immediate Priority (Critical)}
\begin{enumerate}
    \item \textbf{Remediate RDP Exposure:} Immediately close port 3389 on hosts \texttt{10.10.10.51} and \texttt{10.10.10.50}. For all systems requiring remote access, implement a secure solution such as a Virtual Private Network (VPN) or a Zero Trust Network Access (ZTNA) gateway. Access should be granted on a least-privilege basis.
\end{enumerate}

\subsection*{High Priority}
\begin{enumerate}
    \setcounter{enumi}{1}
    \item \textbf{Deploy MFA for All Endpoints:} Implement and enforce MFA for all user logins to workstations and servers. This is one of the most effective controls for preventing unauthorized access from compromised credentials. Prioritize deployment for administrative accounts and then expand to all users.
    \item \textbf{Establish Annual Security Training:} Institute a mandatory security awareness training program for all employees to be completed at least once per year. The training should cover current threats such as phishing, ransomware, and proper data handling.
\end{enumerate}

\subsection*{Medium Priority}
\begin{enumerate}
    \setcounter{enumi}{3}
    \item \textbf{Conduct a Comprehensive Internal Scan:} Perform a credentialed vulnerability scan across the entire internal network to identify other instances of misconfigured services, outdated software, and weak security settings.
    \item \textbf{Develop a Formal Remote Access Policy:} Create and enforce a formal policy that defines the approved methods and procedures for all remote access to the internal network, ensuring that security is a primary consideration.
\end{enumerate}

% --- Document End ---
\end{document}
```