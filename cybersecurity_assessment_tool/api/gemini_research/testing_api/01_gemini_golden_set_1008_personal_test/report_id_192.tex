```latex
\documentclass[12pt]{article}

% Preamble: Required Packages and Document Setup
\usepackage[a4paper, margin=1in]{geometry}
\usepackage{pifont} % For checkmarks and crosses (dingbats)
\usepackage{booktabs} % For professional-looking tables
\usepackage{hyperref} % For clickable links and metadata
\usepackage{url} % For formatting URLs
\usepackage{seqsplit} % For splitting long strings in texttt
\usepackage{xcolor} % For coloring text
\usepackage{graphicx} % For potential logos or diagrams

% --- Document Metadata ---
\hypersetup{
    colorlinks=true,
    linkcolor=blue,
    filecolor=magenta,      
    urlcolor=cyan,
    pdftitle={Cybersecurity Posture Assessment Report},
    pdfauthor={Cybersecurity Analysis Division},
    pdfsubject={Security Analysis},
    pdfkeywords={Cybersecurity, Risk Assessment, Vulnerability},
    bookmarks=true
}

% --- Custom Commands & Color Definitions ---
\newcommand{\yes}{\ding{51}} % Green checkmark
\newcommand{\no}{\ding{55}}  % Red X
\definecolor{critical}{HTML}{990000}
\definecolor{high}{HTML}{D14302}
\definecolor{medium}{HTML}{E5A50A}

\begin{document}

% --- Title Page ---
\title{
    \vspace{2cm}
    \textbf{Cybersecurity Posture Assessment Report} \\
    \large \textit{Prepared for: Urban Jungle Planning}
    \vspace{1cm}
}
\author{Cybersecurity Analysis Division}
\date{\today}
\maketitle
\thispagestyle{empty}
\newpage

% --- Table of Contents ---
\tableofcontents
\newpage

% --- Section 1: Executive Summary ---
\section{Executive Summary}
This report provides a cybersecurity posture assessment for Urban Jungle Planning based on an analysis of organizational data, a security controls questionnaire, and technical network scan data. 

The assessment reveals several critical and high-risk security gaps that require immediate attention. The most significant finding is the complete absence of Multi-Factor Authentication (MFA) across all key systems, including email, computer logins, and access to sensitive data. This represents a critical vulnerability that significantly increases the risk of unauthorized access and account compromise.

Furthermore, foundational security governance controls are missing. The organization lacks a formal Employee Acceptable Use Policy (AUP) and does not conduct mandatory annual security awareness training for all staff. These gaps weaken the organization's human firewall and create legal and operational risks.

\textbf{Important Note:} The provided technical network scan data and the list of current risks were found to be corrupted and could not be analyzed. Consequently, this assessment is limited to the organizational and policy-based information provided. A new technical scan is strongly recommended to identify external-facing vulnerabilities.

The following sections detail these findings and provide prioritized, actionable recommendations to mitigate the identified risks and strengthen the overall security posture of Urban Jungle Planning.

% --- Section 2: Organizational Information ---
\section{Organizational Information}
The following details were provided for the assessment.

\begin{tabular}{@{}ll}
    \toprule
    \textbf{Attribute} & \textbf{Value} \\
    \midrule
    Organization Name & Urban Jungle Planning \\
    Email Domain & \texttt{UrbanJunglePlanning.com} \\
    Website Domain & \url{www.UrbanJunglePlanning.com} \\
    External IP Address & \seqsplit{\texttt{47.149.118.168}} \\
    \bottomrule
\end{tabular}

% --- Section 3: Security Control Review ---
\section{Security Control Review}
The following table summarizes the organization's responses to the security controls questionnaire. Each response has been assessed against industry best practices. "No" answers indicate significant gaps in the security framework.

\begin{tabular}{@{}p{0.6\linewidth} c p{0.25\linewidth}@{}}
    \toprule
    \textbf{Control Question} & \textbf{Response} & \textbf{Assessment} \\
    \midrule
    Do you require MFA to access email? & \no & \textcolor{critical}{\textbf{Critical Gap}} \\
    \addlinespace
    Do you require MFA to log into computers? & \no & \textcolor{critical}{\textbf{Critical Gap}} \\
    \addlinespace
    Do you require MFA to access sensitive data systems? & \no & \textcolor{critical}{\textbf{Critical Gap}} \\
    \addlinespace
    Does your organization have an employee acceptable use policy? & \no & \textcolor{high}{\textbf{High Risk}} \\
    \addlinespace
    Does your organization do security awareness training for new employees? & \yes & Meets Baseline \\
    \addlinespace
    Does your organization do security awareness training for all employees at least once per year? & \no & \textcolor{high}{\textbf{High Risk}} \\
    \bottomrule
\end{tabular}

% --- Section 4: Technical Scan Results ---
\section{Technical Scan Results}
The input data file for the external network scan (\texttt{Input\_1\_Network\_Scan\_JSON}) was found to be corrupted or incomplete. 

\begin{itemize}
    \item \textbf{Scan Status:} Data processing failed.
    \item \textbf{Target IP:} \seqsplit{\texttt{47.149.118.168}} (Derived from organizational data)
    \item \textbf{Findings:} No analysis of open ports, running services, or potential vulnerabilities could be performed due to the corrupted data.
\end{itemize}

Without this technical data, the organization's external attack surface remains unassessed. It is crucial to understand which services are exposed to the internet to identify and remediate potential entry points for attackers.

% --- Section 5: Risk Assessment ---
\section{Risk Assessment}
This assessment is based on the findings from the Security Control Review. The pre-existing risk data (\texttt{Input\_3\_Current\_Risks\_JSON}) was unavailable for correlation. The following table summarizes the newly identified risks, prioritized by severity.

\begin{tabular}{@{}p{0.1\linewidth} p{0.3\linewidth} p{0.15\linewidth} p{0.35\linewidth}@{}}
    \toprule
    \textbf{Risk ID} & \textbf{Risk Name} & \textbf{Severity} & \textbf{Description} \\
    \midrule
    UJP-001 & \textbf{Systemic Lack of Multi-Factor Authentication (MFA)} & \textcolor{critical}{\textbf{Critical}} & The absence of MFA for email, endpoints, and sensitive data systems makes the organization highly vulnerable to credential theft, phishing, and brute-force attacks. A single compromised password could lead to a major breach. \\
    \addlinespace
    UJP-002 & \textbf{Insufficient Security Governance and Training} & \textcolor{high}{\textbf{High}} & The lack of an Acceptable Use Policy (AUP) and mandatory annual security training for all staff creates significant risk. Employees may be unaware of security policies and evolving threats, making them more susceptible to social engineering and accidental data exposure. \\
    \addlinespace
    UJP-003 & \textbf{Unknown External Attack Surface} & \textcolor{medium}{\textbf{Medium}} & Due to the failed network scan, the organization has no current visibility into its internet-facing services and potential vulnerabilities. This unknown exposure presents a medium risk until a successful scan can be completed. \\
    \bottomrule
\end{tabular}

% --- Section 6: Recommendations ---
\section{Recommendations}
The following actionable recommendations are provided to mitigate the identified risks. They are prioritized to address the most critical issues first.

\subsection*{Priority 1: Remediate Critical Risks}
\begin{enumerate}
    \item \textbf{Implement Multi-Factor Authentication (MFA) Immediately:}
    \begin{itemize}
        \item \textbf{Action:} Deploy a robust MFA solution across the entire organization.
        \item \textbf{Scope:} Prioritize enforcement on (1) email access (e.g., Office 365, Google Workspace), (2) all remote access systems (VPNs), and (3) systems containing sensitive or critical data. The goal should be 100\% coverage for all users.
        \item \textbf{Impact:} Drastically reduces the risk of account compromise via stolen credentials.
    \end{itemize}
\end{enumerate}

\subsection*{Priority 2: Remediate High Risks}
\begin{enumerate}
    \setcounter{enumi}{1}
    \item \textbf{Develop and Enforce an Acceptable Use Policy (AUP):}
    \begin{itemize}
        \item \textbf{Action:} Create a formal AUP that clearly defines the rules for using company technology, networks, and data.
        \item \textbf{Scope:} The policy should be distributed to all employees, who must formally acknowledge they have read and understood it.
        \item \textbf{Impact:} Establishes a clear security baseline for employee behavior and reduces legal and operational risk.
    \end{itemize}
    \item \textbf{Establish a Mandatory Annual Security Awareness Program:}
    \begin{itemize}
        \item \textbf{Action:} Implement a recurring, mandatory security awareness training program for all employees.
        \item \textbf{Scope:} Training should cover current threats such as phishing, ransomware, and social engineering. It should be conducted at least annually.
        \item \textbf{Impact:} Strengthens the organization's "human firewall" and fosters a culture of security.
    \end{itemize}
\end{enumerate}

\subsection*{Priority 3: Address Information Gaps}
\begin{enumerate}
    \setcounter{enumi}{3}
    \item \textbf{Conduct a New External Vulnerability Scan:}
    \begin{itemize}
        \item \textbf{Action:} Commission a new, comprehensive vulnerability scan of the external IP address (\seqsplit{\texttt{47.149.118.168}}).
        \item \textbf{Scope:} The scan should identify all open ports, running services, and known vulnerabilities (CVEs).
        \item \textbf{Impact:} Provides critical visibility into the technical attack surface, allowing for the remediation of exploitable weaknesses.
    \end{itemize}
\end{enumerate}

\end{document}
```