```latex
\documentclass[12pt]{article}

% Preamble: Required Packages
\usepackage[margin=1in]{geometry}
\usepackage{pifont} % For checkmarks and crosses
\usepackage{booktabs} % For professional tables
\usepackage{hyperref} % For clickable links
\usepackage{url} % For URL formatting
\usepackage{seqsplit} % To split long monospaced strings
\usepackage{graphicx}
\usepackage{xcolor}
\usepackage{datetime}

% --- Document Metadata ---
\title{Cybersecurity Posture Assessment Report}
\author{Cybersecurity Analysis Division}
\date{\today}

\begin{document}

\maketitle
\thispagestyle{empty}
\newpage

\tableofcontents
\newpage

% --- Section 1: Executive Summary ---
\section{Executive Summary}

This report provides a cybersecurity posture assessment for \textbf{Common Ground}. The analysis is based on organizational data provided via a security questionnaire. It is critical to note that the provided network scan data (\texttt{Input\_1\_Network\_Scan\_JSON}) and the current risks list (\texttt{Input\_3\_Current\_Risks\_JSON}) were corrupted and could not be processed. Therefore, this assessment focuses exclusively on the administrative and policy controls identified in the questionnaire.

The review identified several critical and high-risk security gaps that require immediate attention. The most significant findings include:
\begin{itemize}
    \item \textbf{Lack of Endpoint MFA:} Employee computers are not protected by Multi-Factor Authentication (MFA), exposing the organization to significant risk from compromised credentials.
    \item \textbf{Absence of Foundational Policies:} The organization lacks a formal Employee Acceptable Use Policy (AUP), which is a fundamental component of a security program.
    \item \textbf{No Security Awareness Training:} There is no security awareness training program for new or existing employees. This exposes the organization to a high likelihood of successful social engineering and phishing attacks.
\end{itemize}

While the organization has successfully implemented MFA for email and sensitive data systems, the identified gaps in endpoint security, policy, and user training present an urgent threat. Recommendations are provided in Section \ref{sec:recommendations} to address these deficiencies. A re-assessment including a valid network scan is strongly advised to gain a complete picture of the technical security posture.

% --- Section 2: Organizational Information ---
\section{Organizational Information}

The following information was provided by the client and used as the basis for this assessment.

\begin{table}[h!]
\centering
\begin{tabular}{@{}ll@{}}
\toprule
\textbf{Attribute} & \textbf{Value} \\ \midrule
Organization Name & \textbf{Common Ground} \\
Email Domain & \seqsplit{\texttt{CommonGround.org}} \\
Website Domain & \seqsplit{\texttt{www.CommonGround.org}} \\
External IP Address & \seqsplit{\texttt{152.127.115.78}} \\ \bottomrule
\end{tabular}
\caption{Client Organizational Details}
\label{tab:org_info}
\end{table}

% --- Section 3: Security Control Review ---
\section{Security Control Review}

The following table summarizes the organization's responses to the security controls questionnaire. A green checkmark (\textcolor{green}{\ding{51}}) indicates a positive control is in place, while a red cross (\textcolor{red}{\ding{55}}) indicates a security gap.

\begin{table}[h!]
\centering
\begin{tabular}{@{}p{0.7\textwidth}cc@{}}
\toprule
\textbf{Control Question} & \textbf{Response} & \textbf{Status} \\ \midrule
Do you require MFA to access email? & Yes & \textcolor{green}{\ding{51}} \\
Do you require MFA to log into computers? & No & \textcolor{red}{\ding{55}} \\
Do you require MFA to access sensitive data systems? & Yes & \textcolor{green}{\ding{51}} \\
Does your organization have an employee acceptable use policy? & No & \textcolor{red}{\ding{55}} \\
Does your organization do security awareness training for new employees? & No & \textcolor{red}{\ding{55}} \\
Does your organization do security awareness training for all employees at least once per year? & No & \textcolor{red}{\ding{55}} \\ \bottomrule
\end{tabular}
\caption{Security Controls Questionnaire Analysis}
\label{tab:controls_review}
\end{table}

% --- Section 4: Technical Scan Results ---
\section{Technical Scan Results}

\textbf{Data Not Available:} The input data file containing the network scan results (\texttt{Input\_1\_Network\_Scan\_JSON}) was found to be corrupted or improperly formatted. Consequently, a technical analysis of open ports, running services, and potential software vulnerabilities on the target IP (\seqsplit{\texttt{152.127.115.78}}) could not be performed.

External network scanning is a crucial step for identifying exploitable vulnerabilities on internet-facing systems. Without this data, the organization's external attack surface remains unassessed.

% --- Section 5: Risk Assessment ---
\section{Risk Assessment}

\textbf{Data Not Available:} The input data file containing pre-existing risks (\texttt{Input\_3\_Current\_Risks\_JSON}) was found to be corrupted. The risk assessment below is therefore based solely on the gaps identified during the Security Control Review (Section 3).

\begin{table}[h!]
\centering
\begin{tabular}{@{}p{0.15\textwidth}p{0.25\textwidth}p{0.4\textwidth}l@{}}
\toprule
\textbf{Risk ID} & \textbf{Risk Name} & \textbf{Overview} & \textbf{Severity} \\ \midrule
RISK-001 & Lack of Endpoint MFA & The absence of MFA for computer logins allows an attacker with stolen credentials to gain direct access to an endpoint and the corporate network. & \textbf{Critical} \\
\addlinespace
RISK-002 & No Security Awareness Training Program & Employees are not trained to recognize or respond to phishing, social engineering, or other common cyber threats, making them highly susceptible targets. & \textbf{Critical} \\
\addlinespace
RISK-003 & Absence of Acceptable Use Policy (AUP) & Without a formal AUP, there are no clear rules for employees regarding the use of company assets, data handling, and internet usage, leading to inconsistent security practices and potential insider threats. & \textbf{High} \\ \bottomrule
\end{tabular}
\caption{Identified Risks from Questionnaire Analysis}
\label{tab:risk_assessment}
\end{table}

% --- Section 6: Recommendations ---
\section{Recommendations}
\label{sec:recommendations}

Based on the analysis, the following actions are recommended to mitigate the identified risks and improve the overall security posture of \textbf{Common Ground}.

\begin{enumerate}
    \item \textbf{Implement Mandatory Endpoint MFA (RISK-001):}
    \begin{itemize}
        \item \textbf{Action:} Deploy a mandatory Multi-Factor Authentication solution for all employee computer and laptop logins. This should apply to both on-premise and remote access.
        \item \textbf{Impact:} Drastically reduces the risk of unauthorized access from compromised credentials, a primary vector for ransomware and data breaches.
    \end{itemize}
    \vspace{0.5cm}
    \item \textbf{Establish a Security Awareness Training Program (RISK-002):}
    \begin{itemize}
        \item \textbf{Action:} Develop and implement a formal, mandatory security awareness training program. The program must include initial training for all new hires and annual refresher training for all staff.
        \item \textbf{Impact:} Creates a "human firewall" by empowering employees to identify and report security threats like phishing, reducing the likelihood of a successful attack.
    \end{itemize}
    \vspace{0.5cm}
    \item \textbf{Develop and Enforce an Acceptable Use Policy (RISK-003):}
    \begin{itemize}
        \item \textbf{Action:} Create a comprehensive Employee Acceptable Use Policy (AUP) that clearly defines the rules for using company technology, accessing data, and internet usage. All employees must read and formally acknowledge the policy.
        \item \textbf{Impact:} Establishes a clear security baseline for employee behavior, reduces insider risk, and provides a foundation for enforcing security standards.
    \end{itemize}
    \vspace{0.5cm}
    \item \textbf{Conduct a Full Technical Assessment:}
    \begin{itemize}
        \item \textbf{Action:} Arrange for a new external network vulnerability scan and a comprehensive review of existing vulnerabilities to address the data gaps in this report.
        \item \textbf{Impact:} Provides a complete view of the organization's security posture, enabling the identification and remediation of technical vulnerabilities that could be exploited by attackers.
    \end{itemize}
\end{enumerate}

\end{document}
```