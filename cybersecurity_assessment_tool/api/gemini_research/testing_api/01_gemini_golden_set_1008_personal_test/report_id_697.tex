```latex
\documentclass[12pt]{article}

% --- PACKAGES ---
\usepackage[margin=1in]{geometry} % Set page margins
\usepackage{pifont}                 % For checkmarks and crosses (\ding)
\usepackage{booktabs}               % For professional-looking tables
\usepackage{graphicx}               % For logos, etc.
\usepackage{xcolor}                 % For custom colors
\usepackage{hyperref}               % For hyperlinks
\usepackage{url}                    % For URL formatting
\usepackage{seqsplit}               % For splitting long strings in texttt
\usepackage{array}                  % For table column definitions

% --- DOCUMENT SETUP ---
\definecolor{darkblue}{rgb}{0.0, 0.0, 0.55}
\definecolor{darkred}{rgb}{0.55, 0.0, 0.0}

\hypersetup{
    colorlinks=true,
    linkcolor=darkblue,
    filecolor=darkblue,      
    urlcolor=darkblue,
    citecolor=darkblue,
}

% --- COMMANDS ---
\newcommand{\yes}{\ding{51}}
\newcommand{\no}{\ding{55}}

% --- TITLE ---
\title{Cybersecurity Assessment Report \\ \large for \\ \textbf{Sovereign Trust}}
\author{Cybersecurity Analyst}
\date{November 22, 2025}

\begin{document}

\maketitle

\begin{abstract}
\noindent This report provides a comprehensive cybersecurity assessment for \textbf{Sovereign Trust}, conducted on November 22, 2025. The analysis is based on a network vulnerability scan, a review of organizational security controls, and an evaluation of existing risks. The assessment identified several critical and high-risk vulnerabilities. Key findings include the absence of Multi-Factor Authentication (MFA) on critical systems like email and sensitive data repositories, a lack of foundational security policies, and an externally-facing web server running outdated software. These issues collectively expose the organization to significant threats, including account compromise, data breaches, and service disruption. This report details these findings and provides prioritized, actionable recommendations to mitigate the identified risks and strengthen the organization's overall security posture.
\end{abstract}

\tableofcontents
\newpage

% ===================================================================
\section{Overview and Executive Summary}
% ===================================================================

This assessment synthesizes technical scan data and organizational security questionnaire responses to provide a holistic view of the security posture of \textbf{Sovereign Trust}.

\paragraph{Key Findings:} The organization exhibits significant gaps in its administrative and technical security controls.
\begin{itemize}
    \item \textbf{Critical Control Gaps:} The lack of Multi-Factor Authentication (MFA) for email and sensitive data systems represents a critical vulnerability. Email is a primary vector for phishing and account takeover attacks, which could serve as a gateway to compromise other systems.
    \item \textbf{Outdated Software:} The primary web server is running an outdated version of Nginx (1.18.0), which is no longer receiving security patches for high-severity vulnerabilities. This exposes the organization's public-facing services to known exploits.
    \item \textbf{Policy Deficiencies:} The absence of an employee acceptable use policy and a formal security training program for new hires indicates a weakness in security governance. This can lead to inconsistent security practices and increased susceptibility to social engineering.
\end{itemize}

\paragraph{Overall Posture:} Based on these findings, the overall security risk posture for \textbf{Sovereign Trust} is assessed as \textbf{High}. Immediate action is required to address the critical vulnerabilities identified in this report to prevent potential security incidents.

% ===================================================================
\section{Organizational Information}
% ===================================================================

The following information was provided for the assessment.

\begin{tabular}{@{}ll}
\toprule
\textbf{Attribute} & \textbf{Value} \\
\midrule
Organization Name & \textbf{Sovereign Trust} \\
Primary Email Domain & \texttt{SovereignTrust.net} \\
Primary Website Domain & \url{www.SovereignTrust.net} \\
External IP Address & \texttt{191.137.79.169} \\
\bottomrule
\end{tabular}

% ===================================================================
\section{Security Control Review}
% ===================================================================

A review of administrative security controls was conducted via a standardized questionnaire. The results below highlight gaps in current security practices. "No" answers indicate a deviation from security best practices and represent a potential risk.

\begin{tabular}{@{}p{0.7\textwidth}c}
\toprule
\textbf{Control Question} & \textbf{Status} \\
\midrule
Do you require MFA to access email? & \textcolor{darkred}{\no} \\
Do you require MFA to log into computers? & \textcolor{green!50!black}{\yes} \\
Do you require MFA to access sensitive data systems? & \textcolor{darkred}{\no} \\
Does your organization have an employee acceptable use policy? & \textcolor{darkred}{\no} \\
Does your organization do security awareness training for new employees? & \textcolor{darkred}{\no} \\
Does your organization do security awareness training for all employees at least once per year? & \textcolor{green!50!black}{\yes} \\
\bottomrule
\end{tabular}

% ===================================================================
\section{Technical Scan Results}
% ===================================================================

An external network scan was performed to identify open ports and exposed services.

\subsection{Nmap Scan Summary}
\begin{itemize}
    \item \textbf{Target IP:} \texttt{192.168.10.5}
    \item \textbf{Scan Date:} 2025-11-22T10:00:00Z
\end{itemize}

\subsection{Open Ports and Services}
The following table details the services discovered on the target system.

\begin{tabular}{@{}lllll@{}}
\toprule
\textbf{Port} & \textbf{State} & \textbf{Service} & \textbf{Product} & \textbf{Version} \\
\midrule
443/tcp & open & https & nginx & 1.18.0 \\
\bottomrule
\end{tabular}

\subsection{Analysis of Findings}
The scan identified a single open port, 443 (HTTPS), running an Nginx web server. 
\paragraph{Outdated Nginx Version:} The identified version, \textbf{Nginx 1.18.0}, was released in April 2020. This is a legacy stable version that is no longer supported with security patches. Numerous vulnerabilities affecting this version have been discovered since its release, including those with high and critical severity ratings (e.g., related to request smuggling and memory disclosure). Running unsupported software on an internet-facing system presents a significant and unnecessary risk of compromise.

% ===================================================================
\section{Consolidated Risk Assessment}
% ===================================================================

The following table consolidates all identified risks from the security control review and technical scan. Risks are prioritized by severity to guide remediation efforts.

\begin{tabular}{@{}lp{0.6\textwidth}l@{}}
\toprule
\textbf{Risk ID} & \textbf{Risk Title \& Description} & \textbf{Severity} \\
\midrule
RISK-001 & \textbf{Lack of MFA on Email Accounts} \newline \textit{Without MFA, an attacker who obtains a user's password can gain full access to their mailbox, leading to data theft, phishing, and further system compromise.} & \textbf{Critical} \\
\addlinespace
RISK-002 & \textbf{Lack of MFA on Sensitive Data Systems} \newline \textit{Critical business and client data is protected only by passwords. A single credential leak could result in a catastrophic data breach.} & \textbf{Critical} \\
\addlinespace
RISK-003 & \textbf{Outdated Nginx Web Server} \newline \textit{The public-facing web server is running an unsupported version with known vulnerabilities, making it a prime target for automated attacks and potential server compromise.} & \textbf{High} \\
\addlinespace
RISK-004 & \textbf{Missing Employee Acceptable Use Policy (AUP)} \newline \textit{The absence of a formal AUP creates ambiguity regarding safe technology use, data handling, and security responsibilities, increasing the risk of insider threat and accidental data loss.} & \textbf{High} \\
\addlinespace
RISK-005 & \textbf{No Security Training for New Hires} \newline \textit{New employees are not formally trained on security policies and threats, making them highly susceptible to social engineering and phishing attacks from their first day.} & \textbf{High} \\
\bottomrule
\end{tabular}

% ===================================================================
\section{Recommendations}
% ===================================================================

The following actions are recommended to mitigate the identified risks. Recommendations are prioritized based on the severity of the corresponding risk.

\begin{enumerate}
    \item \textbf{[Critical] Implement MFA for All Email and Sensitive Systems (RISK-001, RISK-002):}
    \begin{itemize}
        \item Immediately enable and enforce MFA for all user accounts across the email platform (e.g., Microsoft 365, Google Workspace).
        \item Identify all systems classified as containing sensitive data and enforce MFA for access. Prioritize systems accessible from the internet.
    \end{itemize}
    
    \item \textbf{[High] Upgrade the Nginx Web Server (RISK-003):}
    \begin{itemize}
        \item Plan and execute an upgrade of the Nginx server from version 1.18.0 to a current, stable, and fully patched version (e.g., 1.26.x or newer).
        \item Establish a patch management policy to ensure internet-facing systems are updated on a regular basis.
    \end{itemize}
    
    \item \textbf{[High] Develop and Implement Foundational Security Policies (RISK-004):}
    \begin{itemize}
        \item Create a formal Acceptable Use Policy (AUP) that clearly defines the rules for using company assets, data handling, and internet access.
        \item Require all employees to read and acknowledge the policy.
    \end{itemize}
    
    \item \textbf{[High] Establish a Security Onboarding Program (RISK-005):}
    \begin{itemize}
        \item Develop a mandatory security awareness training module for all new hires to be completed during their first week of employment.
        \item This training should cover key topics such as phishing, password security, data handling, and the new AUP.
    \end{itemize}
\end{enumerate}

\end{document}
```