```latex
\documentclass[12pt]{article}

% --- PACKAGES ---
\usepackage[a4paper, margin=1in]{geometry}
\usepackage{pifont} % For checkmarks and crosses
\usepackage{booktabs} % For professional tables
\usepackage{hyperref} % For clickable links
\usepackage{url} % For URL formatting
\usepackage{seqsplit} % To split long strings without breaking
\usepackage{graphicx} % For potential logos
\usepackage{xcolor} % For colors

% --- DOCUMENT METADATA ---
\title{Cybersecurity Posture Assessment Report}
\author{Cybersecurity Analysis Division}
\date{\today}

% --- HYPERREF SETUP ---
\hypersetup{
    colorlinks=true,
    linkcolor=blue,
    filecolor=magenta,      
    urlcolor=cyan,
    pdftitle={Cybersecurity Posture Assessment Report},
    pdfpagemode=FullScreen,
}

\begin{document}

\maketitle
\thispagestyle{empty}
\newpage

\tableofcontents
\newpage

% ==============================================================================
% SECTION 1: EXECUTIVE OVERVIEW
% ==============================================================================
\section{Executive Overview}

This report provides a comprehensive cybersecurity posture assessment for \textbf{Cloud Nine Software}, conducted on \today. The analysis synthesizes data from an external network scan, a security controls questionnaire, and a review of previously identified risks.

Overall, the organization demonstrates a solid foundational security posture, with commendable implementation of Multi-Factor Authentication (MFA) for email and computer access, and established policies for acceptable use and new employee training. 

However, two significant areas of concern were identified that elevate the organization's risk profile:
\begin{enumerate}
    \item \textbf{Critical Gap in Data Protection:} Multi-Factor Authentication is not enforced for accessing sensitive data systems. This represents a critical vulnerability, as a single compromised credential could lead to a significant data breach.
    \item \textbf{High Risk of Human Error:} The lack of mandatory, annual security awareness training for all employees increases susceptibility to phishing, social engineering, and other human-targeted attacks.
\end{enumerate}

On a positive note, a technical scan of the external IP address \texttt{111.68.38.219} indicates that a previously identified risk—an unencrypted web server on Port 80—may have been remediated, as the port was found to be closed. This is a positive development that should be internally verified.

This report details these findings and provides actionable recommendations to mitigate the identified risks and strengthen the overall security posture of \textbf{Cloud Nine Software}.

% ==============================================================================
% SECTION 2: ORGANIZATIONAL INFORMATION
% ==============================================================================
\section{Organizational Information}

The following details were provided for the assessment.

\begin{tabular}{@{}ll}
\toprule
\textbf{Attribute} & \textbf{Value} \\
\midrule
Organization Name & \textbf{Cloud Nine Software} \\
Email Domain & \texttt{CloudNineSoftware.net} \\
Website Domain & \url{www.CloudNineSoftware.net} \\
External IP Scanned & \texttt{111.68.38.219} \\
Internal Target Scanned & \texttt{192.168.0.5} \\
\bottomrule
\end{tabular}

% ==============================================================================
% SECTION 3: SECURITY CONTROL REVIEW
% ==============================================================================
\section{Security Control Review}

A review of the organization's security controls was conducted via a questionnaire. The responses are summarized below. Items marked with \ding{55} indicate a deviation from security best practices and represent a gap in the control framework.

\begin{table}[h!]
\centering
\caption{Security Controls Questionnaire Results}
\begin{tabular}{p{0.6\textwidth} c c}
\toprule
\textbf{Control Question} & \textbf{Response} & \textbf{Assessment} \\
\midrule
Do you require MFA to access email? & Yes & \ding{51} \\
Do you require MFA to log into computers? & Yes & \ding{51} \\
\textbf{Do you require MFA to access sensitive data systems?} & \textbf{No} & \textcolor{red}{\ding{55}} \\
Does your organization have an employee acceptable use policy? & Yes & \ding{51} \\
Does your organization do security awareness training for new employees? & Yes & \ding{51} \\
\textbf{Does your organization do security awareness training for all employees at least once per year?} & \textbf{No} & \textcolor{red}{\ding{55}} \\
\bottomrule
\end{tabular}
\end{table}

The two "No" responses are the primary drivers of the risks identified in this report.

% ==============================================================================
% SECTION 4: TECHNICAL SCAN RESULTS
% ==============================================================================
\section{Technical Scan Results}

An Nmap scan was performed on the internal target IP address \texttt{192.168.0.5}. The purpose of this scan was to identify open ports, running services, and potential vulnerabilities visible from the network.

\subsection{Scan Summary}
The scan results were minimal, indicating a well-hardened host or effective firewall rules.
\begin{itemize}
    \item \textbf{Target IP:} \texttt{192.168.0.5}
    \item \textbf{Host Status:} Up
    \item \textbf{Key Finding:} Port 80 (HTTP) was explicitly identified as being in a \textbf{closed} state. No other open ports were discovered during this scan.
\end{itemize}

\subsection{Analysis}
The finding that Port 80 is closed is significant. It directly contradicts a pre-existing risk entry (\textit{Unencrypted Web Server}) which assumed this port was open. This suggests that either the previous finding was a false positive or remediation actions have since been successfully implemented. This is a positive security development.

% ==============================================================================
% SECTION 5: RISK ASSESSMENT
% ==============================================================================
\section{Risk Assessment}

This section synthesizes findings from the security control review, technical scan, and pre-existing risk data.

\begin{table}[h!]
\centering
\caption{Synthesized Risk Register}
\begin{tabular}{p{0.3\textwidth} p{0.5\textwidth} l}
\toprule
\textbf{Risk Name} & \textbf{Overview} & \textbf{Severity} \\
\midrule
\textbf{Lack of MFA on Sensitive Systems} & Failure to enforce MFA on systems containing critical or sensitive data. A single compromised credential could grant an attacker full access. & \textbf{Critical} \\
\addlinespace
\textbf{Inadequate Security Awareness Training} & Employees do not receive annual, recurring security training. This increases the likelihood of successful phishing and social engineering attacks over time. & \textbf{High} \\
\addlinespace
\textbf{Unencrypted Web Server (Port 80)} & Previous risk data indicated an open HTTP port. The current scan shows this port as \textbf{closed}, suggesting the risk is likely remediated or was a false positive. & \textit{Informational} \\
\bottomrule
\end{tabular}
\end{table}

% ==============================================================================
% SECTION 6: RECOMMENDATIONS
% ==============================================================================
\section{Recommendations}

Based on the analysis, the following actions are recommended to mitigate the identified risks and improve the overall security posture.

\begin{enumerate}
    \item \textbf{[Critical] Enforce MFA on All Sensitive Data Systems:}
    \begin{itemize}
        \item \textbf{Action:} Immediately begin a project to identify all systems, applications, and databases containing sensitive information (e.g., PII, financial data, intellectual property).
        \item \textbf{Implementation:} Procure and deploy an MFA solution that integrates with these systems. Enforce its use for all user accounts, especially those with administrative privileges.
        \item \textbf{Justification:} This is the single most effective control to prevent unauthorized access to the organization's most valuable assets.
    \end{itemize}
    \vspace{1em}
    \item \textbf{[High] Implement a Mandatory Annual Security Training Program:}
    \begin{itemize}
        \item \textbf{Action:} Establish a formal, mandatory security awareness training program for all current employees, to be completed annually.
        \item \textbf{Implementation:} Utilize a third-party training platform or develop in-house content covering key topics such as phishing, password security, and incident reporting. Track completion to ensure 100\% compliance.
        \item \textbf{Justification:} A well-trained workforce is the first line of defense against common cyberattacks. Regular training ensures that security remains a top-of-mind concern.
    \end{itemize}
    \vspace{1em}
    \item \textbf{[Informational] Verify Closure of Port 80:}
    \begin{itemize}
        \item \textbf{Action:} Have the system administration team confirm that Port 80 is intentionally and correctly firewalled on all external-facing systems.
        \item \textbf{Implementation:} Update internal documentation and risk registers to reflect the current state.
        \item \textbf{Justification:} This ensures that risk documentation is accurate and that the port closure was an intentional security measure.
    \end{itemize}
\end{enumerate}

\end{document}
```