```latex
\documentclass[12pt, a4paper]{article}

% Preamble: Required Packages
\usepackage[margin=1in]{geometry}
\usepackage{pifont} % For checkmarks and crosses
\usepackage{booktabs} % For professional tables
\usepackage{hyperref} % For clickable links
\usepackage{url} % For formatting URLs
\usepackage{seqsplit} % To split long text strings
\usepackage{graphicx}
\usepackage{xcolor}

% Document Information
\title{Cybersecurity Posture Assessment Report}
\author{Cybersecurity Analyst}
\date{\today}

% Hyperref Setup
\hypersetup{
    colorlinks=true,
    linkcolor=blue,
    filecolor=magenta,      
    urlcolor=cyan,
    pdftitle={Cybersecurity Posture Assessment Report},
    pdfpagemode=FullScreen,
}

\begin{document}

\maketitle
\thispagestyle{empty}
\newpage

\tableofcontents
\newpage

% --- 1. EXECUTIVE SUMMARY ---
\section{Executive Summary}

This report provides a comprehensive cybersecurity assessment for \textbf{Orchid Isle}, based on an analysis of network scan data, organizational security controls, and pre-existing risk information. The assessment reveals a critically high-risk security posture that requires immediate and decisive action.

Key findings indicate a systemic failure to implement foundational security controls. The complete absence of Multi-Factor Authentication (MFA) across all critical systems, including email and remote access, represents a severe vulnerability. This is dangerously compounded by the technical discovery of an exposed Remote Desktop Protocol (RDP) service on host \texttt{10.10.10.51}. This finding mirrors a previously identified risk on another host, suggesting a recurring and unmitigated vulnerability pattern.

Furthermore, the lack of basic security governance, including an employee acceptable use policy and mandatory security awareness training, creates a fertile ground for human error and social engineering attacks. The combination of these technical and procedural gaps places the organization at an imminent risk of a significant security breach, such as a ransomware attack or data exfiltration.

Immediate remediation is required. The recommendations outlined in this report are prioritized to address the most critical threats first.

% --- 2. ORGANIZATIONAL INFORMATION ---
\section{Organizational Information}

The following details were provided for the assessment.

\begin{tabular}{@{}ll}
\toprule
\textbf{Attribute} & \textbf{Value} \\
\midrule
Organization Name & \textbf{Orchid Isle} \\
Primary Email Domain & \texttt{OrchidIsle.org} \\
Primary Website & \url{www.OrchidIsle.org} \\
External IP Address & \texttt{149.193.25.18} \\
\bottomrule
\end{tabular}

% --- 3. SECURITY CONTROL REVIEW ---
\section{Security Control Review}

A review of the organization's security controls was conducted via a questionnaire. The results below highlight critical deficiencies in fundamental security practices. A checkmark (\ding{51}) indicates a positive control is in place, while a cross (\ding{55}) indicates a gap.

\begin{table}[h!]
\centering
\begin{tabular}{@{}lc}
\toprule
\textbf{Security Control Question} & \textbf{Status} \\
\midrule
Do you require MFA to access email? & \ding{55} \\
Do you require MFA to log into computers? & \ding{55} \\
Do you require MFA to access sensitive data systems? & \ding{55} \\
Does your organization have an employee acceptable use policy? & \ding{55} \\
Does your organization do security awareness training for new employees? & \ding{55} \\
Does your organization do security awareness training annually? & \ding{55} \\
\bottomrule
\end{tabular}
\caption{Organizational Security Control Status}
\end{table}

\paragraph{Analysis:} The complete absence of these foundational controls is alarming. The lack of MFA is the most critical technical gap, as it removes a vital layer of defense against credential theft. The absence of policies and training significantly increases the "human factor" risk, making employees more susceptible to phishing and other social engineering tactics.

% --- 4. TECHNICAL SCAN RESULTS ---
\section{Technical Scan Results}

A network scan was performed on the specified target to identify open ports and exposed services.

\begin{itemize}
    \item \textbf{Target IP:} \texttt{10.10.10.51}
    \item \textbf{Scan Date:} \today
\end{itemize}

The following table details the open ports discovered on the target host.

\begin{table}[h!]
\centering
\begin{tabular}{@{}llll@{}}
\toprule
\textbf{Port} & \textbf{State} & \textbf{Service} & \textbf{Analysis} \\
\midrule
3389/tcp & open & ms-wbt-server & \textbf{Critical Risk}. This is the port for Remote \\
& & (RDP) & Desktop Protocol. Exposing RDP directly to \\
& & & the network is a primary vector for ransomware \\
& & & and unauthorized access. \\
\bottomrule
\end{tabular}
\caption{Open Ports on \texttt{10.10.10.51}}
\end{table}

% --- 5. CORRELATED RISK ASSESSMENT ---
\section{Correlated Risk Assessment}

This section synthesizes the findings from the security control review, technical scan, and pre-existing risk data to provide a holistic view of the organization's risk profile.

\begin{table}[h!]
\centering
\begin{tabular}{@{}p{0.25\linewidth}p{0.5\linewidth}p{0.15\linewidth}@{}}
\toprule
\textbf{Risk Name} & \textbf{Overview} & \textbf{Severity} \\
\midrule
\textbf{Systemic RDP Exposure} & The technical scan confirmed RDP is exposed on \texttt{10.10.10.51}. This correlates with a known, existing risk of RDP exposure on \texttt{10.10.10.50}. This indicates a pattern of insecure configuration and a failure to remediate known high-risk vulnerabilities. & \textbf{Critical} \\
\addlinespace
\textbf{Lack of Multi-Factor Authentication (MFA)} & The questionnaire confirmed a complete lack of MFA. When combined with exposed services like RDP, this drastically lowers the barrier for an attacker to gain access via brute-force or stolen credentials. & \textbf{Critical} \\
\addlinespace
\textbf{Absence of Security Governance} & The organization lacks an acceptable use policy and security awareness training. This increases the likelihood of employees using weak passwords or falling for phishing attacks, which could lead to credential compromise and exploitation of the other identified risks. & \textbf{High} \\
\bottomrule
\end{tabular}
\caption{Summary of Correlated Risks}
\end{table}

% --- 6. RECOMMENDATIONS ---
\section{Recommendations}

The following actions are recommended to mitigate the identified risks. They are prioritized based on severity and potential impact.

\subsection{Immediate Actions (Critical Priority)}

\begin{enumerate}
    \item \textbf{Remediate RDP Exposure:}
    \begin{itemize}
        \item Immediately block all inbound traffic to TCP port 3389 on all affected systems (\texttt{10.10.10.51}, \texttt{10.10.10.50}, and any others) at the network firewall.
        \item If remote access is a business requirement, implement a Virtual Private Network (VPN) solution. Access to the internal network should only be granted after a user successfully authenticates to the VPN.
    \end{itemize}

    \item \textbf{Implement Multi-Factor Authentication (MFA):}
    \begin{itemize}
        \item Immediately enable MFA for all users on the primary email system (\texttt{OrchidIsle.org}).
        \item Enforce MFA on any remote access solution (e.g., the new VPN).
        \item Develop a plan to roll out MFA for all sensitive system access and computer logins.
    \end{itemize}
\end{enumerate}

\subsection{High Priority Actions}

\begin{enumerate}
    \setcounter{enumi}{2} % Continue numbering
    \item \textbf{Develop Foundational Security Policies:}
    \begin{itemize}
        \item Draft and implement an \textbf{Acceptable Use Policy (AUP)} that all employees must read and sign. This policy should govern the use of company assets, data handling, and internet usage.
        \item Create a \textbf{Password Policy} that enforces complexity, length, and rotation requirements.
    \end{itemize}

    \item \textbf{Implement Security Awareness Training:}
    \begin{itemize}
        \item Procure and deploy a security awareness training program for all employees.
        \item Training should be mandatory for new hires and conducted at least annually for all staff.
        \item Key topics must include phishing identification, strong password creation, and the importance of the new AUP.
    \end{itemize}
\end{enumerate}

\end{document}
```