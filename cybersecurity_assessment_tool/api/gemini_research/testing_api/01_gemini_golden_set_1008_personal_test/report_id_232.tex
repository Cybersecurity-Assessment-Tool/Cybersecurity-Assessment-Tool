```latex
\documentclass[12pt]{article}

% --- PACKAGES ---
\usepackage[margin=1in]{geometry}
\usepackage{pifont} % For checkmarks and crosses
\usepackage{booktabs} % For professional tables
\usepackage{hyperref} % For clickable links
\usepackage{url} % For URL formatting
\usepackage{seqsplit} % To split long strings without breaking
\usepackage[T1]{fontenc}

% --- DOCUMENT METADATA ---
\title{Cybersecurity Posture Assessment Report}
\author{Cybersecurity Analysis Division}
\date{\today}

% --- HYPERREF SETUP ---
\hypersetup{
    colorlinks=true,
    linkcolor=black,
    urlcolor=blue,
    pdftitle={Cybersecurity Posture Assessment Report},
    pdfauthor={Cybersecurity Analysis Division},
}

% --- BEGIN DOCUMENT ---
\begin{document}

\maketitle
\thispagestyle{empty}
\newpage

\tableofcontents
\thispagestyle{empty}
\newpage

\setcounter{page}{1}

% ==============================================================================
\section*{Executive Summary}
% ==============================================================================
This report provides a comprehensive cybersecurity posture assessment for \textbf{Iron River Finance}, conducted on \today. The analysis synthesizes data from an external network scan, a security controls questionnaire, and a review of pre-existing risks.

The assessment identified a mixed security posture. On a technical level, the scanned external asset (\texttt{192.168.1.100}) demonstrates a strong defensive configuration, with no open ports detected. This indicates effective firewall management and adherence to the principle of least privilege at the network perimeter.

However, significant administrative and access control gaps were identified through the security controls review. The two most critical findings are the lack of Multi-Factor Authentication (MFA) for sensitive data systems and the absence of an employee Acceptable Use Policy (AUP). These gaps represent a \textbf{Critical} and \textbf{High} risk, respectively, as they expose the organization to potential data breaches via compromised credentials and increase the likelihood of insider threats or misuse of company assets.

Immediate remediation should focus on implementing MFA for all sensitive systems and developing a formal AUP to be distributed to all employees.

% ==============================================================================
\section*{1. Organizational Information}
% ==============================================================================
The following information was provided for the assessment:
\begin{itemize}
    \item \textbf{Organization Name:} Iron River Finance
    \item \textbf{Email Domain:} \texttt{IronRiverFinance.com}
    \item \textbf{Website Domain:} \url{www.IronRiverFinance.com}
    \item \textbf{External IP Address:} \texttt{58.82.253.71}
\end{itemize}

% ==============================================================================
\section*{2. Security Control Review}
% ==============================================================================
A review of the organization's security controls was conducted via a questionnaire. The responses indicate key areas of strength and weakness in the current security program. Items marked with \ding{55} represent significant gaps that require immediate attention.

\begin{table}[h!]
\centering
\caption{Security Controls Questionnaire Results}
\begin{tabular}{p{0.75\linewidth} c}
\toprule
\textbf{Control Question} & \textbf{Response} \\
\midrule
Do you require MFA to access email? & \ding{51} \\
Do you require MFA to log into computers? & \ding{51} \\
\textbf{Do you require MFA to access sensitive data systems?} & \textbf{\ding{55}} \\
\textbf{Does your organization have an employee acceptable use policy?} & \textbf{\ding{55}} \\
Does your organization do security awareness training for new employees? & \ding{51} \\
Does your organization do security awareness training for all employees at least once per year? & \ding{51} \\
\bottomrule
\end{tabular}
\end{table}

\paragraph{Analysis:} The organization has successfully implemented MFA for email and computer access, which is a strong foundational control. However, the lack of MFA on sensitive data systems is a critical oversight. Additionally, the absence of an Acceptable Use Policy (AUP) is a high-risk governance gap, as it fails to establish clear rules for employee behavior regarding IT assets.

% ==============================================================================
\section*{3. Technical Scan Results}
% ==============================================================================
An external network scan was performed to identify exposed services and potential vulnerabilities on the perimeter.

\begin{itemize}
    \item \textbf{Target IP Address:} \texttt{192.168.1.100}
    \item \textbf{Scan Date:} \today
    \item \textbf{Host Status:} UP
\end{itemize}

\subsection*{Port Scan Findings}
The scan revealed that the target host is online and responsive. However, \textbf{no open ports were detected}. All 1000 of the most common TCP ports were found to be in a `closed` state.

\paragraph{Analysis:} This is an excellent security finding. It indicates that the external firewall is properly configured to deny all unsolicited inbound traffic, adhering to the principle of least privilege. This significantly reduces the attack surface of the target system from the internet.

% ==============================================================================
\section*{4. Risk Assessment Summary}
% ==============================================================================
This section correlates findings from the security control review, technical scans, and pre-existing risk data. No pre-existing vulnerabilities were reported. The primary risks identified stem from policy and access control gaps.

\begin{table}[h!]
\centering
\caption{Identified Risks}
\begin{tabular}{p{0.1\linewidth} p{0.25\linewidth} p{0.45\linewidth} p{0.1\linewidth}}
\toprule
\textbf{Risk ID} & \textbf{Risk Name} & \textbf{Overview} & \textbf{Severity} \\
\midrule
RISK-001 & Lack of MFA on Sensitive Systems & User accounts for critical systems holding sensitive data are protected only by a password. This elevates the risk of unauthorized access from credential theft or phishing attacks. & \textbf{Critical} \\
\addlinespace
RISK-002 & Absence of Acceptable Use Policy & The lack of a formal AUP creates ambiguity regarding the secure and appropriate use of company IT resources, increasing the risk of insider threat, data leakage, and non-compliance. & \textbf{High} \\
\bottomrule
\end{tabular}
\end{table}

% ==============================================================================
\section*{5. Recommendations}
% ==============================================================================
Based on the findings of this assessment, the following actions are recommended to mitigate the identified risks and improve the overall security posture of \textbf{Iron River Finance}.

\subsection*{Immediate Actions (1-30 Days)}
\begin{enumerate}
    \item \textbf{Implement MFA for Sensitive Systems (RISK-001):}
    \begin{itemize}
        \item \textbf{Action:} Enforce Multi-Factor Authentication on all applications, databases, and administrative interfaces that process or store sensitive customer, financial, or proprietary data.
        \item \textbf{Impact:} Drastically reduces the risk of unauthorized access due to compromised credentials. This is the highest priority remediation task.
    \end{itemize}
    \item \textbf{Develop and Implement an Acceptable Use Policy (RISK-002):}
    \begin{itemize}
        \item \textbf{Action:} Draft a formal AUP that clearly defines the rules for using company networks, computers, and data. The policy should be reviewed by legal/HR, and all employees must read and formally acknowledge it.
        \item \textbf{Impact:} Establishes a clear security baseline for all employees, reduces legal liability, and mitigates risks associated with insider threats and misuse of assets.
    \end{itemize}
\end{enumerate}

\subsection*{Sustaining Actions (Ongoing)}
\begin{enumerate}
    \item \textbf{Maintain Strong Perimeter Security:}
    \begin{itemize}
        \item \textbf{Action:} Continue the excellent practice of maintaining a restrictive firewall policy. Regularly audit firewall rules to ensure they remain aligned with business needs and the principle of least privilege.
        \item \textbf{Impact:} Ensures the external attack surface remains minimal.
    \end{itemize}
    \item \textbf{Annual Policy and Control Review:}
    \begin{itemize}
        \item \textbf{Action:} Conduct an annual review of all security policies and technical controls to ensure they are still effective and relevant to the current threat landscape and business operations.
        \item \textbf{Impact:} Promotes a culture of continuous security improvement.
    \end{itemize}
\end{enumerate}

\end{document}
```