```latex
\documentclass[12pt]{article}

% --- PACKAGES ---
\usepackage[margin=1in]{geometry}
\usepackage{pifont} % For checkmarks and crosses
\usepackage{booktabs} % For professional tables
\usepackage{hyperref} % For clickable links
\usepackage{url} % For formatting URLs
\usepackage{seqsplit} % For splitting long strings in \texttt
\usepackage{xcolor} % For colors

% --- DOCUMENT METADATA ---
\title{Cybersecurity Posture Assessment Report}
\author{Cybersecurity Analyst}
\date{\today}

% --- HYPERREF SETUP ---
\hypersetup{
    colorlinks=true,
    linkcolor=blue,
    filecolor=magenta,      
    urlcolor=cyan,
    pdftitle={Cybersecurity Posture Assessment Report},
    pdfpagemode=FullScreen,
}

% --- DOCUMENT START ---
\begin{document}

\maketitle
\thispagestyle{empty}
\newpage

\tableofcontents
\newpage

% ==============================================================================
% 1. EXECUTIVE SUMMARY
% ==============================================================================
\section{Executive Summary}

This report provides a cybersecurity assessment for \textbf{Orchid Isle}, conducted on \today. The analysis synthesizes data from a network vulnerability scan, a security controls questionnaire, and a review of pre-existing risks.

The assessment reveals several critical and high-risk security gaps that require immediate attention. Key findings include:
\begin{itemize}
    \item \textbf{Critical Gaps in Access Control:} Multi-Factor Authentication (MFA) is not enforced for accessing email or sensitive data systems. This significantly increases the risk of account compromise and subsequent data breaches.
    \item \textbf{Critical Policy Deficiency:} The organization lacks a formal Acceptable Use Policy (AUP), leading to an absence of clear guidelines for employees on the secure use of company assets.
    \item \textbf{Confirmed Technical Vulnerability:} The network scan confirmed a pre-existing critical risk, "Localhost Exposed," with an open SSH port (22) detected on the local loopback interface (\texttt{127.0.0.1}).
\end{itemize}

The combination of weak access controls, missing foundational policies, and a confirmed high-impact technical vulnerability indicates a fragile security posture. Immediate remediation of the identified issues is strongly recommended to mitigate the substantial risk of a security incident.

% ==============================================================================
% 2. ORGANIZATIONAL INFORMATION
% ==============================================================================
\section{Organizational Information}

The following details were provided for the assessment.

\begin{table}[h!]
\centering
\begin{tabular}{@{}ll@{}}
\toprule
\textbf{Attribute} & \textbf{Value} \\
\midrule
Organization Name & \textbf{Orchid Isle} \\
Email Domain & \texttt{OrchidIsle.com} \\
Website Domain & \texttt{www.OrchidIsle.com} \\
External IP Address & \texttt{33.131.175.37} \\
\bottomrule
\end{tabular}
\caption{Client Organizational Details}
\end{table}

% ==============================================================================
% 3. SECURITY CONTROL REVIEW
% ==============================================================================
\section{Security Control Review}

A review of the organization's security controls was conducted via a questionnaire. The responses highlight significant gaps in fundamental security practices. A "No" answer indicates a missing control and a potential area of high risk.

\begin{table}[h!]
\centering
\begin{tabular}{@{}p{0.75\textwidth}c@{}}
\toprule
\textbf{Control Question} & \textbf{Response} \\
\midrule
Do you require MFA to access email? & \textcolor{red}{\ding{55}} \\
Do you require MFA to log into computers? & \textcolor{green}{\ding{51}} \\
Do you require MFA to access sensitive data systems? & \textcolor{red}{\ding{55}} \\
Does your organization have an employee acceptable use policy? & \textcolor{red}{\ding{55}} \\
Does your organization do security awareness training for new employees? & \textcolor{green}{\ding{51}} \\
Does your organization do security awareness training for all employees at least once per year? & \textcolor{green}{\ding{51}} \\
\bottomrule
\end{tabular}
\caption{Security Controls Questionnaire Results}
\end{table}

\subsection*{Analysis of Control Gaps}
\begin{itemize}
    \item \textbf{MFA for Email and Sensitive Data:} The absence of MFA on email and sensitive data systems represents a critical vulnerability. Email is a primary target for phishing and account takeover attacks, which can serve as a gateway to the entire organization.
    \item \textbf{Acceptable Use Policy (AUP):} The lack of an AUP means there are no formally documented rules for employees regarding the use of corporate technology and data. This can lead to unintentional security breaches and complicates enforcement of security standards.
\end{itemize}

% ==============================================================================
% 4. TECHNICAL SCAN RESULTS
% ==============================================================================
\section{Technical Scan Results}

A network scan was performed to identify open ports and services. The scan confirmed the presence of a service running on the local loopback interface, corroborating a known risk.

\begin{itemize}
    \item \textbf{Scan Target:} \texttt{127.0.0.1}
    \item \textbf{Scan Tool:} Nmap
\end{itemize}

\begin{table}[h!]
\centering
\begin{tabular}{@{}lllll@{}}
\toprule
\textbf{Port} & \textbf{State} & \textbf{Service} & \textbf{Product} & \textbf{Version} \\
\midrule
22/tcp & open & ssh & \textit{N/A} & \textit{N/A} \\
\bottomrule
\end{tabular}
\caption{Open Ports Detected on \texttt{127.0.0.1}}
\end{table}

\subsection*{Analysis of Technical Findings}
The scan identified that port 22, commonly used for the Secure Shell (SSH) protocol, is open. While the scan was performed on the localhost address (\texttt{127.0.0.1}), this finding directly validates the pre-existing risk "Localhost Exposed." If this machine has any misconfigurations (e.g., in Docker, networking, or firewall rules), this local service could potentially be exposed to external actors. Further investigation is required to determine the purpose of this service and ensure it is securely configured.

% ==============================================================================
% 5. CONSOLIDATED RISK ASSESSMENT
% ==============================================================================
\section{Consolidated Risk Assessment}

The following table consolidates findings from the security questionnaire, technical scan, and pre-existing risk data into a unified view of the organization's current risk posture.

\begin{table}[h!]
\centering
\begin{tabular}{@{}p{0.3\textwidth}p{0.45\textwidth}l@{}}
\toprule
\textbf{Risk / Vulnerability} & \textbf{Description} & \textbf{Severity} \\
\midrule
\textbf{Localhost Exposed} & The SSH service (port 22) is open on the local interface, confirming a known risk. This could be exploited if network misconfigurations exist. & \textbf{Critical} \\
\addlinespace
\textbf{No MFA on Email} & Lack of a second authentication factor for email access makes accounts highly susceptible to phishing and credential stuffing attacks. & \textbf{Critical} \\
\addlinespace
\textbf{No MFA on Sensitive Data} & Critical business and customer data lacks a fundamental access control, drastically increasing the impact of a potential account compromise. & \textbf{Critical} \\
\addlinespace
\textbf{Missing Acceptable Use Policy} & The absence of a formal AUP creates ambiguity for employees and a lack of enforceable security standards for system usage. & \textbf{High} \\
\bottomrule
\end{tabular}
\caption{Summary of Identified Risks}
\end{table}

% ==============================================================================
% 6. RECOMMENDATIONS
% ==============================================================================
\section{Recommendations}

Based on the findings, the following actions are recommended to improve the security posture of \textbf{Orchid Isle}. Recommendations are prioritized by severity.

\begin{enumerate}
    \item \textbf{Implement MFA for Email (Immediate):}
    \begin{itemize}
        \item \textbf{Action:} Enforce mandatory MFA for all user accounts accessing the corporate email system.
        \item \textbf{Impact:} Drastically reduces the risk of email account takeovers, phishing success, and business email compromise (BEC).
    \end{itemize}
    
    \item \textbf{Implement MFA for Sensitive Systems (Immediate):}
    \begin{itemize}
        \item \textbf{Action:} Identify all systems containing sensitive data and enforce MFA for all access, especially for administrative accounts.
        \item \textbf{Impact:} Protects the organization's most valuable data assets from unauthorized access.
    \end{itemize}

    \item \textbf{Investigate and Secure Exposed SSH Service (High Priority):}
    \begin{itemize}
        \item \textbf{Action:} Determine why the SSH service is running on \texttt{127.0.0.1}. If not required, disable it. If required, ensure it is not exposed externally and is configured securely (e.g., disable root login, use key-based authentication, ensure software is patched).
        \item \textbf{Impact:} Mitigates the risk of a critical service being exploited.
    \end{itemize}
    
    \item \textbf{Develop and Implement an Acceptable Use Policy (High Priority):}
    \begin{itemize}
        \item \textbf{Action:} Draft, approve, and disseminate an AUP that clearly defines the rules for using company networks, devices, and data. Require all employees to read and acknowledge the policy.
        \item \textbf{Impact:} Establishes a baseline for secure employee behavior and provides a framework for security governance.
    \end{itemize}
\end{enumerate}

\end{document}
```