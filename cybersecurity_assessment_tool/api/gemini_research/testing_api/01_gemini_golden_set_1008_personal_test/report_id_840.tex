```latex
\documentclass[12pt]{article}

% Preamble: Required Packages
\usepackage[margin=1in]{geometry} % Set page margins
\usepackage{pifont}               % For checkmark and cross symbols (\ding)
\usepackage{booktabs}             % For professional-looking tables
\usepackage{graphicx}             % For potential logos/images
\usepackage{xcolor}               % For custom colors
\usepackage{hyperref}             % For hyperlinks
\usepackage{url}                  % For URL formatting
\usepackage{seqsplit}             % For splitting long strings without spaces

% Document Metadata
\title{Cybersecurity Posture Assessment Report}
\author{Cybersecurity Analysis Division}
\date{\today}

% Hyperref Setup
\hypersetup{
    colorlinks=true,
    linkcolor=blue,
    filecolor=magenta,      
    urlcolor=cyan,
    pdftitle={Cybersecurity Posture Assessment Report},
    pdfpagemode=FullScreen,
}

\begin{document}

\maketitle
\thispagestyle{empty}
\newpage

\tableofcontents
\newpage

% --- 1. Executive Summary ---
\section{Executive Summary}
This report provides a cybersecurity posture assessment for \textbf{Skyward Bound}, based on organizational data provided and an attempted technical network scan. The analysis reveals several critical and high-risk gaps in the organization's foundational security controls.

The primary findings stem from the security controls questionnaire, as the provided technical scan data and pre-existing risk data were found to be corrupted and unusable for this assessment. Key identified risks include:
\begin{itemize}
    \item \textbf{Critical Risk:} Lack of mandatory Multi-Factor Authentication (MFA) on email accounts, exposing the organization to significant risk of business email compromise and phishing attacks.
    \item \textbf{High Risk:} Absence of a formal employee Acceptable Use Policy (AUP), leading to ambiguity in security responsibilities and acceptable behavior.
    \item \textbf{High Risk:} A complete lack of a security awareness training program for both new and existing employees, which drastically increases the susceptibility to social engineering and human error.
\end{itemize}

While the organization has implemented MFA for computer and sensitive system access, the identified gaps in email security and security governance represent an urgent threat. Immediate remediation of these policy and training-based controls is strongly recommended to establish a baseline security posture and reduce the overall attack surface. A new, validated technical scan is required to assess infrastructure-level vulnerabilities.

% --- 2. Organizational Information ---
\section{Organizational Information}
The following details were provided by the client for the scope of this assessment.

\begin{table}[h!]
\centering
\begin{tabular}{@{}ll@{}}
\toprule
\textbf{Attribute} & \textbf{Value} \\ \midrule
Organization Name    & \textbf{Skyward Bound} \\
Email Domain         & \texttt{SkywardBound.com} \\
Website Domain       & \seqsplit{\url{www.SkywardBound.com}} \\
External IP Address  & \texttt{105.160.161.181} \\ \bottomrule
\end{tabular}
\caption{Client Organizational Details.}
\label{tab:org_info}
\end{table}

% --- 3. Security Control Review ---
\section{Security Control Review}
The following table summarizes the organization's responses to a security controls questionnaire. Items marked with a red cross (\ding{55}) indicate a deviation from security best practices and represent a potential risk.

\begin{table}[h!]
\centering
\begin{tabular}{@{}p{0.6\textwidth}cc@{}}
\toprule
\textbf{Security Control Question} & \textbf{Response} & \textbf{Status} \\ \midrule
Do you require MFA to access email? & No & \textcolor{red}{\ding{55}} \\
Do you require MFA to log into computers? & Yes & \textcolor{green}{\ding{51}} \\
Do you require MFA to access sensitive data systems? & Yes & \textcolor{green}{\ding{51}} \\
Does your organization have an employee acceptable use policy? & No & \textcolor{red}{\ding{55}} \\
Does your organization do security awareness training for new employees? & No & \textcolor{red}{\ding{55}} \\
Does your organization do security awareness training for all employees at least once per year? & No & \textcolor{red}{\ding{55}} \\ \bottomrule
\end{tabular}
\caption{Security Controls Questionnaire Analysis.}
\label{tab:controls_review}
\end{table}

% --- 4. Technical Scan Results ---
\section{Technical Scan Results}
The data provided for the technical network scan (\texttt{Input\_1\_Network\_Scan\_JSON}) was found to be \textbf{corrupted and incomplete}. Consequently, it was not possible to perform an analysis of open ports, running services, or potential software vulnerabilities on the target host \texttt{[Target IP]}.

A comprehensive technical assessment is a critical component of understanding the external attack surface. Without this data, the organization has a significant blind spot regarding its infrastructure-level vulnerabilities. It is imperative that a new network scan is conducted against the target IP address as soon as possible.

% --- 5. Risk Assessment ---
\section{Risk Assessment}
This risk assessment is based exclusively on the findings from the Security Control Review due to the unavailability of technical scan data and pre-existing risk logs (\texttt{Input\_3\_Current\_Risks\_JSON}). The identified risks are foundational and significantly impact the organization's overall security posture.

\begin{table}[h!]
\centering
\begin{tabular}{@{}p{0.15\textwidth}p{0.65\textwidth}l@{}}
\toprule
\textbf{Risk ID} & \textbf{Risk Name \& Overview} & \textbf{Severity} \\ \midrule
\textbf{RISK-001} & \textbf{Lack of MFA on Email} \newline \small{Without MFA, email accounts are vulnerable to takeover via stolen or weak credentials. This is a primary vector for phishing, data exfiltration, and business email compromise (BEC).} & \textcolor{red}{\textbf{Critical}} \\
\addlinespace
\textbf{RISK-002} & \textbf{No Employee Acceptable Use Policy (AUP)} \newline \small{The absence of a formal AUP creates ambiguity regarding safe technology use. It weakens the organization's ability to enforce security standards and hold individuals accountable for risky behavior.} & \textcolor{orange}{\textbf{High}} \\
\addlinespace
\textbf{RISK-003} & \textbf{No Security Awareness Training Program} \newline \small{Employees are not trained to recognize or respond to common cyber threats like phishing or social engineering. This makes them the weakest link and a primary target for attackers.} & \textcolor{orange}{\textbf{High}} \\ \bottomrule
\end{tabular}
\caption{Summary of Identified Risks.}
\label{tab:risk_summary}
\end{table}

% --- 6. Recommendations ---
\section{Recommendations}
The following actions are recommended to mitigate the identified risks and improve the overall security posture of \textbf{Skyward Bound}. Recommendations are prioritized based on risk severity.

\begin{enumerate}
    \item \textbf{[Critical] Implement MFA for Email Immediately:}
    \begin{itemize}
        \item \textbf{Action:} Enforce mandatory Multi-Factor Authentication (MFA) for all user accounts on the \texttt{SkywardBound.com} email platform.
        \item \textbf{Justification:} This is the single most effective control to prevent unauthorized account access and mitigate the risk of business email compromise. It should be treated as the highest priority remediation task.
    \end{itemize}
    \vspace{0.5cm}
    \item \textbf{[High] Develop and Implement a Security Awareness Training Program:}
    \begin{itemize}
        \item \textbf{Action:} Establish a formal security awareness training program. This must include mandatory training for all new employees during onboarding and annual refresher training for all staff.
        \item \textbf{Justification:} A well-trained workforce is a critical layer of defense. Training reduces the likelihood of successful phishing attacks and other forms of social engineering.
    \end{itemize}
    \vspace{0.5cm}
    \item \textbf{[High] Create and Enforce an Acceptable Use Policy (AUP):}
    \begin{itemize}
        \item \textbf{Action:} Draft a comprehensive AUP that clearly defines the rules and expectations for the use of company technology, networks, and data. Require all employees to read and formally acknowledge the policy.
        \item \textbf{Justification:} An AUP establishes a clear security baseline for all employees, reduces insider threats (both malicious and accidental), and provides a legal framework for enforcing security rules.
    \end{itemize}
    \vspace{0.5cm}
    \item \textbf{[Procedural] Conduct a New Technical Network Scan:}
    \begin{itemize}
        \item \textbf{Action:} Schedule and execute a new, validated external network scan against the organization's public-facing IP addresses, including \texttt{105.160.161.181}.
        \item \textbf{Justification:} To gain a complete picture of the security posture, a technical assessment is essential. This will identify vulnerabilities in network services, firewalls, and applications that are not visible through policy review alone.
    \end{itemize}
\end{enumerate}

\end{document}
```