```latex
\documentclass[12pt]{article}

% Preamble: Required Packages
\usepackage[margin=1in]{geometry}
\usepackage{pifont} % For checkmark and X symbols
\usepackage{booktabs} % For professional tables
\usepackage[hidelinks]{hyperref} % For clickable links
\usepackage{url} % For URL formatting
\usepackage{seqsplit} % To split long strings without breaking words
\usepackage{graphicx} % For potential logos/images
\usepackage{fancyhdr} % For headers/footers

% --- Document Metadata ---
\title{Cybersecurity Posture Assessment Report \\ \large For: \textbf{Maple Leaf Logistics}}
\author{Cybersecurity Analysis Division}
\date{\today}

% --- Page Style ---
\pagestyle{fancy}
\fancyhf{}
\lhead{Confidential Report}
\rhead{\textbf{Maple Leaf Logistics}}
\cfoot{\thepage}

\begin{document}

\maketitle
\thispagestyle{empty}
\newpage

\tableofcontents
\newpage

% --- Section 1: Executive Overview ---
\section{Executive Overview}

This report provides a cybersecurity assessment for \textbf{Maple Leaf Logistics}, based on an analysis of network scan data, organizational security controls, and pre-existing risk information.

The assessment reveals a mixed security posture. While the organization has implemented foundational policies such as acceptable use and security awareness training, several critical vulnerabilities exist that expose the organization to significant threats, including ransomware and data breaches.

Key findings include:
\begin{itemize}
    \item \textbf{Critical Lack of Multi-Factor Authentication (MFA):} MFA is not enforced for accessing email or for logging into company computers. This represents a severe security gap, as compromised credentials could grant an attacker direct access to sensitive communications and internal systems.
    \item \textbf{Insecure Remote Access Protocol Exposure:} A network scan identified an open Remote Desktop Protocol (RDP) port (3389) on an internal system (\texttt{10.10.10.51}). This finding, correlated with a pre-existing risk of RDP exposure on another host, indicates a potential systemic configuration weakness. Exposed RDP is a primary attack vector for ransomware groups.
\end{itemize}

Immediate remediation of these high-risk vulnerabilities is strongly recommended to reduce the likelihood of a successful cyberattack. Actionable recommendations are detailed in Section \ref{sec:recommendations}.

% --- Section 2: Organizational Information ---
\section{Organizational Information}

The following details were provided for the assessment:

\begin{tabular}{@{}ll}
    \toprule
    \textbf{Attribute} & \textbf{Value} \\
    \midrule
    Organization Name & \textbf{Maple Leaf Logistics} \\
    Email Domain & \texttt{MapleLeafLogistics.org} \\
    Website Domain & \url{www.MapleLeafLogistics.org} \\
    External IP Address & \texttt{19.196.116.21} \\
    \bottomrule
\end{tabular}

% --- Section 3: Security Control Review ---
\section{Security Control Review}

An analysis of the organization's security questionnaire responses highlights key strengths and weaknesses in its current control environment. "No" answers indicate significant gaps that increase risk.

\begin{table}[h!]
\centering
\caption{Security Controls Questionnaire Analysis}
\begin{tabular}{@{}p{0.6\linewidth}ccp{0.2\linewidth}@{}}
    \toprule
    \textbf{Control Question} & \textbf{Response} & \textbf{Status} & \textbf{Assessment} \\
    \midrule
    Do you require MFA to access email? & No & \ding{55} & \textbf{Critical Gap} \\
    Do you require MFA to log into computers? & No & \ding{55} & \textbf{Critical Gap} \\
    Do you require MFA to access sensitive data systems? & Yes & \ding{51} & Best Practice Met \\
    Does your organization have an employee acceptable use policy? & Yes & \ding{51} & Best Practice Met \\
    Does your organization do security awareness training for new employees? & Yes & \ding{51} & Best Practice Met \\
    Does your organization do security awareness training for all employees at least once per year? & Yes & \ding{51} & Best Practice Met \\
    \bottomrule
\end{tabular}
\end{table}

% --- Section 4: Technical Scan Results ---
\section{Technical Scan Results}

A network scan was conducted on the specified target to identify open ports and exposed services.

\begin{itemize}
    \item \textbf{Target IP Address:} \texttt{10.10.10.51}
\end{itemize}

\begin{table}[h!]
\centering
\caption{Open Port Analysis for \texttt{10.10.10.51}}
\begin{tabular}{@{}llll@{}}
    \toprule
    \textbf{Port} & \textbf{State} & \textbf{Service Name} & \textbf{Analysis} \\
    \midrule
    3389/tcp & Open & \texttt{ms-wbt-server} & This is the default port for Microsoft Remote \\
    & & & Desktop Protocol (RDP). Exposing RDP directly \\
    & & & to any network is a high-risk configuration \\
    & & & and a common entry point for attackers. \\
    \bottomrule
\end{tabular}
\end{table}

% --- Section 5: Correlated Risk Assessment ---
\section{Correlated Risk Assessment}

This section synthesizes findings from the questionnaire, technical scan, and pre-existing risk data into a consolidated list of identified risks.

\begin{table}[h!]
\centering
\caption{Summary of Identified Risks}
\begin{tabular}{@{}p{0.1\linewidth}p{0.25\linewidth}p{0.45\linewidth}l@{}}
    \toprule
    \textbf{Risk ID} & \textbf{Risk Name} & \textbf{Description} & \textbf{Severity} \\
    \midrule
    RISK-001 & Lack of MFA for Email and Endpoints & The absence of MFA on critical access points like email and computer logins means a single compromised password can lead to a full breach. & \textbf{Critical} \\
    \addlinespace
    RISK-002 & Systemic RDP Exposure & The scan identified open RDP on \texttt{10.10.10.51}. This correlates with a pre-existing risk on \texttt{10.10.10.50}, indicating a systemic vulnerability. This is a primary vector for ransomware. & \textbf{Critical} \\
    \bottomrule
\end{tabular}
\end{table}

% --- Section 6: Recommendations ---
\section{Recommendations}
\label{sec:recommendations}

Based on the analysis, the following actions are recommended to mitigate the identified risks and improve the overall security posture of \textbf{Maple Leaf Logistics}.

\subsection{Immediate Priority (Remediate within 7 Days)}
\begin{enumerate}
    \item \textbf{Remediate RDP Exposure (RISK-002):}
    \begin{itemize}
        \item Immediately implement a firewall rule to block all inbound traffic to TCP port 3389 on \texttt{10.10.10.51} and \texttt{10.10.10.50}.
        \item If remote access is a business requirement, deploy a secure remote access solution, such as a Virtual Private Network (VPN) that requires Multi-Factor Authentication.
    \end{itemize}

    \item \textbf{Implement MFA for Email (RISK-001):}
    \begin{itemize}
        \item Enable and enforce MFA for all user accounts on the \texttt{MapleLeafLogistics.org} email platform. This is the single most effective control to prevent email account takeovers.
    \end{itemize}

    \item \textbf{Implement MFA for Endpoint Login (RISK-001):}
    \begin{itemize}
        \item Deploy an MFA solution for all Windows, macOS, and Linux computer logins to prevent unauthorized access from stolen credentials.
    \end{itemize}
\end{enumerate}

\subsection{Strategic Recommendations (Implement within 90 Days)}
\begin{enumerate}
    \item \textbf{Conduct a Comprehensive Vulnerability Assessment:} Perform a credentialed vulnerability scan across the entire internal network to identify other misconfigurations, missing patches, and insecure services.
    \item \textbf{Implement Network Segmentation:} Review the network architecture to segment critical systems from user workstations. This will limit an attacker's ability to move laterally across the network if one system is compromised.
    \item \textbf{Develop a Formal Vulnerability Management Program:} Establish a documented process for regularly identifying, prioritizing, and remediating vulnerabilities to ensure a consistent and proactive security posture.
\end{enumerate}

\end{document}
```