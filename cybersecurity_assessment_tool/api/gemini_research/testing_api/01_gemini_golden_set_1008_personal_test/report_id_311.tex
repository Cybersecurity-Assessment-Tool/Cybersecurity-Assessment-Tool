```latex
\documentclass[12pt, a4paper]{article}

% ----------------------------------------------------------------------
% PREAMBLE
% ----------------------------------------------------------------------
\usepackage[margin=1in]{geometry}
\usepackage{pifont} % For checkmarks and crosses
\usepackage{booktabs} % For professional tables
\usepackage{hyperref} % For clickable links
\usepackage{url} % For URL formatting
\usepackage{seqsplit} % To split long monospaced strings
\usepackage{graphicx}
\usepackage{xcolor}

% --- Hyperref Setup ---
\hypersetup{
    colorlinks=true,
    linkcolor=blue,
    filecolor=magenta,      
    urlcolor=cyan,
    pdftitle={Cybersecurity Posture Assessment},
    pdfpagemode=FullScreen,
}

% --- Document Metadata ---
\title{Cybersecurity Posture Assessment Report \\ \large For \textbf{Gilded Cage Design}}
\author{Cybersecurity Analysis Division}
\date{\today}

% ----------------------------------------------------------------------
% DOCUMENT START
% ----------------------------------------------------------------------
\begin{document}

\maketitle
\thispagestyle{empty}
\newpage

\tableofcontents
\newpage

% ----------------------------------------------------------------------
% SECTION 1: EXECUTIVE SUMMARY
% ----------------------------------------------------------------------
\section{Executive Summary}

This report provides a comprehensive cybersecurity posture assessment for \textbf{Gilded Cage Design}, synthesizing data from technical network scans, a security controls questionnaire, and a review of pre-existing risks. The analysis aims to identify vulnerabilities, policy gaps, and misconfigurations to provide actionable recommendations for improving the organization's overall security.

\paragraph{Key Findings:} The assessment revealed a mixed security posture. While foundational controls like Multi-Factor Authentication (MFA) for email and computers are in place, several critical and high-risk gaps were identified:
\begin{itemize}
    \item \textbf{Critical Technical Risk:} A network service (SSH on port 22) was found exposed on a local loopback interface (\texttt{127.0.0.1}). This finding confirms a pre-existing high-severity risk, "Localhost Exposed," indicating a potentially severe misconfiguration.
    \item \textbf{Critical Policy Gap:} Multi-Factor Authentication is not required for accessing sensitive data systems. This significantly increases the risk of unauthorized access to the organization's most valuable information assets.
    \item \textbf{High-Risk Governance Gap:} The organization lacks a formal employee Acceptable Use Policy (AUP). This absence creates ambiguity regarding security responsibilities and acceptable user behavior, weakening the overall security culture.
\end{itemize}

\paragraph{Recommendations Summary:} Immediate remediation should focus on securing the exposed network service, mandating MFA for all sensitive systems, and developing and implementing a formal Acceptable Use Policy. Detailed recommendations are provided in Section \ref{sec:recommendations}.

% ----------------------------------------------------------------------
% SECTION 2: ORGANIZATIONAL INFORMATION
% ----------------------------------------------------------------------
\section{Organizational Information}

The following information was provided for the assessment.

\begin{tabular}{@{}ll}
    \toprule
    \textbf{Attribute} & \textbf{Value} \\
    \midrule
    Organization Name & \textbf{Gilded Cage Design} \\
    Email Domain & \texttt{GildedCageDesign.org} \\
    Website Domain & \url{www.GildedCageDesign.org} \\
    External IP Address & \seqsplit{\texttt{219.123.243.9}} \\
    \bottomrule
\end{tabular}

% ----------------------------------------------------------------------
% SECTION 3: SECURITY CONTROL REVIEW
% ----------------------------------------------------------------------
\section{Security Control Review (Questionnaire Analysis)}

The following table summarizes the organization's responses to a security controls questionnaire. Items marked with a red cross (\ding{55}) indicate significant gaps in the security framework that require immediate attention.

\begin{table}[h!]
\centering
\begin{tabular}{p{0.7\textwidth}c}
\toprule
\textbf{Control Question} & \textbf{Status} \\
\midrule
Do you require MFA to access email? & \textcolor{green}{\ding{51}} \\
Do you require MFA to log into computers? & \textcolor{green}{\ding{51}} \\
\textbf{Do you require MFA to access sensitive data systems?} & \textcolor{red}{\ding{55}} \\
\textbf{Does your organization have an employee acceptable use policy?} & \textcolor{red}{\ding{55}} \\
Does your organization do security awareness training for new employees? & \textcolor{green}{\ding{51}} \\
Does your organization do security awareness training for all employees at least once per year? & \textcolor{green}{\ding{51}} \\
\bottomrule
\end{tabular}
\caption{Security Controls Questionnaire Results}
\label{tab:controls}
\end{table}

% ----------------------------------------------------------------------
% SECTION 4: TECHNICAL SCAN RESULTS
% ----------------------------------------------------------------------
\section{Technical Scan Results}

A network scan was performed to identify active services and potential exposures.

\subsection{Host Status}
The target host was found to be online and responsive to network probes.
\begin{itemize}
    \item \textbf{Target IP:} \seqsplit{\texttt{127.0.0.1}}
    \item \textbf{Status:} Up
\end{itemize}

\subsection{Open Ports}
The following ports were identified as open on the target system. An open port indicates a listening service that could be a potential entry point for an attacker. The presence of an open SSH port on the localhost interface is highly unusual and directly correlates with the "Localhost Exposed" risk detailed in Section \ref{sec:risks}.

\begin{table}[h!]
\centering
\begin{tabular}{lllll}
\toprule
\textbf{Port} & \textbf{State} & \textbf{Service (Inferred)} & \textbf{Product} & \textbf{Version} \\
\midrule
22/tcp & open & ssh & Not Identified & Not Identified \\
\bottomrule
\end{tabular}
\caption{Open Ports Detected on \texttt{127.0.0.1}}
\label{tab:ports}
\end{table}

% ----------------------------------------------------------------------
% SECTION 5: CONSOLIDATED RISK ASSESSMENT
% ----------------------------------------------------------------------
\section{Consolidated Risk Assessment}
\label{sec:risks}

This section synthesizes findings from all data sources into a consolidated list of identified risks. Each risk is assigned a severity level to aid in prioritization.

\begin{table}[h!]
\centering
\begin{tabular}{p{0.25\textwidth}p{0.5\textwidth}l}
\toprule
\textbf{Risk Name} & \textbf{Description} & \textbf{Severity} \\
\midrule
\textbf{Localhost Exposed} & The technical scan confirmed an active SSH service on the localhost interface (\texttt{127.0.0.1}). This indicates a severe misconfiguration that could be exploited by local processes or chained with other vulnerabilities. This confirms a pre-existing identified risk. & \textbf{Critical (10.0)} \\
\addlinespace
\textbf{Lack of MFA on Sensitive Systems} & The security questionnaire revealed that MFA is not enforced for access to sensitive data systems. This exposes critical assets to compromise via stolen credentials, a common attack vector. & \textbf{Critical} \\
\addlinespace
\textbf{No Acceptable Use Policy (AUP)} & The organization lacks a formal AUP. This governance gap leads to inconsistent security practices and a lack of clear guidelines for employees, increasing the risk of insider threat and accidental data exposure. & \textbf{High} \\
\bottomrule
\end{tabular}
\caption{Summary of Identified Risks}
\label{tab:risks}
\end{table}

% ----------------------------------------------------------------------
% SECTION 6: RECOMMENDATIONS
% ----------------------------------------------------------------------
\section{Recommendations}
\label{sec:recommendations}

The following actions are recommended to mitigate the identified risks and strengthen the overall security posture of \textbf{Gilded Cage Design}.

\begin{enumerate}
    \item \textbf{Remediate Exposed Service (Critical):}
    \begin{itemize}
        \item \textbf{Immediate Action:} Investigate the SSH service running on port 22 of \texttt{127.0.0.1}. If this service is not essential for a specific application's function, it should be disabled immediately.
        \item \textbf{Long-Term Action:} If the service is required, ensure it is properly configured to only allow connections from authorized local processes and is not unintentionally exposed. Implement robust logging and monitoring for this service.
    \end{itemize}
    \vspace{1em}
    \item \textbf{Implement MFA for Sensitive Systems (Critical):}
    \begin{itemize}
        \item \textbf{Immediate Action:} Identify all systems, applications, and databases that store or process sensitive data.
        \item \textbf{Long-Term Action:} Prioritize and execute a plan to deploy and mandate MFA for all user accounts (including administrative and service accounts) that have access to these identified systems.
    \end{itemize}
    \vspace{1em}
    \item \textbf{Develop and Implement an Acceptable Use Policy (High):}
    \begin{itemize}
        \item \textbf{Immediate Action:} Draft a formal AUP that clearly defines the rules and responsibilities for all employees when using company IT assets. The policy should cover data handling, password security, internet usage, and incident reporting.
        \item \textbf{Long-Term Action:} Integrate the AUP into the new employee onboarding process. Require all current employees to read and formally acknowledge the policy. Conduct annual reviews to ensure the policy remains current.
    \end{itemize}
\end{enumerate}

% ----------------------------------------------------------------------
% DOCUMENT END
% ----------------------------------------------------------------------
\end{document}
```