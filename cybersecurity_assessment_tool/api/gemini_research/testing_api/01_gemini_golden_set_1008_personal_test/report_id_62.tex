```latex
\documentclass[12pt, a4paper]{article}

% Preamble: Required Packages
\usepackage[margin=1in]{geometry}
\usepackage{pifont} % For checkmarks and crosses
\usepackage{booktabs} % For professional tables
\usepackage{hyperref} % For clickable links
\usepackage{url} % For URL formatting
\usepackage{seqsplit} % To split long strings without breaking
\usepackage{graphicx}
\usepackage{xcolor}

% Document Information
\title{Cybersecurity Assessment Report}
\author{Cybersecurity Analyst}
\date{\today}

% Hyperref Setup
\hypersetup{
    colorlinks=true,
    linkcolor=blue,
    filecolor=magenta,      
    urlcolor=cyan,
    pdftitle={Cybersecurity Assessment Report},
    pdfpagemode=FullScreen,
}

\begin{document}

\maketitle
\thispagestyle{empty}
\newpage

\tableofcontents
\newpage

% --- 1. Executive Summary ---
\section{Executive Summary}
This report provides a comprehensive cybersecurity assessment for \textbf{Solid State}, conducted on \today. The analysis is based on a network vulnerability scan, a review of organizational security controls, and an evaluation of pre-existing risks.

The assessment has identified several critical and high-risk vulnerabilities that require immediate attention. Key findings include:
\begin{itemize}
    \item \textbf{Critical Vulnerability:} An externally facing FTP service (\texttt{vsftpd 2.3.4}) is active on \texttt{10.0.0.15}. This specific version is known to be vulnerable to a remote command execution backdoor (CVE-2011-2523). Furthermore, it is misconfigured to allow anonymous logins, posing a severe risk of unauthorized access and data breach.
    \item \textbf{Critical Control Gap:} Multi-Factor Authentication (MFA) is not enforced for accessing sensitive data systems. This significantly increases the risk of unauthorized access through compromised credentials.
    \item \textbf{High-Risk Procedural Gap:} New employees do not receive mandatory security awareness training, making them susceptible to social engineering and phishing attacks from their first day.
\end{itemize}

The combination of these findings indicates a security posture with significant weaknesses. This report outlines actionable recommendations to mitigate these risks and strengthen the overall security framework of the organization.

% --- 2. Organizational Information ---
\section{Organizational Information}
The following information was provided by the client and used as a baseline for this assessment.

\begin{table}[h!]
\centering
\begin{tabular}{@{}ll@{}}
\toprule
\textbf{Attribute} & \textbf{Value} \\ \midrule
Organization Name & \textbf{Solid State} \\
Email Domain & \texttt{SolidState.net} \\
Website Domain & \url{www.SolidState.net} \\
External IP Address & \texttt{239.109.170.235} \\ \bottomrule
\end{tabular}
\caption{Client Organizational Details}
\end{table}

% --- 3. Security Control Review ---
\section{Security Control Review}
A review of the organization's security controls was conducted via a standardized questionnaire. The results highlight critical gaps in the current security policies and procedures. A "No" response indicates a deviation from security best practices.

\begin{table}[h!]
\centering
\begin{tabular}{@{}p{0.7\linewidth}c@{}}
\toprule
\textbf{Control Question} & \textbf{Response} \\ \midrule
Do you require MFA to access email? & \ding{51} \\ % Yes
Do you require MFA to log into computers? & \ding{51} \\ % Yes
\textbf{Do you require MFA to access sensitive data systems?} & \textcolor{red}{\ding{55}} \\ % No
Does your organization have an employee acceptable use policy? & \ding{51} \\ % Yes
\textbf{Does your organization do security awareness training for new employees?} & \textcolor{red}{\ding{55}} \\ % No
Does your organization do security awareness training for all employees at least once per year? & \ding{51} \\ % Yes
\bottomrule
\end{tabular}
\caption{Security Controls Questionnaire Results (\ding{51}=Yes, \ding{55}=No)}
\end{table}

\subsection*{Analysis of Control Gaps}
The questionnaire revealed two significant control gaps:
\begin{enumerate}
    \item \textbf{Lack of MFA for Sensitive Systems:} The absence of MFA on systems containing sensitive data is a critical oversight. Should an employee's credentials be compromised, an attacker could gain direct access to the organization's most valuable information assets.
    \item \textbf{No Onboarding Security Training:} Failing to train new employees on security best practices leaves the organization vulnerable. New hires are often targeted by phishing and social engineering attacks and may be unaware of internal policies regarding data handling and acceptable use.
\end{enumerate}

% --- 4. Technical Scan Results ---
\section{Technical Scan Results}
An Nmap scan was performed on the target system to identify open ports and running services.

\begin{itemize}
    \item \textbf{Target IP:} \texttt{10.0.0.15}
    \item \textbf{Scan Date:} \today
\end{itemize}

The scan identified one open port with a critically vulnerable service.

\begin{table}[h!]
\centering
\begin{tabular}{@{}lllll@{}}
\toprule
\textbf{Port} & \textbf{State} & \textbf{Service} & \textbf{Product \& Version} & \textbf{Notes} \\ \midrule
21/tcp & Open & ftp & vsftpd 2.3.4 & Anonymous FTP login allowed \\ \bottomrule
\end{tabular}
\caption{Open Ports and Services Detected on \texttt{10.0.0.15}}
\end{table}

\subsection*{Analysis of Technical Findings}
The technical scan revealed a major security flaw:
\begin{itemize}
    \item \textbf{Vulnerable FTP Service (CVE-2011-2523):} The version of vsftpd detected (\texttt{2.3.4}) contains a well-known, critical backdoor vulnerability. An attacker can gain a command shell on the server by sending a specific string in the username field.
    \item \textbf{Anonymous FTP Enabled:} The service is configured to allow anonymous logins. This allows any unauthenticated user on the network to access, upload, or download files from the FTP server, which can lead to data exfiltration or the introduction of malware.
\end{itemize}
This combination presents an immediate and severe threat to the network's integrity and confidentiality.

% --- 5. Consolidated Risk Assessment ---
\section{Consolidated Risk Assessment}
The following table synthesizes findings from the security control review, the technical scan, and pre-existing risk data. Risks are prioritized based on their potential impact and exploitability.

\begin{table}[h!]
\centering
\begin{tabular}{@{}p{0.3\linewidth}p{0.5\linewidth}l@{}}
\toprule
\textbf{Risk Name} & \textbf{Overview} & \textbf{Severity} \\ \midrule
\textbf{Vulnerable FTP Service} & An exploitable backdoor (CVE-2011-2523) exists in vsftpd 2.3.4, compounded by anonymous login being enabled. & \textbf{Critical} \\
\textbf{No MFA on Sensitive Systems} & Lack of MFA on critical data systems allows for single-factor authentication, making them highly susceptible to credential compromise. & \textbf{Critical} \\
\textbf{No Security Training for New Hires} & New employees are not trained on security policies, increasing susceptibility to social engineering and accidental data breaches. & \textbf{High} \\
\textbf{Outdated Windows Policy} & Computers are running Windows 7, which is an end-of-life operating system and no longer receives security updates. & Medium (5.0) \\ \bottomrule
\end{tabular}
\caption{Summary of Identified Risks}
\end{table}

% --- 6. Recommendations ---
\section{Recommendations}
The following actions are recommended to mitigate the identified risks. Recommendations are prioritized to address critical threats first.

\subsection*{Immediate Actions (To be completed within 72 hours)}
\begin{enumerate}
    \item \textbf{Remediate Vulnerable FTP Service:}
        \begin{itemize}
            \item Immediately take the FTP service on \texttt{10.0.0.15} offline.
            \item If FTP is required, upgrade \texttt{vsftpd} to the latest stable version.
            \item Disable anonymous FTP login.
            \item If possible, replace FTP with a more secure protocol like SFTP (SSH File Transfer Protocol).
        \end{itemize}
    \item \textbf{Implement MFA for Sensitive Systems:}
        \begin{itemize}
            \item Enforce MFA across all systems identified as containing sensitive data.
            \item Develop a project plan to roll out MFA to all remaining systems within the next quarter.
        \end{itemize}
\end{enumerate}

\subsection*{Strategic Actions (To be completed within 90 days)}
\begin{enumerate}
    \setcounter{enumi}{2} % Continue numbering from previous list
    \item \textbf{Establish Onboarding Security Training:}
        \begin{itemize}
            \item Develop a mandatory security awareness training module for all new employees.
            \item Integrate this training into the formal onboarding process.
            \item Topics should include phishing awareness, acceptable use policies, and data handling procedures.
        \end{itemize}
    \item \textbf{Address Outdated Windows Policy:}
        \begin{itemize}
            \item Execute the plan to upgrade all Windows 7 workstations to a supported operating system, such as Windows 10 or 11.
            \item Decommission any hardware that cannot support a modern operating system.
        \end{itemize}
\end{enumerate}

\end{document}
```