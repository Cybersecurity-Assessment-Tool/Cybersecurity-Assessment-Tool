```latex
\documentclass[12pt]{article}

% Preamble: Required Packages
\usepackage[a4paper, margin=1in]{geometry}
\usepackage{pifont} % For checkmarks and crosses
\usepackage{booktabs} % For professional tables
\usepackage[hidelinks]{hyperref} % For clickable links without boxes
\usepackage{url} % For URL formatting
\usepackage{seqsplit} % For splitting long strings in tt font
\usepackage{graphicx}
\usepackage{fancyhdr}
\usepackage{lastpage}

% --- Document Metadata ---
\title{Cybersecurity Posture Assessment Report}
\author{Cybersecurity Analysis Division}
\date{\today}

% --- Header and Footer Configuration ---
\pagestyle{fancy}
\fancyhf{} % Clear all header and footer fields
\fancyhead[L]{Top Tier Security Report}
\fancyfoot[C]{\thepage\ of \pageref{LastPage}}
\renewcommand{\headrulewidth}{0.4pt}
\renewcommand{\footrulewidth}{0.4pt}

\begin{document}

\maketitle
\thispagestyle{empty}
\newpage

\tableofcontents
\newpage

% ==============================================================================
% SECTION 1: EXECUTIVE SUMMARY
% ==============================================================================
\section{Executive Summary}

This report provides a cybersecurity posture assessment for \textbf{Top Tier}, based on an analysis of organizational data, security control questionnaires, and technical network scans. The primary objective is to identify security gaps, assess associated risks, and provide actionable recommendations to enhance the organization's security posture.

\paragraph{Key Findings:}
The assessment revealed several critical and high-risk security gaps stemming from policy and procedural deficiencies. The most significant concerns are the absence of Multi-Factor Authentication (MFA) for email and sensitive data systems, which exposes the organization to account compromise and data breaches. Furthermore, the lack of a formal Acceptable Use Policy (AUP) and a structured security awareness training program indicates a foundational weakness in security governance and culture.

\paragraph{Data Integrity Issues:}
It is crucial to note that the provided technical network scan data (\texttt{Input\_1\_Network\_Scan\_JSON}) and the list of current risks (\texttt{Input\_3\_Current\_Risks\_JSON}) were found to be corrupted and could not be processed. Consequently, this report's findings are based exclusively on the security questionnaire. The inability to analyze technical vulnerabilities from the external scan represents a significant blind spot in this assessment.

\paragraph{Overall Posture:}
The current security posture is rated as \textbf{Weak}. While some controls like MFA for computer logins are in place, the absence of fundamental security measures in other critical areas presents a high likelihood of a security incident. Immediate remediation of the identified risks is strongly recommended.

% ==============================================================================
% SECTION 2: ORGANIZATIONAL INFORMATION
% ==============================================================================
\section{Organizational Information}

The following details were provided for the assessment scope.

\begin{itemize}
    \item \textbf{Organization Name:} Top Tier
    \item \textbf{Email Domain:} \texttt{TopTier.org}
    \item \textbf{Website Domain:} \url{www.TopTier.org}
    \item \textbf{Primary External IP:} \texttt{126.96.164.192}
\end{itemize}

% ==============================================================================
% SECTION 3: SECURITY CONTROL REVIEW
% ==============================================================================
\section{Security Control Review}

The following table summarizes the organization's responses to a security controls questionnaire. A green checkmark (\ding{51}) indicates a positive control is in place, while a red cross (\ding{55}) indicates a security gap.

\begin{table}[h!]
\centering
\caption{Security Controls Questionnaire Results}
\begin{tabular}{p{0.8\linewidth} c}
\toprule
\textbf{Control Question} & \textbf{Response} \\
\midrule
Do you require MFA to access email? & \ding{55} \\
Do you require MFA to log into computers? & \ding{51} \\
Do you require MFA to access sensitive data systems? & \ding{55} \\
Does your organization have an employee acceptable use policy? & \ding{55} \\
Does your organization do security awareness training for new employees? & \ding{55} \\
Does your organization do security awareness training for all employees at least once per year? & \ding{55} \\
\bottomrule
\end{tabular}
\end{table}

The review highlights significant deficiencies in access control, policy enforcement, and employee security training. These gaps are the primary source of the risks identified in Section 5.

% ==============================================================================
% SECTION 4: TECHNICAL SCAN RESULTS
% ==============================================================================
\section{Technical Scan Results}

\subsection{Scan Status}

The input data file containing the network scan results (\texttt{Input\_1\_Network\_Scan\_JSON}) was determined to be corrupted or incomplete. As a result, no analysis of open ports, running services, or potential software vulnerabilities could be performed on the target IP address (\texttt{126.96.164.192}).

\subsection{Implications}

Without valid scan data, the organization's external attack surface remains unverified. This leaves critical questions unanswered regarding:
\begin{itemize}
    \item The presence of exposed and potentially vulnerable services (e.g., RDP, SSH, FTP).
    \item Outdated software versions with known public exploits.
    - Insecure service configurations that could be leveraged by attackers.
\end{itemize}
A comprehensive technical assessment is a mandatory component of a robust security program.

% ==============================================================================
% SECTION 5: RISK ASSESSMENT
% ==============================================================================
\section{Risk Assessment}

The following risks have been identified based on the findings from the Security Control Review. The pre-existing risk list (\texttt{Input\_3\_Current\_Risks\_JSON}) was unavailable for correlation. The severity is rated on a scale of Low, Medium, High, and Critical.

\begin{table}[h!]
\centering
\caption{Identified Risks}
\begin{tabular}{p{0.1\linewidth} p{0.25\linewidth} p{0.4\linewidth} p{0.1\linewidth}}
\toprule
\textbf{Risk ID} & \textbf{Risk Name} & \textbf{Overview} & \textbf{Severity} \\
\midrule
RISK-001 & Lack of MFA for Email Access & Email accounts are protected only by passwords, making them highly susceptible to phishing, credential stuffing, and brute-force attacks, which can lead to business email compromise (BEC) and data exfiltration. & Critical \\
\addlinespace
RISK-002 & Lack of MFA for Sensitive Systems & Critical data systems lack a secondary authentication factor, creating a single point of failure for access control. A compromised password could lead directly to a major data breach. & Critical \\
\addlinespace
RISK-003 & No Employee Acceptable Use Policy (AUP) & Without a formal AUP, there are no clear guidelines for employees on the acceptable use of company assets. This increases the risk of insider threat, misuse of resources, and legal liability. & High \\
\addlinespace
RISK-004 & Inadequate Security Awareness Training & Employees are not trained to recognize or respond to security threats like phishing. This makes them the weakest link and primary target for social engineering attacks. & High \\
\bottomrule
\end{tabular}
\end{table}

% ==============================================================================
% SECTION 6: RECOMMENDATIONS
% ==============================================================================
\section{Recommendations}

The following actions are recommended to mitigate the identified risks and improve the overall security posture of \textbf{Top Tier}.

\subsection{Immediate Actions (1-30 Days)}
\begin{enumerate}
    \item \textbf{Remediate RISK-001: Enforce MFA for Email.} Immediately enable and enforce MFA for all user accounts with access to the \texttt{TopTier.org} email system. Prioritize solutions like authenticator apps over SMS.
    
    \item \textbf{Remediate RISK-002: Enforce MFA for Sensitive Systems.} Identify all systems containing sensitive or critical data and enforce MFA for all user access, especially for administrative accounts.
    
    \item \textbf{Initiate New Technical Scan.} Commission a new, validated external vulnerability scan or penetration test against the public IP address \texttt{126.96.164.192} to identify and remediate technical vulnerabilities.
\end{enumerate}

\subsection{Strategic Actions (30-90 Days)}
\begin{enumerate}
    \setcounter{enumi}{3} % Continue numbering
    \item \textbf{Remediate RISK-003: Develop and Implement AUP.} Draft a comprehensive Acceptable Use Policy that clearly defines rules for computer, network, and data usage. Ensure all employees read and formally acknowledge the policy.
    
    \item \textbf{Remediate RISK-004: Establish Security Awareness Training.} Implement a mandatory security awareness training program. This must include an initial module for all new hires and an annual refresher course for all staff, covering topics such as phishing, password security, and data handling.
\end{enumerate}

\end{document}
```