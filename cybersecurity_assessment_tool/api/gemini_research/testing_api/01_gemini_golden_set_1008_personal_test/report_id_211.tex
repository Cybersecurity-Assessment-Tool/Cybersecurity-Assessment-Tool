```latex
\documentclass[12pt]{article}

% Preamble: Required Packages
\usepackage[a4paper, margin=1in]{geometry}
\usepackage{pifont} % For checkmarks and crosses
\usepackage{booktabs} % For professional tables
\usepackage{hyperref} % For clickable links
\usepackage{url} % For URL formatting
\usepackage{seqsplit} % To split long strings like IPs
\usepackage{graphicx} % For potential logos/images
\usepackage[table]{xcolor} % For coloring table cells

% Document Metadata and Styling
\hypersetup{
    colorlinks=true,
    linkcolor=blue,
    filecolor=magenta,      
    urlcolor=cyan,
    pdftitle={Cybersecurity Posture Report},
    pdfpagemode=FullScreen,
}

\definecolor{lightgray}{gray}{0.9}
\newcommand{\yes}{\ding{51}} % Checkmark
\newcommand{\no}{\ding{55}}  % Cross

% --- Document Start ---
\begin{document}

% --- Title Page ---
\begin{titlepage}
    \centering
    \vspace*{1cm}
    \Huge\textbf{Cybersecurity Posture Report}
    \vspace{1.5cm}
    \Large
    \textbf{Prepared for:}\\
    Opal Sky Media
    \vspace{2cm}
    \large
    \textbf{Date of Report:}\\
    \today
    \vfill
    \large
    \textit{This report contains sensitive information and should be handled with care.}
\end{titlepage}

\tableofcontents
\newpage

% --- Section 1: Executive Summary ---
\section{Executive Summary}
This report provides a comprehensive analysis of the cybersecurity posture for Opal Sky Media, based on a combination of technical network scanning, a review of organizational security controls, and an assessment of known risks. The assessment was conducted on \today.

The overall security posture presents a \textbf{moderate-to-high level of risk}. While foundational controls like Multi-Factor Authentication (MFA) for email and computer access are in place, critical gaps were identified that significantly increase the organization's exposure to common cyber threats.

Key findings include:
\begin{itemize}
    \item \textbf{Critical Control Gaps:} The absence of MFA for sensitive data systems is a critical vulnerability. Furthermore, the lack of a formal security awareness training program for both new and existing employees creates a significant human-element risk, making the organization more susceptible to phishing and social engineering attacks.
    \item \textbf{Network Exposure:} A technical scan of the external network revealed an open Secure Shell (SSH) port (22) on the IPv6 address \seqsplit{\texttt{2001:db8::1}}. While SSH is a standard administrative tool, its public exposure is a common target for brute-force and credential-stuffing attacks.
    \item \textbf{No Pre-existing Risks Logged:} The provided data indicated no previously documented vulnerabilities, meaning the risks identified in this report are newly assessed and require immediate attention.
\end{itemize}

This report concludes with a prioritized list of actionable recommendations designed to mitigate the identified risks and strengthen the overall security posture of Opal Sky Media.

% --- Section 2: Organizational Information ---
\section{Organizational Information}
The following details were provided for the assessment. This information helps to establish the context and scope of the review.

\begin{table}[h!]
\centering
\rowcolors{2}{lightgray}{white}
\begin{tabular}{ll}
\toprule
\textbf{Attribute} & \textbf{Value} \\
\midrule
Organization Name & Opal Sky Media \\
Email Domain & \texttt{OpalSkyMedia.com} \\
Website Domain & \url{www.OpalSkyMedia.com} \\
External IP Address & \texttt{90.113.103.109} \\
\bottomrule
\end{tabular}
\caption{Client Organizational Details}
\end{table}

% --- Section 3: Security Control Review ---
\section{Security Control Review}
A questionnaire was used to evaluate the implementation of key administrative and technical security controls. The results are summarized below. Answers marked with a cross (\no) indicate significant gaps in the security framework.

\begin{table}[h!]
\centering
\rowcolors{2}{lightgray}{white}
\begin{tabular}{p{0.8\linewidth} c}
\toprule
\textbf{Control Question} & \textbf{Status} \\
\midrule
Do you require MFA to access email? & \yes \\
Do you require MFA to log into computers? & \yes \\
\textbf{Do you require MFA to access sensitive data systems?} & \textbf{\color{red}\no} \\
Does your organization have an employee acceptable use policy? & \yes \\
\textbf{Does your organization do security awareness training for new employees?} & \textbf{\color{red}\no} \\
\textbf{Does your organization do security awareness training for all employees at least once per year?} & \textbf{\color{red}\no} \\
\bottomrule
\end{tabular}
\caption{Security Controls Questionnaire Results}
\end{table}

\subsection{Analysis of Control Gaps}
The questionnaire reveals three critical deficiencies:
\begin{itemize}
    \item \textbf{No MFA for Sensitive Data:} Failing to protect sensitive data systems with MFA is a critical oversight. Should an attacker compromise a user's credentials, they would have direct access to high-value data, potentially leading to a significant data breach.
    \item \textbf{No Security Awareness Training:} The complete absence of a security awareness training program is a major weakness. Employees are the first line of defense, and without training, they are far more likely to fall victim to phishing, malware, and other social engineering tactics. This gap undermines all other technical security investments.
\end{itemize}

% --- Section 4: Technical Scan Results ---
\section{Technical Scan Results}
An external network scan was performed to identify exposed services and potential vulnerabilities.

\begin{itemize}
    \item \textbf{Target IP Address:} \seqsplit{\texttt{2001:db8::1}} (IPv6)
    \item \textbf{Scan Date:} \today
\end{itemize}

\subsection{Open Ports}
The following table details the ports found to be open and accessible from the public internet.

\begin{table}[h!]
\centering
\rowcolors{2}{lightgray}{white}
\begin{tabular}{llll}
\toprule
\textbf{Port} & \textbf{State} & \textbf{Service (Inferred)} & \textbf{Notes} \\
\midrule
22/tcp & Open & SSH & No version details were obtained. \\
\bottomrule
\end{tabular}
\caption{Open Ports Detected on \seqsplit{\texttt{2001:db8::1}}}
\end{table}

\subsection{Technical Analysis}
The scan identified that port 22, commonly used for the Secure Shell (SSH) protocol, is open. SSH is a powerful tool for remote server administration but is also a primary target for attackers.
\begin{itemize}
    \item \textbf{Exposure Risk:} An exposed SSH service is vulnerable to automated brute-force attacks, where attackers attempt to guess usernames and passwords.
    \item \textbf{Credential Risk:} If an employee's credentials are stolen (e.g., through a phishing attack, which is more likely given the lack of training), they could be used to gain unauthorized access to the server via this SSH port.
\end{itemize}
Without detailed service and version information, it is not possible to determine if the running SSH server software is vulnerable to any known exploits. However, its public exposure alone constitutes a notable risk.

% --- Section 5: Consolidated Risk Assessment ---
\section{Consolidated Risk Assessment}
The following table synthesizes findings from the security control review and the technical scan into a prioritized list of identified risks.

\begin{table}[h!]
\centering
\begin{tabular}{p{0.1\linewidth} p{0.3\linewidth} p{0.4\linewidth} p{0.1\linewidth}}
\toprule
\textbf{Risk ID} & \textbf{Risk Name} & \textbf{Description} & \textbf{Severity} \\
\midrule
RISK-001 & \textbf{Lack of MFA on Sensitive Systems} & The absence of multi-factor authentication on systems storing or processing sensitive data allows for single-factor (password) compromise, which could lead to a major data breach. & \textbf{Critical} \\
\addlinespace
RISK-002 & \textbf{Inadequate Security Awareness Training} & The lack of a formal training program for new and existing employees significantly increases the likelihood of successful phishing, malware infection, and other human-targeted attacks. & \textbf{High} \\
\addlinespace
RISK-003 & \textbf{Exposed SSH Service} & The SSH management port is open to the public internet, making it a target for brute-force attacks and exploitation if credentials are compromised or the service is unpatched. & \textbf{Medium} \\
\bottomrule
\end{tabular}
\caption{Summary of Identified Risks}
\end{table}

% --- Section 6: Recommendations ---
\section{Recommendations}
The following actions are recommended to mitigate the identified risks and improve the overall security posture of Opal Sky Media.

\subsection{RISK-001: Lack of MFA on Sensitive Systems (Critical)}
\begin{itemize}
    \item \textbf{Immediate Action:} Identify all systems, applications, and databases that contain sensitive client, financial, or proprietary data.
    \item \textbf{Implementation:} Procure and implement a robust MFA solution for all identified systems. This control should be mandatory for all users, including administrators, without exception.
    \item \textbf{Policy:} Update the organization's access control policy to mandate MFA for all current and future systems handling sensitive information.
\end{itemize}

\subsection{RISK-002: Inadequate Security Awareness Training (High)}
\begin{itemize}
    \item \textbf{Onboarding:} Develop and implement a mandatory security awareness training module for all new employees as part of the onboarding process.
    \item \textbf{Annual Training:} Schedule and conduct annual, mandatory security awareness training for all staff. This training should cover topics such as phishing, password hygiene, acceptable use, and incident reporting.
    \item \textbf{Phishing Simulation:} Conduct periodic phishing simulation campaigns to test employee awareness and reinforce training concepts.
\end{itemize}

\subsection{RISK-003: Exposed SSH Service (Medium)}
\begin{itemize}
    \item \textbf{Step 1 (Review):} Determine the business need for remote SSH access to \seqsplit{\texttt{2001:db8::1}}. If it is not required, the port should be blocked at the firewall immediately.
    \item \textbf{Step 2 (Harden):} If SSH access is required, implement the following hardening measures:
        \begin{itemize}
            \item Restrict access to trusted IP addresses only (IP whitelisting).
            \item Disable password-based authentication and enforce the use of public/private key pairs.
            \item Disable root user login over SSH.
            \item Implement an intrusion detection tool like \texttt{fail2ban} to automatically block IPs that exhibit malicious behavior.
        \end{itemize}
\end{itemize}

\end{document}
```