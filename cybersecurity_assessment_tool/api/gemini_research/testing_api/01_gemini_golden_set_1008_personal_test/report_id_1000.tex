```latex
\documentclass[12pt]{article}

% Preamble: Required Packages
\usepackage[a4paper, margin=1in]{geometry}
\usepackage{pifont} % For checkmarks and crosses (\ding)
\usepackage{booktabs} % For professional tables (\toprule, \midrule, \bottomrule)
\usepackage[hidelinks]{hyperref} % For clickable links without boxes
\usepackage{url} % For typesetting URLs
\usepackage{seqsplit} % To split long strings in tt font
\usepackage{graphicx} % To include logos
\usepackage{xcolor} % For custom colors
\usepackage{fancyhdr} % For headers and footers

% --- Document Setup ---

% Define custom colors
\definecolor{darkblue}{rgb}{0.0, 0.0, 0.3}
\definecolor{gray}{rgb}{0.4, 0.4, 0.4}
\definecolor{red}{rgb}{0.8, 0.0, 0.0}
\definecolor{orange}{rgb}{1.0, 0.5, 0.0}
\definecolor{green}{rgb}{0.0, 0.5, 0.0}

% Hyperref Setup
\hypersetup{
    colorlinks=true,
    linkcolor=darkblue,
    filecolor=darkblue,      
    urlcolor=darkblue,
    citecolor=darkblue,
}

% Header and Footer
\pagestyle{fancy}
\fancyhf{} % Clear all header and footer fields
\fancyhead[L]{\textbf{Cybersecurity Posture Report}}
\fancyhead[R]{\textbf{Zenith Point}}
\fancyfoot[C]{\thepage}
\renewcommand{\headrulewidth}{0.4pt}
\renewcommand{\footrulewidth}{0.4pt}

% --- Document Start ---
\begin{document}

% --- Title Page ---
\begin{titlepage}
    \centering
    \vspace*{2cm}
    
    {\Huge\bfseries Cybersecurity Posture Report\par}
    \vspace{1.5cm}
    
    {\Large Prepared for:\par}
    \vspace{0.5cm}
    {\Huge\bfseries Zenith Point\par}
    
    \vfill
    
    {\large \today\par}
    \vspace{1cm}
    
    {\large \textbf{CONFIDENTIAL DOCUMENT}\par}
    \vspace{2cm}
    
    \begin{abstract}
        \noindent This report provides a comprehensive analysis of the cybersecurity posture of Zenith Point. The assessment is based on a synthesis of organizational data, a technical network scan, and a review of existing risks. The findings herein are intended to provide a clear overview of security strengths and weaknesses, followed by actionable recommendations to mitigate identified risks and enhance the organization's overall security resilience.
    \end{abstract}
\end{titlepage}

\tableofcontents
\newpage

% --- Section 1: Executive Summary ---
\section{Executive Summary}

This assessment of \textbf{Zenith Point} reveals a mixed security posture. On the technical front, the external network scan of the target host showed a strong configuration with no open ports detected, which significantly reduces the external attack surface. This is a commendable security practice.

However, the review of organizational security controls identified critical deficiencies. The absence of Multi-Factor Authentication (MFA) for email and sensitive data systems represents a \textbf{critical risk}. An attacker with compromised credentials could gain unauthorized access to key communication channels and confidential information.

Furthermore, the lack of foundational security policies, such as an Acceptable Use Policy, and the complete absence of a security awareness training program for employees, create a high-risk environment susceptible to social engineering, phishing, and insider threats.

While the network perimeter appears secure, the internal policy and procedural gaps must be addressed urgently to prevent a security incident. The highest priority recommendations are the immediate implementation of MFA across all critical systems and the establishment of a formal security awareness program.

% --- Section 2: Organizational Information ---
\section{Organizational Information}

The following information was provided for the assessment. This data forms the basis for understanding the organizational context and scope.

\begin{tabular}{@{}ll}
    \toprule
    \textbf{Attribute} & \textbf{Value} \\
    \midrule
    Organization Name & \textbf{Zenith Point} \\
    Email Domain & \seqsplit{\texttt{ZenithPoint.net}} \\
    Website Domain & \seqsplit{\texttt{www.ZenithPoint.net}} \\
    External IP Address & \seqsplit{\texttt{47.83.243.42}} \\
    \bottomrule
\end{tabular}

% --- Section 3: Security Control Review ---
\section{Security Control Review}

A review of internal security controls was conducted via a questionnaire. The responses indicate significant gaps in identity and access management and employee security awareness. A "No" response indicates a deviation from security best practices and a potential risk.

\begin{tabular}{@{}p{6cm}ccc}
    \toprule
    \textbf{Control Question} & \textbf{Response} & \textbf{Assessment} \\
    \midrule
    Do you require MFA to access email? & \textcolor{red}{\ding{55}} & \textbf{Critical Gap} \\
    \addlinespace
    Do you require MFA to log into computers? & \textcolor{green}{\ding{51}} & Aligns with Best Practice \\
    \addlinespace
    Do you require MFA to access sensitive data systems? & \textcolor{red}{\ding{55}} & \textbf{Critical Gap} \\
    \addlinespace
    Does your organization have an employee acceptable use policy? & \textcolor{red}{\ding{55}} & \textbf{High Risk} \\
    \addlinespace
    Does your organization do security awareness training for new employees? & \textcolor{red}{\ding{55}} & \textbf{High Risk} \\
    \addlinespace
    Does your organization do security awareness training for all employees at least once per year? & \textcolor{red}{\ding{55}} & \textbf{High Risk} \\
    \bottomrule
\end{tabular}

\vspace{1em}
\noindent \textbf{Note:} \textcolor{green}{\ding{51}} indicates a "Yes" response, aligning with security standards. \textcolor{red}{\ding{55}} indicates a "No" response, highlighting a control weakness.

% --- Section 4: Technical Scan Results ---
\section{Technical Scan Results}

A network scan was performed to identify open ports and services on the specified target system. The results provide insight into the external attack surface.

\begin{itemize}
    \item \textbf{Target IP Address:} \seqsplit{\texttt{192.168.1.100}}
    \item \textbf{Scan Date:} \today
    \item \textbf{Target Status:} \textbf{Up}
\end{itemize}

\subsection{Port and Service Analysis}

The scan results were positive, indicating a well-hardened system from an external network perspective.

\begin{tabular}{@{}p{12cm}}
    \toprule
    \textbf{Finding} \\
    \midrule
    \textbf{No Open Ports Detected.} \\
    The Nmap scan confirmed that there were no open TCP ports on the target host. All 1000 scanned ports were reported as being in a \texttt{closed} state. This indicates that the host is either not running any network services or is protected by a firewall that is correctly configured to block incoming connections. This is an excellent security posture for an externally facing or internal segment device. \\
    \bottomrule
\end{tabular}

% --- Section 5: Risk Assessment ---
\section{Risk Assessment}

This section synthesizes the findings from the security control review and technical scans. While no pre-existing vulnerabilities were reported and the technical scan was clean, the policy and procedural gaps represent significant organizational risks.

\begin{tabular}{@{}p{4cm}p{7cm}l}
    \toprule
    \textbf{Risk Name} & \textbf{Overview} & \textbf{Severity} \\
    \midrule
    \textbf{Email Account Compromise} & The lack of MFA on email accounts means a compromised password is all an attacker needs to gain full access, leading to data breaches, phishing, and business email compromise (BEC). & \textcolor{red}{\textbf{Critical}} \\
    \addlinespace
    \textbf{Unauthorized Data Access} & Sensitive data systems are not protected by MFA. A single point of failure (a stolen password) could lead to a major breach of confidential company or customer data. & \textcolor{red}{\textbf{Critical}} \\
    \addlinespace
    \textbf{Lack of Employee Guidance} & Without an Acceptable Use Policy, employees lack clear rules on the safe and appropriate use of company assets, increasing the risk of misuse, data leakage, and legal liability. & \textcolor{orange}{\textbf{High}} \\
    \addlinespace
    \textbf{Untrained Workforce} & With no security awareness training, employees are highly susceptible to phishing, social engineering, and other common attack vectors. They represent the weakest link in the security chain. & \textcolor{orange}{\textbf{High}} \\
    \bottomrule
\end{tabular}

% --- Section 6: Recommendations ---
\section{Recommendations}

The following actionable recommendations are provided to mitigate the identified risks and improve the overall security posture of \textbf{Zenith Point}. Recommendations are prioritized based on risk severity.

\begin{enumerate}
    \item \textbf{[Critical] Implement MFA for Email and Sensitive Systems:}
    \begin{itemize}
        \item \textbf{Action:} Immediately enable and enforce MFA for all user accounts across the email platform (e.g., Microsoft 365, Google Workspace) and all systems identified as containing sensitive data.
        \item \textbf{Impact:} Drastically reduces the risk of account takeover via stolen credentials. This is the single most effective control to implement.
    \end{itemize}
    
    \item \textbf{[High] Develop and Implement a Security Awareness Program:}
    \begin{itemize}
        \item \textbf{Action:} Establish a formal security awareness training program. This must include mandatory training for all new hires and annual refresher training for all staff. Training should cover phishing, password security, and data handling.
        \item \textbf{Impact:} Creates a security-conscious culture and empowers employees to become a human firewall, reducing susceptibility to social engineering.
    \end{itemize}
    
    \item \textbf{[High] Create an Acceptable Use Policy (AUP):}
    \begin{itemize}
        \item \textbf{Action:} Draft, approve, and distribute an AUP that clearly defines the rules for using company networks, computers, and data. Require all employees to read and acknowledge the policy.
        \item \textbf{Impact:} Establishes clear expectations for behavior, reduces insider risk, and provides a basis for disciplinary action if policies are violated.
    \end{itemize}
    
    \item \textbf{[Informational] Maintain Network Hardening:}
    \begin{itemize}
        \item \textbf{Action:} Continue the current practice of maintaining a minimal attack surface on network devices. Regularly audit firewall rules and scan for open ports to ensure the configuration remains secure.
        \item \textbf{Impact:} Ensures the strong network perimeter security observed in this assessment is maintained over time.
    \end{itemize}
\end{enumerate}

\end{document}
```