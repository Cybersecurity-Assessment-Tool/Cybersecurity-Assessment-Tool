```latex
\documentclass[12pt, a4paper]{article}

% Preamble: Required Packages
\usepackage[margin=1in]{geometry}
\usepackage{pifont} % For checkmarks and crosses
\usepackage{booktabs} % For professional tables
\usepackage{hyperref} % For clickable links and ToC
\usepackage{url} % For formatting URLs
\usepackage{seqsplit} % For splitting long text strings
\usepackage{graphicx}
\usepackage[table]{xcolor}
\usepackage{lastpage}
\usepackage{fancyhdr}

% --- Document Setup ---

% Color Definitions
\definecolor{VerveBlue}{RGB}{28, 82, 128}
\definecolor{HighRisk}{RGB}{217, 83, 79}
\definecolor{MediumRisk}{RGB}{240, 173, 78}
\definecolor{LowRisk}{RGB}{92, 184, 92}
\definecolor{tablehead}{gray}{0.9}

% Hyperlink Setup
\hypersetup{
    colorlinks=true,
    linkcolor=VerveBlue,
    urlcolor=VerveBlue,
    citecolor=VerveBlue,
}

% Header & Footer
\pagestyle{fancy}
\fancyhf{} % clear all header and footer fields
\fancyhead[L]{Cybersecurity Posture Report}
\fancyhead[R]{Verve \& Vigor}
\fancyfoot[C]{\thepage\ of \pageref{LastPage}}
\renewcommand{\headrulewidth}{0.4pt}
\renewcommand{\footrulewidth}{0.4pt}

% --- Document Body ---

\begin{document}

% --- Title Page ---
\begin{titlepage}
    \centering
    \vspace*{2cm}
    
    \includegraphics[width=0.4\textwidth]{example-image-a} % Placeholder for company logo
    
    \vspace{1.5cm}
    
    {\Huge\bfseries Cybersecurity Posture Report\par}
    
    \vspace{1cm}
    
    {\Large\bfseries Prepared for: Verve \& Vigor\par}
    
    \vspace{2cm}
    
    {\large \today\par}
    
    \vfill
    
    {\large \textit{This report contains sensitive information and should be handled with care. Distribution is restricted to authorized personnel only.}\par}
    
\end{titlepage}

\tableofcontents
\newpage

% --- Section 1: Executive Summary ---
\section{Executive Summary}
This report provides a cybersecurity assessment for Verve \& Vigor, synthesizing data from organizational questionnaires, external network scans, and a review of pre-existing risks. The analysis aims to identify key vulnerabilities, security control gaps, and provide actionable recommendations to enhance the organization's security posture.

\paragraph{Key Findings:} The organization demonstrates a strong commitment to identity and access management, with Multi-Factor Authentication (MFA) implemented across email, computer logins, and sensitive data systems. This significantly reduces the risk of unauthorized access.

However, two critical administrative control gaps were identified: the absence of a formal Acceptable Use Policy (AUP) and the lack of mandatory, annual security awareness training for all staff. These policy and training deficiencies create a significant risk, as they leave the organization vulnerable to insider threats and social engineering attacks.

From a technical perspective, an external scan identified an open SSH port (22) on an IPv6 address. While this port is necessary for remote administration, its exposure requires robust security configurations to prevent it from becoming an entry point for attackers. The current scan did not provide service version details, preventing a specific vulnerability assessment.

\paragraph{Overall Posture:} The overall security posture is mixed. While technical access controls are strong, foundational policy and training programs are lacking. The recommendations in this report focus on bridging these administrative gaps and hardening the identified external service to build a more resilient and comprehensive security framework.

\newpage

% --- Section 2: Organizational Information ---
\section{Organizational Information}
The following details were provided for the assessment. This information helps contextualize the findings and recommendations.

\begin{table}[h!]
\centering
\caption{Client Organizational Data}
\label{tab:orgdata}
\begin{tabular}{@{}ll@{}}
\toprule
\rowcolor{tablehead}
\textbf{Attribute} & \textbf{Value} \\ \midrule
Organization Name    & Verve \& Vigor \\
Email Domain         & \texttt{VerveVigor.com} \\
Website Domain       & \url{www.VerveVigor.com} \\
External IP Address  & \texttt{195.253.188.61} \\ \bottomrule
\end{tabular}
\end{table}

% --- Section 3: Security Control Review ---
\section{Security Control Review}
A security questionnaire was completed to assess the implementation of key administrative and technical controls. The results are summarized below. "Yes" answers indicate a control is in place, while "No" answers highlight a potential security gap.

\begin{table}[h!]
\centering
\caption{Security Questionnaire Results}
\label{tab:questionnaire}
\begin{tabular}{@{}p{0.8\linewidth}c@{}}
\toprule
\rowcolor{tablehead}
\textbf{Control Question} & \textbf{Response} \\ \midrule
Do you require MFA to access email? & \textcolor{green}{\ding{51}} \\
Do you require MFA to log into computers? & \textcolor{green}{\ding{51}} \\
Do you require MFA to access sensitive data systems? & \textcolor{green}{\ding{51}} \\
Does your organization have an employee acceptable use policy? & \textcolor{red}{\ding{55}} \\
Does your organization do security awareness training for new employees? & \textcolor{green}{\ding{51}} \\
Does your organization do security awareness training for all employees at least once per year? & \textcolor{red}{\ding{55}} \\ \bottomrule
\end{tabular}
\end{table}

\subsection*{Analysis of Control Gaps}
The questionnaire reveals two significant gaps in administrative controls:
\begin{itemize}
    \item \textbf{No Acceptable Use Policy (AUP):} An AUP is a foundational document that defines how employees may use company IT assets. Without it, there is no formal standard for user behavior, making it difficult to enforce security rules or take corrective action against policy violations.
    \item \textbf{No Annual Security Training:} While new hires receive training, the lack of an annual refresher for all employees is a major concern. The threat landscape evolves continuously, and ongoing training is essential to keep staff vigilant against modern threats like phishing, ransomware, and social engineering.
\end{itemize}

\newpage

% --- Section 4: Technical Scan Results ---
\section{Technical Scan Results}
An external network scan was performed to identify open ports and exposed services on the organization's public-facing infrastructure.

\begin{itemize}
    \item \textbf{Target IP Address:} \texttt{2001:db8::1}
\end{itemize}

\begin{table}[h!]
\centering
\caption{Open Ports Detected}
\label{tab:portscan}
\begin{tabular}{@{}llll@{}}
\toprule
\rowcolor{tablehead}
\textbf{Port} & \textbf{State} & \textbf{Service (Inferred)} & \textbf{Product / Version} \\ \midrule
22/tcp        & open           & SSH                         & Not Detected               \\ \bottomrule
\end{tabular}
\end{table}

\subsection*{Analysis of Technical Findings}
The scan identified that port 22, commonly used for the Secure Shell (SSH) protocol, is open to the internet. SSH is a critical tool for secure remote system administration. However, if not properly configured, it can be a primary target for attackers.
\begin{itemize}
    \item \textbf{Attack Vector:} Attackers can perform brute-force password guessing attacks against the SSH service.
    \item \textbf{Vulnerability Risk:} The scan did not retrieve the specific version of the SSH server. Without this information, it is impossible to determine if it is vulnerable to any known exploits. Outdated versions of SSH software often contain critical security flaws.
\end{itemize}

% --- Section 5: Risk Assessment ---
\section{Risk Assessment}
The following table synthesizes findings from the security control review and technical scan into a prioritized list of risks. No pre-existing vulnerabilities were reported.

\begin{table}[h!]
\centering
\caption{Summary of Identified Risks}
\label{tab:risks}
\begin{tabular}{@{}p{0.08\linewidth}p{0.3\linewidth}p{0.15\linewidth}p{0.35\linewidth}@{}}
\toprule
\rowcolor{tablehead}
\textbf{ID} & \textbf{Risk Name} & \textbf{Severity} & \textbf{Description} \\ \midrule
RISK-001 & Lack of Acceptable Use Policy & \cellcolor{HighRisk!25}\textbf{High} & The absence of a formal AUP means there are no enforceable rules for employee use of IT systems, increasing the risk of misuse and insider threat. \\
\addlinespace
RISK-002 & Inadequate Annual Security Training & \cellcolor{HighRisk!25}\textbf{High} & Without regular, mandatory training, employees are more likely to fall victim to phishing and social engineering, which are primary vectors for ransomware and data breaches. \\
\addlinespace
RISK-003 & Unidentified SSH Service Exposure & \cellcolor{MediumRisk!25}\textbf{Medium} & An open SSH port without a known version or confirmed secure configuration presents a potential entry point for attackers via brute-force or exploitation of unknown vulnerabilities. \\ \bottomrule
\end{tabular}
\end{table}

\newpage

% --- Section 6: Recommendations ---
\section{Recommendations}
The following actions are recommended to mitigate the identified risks and improve the overall security posture of Verve \& Vigor.

\subsection*{RISK-001: Lack of Acceptable Use Policy (High)}
\begin{itemize}
    \item \textbf{Action:} Develop and implement a comprehensive Acceptable Use Policy (AUP) for all employees and contractors.
    \item \textbf{Details:} The policy should clearly define rules for email usage, internet access, data handling, and use of personal devices. It must be communicated to all staff and acknowledged via signature.
    \item \textbf{Impact:} Establishes a clear security baseline for user behavior and provides a legal and administrative framework for enforcement.
\end{itemize}

\subsection*{RISK-002: Inadequate Annual Security Training (High)}
\begin{itemize}
    \item \textbf{Action:} Establish a mandatory annual security awareness training program for all employees.
    \item \textbf{Details:} The training should cover current threats such as phishing, ransomware, password security, and social engineering. Consider incorporating periodic phishing simulation tests to measure effectiveness.
    \item \textbf{Impact:} Creates a "human firewall" by empowering employees to recognize and report security threats, drastically reducing the likelihood of a successful social engineering attack.
\end{itemize}

\subsection*{RISK-003: Unidentified SSH Service Exposure (Medium)}
\begin{itemize}
    \item \textbf{Action:} Audit and harden the configuration of the public-facing SSH service on \texttt{2001:db8::1}.
    \item \textbf{Details:}
    \begin{enumerate}
        \item Perform a version detection scan to identify the SSH software and version, and patch if outdated.
        \item Enforce public key authentication and disable direct password-based logins.
        \item Disable root user login via SSH.
        \item Implement rate-limiting or an intrusion prevention tool (e.g., Fail2Ban) to block brute-force attempts.
        \item If possible, restrict access to a whitelist of trusted IP addresses.
    \end{enumerate}
    \item \textbf{Impact:} Significantly reduces the attack surface of the exposed service, preventing unauthorized access and exploitation.
\end{itemize}

\end{document}
```