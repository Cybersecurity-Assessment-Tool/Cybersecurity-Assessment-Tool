```latex
\documentclass[12pt]{article}

% Preamble: Required Packages
\usepackage[margin=1in]{geometry}
\usepackage{pifont} % For \ding symbols
\usepackage{booktabs} % For professional tables
\usepackage{hyperref} % For clickable links
\usepackage{url} % For URL formatting
\usepackage{seqsplit} % To split long strings in tt font
\usepackage{graphicx}
\usepackage[table]{xcolor} % For coloring table cells
\usepackage{tocbibind}

% --- Document Setup ---
\hypersetup{
    colorlinks=true,
    linkcolor=blue,
    filecolor=magenta,      
    urlcolor=cyan,
    pdftitle={Cybersecurity Posture Assessment Report},
    pdfpagemode=FullScreen,
}

% Define colors for severity
\definecolor{severityhigh}{HTML}{D9534F}
\definecolor{severitymedium}{HTML}{F0AD4E}
\definecolor{severitylow}{HTML}{5CB85C}
\definecolor{severityinfo}{HTML}{5BC0DE}
\definecolor{tablehead}{gray}{0.9}

% --- Document Start ---
\begin{document}

% --- Title Page ---
\begin{titlepage}
    \centering
    \vspace*{1cm}
    \Huge\textbf{Cybersecurity Posture Assessment Report}
    \vspace{1.5cm}
    \Large
    \textbf{Prepared for:}\\
    \vspace{0.5cm}
    \textbf{Iron Bridge Legal}
    \vspace{2cm}
    \large
    \textbf{Date of Report:}\\
    \vspace{0.5cm}
    \today
    \vfill
    \large
    \textit{This report contains sensitive and confidential information intended for the exclusive use of the recipient.}
\end{titlepage}

\tableofcontents
\newpage

% --- Section 1: Executive Summary ---
\section{Executive Summary}
This report provides a comprehensive cybersecurity assessment for \textbf{Iron Bridge Legal}, based on a correlation of network scan data, organizational security controls, and pre-existing risk information. The assessment was conducted to identify key vulnerabilities and provide actionable recommendations to enhance the organization's security posture.

The analysis revealed several high-risk security gaps primarily related to identity and access management. The absence of Multi-Factor Authentication (MFA) for accessing email and sensitive data systems represents a critical vulnerability. An attacker with compromised credentials could gain unauthorized access to confidential client information, leading to a significant data breach.

Furthermore, the lack of a formal security awareness training program for new and existing employees increases susceptibility to social engineering attacks, such as phishing, which is a primary vector for initial compromise.

On a positive note, a technical scan of the target system \texttt{192.168.0.5} indicates that a previously identified risk concerning an open unencrypted web port (Port 80) appears to have been remediated, as the port was found to be closed during the current assessment.

Immediate remediation efforts should focus on implementing MFA across all critical systems and establishing a recurring security awareness training program.

% --- Section 2: Organizational Information ---
\section{Organizational Information}
The following details were provided for the assessment. This information is used to establish the context and scope of the review.

\begin{table}[h!]
\centering
\caption{Client Organizational Data}
\label{tab:org_data}
\begin{tabular}{@{}ll@{}}
\toprule
\textbf{Attribute} & \textbf{Value} \\ \midrule
Organization Name & \textbf{Iron Bridge Legal} \\
Email Domain      & \texttt{IronBridgeLegal.org} \\
Website Domain    & \href{http://www.IronBridgeLegal.org}{\texttt{www.IronBridgeLegal.org}} \\
External IP Address & \texttt{56.132.128.126} \\ \bottomrule
\end{tabular}
\end{table}

% --- Section 3: Security Control Review ---
\section{Security Control Review}
A review of administrative and technical security controls was conducted based on a questionnaire. The findings highlight significant gaps in access control and security awareness policies. A "No" answer indicates a deviation from security best practices and a potential risk.

\begin{table}[h!]
\centering
\caption{Security Controls Questionnaire Analysis}
\label{tab:controls}
\rowcolors{2}{gray!10}{white}
\begin{tabular}{@{}p{0.7\textwidth}c@{}}
\toprule
\rowcolor{tablehead}
\textbf{Control Question} & \textbf{Status} \\ \midrule
Do you require MFA to access email? & \ding{55} \\
Do you require MFA to log into computers? & \ding{51} \\
Do you require MFA to access sensitive data systems? & \ding{55} \\
Does your organization have an employee acceptable use policy? & \ding{51} \\
Does your organization do security awareness training for new employees? & \ding{55} \\
Does your organization do security awareness training for all employees at least once per year? & \ding{55} \\ \bottomrule
\end{tabular}
\end{table}

\begin{itemize}
    \item[\ding{51}] \textbf{Control in Place:} The control is implemented.
    \item[\ding{55}] \textbf{Control Gap:} The control is not implemented, representing a security risk.
\end{itemize}

% --- Section 4: Technical Scan Results ---
\section{Technical Scan Results}
A network scan was performed to identify open ports and exposed services on the target system.

\begin{itemize}
    \item \textbf{Target IP Address:} \texttt{192.168.0.5}
    \item \textbf{Scan Date:} \today
\end{itemize}

The scan results are summarized in the table below. The scan revealed no open ports on the target system, which is a positive security finding. Notably, Port 80, which was listed as a concern in pre-existing risk data, was found to be closed.

\begin{table}[h!]
\centering
\caption{Nmap Scan Results for \texttt{192.168.0.5}}
\label{tab:nmap}
\begin{tabular}{@{}lllll@{}}
\toprule
\textbf{Port} & \textbf{State} & \textbf{Service} & \textbf{Product} & \textbf{Version} \\ \midrule
80/tcp      & closed & http & N/A & N/A \\ \bottomrule
\end{tabular}
\end{table}

% --- Section 5: Consolidated Risk Assessment ---
\section{Consolidated Risk Assessment}
The following table synthesizes findings from the security control review, technical scan, and pre-existing risk data. Each risk is assigned a severity level to aid in prioritization.

\begin{table}[h!]
\centering
\caption{Summary of Identified Risks}
\label{tab:risks}
\begin{tabular}{@{}p{0.4\textwidth}p{0.15\textwidth}p{0.15\textwidth}p{0.2\textwidth}@{}}
\toprule
\rowcolor{tablehead}
\textbf{Risk / Vulnerability} & \textbf{Severity} & \textbf{Status} & \textbf{Affected Elements} \\ \midrule
Lack of MFA on Email System & \cellcolor{severityhigh}High & Active & \texttt{IronBridgeLegal.org} email accounts \\
\addlinespace
Lack of MFA on Sensitive Data Systems & \cellcolor{severityhigh}High & Active & All systems storing sensitive client or firm data \\
\addlinespace
Inadequate Security Awareness Program & \cellcolor{severitymedium}Medium & Active & All employees \\
\addlinespace
Unencrypted Web Server (Port 80) & \cellcolor{severityinfo}Resolved & Remediated & Port 80 on \texttt{192.168.0.5} \\
\bottomrule
\end{tabular}
\end{table}

% --- Section 6: Recommendations ---
\section{Recommendations}
Based on the consolidated risk assessment, the following actions are recommended to mitigate the identified vulnerabilities and strengthen the overall security posture of \textbf{Iron Bridge Legal}.

\subsection{High Priority Recommendations}

\subsubsection{Implement Multi-Factor Authentication (MFA)}
\begin{itemize}
    \item \textbf{Risk Addressed:} Lack of MFA on Email and Sensitive Data Systems.
    \item \textbf{Action:} Immediately procure and deploy an MFA solution for all user accounts. Prioritize implementation for email access (e.g., Office 365, Google Workspace) and any systems containing sensitive client data, financial records, or other confidential information.
    \item \textbf{Justification:} MFA is one of the most effective controls to prevent account takeover and unauthorized access resulting from compromised credentials.
\end{itemize}

\subsection{Medium Priority Recommendations}

\subsubsection{Establish a Security Awareness Training Program}
\begin{itemize}
    \item \textbf{Risk Addressed:} Inadequate Security Awareness Program.
    \item \textbf{Action:} Develop and implement a formal security awareness training program. This program must be mandatory for all new employees during onboarding and required for all staff annually. Training should cover topics such as phishing identification, password hygiene, and acceptable use of company assets.
    \item \textbf{Justification:} A well-trained workforce serves as a human firewall, reducing the likelihood of successful social engineering and phishing attacks.
\end{itemize}

\subsection{Informational / Verification}

\subsubsection{Confirm Remediation of Port 80}
\begin{itemize}
    \item \textbf{Risk Addressed:} Unencrypted Web Server.
    \item \textbf{Action:} While the current scan shows Port 80 as closed, it is recommended to confirm with the IT team that this was an intentional change to remediate the previously identified risk. Ensure that if a web server is needed, it is configured securely over HTTPS (Port 443) with a valid SSL/TLS certificate.
    \item \textbf{Justification:} Verifying that security changes are intentional and correctly implemented ensures that risks are fully and permanently mitigated.
\end{itemize}

\end{document}
```