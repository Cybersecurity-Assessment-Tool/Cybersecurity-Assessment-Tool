```latex
\documentclass[12pt]{article}

% Preamble: Required Packages
\usepackage[margin=1in]{geometry}
\usepackage{pifont} % For checkmarks and crosses
\usepackage{booktabs} % For professional tables
\usepackage{hyperref} % For clickable links
\usepackage{url} % For URL formatting
\usepackage{seqsplit} % To split long strings in tt font

% Document Metadata
\title{Cybersecurity Assessment Report}
\author{Cybersecurity Analysis Division}
\date{November 22, 2025}

% Hyperref Setup
\hypersetup{
    colorlinks=true,
    linkcolor=black,
    urlcolor=blue,
    pdftitle={Cybersecurity Assessment Report},
    pdfauthor={Cybersecurity Analysis Division},
    pdfsubject={Security Assessment},
    pdfkeywords={Security, Risk, Analysis}
}

\begin{document}

\maketitle
\thispagestyle{empty}
\newpage
\tableofcontents
\newpage

% --- 1. Executive Overview ---
\section{Executive Overview}
This report details the findings of a cybersecurity assessment conducted for \textbf{Iron Oak Furniture}. The assessment combined a review of organizational security controls, an external network scan, and an analysis of known risks to evaluate the organization's overall security posture.

The analysis identified several critical and high-risk security gaps. The most significant findings include a complete lack of Multi-Factor Authentication (MFA) across all critical systems, including email and sensitive data access. Furthermore, the organization does not conduct security awareness training for its employees, leaving it highly susceptible to social engineering attacks like phishing.

Technically, the external-facing web server was found to be running an outdated version of Nginx (1.18.0), which has multiple publicly disclosed vulnerabilities. The combination of these administrative and technical weaknesses places the organization at a significant risk of unauthorized access, data breach, and business disruption. Immediate remediation of the identified issues is strongly recommended.

% --- 2. Organizational Information ---
\section{Organizational Information}
The following details were provided for the assessment.

\begin{tabular}{@{}ll}
    \toprule
    \textbf{Attribute} & \textbf{Value} \\
    \midrule
    Organization Name & \textbf{Iron Oak Furniture} \\
    Email Domain & \texttt{IronOakFurniture.com} \\
    Website Domain & \texttt{www.IronOakFurniture.com} \\
    External IP Address & \texttt{119.208.213.74} \\
    \bottomrule
\end{tabular}

% --- 3. Security Control Review ---
\section{Security Control Review}
A review of administrative and policy-based security controls was conducted via a questionnaire. The responses reveal critical gaps in fundamental security practices, particularly concerning access control and employee training.

\begin{table}[h!]
\centering
\caption{Security Controls Questionnaire Results}
\begin{tabular}{@{}p{0.75\linewidth}c@{}}
    \toprule
    \textbf{Control Question} & \textbf{Response} \\
    \midrule
    Do you require MFA to access email? & \ding{55} \\
    Do you require MFA to log into computers? & \ding{55} \\
    Do you require MFA to access sensitive data systems? & \ding{55} \\
    Does your organization have an employee acceptable use policy? & \ding{51} \\
    Does your organization do security awareness training for new employees? & \ding{55} \\
    Does your organization do security awareness training for all employees at least once per year? & \ding{55} \\
    \bottomrule
\end{tabular}
\end{table}

\paragraph{Analysis:} The lack of Multi-Factor Authentication (MFA) across email, computers, and sensitive systems is a critical vulnerability. A compromised password would be sufficient for an attacker to gain significant access. Additionally, the absence of any security awareness training program makes employees primary targets for phishing and other social engineering attacks.

% --- 4. Technical Scan Results ---
\section{Technical Scan Results}
An external network scan was performed on \textbf{2025-11-22} to identify accessible services and potential vulnerabilities.

\begin{itemize}
    \item \textbf{Target IP Address:} \texttt{192.168.10.5}
\end{itemize}

\begin{table}[h!]
\centering
\caption{Open Ports and Services}
\begin{tabular}{@{}lllll@{}}
    \toprule
    \textbf{Port} & \textbf{State} & \textbf{Service} & \textbf{Product} & \textbf{Version} \\
    \midrule
    443/tcp & open & https & nginx & 1.18.0 \\
    \bottomrule
\end{tabular}
\end{table}

\subsection{Technical Findings}
\begin{itemize}
    \item \textbf{Outdated Nginx Server:} The web server is running Nginx version 1.18.0, which was released in April 2020. This version is outdated and has several known vulnerabilities, including CVE-2021-23017, which could allow an attacker to cause a denial of service or potentially execute arbitrary code.
    \item \textbf{Potential Certificate Mismatch:} The SSL certificate presented by the server has a Common Name of \texttt{www.acme-corp.com}, which does not match the organization's domain (\texttt{www.IronOakFurniture.com}). This can cause browser trust errors and may indicate a server misconfiguration.
\end{itemize}

% --- 5. Risk Assessment Summary ---
\section{Risk Assessment Summary}
The following table synthesizes findings from the security control review and technical scan into a prioritized list of risks. No pre-existing vulnerabilities were reported.

\begin{table}[h!]
\centering
\caption{Identified Risks}
\begin{tabular}{@{}p{0.1\linewidth}p{0.25\linewidth}p{0.45\linewidth}l@{}}
    \toprule
    \textbf{ID} & \textbf{Risk Name} & \textbf{Description} & \textbf{Severity} \\
    \midrule
    R-01 & No Multi-Factor Authentication (MFA) & The absence of MFA for email, workstations, and sensitive systems exposes the organization to account takeover via credential theft. & \textbf{Critical} \\
    \addlinespace
    R-02 & Outdated Web Server Software & The public-facing Nginx server is running an old version with known vulnerabilities, creating a high risk of server compromise. & \textbf{High} \\
    \addlinespace
    R-03 & Inadequate Security Awareness Training & Employees are not trained to identify or respond to security threats, making the organization highly vulnerable to phishing and social engineering. & \textbf{High} \\
    \addlinespace
    R-04 & SSL Certificate Misconfiguration & The SSL certificate does not match the organization's domain, which can erode user trust and may indicate other configuration issues. & \textbf{Medium} \\
    \bottomrule
\end{tabular}
\end{table}

% --- 6. Recommendations ---
\section{Recommendations}
Based on the identified risks, the following actions are recommended to improve the security posture of \textbf{Iron Oak Furniture}.

\begin{enumerate}
    \item \textbf{[Critical] Implement Multi-Factor Authentication:}
    \begin{itemize}
        \item \textbf{Action:} Immediately deploy MFA for all user accounts, prioritizing access to email, VPNs, and systems containing sensitive data.
        \item \textbf{Justification:} This is the single most effective control to prevent unauthorized access from compromised credentials.
    \end{itemize}
    \vspace{1em}
    \item \textbf{[High] Upgrade Web Server Software:}
    \begin{itemize}
        \item \textbf{Action:} Plan and execute an upgrade of the Nginx server on \texttt{192.168.10.5} to the latest stable version.
        \item \textbf{Justification:} Patching the outdated software will mitigate known vulnerabilities and protect the server from public exploits.
    \end{itemize}
    \vspace{1em}
    \item \textbf{[High] Establish a Security Awareness Training Program:}
    \begin{itemize}
        \item \textbf{Action:} Procure and implement a security awareness training program. All employees should receive training upon hiring and at least annually thereafter.
        \item \textbf{Justification:} A well-trained workforce is the first line of defense against phishing and social engineering attacks.
    \end{itemize}
    \vspace{1em}
    \item \textbf{[Medium] Correct SSL Certificate Configuration:}
    \begin{itemize}
        \item \textbf{Action:} Investigate the SSL certificate mismatch on the web server. Procure and install a valid certificate for the \texttt{www.IronOakFurniture.com} domain.
        \item \textbf{Justification:} A valid certificate ensures encrypted communication is trusted by users and partners, and it resolves a key configuration error.
    \end{itemize}
\end{enumerate}

\end{document}
```