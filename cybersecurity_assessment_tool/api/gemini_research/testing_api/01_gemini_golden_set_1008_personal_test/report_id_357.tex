```latex
\documentclass[12pt]{article}

% --- PACKAGES ---
\usepackage[margin=1in]{geometry}
\usepackage{pifont} % For checkmarks and crosses
\usepackage{booktabs} % For professional tables
\usepackage{hyperref} % For clickable links
\usepackage{url} % For formatting URLs
\usepackage{seqsplit} % For splitting long strings to prevent overflow
\usepackage{graphicx}
\usepackage{fancyhdr}
\usepackage{xcolor}

% --- DOCUMENT SETUP ---
\definecolor{darkblue}{rgb}{0.0, 0.0, 0.55}
\hypersetup{
    colorlinks=true,
    linkcolor=darkblue,
    filecolor=darkblue,      
    urlcolor=darkblue,
    citecolor=darkblue,
}

\pagestyle{fancy}
\fancyhf{}
\lhead{\textbf{Cybersecurity Assessment Report}}
\rhead{\textbf{Midnight Oil Studios}}
\cfoot{\thepage}
\renewcommand{\headrulewidth}{0.4pt}
\renewcommand{\footrulewidth}{0.4pt}

% --- TITLE ---
\title{
    \vspace{1cm}
    \textbf{Cybersecurity Posture Assessment Report} \\
    \large \textit{Confidential}
    \vspace{0.5cm}
}
\author{Cybersecurity Analysis Division}
\date{\today}

\begin{document}

\maketitle
\thispagestyle{empty}
\newpage

\tableofcontents
\newpage

% --- EXECUTIVE SUMMARY ---
\section{Executive Summary}

This report presents a cybersecurity assessment for \textbf{Midnight Oil Studios}, synthesizing data from technical network scans, an organizational security questionnaire, and a review of pre-existing risk documentation. 

The assessment identified several critical and high-risk findings that require immediate attention. The most severe issue is the discovery of a web service on the internal network at \texttt{10.5.5.5:8080} titled \textbf{"TOP SECRET DB"}, suggesting a potentially exposed sensitive database. This finding directly contradicts the current risk register, which lists this port as a secure false positive.

Furthermore, significant procedural gaps were identified, including the lack of Multi-Factor Authentication (MFA) for email and computer access, and the absence of an employee Acceptable Use Policy. These weaknesses substantially increase the risk of account compromise and unauthorized access to sensitive information.

Immediate remediation should focus on securing the exposed database interface and implementing MFA across all critical systems. A comprehensive review of the risk management process is also strongly recommended.

% --- ORGANIZATIONAL INFORMATION ---
\section{Organizational Information}

The following details were provided for the assessment.

\begin{itemize}
    \item \textbf{Organization Name:} Midnight Oil Studios
    \item \textbf{Email Domain:} \texttt{MidnightOilStudios.com}
    \item \textbf{Website Domain:} \url{www.MidnightOilStudios.com}
    \item \textbf{External IP Address:} \texttt{134.82.222.162}
\end{itemize}

% --- SECURITY CONTROL REVIEW ---
\section{Security Control Review}

An analysis of the security questionnaire reveals critical gaps in foundational security controls. While the organization has implemented security awareness training, the lack of MFA and a formal Acceptable Use Policy represents a significant weakness in the overall security posture.

\begin{table}[h!]
\centering
\caption{Security Questionnaire Analysis}
\label{tab:questionnaire}
\begin{tabular}{@{}lc@{}}
\toprule
\textbf{Control Question} & \textbf{Response} \\ \midrule
Do you require MFA to access email? & \ding{55} \\
Do you require MFA to log into computers? & \ding{55} \\
Do you require MFA to access sensitive data systems? & \ding{51} \\
Does your organization have an employee acceptable use policy? & \ding{55} \\
Does your organization do security awareness training for new employees? & \ding{51} \\
Does your organization do security awareness training for all employees at least once per year? & \ding{51} \\ \bottomrule
\end{tabular}
\end{table}

\begin{itemize}
    \item[\textcolor{red}{\ding{55}}] \textbf{Critical Gap:} The absence of MFA for email and computer logins exposes the organization to a high risk of credential theft and account takeover attacks.
    \item[\textcolor{red}{\ding{55}}] \textbf{High Risk:} The lack of an Acceptable Use Policy creates ambiguity regarding proper system usage, increasing the potential for insider threats and accidental data leakage.
    \item[\textcolor{green}{\ding{51}}] \textbf{Strength:} The implementation of annual security awareness training is a positive control that helps mitigate human-based risks.
\end{itemize}

% --- TECHNICAL SCAN RESULTS ---
\section{Technical Scan Results}

A network scan was conducted on the internal target \texttt{10.5.5.5}. The scan identified one open port hosting a web service with a highly concerning title, indicating a potential data exposure.

\begin{table}[h!]
\centering
\caption{Open Port Analysis for Target: \texttt{10.5.5.5}}
\label{tab:scanresults}
\begin{tabular}{@{}llll@{}}
\toprule
\textbf{Port} & \textbf{State} & \textbf{Service} & \textbf{Details} \\ \midrule
8080/tcp & Open & http & HTTP server with title: \textbf{`TOP SECRET DB'} \\ \bottomrule
\end{tabular}
\end{table}

\subsection*{Analysis of Findings}
The discovery of a web service on port 8080 with the title "TOP SECRET DB" is a critical finding. This suggests that an interface to a sensitive database is accessible on the network. This finding is particularly alarming as it directly contradicts the existing risk documentation (\textit{Input\_3\_Current\_Risks\_JSON}), which incorrectly classifies this port as a secure false positive. This discrepancy points to a flaw in the organization's risk assessment and validation process.

% --- CONSOLIDATED RISK ASSESSMENT ---
\section{Consolidated Risk Assessment}

The following table synthesizes findings from the questionnaire, technical scan, and existing risk data into a prioritized list of identified risks.

\begin{table}[h!]
\centering
\caption{Summary of Identified Risks}
\label{tab:risks}
\begin{tabular}{@{}p{0.1\textwidth}p{0.25\textwidth}p{0.45\textwidth}p{0.1\textwidth}@{}}
\toprule
\textbf{Risk ID} & \textbf{Risk Title} & \textbf{Description} & \textbf{Severity} \\ \midrule
\textbf{R-01} & Exposed Sensitive Database Interface & An open web service on \texttt{10.5.5.5:8080} titled "TOP SECRET DB" suggests direct access to highly sensitive data. This contradicts the existing risk register. & \textbf{Critical} \\
\textbf{R-02} & Lack of MFA on Critical Systems & No MFA is enforced for email or computer logins, making user accounts highly susceptible to compromise via phishing or credential stuffing attacks. & \textbf{Critical} \\
\textbf{R-03} & Missing Acceptable Use Policy (AUP) & The absence of a formal AUP leaves the organization without enforceable rules for employee use of IT assets, increasing insider risk. & \textbf{High} \\
\textbf{R-04} & Outdated or Inaccurate Risk Register & The existing risk register incorrectly lists port 8080 as a secure false positive, indicating a failure in the risk management lifecycle to validate and update findings. & \textbf{Medium} \\ \bottomrule
\end{tabular}
\end{table}

% --- RECOMMENDATIONS ---
\section{Recommendations}

Based on the analysis, the following actions are recommended to mitigate the identified risks and improve the overall security posture of \textbf{Midnight Oil Studios}.

\subsection*{Immediate Actions (To Be Completed within 72 Hours)}
\begin{itemize}
    \item \textbf{Investigate and Secure R-01:} Immediately investigate the service running on \texttt{10.5.5.5:8080}.
    \begin{itemize}
        \item Determine the nature and sensitivity of the "TOP SECRET DB".
        \item Restrict all access to the service to only authorized personnel via firewall rules or service configuration.
        \item If the service is not essential, it should be disabled and removed.
    \end{itemize}
    \item \textbf{Deploy MFA for Privileged Accounts:} Begin the rollout of MFA for all administrative and executive accounts across all systems, especially email.
\end{itemize}

\subsection*{High-Priority Actions (To Be Completed within 30 Days)}
\begin{itemize}
    \item \textbf{Full MFA Implementation (R-02):} Complete the deployment of MFA for all user email accounts and computer logins.
    \item \textbf{Develop and Implement AUP (R-03):} Create a comprehensive Acceptable Use Policy that clearly defines the rules for using company IT resources. Ensure all employees read and acknowledge the policy.
    \item \textbf{Review and Update Risk Register (R-04):} Conduct a full review of the existing risk register. Validate all entries against current technical data and ensure a formal process is established for regular review and updates.
\end{itemize}

\end{document}
```