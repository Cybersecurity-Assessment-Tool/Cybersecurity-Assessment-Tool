```latex
\documentclass[12pt]{article}

% Preamble: Required Packages
\usepackage[margin=1in]{geometry}
\usepackage{pifont}
\usepackage{booktabs}
\usepackage{hyperref}
\usepackage{url}
\usepackage{seqsplit}
\usepackage{graphicx}
\usepackage{xcolor}

% Document Metadata
\title{Cybersecurity Posture Assessment Report}
\author{Cybersecurity Analyst}
\date{\today}

% Hyperref Setup
\hypersetup{
    colorlinks=true,
    linkcolor=blue,
    filecolor=magenta,      
    urlcolor=cyan,
    pdftitle={Cybersecurity Posture Assessment Report},
    pdfpagemode=FullScreen,
}

\begin{document}

\maketitle
\thispagestyle{empty}
\newpage

\tableofcontents
\newpage

% --- 1. Executive Summary ---
\section{Executive Summary}
This report provides a comprehensive cybersecurity assessment for \textbf{Iron Bridge Legal}, synthesizing data from technical network scans, an organizational security questionnaire, and a review of pre-existing risks. The analysis reveals several critical vulnerabilities and policy gaps that expose the organization to significant threats, including data breaches, unauthorized access, and ransomware attacks.

\textbf{Critical findings} include an exposed FTP server (\texttt{10.0.0.15}) running a vulnerable version of \texttt{vsftpd} (2.3.4) with anonymous login enabled. This configuration presents an immediate and severe risk. Furthermore, the lack of Multi-Factor Authentication (MFA) on email accounts is a critical security gap that significantly increases the risk of business email compromise.

\textbf{High-risk findings} stem from organizational policy deficiencies, such as the absence of an employee acceptable use policy and a lack of security awareness training for new hires. These gaps cultivate a high-risk environment where employees may be more susceptible to social engineering or accidental data exposure.

Immediate remediation of the technical vulnerabilities and implementation of the recommended security controls are strongly advised to improve the organization's defensive posture.

% --- 2. Organizational Information ---
\section{Organizational Information}
The following information was provided for the assessment.

\begin{tabular}{@{}ll}
\toprule
\textbf{Attribute} & \textbf{Value} \\
\midrule
Organization Name & \textbf{Iron Bridge Legal} \\
Email Domain & \texttt{IronBridgeLegal.com} \\
Website Domain & \href{http://www.IronBridgeLegal.com}{\texttt{www.IronBridgeLegal.com}} \\
External IP Address & \texttt{94.142.119.184} \\
\bottomrule
\end{tabular}

% --- 3. Security Control Review ---
\section{Security Control Review}
The following table summarizes the organization's responses to a security controls questionnaire. "No" answers indicate significant gaps in the security framework and are highlighted for review.

\begin{tabular}{@{}p{0.6\linewidth}cp{0.25\linewidth}@{}}
\toprule
\textbf{Control Question} & \textbf{Response} & \textbf{Analyst Note} \\
\midrule
Do you require MFA to access email? & \textcolor{red}{\ding{55}} & \textbf{Critical Risk.} Email is a primary target for attackers. \\
Do you require MFA to log into computers? & \textcolor{green}{\ding{51}} & Good Practice. \\
Do you require MFA to access sensitive data systems? & \textcolor{green}{\ding{51}} & Good Practice. \\
Does your organization have an employee acceptable use policy? & \textcolor{red}{\ding{55}} & \textbf{High Risk.} Lack of clear policy creates ambiguity. \\
Does your organization do security awareness training for new employees? & \textcolor{red}{\ding{55}} & \textbf{High Risk.} New hires are often targeted by attackers. \\
Does your organization do security awareness training for all employees at least once per year? & \textcolor{green}{\ding{51}} & Good Practice. \\
\bottomrule
\end{tabular}

% --- 4. Technical Scan Results ---
\section{Technical Scan Results}
A network scan was performed on the internal target \texttt{10.0.0.15}. The scan identified one open port with a critically vulnerable service configuration.

\subsection{Open Ports and Services}
\begin{tabular}{@{}llllll@{}}
\toprule
\textbf{Port} & \textbf{State} & \textbf{Service} & \textbf{Product} & \textbf{Version} & \textbf{Notes} \\
\midrule
21/tcp & Open & ftp & vsftpd & 2.3.4 & \textbf{CRITICAL:} Anonymous FTP login allowed. \\
\bottomrule
\end{tabular}

\subsection{Vulnerability Analysis}
The FTP service identified on port 21 presents two major security risks:
\begin{enumerate}
    \item \textbf{Vulnerable Software Version:} \texttt{vsftpd} version 2.3.4 is known to be vulnerable to a critical backdoor command execution vulnerability (CVE-2011-2523). An attacker can exploit this to gain a shell and execute arbitrary commands on the server.
    \item \textbf{Insecure Configuration:} The server permits anonymous FTP logins. This allows any unauthenticated user on the network to connect, potentially accessing, uploading, or downloading files. This could lead to a data breach or be used to stage malware for further attacks.
\end{enumerate}

% --- 5. Overall Risk Assessment ---
\section{Risk Assessment}
The following table consolidates findings from the technical scan, control review, and pre-existing risk data into a unified risk register.

\begin{tabular}{@{}p{0.3\linewidth}p{0.5\linewidth}l@{}}
\toprule
\textbf{Risk Name} & \textbf{Overview} & \textbf{Severity} \\
\midrule
\textbf{Vulnerable FTP Service} & The server at \texttt{10.0.0.15} is running \texttt{vsftpd 2.3.4}, which has a known remote code execution backdoor. & \textbf{Critical} \\
\addlinespace
\textbf{Anonymous FTP Access} & The FTP service allows anonymous, unauthenticated access, enabling potential data exfiltration or malware staging. & \textbf{Critical} \\
\addlinespace
\textbf{No MFA for Email Access} & Lack of MFA on email accounts makes them highly susceptible to phishing and account takeover attacks. & \textbf{Critical} \\
\addlinespace
\textbf{Lack of Acceptable Use Policy} & Without a formal policy, there is no enforceable standard for employee use of company assets, increasing insider risk. & \textbf{High} \\
\addlinespace
\textbf{No Security Training for New Hires} & New employees are not trained on security best practices, making them vulnerable to social engineering. & \textbf{High} \\
\addlinespace
\textbf{Outdated Windows Policy} & Workstations are running Windows 7, which is End-of-Life and no longer receives security updates. & \textbf{Medium} \\
\bottomrule
\end{tabular}

% --- 6. Recommendations ---
\section{Recommendations}
Based on the findings, the following prioritized actions are recommended to mitigate the identified risks.

\subsection{Immediate Priority (0-7 Days)}
\begin{enumerate}
    \item \textbf{Disable or Remediate FTP Service:} Immediately shut down the FTP service on \texttt{10.0.0.15}. If file transfer is a business requirement, replace it with a secure protocol like SFTP (SSH File Transfer Protocol) and ensure anonymous access is disabled.
    \item \textbf{Enforce MFA on Email:} Immediately enable and enforce MFA for all user accounts on the \texttt{IronBridgeLegal.com} email domain.
\end{enumerate}

\subsection{High Priority (1-4 Weeks)}
\begin{enumerate}
    \item \textbf{Develop Acceptable Use Policy (AUP):} Create and implement a formal AUP that clearly defines rules for the use of company networks, systems, and data. Require all employees to read and acknowledge the policy.
    \item \textbf{Implement Onboarding Security Training:} Integrate mandatory security awareness training into the new employee onboarding process. This training should cover phishing, password security, and the new AUP.
\end{enumerate}

\subsection{Medium Priority (1-3 Months)}
\begin{enumerate}
    \item \textbf{Upgrade End-of-Life Systems:} Begin the process of upgrading all workstations running Windows 7 to a modern, supported operating system such as Windows 10 or 11 to ensure they receive critical security patches.
\end{enumerate}

\end{document}
```