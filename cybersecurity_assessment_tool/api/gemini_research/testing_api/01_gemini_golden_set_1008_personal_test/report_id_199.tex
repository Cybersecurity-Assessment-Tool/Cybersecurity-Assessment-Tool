```latex
\documentclass[12pt]{article}

% --- PACKAGES ---
\usepackage[margin=1in]{geometry}
\usepackage{pifont} % For checkmarks and crosses
\usepackage{booktabs} % For professional tables
\usepackage{hyperref} % For hyperlinks
\usepackage{url} % For URL formatting
\usepackage{seqsplit} % To split long strings in tt font
\usepackage{graphicx} % For logo (placeholder)
\usepackage{fancyhdr} % For header/footer

% --- DOCUMENT & HYPERREF SETUP ---
\hypersetup{
    colorlinks=true,
    linkcolor=black,
    urlcolor=blue,
    pdftitle={Cybersecurity Assessment Report},
    pdfauthor={Cybersecurity Analyst},
    pdfsubject={Security Assessment},
    pdfkeywords={Security, Nmap, Risk, Assessment}
}

% --- HEADER & FOOTER ---
\pagestyle{fancy}
\fancyhf{} % Clear all header and footer fields
\fancyhead[L]{Cybersecurity Assessment Report}
\fancyhead[R]{Calyx Botany}
\fancyfoot[C]{\thepage}
\renewcommand{\headrulewidth}{0.4pt}
\renewcommand{\footrulewidth}{0.4pt}

% --- TITLE ---
\title{
    \vspace{2cm}
    \textbf{Cybersecurity Assessment Report} \\
    \large Prepared for: Calyx Botany
    \vspace{1.5cm}
}
\author{Cybersecurity Analyst}
\date{\today}

% --- BEGIN DOCUMENT ---
\begin{document}

\maketitle
\thispagestyle{empty}
\newpage

\tableofcontents
\newpage

% --- EXECUTIVE SUMMARY ---
\section{Executive Summary}
This report details the findings of a cybersecurity assessment conducted for Calyx Botany. The analysis is based on a combination of technical network scanning, a review of organizational security controls, and an evaluation of pre-existing risks.

The assessment identified two significant gaps in organizational security policy and one technical finding requiring immediate attention. The most critical issue is the absence of Multi-Factor Authentication (MFA) for computer logins, which exposes the organization to significant risk from compromised credentials. This is compounded by the lack of mandatory, annual security awareness training for all staff, which increases susceptibility to phishing and social engineering attacks.

Technically, an externally accessible SSH port was discovered. While a common administrative service, its exposure requires robust configuration to prevent unauthorized access. Recommendations are provided to address each identified risk, prioritizing the implementation of MFA and the establishment of a comprehensive security training program.

% --- ORGANIZATIONAL INFORMATION ---
\section{Organizational Information}
The following details were provided for the assessment. This information helps establish the context and scope of the review.

\begin{tabular}{@{}ll}
    \toprule
    \textbf{Detail} & \textbf{Value} \\
    \midrule
    Organization Name & Calyx Botany \\
    Email Domain & \texttt{CalyxBotany.org} \\
    Website Domain & \url{www.CalyxBotany.org} \\
    External IP Address & \seqsplit{\texttt{35.189.149.94}} \\
    \bottomrule
\end{tabular}

% --- SECURITY CONTROL REVIEW ---
\section{Security Control Review}
The following table summarizes the organization's responses to a security controls questionnaire. A red cross (\ding{55}) indicates a potential gap in security posture that increases risk.

\begin{table}[h!]
\centering
\begin{tabular}{p{0.7\textwidth}c}
    \toprule
    \textbf{Control Question} & \textbf{Response} \\
    \midrule
    Do you require MFA to access email? & \ding{51} \\
    \textbf{Do you require MFA to log into computers?} & \textbf{\color{red}\ding{55}} \\
    Do you require MFA to access sensitive data systems? & \ding{51} \\
    Does your organization have an employee acceptable use policy? & \ding{51} \\
    Does your organization do security awareness training for new employees? & \ding{51} \\
    \textbf{Does your organization do security awareness training for all employees at least once per year?} & \textbf{\color{red}\ding{55}} \\
    \bottomrule
\end{tabular}
\caption{Organizational Security Control Status.}
\end{table}

The review highlights two key areas of concern: the lack of MFA on computer endpoints and the absence of recurring, annual security training for all staff members.

% --- TECHNICAL SCAN RESULTS ---
\section{Technical Scan Results}
A network scan was performed on the specified target to identify open ports and exposed services.

\begin{itemize}
    \item \textbf{Target IP Address:} \seqsplit{\texttt{2001:db8::1}}
    \item \textbf{Scan Date:} As per scan metadata.
\end{itemize}

The following table details the open ports discovered on the target system.

\begin{table}[h!]
\centering
\begin{tabular}{lccl}
    \toprule
    \textbf{Port} & \textbf{State} & \textbf{Service (Inferred)} & \textbf{Notes} \\
    \midrule
    22 & Open & SSH & \begin{tabular}[t]{@{}l@{}}Secure Shell is a standard administrative service. \\ No version information was available in the scan. \\ Its exposure requires robust security controls.\end{tabular} \\
    \bottomrule
\end{tabular}
\caption{Open Ports Detected on Target.}
\end{table}

The presence of an open SSH port is a notable finding. If not properly secured, it can serve as a direct entry point for attackers.

% --- RISK ASSESSMENT ---
\section{Risk Assessment}
This section synthesizes the findings from the security control review and technical scan. No pre-existing vulnerabilities were reported. The following new risks have been identified and prioritized.

\begin{table}[h!]
\centering
\begin{tabular}{p{0.15\textwidth}p{0.55\textwidth}l}
    \toprule
    \textbf{Risk ID} & \textbf{Description} & \textbf{Severity} \\
    \midrule
    RISK-001 & \textbf{Lack of Multi-Factor Authentication (MFA) on employee computers.} A single compromised password could grant an attacker full access to an endpoint and the internal network. & \textbf{Critical} \\
    \addlinespace
    RISK-002 & \textbf{Security awareness training is not conducted annually for all employees.} This increases the likelihood of human error, such as falling for phishing attacks, leading to credential theft or malware infection. & \textbf{High} \\
    \addlinespace
    RISK-003 & \textbf{Exposed SSH administrative port on an external-facing system.} Without proper configuration (e.g., strong passwords, key-based authentication, IP whitelisting), this port is a prime target for brute-force and credential stuffing attacks. & \textbf{Medium} \\
    \bottomrule
\end{tabular}
\caption{Summary of Identified Risks.}
\end{table}

% --- RECOMMENDATIONS ---
\section{Recommendations}
The following actionable steps are recommended to mitigate the identified risks and improve the overall security posture of Calyx Botany.

\subsection*{Recommendation for RISK-001 (Critical)}
\textbf{Implement MFA for All Computer Logins:}
\begin{itemize}
    \item Immediately deploy and enforce a mandatory MFA solution for all employee computer and laptop logins (Windows, macOS, Linux).
    \item This is the most effective control to mitigate the risk of unauthorized access resulting from compromised credentials.
\end{itemize}

\subsection*{Recommendation for RISK-002 (High)}
\textbf{Establish a Mandatory Annual Security Training Program:}
\begin{itemize}
    \item Develop or procure a security awareness training program to be completed by all employees annually.
    \item The training should cover critical topics including phishing identification, password hygiene, social engineering tactics, and acceptable use policies.
    \item Track completion to ensure 100\% compliance.
\end{itemize}

\subsection*{Recommendation for RISK-003 (Medium)}
\textbf{Secure the Exposed SSH Service:}
\begin{itemize}
    \item \textbf{Audit Access Needs:} Determine if external SSH access to \seqsplit{\texttt{2001:db8::1}} is necessary. If not, block port 22 at the network firewall immediately.
    \item \textbf{Enforce Strong Controls:} If external access is required:
        \begin{itemize}
            \item Disable password-based authentication and enforce the use of strong cryptographic keys (e.g., ED25519).
            \item Restrict access to a whitelist of trusted IP addresses.
            \item Ensure the SSH server software is up-to-date and patched against known vulnerabilities.
        \end{itemize}
\end{itemize}

\end{document}
```