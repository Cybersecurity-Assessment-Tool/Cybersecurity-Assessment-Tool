```latex
\documentclass[12pt, a4paper]{article}

% Preamble: Required Packages and Document Setup
\usepackage[margin=1in]{geometry}
\usepackage{pifont} % For checkmarks and crosses (\ding)
\usepackage{booktabs} % For professional tables (\toprule, \midrule, \bottomrule)
\usepackage{hyperref} % For clickable links and ToC
\usepackage{url} % For formatting URLs
\usepackage{seqsplit} % For splitting long strings in \texttt
\usepackage{graphicx}
\usepackage{fancyhdr} % For headers and footers
\usepackage{lastpage} % To get the total number of pages
\usepackage[table]{xcolor} % For coloring table cells

% --- Document Metadata ---
\title{Cybersecurity Assessment Report \\ \large For: True Grit}
\author{Cybersecurity Analysis Division}
\date{\today}

% --- Hyperref Setup ---
\hypersetup{
    colorlinks=true,
    linkcolor=blue,
    filecolor=magenta,      
    urlcolor=cyan,
    pdftitle={Cybersecurity Assessment Report for True Grit},
    pdfpagemode=FullScreen,
}

% --- Header and Footer Setup ---
\pagestyle{fancy}
\fancyhf{} % Clear all header and footer fields
\fancyhead[L]{Cybersecurity Assessment Report}
\fancyhead[R]{True Grit}
\fancyfoot[C]{\thepage\ of \pageref{LastPage}}
\renewcommand{\headrulewidth}{0.4pt}
\renewcommand{\footrulewidth}{0.4pt}

% --- Document Start ---
\begin{document}

\maketitle
\thispagestyle{empty}
\newpage

\tableofcontents
\newpage

% ==============================================================================
\section{Executive Overview}
% ==============================================================================

This report presents a cybersecurity assessment for \textbf{True Grit}, conducted on \today. The analysis is based on a combination of technical network scanning, a review of organizational security controls via a questionnaire, and an evaluation of pre-existing risk data.

The assessment reveals a mixed security posture. The organization demonstrates a strong commitment to identity and access management, with Multi-Factor Authentication (MFA) consistently enforced across email, computer logins, and sensitive data systems. This is a commendable and critical security control.

However, significant gaps were identified in foundational security governance. The absence of a formal Employee Acceptable Use Policy (AUP) and the lack of mandatory, annual security awareness training for all staff members represent high-risk deficiencies. These gaps expose the organization to increased threats from insider risk, social engineering, and phishing attacks.

From a technical standpoint, the external network scan identified an open Secure Shell (SSH) port on an IPv6 address. While necessary for remote administration, an improperly secured, internet-facing SSH service is a common target for automated brute-force attacks.

This report provides a detailed breakdown of these findings and offers actionable recommendations to mitigate the identified risks and strengthen the overall security posture of \textbf{True Grit}.

% ==============================================================================
\section{Organizational Information}
% ==============================================================================

The following information was provided for the assessment.

\begin{tabular}{@{}ll}
    \toprule
    \textbf{Attribute} & \textbf{Value} \\
    \midrule
    Organization Name & \textbf{True Grit} \\
    Email Domain & \texttt{TrueGrit.net} \\
    Website Domain & \texttt{www.TrueGrit.net} \\
    External IP Address (IPv4) & \texttt{152.12.40.217} \\
    External IP Address (IPv6 Scan Target) & \seqsplit{\texttt{2001:db8::1}} \\
    \bottomrule
\end{tabular}

% ==============================================================================
\section{Security Control Review}
% ==============================================================================

The following table summarizes the organization's responses to the security controls questionnaire. Items marked with \ding{55} indicate a potential control gap and are discussed in the Risk Assessment section.

\begin{table}[h!]
\centering
\begin{tabular}{p{0.6\linewidth} c p{0.2\linewidth}}
    \toprule
    \textbf{Control Question} & \textbf{Response} & \textbf{Assessment} \\
    \midrule
    Do you require MFA to access email? & \ding{51} & Good Practice \\
    Do you require MFA to log into computers? & \ding{51} & Good Practice \\
    Do you require MFA to access sensitive data systems? & \ding{51} & Good Practice \\
    \rowcolor{red!15} Does your organization have an employee acceptable use policy? & \ding{55} & \textbf{Critical Gap} \\
    Does your organization do security awareness training for new employees? & \ding{51} & Good Practice \\
    \rowcolor{red!15} Does your organization do security awareness training for all employees at least once per year? & \ding{55} & \textbf{High Risk} \\
    \bottomrule
\end{tabular}
\caption{Security Controls Questionnaire Results}
\end{table}

% ==============================================================================
\section{Technical Scan Results}
% ==============================================================================

An external network scan was performed against the provided target IP address. The scan identified the following open ports.

\paragraph{Scan Target:} \seqsplit{\texttt{2001:db8::1}}

\begin{table}[h!]
\centering
\begin{tabular}{c c l l p{0.3\linewidth}}
    \toprule
    \textbf{Port} & \textbf{State} & \textbf{Service} & \textbf{Product/Version} & \textbf{Notes} \\
    \midrule
    22/tcp & Open & SSH (Inferred) & N/A & The Secure Shell service is exposed to the public internet. This is a primary target for automated brute-force login attempts. \\
    \bottomrule
\end{tabular}
\caption{Open Ports Detected on Target Host}
\end{table}

\paragraph{Analysis:} The scan results are minimal but significant. The presence of an open SSH port requires careful management. Without version information, it is not possible to determine if the running SSH server is vulnerable to known exploits. However, regardless of the version, any internet-facing administrative service presents a risk that must be actively managed.

% ==============================================================================
\section{Consolidated Risk Assessment}
% ==============================================================================

The following table consolidates findings from the security control review, technical scan, and pre-existing risk data. No pre-existing vulnerabilities were reported.

\begin{table}[h!]
\centering
\begin{tabular}{p{0.1\linewidth} p{0.2\linewidth} p{0.5\linewidth} c}
    \toprule
    \textbf{Risk ID} & \textbf{Risk Name} & \textbf{Description} & \textbf{Severity} \\
    \midrule
    RISK-001 & Lack of Acceptable Use Policy (AUP) & The organization does not have a formal policy defining the acceptable use of company assets. This leads to ambiguity and increases the risk of misuse and insider threats. & \textbf{High} \\
    \addlinespace
    RISK-002 & Insufficient Annual Security Training & While new hires are trained, there is no recurring annual training for all staff. This allows security knowledge to decay, increasing susceptibility to phishing and social engineering. & \textbf{High} \\
    \addlinespace
    RISK-003 & Exposed SSH Service & The SSH administrative service is open to the internet, creating a target for unauthorized access attempts. If not properly configured, it could lead to a system compromise. & \textbf{Medium} \\
    \bottomrule
\end{tabular}
\caption{Summary of Identified Risks}
\end{table}

% ==============================================================================
\section{Recommendations}
% ==============================================================================

The following actions are recommended to mitigate the identified risks and improve the organization's overall security posture.

\begin{enumerate}
    \item \textbf{Develop and Implement an Acceptable Use Policy (AUP)} \\
    \textit{Addresses: RISK-001 (High)} \\
    A formal AUP should be drafted, approved by management, and distributed to all employees. This policy must clearly define the rules for using company networks, devices, and data. All employees should be required to read and formally acknowledge the policy.

    \item \textbf{Establish a Mandatory Annual Security Awareness Program} \\
    \textit{Addresses: RISK-002 (High)} \\
    Implement a security awareness training program that is mandatory for all employees on an annual basis. The training should cover current threats such as phishing, ransomware, password security, and social engineering. This ensures that security remains a top-of-mind concern for all staff.

    \item \textbf{Harden the Exposed SSH Service} \\
    \textit{Addresses: RISK-003 (Medium)} \\
    Conduct a review to confirm the business necessity of exposing the SSH service on \seqsplit{\texttt{2001:db8::1}}. If it is required, implement the following hardening measures:
    \begin{itemize}
        \item \textbf{Disable password-based authentication} and enforce the use of strong SSH keys.
        \item \textbf{Implement rate-limiting} or an intrusion prevention tool (e.g., Fail2Ban) to block IPs that generate multiple failed login attempts.
        \item \textbf{Restrict access} to trusted source IP addresses using a firewall, if possible.
        \item Ensure the SSH server software is kept up-to-date with the latest security patches.
    \end{itemize}
\end{enumerate}

\end{document}
```