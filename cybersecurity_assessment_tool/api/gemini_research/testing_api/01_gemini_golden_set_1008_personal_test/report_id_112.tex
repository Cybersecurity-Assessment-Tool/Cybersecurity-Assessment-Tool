```latex
\documentclass[12pt]{article}

% Preamble: Required Packages
\usepackage[margin=1in]{geometry}
\usepackage{pifont} % For checkmarks and crosses
\usepackage{booktabs} % For professional tables
\usepackage{hyperref} % For clickable links
\usepackage{url} % For URL formatting
\usepackage{seqsplit} % For splitting long strings without spaces
\usepackage{graphicx}
\usepackage{xcolor}
\usepackage{lastpage}
\usepackage{fancyhdr}

% Document Metadata
\title{Cybersecurity Assessment Report \\ \large For Structure \& Form}
\author{Cybersecurity Analysis Division}
\date{\today}

% Header and Footer Configuration
\pagestyle{fancy}
\fancyhf{} % Clear all header and footer fields
\fancyhead[L]{Cybersecurity Assessment Report}
\fancyhead[R]{Structure \& Form}
\fancyfoot[C]{\thepage\ of \pageref{LastPage}}
\renewcommand{\headrulewidth}{0.4pt}
\renewcommand{\footrulewidth}{0.4pt}

% Hyperref Setup
\hypersetup{
    colorlinks=true,
    linkcolor=blue,
    filecolor=magenta,      
    urlcolor=cyan,
    pdftitle={Cybersecurity Assessment Report},
    pdfpagemode=FullScreen,
}

\begin{document}

\maketitle
\thispagestyle{empty}
\newpage

\tableofcontents
\newpage

% --- 1. Executive Summary ---
\section{Executive Summary}
This report provides a cybersecurity assessment for Structure \& Form, based on a combination of network scanning, organizational data review, and an analysis of pre-existing risks. The assessment was conducted on \today.

The overall security posture presents a mix of strengths and critical weaknesses. The external network scan of the target host \texttt{192.168.1.100} revealed a strong defensive configuration, with no open ports detected. This significantly reduces the external attack surface for that specific asset.

However, the review of security controls identified two significant gaps in administrative and policy-based defenses:
\begin{itemize}
    \item \textbf{Critical Risk:} The absence of Multi-Factor Authentication (MFA) on employee email accounts. Email is a primary vector for phishing and account takeover attacks, and this gap exposes the organization to significant threats like Business Email Compromise (BEC) and data breaches.
    \item \textbf{High Risk:} The lack of mandatory security awareness training for new employees during their onboarding process. This leaves new staff, who are often prime targets for social engineering, unprepared to identify and respond to threats.
\end{itemize}

No pre-existing vulnerabilities were reported. Immediate remediation should focus on implementing MFA for email and integrating security training into the employee onboarding process to mitigate these high-impact risks.

% --- 2. Organizational Information ---
\section{Organizational Information}
The following information was provided for the assessment.
\begin{center}
\begin{tabular}{ll}
\toprule
\textbf{Attribute} & \textbf{Value} \\
\midrule
Organization Name & Structure \& Form \\
Primary Email Domain & \texttt{StructureForm.net} \\
Primary Website Domain & \url{www.StructureForm.net} \\
Primary External IP & \texttt{179.167.59.150} \\
\bottomrule
\end{tabular}
\end{center}

% --- 3. Security Control Review ---
\section{Security Control Review}
A review of key administrative and technical security controls was conducted based on a questionnaire. The results are summarized below. "No" answers indicate potential security gaps that require attention.

\begin{center}
\begin{tabular}{p{0.6\textwidth} c l}
\toprule
\textbf{Control Question} & \textbf{Response} & \textbf{Assessment} \\
\midrule
Do you require MFA to access email? & \ding{55} & \textcolor{red}{\textbf{Critical Gap}} \\
Do you require MFA to log into computers? & \ding{51} & Meets Best Practice \\
Do you require MFA to access sensitive data systems? & \ding{51} & Meets Best Practice \\
Does your organization have an employee acceptable use policy? & \ding{51} & Meets Best Practice \\
Does your organization do security awareness training for new employees? & \ding{55} & \textcolor{orange}{\textbf{High Risk Gap}} \\
Does your organization do security awareness training for all employees at least once per year? & \ding{51} & Meets Best Practice \\
\bottomrule
\end{tabular}
\end{center}

% --- 4. Technical Scan Results ---
\section{Technical Scan Results}
An external network scan was performed to identify open ports and exposed services on the designated target system.

\subsection{Scan Details}
\begin{itemize}
    \item \textbf{Target IP Address:} \texttt{192.168.1.100}
    \item \textbf{Scan Type:} Nmap TCP Scan (Top 1000 Ports)
    \item \textbf{Scan Date:} \today
\end{itemize}

\subsection{Findings}
The scan confirmed that the host at \texttt{192.168.1.100} is online and responsive. However, \textbf{no open TCP ports were detected}. All 1000 scanned ports were found to be in a `closed` state.

\textbf{Conclusion:} This is a positive security finding. A host with no exposed services presents a minimal attack surface to external threats, indicating effective firewalling and network segmentation for this specific asset.

% --- 5. Consolidated Risk Assessment ---
\section{Consolidated Risk Assessment}
This section synthesizes findings from the security control review and technical scans. As no pre-existing vulnerabilities were provided, the risks listed below are derived directly from this assessment.

\begin{center}
\begin{tabular}{lp{0.5\textwidth}ll}
\toprule
\textbf{ID} & \textbf{Risk Description} & \textbf{Source} & \textbf{Severity} \\
\midrule
R-01 & \textbf{Lack of MFA on Email:} The absence of MFA on email accounts allows an attacker with compromised credentials (e.g., from a phishing attack) to gain full access, potentially leading to data exfiltration or BEC. & Questionnaire & \textcolor{red}{\textbf{Critical}} \\
\addlinespace
R-02 & \textbf{No Security Training for New Hires:} New employees are not formally trained on security policies and threat identification, making them highly susceptible to social engineering and phishing attacks during their initial, vulnerable period. & Questionnaire & \textcolor{orange}{\textbf{High}} \\
\bottomrule
\end{tabular}
\end{center}

% --- 6. Recommendations ---
\section{Recommendations}
The following actions are recommended to mitigate the identified risks and improve the overall security posture of Structure \& Form.

\subsection{R-01: Remediate Lack of MFA on Email (Critical)}
\begin{itemize}
    \item \textbf{Action:} Immediately enforce mandatory Multi-Factor Authentication (MFA) for all user accounts on the \texttt{StructureForm.net} email platform.
    \item \textbf{Justification:} MFA is the single most effective control to prevent unauthorized account access, even if user credentials are stolen. It is an industry-standard best practice for protecting critical systems like email.
    \item \textbf{Priority:} Immediate.
\end{itemize}

\subsection{R-02: Implement Onboarding Security Training (High)}
\begin{itemize}
    \item \textbf{Action:} Develop and integrate a mandatory security awareness training module into the new employee onboarding process. This training should be completed before a new hire is granted full access to company systems.
    \item \textbf{Justification:} Educating employees from day one establishes a security-conscious culture and provides them with the necessary knowledge to defend against common threats like phishing, malware, and social engineering. This closes a significant window of vulnerability.
    \item \textbf{Priority:} High.
\end{itemize}

\end{document}
```