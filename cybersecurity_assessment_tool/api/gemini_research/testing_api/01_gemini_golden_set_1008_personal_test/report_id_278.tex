```latex
\documentclass[12pt]{article}

% Preamble: Required Packages
\usepackage[margin=1in]{geometry}
\usepackage{pifont} % For dingbats (checkmarks and X's)
\usepackage{booktabs} % For professional-looking tables
\usepackage{hyperref} % For clickable links and better PDF navigation
\usepackage{url} % For formatting URLs
\usepackage{seqsplit} % To split long monospaced strings
\usepackage{xcolor} % For color definitions
\usepackage{graphicx} % To include graphics
\usepackage{fancyhdr} % For headers and footers

% --- Document Setup ---
\hypersetup{
    colorlinks=true,
    linkcolor=blue,
    filecolor=magenta,      
    urlcolor=cyan,
    pdftitle={Cybersecurity Posture Assessment Report},
    pdfpagemode=FullScreen,
}

% --- Custom Commands ---
\newcommand{\yes}{\ding{51}}
\newcommand{\no}{\ding{55}}
\newcommand{\severitycritical}{\textcolor{red}{\textbf{Critical}}}
\newcommand{\severityhigh}{\textcolor{orange}{\textbf{High}}}
\newcommand{\severitymedium}{\textcolor{yellow!80!black}{\textbf{Medium}}}

% --- Header and Footer ---
\pagestyle{fancy}
\fancyhf{} % Clear all header and footer fields
\fancyhead[L]{Cybersecurity Posture Assessment}
\fancyhead[R]{\textbf{Vertex Solutions}}
\fancyfoot[C]{\thepage}
\renewcommand{\headrulewidth}{0.4pt}
\renewcommand{\footrulewidth}{0.4pt}

\begin{document}

% --- Title Page ---
\begin{titlepage}
    \centering
    \vspace*{1cm}
    \includegraphics[width=0.4\textwidth]{example-image-a} % Placeholder for company logo
    
    \vspace{1.5cm}
    
    \Huge
    \textbf{Cybersecurity Posture Assessment Report}
    
    \vspace{1.5cm}
    
    \Large
    Prepared for: \textbf{Vertex Solutions}
    
    \vspace{0.5cm}
    
    \Large
    Date of Report: \today
    
    \vfill
    
    \large
    \textit{This report contains sensitive information and should be handled with care. Distribution is restricted to authorized personnel only.}
    
\end{titlepage}

\tableofcontents
\newpage

% --- Section 1: Executive Overview ---
\section{Executive Overview}
This report provides a comprehensive analysis of the cybersecurity posture for \textbf{Vertex Solutions}. The assessment is based on a correlation of organizational data, a technical network scan, and a review of pre-existing risks.

The overall security posture is assessed as \severitycritical. This assessment is driven by several critical findings that expose the organization to significant risk of unauthorized access, data breach, and operational disruption.

Key findings include:
\begin{itemize}
    \item \textbf{Critically Vulnerable Public-Facing Service:} An FTP server was identified with a known remote code execution vulnerability (\texttt{vsftpd 2.3.4}) and is configured to allow anonymous access. This represents an immediate and severe threat.
    \item \textbf{Systemic Lack of Multi-Factor Authentication (MFA):} MFA is not enforced for email, computer logins, or access to sensitive data systems. This significantly increases the risk of account compromise through phishing or credential theft.
    \item \textbf{Absence of Foundational Security Policies:} The organization lacks a formal Acceptable Use Policy and a structured security awareness training program. These gaps contribute to a weakened security culture and increased susceptibility to social engineering attacks.
    \item \textbf{Use of End-of-Life Software:} The continued use of Windows 7 workstations, which no longer receive security updates, exposes the internal network to a wide range of unpatchable vulnerabilities.
\end{itemize}

Immediate remediation of these issues is strongly recommended to reduce the organization's risk profile to an acceptable level. Detailed recommendations are provided in Section \ref{sec:recommendations}.

% --- Section 2: Organizational Information ---
\section{Organizational Information}
The following details were provided for the assessment.
\begin{table}[h!]
\centering
\begin{tabular}{@{}ll@{}}
\toprule
\textbf{Attribute} & \textbf{Value} \\ \midrule
Organization Name & \textbf{Vertex Solutions} \\
Email Domain & \texttt{VertexSolutions.net} \\
Website Domain & \url{www.VertexSolutions.net} \\
External IP Address & \texttt{12.11.1.249} \\ \bottomrule
\end{tabular}
\caption{Client Organizational Details.}
\end{table}

% --- Section 3: Security Control Review ---
\section{Security Control Review}
A review of administrative and technical security controls was conducted based on a questionnaire. The responses reveal critical gaps in the organization's defense-in-depth strategy.

\begin{table}[h!]
\centering
\begin{tabular}{@{}p{0.6\linewidth}cp{0.25\linewidth}@{}}
\toprule
\textbf{Control Question} & \textbf{Response} & \textbf{Assessment} \\ \midrule
Do you require MFA to access email? & \no & \severitycritical{} Gap \\
Do you require MFA to log into computers? & \no & \severitycritical{} Gap \\
Do you require MFA to access sensitive data systems? & \no & \severitycritical{} Gap \\
Does your organization have an employee acceptable use policy? & \no & \severityhigh{} Risk \\
Does your organization do security awareness training for new employees? & \no & \severityhigh{} Risk \\
Does your organization do security awareness training for all employees at least once per year? & \no & \severityhigh{} Risk \\ \bottomrule
\end{tabular}
\caption{Security Control Questionnaire Analysis.}
\end{table}

% --- Section 4: Technical Scan Results ---
\section{Technical Scan Results}
An external network scan was performed to identify open ports and exposed services. The scan targeted the host at \texttt{10.0.0.15}.

\subsection{Open Ports and Services}
A single open port was discovered, running a critically outdated and misconfigured service.

\begin{table}[h!]
\centering
\begin{tabular}{@{}cllll@{}}
\toprule
\textbf{Port} & \textbf{State} & \textbf{Service} & \textbf{Version} & \textbf{Details} \\ \midrule
21/tcp & Open & ftp & vsftpd 2.3.4 & \begin{tabular}[t]{@{}l@{}}Anonymous FTP login allowed.\\ \severitycritical{} Vulnerability (CVE-2011-2523).\end{tabular} \\ \bottomrule
\end{tabular}
\caption{Nmap Scan Findings for Host \texttt{10.0.0.15}.}
\end{table}

\subsection{Analysis of Technical Findings}
The FTP service running on port 21 is \textbf{vsftpd version 2.3.4}. This specific version contains a critical backdoor vulnerability (CVE-2011-2523) that was intentionally added to the source code. An attacker can exploit this vulnerability to gain a command shell on the underlying server, leading to a full system compromise.

Furthermore, the service is configured to allow \textbf{anonymous FTP login}, which enables any unauthenticated user on the internet to connect, list files, and potentially upload or download data. This configuration is highly insecure and should be disabled immediately.

% --- Section 5: Consolidated Risk Assessment ---
\section{Consolidated Risk Assessment}
The following table synthesizes findings from the security control review, technical scan, and pre-existing risk data into a consolidated list of identified risks.

\begin{table}[h!]
\centering
\begin{tabular}{@{}lp{0.4\linewidth}ll@{}}
\toprule
\textbf{Risk Name} & \textbf{Description} & \textbf{Source} & \textbf{Severity} \\ \midrule
Vulnerable FTP Server & An internet-facing FTP server is running a version with a known RCE backdoor and allows anonymous access. & Technical Scan & \severitycritical{} \\
No MFA Enforcement & Lack of MFA on all critical systems (email, logins) allows for simple account takeovers via credential theft. & Questionnaire & \severitycritical{} \\
Outdated Windows 7 & Workstations are running an end-of-life operating system (Windows 7) that no longer receives security updates. & Existing Risks & \severityhigh{} \\
Missing Policies \& Training & The absence of an AUP and security awareness training increases human-related risks like phishing and malware. & Questionnaire & \severityhigh{} \\ \bottomrule
\end{tabular}
\caption{Summary of Identified Risks.}
\end{table}

% --- Section 6: Recommendations ---
\section{Recommendations}
\label{sec:recommendations}
Based on the consolidated risk assessment, the following prioritized recommendations are provided to mitigate the identified threats.

\subsection{Immediate Priority (Remediate within 72 hours)}
\begin{enumerate}
    \item \textbf{Remediate Vulnerable FTP Server:} Immediately take the FTP server at \texttt{10.0.0.15} offline. If the service is business-critical, it must be upgraded to the latest stable version of vsftpd (or an alternative secure file transfer solution like SFTP) and reconfigured to disable anonymous access.
    \item \textbf{Enforce MFA for Critical Systems:} Begin emergency rollout of MFA for all external-facing services, administrator accounts, and email access. This is the single most effective control to prevent account compromise.
\end{enumerate}

\subsection{High Priority (Remediate within 30-60 days)}
\begin{enumerate}
    \item \textbf{Develop OS Upgrade Plan:} Create and execute a project to upgrade all Windows 7 workstations to a modern, supported operating system (e.g., Windows 10/11). Isolate remaining Windows 7 machines from the main network if they cannot be immediately upgraded.
    \item \textbf{Establish Foundational Policies:} Draft and implement a formal Acceptable Use Policy (AUP) that all employees must read and sign.
    \item \textbf{Implement Security Awareness Training:} Enroll all employees in a mandatory security awareness training program. This program should be required for all new hires and repeated annually for all staff.
\end{enumerate}

\subsection{Medium Priority (Remediate within 90 days)}
\begin{enumerate}
    \item \textbf{Conduct Comprehensive Vulnerability Scanning:} Perform authenticated and unauthenticated vulnerability scans across the entire internal and external network infrastructure to identify other potential weaknesses.
    \item \textbf{Expand MFA Coverage:} Extend MFA enforcement to all internal systems that support it, including computer logins and access to sensitive data repositories.
\end{enumerate}

\end{document}
```