```latex
\documentclass[12pt]{article}

% 1. DOCUMENT SETUP & PACKAGES
\usepackage[margin=1in]{geometry}
\usepackage{pifont} % For checkmarks and crosses
\usepackage{booktabs} % For professional tables
\usepackage{hyperref} % For hyperlinks
\usepackage{url} % For URL formatting
\usepackage{seqsplit} % To split long strings in tt font
\usepackage{fancyhdr} % For headers and footers
\usepackage{lastpage} % To get the total number of pages
\usepackage{xcolor} % For colors
\usepackage{graphicx} % For logos (placeholder)

% --- Hyperlink Setup ---
\hypersetup{
    colorlinks=true,
    linkcolor=blue,
    filecolor=magenta,      
    urlcolor=cyan,
    pdftitle={Cybersecurity Posture Assessment Report},
    pdfpagemode=FullScreen,
}

% --- Header and Footer Setup ---
\pagestyle{fancy}
\fancyhf{} % Clear all header and footer fields
\fancyhead[L]{\textbf{Cybersecurity Posture Assessment}}
\fancyhead[R]{Ember Glow Hospitality}
\fancyfoot[L]{\textit{Confidential}}
\fancyfoot[C]{Page \thepage\ of \pageref{LastPage}}
\fancyfoot[R]{\today}

% --- Custom Commands ---
\newcommand{\yes}{\ding{51}} % Checkmark
\newcommand{\no}{\ding{55}}  % X-mark
\newcommand{\riskcritical}[1]{\textcolor{red}{\textbf{#1}}}
\newcommand{\riskhigh}[1]{\textcolor{orange}{\textbf{#1}}}
\newcommand{\riskmedium}[1]{\textcolor{yellow!80!black}{\textbf{#1}}}

\begin{document}

% 2. TITLE PAGE
\begin{titlepage}
    \centering
    \vspace*{1cm}
    
    \Huge
    \textbf{Cybersecurity Posture Assessment Report}
    
    \vspace{1.5cm}
    
    \Large
    Prepared for:
    
    \vspace{0.5cm}
    
    \textbf{Ember Glow Hospitality}
    
    \vfill
    
    \large
    \textbf{Date of Report:} \today \\
    \textbf{Analysis Period:} October 2023
    
    \vspace{1.5cm}
    
    \textit{This document contains sensitive and confidential information. Distribution is restricted to authorized personnel only.}
    
\end{titlepage}

\tableofcontents
\newpage

% 3. EXECUTIVE SUMMARY
\section{Executive Summary}
This report details the findings of a cybersecurity posture assessment for \textbf{Ember Glow Hospitality}. The analysis combines a review of organizational security controls, technical network scanning, and a summary of pre-existing risks to provide a holistic view of the current security landscape.

The assessment identified several critical-risk findings that require immediate attention. The most severe issue is the systemic lack of Multi-Factor Authentication (MFA) across all key access points, including computer logins, email, and sensitive data systems. This administrative gap is dangerously compounded by technical findings, specifically the discovery of a new host (\texttt{10.10.10.51}) exposing Remote Desktop Protocol (RDP) to the network. This mirrors a pre-existing risk on another host, indicating a pattern of misconfiguration.

The combination of exposed RDP and the absence of MFA for computer logins creates a direct and high-probability pathway for ransomware attacks and unauthorized network access. An attacker who obtains a single user credential could gain significant control over internal systems.

Overall, the security posture of \textbf{Ember Glow Hospitality} is considered to be at a \riskcritical{High Risk} level. Immediate remediation of the identified MFA and RDP vulnerabilities is strongly recommended to mitigate the threat of a significant security incident.

% 4. ORGANIZATIONAL INFORMATION
\section{Organizational Information}
The following details were provided for the assessment.

\begin{table}[h!]
\centering
\begin{tabular}{@{}ll@{}}
\toprule
\textbf{Attribute} & \textbf{Value} \\ \midrule
Organization Name    & \textbf{Ember Glow Hospitality} \\
Email Domain         & \texttt{EmberGlowHospitality.org} \\
Website Domain       & \seqsplit{\url{www.EmberGlowHospitality.org}} \\
External IP Address  & \texttt{205.70.178.66} \\ \bottomrule
\end{tabular}
\caption{Client Organizational Data.}
\label{tab:org_data}
\end{table}

% 5. SECURITY CONTROL REVIEW (QUESTIONNAIRE)
\section{Security Control Review}
A review of the organization's administrative security controls was conducted via a questionnaire. The results highlight significant gaps in access control policies.

\begin{table}[h!]
\centering
\begin{tabular}{@{}p{0.6\textwidth}cc@{}}
\toprule
\textbf{Control Question} & \textbf{Response} & \textbf{Assessment} \\ \midrule
Do you require MFA to access email? & \no & \riskcritical{Critical Gap} \\
Do you require MFA to log into computers? & \no & \riskcritical{Critical Gap} \\
Do you require MFA to access sensitive data systems? & \no & \riskcritical{Critical Gap} \\
\addlinespace
Does your organization have an employee acceptable use policy? & \yes & Implemented \\
Does your organization do security awareness training for new employees? & \yes & Implemented \\
Does your organization do security awareness training for all employees at least once per year? & \yes & Implemented \\ \bottomrule
\end{tabular}
\caption{Security Controls Questionnaire Analysis.}
\label{tab:controls}
\end{table}

The lack of MFA for email, computer, and data system access represents a fundamental failure in identity and access management. While the organization's security awareness training program is a positive control, it is not sufficient to protect against credential compromise without the technical enforcement provided by MFA.

% 6. TECHNICAL SCAN RESULTS
\section{Technical Scan Results}
A network scan was performed to identify open ports and exposed services on the target system.

\subsection{Target: \texttt{10.10.10.51}}
The scan revealed the following open port on the specified internal host.

\begin{table}[h!]
\centering
\begin{tabular}{@{}llll@{}}
\toprule
\textbf{Port} & \textbf{State} & \textbf{Service Name} & \textbf{Analysis} \\ \midrule
3389/tcp & open & \texttt{ms-wbt-server} & \riskhigh{High Risk}. This is the Microsoft Remote \\
& & & Desktop Protocol (RDP). Exposing RDP without \\
& & & compensating controls (e.g., VPN, MFA) is a \\
& & & primary vector for ransomware attacks. \\ \bottomrule
\end{tabular}
\caption{Open Ports on \texttt{10.10.10.51}.}
\label{tab:nmap_results}
\end{table}

% 7. CONSOLIDATED RISK ASSESSMENT
\section{Consolidated Risk Assessment}
This section synthesizes findings from the security control review, technical scan, and pre-existing risk register into a consolidated list of key vulnerabilities.

\begin{table}[h!]
\centering
\begin{tabular}{@{}lp{0.2\textwidth}p{0.2\textwidth}cp{0.3\textwidth}@{}}
\toprule
\textbf{ID} & \textbf{Risk Name} & \textbf{Affected Asset(s)} & \textbf{Severity} & \textbf{Description} \\ \midrule
\textbf{RISK-001} & \textbf{RDP Exposure on New Host} & \texttt{10.10.10.51} & \riskcritical{Critical (9.8)} & The scan discovered RDP exposed on a new host. This, combined with the lack of MFA for computer logins (RISK-003), makes the network highly vulnerable to compromise via stolen credentials. \\
\addlinespace
\textbf{RISK-002} & \textbf{Systemic RDP Exposure} & \texttt{10.10.10.50} & \riskcritical{Critical (9.0)} & A pre-existing finding confirms RDP is also exposed on another host. This indicates a systemic configuration issue, not an isolated incident. \\
\addlinespace
\textbf{RISK-003} & \textbf{No MFA for Computer Logins} & All Workstations \& Servers & \riskhigh{High (8.8)} & Lack of a second authentication factor for computer access allows an attacker with valid credentials to log in unimpeded, facilitating lateral movement and ransomware deployment. \\
\addlinespace
\textbf{RISK-004} & \textbf{No MFA for Email Access} & \texttt{[Domain]} & \riskhigh{High (8.5)} & The corporate email system is vulnerable to account takeover, leading to Business Email Compromise (BEC), internal phishing, and data exfiltration. \\
\addlinespace
\textbf{RISK-005} & \textbf{No MFA for Sensitive Data} & Internal Data Systems & \riskhigh{High (8.8)} & Critical business and client data lacks a fundamental access control, increasing the risk of a significant data breach. \\
\bottomrule
\end{tabular}
\caption{Consolidated Risk Register.}
\label{tab:risk_register}
\end{table}

% 8. RECOMMENDATIONS
\section{Recommendations}
Based on the consolidated risk assessment, the following remediation actions are recommended. They are prioritized to address the most critical threats first.

\subsection{Immediate Priority (Remediate within 72 hours)}
\begin{itemize}
    \item \textbf{Contain RDP Exposure (RISK-001, RISK-002):} Immediately close TCP port 3389 on the network firewall for hosts \texttt{10.10.10.51} and \texttt{10.10.10.50}. If remote access is a business requirement, it must be placed behind a Virtual Private Network (VPN) that requires Multi-Factor Authentication.
    \item \textbf{Deploy MFA for Privileged Access:} At a minimum, immediately enforce MFA for all administrator accounts and any remote access solutions (e.g., VPN).
\end{itemize}

\subsection{High Priority (Remediate within 30 days)}
\begin{itemize}
    \item \textbf{Full MFA Rollout (RISK-003, RISK-004, RISK-005):} Implement a project to roll out MFA across the entire organization. The priority should be:
    \begin{enumerate}
        \item Email accounts (e.g., Microsoft 365, Google Workspace).
        \item All computer logins (e.g., via Windows Hello for Business, Duo).
        \item All applications hosting sensitive data.
    \end{enumerate}
    \item \textbf{Network Vulnerability Scan:} Conduct a comprehensive, authenticated vulnerability scan of the entire internal network to identify other misconfigurations or unpatched systems.
\end{itemize}

\subsection{Medium Priority (Remediate within 90 days)}
\begin{itemize}
    \item \textbf{Review Network Segmentation:} Ensure that critical servers are on separate network segments from user workstations to limit an attacker's ability to move laterally after an initial compromise.
    \item \textbf{Update Incident Response Plan:} Review and update the corporate Incident Response plan to specifically address scenarios for ransomware and Business Email Compromise, incorporating the new MFA and VPN controls.
\end{itemize}

\vfill
\begin{center}
    \textbf{--- End of Report ---}
\end{center}

\end{document}
```