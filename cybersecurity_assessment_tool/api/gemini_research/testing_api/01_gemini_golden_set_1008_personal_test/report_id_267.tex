```latex
\documentclass[12pt]{article}

% Preamble: Required Packages
\usepackage[a4paper, margin=1in]{geometry}
\usepackage{pifont} % For checkmarks and crosses
\usepackage{booktabs} % For professional tables
\usepackage{hyperref} % For clickable links
\usepackage{url} % For formatting URLs
\usepackage{seqsplit} % For splitting long strings in texttt
\usepackage[T1]{fontenc}

% Document Metadata
\title{Cybersecurity Posture Assessment Report}
\author{Cybersecurity Analysis Division}
\date{\today}

\begin{document}

\maketitle
\thispagestyle{empty}
\newpage

\tableofcontents
\thispagestyle{empty}
\newpage

\setcounter{page}{1}

% --- 1. Executive Summary ---
\section{Executive Summary}

This report provides a comprehensive cybersecurity assessment for \textbf{Signal Flare}, based on a synthesis of network scan data, organizational security controls, and pre-existing risk information.

The analysis reveals several critical and high-risk security gaps that require immediate attention. The most significant finding is a publicly accessible MySQL database (\texttt{172.16.50.20:3306}) running an outdated version (5.7.33), which is missing several years of security patches. This technical vulnerability is severely compounded by organizational policy gaps, most notably the lack of Multi-Factor Authentication (MFA) for email and sensitive data systems.

Furthermore, the absence of a consistent security awareness training program for employees increases the organization's susceptibility to social engineering attacks, such as phishing, which could serve as an initial entry point for an attacker to exploit the aforementioned technical weaknesses.

Immediate remediation should focus on restricting access to the exposed database, enforcing MFA across all critical systems, and patching the vulnerable database software.

% --- 2. Organizational Information ---
\section{Organizational Information}

The following details were provided for the assessment. This information is used to establish the context and scope of the review.

\begin{tabular}{@{}ll}
\toprule
\textbf{Attribute} & \textbf{Value} \\
\midrule
Organization Name & \textbf{Signal Flare} \\
Email Domain & \texttt{SignalFlare.org} \\
Website Domain & \url{www.SignalFlare.org} \\
External IP Address & \texttt{217.65.188.34} \\
\bottomrule
\end{tabular}

% --- 3. Security Control Review ---
\section{Security Control Review}

A review of the organization's security controls was conducted via a questionnaire. The responses highlight critical gaps in identity and access management and employee security training. A summary of the findings is presented in Table \ref{tab:controls}.

\begin{table}[h!]
\centering
\caption{Security Controls Questionnaire Results}
\label{tab:controls}
\begin{tabular}{@{}lc}
\toprule
\textbf{Control Question} & \textbf{Status} \\
\midrule
Do you require MFA to access email? & \ding{55} \\
Do you require MFA to log into computers? & \ding{51} \\
Do you require MFA to access sensitive data systems? & \ding{55} \\
Does your organization have an employee acceptable use policy? & \ding{51} \\
Does your organization do security awareness training for new employees? & \ding{55} \\
Does your organization do security awareness training for all employees annually? & \ding{55} \\
\bottomrule
\end{tabular}
\end{table}

\paragraph{Analysis:} The lack of MFA on email and sensitive data systems represents a critical risk. Email is a primary vector for account takeover and phishing attacks. The absence of security awareness training for both new and existing employees significantly elevates the risk of a successful social engineering attack.

% --- 4. Technical Scan Results ---
\section{Technical Scan Results}

A network scan was performed on the target system to identify open ports and exposed services. The results confirm the presence of a publicly accessible database service.

\subsection{Nmap Scan: \texttt{172.16.50.20}}
The scan identified the following open port:

\begin{table}[h!]
\centering
\caption{Open Ports and Services on \texttt{172.16.50.20}}
\label{tab:nmap}
\begin{tabular}{@{}lllll}
\toprule
\textbf{Port} & \textbf{State} & \textbf{Service} & \textbf{Product} & \textbf{Version} \\
\midrule
3306/tcp & open & mysql & MySQL & 5.7.33 \\
\bottomrule
\end{tabular}
\end{table}

\paragraph{Analysis:} Port 3306 is the default port for MySQL. Exposing a database service directly to the network is a significant security risk. The identified version, \textbf{MySQL 5.7.33}, was released in early 2021 and is missing numerous security patches, leaving it vulnerable to known exploits.

% --- 5. Risk Assessment ---
\section{Risk Assessment}

The following table synthesizes findings from the security control review, technical scan, and pre-existing risk data into a prioritized list of security risks.

\begin{table}[h!]
\centering
\caption{Consolidated Risk Summary}
\label{tab:risks}
\begin{tabular}{@{}p{0.25\linewidth}p{0.1\linewidth}p{0.6\linewidth}@{}}
\toprule
\textbf{Risk Name} & \textbf{Severity} & \textbf{Description \& Affected Elements} \\
\midrule
\textbf{Lack of Multi-Factor Authentication (MFA)} & \textbf{Critical} & The absence of MFA on email and sensitive data systems exposes the organization to account takeover, data breaches, and phishing. \newline \textit{Affected: All employees, all sensitive data.} \\
\addlinespace
\textbf{Database Exposure} & High (7.5) & A MySQL database is directly accessible from the network, allowing attackers to attempt brute-force attacks, exploit vulnerabilities, or exfiltrate data. \newline \textit{Affected: \texttt{172.16.50.20:3306}} \\
\addlinespace
\textbf{Outdated Database Software} & High & The MySQL server is running version 5.7.33, which is outdated and vulnerable to publicly known exploits that have been patched in newer versions. \newline \textit{Affected: \texttt{172.16.50.20:3306}} \\
\addlinespace
\textbf{Insufficient Security Awareness Training} & High & The lack of a formal training program for new and existing employees makes the organization highly susceptible to social engineering and phishing attacks. \newline \textit{Affected: All employees.} \\
\bottomrule
\end{tabular}
\end{table}

% --- 6. Recommendations ---
\section{Recommendations}

The following actionable recommendations are provided to mitigate the identified risks. They are prioritized based on severity and potential impact.

\subsection{Immediate Priority (Critical)}
\begin{enumerate}
    \item \textbf{Restrict Database Access:} Immediately implement strict firewall rules to block all public access to port 3306 on \texttt{172.16.50.20}. Access should only be permitted from trusted application servers or via a secure VPN connection.
    \item \textbf{Enforce MFA Everywhere:} Enable MFA for all employees on all critical systems, with the highest priority on email (e.g., Office 365, Google Workspace) and any systems identified as containing sensitive data.
\end{enumerate}

\subsection{High Priority}
\begin{enumerate}
    \setcounter{enumi}{2} % Continue numbering
    \item \textbf{Patch Database Software:} Schedule and perform an upgrade of the MySQL server from version 5.7.33 to the latest stable and patched version (e.g., 5.7.44 or a newer major version like 8.x). This mitigates risks from known vulnerabilities.
    \item \textbf{Implement Security Awareness Training:} Procure and deploy a security awareness training program. All new hires must complete the training as part of their onboarding, and all existing staff must complete it annually.
\end{enumerate}

\subsection{Long-Term Strategy}
\begin{enumerate}
    \setcounter{enumi}{4} % Continue numbering
    \item \textbf{Adopt a Secure Baseline:} Develop a security baseline for all new systems that includes principles like least privilege, default-deny firewall rules, and mandatory MFA.
    \item \textbf{Implement Network Segmentation:} Review and enhance the network architecture to properly segment critical systems (like databases) from user-facing networks to limit lateral movement in the event of a breach.
\end{enumerate}

\end{document}
```