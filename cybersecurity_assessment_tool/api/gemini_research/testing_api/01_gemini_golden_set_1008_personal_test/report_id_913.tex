```latex
\documentclass[12pt]{article}

% Preamble: Required Packages
\usepackage[margin=1in]{geometry}
\usepackage{pifont} % For checkmarks and crosses
\usepackage{booktabs} % For professional tables
\usepackage{hyperref} % For clickable links
\usepackage{url} % For URL formatting
\usepackage{seqsplit} % To split long text sequences
\usepackage[T1]{fontenc}

% Document Metadata
\hypersetup{
    colorlinks=true,
    linkcolor=blue,
    filecolor=magenta,      
    urlcolor=cyan,
    pdftitle={Cybersecurity Posture Assessment Report},
    pdfauthor={Cybersecurity Analyst},
    pdfsubject={Security Analysis},
    pdfkeywords={Cybersecurity, Risk, Assessment},
}

\begin{document}

% --- Title Page ---
\begin{titlepage}
    \centering
    \vspace*{\stretch{1.0}}
    \Huge{\textbf{Cybersecurity Posture Assessment Report}}
    \vspace{0.5cm}
    \LARGE{Prepared for: Granite Shield}
    \vspace{1.5cm}
    \large{Date of Report: \today}
    \vspace{0.5cm}
    \large{Scan Date: 2025-11-22}
    \vspace*{\stretch{2.0}}
    \normalsize{This report contains sensitive information. Distribution should be limited to authorized personnel only.}
\end{titlepage}

\tableofcontents
\newpage

% --- Executive Summary ---
\section{Executive Summary}
This report provides a comprehensive analysis of the cybersecurity posture of Granite Shield, based on a combination of network scanning, organizational data review, and an assessment of current risks. The analysis was conducted on \today, using data from a network scan performed on 2025-11-22.

The assessment identified several critical and high-risk findings that require immediate attention. Key vulnerabilities include the \textbf{lack of Multi-Factor Authentication (MFA) on email accounts}, the use of an \textbf{outdated and potentially vulnerable web server (Nginx 1.18.0)}, and a procedural gap in \textbf{security awareness training for new employees}. 

While the organization has implemented some positive security controls, such as MFA for computer and sensitive system access, the identified weaknesses present significant risks, including account compromise, data breach, and reputational damage. This report outlines these findings in detail and provides actionable recommendations to mitigate the identified risks and strengthen the organization's overall security posture.

% --- Organizational Information ---
\section{Organizational Information}
The following details were provided for the assessment. This information is used to establish the context for the technical and procedural findings.

\begin{itemize}
    \item \textbf{Organization Name:} Granite Shield
    \item \textbf{Email Domain:} \texttt{GraniteShield.net}
    \item \textbf{Website Domain:} \texttt{www.GraniteShield.net}
    \item \textbf{External IP Address:} \texttt{8.237.10.130}
\end{itemize}

% --- Security Control Review ---
\section{Security Control Review}
A review of the organization's security controls was conducted based on a standard questionnaire. The responses indicate key areas of strength and weakness. Gaps identified here are directly correlated with organizational risk.

\begin{table}[h!]
\centering
\caption{Security Controls Questionnaire Results}
\begin{tabular}{@{}lc@{}}
\toprule
\textbf{Control Question} & \textbf{Response} \\ \midrule
Do you require MFA to access email? & \ding{55} \\
Do you require MFA to log into computers? & \ding{51} \\
Do you require MFA to access sensitive data systems? & \ding{51} \\
Does your organization have an employee acceptable use policy? & \ding{51} \\
Does your organization do security awareness training for new employees? & \ding{55} \\
Does your organization do security awareness training for all employees at least once per year? & \ding{51} \\ \bottomrule
\end{tabular}
\end{table}

\textbf{Analysis:} The absence of MFA for email access is a critical security gap. Email is a primary target for attackers and often serves as the recovery method for other critical accounts. Additionally, the lack of security training during employee onboarding exposes the organization to immediate risk from phishing and social engineering attacks.

% --- Technical Scan Results ---
\section{Technical Scan Results}
An external network scan was performed to identify open ports and exposed services.

\begin{itemize}
    \item \textbf{Scan Target:} \texttt{192.168.10.5}
    \item \textbf{Scan Date:} 2025-11-22T10:00:00Z
\end{itemize}

The scan revealed the following open port(s):

\begin{table}[h!]
\centering
\caption{Open Ports and Services}
\begin{tabular}{@{}lllll@{}}
\toprule
\textbf{Port} & \textbf{State} & \textbf{Service} & \textbf{Product} & \textbf{Version} \\ \midrule
443/TCP & open & https & nginx & 1.18.0 \\ \bottomrule
\end{tabular}
\end{table}

\textbf{Analysis:} The target system is running an Nginx web server, version \textbf{1.18.0}, which was released in April 2020. This version is significantly outdated and is known to be affected by multiple security vulnerabilities (e.g., CVE-2021-23017). Running unsupported or outdated software on internet-facing systems poses a high risk of compromise.

Additionally, the SSL certificate presented by the server has a Common Name of \texttt{www.acme-corp.com}, which does not match the organization's domain. This mismatch can cause browser trust errors and may indicate a server misconfiguration.

% --- Risk Assessment ---
\section{Risk Assessment}
The following table synthesizes findings from the security control review and technical scan into a prioritized list of risks. As per the provided data, there were no pre-existing risks to include.

\begin{table}[h!]
\centering
\caption{Identified Risks and Severity}
\begin{tabular}{@{}p{0.1\linewidth}p{0.3\linewidth}p{0.15\linewidth}p{0.35\linewidth}@{}}
\toprule
\textbf{Risk ID} & \textbf{Risk Name} & \textbf{Severity} & \textbf{Description} \\ \midrule
RISK-001 & Lack of MFA on Email & \textbf{Critical} & Email accounts are not protected by MFA, making them highly susceptible to compromise via phishing or credential stuffing. A compromised email account can lead to a full-scale breach. \\
\addlinespace
RISK-002 & Outdated Web Server Software & \textbf{High} & The Nginx server (v1.18.0) is outdated and has known vulnerabilities, exposing the system at \texttt{192.168.10.5} to potential remote code execution or denial-of-service attacks. \\
\addlinespace
RISK-003 & Inadequate Employee Onboarding Security & \textbf{High} & New employees do not receive security awareness training upon hiring. This creates a window of high vulnerability where new staff are more likely to fall victim to social engineering or policy violations. \\
\addlinespace
RISK-004 & SSL Certificate Mismatch & Low & The server's SSL certificate common name does not match the organization's domain, which can erode user trust and may indicate a configuration error. \\ \bottomrule
\end{tabular}
\end{table}

% --- Recommendations ---
\section{Recommendations}
The following actions are recommended to mitigate the identified risks and improve the overall security posture of Granite Shield.

\begin{enumerate}
    \item \textbf{(RISK-001) Enforce MFA on All Email Accounts:} Immediately implement and enforce the use of Multi-Factor Authentication for all user mailboxes. This is the single most effective control to prevent unauthorized access to email.
    
    \item \textbf{(RISK-002) Upgrade Nginx Web Server:} The Nginx server at \texttt{192.168.10.5} must be upgraded from version 1.18.0 to the latest stable version. Before upgrading, perform a full backup. After upgrading, subscribe to vendor security advisories to ensure timely patching in the future.
    
    \item \textbf{(RISK-003) Implement Onboarding Security Training:} Develop and integrate a mandatory security awareness training module into the new employee onboarding process. This training should cover, at a minimum, phishing, acceptable use, and password security.
    
    \item \textbf{(RISK-004) Correct SSL Certificate Configuration:} Review and reissue the SSL certificate for the service hosted at \texttt{192.168.10.5} to ensure its Common Name and Subject Alternative Names (SANs) correctly match the domain(s) it serves (e.g., \texttt{www.GraniteShield.net}).
\end{enumerate}

\end{document}
```