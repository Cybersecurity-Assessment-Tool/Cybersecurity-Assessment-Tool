```latex
\documentclass[12pt]{article}

% Preamble: Required Packages
\usepackage[margin=1in]{geometry}
\usepackage{pifont} % For checkmarks and crosses
\usepackage{booktabs} % For professional tables
\usepackage{hyperref} % For clickable links
\usepackage{url}      % For URL formatting
\usepackage{seqsplit} % For splitting long strings in texttt

% Document Metadata
\title{Cybersecurity Assessment Report}
\author{Cybersecurity Analyst}
\date{\today}

% Hyperref Setup
\hypersetup{
    colorlinks=true,
    linkcolor=black,
    urlcolor=blue,
    pdftitle={Cybersecurity Assessment Report},
    pdfauthor={Cybersecurity Analyst},
}

\begin{document}

\maketitle

\section*{Executive Overview}
This report provides a summary of the cybersecurity posture for \textbf{Aetheric Systems}, based on a combination of a technical network scan, a security controls questionnaire, and a review of pre-existing risks.

The assessment revealed a mixed security posture. On a positive note, the technical scan of the target host \texttt{192.168.1.100} indicated a strong network security configuration, with no open ports detected. This suggests a well-configured firewall or a system with no exposed network services, significantly reducing its external attack surface.

However, the security controls review identified two high-risk procedural gaps. The absence of Multi-Factor Authentication (MFA) on email accounts presents a critical vulnerability, exposing the organization to significant risks such as Business Email Compromise (BEC) and unauthorized data access. Additionally, the lack of security awareness training for new employees creates an immediate risk, as new hires may be unaware of security policies and more susceptible to social engineering attacks.

Immediate remediation should focus on closing these procedural gaps to mitigate the most significant threats to the organization.

\section{Organizational Information}
The following information was provided for the assessment:
\begin{itemize}
    \item \textbf{Organization Name:} Aetheric Systems
    \item \textbf{Email Domain:} \seqsplit{\texttt{AethericSystems.org}}
    \item \textbf{Website Domain:} \seqsplit{\url{www.AethericSystems.org}}
    \item \textbf{External IP Address:} \seqsplit{\texttt{64.244.122.153}}
\end{itemize}

\section{Security Control Review}
The following table summarizes the organization's responses to a security controls questionnaire. Items marked with \ding{55} represent identified gaps in security posture.

\begin{table}[h!]
\centering
\begin{tabular}{p{0.7\textwidth} c}
\toprule
\textbf{Control Question} & \textbf{Response} \\
\midrule
Do you require MFA to access email? & \ding{55} \\
Do you require MFA to log into computers? & \ding{51} \\
Do you require MFA to access sensitive data systems? & \ding{51} \\
Does your organization have an employee acceptable use policy? & \ding{51} \\
Does your organization do security awareness training for new employees? & \ding{55} \\
Does your organization do security awareness training for all employees at least once per year? & \ding{51} \\
\bottomrule
\end{tabular}
\caption{Security Controls Questionnaire Results}
\end{table}

\section{Technical Scan Results}
A network scan was performed on the specified target to identify its external posture.

\begin{itemize}
    \item \textbf{Target IP:} \seqsplit{\texttt{192.168.1.100}}
    \item \textbf{Scan Summary:} The scan confirmed that the host is online. However, no open ports were detected. All other scanned ports were reported as 'closed'.
    \item \textbf{Analysis:} This is a positive security finding. It indicates a strong firewall configuration or a lack of network-facing services on this host, which effectively minimizes its network-based attack surface.
\end{itemize}

\section{Risk Assessment}
The following table details the risks identified during this assessment, derived from correlating the technical scan and the security control review. No pre-existing risks were provided.

\begin{table}[h!]
\centering
\begin{tabular}{p{0.1\textwidth} p{0.25\textwidth} p{0.45\textwidth} p{0.1\textwidth}}
\toprule
\textbf{Risk ID} & \textbf{Risk Name} & \textbf{Description} & \textbf{Severity} \\
\midrule
RISK-001 & Lack of MFA on Email & The absence of Multi-Factor Authentication on email accounts significantly increases the risk of unauthorized access through credential theft or phishing, potentially leading to Business Email Compromise (BEC) and data breaches. & \textbf{High} \\
\addlinespace
RISK-002 & No Onboarding Security Training & New employees are not provided with security awareness training upon hiring. This gap leaves them vulnerable to social engineering attacks and unaware of organizational security policies, creating an immediate insider risk. & \textbf{High} \\
\bottomrule
\end{tabular}
\caption{Identified Security Risks}
\end{table}

\section{Recommendations}
The following actions are recommended to mitigate the identified risks and improve the overall security posture of \textbf{Aetheric Systems}.

\begin{description}
    \item[High Priority - Remediate RISK-001:] Implement and enforce mandatory Multi-Factor Authentication (MFA) for all user access to the corporate email system. Solutions like authenticator apps (e.g., Google Authenticator, Microsoft Authenticator) or hardware tokens should be deployed immediately to protect against account takeovers.

    \item[High Priority - Remediate RISK-002:] Develop and implement a mandatory security awareness training module as part of the new employee onboarding process. This training should cover, at a minimum, phishing identification, acceptable use policies, password hygiene, and procedures for reporting security incidents.

    \item[General Recommendation:] Continue to maintain the strong network security posture observed on the scanned host. We recommend performing regular, authenticated vulnerability scans on internal assets and periodic external scans to ensure configurations remain secure and to identify any new weaknesses.
\end{description}

\end{document}
```