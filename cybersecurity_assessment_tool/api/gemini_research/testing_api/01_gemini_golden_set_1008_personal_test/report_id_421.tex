```latex
\documentclass[12pt]{article}

% --- PACKAGES ---
\usepackage[margin=1in]{geometry}
\usepackage{pifont} % For checkmarks and crosses
\usepackage{booktabs} % For professional tables
\usepackage{hyperref} % For clickable links
\usepackage{url}      % For URL formatting
\usepackage{seqsplit} % To split long strings without breaking

% --- DOCUMENT METADATA ---
\hypersetup{
    colorlinks=true,
    linkcolor=black,
    urlcolor=blue,
    pdftitle={Cybersecurity Posture Assessment Report},
    pdfauthor={Cybersecurity Analyst},
    pdfsubject={Security Analysis},
    pdfkeywords={Cybersecurity, Risk, Assessment, Nmap, LaTeX}
}

% --- TITLE ---
\title{Cybersecurity Posture Assessment Report \\ \large For: Infinity Loop}
\author{Cybersecurity Analyst}
\date{November 22, 2025}

\begin{document}

\maketitle
\thispagestyle{empty}
\newpage

\tableofcontents
\newpage

% --- 1. EXECUTIVE SUMMARY ---
\section{Executive Summary}
This report provides a comprehensive cybersecurity posture assessment for Infinity Loop, conducted on November 22, 2025. The analysis correlates data from an external network scan, an organizational security questionnaire, and a review of pre-existing risks.

The assessment identified several critical and high-risk security gaps. The most significant findings include the absence of Multi-Factor Authentication (MFA) for computer and sensitive data access, a complete lack of a security awareness training program, and an externally-facing web server running outdated and potentially vulnerable software (nginx 1.18.0).

These findings indicate a high likelihood of compromise from common attack vectors such as phishing, credential theft, and exploitation of known software vulnerabilities. Immediate remediation is strongly recommended to mitigate these risks and improve the organization's overall security posture. This report details each finding and provides actionable recommendations for remediation.

% --- 2. ORGANIZATIONAL INFORMATION ---
\section{Organizational Information}
The following information was provided by the client and used as a baseline for this assessment.

\begin{itemize}
    \item \textbf{Organization Name:} Infinity Loop
    \item \textbf{Email Domain:} \texttt{InfinityLoop.org}
    \item \textbf{Website Domain:} \url{www.InfinityLoop.org}
    \item \textbf{Monitored External IP:} \texttt{55.195.111.226}
\end{itemize}

% --- 3. SECURITY CONTROL REVIEW ---
\section{Security Control Review}
A security questionnaire was completed to evaluate existing administrative and technical controls. The results reveal critical deficiencies in identity and access management and employee security awareness.

\begin{table}[h!]
\centering
\caption{Security Controls Questionnaire Results}
\begin{tabular}{@{}lc@{}}
\toprule
\textbf{Control Question} & \textbf{Status} \\
\midrule
Do you require MFA to access email? & \ding{51} \\
Do you require MFA to log into computers? & \textbf{\color{red}\ding{55}} \\
Do you require MFA to access sensitive data systems? & \textbf{\color{red}\ding{55}} \\
Does your organization have an employee acceptable use policy? & \textbf{\color{red}\ding{55}} \\
Does your organization do security awareness training for new employees? & \textbf{\color{red}\ding{55}} \\
Does your organization do security awareness training for all employees annually? & \textbf{\color{red}\ding{55}} \\
\bottomrule
\end{tabular}
\end{table}

\subsection*{Analysis of Control Gaps}
The responses marked with \textbf{\color{red}\ding{55}} (No) represent significant security gaps:
\begin{itemize}
    \item \textbf{Lack of MFA:} The absence of MFA on computer and sensitive data system logins is a critical vulnerability. If an attacker compromises a single user's password, they could gain widespread access to the internal network and critical data without any secondary challenge.
    \item \textbf{Lack of Security Policies and Training:} Without an acceptable use policy and a formal security awareness training program, employees are more likely to fall victim to social engineering attacks (e.g., phishing) and may mishandle sensitive data, creating a high-risk environment.
\end{itemize}

% --- 4. TECHNICAL SCAN RESULTS ---
\section{Technical Scan Results}
An external network scan was performed against the target IP address \texttt{192.168.10.5} to identify open ports and exposed services.

\subsection*{Nmap Scan Findings}
\begin{table}[h!]
\centering
\caption{Open Ports and Services on \texttt{192.168.10.5}}
\begin{tabular}{@{}lllll@{}}
\toprule
\textbf{Port} & \textbf{State} & \textbf{Service} & \textbf{Product} & \textbf{Version} \\
\midrule
443/tcp & open & https & nginx & 1.18.0 \\
\bottomrule
\end{tabular}
\end{table}

\subsection*{Analysis of Technical Findings}
\begin{itemize}
    \item \textbf{Outdated Web Server Software:} The scan identified an Nginx web server running version \textbf{1.18.0}. This version was released in April 2020 and is significantly outdated. It is known to be vulnerable to multiple Common Vulnerabilities and Exposures (CVEs), including but not limited to CVE-2021-23017. Running outdated software on an internet-facing system presents a high risk of exploitation.
    \item \textbf{Potential Certificate Misconfiguration:} The SSL certificate presented by the server has a Common Name of \texttt{www.acme-corp.com}, which does not match the organization's domain (\texttt{www.InfinityLoop.org}). This misconfiguration can cause trust errors for users and may indicate a deployment oversight.
\end{itemize}

% --- 5. RISK ASSESSMENT SUMMARY ---
\section{Risk Assessment Summary}
The following table synthesizes the findings from the security control review and the technical scan into a prioritized list of identified risks. No pre-existing risks were provided.

\begin{table}[h!]
\centering
\caption{Identified Cybersecurity Risks}
\begin{tabular}{@{}p{0.1\linewidth} p{0.25\linewidth} p{0.45\linewidth} p{0.1\linewidth}@{}}
\toprule
\textbf{Risk ID} & \textbf{Risk Name} & \textbf{Description} & \textbf{Severity} \\
\midrule
RISK-001 & Inadequate MFA Coverage & Lack of MFA on computer and sensitive data systems allows for straightforward account takeovers if credentials are stolen. & \textbf{Critical} \\
\addlinespace
RISK-002 & Outdated Web Server Software & The public-facing nginx server is running an old version with known vulnerabilities, making it a prime target for automated attacks. & \textbf{High} \\
\addlinespace
RISK-003 & Lack of Security Awareness Program & The absence of policies and training makes the organization highly susceptible to phishing, malware, and insider threats. & \textbf{High} \\
\addlinespace
RISK-004 & SSL Certificate Misconfiguration & The server's SSL certificate does not match the organization's domain, which can erode user trust and may indicate other configuration issues. & Medium \\
\bottomrule
\end{tabular}
\end{table}

% --- 6. RECOMMENDATIONS ---
\section{Recommendations}
The following actions are recommended to mitigate the identified risks and strengthen the overall security posture of Infinity Loop.

\begin{enumerate}
    \item \textbf{[RISK-001 - Critical]: Implement Comprehensive MFA.}
    \begin{itemize}
        \item Immediately enforce MFA for all user logins to workstations and servers.
        \item Enforce MFA for access to all applications and systems containing sensitive or critical data.
        \item Prioritize implementation for administrative and privileged accounts.
    \end{itemize}

    \item \textbf{[RISK-002 - High]: Remediate Vulnerable Web Server.}
    \begin{itemize}
        \item Upgrade the nginx server on \texttt{192.168.10.5} to the latest stable version immediately.
        \item Implement a formal patch management policy to ensure all internet-facing systems are updated in a timely manner.
    \end{itemize}
    
    \item \textbf{[RISK-003 - High]: Establish a Security Awareness Program.}
    \begin{itemize}
        \item Develop and enforce an employee Acceptable Use Policy (AUP).
        \item Implement mandatory security awareness training for all new hires upon onboarding.
        \item Conduct annual security awareness training and regular phishing simulations for all employees to build a culture of security.
    \end{itemize}

    \item \textbf{[RISK-004 - Medium]: Correct SSL Certificate.}
    \begin{itemize}
        \item Investigate the certificate mismatch on the web server.
        \item Procure and install a valid SSL certificate that matches the \texttt{www.InfinityLoop.org} domain.
    \end{itemize}
\end{enumerate}

% --- 7. CONCLUSION ---
\section{Conclusion}
The assessment reveals significant and addressable risks within Infinity Loop's security posture. While the organization has taken a positive first step by securing email with MFA, critical gaps remain in protecting internal systems and educating employees. By implementing the recommendations outlined in this report—prioritizing MFA expansion, software patching, and the creation of a security awareness program—the organization can substantially reduce its attack surface and improve its resilience against modern cyber threats.

\end{document}
```