```latex
\documentclass[12pt]{article}

% -----------------------------------------------------------------------------
% PREAMBLE
% -----------------------------------------------------------------------------
\usepackage[margin=1in]{geometry}
\usepackage{pifont} % For checkmarks and crosses
\usepackage{booktabs} % For professional tables
\usepackage{hyperref} % For clickable links
\usepackage{url} % For URL formatting
\usepackage{seqsplit} % To split long strings in texttt
\usepackage{graphicx}
\usepackage{fancyhdr}
\usepackage{lastpage}
\usepackage{xcolor}
\usepackage{datetime}

% --- Document & Hyperlink Setup ---
\hypersetup{
    colorlinks=true,
    linkcolor=blue,
    filecolor=magenta,      
    urlcolor=cyan,
    pdftitle={Cybersecurity Posture Assessment Report},
    pdfauthor={Cybersecurity Analyst},
    pdfsubject={Security Analysis},
    pdfkeywords={Security, Report, Assessment},
    bookmarks=true
}

% --- Header & Footer ---
\pagestyle{fancy}
\fancyhf{} % Clear all header and footer fields
\fancyhead[L]{Cybersecurity Posture Assessment}
\fancyhead[R]{Silent Spring}
\fancyfoot[C]{\thepage\ of \pageref{LastPage}}
\renewcommand{\headrulewidth}{0.4pt}
\renewcommand{\footrulewidth}{0.4pt}

% --- Custom Colors ---
\definecolor{tablehead}{gray}{0.9}
\definecolor{critical}{HTML}{990000}
\definecolor{high}{HTML}{D2691E}
\definecolor{medium}{HTML}{DAA520}

% --- Document Metadata ---
\title{Cybersecurity Posture Assessment Report \\ \large For: Silent Spring}
\author{Cybersecurity Analyst}
\date{\today}


% -----------------------------------------------------------------------------
% DOCUMENT START
% -----------------------------------------------------------------------------
\begin{document}

\maketitle
\thispagestyle{empty}
\newpage

\tableofcontents
\newpage

% -----------------------------------------------------------------------------
% 1. EXECUTIVE SUMMARY
% -----------------------------------------------------------------------------
\section{Executive Summary}

This report details the findings of a cybersecurity posture assessment conducted for \textbf{Silent Spring}. The analysis is based on a combination of technical network scanning, a review of organizational security controls via a questionnaire, and an evaluation of pre-existing risks.

The assessment identified several critical and high-risk security gaps that require immediate attention. The most significant findings include:

\begin{itemize}
    \item \textbf{Critical Lack of Multi-Factor Authentication (MFA):} MFA is not enforced for accessing email, logging into computers, or accessing sensitive data systems. This absence represents a critical vulnerability, as a single compromised password could lead to a widespread system breach.
    \item \textbf{Exposed Administrative Service:} The external network scan revealed an open SSH port (22) on an IPv6 address. Publicly accessible administrative ports are prime targets for automated brute-force attacks and unauthorized access attempts.
    \item \textbf{Inadequate Security Awareness Training:} While new employees receive training, there is no mandatory annual security awareness program for all staff. This gap increases the organization's susceptibility to social engineering and phishing attacks.
\end{itemize}

These findings, particularly the combination of no MFA and an exposed SSH port, create a significant risk of unauthorized access to the organization's network and data. This report provides a detailed breakdown of these risks and offers actionable recommendations to mitigate them effectively.

% -----------------------------------------------------------------------------
% 2. ORGANIZATIONAL INFORMATION
% -----------------------------------------------------------------------------
\section{Organizational Information}

The following details were provided for the assessment.

\begin{tabular}{@{}ll}
    \toprule
    \textbf{Attribute} & \textbf{Value} \\
    \midrule
    Organization Name & \textbf{Silent Spring} \\
    Email Domain & \texttt{SilentSpring.org} \\
    Website Domain & \url{www.SilentSpring.org} \\
    External IP Address & \texttt{185.118.19.247} \\
    \bottomrule
\end{tabular}

% -----------------------------------------------------------------------------
% 3. SECURITY CONTROL REVIEW
% -----------------------------------------------------------------------------
\section{Security Control Review}

The following table summarizes the organization's responses to the security controls questionnaire. Items marked with \ding{55} indicate significant gaps in the current security posture.

\begin{table}[h!]
\centering
\begin{tabular}{p{0.6\textwidth} c p{0.2\textwidth}}
    \toprule
    \rowcolor{tablehead}
    \textbf{Control Question} & \textbf{Response} & \textbf{Assessment} \\
    \midrule
    Do you require MFA to access email? & \ding{55} & \textcolor{critical}{\textbf{Critical Gap}} \\
    Do you require MFA to log into computers? & \ding{55} & \textcolor{critical}{\textbf{Critical Gap}} \\
    Do you require MFA to access sensitive data systems? & \ding{55} & \textcolor{critical}{\textbf{Critical Gap}} \\
    Does your organization have an employee acceptable use policy? & \ding{51} & Best Practice Met \\
    Does your organization do security awareness training for new employees? & \ding{51} & Best Practice Met \\
    Does your organization do security awareness training for all employees at least once per year? & \ding{55} & \textcolor{high}{\textbf{High Risk}} \\
    \bottomrule
\end{tabular}
\caption{Security Controls Questionnaire Analysis}
\end{table}

% -----------------------------------------------------------------------------
% 4. TECHNICAL SCAN RESULTS
% -----------------------------------------------------------------------------
\section{Technical Scan Results}

An external network scan was performed to identify open ports and exposed services.

\begin{itemize}
    \item \textbf{Scan Target:} \seqsplit{\texttt{2001:db8::1}}
    \item \textbf{Scan Date:} \today
\end{itemize}

The scan identified the following open port:

\begin{table}[h!]
\centering
\begin{tabular}{l l l p{0.5\textwidth}}
    \toprule
    \rowcolor{tablehead}
    \textbf{Port} & \textbf{State} & \textbf{Service} & \textbf{Notes} \\
    \midrule
    22/tcp & Open & SSH & No service version was detected. Secure Shell (SSH) is a critical administrative service. Exposing it to the public internet without proper controls (e.g., firewall rules, key-based authentication) poses a high risk of brute-force attacks and unauthorized access. \\
    \bottomrule
\end{tabular}
\caption{Open Ports Detected on Target Host}
\end{table}

% -----------------------------------------------------------------------------
% 5. RISK ASSESSMENT SUMMARY
% -----------------------------------------------------------------------------
\section{Risk Assessment Summary}

This section synthesizes the findings from the security control review and technical scan. No pre-existing vulnerabilities were provided for this assessment.

\begin{table}[h!]
\centering
\begin{tabular}{p{0.1\textwidth} p{0.2\textwidth} p{0.5\textwidth} p{0.1\textwidth}}
    \toprule
    \rowcolor{tablehead}
    \textbf{Risk ID} & \textbf{Risk Title} & \textbf{Description} & \textbf{Severity} \\
    \midrule
    RISK-001 & Widespread Lack of MFA & Multi-Factor Authentication is not enforced on email, computer logins, or sensitive systems. This allows an attacker with valid credentials to gain unauthorized access without a second authentication factor. & \textcolor{critical}{\textbf{Critical}} \\
    \addlinespace
    RISK-002 & Exposed Administrative Service (SSH) & Port 22 (SSH) is open to the internet, creating a direct vector for attackers to attempt brute-force or credential-stuffing attacks against a critical management interface. & \textcolor{high}{\textbf{High}} \\
    \addlinespace
    RISK-003 & Inadequate Security Awareness Training & The lack of mandatory annual training for all employees increases the likelihood of successful phishing and social engineering attacks, which are primary sources of credential theft. & \textcolor{high}{\textbf{High}} \\
    \bottomrule
\end{tabular}
\caption{Identified Risks and Severity}
\end{table}

% -----------------------------------------------------------------------------
% 6. RECOMMENDATIONS
% -----------------------------------------------------------------------------
\section{Recommendations}

Based on the identified risks, the following actions are recommended to improve the cybersecurity posture of \textbf{Silent Spring}.

\begin{enumerate}
    \item \textbf{Implement Multi-Factor Authentication (High Priority):}
    \begin{itemize}
        \item Immediately enforce MFA for all user accounts, especially for administrators and executives.
        \item Prioritize implementation in the following order: (1) Email access (e.g., Office 365, Google Workspace), (2) Remote access systems (VPN, SSH), and (3) Access to all systems containing sensitive data.
    \end{itemize}

    \item \textbf{Secure the Exposed SSH Service (High Priority):}
    \begin{itemize}
        \item If public access to SSH is not required, block port 22 at the network firewall immediately.
        \item If public access is necessary, restrict access to a whitelist of trusted IP addresses.
        \item Enforce public key authentication and completely disable password-based authentication for SSH.
        \item Implement an intrusion prevention tool like \texttt{fail2ban} to automatically block IPs that exhibit malicious behavior (e.g., repeated failed login attempts).
    \end{itemize}

    \item \textbf{Establish a Comprehensive Security Training Program (Medium Priority):}
    \begin{itemize}
        \item Develop and mandate an annual security awareness training program for all employees, contractors, and volunteers.
        \item The training should cover key topics such as phishing identification, password hygiene, and the acceptable use policy.
        \item Conduct periodic phishing simulation campaigns to measure the effectiveness of the training and identify areas for improvement.
    \end{itemize}
\end{enumerate}

\end{document}
```