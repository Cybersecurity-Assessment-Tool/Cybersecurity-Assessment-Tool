```latex
\documentclass[12pt]{article}

% ----------------------------------------------------------------------
%  Required Packages
% ----------------------------------------------------------------------
\usepackage[margin=1in]{geometry} % Set page margins
\usepackage{pifont}               % For checkmarks and crosses (dingbats)
\usepackage{booktabs}             % For professional-looking tables
\usepackage{xcolor}               % For custom colors
\usepackage[hidelinks]{hyperref}  % For hyperlinks without ugly boxes
\usepackage{url}                  % For formatting URLs
\usepackage{seqsplit}             % For splitting long strings in texttt

% ----------------------------------------------------------------------
%  Custom Commands & Document Setup
% ----------------------------------------------------------------------
\newcommand{\yes}{\textcolor{green}{\ding{51}}}
\newcommand{\no}{\textcolor{red}{\ding{55}}}
\definecolor{critical}{HTML}{990000}
\definecolor{high}{HTML}{D16103}
\definecolor{medium}{HTML}{C5921A}
\definecolor{low}{HTML}{005F99}

\hypersetup{
    pdftitle={Cybersecurity Assessment Report},
    pdfauthor={Cybersecurity Analysis Service},
    pdfsubject={Security Posture Analysis},
    pdfkeywords={Cybersecurity, Risk Assessment, Network Scan}
}

% ----------------------------------------------------------------------
%  Title Section
% ----------------------------------------------------------------------
\title{
    Cybersecurity Assessment Report \\
    \large For: \textbf{Hearth \& Home}
}
\author{Cybersecurity Analysis Service}
\date{\today}

% ======================================================================
%  BEGIN DOCUMENT
% ======================================================================
\begin{document}

\maketitle
\thispagestyle{empty}
\newpage

\tableofcontents
\newpage

% ----------------------------------------------------------------------
%  1. Executive Summary
% ----------------------------------------------------------------------
\section{Executive Summary}

This report provides a comprehensive cybersecurity assessment for \textbf{Hearth \& Home}, synthesizing data from a technical network scan, a security controls questionnaire, and a review of pre-existing risks.

The assessment reveals a mixed security posture. The organization has implemented foundational controls such as Multi-Factor Authentication (MFA) for email and computer access, and maintains an acceptable use policy. A significant positive finding is that a previously identified risk, an unencrypted web server on port 80, appears to have been remediated, as the port was found to be closed during the current scan.

However, critical gaps were identified that expose the organization to significant risk. The most severe findings are:
\begin{itemize}
    \item \textbf{Lack of MFA for Sensitive Data Systems:} This is a critical vulnerability that could lead to a severe data breach if credentials are compromised.
    - \textbf{Absence of Security Awareness Training:} The complete lack of security training for both new and existing employees creates a high susceptibility to phishing, social engineering, and other human-vector attacks.
\end{itemize}

Immediate action is required to address these deficiencies. Recommendations in this report prioritize the implementation of MFA on all sensitive systems and the establishment of a comprehensive security awareness training program.

% ----------------------------------------------------------------------
%  2. Organizational Information
% ----------------------------------------------------------------------
\section{Organizational Information}

The following details were provided for the assessment.

\begin{tabular}{@{}ll}
    \toprule
    \textbf{Attribute} & \textbf{Value} \\
    \midrule
    Organization Name & \textbf{Hearth \& Home} \\
    Email Domain & \texttt{HearthHome.net} \\
    Website Domain & \seqsplit{\url{www.HearthHome.net}} \\
    External IP Address & \seqsplit{\texttt{55.223.214.92}} \\
    \bottomrule
\end{tabular}

% ----------------------------------------------------------------------
%  3. Security Control Review
% ----------------------------------------------------------------------
\section{Security Control Review}

A review of organizational security controls was conducted via a questionnaire. The responses indicate key areas of strength and weakness in the current security posture. "No" answers represent significant gaps requiring attention.

\begin{table}[h!]
\centering
\begin{tabular}{@{}p{0.7\textwidth}c@{}}
    \toprule
    \textbf{Control Question} & \textbf{Status} \\
    \midrule
    Do you require MFA to access email? & \yes \\
    Do you require MFA to log into computers? & \yes \\
    Does your organization have an employee acceptable use policy? & \yes \\
    \addlinespace
    \color{red!80!black}Do you require MFA to access sensitive data systems? & \no \\
    \color{red!80!black}Does your organization do security awareness training for new employees? & \no \\
    \color{red!80!black}Does your organization do security awareness training for all employees at least once per year? & \no \\
    \bottomrule
\end{tabular}
\caption{Security Controls Questionnaire Results.}
\end{label{tab:controls}
\end{table}

% ----------------------------------------------------------------------
%  4. Technical Scan Results
% ----------------------------------------------------------------------
\section{Technical Scan Results}

A network scan was performed to identify open ports and exposed services on the specified target system.

\begin{itemize}
    \item \textbf{Target IP Address:} \seqsplit{\texttt{192.168.0.5}}
    \item \textbf{Scan Summary:} The scan was limited in scope and only confirmed the status of a single port. No actively listening services were discovered.
\end{itemize}

\begin{table}[h!]
\centering
\begin{tabular}{@{}llll@{}}
    \toprule
    \textbf{Port} & \textbf{State} & \textbf{Service} & \textbf{Notes} \\
    \midrule
    80/tcp & Closed & http & Port is not listening. \\
    \bottomrule
\end{tabular}
\caption{Nmap Scan Results for \texttt{192.168.0.5}.}
\label{tab:nmap}
\end{table}

\paragraph{Analysis:} The scan indicates that port 80 (HTTP) is closed on the target host. This contradicts a pre-existing risk entry (see Section 5), suggesting that remediation may have occurred. This is a positive development, but it should be formally verified that this change was intentional and is applied across all relevant systems.

% ----------------------------------------------------------------------
%  5. Consolidated Risk Assessment
% ----------------------------------------------------------------------
\section{Consolidated Risk Assessment}

The following table synthesizes findings from the security control review, technical scan, and pre-existing risk data. New risks identified during this assessment are assigned IDs starting with `RR-00`.

\begin{table}[h!]
\centering
\begin{tabular}{@{}p{0.1\textwidth}p{0.25\textwidth}p{0.45\textwidth}p{0.1\textwidth}@{}}
    \toprule
    \textbf{Risk ID} & \textbf{Risk Name} & \textbf{Description} & \textbf{Severity} \\
    \midrule
    RR-001 & \textbf{No MFA on Sensitive Systems} & The lack of MFA on systems storing sensitive data exposes the organization to data breach via credential theft. & \textcolor{critical}{\textbf{Critical}} \\
    \addlinespace
    RR-002 & \textbf{No Security Awareness Training} & Employees are not trained to recognize or respond to phishing, malware, or social engineering, making them a primary target for attackers. & \textcolor{high}{\textbf{High}} \\
    \addlinespace
    EXIST-01 & \textbf{Unencrypted Web Server (Remediated)} & A pre-existing risk stated port 80 was open. The current scan shows it is closed. This risk appears to be remediated. & \textcolor{low}{\textbf{Info}} \\
    \bottomrule
\end{tabular}
\caption{Consolidated Risk Register.}
\label{tab:risks}
\end{table}

% ----------------------------------------------------------------------
%  6. Recommendations
% ----------------------------------------------------------------------
\section{Recommendations}

Based on the consolidated risk assessment, the following actions are recommended to improve the security posture of \textbf{Hearth \& Home}.

\subsection{Critical Priority: Implement MFA for Sensitive Data}

\begin{itemize}
    \item \textbf{Action:} Immediately begin a project to identify all systems classified as containing sensitive data (e.g., financial, customer PII, HR records) and enforce MFA for all user access.
    \item \textbf{Justification:} This is the single most effective control to prevent unauthorized access to critical assets in the event of a password compromise. It directly mitigates the risk of a major data breach.
    \item \textbf{Timeline:} 1-3 Months.
\end{itemize}

\subsection{High Priority: Establish a Security Awareness Program}

\begin{itemize}
    \item \textbf{Action:} Procure and implement a security awareness training solution. The program must include:
    \begin{enumerate}
        \item Mandatory onboarding training for all new employees before they are granted system access.
        \item Annual refresher training for all staff covering topics like phishing, password security, and acceptable use.
        \item Regular simulated phishing campaigns to test and reinforce employee knowledge.
    \end{enumerate}
    \item \textbf{Justification:} A well-trained workforce is the first line of defense. This program will reduce the likelihood of a successful attack that relies on human error.
    \item \textbf{Timeline:} 3-6 Months.
\end{itemize}

\subsection{Informational: Verify and Document Port 80 Closure}

\begin{itemize}
    \item \textbf{Action:} Confirm with the IT/network team that the closure of port 80 on host \texttt{192.168.0.5} was an intentional act of remediation.
    \item \textbf{Justification:} While the port closure is a positive finding, it is crucial to ensure it was deliberate. Update the organization's risk register and network diagrams to reflect this change.
    \item \textbf{Timeline:} 1 Week.
\end{itemize}

\end{document}
% ======================================================================
%  END DOCUMENT
% ======================================================================
```