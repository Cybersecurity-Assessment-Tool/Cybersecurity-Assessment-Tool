```latex
\documentclass[12pt]{article}

% Preamble: Required Packages
\usepackage[margin=1in]{geometry}
\usepackage{pifont} % For \ding symbols (checkmarks/crosses)
\usepackage{booktabs} % For professional-looking tables
\usepackage{hyperref}
\usepackage{url}
\usepackage{seqsplit} % For breaking long strings in tt environments
\usepackage{graphicx}
\usepackage{fancyhdr}
\usepackage{xcolor}

% --- Document Setup ---

% Hyperlink setup for a more interactive PDF
\hypersetup{
    colorlinks=true,
    linkcolor=blue,
    filecolor=magenta,      
    urlcolor=cyan,
    pdftitle={Cybersecurity Posture Assessment Report},
    pdfauthor={Cybersecurity Analysis Division},
}

% Header and Footer Configuration
\pagestyle{fancy}
\fancyhf{} % Clear all header and footer fields
\lhead{Cybersecurity Assessment Report}
\rhead{For: Maple Leaf Logistics}
\cfoot{Page \thepage}
\renewcommand{\headrulewidth}{0.4pt}
\renewcommand{\footrulewidth}{0.4pt}

% Define a custom color for high-risk items
\definecolor{riskred}{rgb}{0.7, 0, 0}

% --- Document Body ---

\begin{document}

% --- Title Page ---
\begin{titlepage}
    \centering
    \vspace*{1cm}
    \includegraphics[width=0.3\textwidth]{example-image-a} % Placeholder for a logo
    
    \vspace{1.5cm}
    
    \Huge
    \textbf{Cybersecurity Posture Assessment Report}
    
    \vspace{1cm}
    
    \Large
    Prepared For: \\
    \vspace{0.5cm}
    \textbf{Maple Leaf Logistics}
    
    \vspace{2cm}
    
    \large
    \textbf{Date of Report:} \today \\
    \textbf{Report ID:} CYBER-2023-0042
    
    \vfill
    
    \large
    \textbf{Generated By:} \\
    Cybersecurity Analysis Division
    
\end{titlepage}

\tableofcontents
\newpage

% --- Section 1: Executive Summary ---
\section{Executive Summary}

This report details the findings of a cybersecurity posture assessment conducted for Maple Leaf Logistics. The assessment combined an external network vulnerability scan with a review of organizational security controls and pre-existing risks.

The key takeaway is a significant disparity between the organization's technical and procedural security postures.

\paragraph{Strengths:} The external network scan of the target system (\texttt{192.168.1.100}) revealed a strong perimeter defense. No open ports were detected, indicating a well-configured firewall and a minimal external attack surface. This is a commendable security practice.

\paragraph{Critical Weaknesses:} Despite the strong network security, the security control review identified critical gaps in internal policies that present a high level of risk. The most severe findings are:
\begin{itemize}
    \item \textbf{Lack of Multi-Factor Authentication (MFA) on Sensitive Systems:} The failure to protect high-value data with MFA exposes the organization to significant risk from credential theft and unauthorized access.
    \item \textbf{Absence of Security Awareness Training:} The organization does not conduct security awareness training for new or existing employees. This makes personnel highly susceptible to phishing, social engineering, and other human-centric attacks, which are the most common initial access vectors for threat actors.
\end{itemize}

\paragraph{Conclusion:} While the network perimeter is secure, the identified procedural gaps could allow an attacker to bypass these technical controls entirely. An attacker who successfully phishes an employee's credentials could potentially gain access to sensitive data without the safeguard of MFA. Immediate remediation of these policy-based risks is strongly recommended to build a defense-in-depth security posture.

\newpage

% --- Section 2: Organizational Information ---
\section{Organizational Information}

The following information was provided for the assessment.

\begin{tabular}{@{}ll}
\toprule
\textbf{Attribute} & \textbf{Value} \\
\midrule
Organization Name & \textbf{Maple Leaf Logistics} \\
Primary Email Domain & \texttt{MapleLeafLogistics.org} \\
Primary Website Domain & \url{www.MapleLeafLogistics.org} \\
Monitored External IP & \texttt{224.119.86.227} \\
\bottomrule
\end{tabular}

% --- Section 3: Security Control Review ---
\section{Security Control Review}

The following table summarizes the organization's responses to a security controls questionnaire. Items marked with a red 'X' represent significant gaps in the security framework and are directly correlated to the risks identified in Section 5.

\vspace{1em}

\begin{tabular}{@{}p{0.8\linewidth}c}
\toprule
\textbf{Control Question} & \textbf{Response} \\
\midrule
Do you require MFA to access email? & \ding{51} \\
Do you require MFA to log into computers? & \ding{51} \\
Do you require MFA to access sensitive data systems? & \textcolor{riskred}{\ding{55}} \\
Does your organization have an employee acceptable use policy? & \ding{51} \\
Does your organization do security awareness training for new employees? & \textcolor{riskred}{\ding{55}} \\
Does your organization do security awareness training for all employees at least once per year? & \textcolor{riskred}{\ding{55}} \\
\bottomrule
\end{tabular}

\vspace{1em}
\noindent
\textbf{Legend:} \ding{51} = Yes/In Place \quad \textcolor{riskred}{\ding{55}} = No/Not In Place

\newpage

% --- Section 4: Technical Scan Results ---
\section{Technical Scan Results}

An external network scan was performed to identify open ports and exposed services on the target system.

\subsection{Nmap Scan Findings}
\begin{itemize}
    \item \textbf{Target IP Address:} \texttt{192.168.1.100}
    \item \textbf{Host Status:} UP
    \item \textbf{Scan Summary:} The scan completed successfully and determined that the host is online. However, the scan did not identify any open TCP or UDP ports. All 1000 scanned ports were reported as being in a \texttt{closed} state.
\end{itemize}

\paragraph{Analyst Interpretation:}
A result of zero open ports is an excellent security finding. It indicates that the host has a properly configured stateful firewall that denies all unsolicited inbound traffic. This configuration drastically reduces the external attack surface, making it difficult for an attacker to probe for vulnerable services from the outside. This is a significant strength in the organization's security posture.

% --- Section 5: Risk Assessment ---
\section{Risk Assessment}

This section synthesizes findings from the security control review (Section 3) and the technical scan (Section 4). No pre-existing vulnerabilities were reported. The primary risks are procedural and human-centric.

\begin{tabular}{@{}lp{0.5\linewidth}l}
\toprule
\textbf{Risk Name} & \textbf{Overview} & \textbf{Severity} \\
\midrule
\addlinespace
\textbf{Lack of MFA on Sensitive Systems} & Failure to protect critical data stores with multi-factor authentication significantly increases the risk of unauthorized access via compromised credentials. An attacker with valid credentials would face no additional barriers. & \textbf{\textcolor{riskred}{High}} \\
\addlinespace
\textbf{No Security Awareness Training Program} & Employees are not trained to recognize or respond to phishing, social engineering, or other common cyber threats. This makes them the weakest link and a primary target for attackers seeking initial access to the network. & \textbf{\textcolor{riskred}{High}} \\
\addlinespace
\bottomrule
\end{tabular}

% --- Section 6: Recommendations ---
\section{Recommendations}

Based on the identified risks, the following prioritized actions are recommended to improve the cybersecurity posture of Maple Leaf Logistics.

\begin{enumerate}
    \item \textbf{Implement Mandatory MFA on Sensitive Systems (Priority: Critical)} \\
    \textit{Action:} Enforce the use of multi-factor authentication on all applications, databases, and administrative interfaces that contain or provide access to sensitive or critical organizational data. This control is essential for mitigating the risk of account compromise.
    
    \item \textbf{Establish a Comprehensive Security Awareness Program (Priority: Critical)} \\
    \textit{Action:} Develop and implement a formal security awareness training program. This program must include:
    \begin{itemize}
        \item \textbf{Onboarding Training:} A mandatory training module for all new employees before they are granted system access.
        \item \textbf{Annual Refresher Training:} A mandatory annual course for all staff covering current threats, best practices, and organizational policies.
        \item \textbf{Phishing Simulations:} Regular, simulated phishing campaigns to test and improve employee vigilance.
    \end{itemize}
\end{enumerate}

By implementing these recommendations, Maple Leaf Logistics can significantly reduce its risk exposure and build a more resilient, defense-in-depth security framework that protects both its technical and human assets.

\end{document}
```