```latex
\documentclass[12pt, a4paper]{article}

% Preamble: Required Packages
\usepackage[margin=1in]{geometry}
\usepackage{pifont} % For checkmarks and crosses
\usepackage{booktabs} % For professional tables
\usepackage{hyperref} % For clickable links
\usepackage{url} % For formatting URLs
\usepackage{seqsplit} % For splitting long strings to prevent overflow
\usepackage{graphicx}
\usepackage{xcolor}
\usepackage{fancyhdr}
\usepackage{lastpage}

% --- Document Setup ---
\definecolor{darkblue}{rgb}{0.0, 0.0, 0.55}
\definecolor{darkred}{rgb}{0.55, 0.0, 0.0}

\hypersetup{
    colorlinks=true,
    linkcolor=darkblue,
    filecolor=darkblue,      
    urlcolor=darkblue,
    citecolor=darkblue,
}

% --- Header & Footer ---
\pagestyle{fancy}
\fancyhf{} % Clear all header and footer fields
\fancyhead[L]{Cybersecurity Assessment Report}
\fancyhead[R]{Vertex Solutions}
\fancyfoot[C]{\thepage\ of \pageref{LastPage}}
\renewcommand{\headrulewidth}{0.4pt}
\renewcommand{\footrulewidth}{0.4pt}

% --- Document Body ---
\begin{document}

% --- Title Page ---
\begin{titlepage}
    \centering
    \vspace*{1cm}
    
    \includegraphics[width=0.4\textwidth]{example-image-a} % Placeholder logo
    
    \vspace{1.5cm}
    
    {\Huge\bfseries Cybersecurity Posture Assessment Report\par}
    
    \vspace{1cm}
    
    {\Large Prepared for:\par}
    {\Large\bfseries Vertex Solutions\par}
    
    \vspace{2cm}
    
    {\large Report Date: \today\par}
    
    \vfill
    
    {\large This report contains sensitive and confidential information. Distribution should be limited to authorized personnel only.\par}
    
\end{titlepage}

\tableofcontents
\newpage

% --- Section 1: Executive Summary ---
\section{Executive Summary}
This report provides a comprehensive analysis of the cybersecurity posture of Vertex Solutions, based on a combination of technical network scanning, a review of existing risks, and an organizational security controls questionnaire.

The assessment identified several critical and high-risk findings that require immediate attention. The primary concerns are a directly exposed management interface (\texttt{SSH} on \texttt{127.0.0.1}) which corresponds to a known critical vulnerability, and significant gaps in access control policies, most notably the lack of Multi-Factor Authentication (MFA) for email and sensitive data systems. Furthermore, foundational security governance elements, such as an employee acceptable use policy and security training for new hires, are absent.

These deficiencies create a high likelihood of security incidents, including business email compromise, data breaches, and unauthorized system access. We strongly recommend prioritizing the remediation steps outlined in Section \ref{sec:recommendations} to mitigate these risks and improve the organization's overall security resilience.

% --- Section 2: Organizational Information ---
\section{Organizational Information}
The following details were provided for the assessment scope.

\begin{table}[h!]
\centering
\begin{tabular}{@{}ll@{}}
\toprule
\textbf{Attribute} & \textbf{Value} \\ \midrule
Organization Name & Vertex Solutions \\
Email Domain & \texttt{VertexSolutions.net} \\
Website Domain & \url{www.VertexSolutions.net} \\
External IP Address & \texttt{56.160.3.114} \\ \bottomrule
\end{tabular}
\caption{Client Organizational Details}
\end{table}

% --- Section 3: Security Control Review ---
\section{Security Control Review}
A questionnaire was completed to evaluate the current state of administrative and technical security controls. The responses reveal critical gaps in identity and access management and employee security governance.

\begin{table}[h!]
\centering
\begin{tabular}{@{}p{0.5\linewidth}ccc@{}}
\toprule
\textbf{Control Question} & \textbf{Response} & \textbf{Assessment} \\ \midrule
Do you require MFA to access email? & \color{darkred}{\ding{55}} & \color{darkred}{\textbf{Critical Gap}} \\
Do you require MFA to log into computers? & \color{darkgreen}{\ding{51}} & Meets Best Practice \\
Do you require MFA to access sensitive data systems? & \color{darkred}{\ding{55}} & \color{darkred}{\textbf{Critical Gap}} \\
Does your organization have an employee acceptable use policy? & \color{darkred}{\ding{55}} & \color{darkred}{\textbf{High Risk}} \\
Does your organization do security awareness training for new employees? & \color{darkred}{\ding{55}} & \color{darkred}{\textbf{High Risk}} \\
Does your organization do security awareness training for all employees at least once per year? & \color{darkgreen}{\ding{51}} & Meets Best Practice \\
\bottomrule
\end{tabular}
\caption{Analysis of Security Controls Questionnaire}
\end{table}

% --- Section 4: Technical Scan Results ---
\section{Technical Scan Results}
An external network scan was performed to identify exposed services. The scan was targeted at the localhost interface, which indicates a potentially severe misconfiguration.

\begin{itemize}
    \item \textbf{Target IP Address:} \texttt{127.0.0.1}
    \item \textbf{Scan Date:} \today
\end{itemize}

The following open ports were discovered:

\begin{table}[h!]
\centering
\begin{tabular}{@{}llll@{}}
\toprule
\textbf{Port} & \textbf{State} & \textbf{Service (Probable)} & \textbf{Notes} \\ \midrule
22/tcp & Open & SSH (Secure Shell) & Exposing SSH is high-risk. This finding \\
& & & validates the pre-existing risk of an \\
& & & exposed localhost interface. \\
\bottomrule
\end{tabular}
\caption{Open Ports Detected on Target}
\end{table}

\subsection{Analysis}
The presence of an open SSH port on the localhost address (\texttt{127.0.0.1}) is a critical finding. This suggests that a service intended only for internal system communication is exposed externally. This misconfiguration could allow an attacker to bypass perimeter defenses and attempt to gain direct administrative access to the system. The scan did not retrieve service version information, but any exposed management service presents a significant threat.

% --- Section 5: Correlated Risk Assessment ---
\section{Correlated Risk Assessment}
The following table synthesizes findings from the security questionnaire, the technical scan, and pre-existing risk data into a prioritized list of security risks.

\begin{table}[h!]
\centering
\begin{tabular}{@{}p{0.1\linewidth}p{0.25\linewidth}p{0.45\linewidth}l@{}}
\toprule
\textbf{Risk ID} & \textbf{Risk Title} & \textbf{Description} & \textbf{Severity} \\ \midrule
RSK-001 & Compromise via Exposed Management Interface & The technical scan confirmed that the SSH service (port 22) is open on the localhost interface, which is exposed. This aligns with the pre-existing CVSS 10.0 risk and could lead to a full system compromise. & \textbf{Critical} \\
\addlinespace
RSK-002 & Business Email Compromise (BEC) & The lack of MFA on email accounts makes them highly susceptible to phishing and credential stuffing attacks, which can lead to financial loss and data exfiltration. & \textbf{High} \\
\addlinespace
RSK-003 & Sensitive Data Breach & Systems containing sensitive data are not protected by MFA. A single compromised password could grant an attacker access to the organization's most valuable information assets. & \textbf{High} \\
\addlinespace
RSK-004 & Insufficient Security Governance & The absence of an Acceptable Use Policy and security training for new hires creates an environment where employees are unaware of security expectations and are more likely to fall victim to social engineering attacks. & \textbf{Medium} \\
\bottomrule
\end{tabular}
\caption{Summary of Identified Risks}
\end{table}

% --- Section 6: Recommendations ---
\section{Recommendations}
\label{sec:recommendations}
Based on the correlated risk assessment, we recommend the following actions, prioritized by severity.

\subsection{Immediate Priority (Critical Risks)}
\begin{itemize}
    \item \textbf{Remediate Exposed Service (RSK-001):} Immediately investigate why the localhost interface is reachable externally. Reconfigure network firewalls and service bindings to ensure that services intended for local use only (like the one on \texttt{127.0.0.1}) are not exposed to the internet.
    \item \textbf{Secure SSH Configuration (RSK-001):} Once the exposure is fixed, harden the SSH configuration by:
    \begin{itemize}
        \item Disabling password-based authentication and enforcing public key authentication.
        \item Disabling root login via SSH.
        \item Implementing IP-based access control lists (whitelisting).
    \end{itemize}
\end{itemize}

\subsection{High Priority}
\begin{itemize}
    \item \textbf{Implement MFA for Email (RSK-002):} Enforce mandatory MFA for all user access to the email system (\texttt{VertexSolutions.net}) to prevent account takeovers.
    \item \textbf{Implement MFA for Sensitive Data (RSK-003):} Deploy MFA as a requirement for accessing all applications and systems identified as storing or processing sensitive data.
    \item \textbf{Develop Acceptable Use Policy (RSK-004):} Create and disseminate a formal Acceptable Use Policy (AUP) that clearly defines rules for all employees regarding the use of company assets, data handling, and security responsibilities.
    \item \textbf{Establish New Hire Training (RSK-004):} Integrate mandatory cybersecurity awareness training into the onboarding process for all new employees to establish a baseline of security knowledge from day one.
\end{itemize}

\subsection{Medium Priority}
\begin{itemize}
    \item \textbf{Conduct Authenticated Vulnerability Scanning:} Perform comprehensive internal and external vulnerability scans with authentication to identify outdated software, missing patches, and further misconfigurations that were not visible in the initial unauthenticated scan.
\end{itemize}

\end{document}
```