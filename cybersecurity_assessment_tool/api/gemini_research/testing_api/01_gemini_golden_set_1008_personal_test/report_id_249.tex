```latex
\documentclass[12pt]{article}

% Preamble: Required Packages
\usepackage[margin=1in]{geometry}
\usepackage{pifont} % For checkmarks and crosses
\usepackage{booktabs} % For professional tables
\usepackage{hyperref} % For clickable links and better PDF metadata
\usepackage{url} % For formatting URLs
\usepackage{seqsplit} % For splitting long strings in tt font
\usepackage{graphicx} % For potential logos (placeholder)
\usepackage{xcolor} % For colors

% Document Information
\title{Cybersecurity Posture Assessment Report}
\author{Cybersecurity Analyst}
\date{\today}

% Hyperref Setup
\hypersetup{
    colorlinks=true,
    linkcolor=blue,
    filecolor=magenta,      
    urlcolor=cyan,
    pdftitle={Cybersecurity Posture Assessment Report},
    pdfpagemode=FullScreen,
}

\begin{document}

\maketitle
\thispagestyle{empty}
\newpage

\tableofcontents
\newpage

% --- 1. Executive Summary ---
\section{Executive Summary}

This report provides a comprehensive cybersecurity assessment for \textbf{True Grit}, based on an analysis of network scan data, organizational security controls, and pre-existing risk information. The assessment was conducted to identify vulnerabilities, evaluate security gaps, and provide actionable recommendations to enhance the organization's security posture.

The analysis revealed several critical-risk findings that require immediate attention. A network scan identified a potentially sensitive database interface, titled \texttt{"TOP SECRET DB"}, exposed on port \texttt{8080}. This finding directly contradicts a previous risk assessment which incorrectly labeled the port as secure.

Furthermore, a review of organizational security controls found a systemic lack of Multi-Factor Authentication (MFA) for accessing email, computers, and sensitive data systems. This gap, combined with the exposed database, creates a significant risk of unauthorized access and data breach. The absence of security awareness training for new employees further exacerbates this risk by increasing susceptibility to social engineering attacks.

Immediate remediation of the exposed service and the enterprise-wide implementation of MFA are the highest priority recommendations.

% --- 2. Organizational Information ---
\section{Organizational Information}

The following details were provided for the assessment.

\begin{tabular}{@{}ll}
\toprule
\textbf{Attribute} & \textbf{Value} \\
\midrule
Organization Name & \textbf{True Grit} \\
Email Domain & \texttt{TrueGrit.net} \\
Website Domain & \url{www.TrueGrit.net} \\
External IP Address & \texttt{181.136.149.93} \\
\bottomrule
\end{tabular}

% --- 3. Security Control Review ---
\section{Security Control Review}

A review of the organization's security questionnaire responses highlights critical gaps in fundamental security controls. "No" answers indicate areas of high risk that weaken the overall security posture.

\begin{tabular}{@{}p{0.6\linewidth}cp{0.25\linewidth}@{}}
\toprule
\textbf{Control Question} & \textbf{Response} & \textbf{Analyst Notes} \\
\midrule
Do you require MFA to access email? & \textcolor{red}{\ding{55}} & Critical Gap. Email is a primary target for account takeover. \\
\addlinespace
Do you require MFA to log into computers? & \textcolor{red}{\ding{55}} & High Risk. Lack of MFA allows for easier lateral movement after a compromise. \\
\addlinespace
Do you require MFA to access sensitive data systems? & \textcolor{red}{\ding{55}} & Critical Gap. Directly exposes sensitive data to credential-based attacks. \\
\addlinespace
Does your organization have an employee acceptable use policy? & \textcolor{green}{\ding{51}} & Good Practice. Establishes a baseline for user behavior. \\
\addlinespace
Does your organization do security awareness training for new employees? & \textcolor{red}{\ding{55}} & High Risk. New hires are often targeted and are unaware of internal policies. \\
\addlinespace
Does your organization do security awareness training for all employees at least once per year? & \textcolor{green}{\ding{51}} & Good Practice. Reinforces security concepts annually. \\
\bottomrule
\end{tabular}

% --- 4. Technical Scan Results ---
\section{Technical Scan Results}

A network scan was performed to identify open ports and exposed services on the target system.

\subsection{Nmap Scan Findings}
\begin{itemize}
    \item \textbf{Target IP:} \texttt{10.5.5.5}
    \item \textbf{Status:} Host is Up
\end{itemize}

The scan revealed the following open port:

\begin{tabular}{@{}llll@{}}
\toprule
\textbf{Port} & \textbf{State} & \textbf{Service/Title} & \textbf{Analyst Notes} \\
\midrule
8080/tcp & Open & HTTP Title: \texttt{TOP SECRET DB} & \textbf{Critical Finding.} The title suggests a \\
& & & highly sensitive database is exposed. This \\
& & & finding contradicts the existing risk data, \\
& & & which incorrectly marked this port as secure. \\
\bottomrule
\end{tabular}

% --- 5. Correlated Risk Assessment ---
\section{Correlated Risk Assessment}

This section synthesizes findings from the security control review, technical scan, and pre-existing risk data. The previous assessment stating that port 8080 was secure is now considered invalid.

\begin{tabular}{@{}p{0.25\linewidth}p{0.15\linewidth}p{0.5\linewidth}@{}}
\toprule
\textbf{Risk Title} & \textbf{Severity} & \textbf{Description} \\
\midrule
\textbf{Exposed Sensitive Database Interface} & \textcolor{red}{\textbf{Critical}} & The service on port \texttt{8080} is titled \texttt{"TOP SECRET DB"}, indicating a high-value target is exposed. This, combined with the lack of MFA for sensitive systems, presents a severe risk of data breach. \\
\addlinespace
\textbf{Inadequate Identity and Access Management} & \textcolor{red}{\textbf{Critical}} & The complete absence of MFA for email, computers, and sensitive data systems makes the organization highly vulnerable to credential theft, phishing, and unauthorized access. \\
\addlinespace
\textbf{Deficient Security Onboarding} & \textcolor{orange}{\textbf{High}} & Failure to provide security awareness training to new employees creates a significant vulnerability, as new hires are prime targets for social engineering attacks. \\
\addlinespace
\textbf{Inaccurate Prior Risk Assessment} & Informational & The existing risk data incorrectly identified port 8080 as a secure false positive. This indicates a failure in the risk validation process that must be addressed to ensure the integrity of future assessments. \\
\bottomrule
\end{tabular}

% --- 6. Recommendations ---
\section{Recommendations}

The following actions are recommended to mitigate the identified risks and improve the overall security posture.

\subsection{Immediate Actions (0-7 Days)}
\begin{enumerate}
    \item \textbf{Secure Exposed Database:} Immediately investigate the service running on \texttt{10.5.5.5:8080}. If it is a sensitive database, restrict access to authorized personnel only via a firewall rule and require strong, multi-factor authentication.
    \item \textbf{Validate Previous Assessments:} Given the critical error in the prior risk assessment for port 8080, an emergency review of all other "low-risk" or "false positive" findings in the risk register should be conducted.
\end{enumerate}

\subsection{High-Priority Actions (1-3 Months)}
\begin{enumerate}
    \item \textbf{Deploy Multi-Factor Authentication (MFA):} Implement a mandatory MFA policy across the entire organization. Prioritize the following systems in order:
    \begin{itemize}
        \item All sensitive data systems.
        \item All email accounts (cloud and on-premise).
        \item All remote access solutions (VPNs, etc.).
        \item All employee workstations and servers.
    \end{itemize}
    \item \textbf{Implement New Hire Security Training:} Integrate mandatory security awareness training into the new employee onboarding process. This training should occur before a new hire is granted access to sensitive systems.
\end{enumerate}

\subsection{Long-Term Improvements (3-6 Months)}
\begin{enumerate}
    \item \textbf{Review Network Segmentation:} Evaluate the network architecture to ensure that sensitive systems, like the one discovered, are not accessible from general user networks.
    \item \textbf{Enhance Risk Management Process:} Revise the vulnerability management and risk assessment process to include mandatory technical validation of all findings before they can be closed or accepted as a false positive.
\end{enumerate}

\end{document}
```