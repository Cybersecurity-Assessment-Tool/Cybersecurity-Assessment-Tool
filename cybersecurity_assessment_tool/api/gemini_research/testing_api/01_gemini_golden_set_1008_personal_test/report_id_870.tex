```latex
\documentclass[12pt]{article}

% -----------------------------------------------------------------------------
% PREAMBLE
% -----------------------------------------------------------------------------
\usepackage[margin=1in]{geometry}
\usepackage{pifont} % For checkmarks and crosses (\ding)
\usepackage{booktabs} % For professional tables
\usepackage{graphicx}
\usepackage{xcolor}
\usepackage{url}
\usepackage{seqsplit} % For splitting long text sequences
\usepackage{hyperref} % For hyperlinks, should be loaded last

% Hyperref Setup
\hypersetup{
    colorlinks=true,
    linkcolor=blue,
    filecolor=magenta,      
    urlcolor=cyan,
    pdftitle={Cybersecurity Posture Report},
    pdfpagemode=FullScreen,
}

% Define checkmark and cross symbols for clarity
\newcommand{\yes}{\ding{51}}
\newcommand{\no}{\ding{55}}

% -----------------------------------------------------------------------------
% DOCUMENT START
% -----------------------------------------------------------------------------
\begin{document}

% -----------------------------------------------------------------------------
% TITLE PAGE
% -----------------------------------------------------------------------------
\title{
    \vspace{2cm}
    \textbf{Cybersecurity Posture Report} \\
    \large \textit{Analysis and Recommendations} \\
    \vspace{1cm}
    \includegraphics[width=0.3\textwidth]{example-image-a} \\ % Placeholder for company logo
    \vspace{1cm}
    \textbf{Apex Legends Group}
}
\author{Cybersecurity Analysis Division}
\date{\today}
\maketitle
\newpage

% -----------------------------------------------------------------------------
% TABLE OF CONTENTS
% -----------------------------------------------------------------------------
\tableofcontents
\newpage

% -----------------------------------------------------------------------------
% SECTION 1: EXECUTIVE OVERVIEW
% -----------------------------------------------------------------------------
\section{Executive Overview}

This report provides a comprehensive analysis of the cybersecurity posture for \textbf{Apex Legends Group}, based on a synthesis of network scan data, organizational security control questionnaires, and a review of pre-existing risks. The assessment was conducted on \today.

The organization demonstrates a solid foundation in security awareness training, with programs in place for both new and existing employees. This is a commendable control that reduces the risk of human error.

However, the analysis revealed several critical gaps in fundamental security controls. The most significant findings are the complete absence of Multi-Factor Authentication (MFA) for accessing email, computers, and sensitive data systems. This exposes the organization to a high risk of account compromise and unauthorized access. Furthermore, the lack of a formal employee Acceptable Use Policy (AUP) creates ambiguity and increases the potential for insider threats and misuse of assets.

On the technical front, a network scan of the target host \texttt{192.168.0.5} revealed no open ports. This is a positive finding and indicates that a previously identified risk concerning an unencrypted web server on port 80 appears to have been remediated.

In summary, while the external network perimeter appears secure at the time of scanning, immediate attention must be directed towards strengthening identity and access management controls and formalizing internal security policies to mitigate significant organizational risks.

% -----------------------------------------------------------------------------
% SECTION 2: ORGANIZATIONAL INFORMATION
% -----------------------------------------------------------------------------
\section{Organizational Information}

The following details were provided for the assessment:

\begin{itemize}
    \item \textbf{Organization Name:} Apex Legends Group
    \item \textbf{Email Domain:} \texttt{ApexLegendsGroup.net}
    \item \textbf{Website Domain:} \texttt{www.ApexLegendsGroup.net}
    \item \textbf{External IP Address:} \texttt{17.146.16.74}
\end{itemize}

% -----------------------------------------------------------------------------
% SECTION 3: SECURITY CONTROL REVIEW
% -----------------------------------------------------------------------------
\section{Security Control Review}

A review of the organization's self-reported security controls was conducted. The following table summarizes the responses and highlights identified gaps. "No" answers indicate a deviation from security best practices and represent a potential risk.

\begin{table}[h!]
\centering
\caption{Security Controls Questionnaire Analysis}
\begin{tabular}{p{0.6\linewidth} c l}
\toprule
\textbf{Control Question} & \textbf{Response} & \textbf{Status} \\
\midrule
Do you require MFA to access email? & \no & \textbf{\textcolor{red}{Critical Gap}} \\
Do you require MFA to log into computers? & \no & \textbf{\textcolor{red}{Critical Gap}} \\
Do you require MFA to access sensitive data systems? & \no & \textbf{\textcolor{red}{Critical Gap}} \\
Does your organization have an employee acceptable use policy? & \no & \textbf{\textcolor{orange}{High Risk}} \\
\addlinespace
Does your organization do security awareness training for new employees? & \yes & Satisfactory \\
Does your organization do security awareness training for all employees at least once per year? & \yes & Satisfactory \\
\bottomrule
\end{tabular}
\end{table}

% -----------------------------------------------------------------------------
% SECTION 4: TECHNICAL SCAN RESULTS
% -----------------------------------------------------------------------------
\section{Technical Scan Results}

A network scan was performed on the specified target to identify open ports and exposed services.

\begin{itemize}
    \item \textbf{Target IP Address:} \texttt{192.168.0.5}
    \item \textbf{Scan Date:} Not provided in scan data. Report generated on \today.
    \item \textbf{Scanner Used:} Nmap
\end{itemize}

\noindent The scan revealed that the host is online, but no open ports were discovered. The state of all tested ports was "closed."

\begin{table}[h!]
\centering
\caption{Scan Results for Target: \texttt{192.168.0.5}}
\begin{tabular}{l l l l}
\toprule
\textbf{Port} & \textbf{State} & \textbf{Service} & \textbf{Version} \\
\midrule
80/tcp & closed & http & N/A \\
\bottomrule
\end{tabular}
\end{table}

\paragraph{Correlation Note:} A pre-existing risk entry noted that port 80 was open, indicating an unencrypted web server. The current scan results show this port as \textbf{closed}. This suggests that the vulnerability has been successfully remediated. This should be internally verified and the risk register updated accordingly.

% -----------------------------------------------------------------------------
% SECTION 5: CONSOLIDATED RISK ASSESSMENT
% -----------------------------------------------------------------------------
\section{Consolidated Risk Assessment}

The following table synthesizes findings from the security control review, technical scan, and pre-existing risk data into a consolidated list of current risks.

\begin{table}[h!]
\centering
\caption{Summary of Identified Risks}
\begin{tabular}{p{0.25\linewidth} p{0.45\linewidth} l l}
\toprule
\textbf{Risk Name} & \textbf{Description} & \textbf{Severity} & \textbf{Source} \\
\midrule
\textbf{No MFA on Sensitive Systems} & Access to sensitive data is not protected by a second authentication factor, creating a high risk of data breach from compromised credentials. & \textbf{Critical} & Questionnaire \\
\addlinespace
\textbf{No MFA on Email} & Email accounts lack MFA, making them vulnerable to takeover attacks, which can lead to business email compromise and further intrusions. & High & Questionnaire \\
\addlinespace
\textbf{No MFA on Endpoints} & Computers can be accessed without MFA, increasing the risk of unauthorized access if credentials are stolen or a device is lost. & High & Questionnaire \\
\addlinespace
\textbf{Missing Acceptable Use Policy} & The absence of a formal AUP leads to inconsistent security practices and a lack of enforceable guidelines for employees. & Medium & Questionnaire \\
\addlinespace
\textbf{Unencrypted Web Server} & Port 80 was previously open. Current scans show it is closed, indicating likely remediation. & Informational (Remediated) & Risk Register \& Scan \\
\bottomrule
\end{tabular}
\end{table}

% -----------------------------------------------------------------------------
% SECTION 6: RECOMMENDATIONS
% -----------------------------------------------------------------------------
\section{Recommendations}

Based on the consolidated risk assessment, the following actions are recommended to improve the cybersecurity posture of \textbf{Apex Legends Group}. Recommendations are prioritized by severity.

\begin{enumerate}
    \item \textbf{Implement Multi-Factor Authentication (Priority: Critical)}
    \begin{itemize}
        \item \textbf{Immediate Action:} Enable and enforce MFA for all users on all email accounts and any systems identified as containing sensitive data.
        \item \textbf{Short-Term Action:} Develop a plan to roll out MFA for all employee computer logins to strengthen endpoint security.
    \end{itemize}
    
    \item \textbf{Develop and Implement an Acceptable Use Policy (AUP) (Priority: High)}
    \begin{itemize}
        \item \textbf{Action:} Draft a formal AUP that clearly defines the rules and responsibilities for employees when using company networks, devices, and data.
        \item \textbf{Action:} Integrate the AUP into the employee onboarding process and distribute it to all current staff for review and acknowledgment. This will complement the existing security awareness training program.
    \end{itemize}
    
    \item \textbf{Verify Remediation and Update Risk Register (Priority: Low)}
    \begin{itemize}
        \item \textbf{Action:} Confirm with the IT/systems administration team that the closure of port 80 on host \texttt{192.168.0.5} was an intentional and permanent remediation.
        \item \textbf{Action:} Formally update the internal risk register to close the "Unencrypted Web Server" finding, documenting the date of remediation.
    \end{itemize}
\end{enumerate}

\end{document}
```