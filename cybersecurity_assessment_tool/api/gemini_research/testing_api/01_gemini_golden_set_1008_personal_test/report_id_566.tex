```latex
\documentclass[12pt]{article}

% Preamble: Required Packages
\usepackage[margin=1in]{geometry}
\usepackage{pifont} % For checkmarks and crosses
\usepackage{booktabs} % For professional tables
\usepackage{hyperref} % For clickable links and better PDF metadata
\usepackage{url} % For formatting URLs
\usepackage{seqsplit} % For splitting long strings like IPs
\usepackage{graphicx} % For potential logos in the future
\usepackage{xcolor} % For colors

% Hyperref Setup
\hypersetup{
    colorlinks=true,
    linkcolor=blue,
    filecolor=magenta,      
    urlcolor=cyan,
    pdftitle={Cybersecurity Assessment Report},
    pdfauthor={Cybersecurity Analyst},
    pdfsubject={Security Assessment},
    pdfkeywords={Security, Risk, Assessment},
    bookmarks=true
}

% Document Title and Author
\title{Cybersecurity Assessment Report \\ \large for \\ \textbf{Iron Oak Furniture}}
\author{Cybersecurity Analyst}
\date{\today}

\begin{document}

\maketitle
\thispagestyle{empty}
\newpage

\tableofcontents
\newpage

% --- 1. Executive Overview ---
\section{Executive Overview}

This report provides a comprehensive cybersecurity assessment for \textbf{Iron Oak Furniture}, based on the analysis of network scan data, organizational security controls, and pre-existing risk information. The assessment reveals several critical and high-risk vulnerabilities that expose the organization to significant threats, including unauthorized access, data breaches, and malware infections.

Key findings indicate a combination of technical misconfigurations and procedural gaps. A publicly accessible FTP server was discovered running a dangerously outdated version with anonymous access enabled, posing an immediate threat. Furthermore, the review of security controls identified a lack of fundamental security practices, such as mandatory multi-factor authentication (MFA) for computer access, a formal acceptable use policy, and a consistent employee security awareness training program.

These issues, compounded by the existing risk of end-of-life operating systems (Windows 7), create a fragile security posture. Immediate and decisive action is required to remediate these vulnerabilities and establish a more resilient defense against cyber threats. This report outlines specific, actionable recommendations to address each identified risk.

% --- 2. Organizational Information ---
\section{Organizational Information}

The following details were provided for the assessment scope.

\begin{itemize}
    \item \textbf{Organization Name:} Iron Oak Furniture
    \item \textbf{Primary Email Domain:} \seqsplit{\texttt{IronOakFurniture.com}}
    \item \textbf{Primary Website Domain:} \seqsplit{\url{www.IronOakFurniture.com}}
    \item \textbf{External IP Address:} \seqsplit{\texttt{227.247.119.56}}
\end{itemize}

% --- 3. Security Control Review ---
\section{Security Control Review}

A review of the organization's security controls was conducted via a questionnaire. The responses highlight critical gaps in foundational security practices. A summary of the findings is presented in Table \ref{tab:controls}.

\begin{table}[h!]
\centering
\caption{Security Controls Questionnaire Results}
\label{tab:controls}
\begin{tabular}{@{}lcc@{}}
\toprule
\textbf{Control Question} & \textbf{Response} & \textbf{Status} \\
\midrule
Do you require MFA to access email? & Yes & \ding{51} \\
Do you require MFA to log into computers? & No & \textcolor{red}{\ding{55}} \\
Do you require MFA to access sensitive data systems? & Yes & \ding{51} \\
Does your organization have an employee acceptable use policy? & No & \textcolor{red}{\ding{55}} \\
Does your organization do security awareness training for new employees? & No & \textcolor{red}{\ding{55}} \\
Does your organization do security awareness training annually? & No & \textcolor{red}{\ding{55}} \\
\bottomrule
\end{tabular}
\end{table}

\subsection*{Analysis of Control Gaps}
The responses marked with a \textcolor{red}{\ding{55}} represent significant weaknesses:
\begin{itemize}
    \item \textbf{No MFA for Computer Logins:} This is a critical vulnerability. Without MFA, compromised credentials (e.g., from a phishing attack) are sufficient for an attacker to gain access to an employee's workstation and, potentially, the entire internal network.
    \item \textbf{Lack of Acceptable Use Policy (AUP):} An AUP is a foundational document that sets clear expectations for employees regarding the use of company assets. Its absence can lead to inconsistent security practices and misuse of resources.
    \item \textbf{No Security Awareness Training:} Employees are the first line of defense. Without training, they are far more susceptible to social engineering, phishing, and other common attack vectors. This gap applies to both new and existing employees, indicating a systemic issue.
\end{itemize}

% --- 4. Technical Scan Results ---
\section{Technical Scan Results}

A network scan was performed to identify open ports and services on the target system.

\subsection{Host: \texttt{10.0.0.15}}
The scan identified one host as active and responsive. The results for this host are detailed in Table \ref{tab:scan}.

\begin{table}[h!]
\centering
\caption{Open Ports and Services on \texttt{10.0.0.15}}
\label{tab:scan}
\begin{tabular}{@{}cllll@{}}
\toprule
\textbf{Port} & \textbf{State} & \textbf{Service} & \textbf{Product / Version} & \textbf{Details} \\
\midrule
21/tcp & Open & ftp & vsftpd 2.3.4 & Anonymous FTP login allowed \\
\bottomrule
\end{tabular}
\end{table}

\subsection*{Analysis of Technical Findings}
The technical scan revealed a \textbf{critical risk}:
\begin{itemize}
    \item \textbf{Outdated and Vulnerable FTP Service:} The server is running \texttt{vsftpd version 2.3.4}. This specific version is notoriously vulnerable to a backdoor command execution vulnerability (CVE-2011-2523), which was inserted into the source code by an attacker. If exploited, it allows a remote attacker to gain a root shell on the server.
    \item \textbf{Anonymous FTP Enabled:} The configuration allows anyone on the network to log in without credentials. This is a severe misconfiguration that can be abused for data exfiltration, staging of malicious files, or as a pivot point for further attacks into the network.
\end{itemize}

% --- 5. Consolidated Risk Assessment ---
\section{Consolidated Risk Assessment}

By correlating the security control gaps, technical findings, and pre-existing risk data, we have compiled a summary of the most significant risks facing the organization. These are prioritized in Table \ref{tab:risks}.

\begin{table}[h!]
\centering
\caption{Summary of Identified Risks}
\label{tab:risks}
\begin{tabular}{@{}p{0.1\linewidth} p{0.25\linewidth} p{0.4\linewidth} p{0.15\linewidth}@{}}
\toprule
\textbf{Risk ID} & \textbf{Risk Name} & \textbf{Description} & \textbf{Severity} \\
\midrule
\textbf{R-01} & Insecure FTP Service & An outdated, vulnerable FTP server (\texttt{vsftpd 2.3.4}) with anonymous login enabled allows for potential remote code execution and unauthorized file access. & \textbf{Critical} \\
\addlinespace
\textbf{R-02} & Lack of Endpoint MFA & The absence of multi-factor authentication on computer logins exposes the organization to account takeover via compromised credentials. & \textbf{High} \\
\addlinespace
\textbf{R-03} & Missing Security Policies \& Training & The lack of an AUP and security awareness training program leaves the organization vulnerable to insider threats and social engineering attacks. & \textbf{High} \\
\addlinespace
\textbf{R-04} & Outdated Windows Policy & Workstations are running Windows 7, an end-of-life operating system that no longer receives security updates, making them highly susceptible to exploitation. & \textbf{Medium} \\
\bottomrule
\end{tabular}
\end{table}

% --- 6. Recommendations ---
\section{Recommendations}

The following actions are recommended to mitigate the identified risks. Recommendations are categorized for immediate and long-term implementation.

\subsection{R-01: Insecure FTP Service (Critical)}
\begin{itemize}
    \item \textbf{Immediate Action:}
    \begin{enumerate}
        \item Take the FTP server offline immediately.
        \item If the service is business-critical, upgrade \texttt{vsftpd} to the latest stable version and disable anonymous access. All access must be protected by strong, unique passwords.
    \end{enumerate}
    \item \textbf{Long-Term Strategy:}
    \begin{enumerate}
        \item Decommission the FTP service entirely.
        \item Migrate all file transfer workflows to a secure protocol such as SFTP (SSH File Transfer Protocol) or FTPS (FTP over SSL/TLS).
    \end{enumerate}
\end{itemize}

\subsection{R-02: Lack of Endpoint MFA (High)}
\begin{itemize}
    \item \textbf{Immediate Action:}
    \begin{enumerate}
        \item Procure and deploy an MFA solution for all employee workstations and servers.
        \item Create a policy mandating the use of MFA for all computer and remote access logins.
    \end{enumerate}
    \item \textbf{Long-Term Strategy:}
    \begin{enumerate}
        \item Integrate MFA into a centralized Identity and Access Management (IAM) or Single Sign-On (SSO) solution to ensure consistent application across all company resources.
    \end{enumerate}
\end{itemize}

\subsection{R-03: Missing Security Policies \& Training (High)}
\begin{itemize}
    \item \textbf{Immediate Action:}
    \begin{enumerate}
        \item Develop and implement a formal Acceptable Use Policy (AUP). Require all employees to read and acknowledge it.
        \item Enroll all employees in a baseline security awareness training course covering topics like phishing, password hygiene, and social engineering.
    \end{enumerate}
    \item \textbf{Long-Term Strategy:}
    \begin{enumerate}
        \item Establish a recurring, annual security awareness training program for all staff.
        \item Implement a continuous training and testing program, such as periodic phishing simulations.
    \end{enumerate}
\end{itemize}

\subsection{R-04: Outdated Windows Policy (Medium)}
\begin{itemize}
    \item \textbf{Immediate Action:}
    \begin{enumerate}
        \item Isolate all Windows 7 machines from sensitive network segments.
        \item Prioritize the upgrade plan for these machines to a supported operating system (e.g., Windows 10 or 11).
    \end{enumerate}
    \item \textbf{Long-Term Strategy:}
    \begin{enumerate}
        \item Fully execute the hardware and software refit to eliminate all end-of-life systems from the environment.
        \item Implement a formal patch and lifecycle management policy to prevent this issue from recurring.
    \end{enumerate}
\end{itemize}

\end{document}
```