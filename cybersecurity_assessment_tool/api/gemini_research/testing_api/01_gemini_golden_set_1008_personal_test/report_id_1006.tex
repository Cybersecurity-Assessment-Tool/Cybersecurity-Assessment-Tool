```latex
\documentclass[12pt]{article}

% -----------------------------------------------------------------------------
% PREAMBLE
% -----------------------------------------------------------------------------
\usepackage[margin=1in]{geometry} % Set page margins
\usepackage{pifont}               % For checkmarks and crosses (\ding)
\usepackage{booktabs}             % For professional-looking tables
\usepackage{hyperref}             % For clickable links and references
\usepackage{url}                  % For formatting URLs
\usepackage{seqsplit}             % For splitting long strings in texttt
\usepackage[utf8]{inputenc}       % For UTF-8 input

% Define colors for hyperref
\hypersetup{
    colorlinks=true,
    linkcolor=black,
    filecolor=magenta,      
    urlcolor=blue,
    pdftitle={Cybersecurity Posture Report},
    pdfpagemode=FullScreen,
}

% Define checkmark and cross symbols for clarity
\newcommand{\cmark}{\ding{51}} % Checkmark
\newcommand{\xmark}{\ding{55}} % Cross

% -----------------------------------------------------------------------------
% DOCUMENT START
% -----------------------------------------------------------------------------
\begin{document}

% -----------------------------------------------------------------------------
% TITLE PAGE
% -----------------------------------------------------------------------------
\title{
    \vspace{2cm}
    \textbf{Cybersecurity Posture Report} \\
    \large \textit{Analysis and Recommendations} \\
    \vspace{1cm}
    \large For: Paper Plane Publishing
    \vspace{2cm}
}
\author{Cybersecurity Analysis Division}
\date{\today}
\maketitle
\thispagestyle{empty}
\newpage

\tableofcontents
\newpage

% -----------------------------------------------------------------------------
% SECTION 1: EXECUTIVE OVERVIEW
% -----------------------------------------------------------------------------
\section{Executive Overview}

This report provides a comprehensive analysis of the cybersecurity posture for \textbf{Paper Plane Publishing}. The assessment combines a review of organizational security controls, an external network scan, and an evaluation of known risks to provide a holistic view of the current security landscape.

The key findings indicate that while some foundational security controls, such as Multi-Factor Authentication (MFA) for email and computer access, are in place, there are \textbf{critical gaps} in both administrative policies and technical access controls for sensitive data. 

Specifically, the following high-priority issues were identified:
\begin{itemize}
    \item \textbf{Lack of MFA for Sensitive Data Systems:} The absence of mandatory MFA for accessing sensitive data exposes the organization's most valuable assets to significant risk from credential compromise.
    \item \textbf{Absence of an Acceptable Use Policy (AUP):} Without a formal AUP, there is no defined standard for employee behavior regarding IT assets, leading to potential misuse and security incidents.
    \item \textbf{Inadequate Security Onboarding:} New employees do not receive security awareness training, creating a window of vulnerability from the moment they join the organization.
\end{itemize}

The external network scan of the target system did not reveal any open ports, suggesting that the perimeter firewall is effectively configured to block unsolicited inbound traffic. No pre-existing vulnerabilities were provided for review.

Immediate action is recommended to address the identified control gaps to mitigate the risk of a security breach and strengthen the overall defense posture.

% -----------------------------------------------------------------------------
% SECTION 2: ORGANIZATIONAL INFORMATION
% -----------------------------------------------------------------------------
\section{Organizational Information}

The following details were provided for the assessment.

\begin{table}[h!]
\centering
\begin{tabular}{@{}ll@{}}
\toprule
\textbf{Attribute} & \textbf{Value} \\ \midrule
Organization Name & Paper Plane Publishing \\
Email Domain & \texttt{PaperPlanePublishing.com} \\
Website Domain & \url{www.PaperPlanePublishing.com} \\
Primary External IP & \texttt{15.180.3.28} \\ \bottomrule
\end{tabular}
\caption{Client Organizational Details}
\end{table}

% -----------------------------------------------------------------------------
% SECTION 3: SECURITY CONTROL REVIEW
% -----------------------------------------------------------------------------
\section{Security Control Review}

A review of administrative and technical security controls was conducted based on a standardized questionnaire. The results highlight areas of both strength and weakness in the current security program. Gaps identified with an \xmark\ represent a deviation from security best practices and are addressed in the Risk Assessment section.

\begin{table}[h!]
\centering
\begin{tabular}{@{}p{0.7\textwidth}cc@{}}
\toprule
\textbf{Control Question} & \textbf{Response} & \textbf{Status} \\ \midrule
Do you require MFA to access email? & Yes & \cmark \\
Do you require MFA to log into computers? & Yes & \cmark \\
\textbf{Do you require MFA to access sensitive data systems?} & \textbf{No} & \textbf{\xmark} \\
\textbf{Does your organization have an employee acceptable use policy?} & \textbf{No} & \textbf{\xmark} \\
\textbf{Does your organization do security awareness training for new employees?} & \textbf{No} & \textbf{\xmark} \\
Does your organization do security awareness training for all employees at least once per year? & Yes & \cmark \\ \bottomrule
\end{tabular}
\caption{Security Control Questionnaire Results}
\end{table}

% -----------------------------------------------------------------------------
% SECTION 4: TECHNICAL SCAN RESULTS
% -----------------------------------------------------------------------------
\section{Technical Scan Results}

An external network vulnerability scan was performed to identify potentially exposed services on the organization's perimeter.

\begin{itemize}
    \item \textbf{Target IP Address:} \texttt{[Target IP]}
    \item \textbf{Scan Summary:} The network scan did not identify any open TCP or UDP ports on the target system. 
    \item \textbf{Analyst Note:} This result indicates that the host may be offline or, more likely, that a perimeter firewall is effectively blocking all inbound connection attempts from the internet. While this is a positive sign for perimeter security, it does not provide insight into the security of internal systems or services that may be misconfigured.
\end{itemize}

% -----------------------------------------------------------------------------
% SECTION 5: RISK ASSESSMENT
% -----------------------------------------------------------------------------
\section{Risk Assessment}

This section synthesizes the findings from the security control review into a list of identified risks. No pre-existing risks were provided for this assessment. The risks below are prioritized by severity and potential impact on the organization.

\begin{table}[h!]
\centering
\begin{tabular}{@{}p{0.1\textwidth}p{0.3\textwidth}p{0.4\textwidth}l@{}}
\toprule
\textbf{ID} & \textbf{Risk Name} & \textbf{Description} & \textbf{Severity} \\ \midrule
\textbf{RISK-001} & \textbf{Lack of MFA on Sensitive Systems} & The absence of MFA on systems containing sensitive data allows an attacker with compromised credentials (e.g., username/password) to gain direct access to critical assets. & \textbf{Critical} \\
\addlinespace
\textbf{RISK-002} & \textbf{No Employee Acceptable Use Policy (AUP)} & Without a formal AUP, employees lack clear guidance on the proper use of company technology. This increases the risk of insider threat, data leakage, and non-compliance. & \textbf{High} \\
\addlinespace
\textbf{RISK-003} & \textbf{No Security Training for New Employees} & New hires are not provided with security awareness training during onboarding. This makes them highly susceptible to phishing, social engineering, and other common attacks from their first day. & \textbf{High} \\ \bottomrule
\end{tabular}
\caption{Identified Risks and Severity}
\end{table}

% -----------------------------------------------------------------------------
% SECTION 6: RECOMMENDATIONS
% -----------------------------------------------------------------------------
\section{Recommendations}

Based on the analysis, the following actions are recommended to mitigate the identified risks and improve the overall security posture of \textbf{Paper Plane Publishing}.

\subsection*{Immediate Priority (Critical Risk Mitigation)}
\begin{enumerate}
    \item \textbf{Enforce MFA on All Sensitive Data Systems (RISK-001):}
    \begin{itemize}
        \item \textbf{Action:} Immediately begin a project to identify all systems containing sensitive or critical data and enforce non-phishable MFA for all user access.
        \item \textbf{Impact:} Drastically reduces the risk of unauthorized access due to stolen credentials, protecting the organization's most valuable information.
    \end{itemize}
\end{enumerate}

\subsection*{High Priority (High Risk Mitigation)}
\begin{enumerate}
    \setcounter{enumi}{1} % Continue numbering
    \item \textbf{Develop and Implement an Acceptable Use Policy (RISK-002):}
    \begin{itemize}
        \item \textbf{Action:} Draft a formal AUP that clearly defines the rules for using company networks, devices, and data. This policy should be reviewed by legal counsel and distributed to all employees for acknowledgment.
        \item \textbf{Impact:} Establishes a baseline for secure employee behavior, reduces legal liability, and provides a framework for enforcing security standards.
    \end{itemize}
    \item \textbf{Establish a Security Onboarding Program (RISK-003):}
    \begin{itemize}
        \item \textbf{Action:} Integrate a mandatory security awareness training module into the new employee onboarding process. This training should cover key topics such as phishing, password security, and the new AUP.
        \item \textbf{Impact:} Ensures that all new staff members understand their security responsibilities from day one, reducing the likelihood of early-tenure security incidents.
    \end{itemize}
\end{enumerate}

\subsection*{General Recommendations}
\begin{enumerate}
    \setcounter{enumi}{3} % Continue numbering
    \item \textbf{Conduct Regular Control Reviews:} Schedule annual or semi-annual reviews of all security controls and policies to ensure they remain effective and aligned with the evolving threat landscape.
\end{enumerate}

% -----------------------------------------------------------------------------
% DOCUMENT END
% -----------------------------------------------------------------------------
\end{document}
```