```latex
\documentclass[12pt]{article}

% Preamble: Required Packages
\usepackage[margin=1in]{geometry}
\usepackage{pifont} % For checkmarks and crosses
\usepackage{booktabs} % For professional tables
\usepackage{hyperref} % For clickable links
\usepackage{url} % For URL formatting
\usepackage{seqsplit} % For splitting long strings without spaces
\usepackage{graphicx}
\usepackage{xcolor}

% Document Metadata
\title{Cybersecurity Posture Assessment Report}
\author{Cybersecurity Analysis Division}
\date{November 22, 2025}

% Hyperref Setup
\hypersetup{
    colorlinks=true,
    linkcolor=blue,
    filecolor=magenta,      
    urlcolor=cyan,
    pdftitle={Cybersecurity Posture Assessment Report},
    pdfpagemode=FullScreen,
}

\begin{document}

\maketitle
\thispagestyle{empty}
\newpage

\tableofcontents
\newpage

% --- 1. Executive Summary ---
\section{Executive Summary}

This report provides a comprehensive cybersecurity assessment for \textbf{Golden Gate Gaming}, conducted on November 22, 2025. The analysis is based on a combination of technical network scanning, a review of organizational security controls, and an evaluation of pre-existing risks.

The assessment reveals a security posture with several significant areas of concern that require immediate attention. While the organization has implemented some foundational security controls, such as Multi-Factor Authentication (MFA) for email and sensitive systems, critical gaps were identified.

Key findings include:
\begin{itemize}
    \item \textbf{Critical Control Gap:} The absence of mandatory MFA for workstation logins presents a high-impact risk, as a single compromised password could grant an attacker direct access to an endpoint.
    \item \textbf{Critical Software Vulnerability:} The external-facing web server is running an outdated version of Nginx (1.18.0), which has multiple known high-severity vulnerabilities. This exposes the organization to potential remote exploitation.
    \item \textbf{High-Risk Onboarding Process:} New employees do not receive security awareness training upon being hired, creating a window of vulnerability where they are more susceptible to phishing and social engineering attacks.
\end{itemize}

This report details these findings and provides actionable recommendations to mitigate the identified risks and strengthen the overall security posture of \textbf{Golden Gate Gaming}.

% --- 2. Organizational Information ---
\section{Organizational Information}

The following information was provided for the assessment.

\begin{tabular}{@{}ll}
\toprule
\textbf{Attribute} & \textbf{Value} \\
\midrule
Organization Name & \textbf{Golden Gate Gaming} \\
Email Domain & \texttt{GoldenGateGaming.com} \\
Website Domain & \url{www.GoldenGateGaming.com} \\
External IP Address & \texttt{190.186.40.192} \\
\bottomrule
\end{tabular}

% --- 3. Security Control Review ---
\section{Security Control Review}

A review of administrative and technical security controls was conducted based on a standardized questionnaire. The responses indicate gaps in endpoint security and employee training protocols. A "No" response (\ding{55}) highlights a deviation from security best practices and a potential risk.

\begin{table}[h!]
\centering
\caption{Security Controls Questionnaire Results}
\begin{tabular}{@{}p{0.8\linewidth}c@{}}
\toprule
\textbf{Control Question} & \textbf{Response} \\
\midrule
Do you require MFA to access email? & \textcolor{green}{\ding{51}} \\
Do you require MFA to log into computers? & \textcolor{red}{\ding{55}} \\
Do you require MFA to access sensitive data systems? & \textcolor{green}{\ding{51}} \\
Does your organization have an employee acceptable use policy? & \textcolor{green}{\ding{51}} \\
Does your organization do security awareness training for new employees? & \textcolor{red}{\ding{55}} \\
Does your organization do security awareness training for all employees at least once per year? & \textcolor{green}{\ding{51}} \\
\bottomrule
\end{tabular}
\end{table}

% --- 4. Technical Scan Results ---
\section{Technical Scan Results}

A network scan was performed on November 22, 2025, to identify open ports and services on the target system.

\subsection{Host: \texttt{192.168.10.5}}
The scan identified the following open port and running service.

\begin{table}[h!]
\centering
\caption{Open Ports and Services for \texttt{192.168.10.5}}
\begin{tabular}{@{}lllll@{}}
\toprule
\textbf{Port} & \textbf{State} & \textbf{Service} & \textbf{Product} & \textbf{Version} \\
\midrule
443/tcp & open & https & nginx & 1.18.0 \\
\bottomrule
\end{tabular}
\end{table}

\paragraph{Analysis:} The scan confirms a web server running \textbf{Nginx version 1.18.0}. This version was released in April 2020 and is now significantly outdated. It is known to be vulnerable to several security issues, including high-severity vulnerabilities such as CVE-2021-23017, which could allow an attacker to crash the process or potentially achieve other impacts. This finding represents a critical technical risk.

% --- 5. Risk Assessment ---
\section{Risk Assessment}

The following table synthesizes findings from the security control review and the technical scan. No pre-existing risks were provided for this assessment.

\begin{table}[h!]
\centering
\caption{Identified Risks and Severity}
\begin{tabular}{@{}p{0.1\linewidth}p{0.3\linewidth}p{0.15\linewidth}p{0.35\linewidth}@{}}
\toprule
\textbf{Risk ID} & \textbf{Risk Name} & \textbf{Severity} & \textbf{Description} \\
\midrule
RISK-001 & Lack of MFA on Workstations & \textbf{Critical} & The absence of MFA for computer logins means that a compromised password is the only barrier to an attacker gaining full access to an employee's workstation and potentially the internal network. \\
\addlinespace
RISK-002 & Outdated Nginx Web Server & \textbf{Critical} & The public-facing web server is running Nginx 1.18.0, a version with multiple documented high-severity vulnerabilities. This could lead to a system compromise, data breach, or denial of service. \\
\addlinespace
RISK-003 & Inadequate New Employee Security Training & \textbf{High} & New hires are not trained on security policies and threats upon joining. This makes them prime targets for phishing and social engineering attacks, which could lead to credential theft or malware infection. \\
\bottomrule
\end{tabular}
\end{table}

% --- 6. Recommendations ---
\section{Recommendations}

The following actions are recommended to mitigate the identified risks and improve the organization's security posture.

\subsection{RISK-001: Lack of MFA on Workstations}
\begin{itemize}
    \item \textbf{Action:} Implement and enforce a mandatory Multi-Factor Authentication (MFA) policy for all workstation and laptop logins (both local and remote).
    \item \textbf{Priority:} Immediate
    \item \textbf{Justification:} This measure drastically reduces the risk of unauthorized access from stolen or weak credentials, protecting endpoints which are the gateway to the corporate network.
\end{itemize}

\subsection{RISK-002: Outdated Nginx Web Server}
\begin{itemize}
    \item \textbf{Action:} Plan and execute an upgrade of the Nginx server on \texttt{192.168.10.5} from version 1.18.0 to the latest stable release. A patch management and vulnerability scanning program should be implemented to prevent future occurrences.
    \item \textbf{Priority:} Immediate
    \item \textbf{Justification:} Upgrading to a patched version will mitigate known vulnerabilities, protecting the server and the data it hosts from public exploits.
\end{itemize}

\subsection{RISK-003: Inadequate New Employee Security Training}
\begin{itemize}
    \item \textbf{Action:} Develop and integrate a mandatory security awareness training module into the new employee onboarding process. This training should cover, at a minimum, phishing awareness, password hygiene, and the acceptable use policy.
    \item \textbf{Priority:} High
    \item \textbf{Justification:} Educating employees from day one is one of the most cost-effective ways to build a human firewall and reduce the likelihood of successful social engineering attacks.
\end{itemize}

\end{document}
```