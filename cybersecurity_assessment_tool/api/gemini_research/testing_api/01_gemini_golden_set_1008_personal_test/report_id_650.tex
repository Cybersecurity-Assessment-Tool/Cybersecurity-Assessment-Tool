```latex
\documentclass[12pt]{article}

% Preamble: Required Packages
\usepackage[margin=1in]{geometry}
\usepackage{pifont} % For checkmarks and crosses
\usepackage{booktabs} % For professional tables
\usepackage{hyperref} % For clickable links
\usepackage{url} % For URL formatting
\usepackage{seqsplit} % For splitting long strings to prevent overflow
\usepackage{graphicx}
\usepackage{xcolor}

% Document Metadata
\title{Cybersecurity Posture Assessment Report}
\author{Cybersecurity Analysis Division}
\date{\today}

% Hyperref Setup
\hypersetup{
    colorlinks=true,
    linkcolor=blue,
    filecolor=magenta,      
    urlcolor=cyan,
    pdftitle={Cybersecurity Posture Assessment Report},
    pdfpagemode=FullScreen,
}

\begin{document}

\maketitle
\thispagestyle{empty}
\newpage

\tableofcontents
\newpage

% --- 1. Executive Summary ---
\section{Executive Summary}
This report provides a comprehensive analysis of the cybersecurity posture for \textbf{Apex Legends Group}. The assessment is based on a correlation of network scan data, a security controls questionnaire, and a review of pre-existing documented risks.

The analysis has identified several critical and high-risk vulnerabilities that require immediate attention. Key findings include:
\begin{itemize}
    \item \textbf{Critically Vulnerable FTP Server:} A public-facing FTP server is running \texttt{vsftpd version 2.3.4}, a version known to contain a critical backdoor vulnerability (CVE-2011-2523). Furthermore, the server is configured to allow anonymous logins, posing a severe risk of unauthorized access and data breach.
    \item \textbf{Insufficient Access Controls:} Multi-Factor Authentication (MFA) is not enforced for accessing sensitive data systems. This significant policy gap dramatically increases the risk of unauthorized access to the organization's most critical assets.
    \item \textbf{Lack of Security Training:} The organization does not provide security awareness training for new or existing employees. This exposes the organization to a high risk of social engineering attacks, human error, and policy violations.
\end{itemize}

The combination of these findings indicates a fragile security posture. Immediate remediation of the identified critical vulnerabilities is strongly recommended to prevent potential exploitation and mitigate significant business impact.

% --- 2. Organizational Information ---
\section{Organizational Information}
The following information was provided for the assessment.

\begin{tabular}{@{}ll}
\toprule
\textbf{Attribute} & \textbf{Value} \\
\midrule
Organization Name & \textbf{Apex Legends Group} \\
Email Domain & \texttt{ApexLegendsGroup.org} \\
Website Domain & \url{www.ApexLegendsGroup.org} \\
External IP Address & \texttt{7.21.164.106} \\
\bottomrule
\end{tabular}

% --- 3. Security Control Review ---
\section{Security Control Review}
A review of the organization's security controls was conducted via a questionnaire. The responses highlight significant gaps in access control and employee security awareness. A "No" response indicates a deviation from security best practices and a potential area of risk.

\begin{tabular}{@{}p{0.8\linewidth}c}
\toprule
\textbf{Control Question} & \textbf{Response} \\
\midrule
Do you require MFA to access email? & \ding{51} \\ % Yes
Do you require MFA to log into computers? & \ding{51} \\ % Yes
\textbf{Do you require MFA to access sensitive data systems?} & \textcolor{red}{\ding{55}} \\ % No - CRITICAL
Does your organization have an employee acceptable use policy? & \ding{51} \\ % Yes
\textbf{Does your organization do security awareness training for new employees?} & \textcolor{red}{\ding{55}} \\ % No - HIGH
\textbf{Does your organization do security awareness training for all employees at least once per year?} & \textcolor{red}{\ding{55}} \\ % No - HIGH
\bottomrule
\end{tabular}

% --- 4. Technical Scan Results ---
\section{Technical Scan Results}
An external network scan was performed on the target IP address \texttt{10.0.0.15}. The scan identified one open port with a critically vulnerable service.

\subsection{Open Ports and Services}
\begin{tabular}{@{}llllll}
\toprule
\textbf{Port} & \textbf{Proto} & \textbf{State} & \textbf{Service} & \textbf{Version} & \textbf{Details} \\
\midrule
21 & tcp & open & ftp & vsftpd 2.3.4 & \parbox[t]{5cm}{\textbf{Critical Finding:} Anonymous FTP login is allowed. This version is known to be vulnerable to a backdoor (CVE-2011-2523).} \\
\bottomrule
\end{tabular}

\subsection{Analysis of Technical Findings}
The presence of an open FTP port running \texttt{vsftpd 2.3.4} is a severe security risk. This specific version contains a well-documented backdoor that, if triggered, opens a command shell on port 6200, allowing a remote attacker to execute arbitrary commands on the system. The configuration allowing anonymous FTP access further exacerbates this risk by permitting unauthenticated users to interact with the server, potentially uploading malicious files or exfiltrating sensitive data.

% --- 5. Consolidated Risk Assessment ---
\section{Consolidated Risk Assessment}
The following table synthesizes findings from the security questionnaire, technical scan, and pre-existing risk documentation into a prioritized list.

\begin{tabular}{@{}p{0.05\linewidth}p{0.3\linewidth}p{0.15\linewidth}p{0.4\linewidth}}
\toprule
\textbf{ID} & \textbf{Risk / Vulnerability} & \textbf{Severity} & \textbf{Description} \\
\midrule
R-01 & Exploitable FTP Server (\texttt{vsftpd 2.3.4}) & \textbf{Critical} & The server is vulnerable to a remote command execution backdoor (CVE-2011-2523). \\
\addlinespace
R-02 & Insecure Anonymous FTP Access & \textbf{Critical} & The FTP server allows unauthenticated access, risking data leakage and malware uploads. \\
\addlinespace
R-03 & No MFA for Sensitive Data Systems & \textbf{Critical} & Lack of MFA on critical systems makes them highly susceptible to compromise via stolen credentials. \\
\addlinespace
R-04 & Lack of Security Awareness Training & \textbf{High} & Employees are not trained to recognize or respond to phishing, social engineering, or other common threats. \\
\addlinespace
R-05 & Outdated Windows Policy (Windows 7) & Medium & Workstations are running an end-of-life operating system that no longer receives security updates. \\
\bottomrule
\end{tabular}

% --- 6. Recommendations ---
\section{Recommendations}
Based on the consolidated risk assessment, the following remediation actions are recommended. They are prioritized by severity to address the most critical threats first.

\subsection{Immediate Priority (Critical Risks)}
\begin{enumerate}
    \item \textbf{Remediate Vulnerable FTP Server (R-01, R-02):}
    \begin{itemize}
        \item Take the FTP server offline immediately.
        \item If FTP is a business requirement, upgrade the \texttt{vsftpd} software to the latest stable version and reconfigure it to disable anonymous access.
        \item It is strongly recommended to replace FTP with a secure alternative like SFTP (SSH File Transfer Protocol) or FTPS (FTP over SSL/TLS).
    \end{itemize}
    \item \textbf{Implement MFA for Sensitive Systems (R-03):}
    \begin{itemize}
        \item Enforce mandatory MFA for all user accounts, especially privileged ones, that have access to sensitive data repositories, databases, and administrative consoles.
    \end{itemize}
\end{enumerate}

\subsection{High Priority Recommendations}
\begin{enumerate}
    \setcounter{enumi}{2} % Continue numbering
    \item \textbf{Establish a Security Awareness Program (R-04):}
    \begin{itemize}
        \item Implement a mandatory security awareness training module for all new employees during their onboarding process.
        \item Conduct annual, mandatory security training for all staff to keep them informed about the latest threats and organizational policies.
    \end{itemize}
\end{enumerate}

\subsection{Medium Priority Recommendations}
\begin{enumerate}
    \setcounter{enumi}{3} % Continue numbering
    \item \textbf{Address Outdated Operating Systems (R-05):}
    \begin{itemize}
        \item Execute the plan to upgrade all workstations from Windows 7 to a modern, supported operating system like Windows 10 or 11 to ensure they receive critical security patches.
    \end{itemize}
\end{enumerate}

% --- 7. Conclusion ---
\section{Conclusion}
The assessment reveals that \textbf{Apex Legends Group} is currently exposed to several critical and high-severity risks that could lead to a significant security incident. The combination of an externally facing, exploitable service, weak access controls on sensitive data, and a lack of employee security training creates a high-risk environment.

We strongly urge the immediate implementation of the recommendations outlined in this report, starting with the critical priority items. Proactive remediation will substantially improve the organization's security posture and reduce the likelihood of a successful cyberattack.

\end{document}
```