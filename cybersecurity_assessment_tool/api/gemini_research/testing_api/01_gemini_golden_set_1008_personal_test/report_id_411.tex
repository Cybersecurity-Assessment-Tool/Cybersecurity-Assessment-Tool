```latex
\documentclass[12pt]{article}

% Required Packages
\usepackage[margin=1in]{geometry}
\usepackage{pifont} % For checkmarks and crosses
\usepackage{booktabs} % For professional tables
\usepackage{hyperref} % For clickable links
\usepackage{url} % For formatting URLs
\usepackage{seqsplit} % To prevent long strings from overflowing

% Document Metadata
\title{Cybersecurity Posture Assessment Report}
\author{Cybersecurity Analyst}
\date{\today}

\begin{document}

\maketitle
\thispagestyle{empty}
\newpage
\tableofcontents
\newpage

% ------------------------------------------------------------------
% Section 1: Executive Summary
% ------------------------------------------------------------------
\section{Executive Summary}

This report provides a comprehensive cybersecurity assessment for \textbf{Nova Terra}, based on an analysis of network scan data, organizational security controls, and pre-existing risk information. The assessment was conducted to identify vulnerabilities, evaluate the current security posture, and provide actionable recommendations to mitigate identified risks.

The key findings indicate a significant risk exposure due to a combination of technical vulnerabilities and procedural gaps. A critical vulnerability was identified: an internet-facing MySQL database server running an end-of-life version (5.7.33), which no longer receives security updates. This finding is compounded by critical gaps in organizational security controls, including the absence of Multi-Factor Authentication (MFA) for computer logins and the lack of security awareness training for new employees.

Immediate remediation is required to address the exposed database to prevent a potential data breach. Further strategic improvements are necessary to strengthen endpoint security and enhance the organization's overall resilience against cyber threats.

% ------------------------------------------------------------------
% Section 2: Organizational Information
% ------------------------------------------------------------------
\section{Organizational Information}

The following details were provided for the assessment.

\begin{itemize}
    \item \textbf{Organization Name:} Nova Terra
    \item \textbf{Email Domain:} \seqsplit{\texttt{NovaTerra.com}}
    \item \textbf{Website Domain:} \seqsplit{\url{www.NovaTerra.com}}
    \item \textbf{External IP Address:} \seqsplit{\texttt{99.112.147.83}}
\end{itemize}

% ------------------------------------------------------------------
% Section 3: Security Control Review
% ------------------------------------------------------------------
\section{Security Control Review}

A review of the organization's security controls was conducted based on a questionnaire. The responses highlight critical areas for improvement in identity and access management and security awareness.

\subsection{Questionnaire Responses}

\begin{table}[h!]
\centering
\begin{tabular}{p{0.7\linewidth}c}
\toprule
\textbf{Control Question} & \textbf{Response} \\
\midrule
Do you require MFA to access email? & \ding{51} \\ % Yes
Do you require MFA to log into computers? & \color{red}\ding{55} \\ % No
Do you require MFA to access sensitive data systems? & \ding{51} \\ % Yes
Does your organization have an employee acceptable use policy? & \ding{51} \\ % Yes
Does your organization do security awareness training for new employees? & \color{red}\ding{55} \\ % No
Does your organization do security awareness training for all employees at least once per year? & \ding{51} \\ % Yes
\bottomrule
\end{tabular}
\caption{Organizational Security Control Status}
\end{table}

\subsection{Analysis of Gaps}
The review identified two significant control gaps:
\begin{itemize}
    \item \textbf{Lack of MFA for Computer Logins:} The absence of MFA on endpoints represents a high risk. If an employee's credentials are stolen, an attacker could gain direct access to the network, facilitating lateral movement and access to sensitive resources.
    \item \textbf{No Security Training for New Employees:} New hires are often targeted by social engineering attacks. Failing to provide immediate security awareness training leaves the organization vulnerable to phishing, malware, and other initial access vectors.
\end{itemize}

% ------------------------------------------------------------------
% Section 4: Technical Scan Results
% ------------------------------------------------------------------
\section{Technical Scan Results}

A network scan was performed on the target system to identify open ports and exposed services.

\subsection{Scan Details}
\begin{itemize}
    \item \textbf{Target IP Address:} \seqsplit{\texttt{172.16.50.20}}
\end{itemize}

\subsection{Open Ports and Services}

\begin{table}[h!]
\centering
\begin{tabular}{llll}
\toprule
\textbf{Port} & \textbf{State} & \textbf{Service} & \textbf{Product \& Version} \\
\midrule
3306/tcp & open & mysql & MySQL 5.7.33 \\
\bottomrule
\end{tabular}
\caption{Network Scan Findings}
\end{table}

\subsection{Technical Analysis}
The scan revealed a critical finding:
\begin{itemize}
    \item \textbf{Exposed End-of-Life Database:} Port 3306 is open, exposing a MySQL database service directly to the network. The running version, \textbf{MySQL 5.7.33}, reached its official End of Life (EOL) in October 2023. EOL software no longer receives security patches from the vendor, meaning any newly discovered vulnerabilities will remain unpatched. This presents a severe risk of data breach through exploitation. This technical finding directly confirms the pre-existing risk titled "Database Exposure".
\end{itemize}

% ------------------------------------------------------------------
% Section 5: Consolidated Risk Assessment
% ------------------------------------------------------------------
\section{Consolidated Risk Assessment}

The following table summarizes the key risks identified by correlating the security control review, technical scan results, and existing risk data.

\begin{table}[h!]
\centering
\begin{tabular}{p{0.25\linewidth}p{0.5\linewidth}l}
\toprule
\textbf{Risk Name} & \textbf{Description} & \textbf{Severity} \\
\midrule
Exposed End-of-Life Database Service & The MySQL database (v5.7.33) on port 3306 is exposed to the network and is no longer supported with security patches. This could lead to a full system compromise or data breach. & \textbf{Critical} \\
\addlinespace
Lack of Endpoint Multi-Factor Authentication & The absence of MFA for computer logins allows an attacker with stolen credentials to easily gain network access, bypassing a fundamental security layer. & \textbf{High} \\
\addlinespace
Inadequate New Hire Security Training & New employees are not receiving security awareness training, making them susceptible to social engineering attacks and increasing the risk of initial compromise. & \textbf{High} \\
\bottomrule
\end{tabular}
\caption{Summary of Identified Risks}
\end{table}

% ------------------------------------------------------------------
% Section 6: Recommendations
% ------------------------------------------------------------------
\section{Recommendations}

The following actions are recommended to mitigate the identified risks, prioritized by severity.

\subsection{Immediate Priority (Critical)}
\begin{enumerate}
    \item \textbf{Restrict Database Access:} Immediately implement strict firewall rules to block all public access to port 3306 on \seqsplit{\texttt{172.16.50.20}}. Access should be restricted to only trusted internal application servers or through a secure VPN connection.
    \item \textbf{Plan Database Upgrade:} Develop and execute a plan to upgrade the MySQL 5.7.33 instance to a fully supported version (e.g., MySQL 8.x). This is essential to ensure the service receives ongoing security patches.
\end{enumerate}

\subsection{High Priority}
\begin{enumerate}
    \item \textbf{Implement MFA for Endpoints:} Enforce Multi-Factor Authentication for all employee computer logins. This can be achieved through solutions like Windows Hello for Business, Duo, or other identity providers.
    \item \textbf{Establish Onboarding Security Training:} Develop and integrate a mandatory security awareness training module into the new employee onboarding process. This training should cover phishing, acceptable use, password hygiene, and incident reporting.
\end{enumerate}

\end{document}
```