```latex
\documentclass[12pt]{article}

% Preamble: Required Packages
\usepackage[margin=1in]{geometry}
\usepackage{pifont} % For checkmarks and crosses
\usepackage{booktabs} % For professional tables
\usepackage[hidelinks]{hyperref} % For clickable links without boxes
\usepackage{url}
\usepackage{seqsplit} % To split long monospaced text
\usepackage{xcolor}   % For colored text

% Custom Commands for Report
\newcommand{\yes}{\textcolor{green!70!black}{\ding{51}}} % Green checkmark
\newcommand{\no}{\textcolor{red}{\ding{55}}} % Red X
\newcommand{\orgname}{Iron Oak Furniture}
\newcommand{\orgdomain}{\texttt{IronOakFurniture.net}}
\newcommand{\orgip}{\texttt{115.135.235.126}}
\newcommand{\targetip}{\texttt{172.16.50.20}}

\begin{document}

% --- Report Header ---
\begin{center}
    \Large\textbf{Cybersecurity Posture Assessment Report} \\
    \vspace{4pt}
    \large Prepared for: \textbf{\orgname} \\
    \vspace{2pt}
    \normalsize \today
\end{center}

\vspace{12pt}

% --- Section 1: Executive Summary ---
\section*{Executive Summary}

This report details a critical risk exposure for \orgname, identified through an analysis of network scan data, organizational security controls, and existing risk documentation. A network scan revealed a publicly accessible database server running an \textbf{End-of-Life (EOL) version of MySQL}. This severe technical vulnerability is significantly compounded by critical organizational security gaps, including a lack of Multi-Factor Authentication (MFA) for sensitive systems and computers, the absence of a formal Acceptable Use Policy, and an incomplete security awareness training program.

The combination of an exposed, unpatchable database and weak access controls creates a high-likelihood path for unauthorized access and a potential data breach. Immediate remediation of the technical findings and strategic improvement of security policies are required to mitigate this risk and improve the organization's overall security posture.

% --- Section 2: Organizational Information ---
\section*{Organizational Information}

The following details were provided for the assessment.

\begin{center}
\begin{tabular}{ll}
\toprule
\textbf{Attribute} & \textbf{Value} \\
\midrule
Organization Name & \orgname \\
Primary Email Domain & \orgdomain \\
External IP Address & \orgip \\
\bottomrule
\end{tabular}
\end{center}

% --- Section 3: Security Control Review ---
\section*{Security Control Review}

A review of the organization's security controls was conducted via a questionnaire. The results below highlight foundational security practices and identify key areas for improvement.

\begin{center}
\begin{tabular}{p{0.8\linewidth}c}
\toprule
\textbf{Control Question} & \textbf{Status} \\
\midrule
Do you require MFA to access email? & \yes \\
Do you require MFA to log into computers? & \no \\
Do you require MFA to access sensitive data systems? & \no \\
Does your organization have an employee acceptable use policy? & \no \\
Does your organization do security awareness training for new employees? & \yes \\
Does your organization do security awareness training for all employees at least once per year? & \no \\
\bottomrule
\end{tabular}
\end{center}

\subsection*{Analysis of Control Gaps}
The review reveals critical gaps in access control and policy. The absence of MFA on computer and sensitive system logins significantly increases the risk of unauthorized access via compromised credentials. This is especially dangerous given the technical findings in this report. Furthermore, the lack of an Acceptable Use Policy and annual security training for all staff creates an environment where employees may be unaware of security best practices, making the organization more susceptible to phishing and other social engineering attacks.

% --- Section 4: Technical Scan Results ---
\section*{Technical Scan Results}

A network scan was performed on the specified target to identify open ports and exposed services.

\begin{itemize}
    \item \textbf{Target IP Address:} \targetip
    \item \textbf{Open Ports Discovered:}
    \begin{itemize}
        \item \textbf{Port 3306/tcp (MySQL):}
        \begin{itemize}
            \item \textbf{Service Banner:} MySQL 5.7.33
            \item \textbf{Status:} \textcolor{red}{\textbf{CRITICAL FINDING}}. The detected version, MySQL 5.7, reached its official \textbf{End of Life (EOL) in October 2023}. EOL software no longer receives security patches from the vendor, leaving it perpetually vulnerable to newly discovered exploits. Its exposure to the network presents a direct and severe threat.
        \end{itemize}
    \end{itemize}
\end{itemize}

% --- Section 5: Consolidated Risk Assessment ---
\section*{Consolidated Risk Assessment}

The following table synthesizes findings from the security control review, technical scan, and pre-existing risk data into a consolidated list of key risks facing the organization.

\begin{center}
\begin{tabular}{p{0.25\linewidth}p{0.5\linewidth}l}
\toprule
\textbf{Risk Name} & \textbf{Overview} & \textbf{Severity} \\
\midrule
\textbf{Exposed End-of-Life Database} & The MySQL database (v5.7.33) is directly exposed to the network and is past its End of Life, meaning it no longer receives security updates. This is a direct confirmation of the pre-existing "Database Exposure" risk, elevated to Critical due to the EOL status. & \textcolor{red}{Critical} \\
\addlinespace
\textbf{Insufficient MFA Implementation} & MFA is not enforced for computer logins or access to sensitive data systems. This weakness would allow an attacker with stolen credentials to directly access critical assets, including the exposed database. & \textcolor{orange}{High} \\
\addlinespace
\textbf{Inadequate Security Training Program} & Security awareness training is not conducted annually for all employees, increasing the organization's susceptibility to phishing and social engineering attacks that could lead to credential compromise. & \textcolor{orange}{High} \\
\addlinespace
\textbf{Missing Acceptable Use Policy} & The lack of a formal policy means there are no documented rules for employees regarding the protection of company data and the secure use of IT assets, creating ambiguity and risk. & \textcolor{yellow!80!black}{Medium} \\
\bottomrule
\end{tabular}
\end{center}

% --- Section 6: Prioritized Recommendations ---
\section*{Prioritized Recommendations}

Based on the consolidated risk assessment, the following actions are recommended to reduce risk and improve the security posture of \orgname.

\begin{enumerate}
    \item \textbf{Immediate (Containment):} Implement a firewall rule to \textbf{block all public access to TCP port 3306} on host \targetip. Access should only be permitted from trusted internal application servers or via a secure Virtual Private Network (VPN).

    \item \textbf{Critical (Remediation):} Develop and execute a plan to \textbf{migrate the MySQL 5.7 database to a fully supported version} (e.g., MySQL 8.x or a managed cloud equivalent). This is essential to ensure the database receives critical security patches.

    \item \textbf{High (Strengthen Controls):} \textbf{Enforce Multi-Factor Authentication (MFA)} across all employee computer logins and for all applications that access or manage sensitive data. This is the single most effective control to prevent unauthorized access from compromised credentials.

    \item \textbf{High (Policy \& Training):} \textbf{Develop and implement a formal Acceptable Use Policy (AUP)} for all employees to sign. Concurrently, establish a mandatory, recurring \textbf{annual security awareness training program} for the entire organization to build a stronger human firewall.
\end{enumerate}

\end{document}
```