```latex
\documentclass[12pt]{article}

% ----------------------------------------------------------------------
% PREAMBLE
% ----------------------------------------------------------------------
\usepackage[margin=1in]{geometry}
\usepackage{pifont} % For checkmarks and crosses
\usepackage{booktabs} % For professional tables
\usepackage{hyperref} % For hyperlinks
\usepackage{url}      % For URL formatting
\usepackage{seqsplit} % For splitting long strings
\usepackage[utf8]{inputenc}

\hypersetup{
    colorlinks=true,
    linkcolor=black,
    filecolor=magenta,      
    urlcolor=blue,
    pdftitle={Cybersecurity Posture Assessment Report},
    pdfpagemode=FullScreen,
}

\newcommand{\yes}{\ding{51}} % Checkmark
\newcommand{\no}{\ding{55}}  % Cross

% ----------------------------------------------------------------------
% DOCUMENT START
% ----------------------------------------------------------------------
\begin{document}

\title{Cybersecurity Posture Assessment Report \\ \large For: \textbf{Harbor Light Foundation}}
\author{Cybersecurity Analysis Division}
\date{\today}
\maketitle

\tableofcontents
\newpage

% ----------------------------------------------------------------------
% SECTION 1: EXECUTIVE SUMMARY
% ----------------------------------------------------------------------
\section{Executive Summary}

This report provides a comprehensive cybersecurity assessment for \textbf{Harbor Light Foundation}, based on a synthesis of network scan data, organizational security controls, and pre-existing risk information. The analysis reveals several critical and high-risk security gaps that require immediate attention.

The most significant findings include the systemic lack of Multi-Factor Authentication (MFA) across all key areas (email, computer logins, and sensitive data systems), the direct exposure of Remote Desktop Protocol (RDP) on an internal server, and deficiencies in the security awareness training program for new employees.

The combination of exposed remote access services and the absence of MFA creates a high-probability attack path for threat actors. A single compromised credential could lead to unauthorized network access, data breach, or a ransomware event. We strongly recommend prioritizing the remediation actions outlined in Section \ref{sec:recommendations} to mitigate these risks and improve the organization's overall security posture.

% ----------------------------------------------------------------------
% SECTION 2: ORGANIZATIONAL INFORMATION
% ----------------------------------------------------------------------
\section{Organizational Information}

The following details were provided for the assessment.

\begin{tabular}{@{}ll}
\toprule
\textbf{Attribute} & \textbf{Value} \\
\midrule
Organization Name & \textbf{Harbor Light Foundation} \\
Email Domain & \texttt{HarborLightFoundation.com} \\
Website Domain & \texttt{www.HarborLightFoundation.com} \\
External IP Address & \texttt{85.84.229.157} \\
\bottomrule
\end{tabular}

% ----------------------------------------------------------------------
% SECTION 3: SECURITY CONTROL REVIEW
% ----------------------------------------------------------------------
\section{Security Control Review}

A review of the organization's security controls was conducted via a questionnaire. The results highlight significant gaps in access control and employee security training.

\begin{table}[h!]
\centering
\caption{Security Controls Questionnaire Analysis}
\begin{tabular}{@{}p{0.6\linewidth} c l@{}}
\toprule
\textbf{Control Question} & \textbf{Status} & \textbf{Assessment} \\
\midrule
Do you require MFA to access email? & \no & \textbf{Critical Gap} \\
Do you require MFA to log into computers? & \no & \textbf{Critical Gap} \\
Do you require MFA to access sensitive data systems? & \no & \textbf{Critical Gap} \\
Does your organization have an employee acceptable use policy? & \yes & Control in Place \\
Does your organization do security awareness training for new employees? & \no & \textbf{High Risk} \\
Does your organization do security awareness training for all employees at least once per year? & \yes & Control in Place \\
\bottomrule
\end{tabular}
\end{table}

\paragraph{Analysis:} The absence of MFA is the most critical finding. It removes a fundamental layer of defense against account compromise attacks. The lack of security training for new hires creates a window of vulnerability where new staff are more susceptible to phishing and social engineering attacks.

% ----------------------------------------------------------------------
% SECTION 4: TECHNICAL SCAN RESULTS
% ----------------------------------------------------------------------
\section{Technical Scan Results}

A network scan was performed to identify open ports and exposed services on the target system.

\begin{itemize}
    \item \textbf{Target IP Address:} \texttt{10.10.10.51}
    \item \textbf{Scan Date:} \textbf{[Scan Date Not Provided]}
\end{itemize}

\begin{table}[h!]
\centering
\caption{Open Port Findings for \texttt{10.10.10.51}}
\begin{tabular}{@{}l l l l@{}}
\toprule
\textbf{Port} & \textbf{State} & \textbf{Service Name} & \textbf{Analysis} \\
\midrule
3389/tcp & Open & \texttt{ms-wbt-server} & High Risk \\
\bottomrule
\end{tabular}
\end{table}

\paragraph{Analysis:} The scan identified that port \textbf{3389/tcp}, used for Microsoft's Remote Desktop Protocol (RDP), is open. RDP is a common vector for unauthorized access and ransomware attacks. This finding is especially critical when correlated with the lack of MFA for computer logins (Section 3) and a pre-existing risk of RDP exposure on another host (\texttt{10.10.10.50}), indicating a systemic configuration issue.

% ----------------------------------------------------------------------
% SECTION 5: CORRELATED RISK ASSESSMENT
% ----------------------------------------------------------------------
\section{Correlated Risk Assessment}

The following table summarizes the key risks identified by correlating the security control review, technical scan results, and pre-existing vulnerability data.

\begin{table}[h!]
\centering
\caption{Summary of Identified Risks}
\begin{tabular}{@{}p{0.15\linewidth} p{0.65\linewidth} l@{}}
\toprule
\textbf{Risk Name} & \textbf{Description} & \textbf{Severity} \\
\midrule
\textbf{Insecure Remote Access} & RDP is exposed on host \texttt{10.10.10.51}, a finding compounded by a known exposure on \texttt{10.10.10.50}. This service is a primary target for brute-force and ransomware attacks. & \textbf{Critical} \\
\addlinespace
\textbf{Lack of MFA} & Multi-Factor Authentication is not enforced for email, computer logins, or sensitive data access. This drastically increases the likelihood of a successful account takeover via compromised credentials. & \textbf{Critical} \\
\addlinespace
\textbf{Inadequate Security Onboarding} & New employees do not receive security awareness training, making them prime targets for phishing and social engineering, which are the primary sources of credential theft. & \textbf{High} \\
\bottomrule
\end{tabular}
\end{table}

% ----------------------------------------------------------------------
% SECTION 6: RECOMMENDATIONS
% ----------------------------------------------------------------------
\section{Recommendations}
\label{sec:recommendations}

The following actions are recommended to mitigate the identified risks. Recommendations are prioritized based on severity and impact.

\subsection{Remediate Insecure Remote Access (Critical)}
\begin{itemize}
    \item \textbf{Immediate Action:} If remote access to \texttt{10.10.10.51} is not business-critical, disable the RDP service and block port 3389 at the network firewall immediately.
    \item \textbf{Strategic Action:} If remote access is required, implement a Virtual Private Network (VPN) or Zero Trust Network Access (ZTNA) solution. All remote administrative access must be routed through this secure, authenticated, and encrypted tunnel. Public-facing RDP should be prohibited organization-wide.
\end{itemize}

\subsection{Implement Multi-Factor Authentication (Critical)}
\begin{itemize}
    \item \textbf{Immediate Action:} Prioritize the deployment of MFA for all user accounts on the email system (\texttt{HarborLightFoundation.com}) and for all administrative or privileged accounts.
    \item \textbf{Strategic Action:} Develop and execute a phased plan to enforce MFA for all computer logins and access to any systems storing sensitive organizational data.
\end{itemize}

\subsection{Strengthen Security Awareness Program (High)}
\begin{itemize}
    \item \textbf{Immediate Action:} Ensure all employees hired since the last annual training cycle are enrolled in and complete the security awareness training immediately.
    \item \textbf{Strategic Action:} Integrate mandatory security awareness training into the formal employee onboarding process. This ensures that all new staff understand their security responsibilities from their first day.
\end{itemize}

% ----------------------------------------------------------------------
% DOCUMENT END
% ----------------------------------------------------------------------
\end{document}
```