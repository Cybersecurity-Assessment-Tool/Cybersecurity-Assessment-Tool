```latex
\documentclass[12pt]{article}

% 1. Document Setup & Packages
\usepackage[margin=1in]{geometry}
\usepackage{pifont} % For checkmarks and crosses
\usepackage{booktabs} % For professional tables
\usepackage{hyperref} % For hyperlinks
\usepackage{url} % For URL formatting
\usepackage{seqsplit} % For splitting long strings
\usepackage{graphicx}
\usepackage{xcolor}
\usepackage{fancyhdr}

% Hyperref Setup
\hypersetup{
    colorlinks=true,
    linkcolor=black,
    urlcolor=blue,
    pdftitle={Cybersecurity Posture Assessment Report},
    pdfauthor={Cybersecurity Analyst},
}

% Header and Footer
\pagestyle{fancy}
\fancyhf{}
\fancyhead[L]{Cybersecurity Posture Assessment}
\fancyhead[R]{Nexus Dynamics}
\fancyfoot[C]{\thepage}
\renewcommand{\headrulewidth}{0.4pt}
\renewcommand{\footrulewidth}{0.4pt}

\begin{document}

% 2. Title Page
\begin{titlepage}
    \centering
    \vspace*{1cm}
    \includegraphics[width=0.4\textwidth]{example-image-a} % Placeholder logo
    \vfill
    \huge\textbf{Cybersecurity Posture Assessment Report}\\[0.5cm]
    \Large For\\[0.5cm]
    \huge\textbf{Nexus Dynamics}\\[2cm]
    \large
    \begin{tabular}{ll}
        \textbf{Date of Report:} & \today \\
        \textbf{Author:} & Lead Cybersecurity Analyst \\
        \textbf{Status:} & Final \\
    \end{tabular}
    \vfill
    \small
    \textit{This document contains sensitive information and is intended for the exclusive use of Nexus Dynamics personnel. Unauthorized distribution is strictly prohibited.}
\end{titlepage}

\tableofcontents
\newpage

% 3. Executive Summary
\section{Executive Summary}

This report provides a comprehensive assessment of the cybersecurity posture for \textbf{Nexus Dynamics}. The analysis is based on a synthesis of data from three sources: a network vulnerability scan, an organizational security controls questionnaire, and a review of pre-existing risk data.

\textbf{Important Note on Data Integrity:} During the analysis phase, it was discovered that the data feeds for the \textbf{Network Scan (Input 1)} and \textbf{Current Risks (Input 3)} were corrupted and could not be parsed. Consequently, this assessment is primarily based on the self-reported data from the organizational security controls questionnaire.

The review of the security controls questionnaire revealed a solid foundation in several key areas, including the enforcement of Multi-Factor Authentication (MFA) for email and sensitive systems, and the presence of an acceptable use policy. However, two significant gaps were identified that present a high level of risk to the organization:
\begin{itemize}
    \item \textbf{Lack of Endpoint MFA:} Employee computers do not require MFA for login, exposing the organization to significant risk from compromised credentials.
    \item \textbf{Inadequate Security Training Cadence:} Security awareness training is not conducted annually for all employees, increasing susceptibility to phishing and social engineering attacks.
\end{itemize}

This report details these findings and provides actionable, prioritized recommendations to mitigate the identified risks and strengthen the overall security posture of \textbf{Nexus Dynamics}. A full technical assessment is recommended once the data integrity issues with the network scanner have been resolved.

\newpage

% 4. Organizational Information
\section{Organizational Information}
The following details were provided by the client and used as the basis for this assessment.

\begin{tabular}{@{}ll}
\toprule
\textbf{Attribute} & \textbf{Value} \\
\midrule
Organization Name & \textbf{Nexus Dynamics} \\
Email Domain & \texttt{NexusDynamics.com} \\
Website Domain & \url{www.NexusDynamics.com} \\
External IP Address & \seqsplit{\texttt{145.84.128.129}} \\
\bottomrule
\end{tabular}

% 5. Security Control Review
\section{Security Control Review (Questionnaire Analysis)}
The following table summarizes the organization's responses to the security controls questionnaire. Items marked with a red cross (\ding{55}) indicate a deviation from security best practices and represent a potential risk.

\begin{center}
\begin{tabular}{p{0.7\textwidth}c}
\toprule
\textbf{Control Question} & \textbf{Response} \\
\midrule
Do you require MFA to access email? & \textcolor{green}{\ding{51}} \\
Do you require MFA to log into computers? & \textcolor{red}{\ding{55}} \\
Do you require MFA to access sensitive data systems? & \textcolor{green}{\ding{51}} \\
Does your organization have an employee acceptable use policy? & \textcolor{green}{\ding{51}} \\
Does your organization do security awareness training for new employees? & \textcolor{green}{\ding{51}} \\
Does your organization do security awareness training for all employees at least once per year? & \textcolor{red}{\ding{55}} \\
\bottomrule
\end{tabular}
\end{center}

\subsection{Analysis of Gaps}
The questionnaire identified two primary control gaps:
\begin{enumerate}
    \item \textbf{Endpoint MFA:} The absence of MFA on computer logins is a critical weakness. If an employee's password is stolen or guessed, an attacker can gain direct access to their workstation and, subsequently, the internal network.
    \item \textbf{Annual Security Training:} While new hires receive training, the lack of an annual refresher for all staff means that knowledge of evolving threats diminishes over time. This makes employees more likely to fall victim to sophisticated phishing or social engineering campaigns.
\end{enumerate}

% 6. Technical Scan Results
\section{Technical Scan Results}
An attempt was made to perform a technical network scan against the organization's external infrastructure.

\begin{itemize}
    \item \textbf{Target IP Address:} \texttt{[Target IP]}
    \item \textbf{Scan Date:} Not Available
    \item \textbf{Scan Status:} \textbf{\textcolor{red}{FAILED}}
\end{itemize}

\textbf{Finding:} The raw data output from the network scanning tool was found to be corrupted and unreadable. As a result, no analysis of open ports, running services, or potential software vulnerabilities could be performed. It is critical to resolve the underlying issue with the scanning infrastructure to enable future technical assessments.

% 7. Risk Assessment
\section{Risk Assessment}
This risk assessment is based on the findings from the Security Control Review. The pre-existing risk data was unavailable for correlation due to data corruption. The identified risks are prioritized by severity.

\begin{center}
\begin{tabular}{p{0.1\textwidth} p{0.25\textwidth} p{0.4\textwidth} p{0.1\textwidth}}
\toprule
\textbf{Risk ID} & \textbf{Risk Name} & \textbf{Description} & \textbf{Severity} \\
\midrule
\textbf{R-001} & Lack of Endpoint Multi-Factor Authentication (MFA) & Employee computers do not require MFA for login. This significantly increases the risk of unauthorized access from compromised credentials, potentially leading to lateral movement and data breaches. & \textbf{Critical} \\
\addlinespace
\textbf{R-002} & Inadequate Security Awareness Training Program & Security awareness training is not conducted annually for all employees. This creates a gap in knowledge regarding evolving threats, making the organization more susceptible to human-error-based attacks like phishing. & \textbf{High} \\
\bottomrule
\end{tabular}
\end{center}

% 8. Recommendations
\section{Recommendations}
The following prioritized recommendations are provided to address the identified risks and improve the overall security posture of \textbf{Nexus Dynamics}.

\subsection{Critical Priority}
\begin{description}
    \item[REC-001: Implement Mandatory MFA for All Endpoints]
    \subitem \textbf{Action:} Deploy a robust Multi-Factor Authentication solution for all employee computer and server logins. Solutions such as Windows Hello for Business, Duo Security, or Okta should be evaluated and implemented.
    \subitem \textbf{Justification:} This is the single most effective control to mitigate the risk of unauthorized access resulting from stolen credentials (Risk ID R-001). It is a foundational element of a modern, zero-trust security architecture.
\end{description}

\subsection{High Priority}
\begin{description}
    \item[REC-002: Establish an Annual Security Awareness Training Program]
    \subitem \textbf{Action:} Institute a mandatory, annual security awareness training program for all staff, including management. This program should be updated yearly to cover the latest threat vectors, with a strong emphasis on phishing, ransomware, and social engineering.
    \subitem \textbf{Justification:} A well-trained workforce is a critical layer of defense. Regular training reduces the organization's "human attack surface" and mitigates the risk of security incidents caused by employee error (Risk ID R-002).
\end{description}

\subsection{Informational}
\begin{description}
    \item[REC-003: Remediate Data Source Integrity]
    \subitem \textbf{Action:} Investigate and resolve the root cause of the data corruption affecting the network scanner output and the current risks database.
    \subitem \textbf{Justification:} To conduct a complete and accurate security assessment in the future, all data sources must be reliable. A comprehensive view of technical vulnerabilities is essential for effective risk management.
\end{description}

\end{document}
```