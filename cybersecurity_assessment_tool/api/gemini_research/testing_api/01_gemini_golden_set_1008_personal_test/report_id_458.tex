```latex
\documentclass[12pt]{article}

% Preamble: Required Packages
\usepackage[margin=1in]{geometry}
\usepackage{pifont} % For checkmarks (\ding{51}) and crosses (\ding{55})
\usepackage{booktabs} % For professional-looking tables
\usepackage{hyperref} % For creating hyperlinks
\usepackage{url}      % For formatting URLs
\usepackage{seqsplit} % For breaking long strings without spaces
\usepackage{xcolor}   % For custom colors

% Hyperlink Setup
\hypersetup{
    colorlinks=true,
    linkcolor=blue,
    filecolor=magenta,
    urlcolor=cyan,
}

% Custom Commands for Readability
\newcommand{\yes}{\textcolor{green}{\ding{51}}}
\newcommand{\no}{\textcolor{red}{\ding{55}}}
\newcommand{\orgname}{\textbf{North Star Education}}
\newcommand{\orgdomain}{\texttt{NorthStarEducation.net}}
\newcommand{\orgip}{\texttt{150.87.54.70}}
\newcommand{\targetip}{\texttt{10.0.0.15}}

\title{Cybersecurity Posture Assessment Report}
\author{Cybersecurity Analysis Division}
\date{\today}

\begin{document}

\maketitle
\thispagestyle{empty}
\newpage

\tableofcontents
\newpage

\section*{1. Executive Summary}

This report details the findings of a cybersecurity assessment conducted for \orgname. The analysis combined a network vulnerability scan, a review of organizational security controls, and an evaluation of pre-existing risks.

The assessment identified several \textbf{critical and high-severity risks} that require immediate attention. A key finding is an externally exposed FTP server running a dangerously outdated and vulnerable version of \texttt{vsftpd} (2.3.4), which is known to have a backdoor. This service also permits anonymous, unauthenticated access, posing a direct and immediate threat of system compromise and data breach.

Furthermore, significant gaps were identified in the organization's access control policies. The lack of Multi-Factor Authentication (MFA) for email and workstation access drastically increases the risk of unauthorized access through credential theft. These issues, compounded by the pre-existing risk of outdated Windows 7 workstations, create a fragile security posture that could be easily exploited by malicious actors.

Immediate remediation of the vulnerable FTP server and the phased implementation of MFA are the highest priorities.

\section*{2. Organizational Information}

The following information was provided for the assessment.

\begin{itemize}
    \item \textbf{Organization Name:} \orgname
    \item \textbf{Email Domain:} \orgdomain
    \item \textbf{Website Domain:} \texttt{www.NorthStarEducation.net}
    \item \textbf{External IP Address:} \orgip
\end{itemize}

\section*{3. Security Control Review}

A review of the organization's security controls was conducted via a questionnaire. The results highlight critical gaps in identity and access management. While security awareness training is in place, the lack of MFA on core systems undermines its effectiveness.

\begin{table}[h!]
\centering
\caption{Security Controls Questionnaire Results}
\begin{tabular}{p{0.75\linewidth} c}
\toprule
\textbf{Control Question} & \textbf{Status} \\
\midrule
Does your organization have an employee acceptable use policy? & \yes \\
Does your organization do security awareness training for new employees? & \yes \\
Does your organization do security awareness training for all employees at least once per year? & \yes \\
Do you require MFA to access sensitive data systems? & \yes \\
\midrule
\textit{Identified Gaps} & \\
\midrule
Do you require MFA to access email? & \no \\
Do you require MFA to log into computers? & \no \\
\bottomrule
\end{tabular}
\end{table}

\section*{4. Technical Scan Results}

An Nmap scan was performed on the target host \targetip. The scan revealed an open FTP port with a vulnerable service version.

\begin{table}[h!]
\centering
\caption{Open Port Scan Details for \targetip}
\begin{tabular}{c c l l}
\toprule
\textbf{Port} & \textbf{State} & \textbf{Service/Version} & \textbf{Notes} \\
\midrule
21/tcp & Open & FTP / vsftpd 2.3.4 & \textbf{Critical Vulnerability.} Anonymous login is allowed. \\
\bottomrule
\end{tabular}
\end{table}

\subsection*{Analysis of Findings}
\begin{itemize}
    \item \textbf{Vulnerable FTP Service (vsftpd 2.3.4):} The version of \texttt{vsftpd} detected, 2.3.4, is widely known to be vulnerable to a critical backdoor (CVE-2011-2523). An attacker can gain a command shell on the underlying server by sending a specific sequence of characters as the username. This provides a direct path to system compromise.
    \item \textbf{Anonymous FTP Login:} The scan confirmed that anonymous FTP login is permitted. This allows any unauthenticated user on the internet to connect to the server, and potentially upload, download, or modify files, depending on the configuration. This is a severe security misconfiguration that can lead to data leakage or the hosting of malicious content.
\end{itemize}

\section*{5. Consolidated Risk Assessment}

The following table synthesizes findings from the security control review, the technical scan, and pre-existing risk data into a prioritized list.

\begin{table}[h!]
\centering
\caption{Summary of Identified Risks}
\begin{tabular}{p{0.3\linewidth} p{0.5\linewidth} l}
\toprule
\textbf{Risk Name} & \textbf{Overview} & \textbf{Severity} \\
\midrule
\textbf{Vulnerable FTP Service} & The server at \targetip is running vsftpd 2.3.4, which contains a known remote code execution backdoor. & \textbf{Critical} \\
\addlinespace
\textbf{Anonymous FTP Access} & The FTP service allows unauthenticated anonymous logins, exposing server files to the public internet. & \textbf{Critical} \\
\addlinespace
\textbf{No MFA for Email Access} & Lack of MFA on email accounts makes them highly susceptible to phishing and credential stuffing attacks. & \textbf{High} \\
\addlinespace
\textbf{No MFA for Workstations} & Lack of MFA on computer logins allows an attacker with stolen credentials to easily gain network access. & \textbf{High} \\
\addlinespace
\textbf{Outdated Windows Policy} & Pre-existing risk: Workstations are running Windows 7, which is end-of-life and no longer receives security updates. & \textbf{Medium} \\
\bottomrule
\end{tabular}
\end{table}

\section*{6. Recommendations}

Based on the consolidated risk assessment, the following remediation actions are recommended in order of priority.

\subsection*{Priority 1: Immediate Actions (Within 24 Hours)}
\begin{enumerate}
    \item \textbf{Remediate Vulnerable FTP Server:}
        \begin{itemize}
            \item \textbf{Option A (Recommended):} If the FTP service is not essential for business operations, \textbf{disable and block port 21 immediately}.
            \item \textbf{Option B:} If the service is essential, immediately upgrade \texttt{vsftpd} to the latest stable version and \textbf{disable anonymous login}. The long-term goal should be to migrate to a secure file transfer protocol like SFTP (SSH File Transfer Protocol) or FTPS (FTP over SSL/TLS).
        \end{itemize}
\end{enumerate}

\subsection*{Priority 2: Short-Term Actions (1-4 Weeks)}
\begin{enumerate}
    \setcounter{enumi}{1}
    \item \textbf{Implement Multi-Factor Authentication (MFA):}
        \begin{itemize}
            \item Begin a phased rollout of MFA for all users on all systems.
            \item \textbf{Phase 1 (Week 1):} Enforce MFA for all administrator and executive email accounts and workstation logins.
            \item \textbf{Phase 2 (Weeks 2-4):} Enforce MFA for all remaining users for both email and workstation access.
        \end{itemize}
\end{enumerate}

\subsection*{Priority 3: Mid-Term Actions (1-6 Months)}
\begin{enumerate}
    \setcounter{enumi}{2}
    \item \textbf{Accelerate Operating System Upgrades:}
        \begin{itemize}
            \item Continue with the existing plan to upgrade all Windows 7 workstations to a modern, supported operating system (e.g., Windows 10/11).
            \item Prioritize upgrading systems used by employees with access to sensitive data.
        \end{itemize}
\end{enumerate}

\end{document}
```