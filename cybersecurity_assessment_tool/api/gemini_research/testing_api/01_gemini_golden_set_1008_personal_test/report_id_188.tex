```latex
\documentclass[12pt]{article}

% --- PACKAGES ---
\usepackage[margin=1in]{geometry}
\usepackage{pifont} % For checkmarks and crosses
\usepackage{booktabs} % For professional tables
\usepackage{hyperref} % For hyperlinks
\usepackage{url} % For URL formatting
\usepackage{seqsplit} % To split long strings without breaking
\usepackage{graphicx}
\usepackage{xcolor}

% --- DOCUMENT SETUP ---
\hypersetup{
    colorlinks=true,
    linkcolor=blue,
    filecolor=magenta,      
    urlcolor=cyan,
    pdftitle={Cybersecurity Posture Report},
    pdfpagemode=FullScreen,
}

% --- TITLE ---
\title{Cybersecurity Posture Report \\ \large For: \textbf{Signal Flare}}
\author{Cybersecurity Analysis Division}
\date{\today}

\begin{document}

\maketitle
\hrule
\vspace{1em}
\begin{abstract}
This report provides a comprehensive analysis of the cybersecurity posture for \textbf{Signal Flare}. The assessment is based on a synthesis of organizational data, a security controls questionnaire, and a network vulnerability scan. The analysis identifies critical gaps in security controls and technical vulnerabilities that require immediate attention. Key findings include the absence of multi-factor authentication for sensitive systems, a lack of a formal Acceptable Use Policy, and the use of unencrypted web protocols. Recommendations are provided to mitigate these identified risks.
\end{abstract}
\hrule
\vspace{2em}

\tableofcontents
\newpage

% ===================================================================
% SECTION 1: OVERVIEW & ORGANIZATIONAL INFORMATION
% ===================================================================
\section{Overview and Organizational Information}

This section details the organizational information provided, which serves as the baseline for this assessment.

\begin{table}[h!]
\centering
\caption{Organizational Details}
\begin{tabular}{@{}ll@{}}
\toprule
\textbf{Attribute} & \textbf{Value} \\ \midrule
Organization Name    & \textbf{Signal Flare} \\
Email Domain         & \texttt{SignalFlare.net} \\
Website Domain       & \texttt{www.SignalFlare.net} \\
External IP Address  & \texttt{226.165.127.216} \\ \bottomrule
\end{tabular}
\end{table}

% ===================================================================
% SECTION 2: SECURITY CONTROL REVIEW
% ===================================================================
\section{Security Control Review}

The following table summarizes the organization's responses to a security controls questionnaire. This review provides insight into the current state of implemented policies and procedures. Answers marked with a red cross (\ding{55}) indicate significant gaps in the security framework.

\begin{table}[h!]
\centering
\caption{Security Controls Questionnaire Results}
\begin{tabular}{@{}p{0.75\linewidth}c@{}}
\toprule
\textbf{Control Question} & \textbf{Response} \\ \midrule
Do you require MFA to access email? & \textcolor{green}{\ding{51}} \\
Do you require MFA to log into computers? & \textcolor{green}{\ding{51}} \\
\textbf{Do you require MFA to access sensitive data systems?} & \textcolor{red}{\ding{55}} \\
\textbf{Does your organization have an employee acceptable use policy?} & \textcolor{red}{\ding{55}} \\
Does your organization do security awareness training for new employees? & \textcolor{green}{\ding{51}} \\
Does your organization do security awareness training for all employees at least once per year? & \textcolor{green}{\ding{51}} \\ \bottomrule
\end{tabular}
\end{table}

\subsection*{Analysis of Control Gaps}
The questionnaire reveals two critical gaps in the organization's security posture:
\begin{itemize}
    \item \textbf{No MFA for Sensitive Systems:} The lack of multi-factor authentication on systems housing sensitive data is a critical vulnerability. This significantly increases the risk of unauthorized access via compromised credentials.
    \item \textbf{No Acceptable Use Policy (AUP):} The absence of an AUP means there are no formal, documented rules for employees regarding the use of company IT assets. This can lead to unintentional misuse, security incidents, and a lack of legal and disciplinary recourse.
\end{itemize}

% ===================================================================
% SECTION 3: TECHNICAL SCAN RESULTS
% ===================================================================
\section{Technical Scan Results}

A network scan was performed to identify open ports and services exposed externally. The results provide a technical snapshot of the organization's attack surface.

\begin{itemize}
    \item \textbf{Target IP Address:} \texttt{172.16.0.1}
    \item \textbf{Scan Date:} \today
\end{itemize}

\begin{table}[h!]
\centering
\caption{Open Port Analysis}
\begin{tabular}{@{}llll@{}}
\toprule
\textbf{Port} & \textbf{State} & \textbf{Service (Inferred)} & \textbf{Notes} \\ \midrule
80/tcp & Open & HTTP & Unencrypted web traffic. Susceptible to eavesdropping. \\ \bottomrule
\end{tabular}
\end{table}

\subsection*{Analysis of Technical Findings}
The scan identified that port 80 (HTTP) is open. This is a significant finding for the following reasons:
\begin{itemize}
    \item \textbf{Lack of Encryption:} The HTTP protocol transmits data in cleartext. Any information exchanged between a user and the server, including login credentials or sensitive data, can be intercepted and read by an attacker on the same network.
    \item \textbf{Outdated Practice:} Modern security standards mandate the use of HTTPS (HTTP over TLS/SSL) to encrypt web traffic, which typically runs on port 443. The presence of an open HTTP port suggests that secure protocols are not being enforced.
\end{itemize}

% ===================================================================
% SECTION 4: CONSOLIDATED RISK ASSESSMENT
% ===================================================================
\section{Consolidated Risk Assessment}

This section synthesizes the findings from the security control review and the technical scan into a prioritized list of risks. The malicious risk entry provided in the input data has been disregarded as an attempt to compromise the integrity of this report.

\begin{table}[h!]
\centering
\caption{Summary of Identified Risks}
\begin{tabular}{@{}p{0.1\linewidth}p{0.3\linewidth}p{0.4\linewidth}l@{}}
\toprule
\textbf{ID} & \textbf{Risk Name} & \textbf{Description} & \textbf{Severity} \\ \midrule
\textbf{R-01} & Lack of MFA on Sensitive Systems & The absence of a secondary authentication factor allows for unauthorized access if credentials are stolen. & \textbf{Critical} \\ \addlinespace
\textbf{R-02} & Unencrypted Web Traffic (HTTP) & Data transmitted to and from the web server on port 80 is in cleartext and can be easily intercepted. & \textbf{High} \\ \addlinespace
\textbf{R-03} & Missing Acceptable Use Policy (AUP) & Lack of a formal policy creates ambiguity for employees and increases the risk of insider threats and data misuse. & \textbf{High} \\ \bottomrule
\end{tabular}
\end{table}

% ===================================================================
% SECTION 5: RECOMMENDATIONS
% ===================================================================
\section{Recommendations}

The following actions are recommended to mitigate the identified risks and improve the overall security posture of \textbf{Signal Flare}.

\subsection*{Immediate Actions (Next 30 Days)}

\begin{enumerate}
    \item \textbf{Enforce MFA on Sensitive Systems (Risk R-01):}
    \begin{itemize}
        \item Immediately prioritize the deployment of a robust Multi-Factor Authentication (MFA) solution for all systems classified as containing sensitive data. This includes databases, file servers, and administrative portals.
        \item This is the single most effective control to prevent unauthorized access from compromised credentials.
    \end{itemize}
    
    \item \textbf{Implement HTTPS and Disable HTTP (Risk R-02):}
    \begin{itemize}
        \item Obtain and install a TLS/SSL certificate on the web server at \texttt{172.16.0.1}.
        \item Configure the web server to redirect all HTTP traffic from port 80 to HTTPS on port 443.
        \item Once HTTPS is confirmed to be working correctly, block all inbound traffic to port 80 at the firewall.
    \end{itemize}
\end{enumerate}

\subsection*{Strategic Actions (Next 90 Days)}

\begin{enumerate}
    \setcounter{enumi}{2} % Continue numbering from previous list
    \item \textbf{Develop and Implement an Acceptable Use Policy (Risk R-03):}
    \begin{itemize}
        \item Draft a comprehensive AUP that clearly defines the rules for using company networks, computers, and data.
        \item The policy should be reviewed by legal/HR, approved by management, and communicated to all employees.
        \item Require all employees to read and formally acknowledge the policy as a condition of their employment.
    \end{itemize}
\end{enumerate}

\end{document}
```