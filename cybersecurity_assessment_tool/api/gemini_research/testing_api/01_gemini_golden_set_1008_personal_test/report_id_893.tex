```latex
\documentclass[12pt]{article}

% ----------------------------------------------------------------------
% PREAMBLE
% ----------------------------------------------------------------------
\usepackage[margin=1in]{geometry}
\usepackage{pifont} % For checkmarks and crosses
\usepackage{booktabs} % For professional tables
\usepackage{hyperref} % For clickable links and metadata
\usepackage{url}      % For URL formatting
\usepackage{seqsplit} % For splitting long strings to prevent overflow
\usepackage{graphicx} % For potential logos
\usepackage{xcolor}   % For custom colors

% Define colors for severity
\definecolor{criticalred}{HTML}{D12727}
\definecolor{highorange}{HTML}{E48824}

% Hyperref setup
\hypersetup{
    colorlinks=true,
    linkcolor=blue,
    filecolor=magenta,      
    urlcolor=cyan,
    pdftitle={Cybersecurity Posture Assessment Report},
    pdfauthor={Cybersecurity Analyst},
    pdfsubject={Security Analysis},
    pdfkeywords={Cybersecurity, Risk, Assessment},
    bookmarks=true
}

% Checkmark and Cross definitions
\newcommand{\cmark}{\ding{51}}%
\newcommand{\xmark}{\ding{55}}%

% ----------------------------------------------------------------------
% DOCUMENT START
% ----------------------------------------------------------------------
\begin{document}

% --- TITLE PAGE ---
\begin{titlepage}
    \centering
    \vspace*{1cm}
    \Huge{\textbf{Cybersecurity Posture Assessment Report}}
    \vspace{1.5cm}
    \Large{\textbf{Prepared for:}} \\
    \vspace{0.5cm}
    \huge{Arcane Security}
    \vspace{2cm}
    \large{\textbf{Date of Report:}} \\
    \vspace{0.2cm}
    \today
    \vfill
    \large{\textit{This report contains sensitive information and should be handled with care. Distribution is restricted to authorized personnel only.}}
\end{titlepage}

\tableofcontents
\newpage

% ----------------------------------------------------------------------
% 1. EXECUTIVE OVERVIEW
% ----------------------------------------------------------------------
\section{Executive Overview}

This report provides a comprehensive assessment of the cybersecurity posture for \textbf{Arcane Security}. The analysis is based on a correlation of technical network scan data, a review of existing risks, and an evaluation of organizational security controls provided via a questionnaire.

The assessment identified several critical and high-risk findings that require immediate attention. A key technical vulnerability, the exposure of Remote Desktop Protocol (RDP) services, was discovered on a new host (\texttt{10.10.10.51}), indicating a potential systemic issue as this risk was previously identified on another system (\texttt{10.10.10.50}).

Furthermore, significant gaps in administrative controls were noted. The lack of mandatory Multi-Factor Authentication (MFA) for email access represents a critical vulnerability to phishing and account takeover attacks. This is compounded by the absence of a formal employee acceptable use policy and a mandatory security awareness training program for new hires, leaving the organization susceptible to insider threats, whether malicious or unintentional.

The combination of easily exploitable technical vulnerabilities and weaknesses in foundational security policies creates a high-risk environment. This report provides prioritized, actionable recommendations to mitigate these risks and strengthen the overall security posture.

% ----------------------------------------------------------------------
% 2. ORGANIZATIONAL INFORMATION
% ----------------------------------------------------------------------
\section{Organizational Information}

The following details were provided for the assessment.

\begin{table}[h!]
\centering
\begin{tabular}{@{}ll@{}}
\toprule
\textbf{Attribute} & \textbf{Value} \\ \midrule
Organization Name & Arcane Security \\
Email Domain & \texttt{ArcaneSecurity.org} \\
Website Domain & \url{www.ArcaneSecurity.org} \\
External IP Address & \texttt{172.159.184.15} \\ \bottomrule
\end{tabular}
\caption{Client Organizational Data.}
\label{tab:org_data}
\end{table}

% ----------------------------------------------------------------------
% 3. SECURITY CONTROL REVIEW (QUESTIONNAIRE)
% ----------------------------------------------------------------------
\section{Security Control Review}

An analysis of the security questionnaire reveals critical gaps in administrative and access controls. "No" answers indicate a deviation from security best practices and are flagged as significant risks.

\begin{table}[h!]
\centering
\begin{tabular}{@{}lc@{}}
\toprule
\textbf{Control Question} & \textbf{Status} \\ \midrule
Do you require MFA to access email? & \xmark \\
Do you require MFA to log into computers? & \cmark \\
Do you require MFA to access sensitive data systems? & \cmark \\
Does your organization have an employee acceptable use policy? & \xmark \\
Does your organization do security awareness training for new employees? & \xmark \\
Does your organization do security awareness training for all employees at least once per year? & \cmark \\ \bottomrule
\end{tabular}
\caption{Security Controls Questionnaire Analysis.}
\label{tab:controls}
\end{table}

\paragraph{Analysis:} The lack of MFA on email is a critical weakness, as email is a primary vector for initial compromise. The absence of an acceptable use policy and security training for new hires creates an environment where employees may be unaware of security expectations, increasing the likelihood of security incidents.

% ----------------------------------------------------------------------
% 4. TECHNICAL SCAN RESULTS
% ----------------------------------------------------------------------
\section{Technical Scan Results}

A network scan was performed on the specified target to identify open ports and exposed services.

\paragraph{Target Host:} \texttt{10.10.10.51}

\begin{table}[h!]
\centering
\begin{tabular}{@{}llll@{}}
\toprule
\textbf{Port} & \textbf{State} & \textbf{Service Name} & \textbf{Notes} \\ \midrule
3389/tcp & open & \texttt{ms-wbt-server} & Microsoft Remote Desktop Protocol (RDP) \\ \bottomrule
\end{tabular}
\caption{Open Ports and Services Detected on \texttt{10.10.10.51}.}
\label{tab:scan_results}
\end{table}

\paragraph{Analysis:} The scan confirmed that port 3389 (RDP) is open. RDP is a high-value target for attackers who use it for initial access, lateral movement, and ransomware deployment. Exposing this service directly to a network without proper controls (e.g., VPN, Network Level Authentication, MFA) is a critical security risk.

% ----------------------------------------------------------------------
% 5. CONSOLIDATED RISK ASSESSMENT
% ----------------------------------------------------------------------
\section{Consolidated Risk Assessment}

The following table synthesizes findings from the questionnaire, the network scan, and pre-existing risk data into a prioritized list.

\begin{table}[h!]
\centering
\resizebox{\textwidth}{!}{%
\begin{tabular}{@{}llll@{}}
\toprule
\textbf{Risk Name} & \textbf{Description} & \textbf{Severity} & \textbf{Source} \\ \midrule
\textbf{Systemic RDP Exposure} & RDP is exposed on \texttt{10.10.10.51} (new) and \texttt{10.10.10.50} (existing). & \color{criticalred}\textbf{Critical} & Network Scan \& Existing Risks \\
\textbf{No MFA for Email} & Email accounts lack MFA, making them vulnerable to takeover. & \color{criticalred}\textbf{Critical} & Questionnaire \\
\textbf{Missing Policies \& Training} & No acceptable use policy or new hire security training. & \color{highorange}\textbf{High} & Questionnaire \\
\bottomrule
\end{tabular}%
}
\caption{Summary of Identified Risks.}
\label{tab:risk_summary}
\end{table}

% ----------------------------------------------------------------------
% 6. RECOMMENDATIONS
% ----------------------------------------------------------------------
\section{Recommendations}

The following actions are recommended to mitigate the identified risks and improve the organization's security posture.

\subsection{Immediate Actions (Critical Priority)}
\begin{enumerate}
    \item \textbf{Remediate RDP Exposure:} Immediately close port 3389 on both \texttt{10.10.10.50} and \texttt{10.10.10.51} to all untrusted networks. If remote access is required, it must be placed behind a secure gateway, such as a Virtual Private Network (VPN) or a Remote Desktop Gateway, with MFA enforced.
    \item \textbf{Enforce MFA on All Email Accounts:} Immediately enable and enforce MFA for all user accounts on the \texttt{ArcaneSecurity.org} email domain. This is the single most effective control to prevent unauthorized access to email.
\end{enumerate}

\subsection{High Priority Actions}
\begin{enumerate}
    \item \textbf{Develop and Implement an Acceptable Use Policy (AUP):} Create a formal AUP that clearly defines the rules and expectations for employees when using company assets, data, and networks. All employees must read and acknowledge this policy.
    \item \textbf{Establish a New Hire Security Training Program:} Institute a mandatory security awareness training module as part of the new employee onboarding process. This ensures that security principles are established from day one.
\end{enumerate}

\subsection{Strategic Recommendations}
\begin{enumerate}
    \item \textbf{Conduct a Comprehensive Internal Vulnerability Scan:} The discovery of a second system with exposed RDP suggests this may be a wider issue. A full internal scan should be conducted to identify and remediate other misconfigurations or vulnerabilities.
    \item \textbf{Review Network Segmentation:} Evaluate the current network architecture to ensure that critical systems are properly segmented and isolated, limiting the potential impact of a breach.
\end{enumerate}

% ----------------------------------------------------------------------
% DOCUMENT END
% ----------------------------------------------------------------------
\end{document}
```