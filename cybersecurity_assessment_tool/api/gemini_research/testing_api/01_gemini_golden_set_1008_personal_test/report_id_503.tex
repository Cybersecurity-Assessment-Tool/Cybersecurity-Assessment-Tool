```latex
\documentclass[12pt]{article}

% Preamble: Required Packages and Document Setup
\usepackage[margin=1in]{geometry}
\usepackage{pifont} % For checkmarks and crosses
\usepackage{booktabs} % For professional tables
\usepackage{hyperref} % For hyperlinks
\usepackage{url} % For URL formatting
\usepackage{seqsplit} % To split long strings in tt font
\usepackage{graphicx}
\usepackage{xcolor}
\usepackage{tocloft}

% --- Document Metadata and Hyperlink Setup ---
\hypersetup{
    colorlinks=true,
    linkcolor=blue,
    filecolor=magenta,      
    urlcolor=cyan,
    pdftitle={Cybersecurity Posture Assessment Report},
    pdfauthor={Cybersecurity Analysis Division},
    pdfsubject={Security Assessment},
    pdfkeywords={Cybersecurity, Risk, Assessment},
    bookmarks=true
}

% --- Custom Commands for Readability ---
\newcommand{\yes}{\ding{51}}
\newcommand{\no}{\ding{55}}
\definecolor{severitycritical}{HTML}{940000}
\definecolor{severityhigh}{HTML}{D14124}
\newcommand{\sev_critical}[1]{\textcolor{severitycritical}{\textbf{#1}}}
\newcommand{\sev_high}[1]{\textcolor{severityhigh}{\textbf{#1}}}

% --- Document Start ---
\begin{document}

% --- Title Page ---
\title{
    \vspace{2cm}
    \textbf{Cybersecurity Posture Assessment Report} \\
    \large \textit{For: Maple Leaf Logistics} \\
    \vspace{1.5cm}
}
\author{Cybersecurity Analysis Division}
\date{\today}
\maketitle
\thispagestyle{empty}
\newpage

% --- Table of Contents ---
\tableofcontents
\newpage

% --- Section 1: Executive Summary ---
\section{Executive Summary}
This report provides a comprehensive cybersecurity posture assessment for \textbf{Maple Leaf Logistics}, based on the analysis and correlation of organizational data, network scan results, and pre-existing risk information. The assessment was conducted to identify security gaps, technical vulnerabilities, and areas for improvement in the organization's overall security framework.

The analysis reveals a mixed security posture. \textbf{Maple Leaf Logistics} has implemented strong foundational controls, particularly in the mandatory use of Multi-Factor Authentication (MFA) across key systems. However, two significant risks were identified that require immediate attention:

\begin{enumerate}
    \item \textbf{Critical Technical Vulnerability:} A network scan validated a pre-existing critical risk, "Localhost Exposed," on asset \texttt{127.0.0.1}. An open SSH port (22) on this internal interface represents a severe misconfiguration that could be leveraged by an attacker for privilege escalation or unauthorized system access.
    
    \item \textbf{High-Impact Policy Gap:} The organization does not conduct mandatory annual security awareness training for all employees. This gap significantly increases the organization's susceptibility to social engineering attacks, such as phishing, which remain a primary vector for security breaches.
\end{enumerate}

This report details these findings and provides actionable recommendations to mitigate the identified risks and enhance the organization's resilience against cyber threats.

% --- Section 2: Organizational Information ---
\section{Organizational Information}
The following details were provided for the assessment.

\begin{tabular}{@{}ll}
    \toprule
    \textbf{Attribute} & \textbf{Value} \\
    \midrule
    Organization Name & \textbf{Maple Leaf Logistics} \\
    Email Domain & \texttt{MapleLeafLogistics.org} \\
    Website Domain & \seqsplit{\url{www.MapleLeafLogistics.org}} \\
    External IP Address & \texttt{133.57.41.15} \\
    \bottomrule
\end{tabular}

% --- Section 3: Security Control Review ---
\section{Security Control Review}
A review of the organization's security controls was conducted based on a standardized questionnaire. The responses indicate a strong commitment to identity and access management but highlight a critical deficiency in ongoing employee security education.

\begin{table}[h!]
\centering
\caption{Security Controls Questionnaire Results}
\begin{tabular}{@{}lc@{}}
    \toprule
    \textbf{Control Question} & \textbf{Response} \\
    \midrule
    Do you require MFA to access email? & \yes \\
    Do you require MFA to log into computers? & \yes \\
    Do you require MFA to access sensitive data systems? & \yes \\
    Does your organization have an employee acceptable use policy? & \yes \\
    Does your organization do security awareness training for new employees? & \yes \\
    \textbf{Does your organization do security awareness training for all employees at least once per year?} & \no \\
    \bottomrule
\end{tabular}
\end{table}

\subsection*{Analysis}
The "No" response to annual security awareness training is a significant finding. While training new hires is a crucial first step, the threat landscape evolves continuously. Without regular, recurring training, employees are more likely to fall victim to new phishing techniques and other forms of social engineering, potentially compromising the very systems protected by MFA.

% --- Section 4: Technical Scan Results ---
\section{Technical Scan Results}
An external network scan was performed to identify open ports and exposed services on the target system.

\begin{itemize}
    \item \textbf{Target IP Address:} \texttt{127.0.0.1}
    \item \textbf{Scan Status:} Host is up.
\end{itemize}

\begin{table}[h!]
\centering
\caption{Open Ports Identified on \texttt{127.0.0.1}}
\begin{tabular}{@{}llll@{}}
    \toprule
    \textbf{Port} & \textbf{State} & \textbf{Service (Inferred)} & \textbf{Notes} \\
    \midrule
    22 & Open & SSH & Secure Shell access is enabled. \\
    \bottomrule
\end{tabular}
\end{table}

\subsection*{Analysis}
The scan identified that port 22 (SSH) is open. While SSH is a standard and secure protocol for remote administration, its presence on the localhost interface (\texttt{127.0.0.1}) is highly unusual and often indicative of a misconfiguration. This finding directly validates the pre-existing risk documented in the following section and elevates its urgency.

% --- Section 5: Consolidated Risk Assessment ---
\section{Consolidated Risk Assessment}
The following table synthesizes findings from the security control review, technical scan, and pre-existing risk data into a consolidated list of key risks.

\begin{table}[h!]
\centering
\caption{Summary of Identified Risks}
\begin{tabular}{@{}p{0.1\linewidth}p{0.25\linewidth}p{0.1\linewidth}p{0.45\linewidth}@{}}
    \toprule
    \textbf{Risk ID} & \textbf{Risk Name} & \textbf{Severity} & \textbf{Description \& Impact} \\
    \midrule
    R-01 & \textbf{Localhost Exposed via Open Port} & \sev_critical{Critical} & The SSH service is exposed on the localhost interface. This critical misconfiguration could be exploited by local processes or attackers who have gained initial access to escalate privileges or pivot within the network. This correlates with a known vulnerability with a CVSS score of 10.0. \\
    \addlinespace
    R-02 & \textbf{Lack of Annual Security Training} & \sev_high{High} & The absence of a mandatory, recurring security awareness program for all staff leaves the organization vulnerable to human error. This increases the likelihood of a successful phishing or social engineering attack, which could bypass technical controls. \\
    \bottomrule
\end{tabular}
\end{table}

% --- Section 6: Recommendations ---
\section{Recommendations}
The following actionable recommendations are provided to address the identified risks. They are prioritized based on severity and potential impact.

\subsection*{R-01: Remediate Localhost Exposure (Priority: Critical)}
This vulnerability requires immediate investigation and remediation due to its critical nature.
\begin{itemize}
    \item \textbf{Immediate Action:} Investigate the purpose of the SSH service running on the localhost interface of asset \texttt{127.0.0.1}.
    \item \textbf{Containment:} If the service is not essential for a specific application's function, it should be disabled immediately, and the configuration should be updated to prevent it from starting on boot.
    \item \textbf{Hardening:} If the service is required, ensure it is properly configured. This includes enforcing key-based authentication only, disabling root login via SSH, and implementing strict firewall rules to ensure the service cannot be accessed from unintended sources.
\end{itemize}

\subsection*{R-02: Implement Annual Security Awareness Training (Priority: High)}
This policy gap should be closed to strengthen the organization's human firewall.
\begin{itemize}
    \item \textbf{Immediate Action:} Procure or develop a comprehensive security awareness training program. The program should be mandatory for all employees, including management and executive staff.
    \item \textbf{Content:} Training modules should cover modern threats, including but not limited to: phishing, spear phishing, business email compromise (BEC), password hygiene, and safe data handling.
    \item \textbf{Long-Term Strategy:} Establish a recurring, annual schedule for this training. Implement a tracking system to monitor completion rates and ensure 100\% compliance. Consider periodic phishing simulations to test and reinforce the training's effectiveness.
\end{itemize}

\end{document}
```