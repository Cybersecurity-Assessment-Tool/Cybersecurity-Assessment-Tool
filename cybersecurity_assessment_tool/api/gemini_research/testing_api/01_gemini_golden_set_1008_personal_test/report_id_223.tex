```latex
\documentclass[12pt]{article}

% ----------------------------------------------------------------------
% PREAMBLE
% ----------------------------------------------------------------------
\usepackage[margin=1in]{geometry}
\usepackage{pifont} % For checkmarks and crosses
\usepackage{booktabs} % For professional tables
\usepackage{hyperref} % For clickable links
\usepackage{url}      % For URL formatting
\usepackage{seqsplit} % To split long strings in tt font
\usepackage{graphicx}
\usepackage{xcolor}

% --- Document Metadata ---
\title{Cybersecurity Posture Assessment Report}
\author{Cybersecurity Analysis Cell}
\date{\today}

% --- Hyperref Setup ---
\hypersetup{
    colorlinks=true,
    linkcolor=black,
    urlcolor=blue,
    pdftitle={Cybersecurity Posture Assessment},
    pdfauthor={Cybersecurity Analysis Cell},
}

% --- Custom Commands ---
\newcommand{\yes}{\ding{51}}
\newcommand{\no}{\ding{55}}

% ----------------------------------------------------------------------
% DOCUMENT START
% ----------------------------------------------------------------------
\begin{document}

\maketitle
\hrule
\vspace{1cm}

\begin{center}
\textbf{CONFIDENTIAL REPORT} \\
\vspace{0.5cm}
Prepared for: \textbf{Ironclad Logistics}
\end{center}

\newpage
\tableofcontents
\newpage

% ----------------------------------------------------------------------
% SECTION 1: EXECUTIVE OVERVIEW
% ----------------------------------------------------------------------
\section{Executive Overview}

This report details the findings of a cybersecurity posture assessment conducted for \textbf{Ironclad Logistics}. The assessment combines an analysis of organizational security controls, a technical network scan, and a review of pre-existing risks to provide a holistic view of the organization's security posture.

The analysis identified several critical and high-risk security gaps that require immediate attention. The most significant findings are the absence of Multi-Factor Authentication (MFA) for email and sensitive data systems. These gaps expose the organization to a high likelihood of account compromise, data breach, and business email compromise (BEC) attacks.

Furthermore, the lack of security awareness training during employee onboarding creates a window of vulnerability where new staff are more susceptible to social engineering attacks.

From a technical perspective, an external-facing administrative service (SSH on port 22) was identified on the IPv6 address \seqsplit{\texttt{2001:db8::1}}. While necessary for remote management, its public exposure increases the attack surface and must be appropriately hardened.

While the organization has implemented some positive security controls, such as MFA for computer logins and an annual training program, the identified deficiencies must be remediated promptly to reduce the overall risk profile to an acceptable level.

% ----------------------------------------------------------------------
% SECTION 2: ORGANIZATIONAL INFORMATION
% ----------------------------------------------------------------------
\section{Organizational Information}

The following details were provided for the assessment:

\begin{itemize}
    \item \textbf{Organization Name:} Ironclad Logistics
    \item \textbf{Email Domain:} \texttt{IroncladLogistics.com}
    \item \textbf{Website Domain:} \url{www.IroncladLogistics.com}
    \item \textbf{External IP Address:} \texttt{169.218.179.235}
\end{itemize}

% ----------------------------------------------------------------------
% SECTION 3: SECURITY CONTROL REVIEW
% ----------------------------------------------------------------------
\section{Security Control Review (Questionnaire Analysis)}

The following table summarizes the organization's responses to a security controls questionnaire. Each response is assessed against industry best practices. "No" answers indicate significant gaps in the security framework.

\begin{table}[h!]
\centering
\caption{Security Controls Questionnaire Results}
\begin{tabular}{p{8cm} c p{4cm}}
\toprule
\textbf{Control Question} & \textbf{Response} & \textbf{Assessment} \\
\midrule
Do you require MFA to access email? & \no & \textcolor{red}{\textbf{Critical Gap}} \\
Do you require MFA to log into computers? & \yes & Meets Best Practice \\
Do you require MFA to access sensitive data systems? & \no & \textcolor{red}{\textbf{Critical Gap}} \\
Does your organization have an employee acceptable use policy? & \yes & Meets Best Practice \\
Does your organization do security awareness training for new employees? & \no & \textcolor{orange}{\textbf{High Risk}} \\
Does your organization do security awareness training for all employees at least once per year? & \yes & Meets Best Practice \\
\bottomrule
\end{tabular}
\end{table}

% ----------------------------------------------------------------------
% SECTION 4: TECHNICAL SCAN RESULTS
% ----------------------------------------------------------------------
\section{Technical Scan Results}

An external network scan was performed on the target system to identify open ports and exposed services.

\begin{itemize}
    \item \textbf{Target IP Address:} \seqsplit{\texttt{2001:db8::1}}
    \item \textbf{Target Status:} Host is Up
\end{itemize}

\begin{table}[h!]
\centering
\caption{Open Ports Detected on \seqsplit{\texttt{2001:db8::1}}}
\begin{tabular}{l l l p{6cm}}
\toprule
\textbf{Port} & \textbf{State} & \textbf{Service (Inferred)} & \textbf{Notes} \\
\midrule
22/TCP & Open & SSH (Secure Shell) & This port is commonly used for remote system administration. The scan data did not include service version information. Public exposure of administrative services increases the attack surface. \\
\bottomrule
\end{tabular}
\end{table}

% ----------------------------------------------------------------------
% SECTION 5: RISK ASSESSMENT SUMMARY
% ----------------------------------------------------------------------
\section{Risk Assessment Summary}

The following table consolidates risks identified from the security control review and technical scan. As no pre-existing vulnerabilities were provided, this table reflects only the new findings from this assessment.

\begin{table}[h!]
\centering
\caption{Consolidated Risk Register}
\begin{tabular}{p{2cm} p{4.5cm} l p{5.5cm}}
\toprule
\textbf{Risk ID} & \textbf{Risk Name} & \textbf{Severity} & \textbf{Overview} \\
\midrule
RISK-001 & Lack of MFA on Email & \textcolor{red}{\textbf{Critical}} & The absence of MFA on email accounts significantly increases the risk of unauthorized access through credential theft or phishing, leading to potential data breaches and BEC. \\
\addlinespace
RISK-002 & Lack of MFA on Sensitive Systems & \textcolor{red}{\textbf{Critical}} & Sensitive data systems without MFA are prime targets for attackers. A single compromised credential could lead to a catastrophic breach of confidential information. \\
\addlinespace
RISK-003 & Inadequate New Employee Onboarding Security & \textcolor{orange}{\textbf{High}} & New employees are not receiving security awareness training upon being hired. This makes them highly vulnerable to social engineering and phishing attacks during their initial employment period. \\
\addlinespace
RISK-004 & Exposed Administrative Service (SSH) & \textbf{Medium} & The SSH service on a publicly accessible server is exposed to brute-force attacks and exploitation of potential vulnerabilities. Without proper hardening, it presents a direct path into the network. \\
\bottomrule
\end{tabular}
\end{table}

% ----------------------------------------------------------------------
% SECTION 6: RECOMMENDATIONS
% ----------------------------------------------------------------------
\section{Recommendations}

Based on the findings of this assessment, the following actions are recommended to mitigate the identified risks and improve the overall security posture of \textbf{Ironclad Logistics}.

\begin{enumerate}
    \item \textbf{[Critical] Implement Mandatory MFA:}
    \begin{itemize}
        \item Immediately develop and execute a plan to enforce MFA for all user access to email (RISK-001) and all systems classified as containing sensitive data (RISK-002).
        \item Prioritize deployment for privileged accounts (administrators, executives) and then roll out to the general user base.
    \end{itemize}
    \vspace{0.5cm}
    \item \textbf{[High] Enhance Security Onboarding Process:}
    \begin{itemize}
        \item Integrate a mandatory security awareness training module into the new employee onboarding process (RISK-003). This training should occur before the employee is granted access to sensitive systems and should cover topics like phishing, password hygiene, and acceptable use.
    \end{itemize}
    \vspace{0.5cm}
    \item \textbf{[Medium] Harden External-Facing Services:}
    \begin{itemize}
        \item For the exposed SSH service on \seqsplit{\texttt{2001:db8::1}} (RISK-004), conduct a review to determine if public access is necessary.
        \item If it is required, implement security hardening measures, including but not limited to:
            \begin{itemize}
                \item Disabling password-based authentication in favor of public key cryptography.
                \item Restricting access to a whitelist of trusted IP addresses.
                \item Implementing an intrusion detection/prevention tool like Fail2ban to block brute-force attempts.
                \item Ensuring the SSH server software is patched and up-to-date.
            \end{itemize}
    \end{itemize}
\end{enumerate}

\end{document}
% ----------------------------------------------------------------------
% DOCUMENT END
% ----------------------------------------------------------------------
```