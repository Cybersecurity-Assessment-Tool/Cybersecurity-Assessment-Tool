```latex
\documentclass[12pt]{article}

% --- PACKAGE IMPORTS ---
\usepackage[margin=1in]{geometry}
\usepackage{pifont} % For checkmarks and crosses
\usepackage{booktabs} % For professional tables
\usepackage{hyperref} % For hyperlinks
\usepackage{url} % For URL formatting
\usepackage{seqsplit} % To split long strings without breaking
\usepackage{graphicx} % For potential logos
\usepackage{xcolor} % For colors

% --- DOCUMENT METADATA ---
\hypersetup{
    colorlinks=true,
    linkcolor=blue,
    filecolor=magenta,      
    urlcolor=cyan,
    pdftitle={Cybersecurity Assessment Report},
    pdfauthor={Cybersecurity Analyst},
    pdfsubject={Security Analysis},
    pdfkeywords={Cybersecurity, Risk, Assessment},
}

% --- DOCUMENT START ---
\begin{document}

% --- TITLE PAGE ---
\begin{titlepage}
    \centering
    \vspace*{1cm}
    \Huge{\textbf{Cybersecurity Assessment Report}}
    \vspace{0.5cm}
    \Large{Prepared for: \textbf{Sterling Silver}}
    
    \vspace{1.5cm}
    
    \textbf{Date of Report:} \today
    
    \vfill
    
    \large
    \textbf{Generated By:} \\
    Expert-Level Cybersecurity Analyst
    
    \vspace{0.8cm}
    \textit{This report contains sensitive information and is intended solely for the designated recipient. Unauthorized distribution is prohibited.}
    
\end{titlepage}

\tableofcontents
\newpage

% --- EXECUTIVE OVERVIEW ---
\section{Executive Overview}
This report provides a comprehensive cybersecurity assessment for \textbf{Sterling Silver}, based on a technical network scan, a review of organizational security controls, and an analysis of known risks. The evaluation was conducted on \today.

The assessment reveals a mixed security posture. On one hand, the technical scan of the target host (\texttt{192.168.1.100}) indicates a very strong network security configuration, with no open ports detected. This significantly reduces the external attack surface of the scanned asset and is a commendable finding.

On the other hand, a review of organizational security controls identified several critical gaps. The most significant risks stem from the lack of Multi-Factor Authentication (MFA) for logging into computers and accessing sensitive data systems. These gaps expose the organization to significant risk from credential theft and unauthorized access. Furthermore, the absence of annual security awareness training for all employees constitutes a high risk, as it increases susceptibility to social engineering and phishing attacks.

Our primary recommendations focus on the immediate implementation of a comprehensive MFA strategy and the establishment of a recurring security training program to mitigate these identified risks effectively.

% --- ORGANIZATIONAL INFORMATION ---
\section{Organizational Information}
The following details were provided for the assessment. This information is used to establish the context and scope of the review.

\begin{itemize}
    \item \textbf{Organization Name:} Sterling Silver
    \item \textbf{Email Domain:} \texttt{SterlingSilver.net}
    \item \textbf{Website Domain:} \url{www.SterlingSilver.net}
    \item \textbf{External IP Address:} \texttt{74.87.98.5}
\end{itemize}

% --- SECURITY CONTROL REVIEW ---
\section{Security Control Review}
The following table summarizes the organization's responses to a security controls questionnaire. A green checkmark (\ding{51}) indicates a positive control is in place, while a red cross (\ding{55}) highlights a potential security gap.

\begin{table}[h!]
\centering
\caption{Security Controls Questionnaire Results}
\begin{tabular}{p{0.6\textwidth} c c}
\toprule
\textbf{Control Question} & \textbf{Response} & \textbf{Status} \\
\midrule
Do you require MFA to access email? & Yes & \textcolor{green}{\ding{51}} \\
Do you require MFA to log into computers? & No & \textcolor{red}{\ding{55}} \\
Do you require MFA to access sensitive data systems? & No & \textcolor{red}{\ding{55}} \\
Does your organization have an employee acceptable use policy? & Yes & \textcolor{green}{\ding{51}} \\
Does your organization do security awareness training for new employees? & Yes & \textcolor{green}{\ding{51}} \\
Does your organization do security awareness training for all employees at least once per year? & No & \textcolor{red}{\ding{55}} \\
\bottomrule
\end{tabular}
\end{table}

The identified gaps, particularly concerning MFA and ongoing training, are addressed in the Risk Assessment (Section 5) and Recommendations (Section 6) of this report.

% --- TECHNICAL SCAN RESULTS ---
\section{Technical Scan Results}
A network scan was performed to identify exposed services and potential vulnerabilities on the specified target system.

\subsection{Nmap Scan: \texttt{192.168.1.100}}
\begin{itemize}
    \item \textbf{Scan Date:} \today
    \item \textbf{Target IP:} \texttt{192.168.1.100}
    \item \textbf{Host Status:} UP
\end{itemize}

\textbf{Findings:}
The scan concluded that the host is online and responsive. However, \textbf{no open TCP or UDP ports were discovered}. All other scanned ports were reported as being in a `closed` state.

\textbf{Analysis:}
A host with no open ports presents a minimal network attack surface. This is an excellent security posture, indicating that the device is likely protected by a well-configured firewall or has no network-facing services running. This configuration adheres to the principle of least privilege and is a significant strength.

% --- RISK ASSESSMENT ---
\section{Risk Assessment}
This section synthesizes findings from the security control review and technical scan to provide a consolidated list of identified risks. As no pre-existing vulnerabilities were reported, the following risks are derived directly from this assessment.

\begin{table}[h!]
\centering
\caption{Summary of Identified Risks}
\begin{tabular}{p{0.25\textwidth} p{0.5\textwidth} l}
\toprule
\textbf{Identified Risk} & \textbf{Description} & \textbf{Severity} \\
\midrule
\textbf{Lack of MFA on Endpoints} & User accounts for computer logins are protected only by passwords. A compromised password could lead to direct endpoint compromise and lateral movement within the network. & \textbf{Critical} \\
\addlinespace
\textbf{Lack of MFA on Sensitive Systems} & Critical data systems can be accessed without a second factor of authentication. This creates a high-impact risk, as a single credential breach could lead to a major data exfiltration event. & \textbf{Critical} \\
\addlinespace
\textbf{Insufficient Security Training} & Without mandatory annual training, employees' ability to recognize and respond to modern threats like phishing and social engineering diminishes over time, making them the weakest link in the security chain. & \textbf{High} \\
\bottomrule
\end{tabular}
\end{table}

% --- RECOMMENDATIONS ---
\section{Recommendations}
Based on the analysis, the following actionable recommendations are provided to enhance the cybersecurity posture of \textbf{Sterling Silver}. Recommendations are prioritized by severity.

\begin{enumerate}
    \item \textbf{[Critical] Deploy MFA for Sensitive System Access:} Immediately prioritize the implementation of MFA on all systems classified as containing sensitive or critical data. This is the single most effective control to prevent unauthorized access to key organizational assets.
    
    \item \textbf{[Critical] Enforce MFA for Endpoint Logon:} Implement MFA for all user logins to company computers (desktops and laptops). This dramatically increases the difficulty for an attacker to gain an initial foothold in the network, even with stolen credentials. Solutions like Windows Hello for Business, Duo, or other third-party tools can be leveraged.
    
    \item \textbf{[High] Establish a Mandatory Annual Security Training Program:} Develop and enforce a security awareness training program that is mandatory for all employees on an annual basis. This program should cover current threats such as phishing, ransomware, and proper data handling. Phishing simulations should be used to measure the program's effectiveness.
    
    \item \textbf{[Informational] Maintain Strong Network Hardening:} The excellent network posture of the scanned host (\texttt{192.168.1.100}) should be used as a baseline standard for other servers and critical devices within the network. Continue to apply the principle of least privilege by ensuring that only necessary ports and services are exposed.
\end{enumerate}

% --- DOCUMENT END ---
\end{document}
```