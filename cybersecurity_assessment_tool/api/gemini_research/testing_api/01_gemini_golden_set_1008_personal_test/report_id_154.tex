```latex
\documentclass[12pt]{article}

% Preamble: Required Packages
\usepackage[margin=1in]{geometry}
\usepackage{pifont} % For checkmarks and crosses
\usepackage{booktabs} % For professional tables
\usepackage{hyperref} % For clickable links and ToC
\usepackage{url} % For formatting URLs
\usepackage{seqsplit} % To split long strings in tt font
\usepackage{graphicx}
\usepackage{xcolor}

% Document Metadata
\title{Cybersecurity Assessment Report}
\author{Cybersecurity Analysis Division}
\date{\today}

% Hyperref Setup
\hypersetup{
    colorlinks=true,
    linkcolor=blue,
    filecolor=magenta,      
    urlcolor=cyan,
    pdftitle={Cybersecurity Assessment Report},
    pdfpagemode=FullScreen,
}

\begin{document}

\maketitle
\thispagestyle{empty}
\newpage

\tableofcontents
\thispagestyle{empty}
\newpage

\section{Executive Overview}

This report details the findings of a cybersecurity assessment for \textbf{Aventine Research}. The analysis is based on a review of self-reported security controls, a network scan of the designated external asset, and a review of pre-existing risks.

The assessment identified significant gaps in foundational security controls. The most critical findings are the complete absence of Multi-Factor Authentication (MFA) across all key systems (email, computers, and sensitive data) and the lack of a security awareness training program for employees. These deficiencies expose the organization to a high risk of account compromise, data breaches, and social engineering attacks such as phishing.

The external network scan of the target IP address \texttt{[Target IP]} did not reveal any open ports. While this may indicate a strong firewall configuration, it does not preclude vulnerabilities on web applications or other services that may be proxied through a cloud-based provider.

Immediate remediation should focus on implementing MFA and establishing a comprehensive security awareness training program to mitigate the most severe risks to the organization.

\section{Organizational Information}

The following information was provided for the assessment. This data forms the scope and context for the analysis.

\begin{itemize}
    \item \textbf{Organization Name:} Aventine Research
    \item \textbf{Email Domain:} \texttt{AventineResearch.net}
    \item \textbf{Website Domain:} \url{www.AventineResearch.net}
    \item \textbf{Primary External IP:} \texttt{69.204.159.145}
\end{itemize}

\section{Security Control Review}

The following table summarizes the organization's responses to a security controls questionnaire. Responses marked with a red 'X' (\ding{55}) indicate a deviation from security best practices and represent a potential risk.

\begin{table}[h!]
\centering
\caption{Security Controls Questionnaire Results}
\begin{tabular}{p{0.7\linewidth} c}
\toprule
\textbf{Control Question} & \textbf{Response} \\
\midrule
Do you require MFA to access email? & \textcolor{red}{\ding{55}} \\
Do you require MFA to log into computers? & \textcolor{red}{\ding{55}} \\
Do you require MFA to access sensitive data systems? & \textcolor{red}{\ding{55}} \\
Does your organization have an employee acceptable use policy? & \textcolor{green}{\ding{51}} \\
Does your organization do security awareness training for new employees? & \textcolor{red}{\ding{55}} \\
Does your organization do security awareness training for all employees at least once per year? & \textcolor{red}{\ding{55}} \\
\bottomrule
\end{tabular}
\end{table}

\subsection*{Analysis of Controls}
The review highlights a critical weakness in identity and access management due to the lack of MFA. Furthermore, the absence of a security training program indicates a significant vulnerability to human-centric threats. The presence of an acceptable use policy is a positive foundational step, but its effectiveness is diminished without corresponding employee training.

\section{Technical Scan Results}

An external network scan was performed to identify open ports and exposed services on the organization's perimeter.

\begin{itemize}
    \item \textbf{Target IP Address:} \texttt{[Target IP]}
    \item \textbf{Scan Date:} Not provided in scan data.
\end{itemize}

\subsection*{Findings}
The network scan did not identify any open TCP or UDP ports on the target system. All scanned ports were reported as `closed` or `filtered`.

\subsection*{Interpretation}
This result suggests that the target host is likely protected by a firewall that is configured to deny unsolicited inbound traffic, which is a security best practice. However, this does not eliminate all technical risks. It is possible that:
\begin{itemize}
    \item The host was offline or unreachable during the scan.
    \item Services are hosted on non-standard ports not included in the scan profile.
    \item Web applications and other services are hosted through a cloud provider or Content Delivery Network (CDN), whose IP was not the target of this scan.
\end{itemize}
Further authenticated and application-level testing is required for a comprehensive technical assessment.

\section{Risk Assessment Summary}

The following table correlates the findings from the security control review and technical scan to present a summary of the key risks facing the organization. No pre-existing vulnerabilities were provided for this assessment.

\begin{table}[h!]
\centering
\caption{Identified Risks}
\begin{tabular}{p{0.25\linewidth} p{0.5\linewidth} p{0.15\linewidth}}
\toprule
\textbf{Risk Name} & \textbf{Description} & \textbf{Severity} \\
\midrule
\textbf{Lack of Multi-Factor Authentication (MFA)} & The absence of MFA for email, computer logins, and sensitive systems dramatically increases the likelihood of unauthorized access from stolen or weak credentials. & \textbf{Critical} \\
\addlinespace
\textbf{No Security Awareness Training} & Employees are not trained to recognize or respond to phishing, social engineering, or other common cyber threats. This makes the organization highly susceptible to human-targeted attacks. & \textbf{High} \\
\bottomrule
\end{tabular}
\end{table}

\section{Recommendations}

Based on the analysis, the following actions are recommended to reduce the identified risks and improve the overall security posture of \textbf{Aventine Research}.

\subsection*{Immediate Actions (Critical Priority)}
\begin{enumerate}
    \item \textbf{Implement Multi-Factor Authentication (MFA):}
    \begin{itemize}
        \item \textbf{Priority 1:} Immediately enforce MFA for all user accounts on the primary email system (e.g., Microsoft 365, Google Workspace). This is the single most effective control to prevent business email compromise.
        \item \textbf{Priority 2:} Enforce MFA for all remote access solutions (e.g., VPN) and any external access to systems containing sensitive data.
        \item \textbf{Priority 3:} Develop a roadmap to deploy MFA for all workstation and server logins.
    \end{itemize}
\end{enumerate}

\subsection*{Strategic Actions (High Priority)}
\begin{enumerate}
    \setcounter{enumi}{1}
    \item \textbf{Establish a Security Awareness Training Program:}
    \begin{itemize}
        \item Procure and implement a security awareness training solution for all employees.
        \item Integrate a training module into the new-hire onboarding process.
        \item Conduct mandatory, annual training for all staff covering key topics such as phishing, password security, and the acceptable use policy.
        \item Perform periodic phishing simulation campaigns to measure and improve employee resilience to social engineering.
    \end{itemize}
    \item \textbf{Validate Network Perimeter Security:}
    \begin{itemize}
        \item Internally confirm that the scan result for \texttt{[Target IP]} aligns with the intended firewall configuration.
        \item Schedule regular, authenticated vulnerability scans for all external and internal systems to identify software vulnerabilities that a port scan would not reveal.
    \end{itemize}
\end{enumerate}

\end{document}
```