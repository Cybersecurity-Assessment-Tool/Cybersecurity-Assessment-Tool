```latex
\documentclass[12pt]{article}

% Required Packages
\usepackage[margin=1in]{geometry}
\usepackage{pifont} % For checkmarks and crosses
\usepackage{booktabs} % For professional tables
\usepackage{hyperref} % For clickable links
\usepackage{url} % For URL formatting
\usepackage{seqsplit} % For splitting long strings in tt font
\usepackage{graphicx}
\usepackage{xcolor}

% Hyperref Setup
\hypersetup{
    colorlinks=true,
    linkcolor=blue,
    filecolor=magenta,      
    urlcolor=cyan,
    pdftitle={Cybersecurity Assessment Report},
    pdfpagemode=FullScreen,
}

% Define check and cross symbols
\newcommand{\cmark}{\ding{51}}%
\newcommand{\xmark}{\ding{55}}%

% Document Title
\title{Cybersecurity Assessment Report \\ \large For: \textbf{Signal Flare}}
\author{Cybersecurity Analyst Group}
\date{\today}

\begin{document}

\maketitle
\thispagestyle{empty}
\newpage

\tableofcontents
\newpage

% --- 1. Executive Summary ---
\section{Executive Summary}
This report details the findings of a cybersecurity assessment conducted for \textbf{Signal Flare}. The assessment combined a review of organizational security controls, a technical network scan, and an analysis of pre-existing risk data.

The primary findings indicate critical gaps in access control policies. Specifically, the absence of Multi-Factor Authentication (MFA) across all key systems, including email, workstations, and sensitive data repositories, presents a \textbf{Critical} risk to the organization. An attacker with compromised credentials could gain significant and potentially unrestricted access to corporate resources.

Additionally, while security training is provided to new hires, the lack of a mandatory annual refresher course for all employees constitutes a \textbf{High} risk, leaving the organization vulnerable to evolving social engineering and phishing tactics.

A technical scan of the target host \texttt{192.168.0.5} did not identify any open ports. This conflicts with a pre-existing risk report indicating an open and unencrypted web server on Port 80. This suggests the risk may have been remediated, but further verification is required.

Immediate remediation should focus on the phased implementation of a mandatory MFA policy, starting with sensitive systems, and the establishment of a recurring security awareness training program.

% --- 2. Organizational Information ---
\section{Organizational Information}
The following information was provided for the assessment.

\begin{tabular}{@{}ll}
\toprule
\textbf{Attribute} & \textbf{Value} \\
\midrule
Organization Name & \textbf{Signal Flare} \\
Email Domain & \texttt{SignalFlare.net} \\
Website Domain & \url{www.SignalFlare.net} \\
External IP Address & \texttt{61.26.139.145} \\
\bottomrule
\end{tabular}

% --- 3. Security Control Review ---
\section{Security Control Review}
A security questionnaire was completed to evaluate existing administrative and technical controls. The results are summarized below. Answers marked with \xmark{} represent significant gaps in the organization's security posture.

\begin{table}[h!]
\centering
\begin{tabular}{@{}p{8cm} c p{4cm}@{}}
\toprule
\textbf{Control Question} & \textbf{Response} & \textbf{Analyst Note} \\
\midrule
Do you require MFA to access email? & \textcolor{red}{\xmark} & \textbf{High Risk}. Email is a primary target for account takeover. \\
\addlinespace
Do you require MFA to log into computers? & \textcolor{red}{\xmark} & \textbf{High Risk}. Lack of MFA allows for easier lateral movement. \\
\addlinespace
Do you require MFA to access sensitive data systems? & \textcolor{red}{\xmark} & \textbf{Critical Risk}. Core assets are not adequately protected. \\
\addlinespace
Does your organization have an employee acceptable use policy? & \textcolor{green}{\cmark} & Control is in place. \\
\addlinespace
Does your organization do security awareness training for new employees? & \textcolor{green}{\cmark} & Good baseline for new staff. \\
\addlinespace
Does your organization do security awareness training for all employees at least once per year? & \textcolor{red}{\xmark} & \textbf{High Risk}. Skills degrade and threats evolve. \\
\bottomrule
\end{tabular}
\caption{Security Controls Questionnaire Results}
\label{tab:controls}
\end{table}

% --- 4. Technical Scan Results ---
\section{Technical Scan Results}
A network scan was performed to identify listening services and potential vulnerabilities on the specified target system.

\begin{itemize}
    \item \textbf{Target IP:} \texttt{192.168.0.5}
    \item \textbf{Target Status:} Host is Up
\end{itemize}

\subsection{Port Scan Details}
The scan revealed the following port status. No open ports were detected at the time of the scan.

\begin{table}[h!]
\centering
\begin{tabular}{@{}lllll@{}}
\toprule
\textbf{Port} & \textbf{State} & \textbf{Service} & \textbf{Product} & \textbf{Version} \\
\midrule
80/tcp & closed & http & N/A & N/A \\
\bottomrule
\end{tabular}
\caption{Nmap Scan Results for \texttt{192.168.0.5}}
\label{tab:nmap}
\end{table}

\subsection{Analysis of Technical Findings}
The scan results are minimal, showing that the target host is online but has no externally accessible services on the scanned ports. 

Notably, this scan result contradicts a pre-existing risk entry (\textit{Unencrypted Web Server}) which states that Port 80 is open. This discrepancy suggests that the previously identified risk may have been remediated. However, this should be formally verified across all relevant assets.

% --- 5. Risk Assessment & Findings Summary ---
\section{Risk Assessment \& Findings Summary}
The following table synthesizes findings from the security control review, technical scan, and pre-existing risk data.

\begin{table}[h!]
\centering
\begin{tabular}{@{}p{2cm} p{4.5cm} p{2.5cm} p{4cm}@{}}
\toprule
\textbf{Finding ID} & \textbf{Description} & \textbf{Severity} & \textbf{Affected Elements} \\
\midrule
\textbf{RISK-001} & Lack of MFA on systems handling sensitive data. & \textbf{Critical} & All sensitive data systems, core business applications. \\
\addlinespace
\textbf{RISK-002} & Lack of MFA on standard user endpoints and email accounts. & \textbf{High} & All employee workstations and email accounts (\texttt{SignalFlare.net}). \\
\addlinespace
\textbf{RISK-003} & Inadequate security awareness training program (no annual refresher). & \textbf{High} & All employees. \\
\addlinespace
\textbf{OBS-001} & The current scan shows Port 80 is closed, which conflicts with pre-existing risk data. & Informational & Web server infrastructure, risk management process. \\
\bottomrule
\end{tabular}
\caption{Summary of Identified Risks and Observations}
\label{tab:risks}
\end{table}

% --- 6. Recommendations ---
\section{Recommendations}
The following actions are recommended to mitigate the identified risks and improve the overall security posture of \textbf{Signal Flare}.

\subsection{RISK-001: Implement MFA on Sensitive Systems (Critical)}
\begin{itemize}
    \item \textbf{Immediate Action:} Immediately enable and enforce MFA for all administrative and privileged accounts across all infrastructure and applications.
    \item \textbf{Short-Term Action:} Develop a phased rollout plan to enforce MFA for all users accessing systems designated as containing sensitive or critical data. Prioritize systems with financial, customer, or proprietary information.
\end{itemize}

\subsection{RISK-002: Enforce MFA for All Users (High)}
\begin{itemize}
    \item \textbf{Action:} Following the critical system rollout, expand the mandatory MFA policy to cover all user-facing systems, including email access (e.g., Office 365, Google Workspace) and workstation logins (e.g., Windows Hello, Duo).
\end{itemize}

\subsection{RISK-003: Establish Annual Security Training (High)}
\begin{itemize}
    \item \textbf{Action:} Procure or develop a security awareness training module and make it mandatory for all employees to complete on an annual basis. The training should cover current threats such as phishing, ransomware, and proper data handling. Track completion to ensure compliance.
\end{itemize}

\subsection{OBS-001: Verify Port 80 Status (Informational)}
\begin{itemize}
    \item \textbf{Action:} Conduct a comprehensive internal and external scan of all public-facing assets to verify that Port 80 is closed where it should be. Update the organizational risk register to reflect that the "Unencrypted Web Server" risk is either remediated or still present on other systems. If unencrypted web services are required, develop a plan to migrate them to HTTPS (Port 443).
\end{itemize}

\end{document}
```