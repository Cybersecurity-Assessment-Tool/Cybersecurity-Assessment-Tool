```latex
\documentclass[12pt]{article}

% Preamble: Required Packages
\usepackage[margin=1in]{geometry} % Set page margins
\usepackage{pifont}               % For checkmark and X symbols (\ding)
\usepackage{booktabs}             % For professional-looking tables
\usepackage{hyperref}             % For clickable links and metadata
\usepackage{url}                  % For formatting URLs
\usepackage{seqsplit}             % To split long strings in texttt
\usepackage[utf8]{inputenc}       % For UTF-8 input encoding

% Document Metadata
\hypersetup{
    colorlinks=true,
    linkcolor=black,
    urlcolor=blue,
    pdftitle={Cybersecurity Posture Report},
    pdfauthor={Cybersecurity Analysis Unit},
    pdfsubject={Security Assessment},
    pdfkeywords={Security, Risk, Analysis, Report}
}

\title{Cybersecurity Posture Report \\ \large For: Aventine Research}
\author{Cybersecurity Analysis Unit}
\date{\today}

\begin{document}

\maketitle
\thispagestyle{empty}
\newpage

\tableofcontents
\newpage

\section{Executive Summary}
This report provides a cybersecurity posture assessment for Aventine Research, based on a combination of network scanning, a security controls questionnaire, and a review of pre-existing risks.

The assessment reveals a significant disparity between the organization's technical and administrative security controls. The external network scan of the target host \texttt{192.168.1.100} indicated a strong security posture, with no open ports detected. This suggests that the specific asset is well-hardened against external network-based attacks.

However, the security control review identified critical deficiencies in foundational security practices. The absence of Multi-Factor Authentication (MFA) for computer and sensitive data system access, coupled with a complete lack of a security awareness training program and an acceptable use policy, presents a high level of risk. These administrative gaps could easily be exploited by threat actors through phishing or social engineering, bypassing the strong network perimeter. The most immediate threats stem from potential credential compromise and unauthorized access to internal systems.

Urgent remediation is required to address these policy and procedural gaps to build a resilient security posture.

\section{Organizational Information}
The following details were provided for the assessment.

\begin{itemize}
    \item \textbf{Organization Name:} Aventine Research
    \item \textbf{Email Domain:} \texttt{AventineResearch.net}
    \item \textbf{Website Domain:} \url{www.AventineResearch.net}
    \item \textbf{External IP Address:} \texttt{178.1.12.129}
\end{itemize}

\section{Security Control Review}
The following table summarizes the organization's responses to the security controls questionnaire. "No" answers indicate significant gaps in the security framework and are flagged as high-risk areas.

\begin{table}[h!]
\centering
\begin{tabular}{p{8cm} c l}
\toprule
\textbf{Control Question} & \textbf{Response} & \textbf{Assessment} \\
\midrule
Do you require MFA to access email? & \ding{51} Yes & Good Practice \\
Do you require MFA to log into computers? & \ding{55} No & \textbf{High Risk Gap} \\
Do you require MFA to access sensitive data systems? & \ding{55} No & \textbf{High Risk Gap} \\
Does your organization have an employee acceptable use policy? & \ding{55} No & \textbf{Critical Gap} \\
Does your organization do security awareness training for new employees? & \ding{55} No & \textbf{Critical Gap} \\
Does your organization do security awareness training for all employees at least once per year? & \ding{55} No & \textbf{Critical Gap} \\
\bottomrule
\end{tabular}
\caption{Security Controls Questionnaire Analysis}
\label{tab:controls}
\end{table}

\section{Technical Scan Results}
A network port scan was conducted to identify accessible services on the specified target system.

\begin{itemize}
    \item \textbf{Target IP Address:} \texttt{192.168.1.100}
    \item \textbf{Scan Date:} \today
    \item \textbf{Host Status:} Up
\end{itemize}

\textbf{Summary of Findings:}
The scan confirmed that the host is online and responsive. However, \textbf{no open TCP or UDP ports were discovered} within the scanned range. All other ports were reported as "closed," meaning they are accessible but have no application listening on them. This is a positive security finding, as it indicates a minimal network attack surface for this specific host.

\section{Consolidated Risk Assessment}
This section synthesizes findings from the security control review, technical scan, and pre-existing risk data. As no pre-existing vulnerabilities were reported, the following risks are derived directly from this assessment's findings. The primary risks are administrative and procedural rather than technical.

\begin{table}[h!]
\centering
\begin{tabular}{p{2.5cm} p{8cm} l}
\toprule
\textbf{Risk Name} & \textbf{Description} & \textbf{Severity} \\
\midrule
\textbf{Lack of MFA on Endpoints \& Systems} & The absence of MFA on computers and sensitive data systems means a single compromised password could grant an attacker full access, facilitating lateral movement and data exfiltration. & \textbf{High} \\
\addlinespace
\textbf{No Security Awareness Program} & Without training, employees are highly susceptible to phishing and social engineering attacks. This significantly increases the likelihood of an initial breach through human error. & \textbf{High} \\
\addlinespace
\textbf{Missing Acceptable Use Policy (AUP)} & The lack of a formal AUP creates ambiguity regarding the secure and appropriate use of company assets, leading to inconsistent practices and a weakened ability to enforce security standards. & \textbf{Medium} \\
\bottomrule
\end{tabular}
\caption{Summary of Identified Risks}
\label{tab:risks}
\end{table}

\section{Recommendations}
To mitigate the identified risks and improve the overall security posture of Aventine Research, the following actions are recommended with urgency.

\begin{enumerate}
    \item \textbf{Implement Comprehensive MFA:}
    \begin{itemize}
        \item \textbf{Action:} Deploy a robust Multi-Factor Authentication solution for all employee computer logins (desktops, laptops) and for access to all systems storing or processing sensitive data.
        \item \textbf{Priority:} Critical. This is the single most effective technical control to prevent unauthorized access from compromised credentials.
    \end{itemize}
    \vspace{0.5cm}
    \item \textbf{Establish a Security Awareness Training Program:}
    \begin{itemize}
        \item \textbf{Action:} Develop and mandate a security awareness program that includes initial training for all new hires and annual refresher training for all staff. Training should cover phishing, password security, and social engineering.
        \item \textbf{Priority:} High. An educated workforce is a critical layer of defense against common attack vectors.
    \end{itemize}
    \vspace{0.5cm}
    \item \textbf{Develop and Enforce an Acceptable Use Policy (AUP):}
    \begin{itemize}
        \item \textbf{Action:} Draft a clear AUP that outlines the rules and responsibilities for using company networks, devices, and data. This policy should be formally acknowledged by all employees.
        \item \textbf{Priority:} Medium. This policy provides the foundational governance needed to enforce other security controls.
    \end{itemize}
\end{enumerate}

\end{document}
```