```latex
\documentclass[12pt, a4paper]{article}

% Preamble: Required Packages
\usepackage[margin=1in]{geometry}
\usepackage{pifont} % For checkmarks and crosses
\usepackage{booktabs} % For professional-looking tables
\usepackage{hyperref} % For clickable links
\usepackage{url} % For formatting URLs
\usepackage{seqsplit} % To split long strings in texttt
\usepackage[table]{xcolor} % For coloring table cells
\usepackage{graphicx} % For potential logos (not used here, but good practice)
\usepackage{fancyhdr} % For headers and footers

% --- Document Setup ---
\hypersetup{
    colorlinks=true,
    linkcolor=blue,
    filecolor=magenta,      
    urlcolor=cyan,
    pdftitle={Cybersecurity Posture Report},
    pdfpagemode=FullScreen,
}

% Custom commands for severity levels for consistency and readability
\newcommand{\sevCRITICAL}{\colorbox{red!80!black}{\color{white}\textbf{\strut CRITICAL}}}
\newcommand{\sevHIGH}{\colorbox{orange!90!black}{\color{white}\textbf{\strut HIGH}}}
\newcommand{\sevMEDIUM}{\colorbox{yellow!80!black}{\color{black}\textbf{\strut MEDIUM}}}
\newcommand{\sevLOW}{\colorbox{green!70!black}{\color{white}\textbf{\strut LOW}}}

% Checkmark and Cross definitions
\newcommand{\cmark}{\ding{51}} % Checkmark
\newcommand{\xmark}{\ding{55}} % Cross

% --- Header and Footer ---
\pagestyle{fancy}
\fancyhf{}
\lhead{Cybersecurity Posture Report}
\rhead{Blue Horizon Initiative}
\cfoot{\thepage}

% --- Document Start ---
\begin{document}

% --- Title Page ---
\begin{titlepage}
    \centering
    \vspace*{1cm}
    \Huge\textbf{Cybersecurity Posture Report}
    \vspace{0.5cm}
    \Large For
    \vspace{1.5cm}
    \textbf{\Large Blue Horizon Initiative}
    \vfill
    \large
    \textbf{Date of Report:} \today \\
    \textbf{Report ID:} CSR-2023-451
    \vspace{0.8cm}
    \normalsize
    This report contains a summary of findings from a recent security assessment. It includes an analysis of organizational security controls, technical network scan results, and a consolidated risk assessment with actionable recommendations.
\end{titlepage}

\tableofcontents
\newpage

% --- Section 1: Executive Summary ---
\section{Executive Summary}
This report provides a comprehensive analysis of the cybersecurity posture for \textbf{Blue Horizon Initiative}. The assessment combined a review of organizational security controls, an external network scan, and an evaluation of known risks.

The overall security posture is considered \textbf{CRITICAL}. Two immediate and severe risks were identified that require immediate attention:
\begin{enumerate}
    \item \textbf{Exposed FTP Server with Critical Vulnerability:} A network scan identified an FTP server (\texttt{vsftpd 2.3.4}) running on an internal host (\texttt{10.0.0.15}). This specific version contains a well-known, publicly documented backdoor that allows for remote code execution (CVE-2011-2523). The server is also configured to allow anonymous logins, posing a significant data exfiltration and system compromise risk.
    \item \textbf{Lack of Multi-Factor Authentication (MFA) on Email:} The organizational questionnaire revealed that MFA is not required for email access. Email is a primary target for phishing and business email compromise (BEC) attacks. The absence of MFA makes user accounts highly susceptible to takeover with just a compromised password.
\end{enumerate}

Additionally, a pre-existing medium-risk finding regarding outdated Windows 7 workstations remains a concern. Prioritized, actionable recommendations are provided in Section \ref{sec:recommendations} to address these findings and improve the organization's defensive capabilities.

% --- Section 2: Organizational Information ---
\section{Organizational Information}
The following information was provided for the assessment.
\begin{center}
\begin{tabular}{ll}
\toprule
\textbf{Attribute} & \textbf{Value} \\
\midrule
Organization Name & Blue Horizon Initiative \\
Email Domain & \seqsplit{\texttt{BlueHorizonInitiative.org}} \\
Website Domain & \seqsplit{\url{www.BlueHorizonInitiative.org}} \\
External IP Address & \texttt{3.67.36.236} \\
\bottomrule
\end{tabular}
\end{center}

% --- Section 3: Security Control Review ---
\section{Security Control Review (Questionnaire)}
A review of the organization's security controls was conducted via a questionnaire. The responses indicate a solid foundation in policy and training, but highlight a critical gap in access control for a primary communication system.

\begin{center}
\begin{tabular}{p{0.6\textwidth} c c}
\toprule
\textbf{Control Question} & \textbf{Response} & \textbf{Status} \\
\midrule
Do you require MFA to access email? & \xmark & \sevCRITICAL \\
Do you require MFA to log into computers? & \cmark & OK \\
Do you require MFA to access sensitive data systems? & \cmark & OK \\
Does your organization have an employee acceptable use policy? & \cmark & OK \\
Does your organization do security awareness training for new employees? & \cmark & OK \\
Does your organization do security awareness training for all employees at least once per year? & \cmark & OK \\
\bottomrule
\end{tabular}
\end{center}

% --- Section 4: Technical Scan Results ---
\section{Technical Scan Results}
A network reconnaissance scan was performed to identify exposed services and potential vulnerabilities.

\begin{itemize}
    \item \textbf{Target IP Address:} \texttt{10.0.0.15}
    \item \textbf{Scan Date:} Not provided, report generated on \today.
\end{itemize}

\subsection{Open Ports and Services}
The following open port was discovered on the target system.

\begin{center}
\begin{tabular}{l l l l l p{0.3\textwidth}}
\toprule
\textbf{Port} & \textbf{State} & \textbf{Service} & \textbf{Product} & \textbf{Version} & \textbf{Notes} \\
\midrule
21/tcp & Open & ftp & vsftpd & 2.3.4 & \textbf{Critical Finding.} Anonymous FTP login is allowed. This version is known to be backdoored (CVE-2011-2523), allowing remote command execution. \\
\bottomrule
\end{tabular}
\end{center}

% --- Section 5: Risk Assessment ---
\section{Risk Assessment}
This section consolidates findings from the questionnaire, technical scan, and pre-existing risk data into a unified view.

\begin{center}
\rowcolors{2}{gray!10}{white}
\begin{tabular}{p{0.05\textwidth} p{0.35\textwidth} l p{0.2\textwidth} p{0.15\textwidth}}
\toprule
\textbf{ID} & \textbf{Risk Description} & \textbf{Severity} & \textbf{Affected Systems} & \textbf{Source} \\
\midrule
R-01 & A publicly known backdoor (CVE-2011-2523) exists in the running FTP server (vsftpd 2.3.4), allowing for total system compromise. & \sevCRITICAL & FTP Server \texttt{10.0.0.15} & Network Scan \\
\addlinespace
R-02 & Multi-Factor Authentication (MFA) is not enforced for email accounts, exposing the organization to phishing and account takeover. & \sevCRITICAL & Email System, All Users & Questionnaire \\
\addlinespace
R-03 & The FTP server is configured to allow anonymous, unauthenticated logins, posing a high risk of unauthorized data access and exfiltration. & \sevHIGH & FTP Server \texttt{10.0.0.15} & Network Scan \\
\addlinespace
R-04 & Workstations are running Windows 7, an unsupported operating system that no longer receives security updates. & \sevMEDIUM & Workstations & Pre-existing Risk \\
\bottomrule
\end{tabular}
\end{center}

% --- Section 6: Recommendations ---
\section{Recommendations}
\label{sec:recommendations}
Based on the findings, the following prioritized recommendations are provided to mitigate the identified risks.

\subsection*{Priority 1: Immediate Actions (Due within 24 hours)}
\begin{enumerate}
    \item \textbf{Mitigate FTP Server Vulnerability (R-01, R-03):}
    \begin{itemize}
        \item \textbf{Immediate:} Take the FTP server at \texttt{10.0.0.15} offline immediately by shutting down the service or applying a firewall rule to block all access to port 21.
        \item \textbf{Follow-up:} Conduct a forensic analysis of the server to determine if it has been compromised. Decommission the server permanently. If file transfer is a business requirement, deploy a new, fully patched server using a secure protocol such as SFTP (SSH File Transfer Protocol).
    \end{itemize}
    \item \textbf{Enforce MFA on Email (R-02):}
    \begin{itemize}
        \item \textbf{Immediate:} Enable and enforce MFA for all user accounts, including administrative and service accounts, within your email system (e.g., Microsoft 365, Google Workspace).
        \item \textbf{Follow-up:} Provide clear instructions and support to all employees to ensure a smooth rollout.
    \end{itemize}
\end{enumerate}

\subsection*{Priority 2: High-Priority Actions (Due within 90 days)}
\begin{enumerate}
    \setcounter{enumi}{2} % Continue numbering
    \item \textbf{Upgrade Outdated Operating Systems (R-04):}
    \begin{itemize}
        \item \textbf{Action:} Develop and execute a plan to upgrade all workstations from Windows 7 to a modern, supported operating system such as Windows 10 or Windows 11. This is essential for receiving security patches against new vulnerabilities.
    \end{itemize}
\end{enumerate}

\end{document}
```