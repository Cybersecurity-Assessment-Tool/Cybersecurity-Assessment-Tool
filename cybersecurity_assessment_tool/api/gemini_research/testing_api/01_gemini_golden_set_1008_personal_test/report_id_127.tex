```latex
\documentclass[12pt]{article}

% Preamble: Required Packages
\usepackage[margin=1in]{geometry}
\usepackage{pifont} % For checkmarks and crosses
\usepackage{booktabs} % For professional tables
\usepackage{hyperref} % For clickable links
\usepackage{url} % For formatting URLs
\usepackage{seqsplit} % For splitting long strings like IPs
\usepackage{graphicx}
\usepackage{xcolor}

% Hyperref Setup
\hypersetup{
    colorlinks=true,
    linkcolor=blue,
    filecolor=magenta,      
    urlcolor=cyan,
    pdftitle={Cybersecurity Posture Assessment Report},
    pdfpagemode=FullScreen,
}

% Document Title and Author
\title{Cybersecurity Posture Assessment Report}
\author{Cybersecurity Analyst}
\date{\today}

\begin{document}

\maketitle
\thispagestyle{empty}
\newpage

\tableofcontents
\newpage

% --- 1. Executive Summary ---
\section{Executive Summary}
This report provides a comprehensive cybersecurity posture assessment for \textbf{Bluebird Bio}. The analysis is based on a synthesis of network scan data, an organizational security controls questionnaire, and a review of pre-existing risks.

The assessment reveals several critical and high-risk gaps in the current security framework. Key findings include the lack of Multi-Factor Authentication (MFA) for computer and sensitive data system access, the absence of fundamental security policies and training for new employees, and the exposure of a network management service (SSH) on an external IPv6 address.

While the organization has implemented some positive controls, such as MFA for email and annual security training, the identified weaknesses create significant exposure to credential theft, unauthorized access, and potential system compromise. This report outlines these risks in detail and provides a prioritized list of actionable recommendations to mitigate them and strengthen the overall security posture.

% --- 2. Organizational Information ---
\section{Organizational Information}
The following details were provided for the assessment.

\begin{tabular}{@{}ll}
\toprule
\textbf{Attribute} & \textbf{Value} \\
\midrule
Organization Name & \textbf{Bluebird Bio} \\
Email Domain & \texttt{BluebirdBio.com} \\
Website Domain & \url{www.BluebirdBio.com} \\
External IP Address & \texttt{49.117.35.10} \\
\bottomrule
\end{tabular}

% --- 3. Security Control Review ---
\section{Security Control Review}
A review of the organization's security controls was conducted via a questionnaire. The responses indicate significant gaps in foundational security practices. A "No" response highlights a missing control and a potential area of high risk.

\begin{table}[h!]
\centering
\caption{Security Controls Questionnaire Analysis}
\begin{tabular}{p{0.6\linewidth} c l}
\toprule
\textbf{Control Question} & \textbf{Response} & \textbf{Assessment} \\
\midrule
Do you require MFA to access email? & \ding{51} Yes & Control in Place \\
Do you require MFA to log into computers? & \ding{55} No & \textbf{Critical Gap} \\
Do you require MFA to access sensitive data systems? & \ding{55} No & \textbf{Critical Gap} \\
Does your organization have an employee acceptable use policy? & \ding{55} No & \textbf{High Risk} \\
Does your organization do security awareness training for new employees? & \ding{55} No & \textbf{High Risk} \\
Does your organization do security awareness training for all employees at least once per year? & \ding{51} Yes & Control in Place \\
\bottomrule
\end{tabular}
\end{table}

% --- 4. Technical Scan Results ---
\section{Technical Scan Results}
An external network scan was performed to identify open ports and exposed services on the organization's infrastructure.

\begin{itemize}
    \item \textbf{Target IP Address:} \seqsplit{\texttt{2001:db8::1}}
    \item \textbf{Scan Date:} Not provided in scan data.
\end{itemize}

The scan identified the following open port:

\begin{table}[h!]
\centering
\caption{Open Port Analysis}
\begin{tabular}{l l l p{0.4\linewidth}}
\toprule
\textbf{Port} & \textbf{State} & \textbf{Service (Inferred)} & \textbf{Notes} \\
\midrule
22/TCP & Open & SSH & Secure Shell is a common remote management protocol. Exposing it to the public internet without compensating controls (e.g., IP whitelisting, MFA, key-only authentication) is a significant security risk. The specific software version was not identified in this scan. \\
\bottomrule
\end{tabular}
\end{table}

% --- 5. Risk Assessment Summary ---
\section{Risk Assessment Summary}
The following table synthesizes findings from the security control review and the technical scan. No pre-existing vulnerabilities were reported.

\begin{table}[h!]
\centering
\caption{Identified Risks}
\begin{tabular}{p{0.1\linewidth} p{0.25\linewidth} p{0.4\linewidth} p{0.1\linewidth}}
\toprule
\textbf{Risk ID} & \textbf{Risk Name} & \textbf{Description} & \textbf{Severity} \\
\midrule
RISK-001 & Lack of Multi-Factor Authentication & MFA is not enforced on computer logins or for access to sensitive data systems. This exposes the organization to severe risk from credential theft, as a single compromised password could grant an attacker broad access. & \textbf{Critical} \\
\addlinespace
RISK-002 & Inadequate Employee Policies \& Training & The absence of an Acceptable Use Policy and security training for new hires creates a weak human firewall. Employees may be unaware of their responsibilities, making them more susceptible to social engineering and policy violations. & \textbf{High} \\
\addlinespace
RISK-003 & Exposed Management Service (SSH) & Port 22 (SSH) is open on an external IPv6 address. This provides a direct vector for brute-force attacks and exploitation of potential vulnerabilities in the SSH server software. This risk is amplified by the lack of MFA (RISK-001). & \textbf{High} \\
\bottomrule
\end{tabular}
\end{table}

% --- 6. Recommendations ---
\section{Recommendations}
Based on the analysis, the following prioritized actions are recommended to mitigate the identified risks and improve the overall security posture.

\begin{enumerate}
    \item \textbf{[High Priority] Implement Comprehensive MFA:} Immediately deploy and enforce MFA for all employee computer logins and for access to all systems containing sensitive data. This is the single most effective control to mitigate RISK-001.

    \item \textbf{[High Priority] Establish Foundational Policies and Training:}
    \begin{itemize}
        \item Develop and implement a formal Acceptable Use Policy (AUP) that all employees must read and acknowledge.
        \item Integrate mandatory security awareness training into the onboarding process for all new employees before they are granted system access. This directly addresses RISK-002.
    \end{itemize}

    \item \textbf{[Medium Priority] Secure the Exposed SSH Service:}
    \begin{itemize}
        \item Review the business justification for the publicly accessible SSH service on \seqsplit{\texttt{2001:db8::1}}.
        \item If it is not required, block port 22 at the network firewall.
        \item If it is required, restrict access to a whitelist of trusted source IP addresses, enforce public-key authentication, and disable password-based logins.
    \end{itemize}

    \item \textbf{[Medium Priority] Conduct In-Depth Vulnerability Scanning:} Perform a comprehensive, credentialed vulnerability scan of the external infrastructure. This will identify specific software versions and associated vulnerabilities (CVEs) that are not visible in a simple port scan, providing a more complete technical risk picture.
\end{enumerate}

% --- 7. Conclusion ---
\section{Conclusion}
The assessment for \textbf{Bluebird Bio} has identified actionable opportunities to significantly enhance its cybersecurity defenses. While foundational controls are in place for email and annual training, critical weaknesses in authentication, policy, and network configuration present a clear and present risk. By implementing the high-priority recommendations outlined in this report, the organization can substantially reduce its attack surface and build a more resilient security posture.

\end{document}
```