```latex
\documentclass[12pt]{article}

% Required Packages
\usepackage[margin=1in]{geometry}
\usepackage{pifont} % For \ding{51} and \ding{55}
\usepackage{booktabs} % For professional tables
\usepackage{hyperref} % For clickable links
\usepackage{url} % For URL formatting
\usepackage{seqsplit} % For splitting long strings in tt font
\usepackage{graphicx} % For potential logos
\usepackage{xcolor} % For colors

% Document Information
\title{Cybersecurity Posture Assessment Report}
\author{Cybersecurity Analysis Division}
\date{\today}

% Hyperref Setup
\hypersetup{
    colorlinks=true,
    linkcolor=blue,
    filecolor=magenta,      
    urlcolor=cyan,
    pdftitle={Cybersecurity Posture Assessment Report},
    pdfpagemode=FullScreen,
}

\begin{document}

\maketitle
\thispagestyle{empty}
\newpage

\tableofcontents
\newpage

% --- 1. Executive Summary ---
\section{Executive Summary}

This report details the findings of a cybersecurity assessment for \textbf{Astraeus Aerospace}. The assessment incorporated a technical network scan, a review of organizational security controls, and an analysis of pre-existing risks.

The analysis revealed several high-impact vulnerabilities and security control gaps that expose the organization to significant risk. A critical vulnerability was identified on an internal server (\texttt{10.0.0.15}), which is running a dangerously outdated and misconfigured FTP service (\texttt{vsftpd 2.3.4}). This specific version is known to contain a backdoor (CVE-2011-2523), and the service is configured to allow anonymous, unauthenticated access.

Furthermore, significant procedural and policy gaps were identified through the security questionnaire. The lack of Multi-Factor Authentication (MFA) on employee computers, the absence of a formal Acceptable Use Policy, and the failure to provide security training to new hires represent fundamental weaknesses in the organization's defense-in-depth strategy. These issues, combined with the pre-existing risk of outdated Windows 7 workstations, create a high-risk environment.

Immediate remediation of the vulnerable FTP server is paramount. We strongly recommend a comprehensive effort to address the identified policy and control deficiencies to improve the overall security posture.

% --- 2. Organizational Information ---
\section{Organizational Information}

The following details were provided for the assessment.

\begin{tabular}{@{}ll}
\toprule
\textbf{Attribute} & \textbf{Value} \\
\midrule
Organization Name & \textbf{Astraeus Aerospace} \\
Email Domain & \texttt{AstraeusAerospace.com} \\
Website Domain & \texttt{www.AstraeusAerospace.com} \\
External IP Address & \texttt{166.160.225.226} \\
\bottomrule
\end{tabular}

% --- 3. Security Control Review ---
\section{Security Control Review}

A review of administrative and technical security controls was conducted via a standardized questionnaire. The results are summarized below. Answers marked with a red 'X' (\ding{55}) indicate a deviation from security best practices and represent a significant gap in the organization's defenses.

\begin{table}[h!]
\centering
\begin{tabular}{p{0.7\textwidth}c}
\toprule
\textbf{Control Question} & \textbf{Response} \\
\midrule
Do you require MFA to access email? & \ding{51} \\
Do you require MFA to log into computers? & \textcolor{red}{\ding{55}} \\
Do you require MFA to access sensitive data systems? & \ding{51} \\
Does your organization have an employee acceptable use policy? & \textcolor{red}{\ding{55}} \\
Does your organization do security awareness training for new employees? & \textcolor{red}{\ding{55}} \\
Does your organization do security awareness training for all employees at least once per year? & \ding{51} \\
\bottomrule
\end{tabular}
\caption{Security Controls Questionnaire Results}
\end{table}

\subsection*{Analysis of Control Gaps}
\begin{itemize}
    \item \textbf{No MFA for Computer Logins:} This is a high-risk gap. If an employee's credentials are stolen (e.g., through phishing), an attacker can gain direct access to their workstation and the internal network without needing a second factor of authentication.
    \item \textbf{No Acceptable Use Policy (AUP):} An AUP is a foundational document that sets expectations for employee behavior on company networks and systems. Its absence can lead to inconsistent security practices and difficulty in enforcing security rules.
    \item \textbf{No Security Training for New Hires:} New employees are often prime targets for social engineering and phishing attacks. Failing to provide immediate security training upon hiring leaves a critical window of vulnerability.
\end{itemize}

% --- 4. Technical Scan Results ---
\section{Technical Scan Results}

A network scan was performed to identify open ports, running services, and potential vulnerabilities on the specified target system.

\subsection*{Scan Details}
\begin{itemize}
    \item \textbf{Target IP Address:} \texttt{10.0.0.15}
    \item \textbf{Scan Type:} Nmap TCP Port Scan with Service and Script Enumeration
\end{itemize}

\subsection*{Open Ports and Services}
A critical vulnerability was discovered on the target system. The details are outlined in the table below.

\begin{table}[h!]
\centering
\begin{tabular}{llllll}
\toprule
\textbf{Port} & \textbf{State} & \textbf{Service} & \textbf{Product} & \textbf{Version} & \textbf{Notes} \\
\midrule
21/tcp & Open & ftp & vsftpd & 2.3.4 & \begin{tabular}[t]{@{}l@{}}\textbf{CRITICAL:} Anonymous login allowed. \\ This version is vulnerable to a \\ well-known backdoor (CVE-2011-2523).\end{tabular} \\
\bottomrule
\end{tabular}
\caption{Network Scan Findings for \texttt{10.0.0.15}}
\end{table}

% --- 5. Consolidated Risk Assessment ---
\section{Consolidated Risk Assessment}

The following table synthesizes findings from the technical scan, the controls review, and pre-existing risk data to provide a consolidated view of the organization's current risk profile. Risks are prioritized by severity.

\begin{table}[h!]
\centering
\begin{tabular}{p{0.3\textwidth}p{0.12\textwidth}p{0.15\textwidth}p{0.33\textwidth}}
\toprule
\textbf{Risk / Vulnerability} & \textbf{Severity} & \textbf{Source} & \textbf{Description} \\
\midrule
\textbf{Vulnerable FTP Server} & \textbf{Critical} & Network Scan & The server at \texttt{10.0.0.15} is running vsftpd 2.3.4, which contains a known backdoor. Anonymous login is also enabled, allowing unauthenticated access. \\
\addlinespace
\textbf{Lack of Workstation MFA} & High & Questionnaire & The absence of MFA on computer logins exposes the organization to account takeover and unauthorized internal network access if a single password is compromised. \\
\addlinespace
\textbf{Inadequate Security Policies \& Training} & High & Questionnaire & The lack of an AUP and security training for new hires indicates a weak security culture and increases the risk of human error leading to a breach. \\
\addlinespace
\textbf{Outdated Windows Policy} & Medium & Existing Risks & Workstations are running Windows 7, an unsupported OS that no longer receives security updates, making them highly susceptible to known exploits. \\
\bottomrule
\end{tabular}
\caption{Summary of Identified Risks}
\end{table}

% --- 6. Recommendations ---
\section{Recommendations}

The following actions are recommended to mitigate the identified risks, prioritized by severity.

\subsection*{Critical Priority}
\subsubsection*{Remediate Vulnerable FTP Server (10.0.0.15)}
\begin{itemize}
    \item \textbf{Immediate Action:} Disable the FTP service on this server immediately to prevent exploitation. Take the server offline if its function is not business-critical.
    \item \textbf{Short-Term Fix:} If the service is required, upgrade \texttt{vsftpd} to the latest stable version and reconfigure it to disable anonymous access and enforce strong authentication.
    \item \textbf{Long-Term Strategy:} Migrate all file transfer processes to a secure protocol such as SFTP (SSH File Transfer Protocol) or FTPS (FTP over SSL/TLS). Decommission legacy FTP entirely.
\end{itemize}

\subsection*{High Priority}
\subsubsection*{Implement Multi-Factor Authentication (MFA)}
\begin{itemize}
    \item Enforce MFA for all employee logins to company-managed computers and laptops. This adds a critical layer of security to prevent unauthorized access from compromised credentials.
\end{itemize}

\subsubsection*{Develop and Implement Foundational Security Policies}
\begin{itemize}
    \item \textbf{Acceptable Use Policy (AUP):} Create and enforce an AUP that clearly defines the rules for using company IT assets, including data handling, internet usage, and password requirements.
    \item \textbf{New Hire Training:} Institute a mandatory security awareness training module as part of the new employee onboarding process. This should be completed before a new hire is granted access to sensitive systems.
\end{itemize}

\subsection*{Medium Priority}
\subsubsection*{Address Outdated Windows 7 Workstations}
\begin{itemize}
    \item Continue with the existing plan to upgrade all Windows 7 machines to a modern, supported operating system (e.g., Windows 10 or 11). This project should be prioritized to reduce the attack surface of the internal network.
\end{itemize}

\end{document}
```