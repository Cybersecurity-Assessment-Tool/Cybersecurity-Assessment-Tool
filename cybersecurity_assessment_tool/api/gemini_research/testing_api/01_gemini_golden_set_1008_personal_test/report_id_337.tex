```latex
\documentclass[12pt]{article}

% Preamble: Required Packages
\usepackage[margin=1in]{geometry}
\usepackage{pifont} % For checkmarks and crosses (\ding)
\usepackage{booktabs} % For professional-looking tables
\usepackage{hyperref} % For clickable links
\usepackage{url}      % For proper URL formatting
\usepackage{seqsplit} % To split long strings without breaking words
\usepackage{graphicx} % For potential logo inclusion
\usepackage{fancyhdr} % For headers and footers

% Document Metadata and Hyperlink Setup
\hypersetup{
    colorlinks=true,
    linkcolor=black,
    urlcolor=blue,
    pdftitle={Cybersecurity Posture Assessment Report},
    pdfauthor={Cybersecurity Analysis Division},
    pdfsubject={Security Report},
    pdfkeywords={Cybersecurity, Risk Assessment, Nmap, Security Controls}
}

% Define custom commands for consistency
\newcommand{\yes}{\ding{51}}
\newcommand{\no}{\ding{55}}
\newcommand{\orgname}{\textbf{Ironclad Logistics}}
\newcommand{\orgdomain}{\texttt{IroncladLogistics.com}}
\newcommand{\orgip}{\texttt{34.31.60.252}}
\newcommand{\targetip}{\texttt{192.168.10.5}}

% Header and Footer Configuration
\pagestyle{fancy}
\fancyhf{} % Clear all header and footer fields
\fancyhead[L]{\orgname{} - Confidential}
\fancyhead[R]{Security Assessment Report}
\fancyfoot[C]{\thepage}
\renewcommand{\headrulewidth}{0.4pt}
\renewcommand{\footrulewidth}{0.4pt}

% --- Document Start ---
\begin{document}

\begin{titlepage}
    \centering
    \vspace*{1cm}
    \Huge \textbf{Cybersecurity Posture Assessment Report}
    \vspace{1.5cm}
    \Large \textbf{Prepared for:} \\
    \vspace{0.5cm}
    \Large \orgname{}
    \vspace{2cm}
    \rule{0.8\textwidth}{0.4pt}
    \vspace{0.5cm}
    \textbf{Date of Report:} \today \\
    \textbf{Date of Assessment:} 2025-11-22
    \rule{0.8\textwidth}{0.4pt}
    \vfill
    \small \textit{This document contains sensitive information and is intended solely for the use of the designated recipient. Unauthorized distribution is strictly prohibited.}
\end{titlepage}

\tableofcontents
\newpage

\section{Executive Summary}

This report details the findings of a cybersecurity posture assessment conducted for \orgname{} on November 22, 2025. The assessment combined a review of organizational security controls, an external network scan, and an analysis of pre-existing risks.

The overall security posture requires immediate attention. Key findings include a critical technical vulnerability and a significant gap in the organization's security awareness program.

\begin{itemize}
    \item \textbf{High-Risk Technical Finding:} The external-facing web server at \targetip{} is running an outdated version of Nginx (\texttt{1.18.0}). This version, released in 2020, has multiple publicly disclosed vulnerabilities (CVEs) that could be exploited by attackers to compromise the server.
    \item \textbf{High-Risk Organizational Finding:} The organization does not conduct mandatory annual security awareness training for all employees. This significantly increases the risk of successful phishing, social engineering, and other human-centric attacks.
\end{itemize}

Immediate remediation of these issues is strongly recommended to reduce the organization's attack surface and strengthen its defense against common cyber threats. Detailed recommendations are provided in Section \ref{sec:recommendations}.

\section{Organizational Information}

The following information was provided for the assessment.
\begin{table}[h!]
\centering
\caption{Client Organizational Data}
\begin{tabular}{@{}ll@{}}
\toprule
\textbf{Attribute} & \textbf{Value} \\ \midrule
Organization Name & \orgname{} \\
Email Domain & \orgdomain{} \\
Website Domain & \seqsplit{\texttt{www.IroncladLogistics.com}} \\
External IP Address & \orgip{} \\ \bottomrule
\end{tabular}
\end{table}

\section{Security Control Review}

A review of administrative and technical security controls was conducted based on a standardized questionnaire. The results indicate a strong foundation in access control but reveal a critical weakness in ongoing employee security education.

\begin{table}[h!]
\centering
\caption{Security Controls Questionnaire Results}
\label{tab:controls}
\begin{tabular}{@{}p{0.8\linewidth}c@{}}
\toprule
\textbf{Control Question} & \textbf{Status} \\ \midrule
Do you require MFA to access email? & \yes \\
Do you require MFA to log into computers? & \yes \\
Do you require MFA to access sensitive data systems? & \yes \\
Does your organization have an employee acceptable use policy? & \yes \\
Does your organization do security awareness training for new employees? & \yes \\
\textbf{Does your organization do security awareness training for all employees at least once per year?} & \no \\ \bottomrule
\end{tabular}
\end{table}

The failure to provide recurring annual security training for all staff is a significant gap. While onboarding training is a good first step, the threat landscape evolves continuously, and so must employee awareness.

\section{Technical Scan Results}

An Nmap scan was performed against the target host \targetip{} on November 22, 2025. The scan identified one open port running a web server with an outdated software version.

\begin{table}[h!]
\centering
\caption{Nmap Scan Findings for \targetip{}}
\label{tab:nmap}
\begin{tabular}{@{}llllll@{}}
\toprule
\textbf{Port} & \textbf{State} & \textbf{Service} & \textbf{Product} & \textbf{Version} \\ \midrule
443/tcp & open & https & nginx & \texttt{1.18.0} \\ \bottomrule
\end{tabular}
\end{table}

\subsection{Analysis of Findings}
The service identified on port 443 is \textbf{Nginx version 1.18.0}. This is a mainline version released in April 2020. Since its release, numerous security vulnerabilities have been discovered and patched in subsequent versions. Running this outdated software exposes the server to risks such as request smuggling, denial-of-service, and other potential exploits that could lead to a system compromise.

\section{Consolidated Risk Assessment}

The following table synthesizes findings from the security control review and the technical scan. No pre-existing risks were provided for this assessment.

\begin{table}[h!]
\centering
\caption{Identified Risks and Severity}
\label{tab:risks}
\begin{tabular}{@{}p{0.1\linewidth}p{0.25\linewidth}p{0.45\linewidth}l@{}}
\toprule
\textbf{Risk ID} & \textbf{Risk Name} & \textbf{Description} & \textbf{Severity} \\ \midrule
RISK-001 & Outdated Web Server Software & The web server at \targetip{} is running Nginx \texttt{1.18.0}, a version with multiple known public vulnerabilities (CVEs). & \textbf{High} \\
\addlinespace
RISK-002 & Insufficient Security Awareness Training & The lack of mandatory, recurring annual training for all employees increases organizational susceptibility to phishing and social engineering attacks. & \textbf{High} \\ \bottomrule
\end{tabular}
\end{table}

\section{Recommendations}
\label{sec:recommendations}

To mitigate the identified risks, the following actions are recommended with high priority.

\begin{enumerate}
    \item \textbf{Upgrade Nginx Web Server (RISK-001):}
    \begin{itemize}
        \item \textbf{Action:} Plan and execute an upgrade of the Nginx server on host \targetip{} from version \texttt{1.18.0} to the latest stable version.
        \item \textbf{Justification:} Upgrading will patch known vulnerabilities, protecting the server and its data from potential exploitation. This is a critical step in reducing the external attack surface.
        \item \textbf{Timeline:} Immediate (within 30 days).
    \end{itemize}
    \vspace{0.5cm}
    \item \textbf{Implement Annual Security Training (RISK-002):}
    \begin{itemize}
        \item \textbf{Action:} Procure and implement a mandatory security awareness training program for all employees, to be completed annually. The program should cover modern threats such as phishing, ransomware, and social engineering.
        \item \textbf{Justification:} A well-trained workforce is the first line of defense. Regular training ensures that security remains a top-of-mind concern and equips employees to recognize and report suspicious activity.
        \item \textbf{Timeline:} Within 90 days.
    \end{itemize}
\end{enumerate}

\end{document}
```