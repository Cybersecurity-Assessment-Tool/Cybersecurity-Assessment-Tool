```latex
\documentclass[12pt, a4paper]{article}

% Preamble: Required Packages
\usepackage[margin=1in]{geometry}
\usepackage{pifont} % For checkmarks and crosses
\usepackage{booktabs} % For professional tables
\usepackage{hyperref} % For clickable links
\usepackage{url} % For formatting URLs
\usepackage{seqsplit} % For splitting long strings in tt font
\usepackage{graphicx}
\usepackage[table]{xcolor} % For coloring table cells
\usepackage{fancyhdr} % For custom headers/footers
\usepackage{lastpage} % To get the total number of pages

% --- Document Setup ---

% Define colors for risk levels
\definecolor{criticalred}{HTML}{D10000}
\definecolor{highorange}{HTML}{E97400}
\definecolor{mediumyellow}{HTML}{FFBF00}
\definecolor{lowgreen}{HTML}{008000}

% Hyperref setup
\hypersetup{
    colorlinks=true,
    linkcolor=blue,
    filecolor=magenta,      
    urlcolor=cyan,
    pdftitle={Cybersecurity Posture Report},
    pdfauthor={Cybersecurity Analyst},
    pdfsubject={Security Assessment},
    pdfkeywords={Cybersecurity, Risk, Assessment},
}

% Header and Footer Configuration
\pagestyle{fancy}
\fancyhf{} % Clear all header and footer fields
\fancyhead[L]{Cybersecurity Posture Report}
\fancyhead[R]{Sovereign Trust}
\fancyfoot[C]{\thepage\ of \pageref{LastPage}}
\renewcommand{\headrulewidth}{0.4pt}
\renewcommand{\footrulewidth}{0.4pt}

% --- Document Start ---
\begin{document}

% --- Title Page ---
\begin{titlepage}
    \centering
    \vspace*{2cm}
    
    {\Huge \textbf{Cybersecurity Posture Report}\par}
    \vspace{1.5cm}
    
    {\Large \textbf{Prepared for:}\par}
    \vspace{0.5cm}
    {\Large Sovereign Trust\par}
    
    \vfill % Pushes content to the bottom
    
    {\large \today\par}
    \vspace{0.5cm}
    {\large Confidential\par}
    
\end{titlepage}

\newpage
\tableofcontents
\newpage

% --- Section 1: Executive Summary ---
\section{Executive Summary}

This report provides a comprehensive analysis of the cybersecurity posture of \textbf{Sovereign Trust}, based on a correlation of technical network scans, a security controls questionnaire, and a review of pre-existing risks.

The assessment reveals a high-risk environment characterized by critical vulnerabilities and significant gaps in foundational security controls. Key findings include:

\begin{itemize}
    \item \textbf{Critical Database Exposure:} An externally accessible MySQL database was identified on an internal IP address (\texttt{172.16.50.20}). This database is running version \texttt{5.7.33}, which is End-of-Life (EOL) and no longer receives security updates, making it an easy target for known exploits.
    \item \textbf{Weak Access Controls:} Multi-Factor Authentication (MFA) is not enforced for email access. This represents a critical vulnerability, as compromised credentials could lead to a full email account takeover, data breaches, and further internal network compromise via phishing.
    \item \textbf{Policy and Training Deficiencies:} The organization lacks a formal Acceptable Use Policy (AUP) and does not provide security awareness training for new employees. These gaps increase the risk of insider threats and susceptibility to social engineering attacks.
\end{itemize}

Immediate remediation is required to address the database exposure and enforce MFA. Strategic initiatives must be undertaken to develop missing policies and integrate security training into the employee onboarding process to mitigate these risks and improve the organization's overall security resilience.

\newpage

% --- Section 2: Organizational Information ---
\section{Organizational Information}
The following details were provided for the assessment.

\begin{tabular}{@{}ll}
    \toprule
    \textbf{Attribute} & \textbf{Value} \\
    \midrule
    Organization Name & \textbf{Sovereign Trust} \\
    Email Domain & \texttt{SovereignTrust.com} \\
    Website Domain & \seqsplit{\texttt{www.SovereignTrust.com}} \\
    External IP Address & \texttt{49.244.132.43} \\
    \bottomrule
\end{tabular}

% --- Section 3: Security Control Review ---
\section{Security Control Review}
A review of the organization's security controls was conducted via a questionnaire. The responses indicate several critical gaps in security best practices. A summary of the findings is presented in Table \ref{tab:controls}.

\begin{table}[h!]
    \centering
    \caption{Security Controls Questionnaire Results}
    \label{tab:controls}
    \begin{tabular}{@{}p{0.7\textwidth}cc@{}}
        \toprule
        \textbf{Control Question} & \textbf{Response} & \textbf{Status} \\
        \midrule
        Do you require MFA to access email? & No & \ding{55} \\
        Do you require MFA to log into computers? & Yes & \ding{51} \\
        Do you require MFA to access sensitive data systems? & Yes & \ding{51} \\
        Does your organization have an employee acceptable use policy? & No & \ding{55} \\
        Does your organization do security awareness training for new employees? & No & \ding{55} \\
        Does your organization do security awareness training for all employees at least once per year? & Yes & \ding{51} \\
        \bottomrule
    \end{tabular}
\end{table}

\subsection*{Analysis of Gaps}
The "No" responses highlight significant areas of risk:
\begin{itemize}
    \item \textbf{No MFA for Email:} Email is a primary target for attackers. The lack of MFA makes business email compromise (BEC), phishing, and data exfiltration significantly easier for an adversary with stolen credentials.
    \item \textbf{No Acceptable Use Policy (AUP):} Without an AUP, there are no clear, enforceable rules for employees regarding the use of company assets. This creates legal and security ambiguities.
    \item \textbf{No New Employee Security Training:} New hires are often targeted by attackers. Failing to provide immediate security training leaves a critical window of vulnerability.
\end{itemize}

\newpage

% --- Section 4: Technical Scan Results ---
\section{Technical Scan Results}
A network scan was performed to identify open ports and services on the target system.

\begin{itemize}
    \item \textbf{Target IP Address:} \texttt{172.16.50.20}
\end{itemize}

The scan identified one open port, detailed in Table \ref{tab:scan}.

\begin{table}[h!]
    \centering
    \caption{Open Port Scan Findings}
    \label{tab:scan}
    \begin{tabular}{@{}llll@{}}
        \toprule
        \textbf{Port} & \textbf{State} & \textbf{Service} & \textbf{Product \& Version} \\
        \midrule
        3306/tcp & Open & mysql & MySQL 5.7.33 \\
        \bottomrule
    \end{tabular}
\end{table}

\subsection*{Technical Analysis}
The primary finding is the open MySQL port (\texttt{3306}). This directly confirms the pre-existing risk of "Database Exposure." More critically, the identified version, \textbf{MySQL 5.7.33}, reached its official End-of-Life (EOL) in October 2023. EOL software no longer receives security patches from the vendor, meaning any vulnerabilities discovered after this date will remain unpatched. This elevates the risk of compromise from "High" to "Critical," as public exploits may exist for this version.

% --- Section 5: Consolidated Risk Assessment ---
\section{Consolidated Risk Assessment}
The following table synthesizes findings from the security questionnaire, technical scan, and pre-existing risk data into a consolidated list of current risks.

\begin{table}[h!]
    \centering
    \caption{Summary of Identified Risks}
    \label{tab:risks}
    \renewcommand{\arraystretch}{1.5}
    \begin{tabular}{@{}p{0.15\textwidth}p{0.55\textwidth}p{0.2\textwidth}@{}}
        \toprule
        \textbf{Risk Name} & \textbf{Description} & \textbf{Severity} \\
        \midrule
        \textbf{Outdated Database Software} & The MySQL server is running version 5.7.33, which is End-of-Life and is no longer supported with security patches. & \colorbox{criticalred!80}{\color{white}\textbf{\phantom{..}Critical\phantom{..}}} \\
        \hline
        \textbf{Lack of MFA on Email} & The absence of MFA on email accounts exposes the organization to a high risk of account takeover and business email compromise. & \colorbox{criticalred!80}{\color{white}\textbf{\phantom{..}Critical\phantom{..}}} \\
        \hline
        \textbf{Database Exposure} & The MySQL database port (3306) is open to the network, allowing direct connection attempts from unauthorized sources. & \colorbox{highorange!90}{\color{white}\textbf{\phantom{....}High\phantom{....}}} \\
        \hline
        \textbf{Missing Acceptable Use Policy} & Lack of a formal AUP creates ambiguity for employees and weakens the organization's ability to enforce security standards. & \colorbox{highorange!90}{\color{white}\textbf{\phantom{....}High\phantom{....}}} \\
        \hline
        \textbf{Inadequate Employee Onboarding} & New employees are not provided with security awareness training, making them more susceptible to phishing and social engineering. & \colorbox{highorange!90}{\color{white}\textbf{\phantom{....}High\phantom{....}}} \\
        \bottomrule
    \end{tabular}
\end{table}

\newpage

% --- Section 6: Recommendations ---
\section{Recommendations}
The following prioritized recommendations are provided to address the identified risks and improve the overall security posture of \textbf{Sovereign Trust}.

\subsection*{Immediate Actions (Critical Priority)}
\begin{enumerate}
    \item \textbf{Restrict Database Access:} Immediately implement firewall rules to restrict access to TCP port 3306 on server \texttt{172.16.50.20}. Access should only be permitted from specific, authorized application servers. Public or wide-network access must be disabled.
    \item \textbf{Enforce MFA for Email:} Immediately enable and enforce Multi-Factor Authentication for all user email accounts. This is the single most effective control to prevent account takeovers.
\end{enumerate}

\subsection*{High Priority Actions}
\begin{enumerate}
    \setcounter{enumi}{2} % Continue numbering from previous list
    \item \textbf{Upgrade Database Server:} Plan and execute the upgrade of the MySQL 5.7 server to a currently supported version (e.g., MySQL 8.0 or later). This is essential to ensure the system receives ongoing security patches.
    \item \textbf{Develop and Implement an AUP:} Draft, approve, and disseminate an Acceptable Use Policy that clearly defines the rules for using company technology and data. Require all employees to read and acknowledge the policy.
    \item \textbf{Integrate Security into Onboarding:} Develop a security awareness training module and make it a mandatory component of the new employee onboarding process. This training should cover key topics such as phishing, password security, and the new AUP.
\end{enumerate}

\end{document}
```