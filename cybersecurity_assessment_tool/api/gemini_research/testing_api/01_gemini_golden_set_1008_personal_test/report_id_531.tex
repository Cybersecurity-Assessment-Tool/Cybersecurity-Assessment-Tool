```latex
\documentclass[12pt]{article}

% --- PACKAGES ---
\usepackage[margin=1in]{geometry} % Set page margins
\usepackage{pifont}               % For checkmarks and crosses (\ding)
\usepackage{booktabs}             % For professional-looking tables
\usepackage[hidelinks]{hyperref}  % For hyperlinks without ugly boxes
\usepackage{url}                  % For URL formatting
\usepackage{seqsplit}             % To split long strings in texttt
\usepackage{graphicx}             % For logos, etc. (optional but good practice)
\usepackage{xcolor}               % For colors

% --- DOCUMENT DEFINITIONS ---
\newcommand{\yes}{\ding{51}}
\newcommand{\no}{\ding{55}}
\definecolor{darkred}{rgb}{0.55, 0.0, 0.0}
\definecolor{darkorange}{rgb}{1.0, 0.55, 0.0}

% --- DOCUMENT START ---
\begin{document}

% --- TITLE PAGE ---
\begin{titlepage}
    \centering
    \vspace*{1cm}
    \Huge\textbf{Cybersecurity Posture Assessment Report}
    \vspace{1.5cm}
    \Large\textbf{Prepared for:} \\
    \vspace{0.5cm}
    \huge Structure \& Form
    \vfill
    \large
    \textbf{Date of Report:} \today \\
    \textbf{Report ID:} CSA-2024-08-01
    \vspace{1.5cm}
    \large
    \textbf{Generated by:} \\
    Cybersecurity Analysis Division
\end{titlepage}

\tableofcontents
\newpage

% --- EXECUTIVE SUMMARY ---
\section{Executive Summary}

This report provides a comprehensive analysis of the cybersecurity posture for \textbf{Structure \& Form}, based on network scans, a security controls questionnaire, and a review of existing risks. The assessment identified two critical areas of concern that significantly elevate the organization's risk profile.

First, a network scan revealed an exposed MySQL database service (\texttt{172.16.50.20:3306}). This service is running an outdated and unsupported version (MySQL 5.7.33), which reached its End-of-Life in October 2023. This exposes the organization to numerous known vulnerabilities that will not be patched, presenting a \textbf{Critical} risk of data breach.

Second, the security controls review highlighted a critical gap in endpoint security: the absence of mandatory Multi-Factor Authentication (MFA) for computer logins. This lack of a fundamental security control creates a high-risk environment for credential theft, unauthorized access, and lateral movement within the network.

Immediate remediation of these findings is strongly recommended to reduce the likelihood of a significant security incident.

% --- ORGANIZATIONAL INFORMATION ---
\section{Organizational Information}

The following information was provided for the assessment.

\begin{tabular}{@{}ll}
    \toprule
    \textbf{Attribute} & \textbf{Value} \\
    \midrule
    Organization Name & Structure \& Form \\
    Email Domain & \texttt{StructureForm.net} \\
    Website Domain & \seqsplit{\url{www.StructureForm.net}} \\
    External IP Address & \texttt{229.239.103.9} \\
    \bottomrule
\end{tabular}

% --- SECURITY CONTROL REVIEW ---
\section{Security Control Review}

A review of the organization's security controls was conducted via a questionnaire. The results are summarized below. A green checkmark (\yes) indicates a positive control is in place, while a red cross (\no) indicates a control gap.

\begin{table}[h!]
\centering
\begin{tabular}{@{}p{0.8\textwidth}c@{}}
    \toprule
    \textbf{Security Control Question} & \textbf{Status} \\
    \midrule
    Do you require MFA to access email? & \yes \\
    Do you require MFA to log into computers? & \textcolor{darkred}{\no} \\
    Do you require MFA to access sensitive data systems? & \yes \\
    Does your organization have an employee acceptable use policy? & \yes \\
    Does your organization do security awareness training for new employees? & \yes \\
    Does your organization do security awareness training for all employees at least once per year? & \yes \\
    \bottomrule
\end{tabular}
\caption{Security Controls Questionnaire Results}
\end{table}

\subsection*{Analysis}
The organization has implemented several key security controls, including MFA for email and sensitive systems, as well as a robust security awareness program. However, the lack of MFA for computer logins is a \textbf{critical security gap}. This weakness significantly increases the risk of unauthorized access to endpoints, which are often the starting point for more extensive network compromises. An attacker with valid user credentials could bypass primary defenses and gain an initial foothold within the internal network.

% --- TECHNICAL SCAN RESULTS ---
\section{Technical Scan Results}

An Nmap scan was performed on the target system to identify open ports and running services.

\begin{itemize}
    \item \textbf{Target IP Address:} \texttt{172.16.50.20}
    \item \textbf{Host Status:} Up
\end{itemize}

\begin{table}[h!]
\centering
\begin{tabular}{@{}lllll@{}}
    \toprule
    \textbf{Port} & \textbf{State} & \textbf{Service} & \textbf{Product} & \textbf{Version} \\
    \midrule
    3306/tcp & Open & mysql & MySQL & 5.7.33 \\
    \bottomrule
\end{tabular}
\caption{Open Ports and Services Detected}
\end{table}

\subsection*{Analysis}
The scan confirms that port \texttt{3306} is open, exposing a MySQL database service directly to the network. The identified version, \textbf{MySQL 5.7.33}, is a significant finding. The MySQL 5.7 series reached its official End-of-Life (EOL) in October 2023. This means the software no longer receives security updates, bug fixes, or technical support from the vendor. Running EOL software, especially for a critical database service, exposes the system to a wide range of publicly known and unpatched vulnerabilities. This finding validates and elevates the severity of the pre-existing "Database Exposure" risk.

% --- RISK ASSESSMENT SUMMARY ---
\section{Risk Assessment Summary}

The following table synthesizes findings from the security questionnaire, technical scans, and pre-existing risk data into a prioritized list.

\begin{table}[h!]
\centering
\begin{tabular}{@{}p{0.1\textwidth}p{0.25\textwidth}p{0.15\textwidth}p{0.4\textwidth}@{}}
    \toprule
    \textbf{Risk ID} & \textbf{Risk Name} & \textbf{Severity} & \textbf{Description} \\
    \midrule
    CSA-001 & Exposed \& Outdated Database Service & \textbf{\textcolor{darkred}{Critical (9.1)}} & A MySQL database on an unsupported, End-of-Life version (5.7.33) is directly exposed to the network. This allows attackers to exploit known, unpatched vulnerabilities to achieve remote code execution or data exfiltration. \\
    \addlinespace
    CSA-002 & Lack of Endpoint Multi-Factor Authentication & \textbf{\textcolor{darkorange}{High (8.0)}} & The absence of MFA for computer logins means that a compromised password is all an attacker needs to gain access to an employee's workstation, from which they can escalate privileges and move laterally through the network. \\
    \bottomrule
\end{tabular}
\caption{Synthesized Risk Register}
\end{table}

% --- RECOMMENDATIONS ---
\section{Recommendations}

The following actions are recommended to mitigate the identified risks.

\subsection{CSA-001: Exposed \& Outdated Database Service (Critical)}
\begin{itemize}
    \item \textbf{Immediate (Containment):} Implement strict firewall rules to restrict all access to port \texttt{3306} on \texttt{172.16.50.20}. Access should only be permitted from specific, authorized application server IP addresses. Public access must be disabled immediately.
    \item \textbf{Short-Term (Remediation):} Initiate a project to upgrade the MySQL 5.7 database to a currently supported version (e.g., MySQL 8.x). This is crucial to ensure the service receives ongoing security patches.
    \item \textbf{Long-Term (Strategic):} Evaluate a Zero Trust network architecture where database access is not based on network location. Access should be brokered through a secure, authenticated proxy or VPN connection, eliminating direct exposure.
\end{itemize}

\subsection{CSA-002: Lack of Endpoint MFA (High)}
\begin{itemize}
    \item \textbf{Immediate (Prioritization):} Procure and deploy an MFA solution for endpoint logins (e.g., Windows Hello for Business, Duo, etc.). Begin a phased rollout, prioritizing users with administrative or privileged access to critical systems.
    \item \textbf{Short-Term (Implementation):} Expand the MFA rollout to all employees across the organization.
    \item \textbf{Long-Term (Policy):} Update the organization's access control policy to mandate the use of MFA for all computer and system logins as a baseline security requirement.
\end{itemize}

\end{document}
```