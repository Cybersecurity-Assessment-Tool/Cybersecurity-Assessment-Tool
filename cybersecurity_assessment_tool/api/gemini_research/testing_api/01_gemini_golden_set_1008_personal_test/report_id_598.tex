```latex
\documentclass[12pt]{article}

% Preamble: Required Packages
\usepackage[margin=1in]{geometry} % For setting page margins
\usepackage{pifont}               % For using dingbats like checkmarks and crosses
\usepackage{booktabs}             % For professional-looking tables
\usepackage{hyperref}             % For hyperlinks and document metadata
\usepackage{url}                  % For formatting URLs
\usepackage{seqsplit}             % For splitting long strings in texttt
\usepackage{graphicx}             % For including logos (optional, but good practice)
\usepackage{xcolor}               % For custom colors

% Define colors for severity
\definecolor{criticalred}{HTML}{D73B3E}
\definecolor{highorange}{HTML}{F58426}
\definecolor{mediumyellow}{HTML}{FFC842}
\definecolor{lowblue}{HTML}{3498DB}

% Hyperref setup
\hypersetup{
    colorlinks=true,
    linkcolor=blue,
    filecolor=magenta,      
    urlcolor=cyan,
    pdftitle={Cybersecurity Posture Assessment Report},
    pdfauthor={Cybersecurity Analyst},
    pdfsubject={Security Analysis},
    pdfkeywords={Cybersecurity, Risk Assessment, Network Scan},
    bookmarks=true
}

% Document Start
\begin{document}

% --- Title Page ---
\begin{titlepage}
    \centering
    \vspace*{1cm}
    
    \Huge
    \textbf{Cybersecurity Posture Assessment Report}
    
    \vspace{1.5cm}
    
    \Large
    Prepared for: \\
    \vspace{0.5cm}
    \textbf{Structure \& Form}
    
    \vfill
    
    \Large
    \today
    
    \vspace{1.5cm}
    
    \normalsize
    Generated by: Cybersecurity Analyst
\end{titlepage}

% --- Table of Contents ---
\tableofcontents
\newpage

% --- Executive Overview ---
\section*{Executive Overview}
This report provides a comprehensive cybersecurity assessment for \textbf{Structure \& Form}, synthesized from a network vulnerability scan, an organizational security questionnaire, and a review of pre-existing risks. The analysis reveals a mixed security posture. While foundational controls such as Multi-Factor Authentication (MFA) for email and computer access are in place, significant gaps exist in policy and procedural controls.

Key findings indicate critical vulnerabilities related to inconsistent MFA application, the absence of a formal Acceptable Use Policy (AUP), and a lack of recurring security awareness training for all employees. These gaps expose the organization to increased risks of unauthorized access, data breaches, and insider threats.

The external network scan of the target IP address \texttt{[Target IP]} did not identify any open ports, which is a positive indicator of a strong firewall configuration at the network perimeter. However, the internal policy and procedural weaknesses identified are of high concern and require immediate attention. This report outlines these risks in detail and provides prioritized, actionable recommendations to mitigate them and strengthen the organization's overall security posture.

% --- Organizational Information ---
\section*{Organizational Information}
The following details were provided by the client and used as a baseline for this assessment.

\begin{table}[h!]
\centering
\begin{tabular}{@{}ll@{}}
\toprule
\textbf{Attribute} & \textbf{Value} \\
\midrule
Organization Name & \textbf{Structure \& Form} \\
Email Domain      & \texttt{StructureForm.net} \\
Website Domain    & \texttt{www.StructureForm.net} \\
External IP Address & \texttt{18.20.187.133} \\
\bottomrule
\end{tabular}
\caption{Client Organizational Details}
\end{table}

% --- Security Control Review ---
\section*{Security Control Review}
A security questionnaire was completed to evaluate existing administrative and procedural controls. The results below highlight both strengths and critical areas for improvement. Answers marked with \ding{55} represent significant gaps in the current security framework.

\begin{table}[h!]
\centering
\begin{tabular}{@{}p{0.8\textwidth}c@{}}
\toprule
\textbf{Question / Control} & \textbf{Status} \\
\midrule
Do you require MFA to access email? & \textcolor{green}{\ding{51}} \\
Do you require MFA to log into computers? & \textcolor{green}{\ding{51}} \\
Do you require MFA to access sensitive data systems? & \textcolor{criticalred}{\ding{55}} \\
Does your organization have an employee acceptable use policy? & \textcolor{criticalred}{\ding{55}} \\
Does your organization do security awareness training for new employees? & \textcolor{green}{\ding{51}} \\
Does your organization do security awareness training for all employees at least once per year? & \textcolor{criticalred}{\ding{55}} \\
\bottomrule
\end{tabular}
\caption{Security Controls Questionnaire Results}
\end{table}

% --- Technical Scan Results ---
\section*{Technical Scan Results}
An external network scan was performed to identify potential vulnerabilities visible from the public internet.

\begin{itemize}
    \item \textbf{Target IP Address:} \texttt{[Target IP]}
    \item \textbf{Scan Date:} \today
\end{itemize}

\subsection*{Summary of Findings}
The scan completed successfully and found \textbf{no open ports or exposed services} on the target system. This indicates a robust firewall policy and a strong network perimeter security posture, effectively minimizing the external attack surface. No vulnerabilities were identified from this scan.

% --- Identified Risks and Assessment ---
\section*{Identified Risks and Assessment}
This section correlates findings from the security control review, technical scan, and pre-existing risk data. The primary risks identified are procedural and policy-based, originating from the security questionnaire. No pre-existing risks were reported and no technical vulnerabilities were discovered during the external scan.

\begin{table}[h!]
\centering
\begin{tabular}{@{}p{0.25\linewidth}p{0.5\linewidth}p{0.15\linewidth}@{}}
\toprule
\textbf{Risk Name} & \textbf{Overview} & \textbf{Severity} \\
\midrule
\textbf{Incomplete MFA Coverage} & MFA is not enforced for accessing sensitive data systems. This creates a critical vulnerability, as a single compromised password could lead to a major data breach. & \textcolor{criticalred}{\textbf{Critical}} \\
\addlinespace
\textbf{No Acceptable Use Policy (AUP)} & The absence of a formal AUP means there are no documented rules for employees regarding the use of company assets, data handling, and online behavior, increasing the risk of insider threats and misuse. & \textcolor{highorange}{\textbf{High}} \\
\addlinespace
\textbf{Lack of Annual Security Training} & While new hires receive training, there is no recurring annual training for all staff. This leads to knowledge decay and an inability to adapt to new and evolving cyber threats, making employees more susceptible to phishing and social engineering. & \textcolor{highorange}{\textbf{High}} \\
\bottomrule
\end{tabular}
\caption{Summary of Identified Risks}
\end{table}

% --- Recommendations ---
\section*{Recommendations}
Based on the risk assessment, the following prioritized actions are recommended to enhance the cybersecurity posture of \textbf{Structure \& Form}.

\subsection*{Priority 1: Critical}
\begin{enumerate}
    \item \textbf{Enforce MFA on Sensitive Systems:} Immediately deploy and enforce Multi-Factor Authentication (MFA) across all systems classified as containing sensitive or critical data. This is the single most effective control to prevent unauthorized access resulting from credential compromise.
\end{enumerate}

\subsection*{Priority 2: High}
\begin{enumerate}
    \setcounter{enumi}{1}
    \item \textbf{Develop and Implement an Acceptable Use Policy (AUP):} Create a comprehensive AUP that clearly defines the rules and responsibilities for all employees when using company technology and data. This policy should be formally acknowledged by all staff members upon implementation and during onboarding.
    \item \textbf{Establish a Mandatory Annual Security Awareness Program:} Institute a mandatory security awareness training program for all employees, to be completed at least once per year. The training should cover current threats such as phishing, ransomware, social engineering, and secure data handling practices.
\end{enumerate}

\end{document}
```