```latex
\documentclass[12pt, a4paper]{article}

% Preamble: Required Packages
\usepackage[margin=1in]{geometry}
\usepackage{pifont} % For checkmarks and crosses
\usepackage{booktabs} % For professional tables
\usepackage{hyperref} % For clickable links and better PDF navigation
\usepackage{url} % For formatting URLs
\usepackage{seqsplit} % To split long monospaced strings
\usepackage{graphicx} % For potential logos
\usepackage{fancyhdr} % For headers and footers

% --- Document Setup ---
\hypersetup{
    colorlinks=true,
    linkcolor=black,
    filecolor=magenta,      
    urlcolor=blue,
    pdftitle={Cybersecurity Assessment Report},
    pdfauthor={Cybersecurity Analyst},
    pdfsubject={Security Posture Analysis},
    pdfkeywords={Security, Risk, Assessment, Nmap, RDP},
    bookmarks=true
}

\pagestyle{fancy}
\fancyhf{}
\fancyhead[L]{Cybersecurity Assessment Report}
\fancyhead[R]{Pioneer Pulse}
\fancyfoot[C]{\thepage}

% --- Document Body ---
\begin{document}

% --- Title Page ---
\begin{titlepage}
    \centering
    \vspace*{1cm}
    \Huge{\textbf{Cybersecurity Assessment Report}}
    \vspace{1.5cm}
    \Large{\textbf{Prepared for:}} \\
    \vspace{0.5cm}
    \Large{Pioneer Pulse}
    \vfill
    \large{\textbf{Date of Report:}} \\
    \vspace{0.2cm}
    \large{\today}
    \vspace{1.5cm}
    \large{\textbf{Generated By:}} \\
    \vspace{0.2cm}
    \large{Expert Cybersecurity Analyst}
\end{titlepage}

\tableofcontents
\newpage

% --- Executive Summary ---
\section*{Executive Summary}

This report provides a comprehensive analysis of the cybersecurity posture of Pioneer Pulse, based on network scan data, an organizational security questionnaire, and a review of pre-existing risks.

The assessment identified two critical-risk findings that require immediate attention. A network scan revealed a new host, \texttt{10.10.10.51}, with an exposed Remote Desktop Protocol (RDP) port (3389). This finding, when correlated with a pre-existing risk of RDP exposure on another host (\texttt{10.10.10.50}), indicates a systemic vulnerability to unauthorized access and ransomware attacks.

Furthermore, a review of administrative controls revealed a significant gap: the organization does not conduct annual security awareness training for all employees. This failure to maintain security awareness significantly increases the organization's susceptibility to social engineering and phishing attacks, which are primary vectors for initial compromise.

Immediate remediation of the exposed RDP services and the implementation of a mandatory, recurring security awareness training program are strongly recommended to mitigate these high-impact risks.

% --- Organizational Information ---
\section*{1. Organizational Information}
The following details were provided for the assessment.

\begin{itemize}
    \item \textbf{Organization Name:} Pioneer Pulse
    \item \textbf{Email Domain:} \seqsplit{\texttt{PioneerPulse.com}}
    \item \textbf{Website Domain:} \seqsplit{\url{http://www.PioneerPulse.com}}
    \item \textbf{External IP Address:} \texttt{49.225.119.97}
\end{itemize}

% --- Security Control Review ---
\section*{2. Security Control Review}
A review of administrative and technical security controls was conducted via a questionnaire. The responses are summarized below. A checkmark (\ding{51}) indicates a positive control is in place, while a cross (\ding{55}) indicates a control gap.

\begin{table}[h!]
\centering
\caption{Security Controls Questionnaire Results}
\begin{tabular}{p{0.8\linewidth} c}
\toprule
\textbf{Control Question} & \textbf{Response} \\
\midrule
Do you require MFA to access email? & \ding{51} \\
Do you require MFA to log into computers? & \ding{51} \\
Do you require MFA to access sensitive data systems? & \ding{51} \\
Does your organization have an employee acceptable use policy? & \ding{51} \\
Does your organization do security awareness training for new employees? & \ding{51} \\
\textbf{Does your organization do security awareness training for all employees at least once per year?} & \textbf{\ding{55}} \\
\bottomrule
\end{tabular}
\end{table}

\subsection*{Analysis of Control Gaps}
The primary control gap identified is the \textbf{lack of annual security awareness training for all employees}. While initial training for new hires is a good practice, the threat landscape evolves continuously. Without regular, recurring training, employees are more likely to fall victim to modern phishing, social engineering, and malware campaigns. This is considered a high-risk finding as it directly impacts the organization's resilience against common attack vectors.

% --- Technical Scan Results ---
\section*{3. Technical Scan Results}
A network scan was performed to identify active services and potential vulnerabilities on the target host.

\begin{itemize}
    \item \textbf{Target IP Address:} \texttt{10.10.10.51}
    \item \textbf{Scan Tool:} Nmap
\end{itemize}

\begin{table}[h!]
\centering
\caption{Open Ports on \texttt{10.10.10.51}}
\begin{tabular}{l l l}
\toprule
\textbf{Port} & \textbf{State} & \textbf{Service Name} \\
\midrule
3389/tcp & open & ms-wbt-server (Microsoft RDP) \\
\bottomrule
\end{tabular}
\end{table}

\subsection*{Analysis of Technical Findings}
The scan identified that port \textbf{3389/tcp}, used for Microsoft Remote Desktop Protocol (RDP), is open on host \texttt{10.10.10.51}. RDP is a common target for attackers seeking to gain unauthorized access to internal networks. Exposed RDP services are frequently exploited through:
\begin{itemize}
    \item \textbf{Brute-force attacks:} Guessing common or weak passwords.
    \item \textbf{Credential stuffing:} Using credentials stolen from other data breaches.
    \item \textbf{Exploitation of vulnerabilities:} Such as the infamous "BlueKeep" (CVE-2019-0708).
\end{itemize}
This finding represents a critical risk to the internal network.

% --- Risk Assessment ---
\section*{4. Correlated Risk Assessment}
This section synthesizes findings from the security control review, the technical scan, and pre-existing risk data to provide a holistic view of the current risk posture.

\begin{table}[h!]
\centering
\caption{Summary of Identified Risks}
\begin{tabular}{p{0.25\linewidth} p{0.45\linewidth} l l}
\toprule
\textbf{Risk Name} & \textbf{Description} & \textbf{Affected Systems} & \textbf{Severity} \\
\midrule
\textbf{Unsecured RDP Access (New)} & Port 3389 (RDP) is open, exposing the host to unauthorized access and lateral movement. & \texttt{10.10.10.51} & \textbf{Critical} \\
\addlinespace
\textbf{Systemic RDP Exposure (Pattern)} & A pre-existing risk on host \texttt{10.10.10.50} combined with the new finding indicates a pattern of insecure RDP configuration across the network. & Network-wide & \textbf{Critical} \\
\addlinespace
\textbf{Lack of Annual Security Training} & The absence of yearly security training for all staff increases susceptibility to phishing and social engineering. & All Employees & \textbf{High} \\
\bottomrule
\end{tabular}
\end{table}

% --- Recommendations ---
\section*{5. Recommendations}
The following actions are recommended to mitigate the identified risks. Recommendations are prioritized based on severity.

\subsection*{Immediate Actions (Critical Risks)}
\begin{enumerate}
    \item \textbf{Remediate RDP Exposure on \texttt{10.10.10.51}:}
    \begin{itemize}
        \item \textbf{Validate Business Need:} Immediately confirm if RDP access to this host is required.
        \item \textbf{Close Port:} If access is not required, close port 3389 on the host's firewall.
        \item \textbf{Restrict Access:} If required, implement strict firewall rules to allow RDP connections only from specific, authorized IP addresses (e.g., a management jump box).
    \end{itemize}
    
    \item \textbf{Conduct Network-Wide RDP Audit:}
    \begin{itemize}
        \item Perform an internal network scan to identify ALL systems with port 3389 open.
        \item Apply the remediation steps above to every discovered host.
    \end{itemize}
\end{enumerate}

\subsection*{Strategic Actions (High Risks \& Long-Term Fixes)}
\begin{enumerate}
    \setcounter{enumi}{2} % Continue numbering
    \item \textbf{Implement a Secure Remote Access Policy:}
    \begin{itemize}
        \item Prohibit direct RDP access from general user workstations.
        \item Implement a centralized, secure remote access solution, such as a Virtual Private Network (VPN) or a bastion host (jump server), with Multi-Factor Authentication (MFA) enabled.
    \end{itemize}

    \item \textbf{Establish an Annual Security Awareness Program:}
    \begin{itemize}
        \item \textbf{Procure Training:} Select and procure a security awareness training platform or service.
        \item \textbf{Mandate Completion:} Make annual training mandatory for all employees, including management.
        \item \textbf{Track Compliance:} Implement a system to track completion rates and follow up with non-compliant staff.
        \item \textbf{Content:} Ensure training covers modern threats, including phishing, password security, and social engineering.
    \end{itemize}
\end{enumerate}

% --- Conclusion ---
\section*{Conclusion}
The assessment has identified critical vulnerabilities related to systemic RDP exposure and a high-risk gap in security training. The combination of these findings presents a significant threat to the confidentiality, integrity, and availability of Pioneer Pulse's data and systems.

Prompt implementation of the recommendations outlined in this report is essential to strengthen the organization's security posture and reduce the likelihood of a successful cyberattack.

\end{document}
```