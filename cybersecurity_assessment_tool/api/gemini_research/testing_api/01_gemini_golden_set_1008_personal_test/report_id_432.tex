```latex
\documentclass[12pt]{article}

% Preamble: Required Packages
\usepackage[margin=1in]{geometry}
\usepackage{pifont} % For checkmarks and crosses
\usepackage{booktabs} % For professional tables
\usepackage{hyperref} % For clickable links
\usepackage{url} % For formatting URLs
\usepackage{seqsplit} % For splitting long strings
\usepackage{graphicx} % For logo (placeholder)
\usepackage{fancyhdr} % For header/footer
\usepackage{lastpage} % To get total page count

% Document Metadata
\hypersetup{
    colorlinks=true,
    linkcolor=black,
    urlcolor=blue,
    pdftitle={Cybersecurity Posture Assessment Report},
    pdfauthor={Cybersecurity Analysis Division},
    pdfsubject={Security Assessment},
    pdfkeywords={Security, Risk, Analysis}
}

% Header and Footer Configuration
\pagestyle{fancy}
\fancyhf{} % Clear all header and footer fields
\fancyhead[L]{Cybersecurity Posture Assessment Report}
\fancyhead[R]{Vertex Solutions}
\fancyfoot[C]{\thepage\ of \pageref{LastPage}}
\renewcommand{\headrulewidth}{0.4pt}
\renewcommand{\footrulewidth}{0.4pt}

\begin{document}

% --- Title Page ---
\begin{titlepage}
    \centering
    \vspace*{1cm}
    
    \Huge
    \textbf{Cybersecurity Posture Assessment Report}
    
    \vspace{1.5cm}
    
    \Large
    Prepared for:
    
    \vspace{0.5cm}
    
    \textbf{Vertex Solutions}
    
    \vspace{2cm}
    
    \large
    \textbf{Date of Report:} \today
    
    \vfill
    
    \normalsize
    \textit{This report contains sensitive information and is intended solely for the use of the recipient organization. Distribution without prior consent is prohibited.}
    
\end{titlepage}

\tableofcontents
\newpage

% --- Section 1: Executive Summary ---
\section{Executive Summary}

This report provides a cybersecurity posture assessment for \textbf{Vertex Solutions}, based on a review of organizational security controls and an analysis of technical scan data. The assessment aims to identify key security gaps, evaluate existing risks, and provide actionable recommendations to enhance the organization's overall security resilience.

\paragraph{Key Findings:} The analysis revealed several critical and high-risk security gaps originating from organizational policies. The most significant concerns are:
\begin{itemize}
    \item \textbf{Critical Risk: Lack of Multi-Factor Authentication (MFA):} MFA is not enforced for accessing company email or for logging into employee computers. This exposes the organization to a high risk of account compromise, business email compromise (BEC), and unauthorized access to internal systems.
    \item \textbf{High Risk: Inadequate Security Awareness Training:} While new employees receive training, there is no mandatory annual security training for all staff. This leads to a gradual decay in security awareness, making the organization more susceptible to evolving threats like phishing and social engineering.
\end{itemize}

\paragraph{Data Limitations:} It is crucial to note that the provided network scan data (\texttt{Input\_1\_Network\_Scan\_JSON}) and the list of current risks (\texttt{Input\_3\_Current\_Risks\_JSON}) were corrupted and could not be processed. Consequently, this assessment is based primarily on the security questionnaire. A comprehensive understanding of the external attack surface and existing vulnerabilities is not possible without valid technical data.

\paragraph{Overall Posture:} While \textbf{Vertex Solutions} has foundational security controls in place, such as an acceptable use policy and MFA for sensitive systems, the identified gaps represent fundamental weaknesses that significantly elevate the risk of a security incident. The recommendations outlined in this report should be prioritized to mitigate these immediate threats.

\newpage

% --- Section 2: Organizational Information ---
\section{Organizational Information}
The following details were provided for the assessment.

\begin{tabular}{@{}ll}
\toprule
\textbf{Attribute} & \textbf{Value} \\
\midrule
Organization Name & \textbf{Vertex Solutions} \\
Email Domain & \texttt{VertexSolutions.net} \\
Website Domain & \url{www.VertexSolutions.net} \\
External IP Address & \texttt{211.204.106.102} \\
\bottomrule
\end{tabular}

% --- Section 3: Security Control Review ---
\section{Security Control Review (Questionnaire Analysis)}
A security questionnaire was completed to evaluate the implementation of key administrative and technical controls. The responses are summarized below. Answers marked with a cross (\ding{55}) indicate a potential security gap that requires attention.

\begin{table}[h!]
\centering
\caption{Organizational Security Controls Questionnaire}
\begin{tabular}{@{}p{0.8\linewidth}c@{}}
\toprule
\textbf{Control Question} & \textbf{Response} \\
\midrule
Do you require MFA to access email? & \ding{55} \\
Do you require MFA to log into computers? & \ding{55} \\
Do you require MFA to access sensitive data systems? & \ding{51} \\
Does your organization have an employee acceptable use policy? & \ding{51} \\
Does your organization do security awareness training for new employees? & \ding{51} \\
Does your organization do security awareness training for all employees at least once per year? & \ding{55} \\
\bottomrule
\end{tabular}
\end{table}

\paragraph{Analysis:} The questionnaire reveals critical deficiencies in access control and employee security education. The absence of MFA on email and computer logins is a major vulnerability. Furthermore, the lack of recurring annual security training for all employees means that staff awareness of current cyber threats is likely to be outdated, increasing the human element of risk.

% --- Section 4: Technical Scan Results ---
\section{Technical Scan Results}
\subsection{External Network Scan}
The external network scan data provided in \texttt{Input\_1\_Network\_Scan\_JSON} was found to be corrupted or incomplete. As a result, an analysis of open ports, exposed services, and potential vulnerabilities on the external IP address (\texttt{211.204.106.102}) could not be performed.

\paragraph{Impact:} Without this data, the organization's external attack surface remains unknown. There may be unpatched services, misconfigured applications, or outdated software exposed to the internet, which could be exploited by attackers. It is strongly recommended to perform a new, comprehensive network scan as a matter of priority.

\newpage

% --- Section 5: Risk Assessment ---
\section{Risk Assessment}
This risk assessment is based on the findings from the Security Control Review. Due to corrupted input data, pre-existing risks and technical vulnerabilities could not be included. The following table summarizes the newly identified risks.

\begin{table}[h!]
\centering
\caption{Summary of Identified Risks}
\begin{tabular}{@{}p{0.1\linewidth}p{0.3\linewidth}p{0.15\linewidth}p{0.35\linewidth}@{}}
\toprule
\textbf{Risk ID} & \textbf{Risk Name} & \textbf{Severity} & \textbf{Description} \\
\midrule
R-01 & Lack of MFA on Email & \textbf{Critical} & Absence of MFA on email accounts makes them highly susceptible to phishing and credential stuffing attacks, leading to Business Email Compromise (BEC) and data breaches. \\
\addlinespace
R-02 & Lack of MFA on Endpoint Logins & \textbf{Critical} & Without MFA, compromised credentials can be used to gain direct access to employee computers, enabling lateral movement, data theft, and ransomware deployment. \\
\addlinespace
R-03 & Inadequate Security Awareness Training & \textbf{High} & The lack of annual, recurring training for all staff increases the likelihood of employees falling victim to phishing, social engineering, and other common attack vectors. \\
\bottomrule
\end{tabular}
\end{table}

% --- Section 6: Recommendations ---
\section{Recommendations}
The following actions are recommended to mitigate the identified risks and strengthen the security posture of \textbf{Vertex Solutions}. Recommendations are prioritized based on risk severity.

\begin{enumerate}
    \item \textbf{Implement Mandatory Multi-Factor Authentication (Critical)}
    \begin{itemize}
        \item \textbf{Action:} Enforce MFA across all user accounts for email access (e.g., Office 365, Google Workspace) and for logging into company-managed computers (e.g., Windows Hello, Duo).
        \item \textbf{Justification:} This is the single most effective control to prevent unauthorized account access (mitigates R-01 and R-02). It provides a critical layer of defense even if user credentials are stolen.
    \end{itemize}
    
    \item \textbf{Establish an Annual Security Awareness Program (High)}
    \begin{itemize}
        \item \textbf{Action:} Develop and mandate an annual security awareness training program for all employees. The training should cover current threats, phishing identification, password hygiene, and company security policies.
        \item \textbf{Justification:} A well-informed workforce is a crucial part of the security defense layer. Regular training ensures that security remains a top-of-mind concern for all staff and reduces the risk of human error (mitigates R-03).
    \end{itemize}
    
    \item \textbf{Conduct a New External Network Vulnerability Scan (High)}
    \begin{itemize}
        \item \textbf{Action:} Commission a new, authenticated and unauthenticated vulnerability scan of the organization's external network perimeter, targeting the IP address \texttt{211.204.106.102}.
        \item \textbf{Justification:} The previous scan data was unusable. A new scan is essential to identify and remediate technical vulnerabilities on internet-facing systems before they can be exploited by attackers.
    \end{itemize}
\end{enumerate}

\end{document}
```