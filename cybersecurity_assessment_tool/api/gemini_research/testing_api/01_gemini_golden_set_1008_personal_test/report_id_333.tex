```latex
\documentclass[12pt]{article}

% --- PACKAGES ---
\usepackage[margin=1in]{geometry}
\usepackage{pifont} % For checkmarks and crosses
\usepackage{booktabs} % For professional tables
\usepackage{hyperref} % For hyperlinks
\usepackage{url} % For URL formatting
\usepackage{seqsplit} % To split long strings in tt font
\usepackage{graphicx}
\usepackage{xcolor}

% --- DOCUMENT SETUP ---
\hypersetup{
    colorlinks=true,
    linkcolor=blue,
    filecolor=magenta,      
    urlcolor=cyan,
    pdftitle={Cybersecurity Posture Report},
    pdfpagemode=FullScreen,
}

% --- TITLE ---
\title{Cybersecurity Posture Report \\ \large For: \textbf{Cinder \& Ash}}
\author{Cybersecurity Analyst}
\date{\today}

% --- BEGIN DOCUMENT ---
\begin{document}
\maketitle
\thispagestyle{empty}
\newpage

\tableofcontents
\newpage

% --- EXECUTIVE SUMMARY ---
\section*{Executive Summary}
This report provides a comprehensive analysis of the cybersecurity posture for \textbf{Cinder \& Ash}, based on a synthesis of network scan data, organizational security controls, and existing risk documentation. The assessment reveals several critical and high-risk vulnerabilities that require immediate attention.

A network scan identified a publicly accessible service on an internal host (\texttt{10.5.5.5:8080}) with an alarming HTTP title: \textbf{"TOP SECRET DB"}. This finding directly contradicts the current risk register, which incorrectly classifies this port as secure. This discrepancy indicates a severe, unmitigated exposure of a potentially highly sensitive data system.

Furthermore, a review of organizational security controls highlights critical gaps. The lack of mandatory Multi-Factor Authentication (MFA) for email and computer access, combined with the absence of a security awareness training program, exposes the organization to significant threats from phishing, credential theft, and social engineering attacks.

Immediate remediation is required to address the exposed database. Strategic initiatives must be launched to implement foundational security controls like MFA and employee training to build a more resilient security posture.

% --- ORGANIZATIONAL INFORMATION ---
\section{Organizational Information}
The following details were provided for the assessment.

\begin{itemize}
    \item \textbf{Organization Name:} Cinder \& Ash
    \item \textbf{Email Domain:} \texttt{CinderAsh.com}
    \item \textbf{Website Domain:} \url{www.CinderAsh.com}
    \item \textbf{External IP Address:} \texttt{51.211.27.170}
\end{itemize}

% --- SECURITY CONTROL REVIEW ---
\section{Security Control Review}
A review of the organization's security controls was conducted via a questionnaire. The results below highlight significant gaps in foundational security practices. "No" answers indicate a lack of a critical control and are considered high-risk findings.

\begin{table}[h!]
\centering
\caption{Security Control Questionnaire Results}
\begin{tabular}{p{0.75\linewidth} c}
\toprule
\textbf{Control Question} & \textbf{Status} \\
\midrule
Do you require MFA to access email? & \ding{55} \\
Do you require MFA to log into computers? & \ding{55} \\
Do you require MFA to access sensitive data systems? & \ding{51} \\
Does your organization have an employee acceptable use policy? & \ding{51} \\
Does your organization do security awareness training for new employees? & \ding{55} \\
Does your organization do security awareness training for all employees at least once per year? & \ding{55} \\
\bottomrule
\end{tabular}
\end{table}

\subsection*{Analysis}
The absence of MFA for email and computer logins is a critical vulnerability. Email is a primary target for attackers seeking to compromise accounts for phishing, business email compromise (BEC), and lateral movement. Similarly, unprotected computer logins allow a compromised password to grant an attacker direct access to the internal network. The lack of security awareness training for any employees leaves the organization highly susceptible to social engineering attacks, as staff are not equipped to recognize or respond to threats.

% --- TECHNICAL SCAN RESULTS ---
\section{Technical Scan Results}
An Nmap scan was performed on the specified target to identify open ports and running services. The results indicate a critical finding that requires immediate investigation.

\begin{table}[h!]
\centering
\caption{Nmap Scan Findings}
\begin{tabular}{llll}
\toprule
\textbf{Target IP} & \textbf{Port} & \textbf{State} & \textbf{Service Information} \\
\midrule
\texttt{10.5.5.5} & 8080/tcp & Open & HTTP Title: \textbf{TOP SECRET DB} \\
\bottomrule
\end{tabular}
\end{table}

\subsection*{Analysis}
The scan identified an open HTTP service on port 8080. The service's title, "TOP SECRET DB," is extremely concerning. This suggests that a sensitive, possibly confidential, database is accessible on the network. This finding is particularly alarming because the existing risk documentation (\textit{Input\_3\_Current\_Risks\_JSON}) explicitly states that this port is "confirmed secure and false positive." This indicates a critical failure in the risk management process, as a high-risk vulnerability was incorrectly dismissed and left unmitigated.

% --- RISK ASSESSMENT ---
\section{Risk Assessment}
Based on the correlation of all provided data, the following risks have been identified and prioritized. The severity levels reflect the potential impact and likelihood of exploitation.

\begin{table}[h!]
\centering
\caption{Summary of Identified Risks}
\begin{tabular}{p{0.15\linewidth} p{0.65\linewidth} p{0.1\linewidth}}
\toprule
\textbf{Risk ID} & \textbf{Description} & \textbf{Severity} \\
\midrule
\textbf{RISK-001} & \textbf{Unmitigated Exposure of Sensitive Database:} An open service on \texttt{10.5.5.5:8080} is titled "TOP SECRET DB". This directly contradicts the risk register and presents an immediate and severe threat of a data breach. & \textbf{Critical} \\
\addlinespace
\textbf{RISK-002} & \textbf{Widespread Lack of MFA:} The absence of MFA for email and computer logins exposes the organization to a high likelihood of account compromise, which could lead to data theft, financial loss, or ransomware. & \textbf{Critical} \\
\addlinespace
\textbf{RISK-003} & \textbf{Inadequate Security Awareness Program:} With no security training, employees are unprepared to defend against phishing and social engineering, making them the weakest link in the organization's security posture. & \textbf{High} \\
\bottomrule
\end{tabular}
\end{table}

% --- RECOMMENDATIONS ---
\section{Recommendations}
The following actions are recommended to mitigate the identified risks. They are prioritized based on severity and the urgency of the threat.

\subsection*{Immediate Priority (To Be Completed Within 24 Hours)}
\begin{enumerate}
    \item \textbf{Secure Exposed Database (RISK-001):}
    \begin{itemize}
        \item Immediately investigate the service running on \texttt{10.5.5.5:8080}.
        \item If the service is not business-critical, shut it down.
        \item If it is required, place it behind a firewall and restrict access to only authorized personnel.
        \item Conduct a forensic analysis to determine if the system has already been compromised.
    \end{itemize}
    \item \textbf{Update Risk Register:} The existing risk documentation must be immediately updated to reflect the critical nature of RISK-001. The process for validating and closing risks must be reviewed to prevent future failures.
\end{enumerate}

\subsection*{High Priority (To Be Completed Within 30 Days)}
\begin{enumerate}
    \setcounter{enumi}{2} % Continue numbering
    \item \textbf{Implement Multi-Factor Authentication (RISK-002):}
    \begin{itemize}
        \item Enforce mandatory MFA for all user accounts on all systems, starting with email and remote access solutions.
        \item Develop a plan to roll out MFA for all computer logins within the next quarter.
    \end{itemize}
    \item \textbf{Deploy Security Awareness Training (RISK-003):}
    \begin{itemize}
        \item Procure and deploy a security awareness training platform.
        \item Mandate that all new employees complete training during onboarding.
        \item Require all current employees to complete training annually and conduct regular phishing simulations to reinforce learning.
    \end{itemize}
\end{enumerate}

\end{document}
```