```latex
\documentclass[12pt]{article}

% Preamble: Required Packages
\usepackage[margin=1in]{geometry}
\usepackage{pifont} % For checkmarks and crosses
\usepackage{booktabs} % For professional tables
\usepackage{hyperref} % For clickable links
\usepackage{url}      % For formatting URLs
\usepackage{seqsplit} % For splitting long strings in texttt
\usepackage[T1]{fontenc}

% Document Metadata
\title{Cybersecurity Assessment Report \\ \large For: Atlas Mapping}
\author{Cybersecurity Analyst Group}
\date{\today}

\begin{document}

\maketitle
\thispagestyle{empty}
\newpage

\tableofcontents
\thispagestyle{empty}
\newpage

\setcounter{page}{1}

% ==============================================================================
% 1. Executive Summary
% ==============================================================================
\section{Executive Summary}

This report details the findings of a cybersecurity assessment for Atlas Mapping, conducted on \today. The analysis combines a review of organizational security controls, a technical network scan, and an evaluation of pre-existing risks.

The assessment has identified several critical and high-risk security gaps that require immediate attention. The most severe findings relate to a systemic lack of Multi-Factor Authentication (MFA) across all key access points, including email, computer logins, and sensitive data systems. This absence of a fundamental security control exposes the organization to a high likelihood of account compromise and unauthorized access.

Additionally, administrative controls are weakened by the lack of an employee Acceptable Use Policy. Technical scans revealed an externally exposed Secure Shell (SSH) service on the network, which, combined with the lack of MFA, presents a significant vector for potential intrusion.

While the organization has a commendable security awareness training program, the foundational technical and administrative controls are insufficient. We strongly recommend prioritizing the implementation of MFA and developing a comprehensive set of security policies to mitigate these identified risks and improve the overall security posture.

% ==============================================================================
% 2. Organizational Information
% ==============================================================================
\section{Organizational Information}

The following details were provided for the assessment.

\begin{tabular}{@{}ll}
\toprule
\textbf{Attribute} & \textbf{Value} \\
\midrule
Organization Name & Atlas Mapping \\
Email Domain & \texttt{AtlasMapping.com} \\
Website Domain & \url{www.AtlasMapping.com} \\
External IP Address & \texttt{190.75.128.85} \\
\bottomrule
\end{tabular}

% ==============================================================================
% 3. Security Control Review
% ==============================================================================
\section{Security Control Review}

A review of administrative and technical security controls was conducted via a standardized questionnaire. The results are summarized below. Answers marked with \ding{55} indicate a deviation from security best practices and represent a potential risk.

\begin{table}[h!]
\centering
\begin{tabular}{@{}p{0.8\linewidth}c}
\toprule
\textbf{Control Question} & \textbf{Status} \\
\midrule
Do you require MFA to access email? & \ding{55} \\
Do you require MFA to log into computers? & \ding{55} \\
Do you require MFA to access sensitive data systems? & \ding{55} \\
Does your organization have an employee acceptable use policy? & \ding{55} \\
\addlinespace
Does your organization do security awareness training for new employees? & \ding{51} \\
Does your organization do security awareness training for all employees at least once per year? & \ding{51} \\
\bottomrule
\end{tabular}
\caption{Organizational Security Control Status}
\label{tab:controls}
\end{table}

\subsection*{Analysis}
The questionnaire reveals critical gaps in access control. The lack of MFA for email, computer, and sensitive system access is a significant vulnerability. An attacker who obtains a user's password would have direct, unrestricted access. Furthermore, the absence of an Acceptable Use Policy means there are no formal guidelines for employees regarding the secure use of company assets, increasing the risk of insider threats and accidental data exposure.

The organization's commitment to security awareness training is a positive control that should be maintained.

% ==============================================================================
% 4. Technical Scan Results
% ==============================================================================
\section{Technical Scan Results}

An external network scan was performed to identify open ports and exposed services.

\begin{itemize}
    \item \textbf{Target IP Address:} \texttt{2001:db8::1}
\end{itemize}

The following table details the services found to be accessible from the public internet.

\begin{table}[h!]
\centering
\begin{tabular}{@{}llll@{}}
\toprule
\textbf{Port} & \textbf{State} & \textbf{Service (Inferred)} & \textbf{Notes} \\
\midrule
22/tcp & Open & SSH & Service is exposed externally. \\
\bottomrule
\end{tabular}
\caption{Open Ports Detected on \texttt{2001:db8::1}}
\label{tab:scanresults}
\end{table}

\subsection*{Analysis}
The scan identified that port 22, commonly used for the Secure Shell (SSH) protocol, is open. SSH is a powerful administrative tool, and its exposure to the internet makes it a primary target for brute-force and credential-stuffing attacks. Without robust controls such as IP whitelisting, key-based authentication, and intrusion prevention systems, this service poses a direct risk to the network's integrity. This risk is amplified by the lack of MFA for computer logins, as a compromised password could lead to a successful SSH login.

% ==============================================================================
% 5. Risk Assessment
% ==============================================================================
\section{Risk Assessment}

The following table synthesizes findings from the security control review and technical scan into a prioritized list of risks. No pre-existing vulnerabilities were reported.

\begin{table}[h!]
\centering
\begin{tabular}{@{}p{0.1\linewidth}p{0.25\linewidth}p{0.45\linewidth}l@{}}
\toprule
\textbf{ID} & \textbf{Risk Name} & \textbf{Description} & \textbf{Severity} \\
\midrule
\textbf{RISK-001} & Widespread Lack of MFA & The absence of MFA on email, computers, and sensitive systems allows for account takeover with only a single factor (password). & \textbf{Critical} \\
\addlinespace
\textbf{RISK-002} & Absence of Acceptable Use Policy & Lack of a formal policy creates ambiguity for employees on security responsibilities, increasing the likelihood of insecure behavior. & High \\
\addlinespace
\textbf{RISK-003} & Exposed SSH Service & The SSH management port is open to the internet, making it a target for brute-force attacks and unauthorized access attempts. & Medium \\
\bottomrule
\end{tabular}
\caption{Summary of Identified Risks}
\label{tab:risks}
\end{table}

% ==============================================================================
% 6. Recommendations
% ==============================================================================
\section{Recommendations}

Based on the identified risks, we provide the following actionable recommendations to improve the security posture of Atlas Mapping.

\subsection*{RISK-001: Widespread Lack of MFA (Critical)}
\begin{itemize}
    \item \textbf{Immediate Action:} Prioritize the deployment of a robust MFA solution across the entire organization.
    \item \textbf{Phase 1:} Enforce MFA for all email accounts (e.g., via O365 or Google Workspace policies) and any systems containing sensitive data.
    \item \textbf{Phase 2:} Enforce MFA for all employee computer logins (e.g., via Windows Hello for Business, Duo, or similar).
    \item \textbf{Phase 3:} Enforce MFA for all other applications and services, including VPN access.
\end{itemize}

\subsection*{RISK-002: Absence of Acceptable Use Policy (High)}
\begin{itemize}
    \item \textbf{Action:} Develop, approve, and implement a formal Acceptable Use Policy (AUP).
    \item \textbf{Details:} The policy should clearly define rules for the use of company computers, networks, email, and internet access. It must be communicated to all employees, and acknowledgement of receipt should be tracked.
\end{itemize}

\subsection*{RISK-003: Exposed SSH Service (Medium)}
\begin{itemize}
    \item \textbf{Action:} Review the business justification for exposing the SSH service on \texttt{2001:db8::1} to the public internet.
    \item \textbf{If Not Needed:} Block port 22 at the network firewall immediately.
    \item \textbf{If Required:} Implement security hardening measures:
    \begin{enumerate}
        \item Restrict access to a whitelist of trusted IP addresses.
        \item Disable password-based authentication and enforce the use of strong SSH keys.
        \item Disable root login over SSH.
        \item Implement an intrusion detection/prevention tool like Fail2ban to automatically block IPs that exhibit malicious behavior.
    \end{enumerate}
\end{itemize}

\end{document}
```