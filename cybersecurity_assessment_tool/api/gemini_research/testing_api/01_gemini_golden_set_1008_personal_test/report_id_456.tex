```latex
\documentclass[12pt]{article}

% Preamble: Required Packages
\usepackage[margin=1in]{geometry}
\usepackage{pifont} % For checkmarks and crosses
\usepackage{booktabs} % For professional tables
\usepackage{hyperref} % For clickable links
\usepackage{url} % For URL formatting
\usepackage{seqsplit} % To split long text strings
\usepackage{graphicx}
\usepackage{xcolor}
\usepackage{fancyhdr}
\usepackage{lastpage}

% Document Metadata and Styling
\hypersetup{
    colorlinks=true,
    linkcolor=blue,
    filecolor=magenta,      
    urlcolor=cyan,
    pdftitle={Cybersecurity Posture Assessment Report},
    pdfauthor={Cybersecurity Analyst},
    pdfkeywords={Security, Assessment, Report},
}

% Define colors for severity
\definecolor{criticalred}{HTML}{D10000}
\definecolor{highorange}{HTML}{E25F00}
\definecolor{mediumyellow}{HTML}{F2C000}
\definecolor{lowgreen}{HTML}{3B8700}

% Header and Footer
\pagestyle{fancy}
\fancyhf{}
\fancyhead[L]{Cybersecurity Posture Assessment}
\fancyhead[R]{Apex Legends Group}
\fancyfoot[C]{\thepage\ of \pageref{LastPage}}
\renewcommand{\headrulewidth}{0.4pt}
\renewcommand{\footrulewidth}{0.4pt}

\begin{document}

% --- Title Page ---
\begin{titlepage}
    \centering
    \vspace*{2cm}
    
    \Huge
    \textbf{Cybersecurity Posture Assessment Report}
    
    \vspace{1.5cm}
    
    \Large
    Prepared for: \\
    \vspace{0.5cm}
    \textbf{Apex Legends Group}
    
    \vspace{2cm}
    
    \large
    \textbf{Date of Report:} \today
    
    \vfill
    
    \normalsize
    \textit{This report contains sensitive information and is intended solely for the designated recipient. Unauthorized distribution is strictly prohibited.}
    
\end{titlepage}

\tableofcontents
\newpage

% --- Section 1: Executive Summary ---
\section{Executive Summary}
This report provides a cybersecurity posture assessment for \textbf{Apex Legends Group}, based on a review of organizational security controls and an analysis of network scan data. The assessment aims to identify key security gaps, evaluate existing risks, and provide actionable recommendations to enhance the organization's overall security resilience.

The primary findings indicate that while \textbf{Apex Legends Group} has implemented several important security controls, such as requiring Multi-Factor Authentication (MFA) for email and sensitive systems, two critical gaps were identified:
\begin{itemize}
    \item \textbf{Lack of MFA for Endpoint Logins:} The absence of mandatory MFA for computer logins presents a significant risk, as compromised credentials could lead to direct network access.
    \item \textbf{Inadequate Security Awareness Training:} The lack of a mandatory, annual security awareness training program for all employees increases susceptibility to phishing and social engineering attacks.
\end{itemize}

It is important to note that the provided technical network scan data and pre-existing risk data were found to be corrupted and could not be parsed. Therefore, this assessment is primarily based on the security questionnaire. A new technical scan is strongly recommended to identify potential vulnerabilities on the external network perimeter.

This report outlines specific, prioritized recommendations to address these findings and strengthen the security posture of \textbf{Apex Legends Group}.

% --- Section 2: Organizational Information ---
\section{Organizational Information}
The following details were provided for the assessment. This information establishes the context and scope for the review.

\begin{table}[h!]
\centering
\begin{tabular}{@{}ll@{}}
\toprule
\textbf{Attribute} & \textbf{Value} \\
\midrule
Organization Name & Apex Legends Group \\
Email Domain & \texttt{ApexLegendsGroup.com} \\
Website Domain & \url{www.ApexLegendsGroup.com} \\
External IP Address & \texttt{125.9.208.11} \\
\bottomrule
\end{tabular}
\caption{Client Organizational Details}
\end{table}

% --- Section 3: Security Control Review ---
\section{Security Control Review}
The following table summarizes the organization's responses to a security controls questionnaire. The status indicates whether the control aligns with standard cybersecurity best practices. A cross (\ding{55}) highlights a potential security gap that requires attention.

\begin{table}[h!]
\centering
\begin{tabular}{@{}p{0.65\linewidth}cc@{}}
\toprule
\textbf{Control Question} & \textbf{Response} & \textbf{Status} \\
\midrule
Do you require MFA to access email? & Yes & \ding{51} \\
\addlinespace
Do you require MFA to log into computers? & No & \textcolor{criticalred}{\ding{55}} \\
\addlinespace
Do you require MFA to access sensitive data systems? & Yes & \ding{51} \\
\addlinespace
Does your organization have an employee acceptable use policy? & Yes & \ding{51} \\
\addlinespace
Does your organization do security awareness training for new employees? & Yes & \ding{51} \\
\addlinespace
Does your organization do security awareness training for all employees at least once per year? & No & \textcolor{highorange}{\ding{55}} \\
\bottomrule
\end{tabular}
\caption{Security Controls Questionnaire Analysis}
\end{table}

\subsection*{Analysis of Findings}
The questionnaire reveals a mixed security posture. The implementation of MFA for email and sensitive data access is a commendable strength. However, the two "No" responses represent significant weaknesses that undermine other security efforts and are detailed in the Risk Assessment section.

% --- Section 4: Technical Scan Results ---
\section{Technical Scan Results}
An attempt was made to analyze the external network perimeter of the provided IP address.

\begin{itemize}
    \item \textbf{Target IP Address:} \texttt{125.9.208.11}
    \item \textbf{Scan Date:} Data Not Available
    \item \textbf{Scan Status:} \textbf{\textcolor{criticalred}{Failed - Corrupted Data}}
\end{itemize}

The provided network scan data file (Input\_1\_Network\_Scan\_JSON) was found to be corrupted or malformed and could not be processed. As a result, no analysis of open ports, running services, or potential software vulnerabilities could be performed. This represents a significant gap in visibility into the organization's external attack surface. A comprehensive external vulnerability scan is crucial for identifying and mitigating technical risks.

% --- Section 5: Risk Assessment ---
\section{Risk Assessment}
This assessment is based on the findings from the Security Control Review. Due to corrupted input data, pre-existing risks and technical scan findings could not be included. The following table details the identified risks, their potential impact, and an assigned severity level.

\begin{table}[h!]
\centering
\begin{tabular}{@{}p{0.1\linewidth}p{0.25\linewidth}p{0.45\linewidth}p{0.1\linewidth}@{}}
\toprule
\textbf{Risk ID} & \textbf{Risk Name} & \textbf{Overview} & \textbf{Severity} \\
\midrule
\textbf{ALG-001} & Lack of Endpoint MFA & The absence of MFA on computer logins means that a single compromised password could grant an attacker full access to an employee's workstation and, potentially, the internal network. & \textcolor{criticalred}{Critical} \\
\addlinespace
\textbf{ALG-002} & Inadequate Security Awareness Training & Without mandatory annual training, employees' ability to recognize and report modern phishing and social engineering attacks diminishes over time, making them the weakest link in the security chain. & \textcolor{highorange}{High} \\
\bottomrule
\end{tabular}
\caption{Identified Risks and Severity}
\end{table}

% --- Section 6: Recommendations ---
\section{Recommendations}
The following actions are recommended to mitigate the identified risks and improve the overall cybersecurity posture of \textbf{Apex Legends Group}. Recommendations are prioritized based on the severity of the associated risk.

\subsection*{Priority: Critical}
\begin{description}
    \item[Recommendation ALG-R1:] \textbf{Implement Mandatory MFA for All Endpoint Logins.}
    \begin{itemize}
        \item \textbf{Action:} Deploy a robust Multi-Factor Authentication solution for all employee computer and laptop logins. This could include authenticator apps (TOTP), hardware tokens (YubiKey), or biometrics.
        \item \textbf{Justification:} This action directly mitigates risk \textbf{ALG-001}. It adds a critical layer of security that protects against unauthorized access even if user credentials are stolen.
    \end{itemize}
\end{description}

\subsection*{Priority: High}
\begin{description}
    \item[Recommendation ALG-R2:] \textbf{Establish a Continuous Security Awareness Program.}
    \begin{itemize}
        \item \textbf{Action:} Institute a mandatory annual security awareness training program for all employees. This program should cover current threats, such as phishing, ransomware, and business email compromise. Supplement this with periodic phishing simulation campaigns.
        \item \textbf{Justification:} This addresses risk \textbf{ALG-002} by creating a more security-conscious culture and empowering employees to act as a human firewall.
    \end{itemize}
    \item[Recommendation ALG-R3:] \textbf{Conduct a New Network Vulnerability Scan.}
    \begin{itemize}
        \item \textbf{Action:} Schedule and execute a new, comprehensive external network vulnerability scan against the public-facing IP address (\texttt{125.9.208.11}).
        \item \textbf{Justification:} The initial scan data was unusable. A new scan is essential to identify and remediate technical vulnerabilities such as unpatched services, weak configurations, or exposed management interfaces.
    \end{itemize}
\end{description}

\end{document}
```