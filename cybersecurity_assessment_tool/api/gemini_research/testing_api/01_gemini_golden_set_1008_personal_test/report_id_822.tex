```latex
\documentclass[12pt]{article}

% Preamble: Required Packages
\usepackage[a4paper, margin=1in]{geometry}
\usepackage{pifont} % For checkmarks and crosses (\ding{51}, \ding{55})
\usepackage{booktabs} % For professional-looking tables
\usepackage{hyperref} % For clickable links
\usepackage{url}      % For proper URL formatting
\usepackage{seqsplit} % For splitting long strings in texttt
\usepackage[T1]{fontenc}

% Document Metadata
\title{Cybersecurity Posture Assessment Report}
\author{Cybersecurity Analysis Division}
\date{\today}

\begin{document}

\maketitle
\thispagestyle{empty}
\newpage
\tableofcontents
\newpage

\section*{Executive Summary}

This report provides a comprehensive cybersecurity posture assessment for \textbf{Bluebird Bio}. The analysis is based on a correlation of network scan data, a security controls questionnaire, and a review of pre-existing documented risks.

The assessment reveals critical deficiencies in fundamental administrative security controls. The absence of Multi-Factor Authentication (MFA) for email and computer access, combined with a lack of a formal security awareness training program, exposes the organization to a high risk of credential theft, phishing attacks, and subsequent unauthorized access. These policy and procedure gaps currently represent a more significant threat than the organization's technical network perimeter.

On a positive note, the technical network scan of the target host \texttt{192.168.0.5} revealed a minimal attack surface, with no open ports detected. This contradicts a pre-existing risk item concerning an unencrypted web server on Port 80, suggesting the vulnerability may have been recently remediated.

Immediate action is required to address the identified gaps in authentication and employee security awareness to significantly improve the organization's overall security posture.

\section*{Organizational Information}

The following details were provided for the assessment.

\begin{tabular}{@{}ll}
    \toprule
    \textbf{Attribute} & \textbf{Value} \\
    \midrule
    Organization Name & \textbf{Bluebird Bio} \\
    Email Domain & \texttt{BluebirdBio.org} \\
    Website Domain & \href{http://www.BluebirdBio.org}{\texttt{www.BluebirdBio.org}} \\
    External IP Address & \texttt{232.251.35.188} \\
    \bottomrule
\end{tabular}

\section*{Security Control Review}

The following table summarizes the organization's responses to the security controls questionnaire. Items marked with \ding{55} (No) indicate significant gaps in the security framework and are discussed in the Risk Assessment section.

\begin{table}[h!]
\centering
\begin{tabular}{@{}lcc}
    \toprule
    \textbf{Control Question} & \textbf{Response} & \textbf{Status} \\
    \midrule
    Do you require MFA to access email? & No & \ding{55} \\
    Do you require MFA to log into computers? & No & \ding{55} \\
    Do you require MFA to access sensitive data systems? & Yes & \ding{51} \\
    Does your organization have an employee acceptable use policy? & No & \ding{55} \\
    Does your organization do security awareness training for new employees? & No & \ding{55} \\
    Does your organization do security awareness training annually? & No & \ding{55} \\
    \bottomrule
\end{tabular}
\caption{Security Controls Questionnaire Results}
\end{table}

\section*{Technical Scan Results}

A network scan was performed on the specified target to identify its external-facing services and potential vulnerabilities.

\begin{itemize}
    \item \textbf{Target IP Address:} \texttt{192.168.0.5}
    \item \textbf{Scan Date:} \today
    \item \textbf{Scanner Used:} Nmap
\end{itemize}

The scan confirmed that the host is online, but found no open ports. This indicates a very limited external attack surface for this specific host at the time of the scan.

\begin{table}[h!]
\centering
\begin{tabular}{@{}llll}
    \toprule
    \textbf{Port} & \textbf{State} & \textbf{Service} & \textbf{Version} \\
    \midrule
    80/tcp & closed & http & N/A \\
    \bottomrule
\end{tabular}
\caption{Scan Results for Target: \texttt{192.168.0.5}}
\end{table}

\textbf{Note on Discrepancy:} The current scan shows Port 80 as \texttt{closed}. This finding conflicts with the pre-existing risk titled "Unencrypted Web Server," which assumes the port is open. This suggests that the risk may have been remediated. This should be validated internally.

\section*{Consolidated Risk Assessment}

The following table synthesizes findings from the security questionnaire, technical scan, and pre-existing risk data into a prioritized list of current risks.

\begin{table}[h!]
\centering
\begin{tabular}{@{}p{0.25\linewidth}p{0.55\linewidth}p{0.1\linewidth}@{}}
    \toprule
    \textbf{Risk Title} & \textbf{Description} & \textbf{Severity} \\
    \midrule
    \textbf{Lack of MFA on Core Systems} & The absence of MFA on email and computer logins creates a critical vulnerability. A single compromised password could lead to a full account takeover, data breach, or ransomware deployment. & \textbf{Critical} \\
    \addlinespace
    \textbf{Insufficient Security Awareness Program} & The lack of an acceptable use policy and any form of security awareness training leaves employees unprepared to identify and resist social engineering, phishing, and other common cyber attacks. & \textbf{High} \\
    \addlinespace
    \textbf{Unencrypted Web Server (Potentially Remediated)} & A pre-existing risk noted that Port 80 was open, exposing unencrypted HTTP traffic. The most recent scan shows this port is now closed. If this was a deliberate remediation, the risk is mitigated. If not, the configuration should be validated. & Low \\
    \bottomrule
\end{tabular}
\caption{Prioritized Risk Summary}
\end{table}

\section*{Recommendations}

Based on the analysis, the following actions are recommended to mitigate the identified risks and strengthen the overall security posture of \textbf{Bluebird Bio}.

\begin{enumerate}
    \item \textbf{Immediate Priority (Critical): Deploy Multi-Factor Authentication (MFA)}
    \begin{itemize}
        \item Immediately enable MFA for all user accounts across all email systems (e.g., Office 365, Google Workspace).
        \item Plan and execute the deployment of MFA for all employee computer and remote access logins.
    \end{itemize}

    \item \textbf{High Priority: Establish a Security Awareness Program}
    \begin{itemize}
        \item Develop and implement a formal Employee Acceptable Use Policy (AUP) that all employees must read and acknowledge.
        \item Institute a mandatory security awareness training module for all new hires during their onboarding process.
        \item Conduct annual security awareness training for all staff to keep them informed about evolving threats like phishing and business email compromise.
    \end{itemize}
    
    \item \textbf{Medium Priority: Risk Register and Scan Policy Validation}
    \begin{itemize}
        \item Internally validate that the closure of Port 80 on host \texttt{192.168.0.5} was intentional.
        \item If remediated, formally update and close the "Unencrypted Web Server" risk in the organization's risk register.
        \item Establish a policy for regular, automated vulnerability scanning of all internal and external network assets to ensure continuous monitoring.
    \end{itemize}
\end{enumerate}

\end{document}
```