```latex
\documentclass[12pt]{article}

% Preamble: Required Packages
\usepackage[margin=1in]{geometry}
\usepackage{pifont} % For checkmarks (\ding{51}) and crosses (\ding{55})
\usepackage{booktabs} % For professional-looking tables
\usepackage{hyperref} % For creating hyperlinks in the document
\usepackage{url}      % For formatting URLs
\usepackage{seqsplit} % For splitting long sequences of text like hashes or URLs
\usepackage[T1]{fontenc}
\usepackage{graphicx}
\usepackage{fancyhdr}
\usepackage{xcolor}

% --- Document Setup ---
\hypersetup{
    colorlinks=true,
    linkcolor=blue,
    filecolor=magenta,      
    urlcolor=cyan,
    pdftitle={Cybersecurity Posture Assessment Report},
    pdfpagemode=FullScreen,
}

\pagestyle{fancy}
\fancyhf{}
\fancyhead[L]{Cybersecurity Posture Assessment}
\fancyhead[R]{Astraeus Aerospace}
\fancyfoot[C]{\thepage}

% --- Document Body ---
\begin{document}

% --- Title Page ---
\begin{titlepage}
    \centering
    \vspace*{1cm}
    \Huge\textbf{Cybersecurity Posture Assessment Report}
    \vspace{1.5cm}
    \Large
    \textbf{Prepared for:}\\
    Astraeus Aerospace
    \vspace{2cm}
    \large
    \textbf{Date of Report:}\\
    \today
    \vfill
    \large
    \textbf{Generated by:}\\
    Expert-Level Cybersecurity Analyst
\end{titlepage}

\tableofcontents
\newpage

% --- Section 1: Executive Summary ---
\section{Executive Summary}
This report provides a comprehensive analysis of the cybersecurity posture for Astraeus Aerospace, based on a combination of self-reported security controls, an external network scan, and a review of pre-existing risks.

The assessment identified several critical and high-risk gaps in the organization's security controls. While Multi-Factor Authentication (MFA) is commendably enforced for email access, its absence on computer logins and sensitive data systems presents a significant risk of unauthorized access. Furthermore, the lack of a formal Acceptable Use Policy (AUP) and mandatory annual security awareness training for all employees weakens the human element of the security framework.

On a positive note, a technical scan of the host at \texttt{192.168.0.5} found that port 80 (HTTP) was closed. This contradicts a previously identified risk of an "Unencrypted Web Server," suggesting that remediation may have occurred on this specific asset.

Immediate action is recommended to address the identified gaps, focusing on the broad implementation of MFA, the development of foundational security policies, and the establishment of a continuous security training program.

% --- Section 2: Organizational Information ---
\section{Organizational Information}
The following details were provided for the assessment:
\begin{itemize}
    \item \textbf{Organization Name:} Astraeus Aerospace
    \item \textbf{Primary Email Domain:} \texttt{AstraeusAerospace.com}
    \item \textbf{External IP Address:} \texttt{213.110.255.34}
    \item \textbf{Website Domain:} \url{www.AstraeusAerospace.com}
\end{itemize}

% --- Section 3: Security Control Review ---
\section{Security Control Review}
The following table summarizes the organization's responses to a security controls questionnaire. Items marked with a red cross (\ding{55}) indicate significant gaps in the security posture and are classified as high-risk findings.

\begin{table}[h!]
\centering
\caption{Security Controls Questionnaire Analysis}
\begin{tabular}{p{0.7\textwidth} c c}
\toprule
\textbf{Control Question} & \textbf{Response} & \textbf{Status} \\
\midrule
Do you require MFA to access email? & Yes & \ding{51} \\
Do you require MFA to log into computers? & No & \textcolor{red}{\ding{55}} \\
Do you require MFA to access sensitive data systems? & No & \textcolor{red}{\ding{55}} \\
Does your organization have an employee acceptable use policy? & No & \textcolor{red}{\ding{55}} \\
Does your organization do security awareness training for new employees? & Yes & \ding{51} \\
Does your organization do security awareness training for all employees at least once per year? & No & \textcolor{red}{\ding{55}} \\
\bottomrule
\end{tabular}
\end{table}

% --- Section 4: Technical Scan Results ---
\section{Technical Scan Results}
A network scan was performed to identify externally exposed services and potential vulnerabilities.

\begin{itemize}
    \item \textbf{Target IP Address:} \texttt{192.168.0.5}
    \item \textbf{Scan Tool:} Nmap
    \item \textbf{Host Status:} Up
\end{itemize}

\textbf{Port Analysis:}
The scan specifically checked for common web service ports. The results are as follows:
\begin{itemize}
    \item \textbf{Port 80 (HTTP):} The port was found to be \textbf{closed}. This is a positive security finding, as it prevents unencrypted web traffic. This result contradicts a previously identified risk (see Section 5), indicating the risk may be remediated on this host or is present on other network assets.
\end{itemize}

% --- Section 5: Correlated Risk Assessment ---
\section{Correlated Risk Assessment}
This section synthesizes findings from the security control review, technical scan, and pre-existing risk data into a prioritized list.

\begin{table}[h!]
\centering
\caption{Summary of Identified Risks}
\begin{tabular}{p{0.25\textwidth} p{0.5\textwidth} c}
\toprule
\textbf{Risk Name} & \textbf{Description} & \textbf{Severity} \\
\midrule
\textbf{Lack of MFA on Endpoints and Systems} & The absence of MFA on computer logins and sensitive systems drastically increases the risk of account compromise via stolen credentials. & \textbf{Critical} \\
\addlinespace
\textbf{No Acceptable Use Policy (AUP)} & Without a formal AUP, employees lack clear guidelines on the secure and acceptable use of company assets, leading to inconsistent practices and increased insider risk. & \textbf{High} \\
\addlinespace
\textbf{Insufficient Security Awareness Training} & While new hires receive training, the lack of an annual refresher for all employees allows security knowledge to degrade, making the organization more susceptible to phishing and social engineering. & \textbf{High} \\
\addlinespace
\textbf{Unencrypted Web Server (Historical)} & A pre-existing risk noted that port 80 was open, exposing the organization to unencrypted communications. \textbf{Note:} The recent scan of \texttt{192.168.0.5} found this port closed, suggesting this risk may be mitigated or isolated to other assets. & Medium \\
\bottomrule
\end{tabular}
\end{table}

% --- Section 6: Recommendations ---
\section{Recommendations}
Based on the correlated risk assessment, the following actions are recommended to improve the cybersecurity posture of Astraeus Aerospace.

\begin{enumerate}
    \item \textbf{Implement Comprehensive MFA (Critical Priority):}
    \begin{itemize}
        \item \textbf{Action:} Enforce MFA for all user logins to workstations, laptops, and servers.
        \item \textbf{Action:} Mandate MFA for access to all systems containing sensitive or critical business data, including databases, file shares, and cloud applications.
        \item \textbf{Impact:} This is the single most effective control to prevent unauthorized access resulting from credential theft.
    \end{itemize}
    \vspace{0.5cm}
    \item \textbf{Develop and Implement Foundational Policies (High Priority):}
    \begin{itemize}
        \item \textbf{Action:} Create a formal Acceptable Use Policy (AUP) that clearly defines rules for using company networks, devices, and data.
        \item \textbf{Action:} Require all employees to read and acknowledge the AUP as a condition of their access.
        \item \textbf{Impact:} Establishes a baseline for secure employee behavior and provides a framework for enforcement.
    \end{itemize}
    \vspace{0.5cm}
    \item \textbf{Establish a Continuous Security Training Program (High Priority):}
    \begin{itemize}
        \item \textbf{Action:} Institute a mandatory security awareness training program that all employees must complete at least once per year.
        \item \textbf{Action:} Supplement annual training with regular phishing simulations and security bulletins to keep staff vigilant.
        \item \textbf{Impact:} Strengthens the "human firewall" and reduces the likelihood of successful social engineering attacks.
    \end{itemize}
    \vspace{0.5cm}
    \item \textbf{Verify Network-Wide Risk Mitigation (Medium Priority):}
    \begin{itemize}
        \item \textbf{Action:} Conduct a comprehensive internal and external vulnerability scan across the entire network to confirm if the "Unencrypted Web Server" risk has been fully remediated or if other instances of open port 80 exist.
        \item \textbf{Impact:} Ensures that the positive finding on a single host is representative of the entire network posture.
    \end{itemize}
\end{enumerate}

\end{document}
```