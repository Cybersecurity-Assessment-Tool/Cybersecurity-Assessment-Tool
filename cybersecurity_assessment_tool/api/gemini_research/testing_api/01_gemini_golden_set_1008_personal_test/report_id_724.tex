```latex
\documentclass[12pt]{article}

% Preamble: Required Packages
\usepackage[margin=1in]{geometry}
\usepackage{pifont} % For checkmarks and crosses
\usepackage{booktabs} % For professional tables
\usepackage{hyperref} % For clickable links
\usepackage{url} % For formatting URLs
\usepackage{seqsplit} % To split long strings in tt font
\usepackage{graphicx} % For potential logo inclusion

% Document Metadata
\title{Cybersecurity Assessment Report \\ \large For: Silent Spring}
\author{Cybersecurity Analyst Group}
\date{\today}

\begin{document}

\maketitle
\thispagestyle{empty}
\newpage

\tableofcontents
\newpage

% --- 1. Executive Overview ---
\section{Executive Overview}

This report details the findings of a cybersecurity assessment conducted for \textbf{Silent Spring}. The assessment incorporated a review of organizational security controls, an external network scan, and an analysis of pre-existing risks.

The overall security posture presents a mixed landscape. On a technical level, the scanned host (\texttt{192.168.1.100}) demonstrated a strong network security configuration, with no open ports detected. This indicates a robust firewall policy that effectively minimizes the external attack surface for that specific asset.

However, significant gaps were identified in the organization's access control and employee training policies. The absence of Multi-Factor Authentication (MFA) for computer logins and, most critically, for access to sensitive data systems, constitutes a \textbf{Critical Risk}. A single compromised password could lead to unauthorized system access and a potential data breach. Furthermore, the lack of mandatory security awareness training for new employees creates a vulnerability, as new hires are often prime targets for social engineering attacks.

Immediate remediation should focus on implementing MFA across all critical systems and integrating security training into the employee onboarding process.

% --- 2. Organizational Information ---
\section{Organizational Information}

The following information was provided for the assessment.

\begin{itemize}
    \item \textbf{Organization Name:} Silent Spring
    \item \textbf{Email Domain:} \texttt{SilentSpring.com}
    \item \textbf{Website Domain:} \url{www.SilentSpring.com}
    \item \textbf{External IP Address:} \texttt{113.242.87.140}
\end{itemize}

% --- 3. Security Control Review ---
\section{Security Control Review}

A review of internal security controls was conducted based on a standardized questionnaire. The responses indicate several areas requiring immediate attention. A summary of the findings is presented in Table \ref{tab:controls}.

\begin{table}[h!]
\centering
\caption{Security Controls Questionnaire Results}
\label{tab:controls}
\begin{tabular}{@{}lc@{}}
\toprule
\textbf{Control Question} & \textbf{Response} \\
\midrule
Do you require MFA to access email? & \ding{51} \\
Do you require MFA to log into computers? & \textbf{\color{red}\ding{55}} \\
Do you require MFA to access sensitive data systems? & \textbf{\color{red}\ding{55}} \\
Does your organization have an employee acceptable use policy? & \ding{51} \\
Does your organization do security awareness training for new employees? & \textbf{\color{red}\ding{55}} \\
Does your organization do security awareness training for all employees at least once per year? & \ding{51} \\
\bottomrule
\end{tabular}
\end{table}

% --- 4. Technical Scan Results ---
\section{Technical Scan Results}

A network scan was performed to identify exposed services and potential vulnerabilities on the specified target system.

\begin{itemize}
    \item \textbf{Scan Target:} \texttt{192.168.1.100}
    \item \textbf{Scan Date:} [Scan Date]
\end{itemize}

\subsection{Summary of Findings}
The scan revealed that the target host is online and responsive. However, \textbf{no open TCP or UDP ports were discovered}. All 65,535 ports on the target machine were found to be in a `closed` state.

\subsection{Analysis}
A host with no open ports is considered to have a very strong network security posture. This configuration significantly reduces the attack surface, as there are no listening services (e.g., web servers, remote access protocols, databases) for an attacker to target from the network. This finding indicates an effective and well-configured firewall policy.

% --- 5. Risk Assessment ---
\section{Risk Assessment}

This section synthesizes findings from the security control review and the technical scan. While no pre-existing risks were reported and the network scan was positive, the policy gaps identified in the questionnaire introduce significant organizational risks.

\begin{table}[h!]
\centering
\caption{Identified Risks and Severity}
\label{tab:risks}
\begin{tabular}{@{}p{0.3\linewidth}p{0.5\linewidth}l@{}}
\toprule
\textbf{Risk Name} & \textbf{Overview} & \textbf{Severity} \\
\midrule
\textbf{Inadequate MFA for Sensitive Systems} & Critical data systems lack MFA, a fundamental security layer. A compromised password could grant an attacker direct access to the organization's most valuable data. & \textbf{Critical} \\
\addlinespace
\textbf{Inadequate MFA for Endpoints} & Employee computers can be accessed with only a password. This elevates the risk of lateral movement and further system compromise following a successful phishing or credential theft attack. & \textbf{High} \\
\addlinespace
\textbf{Lack of Onboarding Security Training} & New employees are not formally trained on security policies and common threats. This makes them highly susceptible to social engineering attacks and unintentional policy violations. & \textbf{High} \\
\bottomrule
\end{tabular}
\end{table}

% --- 6. Recommendations ---
\section{Recommendations}

Based on the risk assessment, the following prioritized actions are recommended to enhance the security posture of \textbf{Silent Spring}.

\subsection{Priority 1: Critical}
\begin{itemize}
    \item \textbf{Implement MFA on Sensitive Systems:} Immediately deploy mandatory Multi-Factor Authentication for all user accounts (including administrative and service accounts) that have access to systems designated as containing sensitive or critical data.
\end{itemize}

\subsection{Priority 2: High}
\begin{itemize}
    \item \textbf{Enforce MFA for Endpoint Logins:} Roll out MFA for all employee computer logins (e.g., Windows, macOS). This measure drastically reduces the risk of unauthorized endpoint access via stolen credentials.
    \item \textbf{Develop New-Hire Security Training:} Create a mandatory security awareness training module to be completed by all new employees during their onboarding process. This training should cover the acceptable use policy, phishing identification, password hygiene, and incident reporting procedures.
\end{itemize}

\subsection{Priority 3: Informational}
\begin{itemize}
    \item \textbf{Maintain Strong Firewall Policies:} Continue the excellent practice of maintaining a default-deny firewall policy for network assets, as observed on the scanned host. Regularly review and audit firewall rules to ensure they remain effective.
\end{itemize}

\end{document}
```