```latex
\documentclass[12pt, a4paper]{article}

% Preamble: Required Packages
\usepackage[margin=1in]{geometry}
\usepackage{pifont} % For checkmarks and crosses
\usepackage{booktabs} % For professional tables
\usepackage{hyperref} % For clickable links
\usepackage{url} % For URL formatting
\usepackage{seqsplit} % To split long strings in tt font
\usepackage{graphicx}
\usepackage{xcolor}
\usepackage{datetime}

% Document Metadata
\hypersetup{
    colorlinks=true,
    linkcolor=blue,
    filecolor=magenta,      
    urlcolor=cyan,
    pdftitle={Cybersecurity Posture Assessment Report},
    pdfauthor={Cybersecurity Analyst},
    pdfsubject={Security Analysis},
    pdfkeywords={Cybersecurity, Nmap, Risk Assessment},
}

\newcommand{\yes}{\ding{51}}
\newcommand{\no}{\ding{55}}

\begin{document}

% --- Title Page ---
\begin{titlepage}
    \centering
    \vspace*{1cm}
    \Huge\textbf{Cybersecurity Posture Assessment Report}
    \vspace{1.5cm}
    \Large
    \textbf{Prepared for:} \\
    Echo Chamber Arts
    \vspace{2cm}
    \rule{\linewidth}{0.5mm}
    \vspace{0.5cm}
    \large
    \textbf{Date of Report:} \today \\
    \textbf{Analysis Period:} \today
    \vspace{0.5cm}
    \rule{\linewidth}{0.5mm}
    \vfill
    \small
    \textit{This report contains sensitive information and is intended solely for the use of the recipient organization. Unauthorized distribution is strictly prohibited.}
\end{titlepage}

\tableofcontents
\newpage

% --- Section 1: Executive Summary ---
\section{Executive Summary}
This report provides a comprehensive cybersecurity posture assessment for Echo Chamber Arts, based on a combination of technical network scanning, a review of existing risks, and an organizational security controls questionnaire.

The assessment identified several critical and high-risk security gaps that require immediate attention. Key findings include:
\begin{itemize}
    \item \textbf{Critical Gaps in Access Control:} Multi-Factor Authentication (MFA) is not enforced for logging into computers or accessing sensitive data systems. This significantly increases the risk of unauthorized access and potential data breaches resulting from compromised credentials.
    \item \textbf{Deficiencies in Security Governance:} The organization lacks a formal employee Acceptable Use Policy and does not provide security awareness training for new hires. These foundational policy gaps create an environment where employees may be unaware of security best practices, increasing susceptibility to social engineering and insider threats.
    \item \textbf{Confirmed Technical Vulnerability:} A network service (SSH on port 22) was found exposed on the localhost interface (\texttt{127.0.0.1}). This confirms a pre-existing high-severity risk and could be exploited by local malware or an attacker with initial access to the system.
\end{itemize}

The overall security posture is considered weak due to the fundamental nature of these findings. We strongly recommend prioritizing the implementation of MFA across all critical systems and developing a foundational set of security policies and training programs. Detailed, actionable recommendations are provided in Section \ref{sec:recommendations}.

% --- Section 2: Organizational Information ---
\section{Organizational Information}
The following details were provided for the assessment.
\begin{center}
\begin{tabular}{ll}
\toprule
\textbf{Attribute} & \textbf{Value} \\
\midrule
Organization Name & Echo Chamber Arts \\
Email Domain & \seqsplit{\texttt{EchoChamberArts.org}} \\
Website Domain & \seqsplit{\texttt{www.EchoChamberArts.org}} \\
External IP Address & \seqsplit{\texttt{144.67.216.195}} \\
\bottomrule
\end{tabular}
\end{center}

% --- Section 3: Security Control Review ---
\section{Security Control Review}
An analysis of the security questionnaire reveals the current state of administrative and technical controls. "No" answers indicate significant gaps when measured against cybersecurity best practices.

\begin{table}[h!]
\centering
\caption{Security Controls Questionnaire Analysis}
\begin{tabular}{p{0.6\linewidth} c c}
\toprule
\textbf{Control Question} & \textbf{Response} & \textbf{Status} \\
\midrule
Do you require MFA to access email? & \yes & Implemented \\
Do you require MFA to log into computers? & \no & \textcolor{red}{\textbf{Critical Gap}} \\
Do you require MFA to access sensitive data systems? & \no & \textcolor{red}{\textbf{Critical Gap}} \\
Does your organization have an employee acceptable use policy? & \no & \textcolor{orange}{High Risk} \\
Does your organization do security awareness training for new employees? & \no & \textcolor{orange}{High Risk} \\
Does your organization do security awareness training for all employees at least once per year? & \yes & Implemented \\
\bottomrule
\end{tabular}
\end{table}

% --- Section 4: Technical Scan Results ---
\section{Technical Scan Results}
A network scan was performed to identify open ports and services on the target system. The results provide insight into the external and internal attack surface.

\begin{itemize}
    \item \textbf{Target IP Address:} \seqsplit{\texttt{127.0.0.1}}
    \item \textbf{Scan Date:} \today
\end{itemize}

The scan identified the following open port:

\begin{table}[h!]
\centering
\caption{Open Port Analysis for \texttt{127.0.0.1}}
\begin{tabular}{c c p{0.5\linewidth}}
\toprule
\textbf{Port} & \textbf{State} & \textbf{Analysis} \\
\midrule
22/tcp & Open & This port is standard for the Secure Shell (SSH) protocol. The scan did not retrieve service version details. An open SSH port on the localhost interface confirms the pre-existing risk "Localhost Exposed". This could be leveraged by malicious software on the machine to establish persistence or escalate privileges. \\
\bottomrule
\end{tabular}
\end{table}

% --- Section 5: Consolidated Risk Assessment ---
\section{Consolidated Risk Assessment}
The following table synthesizes findings from the security questionnaire, the technical scan, and pre-existing risk data into a prioritized list.

\begin{table}[h!]
\centering
\caption{Prioritized Risk Register}
\begin{tabular}{p{0.35\linewidth} p{0.45\linewidth} l}
\toprule
\textbf{Risk Name} & \textbf{Overview} & \textbf{Severity} \\
\midrule
\textbf{Localhost Exposed} & The SSH service is running and accessible on the local machine, posing a risk of local privilege escalation or misuse by other processes. & \textcolor{red}{Critical} \\
\textbf{No MFA on Sensitive Systems} & Lack of MFA on critical data systems allows an attacker with stolen credentials to gain direct access to sensitive information. & \textcolor{red}{Critical} \\
\textbf{No MFA on Endpoints} & Lack of MFA on computer logins allows for trivial takeover of an employee's workstation if their password is compromised. & \textcolor{red}{Critical} \\
\textbf{Missing Acceptable Use Policy} & Without a formal policy, there are no established rules for employee use of company assets, leading to potential misuse and security incidents. & \textcolor{orange}{High} \\
\textbf{No Onboarding Security Training} & New employees are not trained on security best practices, making them highly susceptible to phishing and other social engineering attacks from day one. & \textcolor{orange}{High} \\
\bottomrule
\end{tabular}
\end{table}

% --- Section 6: Recommendations ---
\section{Recommendations}
\label{sec:recommendations}
Based on the consolidated risk assessment, we provide the following actionable recommendations to mitigate the identified vulnerabilities and improve the overall security posture of Echo Chamber Arts.

\subsection{Immediate Actions (1-30 Days)}
\begin{enumerate}
    \item \textbf{Implement Multi-Factor Authentication (MFA):}
    \begin{itemize}
        \item \textbf{Priority 1:} Immediately enforce MFA for all access to systems classified as containing sensitive data.
        \item \textbf{Priority 2:} Roll out mandatory MFA for all employee computer logins (e.g., via Windows Hello, Duo, or similar).
    \end{itemize}
    
    \item \textbf{Investigate and Remediate Exposed Localhost Service:}
    \begin{itemize}
        \item Identify the purpose of the SSH service running on \texttt{127.0.0.1}.
        \item If the service is not required for a legitimate business function, it should be disabled immediately.
        \item If it is required, ensure it is securely configured, patched, and access is logged and monitored.
    \end{itemize}
\end{enumerate}

\subsection{Strategic Initiatives (30-90 Days)}
\begin{enumerate}
    \setcounter{enumi}{2} % Continue numbering from previous list
    \item \textbf{Develop and Implement Foundational Security Policies:}
    \begin{itemize}
        \item Draft and formally approve an \textbf{Acceptable Use Policy (AUP)} that clearly defines the rules for using company IT assets, data, and internet access.
        \item Require all current employees to read and acknowledge the policy.
    \end{itemize}

    \item \textbf{Establish a Security Awareness Training Program:}
    \begin{itemize}
        \item Integrate a mandatory security awareness training module into the new employee onboarding process. This should cover topics such as phishing, password security, and the new AUP.
        \item Continue the existing annual refresher training for all staff to ensure ongoing awareness.
    \end{itemize}
\end{enumerate}

\end{document}
```