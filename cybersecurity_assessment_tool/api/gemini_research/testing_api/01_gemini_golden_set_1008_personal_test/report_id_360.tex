Of course. As an expert-level Cybersecurity Analyst and LaTeX Report Generator, I will analyze the provided data, including the corrupted inputs, and synthesize it into a complete and professional security report.

The report will clearly state that the technical scan and pre-existing risk data were unavailable due to data corruption and will focus the analysis on the provided organizational questionnaire.

Here is the complete LaTeX report:

```latex
\documentclass[12pt]{article}

% --- PACKAGES ---
\usepackage[margin=1in]{geometry}
\usepackage{pifont} % For checkmarks and X's (\ding{51}, \ding{55})
\usepackage{booktabs} % For professional tables
\usepackage{hyperref} % For clickable links
\usepackage{url} % For URL formatting
\usepackage{seqsplit} % To split long strings without breaking
\usepackage[table]{xcolor} % For coloring table rows

% --- DOCUMENT METADATA ---
\hypersetup{
    colorlinks=true,
    linkcolor=blue,
    filecolor=magenta,      
    urlcolor=cyan,
    pdftitle={Cybersecurity Posture Report},
    pdfauthor={Cybersecurity Analyst},
    pdfsubject={Security Assessment},
    pdfkeywords={Security, Risk, Analysis},
}

% --- DOCUMENT START ---
\begin{document}

% --- TITLE PAGE ---
\begin{titlepage}
    \centering
    \vspace*{1cm}
    \Huge \textbf{Cybersecurity Posture Report}
    \vspace{1.5cm}
    \Large \textbf{Prepared for:} \\
    \vspace{0.5cm}
    \huge Apex Legends Group
    \vfill
    \large \textbf{Date of Report:} \\
    \vspace{0.5cm}
    \today
    \vspace{1.5cm}
    
    \normalsize This report is confidential and intended solely for the use of Apex Legends Group.
\end{titlepage}

\newpage

% --- TABLE OF CONTENTS ---
\tableofcontents
\newpage

% --- EXECUTIVE SUMMARY ---
\section*{1.0 Executive Summary}
This report provides an assessment of the cybersecurity posture for Apex Legends Group. The analysis is primarily based on a review of organizational security controls provided via a questionnaire. 

\textbf{Important Note:} The technical network scan data (\texttt{Input\_1\_Network\_Scan\_JSON}) and the list of pre-existing risks (\texttt{Input\_3\_Current\_Risks\_JSON}) were found to be corrupted and could not be processed. Consequently, this assessment does not include an analysis of external-facing technical vulnerabilities or previously tracked risks.

The key findings from the security control review indicate a mixed security posture. While the organization has implemented foundational controls such as an acceptable use policy and annual security training, there are \textbf{critical gaps} in access control. The absence of Multi-Factor Authentication (MFA) for computer logins and access to sensitive data systems represents a significant risk. Furthermore, the lack of security awareness training for new employees creates an immediate vulnerability upon hiring.

Immediate remediation should focus on deploying a comprehensive MFA solution and integrating security training into the employee onboarding process. A new technical scan should be conducted to identify and address potential vulnerabilities on the external network perimeter.

% --- ORGANIZATIONAL INFORMATION ---
\section*{2.0 Organizational Information}
The following details were provided for the assessment.

\begin{tabular}{@{}ll}
\toprule
\textbf{Attribute} & \textbf{Value} \\
\midrule
Organization Name & Apex Legends Group \\
Email Domain & \texttt{ApexLegendsGroup.org} \\
Website Domain & \url{www.ApexLegendsGroup.org} \\
External IP Address & \texttt{107.196.74.241} \\
\bottomrule
\end{tabular}

% --- SECURITY CONTROL REVIEW ---
\section*{3.0 Security Control Review}
The following table summarizes the organization's responses to the security controls questionnaire. "No" answers indicate significant gaps in the security framework and are highlighted as areas of concern.

\begin{table}[h!]
\centering
\caption{Security Controls Questionnaire Analysis}
\begin{tabular}{@{}p{8cm}cc}
\toprule
\textbf{Control Question} & \textbf{Response} & \textbf{Status} \\
\midrule
Do you require MFA to access email? & \ding{51} Yes & Implemented \\
\rowcolor{red!15} Do you require MFA to log into computers? & \ding{55} No & \textbf{Critical Gap} \\
\rowcolor{red!15} Do you require MFA to access sensitive data systems? & \ding{55} No & \textbf{Critical Gap} \\
Does your organization have an employee acceptable use policy? & \ding{51} Yes & Implemented \\
\rowcolor{red!15} Does your organization do security awareness training for new employees? & \ding{55} No & \textbf{High Risk Gap} \\
Does your organization do security awareness training for all employees at least once per year? & \ding{51} Yes & Implemented \\
\bottomrule
\end{tabular}
\end{table}

% --- TECHNICAL SCAN RESULTS ---
\section*{4.0 Technical Scan Results}
\subsection*{4.1 External Network Scan}
The external network scan targeting \texttt{[Target IP]} could not be completed successfully. The provided data file (\texttt{Input\_1\_Network\_Scan\_JSON}) was malformed or corrupted, preventing the extraction of technical details such as open ports, running services, and software versions. 

Without this data, a technical vulnerability assessment of the external perimeter cannot be performed. It is crucial to re-run this scan to identify potential weaknesses that could be exploited by external attackers.

% --- RISK ASSESSMENT ---
\section*{5.0 Risk Assessment}
This section details the risks identified based on the security control review. As the pre-existing risk data was unavailable, this table only reflects newly identified risks. Severity is rated on a scale of Critical, High, Medium, and Low.

\begin{table}[h!]
\centering
\caption{Identified Risks and Severity}
\begin{tabular}{@{}p{3.5cm}p{7.5cm}l}
\toprule
\textbf{Risk Name} & \textbf{Overview} & \textbf{Severity} \\
\midrule
\rowcolor{red!25} No MFA on Sensitive Systems & The absence of MFA on systems containing sensitive data means that a single compromised password could lead to a major data breach. & Critical \\
\addlinespace
\rowcolor{orange!25} No MFA on Endpoints & Lack of MFA on computer logins weakens defenses against unauthorized access and lateral movement within the network should an employee's credentials be stolen. & High \\
\addlinespace
\rowcolor{orange!25} Inadequate New Hire Training & New employees are not receiving security awareness training, making them highly susceptible to phishing and social engineering attacks from their first day. This creates a persistent weak link in the human firewall. & High \\
\bottomrule
\end{tabular}
\end{table}

% --- RECOMMENDATIONS ---
\section*{6.0 Recommendations}
The following actionable recommendations are provided to mitigate the identified risks and improve the overall security posture of Apex Legends Group.

\begin{table}[h!]
\centering
\caption{Remediation Plan}
\begin{tabular}{@{}p{3.5cm}p{9cm}l}
\toprule
\textbf{Priority} & \textbf{Recommendation} & \textbf{Associated Risk} \\
\midrule
\textbf{1. Critical} & Immediately deploy and enforce a strong, phishing-resistant MFA solution (e.g., FIDO2, TOTP) for all user and administrative access to sensitive data systems and critical infrastructure. & No MFA on Sensitive Systems \\
\addlinespace
\textbf{2. High} & Implement MFA for all employee and privileged user logins to company computers and laptops. This significantly raises the difficulty for an attacker to gain a foothold in the network. & No MFA on Endpoints \\
\addlinespace
\textbf{3. High} & Integrate mandatory security awareness training into the standard onboarding process for all new employees and contractors before they are granted system access. & Inadequate New Hire Training \\
\addlinespace
\textbf{4. Informational} & Commission a new, comprehensive external network vulnerability scan to identify and remediate technical weaknesses on the internet-facing perimeter. & Unassessed Technical Risk \\
\bottomrule
\end{tabular}
\end{table}

\end{document}
```