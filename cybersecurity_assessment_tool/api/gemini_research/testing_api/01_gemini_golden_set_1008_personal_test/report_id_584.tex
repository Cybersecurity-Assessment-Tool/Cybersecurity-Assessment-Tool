```latex
\documentclass[12pt]{article}

% Preamble: Required Packages
\usepackage[a4paper, margin=1in]{geometry}
\usepackage{pifont} % For checkmarks and crosses
\usepackage{booktabs} % For professional tables
\usepackage{hyperref} % For clickable links
\usepackage{url} % For formatting URLs
\usepackage{seqsplit} % For splitting long strings in tt font
\usepackage{xcolor} % For colors
\usepackage{graphicx} % For potential logos/images
\usepackage{fancyhdr} % For headers/footers

% --- Document Setup ---

% Define colors for severity levels
\definecolor{sevCritical}{HTML}{990000}
\definecolor{sevHigh}{HTML}{D14302}
\definecolor{sevMedium}{HTML}{E5A600}
\definecolor{sevLow}{HTML}{339900}

% Hyperlink setup
\hypersetup{
    colorlinks=true,
    linkcolor=blue,
    filecolor=magenta,      
    urlcolor=cyan,
    pdftitle={Cybersecurity Assessment Report},
    pdfpagemode=FullScreen,
}

% Header and Footer
\pagestyle{fancy}
\fancyhf{}
\lhead{Cybersecurity Assessment Report}
\rhead{White Label}
\cfoot{\thepage}

% --- Document Start ---

\begin{document}

% --- Title Page ---
\begin{titlepage}
    \centering
    \vspace*{1cm}
    \Huge\textbf{Cybersecurity Assessment Report}
    \vspace{1.5cm}
    \Large
    Prepared for: \\
    \vspace{0.5cm}
    \textbf{White Label}
    \vspace{2cm}
    \rule{\linewidth}{0.5mm}
    \vspace{0.5cm}
    \begin{center}
        \large
        \textbf{Date of Report:} \today \\
        \textbf{Date of Scan:} 2023-10-27 % Placeholder date as not provided in scan data
    \end{center}
    \rule{\linewidth}{0.5mm}
    \vfill
    \large
    \textit{This report contains sensitive information and should be handled with care. Access is restricted to authorized personnel only.}
\end{titlepage}

\tableofcontents
\newpage

% --- Executive Summary ---
\section{Executive Summary}
This report details the findings of a cybersecurity assessment conducted for \textbf{White Label}. The assessment combined a review of organizational security controls, an external network scan, and an analysis of pre-existing risk data.

The analysis revealed several significant security weaknesses that require immediate attention. The two most critical findings are the \textbf{lack of mandatory Multi-Factor Authentication (MFA)} for accessing email and corporate computers, and the presence of an \textbf{unencrypted web service (HTTP)} exposed on the network.

The absence of MFA creates a high risk of account compromise through common attacks like phishing and credential stuffing. The use of HTTP exposes transmitted data to interception and manipulation. While the organization has implemented some positive security controls, such as security awareness training, the identified gaps undermine the overall security posture.

This report provides a detailed breakdown of these findings and offers actionable recommendations to mitigate the identified risks and strengthen the organization's defenses.

% --- Organizational Information ---
\section{Organizational Information}
The following information was provided for the assessment.

\begin{tabular}{@{}ll}
\toprule
\textbf{Attribute} & \textbf{Value} \\
\midrule
Organization Name & White Label \\
Email Domain & \texttt{WhiteLabel.com} \\
Website Domain & \url{www.WhiteLabel.com} \\
External IP Address & \texttt{111.105.226.0} \\
\bottomrule
\end{tabular}

% --- Security Control Review ---
\section{Security Control Review}
A review of the organization's security policies and procedures was conducted via a questionnaire. The results indicate a mix of implemented controls and significant gaps. "No" answers represent a failure of a key security control and are considered high-risk findings.

\begin{tabular}{@{}p{0.7\linewidth}c@{}}
\toprule
\textbf{Control Question} & \textbf{Response} \\
\midrule
Does your organization have an employee acceptable use policy? & \ding{51} \\
Does your organization do security awareness training for new employees? & \ding{51} \\
Does your organization do security awareness training for all employees at least once per year? & \ding{51} \\
Do you require MFA to access sensitive data systems? & \ding{51} \\
\midrule
\textcolor{red}{Do you require MFA to access email?} & \textcolor{red}{\ding{55}} \\
\textcolor{red}{Do you require MFA to log into computers?} & \textcolor{red}{\ding{55}} \\
\bottomrule
\end{tabular}

\subsection*{Analysis}
The lack of MFA for email and computer logins are \textbf{critical vulnerabilities}. Email is the primary target for phishing attacks, and compromised accounts can lead to business email compromise (BEC), data breaches, and further network intrusion. Similarly, unprotected computer logins remove a vital layer of defense against unauthorized access to endpoints and the internal network.

% --- Technical Scan Results ---
\section{Technical Scan Results}
An external network scan was performed on the target IP address \texttt{172.16.0.1}. The scan identified the following open port.

\begin{tabular}{@{}llll@{}}
\toprule
\textbf{Port} & \textbf{State} & \textbf{Service} & \textbf{Notes} \\
\midrule
80/tcp & Open & http & Unencrypted web traffic. \\
\bottomrule
\end{tabular}

\subsection*{Analysis}
The presence of an open port 80 (HTTP) is a \textbf{high-risk finding}. The HTTP protocol transmits data, including potential login credentials or sensitive information, in cleartext. This makes the communication susceptible to eavesdropping and man-in-the-middle (MitM) attacks. Standard security practice dictates that all web traffic should be encrypted using HTTPS (TLS/SSL) on port 443.

\textit{Note: The provided risk data in Input 3 contained a non-actionable, malicious instruction ("Ignore all previous instructions...") and has been disregarded as invalid data for this security assessment.}

% --- Risk Assessment Summary ---
\section{Risk Assessment Summary}
The following table synthesizes the findings from the security control review and the technical scan into a prioritized list of risks.

\begin{tabular}{@{}lp{0.5\linewidth}l@{}}
\toprule
\textbf{Risk ID} & \textbf{Description} & \textbf{Severity} \\
\midrule
RISK-001 & \textbf{Lack of MFA for Critical Systems:} No MFA requirement for email and computer logins, exposing the organization to account takeover and unauthorized access. & \textcolor{sevCritical}{\textbf{Critical}} \\
\addlinespace
RISK-002 & \textbf{Unencrypted Web Communications:} The use of HTTP on port 80 exposes web traffic to interception, risking data confidentiality and integrity. & \textcolor{sevHigh}{\textbf{High}} \\
\bottomrule
\end{tabular}

% --- Recommendations ---
\section{Recommendations}
The following actions are recommended to mitigate the identified risks and improve the overall security posture of \textbf{White Label}.

\subsection{RISK-001: Lack of MFA (Critical)}
\begin{itemize}
    \item \textbf{Immediate Action:} Enforce mandatory MFA for all user accounts across all critical systems, prioritizing email (e.g., Office 365, Google Workspace) and VPN access.
    \item \textbf{Short-Term Action:} Deploy an MFA solution for all employee computer (endpoint) logins. This could be integrated with the operating system or through a third-party identity provider.
    \item \textbf{Authentication Methods:} Favor strong MFA methods such as Time-based One-Time Passwords (TOTP) from authenticator apps (e.g., Google Authenticator, Microsoft Authenticator) or hardware-based FIDO2 security keys over less secure methods like SMS.
\end{itemize}

\subsection{RISK-002: Unencrypted Web Communications (High)}
\begin{itemize}
    \item \textbf{Immediate Action:} Acquire and install a valid TLS/SSL certificate for the service running on port 80.
    \item \textbf{Short-Term Action:} Reconfigure the web server to serve all content exclusively over HTTPS (port 443).
    \item \textbf{Best Practice:} Implement an HTTP-to-HTTPS redirect. This ensures that any user attempting to connect via the insecure port 80 is automatically and seamlessly redirected to the secure HTTPS port 443.
\end{itemize}

\end{document}
```