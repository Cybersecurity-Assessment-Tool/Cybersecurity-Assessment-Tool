```latex
\documentclass[12pt]{article}

% Preamble: Required Packages
\usepackage[margin=1in]{geometry}
\usepackage{pifont} % For checkmarks and crosses (\ding{51} and \ding{55})
\usepackage{booktabs} % For professional-looking tables
\usepackage{hyperref} % For clickable links
\usepackage{url} % For formatting URLs
\usepackage{seqsplit} % For splitting long strings in texttt
\usepackage{graphicx}
\usepackage{xcolor}

% Document Information
\title{Cybersecurity Posture Assessment Report}
\author{Cybersecurity Analysis Division}
\date{\today}

% Hyperref Setup
\hypersetup{
    colorlinks=true,
    linkcolor=blue,
    filecolor=magenta,      
    urlcolor=cyan,
    pdftitle={Cybersecurity Posture Assessment Report},
    pdfpagemode=FullScreen,
}

\begin{document}

\maketitle
\thispagestyle{empty}
\newpage

\tableofcontents
\newpage

% --- 1. Executive Summary ---
\section{Executive Summary}

This report provides a cybersecurity posture assessment for \textbf{Astraeus Aerospace}, conducted on \today. The analysis is based on a review of organizational security controls, an external network scan, and a summary of pre-existing risks.

The overall security posture of the organization shows a strong foundation in policy and user awareness. Controls such as Multi-Factor Authentication (MFA) for email and sensitive systems are commendable. However, a \textbf{critical control gap} was identified: the absence of mandatory MFA for logging into employee computers. This represents a significant risk, as compromised user credentials could lead directly to endpoint and internal network access.

The external network scan performed against the target IP address \seqsplit{\texttt{55.13.126.96}} did not identify any open ports. This is a positive finding, suggesting a well-configured perimeter firewall that properly implements the principle of least privilege by denying all unsolicited inbound traffic. No pre-existing vulnerabilities were reported for assessment.

Our primary recommendation is the immediate implementation of an MFA solution for all endpoint logins to mitigate the risk of unauthorized access.

% --- 2. Organizational Information ---
\section{Organizational Information}

The following information was provided for the assessment.

\begin{table}[h!]
\centering
\begin{tabular}{@{}ll@{}}
\toprule
\textbf{Attribute} & \textbf{Value} \\ \midrule
Organization Name    & \textbf{Astraeus Aerospace} \\
Email Domain         & \texttt{AstraeusAerospace.com} \\
Website Domain       & \url{www.AstraeusAerospace.com} \\
External IP Address  & \seqsplit{\texttt{55.13.126.96}} \\ \bottomrule
\end{tabular}
\caption{Client Organizational Data}
\label{tab:org_data}
\end{table}

% --- 3. Security Control Review ---
\section{Security Control Review}

A review of self-reported security controls was conducted based on a standardized questionnaire. The results highlight the organization's current policies and identify potential areas for improvement. A (\ding{51}) indicates a positive control is in place, while a (\ding{55}) indicates a control gap.

\begin{table}[h!]
\centering
\begin{tabular}{@{}lc@{}}
\toprule
\textbf{Security Control Question} & \textbf{Status} \\ \midrule
Do you require MFA to access email? & \ding{51} \\
\textbf{Do you require MFA to log into computers?} & \textcolor{red}{\ding{55}} \\
Do you require MFA to access sensitive data systems? & \ding{51} \\
Does your organization have an employee acceptable use policy? & \ding{51} \\
Does your organization do security awareness training for new employees? & \ding{51} \\
Does your organization do security awareness training for all employees at least once per year? & \ding{51} \\ \bottomrule
\end{tabular}
\caption{Security Controls Questionnaire Results}
\label{tab:controls}
\end{table}

\subsection*{Analysis}
The organization has implemented several key security controls effectively, particularly concerning MFA for critical systems and a robust security awareness program. However, the lack of MFA for computer logins is a significant weakness. If an employee's credentials are stolen (e.g., through a phishing attack), an attacker could gain direct access to their workstation and, consequently, the internal network.

% --- 4. Technical Scan Results ---
\section{Technical Scan Results}

An external network vulnerability scan was conducted against the provided IP address.

\begin{itemize}
    \item \textbf{Target IP:} \texttt{[Target IP]} (Note: Target IP was not specified in the scan data; the organizational IP \seqsplit{\texttt{55.13.126.96}} is assumed to be the intended target).
    \item \textbf{Scan Date:} Not provided in scan data.
\end{itemize}

\subsection*{Findings}
The scan results were empty, indicating that \textbf{no open ports or services were discovered} on the target host. This is a positive security finding. It suggests that the perimeter firewall is effectively configured to block all unsolicited incoming connections, which significantly reduces the external attack surface. This could also indicate the host was offline or configured not to respond to scan probes.

% --- 5. Risk Assessment and Findings ---
\section{Risk Assessment and Findings}

This section synthesizes the findings from the security control review and technical scan to provide a consolidated list of identified risks.

\begin{table}[h!]
\centering
\begin{tabular}{@{}p{0.2\linewidth}p{0.6\linewidth}p{0.1\linewidth}@{}}
\toprule
\textbf{Risk Name} & \textbf{Overview} & \textbf{Severity} \\ \midrule
\textbf{Lack of Endpoint MFA} & The absence of Multi-Factor Authentication for computer logins exposes the organization to significant risk. A single compromised password could grant an attacker full access to an employee's workstation and the internal network, bypassing other security controls. & \textbf{High} \\
\addlinespace
No External Services Exposed & The network scan found no open ports on the target IP. While this is a positive finding, it underscores the importance of securing internal assets, as the primary threat vector shifts from external exploitation to attacks originating from compromised endpoints. & Low \\
\bottomrule
\end{tabular}
\caption{Consolidated Risk Summary}
\label{tab:risks}
\end{table}

% --- 6. Recommendations ---
\section{Recommendations}

Based on the analysis, the following recommendations are provided to enhance the cybersecurity posture of \textbf{Astraeus Aerospace}.

\subsection*{Priority 1: Critical}
\begin{itemize}
    \item \textbf{Implement MFA for All Endpoint Logins:}
    \begin{itemize}
        \item \textbf{Action:} Deploy a mandatory Multi-Factor Authentication solution for all employee computer and laptop logins (Windows, macOS, Linux).
        \item \textbf{Justification:} This is the most critical action to take. It closes the security gap identified in the control review and provides a vital layer of defense against credential theft and unauthorized access.
        \item \textbf{Examples:} Solutions include Windows Hello for Business, Duo Security, Okta, or other FIDO2-compliant hardware keys.
    \end{itemize}
\end{itemize}

\subsection*{Priority 2: High}
\begin{itemize}
    \item No high-priority recommendations were identified beyond the critical need for endpoint MFA.
\end{itemize}

\subsection*{Priority 3: Medium}
\begin{itemize}
    \item \textbf{Conduct Regular Vulnerability Scanning:}
    \begin{itemize}
        \item \textbf{Action:} Establish a formal program for regular, authenticated internal and external vulnerability scanning.
        \item \textbf{Justification:} While the external posture appears strong, continuous scanning ensures that new vulnerabilities or misconfigurations are identified and remediated promptly. Authenticated internal scans are necessary to identify weaknesses that are not visible from the outside.
    \end{itemize}
\end{itemize}

\end{document}
```