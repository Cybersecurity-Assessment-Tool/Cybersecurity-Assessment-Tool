```latex
\documentclass[12pt]{article}

% --- PACKAGES ---
\usepackage[margin=1in]{geometry}
\usepackage{pifont} % Required for \ding
\usepackage{booktabs} % For professional tables
\usepackage{hyperref} % For hyperlinks, URLs
\usepackage{url}
\usepackage{seqsplit} % To split long strings in \texttt
\usepackage{graphicx}
\usepackage{fancyhdr}
\usepackage{xcolor}
\usepackage[utf8]{inputenc}

% --- DOCUMENT CONFIGURATION ---
\hypersetup{
    colorlinks=true,
    linkcolor=blue,
    filecolor=magenta,      
    urlcolor=cyan,
    pdftitle={Cybersecurity Posture Assessment Report},
    pdfpagemode=FullScreen,
}

\pagestyle{fancy}
\fancyhf{}
\fancyhead[L]{Cybersecurity Posture Assessment}
\fancyhead[R]{\textbf{Mainframe Managed}}
\fancyfoot[C]{\thepage}

% --- DOCUMENT START ---
\begin{document}

\title{
    \vspace{2cm}
    \textbf{Cybersecurity Posture Assessment Report} \\
    \large \textbf{Prepared for: Mainframe Managed}
    \vspace{1.5cm}
}
\author{Cybersecurity Analysis Division}
\date{\today}
\maketitle

\thispagestyle{empty}
\newpage

\tableofcontents
\newpage

% ==============================================================================
\section{Executive Summary}
% ==============================================================================

This report provides a comprehensive cybersecurity assessment for \textbf{Mainframe Managed}, based on data from organizational questionnaires, external network scans, and a review of known risks. The analysis was conducted on \today.

\paragraph{Key Findings:} The organization demonstrates a strong commitment to identity and access management, with Multi-Factor Authentication (MFA) consistently enforced across email, workstations, and sensitive data systems. This significantly reduces the risk of unauthorized access via compromised credentials.

However, two critical administrative control gaps were identified:
\begin{itemize}
    \item \textbf{Lack of an Acceptable Use Policy (AUP):} The absence of a formal AUP creates ambiguity for employees and increases the risk of insider threats, whether malicious or unintentional.
    * \textbf{Inadequate Security Awareness Training:} While new employees receive training, the lack of a mandatory annual refresher for all staff leaves the organization vulnerable to evolving social engineering tactics like phishing.
\end{itemize}

The external network scan of the target IP address, \texttt{[Target IP]}, did not reveal any open ports or exposed services. This suggests a well-configured perimeter firewall. No pre-existing vulnerabilities were reported for assessment.

\paragraph{Overall Posture:} \textbf{Mainframe Managed} has established a solid technical foundation for security. The immediate priority should be to address the identified policy and training deficiencies to build a more resilient, security-conscious culture and mitigate human-centric risks.

% ==============================================================================
\section{Organizational Information}
% ==============================================================================

The following information was provided for the assessment:

\begin{tabular}{@{}ll}
    \toprule
    \textbf{Attribute} & \textbf{Value} \\
    \midrule
    Organization Name & Mainframe Managed \\
    Email Domain & \texttt{MainframeManaged.com} \\
    Website Domain & \url{www.MainframeManaged.com} \\
    External IP Address & \texttt{23.222.222.253} \\
    \bottomrule
\end{tabular}

% ==============================================================================
\section{Security Control Review}
% ==============================================================================

The following table summarizes the organization's responses to the security controls questionnaire. Items marked with a red 'X' (\textcolor{red}{\ding{55}}) represent significant gaps in the security program and are addressed in the Risk Assessment section.

\begin{table}[h!]
\centering
\begin{tabular}{@{}p{0.8\linewidth}c@{}}
    \toprule
    \textbf{Control Question} & \textbf{Response} \\
    \midrule
    Do you require MFA to access email? & \ding{51} \\
    Do you require MFA to log into computers? & \ding{51} \\
    Do you require MFA to access sensitive data systems? & \ding{51} \\
    Does your organization have an employee acceptable use policy? & \textcolor{red}{\ding{55}} \\
    Does your organization do security awareness training for new employees? & \ding{51} \\
    Does your organization do security awareness training for all employees at least once per year? & \textcolor{red}{\ding{55}} \\
    \bottomrule
\end{tabular}
\caption{Security Controls Questionnaire Results}
\end{table}

% ==============================================================================
\section{Technical Scan Results}
% ==============================================================================

An external network vulnerability scan was conducted to identify exposed services and potential weaknesses on the organization's perimeter.

\begin{itemize}
    \item \textbf{Target IP Address:} \texttt{[Target IP]}
    \item \textbf{Scan Date:} \today
\end{itemize}

\paragraph{Findings:} The scan did not identify any open TCP or UDP ports on the target system. This is a positive finding, suggesting that perimeter firewalls are effectively configured to deny unsolicited inbound traffic, adhering to the principle of least privilege. No vulnerabilities were discovered.

% ==============================================================================
\section{Risk Assessment}
% ==============================================================================

This section synthesizes findings from the security control review and technical scans to identify and prioritize risks.

\begin{table}[h!]
\centering
\begin{tabular}{@{}p{0.15\linewidth} p{0.55\linewidth} p{0.15\linewidth}@{}}
    \toprule
    \textbf{Risk Name} & \textbf{Overview} & \textbf{Severity} \\
    \midrule
    \textbf{Lack of Acceptable Use Policy} & The absence of a formal policy defining the proper use of company assets (e.g., internet, email, devices) increases the likelihood of misuse, data leakage, and legal liability. & \textbf{High} \\
    \addlinespace
    \textbf{Inadequate Security Training} & Without mandatory, recurring security awareness training, employees are less prepared to identify and resist social engineering attacks (e.g., phishing, pretexting), making them a primary target for threat actors. & \textbf{High} \\
    \bottomrule
\end{tabular}
\caption{Identified Risks and Severity}
\end{table}

% ==============================================================================
\section{Recommendations}
% ==============================================================================

The following actions are recommended to mitigate the identified risks and strengthen the overall security posture of \textbf{Mainframe Managed}.

\begin{enumerate}
    \item \textbf{Develop and Implement an Acceptable Use Policy (AUP):}
    \begin{itemize}
        \item \textbf{Action:} Draft a comprehensive AUP that clearly outlines the rules and expectations for using company technology and data. This policy should be reviewed by legal counsel.
        \item \textbf{Details:} The AUP should be distributed to all current employees and integrated into the onboarding process for new hires. Require all employees to read and formally acknowledge the policy with a signature.
        \item \textbf{Impact:} Establishes a clear security baseline for employee behavior and provides an enforceable framework for policy violations.
    \end{itemize}
    \vspace{0.5cm}
    \item \textbf{Establish a Mandatory Annual Security Awareness Training Program:}
    \begin{itemize}
        \item \textbf{Action:} Implement a formal, mandatory security awareness training program that is required for all employees on an annual basis.
        \item \textbf{Details:} The training should cover current threats, including phishing identification, password security, safe browsing habits, and physical security. Consider running periodic, unannounced phishing simulations to test and reinforce the training.
        \item \textbf{Impact:} Reduces the organization's susceptibility to human-centric attacks and fosters a more security-aware culture.
    \end{itemize}
\end{enumerate}

\end{document}
```