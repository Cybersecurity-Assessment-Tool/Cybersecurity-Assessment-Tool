```latex
\documentclass[12pt, a4paper]{article}

% Preamble: Required Packages
\usepackage[margin=1in]{geometry}
\usepackage{pifont} % For checkmarks and crosses (\ding)
\usepackage{booktabs} % For professional tables
\usepackage{hyperref} % For clickable links and references
\usepackage{url} % For formatting URLs
\usepackage{seqsplit} % To split long strings in tt font
\usepackage{graphicx}
\usepackage{xcolor}
\usepackage{fancyhdr}
\usepackage{lastpage}

% --- Document Setup ---

% Define colors for the report
\definecolor{PrimaryColor}{RGB}{0, 32, 96} % A deep blue
\definecolor{SecondaryColor}{RGB}{100, 100, 100} % A medium gray

% Hyperlink setup
\hypersetup{
    colorlinks=true,
    linkcolor=PrimaryColor,
    urlcolor=PrimaryColor,
    citecolor=PrimaryColor,
    pdftitle={Cybersecurity Posture Report},
    pdfauthor={Cybersecurity Analyst},
    pdfsubject={Security Assessment}
}

% Header and Footer Configuration
\pagestyle{fancy}
\fancyhf{} % Clear all header and footer fields
\fancyhead[L]{\textbf{Cybersecurity Posture Report}}
\fancyhead[R]{\textbf{Nomad Gear Co.}}
\fancyfoot[C]{\thepage\ of \pageref{LastPage}}
\renewcommand{\headrulewidth}{0.4pt}
\renewcommand{\footrulewidth}{0.4pt}

% --- Document Start ---
\begin{document}

% --- Title Page ---
\begin{titlepage}
    \centering
    \vspace*{2cm}
    
    {\Huge \textbf{Cybersecurity Posture Report}\par}
    \vspace{1.5cm}
    
    {\Large \textbf{Prepared For:}\par}
    \vspace{0.5cm}
    {\Large Nomad Gear Co.\par}
    
    \vfill % Pushes content to the bottom
    
    {\large \today\par}
    \vspace{0.5cm}
    {\large Confidential\par}
    
\end{titlepage}

\newpage
\tableofcontents
\newpage

% --- Section 1: Executive Summary ---
\section{Executive Summary}

This report provides a comprehensive analysis of the cybersecurity posture for \textbf{Nomad Gear Co.}. The assessment is based on a correlation of data from a security controls questionnaire, an external network scan, and a review of pre-existing risks.

The assessment identified two significant, high-impact security gaps in the organization's current policies and procedures. While foundational controls like Multi-Factor Authentication (MFA) for email and computer access are in place, critical weaknesses exist that expose the organization to substantial risk:
\begin{itemize}
    \item \textbf{Critical Risk - Lack of MFA on Sensitive Systems:} The absence of MFA for accessing sensitive data systems is a critical vulnerability. This significantly increases the risk of unauthorized access and data breach should an attacker compromise user credentials.
    \item \textbf{High Risk - No Security Training for New Employees:} New hires are not provided with security awareness training. This oversight makes them highly susceptible to social engineering attacks, such as phishing, which is a primary vector for initial network compromise.
\end{itemize}

The external network scan conducted against the target IP address \texttt{[Target IP]} did not detect any open ports. While this can indicate a strong firewall configuration, it may also suggest the host was offline or the scan was blocked. This finding requires further investigation to be validated.

In summary, while Nomad Gear Co. has implemented some important security measures, the identified gaps require immediate attention. The recommendations in this report provide a clear, actionable roadmap to mitigate these risks and strengthen the organization's overall security posture.

\newpage

% --- Section 2: Organizational Information ---
\section{Organizational Information}

This section details the organizational data provided for this assessment.

\begin{table}[h!]
\centering
\begin{tabular}{@{}ll@{}}
\toprule
\textbf{Attribute} & \textbf{Value} \\ \midrule
Organization Name & Nomad Gear Co. \\
Email Domain & \texttt{NomadGearCo.net} \\
Website Domain & \seqsplit{\url{www.NomadGearCo.net}} \\
External IP Address & \texttt{32.63.17.171} \\ \bottomrule
\end{tabular}
\caption{Client Organizational Details.}
\label{tab:org_info}
\end{table}


% --- Section 3: Security Control Review ---
\section{Security Control Review}

The following table summarizes the responses from the security controls questionnaire. Each response has been assessed against industry best practices. Items marked with \ding{55} represent significant gaps in the security framework.

\begin{table}[h!]
\centering
\begin{tabular}{@{}p{8cm}cc@{}}
\toprule
\textbf{Control Question} & \textbf{Response} & \textbf{Assessment} \\ \midrule
Do you require MFA to access email? & Yes & \ding{51} \\
Do you require MFA to log into computers? & Yes & \ding{51} \\
\textbf{Do you require MFA to access sensitive data systems?} & \textbf{No} & \textbf{\color{red}\ding{55} Critical Gap} \\
Does your organization have an employee acceptable use policy? & Yes & \ding{51} \\
\textbf{Does your organization do security awareness training for new employees?} & \textbf{No} & \textbf{\color{orange}\ding{55} High Risk} \\
Does your organization do security awareness training for all employees at least once per year? & Yes & \ding{51} \\ \bottomrule
\end{tabular}
\caption{Security Controls Questionnaire Analysis.}
\label{tab:controls}
\end{table}

\subsection*{Analysis of Gaps}
\begin{itemize}
    \item \textbf{MFA on Sensitive Systems:} The lack of MFA on systems holding sensitive data is a critical oversight. Stolen credentials are a common attack vector, and MFA is the single most effective control to prevent credential-based account takeovers.
    \item \textbf{New Employee Training:} Failing to train new employees on security best practices from day one leaves a window of vulnerability. New staff are often eager to be helpful and may be more susceptible to phishing or other social engineering tactics.
\end{itemize}

\newpage

% --- Section 4: Technical Scan Results ---
\section{Technical Scan Results}

An external network vulnerability scan was performed to identify open ports and services exposed to the internet.

\begin{table}[h!]
\centering
\begin{tabular}{@{}ll@{}}
\toprule
\textbf{Scan Parameter} & \textbf{Value} \\ \midrule
Target IP Address & \texttt{[Target IP]} \\
Scan Date & \today \\
\textbf{Result} & \textbf{No open ports detected.} \\ \bottomrule
\end{tabular}
\caption{Network Scan Summary.}
\label{tab:scan_summary}
\end{table}

\subsection*{Analyst's Note}
The scan did not identify any open TCP or UDP ports on the target host. This outcome can be interpreted in several ways:
\begin{enumerate}
    \item The host is protected by a well-configured firewall that correctly blocks all unsolicited inbound traffic.
    \item The host was offline or unreachable at the time of the scan.
    \item The scan was actively blocked by an Intrusion Prevention System (IPS) or similar network security appliance.
\end{enumerate}
While a lack of open ports is generally a positive security finding, it should be validated to ensure it is the result of intentional configuration and not a network issue.

% --- Section 5: Risk Assessment ---
\section{Risk Assessment}

This section synthesizes the findings from the questionnaire and technical scan into a prioritized list of identified risks. No pre-existing vulnerabilities were provided for this assessment.

\begin{table}[h!]
\centering
\begin{tabular}{@{}lp{6cm}l@{}}
\toprule
\textbf{Risk ID} & \textbf{Risk Name \& Description} & \textbf{Severity} \\ \midrule
\textbf{RISK-001} & \textbf{Lack of MFA on Sensitive Systems} \newline \small{Without MFA, sensitive data repositories are vulnerable to compromise via stolen or weak credentials, potentially leading to a major data breach.} & \textbf{\color{red}Critical} \\
\addlinespace
\textbf{RISK-002} & \textbf{Inadequate New Employee Onboarding} \newline \small{New employees without initial security training represent a significant weak point and are prime targets for phishing and social engineering attacks.} & \textbf{\color{orange}High} \\
\bottomrule
\end{tabular}
\caption{Summary of Identified Risks.}
\label{tab:risks}
\end{table}

\newpage

% --- Section 6: Recommendations ---
\section{Recommendations}

The following actions are recommended to mitigate the identified risks and improve the overall security posture of Nomad Gear Co.

\subsection*{Remediation for Identified Risks}
\begin{enumerate}
    \item \textbf{Implement MFA on All Sensitive Systems (RISK-001 - Critical):}
    \begin{itemize}
        \item \textbf{Action:} Immediately begin a project to enforce MFA on all applications, databases, and administrative interfaces that contain or provide access to sensitive company or customer data (e.g., financial systems, CRM, HR platforms).
        \item \textbf{Priority:} Immediate. This should be the top security priority.
    \end{itemize}
    \vspace{0.5cm}
    
    \item \textbf{Integrate Security Training into Employee Onboarding (RISK-002 - High):}
    \begin{itemize}
        \item \textbf{Action:} Develop a mandatory security awareness training module for all new hires. This training must be completed during their first week of employment. It should cover, at a minimum: phishing awareness, password security, acceptable use of company assets, and how to report a security incident.
        \item \textbf{Priority:} High. This should be implemented within the next quarter.
    \end{itemize}
\end{enumerate}

\subsection*{Validation of Technical Findings}
\begin{enumerate}
    \setcounter{enumi}{2} % Continue numbering
    \item \textbf{Validate Network Scan Results:}
    \begin{itemize}
        \item \textbf{Action:} Coordinate with the network and system administration teams to confirm that the target host \texttt{[Target IP]} was online and accessible during the scan period. A re-scan should be scheduled to validate that the lack of open ports is the expected behavior from the external firewall.
        \item \textbf{Priority:} Medium.
    \end{itemize}
\end{enumerate}

\end{document}
```