```latex
\documentclass[12pt]{article}

% --- PACKAGE IMPORTS ---
\usepackage[margin=1in]{geometry}
\usepackage{pifont} % For checkmarks and crosses
\usepackage{booktabs} % For professional-looking tables
\usepackage{hyperref} % For clickable links
\usepackage{url} % For URL formatting
\usepackage{seqsplit} % To split long strings in texttt
\usepackage{graphicx}
\usepackage{xcolor}

% --- DOCUMENT METADATA ---
\hypersetup{
    colorlinks=true,
    linkcolor=blue,
    filecolor=magenta,      
    urlcolor=cyan,
    pdftitle={Cybersecurity Posture Assessment Report},
    pdfauthor={Cybersecurity Analyst},
    pdfsubject={Security Analysis},
    pdfkeywords={Security, Nmap, Risk Assessment},
}

% --- COMMAND DEFINITIONS ---
\newcommand{\yes}{\ding{51}} % Checkmark
\newcommand{\no}{\ding{55}}  % Cross

% --- DOCUMENT START ---
\begin{document}

% --- TITLE PAGE ---
\begin{titlepage}
    \centering
    \vspace*{1cm}
    \Huge\textbf{Cybersecurity Posture Assessment Report}
    \vspace{1.5cm}
    \Large
    \textbf{Prepared for:}\\
    Hearth \& Home
    \vspace{2cm}
    \rule{\linewidth}{0.5mm}
    \vspace{0.4cm}
    \Large \textbf{CONFIDENTIAL}
    \vspace{0.4cm}
    \rule{\linewidth}{0.5mm}
    \vfill
    \large
    \textbf{Date of Report:}\\
    \today
    \vspace{1cm}
\end{titlepage}

\tableofcontents
\newpage

% --- EXECUTIVE SUMMARY ---
\section*{Executive Summary}
This report provides a comprehensive cybersecurity assessment for Hearth \& Home, based on a combination of network scanning, organizational data review, and an analysis of pre-existing risks. The assessment reveals several critical and high-risk vulnerabilities that require immediate attention to mitigate the potential for significant security incidents.

The most critical finding is an externally exposed FTP server running a dangerously outdated and vulnerable version of \texttt{vsftpd} (2.3.4), which is susceptible to a known remote code execution backdoor (CVE-2011-2523). This is compounded by the server's configuration, which permits anonymous, unauthenticated access.

Furthermore, significant policy gaps were identified, most notably the lack of Multi-Factor Authentication (MFA) for computer and sensitive data system access. The absence of annual security awareness training for all employees also presents a considerable risk, increasing the organization's susceptibility to social engineering and phishing attacks.

Immediate remediation of the vulnerable FTP server and the phased implementation of MFA are the highest priorities. Addressing these issues will substantially improve the organization's security posture and resilience against common cyber threats.

% --- ORGANIZATIONAL INFORMATION ---
\section*{1. Organizational Information}
The following details were provided for the assessment. This information forms the baseline for understanding the organization's digital footprint.

\begin{tabular}{@{}ll}
    \toprule
    \textbf{Attribute} & \textbf{Value} \\
    \midrule
    Organization Name & Hearth \& Home \\
    Email Domain & \seqsplit{\texttt{HearthHome.net}} \\
    Website Domain & \seqsplit{\url{www.HearthHome.net}} \\
    External IP Address & \seqsplit{\texttt{45.188.26.170}} \\
    \bottomrule
\end{tabular}

% --- SECURITY CONTROL REVIEW ---
\section*{2. Security Control Review}
An analysis of the organization's security questionnaire responses indicates several gaps in foundational security controls. "No" answers highlight areas where current policies or implementations fall short of established best practices.

\begin{table}[h!]
\centering
\caption{Security Questionnaire Analysis}
\begin{tabular}{@{}p{0.8\linewidth}c@{}}
    \toprule
    \textbf{Question / Control} & \textbf{Response} \\
    \midrule
    Do you require MFA to access email? & \yes \\
    Do you require MFA to log into computers? & \no \\
    Do you require MFA to access sensitive data systems? & \no \\
    Does your organization have an employee acceptable use policy? & \yes \\
    Does your organization do security awareness training for new employees? & \yes \\
    Does your organization do security awareness training for all employees at least once per year? & \no \\
    \bottomrule
\end{tabular}
\end{table}

\subsection*{Analysis of Gaps}
\begin{itemize}
    \item \textbf{Lack of MFA on Computers \& Sensitive Systems:} This is a high-risk gap. Without MFA, a single compromised password could grant an attacker full access to an employee's workstation and, subsequently, to critical company data.
    \item \textbf{Lack of Annual Security Training:} While training new hires is a good first step, the threat landscape evolves continuously. The absence of annual refresher training for all staff leaves the organization vulnerable to modern phishing and social engineering tactics.
\end{itemize}

% --- TECHNICAL SCAN RESULTS ---
\section*{3. Technical Scan Results}
An active network scan was performed on the target system to identify open ports and exposed services. The results indicate a critical misconfiguration.

\begin{itemize}
    \item \textbf{Target IP Address:} \texttt{10.0.0.15}
    \item \textbf{Scan Date:} 2023-10-27 (Simulated)
\end{itemize}

\begin{table}[h!]
\centering
\caption{Open Port Analysis}
\begin{tabular}{@{}llllll@{}}
    \toprule
    \textbf{Port} & \textbf{State} & \textbf{Service} & \textbf{Product} & \textbf{Version} & \textbf{Notes} \\
    \midrule
    21/tcp & open & ftp & vsftpd & 2.3.4 & \begin{tabular}[t]{@{}l@{}}\textbf{CRITICAL:} Anonymous FTP login allowed. \\ Version is vulnerable to CVE-2011-2523 \\ (Backdoor Command Execution).\end{tabular} \\
    \bottomrule
\end{tabular}
\end{table}

\subsection*{Analysis of Technical Findings}
The scan identified a single open port, 21 (FTP), running \textbf{vsftpd version 2.3.4}. This version is over a decade old and contains a well-documented, critical backdoor vulnerability (\textbf{CVE-2011-2523}). An attacker can exploit this flaw to execute arbitrary commands on the server with root privileges. The issue is exacerbated by the server's configuration, which allows \textbf{anonymous FTP login}, meaning no authentication is required to access the service. This represents an immediate and severe threat to the integrity and confidentiality of the server and the network it resides on.

% --- RISK ASSESSMENT ---
\section*{4. Risk Assessment Summary}
The following table synthesizes findings from the security control review, technical scan, and pre-existing risk data into a prioritized list.

\begin{table}[h!]
\centering
\caption{Consolidated Risk Register}
\begin{tabular}{@{}lp{0.5\linewidth}ll@{}}
    \toprule
    \textbf{ID} & \textbf{Risk Description} & \textbf{Severity} & \textbf{Affected Systems} \\
    \midrule
    R-01 & Vulnerable FTP server (\texttt{vsftpd 2.3.4}) with anonymous access enabled. & \textbf{Critical} & Server at \texttt{10.0.0.15} \\
    R-02 & No MFA required for access to sensitive data systems. & High & Data Systems, User Accounts \\
    R-03 & No MFA required for computer logins. & High & Workstations, User Accounts \\
    R-04 & Lack of annual security awareness training for all employees. & Medium & All Employees \\
    R-05 & Workstations running outdated Windows 7 operating systems. & Medium & Workstations \\
    \bottomrule
\end{tabular}
\end{table}

% --- RECOMMENDATIONS ---
\section*{5. Recommendations}
Based on the risk assessment, the following actions are recommended to improve the organization's security posture. Recommendations are prioritized by severity.

\subsection*{Immediate Priority (Critical)}
\begin{enumerate}
    \item \textbf{Remediate Vulnerable FTP Server (R-01):}
    \begin{itemize}
        \item \textbf{Immediate Action:} Take the server at \texttt{10.0.0.15} offline immediately to prevent exploitation.
        \item \textbf{Short-Term:} If the FTP service is required, upgrade \texttt{vsftpd} to the latest stable version and disable anonymous access.
        \item \textbf{Long-Term:} Evaluate the business need for FTP. If possible, replace it with a more secure file transfer protocol like SFTP (SSH File Transfer Protocol).
    \end{itemize}
\end{enumerate}

\subsection*{High Priority}
\begin{enumerate}
    \setcounter{enumi}{1}
    \item \textbf{Implement Mandatory MFA (R-02, R-03):}
    \begin{itemize}
        \item Develop and execute a phased rollout plan to enforce MFA for all computer logins and access to systems containing sensitive data.
        \item Prioritize implementation for administrative accounts and employees with access to financial, HR, or customer data.
    \end{itemize}
\end{enumerate}

\subsection*{Medium Priority}
\begin{enumerate}
    \setcounter{enumi}{2}
    \item \textbf{Establish Annual Security Training (R-04):}
    \begin{itemize}
        \item Procure or develop a security awareness training program that covers modern threats like phishing, ransomware, and social engineering.
        \item Mandate that all employees complete this training annually and track completion.
    \end{itemize}
    \item \textbf{Accelerate OS Upgrades (R-05):}
    \begin{itemize}
        \item Continue with the established plan to upgrade all Windows 7 machines to a modern, supported operating system (e.g., Windows 10/11). Windows 7 is end-of-life and no longer receives security updates.
    \end{itemize}
\end{enumerate}

% --- CONCLUSION ---
\section*{Conclusion}
The assessment of Hearth \& Home's security posture has identified significant risks that require urgent action. The combination of a critically vulnerable public-facing service and internal policy gaps, such as the lack of widespread MFA, creates a high-risk environment.

By following the prioritized recommendations outlined in this report, Hearth \& Home can significantly reduce its attack surface, strengthen its defenses, and build a more resilient security foundation for the future.

\end{document}
```