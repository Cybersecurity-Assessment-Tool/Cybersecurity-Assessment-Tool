```latex
\documentclass[12pt]{article}

% Preamble: Required Packages
\usepackage[utf8]{inputenc}
\usepackage[a4paper, margin=1in]{geometry}
\usepackage{pifont} % Required for \ding symbols
\usepackage{booktabs} % For professional-looking tables (\toprule, \midrule, \bottomrule)
\usepackage{hyperref} % For creating hyperlinks
\usepackage{url} % For formatting URLs
\usepackage{seqsplit} % To split long strings without breaking words
\usepackage{xcolor} % For colors
\usepackage{graphicx} % For images, if needed

% Hyperlink Setup
\hypersetup{
    colorlinks=true,
    linkcolor=black,
    filecolor=magenta,
    urlcolor=blue,
    pdftitle={Cybersecurity Posture Assessment Report},
    pdfpagemode=FullScreen,
}

% --- Document Start ---
\begin{document}

% --- Title Page ---
\begin{titlepage}
    \centering
    \vspace*{1cm}
    \Huge\textbf{Cybersecurity Posture Assessment Report}
    \vspace{1.5cm}
    \Large
    \textbf{Prepared For:} \\
    \vspace{0.5cm}
    Silver Leaf Collective \\
    \vspace{2cm}
    \textbf{Date of Report:} \\
    \vspace{0.5cm}
    \today
    \vfill
    \large
    \textbf{Generated By:} \\
    \vspace{0.5cm}
    Cybersecurity Analysis Division
\end{titlepage}

\newpage
\tableofcontents
\newpage

% --- Section 1: Executive Summary ---
\section{Executive Summary}
This report provides a comprehensive cybersecurity posture assessment for Silver Leaf Collective, based on a combination of network scanning, a security controls questionnaire, and a review of pre-existing risk data. The analysis synthesizes these data points to provide a holistic view of the organization's security landscape.

The assessment identified several areas of significant concern requiring immediate attention. The most critical finding is the external exposure of a MySQL database service (\texttt{172.16.50.20:3306}). This service is running an outdated and End-of-Life (EOL) version of MySQL (5.7.33), which no longer receives security updates, posing a severe risk of data breach.

Furthermore, critical gaps were identified in administrative controls. The organization lacks a formal employee Acceptable Use Policy (AUP) and does not provide security awareness training for new employees. These policy and training deficiencies significantly increase the risk of insider threats and human error, which could directly lead to the compromise of vulnerable systems like the exposed database.

This report details these findings and provides actionable recommendations to mitigate the identified risks and strengthen the overall security posture of Silver Leaf Collective.

% --- Section 2: Organizational Information ---
\section{Organizational Information}
The following details were provided for the assessment. This information helps establish the context for the technical and administrative findings.

\begin{itemize}
    \item \textbf{Organization Name:} Silver Leaf Collective
    \item \textbf{Email Domain:} \texttt{SilverLeafCollective.net}
    \item \textbf{Website Domain:} \texttt{www.SilverLeafCollective.net}
    \item \textbf{External IP Address:} \texttt{198.70.227.101}
\end{itemize}

% --- Section 3: Security Control Review ---
\section{Security Control Review}
A security questionnaire was completed to evaluate the implementation of key administrative and technical controls. While the organization demonstrates strong practices in Multi-Factor Authentication (MFA), critical gaps were identified in policy and employee training.

\subsection{Questionnaire Results}
The table below summarizes the responses. A checkmark (\ding{51}) indicates a positive control is in place, while an X (\ding{55}) indicates a control gap.

\begin{table}[h!]
\centering
\begin{tabular}{p{0.75\linewidth}c}
\toprule
\textbf{Control Question} & \textbf{Status} \\
\midrule
Do you require MFA to access email? & \textcolor{green}{\ding{51}} \\
Do you require MFA to log into computers? & \textcolor{green}{\ding{51}} \\
Do you require MFA to access sensitive data systems? & \textcolor{green}{\ding{51}} \\
Does your organization have an employee acceptable use policy? & \textcolor{red}{\ding{55}} \\
Does your organization do security awareness training for new employees? & \textcolor{red}{\ding{55}} \\
Does your organization do security awareness training for all employees at least once per year? & \textcolor{green}{\ding{51}} \\
\bottomrule
\end{tabular}
\caption{Security Controls Questionnaire Summary}
\end{table}

\subsection{Analysis of Control Gaps}
\begin{itemize}
    \item \textbf{Lack of Acceptable Use Policy (AUP):} The absence of a formal AUP is a critical governance gap. An AUP defines the rules and expectations for employee use of company technology and data. Without it, there is no clear standard for behavior, making it difficult to enforce security policies or take disciplinary action in case of a violation.
    \item \textbf{No Security Training for New Employees:} New hires are often a primary target for phishing and social engineering attacks. Failing to provide security awareness training during the onboarding process leaves the organization vulnerable from day one of an employee's tenure. This gap undermines the effectiveness of the annual training program.
\end{itemize}

% --- Section 4: Technical Scan Results ---
\section{Technical Scan Results}
A network scan was performed on the target system \texttt{172.16.50.20} to identify open ports and exposed services. The results confirm the presence of an open database port.

\begin{table}[h!]
\centering
\begin{tabular}{lllll}
\toprule
\textbf{Port} & \textbf{State} & \textbf{Service} & \textbf{Product} & \textbf{Version} \\
\midrule
3306/tcp & open & mysql & MySQL & 5.7.33 \\
\bottomrule
\end{tabular}
\caption{Open Ports Detected on \texttt{172.16.50.20}}
\end{table}

\subsection{Technical Analysis}
The scan reveals that port \texttt{3306}, the default port for MySQL databases, is open to the network. The running service was identified as \textbf{MySQL version 5.7.33}. This presents two major risks:
\begin{enumerate}
    \item \textbf{Service Exposure:} Database services should not be directly exposed to untrusted networks. An open port allows attackers to directly interact with the database, enabling brute-force attacks, credential stuffing, and exploitation of potential vulnerabilities.
    \item \textbf{Outdated and EOL Software:} MySQL version 5.7 reached its official End-of-Life (EOL) in October 2023. This means it no longer receives security patches from the vendor. The specific version detected, 5.7.33, was released in early 2021 and is missing years of security updates, making it highly susceptible to known exploits.
\end{enumerate}

% --- Section 5: Synthesized Risk Assessment ---
\section{Synthesized Risk Assessment}
By correlating the administrative control gaps, technical scan results, and pre-existing risk data, we have compiled a summary of the most significant risks facing the organization.

\begin{table}[h!]
\centering
\begin{tabular}{p{0.3\linewidth}p{0.5\linewidth}l}
\toprule
\textbf{Risk Title} & \textbf{Description} & \textbf{Severity} \\
\midrule
\textbf{Exposed \& Outdated Database Service} & A publicly accessible MySQL database is running on an End-of-Life version (5.7.33), which is no longer receiving security updates. This poses a direct threat of data breach, ransomware, or system compromise. & \textbf{Critical} \\
\addlinespace
\textbf{Lack of Foundational Security Policies} & The absence of an Acceptable Use Policy prevents the enforcement of secure behavior and creates ambiguity regarding employee responsibilities for protecting company assets. & \textbf{High} \\
\addlinespace
\textbf{Inadequate Employee Onboarding} & New employees are not provided with security awareness training, making them highly susceptible to social engineering and phishing attacks from their first day of employment. & \textbf{High} \\
\bottomrule
\end{tabular}
\caption{Summary of Identified Risks}
\end{table}

% --- Section 6: Recommendations ---
\section{Recommendations}
The following actions are recommended to mitigate the identified risks. Recommendations are prioritized based on severity and potential impact.

\subsection{Remediation for Exposed \& Outdated Database (Critical)}
\begin{enumerate}
    \item \textbf{Immediate Action:} Implement strict firewall rules to block all public access to port \texttt{3306} on host \texttt{172.16.50.20}. Access should be restricted to only trusted internal IP addresses that explicitly require it.
    \item \textbf{Short-Term Plan:} Develop a migration plan to upgrade the MySQL 5.7 database to a currently supported version, such as MySQL 8.x. This is essential to ensure the service receives ongoing security patches.
    \item \textbf{Long-Term Strategy:} For any remote database administration, implement a Virtual Private Network (VPN) with MFA to ensure all connections are encrypted and authenticated.
\end{enumerate}

\subsection{Remediation for Policy Gaps (High)}
\begin{enumerate}
    \item \textbf{Develop an AUP:} Draft and implement a comprehensive Acceptable Use Policy that covers topics such as data handling, password security, internet usage, and reporting security incidents.
    \item \textbf{Policy Acknowledgement:} Require all current and new employees to read and formally acknowledge the AUP as a condition of using company resources.
\end{enumerate}

\subsection{Remediation for Training Gaps (High)}
\begin{enumerate}
    \item \textbf{Implement Onboarding Training:} Integrate a mandatory security awareness training module into the new employee onboarding process. This should cover key topics like phishing, password hygiene, and the new AUP.
    \item \textbf{Enhance Annual Training:} Review and enhance the existing annual security training program to ensure it is engaging and covers current threat trends relevant to the organization.
\end{enumerate}

% --- Document End ---
\end{document}
```