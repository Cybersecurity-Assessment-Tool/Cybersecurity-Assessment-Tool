```latex
\documentclass[12pt]{article}

% --- PACKAGES ---
\usepackage[margin=1in]{geometry}
\usepackage{pifont} % For checkmarks and crosses
\usepackage{booktabs} % For professional tables
\usepackage{hyperref} % For clickable links
\usepackage{url} % For URL formatting
\usepackage{seqsplit} % To split long strings in texttt
\usepackage{xcolor} % For colors
\usepackage{graphicx} % For potential logos/images

% --- DOCUMENT SETUP ---
\hypersetup{
    colorlinks=true,
    linkcolor=blue,
    filecolor=magenta,      
    urlcolor=cyan,
    pdftitle={Cybersecurity Posture Report},
    pdfpagemode=FullScreen,
}

% --- CUSTOM COMMANDS ---
\newcommand{\yes}{\ding{51}}
\newcommand{\no}{\ding{55}}
\newcommand{\riskcritical}[1]{\textcolor{red}{\textbf{#1}}}
\newcommand{\riskhigh}[1]{\textcolor{orange}{\textbf{#1}}}
\newcommand{\riskmedium}[1]{\textcolor{yellow!80!black}{\textbf{#1}}}
\newcommand{\risklow}[1]{\textcolor{green}{\textbf{#1}}}

% --- DOCUMENT START ---
\begin{document}

% --- TITLE PAGE ---
\begin{titlepage}
    \centering
    \vspace*{\stretch{1.0}}
    \Huge \textbf{Cybersecurity Posture Report} \\
    \vspace{0.5cm}
    \LARGE \textbf{Grizzly Peak} \\
    \vspace{1.5cm}
    \large Report Date: \today \\
    \vspace*{\stretch{2.0}}
    \normalsize
    Prepared by: \\
    \textbf{Cybersecurity Analyst} \\
    \vspace{0.2cm}
    \textit{This report is confidential and intended for the exclusive use of the recipient.}
    \vfill
\end{titlepage}

\tableofcontents
\newpage

% --- EXECUTIVE OVERVIEW ---
\section{Executive Overview}
This report provides a comprehensive assessment of the cybersecurity posture for \textbf{Grizzly Peak}. The analysis is based on a synthesis of organizational data from a security questionnaire, results from a network port scan, and a review of pre-existing risks.

The assessment reveals a mixed security posture. On the one hand, the technical scan of the target host \texttt{192.168.1.100} indicates a strong network configuration with no open ports detected, which significantly reduces the external attack surface of that specific device.

On the other hand, critical procedural and policy gaps were identified through the security questionnaire. The two most significant findings are:
\begin{itemize}
    \item \riskcritical{Critical Risk:} The absence of mandatory Multi-Factor Authentication (MFA) for accessing sensitive data systems. This gap exposes the organization to significant risk from credential compromise, which could lead to a major data breach.
    \item \riskhigh{High Risk:} The lack of security awareness training for new employees. This oversight leaves the organization vulnerable to social engineering and phishing attacks, as new hires are often prime targets.
\end{itemize}
While no pre-existing vulnerabilities were reported, the identified policy gaps require immediate attention. This report outlines these findings in detail and provides actionable recommendations to mitigate the identified risks and strengthen the overall security posture.

% --- ORGANIZATIONAL INFORMATION ---
\section{Organizational Information}
The following details were provided for the assessment.
\begin{table}[h!]
\centering
\begin{tabular}{ll}
\toprule
\textbf{Attribute} & \textbf{Value} \\
\midrule
Organization Name & \textbf{Grizzly Peak} \\
Email Domain & \texttt{GrizzlyPeak.net} \\
Website Domain & \seqsplit{\texttt{www.GrizzlyPeak.net}} \\
External IP Address & \texttt{208.176.20.98} \\
\bottomrule
\end{tabular}
\caption{Client Organizational Data}
\label{tab:org_data}
\end{table}

% --- SECURITY CONTROL REVIEW ---
\section{Security Control Review}
The following table summarizes the organization's responses to a security controls questionnaire. The assessment column highlights areas of concern where current practices deviate from established security best practices.

\begin{table}[h!]
\centering
\begin{tabular}{p{0.6\textwidth} c l}
\toprule
\textbf{Control Question} & \textbf{Response} & \textbf{Assessment} \\
\midrule
Do you require MFA to access email? & \yes & Compliant \\
\addlinespace
Do you require MFA to log into computers? & \yes & Compliant \\
\addlinespace
Do you require MFA to access sensitive data systems? & \no & \riskcritical{Critical Gap} \\
\addlinespace
Does your organization have an employee acceptable use policy? & \yes & Compliant \\
\addlinespace
Does your organization do security awareness training for new employees? & \no & \riskhigh{High Risk} \\
\addlinespace
Does your organization do security awareness training for all employees at least once per year? & \yes & Compliant \\
\bottomrule
\end{tabular}
\caption{Security Controls Questionnaire Results}
\label{tab:controls}
\end{table}

% --- TECHNICAL SCAN RESULTS ---
\section{Technical Scan Results}
A network port scan was conducted to identify exposed services on the target system. The results indicate a well-hardened device from a network perspective.

\begin{itemize}
    \item \textbf{Target IP Address:} \texttt{192.168.1.100}
    \item \textbf{Host Status:} Up
    \item \textbf{Scan Summary:} The scan confirmed that the host was online but found \textbf{zero open ports}. All 1000 scanned TCP ports were in a "closed" state.
\end{itemize}

\subsection{Analysis}
A finding of no open ports is a significant strength. It indicates that the device is not exposing any network services to the scanner, likely due to a properly configured host-based or network firewall. This configuration drastically reduces the attack surface of the device, protecting it from network-based exploits targeting common services.

% --- OVERALL RISK ASSESSMENT ---
\section{Overall Risk Assessment}
This section synthesizes findings from all data sources into a prioritized list of risks. As no pre-existing vulnerabilities were provided, the risks below are derived from the security control review.

\begin{table}[h!]
\centering
\begin{tabular}{p{0.1\textwidth} p{0.3\textwidth} p{0.15\textwidth} p{0.35\textwidth}}
\toprule
\textbf{ID} & \textbf{Risk Name} & \textbf{Severity} & \textbf{Description} \\
\midrule
RISK-001 & Lack of MFA on Sensitive Systems & \riskcritical{Critical} & The absence of MFA on systems containing sensitive data means that a single compromised password could grant an attacker full access. This dramatically increases the likelihood and potential impact of a data breach. \\
\addlinespace
RISK-002 & No Security Training for New Hires & \riskhigh{High} & New employees are a common target for phishing and social engineering attacks. Without immediate training, they are more likely to fall victim to these attacks, potentially providing an initial foothold for an attacker into the network. \\
\bottomrule
\end{tabular}
\caption{Synthesized Risk Register}
\label{tab:risks}
\end{table}

% --- RECOMMENDATIONS ---
\section{Recommendations}
Based on the analysis, the following actions are recommended to mitigate the identified risks and improve the overall security posture of \textbf{Grizzly Peak}.

\subsection{Immediate Actions (Critical Priority)}
\begin{description}
    \item[Remediate RISK-001:] Implement mandatory Multi-Factor Authentication (MFA) across all systems, applications, and databases that store or process sensitive organizational or customer data. This should be treated as the highest priority remediation effort.
\end{description}

\subsection{Near-Term Actions (High Priority)}
\begin{description}
    \item[Remediate RISK-002:] Develop and integrate a mandatory security awareness training module into the new employee onboarding process. This training must be completed before a new hire is granted access to sensitive systems. The curriculum should cover phishing identification, password hygiene, and the acceptable use policy.
\end{description}

\subsection{Ongoing Improvements}
\begin{description}
    \item[Maintain Technical Controls:] Continue the practice of regular, automated network scanning and vulnerability assessments across all organizational assets. The strong posture of the scanned device should be used as a baseline standard for all other systems.
\end{description}

\end{document}
```