```latex
\documentclass[12pt]{article}

% Preamble: Required Packages
\usepackage[margin=1in]{geometry}
\usepackage{pifont} % For checkmarks and crosses
\usepackage{booktabs} % For professional tables
\usepackage{hyperref} % For clickable links and ToC
\usepackage{url} % For formatting URLs
\usepackage{seqsplit} % For splitting long strings in texttt
\usepackage{xcolor} % For colors
\usepackage{graphicx} % For potential logos/images
\usepackage{datetime} % For report date

% --- Customization ---
% Define colors for risk levels
\definecolor{criticalred}{HTML}{D10000}
\definecolor{highorange}{HTML}{E97400}
\definecolor{mediumyellow}{HTML}{FFBF00}
\definecolor{lowblue}{HTML}{0072B2}

% Hyperref setup
\hypersetup{
    colorlinks=true,
    linkcolor=blue,
    filecolor=magenta,      
    urlcolor=cyan,
    pdftitle={Cybersecurity Posture Report},
    pdfpagemode=FullScreen,
}

% --- Document Start ---
\begin{document}

% --- Title Page ---
\begin{titlepage}
    \centering
    \vspace*{1cm}
    \includegraphics[width=0.3\textwidth]{example-image-a} % Placeholder for company logo
    
    \vspace{1.5cm}
    
    \Huge
    \textbf{Cybersecurity Posture Report}
    
    \vspace{1.5cm}
    
    \Large
    Prepared for: \textbf{Summit Peak Partners}
    
    \vspace{2cm}
    
    \large
    Report Date: \today \\
    Analysis Period: October 2023
    
    \vfill
    
    \large
    \textbf{Confidential} \\
    This document contains sensitive information and is intended solely for the use of the designated recipient.
    
\end{titlepage}

\tableofcontents
\newpage

% --- Section 1: Executive Summary ---
\section{Executive Summary}
This report provides a comprehensive analysis of the cybersecurity posture for \textbf{Summit Peak Partners}, based on a review of organizational security controls, technical network scanning, and pre-existing risk data. The assessment identified several critical and high-risk security gaps that require immediate attention to mitigate the potential for unauthorized access and data compromise.

Key findings indicate significant weaknesses in access control, specifically the absence of Multi-Factor Authentication (MFA) for computer logins and sensitive data systems. These gaps are compounded by technical findings, such as an internally exposed Remote Desktop Protocol (RDP) service on host \texttt{10.10.10.51}. This combination creates a high-risk scenario where a single compromised credential could lead to significant network intrusion.

While the organization demonstrates a solid foundation in security policy and employee awareness training, the identified control deficiencies present a tangible threat. This report outlines these risks in detail and provides actionable recommendations to strengthen the organization's defenses and improve its overall security resilience.

% --- Section 2: Organizational Information ---
\section{Organizational Information}
The following details were provided for the assessment. This information helps contextualize the findings within the organization's environment.

\begin{tabular}{@{}ll}
\toprule
\textbf{Attribute} & \textbf{Value} \\
\midrule
Organization Name & \textbf{Summit Peak Partners} \\
Email Domain & \texttt{SummitPeakPartners.net} \\
Website Domain & \url{www.SummitPeakPartners.net} \\
External IP Address & \texttt{203.65.32.201} \\
\bottomrule
\end{tabular}

% --- Section 3: Security Control Review ---
\section{Security Control Review}
A review of organizational security controls was conducted via a standardized questionnaire. The responses are summarized below. Items marked with \ding{55} indicate a deviation from security best practices and represent a significant gap in the control environment.

\begin{tabular}{@{}p{0.6\linewidth}cp{0.25\linewidth}@{}}
\toprule
\textbf{Control Question} & \textbf{Response} & \textbf{Analyst Notes} \\
\midrule
Do you require MFA to access email? & \textcolor{green}{\ding{51}} & Strong control. Protects a primary communication vector. \\
\addlinespace
Do you require MFA to log into computers? & \textcolor{red}{\ding{55}} & \textbf{Critical Gap.} Lack of endpoint MFA allows compromised credentials to be used for network access. \\
\addlinespace
Do you require MFA to access sensitive data systems? & \textcolor{red}{\ding{55}} & \textbf{Critical Gap.} High-value data is not adequately protected from unauthorized access. \\
\addlinespace
Does your organization have an employee acceptable use policy? & \textcolor{green}{\ding{51}} & Good. Establishes a baseline for employee security responsibilities. \\
\addlinespace
Does your organization do security awareness training for new employees? & \textcolor{green}{\ding{51}} & Good. Ensures new hires are aware of security policies from the start. \\
\addlinespace
Does your organization do security awareness training for all employees at least once per year? & \textcolor{green}{\ding{51}} & Good. Reinforces security concepts and addresses evolving threats. \\
\bottomrule
\end{tabular}

% --- Section 4: Technical Scan Results ---
\section{Technical Scan Results}
A network scan was performed to identify active services on the target system. The results provide insight into the host's configuration and potential attack surface.

\subsection{Scan Details}
\begin{itemize}
    \item \textbf{Target IP Address:} \texttt{10.10.10.51}
    \item \textbf{Scan Tool:} Nmap
\end{itemize}

\subsection{Open Ports and Services}
The following open port was discovered on the target host:

\begin{tabular}{@{}llll@{}}
\toprule
\textbf{Port} & \textbf{State} & \textbf{Service Name} & \textbf{Analysis} \\
\midrule
3389/tcp & open & \texttt{ms-wbt-server} & This is the default port for Microsoft Remote Desktop Protocol (RDP). \\
\bottomrule
\end{tabular}

\subsection{Technical Analysis}
The discovery of an open RDP port on \texttt{10.10.10.51} is a significant finding. RDP is a common vector for initial access and lateral movement by attackers. When combined with the lack of MFA on computer logins, this creates a direct path for an adversary with valid (e.g., stolen or brute-forced) credentials to gain interactive control of the system. This finding confirms a pattern of RDP exposure, as a similar risk was previously identified on host \texttt{10.10.10.50}.

% --- Section 5: Correlated Risk Assessment ---
\section{Correlated Risk Assessment}
The following table synthesizes findings from the security control review, technical scan, and pre-existing risk data. Each risk is assigned a severity level based on its potential impact and likelihood.

\begin{tabular}{@{}lp{0.5\linewidth}l@{}}
\toprule
\textbf{Risk Name} & \textbf{Description} & \textbf{Severity} \\
\midrule
\textbf{Lack of MFA on Sensitive Systems} & Sensitive data systems can be accessed with only a username and password, making them highly vulnerable to credential theft and data breaches. & \textcolor{criticalred}{\textbf{Critical}} \\
\addlinespace
\textbf{Lack of MFA on Endpoints} & The absence of MFA on computer logins allows an attacker with compromised credentials to easily gain a foothold on the internal network. & \textcolor{highorange}{\textbf{High}} \\
\addlinespace
\textbf{Internal RDP Exposure} & RDP is exposed on internal systems (\texttt{10.10.10.50}, \texttt{10.10.10.51}), providing a direct vector for lateral movement. This risk is critically amplified by the lack of MFA. & \textcolor{highorange}{\textbf{High}} \\
\bottomrule
\end{tabular}

% --- Section 6: Recommendations ---
\section{Recommendations}
Based on the correlated risk assessment, the following actions are recommended to mitigate the identified vulnerabilities and improve the overall security posture of \textbf{Summit Peak Partners}.

\subsection{Immediate Priority (Remediate within 30 days)}
\begin{enumerate}
    \item \textbf{Implement MFA on Sensitive Systems (Critical Risk):}
    \begin{itemize}
        \item \textbf{Action:} Enforce MFA for all user accounts accessing systems identified as containing sensitive data.
        \item \textbf{Justification:} This is the most effective control to prevent unauthorized access to critical data assets.
    \end{itemize}
    
    \item \textbf{Implement MFA on All Endpoints (High Risk):}
    \begin{itemize}
        \item \textbf{Action:} Deploy and enforce MFA for all remote and on-premise computer logins, prioritizing administrative accounts and remote users.
        \item \textbf{Justification:} Prevents attackers from using compromised credentials to log into workstations and servers, thereby disrupting a key stage of the attack lifecycle.
    \end{itemize}
    
    \item \textbf{Restrict RDP Access (High Risk):}
    \begin{itemize}
        \item \textbf{Action:} Conduct a review of all systems with RDP enabled. If the service is not essential, disable it. If required, restrict access using host-based firewalls to allow connections only from authorized jump boxes or administrative subnets.
        \item \textbf{Justification:} Reduces the attack surface and limits the pathways for lateral movement within the network.
    \end{itemize}
\end{enumerate}

\subsection{Strategic Priority (Remediate within 90-180 days)}
\begin{enumerate}
    \item \textbf{Deploy a Privileged Access Management (PAM) Solution:}
    \begin{itemize}
        \item \textbf{Action:} Implement a PAM solution to manage, monitor, and secure all privileged access, including RDP sessions. This provides session recording, credential vaulting, and just-in-time access.
        \item \textbf{Justification:} Provides granular control and auditability over administrative activities, significantly reducing the risk of privilege abuse.
    \end{itemize}
    
    \item \textbf{Conduct Comprehensive Network Vulnerability Scanning:}
    \begin{itemize}
        \item \textbf{Action:} Establish a recurring, authenticated vulnerability scanning program for the entire internal network.
        \item \textbf{Justification:} Proactively identifies misconfigurations, outdated software, and exposed services beyond the scope of this initial assessment, enabling continuous risk reduction.
    \end{itemize}
\end{enumerate}

\end{document}
```