```latex
\documentclass[12pt]{article}

% Preamble: Required Packages
\usepackage[a4paper, margin=1in]{geometry}
\usepackage{pifont} % For checkmarks and crosses
\usepackage{booktabs} % For professional tables
\usepackage{hyperref} % For clickable links
\usepackage{url} % For URL formatting
\usepackage{seqsplit} % For splitting long strings
\usepackage{graphicx}
\usepackage{xcolor}

% Hyperref Setup
\hypersetup{
    colorlinks=true,
    linkcolor=blue,
    filecolor=magenta,      
    urlcolor=cyan,
    pdftitle={Cybersecurity Assessment Report},
    pdfpagemode=FullScreen,
}

% Define checkmark and crossmark
\newcommand{\cmark}{\ding{51}}%
\newcommand{\xmark}{\ding{55}}%

% Document Information
\title{Cybersecurity Assessment Report \\ \large For: \textbf{Borealis Tech}}
\author{Cybersecurity Analyst}
\date{November 22, 2025}

\begin{document}

\maketitle
\thispagestyle{empty}
\newpage
\tableofcontents
\newpage

% --- 1. Executive Overview ---
\section*{1. Executive Overview}

This report details the findings of a cybersecurity assessment conducted for \textbf{Borealis Tech} on November 22, 2025. The assessment combined a review of organizational security controls, an external network scan, and an analysis of pre-existing risks.

The organization demonstrates a foundational level of security by implementing Multi-Factor Authentication (MFA) for email and computer access. However, several critical and high-risk gaps were identified that require immediate attention. These include the lack of MFA for sensitive data systems, the absence of an employee acceptable use policy, and incomplete security awareness training.

Furthermore, the technical scan revealed a public-facing web server running an outdated and vulnerable version of Nginx (1.18.0). This vulnerability, combined with the identified policy and control gaps, significantly increases the organization's risk profile. Actionable recommendations are provided to mitigate these risks and enhance the overall security posture.

% --- 2. Organizational Information ---
\section*{2. Organizational Information}

The following information was provided by the client and used as the basis for this assessment.

\begin{table}[h!]
\centering
\caption{Client Profile}
\begin{tabular}{@{}ll@{}}
\toprule
\textbf{Attribute} & \textbf{Value} \\
\midrule
Organization Name & \textbf{Borealis Tech} \\
Email Domain & \texttt{BorealisTech.com} \\
Website Domain & \url{www.BorealisTech.com} \\
External IP Address & \texttt{31.95.32.82} \\
\bottomrule
\end{tabular}
\end{table}

% --- 3. Security Control Review ---
\section*{3. Security Control Review}

A questionnaire was completed to evaluate the current state of administrative and technical security controls. The results are summarized below. Answers marked with a red \xmark{} indicate significant gaps in the security program.

\begin{table}[h!]
\centering
\caption{Security Controls Questionnaire Results}
\begin{tabular}{@{}p{0.8\linewidth}c@{}}
\toprule
\textbf{Question} & \textbf{Response} \\
\midrule
Do you require MFA to access email? & \textcolor{green}{\cmark} \\
Do you require MFA to log into computers? & \textcolor{green}{\cmark} \\
Do you require MFA to access sensitive data systems? & \textcolor{red}{\xmark} \\
Does your organization have an employee acceptable use policy? & \textcolor{red}{\xmark} \\
Does your organization do security awareness training for new employees? & \textcolor{green}{\cmark} \\
Does your organization do security awareness training for all employees at least once per year? & \textcolor{red}{\xmark} \\
\bottomrule
\end{tabular}
\end{table}

\subsection*{Analysis of Control Gaps}
The review identified three primary areas of concern:
\begin{itemize}
    \item \textbf{MFA on Sensitive Systems:} The absence of MFA on systems containing sensitive data is a critical vulnerability. This allows a single point of failure (a compromised password) to potentially lead to a major data breach.
    \item \textbf{Acceptable Use Policy (AUP):} Without a formal AUP, employees may be unaware of their responsibilities regarding the secure use of company assets, leading to unintentional policy violations and security incidents.
    \item \textbf{Annual Security Training:} Security is an ongoing process. The lack of annual refresher training for all employees means that knowledge of current threats (e.g., phishing, social engineering) degrades over time, increasing susceptibility to attack.
\end{itemize}

% --- 4. Technical Scan Results ---
\section*{4. Technical Scan Results}

An Nmap scan was performed against the target IP address \texttt{192.168.10.5} to identify open ports and exposed services.

\begin{table}[h!]
\centering
\caption{Network Scan Findings for \texttt{192.168.10.5}}
\begin{tabular}{@{}lllll@{}}
\toprule
\textbf{Port} & \textbf{State} & \textbf{Service} & \textbf{Product} & \textbf{Version} \\
\midrule
443/tcp & open & https & nginx & 1.18.0 \\
\bottomrule
\end{tabular}
\end{table}

\subsection*{Analysis of Technical Findings}
\begin{itemize}
    \item \textbf{Outdated Nginx Server:} The web server is running Nginx version 1.18.0, which was released in April 2020. This version is considered end-of-life and contains multiple known vulnerabilities, including CVE-2021-23017, which could allow a request smuggling attack. Running outdated software on internet-facing systems poses a high risk of compromise.
    \item \textbf{SSL Certificate Mismatch:} The SSL certificate presented by the server has a Common Name of \texttt{www.acme-corp.com}, which does not match the organization's domain (\texttt{www.BorealisTech.com}). This is a misconfiguration that will cause browser trust errors and could be indicative of other underlying setup issues.
\end{itemize}

% --- 5. Risk Assessment Summary ---
\section*{5. Risk Assessment Summary}

The following table synthesizes findings from the security control review and the technical scan. No pre-existing risks were provided for this assessment.

\begin{table}[h!]
\centering
\caption{Identified Risks}
\begin{tabular}{@{}p{0.5\linewidth}p{0.2\linewidth}l@{}}
\toprule
\textbf{Risk Description} & \textbf{Source} & \textbf{Severity} \\
\midrule
Lack of MFA on systems housing sensitive data. & Questionnaire & \textbf{Critical} \\
Outdated and vulnerable Nginx web server exposed to the internet. & Network Scan & \textbf{High} \\
Lack of mandatory annual security awareness training for all staff. & Questionnaire & \textbf{High} \\
Absence of a formal Employee Acceptable Use Policy (AUP). & Questionnaire & \textbf{Medium} \\
SSL certificate common name does not match the server's domain. & Network Scan & \textbf{Medium} \\
\bottomrule
\end{tabular}
\end{table}

% --- 6. Recommendations ---
\section*{6. Recommendations}

The following actions are recommended to mitigate the identified risks and improve the overall security posture of \textbf{Borealis Tech}.

\begin{enumerate}
    \item \textbf{[Critical] Implement MFA for Sensitive Systems:} Immediately prioritize the deployment of MFA across all applications, databases, and administrative interfaces that store or process sensitive information. This is the single most effective control to prevent unauthorized access.
    
    \item \textbf{[High] Upgrade Nginx Server:} Plan and execute an upgrade of the Nginx server from version 1.18.0 to the latest stable version. Implement a formal patch management policy to ensure all internet-facing systems are updated in a timely manner.
    
    \item \textbf{[High] Establish Annual Security Training:} Develop and deploy a mandatory security awareness training program for all employees to be completed annually. The training should cover modern threats such as phishing, ransomware, and social engineering.
    
    \item \textbf{[Medium] Develop an Acceptable Use Policy:} Create a comprehensive AUP that clearly defines the rules and responsibilities for all users of company IT assets. This policy should be reviewed and signed by all employees.
    
    \item \textbf{[Medium] Correct SSL Certificate Configuration:} Replace the current SSL certificate on the web server with a valid certificate that correctly corresponds to the \texttt{www.BorealisTech.com} domain. This will resolve browser trust errors and ensure encrypted traffic is properly secured.
\end{enumerate}

\end{document}
```