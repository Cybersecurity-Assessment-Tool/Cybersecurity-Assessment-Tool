```latex
\documentclass[12pt]{article}

% Preamble: Required Packages
\usepackage[margin=1in]{geometry}
\usepackage{pifont} % For checkmarks and crosses
\usepackage{booktabs} % For professional tables
\usepackage{hyperref} % For clickable links
\usepackage{url} % For URL formatting
\usepackage{seqsplit} % For splitting long text strings
\usepackage{graphicx}
\usepackage{xcolor}

% Hyperref Setup
\hypersetup{
    colorlinks=true,
    linkcolor=blue,
    filecolor=magenta,      
    urlcolor=cyan,
    pdftitle={Cybersecurity Posture Report},
    pdfpagemode=FullScreen,
}

% Define checkmark and cross symbols for clarity
\newcommand{\cmark}{\ding{51}}%
\newcommand{\xmark}{\ding{55}}%

% --- Document Start ---
\begin{document}

% --- Title Page ---
\begin{titlepage}
    \centering
    \vspace*{1cm}
    \Huge{\textbf{Cybersecurity Posture Report}}
    \vspace{1.5cm}
    \Large{\textbf{Prepared for: Neon Pulse Entertainment}}
    \vspace{2cm}
    \includegraphics[width=0.4\textwidth]{example-image-a} % Placeholder for a logo
    \vfill
    \large{
        \textbf{Date of Report:} \today \\
        \textbf{Author:} Cybersecurity Analyst
    }
\end{titlepage}

% --- Table of Contents ---
\tableofcontents
\newpage

% --- Section 1: Executive Summary ---
\section{Executive Summary}
This report provides a comprehensive analysis of the cybersecurity posture for Neon Pulse Entertainment. The assessment is based on a review of organizational security controls, an external network scan, and an evaluation of known risks.

The organization has implemented several positive security controls, including mandatory Multi-Factor Authentication (MFA) for computer and sensitive system access, and security awareness training for new hires.

However, two significant gaps were identified that present a high degree of risk to the organization:
\begin{itemize}
    \item \textbf{Critical Risk:} The absence of MFA on employee email accounts. This is a primary vector for Business Email Compromise (BEC), phishing, and account takeover attacks.
    \item \textbf{High Risk:} The lack of mandatory, annual security awareness training for all staff. This can lead to a degradation of security consciousness over time, making the organization more susceptible to social engineering attacks.
\end{itemize}

The external network scan against the target IP address was inconclusive and returned no open ports. This may indicate a strong firewall configuration, but it could also be the result of a scan error or an incorrect target.

Recommendations in this report focus on immediate remediation of the critical email security gap, followed by the implementation of a recurring security training program.

% --- Section 2: Organizational Information ---
\section{Organizational Information}
The following details were provided for the assessment. This information forms the basis for understanding the organization's digital footprint.

\begin{tabular}{@{}ll}
    \toprule
    \textbf{Attribute} & \textbf{Value} \\
    \midrule
    Organization Name & Neon Pulse Entertainment \\
    Email Domain & \texttt{NeonPulseEntertainment.com} \\
    Website Domain & \url{www.NeonPulseEntertainment.com} \\
    External IP Address & \texttt{28.189.169.137} \\
    \bottomrule
\end{tabular}

% --- Section 3: Security Control Review ---
\section{Security Control Review}
A review of foundational security controls was conducted via a questionnaire. The results below highlight implemented controls and identify significant gaps in the organization's defensive posture.

\begin{table}[h!]
\centering
\caption{Organizational Security Controls Questionnaire}
\begin{tabular}{@{}p{0.6\linewidth} c p{0.25\linewidth}@{}}
    \toprule
    \textbf{Control Question} & \textbf{Status} & \textbf{Analyst Note} \\
    \midrule
    Do you require MFA to access email? & \textcolor{red}{\xmark} & \textbf{Critical Gap.} Email is a primary target. Lack of MFA significantly increases risk. \\
    \addlinespace
    Do you require MFA to log into computers? & \textcolor{green}{\cmark} & Good Practice. Protects against unauthorized physical or remote access. \\
    \addlinespace
    Do you require MFA to access sensitive data systems? & \textcolor{green}{\cmark} & Good Practice. Essential for protecting critical business and customer data. \\
    \addlinespace
    Does your organization have an employee acceptable use policy? & \textcolor{green}{\cmark} & Good Practice. Sets clear expectations for employee behavior. \\
    \addlinespace
    Does your organization do security awareness training for new employees? & \textcolor{green}{\cmark} & Good Practice. Establishes a security baseline for new staff. \\
    \addlinespace
    Does your organization do security awareness training for all employees at least once per year? & \textcolor{red}{\xmark} & \textbf{High Risk.} Security knowledge degrades over time. Annual training is crucial. \\
    \bottomrule
\end{tabular}
\end{table}

% --- Section 4: Technical Scan Results ---
\section{Technical Scan Results}
An external network vulnerability scan was performed to identify open ports and exposed services.

\begin{itemize}
    \item \textbf{Target IP Address:} \texttt{[Target IP]}
    \item \textbf{Scan Date:} Not available in scan data.
    \item \textbf{Findings:} The scan completed but did not identify any open TCP or UDP ports on the target host.
\end{itemize}

\subsection{Analysis}
An absence of open ports can indicate several possibilities:
\begin{enumerate}
    \item \textbf{Effective Firewall Configuration:} The perimeter firewall is correctly configured to block all unsolicited inbound traffic, which is a strong security posture.
    \item \textbf{Incorrect Target or Offline Host:} The scan may have been directed at an incorrect IP address, or the target host may have been offline at the time of the scan.
    \item \textbf{Scan Blocking:} An Intrusion Prevention System (IPS) or other security appliance may have detected and blocked the scan traffic.
\end{enumerate}
While this result may be positive, it is inconclusive without further verification. A follow-up action to confirm the scan's accuracy is recommended.

% --- Section 5: Risk Assessment ---
\section{Risk Assessment}
This section synthesizes findings from the security control review and technical scan. No pre-existing vulnerabilities were provided for this assessment. The primary risks identified are related to organizational policy and procedure.

\begin{table}[h!]
\centering
\caption{Summary of Identified Risks}
\begin{tabular}{@{}p{0.25\linewidth} p{0.5\linewidth} l@{}}
    \toprule
    \textbf{Risk Name} & \textbf{Overview} & \textbf{Severity} \\
    \midrule
    \addlinespace
    Lack of MFA on Email & The absence of MFA on email accounts exposes the organization to a high likelihood of account compromise via phishing or credential stuffing. A compromised email account can lead to data breaches, financial fraud (BEC), and further infiltration of the network. & \textbf{Critical} \\
    \addlinespace
    Insufficient Security Awareness Training & Without mandatory, annual security training, employees are less likely to recognize and appropriately respond to evolving threats like sophisticated phishing and social engineering attacks. This creates a weak "human firewall." & \textbf{High} \\
    \addlinespace
    Inconclusive External Scan & The network scan results could not confirm the external security posture. This leaves a blind spot regarding potentially exposed services or misconfigurations on the network perimeter. & \textbf{Low} \\
    \bottomrule
\end{tabular}
\end{table}

% --- Section 6: Recommendations ---
\section{Recommendations}
The following actions are recommended to mitigate the identified risks and improve the overall security posture of Neon Pulse Entertainment. Recommendations are prioritized by severity.

\subsection{Immediate Priority (Critical)}
\begin{itemize}
    \item \textbf{Enforce MFA on All Email Accounts:}
    \begin{itemize}
        \item \textbf{Action:} Immediately enable and enforce MFA for all user mailboxes.
        \item \textbf{Justification:} This is the single most effective control to prevent unauthorized access to email accounts and mitigate the risk of Business Email Compromise. It should be treated as the highest priority remediation task.
    \end{itemize}
\end{itemize}

\subsection{High Priority}
\begin{itemize}
    \item \textbf{Implement Annual Security Awareness Training:}
    \begin{itemize}
        \item \textbf{Action:} Procure and roll out a mandatory security awareness training program for all employees, to be completed annually. This program should include modules on phishing, password security, and acceptable use.
        \item \textbf{Justification:} A well-trained workforce is a critical layer of defense. Regular training ensures that security remains a top-of-mind concern and equips employees to defend against modern threats.
    \end{itemize}
\end{itemize}

\subsection{Informational / Follow-Up}
\begin{itemize}
    \item \textbf{Verify External Network Posture:}
    \begin{itemize}
        \item \textbf{Action:} Internally verify that the target IP address (\texttt{[Target IP]}) was correct and that the host was online. Coordinate with the network team to schedule an authenticated or white-listed re-scan to validate the external firewall configuration.
        \item \textbf{Justification:} This action will provide conclusive evidence of the network perimeter's security and ensure no services are unintentionally exposed.
    \end{itemize}
\end{itemize}


\end{document}
```