```latex
\documentclass[12pt]{article}

% 1. DOCUMENT SETUP & PACKAGES
\usepackage[margin=1in]{geometry}
\usepackage{pifont} % For checkmarks and crosses
\usepackage{booktabs} % For professional tables
\usepackage{hyperref} % For hyperlinks
\usepackage{url} % For URL formatting
\usepackage{seqsplit} % For splitting long strings in tt font
\usepackage[utf8]{inputenc}

% 2. CUSTOM COMMANDS & HYPERREF SETUP
\newcommand{\check}{\ding{51}}
\newcommand{\cross}{\ding{55}}

\hypersetup{
    colorlinks=true,
    linkcolor=blue,
    filecolor=magenta,      
    urlcolor=cyan,
    pdftitle={Cybersecurity Assessment Report},
    pdfauthor={Cybersecurity Analyst},
    pdfsubject={Security Assessment},
    pdfkeywords={Security, Report, Analysis},
    bookmarks=true
}

% 3. DOCUMENT START
\begin{document}

% --- TITLE PAGE ---
\begin{titlepage}
    \centering
    \vspace*{1cm}
    \Huge
    \textbf{Cybersecurity Assessment Report}
    
    \vspace{1.5cm}
    \Large
    Prepared for: \\
    \vspace{0.5cm}
    \textbf{True North Travel}
    
    \vspace{2cm}
    \large
    Date of Report: \today \\
    Scan Date: 2025-11-22
    
    \vfill
    
    \large
    \textit{This report contains sensitive information and should be handled with care. Access is restricted to authorized personnel only.}
    
\end{titlepage}

\tableofcontents
\newpage

% --- SECTION 1: EXECUTIVE OVERVIEW ---
\section{Executive Overview}
This report provides a cybersecurity assessment for \textbf{True North Travel}, based on a technical network scan, a review of existing security controls, and an analysis of pre-identified risks. The assessment was conducted to identify vulnerabilities and provide actionable recommendations to improve the organization's overall security posture.

The analysis revealed several areas of significant concern that require immediate attention. Key findings include:
\begin{itemize}
    \item \textbf{Critical Gaps in Access Control:} Multi-Factor Authentication (MFA) is not enforced for email access or computer logins. This represents a critical vulnerability, as compromised credentials could lead to unauthorized access to sensitive company and client data.
    \item \textbf{Outdated Web Server Software:} The external-facing web server is running an outdated version of Nginx (1.18.0). Outdated software is a primary target for attackers, as it often contains publicly known and unpatched vulnerabilities.
    \item \textbf{Deficiencies in Security Governance:} The organization lacks a formal employee acceptable use policy and does not provide mandatory annual security awareness training for all staff. These gaps increase the risk of security incidents caused by human error.
\end{itemize}
The overall security posture is considered weak due to these foundational gaps. The recommendations outlined in this report are prioritized to address the most critical risks first.

% --- SECTION 2: ORGANIZATIONAL INFORMATION ---
\section{Organizational Information}
The following details were provided for the assessment.
\begin{itemize}
    \item \textbf{Organization Name:} True North Travel
    \item \textbf{Email Domain:} \texttt{TrueNorthTravel.net}
    \item \textbf{Website Domain:} \url{www.TrueNorthTravel.net}
    \item \textbf{Primary External IP:} \texttt{223.136.133.44}
\end{itemize}

% --- SECTION 3: SECURITY CONTROL REVIEW ---
\section{Security Control Review}
A review of organizational security controls was conducted via a questionnaire. The responses highlight significant gaps in identity and access management and security governance.
\begin{table}[h!]
\centering
\caption{Security Controls Questionnaire Results}
\begin{tabular}{p{8cm} c l}
\toprule
\textbf{Control Question} & \textbf{Response} & \textbf{Assessment} \\
\midrule
Do you require MFA to access email? & \cross & Critical Gap \\
Do you require MFA to log into computers? & \cross & High Risk \\
Do you require MFA to access sensitive data systems? & \check & Good Practice \\
Does your organization have an employee acceptable use policy? & \cross & High Risk \\
Does your organization do security awareness training for new employees? & \check & Good Practice \\
Does your organization do security awareness training for all employees at least once per year? & \cross & High Risk \\
\bottomrule
\end{tabular}
\end{table}

% --- SECTION 4: TECHNICAL SCAN RESULTS ---
\section{Technical Scan Results}
A network scan was performed to identify open ports and services on the target system.
\begin{itemize}
    \item \textbf{Target IP Address:} \texttt{192.168.10.5}
    \item \textbf{Scan Date:} 2025-11-22T10:00:00Z
\end{itemize}

\subsection{Open Ports and Services}
The following table details the open ports and services discovered on the target system.
\begin{table}[h!]
\centering
\caption{Discovered Open Ports}
\begin{tabular}{c c l l l}
\toprule
\textbf{Port} & \textbf{State} & \textbf{Service} & \textbf{Product} & \textbf{Version} \\
\midrule
443/tcp & open & https & nginx & 1.18.0 \\
\bottomrule
\end{tabular}
\end{table}

\subsection{Technical Findings Analysis}
The scan identified a single open port, 443 (HTTPS), served by an Nginx web server. The detected version, \textbf{1.18.0}, was released in April 2020. As of late 2025, this version is severely outdated and is no longer receiving security updates. Running outdated software, especially on an internet-facing server, exposes the organization to a wide range of known vulnerabilities that could be exploited to compromise the server and gain access to the internal network.

% --- SECTION 5: RISK ASSESSMENT SUMMARY ---
\section{Risk Assessment Summary}
The following table synthesizes findings from the security control review and the technical scan into a list of identified risks. No pre-existing risks were provided for this assessment.
\begin{table}[h!]
\centering
\caption{Identified Risks}
\begin{tabular}{p{2cm} p{6.5cm} p{3.5cm} l}
\toprule
\textbf{Risk ID} & \textbf{Risk Description} & \textbf{Affected Asset(s)} & \textbf{Severity} \\
\midrule
TNT-RISK-001 & Lack of MFA on email accounts significantly increases the risk of Business Email Compromise (BEC) and unauthorized data access. & All user email accounts & Critical \\
\addlinespace
TNT-RISK-002 & The public-facing web server is running an outdated and unsupported version of Nginx, making it susceptible to known exploits. & Web Server (\texttt{192.168.10.5}) & High \\
\addlinespace
TNT-RISK-003 & The absence of MFA for computer logins weakens endpoint security and could facilitate unauthorized lateral movement within the network. & All employee computers & High \\
\addlinespace
TNT-RISK-004 & The lack of an Acceptable Use Policy and mandatory annual security training leads to inconsistent user behavior and a higher likelihood of security incidents. & Entire Organization (People \& Processes) & High \\
\bottomrule
\end{tabular}
\end{table}

% --- SECTION 6: RECOMMENDATIONS ---
\section{Recommendations}
Based on the findings, the following actions are recommended to mitigate the identified risks and strengthen the security posture of \textbf{True North Travel}.

\begin{enumerate}
    \item \textbf{Implement MFA for Email (Critical):} Immediately enforce MFA for all user accounts accessing the email system. This is the single most effective control to prevent account takeovers and Business Email Compromise.
    
    \item \textbf{Upgrade Web Server Software (High):} Plan and execute an upgrade of the Nginx server at \texttt{192.168.10.5} to the latest stable version. Establish a patch management process to ensure the server software is kept up-to-date going forward.
    
    \item \textbf{Enforce MFA for Computer Logins (High):} Deploy MFA for all employee logins to company computers and laptops. This adds a critical layer of defense against the use of stolen credentials.
    
    \item \textbf{Develop and Implement Security Policies (High):}
    \begin{itemize}
        \item Create a formal Acceptable Use Policy (AUP) that all employees must read and acknowledge.
        \item Establish a mandatory annual security awareness training program for all employees to reinforce best practices and educate them on current threats.
    \end{itemize}
\end{enumerate}

% 4. DOCUMENT END
\end{document}
```