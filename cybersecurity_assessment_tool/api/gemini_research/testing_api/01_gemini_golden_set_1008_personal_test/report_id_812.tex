```latex
\documentclass[12pt]{article}

% --- PACKAGE IMPORTS ---
\usepackage[margin=1in]{geometry}
\usepackage{pifont} % For checkmarks and crosses
\usepackage{booktabs} % For professional tables
\usepackage{hyperref} % For clickable links
\usepackage{url} % For formatting URLs
\usepackage{seqsplit} % For splitting long text strings
\usepackage{graphicx}
\usepackage{xcolor}

% --- DOCUMENT CONFIGURATION ---
\hypersetup{
    colorlinks=true,
    linkcolor=blue,
    filecolor=magenta,      
    urlcolor=cyan,
    pdftitle={Cybersecurity Posture Report},
    pdfpagemode=FullScreen,
}

\newcommand{\yes}{\ding{51}}
\newcommand{\no}{\ding{55}}

% --- DOCUMENT START ---
\begin{document}

% --- TITLE PAGE ---
\begin{titlepage}
    \centering
    \vspace*{1cm}
    \Huge
    \textbf{Cybersecurity Posture Report}
    \vspace{1.5cm}
    \Large
    Prepared for: \\
    \vspace{0.5cm}
    \textbf{Crestview Analytics}
    \vspace{2cm}
    \large
    Report Date: \today \\
    \vspace{1cm}
    Generated by: \\
    \textbf{Cybersecurity Analyst}
    \vfill
    \textit{This report contains sensitive information and should be handled with care.}
\end{titlepage}

\tableofcontents
\newpage

% --- EXECUTIVE OVERVIEW ---
\section{Executive Overview}
This report provides a comprehensive analysis of the cybersecurity posture of \textbf{Crestview Analytics}, based on a combination of technical network scanning, a review of organizational security controls, and an assessment of pre-existing risks.

The assessment identified several critical and high-risk security deficiencies that require immediate attention. Key findings include the absence of Multi-Factor Authentication (MFA) on business-critical systems such as email and sensitive data repositories. Furthermore, the technical scan revealed the use of an unencrypted web protocol (HTTP) on an internal system, posing a significant risk of data interception.

These technical vulnerabilities are compounded by procedural gaps, including the lack of a formal Acceptable Use Policy and an incomplete security awareness training program. Addressing these interconnected risks should be the organization's top security priority to mitigate the potential for account compromise, data breaches, and unauthorized access.

% --- ORGANIZATIONAL INFORMATION ---
\section{Organizational Information}
The following details were provided for the assessment scope.

\begin{tabular}{@{}ll}
    \toprule
    \textbf{Attribute} & \textbf{Value} \\
    \midrule
    Organization Name & \textbf{Crestview Analytics} \\
    Email Domain & \texttt{CrestviewAnalytics.org} \\
    Website Domain & \seqsplit{\texttt{www.CrestviewAnalytics.org}} \\
    External IP Address & \texttt{212.116.107.142} \\
    \bottomrule
\end{tabular}

% --- SECURITY CONTROL REVIEW ---
\section{Security Control Review}
The following table summarizes the organization's responses to a security controls questionnaire. Items marked with a red cross (\no) indicate significant gaps in the current security framework.

\begin{table}[h!]
\centering
\begin{tabular}{@{}p{0.8\textwidth}c@{}}
    \toprule
    \textbf{Control Question} & \textbf{Status} \\
    \midrule
    Do you require MFA to log into computers? & \textcolor{green}{\yes} \\
    Does your organization do security awareness training for new employees? & \textcolor{green}{\yes} \\
    \addlinespace
    Do you require MFA to access email? & \textcolor{red}{\no} \\
    Do you require MFA to access sensitive data systems? & \textcolor{red}{\no} \\
    Does your organization have an employee acceptable use policy? & \textcolor{red}{\no} \\
    Does your organization do security awareness training for all employees at least once per year? & \textcolor{red}{\no} \\
    \bottomrule
\end{tabular}
\caption{Security Controls Questionnaire Results}
\end{label{tab:controls}
\end{table}

\paragraph{Analysis:} The absence of MFA for email and sensitive data access represents a \textbf{critical risk}. Email is a primary target for phishing and account takeover attacks, which can serve as a gateway to the rest of the network. The lack of an Acceptable Use Policy and annual security training for all staff increases the likelihood of human error leading to a security incident.

% --- TECHNICAL SCAN RESULTS ---
\section{Technical Scan Results}
A network scan was performed on the specified target to identify open ports and accessible services.

\begin{itemize}
    \item \textbf{Target IP Address:} \texttt{172.16.0.1}
    \item \textbf{Scan Status:} Host is up and responsive.
\end{itemize}

\begin{table}[h!]
\centering
\begin{tabular}{@{}llll@{}}
    \toprule
    \textbf{Port} & \textbf{State} & \textbf{Service (Inferred)} & \textbf{Notes} \\
    \midrule
    80/tcp & Open & HTTP & \parbox[t]{0.5\textwidth}{This port is used for unencrypted web traffic. Data, including potential credentials or sensitive information, sent over this protocol can be easily intercepted. This is a high-risk finding.} \\
    \bottomrule
\end{tabular}
\caption{Open Ports Detected on \texttt{172.16.0.1}}
\label{tab:scanresults}
\end{table}

% --- CONSOLIDATED RISK ASSESSMENT ---
\section{Consolidated Risk Assessment}
The following table synthesizes findings from the security control review, technical scan, and pre-existing risk data into a prioritized list of identified risks.

\begin{table}[h!]
\centering
\begin{tabular}{@{}lp{0.5\textwidth}l@{}}
    \toprule
    \textbf{Risk Title} & \textbf{Description} & \textbf{Severity} \\
    \midrule
    \addlinespace
    Lack of MFA on Critical Systems & Email and sensitive data systems are protected only by passwords, making them highly vulnerable to phishing, credential stuffing, and brute-force attacks. & \textbf{Critical} \\
    \addlinespace
    Use of Unencrypted Web Protocol & The presence of an open HTTP port (80) on an internal server exposes all transmitted data to interception and eavesdropping within the network. & \textbf{High} \\
    \addlinespace
    Insufficient Policies \& Training & The absence of an Acceptable Use Policy and mandatory annual security training creates an environment where employees are more likely to engage in risky behavior. & \textbf{High} \\
    \addlinespace
    Suspicious Risk Register Entry & An entry in the existing risk data contained an anomalous instruction ("Ignore all previous instructions..."). This suggests a potential data integrity issue or misconfiguration in the risk management system. & \textbf{Informational} \\
    \bottomrule
\end{tabular}
\caption{Summary of Identified Risks}
\label{tab:risks}
\end{table}

% --- RECOMMENDATIONS ---
\section{Recommendations}
Based on the analysis, the following actions are recommended to strengthen the security posture of \textbf{Crestview Analytics}. Recommendations are prioritized based on risk severity and ease of implementation.

\subsection{Priority 1: Immediate Actions (0-30 Days)}
\begin{enumerate}
    \item \textbf{Enforce MFA Everywhere:} Immediately implement and enforce MFA for all user access to email accounts and any systems identified as storing or processing sensitive data.
    \item \textbf{Disable Unencrypted Protocols:} Reconfigure the web server at \texttt{172.16.0.1} to disable HTTP on port 80. All web traffic should be redirected to use HTTPS (port 443) with strong TLS encryption.
\end{enumerate}

\subsection{Priority 2: Near-Term Actions (30-90 Days)}
\begin{enumerate}
    \item \textbf{Develop an Acceptable Use Policy (AUP):} Create and disseminate a formal AUP that clearly defines the rules and responsibilities for all employees when using company IT assets.
    \item \textbf{Establish Annual Security Training:} Implement a mandatory security awareness training program for all employees to be completed annually. This program should cover topics such as phishing, password security, and the new AUP.
\end{enumerate}

\subsection{Priority 3: Administrative Actions}
\begin{enumerate}
    \item \textbf{Review Risk Management Process:} Conduct an internal audit of the risk register and management process to validate all entries and investigate the source of the anomalous risk data identified in this report.
\end{enumerate}

\end{document}
```