```latex
\documentclass[12pt]{article}

% Preamble: Required Packages
\usepackage[a4paper, margin=1in]{geometry}
\usepackage{pifont} % For checkmarks and crosses
\usepackage{booktabs} % For professional tables
\usepackage{hyperref} % For clickable links
\usepackage{url}      % For URL formatting
\usepackage{seqsplit} % For splitting long strings without breaking
\usepackage{graphicx} % For potential logos
\usepackage{xcolor}   % For colors

% Document Information
\title{Cybersecurity Posture Assessment Report}
\author{Cybersecurity Analysis Division}
\date{\today}

% Hyperref Setup
\hypersetup{
    colorlinks=true,
    linkcolor=blue,
    filecolor=magenta,      
    urlcolor=cyan,
    pdftitle={Cybersecurity Posture Assessment Report},
    pdfpagemode=FullScreen,
}

\begin{document}

\maketitle
\thispagestyle{empty}
\newpage

\tableofcontents
\newpage

% --- 1. Executive Summary ---
\section{Executive Summary}

This report provides a comprehensive cybersecurity posture assessment for \textbf{Radiant Life}. The analysis is based on a synthesis of network scan data, organizational security control questionnaires, and a review of pre-existing risks.

The assessment reveals several critical and high-risk security gaps that require immediate attention. Key findings include the public exposure of Remote Desktop Protocol (RDP) services on internal systems, a critical lack of Multi-Factor Authentication (MFA) for email and sensitive data systems, and the absence of a formal Acceptable Use Policy for employees.

These vulnerabilities, when correlated, present a significant risk of unauthorized access, data breach, and potential ransomware attacks. The combination of exposed RDP and weak authentication controls on critical assets creates a direct pathway for threat actors. This report outlines these risks in detail and provides actionable recommendations prioritized by severity to mitigate them effectively.

% --- 2. Organizational Information ---
\section{Organizational Information}

The following details were provided for the assessment. This information is used to establish the context and scope of the review.

\begin{tabular}{@{}ll}
\toprule
\textbf{Attribute} & \textbf{Value} \\
\midrule
Organization Name & \textbf{Radiant Life} \\
Email Domain & \texttt{RadiantLife.net} \\
Website Domain & \url{www.RadiantLife.net} \\
External IP Address & \texttt{160.243.1.7} \\
\bottomrule
\end{tabular}

% --- 3. Security Control Review ---
\section{Security Control Review}

A review of the organization's security controls was conducted via a questionnaire. The responses highlight significant gaps in foundational security practices, particularly concerning identity and access management. A "No" response indicates a deviation from security best practices and is flagged as a risk.

\begin{tabular}{@{}p{0.7\linewidth}c@{}}
\toprule
\textbf{Control Question} & \textbf{Status} \\
\midrule
Do you require MFA to access email? & \textcolor{red}{\ding{55}} \\
Do you require MFA to log into computers? & \textcolor{green}{\ding{51}} \\
Do you require MFA to access sensitive data systems? & \textcolor{red}{\ding{55}} \\
Does your organization have an employee acceptable use policy? & \textcolor{red}{\ding{55}} \\
Does your organization do security awareness training for new employees? & \textcolor{green}{\ding{51}} \\
Does your organization do security awareness training for all employees at least once per year? & \textcolor{green}{\ding{51}} \\
\bottomrule
\end{tabular}

\subsection*{Analysis of Gaps}
\begin{itemize}
    \item \textbf{MFA for Email (Critical Gap):} The absence of MFA on email makes the organization highly susceptible to phishing attacks and business email compromise (BEC). An attacker who obtains a user's password can gain full access to their mailbox.
    \item \textbf{MFA for Sensitive Data (Critical Gap):} Sensitive systems without MFA are at high risk of unauthorized access. Stolen credentials could be used to directly access and exfiltrate confidential information.
    \item \textbf{Acceptable Use Policy (High Risk):} The lack of a formal policy creates ambiguity regarding security responsibilities for employees and contractors, increasing the likelihood of unintentional security incidents.
\end{itemize}

% --- 4. Technical Scan Results ---
\section{Technical Scan Results}

An Nmap scan was performed on the target system to identify open ports and exposed services. The results indicate a critical service is accessible, which aligns with pre-existing risk data and points to a systemic issue.

\begin{itemize}
    \item \textbf{Target IP Address:} \texttt{10.10.10.51}
    \item \textbf{Scan Status:} Host is up.
\end{itemize}

\begin{tabular}{@{}llll@{}}
\toprule
\textbf{Port} & \textbf{State} & \textbf{Service} & \textbf{Analysis} \\
\midrule
3389/tcp & Open & ms-wbt-server & \parbox[t]{0.5\linewidth}{This port is used for Microsoft Remote Desktop Protocol (RDP). Exposing RDP directly to a network is a severe security risk and a primary vector for ransomware attacks.} \\
\bottomrule
\end{tabular}

% --- 5. Correlated Risk Assessment ---
\section{Correlated Risk Assessment}

This section synthesizes findings from the security control review, the technical scan, and pre-existing risk data into a unified risk register. The correlation of these items reveals a significantly elevated risk posture.

\begin{tabular}{@{}p{0.3\linewidth}p{0.5\linewidth}l@{}}
\toprule
\textbf{Risk Title} & \textbf{Description} & \textbf{Severity} \\
\midrule
\textbf{Systemic RDP Exposure} & RDP is exposed on multiple internal systems (\texttt{10.10.10.50} from prior risk data and \texttt{10.10.10.51} from the new scan). This indicates a lack of network segmentation and insecure remote access practices. & \textbf{Critical} \\
\addlinespace
\textbf{Lack of MFA on Email} & The primary communication channel is vulnerable to account takeover via phishing or credential stuffing, as it is not protected by MFA. & \textbf{Critical} \\
\addlinespace
\textbf{Lack of MFA on Sensitive Systems} & Critical data systems are accessible with only a username and password, posing a direct risk of a data breach if credentials are compromised. & \textbf{Critical} \\
\addlinespace
\textbf{No Acceptable Use Policy} & The absence of a guiding policy for employees on the proper use of company assets makes it difficult to enforce security standards and increases insider threat risk. & \textbf{High} \\
\bottomrule
\end{tabular}

% --- 6. Recommendations ---
\section{Recommendations}

The following prioritized recommendations are provided to address the identified risks and improve the overall security posture of \textbf{Radiant Life}.

\subsection*{Priority 1: Immediate Actions (0-7 Days)}
\begin{enumerate}
    \item \textbf{Isolate Exposed RDP Services:} Immediately implement firewall rules to block access to TCP port 3389 on all identified systems (\texttt{10.10.10.50}, \texttt{10.10.10.51}, and any others) from untrusted networks. Access should only be permitted through a secure gateway.
    \item \textbf{Enforce MFA on Email:} Enable and enforce MFA for all user accounts on the \texttt{RadiantLife.net} email domain. This is the single most effective control to prevent business email compromise.
    \item \textbf{Enforce MFA on Sensitive Systems:} Immediately deploy and require MFA for all systems identified as containing sensitive data.
\end{enumerate}

\subsection*{Priority 2: Short-Term Actions (1-3 Months)}
\begin{enumerate}
    \item \textbf{Develop and Implement an Acceptable Use Policy (AUP):} Create a formal AUP that clearly defines the rules for using company IT assets, including remote access, data handling, and internet usage. Ensure all employees read and acknowledge the policy.
    \item \textbf{Deploy a Secure Remote Access Solution:} Replace direct RDP access with a modern, secure solution such as a Virtual Private Network (VPN) with MFA or a Zero Trust Network Access (ZTNA) gateway. This provides a single, controlled, and audited entry point for remote administration.
\end{enumerate}

\subsection*{Priority 3: Long-Term Actions (3-6 Months)}
\begin{enumerate}
    \item \textbf{Conduct Network-Wide Vulnerability Scanning:} Implement a regular vulnerability scanning program to proactively identify and remediate exposed services, misconfigurations, and outdated software across the entire network.
    \item \textbf{Review and Enhance Network Segmentation:} Evaluate the internal network architecture to ensure that critical systems are properly isolated from user workstations and other less-sensitive assets, limiting the lateral movement of potential attackers.
\end{enumerate}

\end{document}
```