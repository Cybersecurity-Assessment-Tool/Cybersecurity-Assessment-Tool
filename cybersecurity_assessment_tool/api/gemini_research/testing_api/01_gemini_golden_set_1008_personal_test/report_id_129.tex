```latex
\documentclass[12pt]{article}

% Preamble: Required Packages
\usepackage[margin=1in]{geometry}
\usepackage{pifont} % For checkmarks and crosses
\usepackage{booktabs} % For professional tables
\usepackage{hyperref} % For clickable links
\usepackage{url} % For URL formatting
\usepackage{seqsplit} % To split long strings without breaking
\usepackage{graphicx}
\usepackage{fancyhdr}
\usepackage{xcolor}
\usepackage{datetime}

% --- Document Setup ---
\hypersetup{
    colorlinks=true,
    linkcolor=blue,
    filecolor=magenta,      
    urlcolor=cyan,
    pdftitle={Cybersecurity Posture Assessment Report},
    pdfpagemode=FullScreen,
}

\pagestyle{fancy}
\fancyhf{}
\fancyhead[L]{\textbf{Cybersecurity Posture Assessment}}
\fancyhead[R]{\textbf{Signal Flare}}
\fancyfoot[C]{\thepage}

% --- Document Start ---
\begin{document}

% --- Title Page ---
\begin{titlepage}
    \centering
    \vspace*{1cm}
    \Huge\textbf{Cybersecurity Posture Assessment Report}
    \vspace{1.5cm}
    \Large
    \textbf{Prepared for:}\\
    \vspace{0.5cm}
    Signal Flare
    \vfill
    \large
    \textbf{Date of Report:}\\
    \today
    \vspace{1.5cm}
    \textbf{Analysis Conducted By:}\\
    Cybersecurity Analysis Division
\end{titlepage}

\tableofcontents
\newpage

% --- Section 1: Executive Summary ---
\section{Executive Summary}
This report provides a comprehensive analysis of the cybersecurity posture for Signal Flare, based on network scans, organizational questionnaires, and a review of existing risk documentation. The assessment was conducted to identify vulnerabilities, security control gaps, and areas of non-compliance with cybersecurity best practices.

The overall security posture is rated as \textbf{Weak}. Several critical-risk findings require immediate attention. The analysis revealed significant gaps in fundamental security controls, particularly the lack of Multi-Factor Authentication (MFA) for email and computer access. This exposes the organization to a high risk of account compromise and unauthorized access.

Furthermore, a technical scan identified an exposed service on port 8080 of an internal system (\texttt{10.5.5.5}) with a highly concerning banner: \texttt{"TOP SECRET DB"}. This suggests a sensitive database or application is accessible on the network without adequate protection. Critically, this finding directly contradicts the existing risk documentation, which lists this port as a secured false positive. This discrepancy indicates that the current risk register is outdated and does not reflect the actual state of the network, posing a significant meta-risk to the organization's security management program.

Immediate remediation of the exposed service and the implementation of MFA are the highest priorities.

% --- Section 2: Organizational Information ---
\section{Organizational Information}
The following details were provided by the client and used as a baseline for this assessment.

\begin{tabular}{@{}ll}
    \toprule
    \textbf{Attribute} & \textbf{Value} \\
    \midrule
    Organization Name & Signal Flare \\
    Email Domain & \texttt{SignalFlare.net} \\
    Website Domain & \url{www.SignalFlare.net} \\
    External IP Address & \texttt{180.44.245.252} \\
    \bottomrule
\end{tabular}

% --- Section 3: Security Control Review ---
\section{Security Control Review}
A review of organizational security controls was conducted via a standardized questionnaire. The responses indicate several critical gaps in foundational security practices. A summary is provided below.

\begin{table}[h!]
\centering
\caption{Organizational Security Controls Questionnaire}
\begin{tabular}{@{}p{9cm}cc@{}}
    \toprule
    \textbf{Control Question} & \textbf{Response} & \textbf{Status} \\
    \midrule
    Do you require MFA to access email? & No & \textcolor{red}{\ding{55}} \\
    Do you require MFA to log into computers? & No & \textcolor{red}{\ding{55}} \\
    Do you require MFA to access sensitive data systems? & Yes & \textcolor{green}{\ding{51}} \\
    Does your organization have an employee acceptable use policy? & No & \textcolor{red}{\ding{55}} \\
    Does your organization do security awareness training for new employees? & Yes & \textcolor{green}{\ding{51}} \\
    Does your organization do security awareness training for all employees at least once per year? & Yes & \textcolor{green}{\ding{51}} \\
    \bottomrule
\end{tabular}
\end{table}

The absence of MFA for primary access vectors like email and workstations, combined with the lack of an Acceptable Use Policy, represents a significant risk to the organization's data and systems.

% --- Section 4: Technical Scan Results ---
\section{Technical Scan Results}
An active network scan was performed to identify open ports and exposed services on the target system.

\begin{itemize}
    \item \textbf{Target IP Address:} \texttt{10.5.5.5}
    \item \textbf{Scan Date:} 2024-05-21 (Assumed from current date)
\end{itemize}

The scan revealed the following open port:

\begin{table}[h!]
\centering
\caption{Open Port Analysis for \texttt{10.5.5.5}}
\begin{tabular}{@{}llll@{}}
    \toprule
    \textbf{Port} & \textbf{State} & \textbf{Service Info} \\
    \midrule
    8080/tcp & Open & HTTP service with title: \textbf{\texttt{"TOP SECRET DB"}} \\
    \bottomrule
\end{tabular}
\end{table}

\textbf{Analysis:} The presence of an open port with a title explicitly mentioning a "TOP SECRET DB" is a critical finding. This strongly suggests that a sensitive, possibly unauthenticated, database management interface or application is exposed on the network. This service must be investigated immediately to determine its purpose and secure it from unauthorized access.

% --- Section 5: Risk Assessment & Correlation ---
\section{Risk Assessment \& Correlation}
This section synthesizes findings from the security control review, technical scans, and existing risk documentation.

\begin{table}[h!]
\centering
\caption{Summary of Identified Risks}
\begin{tabular}{@{}p{2.5cm}p{9.5cm}l@{}}
    \toprule
    \textbf{Risk Title} & \textbf{Description} & \textbf{Severity} \\
    \midrule
    \textbf{Exposed Sensitive Service} & An HTTP service on \texttt{10.5.5.5:8080} is open and identifies itself as \texttt{"TOP SECRET DB"}. This poses an immediate and severe risk of data exfiltration or system compromise. & \textbf{Critical} \\
    \addlinespace
    \textbf{Lack of Multi-Factor Authentication} & MFA is not enforced for email or computer logins, making user accounts highly susceptible to takeover via phishing or credential stuffing attacks. & \textbf{Critical} \\
    \addlinespace
    \textbf{Outdated Risk Register} & The existing risk register incorrectly lists Port 8080 as a secured false positive. This live finding proves the register is inaccurate, undermining the reliability of the organization's risk management program. & \textbf{High} \\
    \addlinespace
    \textbf{Missing Acceptable Use Policy} & The absence of a formal Acceptable Use Policy (AUP) creates ambiguity regarding security responsibilities for employees and increases the risk of insider threat and policy violations. & \textbf{High} \\
    \bottomrule
\end{tabular}
\end{table}

% --- Section 6: Recommendations ---
\section{Recommendations}
The following prioritized recommendations are provided to mitigate the identified risks and improve the overall security posture of Signal Flare.

\subsection{Priority 1: Immediate Actions (Critical Risks)}
\begin{enumerate}
    \item \textbf{Secure the Exposed Service on Port 8080:}
        \begin{itemize}
            \item Immediately investigate the service running on \texttt{10.5.5.5:8080}.
            \item Identify the owner and purpose of the "TOP SECRET DB".
            \item If the service is not essential, disable it.
            \item If essential, restrict access to authorized personnel only via firewall rules and implement strong authentication and encryption.
        \end{itemize}
    \item \textbf{Implement Multi-Factor Authentication (MFA):}
        \begin{itemize}
            \item Enforce MFA for all user accounts across all critical systems, starting with email (e.g., Office 365, Google Workspace) and workstation logins (e.g., Windows Hello, Duo).
        \end{itemize}
\end{enumerate}

\subsection{Priority 2: Foundational Improvements (High Risks)}
\begin{enumerate}
    \setcounter{enumi}{2} % Continue numbering
    \item \textbf{Develop and Implement an Acceptable Use Policy (AUP):}
        \begin{itemize}
            \item Create a formal AUP that clearly defines the rules for using company IT assets, data handling, and security responsibilities.
            \item Require all employees to read and acknowledge the policy.
        \end{itemize}
    \item \textbf{Review and Update the Risk Register:}
        \begin{itemize}
            \item Conduct a full review of the existing risk register to validate all entries against the current environment.
            \item Correct the entry for Port 8080 to reflect its status as a critical vulnerability.
            \item Establish a formal process for regular review and validation of the risk register.
        \end{itemize}
\end{enumerate}

\end{document}
```