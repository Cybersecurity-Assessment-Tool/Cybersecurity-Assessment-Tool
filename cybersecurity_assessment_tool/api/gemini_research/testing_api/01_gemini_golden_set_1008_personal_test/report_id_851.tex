```latex
\documentclass[12pt, a4paper]{article}

% Preamble: Required Packages
\usepackage[margin=1in]{geometry}
\usepackage{pifont} % For checkmarks and crosses
\usepackage{booktabs} % For professional tables
\usepackage{hyperref} % For clickable links and TOC
\usepackage{url} % For formatting URLs
\usepackage{seqsplit} % To split long monospaced text
\usepackage{graphicx}
\usepackage[table]{xcolor}
\usepackage{tocbibind}

% --- Document Setup ---
\definecolor{criticalred}{HTML}{D12B2B}
\definecolor{highorange}{HTML}{E57300}
\definecolor{mediumyellow}{HTML}{FFBF00}
\definecolor{lowblue}{HTML}{0073E6}
\definecolor{infogray}{HTML}{808080}
\definecolor{tablehead}{gray}{0.9}

\hypersetup{
    colorlinks=true,
    linkcolor=blue,
    filecolor=magenta,      
    urlcolor=cyan,
    pdftitle={Cybersecurity Posture Assessment Report},
    pdfpagemode=FullScreen,
}

\newcommand{\yes}{\ding{51}}
\newcommand{\no}{\ding{55}}

% --- Document Start ---
\begin{document}

% --- Title Page ---
\begin{titlepage}
    \centering
    \vspace*{1cm}
    \includegraphics[width=0.4\textwidth]{example-image-a} % Placeholder for company logo
    
    \vspace{1.5cm}
    
    \Huge
    \textbf{Cybersecurity Posture Assessment Report}
    
    \vspace{1.5cm}
    
    \Large
    \textbf{Prepared for:} \\
    Moxie Marketing
    
    \vspace{2cm}
    
    \large
    \textbf{Report Date:} \\
    \today
    
    \vfill
    
    \large
    \textbf{Generated by:} \\
    Expert Cybersecurity Analyst
    
\end{titlepage}

\newpage
\tableofcontents
\newpage

% --- Section 1: Executive Summary ---
\section{Executive Summary}

This report provides a comprehensive analysis of the cybersecurity posture for \textbf{Moxie Marketing}, conducted on \today. The assessment combines a review of organizational security controls, an external network scan, and an analysis of pre-existing risk data.

The overall security posture is assessed as \textbf{Poor} due to the identification of several critical and high-risk vulnerabilities. These findings indicate significant exposure to common cyber threats such as Business Email Compromise (BEC), unauthorized access, and social engineering.

\textbf{Key Critical Findings include:}
\begin{itemize}
    \item \textbf{Lack of Multi-Factor Authentication (MFA) on Email:} The absence of MFA on the primary email system (\texttt{MoxieMarketing.net}) presents a critical vulnerability. This significantly increases the risk of account takeovers through phishing or credential stuffing attacks.
    \item \textbf{Exposed Localhost Service:} A network service (SSH on port 22) was identified running on the localhost interface (\texttt{127.0.0.1}). This aligns with a pre-identified critical risk and suggests a severe misconfiguration that could allow unauthorized access if the service is improperly exposed to external networks.
\end{itemize}

\textbf{Key High-Risk Findings include:}
\begin{itemize}
    \item \textbf{Inadequate Security Awareness Training:} The organization does not provide mandatory annual security awareness training for all employees. This gap allows security knowledge to become outdated, making the organization more susceptible to evolving threats like sophisticated phishing attacks.
\end{itemize}

Immediate remediation of these issues is strongly recommended to reduce the organization's attack surface and mitigate the risk of a significant security incident. Detailed analysis and actionable recommendations are provided in the subsequent sections of this report.

% --- Section 2: Organizational Information ---
\section{Organizational Information}
The following information was provided for the assessment.

\begin{table}[h!]
\centering
\rowcolors{2}{gray!10}{white}
\begin{tabular}{ll}
\toprule
\textbf{Attribute} & \textbf{Value} \\
\midrule
Organization Name & Moxie Marketing \\
Email Domain & \texttt{MoxieMarketing.net} \\
Website Domain & \seqsplit{\url{www.MoxieMarketing.net}} \\
External IP Address & \texttt{57.21.58.23} \\
\bottomrule
\end{tabular}
\caption{Client Organizational Data}
\label{tab:org_info}
\end{table}

% --- Section 3: Security Control Review ---
\section{Security Control Review (Questionnaire)}
A review of the organization's security controls was conducted via a questionnaire. The results highlight critical gaps in the current security framework.

\begin{table}[h!]
\centering
\begin{tabular}{p{0.7\linewidth}c}
\toprule
\rowcolor{tablehead}
\textbf{Control Question} & \textbf{Response} \\
\midrule
Do you require MFA to access email? & \textcolor{criticalred}{\no} \\
Do you require MFA to log into computers? & \textcolor{lowblue}{\yes} \\
Do you require MFA to access sensitive data systems? & \textcolor{lowblue}{\yes} \\
Does your organization have an employee acceptable use policy? & \textcolor{lowblue}{\yes} \\
Does your organization do security awareness training for new employees? & \textcolor{lowblue}{\yes} \\
Does your organization do security awareness training for all employees at least once per year? & \textcolor{criticalred}{\no} \\
\bottomrule
\end{tabular}
\caption{Security Controls Questionnaire Results}
\label{tab:controls}
\end{table}

\subsection*{Analysis of Controls}
The questionnaire reveals two significant weaknesses:
\begin{enumerate}
    \item \textbf{MFA on Email:} The lack of enforced MFA for email is a critical oversight. Email accounts are a primary target for attackers seeking to launch internal phishing campaigns, commit financial fraud, or access sensitive company data.
    \item \textbf{Annual Security Training:} While new employees receive training, the absence of an annual refresher for all staff is a high-risk gap. The threat landscape evolves continuously, and without ongoing education, employees become more vulnerable to new attack techniques.
\end{enumerate}

% --- Section 4: Technical Scan Results ---
\section{Technical Scan Results}
An Nmap scan was performed to identify open ports and services on the target system.

\begin{itemize}
    \item \textbf{Scan Target:} \texttt{127.0.0.1}
    \item \textbf{Scan Date:} \today
\end{itemize}

\begin{table}[h!]
\centering
\begin{tabular}{ccccc}
\toprule
\rowcolor{tablehead}
\textbf{Port} & \textbf{State} & \textbf{Service (Inferred)} & \textbf{Product} & \textbf{Version} \\
\midrule
22/tcp & open & SSH & \textit{Not Detected} & \textit{Not Detected} \\
\bottomrule
\end{tabular}
\caption{Open Ports Detected on 127.0.0.1}
\label{tab:nmap_results}
\end{table}

\subsection*{Analysis of Technical Findings}
The scan identified that port 22 (SSH - Secure Shell) is open. The scan target of \texttt{127.0.0.1} (localhost) is highly unusual for an external assessment and directly correlates with the pre-existing risk data provided in Input 3 ("Localhost Exposed").

This finding indicates a service intended for local machine administration is running. If this host is a gateway, firewall, or multi-homed server, there is a critical risk that this internal service is improperly exposed to the internet. This could allow an attacker to attempt to brute-force credentials or exploit a vulnerability in the SSH service to gain unauthorized access to the system. The lack of version information from the scan prevents a detailed vulnerability assessment, but the exposure itself is a critical issue.

% --- Section 5: Consolidated Risk Assessment ---
\section{Consolidated Risk Assessment}
The following table synthesizes findings from the security control review, technical scan, and pre-existing risk data into a consolidated list of identified risks.

\begin{table}[h!]
\centering
\resizebox{\textwidth}{!}{%
\begin{tabular}{lp{0.4\linewidth}lll}
\toprule
\rowcolor{tablehead}
\textbf{Risk ID} & \textbf{Risk Name} & \textbf{Severity} & \textbf{Affected Asset(s)} & \textbf{Source} \\
\midrule
RISK-001 & Lack of MFA on Email Accounts & \cellcolor{criticalred!25}Critical & All email accounts & Questionnaire \\
RISK-002 & Exposed Localhost Service (SSH) & \cellcolor{criticalred!25}Critical & Network Host (\texttt{127.0.0.1}) & Nmap Scan, Input 3 \\
RISK-003 & Inadequate Annual Security Training & \cellcolor{highorange!25}High & All Employees & Questionnaire \\
\bottomrule
\end{tabular}
}
\caption{Summary of Identified Risks}
\label{tab:risks}
\end{table}

% --- Section 6: Recommendations ---
\section{Recommendations}
The following actionable recommendations are provided to address the identified risks. Remediation should be prioritized based on severity.

\subsection*{RISK-001: Lack of MFA on Email Accounts (Critical)}
\begin{itemize}
    \item \textbf{Immediate Action:} Procure and enforce an MFA solution for all user accounts on the \texttt{MoxieMarketing.net} email platform immediately.
    \item \textbf{Guidance:} Prioritize phishing-resistant MFA methods such as FIDO2 security keys or authenticator apps (TOTP). Avoid less secure methods like SMS-based MFA if possible.
    \item \textbf{Long-Term:} Integrate MFA across all other critical cloud and on-premise applications to establish a consistent security baseline.
\end{itemize}

\subsection*{RISK-002: Exposed Localhost Service (SSH) (Critical)}
\begin{itemize}
    \item \textbf{Immediate Action:} Urgently investigate the system at \texttt{57.21.58.23} to determine why a service is listening on its localhost interface (\texttt{127.0.0.1}) and how this was detected externally.
    \item \textbf{Guidance:}
        \begin{enumerate}
            \item \textbf{Firewall/Configuration Review:} Ensure that firewall rules block all inbound access to management ports (like SSH) from the internet. The SSH service should be reconfigured to bind \textit{only} to a private management network interface, not a public-facing one.
            \item \textbf{Access Control:} If remote administration is required, access should be restricted to a trusted IP whitelist and ideally placed behind a Virtual Private Network (VPN).
        \end{enumerate}
\end{itemize}

\subsection*{RISK-003: Inadequate Annual Security Training (High)}
\begin{itemize}
    \item \textbf{Immediate Action:} Schedule and mandate a security awareness training session for all current employees within the next quarter.
    \item \textbf{Guidance:} Implement a formal, ongoing security awareness program that includes:
        \begin{itemize}
            \item Mandatory annual training covering topics like phishing, password security, data handling, and acceptable use.
            \item Regular (e.g., quarterly) simulated phishing campaigns to test and reinforce employee knowledge.
            \item A clear process for employees to report suspected security incidents.
        \end{itemize}
\end{itemize}

\end{document}
```