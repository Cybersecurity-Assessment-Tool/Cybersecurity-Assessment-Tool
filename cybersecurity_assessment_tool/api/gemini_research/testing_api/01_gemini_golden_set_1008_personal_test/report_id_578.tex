```latex
\documentclass[12pt]{article}

% --- PACKAGES ---
\usepackage[margin=1in]{geometry}
\usepackage{pifont} % For checkmarks and crosses
\usepackage{booktabs} % For professional tables
\usepackage{hyperref} % For clickable links
\usepackage{url} % For URL formatting
\usepackage{seqsplit} % To split long strings in tt font
\usepackage{graphicx}
\usepackage{xcolor}

% --- DOCUMENT METADATA ---
\title{Cybersecurity Assessment Report \\ \large For: \textbf{Ironclad Logistics}}
\author{Cybersecurity Analysis Division}
\date{\today}

% --- HYPERREF SETUP ---
\hypersetup{
    colorlinks=true,
    linkcolor=blue,
    filecolor=magenta,      
    urlcolor=cyan,
    pdftitle={Cybersecurity Assessment Report for Ironclad Logistics},
    pdfpagemode=FullScreen,
}

% --- DOCUMENT START ---
\begin{document}

\maketitle
\thispagestyle{empty}
\newpage

\tableofcontents
\newpage

% ==============================================================================
% SECTION 1: EXECUTIVE SUMMARY
% ==============================================================================
\section{Executive Summary}

This report details the findings of a cybersecurity assessment conducted for \textbf{Ironclad Logistics}. The analysis synthesizes data from a network vulnerability scan, a security controls questionnaire, and a review of pre-existing risks.

The overall security posture is determined to be critically weak, with significant deficiencies in both technical and administrative controls. Several high-impact vulnerabilities were identified that expose the organization to immediate threats, including potential system compromise and data breaches.

Key findings include:
\begin{itemize}
    \item \textbf{Critical Network Vulnerability:} An internal server (\texttt{10.0.0.15}) is running a dangerously outdated and vulnerable FTP service (\texttt{vsftpd 2.3.4}) with anonymous login enabled. This version is known to contain a backdoor, presenting a direct path for an attacker to gain unauthorized access to the network.
    \item \textbf{Systemic Lack of Multi-Factor Authentication (MFA):} MFA is not enforced for email, computer logins, or access to sensitive data systems. This represents a critical failure in access control, making the organization highly susceptible to credential theft and phishing attacks.
    \item \textbf{Inadequate Security Policies:} The organization lacks a formal employee acceptable use policy and does not provide security awareness training to new hires, creating a high-risk environment where human error can easily lead to security incidents.
    \item \textbf{Pre-existing Risk:} The continued use of Windows 7 workstations, an unsupported operating system, remains a medium-level risk that contributes to the overall insecure environment.
\end{itemize}

Immediate and decisive action is required to remediate these vulnerabilities. Recommendations are prioritized to address the most critical risks first.

% ==============================================================================
% SECTION 2: ORGANIZATIONAL INFORMATION
% ==============================================================================
\section{Organizational Information}

The following details were provided for the assessment.

\begin{tabular}{@{}ll}
    \toprule
    \textbf{Attribute} & \textbf{Value} \\
    \midrule
    Organization Name & \textbf{Ironclad Logistics} \\
    Email Domain & \texttt{IroncladLogistics.net} \\
    Website Domain & \url{www.IroncladLogistics.net} \\
    External IP Address & \texttt{200.82.3.123} \\
    \bottomrule
\end{tabular}

% ==============================================================================
% SECTION 3: SECURITY CONTROL REVIEW
% ==============================================================================
\section{Security Control Review}

The following table summarizes the organization's responses to the security controls questionnaire. The status indicates whether the control aligns with cybersecurity best practices. A cross (\ding{55}) signifies a significant gap in the security framework.

\begin{table}[h!]
\centering
\begin{tabular}{@{}lcc@{}}
    \toprule
    \textbf{Control Question} & \textbf{Response} & \textbf{Status} \\
    \midrule
    Do you require MFA to access email? & No & \textcolor{red}{\ding{55}} \\
    Do you require MFA to log into computers? & No & \textcolor{red}{\ding{55}} \\
    Do you require MFA to access sensitive data systems? & No & \textcolor{red}{\ding{55}} \\
    Does your organization have an employee acceptable use policy? & No & \textcolor{red}{\ding{55}} \\
    Does your organization do security awareness training for new employees? & No & \textcolor{red}{\ding{55}} \\
    Does your organization do security awareness training for all employees at least once per year? & Yes & \textcolor{green}{\ding{51}} \\
    \bottomrule
\end{tabular}
\caption{Security Controls Questionnaire Analysis}
\end{table}

\paragraph{Analysis:} The questionnaire reveals critical deficiencies in foundational security controls. The complete absence of MFA for email, workstations, and sensitive systems is a major vulnerability. Furthermore, the lack of an acceptable use policy and security training for new hires indicates a weak security culture, undermining the effectiveness of the annual training program.

% ==============================================================================
% SECTION 4: TECHNICAL SCAN RESULTS
% ==============================================================================
\section{Technical Scan Results}

A network scan was performed on the target system to identify open ports and exposed services.

\begin{itemize}
    \item \textbf{Target IP Address:} \texttt{10.0.0.15}
\end{itemize}

\begin{table}[h!]
\centering
\begin{tabular}{@{}lllll@{}}
    \toprule
    \textbf{Port} & \textbf{State} & \textbf{Service} & \textbf{Product \& Version} & \textbf{Notes} \\
    \midrule
    21/tcp & Open & ftp & vsftpd 2.3.4 & \begin{tabular}[c]{@{}l@{}}\textbf{CRITICAL:} Anonymous FTP login allowed. \\ This version is known to be vulnerable \\ to a backdoor (CVE-2011-2523).\end{tabular} \\
    \bottomrule
\end{tabular}
\caption{Open Port Analysis}
\end{table}

\paragraph{Analysis:} The scan identified a critical vulnerability on the internal server at \texttt{10.0.0.15}. The FTP service is not only configured to allow anonymous logins, which permits unauthenticated access, but the software version (\texttt{vsftpd 2.3.4}) is over a decade old and contains a well-documented remote command execution backdoor. An attacker on the local network could exploit this to gain complete control of the server.

% ==============================================================================
% SECTION 5: CONSOLIDATED RISK ASSESSMENT
% ==============================================================================
\section{Consolidated Risk Assessment}

The following table correlates findings from the technical scan, security questionnaire, and pre-existing risk data into a prioritized list.

\begin{table}[h!]
\centering
\begin{tabular}{@{}llll@{}}
    \toprule
    \textbf{Risk ID} & \textbf{Risk Description} & \textbf{Source} & \textbf{Severity} \\
    \midrule
    RISK-001 & \begin{tabular}[c]{@{}l@{}}Vulnerable FTP service with anonymous \\ login enabled.\end{tabular} & Technical Scan & \textbf{Critical} \\
    \addlinespace
    RISK-002 & \begin{tabular}[c]{@{}l@{}}Systemic lack of Multi-Factor \\ Authentication (MFA).\end{tabular} & Questionnaire & \textbf{Critical} \\
    \addlinespace
    RISK-003 & \begin{tabular}[c]{@{}l@{}}Inadequate security policies and \\ new-hire training.\end{tabular} & Questionnaire & High \\
    \addlinespace
    RISK-004 & \begin{tabular}[c]{@{}l@{}}Workstations running unsupported \\ Windows 7 OS.\end{tabular} & Pre-existing & Medium \\
    \bottomrule
\end{tabular}
\caption{Summary of Identified Risks}
\end{table}

% ==============================================================================
% SECTION 6: RECOMMENDATIONS
% ==============================================================================
\section{Recommendations}

The following actions are recommended to mitigate the identified risks. They are prioritized based on severity and potential impact.

\subsection*{Immediate Priority (Critical Risks)}
\begin{enumerate}
    \item \textbf{Remediate Vulnerable FTP Service (RISK-001):}
    \begin{itemize}
        \item Immediately take the FTP service on \texttt{10.0.0.15} offline.
        \item If the service is business-critical, upgrade \texttt{vsftpd} to the latest stable version and disable anonymous access.
        \item If FTP is not required, disable and uninstall the service entirely.
    \end{itemize}
    
    \item \textbf{Implement Multi-Factor Authentication (RISK-002):}
    \begin{itemize}
        \item Begin a phased rollout of MFA across the organization, starting with email and all other externally-facing services.
        \item Subsequently, enforce MFA for all privileged user accounts, access to sensitive data systems, and standard workstation logins.
    \end{itemize}
\end{enumerate}

\subsection*{High Priority}
\begin{enumerate}
    \setcounter{enumi}{2} % Continue numbering
    \item \textbf{Develop Foundational Security Policies (RISK-003):}
    \begin{itemize}
        \item Draft and implement a formal Acceptable Use Policy (AUP) that all employees must read and acknowledge.
        \item Institute a mandatory security awareness training module for all new employees as part of the onboarding process.
    \end{itemize}
\end{enumerate}

\subsection*{Medium Priority}
\begin{enumerate}
    \setcounter{enumi}{3} % Continue numbering
    \item \textbf{Upgrade Legacy Operating Systems (RISK-004):}
    \begin{itemize}
        \item Create and execute a plan to upgrade all workstations from Windows 7 to a modern, supported operating system such as Windows 10 or 11.
    \end{itemize}
\end{enumerate}

\end{document}
```