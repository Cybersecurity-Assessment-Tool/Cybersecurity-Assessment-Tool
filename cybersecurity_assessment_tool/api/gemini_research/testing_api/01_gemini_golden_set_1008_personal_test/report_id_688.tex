```latex
\documentclass[12pt]{article}

% Preamble: Required Packages
\usepackage[margin=1in]{geometry}
\usepackage{pifont} % For checkmarks and crosses
\usepackage{booktabs} % For professional-looking tables
\usepackage{hyperref} % For clickable links and better PDF navigation
\usepackage{url} % For formatting URLs
\usepackage{seqsplit} % For splitting long strings without spaces
\usepackage{graphicx}
\usepackage{xcolor}

% Hyperref Setup
\hypersetup{
    colorlinks=true,
    linkcolor=black,
    urlcolor=blue,
    pdftitle={Cybersecurity Posture Assessment Report},
    pdfauthor={Cybersecurity Analyst},
    pdfsubject={Security Assessment},
    pdfkeywords={Cybersecurity, Risk, Assessment, Nmap, Policy},
}

% Define custom colors
\definecolor{darkblue}{rgb}{0.0, 0.0, 0.55}
\definecolor{darkred}{rgb}{0.55, 0.0, 0.0}

% Document Start
\begin{document}

% --- Title Page ---
\begin{titlepage}
    \centering
    \vspace*{1cm}
    
    \Huge\textbf{Cybersecurity Posture Assessment Report}
    
    \vspace{1.5cm}
    
    \Large\textbf{Prepared for:} \\
    \vspace{0.5cm}
    \textbf{Stellar Pathways}
    
    \vfill
    
    \large
    \textbf{Date of Report:} \today \\
    \textbf{Analysis Period:} October 2023
    
    \vspace{1cm}
    
    \textit{This document contains sensitive information and is intended for the exclusive use of the recipient organization. Distribution is strictly prohibited.}
    
\end{titlepage}

\tableofcontents
\newpage

% --- Executive Summary ---
\section*{Executive Summary}

This report presents a cybersecurity posture assessment for \textbf{Stellar Pathways}, conducted by synthesizing technical network scan data, an organizational security controls questionnaire, and a review of pre-existing risks.

The assessment reveals a mixed security posture. On a positive note, the external-facing network scan of the target host \texttt{192.168.1.100} found no open ports, indicating a strong firewall configuration and a hardened network perimeter for that specific asset. This significantly reduces the external attack surface.

However, the review of internal security controls identified two critical gaps that present a high level of risk to the organization:
\begin{enumerate}
    \item \textbf{Lack of Multi-Factor Authentication (MFA) on Computers:} The absence of MFA for endpoint logins creates a significant vulnerability. A single compromised password could grant an attacker full access to an employee's workstation, leading to data theft, lateral movement within the network, and deployment of ransomware.
    \item \textbf{Absence of an Employee Acceptable Use Policy (AUP):} Without a formal AUP, there is no clear guidance for employees on the secure and appropriate use of company assets. This increases the risk of insider threats (both malicious and accidental) and creates potential legal and compliance liabilities.
\end{enumerate}

No pre-existing vulnerabilities were reported for correlation. The primary focus of remediation efforts should be on addressing the identified internal policy and authentication control gaps. Recommendations are provided to mitigate these high-priority risks effectively.

% --- Organizational Information ---
\section*{1. Organizational Information}

The following details were provided by the client for this assessment.

\begin{tabular}{@{}ll}
    \toprule
    \textbf{Attribute} & \textbf{Value} \\
    \midrule
    Organization Name & \textbf{Stellar Pathways} \\
    Primary Email Domain & \texttt{StellarPathways.org} \\
    Primary Website & \url{www.StellarPathways.org} \\
    Known External IP & \texttt{179.120.54.121} \\
    \bottomrule
\end{tabular}

% --- Security Control Review ---
\section*{2. Security Control Review}

A review of the organization's security controls was conducted via a questionnaire. The results below highlight current practices and identify key areas for improvement. A red cross (\textcolor{darkred}{\ding{55}}) indicates a significant control gap.

\begin{tabular}{@{}p{0.6\linewidth} c c}
    \toprule
    \textbf{Control Question} & \textbf{Response} & \textbf{Status} \\
    \midrule
    Do you require MFA to access email? & Yes & \textcolor{darkblue}{\ding{51}} \\
    \textbf{Do you require MFA to log into computers?} & \textbf{No} & \textcolor{darkred}{\ding{55}} \\
    Do you require MFA to access sensitive data systems? & Yes & \textcolor{darkblue}{\ding{51}} \\
    \textbf{Does your organization have an employee acceptable use policy?} & \textbf{No} & \textcolor{darkred}{\ding{55}} \\
    Does your organization do security awareness training for new employees? & Yes & \textcolor{darkblue}{\ding{51}} \\
    Does your organization do security awareness training for all employees at least once per year? & Yes & \textcolor{darkblue}{\ding{51}} \\
    \bottomrule
\end{tabular}

% --- Technical Scan Results ---
\section*{3. Technical Scan Results}

A network port scan was performed using Nmap to identify exposed services on the target system.

\begin{itemize}
    \item \textbf{Target IP Address:} \texttt{192.168.1.100}
    \item \textbf{Scan Status:} Completed
    \item \textbf{Host Status:} Up
\end{itemize}

\textbf{Findings:} The scan confirmed that the host is online but found \textbf{zero open TCP ports}. All 1000 scanned ports were reported as 'closed'. This is a positive security finding, as it indicates the host is not exposing any network services to the scanner's location and is likely protected by a well-configured firewall.

% --- Risk Assessment ---
\section*{4. Risk Assessment}

This section correlates the findings from the security control review and technical scans. The following risks have been identified and prioritized based on their potential impact on the organization.

\begin{tabular}{@{}p{0.15\linewidth} p{0.5\linewidth} p{0.2\linewidth}@{}}
    \toprule
    \textbf{Risk ID} & \textbf{Risk Description} & \textbf{Severity} \\
    \midrule
    RISK-001 & \textbf{Lack of Endpoint MFA.} A threat actor with compromised credentials (e.g., from a phishing attack) can directly access an employee's computer, leading to data exfiltration, lateral movement, or ransomware deployment. & \textbf{High} \\
    \addlinespace
    RISK-002 & \textbf{Absence of Acceptable Use Policy (AUP).} Lack of a formal policy creates ambiguity regarding proper use of IT assets. This increases the risk of insider threat, data leakage, and non-compliance with regulatory standards. & \textbf{High} \\
    \bottomrule
\end{tabular}

% --- Recommendations ---
\section*{5. Recommendations}

The following actions are recommended to mitigate the identified risks and improve the overall security posture of \textbf{Stellar Pathways}.

\begin{enumerate}
    \item \textbf{Implement MFA for All Computer Logins (RISK-001):}
    \begin{itemize}
        \item \textbf{Priority:} \textcolor{darkred}{CRITICAL}
        \item \textbf{Action:} Procure and deploy a Multi-Factor Authentication solution for all Windows, macOS, and Linux endpoints. This should be mandatory for all employees and contractors. This single control dramatically reduces the risk of unauthorized access from compromised credentials.
    \end{itemize}
    \vspace{0.5cm}
    
    \item \textbf{Develop and Enforce an Acceptable Use Policy (RISK-002):}
    \begin{itemize}
        \item \textbf{Priority:} \textcolor{darkred}{HIGH}
        \item \textbf{Action:} Draft a formal AUP that clearly defines the rules and expectations for using company networks, devices, and data. The policy should be reviewed by legal counsel, approved by management, and communicated to all employees. Require employees to formally acknowledge they have read and understood the policy.
    \end{itemize}
\end{enumerate}

% --- Conclusion ---
\section*{6. Conclusion}

The security assessment of \textbf{Stellar Pathways} highlights a commendable network security posture on the scanned asset but reveals critical deficiencies in foundational internal security controls. While the hardened network perimeter reduces external threats, the identified gaps in authentication and policy represent a significant internal risk.

We strongly urge the organization to prioritize the implementation of Multi-Factor Authentication for all computer endpoints and the development of a formal Acceptable Use Policy. Addressing these two high-severity risks will substantially strengthen the organization's defense against common and impactful cyber threats.

\end{document}
```