```latex
\documentclass[12pt]{article}

% ----------------------------------------------------------------------
% PREAMBLE
% ----------------------------------------------------------------------
\usepackage[margin=1in]{geometry}
\usepackage{pifont} % For checkmarks and crosses
\usepackage{booktabs} % For professional tables
\usepackage{hyperref} % For hyperlinks and document metadata
\usepackage{url} % For URL formatting
\usepackage{seqsplit} % To split long monospaced strings
\usepackage[T1]{fontenc}

% Define checkmark and cross symbols for convenience
\newcommand{\cmark}{\ding{51}}
\newcommand{\xmark}{\ding{55}}

% Document Metadata
\hypersetup{
    colorlinks=true,
    linkcolor=black,
    filecolor=magenta,      
    urlcolor=blue,
    pdftitle={Cybersecurity Posture Assessment Report},
    pdfauthor={Cybersecurity Analysis Division},
    pdfsubject={Security Assessment},
    pdfkeywords={Cybersecurity, Risk, Assessment, LaTeX},
    bookmarks=true
}

% ----------------------------------------------------------------------
% DOCUMENT START
% ----------------------------------------------------------------------
\begin{document}

\title{Cybersecurity Posture Assessment Report \\ \large For: \textbf{Falcon Heavy}}
\author{Cybersecurity Analysis Division}
\date{\today}
\maketitle

\hrule
\vspace{1em}
\begin{abstract}
\noindent This report provides a cybersecurity posture assessment for \textbf{Falcon Heavy}. The analysis is based on a combination of organizational data, a security controls questionnaire, and technical network scan results. This document outlines identified risks, analyzes their potential impact, and provides actionable recommendations to enhance the organization's security posture.

\vspace{1em}
\noindent \textbf{Note on Data Integrity:} During this assessment, it was determined that the provided technical network scan data (\texttt{Input\_1\_Network\_Scan\_JSON}) and the list of current risks (\texttt{Input\_3\_Current\_Risks\_JSON}) were corrupted or incomplete. Therefore, this report's findings are primarily based on the analysis of the organizational and security questionnaire data (\texttt{Input\_2\_Org\_Data\_JSON}). A comprehensive technical assessment requires a successful network scan.
\end{abstract}
\hrule

\newpage
\tableofcontents
\newpage

% ----------------------------------------------------------------------
% SECTION 1: EXECUTIVE OVERVIEW
% ----------------------------------------------------------------------
\section{Executive Overview}

The primary objective of this assessment was to evaluate the current security posture of \textbf{Falcon Heavy} by correlating organizational policies, existing risks, and technical vulnerabilities.

Due to data corruption in the technical scan and current risk inputs, this analysis focuses on critical policy and procedural gaps identified through the security controls questionnaire. The assessment reveals significant risks related to access control, corporate policy, and employee security training.

Key findings include:
\begin{itemize}
    \item \textbf{Critical Risk:} Multi-Factor Authentication (MFA) is not enforced for accessing sensitive data systems, leaving critical assets vulnerable to unauthorized access.
    \item \textbf{High Risk:} The absence of a formal Employee Acceptable Use Policy (AUP) creates ambiguity regarding security responsibilities and acceptable behavior.
    * \textbf{High Risk:} New employees do not receive security awareness training during onboarding, representing a missed opportunity to establish a security-conscious culture from day one.
\end{itemize}

While the organization has implemented some positive security controls, such as MFA for email and computer logins, the identified gaps require immediate attention to mitigate potential threats. Recommendations are provided to address each finding systematically.

% ----------------------------------------------------------------------
% SECTION 2: ORGANIZATIONAL INFORMATION
% ----------------------------------------------------------------------
\section{Organizational Information}

The following details were provided for the assessment. This information establishes the context and scope for the analysis.

\begin{table}[h!]
\centering
\begin{tabular}{@{}ll@{}}
\toprule
\textbf{Attribute} & \textbf{Value} \\ \midrule
Organization Name & \textbf{Falcon Heavy} \\
Email Domain & \texttt{FalconHeavy.org} \\
Website Domain & \url{www.FalconHeavy.org} \\
External IP Address & \seqsplit{\texttt{199.145.173.167}} \\ \bottomrule
\end{tabular}
\caption{Client Organizational Details.}
\label{tab:org_info}
\end{table}

% ----------------------------------------------------------------------
% SECTION 3: SECURITY CONTROL REVIEW
% ----------------------------------------------------------------------
\section{Security Control Review (Questionnaire Analysis)}

The following table summarizes the organization's responses to a security controls questionnaire. This review provides insight into the current policies and procedures governing the security environment. Answers marked with a red \xmark\ indicate a deviation from security best practices and represent a potential risk.

\begin{table}[h!]
\centering
\begin{tabular}{@{}p{0.7\textwidth}c@{}}
\toprule
\textbf{Control Question} & \textbf{Response} \\ \midrule
Do you require MFA to access email? & \cmark \\
Do you require MFA to log into computers? & \cmark \\
Do you require MFA to access sensitive data systems? & \xmark \\
Does your organization have an employee acceptable use policy? & \xmark \\
Does your organization do security awareness training for new employees? & \xmark \\
Does your organization do security awareness training for all employees at least once per year? & \cmark \\ \bottomrule
\end{tabular}
\caption{Security Controls Questionnaire Results.}
\label{tab:controls}
\end{table}

\subsection{Analysis of Gaps}
The questionnaire reveals three significant gaps in the organization's security program:
\begin{itemize}
    \item \textbf{MFA on Sensitive Systems:} The lack of MFA on systems holding sensitive data is a critical vulnerability. Should an attacker compromise a user's credentials, they would have direct access to the organization's most valuable information.
    \item \textbf{Acceptable Use Policy (AUP):} An AUP is a foundational document that defines how employees can use company resources. Without one, there is no formal standard for user behavior, data handling, or consequences for misuse, increasing the risk of insider threats and accidental data loss.
    \item \textbf{New Employee Training:} Failing to train new employees on security best practices during onboarding means they may be unaware of company policies, common threats like phishing, and their role in protecting the organization. This gap makes them a primary target for social engineering attacks.
\end{itemize}

% ----------------------------------------------------------------------
% SECTION 4: TECHNICAL SCAN RESULTS
% ----------------------------------------------------------------------
\section{Technical Scan Results}

\textbf{The network scan data provided for this assessment was found to be corrupted and could not be parsed.}

A technical network scan is essential for identifying vulnerabilities at the infrastructure level, such as:
\begin{itemize}
    \item Open ports and exposed services.
    \item Outdated software with known vulnerabilities (e.g., CVEs).
    \item Insecure service configurations.
\end{itemize}
Without this data, a complete picture of the organization's external attack surface cannot be formed. It is strongly recommended to conduct a new network scan against the target IP address (\seqsplit{\texttt{199.145.173.167}}) to complete this portion of the assessment.

% ----------------------------------------------------------------------
% SECTION 5: RISK ASSESSMENT
% ----------------------------------------------------------------------
\section{Risk Assessment}

This risk assessment is based on the findings from the Security Control Review. The severity level (Critical, High, Medium, Low) is assigned based on the potential impact of the risk and the likelihood of its occurrence. Due to incomplete input data, pre-existing risks could not be included.

\begin{table}[h!]
\centering
\begin{tabular}{@{}p{0.15\textwidth}p{0.3\textwidth}p{0.35\textwidth}p{0.1\textwidth}@{}}
\toprule
\textbf{Risk ID} & \textbf{Risk Name} & \textbf{Description} & \textbf{Severity} \\ \midrule
RISK-001 & Lack of MFA on Sensitive Systems & User credentials are the only barrier to sensitive data, making it highly susceptible to compromise via phishing or password spraying. & Critical \\
\addlinespace
RISK-002 & No Employee Acceptable Use Policy & Lack of a formal policy leads to inconsistent security practices and increases the risk of insider threat and non-compliance. & High \\
\addlinespace
RISK-003 & Inadequate Onboarding Security Training & New employees are not trained on security policies and threats, making them more vulnerable to social engineering and accidental breaches. & High \\ \bottomrule
\end{tabular}
\caption{Summary of Identified Risks.}
\label{tab:risks}
\end{table}

% ----------------------------------------------------------------------
% SECTION 6: RECOMMENDATIONS
% ----------------------------------------------------------------------
\section{Recommendations}

The following actionable recommendations are provided to address the identified risks and strengthen the overall security posture of \textbf{Falcon Heavy}.

\subsection{Immediate Actions (1-30 Days)}
\begin{description}
    \item[For RISK-001:] \textbf{Implement MFA for Sensitive Systems.}
    Prioritize the deployment of a robust Multi-Factor Authentication solution across all applications and systems that store, process, or transmit sensitive or critical data. This is the single most effective control to prevent unauthorized access.

    \item[For RISK-002:] \textbf{Develop and Implement an Acceptable Use Policy (AUP).}
    Draft a formal AUP that clearly outlines the rules for using company IT assets, data handling responsibilities, and the consequences for violations. Require all current employees to read and acknowledge the policy.
\end{description}

\subsection{Short-Term Actions (30-90 Days)}
\begin{description}
    \item[For RISK-003:] \textbf{Integrate Security Training into Onboarding.}
    Develop a mandatory security awareness training module for the new employee onboarding process. This module should cover key topics such as phishing, password hygiene, data handling, and the new AUP.

    \item[For Data Integrity:] \textbf{Conduct a New Network Scan.}
    Perform a new, authenticated and unauthenticated, external network scan to gather the necessary technical data. This will allow for a complete assessment of the organization's attack surface and identify any unpatched or misconfigured services.
\end{description}

% ----------------------------------------------------------------------
% DOCUMENT END
% ----------------------------------------------------------------------
\end{document}
```