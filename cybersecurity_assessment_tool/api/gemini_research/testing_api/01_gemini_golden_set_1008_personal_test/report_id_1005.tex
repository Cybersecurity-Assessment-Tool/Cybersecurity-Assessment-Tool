```latex
\documentclass[12pt, a4paper]{article}

% Preamble: Required Packages
\usepackage[margin=1in]{geometry}
\usepackage{pifont} % For \ding
\usepackage{booktabs} % For professional tables
\usepackage[hidelinks]{hyperref}
\usepackage{url}
\usepackage{seqsplit} % For breaking long strings in tt font
\usepackage[table]{xcolor} % For coloring table cells

% Document Metadata
\title{Cybersecurity Posture Assessment Report}
\author{Cybersecurity Analysis Division}
\date{\today}

\begin{document}

\maketitle
\thispagestyle{empty}
\newpage
\tableofcontents
\newpage

% --- 1. Executive Overview ---
\section*{1. Executive Overview}
This report provides a comprehensive cybersecurity assessment for \textbf{Harbor Light Foundation}, synthesizing data from a network vulnerability scan, an organizational security questionnaire, and a review of pre-existing risk documentation. 

The analysis reveals several critical and high-risk security gaps that require immediate attention. Most notably, a potentially sensitive internal service with the title \texttt{"TOP SECRET DB"} was discovered on port 8080. This finding directly contradicts the existing risk documentation, which incorrectly classifies this port as secure. This discrepancy highlights a potential failure in the vulnerability management and validation process.

Furthermore, critical deficiencies were identified in access control, with a lack of Multi-Factor Authentication (MFA) for email and computer logins. Foundational security policies and recurring employee training programs are also absent, significantly increasing the organization's susceptibility to both external threats and insider risk. Immediate remediation of these issues is strongly recommended to mitigate the risk of a significant security breach.

% --- 2. Organizational Information ---
\section*{2. Organizational Information}
The following information was provided by the client organization and used as a baseline for this assessment.

\begin{table}[h!]
\centering
\begin{tabular}{@{}ll@{}}
\toprule
\textbf{Attribute} & \textbf{Value} \\ \midrule
Organization Name    & \textbf{Harbor Light Foundation} \\
Email Domain         & \texttt{HarborLightFoundation.org} \\
External IP Address  & \texttt{119.38.18.158} \\ \bottomrule
\end{tabular}
\caption{Client Organizational Details.}
\end{table}

% --- 3. Security Control Review ---
\section*{3. Security Control Review}
A review of the organization's security controls was conducted via a questionnaire. The responses indicate significant gaps in foundational security practices. "No" answers, marked with a red \ding{55}, represent areas of high risk that weaken the overall security posture.

\begin{table}[h!]
\centering
\renewcommand{\arraystretch}{1.2}
\begin{tabular}{@{}p{0.7\textwidth}c@{}}
\toprule
\textbf{Control Question} & \textbf{Response} \\ \midrule
Do you require MFA to access email? & \textcolor{red}{\ding{55}} \\
Do you require MFA to log into computers? & \textcolor{red}{\ding{55}} \\
Do you require MFA to access sensitive data systems? & \textcolor{green}{\ding{51}} \\
Does your organization have an employee acceptable use policy? & \textcolor{red}{\ding{55}} \\
Does your organization do security awareness training for new employees? & \textcolor{green}{\ding{51}} \\
Does your organization do security awareness training for all employees at least once per year? & \textcolor{red}{\ding{55}} \\ \bottomrule
\end{tabular}
\caption{Security Controls Questionnaire Results.}
\end{table}

% --- 4. Technical Scan Results ---
\section*{4. Technical Scan Results}
An external network scan was performed to identify open ports and exposed services. A critical finding was discovered on an internal host, indicating a potentially misconfigured and highly sensitive service is accessible.

\begin{table}[h!]
\centering
\renewcommand{\arraystretch}{1.2}
\begin{tabular}{@{}llll@{}}
\toprule
\textbf{Target IP} & \textbf{Port} & \textbf{State} & \textbf{Service Details} \\ \midrule
\texttt{10.5.5.5} & 8080 & Open & HTTP Title: \seqsplit{\texttt{"TOP SECRET DB"}} \\ \bottomrule
\end{tabular}
\caption{Nmap Scan Findings.}
\end{table}

% --- 5. Correlated Risk Assessment ---
\section*{5. Correlated Risk Assessment}
The following table synthesizes findings from the security questionnaire, technical scan, and existing risk data. The risks are prioritized based on their potential impact on the organization's confidentiality, integrity, and availability.

\begin{table}[h!]
\centering
\renewcommand{\arraystretch}{1.5}
\begin{tabular}{@{}p{0.1\textwidth}p{0.2\textwidth}p{0.45\textwidth}p{0.1\textwidth}@{}}
\toprule
\textbf{Risk ID} & \textbf{Risk Title} & \textbf{Description} & \textbf{Severity} \\ \midrule
\rowcolor{red!20}
R-001 & Exposed Sensitive Service & Port 8080 on host \texttt{10.5.5.5} is open and presents a title suggesting it is a sensitive database. This is a high-value target for attackers. & Critical \\
\rowcolor{red!20}
R-002 & Critical MFA Gaps & Lack of MFA on email and computer logins exposes the organization to account takeover, phishing, and subsequent data breaches. & Critical \\
\rowcolor{orange!20}
R-003 & Inaccurate Risk Register & The existing risk register incorrectly states Port 8080 is secure. This indicates a flawed vulnerability management or validation process. & High \\
\rowcolor{orange!20}
R-004 & Policy and Training Deficiencies & The absence of an Acceptable Use Policy and mandatory annual security training increases human-related risks and liability. & High \\
\bottomrule
\end{tabular}
\caption{Summary of Identified Risks.}
\end{table}

% --- 6. Recommendations ---
\section*{6. Recommendations}
Based on the correlated risk assessment, the following actions are recommended to improve the security posture of \textbf{Harbor Light Foundation}.

\subsection*{Immediate Actions (0-7 Days)}
\begin{itemize}
    \item \textbf{Investigate and Secure Port 8080 (Risk R-001):} Immediately investigate the service running on \texttt{10.5.5.5:8080}. Determine its purpose and owner. If sensitive, restrict access immediately via firewall rules to only authorized personnel. If the service is non-essential, it should be disabled.
    \item \textbf{Deploy MFA on Email (Risk R-002):} Prioritize the enforcement of MFA for all email accounts. This is the single most effective control to prevent business email compromise.
\end{itemize}

\subsection*{Short-Term Actions (1-3 Months)}
\begin{itemize}
    \item \textbf{Deploy MFA on Endpoints (Risk R-002):} Implement MFA for all employee computer logins to protect against unauthorized physical and remote access.
    \item \textbf{Review Risk Management Process (Risk R-003):} Conduct a full review of the vulnerability management and risk assessment process. Determine why the Port 8080 risk was previously misclassified and implement corrective actions to ensure the accuracy of future assessments.
    \item \textbf{Develop an Acceptable Use Policy (AUP) (Risk R-004):} Create and enforce a formal AUP that clearly defines the rules for using company technology and data.
\end{itemize}

\subsection*{Long-Term Actions (3-6 Months)}
\begin{itemize}
    \item \textbf{Establish Annual Security Training (Risk R-004):} Implement a mandatory security awareness training program for all employees, to be conducted at least annually. This program should cover topics such as phishing, password security, and the new AUP.
\end{itemize}

\end{document}
```