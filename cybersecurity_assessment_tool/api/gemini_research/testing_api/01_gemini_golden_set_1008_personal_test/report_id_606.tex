```latex
\documentclass[12pt]{article}

% --- PACKAGES ---
\usepackage[a4paper, margin=1in]{geometry}
\usepackage{pifont} % For checkmarks and crosses
\usepackage{booktabs} % For professional tables
\usepackage{hyperref} % For hyperlinks and PDF metadata
\usepackage{url}      % For formatting URLs
\usepackage{seqsplit} % For splitting long strings in tt font
\usepackage{graphicx} % For logo (placeholder)
\usepackage{xcolor}   % For colors

% --- DOCUMENT METADATA ---
\hypersetup{
    colorlinks=true,
    linkcolor=blue,
    filecolor=magenta,      
    urlcolor=cyan,
    pdftitle={Cybersecurity Posture Report},
    pdfauthor={Automated Security Analysis System},
    pdfsubject={Security Assessment},
    pdfkeywords={Cybersecurity, Risk, Analysis},
    bookmarks=true
}

% --- HELPER COMMANDS ---
\newcommand{\yes}{\ding{51}}
\newcommand{\no}{\ding{55}}

% --- DOCUMENT START ---
\begin{document}

% --- TITLE PAGE ---
\begin{titlepage}
    \centering
    \vspace*{1cm}
    
    \Huge
    \textbf{Cybersecurity Posture Report}
    
    \vspace{1.5cm}
    
    \Large
    Prepared for: \\
    \vspace{0.5cm}
    \textbf{Tidal Wave Sports}
    
    \vspace{2cm}
    
    \large
    \textbf{Date of Analysis:} \today
    
    \vfill
    
    \large
    \textit{This report contains a summary of findings based on network scans, organizational data, and pre-existing risk information. All information should be considered confidential.}
    
\end{titlepage}

\tableofcontents
\newpage

% --- EXECUTIVE OVERVIEW ---
\section{Executive Overview}
This report provides a consolidated cybersecurity assessment for \textbf{Tidal Wave Sports}, synthesizing technical network scan data, a security controls questionnaire, and a review of previously identified risks.

The analysis reveals a critical security gap: the absence of Multi-Factor Authentication (MFA) for email access. Email is a primary vector for sophisticated attacks, and this gap exposes the organization to significant risks, including Business Email Compromise (BEC), phishing, and account takeovers.

On a positive note, the technical scan indicates that a previously documented risk—an unencrypted web server on port 80—appears to have been remediated, as the port was found to be closed during the assessment. While the organization demonstrates a solid foundation in security policies and employee training, the immediate priority must be the enforcement of MFA on the \texttt{TidalWaveSports.org} email domain to mitigate the most pressing threat.

% --- ORGANIZATIONAL INFORMATION ---
\section{Organizational Information}
The following details were provided for the assessment. This information is used to contextualize the findings and recommendations.

\begin{itemize}
    \item \textbf{Organization Name:} Tidal Wave Sports
    \item \textbf{Email Domain:} \texttt{TidalWaveSports.org}
    \item \textbf{Website Domain:} \url{www.TidalWaveSports.org}
    \item \textbf{External IP Address:} \texttt{230.112.73.48}
\end{itemize}

% --- SECURITY CONTROL REVIEW ---
\section{Security Control Review}
A review of the organization's security controls was conducted via a questionnaire. The responses indicate a good baseline for security policies and awareness training. However, a critical gap was identified in access control for email systems.

\begin{table}[h!]
\centering
\caption{Security Controls Questionnaire Results}
\begin{tabular}{p{0.75\linewidth} c}
\toprule
\textbf{Control Question} & \textbf{Response} \\
\midrule
Does your organization have an employee acceptable use policy? & \yes \\
Does your organization do security awareness training for new employees? & \yes \\
Does your organization do security awareness training for all employees at least once per year? & \yes \\
Do you require MFA to log into computers? & \yes \\
Do you require MFA to access sensitive data systems? & \yes \\
\textbf{Do you require MFA to access email?} & \textcolor{red}{\no} \\
\bottomrule
\end{tabular}
\end{table}

The failure to enforce MFA on email (\texttt{TidalWaveSports.org}) is the most significant weakness identified in this review. This control is fundamental to protecting against credential theft and unauthorized access.

% --- TECHNICAL SCAN RESULTS ---
\section{Technical Scan Results}
An external network scan was performed to identify open ports and exposed services. The scan provides a snapshot of the target's network perimeter.

\begin{itemize}
    \item \textbf{Target IP Address:} \texttt{192.168.0.5}
\end{itemize}

\begin{table}[h!]
\centering
\caption{Nmap Port Scan Findings}
\begin{tabular}{llll}
\toprule
\textbf{Port} & \textbf{State} & \textbf{Service} & \textbf{Product / Version} \\
\midrule
80/tcp & closed & http & N/A \\
\bottomrule
\end{tabular}
\end{table}

\textbf{Analysis:} The scan of the target system revealed no open ports. The finding that port 80 is closed is particularly noteworthy, as it contradicts a previously documented risk. This suggests that remediation has occurred, which is a positive security development.

% --- RISK ASSESSMENT & CORRELATION ---
\section{Risk Assessment \& Correlation}
This section correlates findings from the security questionnaire, the technical scan, and pre-existing risk data to provide a holistic view of the current risk posture.

\begin{table}[h!]
\centering
\caption{Consolidated Risk Summary}
\begin{tabular}{p{0.25\linewidth} p{0.5\linewidth} l}
\toprule
\textbf{Risk Name} & \textbf{Description} & \textbf{Severity} \\
\midrule
\textbf{Lack of MFA on Email} & Email accounts are protected only by passwords, making them highly vulnerable to phishing, credential stuffing, and takeover attacks. This exposes sensitive communications and can be a launchpad for internal attacks. & \textbf{High} \\
\addlinespace
\textbf{Unencrypted Web Server (Resolved)} & A previously identified risk (CVSS 5.0) indicated that port 80 was open. The current scan confirms this port is now closed, mitigating the risk of unencrypted data transmission and associated attacks. & \textbf{Informational} \\
\bottomrule
\end{tabular}
\end{table}

% --- RECOMMENDATIONS ---
\section{Recommendations}
Based on the correlated findings, the following actions are recommended to improve the security posture of \textbf{Tidal Wave Sports}.

\subsection{Priority 1: Implement MFA for Email (Critical)}
\begin{itemize}
    \item \textbf{Action:} Immediately enable and enforce Multi-Factor Authentication (MFA) for all user mailboxes on the \texttt{TidalWaveSports.org} domain.
    \item \textbf{Justification:} This is the single most effective control to prevent email account compromise. It provides a critical layer of defense against password-based attacks, safeguarding sensitive data and preventing attackers from using a compromised email account to pivot into other systems.
\end{itemize}

\subsection{Priority 2: Validate Risk Register (Recommended)}
\begin{itemize}
    \item \textbf{Action:} Formally update the internal risk register to mark the "Unencrypted Web Server" vulnerability as resolved based on the recent scan data.
    \item \textbf{Justification:} Maintaining an accurate and up-to-date risk register is essential for effective security management. It ensures that resources are allocated to current, relevant threats and provides an accurate picture of the organization's security posture to leadership.
\end{itemize}

\end{document}
```