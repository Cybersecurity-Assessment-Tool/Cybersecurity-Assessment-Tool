```latex
\documentclass[12pt]{article}

% Required Packages
\usepackage[margin=1in]{geometry}
\usepackage{pifont} % For \ding
\usepackage{booktabs} % For professional tables
\usepackage{hyperref} % For hyperlinks
\usepackage{url} % For URL formatting
\usepackage{seqsplit} % For splitting long strings in tt font
\usepackage{graphicx}
\usepackage{xcolor}

% --- Document Setup ---
\hypersetup{
    colorlinks=true,
    linkcolor=blue,
    filecolor=magenta,      
    urlcolor=cyan,
    pdftitle={Cybersecurity Posture Report},
    pdfpagemode=FullScreen,
}

\newcommand{\yes}{\ding{51}}
\newcommand{\no}{\ding{55}}

% --- Document Start ---
\begin{document}

% --- Title Page ---
\begin{titlepage}
    \centering
    \vspace*{1cm}
    \Huge\textbf{Cybersecurity Posture Report}
    \vspace{1.5cm}
    \Large
    \textbf{Prepared for:}\\
    \vspace{0.5cm}
    \textbf{Paper Plane Publishing}
    \vspace{2cm}
    \large
    \textbf{Report Date:}\\
    \vspace{0.5cm}
    \today
    \vfill
    \small
    \textit{This report contains sensitive information and should be handled with care. Distribution is restricted to authorized personnel only.}
\end{titlepage}

\tableofcontents
\newpage

% --- Section 1: Executive Summary ---
\section{Executive Summary}

This report provides a comprehensive analysis of the cybersecurity posture for \textbf{Paper Plane Publishing}, based on a network scan, a review of organizational security controls, and an assessment of pre-existing risks.

The analysis reveals several critical-risk findings that require immediate attention. The primary areas of concern are:
\begin{itemize}
    \item \textbf{Critical Gaps in Multi-Factor Authentication (MFA):} MFA is not enforced for accessing email or sensitive data systems. This significantly increases the risk of account compromise and unauthorized data access.
    \item \textbf{Inadequate Security Training:} Security awareness training is not conducted annually for all employees, leaving the organization more susceptible to social engineering and phishing attacks.
    \item \textbf{Insecure Service Exposure:} The network scan identified an open Remote Desktop Protocol (RDP) port on an internal host. This finding, correlated with pre-existing risk data showing a similar issue on another host, points to a systemic problem in network configuration that presents a common and high-impact vector for ransomware attacks.
\end{itemize}

The combination of these vulnerabilities creates a high-risk environment. Immediate remediation is strongly recommended to mitigate the likelihood of a significant security incident.

% --- Section 2: Organizational Information ---
\section{Organizational Information}
The following details were provided for the assessment.

\begin{tabular}{@{}ll}
    \toprule
    \textbf{Attribute} & \textbf{Value} \\
    \midrule
    Organization Name & Paper Plane Publishing \\
    Email Domain & \seqsplit{\texttt{PaperPlanePublishing.net}} \\
    Website Domain & \seqsplit{\url{www.PaperPlanePublishing.net}} \\
    External IP Address & \seqsplit{\texttt{126.200.103.97}} \\
    \bottomrule
\end{tabular}

% --- Section 3: Security Control Review ---
\section{Security Control Review}
A review of administrative and technical security controls was conducted based on a standardized questionnaire. The results highlight significant gaps in the organization's defense-in-depth strategy.

\begin{table}[h!]
\centering
\caption{Security Control Questionnaire Analysis}
\begin{tabular}{@{}p{8cm}ccp{4cm}@{}}
    \toprule
    \textbf{Control Question} & \textbf{Response} & \textbf{Assessment} \\
    \midrule
    Do you require MFA to access email? & No & \no & \textcolor{red}{\textbf{Critical Gap}} \\
    Do you require MFA to log into computers? & Yes & \yes & Meets Best Practice \\
    Do you require MFA to access sensitive data systems? & No & \no & \textcolor{red}{\textbf{Critical Gap}} \\
    Does your organization have an employee acceptable use policy? & Yes & \yes & Meets Best Practice \\
    Does your organization do security awareness training for new employees? & Yes & \yes & Meets Best Practice \\
    Does your organization do security awareness training for all employees at least once per year? & No & \no & \textcolor{orange}{\textbf{High Risk}} \\
    \bottomrule
\end{tabular}
\end{table}

% --- Section 4: Technical Scan Results ---
\section{Technical Scan Results}
A network scan was performed on the target host \texttt{10.10.10.51} to identify open ports and exposed services.

\subsection{Host: \texttt{10.10.10.51}}
The host was found to be online and responsive. The scan identified one open port.

\begin{table}[h!]
\centering
\caption{Open Ports on \texttt{10.10.10.51}}
\begin{tabular}{@{}llll@{}}
    \toprule
    \textbf{Port} & \textbf{State} & \textbf{Service} & \textbf{Analysis} \\
    \midrule
    3389/tcp & open & ms-wbt-server & This port is used for Microsoft Remote Desktop \\
             &        & (RDP) & Protocol. Exposing RDP without proper controls \\
             &        &       & (e.g., VPN, MFA) is a major security risk. \\
    \bottomrule
\end{tabular}
\end{table}

% --- Section 5: Correlated Risk Assessment ---
\section{Correlated Risk Assessment}
This section synthesizes findings from the security control review, the technical scan, and pre-existing risk data to provide a holistic view of the organization's risk profile.

\begin{table}[h!]
\centering
\caption{Summary of Identified Risks}
\begin{tabular}{@{}p{5cm}p{6cm}l@{}}
    \toprule
    \textbf{Risk / Vulnerability} & \textbf{Description} & \textbf{Severity} \\
    \midrule
    \textbf{Lack of MFA for Email and Sensitive Data} & The absence of MFA on critical systems makes user accounts highly vulnerable to takeover via credential stuffing or phishing. & \textcolor{red}{\textbf{Critical}} \\
    \addlinespace
    \textbf{Systemic RDP Exposure} & Open RDP was found on \texttt{10.10.10.51} (new finding) and was previously reported on \texttt{10.10.10.50}. This indicates a pattern of insecure configuration and is a primary target for ransomware actors. & \textcolor{red}{\textbf{Critical}} \\
    \addlinespace
    \textbf{Insufficient Security Awareness Training} & Without annual training, employees' ability to recognize and report modern threats like phishing and social engineering degrades over time. & \textcolor{orange}{\textbf{High}} \\
    \bottomrule
\end{tabular}
\end{table}

% --- Section 6: Recommendations ---
\section{Recommendations}
The following actions are recommended to mitigate the identified risks. They are prioritized based on severity and potential impact.

\subsection{Priority 1: Immediate Actions (0-30 Days)}
\begin{enumerate}
    \item \textbf{Remediate RDP Exposure:}
        \begin{itemize}
            \item Immediately close port 3389 on hosts \texttt{10.10.10.51} and \texttt{10.10.10.50} to all traffic, unless there is an explicit and documented business need.
            \item If RDP access is required, it \textbf{must} be placed behind a Virtual Private Network (VPN) that requires Multi-Factor Authentication. Direct exposure is not an acceptable configuration.
        \end{itemize}
    \item \textbf{Enforce Multi-Factor Authentication:}
        \begin{itemize}
            \item Enable and enforce MFA for all user accounts on the email platform (\texttt{PaperPlanePublishing.net}).
            \item Enable and enforce MFA for all systems identified as containing sensitive data.
        \end{itemize}
\end{enumerate}

\subsection{Priority 2: Short-Term Actions (30-90 Days)}
\begin{enumerate}
    \setcounter{enumi}{2} % Continue numbering from previous list
    \item \textbf{Implement Annual Security Training:}
        \begin{itemize}
            \item Procure and schedule a mandatory security awareness training program for all current employees.
            \item Integrate this training into an annual cycle to ensure continuous education on evolving threats.
        \end{itemize}
\end{enumerate}

\subsection{Priority 3: Strategic Actions (90+ Days)}
\begin{enumerate}
    \setcounter{enumi}{3} % Continue numbering
    \item \textbf{Conduct Comprehensive Vulnerability Assessment:}
        \begin{itemize}
            \item The identified systemic RDP issue suggests other misconfigurations may exist. A full internal and external vulnerability assessment should be performed to create a complete risk baseline.
        \end{itemize}
    \item \textbf{Develop a Vulnerability Management Program:}
        \begin{itemize}
            \item Establish a formal process for regularly scanning, identifying, prioritizing, and remediating vulnerabilities across the entire IT environment.
        \end{itemize}
\end{enumerate}

\end{document}
```