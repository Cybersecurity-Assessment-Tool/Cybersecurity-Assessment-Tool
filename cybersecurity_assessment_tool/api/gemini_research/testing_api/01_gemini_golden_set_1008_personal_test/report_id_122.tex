```latex
\documentclass[12pt]{article}

% Preamble: Required Packages
\usepackage[margin=1in]{geometry}
\usepackage{pifont} % For checkmarks and crosses
\usepackage{booktabs} % For professional tables
\usepackage{hyperref} % For clickable links and metadata
\usepackage{url} % For formatting URLs
\usepackage{seqsplit} % For splitting long strings in tt font
\usepackage[utf8]{inputenc}

% Document Metadata
\hypersetup{
    colorlinks=true,
    linkcolor=black,
    filecolor=magenta,      
    urlcolor=blue,
    pdftitle={Cybersecurity Posture Assessment Report},
    pdfauthor={Cybersecurity Analyst},
    pdfsubject={Security Assessment},
    pdfkeywords={Security, Risk, Assessment},
    bookmarks=true
}

% --- Document Start ---
\begin{document}

% --- Title Page ---
\title{
    \vspace{2cm}
    \textbf{Cybersecurity Posture Assessment Report} \\
    \large For: \textbf{True Grit}
    \vspace{1cm}
}
\author{Cybersecurity Analyst}
\date{\today}
\maketitle
\thispagestyle{empty}
\newpage

% --- Table of Contents ---
\tableofcontents
\newpage

% --- Section 1: Executive Summary ---
\section{Executive Summary}

This report provides a comprehensive analysis of the cybersecurity posture for \textbf{True Grit}, based on a synthesis of technical network scans, a security controls questionnaire, and a review of pre-existing risks.

The assessment identified several high-impact risks that require immediate attention. A critical vulnerability was discovered on an internal-facing server (\texttt{10.0.0.15}), which is running an outdated and insecure FTP service (\texttt{vsftpd 2.3.4}) that permits anonymous access. This version is associated with a known remote code execution vulnerability (CVE-2011-2523) and presents a clear and present danger to the internal network.

Furthermore, significant gaps were identified in organizational security policies and controls. The absence of mandatory Multi-Factor Authentication (MFA) for computer logins, the lack of an employee Acceptable Use Policy (AUP), and the failure to conduct annual security awareness training for all staff create substantial vulnerabilities. These procedural weaknesses are compounded by the continued use of Windows 7, an End-of-Life operating system, on company workstations.

This report outlines these findings in detail and provides a prioritized list of actionable recommendations to mitigate the identified risks and strengthen the organization's overall security posture.

% --- Section 2: Organizational Information ---
\section{Organizational Information}

The following information was provided for the assessment.
\begin{itemize}
    \item \textbf{Organization Name:} True Grit
    \item \textbf{Primary Domain:} \texttt{TrueGrit.net}
    \item \textbf{External IP Address:} \texttt{142.37.148.171}
\end{itemize}

% --- Section 3: Security Control Review ---
\section{Security Control Review}

A review of the security controls questionnaire was conducted to evaluate the organization's adherence to fundamental security best practices. The responses are summarized below. Items marked with \ding{55} indicate significant control gaps that increase organizational risk.

\begin{table}[h!]
\centering
\caption{Security Controls Questionnaire Results}
\begin{tabular}{@{}lc@{}}
\toprule
\textbf{Control Question} & \textbf{Response} \\
\midrule
Do you require MFA to access email? & \ding{51} \\
\textbf{Do you require MFA to log into computers?} & \textbf{\ding{55}} \\
Do you require MFA to access sensitive data systems? & \ding{51} \\
\textbf{Does your organization have an employee acceptable use policy?} & \textbf{\ding{55}} \\
Does your organization do security awareness training for new employees? & \ding{51} \\
\textbf{Does your organization do security awareness training for all employees at least once per year?} & \textbf{\ding{55}} \\
\bottomrule
\end{tabular}
\end{table}

\subsection*{Analysis of Control Gaps}
\begin{itemize}
    \item \textbf{No MFA on Computers:} The lack of MFA on workstations is a critical vulnerability. It exposes the organization to significant risk from credential theft, as a compromised password is all an attacker needs to gain access to an employee's computer and potentially the network.
    \item \textbf{No Acceptable Use Policy (AUP):} An AUP is a foundational governance document. Its absence means there are no formally communicated rules for employees regarding the use of company assets, data handling, or security responsibilities, which can lead to unintentional insider threats.
    \item \textbf{No Annual Security Training:} Security is a continuous process. Without regular, annual training, employee awareness of new threats (like phishing and social engineering) diminishes, making them more likely to fall victim to attacks.
\end{itemize}

% --- Section 4: Technical Scan Results ---
\section{Technical Scan Results}

An Nmap scan was performed on the internal network to identify active services and potential vulnerabilities.

\begin{table}[h!]
\centering
\caption{Open Ports and Services on Target: \texttt{10.0.0.15}}
\begin{tabular}{@{}lllll@{}}
\toprule
\textbf{Port} & \textbf{State} & \textbf{Service} & \textbf{Version} & \textbf{Details} \\
\midrule
21/tcp & Open & ftp & vsftpd 2.3.4 & Anonymous FTP login allowed \\
\bottomrule
\end{tabular}
\end{table}

\subsection*{Analysis of Technical Findings}
The scan revealed a single, but critical, finding:
\begin{itemize}
    \item \textbf{Insecure FTP Server:} The target is running \texttt{vsftpd version 2.3.4}. This specific version, released in 2011, contains a critical backdoor vulnerability (\textbf{CVE-2011-2523}) that allows for remote command execution.
    \item \textbf{Anonymous Access:} The server is configured to allow anonymous FTP logins. This enables any user on the network to connect and potentially upload or download files without authentication, creating a perfect staging ground for malware or data exfiltration. This configuration is a severe security risk and should be disabled immediately.
\end{itemize}

% --- Section 5: Consolidated Risk Assessment ---
\section{Consolidated Risk Assessment}

The following table synthesizes findings from the security questionnaire, the technical scan, and pre-existing risk data into a unified view of the primary risks facing the organization.

\begin{table}[h!]
\centering
\caption{Summary of Identified Risks}
\begin{tabular}{@{}p{0.1\linewidth} p{0.4\linewidth} p{0.25\linewidth} p{0.1\linewidth}@{}}
\toprule
\textbf{Risk ID} & \textbf{Risk Description} & \textbf{Source of Finding} & \textbf{Severity} \\
\midrule
RISK-001 & Vulnerable FTP server (\texttt{vsftpd 2.3.4}) with anonymous login enabled on the internal network. & Technical Scan & \textbf{Critical} \\
\addlinespace
RISK-002 & Lack of Multi-Factor Authentication (MFA) for workstation logins. & Questionnaire & High \\
\addlinespace
RISK-003 & Absence of a formal employee Acceptable Use Policy (AUP). & Questionnaire & High \\
\addlinespace
RISK-004 & No mandatory annual security awareness training program for all employees. & Questionnaire & High \\
\addlinespace
RISK-005 & Workstations are running Windows 7, an End-of-Life (EOL) operating system that no longer receives security updates. & Pre-existing Risk Data & High \\
\bottomrule
\end{tabular}
\end{table}

% --- Section 6: Recommendations ---
\section{Recommendations}

Based on the consolidated risk assessment, the following actions are recommended. They are prioritized to address the most critical threats first.

\subsection*{Immediate Priority (Within 7 Days)}
\begin{enumerate}
    \item \textbf{Mitigate FTP Server Vulnerability (RISK-001):}
    \begin{itemize}
        \item Immediately disable anonymous access on the FTP server at \texttt{10.0.0.15}.
        \item If the FTP service is not business-critical, decommission the server.
        \item If required, upgrade \texttt{vsftpd} to the latest stable version or replace it with a secure file transfer protocol like SFTP (SSH File Transfer Protocol).
    \end{itemize}
\end{enumerate}

\subsection*{High Priority (Within 90 Days)}
\begin{enumerate}
    \setcounter{enumi}{1} % Continue numbering from previous list
    \item \textbf{Implement MFA for Workstations (RISK-002):}
    \begin{itemize}
        \item Deploy a mandatory MFA solution for all employee computer and remote access logins to protect against credential theft.
    \end{itemize}
    \item \textbf{Develop and Implement an AUP (RISK-003):}
    \begin{itemize}
        \item Draft a comprehensive Acceptable Use Policy that clearly defines rules for technology use, data handling, and security responsibilities.
        \item Require all employees to read and formally acknowledge the policy.
    \end{itemize}
    \item \textbf{Establish Annual Security Training (RISK-004):}
    \begin{itemize}
        \item Institute a mandatory security awareness training program for all employees, to be completed annually. The training should cover modern threats such as phishing, ransomware, and social engineering.
    \end{itemize}
    \item \textbf{Plan Windows 7 Migration (RISK-005):}
    \begin{itemize}
        \item Immediately begin planning the migration of all remaining Windows 7 workstations to a modern, supported operating system like Windows 10 or 11. This is critical for receiving security patches.
    \end{itemize}
\end{enumerate}

% --- Document End ---
\end{document}
```