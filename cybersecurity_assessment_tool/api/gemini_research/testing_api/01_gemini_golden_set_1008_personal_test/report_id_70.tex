```latex
\documentclass[12pt]{article}

% ----------------------------------------------------------------------
% PREAMBLE
% ----------------------------------------------------------------------
\usepackage[margin=1in]{geometry}
\usepackage{pifont} % For checkmark and x-mark symbols
\usepackage{booktabs} % For professional-looking tables
\usepackage{hyperref} % For clickable links and metadata
\usepackage{url} % For typesetting URLs
\usepackage{seqsplit} % To split long strings in texttt
\usepackage[utf8]{inputenc}

% Hyperref setup
\hypersetup{
    colorlinks=true,
    linkcolor=black,
    urlcolor=blue,
    pdftitle={Cybersecurity Posture Assessment Report},
    pdfauthor={Cybersecurity Analysis Division},
    pdfsubject={Security Assessment},
    pdfkeywords={Cybersecurity, Risk, Assessment, Scan}
}

% Define custom commands for Yes/No symbols
\newcommand{\cmark}{\ding{51}}%
\newcommand{\xmark}{\ding{55}}%

% ----------------------------------------------------------------------
% DOCUMENT START
% ----------------------------------------------------------------------
\begin{document}

\title{Cybersecurity Posture Assessment Report \\ \large For: \textbf{Foresight Strategies}}
\author{Cybersecurity Analysis Division}
\date{\today}
\maketitle

\begin{abstract}
\noindent This report provides a comprehensive cybersecurity posture assessment for \textbf{Foresight Strategies}. The analysis is based on a review of self-reported security controls, an external network vulnerability scan, and a list of pre-existing risks. The assessment identifies key strengths and critical areas for improvement. The primary findings indicate a strong network perimeter but significant gaps in internal security policies and user-level controls. Immediate attention should be directed towards implementing multi-factor authentication for computer access, establishing a formal Acceptable Use Policy, and integrating security training into the employee onboarding process.
\end{abstract}

\tableofcontents
\newpage

% ----------------------------------------------------------------------
% SECTION 1: OVERVIEW
% ----------------------------------------------------------------------
\section{Executive Overview}
This assessment synthesizes data from three sources: an organizational security questionnaire, an external network scan, and a list of known vulnerabilities. The goal is to provide a holistic view of the organization's current security posture and offer actionable recommendations to mitigate identified risks.

\paragraph{Key Findings:}
\begin{itemize}
    \item \textbf{Positive Findings:} The external network scan of the target IP address revealed no open ports. This is a strong indicator of a well-configured firewall and effective network perimeter security, which significantly reduces the external attack surface.
    \item \textbf{Critical Gaps:} The security control review identified several high-risk gaps in administrative and policy-based controls. These include:
    \begin{itemize}
        \item Lack of mandatory Multi-Factor Authentication (MFA) for computer logins.
        \item Absence of a formal employee Acceptable Use Policy (AUP).
        \item No security awareness training for new employees during onboarding.
    \end{itemize}
    \item \textbf{Overall Posture:} While the organization's perimeter defense appears robust, its internal security posture is weakened by policy and access control deficiencies. These gaps expose the organization to significant risks from insider threats, phishing, and credential compromise.
\end{itemize}

% ----------------------------------------------------------------------
% SECTION 2: ORGANIZATIONAL INFORMATION
% ----------------------------------------------------------------------
\section{Organizational Information}
The following details were provided for the assessment.

\begin{tabular}{@{}ll}
    \toprule
    \textbf{Attribute} & \textbf{Value} \\
    \midrule
    Organization Name & \textbf{Foresight Strategies} \\
    Email Domain & \texttt{ForesightStrategies.org} \\
    Website Domain & \texttt{www.ForesightStrategies.org} \\
    External IP Address & \texttt{42.72.92.119} \\
    \bottomrule
\end{tabular}

% ----------------------------------------------------------------------
% SECTION 3: SECURITY CONTROL REVIEW
% ----------------------------------------------------------------------
\section{Security Control Review}
The following table summarizes the organization's responses to a security controls questionnaire. A green checkmark (\cmark) indicates a positive control is in place, while a red 'X' (\xmark) indicates a control gap that introduces risk.

\begin{table}[h!]
\centering
\begin{tabular}{@{}lc@{}}
\toprule
\textbf{Control Question} & \textbf{Response} \\
\midrule
Do you require MFA to access email? & \cmark \\
Do you require MFA to log into computers? & \xmark \\
Do you require MFA to access sensitive data systems? & \cmark \\
Does your organization have an employee acceptable use policy? & \xmark \\
Does your organization do security awareness training for new employees? & \xmark \\
Does your organization do security awareness training for all employees at least once per year? & \cmark \\
\bottomrule
\end{tabular}
\caption{Summary of Self-Reported Security Controls.}
\label{tab:controls}
\end{table}

The identified gaps are considered high-risk and are detailed further in the Risk Assessment section of this report.

% ----------------------------------------------------------------------
% SECTION 4: TECHNICAL SCAN RESULTS
% ----------------------------------------------------------------------
\section{Technical Scan Results}
An external network vulnerability scan was conducted to identify open ports, running services, and potential vulnerabilities visible from the public internet.

\begin{itemize}
    \item \textbf{Target IP Address:} \texttt{[Target IP]}
    \item \textbf{Scan Date:} \today
\end{itemize}

\paragraph{Findings:} The network scan completed successfully but did not identify any open TCP or UDP ports on the target host.

\paragraph{Analysis:} This is a positive security finding. It suggests that a well-configured firewall or network access control list (ACL) is in place, effectively implementing the principle of least privilege by denying all unsolicited inbound traffic. This configuration significantly minimizes the external attack surface and reduces the risk of network-based attacks.

% ----------------------------------------------------------------------
% SECTION 5: RISK ASSESSMENT
% ----------------------------------------------------------------------
\section{Risk Assessment}
This section correlates the findings from the security control review and technical scan. No pre-existing vulnerabilities were provided for this assessment. The following new risks have been identified based on control gaps.

\begin{table}[h!]
\centering
\begin{tabular}{@{}p{0.1\linewidth} p{0.6\linewidth} p{0.15\linewidth}@{}}
\toprule
\textbf{Risk ID} & \textbf{Risk Description} & \textbf{Severity} \\
\midrule
RISK-001 & \textbf{No MFA on Workstations:} The absence of MFA for computer logins creates a significant risk. If an employee's password is stolen or guessed, an attacker can gain direct access to their workstation and potentially the corporate network. & \textbf{High} \\
\addlinespace
RISK-002 & \textbf{Lack of Acceptable Use Policy (AUP):} Without a formal AUP, there are no established rules for how employees should use company technology and data. This can lead to unintentional data exposure, misuse of assets, and a weakened legal standing in case of an insider incident. & \textbf{High} \\
\addlinespace
RISK-003 & \textbf{No Security Training for New Hires:} New employees are a prime target for social engineering and phishing attacks. Failing to provide security training during onboarding leaves them unaware of threats and policies, making them more likely to fall victim to an attack. & \textbf{High} \\
\bottomrule
\end{tabular}
\caption{Summary of Identified Risks.}
\label{tab:risks}
\end{table}

% ----------------------------------------------------------------------
% SECTION 6: RECOMMENDATIONS
% ----------------------------------------------------------------------
\section{Recommendations}
The following actions are recommended to mitigate the identified risks and improve the overall security posture of \textbf{Foresight Strategies}.

\subsection{Remediation for RISK-001: No MFA on Workstations}
\begin{itemize}
    \item \textbf{Action:} Implement and enforce Multi-Factor Authentication (MFA) for all employee and contractor computer/workstation logins (e.g., Windows, macOS).
    \item \textbf{Justification:} This is one of the most effective controls to prevent unauthorized access resulting from compromised credentials. It acts as a critical barrier against attackers who have stolen a user's password.
\end{itemize}

\subsection{Remediation for RISK-002: Lack of Acceptable Use Policy}
\begin{itemize}
    \item \textbf{Action:} Develop, approve, and implement a comprehensive Acceptable Use Policy (AUP). This policy should be distributed to all employees, who must formally acknowledge their understanding and agreement to comply.
    \item \textbf{Justification:} An AUP establishes clear expectations and rules of behavior for the use of company IT assets. It is a foundational governance document that reduces ambiguity and protects the organization.
\end{itemize}

\subsection{Remediation for RISK-003: No Security Training for New Hires}
\begin{itemize}
    \item \textbf{Action:} Integrate mandatory security awareness training into the new employee onboarding process. This training should cover key topics such as phishing identification, password hygiene, data handling, and the new AUP.
    \item \textbf{Justification:} Equipping new employees with security knowledge from day one is crucial. It helps build a security-conscious culture and turns the "human firewall" into a stronger line of defense.
\end{itemize}

% ----------------------------------------------------------------------
% DOCUMENT END
% ----------------------------------------------------------------------
\end{document}
```