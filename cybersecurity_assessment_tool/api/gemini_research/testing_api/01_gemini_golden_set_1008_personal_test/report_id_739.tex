```latex
\documentclass[12pt]{article}

% Required Packages
\usepackage[margin=1in]{geometry}
\usepackage{pifont} % For checkmarks and crosses
\usepackage{booktabs} % For professional tables
\usepackage{hyperref} % For clickable links
\usepackage{url} % For URL formatting
\usepackage{seqsplit} % To split long strings in tt font
\usepackage{graphicx}
\usepackage{xcolor}

% Hyperref Setup
\hypersetup{
    colorlinks=true,
    linkcolor=blue,
    filecolor=magenta,      
    urlcolor=cyan,
    pdftitle={Cybersecurity Posture Report},
    pdfpagemode=FullScreen,
}

% Define custom colors
\definecolor{darkred}{rgb}{0.55, 0.0, 0.0}
\definecolor{darkorange}{rgb}{0.8, 0.33, 0.0}
\definecolor{darkblue}{rgb}{0.0, 0.0, 0.55}

% Document Title
\title{Cybersecurity Posture Report \\ \large For: Foresight Strategies}
\author{Cybersecurity Analysis Division}
\date{\today}

\begin{document}

\maketitle
\thispagestyle{empty}
\newpage

\tableofcontents
\newpage

% --- 1. Executive Summary ---
\section{Executive Summary}
This report provides a comprehensive analysis of the cybersecurity posture for Foresight Strategies. The assessment is based on a correlation of organizational data, a security controls questionnaire, and an external network scan.

The analysis identified several positive security controls, including the enforcement of multi-factor authentication (MFA) for computer and sensitive data access, as well as a robust security awareness training program for all employees.

However, two critical administrative gaps and one medium-risk technical exposure were discovered. The most significant risk is the lack of MFA for email access, which exposes the organization to a high likelihood of account compromise and business email compromise (BEC) attacks. Additionally, the absence of a formal Acceptable Use Policy (AUP) represents a significant governance and compliance gap.

Technically, an exposed Secure Shell (SSH) service was identified on an IPv6 address. While not immediately vulnerable, any public-facing management service presents a risk and should be secured or removed.

This report details these findings and provides actionable recommendations to mitigate the identified risks and strengthen the overall security posture.

% --- 2. Organizational Information ---
\section{Organizational Information}
The following details were provided for the assessment.

\begin{tabular}{@{}ll}
\toprule
\textbf{Attribute} & \textbf{Value} \\
\midrule
Organization Name & Foresight Strategies \\
Email Domain & \texttt{ForesightStrategies.com} \\
Website Domain & \url{www.ForesightStrategies.com} \\
External IP (IPv4) & \texttt{86.214.110.184} \\
Scanned Target (IPv6) & \seqsplit{\texttt{2001:db8::1}} \\
\bottomrule
\end{tabular}

% --- 3. Security Control Review ---
\section{Security Control Review}
A review of the organization's security controls was conducted via a questionnaire. The results below highlight implemented controls and identify significant gaps. A checkmark (\ding{51}) indicates a positive response, while a cross (\ding{55}) indicates a negative response that requires attention.

\begin{table}[h!]
\centering
\begin{tabular}{@{}lc}
\toprule
\textbf{Control Question} & \textbf{Response} \\
\midrule
Do you require MFA to access email? & \textcolor{darkred}{\ding{55}} \\
Do you require MFA to log into computers? & \textcolor{darkgreen}{\ding{51}} \\
Do you require MFA to access sensitive data systems? & \textcolor{darkgreen}{\ding{51}} \\
Does your organization have an employee acceptable use policy? & \textcolor{darkred}{\ding{55}} \\
Does your organization do security awareness training for new employees? & \textcolor{darkgreen}{\ding{51}} \\
Does your organization do security awareness training for all employees at least once per year? & \textcolor{darkgreen}{\ding{51}} \\
\bottomrule
\end{tabular}
\caption{Security Controls Questionnaire Results}
\end{table}

\subsection*{Analysis of Controls}
The questionnaire reveals two primary areas of concern:
\begin{itemize}
    \item \textbf{Email Security:} The lack of MFA on email is a critical vulnerability. Email accounts are a primary target for attackers seeking to conduct phishing, social engineering, and business email compromise (BEC) attacks.
    \item \textbf{Governance:} The absence of an Acceptable Use Policy (AUP) creates ambiguity regarding the proper use of company assets. This can lead to unintentional misuse of systems and data, and complicates enforcement actions in the event of a policy violation.
\end{itemize}

% --- 4. Technical Scan Results ---
\section{Technical Scan Results}
An external network scan was performed on the specified target IP address to identify accessible services.

\subsection*{Scan Details}
\begin{itemize}
    \item \textbf{Target IP Address:} \seqsplit{\texttt{2001:db8::1}}
    \item \textbf{Scan Status:} Host was responsive (up).
\end{itemize}

\subsection*{Open Ports}
The following table details the open ports discovered on the target system.

\begin{table}[h!]
\centering
\begin{tabular}{@{}llll@{}}
\toprule
\textbf{Port} & \textbf{State} & \textbf{Service (Presumed)} & \textbf{Notes} \\
\midrule
22/TCP & open & SSH & Service version not enumerated. \\
\bottomrule
\end{tabular}
\caption{Discovered Open Ports}
\end{table}

\subsection*{Technical Findings}
The scan identified an open SSH port (22) on an IPv6 address. Exposing management services like SSH directly to the internet increases the attack surface of the organization. This service is a common target for automated brute-force attacks. The security of the organization's IPv6 infrastructure should be reviewed to ensure it is as robust as its IPv4 counterpart.

% --- 5. Risk Assessment ---
\section{Risk Assessment}
This section synthesizes the findings from the security control review and the technical scan into a prioritized list of risks.

\begin{table}[h!]
\centering
\begin{tabular}{@{}lp{6cm}l@{}}
\toprule
\textbf{ID} & \textbf{Risk / Vulnerability} & \textbf{Severity} \\
\midrule
RISK-001 & Lack of Multi-Factor Authentication (MFA) on Email & \textcolor{darkred}{\textbf{Critical}} \\
\textit{Overview} & \multicolumn{2}{p{12cm}}{\small Without MFA, a compromised password is all an attacker needs to gain full access to an employee's mailbox, leading to data breaches, financial fraud, and further internal compromise.} \\
\addlinespace[0.5em]
RISK-002 & Missing Employee Acceptable Use Policy (AUP) & \textcolor{darkorange}{\textbf{High}} \\
\textit{Overview} & \multicolumn{2}{p{12cm}}{\small The absence of a formal AUP creates legal and operational risks. It fails to establish clear guidelines for employees on the secure and appropriate use of corporate IT resources.} \\
\addlinespace[0.5em]
RISK-003 & Exposed SSH Management Service & \textbf{Medium} \\
\textit{Overview} & \multicolumn{2}{p{12cm}}{\small The SSH service on \seqsplit{\texttt{2001:db8::1}} is exposed to the public internet. This allows attackers to perform password guessing attacks and exploit potential vulnerabilities in the SSH software.} \\
\bottomrule
\end{tabular}
\caption{Summary of Identified Risks}
\end{table}

% --- 6. Recommendations ---
\section{Recommendations}
The following actions are recommended to mitigate the identified risks and improve the overall security posture of Foresight Strategies.

\subsection*{RISK-001: Implement MFA on Email (Critical)}
\begin{itemize}
    \item \textbf{Immediate Action:} Enable MFA for all user email accounts without exception. This is the single most effective control to prevent unauthorized account access.
    \item \textbf{Strategic Action:} Prioritize modern authentication methods (e.g., FIDO2/WebAuthn, authenticator apps) over less secure methods like SMS.
    \item \textbf{User Training:} Communicate the change to all employees and provide clear instructions on how to enroll in and use MFA.
\end{itemize}

\subsection*{RISK-002: Develop an Acceptable Use Policy (High)}
\begin{itemize}
    \item \textbf{Immediate Action:} Draft a formal Acceptable Use Policy (AUP) that covers topics such as data handling, internet usage, personal device usage (BYOD), and security responsibilities.
    \item \textbf{Implementation:} Require all current employees to read and formally acknowledge the policy. Incorporate AUP review and acknowledgement into the new employee onboarding process.
    \item \textbf{Review Cycle:} Establish a process to review and update the AUP at least annually or when significant organizational changes occur.
\end{itemize}

\subsection*{RISK-003: Secure the Exposed SSH Service (Medium)}
\begin{itemize}
    \item \textbf{Review Business Need:} Determine if public access to the SSH service on \seqsplit{\texttt{2001:db8::1}} is necessary. If not, block access at the network firewall immediately.
    \item \textbf{Implement Compensating Controls:} If public access is required, implement the following hardening measures:
    \begin{itemize}
        \item Restrict access to specific, trusted source IP addresses (IP whitelisting).
        \item Disable password-based authentication and enforce the use of public key cryptography.
        \item Disable root user login over SSH.
        \item Ensure the SSH server software is patched to the latest stable version to protect against known vulnerabilities.
    \end{itemize}
\end{itemize}

\end{document}
```