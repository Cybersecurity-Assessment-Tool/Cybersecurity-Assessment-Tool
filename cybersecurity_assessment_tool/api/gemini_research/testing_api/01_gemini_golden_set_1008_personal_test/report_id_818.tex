```latex
\documentclass[12pt]{article}

% Preamble: Required Packages
\usepackage[margin=1in]{geometry}
\usepackage{pifont} % For checkmarks and crosses (\ding)
\usepackage{booktabs} % For professional-looking tables
\usepackage{hyperref} % For clickable links and references
\usepackage{url} % For formatting URLs
\usepackage{seqsplit} % To split long strings in \texttt
\usepackage{xcolor} % For colors

% Hyperref and Color Setup
\hypersetup{
    colorlinks=true,
    linkcolor=black,
    filecolor=magenta,
    urlcolor=blue,
    pdftitle={Cybersecurity Posture Assessment Report},
    pdfpagemode=FullScreen,
}

\newcommand{\yes}{\ding{51}}
\newcommand{\no}{\ding{55}}

\begin{document}

% --- Title Page ---
\title{Cybersecurity Posture Assessment Report \\ \large For: Clear Path}
\author{Cybersecurity Analysis Division}
\date{\today}
\maketitle

\newpage

% --- Table of Contents ---
\tableofcontents
\newpage

% --- Section 1: Executive Summary ---
\section*{1. Executive Summary}

This report details the findings of a cybersecurity posture assessment conducted for Clear Path. The assessment combined an automated network scan, a review of existing documented risks, and an analysis of organizational security controls via a questionnaire.

The overall security posture is assessed as \textbf{High-Risk}. Several critical vulnerabilities and security gaps were identified that require immediate attention. Key findings include:

\begin{itemize}
    \item \textbf{Critical Service Vulnerability:} An internal server (\texttt{10.0.0.15}) is running a version of \texttt{vsftpd} (2.3.4) with a known, severe remote code execution vulnerability. This service is also misconfigured to allow anonymous FTP access.
    \item \textbf{Critical Authentication Gaps:} Multi-Factor Authentication (MFA) is not enforced for accessing email or other sensitive data systems, leaving these critical assets vulnerable to credential-based attacks.
    \item \textbf{Procedural Deficiencies:} New employees do not receive security awareness training, creating a significant gap in the organization's human firewall from day one.
    \item \textbf{Pre-existing Risks:} An existing risk concerning outdated Windows 7 workstations remains a medium-level threat that could be exacerbated by the new findings.
\end{itemize}

Immediate remediation of the vulnerable FTP server and implementation of MFA are paramount to reducing the organization's risk of a significant security breach.

% --- Section 2: Organizational Information ---
\section*{2. Organizational Information}
This section provides the organizational details as provided for this assessment.

\begin{tabular}{@{}ll}
\toprule
\textbf{Attribute} & \textbf{Value} \\
\midrule
Organization Name & Clear Path \\
Email Domain & \texttt{ClearPath.com} \\
Website Domain & \url{www.ClearPath.com} \\
External IP Address & \texttt{46.20.6.2} \\
\bottomrule
\end{tabular}

% --- Section 3: Security Control Review ---
\section*{3. Security Control Review}
The following table summarizes the organization's responses to the security controls questionnaire. Items marked with \no\ represent significant gaps in the security framework and are discussed in the Risk Assessment section.

\begin{table}[h!]
\centering
\begin{tabular}{p{0.6\textwidth} c l}
\toprule
\textbf{Control Question} & \textbf{Response} & \textbf{Assessment} \\
\midrule
Do you require MFA to access email? & \no & \textcolor{red}{\textbf{Critical Gap}} \\
Do you require MFA to log into computers? & \yes & Good Practice \\
Do you require MFA to access sensitive data systems? & \no & \textcolor{red}{\textbf{Critical Gap}} \\
Does your organization have an employee acceptable use policy? & \yes & Good Practice \\
Does your organization do security awareness training for new employees? & \no & \textcolor{orange}{\textbf{High Risk}} \\
Does your organization do security awareness training for all employees at least once per year? & \yes & Good Practice \\
\bottomrule
\end{tabular}
\caption{Security Controls Questionnaire Analysis}
\end{table}

% --- Section 4: Technical Scan Results ---
\section*{4. Technical Scan Results}
An Nmap scan was performed on the internal network target \texttt{10.0.0.15}. The scan identified one open port with a critically vulnerable service.

\begin{table}[h!]
\centering
\begin{tabular}{l l l l}
\toprule
\textbf{Port/Proto} & \textbf{State} & \textbf{Service/Version} & \textbf{Analyst Notes} \\
\midrule
21/tcp & Open & FTP / vsftpd 2.3.4 & \parbox[t]{0.5\textwidth}{
    \textbf{1. Critical Vulnerability:} Version 2.3.4 is vulnerable to a backdoor command execution flaw (CVE-2011-2523). An attacker can gain a root shell. \\
    \textbf{2. Insecure Configuration:} Anonymous FTP login is allowed, permitting unauthenticated access to the file system.
} \\
\bottomrule
\end{tabular}
\caption{Open Port Analysis for Target \texttt{10.0.0.15}}
\end{table}

% --- Section 5: Risk Assessment Summary ---
\section*{5. Risk Assessment Summary}
This section correlates findings from the security control review, technical scan, and pre-existing risk documentation into a unified risk register.

\begin{table}[h!]
\centering
\begin{tabular}{p{0.15\textwidth} p{0.55\textwidth} l}
\toprule
\textbf{Risk Title} & \textbf{Overview} & \textbf{Severity} \\
\midrule
\textbf{Vulnerable FTP Service} & The FTP server (\texttt{10.0.0.15}) is running vsftpd 2.3.4, which is susceptible to a well-known remote code execution vulnerability. & \textcolor{red}{\textbf{Critical}} \\
\addlinespace
\textbf{MFA Not Enforced on Critical Systems} & Email and sensitive data systems lack MFA protection, making them highly susceptible to compromise via stolen or weak credentials. & \textcolor{red}{\textbf{Critical}} \\
\addlinespace
\textbf{Anonymous FTP Access} & The vulnerable FTP server is configured to allow anonymous logins, significantly lowering the barrier for an attacker to access and exploit the system. & \textcolor{orange}{\textbf{High}} \\
\addlinespace
\textbf{No Onboarding Security Training} & New employees are not trained on security best practices, acceptable use, or threat identification, making them easy targets for social engineering. & \textcolor{orange}{\textbf{High}} \\
\addlinespace
\textbf{Outdated Windows Policy} & (Pre-existing) Workstations are running Windows 7, an unsupported OS that no longer receives security updates. This increases the attack surface. & \textcolor{yellow!80!black}{\textbf{Medium}} \\
\bottomrule
\end{tabular}
\caption{Consolidated Risk Register}
\end{table}

% --- Section 6: Recommendations ---
\section*{6. Recommendations}
The following actions are recommended to mitigate the identified risks. They are prioritized based on severity and potential impact.

\subsection*{Immediate Actions (0-7 Days)}
\begin{enumerate}
    \item \textbf{Remediate Vulnerable FTP Server:}
    \begin{itemize}
        \item Immediately take the FTP server at \texttt{10.0.0.15} offline.
        \item If the service is business-critical, upgrade \texttt{vsftpd} to the latest stable version.
        \item Disable anonymous FTP login.
        \item Conduct a forensic analysis to determine if the vulnerability has already been exploited.
    \end{itemize}
    \item \textbf{Enforce Multi-Factor Authentication (MFA):}
    \begin{itemize}
        \item Immediately enable and enforce MFA for all users on the email system (\texttt{ClearPath.com}).
        \item Immediately enable and enforce MFA for all users on systems identified as containing sensitive data.
    \end{itemize}
\end{enumerate}

\subsection*{Short-Term Actions (1-3 Months)}
\begin{enumerate}
    \setcounter{enumi}{2} % Continue numbering from previous list
    \item \textbf{Implement Onboarding Security Training:}
    \begin{itemize}
        \item Develop a mandatory security awareness training module for all new hires.
        \item The training should cover the acceptable use policy, phishing identification, password hygiene, and incident reporting procedures.
    \end{itemize}
    \item \textbf{Replace Insecure Protocols:}
    \begin{itemize}
        \item Create a plan to decommission FTP in favor of a secure alternative, such as SFTP (SSH File Transfer Protocol) or a managed file transfer solution.
    \end{itemize}
\end{enumerate}

\subsection*{Ongoing / Long-Term Actions}
\begin{enumerate}
    \setcounter{enumi}{4}
    \item \textbf{Upgrade Outdated Operating Systems:}
    \begin{itemize}
        \item Continue and accelerate the project to upgrade all Windows 7 workstations to a supported operating system (e.g., Windows 10/11).
    \end{itemize}
\end{enumerate}

\end{document}
```