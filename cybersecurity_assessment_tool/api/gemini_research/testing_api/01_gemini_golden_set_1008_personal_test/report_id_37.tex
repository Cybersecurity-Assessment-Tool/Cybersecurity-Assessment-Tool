```latex
\documentclass[12pt]{article}

% === PACKAGES ===
\usepackage[margin=1in]{geometry}
\usepackage{pifont} % For \ding
\usepackage{booktabs} % For professional tables (\toprule, \midrule, \bottomrule)
\usepackage{hyperref} % For clickable links and metadata
\usepackage{url}
\usepackage{seqsplit} % To split long strings in \texttt
\usepackage{graphicx}
\usepackage{fancyhdr}
\usepackage{lastpage}
\usepackage{xcolor}

% === DOCUMENT SETUP ===
\hypersetup{
    colorlinks=true,
    linkcolor=blue,
    urlcolor=cyan,
    pdftitle={Cybersecurity Posture Report},
    pdfauthor={Cybersecurity Analysis Division},
    pdfsubject={Security Assessment for Hidden Gem}
}

% Define colors for severity
\definecolor{criticalred}{HTML}{D7263D}
\definecolor{highorange}{HTML}{F49D4E}
\definecolor{mediumyellow}{HTML}{F3E96B}

% === HEADER & FOOTER ===
\pagestyle{fancy}
\fancyhf{} % Clear all header and footer fields
\lhead{\textbf{Cybersecurity Posture Report}}
\rhead{\textbf{Hidden Gem}}
\cfoot{Page \thepage\ of \pageref{LastPage}}
\renewcommand{\headrulewidth}{0.4pt}
\renewcommand{\footrulewidth}{0.4pt}

% ==============================================================================
% === DOCUMENT START ===
% ==============================================================================
\begin{document}

\title{
    \vspace{1cm}
    \textbf{Cybersecurity Posture Report} \\
    \large \vspace{0.5cm}
    Prepared for: \textbf{Hidden Gem}
}
\author{Cybersecurity Analysis Division}
\date{\today}
\maketitle
\thispagestyle{fancy}

\newpage

\tableofcontents

\newpage

% ==============================================================================
\section{Executive Overview}
% ==============================================================================

This report provides a comprehensive analysis of the cybersecurity posture for \textbf{Hidden Gem}, based on a review of organizational security controls, a technical network scan, and pre-existing risk data. The assessment was conducted on \textbf{2025-11-22}.

The analysis revealed several high-priority risks requiring immediate attention. A critical gap was identified in endpoint security: the absence of Multi-Factor Authentication (MFA) for computer logins, which significantly increases the risk of unauthorized access via compromised credentials.

Furthermore, the external network scan discovered a public-facing web server running a severely outdated version of \textbf{Nginx (1.18.0)}. This version is susceptible to numerous publicly known vulnerabilities, posing a high risk of system compromise. A mismatch between the organization's domain and the server's SSL certificate was also noted, which can erode user trust and may indicate a configuration error.

Finally, organizational process gaps, such as the lack of annual security awareness training for all employees, compound these technical risks by increasing susceptibility to social engineering attacks like phishing.

We strongly recommend prioritizing the remediation actions outlined in Section 6 to mitigate these identified risks and strengthen the overall security posture.

% ==============================================================================
\section{Organizational Information}
% ==============================================================================

The following information was provided by the organization and serves as the baseline for this assessment.

\begin{table}[h!]
\centering
\caption{Client Organizational Data}
\begin{tabular}{@{}ll@{}}
\toprule
\textbf{Attribute} & \textbf{Value} \\
\midrule
Organization Name & \textbf{Hidden Gem} \\
Email Domain & \texttt{HiddenGem.com} \\
Website Domain & \url{www.HiddenGem.com} \\
External IP Address & \texttt{117.202.173.93} \\
\bottomrule
\end{tabular}
\end{table}

% ==============================================================================
\section{Security Control Review}
% ==============================================================================

The following table summarizes the organization's responses to a security controls questionnaire. Items marked with \textcolor{red}{\ding{55}} represent significant gaps in the security framework and are correlated with findings in the Risk Assessment section.

\begin{table}[h!]
\centering
\caption{Security Controls Questionnaire Analysis}
\begin{tabular}{@{}p{0.65\textwidth} c l@{}}
\toprule
\textbf{Control Question} & \textbf{Response} & \textbf{Assessment} \\
\midrule
Do you require MFA to access email? & \textcolor{green}{\ding{51}} & Best practice met. \\
Do you require MFA to log into computers? & \textcolor{red}{\ding{55}} & \textbf{Critical Gap.} \\
Do you require MFA to access sensitive data systems? & \textcolor{green}{\ding{51}} & Best practice met. \\
Does your organization have an employee acceptable use policy? & \textcolor{green}{\ding{51}} & Foundational policy in place. \\
Does your organization do security awareness training for new employees? & \textcolor{green}{\ding{51}} & Good onboarding practice. \\
Does your organization do security awareness training for all employees at least once per year? & \textcolor{red}{\ding{55}} & \textbf{High Risk.} \\
\bottomrule
\end{tabular}
\end{table}

% ==============================================================================
\section{Technical Scan Results}
% ==============================================================================

An external network scan was performed to identify open ports and exposed services.

\begin{itemize}
    \item \textbf{Target IP:} \texttt{192.168.10.5}
    \item \textbf{Scan Date:} 2025-11-22
\end{itemize}

The scan identified the following open port:

\begin{table}[h!]
\centering
\caption{Open Port Analysis}
\begin{tabular}{@{}llllll@{}}
\toprule
\textbf{Port} & \textbf{State} & \textbf{Service} & \textbf{Product} & \textbf{Version} & \textbf{Notes} \\
\midrule
443/tcp & Open & https & nginx & 1.18.0 & \parbox[t]{4cm}{\textbf{Outdated Version.} SSL cert common name is \texttt{www.acme-corp.com}, a mismatch with the client's domain.} \\
\bottomrule
\end{tabular}
\end{table}

% ==============================================================================
\section{Risk Assessment Summary}
% ==============================================================================

This section synthesizes findings from the security control review and technical scans into a prioritized list of risks. No pre-existing vulnerabilities were reported.

\begin{table}[h!]
\centering
\caption{Identified Risks}
\begin{tabular}{@{}lp{0.5\textwidth}ll@{}}
\toprule
\textbf{Risk ID} & \textbf{Description} & \textbf{Severity} & \textbf{Source} \\
\midrule
R-01 & Lack of mandatory MFA on all employee computers allows for trivial account takeover if credentials are stolen. & \textcolor{criticalred}{\textbf{Critical}} & Questionnaire \\
\addlinespace
R-02 & The public-facing web server runs an outdated Nginx version (1.18.0) with multiple known vulnerabilities, potentially leading to remote code execution. & \textcolor{highorange}{\textbf{High}} & Network Scan \\
\addlinespace
R-03 & Absence of annual security awareness training for all staff increases susceptibility to phishing and other social engineering attacks. & \textcolor{highorange}{\textbf{High}} & Questionnaire \\
\addlinespace
R-04 & The SSL certificate on the web server does not match the organization's domain, which erodes user trust and could be exploited in phishing attacks. & \textcolor{mediumyellow}{\textbf{Medium}} & Network Scan \\
\bottomrule
\end{tabular}
\end{table}

% ==============================================================================
\section{Recommendations}
% ==============================================================================

The following actions are recommended to mitigate the identified risks. Recommendations are prioritized by severity.

\paragraph{Priority 1 (Critical): Remediate R-01 - Implement Endpoint MFA}
Deploy a mandatory Multi-Factor Authentication (MFA) solution for all employee computer and remote access logins. This is the single most effective control to prevent unauthorized access resulting from compromised passwords.
\begin{itemize}
    \item \textbf{Action:} Enforce MFA for all operating system logins (Windows, macOS) and VPN connections.
    \item \textbf{Impact:} Significantly reduces the risk of unauthorized access.
\end{itemize}

\paragraph{Priority 2 (High): Remediate R-02 - Upgrade Nginx Web Server}
The Nginx server at \texttt{192.168.10.5} must be upgraded from version 1.18.0 to the latest stable release. Version 1.18.0 is end-of-life and contains numerous security flaws that have since been patched.
\begin{itemize}
    \item \textbf{Action:} Plan and execute an upgrade of Nginx to a current, supported version.
    \item \textbf{Impact:} Mitigates risk of server compromise from known exploits.
\end{itemize}

\paragraph{Priority 3 (High): Remediate R-03 - Establish Annual Security Training}
Implement a mandatory, annual security awareness training program for all employees. Effective training reduces the likelihood of successful phishing and social engineering attacks.
\begin{itemize}
    \item \textbf{Action:} Procure and deploy a training module covering topics such as phishing, password hygiene, and acceptable use. Track completion for all staff.
    \item \textbf{Impact:} Strengthens the human element of the security program.
\end{itemize}

\paragraph{Priority 4 (Medium): Remediate R-04 - Correct SSL Certificate Mismatch}
Replace the incorrect SSL certificate on the web server at \texttt{192.168.10.5}. The current certificate for \texttt{www.acme-corp.com} should be replaced with one that correctly identifies \texttt{www.HiddenGem.com}.
\begin{itemize}
    \item \textbf{Action:} Provision and install a valid SSL/TLS certificate for the correct domain.
    \item \textbf{Impact:} Ensures user trust and secure, encrypted communications.
\end{itemize}

\end{document}
```