```latex
\documentclass[12pt]{article}

% Preamble: Required Packages
\usepackage[margin=1in]{geometry}
\usepackage{pifont} % For checkmarks and crosses
\usepackage{booktabs} % For professional tables
\usepackage{hyperref} % For clickable links
\usepackage{url} % For formatting URLs
\usepackage{seqsplit} % For splitting long strings in tt font
\usepackage{graphicx}
\usepackage{xcolor}

% Document Information
\title{Cybersecurity Posture Assessment Report \\ \large For: Open Door}
\author{Cybersecurity Analyst Group}
\date{\today}

% Hyperref Setup
\hypersetup{
    colorlinks=true,
    linkcolor=blue,
    filecolor=magenta,      
    urlcolor=cyan,
    pdftitle={Cybersecurity Posture Assessment Report},
    pdfpagemode=FullScreen,
}

\begin{document}

\maketitle
\thispagestyle{empty}
\newpage

\tableofcontents
\newpage

% --- 1. Executive Overview ---
\section*{1. Executive Overview}

This report details the findings of a cybersecurity posture assessment for \textbf{Open Door}. The evaluation was conducted by synthesizing data from three key sources: a review of organizational security controls, an external network vulnerability scan, and an analysis of pre-existing risk registers.

The assessment reveals a mixed security posture. \textbf{Positive findings} include the robust implementation of Multi-Factor Authentication (MFA) across critical systems and a strong network perimeter, as evidenced by a technical scan that found no open external ports. These controls significantly reduce the risk of unauthorized access and external attacks.

However, the assessment identified \textbf{two critical gaps} in administrative and procedural controls. The absence of a formal Employee Acceptable Use Policy and the lack of mandatory, annual security awareness training for all staff represent a high level of risk. These gaps expose the organization to significant threats stemming from human error, such as falling victim to phishing attacks, mishandling sensitive data, or misusing company assets.

Immediate action is recommended to address these policy and training deficiencies to build a more resilient and comprehensive security program.

% --- 2. Organizational Information ---
\section*{2. Organizational Information}

The following details were provided for the assessment.

\begin{tabular}{@{}ll}
\toprule
\textbf{Attribute} & \textbf{Value} \\
\midrule
Organization Name & \textbf{Open Door} \\
Email Domain      & \texttt{OpenDoor.org} \\
Website Domain    & \url{www.OpenDoor.org} \\
External IP Address & \texttt{15.9.229.31} \\
\bottomrule
\end{tabular}

% --- 3. Security Control Review ---
\section*{3. Security Control Review}

A questionnaire was used to evaluate the implementation of key administrative and technical security controls. The responses indicate a strong focus on authentication security but highlight significant weaknesses in policy and employee training. "No" answers represent critical control gaps that require immediate attention.

\begin{tabular}{@{}p{0.7\linewidth}c@{}}
\toprule
\textbf{Control Question} & \textbf{Response} \\
\midrule
Do you require MFA to access email? & \ding{51} \\
Do you require MFA to log into computers? & \ding{51} \\
Do you require MFA to access sensitive data systems? & \ding{51} \\
Does your organization have an employee acceptable use policy? & \textcolor{red}{\ding{55}} \\
Does your organization do security awareness training for new employees? & \ding{51} \\
Does your organization do security awareness training for all employees at least once per year? & \textcolor{red}{\ding{55}} \\
\bottomrule
\end{tabular}

% --- 4. Technical Scan Results ---
\section*{4. Technical Scan Results}

An external network scan was performed to identify open ports and exposed services on the organization's perimeter.

\begin{itemize}
    \item \textbf{Scan Target:} \texttt{[Target IP]}
    \item \textbf{Scan Date:} \today
\end{itemize}

\subsection*{Findings}
The scan completed successfully and found \textbf{no open ports or exposed services}. This is a positive security finding, indicating that the network firewall and perimeter security configurations are effectively preventing unauthorized external access. No further action is required based on these technical results.

% --- 5. Risk Assessment ---
\section*{5. Risk Assessment}

This section correlates the findings from the security control review and technical scan. The primary risks identified are administrative in nature and stem from gaps in policy and training. No pre-existing vulnerabilities were reported.

\begin{tabular}{@{}p{0.1\linewidth}p{0.2\linewidth}p{0.5\linewidth}p{0.1\linewidth}@{}}
\toprule
\textbf{Risk ID} & \textbf{Risk Name} & \textbf{Description} & \textbf{Severity} \\
\midrule
RISK-001 & Lack of Acceptable Use Policy (AUP) & The absence of a formal AUP creates ambiguity regarding the proper use of company assets and data. This increases the risk of insider threat, data leakage, and legal liability. & \textbf{High} \\
\addlinespace
RISK-002 & No Annual Security Awareness Training & Without recurring training, employees' ability to recognize and respond to modern threats like phishing and social engineering degrades over time, making the organization highly susceptible to human-targeted attacks. & \textbf{High} \\
\bottomrule
\end{tabular}

% --- 6. Recommendations ---
\section*{6. Recommendations}

The following actions are recommended to mitigate the identified risks and improve the overall security posture of \textbf{Open Door}. These recommendations are prioritized as critical due to their impact on reducing human-related security risks.

\subsection*{Recommendation for RISK-001: Develop and Implement an Acceptable Use Policy}
\begin{itemize}
    \item \textbf{Priority:} Critical
    \item \textbf{Action:} Draft a comprehensive Acceptable Use Policy (AUP) that clearly defines the rules and responsibilities for all employees when using company networks, computers, software, and data. The policy should cover topics such as data privacy, internet usage, password security, and incident reporting. It must be formally approved, distributed to all current and new employees, and require a signed acknowledgment of review and acceptance.
\end{itemize}

\subsection*{Recommendation for RISK-002: Establish a Recurring Security Awareness Training Program}
\begin{itemize}
    \item \textbf{Priority:} Critical
    \item \textbf{Action:} Implement a mandatory, annual security awareness training program for all employees. This program should be updated regularly to address the latest threat landscape, including sophisticated phishing tactics, ransomware, and social engineering. Consider incorporating phishing simulation exercises to test and reinforce the training concepts. Completion of training should be tracked for compliance purposes.
\end{itemize}

\end{document}
```