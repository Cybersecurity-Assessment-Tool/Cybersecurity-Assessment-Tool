```latex
\documentclass[12pt]{article}

% ----------------------------------------------------------------------
% PREAMBLE
% ----------------------------------------------------------------------
\usepackage[margin=1in]{geometry}
\usepackage{pifont} % For checkmarks and crosses
\usepackage{booktabs} % For professional tables
\usepackage{hyperref} % For clickable links and references
\usepackage{url} % For formatting URLs
\usepackage{seqsplit} % To split long monospaced strings
\usepackage{datetime} % For report date

\hypersetup{
    colorlinks=true,
    linkcolor=black,
    urlcolor=blue,
    pdftitle={Cybersecurity Posture Assessment Report},
    pdfauthor={Cybersecurity Analyst},
    pdfsubject={Security Assessment}
}

\newcommand{\yes}{\ding{51}} % Green checkmark
\newcommand{\no}{\ding{55}}  % Red X

% ----------------------------------------------------------------------
% DOCUMENT START
% ----------------------------------------------------------------------
\begin{document}

% ----------------------------------------------------------------------
% TITLE PAGE
% ----------------------------------------------------------------------
\begin{titlepage}
    \centering
    \vspace*{\fill}
    \Huge\textbf{Cybersecurity Posture Assessment Report}
    \vspace{1.5cm}
    \Large\textbf{Prepared for: Midnight Oil Studios}
    \vspace{2cm}
    \large
    \begin{tabular}{ll}
        \textbf{Report Date:} & \today \\
        \textbf{Analysis Period:} & \today \\
        \textbf{Classification:} & Confidential \\
    \end{tabular}
    \vspace*{\fill}
    \vfill
    \textit{This report contains sensitive information regarding the security posture of the organization. Distribution should be limited to authorized personnel only.}
\end{titlepage}

\tableofcontents
\newpage

% ----------------------------------------------------------------------
% SECTION 1: EXECUTIVE SUMMARY
% ----------------------------------------------------------------------
\section{Executive Summary}
This report provides a cybersecurity posture assessment for \textbf{Midnight Oil Studios}, based on an analysis of organizational security controls, technical scan data, and pre-existing risk information. The assessment identified several critical gaps that significantly increase the organization's risk of a security incident.

\textbf{Key Findings:}
\begin{itemize}
    \item \textbf{Critical Risk - Lack of Email MFA:} The absence of mandatory Multi-Factor Authentication (MFA) for email access is a critical vulnerability. This exposes the organization to a high risk of Business Email Compromise (BEC), phishing attacks, and data breaches originating from compromised accounts.
    \item \textbf{High Risk - No Security Awareness Training:} The organization does not conduct security awareness training for new or existing employees. This lack of training makes staff highly susceptible to social engineering and phishing attacks, turning the human element into the weakest link in the security chain.
    \item \textbf{Data Inconclusive - Corrupted Inputs:} The provided technical network scan data (\texttt{Input\_1\_Network\_Scan\_JSON}) and the list of current risks (\texttt{Input\_3\_Current\_Risks\_JSON}) were corrupted and could not be processed. This creates a significant blind spot regarding external-facing vulnerabilities and previously tracked issues.
\end{itemize}

The overall security posture is weakened by fundamental control gaps. While some controls like MFA for computer and sensitive system access are in place, the vulnerabilities in email security and employee awareness present clear and immediate threats. Immediate remediation of the identified critical and high-risk findings is strongly recommended to reduce the likelihood of a successful cyberattack. A comprehensive rescan of the network perimeter is also essential to gain visibility into technical vulnerabilities.

% ----------------------------------------------------------------------
% SECTION 2: ORGANIZATIONAL INFORMATION
% ----------------------------------------------------------------------
\section{Organizational Information}
The following details were provided for the assessment.

\begin{tabular}{@{}ll}
    \toprule
    \textbf{Attribute} & \textbf{Value} \\
    \midrule
    Organization Name & \textbf{Midnight Oil Studios} \\
    Email Domain & \seqsplit{\texttt{MidnightOilStudios.org}} \\
    Website Domain & \seqsplit{\url{www.MidnightOilStudios.org}} \\
    External IP Address & \texttt{203.67.47.66} \\
    \bottomrule
\end{tabular}

% ----------------------------------------------------------------------
% SECTION 3: SECURITY CONTROL REVIEW
% ----------------------------------------------------------------------
\section{Security Control Review (Questionnaire Analysis)}
A review of the organization's self-reported security controls was conducted. The following table summarizes the responses and provides an initial assessment of each control's status. "No" answers indicate significant gaps in the security framework.

\begin{table}[h!]
\centering
\caption{Security Controls Questionnaire Results}
\begin{tabular}{@{}p{0.6\linewidth} c l@{}}
    \toprule
    \textbf{Control Question} & \textbf{Response} & \textbf{Assessment} \\
    \midrule
    Do you require MFA to access email? & \no & \textbf{Critical Gap} \\
    Do you require MFA to log into computers? & \yes & Best Practice Met \\
    Do you require MFA to access sensitive data systems? & \yes & Best Practice Met \\
    Does your organization have an employee acceptable use policy? & \yes & Best Practice Met \\
    Does your organization do security awareness training for new employees? & \no & \textbf{High Risk} \\
    Does your organization do security awareness training for all employees at least once per year? & \no & \textbf{High Risk} \\
    \bottomrule
\end{tabular}
\end{table}

% ----------------------------------------------------------------------
% SECTION 4: TECHNICAL SCAN RESULTS
% ----------------------------------------------------------------------
\section{Technical Scan Results}
The technical network scan data provided for this assessment (\texttt{Input\_1\_Network\_Scan\_JSON}) was found to be corrupted and could not be parsed. 

\textbf{Target IP Address:} \texttt{203.67.47.66}

\textbf{Status: Scan Data Unavailable}

\textbf{Impact:}
Due to the corrupted data, it was not possible to analyze the external network perimeter of \textbf{Midnight Oil Studios}. This analysis would typically identify open ports, exposed services, and potential software vulnerabilities that could be exploited by an external attacker. Without this information, there is no visibility into the technical attack surface of the organization's public-facing infrastructure.

A complete and accurate network scan is a foundational component of any security assessment. It is strongly recommended that a new scan be commissioned immediately.

% ----------------------------------------------------------------------
% SECTION 5: RISK ASSESSMENT
% ----------------------------------------------------------------------
\section{Risk Assessment}
This section synthesizes findings from the security control review. The risks identified below are based solely on the questionnaire, as technical scan and pre-existing risk data were unavailable.

\begin{table}[h!]
\centering
\caption{Summary of Identified Risks}
\begin{tabular}{@{}p{0.1\linewidth} p{0.25\linewidth} p{0.4\linewidth} l@{}}
    \toprule
    \textbf{Risk ID} & \textbf{Risk Name} & \textbf{Description} & \textbf{Severity} \\
    \midrule
    R-001 & Lack of MFA on Email & Email accounts are highly vulnerable to compromise via credential theft or password spraying. A compromised account can lead to data exfiltration, internal phishing, and Business Email Compromise (BEC). & \textbf{Critical} \\
    \addlinespace
    R-002 & No Security Awareness Training Program & Employees are not equipped to identify or respond to phishing, malware, or social engineering attacks. This makes them a primary vector for initial compromise of the network. & \textbf{High} \\
    \addlinespace
    R-003 & Incomplete Risk Visibility & The failure to process technical scan data and pre-existing risk logs means the organization is operating with significant blind spots regarding its security posture. & \textbf{High} \\
    \bottomrule
\end{tabular}
\end{table}

\textit{Note: The list of pre-existing vulnerabilities (\texttt{Input\_3\_Current\_Risks\_JSON}) was unavailable due to data corruption and could not be correlated with these findings.}

% ----------------------------------------------------------------------
% SECTION 6: RECOMMENDATIONS
% ----------------------------------------------------------------------
\section{Recommendations}
The following actionable recommendations are prioritized based on the severity of the associated risks.

\subsection{Priority 1: Remediate Critical Risks}
\begin{description}
    \item[R-001: Implement Mandatory MFA for Email Access] \\
    \textbf{Action:} Immediately enforce MFA for all users accessing the email system (\seqsplit{\texttt{MidnightOilStudios.org}}). This is the single most effective control to prevent unauthorized account access.
    \begin{itemize}
        \item Configure the email provider (e.g., Microsoft 365, Google Workspace) to require MFA for all logins.
        \item Mandate the use of strong MFA methods, such as authenticator apps (e.g., Google Authenticator, Microsoft Authenticator) or hardware security keys.
        \item Provide clear instructions and support to all employees during the rollout.
    \end{itemize}
\end{description}

\subsection{Priority 2: Remediate High Risks}
\begin{description}
    \item[R-002: Establish a Security Awareness Training Program] \\
    \textbf{Action:} Develop and implement a formal security awareness training program to build a culture of security and reduce human-based risk.
    \begin{itemize}
        \item Implement mandatory security training for all new employees as part of the onboarding process.
        \item Conduct annual refresher training for all staff covering topics like phishing, password security, and acceptable use.
        \item Consider periodic phishing simulations to test and reinforce employee awareness.
    \end{itemize}
    \item[R-003: Conduct a Full Network Vulnerability Scan] \\
    \textbf{Action:} Commission a new, comprehensive vulnerability scan of the organization's external network perimeter, specifically targeting the IP address \texttt{203.67.47.66}.
    \begin{itemize}
        \item The scan should include, at a minimum, a full TCP/UDP port scan and service version detection.
        \item Results from the new scan must be analyzed promptly to identify and remediate any discovered vulnerabilities.
    \end{itemize}
\end{description}

% ----------------------------------------------------------------------
% DOCUMENT END
% ----------------------------------------------------------------------
\end{document}
```