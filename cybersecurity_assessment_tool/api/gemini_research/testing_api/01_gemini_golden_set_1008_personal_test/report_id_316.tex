```latex
\documentclass[12pt]{article}

% Preamble: Required Packages
\usepackage[margin=1in]{geometry}
\usepackage{pifont} % For checkmarks and crosses
\usepackage{booktabs} % For professional tables
\usepackage{hyperref} % For clickable links and better PDF navigation
\usepackage{url} % For formatting URLs
\usepackage{seqsplit} % To split long strings without breaking words

% Document Metadata and Styling
\hypersetup{
    colorlinks=true,
    linkcolor=blue,
    filecolor=magenta,      
    urlcolor=cyan,
    pdftitle={Cybersecurity Assessment Report},
    pdfpagemode=FullScreen,
}

\newcommand{\yes}{\ding{51}} % Green checkmark
\newcommand{\no}{\ding{55}}  % Red X

\begin{document}

% --- Title Page ---
\begin{titlepage}
    \centering
    \vspace*{\fill}
    \Huge{\textbf{Cybersecurity Assessment Report}}
    \vspace{1.5cm}
    \Large{\textbf{Prepared for: Opal Sky Media}}
    \vspace{2cm}
    \normalsize
    \begin{tabular}{ll}
        \textbf{Date of Report:} & \today \\
        \textbf{Author:} & Cybersecurity Analyst \\
    \end{tabular}
    \vspace*{\fill}
    \textit{This report is confidential and intended solely for the use of Opal Sky Media.}
\end{titlepage}

\tableofcontents
\newpage

% --- Section 1: Executive Overview ---
\section{Executive Overview}
This report provides a comprehensive cybersecurity assessment for Opal Sky Media, synthesizing findings from a technical network scan, a review of organizational security controls, and an analysis of existing risks.

The external network scan of the target host \texttt{192.168.1.100} revealed a strong perimeter security posture, with no open ports detected. This indicates that the host is not exposing any services to the network, which is a significant security strength.

However, the review of organizational security controls identified several critical gaps that present a high level of risk to the organization. The most pressing concerns are:
\begin{itemize}
    \item \textbf{Lack of Multi-Factor Authentication (MFA):} The absence of MFA for both email access and computer logins creates a substantial risk of account compromise and unauthorized access.
    \item \textbf{Inadequate Employee Onboarding:} New employees do not receive security awareness training, leaving them vulnerable to social engineering attacks like phishing from their first day.
\end{itemize}

While the technical posture is commendable, these procedural and policy-based vulnerabilities must be addressed urgently to protect against common and effective cyberattacks, such as Business Email Compromise (BEC) and ransomware. This report outlines these risks in detail and provides actionable recommendations for remediation.

% --- Section 2: Organizational Information ---
\section{Organizational Information}
The following details were provided for the assessment.
\begin{itemize}
    \item \textbf{Organization Name:} Opal Sky Media
    \item \textbf{Email Domain:} \texttt{OpalSkyMedia.org}
    \item \textbf{Website Domain:} \url{www.OpalSkyMedia.org}
    \item \textbf{External IP Address:} \texttt{78.180.90.55}
\end{itemize}

% --- Section 3: Security Control Review ---
\section{Security Control Review}
An assessment of key security controls was conducted via a questionnaire. The responses are summarized below. Gaps identified with a \no\ represent significant areas for improvement.

\begin{table}[h!]
\centering
\caption{Organizational Security Control Status}
\begin{tabular}{p{0.8\linewidth} c}
\toprule
\textbf{Control Question} & \textbf{Response} \\
\midrule
Do you require MFA to access email? & \no \\
Do you require MFA to log into computers? & \no \\
Do you require MFA to access sensitive data systems? & \yes \\
Does your organization have an employee acceptable use policy? & \yes \\
Does your organization do security awareness training for new employees? & \no \\
Does your organization do security awareness training for all employees at least once per year? & \yes \\
\bottomrule
\end{tabular}
\end{table}

% --- Section 4: Technical Scan Results ---
\section{Technical Scan Results}
A network scan was performed to identify open ports and exposed services on the designated target system.

\begin{itemize}
    \item \textbf{Target IP Address:} \texttt{192.168.1.100}
    \item \textbf{Scan Date:} \today
\end{itemize}

\subsection{Summary of Findings}
The scan revealed that the host is online, but no open TCP ports were detected. All 1000 scanned ports were reported as "closed". This is a positive security finding, as it indicates a properly configured firewall or a host with no network-facing services, minimizing its attack surface.

% --- Section 5: Risk Assessment ---
\section{Risk Assessment}
This section synthesizes the findings from the security control review and technical scan. The primary risks identified are related to organizational policies and procedures rather than technical vulnerabilities. No pre-existing vulnerabilities were provided for this assessment.

\begin{table}[h!]
\centering
\caption{Identified Risks and Severity}
\begin{tabular}{p{0.25\linewidth} p{0.5\linewidth} l}
\toprule
\textbf{Risk Name} & \textbf{Overview} & \textbf{Severity} \\
\midrule
\textbf{Lack of MFA for Email Access} & Failure to protect email accounts with MFA makes them highly susceptible to compromise via stolen credentials. This is a primary vector for Business Email Compromise (BEC) and phishing attacks. & \textbf{High} \\
\addlinespace
\textbf{Lack of MFA for Endpoint Access} & The absence of MFA for computer logins means that a single compromised password could grant an attacker full access to an employee's workstation, facilitating data theft and lateral movement. & \textbf{High} \\
\addlinespace
\textbf{Inadequate New Hire Security Training} & New employees are not trained on security policies and common threats upon joining. This makes them significantly more susceptible to social engineering and accidental policy violations. & \textbf{High} \\
\bottomrule
\end{tabular}
\end{table}

% --- Section 6: Recommendations ---
\section{Recommendations}
The following prioritized recommendations are provided to mitigate the identified risks and improve the overall security posture of Opal Sky Media.

\subsection{Immediate Priority}
\begin{enumerate}
    \item \textbf{Implement MFA for All Email Accounts:}
        \begin{itemize}
            \item \textbf{Action:} Enforce MFA for all user accounts on the \texttt{OpalSkyMedia.org} email domain immediately.
            \item \textbf{Justification:} This is the single most effective control to prevent unauthorized email access and mitigate the risk of Business Email Compromise (BEC).
        \end{itemize}
    \item \textbf{Establish New Hire Security Training:}
        \begin{itemize}
            \item \textbf{Action:} Develop and mandate a security awareness training module as a required step in the new employee onboarding process. This training should cover the acceptable use policy, phishing identification, and password security.
            \item \textbf{Justification:} This closes a critical window of vulnerability and ensures all employees have a baseline understanding of security expectations from day one.
        \end{itemize}
\end{enumerate}

\subsection{High Priority}
\begin{enumerate}
    \setcounter{enumi}{2} % Continue numbering
    \item \textbf{Deploy MFA for Endpoint Logins:}
        \begin{itemize}
            \item \textbf{Action:} Implement an MFA solution for all company computer and laptop logins.
            \item \textbf{Justification:} This adds a critical layer of defense to protect endpoints, preventing unauthorized access even if user credentials are stolen.
        \end{itemize}
\end{enumerate}

\subsection{General Recommendations}
\begin{enumerate}
    \setcounter{enumi}{3} % Continue numbering
    \item \textbf{Conduct Regular Security Reviews:}
        \begin{itemize}
            \item \textbf{Action:} Schedule and perform periodic reviews of security controls and conduct regular network vulnerability scans.
            \item \textbf{Justification:} The threat landscape is constantly evolving. Continuous monitoring ensures that the security posture remains strong and adapts to new risks.
        \end{itemize}
\end{enumerate}

\end{document}
```