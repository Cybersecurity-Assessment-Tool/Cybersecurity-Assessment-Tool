```latex
\documentclass[12pt]{article}

% Preamble: Required Packages
\usepackage[margin=1in]{geometry}
\usepackage{pifont} % For checkmarks and crosses
\usepackage{booktabs} % For professional tables
\usepackage{hyperref} % For clickable links and metadata
\usepackage{url} % For formatting URLs
\usepackage{seqsplit} % For splitting long strings in tt font
\usepackage{graphicx} % For potential logos
\usepackage{fancyhdr} % For headers/footers
\usepackage{lastpage} % To get the total number of pages

% --- Document Metadata ---
\hypersetup{
    colorlinks=true,
    linkcolor=blue,
    filecolor=magenta,      
    urlcolor=cyan,
    pdftitle={Cybersecurity Posture Assessment Report},
    pdfauthor={Cybersecurity Analyst},
    pdfsubject={Security Analysis},
    pdfkeywords={Security, Report, Analysis},
    pdftoolbar=true,
}

% --- Header and Footer Configuration ---
\pagestyle{fancy}
\fancyhf{} % Clear all header and footer fields
\fancyhead[L]{Cybersecurity Posture Assessment}
\fancyhead[R]{Catalyst Consulting}
\fancyfoot[C]{\thepage\ of \pageref{LastPage}}
\renewcommand{\headrulewidth}{0.4pt}
\renewcommand{\footrulewidth}{0.4pt}

% --- Document Start ---
\begin{document}

% --- Title Page ---
\begin{titlepage}
    \centering
    \vspace*{1cm}
    \Huge{\textbf{Cybersecurity Posture Assessment Report}}
    \vspace{1.5cm}
    \Large{\textbf{Prepared for:}}
    \vspace{0.5cm}
    \Large{Catalyst Consulting}
    \vfill
    \large{
        \begin{tabular}{ll}
            \textbf{Date of Report:} & \today \\
            \textbf{Scan Date:} & See Technical Scan Section \\
            \textbf{Report ID:} & CSR-2024-001 \\
        \end{tabular}
    }
    \vspace{1cm}
    \small{\textit{This document contains sensitive information and is intended for the exclusive use of the recipient.}}
\end{titlepage}

\tableofcontents
\newpage

% --- Section 1: Executive Summary ---
\section{Executive Summary}
This report provides a comprehensive assessment of the cybersecurity posture for \textbf{Catalyst Consulting}, based on a combination of technical network scanning, a review of existing security risks, and an analysis of organizational security controls.

The assessment identified several critical and high-risk areas requiring immediate attention. While the organization has implemented Multi-Factor Authentication (MFA) for email and sensitive data systems, significant gaps exist in foundational security practices.

Key findings include:
\begin{itemize}
    \item \textbf{Critical Database Vulnerability:} A network scan confirmed the presence of a publicly accessible MySQL database. This system is running version 5.7.33, which is past its End-of-Life (EOL) and no longer receives security updates, posing a severe risk of data breach.
    \item \textbf{Endpoint Security Gaps:} The absence of MFA for computer logins presents a high risk, as compromised credentials could lead to direct workstation and network access.
    \item \textbf{Policy and Training Deficiencies:} The organization lacks a formal Acceptable Use Policy and does not conduct security awareness training for employees. These gaps significantly increase the susceptibility to human-centric attacks like phishing and social engineering.
\end{itemize}

This report outlines these findings in detail and provides a series of actionable recommendations to mitigate the identified risks and strengthen the overall security posture of \textbf{Catalyst Consulting}.

% --- Section 2: Organizational Information ---
\section{Organizational Information}
The following details were provided for the assessment.
\begin{table}[h!]
\centering
\begin{tabular}{@{}ll@{}}
\toprule
\textbf{Attribute} & \textbf{Value} \\ \midrule
Organization Name    & Catalyst Consulting \\
Email Domain         & \texttt{CatalystConsulting.com} \\
Website Domain       & \url{www.CatalystConsulting.com} \\
External IP Address  & \texttt{14.234.33.132} \\ \bottomrule
\end{tabular}
\caption{Client Organizational Details}
\end{table}

% --- Section 3: Security Control Review ---
\section{Security Control Review}
An analysis of the security questionnaire reveals critical gaps in administrative and technical controls. "No" responses indicate a lack of a necessary security measure and are correlated with specific risks in Section 5.

\begin{table}[h!]
\centering
\begin{tabular}{@{}lc@{}}
\toprule
\textbf{Control Question} & \textbf{Response} \\ \midrule
Do you require MFA to access email? & \ding{51} \\
Do you require MFA to log into computers? & \textbf{\color{red}\ding{55}} \\
Do you require MFA to access sensitive data systems? & \ding{51} \\
Does your organization have an employee acceptable use policy? & \textbf{\color{red}\ding{55}} \\
Does your organization do security awareness training for new employees? & \textbf{\color{red}\ding{55}} \\
Does your organization do security awareness training for all employees annually? & \textbf{\color{red}\ding{55}} \\ \bottomrule
\end{tabular}
\caption{Security Controls Questionnaire Analysis}
\end{table}

The lack of MFA on computers, an acceptable use policy, and any form of security awareness training are considered high-risk findings. These represent fundamental weaknesses that can be exploited by threat actors.

% --- Section 4: Technical Scan Results ---
\section{Technical Scan Results}
A network scan was performed on the target IP address to identify open ports and exposed services. The findings validate and expand upon the pre-existing risk data.

\begin{itemize}
    \item \textbf{Target IP:} \texttt{172.16.50.20}
    \item \textbf{Scan Tool:} Nmap
\end{itemize}

\begin{table}[h!]
\centering
\begin{tabular}{@{}llllll@{}}
\toprule
\textbf{Port} & \textbf{State} & \textbf{Service} & \textbf{Product} & \textbf{Version} & \textbf{Analyst Notes} \\ \midrule
3306 & Open & mysql & MySQL & 5.7.33 & \textbf{Critical.} EOL Version. \\ \bottomrule
\end{tabular}
\caption{Open Port Analysis}
\end{table}

\subsection{Analysis of Findings}
The scan identified an open MySQL port (\texttt{3306}). The running version, \textbf{MySQL 5.7.33}, reached its official End-of-Life (EOL) in October 2023. EOL software no longer receives security patches from the vendor, making it an easy target for exploitation of known vulnerabilities. This finding confirms the "Database Exposure" risk provided in the initial data.

% --- Section 5: Risk Assessment ---
\section{Risk Assessment}
The following table synthesizes findings from the security questionnaire, technical scan, and pre-existing risk data into a prioritized list.

\begin{table}[h!]
\centering
\begin{tabular}{@{}p{0.1\linewidth}p{0.25\linewidth}p{0.45\linewidth}p{0.1\linewidth}@{}}
\toprule
\textbf{Risk ID} & \textbf{Risk Name} & \textbf{Description} & \textbf{Severity} \\ \midrule
\textbf{R-01} & Publicly Exposed End-of-Life Database & The MySQL server is accessible from the network and is running an unsupported version (5.7.33), which no longer receives security updates. This directly validates a known high-risk finding. & \textbf{Critical} \\
\addlinespace
\textbf{R-02} & Lack of Endpoint MFA & Workstations do not require Multi-Factor Authentication, making them highly vulnerable to unauthorized access via stolen or weak credentials. & High \\
\addlinespace
\textbf{R-03} & Insufficient Security Awareness Training & The absence of security training for new and current employees increases the likelihood of successful phishing, social engineering, and malware incidents. & High \\
\addlinespace
\textbf{R-04} & Missing Foundational Security Policies & The lack of an Acceptable Use Policy creates ambiguity and increases the risk of insider threat, whether malicious or accidental. & High \\
\bottomrule
\end{tabular}
\caption{Synthesized Risk Summary}
\end{table}

% --- Section 6: Recommendations ---
\section{Recommendations}
The following actions are recommended to mitigate the identified risks. Recommendations are categorized by the associated risk ID.

\subsection{R-01: Publicly Exposed End-of-Life Database}
\begin{itemize}
    \item \textbf{Immediate (0-7 Days):} Implement strict firewall rules to restrict all access to port \texttt{3306} on \texttt{172.16.50.20} to only trusted, internal IP addresses.
    \item \textbf{Short-Term (1-3 Months):} Plan and execute a migration from MySQL 5.7 to a currently supported version (e.g., MySQL 8.x) to ensure security patches are available.
    \item \textbf{Long-Term (3-6 Months):} Re-architect the network so that the database server resides in a private, non-internet-facing subnet, accessible only via a bastion host or VPN.
\end{itemize}

\subsection{R-02: Lack of Endpoint MFA}
\begin{itemize}
    \item \textbf{Immediate (0-30 Days):} Procure and pilot an MFA solution for workstation logins with the IT department and key executives.
    \item \textbf{Short-Term (1-3 Months):} Develop a phased rollout plan to deploy MFA across the entire organization.
    \item \textbf{Long-Term (3-6 Months):} Enforce MFA for all user logins to company workstations and laptops without exception.
\end{itemize}

\subsection{R-03: Insufficient Security Awareness Training}
\begin{itemize}
    \item \textbf{Immediate (0-30 Days):} Select and procure a security awareness training platform.
    \item \textbf{Short-Term (1-3 Months):} Enroll all employees in foundational training covering phishing, password hygiene, and data handling. Make this training mandatory for all new hires during onboarding.
    \item \textbf{Long-Term (Ongoing):} Establish a continuous security culture program, including mandatory annual refresher training and quarterly simulated phishing campaigns.
\end{itemize}

\subsection{R-04: Missing Foundational Security Policies}
\begin{itemize}
    \item \textbf{Immediate (0-30 Days):} Draft a formal Acceptable Use Policy (AUP) that clearly defines rules for using company assets, data handling, and internet access.
    \item \textbf{Short-Term (1-2 Months):} Have the draft AUP reviewed by HR, legal, and senior management for approval.
    \item \textbf{Long-Term (Ongoing):} Implement a process to have all new and existing employees formally read and acknowledge the AUP. Review and update the policy annually.
\end{itemize}

\end{document}
```