```latex
\documentclass[12pt]{article}

% ----------------------------------------------------------------------
% METADATA and PACKAGES
% ----------------------------------------------------------------------
\usepackage[margin=1in]{geometry}
\usepackage{pifont} % For checkmarks and crosses
\usepackage{booktabs} % For professional tables
\usepackage{hyperref} % For clickable links
\usepackage{url}      % For URL formatting
\usepackage{seqsplit} % To split long strings in texttt
\usepackage[T1]{fontenc}

\hypersetup{
    colorlinks=true,
    linkcolor=black,
    filecolor=magenta,      
    urlcolor=blue,
    pdftitle={Cybersecurity Posture Report},
    pdfauthor={Cybersecurity Analyst},
    pdfsubject={Security Assessment},
    pdfkeywords={Security, Report, Analysis},
    bookmarks=true
}

\newcommand{\yes}{\ding{51}}
\newcommand{\no}{\ding{55}}

\author{Cybersecurity Analyst}
\title{Cybersecurity Posture Report for \textbf{Hidden Gem}}
\date{November 22, 2025}

% ----------------------------------------------------------------------
% DOCUMENT START
% ----------------------------------------------------------------------
\begin{document}

\maketitle
\hrule
\vspace{1em}

% ----------------------------------------------------------------------
% 1. EXECUTIVE SUMMARY
% ----------------------------------------------------------------------
\section*{Executive Summary}

This report provides a comprehensive cybersecurity assessment for \textbf{Hidden Gem}, based on data collected on November 22, 2025. The analysis combines a review of organizational security controls, a technical network scan, and an evaluation of known risks.

The organization demonstrates a strong commitment to identity and access management, with multi-factor authentication (MFA) widely deployed across key systems. Security awareness training programs are also well-established.

However, two significant risks were identified that require immediate attention. First, a critical administrative gap exists due to the lack of a formal Employee Acceptable Use Policy (AUP). Second, a technical vulnerability was discovered: the external-facing web server is running an outdated and vulnerable version of Nginx (1.18.0). These findings present a tangible risk of system compromise and data misuse.

This report details these findings and provides actionable recommendations to mitigate the identified risks and improve the overall security posture.

% ----------------------------------------------------------------------
% 2. ORGANIZATIONAL INFORMATION
% ----------------------------------------------------------------------
\section*{Organizational Information}

The following details were provided for the assessment.

\begin{itemize}
    \item \textbf{Organization Name:} Hidden Gem
    \item \textbf{Email Domain:} \texttt{HiddenGem.com}
    \item \textbf{Website Domain:} \url{www.HiddenGem.com}
    \item \textbf{External IP Address:} \texttt{129.145.173.200}
\end{itemize}

% ----------------------------------------------------------------------
% 3. SECURITY CONTROL REVIEW
% ----------------------------------------------------------------------
\section*{Security Control Review}

A review of administrative and organizational security controls was conducted based on a supplied questionnaire. The results indicate a strong foundation in some areas but also highlight a critical policy gap.

\begin{table}[h!]
\centering
\caption{Organizational Security Control Questionnaire}
\begin{tabular}{@{}lc@{}}
\toprule
\textbf{Control Question} & \textbf{Response} \\ \midrule
Do you require MFA to access email? & \yes \\
Do you require MFA to log into computers? & \yes \\
Do you require MFA to access sensitive data systems? & \yes \\
Does your organization have an employee acceptable use policy? & \no \\
Does your organization do security awareness training for new employees? & \yes \\
Does your organization do security awareness training for all employees annually? & \yes \\
\bottomrule
\end{tabular}
\end{table}

\paragraph{Analysis:} The consistent implementation of MFA and security awareness training is commendable and significantly reduces risks related to phishing and credential theft. However, the absence of an Employee Acceptable Use Policy is a \textbf{High Risk} finding. An AUP is a foundational document that sets clear expectations for employees on how to use company resources securely and ethically. Without it, the organization lacks a formal basis for enforcing security standards and holding individuals accountable for misuse.

% ----------------------------------------------------------------------
% 4. TECHNICAL SCAN RESULTS
% ----------------------------------------------------------------------
\section*{Technical Scan Results}

An Nmap scan was performed on \texttt{192.168.10.5} on November 22, 2025. The scan identified one open port, detailed below.

\begin{table}[h!]
\centering
\caption{Open Port Analysis for Target: \texttt{192.168.10.5}}
\begin{tabular}{@{}lllll@{}}
\toprule
\textbf{Port} & \textbf{State} & \textbf{Service} & \textbf{Product} & \textbf{Version} \\ \midrule
443/tcp & open & https & nginx & 1.18.0 \\ \bottomrule
\end{tabular}
\end{table}

\paragraph{Analysis:} The scan revealed that the server is running Nginx version 1.18.0. This version was released in April 2020 and is now considered outdated. It is known to be vulnerable to several security issues, including but not limited to request smuggling and denial-of-service vulnerabilities (e.g., CVE-2022-41741, CVE-2022-41742). Running outdated software on an internet-facing service constitutes a \textbf{High Risk}, as it provides a clear vector for attackers to compromise the server and potentially gain access to the internal network.

% ----------------------------------------------------------------------
% 5. RISK ASSESSMENT SUMMARY
% ----------------------------------------------------------------------
\section*{Risk Assessment Summary}

The following table synthesizes the findings from the security control review and technical scan. No pre-existing vulnerabilities were reported.

\begin{table}[h!]
\centering
\caption{Identified Risks}
\begin{tabular}{@{}p{0.1\linewidth} p{0.3\linewidth} p{0.1\linewidth} p{0.4\linewidth}@{}}
\toprule
\textbf{Risk ID} & \textbf{Risk Name} & \textbf{Severity} & \textbf{Description} \\ \midrule
RISK-001 & Lack of Acceptable Use Policy & \textbf{High} & The absence of a formal AUP creates ambiguity for employees regarding the proper use of company assets, increasing the risk of insider threats and making security rule enforcement difficult. \\
\addlinespace
RISK-002 & Outdated Web Server Software & \textbf{High} & The web server at \texttt{192.168.10.5} is running a vulnerable version of Nginx (1.18.0), exposing the organization to potential remote compromise, data breaches, or service disruption. \\ \bottomrule
\end{tabular}
\end{table}

% ----------------------------------------------------------------------
% 6. RECOMMENDATIONS
% ----------------------------------------------------------------------
\section*{Recommendations}

To address the identified risks and enhance the organization's security posture, the following actions are recommended with high priority.

\begin{enumerate}
    \item \textbf{Develop and Implement an Acceptable Use Policy (AUP):}
    \begin{itemize}
        \item \textbf{Action:} Draft a comprehensive AUP that clearly defines the rules for using company networks, devices, software, and data.
        \item \textbf{Impact:} Establishes a clear security baseline for all employees, reduces the risk of accidental or malicious misuse of resources, and provides a legal framework for enforcement.
        \item \textbf{Timeline:} High Priority (within 30 days).
    \end{itemize}
    \vspace{1em}
    \item \textbf{Upgrade the Nginx Web Server:}
    \begin{itemize}
        \item \textbf{Action:} Plan and execute an upgrade of the Nginx instance on server \texttt{192.168.10.5} to the latest stable version.
        \item \textbf{Impact:} Mitigates multiple known vulnerabilities, protecting the server from compromise and safeguarding the data it processes.
        \item \textbf{Timeline:} Critical Priority (within 7 days).
    \end{itemize}
    \vspace{1em}
    \item \textbf{Establish a Patch Management Program:}
    \begin{itemize}
        \item \textbf{Action:} Formalize a process for regularly identifying, testing, and deploying security patches for all critical software and systems.
        \item \textbf{Impact:} Proactively reduces the attack surface by ensuring vulnerabilities are addressed in a timely manner, preventing future occurrences of issues like RISK-002.
        \item \textbf{Timeline:} High Priority (begin planning immediately).
    \end{itemize}
\end{enumerate}

% ----------------------------------------------------------------------
% 7. CONCLUSION
% ----------------------------------------------------------------------
\section*{Conclusion}
\textbf{Hidden Gem} has implemented several effective security controls, particularly in user authentication and awareness. However, the identified governance and technical vulnerabilities require immediate remediation to prevent potential exploitation. By implementing the recommendations outlined in this report—specifically by creating an AUP and upgrading critical software—the organization can significantly strengthen its defenses and build a more resilient security posture.

\end{document}
```