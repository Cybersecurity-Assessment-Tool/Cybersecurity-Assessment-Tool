```latex
\documentclass[12pt]{article}

% Required Packages
\usepackage[margin=1in]{geometry}
\usepackage{pifont} % For \ding
\usepackage{booktabs} % For professional tables
\usepackage{hyperref} % For clickable links
\usepackage{url} % For URL formatting
\usepackage{seqsplit} % For splitting long strings like IPs
\usepackage{graphicx}
\usepackage{fancyhdr}
\usepackage{xcolor}
\usepackage{array}

% --- Document Setup ---
\hypersetup{
    colorlinks=true,
    linkcolor=blue,
    urlcolor=blue,
    pdftitle={Cybersecurity Posture Assessment Report},
    pdfauthor={Cybersecurity Analyst},
}

% Define colors for severity
\definecolor{critical}{HTML}{990000}
\definecolor{high}{HTML}{D13F3F}
\definecolor{medium}{HTML}{E89829}
\definecolor{low}{HTML}{4CAF50}

% Define a command for severity labels
\newcommand{\severity}[2]{\colorbox{#1}{\textcolor{white}{\textbf{\sffamily\small #2}}}}

% --- Header and Footer ---
\pagestyle{fancy}
\fancyhf{}
\fancyhead[L]{\textbf{Cybersecurity Posture Assessment Report}}
\fancyhead[R]{\textbf{Modern Myth}}
\fancyfoot[C]{\thepage}
\renewcommand{\headrulewidth}{0.4pt}
\renewcommand{\footrulewidth}{0.4pt}

\begin{document}

% --- Title Page ---
\begin{titlepage}
    \centering
    \vspace*{2cm}
    {\Huge \textbf{Cybersecurity Posture Assessment Report}\par}
    \vspace{1.5cm}
    {\Large \textbf{Prepared For:}\par}
    \vspace{0.5cm}
    {\Large Modern Myth\par}
    \vfill
    {\large \today\par}
    \vspace{1cm}
    \textit{This report contains sensitive information and should be handled with care.}
\end{titlepage}

\newpage

% --- Table of Contents ---
\tableofcontents
\newpage

% --- Section 1: Executive Summary ---
\section{Executive Summary}

This report provides a comprehensive assessment of the cybersecurity posture for Modern Myth, based on an analysis of organizational security controls, a technical network scan, and a review of known risks. The assessment was conducted on \today.

The analysis reveals a mixed security posture. While the organization has implemented some important controls, such as requiring Multi-Factor Authentication (MFA) for computer and sensitive system access, several critical and high-risk gaps were identified.

\textbf{Key Findings Include:}
\begin{itemize}
    \item \textbf{Critical Risk:} The lack of mandatory MFA for email access represents a significant vulnerability. Email is a primary target for phishing and account takeover attacks, which can lead to data breaches and further network compromise.
    \item \textbf{High Risk:} The absence of an employee Acceptable Use Policy and a formal security awareness training program creates a substantial risk from insider threats, both malicious and unintentional. A well-informed workforce is the first line of defense.
    \item \textbf{Technical Finding:} An external-facing SSH service was identified on the IPv6 address \seqsplit{\texttt{2001:db8::1}}. While necessary for remote administration, if not properly secured, this service can be a direct entry point for attackers.
\end{itemize}

This report details these findings and provides actionable recommendations to mitigate the identified risks. Prioritizing the implementation of MFA for email, developing foundational security policies, and securing exposed services is crucial to improving the overall security posture of Modern Myth.

\newpage

% --- Section 2: Organizational Information ---
\section{Organizational Information}
This section provides the organizational details used as the basis for this assessment.

\begin{tabular}{@{}ll}
\toprule
\textbf{Attribute} & \textbf{Value} \\
\midrule
Organization Name & Modern Myth \\
Email Domain & \texttt{ModernMyth.net} \\
Website Domain & \url{www.ModernMyth.net} \\
Monitored External IP & \texttt{14.186.21.45} \\
\bottomrule
\end{tabular}

% --- Section 3: Security Control Review ---
\section{Security Control Review}
The following table summarizes the organization's current security controls based on the provided questionnaire. Items marked with \ding{55} indicate significant gaps in the security framework that increase risk.

\begin{table}[h!]
\centering
\begin{tabular}{>{\raggedright\arraybackslash}p{12cm} c}
\toprule
\textbf{Control Question} & \textbf{Status} \\
\midrule
Do you require MFA to log into computers? & \ding{51} \\
Do you require MFA to access sensitive data systems? & \ding{51} \\
\midrule
\textcolor{red}{Do you require MFA to access email?} & \textcolor{red}{\ding{55}} \\
\textcolor{red}{Does your organization have an employee acceptable use policy?} & \textcolor{red}{\ding{55}} \\
\textcolor{red}{Does your organization do security awareness training for new employees?} & \textcolor{red}{\ding{55}} \\
\textcolor{red}{Does your organization do security awareness training for all employees at least once per year?} & \textcolor{red}{\ding{55}} \\
\bottomrule
\end{tabular}
\caption{Security Control Questionnaire Results (\ding{51} = Yes, \ding{55} = No)}
\label{tab:controls}
\end{table}

The review highlights a critical weakness in email security and a complete lack of foundational user policies and training. These gaps leave the organization vulnerable to phishing, social engineering, and insider threats.

% --- Section 4: Technical Scan Results ---
\section{Technical Scan Results}
An external network scan was performed on the target IP address to identify open ports and exposed services.

\begin{itemize}
    \item \textbf{Target IP Address:} \seqsplit{\texttt{2001:db8::1}}
    \item \textbf{Host Status:} Up
\end{itemize}

The following table details the open ports discovered during the scan.

\begin{table}[h!]
\centering
\begin{tabular}{llll}
\toprule
\textbf{Port} & \textbf{Protocol} & \textbf{State} & \textbf{Service} \\
\midrule
22 & TCP & Open & ssh (Secure Shell) \\
\bottomrule
\end{tabular}
\caption{Open Ports on \seqsplit{\texttt{2001:db8::1}}}
\label{tab:ports}
\end{table}

\textbf{Analysis:} The discovery of an open SSH port (22) indicates that remote administrative access is enabled on this host. This is a common attack vector. It is essential to ensure that this service is configured securely by enforcing strong authentication (e.g., key-based instead of password-based), restricting access to trusted IP addresses, and monitoring for brute-force attempts.

\newpage

% --- Section 5: Risk Assessment ---
\section{Risk Assessment}
This section synthesizes findings from the security control review and technical scan into a prioritized list of risks. No pre-existing vulnerabilities were reported.

\begin{table}[h!]
\centering
\begin{tabular}{p{0.5cm} p{6cm} l p{3.5cm}}
\toprule
\textbf{ID} & \textbf{Risk Description} & \textbf{Severity} & \textbf{Finding Source} \\
\midrule
R-01 & Lack of MFA on email accounts allows for easy account takeover via stolen credentials, leading to data breaches and phishing. & \severity{critical}{CRITICAL} & Questionnaire \\
\addlinespace
R-02 & No formal security awareness training program leaves employees unable to recognize and respond to threats like phishing. & \severity{high}{HIGH} & Questionnaire \\
\addlinespace
R-03 & Absence of an Acceptable Use Policy (AUP) leads to inconsistent and unsafe use of company assets, increasing insider risk. & \severity{high}{HIGH} & Questionnaire \\
\addlinespace
R-04 & Exposed SSH service on a public-facing server could be exploited by brute-force or credential stuffing attacks to gain unauthorized access. & \severity{medium}{MEDIUM} & Network Scan \\
\bottomrule
\end{tabular}
\caption{Summary of Identified Risks}
\label{tab:risks}
\end{table}

% --- Section 6: Recommendations ---
\section{Recommendations}
The following actionable recommendations are provided to mitigate the risks identified in this report.

\subsection{R-01: Implement MFA for Email (CRITICAL)}
Immediately enforce Multi-Factor Authentication (MFA) for all user accounts on the \texttt{ModernMyth.net} email system. This is the single most effective control to prevent unauthorized access to email, which is a gateway to sensitive company data and other systems.
\begin{itemize}
    \item \textbf{Action:} Enable MFA within the email provider's security settings.
    \item \textbf{Priority:} Immediate
\end{itemize}

\subsection{R-02: Establish Security Awareness Training (HIGH)}
Develop and implement a mandatory security awareness training program. This program should be required for all new hires during onboarding and annually for all existing staff.
\begin{itemize}
    \item \textbf{Action:} Procure a training solution and create a formal schedule. Topics must include phishing identification, password hygiene, and social engineering defense.
    \item \textbf{Priority:} High
\end{itemize}

\subsection{R-03: Develop an Acceptable Use Policy (HIGH)}
Create a formal Acceptable Use Policy (AUP) that clearly defines the rules for using company computers, networks, and data.
\begin{itemize}
    \item \textbf{Action:} Draft an AUP, have it reviewed by legal/HR, and require all employees to read and formally acknowledge it.
    \item \textbf{Priority:} High
\end{itemize}

\subsection{R-04: Secure the Exposed SSH Service (MEDIUM)}
Review the business justification for the publicly accessible SSH port on \seqsplit{\texttt{2001:db8::1}}.
\begin{itemize}
    \item \textbf{Action:} If the service is not required, disable it. If it is required, implement the following controls:
    \begin{enumerate}
        \item Use a firewall to restrict access to only known, trusted IP addresses.
        \item Disable password-based authentication and enforce the use of public/private key pairs.
        \item Disable direct root login.
        \item Implement intrusion detection and log all login attempts.
    \end{enumerate}
    \item \textbf{Priority:} Medium
\end{itemize}

\end{document}
```