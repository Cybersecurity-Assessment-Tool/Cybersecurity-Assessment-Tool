```latex
\documentclass[12pt]{article}

% Preamble: Required Packages
\usepackage[margin=1in]{geometry}
\usepackage{pifont} % For checkmarks and crosses
\usepackage{booktabs} % For professional tables
\usepackage{hyperref} % For clickable links
\usepackage{url} % For URL formatting
\usepackage{seqsplit} % To split long strings without breaking
\usepackage{graphicx}
\usepackage{xcolor}

% Document Metadata
\title{Cybersecurity Posture Assessment Report}
\author{Cybersecurity Analysis Division}
\date{November 22, 2025}

% Hyperref Setup
\hypersetup{
    colorlinks=true,
    linkcolor=blue,
    filecolor=magenta,      
    urlcolor=cyan,
    pdftitle={Cybersecurity Posture Assessment Report},
    pdfpagemode=FullScreen,
}

% Define checkmark and cross symbols
\newcommand{\cmark}{\ding{51}}
\newcommand{\xmark}{\ding{55}}

\begin{document}

\maketitle
\thispagestyle{empty}
\newpage
\tableofcontents
\thispagestyle{empty}
\newpage

% --- 1. Executive Summary ---
\section*{1. Executive Summary}
This report provides a comprehensive cybersecurity posture assessment for \textbf{Symmetry Architecture}, conducted on November 22, 2025. The analysis is based on a synthesis of network scan data, a security controls questionnaire, and a review of pre-existing risks.

The assessment reveals a mixed security posture. The organization demonstrates strong identity and access management controls, with Multi-Factor Authentication (MFA) widely implemented across key systems. However, significant risks were identified in two primary areas: human-factor security and technical vulnerability management.

\textbf{Key Findings:}
\begin{itemize}
    \item \textbf{Critical Gap in Employee Onboarding:} A critical gap was identified in the security training process for new employees. This oversight exposes the organization to a heightened risk of social engineering, phishing, and accidental data breaches.
    \item \textbf{Vulnerable Web Server:} The external-facing web server at \texttt{192.168.10.5} is running an outdated version of Nginx (1.18.0), which is several years old and known to have multiple security vulnerabilities. This presents a direct and exploitable attack vector.
    \item \textbf{SSL Certificate Misconfiguration:} The SSL certificate on the web server does not match the organization's domain, which can erode user trust and may indicate a more significant server misconfiguration.
\end{itemize}

Immediate action is recommended to address these findings. Recommendations focus on implementing a mandatory security training program for new hires and upgrading the vulnerable web server software to mitigate imminent threats.

% --- 2. Organizational Information ---
\section*{2. Organizational Information}
The following details were provided by the client and used as a baseline for this assessment.

\begin{tabular}{@{}ll}
    \toprule
    \textbf{Attribute} & \textbf{Value} \\
    \midrule
    Organization Name & \textbf{Symmetry Architecture} \\
    Email Domain & \texttt{SymmetryArchitecture.com} \\
    Website Domain & \url{www.SymmetryArchitecture.com} \\
    External IP Address & \texttt{25.15.146.154} \\
    \bottomrule
\end{tabular}

% --- 3. Security Control Review ---
\section*{3. Security Control Review}
This section details the organization's self-reported security controls based on the provided questionnaire. "No" answers indicate potential gaps in the security framework that require attention.

\begin{tabular}{@{}p{0.75\linewidth}c}
    \toprule
    \textbf{Control Question} & \textbf{Response} \\
    \midrule
    Do you require MFA to access email? & \textcolor{green}{\cmark} \\
    Do you require MFA to log into computers? & \textcolor{green}{\cmark} \\
    Do you require MFA to access sensitive data systems? & \textcolor{green}{\cmark} \\
    Does your organization have an employee acceptable use policy? & \textcolor{green}{\cmark} \\
    Does your organization do security awareness training for new employees? & \textcolor{red}{\xmark} \\
    Does your organization do security awareness training for all employees at least once per year? & \textcolor{green}{\cmark} \\
    \bottomrule
\end{tabular}

\subsection*{Analysis of Controls}
The organization has a robust MFA implementation, which significantly strengthens its defense against credential theft and unauthorized access. However, the lack of mandatory security awareness training for new employees is a critical procedural flaw. New hires are often prime targets for attackers, and without proper initial training, they represent a high-risk entry point into the organization.

% --- 4. Technical Scan Results ---
\section*{4. Technical Scan Results}
An external network scan was performed to identify open ports and exposed services.

\begin{itemize}
    \item \textbf{Scan Target:} \texttt{192.168.10.5}
    \item \textbf{Scan Date:} 2025-11-22T10:00:00Z
\end{itemize}

\begin{tabular}{@{}llllll}
    \toprule
    \textbf{Port} & \textbf{State} & \textbf{Service} & \textbf{Product} & \textbf{Version} & \textbf{Notes} \\
    \midrule
    443/tcp & open & https & nginx & 1.18.0 & \parbox[t]{4cm}{SSL certificate common name is \texttt{www.acme-corp.com}, a mismatch with the client's domain.} \\
    \bottomrule
\end{tabular}

\subsection*{Analysis of Technical Findings}
The scan identified a single open port (443/tcp) running an Nginx web server. The version detected, \textbf{1.18.0}, was released in April 2020. This version is outdated and has numerous publicly disclosed vulnerabilities (CVEs). Continuing to run this software version poses a significant and unnecessary risk of compromise. The SSL certificate mismatch further suggests a lack of proper configuration management and security hygiene for this critical public-facing asset.

% --- 5. Consolidated Risk Assessment ---
\section*{5. Consolidated Risk Assessment}
The following table synthesizes findings from the security control review and technical scan into a prioritized list of risks. No pre-existing vulnerabilities were reported.

\begin{tabular}{@{}lp{0.3\linewidth}p{0.45\linewidth}l}
    \toprule
    \textbf{ID} & \textbf{Risk Name} & \textbf{Description} & \textbf{Severity} \\
    \midrule
    RISK-001 & Lack of Onboarding Security Training & New employees are not provided with security awareness training, making them highly susceptible to phishing, social engineering, and policy violations. & \textbf{High} \\
    \addlinespace
    RISK-002 & Outdated Web Server Software & The public-facing web server runs Nginx 1.18.0, a version with known, exploitable vulnerabilities. This could lead to server compromise, data breach, or defacement. & \textbf{High} \\
    \addlinespace
    RISK-003 & SSL Certificate Mismatch & The SSL certificate on the web server does not match the organization's domain, which can cause browser trust errors and damage brand reputation. & \textbf{Medium} \\
    \bottomrule
\end{tabular}

% --- 6. Recommendations ---
\section*{6. Recommendations}
The following actionable steps are recommended to mitigate the identified risks and improve the overall security posture of \textbf{Symmetry Architecture}.

\subsection*{RISK-001: Lack of Onboarding Security Training (High)}
\begin{itemize}
    \item \textbf{Action:} Implement a mandatory security awareness training module as a standard part of the new employee onboarding process.
    \item \textbf{Details:} This training should cover, at a minimum, phishing identification, password security, acceptable use of company assets, and procedures for reporting security incidents. This closes a critical gap in the human firewall.
\end{itemize}

\subsection*{RISK-002: Outdated Web Server Software (High)}
\begin{itemize}
    \item \textbf{Action:} Plan and execute an upgrade of the Nginx server at \texttt{192.168.10.5} to the latest stable version.
    \item \textbf{Details:} Before deployment, the new version should be tested in a staging environment to ensure compatibility with the existing web application. A vulnerability scan should be performed post-upgrade to confirm that known vulnerabilities have been remediated.
\end{itemize}

\subsection*{RISK-003: SSL Certificate Mismatch (Medium)}
\begin{itemize}
    \item \textbf{Action:} Replace the incorrect SSL certificate with a valid one issued for the correct domain (\texttt{www.SymmetryArchitecture.com}).
    \item \textbf{Details:} Procure and install a valid TLS/SSL certificate from a trusted Certificate Authority (CA). Ensure the server is configured to use this new certificate for all HTTPS traffic to build user trust and secure communications.
\end{itemize}

\end{document}
```