```latex
\documentclass[12pt]{article}

% Preamble: Required Packages
\usepackage[margin=1in]{geometry}
\usepackage{pifont} % For checkmarks and crosses
\usepackage{booktabs} % For professional tables
\usepackage{hyperref} % For clickable links
\usepackage{url} % For URL formatting
\usepackage{seqsplit} % To split long strings in tt font
\usepackage{graphicx} % For potential logos
\usepackage{xcolor} % For colors

% Document Information
\title{Cybersecurity Posture Assessment Report}
\author{Cybersecurity Analyst}
\date{\today}

% Hyperref Setup
\hypersetup{
    colorlinks=true,
    linkcolor=blue,
    filecolor=magenta,      
    urlcolor=cyan,
    pdftitle={Cybersecurity Posture Assessment Report},
    pdfpagemode=FullScreen,
}

\begin{document}

\maketitle
\thispagestyle{empty}
\newpage

\tableofcontents
\newpage

% --- 1. Overview and Executive Summary ---
\section{Overview and Executive Summary}

This report details the findings of a cybersecurity posture assessment for \textbf{Iron Oak Furniture}. The evaluation was conducted by analyzing network scan data, a security controls questionnaire, and a list of pre-existing risks.

The overall security posture requires significant improvement. While some essential controls like Multi-Factor Authentication (MFA) for email are in place, critical gaps exist in foundational areas. Key findings include the absence of a mandatory security awareness training program, a lack of an Acceptable Use Policy (AUP), and failure to enforce MFA for computer logins.

Technically, an externally exposed Secure Shell (SSH) service was identified on an IPv6 address. When combined with the identified policy and access control weaknesses, this presents a high-risk attack vector. Immediate remediation of the identified risks is strongly recommended to reduce the organization's exposure to common cyber threats such as ransomware, data breaches, and unauthorized access.

% --- 2. Organizational Information ---
\section{Organizational Information}

The following details were provided for the assessment.

\begin{tabular}{@{}ll}
\toprule
\textbf{Attribute} & \textbf{Value} \\
\midrule
Organization Name & \textbf{Iron Oak Furniture} \\
Email Domain & \texttt{IronOakFurniture.org} \\
Website Domain & \url{www.IronOakFurniture.org} \\
External IP (IPv4) & \texttt{232.63.1.231} \\
External IP (IPv6 Scanned) & \seqsplit{\texttt{2001:db8::1}} \\
\bottomrule
\end{tabular}

% --- 3. Security Control Review ---
\section{Security Control Review}

The following table summarizes the organization's responses to a security controls questionnaire. A green checkmark (\textcolor{green}{\ding{51}}) indicates a positive control is in place, while a red cross (\textcolor{red}{\ding{55}}) highlights a control gap that introduces risk.

\begin{table}[h!]
\centering
\begin{tabular}{p{0.7\textwidth} c c}
\toprule
\textbf{Control Question} & \textbf{Response} & \textbf{Status} \\
\midrule
Do you require MFA to access email? & Yes & \textcolor{green}{\ding{51}} \\
Do you require MFA to log into computers? & No & \textcolor{red}{\ding{55}} \\
Do you require MFA to access sensitive data systems? & Yes & \textcolor{green}{\ding{51}} \\
Does your organization have an employee acceptable use policy? & No & \textcolor{red}{\ding{55}} \\
Does your organization do security awareness training for new employees? & No & \textcolor{red}{\ding{55}} \\
Does your organization do security awareness training for all employees at least once per year? & No & \textcolor{red}{\ding{55}} \\
\bottomrule
\end{tabular}
\caption{Security Controls Questionnaire Results}
\end{table}

\paragraph{Analysis:} The questionnaire reveals critical gaps in foundational security practices. The lack of MFA on computer logins significantly increases the risk of lateral movement should an attacker gain a foothold. Furthermore, the complete absence of a security awareness training program and an Acceptable Use Policy leaves the organization highly vulnerable to social engineering attacks and insider threats.

% --- 4. Technical Scan Results ---
\section{Technical Scan Results}

A network scan was performed on the organization's external infrastructure. The scan identified the following open ports and services.

\begin{itemize}
    \item \textbf{Scan Target:} \seqsplit{\texttt{2001:db8::1}}
    \item \textbf{Scan Tool:} Nmap
\end{itemize}

\begin{table}[h!]
\centering
\begin{tabular}{l l l p{0.5\textwidth}}
\toprule
\textbf{Port} & \textbf{State} & \textbf{Service} & \textbf{Notes} \\
\midrule
22/tcp & open & SSH & The Secure Shell service is exposed to the public internet. This service is a common target for brute-force and credential stuffing attacks. No version information was obtained, but any exposed management protocol presents a significant risk. \\
\bottomrule
\end{tabular}
\caption{Open Port Findings}
\end{table}

% --- 5. Risk Assessment ---
\section{Risk Assessment}

The following table synthesizes findings from the security questionnaire and technical scan. No pre-existing vulnerabilities were reported.

\begin{table}[h!]
\centering
\begin{tabular}{p{0.1\textwidth} p{0.25\textwidth} p{0.45\textwidth} l}
\toprule
\textbf{Risk ID} & \textbf{Risk Name} & \textbf{Description} & \textbf{Severity} \\
\midrule
RISK-001 & Inadequate Access Control & MFA is not enforced for computer logins. A compromised password could grant an attacker direct access to an endpoint, facilitating lateral movement and data exfiltration. & \textbf{High} \\
\addlinespace
RISK-002 & Inadequate Security Awareness Program & The organization does not provide security training. Employees are likely unable to recognize and report phishing attempts or other social engineering tactics, making them a primary target. & \textbf{High} \\
\addlinespace
RISK-003 & Exposed Management Service & The SSH service (port 22) is publicly accessible. This allows attackers to directly attempt to compromise a critical administrative interface from anywhere on the internet. & \textbf{High} \\
\addlinespace
RISK-004 & Missing Foundational Policies & The lack of an Acceptable Use Policy (AUP) means there are no formal rules governing the use of company IT assets, leading to inconsistent security practices and ambiguity. & \textbf{Medium} \\
\bottomrule
\end{tabular}
\caption{Summary of Identified Risks}
\end{table}

% --- 6. Recommendations ---
\section{Recommendations}

Based on the analysis, the following actions are recommended to mitigate the identified risks and improve the overall security posture of \textbf{Iron Oak Furniture}.

\begin{description}
    \item[For RISK-001 (Inadequate Access Control):]
    \begin{itemize}
        \item \textbf{Immediate Action:} Implement and enforce MFA for all employee computer and laptop logins using a solution compatible with your operating systems.
        \item \textbf{Strategic Action:} Review all systems, including internal applications, and enforce MFA wherever possible, especially for privileged access.
    \end{itemize}
    
    \item[For RISK-002 (Inadequate Security Awareness):]
    \begin{itemize}
        \item \textbf{Immediate Action:} Enroll all employees in a reputable security awareness training program. Conduct a baseline phishing simulation to measure initial susceptibility.
        \item \textbf{Strategic Action:} Formalize the program to include mandatory training for new hires and annual refresher courses for all staff.
    \end{itemize}
    
    \item[For RISK-003 (Exposed Management Service):]
    \begin{itemize}
        \item \textbf{Immediate Action:} If SSH access is not required from the public internet, block port 22 at the network firewall immediately.
        \item \textbf{Strategic Action:} If external access is required, restrict it to known, trusted IP addresses via firewall rules. Implement a bastion host or VPN for administrative access. Enforce public key authentication and disable password-based logins for SSH.
    \end{itemize}
    
    \item[For RISK-004 (Missing Foundational Policies):]
    \begin{itemize}
        \item \textbf{Immediate Action:} Develop and ratify a formal Acceptable Use Policy (AUP). The policy should clearly define the rules for using company networks, devices, and data.
        \item \textbf{Strategic Action:} Require all employees to read and acknowledge the AUP as a condition of using company resources.
    \end{itemize}
\end{description}

\end{document}
```