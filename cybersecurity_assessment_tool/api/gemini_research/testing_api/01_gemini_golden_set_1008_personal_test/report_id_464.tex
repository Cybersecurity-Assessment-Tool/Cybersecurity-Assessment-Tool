```latex
\documentclass[12pt]{article}

% Preamble: Required Packages
\usepackage[margin=1in]{geometry}
\usepackage{pifont} % For checkmarks and crosses (\ding{51} and \ding{55})
\usepackage{booktabs} % For professional-looking tables
\usepackage{hyperref} % For clickable links and references
\usepackage{url}      % For formatting URLs
\usepackage{seqsplit} % For breaking long strings in texttt
\usepackage[T1]{fontenc}
\usepackage{fancyhdr} % For headers and footers

% Document Information
\title{Cybersecurity Posture Assessment Report \\ \large For: \textbf{Quantum Reach}}
\author{Cybersecurity Analysis Division}
\date{\today}

% Header and Footer Configuration
\pagestyle{fancy}
\fancyhf{} % Clear all header and footer fields
\fancyhead[L]{\textbf{Quantum Reach} // Cybersecurity Assessment}
\fancyfoot[C]{\thepage}

\begin{document}

\maketitle
\thispagestyle{empty}
\tableofcontents
\newpage

\section*{Executive Summary}

This report provides a comprehensive cybersecurity assessment for \textbf{Quantum Reach}, based on an analysis of network scan data, organizational security controls, and existing risk registers. The assessment was conducted on \today.

The analysis revealed several high-risk security gaps in current policies and technical configurations. Key findings include critical deficiencies in access control, employee security awareness, and network service security. Specifically, the absence of Multi-Factor Authentication (MFA) for sensitive systems, the lack of an Acceptable Use Policy (AUP), and incomplete security training for new employees represent significant organizational risks.

From a technical perspective, the external network scan identified an open HTTP port (80), which exposes web traffic to interception and eavesdropping. This is a critical vulnerability that could lead to the compromise of sensitive data or user credentials.

This report details these findings and provides actionable, prioritized recommendations to mitigate the identified risks and strengthen the overall security posture of \textbf{Quantum Reach}. We urge management to review these recommendations and implement them promptly.

\section{Organizational Information}

The following information was provided by the client and used as a baseline for this assessment.

\begin{table}[h!]
\centering
\begin{tabular}{@{}ll@{}}
\toprule
\textbf{Attribute} & \textbf{Value} \\ \midrule
Organization Name & \textbf{Quantum Reach} \\
Primary Email Domain & \texttt{QuantumReach.net} \\
Primary Website Domain & \url{www.QuantumReach.net} \\
External IP Address & \texttt{7.190.25.214} \\ \bottomrule
\end{tabular}
\caption{Client Profile}
\label{tab:client_profile}
\end{table}

\section{Security Control Review}

A review of the organization's security controls was conducted via a standardized questionnaire. The responses highlight critical gaps in the current security framework. "No" answers indicate a deviation from security best practices and are flagged as high-risk areas requiring immediate attention.

\begin{table}[h!]
\centering
\begin{tabular}{@{}p{0.6\linewidth}cc@{}}
\toprule
\textbf{Control Question} & \textbf{Response} & \textbf{Assessment} \\ \midrule
Do you require MFA to access email? & \ding{51} (Yes) & Compliant \\
Do you require MFA to log into computers? & \ding{51} (Yes) & Compliant \\
Do you require MFA to access sensitive data systems? & \ding{55} (No) & \textbf{High Risk Gap} \\
Does your organization have an employee acceptable use policy? & \ding{55} (No) & \textbf{High Risk Gap} \\
Does your organization do security awareness training for new employees? & \ding{55} (No) & \textbf{High Risk Gap} \\
Does your organization do security awareness training for all employees at least once per year? & \ding{51} (Yes) & Compliant \\ \bottomrule
\end{tabular}
\caption{Security Controls Questionnaire Analysis}
\label{tab:controls_analysis}
\end{table}

\section{Technical Scan Results}

An external network scan was performed against the target IP address \texttt{172.16.0.1}. The scan identified the following open ports and services.

\begin{table}[h!]
\centering
\begin{tabular}{@{}llll@{}}
\toprule
\textbf{Port} & \textbf{State} & \textbf{Inferred Service} & \textbf{Finding} \\ \midrule
80/tcp & Open & HTTP & \textbf{Critical Risk.} Unencrypted web traffic. \\ \bottomrule
\end{tabular}
\caption{Open Port Analysis}
\label{tab:port_analysis}
\end{table}

\subsection*{Analysis of Technical Findings}
The presence of an open port 80 (HTTP) is a critical security vulnerability. The HTTP protocol transmits data in cleartext, making it susceptible to interception by malicious actors. Any information, including user credentials, session cookies, or sensitive data exchanged with the web server, can be captured and read. This configuration fails to meet modern security standards, which mandate the use of encrypted channels (HTTPS) for all web communications.

\section{Risk Assessment and Findings}

This section correlates the findings from the security control review and the technical scan to present a consolidated list of identified risks. The pre-existing risk data provided in \texttt{Input\_3\_Current\_Risks\_JSON} was found to contain invalid entries and has been excluded from this analysis to ensure accuracy.

\begin{table}[h!]
\centering
\begin{tabular}{@{}p{0.05\linewidth}p{0.4\linewidth}p{0.25\linewidth}l@{}}
\toprule
\textbf{ID} & \textbf{Risk Description} & \textbf{Source} & \textbf{Severity} \\ \midrule
\textbf{R-01} & \textbf{Unencrypted Web Traffic (HTTP)}: Data transmitted to and from the web server is in cleartext and can be easily intercepted. & Technical Scan & \textbf{Critical} \\
\textbf{R-02} & \textbf{No MFA for Sensitive Systems}: Lack of MFA on critical data systems allows for single-factor authentication, making them vulnerable to credential theft. & Questionnaire & \textbf{High} \\
\textbf{R-03} & \textbf{No Employee Acceptable Use Policy (AUP)}: Absence of a formal AUP creates ambiguity regarding secure practices and limits enforceability. & Questionnaire & \textbf{High} \\
\textbf{R-04} & \textbf{No Security Training for New Hires}: New employees are not trained on security policies upon joining, making them a primary target for social engineering. & Questionnaire & \textbf{High} \\ \bottomrule
\end{tabular}
\caption{Consolidated Risk Register}
\label{tab:risk_register}
\end{table}

\section{Recommendations}

The following actions are recommended to mitigate the identified risks. Recommendations are prioritized based on severity.

\subsection*{R-01: Remediate Unencrypted Web Traffic (Critical)}
\begin{itemize}
    \item \textbf{Immediate Action:} Migrate the web service running on port 80 to HTTPS (port 443). This requires obtaining and installing a valid TLS/SSL certificate from a trusted Certificate Authority (e.g., Let's Encrypt, DigiCert).
    \item \textbf{Configuration:} Configure the web server to automatically redirect all HTTP traffic to HTTPS to ensure all connections are encrypted.
    \item \textbf{Policy:} Implement HTTP Strict Transport Security (HSTS) to enforce the use of HTTPS in browsers.
\end{itemize}

\subsection*{R-02: Implement MFA for Sensitive Systems (High)}
\begin{itemize}
    \item \textbf{Immediate Action:} Conduct an inventory of all systems that store, process, or transmit sensitive data.
    \item \textbf{Implementation:} Enable and enforce MFA for all user accounts (including administrative accounts) that have access to these identified systems.
\end{itemize}

\subsection*{R-03: Develop and Implement an AUP (High)}
\begin{itemize}
    \item \textbf{Action:} Draft a formal Acceptable Use Policy that clearly defines the rules and responsibilities for all employees when using company IT assets.
    \item \textbf{Distribution:} Require all current employees to read and formally acknowledge the policy. Integrate this process into the onboarding for new hires.
\end{itemize}

\subsection*{R-04: Mandate Security Training for New Hires (High)}
\begin{itemize}
    \item \textbf{Action:} Integrate a mandatory security awareness training module into the new employee onboarding process. This training should occur within the first week of employment.
    \item \textbf{Content:} The training should cover key topics such as phishing, password security, data handling, and the new AUP.
\end{itemize}

\section{Conclusion}
The assessment identified significant, high-impact risks within \textbf{Quantum Reach}'s security posture. While the organization has implemented some foundational controls like MFA for email, critical gaps remain in policy and technical enforcement. The recommendations provided in this report are designed to be actionable and to produce a measurable improvement in security. We strongly advise that these recommendations, particularly the remediation of the unencrypted web service, be addressed with the highest priority to protect the organization's data and reputation.

\end{document}
```