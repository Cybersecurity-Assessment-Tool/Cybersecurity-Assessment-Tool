```latex
\documentclass[12pt]{article}

% Preamble: Required Packages
\usepackage[margin=1in]{geometry}
\usepackage{pifont} % For checkmarks and crosses
\usepackage{booktabs} % For professional tables
\usepackage[colorlinks=true, urlcolor=blue]{hyperref}
\usepackage{url}
\usepackage{seqsplit} % For splitting long strings like URLs or hashes
\usepackage{xcolor} % For colors
\usepackage{graphicx}

% Define custom colors for severity
\definecolor{sev_critical}{HTML}{990000}
\definecolor{sev_high}{HTML}{DD4B39}

% --- Document Start ---
\begin{document}

% --- Title Page ---
\title{Cybersecurity Posture Assessment Report \\ \large For Urban Jungle Planning}
\author{Cybersecurity Analyst Group}
\date{\today}
\maketitle
\thispagestyle{empty}
\newpage

\tableofcontents
\newpage

% --- Section 1: Executive Summary ---
\section{Executive Summary}

This report provides a comprehensive cybersecurity assessment for Urban Jungle Planning, based on the analysis of network scan data, organizational security controls, and pre-existing risk information. The assessment was conducted to identify key vulnerabilities, security gaps, and areas of non-compliance with cybersecurity best practices.

The analysis revealed several critical and high-risk findings that require immediate attention. Key issues include:

\begin{itemize}
    \item \textbf{Systemic Network Vulnerabilities:} The technical scan identified an open Remote Desktop Protocol (RDP) port on host \texttt{10.10.10.51}. When correlated with pre-existing risk data, which notes a similar exposure on \texttt{10.10.10.50}, this indicates a systemic and dangerous misconfiguration. RDP is a primary vector for ransomware attacks.
    
    \item \textbf{Critical Gaps in Access Control:} The organization does not enforce Multi-Factor Authentication (MFA) for accessing email or sensitive data systems. This significantly increases the risk of account compromise and subsequent data breaches.
    
    \item \textbf{Deficiencies in Security Governance:} There is no formal employee Acceptable Use Policy, and mandatory security awareness training is not conducted annually for all employees. These policy gaps weaken the organization's human firewall and increase susceptibility to social engineering and insider threats.
\end{itemize}

These findings collectively create a high-risk environment. This report provides a detailed breakdown of each risk and offers prioritized, actionable recommendations to mitigate them and improve the overall security posture of Urban Jungle Planning.

% --- Section 2: Organizational Information ---
\section{Organizational Information}

The following details were provided for the assessment.

\begin{itemize}
    \item \textbf{Organization Name:} Urban Jungle Planning
    \item \textbf{Email Domain:} \texttt{UrbanJunglePlanning.com}
    \item \textbf{Website Domain:} \url{www.UrbanJunglePlanning.com}
    \item \textbf{Known External IP:} \texttt{43.200.20.253}
\end{itemize}

% --- Section 3: Security Control Review ---
\section{Security Control Review (Questionnaire Analysis)}

A review of the organization's security controls via a questionnaire identified significant gaps in policy and procedure. A "No" response indicates a deviation from security best practices and represents a potential risk.

\begin{table}[h!]
\centering
\caption{Security Controls Questionnaire Results}
\begin{tabular}{p{0.6\linewidth} c p{0.2\linewidth}}
\toprule
\textbf{Control Question} & \textbf{Response} & \textbf{Analyst Assessment} \\
\midrule
Do you require MFA to access email? & \textcolor{red}{\ding{55}} & Critical Gap \\
Do you require MFA to log into computers? & \textcolor{green}{\ding{51}} & Control in Place \\
Do you require MFA to access sensitive data systems? & \textcolor{red}{\ding{55}} & Critical Gap \\
Does your organization have an employee acceptable use policy? & \textcolor{red}{\ding{55}} & High Risk \\
Does your organization do security awareness training for new employees? & \textcolor{green}{\ding{51}} & Control in Place \\
Does your organization do security awareness training for all employees at least once per year? & \textcolor{red}{\ding{55}} & High Risk \\
\bottomrule
\end{tabular}
\end{table}

The absence of MFA for email and sensitive data is a critical vulnerability. Email accounts are frequent targets for attackers, and a compromise could lead to widespread system access. The lack of an Acceptable Use Policy and annual training creates an environment where employees may be unaware of their security responsibilities.

% --- Section 4: Technical Scan Results ---
\section{Technical Scan Results}

An external network scan was performed to identify open ports and exposed services on the target system.

\subsection{Nmap Scan on Target: \texttt{10.10.10.51}}
The scan revealed the following open port:

\begin{table}[h!]
\centering
\caption{Open Ports on \texttt{10.10.10.51}}
\begin{tabular}{c c l}
\toprule
\textbf{Port} & \textbf{State} & \textbf{Service Name} \\
\midrule
3389/tcp & open & ms-wbt-server (Remote Desktop Protocol) \\
\bottomrule
\end{tabular}
\end{table}

\paragraph{Analysis:}
The presence of an open port 3389/TCP indicates that Microsoft's Remote Desktop Protocol (RDP) is accessible on this host. RDP is a common target for brute-force attacks and exploitation of vulnerabilities (e.g., BlueKeep). When correlated with the existing risk of RDP exposure on host \texttt{10.10.10.50}, this finding suggests a pattern of insecure remote access configurations within the network.

% --- Section 5: Consolidated Risk Assessment ---
\section{Consolidated Risk Assessment}

The following table synthesizes findings from the technical scan, the controls review, and pre-existing risk data into a prioritized list.

\begin{table}[h!]
\centering
\caption{Summary of Identified Risks}
\begin{tabular}{p{0.1\linewidth} p{0.2\linewidth} p{0.4\linewidth} p{0.15\linewidth}}
\toprule
\textbf{Risk ID} & \textbf{Risk Title} & \textbf{Description} & \textbf{Severity} \\
\midrule
RISK-001 & Lack of MFA for Email Access & User email accounts are not protected by a second factor of authentication, making them vulnerable to phishing and credential stuffing attacks. & \textcolor{sev_critical}{\textbf{Critical}} \\
\addlinespace
RISK-002 & Lack of MFA for Sensitive Data & Systems containing sensitive data can be accessed with only a password, creating a high risk of data exfiltration if credentials are compromised. & \textcolor{sev_critical}{\textbf{Critical}} \\
\addlinespace
RISK-003 & Systemic RDP Exposure & RDP is exposed on hosts \texttt{10.10.10.50} and \texttt{10.10.10.51}, making them prime targets for ransomware and unauthorized access. & \textcolor{sev_critical}{\textbf{Critical}} \\
\addlinespace
RISK-004 & No Acceptable Use Policy (AUP) & The absence of a formal AUP means there are no clear rules for employees regarding technology and data handling, increasing the risk of misuse. & \textcolor{sev_high}{\textbf{High}} \\
\addlinespace
RISK-005 & No Annual Security Training & Security knowledge degrades over time. Without annual training, employees are more likely to fall victim to evolving social engineering tactics. & \textcolor{sev_high}{\textbf{High}} \\
\bottomrule
\end{tabular}
\end{table}

% --- Section 6: Recommendations ---
\section{Recommendations}

The following actions are recommended to mitigate the identified risks and strengthen the security posture of Urban Jungle Planning. Recommendations are prioritized based on severity.

\subsection{Critical Priority Actions (Immediate)}
\begin{enumerate}
    \item \textbf{Remediate RDP Exposure (RISK-003):}
    \begin{itemize}
        \item Immediately disable external access to TCP port 3389 on hosts \texttt{10.10.10.50}, \texttt{10.10.10.51}, and any other systems where it is not strictly required.
        \item If remote administrative access is necessary, implement a secure Virtual Private Network (VPN) solution that requires MFA for access.
    \end{itemize}

    \item \textbf{Implement Multi-Factor Authentication (RISK-001, RISK-002):}
    \begin{itemize}
        \item Enforce MFA for all user accounts across all email platforms (e.g., Office 365, G Suite).
        \item Identify all systems containing sensitive data and enforce MFA for all access, both privileged and standard.
    \end{itemize}
\end{enumerate}

\subsection{High Priority Actions (Near-Term)}
\begin{enumerate}
    \setcounter{enumi}{2} % Continue numbering from previous list
    \item \textbf{Develop and Implement an Acceptable Use Policy (RISK-004):}
    \begin{itemize}
        \item Draft a formal AUP that clearly defines the rules for using company technology, networks, and data.
        \item Ensure all current employees read and acknowledge the policy, and incorporate it into the onboarding process for new hires.
    \end{itemize}

    \item \textbf{Establish Annual Security Awareness Training (RISK-005):}
    \begin{itemize}
        \item Implement a mandatory security awareness training program for all employees to be completed annually.
        \item Training should cover key topics such as phishing identification, password hygiene, and the new Acceptable Use Policy.
    \end{itemize}
\end{enumerate}

% --- Document End ---
\end{document}
```