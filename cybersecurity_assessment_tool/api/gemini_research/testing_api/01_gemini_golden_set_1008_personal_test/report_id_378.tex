```latex
\documentclass[12pt]{article}

% Preamble: Required Packages
\usepackage[margin=1in]{geometry}
\usepackage{pifont} % For checkmarks and crosses
\usepackage{booktabs} % For professional tables
\usepackage{hyperref} % For clickable links
\usepackage{url} % For formatting URLs
\usepackage{seqsplit} % For splitting long strings in tt font
\usepackage{graphicx}
\usepackage{xcolor}

% Document Metadata
\title{Cybersecurity Posture Assessment Report}
\author{Cybersecurity Analyst}
\date{\today}

% Hyperref Setup
\hypersetup{
    colorlinks=true,
    linkcolor=blue,
    filecolor=magenta,      
    urlcolor=cyan,
    pdftitle={Cybersecurity Posture Assessment Report},
    pdfpagemode=FullScreen,
}

\begin{document}

\maketitle
\thispagestyle{empty}
\newpage

\tableofcontents
\newpage

% --- 1. Executive Overview ---
\section{Executive Overview}

This report details the findings of a cybersecurity posture assessment for \textbf{Harbor Light Foundation}. The analysis is based on a combination of network scanning, a review of organizational security controls, and an evaluation of pre-existing risk data.

The assessment reveals a mixed security posture. On a positive note, a previously identified risk concerning an unencrypted web server on port 80 appears to have been remediated, as the corresponding port was found to be closed during our technical scan.

However, significant and critical gaps were identified in the organization's access control and employee security policies. The absence of multi-factor authentication (MFA) for computer and sensitive data system access represents a high risk of unauthorized access. Furthermore, the lack of a formal Acceptable Use Policy and mandatory security training for new employees creates vulnerabilities related to insider threats and human error.

Immediate action is required to address these policy and access control deficiencies to mitigate substantial risks to the organization's data and systems.

% --- 2. Organizational Information ---
\section{Organizational Information}

The following details were provided for the assessment.

\begin{itemize}
    \item \textbf{Organization Name:} Harbor Light Foundation
    \item \textbf{Email Domain:} \seqsplit{\texttt{HarborLightFoundation.com}}
    \item \textbf{Website Domain:} \seqsplit{\url{www.HarborLightFoundation.com}}
    \item \textbf{External IP Address:} \seqsplit{\texttt{44.97.214.90}}
\end{itemize}

% --- 3. Security Control Review ---
\section{Security Control Review}

A review of the organization's security controls was conducted via a questionnaire. The responses highlight critical areas requiring immediate attention. A checkmark (\ding{51}) indicates a positive control is in place, while a cross (\ding{55}) indicates a control gap.

\begin{table}[h!]
\centering
\caption{Organizational Security Control Status}
\begin{tabular}{p{0.7\textwidth} c}
\toprule
\textbf{Control Question} & \textbf{Status} \\
\midrule
Do you require MFA to access email? & \textcolor{green}{\ding{51}} \\
Do you require MFA to log into computers? & \textcolor{red}{\ding{55}} \\
Do you require MFA to access sensitive data systems? & \textcolor{red}{\ding{55}} \\
Does your organization have an employee acceptable use policy? & \textcolor{red}{\ding{55}} \\
Does your organization do security awareness training for new employees? & \textcolor{red}{\ding{55}} \\
Does your organization do security awareness training for all employees at least once per year? & \textcolor{green}{\ding{51}} \\
\bottomrule
\end{tabular}
\end{table}

\subsection*{Analysis of Control Gaps}
The lack of MFA on employee computers and sensitive data systems is a \textbf{High Risk}. This significantly increases the likelihood of a successful breach if an employee's credentials are compromised. The absence of an Acceptable Use Policy and security training for new hires are \textbf{Critical Gaps} that expose the organization to unmitigated risks from insider threats and accidental data exposure.

% --- 4. Technical Scan Results ---
\section{Technical Scan Results}

A network scan was performed to identify exposed services on the organization's infrastructure.

\begin{itemize}
    \item \textbf{Target IP Address:} \seqsplit{\texttt{192.168.0.5}}
    \item \textbf{Scan Date:} \today
\end{itemize}

\begin{table}[h!]
\centering
\caption{Network Port Scan Findings}
\begin{tabular}{c c c c}
\toprule
\textbf{Port} & \textbf{State} & \textbf{Service} & \textbf{Version} \\
\midrule
80 & closed & http & N/A \\
\bottomrule
\end{tabular}
\end{table}

\subsection*{Analysis of Technical Findings}
The scan indicates that the target host is online, but no open ports were discovered. Notably, Port 80 (HTTP), which was identified as a risk in previous assessments (\textit{Input\_3\_Current\_Risks\_JSON}), was found to be \textbf{closed}. This is a positive security development, suggesting the risk of an unencrypted web server has been successfully mitigated. No other vulnerabilities were identified from this scan.

% --- 5. Consolidated Risk Assessment ---
\section{Consolidated Risk Assessment}

The following table synthesizes findings from the security control review, technical scan, and pre-existing risk data into a consolidated list of current risks.

\begin{table}[h!]
\centering
\caption{Summary of Identified Risks}
\begin{tabular}{p{0.2\textwidth} p{0.5\textwidth} l}
\toprule
\textbf{Risk Name} & \textbf{Description} & \textbf{Severity} \\
\midrule
\textbf{Lack of Endpoint MFA} & User computers do not require MFA for login, exposing them to takeover if credentials are stolen. & \textbf{High} \\
\addlinespace
\textbf{Inadequate Data Access Controls} & Sensitive data systems can be accessed without MFA, failing to protect critical information assets. & \textbf{High} \\
\addlinespace
\textbf{Missing Acceptable Use Policy} & No formal policy exists to govern the use of company assets, leading to potential misuse and insider threats. & \textbf{Medium} \\
\addlinespace
\textbf{Insufficient Employee Onboarding} & New employees do not receive security awareness training, making them more susceptible to phishing and social engineering. & \textbf{Medium} \\
\addlinespace
Unencrypted Web Server & Port 80 was previously identified as open. The current scan shows this port is now closed. & \textcolor{green}{Mitigated} \\
\bottomrule
\end{tabular}
\end{table}

% --- 6. Recommendations ---
\section{Recommendations}

Based on the consolidated risk assessment, the following actions are recommended to improve the security posture of \textbf{Harbor Light Foundation}.

\subsection*{High Priority}
\begin{enumerate}
    \item \textbf{Implement MFA Across All Systems:} Immediately deploy a mandatory MFA policy for logging into all employee computers and for accessing any systems containing sensitive or critical data. This is the single most effective control to mitigate credential theft.
    \item \textbf{Develop and Enforce an Acceptable Use Policy (AUP):} Create a formal AUP that all employees must read and sign. This policy should clearly define the rules for using company networks, devices, and data.
\end{enumerate}

\subsection*{Medium Priority}
\begin{enumerate}
    \item \textbf{Integrate Security Training into Onboarding:} Develop a security awareness training module and make it a mandatory part of the onboarding process for all new employees before they are granted system access.
\end{enumerate}

\subsection*{Verification}
\begin{enumerate}
    \item \textbf{Confirm Port 80 Closure:} While the scan shows Port 80 is closed, it is recommended to formally verify with the IT team that this was an intentional change and that the underlying service is no longer required. Document this change for future reference.
\end{enumerate}

\end{document}
```