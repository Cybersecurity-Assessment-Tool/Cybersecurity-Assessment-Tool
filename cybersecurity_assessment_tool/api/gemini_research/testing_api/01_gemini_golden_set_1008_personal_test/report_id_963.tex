```latex
\documentclass[12pt]{article}

% Preamble: Required Packages and Document Setup
\usepackage[margin=1in]{geometry}
\usepackage{pifont} % For checkmarks and crosses (\ding{51}, \ding{55})
\usepackage{booktabs} % For professional-looking tables
\usepackage{hyperref} % For clickable links
\usepackage{url} % For formatting URLs
\usepackage{seqsplit} % For splitting long strings in texttt
\usepackage{xcolor} % For colors
\usepackage{fancyhdr} % For headers/footers
\usepackage{graphicx} % For images (e.g., logo)

% Define colors for severity
\definecolor{criticalred}{HTML}{D73B3E}
\definecolor{highorange}{HTML}{F08C00}
\definecolor{mediumyellow}{HTML}{F0E68C}
\definecolor{lowblue}{HTML}{3498DB}

% Hyperref Setup
\hypersetup{
    colorlinks=true,
    linkcolor=blue,
    filecolor=magenta,      
    urlcolor=cyan,
    pdftitle={Cybersecurity Posture Assessment Report},
    pdfpagemode=FullScreen,
}

% Header and Footer
\pagestyle{fancy}
\fancyhf{}
\fancyhead[L]{Cybersecurity Posture Assessment}
\fancyhead[R]{For: Clear Path}
\fancyfoot[C]{\thepage}

% Document Title
\title{Cybersecurity Posture Assessment Report \\ \large For: Clear Path}
\author{Cybersecurity Analysis Division}
\date{\today}

\begin{document}

\maketitle
\thispagestyle{empty}
\newpage

\tableofcontents
\newpage

% --- 1. Executive Overview ---
\section{Executive Overview}

This report details the findings of a cybersecurity posture assessment for \textbf{Clear Path}, conducted on \today. The assessment combined a technical network scan, a review of existing risks, and an analysis of organizational security controls based on a questionnaire.

The overall security posture is assessed as \textbf{High-Risk}. Several critical vulnerabilities and control gaps were identified that expose the organization to significant threats, including data breaches, unauthorized access, and service disruption.

Key findings include:
\begin{itemize}
    \item \textbf{Critical Database Exposure:} A MySQL database (\texttt{172.16.50.20:3306}) is directly exposed to the network. The service is running MySQL version 5.7.33, which is an \textbf{End-of-Life (EOL)} product and no longer receives security updates. This finding from the technical scan directly corroborates a pre-existing known risk.
    \item \textbf{Insufficient Access Controls:} Multi-Factor Authentication (MFA) is not enforced for accessing email or sensitive data systems. This significantly increases the risk of account compromise and subsequent unauthorized data access.
    \item \textbf{Foundational Policy Gaps:} The organization lacks a formal Acceptable Use Policy (AUP) and does not provide security awareness training for new employees, indicating gaps in security governance and culture.
\end{itemize}

Immediate remediation is required to address these findings. Recommendations are provided in Section \ref{sec:recommendations} to mitigate the identified risks and improve the organization's overall security posture.

% --- 2. Organizational Information ---
\section{Organizational Information}

The following information was provided for the assessment.

\begin{tabular}{@{}ll}
    \toprule
    \textbf{Attribute} & \textbf{Value} \\
    \midrule
    Organization Name & \textbf{Clear Path} \\
    Email Domain & \texttt{ClearPath.org} \\
    Website Domain & \url{www.ClearPath.org} \\
    External IP Address & \texttt{211.90.90.117} \\
    Internal Target IP & \texttt{172.16.50.20} \\
    \bottomrule
\end{tabular}

% --- 3. Security Control Review ---
\section{Security Control Review (Questionnaire Analysis)}

An analysis of the security questionnaire reveals significant gaps in fundamental security controls. "No" answers indicate a lack of a necessary control and are flagged as high-impact risks.

\begin{table}[h!]
\centering
\caption{Security Controls Questionnaire Results}
\begin{tabular}{@{}p{0.7\linewidth} c c}
    \toprule
    \textbf{Control Question} & \textbf{Response} & \textbf{Status} \\
    \midrule
    Do you require MFA to access email? & No & \ding{55} \\
    Do you require MFA to log into computers? & Yes & \ding{51} \\
    Do you require MFA to access sensitive data systems? & No & \ding{55} \\
    Does your organization have an employee acceptable use policy? & No & \ding{55} \\
    Does your organization do security awareness training for new employees? & No & \ding{55} \\
    Does your organization do security awareness training for all employees at least once per year? & Yes & \ding{51} \\
    \bottomrule
\end{tabular}
\end{table}

\subsection*{Analysis of Control Gaps}
The lack of MFA on email and sensitive systems is a critical vulnerability. Email is a primary target for phishing attacks, and a compromised account can be a gateway to the entire organization. The absence of an Acceptable Use Policy and new-hire training indicates a reactive, rather than proactive, approach to security.

% --- 4. Technical Scan Results ---
\section{Technical Scan Results}

A network scan was performed on the target IP address \texttt{172.16.50.20}. The scan identified one open port with a publicly exposed service.

\begin{table}[h!]
\centering
\caption{Open Port Scan Results for \texttt{172.16.50.20}}
\begin{tabular}{@{}lllll@{}}
    \toprule
    \textbf{Port} & \textbf{State} & \textbf{Service} & \textbf{Product} & \textbf{Version} \\
    \midrule
    3306/tcp & open & mysql & MySQL & 5.7.33 \\
    \bottomrule
\end{tabular}
\end{table}

\subsection*{Analysis of Technical Findings}
The scan confirms that a MySQL database is directly accessible on port 3306. This configuration is highly discouraged as it exposes the database to brute-force attacks, credential stuffing, and exploitation of vulnerabilities.

Furthermore, the detected version, \textbf{MySQL 5.7.33}, reached its official End-of-Life (EOL) in October 2023. EOL software no longer receives security patches from the vendor, meaning any newly discovered vulnerabilities will remain unpatched, posing a severe and unmitigable risk to the data stored within.

% --- 5. Consolidated Risk Assessment ---
\section{Consolidated Risk Assessment}

The following table synthesizes findings from the technical scan, the control review, and pre-existing risk data into a prioritized list.

\begin{table}[h!]
\centering
\caption{Summary of Identified Risks}
\begin{tabular}{@{}p{0.15\linewidth} p{0.55\linewidth} p{0.2\linewidth}@{}}
    \toprule
    \textbf{Risk Name} & \textbf{Description} & \textbf{Severity} \\
    \midrule
    Exposed \& Outdated Database Service & MySQL port 3306 is open to the network, and the service (v5.7.33) is End-of-Life, making it vulnerable to unpatched exploits. This directly confirms a known risk. & \colorbox{criticalred}{\color{white}\textbf{CRITICAL (9.8)}} \\
    \addlinespace
    Lack of Critical MFA Enforcement & MFA is not required for email or sensitive data systems. This exposes the organization to account takeovers and subsequent data breaches. & \colorbox{criticalred}{\color{white}\textbf{CRITICAL (9.1)}} \\
    \addlinespace
    Insufficient Security Policies \& Training & The absence of an Acceptable Use Policy and security training for new hires creates an environment where employees are more likely to engage in risky behavior. & \colorbox{highorange}{\color{white}\textbf{HIGH (7.2)}} \\
    \bottomrule
\end{tabular}
\end{table}

% --- 6. Recommendations ---
\section{Recommendations}
\label{sec:recommendations}

The following actions are recommended to mitigate the identified risks. Recommendations are prioritized based on severity and ease of implementation.

\subsection{Risk: Exposed \& Outdated Database Service}
\begin{itemize}
    \item \textbf{Immediate (Containment):} Implement strict firewall rules to deny all public access to TCP port 3306. Access should be restricted to a limited set of trusted internal IP addresses only.
    \item \textbf{Short-Term (Remediation):} Plan and execute an upgrade of the MySQL 5.7 database to a currently supported version (e.g., MySQL 8.x). This is critical to ensure security patches are available.
    \item \textbf{Long-Term (Architectural):} Re-architect the network so that the database server resides in a private, non-internet-facing network segment. Access should only be permitted through a secure bastion host or a VPN connection.
\end{itemize}

\subsection{Risk: Lack of Critical MFA Enforcement}
\begin{itemize}
    \item \textbf{Immediate (Remediation):} Procure and enforce an MFA solution for the organization's email system (\texttt{ClearPath.org}).
    \item \textbf{Short-Term (Remediation):} Identify all systems classified as containing sensitive data and enforce MFA for all user access, both privileged and standard.
\end{itemize}

\subsection{Risk: Insufficient Security Policies \& Training}
\begin{itemize}
    \item \textbf{Immediate (Foundation):} Develop and implement a mandatory security awareness training module as part of the new employee onboarding process.
    \item \textbf{Short-Term (Governance):} Draft, approve, and disseminate a formal Acceptable Use Policy (AUP) that all employees must read and acknowledge. This policy should clearly define the rules for using company IT assets.
\end{itemize}

\end{document}
```