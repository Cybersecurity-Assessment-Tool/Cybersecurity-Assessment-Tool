```latex
\documentclass[12pt]{article}

% --- PACKAGE IMPORTS ---
\usepackage[margin=1in]{geometry} % For setting page margins
\usepackage{pifont}               % For checkmark and cross symbols (\ding)
\usepackage{booktabs}             % For professional-looking tables
\usepackage{hyperref}             % For clickable links and metadata
\usepackage{url}                  % For formatting URLs
\usepackage{seqsplit}             % To split long strings in texttt
\usepackage{graphicx}             % For logos, etc. (optional but good practice)
\usepackage{xcolor}               % For custom colors

% --- DOCUMENT METADATA ---
\hypersetup{
    colorlinks=true,
    linkcolor=blue,
    filecolor=magenta,      
    urlcolor=cyan,
    pdftitle={Cybersecurity Posture Assessment Report},
    pdfauthor={Cybersecurity Analysis Division},
    pdfsubject={Security Assessment},
    pdfkeywords={Cybersecurity, Risk, Analysis},
    bookmarks=true
}

% --- DEFINE COLORS ---
\definecolor{darkred}{rgb}{0.55, 0.0, 0.0}
\definecolor{darkgreen}{rgb}{0.0, 0.39, 0.0}

% --- DOCUMENT START ---
\begin{document}

% --- TITLE PAGE ---
\title{
    \vspace{2cm}
    \textbf{Cybersecurity Posture Assessment Report} \\
    \large \textit{Prepared for: \textbf{North Star Education}}
    \vspace{1cm}
}
\author{Cybersecurity Analysis Division}
\date{\today}
\maketitle
\thispagestyle{empty}
\newpage

% --- TABLE OF CONTENTS ---
\tableofcontents
\newpage

% --- EXECUTIVE SUMMARY ---
\section*{Executive Summary}
This report provides a cybersecurity posture assessment for \textbf{North Star Education}, based on an analysis of organizational data, security control questionnaires, and technical network scans. 

The assessment reveals several \textbf{critical and high-risk gaps} in fundamental security controls. The most severe finding is the complete absence of Multi-Factor Authentication (MFA) for accessing email, computers, and sensitive data systems. This exposes the organization to a significant risk of account compromise and unauthorized access. 

Furthermore, the lack of an employee acceptable use policy and security training for new hires indicates deficiencies in security governance and culture. While annual training is in place, the initial onboarding gap leaves new employees vulnerable.

It is crucial to note that the provided technical network scan data and the list of current organizational risks were corrupted and could not be analyzed. This report is therefore based solely on the organizational questionnaire. A complete technical assessment is not possible without a successful network scan.

Immediate remediation of the identified access control and policy gaps is strongly recommended to reduce the organization's attack surface and improve its overall security posture.

% --- ORGANIZATIONAL INFORMATION ---
\section*{1. Organizational Information}
This section details the organizational data provided for this assessment.

\begin{tabular}{@{}ll}
\toprule
\textbf{Attribute} & \textbf{Value} \\
\midrule
Organization Name & \textbf{North Star Education} \\
Primary Email Domain & \texttt{NorthStarEducation.com} \\
Primary Website & \url{www.NorthStarEducation.com} \\
External IP Address & \texttt{122.249.63.49} \\
\bottomrule
\end{tabular}

% --- SECURITY CONTROL REVIEW ---
\section*{2. Security Control Review}
The following table summarizes the organization's responses to a security controls questionnaire. Answers marked with a red cross (\textcolor{darkred}{\ding{55}}) indicate significant gaps in the security framework and are discussed in the Risk Assessment section.

\begin{table}[h!]
\centering
\begin{tabular}{@{}p{0.7\textwidth}c}
\toprule
\textbf{Control Question} & \textbf{Status} \\
\midrule
Do you require MFA to access email? & \textcolor{darkred}{\ding{55}} \\
Do you require MFA to log into computers? & \textcolor{darkred}{\ding{55}} \\
Do you require MFA to access sensitive data systems? & \textcolor{darkred}{\ding{55}} \\
Does your organization have an employee acceptable use policy? & \textcolor{darkred}{\ding{55}} \\
Does your organization do security awareness training for new employees? & \textcolor{darkred}{\ding{55}} \\
Does your organization do security awareness training for all employees at least once per year? & \textcolor{darkgreen}{\ding{51}} \\
\bottomrule
\end{tabular}
\caption{Security Controls Questionnaire Results}
\end{table}

% --- TECHNICAL SCAN RESULTS ---
\section*{3. Technical Scan Results}
This section is intended to detail findings from external network vulnerability scans.

\subsection*{Scan Data Integrity}
\textbf{Status: Data Not Available.}

The input data file for the network scan against target \texttt{[Target IP]} was found to be corrupted and could not be parsed. Therefore, no analysis of open ports, running services, or potential software vulnerabilities could be performed. Without this data, the organization's external technical attack surface remains unassessed.

\subsection*{Pre-existing Risk Data}
\textbf{Status: Data Not Available.}

Similarly, the input data file containing the organization's list of current known risks was also corrupted. This prevents a correlated analysis of pre-existing vulnerabilities against new findings.

% --- RISK ASSESSMENT ---
\section*{4. Risk Assessment}
The following risks have been identified based on the available data. The severity of these risks is high due to their fundamental nature. This list is considered incomplete pending the results of a successful technical scan.

\begin{table}[h!]
\centering
\begin{tabular}{@{}p{0.2\textwidth}p{0.6\textwidth}l}
\toprule
\textbf{Risk ID} & \textbf{Risk Name \& Overview} & \textbf{Severity} \\
\midrule
RISK-001 & \textbf{Lack of Multi-Factor Authentication (MFA)} \newline The absence of MFA on email, computer logins, and sensitive systems creates a critical vulnerability. A compromised password is all an attacker needs to gain significant access to organizational data. & \textbf{Critical} \\
\addlinespace
RISK-002 & \textbf{Absence of Acceptable Use Policy (AUP)} \newline Without a formal AUP, there are no clear guidelines for employees on the acceptable use of company assets. This increases the risk of insider threat, data misuse, and non-compliance. & \textbf{High} \\
\addlinespace
RISK-003 & \textbf{Inadequate New Hire Security Training} \newline Failing to provide security awareness training during onboarding means new employees may be unaware of policies and common threats like phishing, potentially exposing the organization from their first day. & \textbf{High} \\
\bottomrule
\end{tabular}
\caption{Identified Risks from Questionnaire Data}
\end{table}

% --- RECOMMENDATIONS ---
\section*{5. Recommendations}
Based on the analysis, the following actions are recommended to mitigate the identified risks and strengthen the security posture of \textbf{North Star Education}.

\subsection*{Immediate Priority (Critical)}
\begin{itemize}
    \item \textbf{Implement MFA (RISK-001):} Immediately begin a phased rollout of MFA across all systems.
    \begin{enumerate}
        \item \textbf{Phase 1:} Enforce MFA for email access (e.g., Office 365, Google Workspace).
        \item \textbf{Phase 2:} Enforce MFA for all remote access and access to sensitive data systems (e.g., student records, financial platforms).
        \item \textbf{Phase 3:} Enforce MFA for all computer logins.
    \end{enumerate}
\end{itemize}

\subsection*{High Priority}
\begin{itemize}
    \item \textbf{Develop and Implement an AUP (RISK-002):} Draft a formal Employee Acceptable Use Policy that covers topics such as data handling, internet usage, and personal device usage. Require all employees to read and acknowledge the policy.
    \item \textbf{Institute Onboarding Security Training (RISK-003):} Integrate a mandatory security awareness training module into the new employee onboarding process. This should cover the AUP, phishing identification, password hygiene, and incident reporting.
\end{itemize}

\subsection*{Action Items for Further Analysis}
\begin{itemize}
    \item \textbf{Re-run Network Scan:} A new, uncorrupted external network vulnerability scan must be performed against the public-facing IP address (\texttt{122.249.63.49}) to identify and assess technical vulnerabilities.
    \item \textbf{Provide Risk Register:} The list of current, known organizational risks should be provided to allow for a comprehensive and correlated threat landscape analysis.
\end{itemize}

% --- DOCUMENT END ---
\end{document}
```