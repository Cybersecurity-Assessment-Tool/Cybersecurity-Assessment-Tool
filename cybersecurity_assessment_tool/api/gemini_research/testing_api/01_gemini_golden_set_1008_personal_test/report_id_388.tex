```latex
\documentclass[12pt]{article}

% Preamble: Required Packages
\usepackage[margin=1in]{geometry}
\usepackage{pifont} % For checkmarks and crosses
\usepackage{booktabs} % For professional tables
\usepackage{hyperref} % For hyperlinks
\usepackage{url} % For URL formatting
\usepackage{seqsplit} % For splitting long text strings
\usepackage{graphicx}
\usepackage{xcolor}

% Document Information
\title{Cybersecurity Assessment Report \\ \large Prepared for Solid State}
\author{Cybersecurity Analyst}
\date{\today}

% Hyperref Setup
\hypersetup{
    colorlinks=true,
    linkcolor=blue,
    filecolor=magenta,      
    urlcolor=cyan,
    pdftitle={Cybersecurity Assessment Report},
    pdfpagemode=FullScreen,
}

\begin{document}

\maketitle
\thispagestyle{empty}
\newpage

\tableofcontents
\newpage

% ------------------------------------------------------------------------------
% 1. Executive Summary
% ------------------------------------------------------------------------------
\section*{1. Executive Summary}

This report details the findings of a cybersecurity assessment conducted for \textbf{Solid State}. The evaluation combined a technical network scan, a review of organizational security controls, and an analysis of pre-existing risks.

The assessment reveals a mixed security posture. \textbf{Solid State} demonstrates commendable strength in technical access controls, particularly with the comprehensive implementation of Multi-Factor Authentication (MFA) across key systems. The external network scan of the target host \texttt{192.168.1.100} found no open ports, indicating a robust firewall configuration and a minimal attack surface for that specific asset.

However, significant and critical gaps were identified in the organization's foundational security policies and employee training programs. The absence of an Acceptable Use Policy and a formal security awareness training program creates a high risk of security incidents originating from human error, such as falling victim to phishing or social engineering attacks. These procedural and administrative weaknesses currently represent the most significant threat to the organization's security and undermine its strong technical controls.

Immediate action is required to develop and implement the recommended policies and training programs to mitigate these critical risks.

% ------------------------------------------------------------------------------
% 2. Organizational Information
% ------------------------------------------------------------------------------
\section*{2. Organizational Information}

The following details were provided for the assessment.

\begin{itemize}
    \item \textbf{Organization Name:} Solid State
    \item \textbf{Email Domain:} \texttt{SolidState.com}
    \item \textbf{Website Domain:} \url{www.SolidState.com}
    \item \textbf{External IP Address:} \texttt{212.80.52.83}
\end{itemize}

% ------------------------------------------------------------------------------
% 3. Security Control Review
% ------------------------------------------------------------------------------
\section*{3. Security Control Review}

A review of administrative and technical security controls was conducted based on a standardized questionnaire. The responses indicate strong preventative measures in user authentication but reveal critical deficiencies in policy and user education.

\begin{table}[h!]
\centering
\caption{Security Control Questionnaire Responses}
\begin{tabular}{p{0.8\linewidth} c}
\toprule
\textbf{Control Question} & \textbf{Response} \\
\midrule
Do you require MFA to access email? & \textcolor{green}{\ding{51}} \\
Do you require MFA to log into computers? & \textcolor{green}{\ding{51}} \\
Do you require MFA to access sensitive data systems? & \textcolor{green}{\ding{51}} \\
\addlinespace
Does your organization have an employee acceptable use policy? & \textcolor{red}{\ding{55}} \\
Does your organization do security awareness training for new employees? & \textcolor{red}{\ding{55}} \\
Does your organization do security awareness training for all employees at least once per year? & \textcolor{red}{\ding{55}} \\
\bottomrule
\end{tabular}
\end{table}

\textbf{Analysis:} The consistent "No" responses to fundamental policy and training questions are a major concern. Without these controls, the organization is highly susceptible to threats that target employees directly.

% ------------------------------------------------------------------------------
% 4. Technical Scan Results
% ------------------------------------------------------------------------------
\section*{4. Technical Scan Results}

An external network scan was performed to identify open ports and exposed services on the designated target system.

\begin{itemize}
    \item \textbf{Target IP Address:} \texttt{192.168.1.100}
    \item \textbf{Scan Summary:} The scan reported the host as "up". All 1000 scanned TCP ports were found to be in a \textbf{"closed"} state.
\end{itemize}

\textbf{Finding:} No open ports or active services were detected on the target host.

\textbf{Interpretation:} This is a positive security finding. A host with no externally accessible ports presents a minimal attack surface from a network perspective. This result suggests the presence of a well-configured firewall or that the host does not run any network-facing services, which aligns with security best practices for systems not intended for public access.

% ------------------------------------------------------------------------------
% 5. Risk Assessment
% ------------------------------------------------------------------------------
\section*{5. Risk Assessment}

This section correlates the findings from the security control review and technical scan to identify and prioritize risks. No pre-existing vulnerabilities were provided for this assessment. The primary risks identified are administrative and procedural in nature.

\begin{table}[h!]
\centering
\caption{Identified Risks and Severity}
\begin{tabular}{p{0.1\linewidth} p{0.25\linewidth} p{0.45\linewidth} l}
\toprule
\textbf{Risk ID} & \textbf{Risk Name} & \textbf{Description} & \textbf{Severity} \\
\midrule
RISK-001 & Lack of Security Awareness Training & Employees are not formally trained to recognize and respond to cyber threats like phishing, malware, and social engineering. This makes them vulnerable targets and increases the likelihood of a security breach caused by human error. & \textbf{Critical} \\
\addlinespace
RISK-002 & Missing Acceptable Use Policy (AUP) & The absence of a formal AUP means there are no defined rules for using company IT assets. This can lead to unintentional data exposure, misuse of resources, and a lack of legal and disciplinary recourse in the event of a policy violation. & \textbf{High} \\
\bottomrule
\end{tabular}
\end{table}

% ------------------------------------------------------------------------------
% 6. Recommendations
% ------------------------------------------------------------------------------
\section*{6. Recommendations}

Based on the risk assessment, the following actions are recommended to improve the security posture of \textbf{Solid State}.

\begin{enumerate}
    \item \textbf{Implement a Mandatory Security Awareness Program (Addresses RISK-001):}
    \begin{itemize}
        \item Develop a formal training program for all employees covering essential topics such as phishing and spear-phishing identification, password security, safe browsing habits, and incident reporting procedures.
        \item This training must be mandatory for all new hires during their onboarding process.
        \item Conduct annual refresher training for all staff to ensure their knowledge remains current with the evolving threat landscape. Consider periodic phishing simulations to test and reinforce learning.
    \end{itemize}
    \vspace{1em}
    \item \textbf{Develop and Enforce an Acceptable Use Policy (Addresses RISK-002):}
    \begin{itemize}
        \item Create a comprehensive AUP that clearly outlines the rules and responsibilities for employees when using company computers, networks, email, and data.
        \item The policy should explicitly prohibit activities that pose a security risk, such as installing unauthorized software or sharing credentials.
        \item Require all employees to read and formally acknowledge the policy upon hiring and after any significant updates. Ensure the policy is easily accessible to all staff.
    \end{itemize}
\end{enumerate}

% ------------------------------------------------------------------------------
% 7. Conclusion
% ------------------------------------------------------------------------------
\section*{7. Conclusion}

The security posture of \textbf{Solid State} is characterized by a dichotomy between strong technical controls and weak administrative safeguards. While the robust implementation of MFA and a secure network perimeter are highly effective at preventing direct technical exploits, they do not protect against attacks that exploit human behavior.

The identified gaps in security training and policy represent the path of least resistance for an attacker. It is imperative that the organization prioritizes the recommendations in this report to build a "human firewall" that complements its existing technical defenses. Addressing these foundational issues will significantly mature the organization's overall cybersecurity resilience.

\end{document}
```