```latex
\documentclass[12pt]{article}

% Preamble: Required Packages
\usepackage[margin=1in]{geometry}
\usepackage{pifont} % For check and cross marks
\usepackage{booktabs} % For professional tables
\usepackage{hyperref} % For clickable links
\usepackage{url} % For URL formatting
\usepackage{seqsplit} % For splitting long strings in tt font
\usepackage{graphicx}
\usepackage{xcolor}

% Document Metadata
\title{Cybersecurity Posture Assessment Report}
\author{Cybersecurity Analysis Division}
\date{\today}

% Hyperref Setup
\hypersetup{
    colorlinks=true,
    linkcolor=blue,
    filecolor=magenta,      
    urlcolor=cyan,
    pdftitle={Cybersecurity Posture Assessment Report},
    pdfpagemode=FullScreen,
}

\begin{document}

\maketitle
\thispagestyle{empty}
\newpage

\tableofcontents
\newpage

% --- 1. Executive Summary ---
\section{Executive Summary}
This report provides a comprehensive analysis of the cybersecurity posture for \textbf{Harbor Light Foundation}. The assessment is based on a correlation of organizational data, a review of existing security controls, and an external network scan.

The organization demonstrates a foundational understanding of security by implementing Multi-Factor Authentication (MFA) for critical systems like email and sensitive data access. An employee acceptable use policy is also in place, which is a positive control.

However, several critical and high-risk gaps were identified that significantly increase the organization's vulnerability to common cyber threats. The most pressing concerns are the complete absence of a security awareness training program for both new and existing employees, and the lack of mandatory MFA for workstation logins. These deficiencies create substantial risk, particularly from phishing, social engineering, and credential compromise attacks.

The external network scan of the target IP address, \texttt{[Target IP]}, did not identify any open ports. This suggests a properly configured firewall is in place, which is a strong defensive measure for the network perimeter.

Immediate action should be taken to address the identified gaps in employee training and endpoint security to build a more resilient and secure operational environment.

% --- 2. Organizational Information ---
\section{Organizational Information}
The following details were provided for this assessment:

\begin{itemize}
    \item \textbf{Organization Name:} Harbor Light Foundation
    \item \textbf{Email Domain:} \texttt{HarborLightFoundation.com}
    \item \textbf{Website Domain:} \url{www.HarborLightFoundation.com}
    \item \textbf{External IP Address:} \texttt{141.17.219.44}
\end{itemize}

% --- 3. Security Control Review ---
\section{Security Control Review}
A review of key administrative and technical security controls was conducted based on a questionnaire. The results are summarized below. Gaps identified with a \ding{55} (cross mark) represent significant areas of risk.

\begin{table}[h!]
\centering
\caption{Security Controls Questionnaire Results}
\begin{tabular}{p{0.75\linewidth} c}
\toprule
\textbf{Control Question} & \textbf{Response} \\
\midrule
Do you require MFA to access email? & \ding{51} \\
Do you require MFA to log into computers? & \textcolor{red}{\ding{55}} \\
Do you require MFA to access sensitive data systems? & \ding{51} \\
Does your organization have an employee acceptable use policy? & \ding{51} \\
Does your organization do security awareness training for new employees? & \textcolor{red}{\ding{55}} \\
Does your organization do security awareness training for all employees at least once per year? & \textcolor{red}{\ding{55}} \\
\bottomrule
\end{tabular}
\end{table}

% --- 4. Technical Scan Results ---
\section{Technical Scan Results}
An external network vulnerability scan was performed to identify exposed services and potential vulnerabilities on the public-facing infrastructure.

\begin{itemize}
    \item \textbf{Target IP:} \texttt{[Target IP]}
    \item \textbf{Scan Date:} \today
    \item \textbf{Summary:} The scan completed successfully and found \textbf{no open ports}.
\end{itemize}

\subsection*{Analysis}
The absence of open ports is a positive finding. It indicates that the network perimeter firewall is correctly configured to block unsolicited inbound traffic, effectively reducing the external attack surface. No publicly exposed services were detected on the scanned host.

% --- 5. Risk Assessment ---
\section{Risk Assessment}
This section synthesizes findings from the security control review and technical scan. The following risks have been identified and prioritized based on their potential impact on the organization.

\begin{table}[h!]
\centering
\caption{Identified Risks Summary}
\begin{tabular}{p{0.1\linewidth} p{0.25\linewidth} p{0.45\linewidth} p{0.1\linewidth}}
\toprule
\textbf{Risk ID} & \textbf{Risk Name} & \textbf{Description} & \textbf{Severity} \\
\midrule
RISK-001 & Lack of Endpoint MFA & User workstations do not require MFA for login. A compromised password could grant an attacker full access to a user's computer and any connected network resources. & \textbf{Critical} \\
\addlinespace
RISK-002 & No Security Awareness Training Program & The lack of security training for new and existing employees makes the organization highly susceptible to phishing, malware, and social engineering attacks. The human element remains the weakest link in the security chain. & \textbf{High} \\
\addlinespace
RISK-003 & No Onboarding Security Training & New employees are not provided with security training upon being hired. This is a critical missed opportunity to establish a security-first mindset and educate them on organizational policies and common threats. & \textbf{High} \\
\bottomrule
\end{tabular}
\end{table}

% --- 6. Recommendations ---
\section{Recommendations}
Based on the risks identified in the previous section, the following actions are recommended to improve the overall security posture of \textbf{Harbor Light Foundation}.

\begin{enumerate}
    \item \textbf{Implement MFA for All Workstation Logins (Addresses RISK-001):}
    \begin{itemize}
        \item \textbf{Action:} Deploy a mandatory Multi-Factor Authentication solution for all employee computers (desktops and laptops). This could include methods like push notifications, authenticator app codes, or physical security keys.
        \item \textbf{Justification:} This control provides a critical layer of defense against credential theft. Even if an attacker obtains a user's password, they will be unable to log in without the second factor, preventing unauthorized access and lateral movement.
    \end{itemize}
    \vspace{1em}

    \item \textbf{Establish a Comprehensive Security Awareness Training Program (Addresses RISK-002 \& RISK-003):}
    \begin{itemize}
        \item \textbf{Action:} Develop and implement a formal security awareness training program. This program must include:
        \begin{enumerate}
            \item \textbf{Mandatory Onboarding Training:} All new hires must complete a security training module before being granted full access to systems. This should cover the acceptable use policy, data handling, and how to identify and report phishing attempts.
            \item \textbf{Annual Refresher Training:} All employees must complete a security awareness course at least once per year to stay informed about evolving threats.
            \item \textbf{Regular Phishing Simulations:} Conduct periodic, simulated phishing campaigns to test employee vigilance and provide targeted, just-in-time training to those who click malicious links.
        \end{enumerate}
        \item \textbf{Justification:} An educated workforce is the first line of defense. A robust training program drastically reduces the likelihood of a successful social engineering or phishing attack, which is the most common initial access vector for breaches.
    \end{itemize}
\end{enumerate}

\end{document}
```