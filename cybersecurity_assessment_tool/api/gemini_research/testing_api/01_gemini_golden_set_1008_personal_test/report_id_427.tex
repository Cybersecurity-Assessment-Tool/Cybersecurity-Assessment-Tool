```latex
\documentclass[12pt]{article}

% 1. REQUIRED PACKAGES
\usepackage[margin=1in]{geometry}
\usepackage{pifont} % For checkmarks and crosses (\ding)
\usepackage{booktabs} % For professional tables
\usepackage{hyperref} % For clickable links
\usepackage{url}      % For proper URL formatting
\usepackage{seqsplit} % To split long strings like IPs

% --- Document Metadata ---
\title{Cybersecurity Posture Assessment Report}
\author{Cybersecurity Analysis Division}
\date{\today}

\begin{document}

\maketitle
\thispagestyle{empty}
\newpage

\tableofcontents
\newpage

% 2. EXECUTIVE SUMMARY / OVERVIEW
\section*{Executive Summary}

This report details the findings of a cybersecurity assessment for \textbf{Open Door}. The analysis is based on a network scan, a review of organizational security controls, and an evaluation of pre-existing risks.

The assessment identified several critical and high-risk security gaps that expose the organization to significant threats, primarily related to account compromise and unauthorized access. The most pressing issue is a complete lack of Multi-Factor Authentication (MFA) across all key systems, including email, user computers, and sensitive data repositories. This is compounded by an externally accessible management service (SSH) on a public-facing server.

Furthermore, foundational security policies and training programs are underdeveloped. The absence of an Acceptable Use Policy and a mandatory annual security awareness training program for all staff increases the likelihood of human error leading to a security incident.

Immediate remediation is required to address the MFA and policy deficiencies to reduce the risk of a material security breach.

% 3. ORGANIZATIONAL INFORMATION
\section*{Organizational Information}

The following information was provided for the assessment.

\begin{tabular}{@{}ll}
\toprule
\textbf{Attribute} & \textbf{Value} \\
\midrule
Organization Name & \textbf{Open Door} \\
Email Domain & \texttt{OpenDoor.org} \\
Website Domain & \url{www.OpenDoor.org} \\
Primary External IP & \texttt{3.198.35.233} \\
\bottomrule
\end{tabular}

% 4. SECURITY CONTROL REVIEW (from Questionnaire)
\section*{Security Control Review}

A review of administrative and technical security controls was conducted via a standardized questionnaire. The responses reveal significant gaps in the organization's identity and access management and security governance frameworks.

\begin{table}[h!]
\centering
\caption{Security Controls Questionnaire Results}
\begin{tabular}{@{}p{8cm}cc@{}}
\toprule
\textbf{Control Question} & \textbf{Response} & \textbf{Assessment} \\
\midrule
Do you require MFA to access email? & \ding{55} & \textbf{Critical Gap} \\
Do you require MFA to log into computers? & \ding{55} & \textbf{Critical Gap} \\
Do you require MFA to access sensitive data systems? & \ding{55} & \textbf{Critical Gap} \\
Does your organization have an employee acceptable use policy? & \ding{55} & High Risk \\
Does your organization do security awareness training for new employees? & \ding{51} & Control in Place \\
Does your organization do security awareness training for all employees at least once per year? & \ding{55} & High Risk \\
\bottomrule
\end{tabular}
\end{table}

% 5. TECHNICAL SCAN RESULTS
\section*{Technical Scan Results}

An external network scan was performed to identify accessible services on the organization's perimeter.

\begin{itemize}
    \item \textbf{Target IP Address:} \seqsplit{\texttt{2001:db8::1}}
    \item \textbf{Scan Status:} Host is up and responsive.
\end{itemize}

The following open ports were discovered:

\begin{table}[h!]
\centering
\caption{Open Port Analysis}
\begin{tabular}{@{}llll@{}}
\toprule
\textbf{Port} & \textbf{State} & \textbf{Service (Presumed)} & \textbf{Notes} \\
\midrule
22/tcp & open & SSH & Secure Shell is a remote management protocol. \\
& & & Exposing SSH to the internet creates a significant \\
& & & attack vector. The scan data did not include \\
& & & version information, preventing automated checks \\
& & & for known vulnerabilities. \\
\bottomrule
\end{tabular}
\end{table}

\textbf{Analysis:} The presence of an open SSH port is a notable risk, especially when correlated with the lack of MFA controls. This service is a primary target for brute-force and credential-stuffing attacks.

% 6. RISK ASSESSMENT
\section*{Risk Assessment}

The following table synthesizes findings from the security control review and technical scan into a prioritized list of risks. No pre-existing vulnerabilities were reported.

\begin{table}[h!]
\centering
\caption{Summary of Identified Risks}
\begin{tabular}{@{}p{1.5cm}p{3cm}p{6.5cm}l@{}}
\toprule
\textbf{Risk ID} & \textbf{Risk Name} & \textbf{Description} & \textbf{Severity} \\
\midrule
RISK-001 & Absence of MFA & The lack of MFA on email, endpoints, and sensitive systems exposes the organization to a high likelihood of account compromise via phishing or credential theft. & \textbf{Critical} \\
\addlinespace
RISK-002 & Exposed Management Service & The public-facing SSH service, combined with the lack of MFA, provides a direct path for attackers to attempt to gain unauthorized server access. & High \\
\addlinespace
RISK-003 & Inadequate Security Governance & The absence of a formal Acceptable Use Policy and mandatory annual security training for all staff increases the risk of insider threat and human error. & High \\
\bottomrule
\end{tabular}
\end{table}

% 7. RECOMMENDATIONS
\section*{Recommendations}

Based on the findings, the following actions are recommended to mitigate the identified risks and improve the overall security posture of \textbf{Open Door}.

\subsection*{Immediate Actions (Critical Priority)}
\begin{enumerate}
    \item \textbf{Deploy Multi-Factor Authentication (MFA):} Immediately begin a phased rollout of MFA for all users. Prioritize the following systems:
    \begin{itemize}
        \item Email (e.g., Office 365, Google Workspace).
        \item Remote access systems (VPN, SSH).
        \item Access to all systems containing sensitive or critical data.
    \end{itemize}
\end{enumerate}

\subsection*{High Priority Actions}
\begin{enumerate}
    \setcounter{enumi}{1}
    \item \textbf{Harden Exposed Services:} For the SSH service on \seqsplit{\texttt{2001:db8::1}}:
    \begin{itemize}
        \item If possible, restrict access to trusted IP addresses using a firewall.
        \item Disable password-based authentication and enforce the use of public-key cryptography.
        \item Ensure the SSH server software is fully patched and running a current version.
    \end{itemize}
    \item \textbf{Develop an Acceptable Use Policy (AUP):} Create and enforce a formal AUP that clearly defines the rules for using company technology and data. All employees must read and acknowledge this policy.
    \item \textbf{Implement Annual Security Training:} Establish a mandatory security awareness training program that all employees must complete at least once per year. This reinforces concepts and addresses evolving threats.
\end{enumerate}

\subsection*{Further Security Enhancements}
\begin{enumerate}
    \setcounter{enumi}{4}
    \item \textbf{Conduct Comprehensive Vulnerability Scanning:} Perform authenticated and unauthenticated vulnerability scans across the organization's infrastructure to identify outdated software and specific configuration weaknesses in greater detail.
\end{enumerate}

\end{document}
```