```latex
\documentclass[12pt]{article}

% --- PACKAGE IMPORTS ---
\usepackage[margin=1in]{geometry}
\usepackage{pifont} % For checkmarks and crosses
\usepackage{booktabs} % For professional tables
\usepackage{hyperref} % For clickable links
\usepackage{url} % For URL formatting
\usepackage{seqsplit} % To split long strings without breaking
\usepackage{graphicx} % For potential logos
\usepackage{xcolor} % For colors

% --- DOCUMENT METADATA ---
\title{Cybersecurity Posture Assessment Report}
\author{Cybersecurity Analyst}
\date{\today}

% --- HYPERREF SETUP ---
\hypersetup{
    colorlinks=true,
    linkcolor=blue,
    filecolor=magenta,      
    urlcolor=cyan,
    pdftitle={Cybersecurity Posture Assessment Report},
    pdfpagemode=FullScreen,
}

% --- DOCUMENT START ---
\begin{document}

\maketitle
\hrule
\vspace{1em}

% ===================================================================
% SECTION 1: EXECUTIVE SUMMARY
% ===================================================================
\section*{Executive Summary}

This report provides a cybersecurity posture assessment for \textbf{Brimstone Manufacturing}, conducted on \today. The analysis is based on a review of organizational security controls, a technical network scan, and a summary of known risks.

The assessment reveals a mixed security posture. The organization has established a solid foundation with mandatory Multi-Factor Authentication (MFA) for email access and a comprehensive security awareness training program. These controls are commendable and significantly reduce the risk of phishing and social engineering attacks.

However, critical gaps were identified in access control policies. The absence of MFA for workstation logins and, most importantly, for systems containing sensitive data, presents a severe risk. A compromised user credential could grant an attacker direct access to an employee's computer and the organization's most valuable data assets.

The external network scan of the target IP address \texttt{[Target IP]} did not identify any open ports. While this may indicate a strong firewall configuration, it also underscores the importance of securing internal assets and user credentials, as the primary risks identified are related to authentication and access control rather than external exposure.

Immediate remediation should focus on implementing a robust MFA solution across all workstations and sensitive data systems to mitigate the risk of unauthorized access.

% ===================================================================
% SECTION 2: ORGANIZATIONAL INFORMATION
% ===================================================================
\section*{Organizational Information}

The following details were provided for the assessment.

\begin{tabular}{@{}ll}
    \toprule
    \textbf{Attribute} & \textbf{Value} \\
    \midrule
    Organization Name & \textbf{Brimstone Manufacturing} \\
    Email Domain & \seqsplit{\texttt{BrimstoneManufacturing.com}} \\
    Website Domain & \seqsplit{\url{www.BrimstoneManufacturing.com}} \\
    External IP Address & \texttt{128.2.217.227} \\
    \bottomrule
\end{tabular}

% ===================================================================
% SECTION 3: SECURITY CONTROL REVIEW
% ===================================================================
\section*{Security Control Review}

A review of administrative and technical security controls was conducted based on a standardized questionnaire. The results below highlight implemented controls and identify significant gaps. A green checkmark (\textcolor{green}{\ding{51}}) indicates a positive control, while a red cross (\textcolor{red}{\ding{55}}) indicates a gap requiring attention.

\begin{table}[h!]
\centering
\begin{tabular}{@{}lc}
    \toprule
    \textbf{Security Control Question} & \textbf{Status} \\
    \midrule
    Do you require MFA to access email? & \textcolor{green}{\ding{51}} \\
    Do you require MFA to log into computers? & \textcolor{red}{\ding{55}} \\
    Do you require MFA to access sensitive data systems? & \textcolor{red}{\ding{55}} \\
    \addlinespace
    Does your organization have an employee acceptable use policy? & \textcolor{green}{\ding{51}} \\
    Does your organization do security awareness training for new employees? & \textcolor{green}{\ding{51}} \\
    Does your organization do security awareness training for all employees at least once per year? & \textcolor{green}{\ding{51}} \\
    \bottomrule
\end{tabular}
\caption{Organizational Security Control Status}
\end{table}

% ===================================================================
% SECTION 4: TECHNICAL SCAN RESULTS
% ===================================================================
\section*{Technical Scan Results}

An external network vulnerability scan was performed on the designated target IP address.

\begin{itemize}
    \item \textbf{Target IP Address:} \texttt{[Target IP]}
    \item \textbf{Scan Date:} Not specified in scan data.
    \item \textbf{Summary of Findings:} The scan completed successfully but did not detect any open TCP or UDP ports on the target system. This suggests that the host may be protected by a well-configured firewall that drops or rejects unsolicited incoming traffic, or the host was not online during the scan. No vulnerabilities related to exposed services could be identified.
\end{itemize}

% ===================================================================
% SECTION 5: RISK ASSESSMENT
% ===================================================================
\section*{Risk Assessment}

This section synthesizes findings from the security control review and technical scan. The primary risks identified are related to inadequate access control measures, which could be exploited if an attacker obtains valid user credentials.

\begin{table}[h!]
\centering
\begin{tabular}{@{}p{0.3\linewidth} p{0.5\linewidth} p{0.15\linewidth}@{}}
    \toprule
    \textbf{Risk Name} & \textbf{Overview} & \textbf{Severity} \\
    \midrule
    \textbf{Lack of MFA for Sensitive Systems} & The absence of MFA on systems storing or processing sensitive data means that a single compromised password could lead to a significant data breach. This is the most critical security gap identified. & \textbf{Critical} \\
    \addlinespace
    \textbf{Lack of MFA for Workstation Access} & Without MFA, compromised employee credentials (e.g., from a phishing attack) can be used to gain direct access to company workstations, providing a foothold for lateral movement within the network. & \textbf{High} \\
    \bottomrule
\end{tabular}
\caption{Identified Risks and Severity}
\end{table}

% ===================================================================
% SECTION 6: RECOMMENDATIONS
% ===================================================================
\section*{Recommendations}

Based on the analysis, the following actions are recommended to enhance the organization's security posture. Recommendations are prioritized by severity.

\begin{enumerate}
    \item \textbf{Implement MFA for All Sensitive Systems (Critical)} \\
    \textit{Action:} Deploy a mandatory MFA solution for all user accounts (including administrative and service accounts) that can access systems containing sensitive financial, customer, or proprietary data. This control should be the highest priority for remediation.
    \textit{Justification:} This single control provides a powerful defense against credential theft and is a standard industry best practice for protecting critical assets.

    \item \textbf{Enforce MFA for Workstation Logins (High)} \\
    \textit{Action:} Require MFA for all local and remote logins to employee workstations and servers. Solutions like Windows Hello for Business, Duo, or other third-party integrations can fulfill this requirement.
    \textit{Justification:} This prevents an attacker with a stolen password from accessing a user's machine, which is often the entry point for more sophisticated attacks like ransomware deployment or internal network reconnaissance.

    \item \textbf{Conduct Regular Security Control Audits} \\
    \textit{Action:} Periodically review and audit the implementation of security controls, including MFA policies, to ensure they are functioning as intended and covering all new systems and employees.
    \textit{Justification:} Continuous verification ensures that security policies remain effective over time and adapt to changes in the IT environment.
\end{enumerate}

% ===================================================================
% SECTION 7: CONCLUSION
% ===================================================================
\section*{Conclusion}

\textbf{Brimstone Manufacturing} has successfully implemented important foundational security controls, particularly regarding email security and employee awareness. However, the current reliance on passwords as the sole authentication factor for workstations and sensitive systems creates an unacceptable level of risk.

By addressing the critical and high-priority recommendations outlined in this report, the organization can significantly strengthen its defenses against common cyber threats and protect its valuable data assets from unauthorized access.

\end{document}
```