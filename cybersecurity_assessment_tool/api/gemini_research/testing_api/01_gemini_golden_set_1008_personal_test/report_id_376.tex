```latex
\documentclass[12pt]{article}

% Required Packages
\usepackage[margin=1in]{geometry}
\usepackage{pifont} % For checkmarks and crosses
\usepackage{booktabs} % For professional tables
\usepackage{hyperref} % For clickable links
\usepackage{url} % For URL formatting
\usepackage{seqsplit} % For splitting long strings
\usepackage{graphicx} % For logo
\usepackage{xcolor} % For colors

% Document Metadata
\title{Cybersecurity Assessment Report}
\author{Cybersecurity Analysis Division}
\date{\today}

% Hyperref Setup
\hypersetup{
    colorlinks=true,
    linkcolor=blue,
    filecolor=magenta,      
    urlcolor=cyan,
    pdftitle={Cybersecurity Assessment Report},
    pdfpagemode=FullScreen,
}

\begin{document}

\begin{titlepage}
    \centering
    \vspace*{1cm}
    \Huge\textbf{Cybersecurity Assessment Report}
    \vspace{1.5cm}
    \Large
    \textbf{Prepared for:}\\
    Nexus Dynamics
    \vspace{2cm}
    \large
    \textbf{Report Date:}\\
    \today
    \vfill
    \large
    \textbf{CONFIDENTIAL}
    \vspace{0.8cm}
    \small
    This document contains sensitive information and is intended solely for the use of the designated recipient. Unauthorized distribution is strictly prohibited.
\end{titlepage}

\maketitle

\tableofcontents
\newpage

\section{Executive Summary}
This report provides a comprehensive analysis of the cybersecurity posture of Nexus Dynamics, based on a network scan, a security controls questionnaire, and a review of pre-existing risks.

The assessment reveals a mixed security posture. On the technical front, the external network scan of the target IP address (\texttt{9.229.165.219}) indicates a strong perimeter defense, with no open ports detected. This significantly reduces the external attack surface and is a commendable security practice.

However, a critical gap was identified in the organization's security policies. While new employees receive security awareness training, there is no program for mandatory, annual refresher training for all staff. This exposes the organization to a high risk of human error, particularly from evolving threats like sophisticated phishing and social engineering attacks. As employees are often the first line of defense, this gap undermines other technical controls.

No other pre-existing vulnerabilities were reported. The primary recommendation of this report is to immediately implement a mandatory, recurring security awareness training program for all employees to mitigate this high-risk finding.

\section{Organizational Information}
The following details were provided for the assessment.

\begin{tabular}{@{}ll}
\toprule
\textbf{Attribute} & \textbf{Value} \\
\midrule
Organization Name & Nexus Dynamics \\
Email Domain & \texttt{NexusDynamics.net} \\
Website Domain & \url{www.NexusDynamics.net} \\
External IP Scanned & \texttt{9.229.165.219} \\
\bottomrule
\end{tabular}

\section{Security Control Review}
The following table summarizes the responses from the security controls questionnaire. A checkmark (\ding{51}) indicates a positive control is in place, while a cross (\ding{55}) indicates a potential security gap.

\begin{table}[h!]
\centering
\begin{tabular}{@{}lc}
\toprule
\textbf{Security Control Question} & \textbf{Status} \\
\midrule
Do you require MFA to access email? & \ding{51} \\
Do you require MFA to log into computers? & \ding{51} \\
Do you require MFA to access sensitive data systems? & \ding{51} \\
Does your organization have an employee acceptable use policy? & \ding{51} \\
Does your organization do security awareness training for new employees? & \ding{51} \\
\textbf{Does your organization do security awareness training for all employees at least once per year?} & \textcolor{red}{\ding{55}} \\
\bottomrule
\end{tabular}
\caption{Security Controls Questionnaire Results}
\end{table}

\subsection*{Analysis of Findings}
The organization has implemented strong identity and access management controls, with Multi-Factor Authentication (MFA) enforced across key systems. Foundational policies, such as an acceptable use policy and new hire training, are also in place.

However, the lack of \textbf{annual, recurring security awareness training for all employees} is a critical deficiency. Cyber threats evolve rapidly, and without continuous education, employees' ability to recognize and respond to new attack vectors diminishes over time. This gap significantly increases the risk of a security breach originating from human error.

\section{Technical Scan Results}
An external network scan was performed to identify accessible services and potential vulnerabilities on the organization's perimeter.

\begin{itemize}
    \item \textbf{Target IP Address:} \texttt{192.168.1.100} (Internal scan target provided)
    \item \textbf{Scan Date:} \today
\end{itemize}

\begin{table}[h!]
\centering
\begin{tabular}{@{}ll}
\toprule
\textbf{Scan Finding} & \textbf{Result} \\
\midrule
Host Status & Up \\
Open Ports & 0 \\
Filtered/Closed Ports & All scanned ports were closed \\
\bottomrule
\end{tabular}
\caption{Nmap Scan Summary}
\end{table}

\subsection*{Analysis of Findings}
The scan results are positive. The target host was responsive, but no open ports were discovered. This indicates a well-configured firewall that effectively blocks unsolicited inbound traffic, presenting a minimal attack surface to the public internet. No vulnerabilities could be identified at the network level due to the absence of exposed services.

\section{Consolidated Risk Assessment}
This section synthesizes findings from the security questionnaire, technical scans, and pre-existing risk data into a consolidated list.

\begin{table}[h!]
\centering
\begin{tabular}{@{}p{0.15\linewidth}p{0.5\linewidth}p{0.1\linewidth}p{0.15\linewidth}@{}}
\toprule
\textbf{Risk ID} & \textbf{Description} & \textbf{Severity} & \textbf{Source} \\
\midrule
ORG-001 & Lack of mandatory annual security awareness training for all employees increases susceptibility to phishing, social engineering, and other human-vector attacks. & \textbf{High} & Questionnaire \\
\addlinespace
\bottomrule
\end{tabular}
\caption{Identified Risks}
\end{table}

\section{Recommendations}
Based on the consolidated risk assessment, the following actions are recommended to improve the cybersecurity posture of Nexus Dynamics.

\subsection{Risk ID: ORG-001 - Lack of Annual Security Training}
\begin{itemize}
    \item \textbf{Recommendation:} Implement a mandatory, comprehensive security awareness training program that is required for all employees to complete on an annual basis.
    \item \textbf{Justification:} The threat landscape is constantly changing. Annual training ensures that employees' knowledge remains current regarding new threats (e.g., advanced phishing techniques, ransomware delivery methods) and reinforces best practices for data handling, password security, and incident reporting. This is one of the most cost-effective measures to reduce the likelihood of a successful cyberattack.
    \item \textbf{Actionable Steps:}
    \begin{enumerate}
        \item Select a reputable security awareness training platform or develop an in-house curriculum.
        \item The curriculum should cover, at a minimum: phishing and spear-phishing, password hygiene, social engineering, secure data handling, and incident reporting procedures.
        \item Mandate completion for all employees, including management and executive staff.
        \item Track completion rates to ensure 100\% compliance.
        \item Consider periodic phishing simulations to test and reinforce the training.
    \end{enumerate}
\end{itemize}

\end{document}
```