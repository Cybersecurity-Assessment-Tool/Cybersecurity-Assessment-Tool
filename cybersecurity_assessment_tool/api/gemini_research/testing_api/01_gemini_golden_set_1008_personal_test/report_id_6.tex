```latex
\documentclass[12pt]{article}

% Preamble: Required Packages
\usepackage[margin=1in]{geometry}
\usepackage{pifont} % For check and cross marks
\usepackage{booktabs} % For professional tables
\usepackage{hyperref} % For hyperlinks
\usepackage{url} % For URL formatting
\usepackage{seqsplit} % To split long strings in tt font
\usepackage{graphicx}
\usepackage{xcolor}
\usepackage{fancyhdr}

% Document Metadata
\title{Cybersecurity Assessment Report}
\author{Cybersecurity Analysis Division}
\date{\today}

% Hyperref Setup
\hypersetup{
    colorlinks=true,
    linkcolor=black,
    urlcolor=blue,
    pdftitle={Cybersecurity Assessment Report},
    pdfauthor={Cybersecurity Analysis Division},
}

% Header and Footer
\pagestyle{fancy}
\fancyhf{}
\lhead{Tidal Wave Sports}
\rhead{Confidential}
\cfoot{\thepage}

\begin{document}

\maketitle
\thispagestyle{empty}
\newpage

\tableofcontents
\newpage

% --- 1. Executive Summary ---
\section{Executive Summary}
This report details the findings of a cybersecurity assessment conducted for \textbf{Tidal Wave Sports}. The assessment incorporated a technical network scan, a review of pre-existing risk documentation, and an analysis of the organization's security controls based on a questionnaire.

The primary finding of this assessment is the identification of critical gaps in foundational security controls. The complete absence of Multi-Factor Authentication (MFA) across email, computers, and sensitive data systems presents a significant and immediate risk of account compromise and unauthorized access. Furthermore, the lack of formal security policies and employee training programs creates an environment where human error is more likely to result in a security incident.

The technical network scan of the target host \texttt{192.168.0.5} did not reveal any open ports, indicating a minimal external attack surface on that specific device. This finding contradicts a pre-existing documented risk concerning an unencrypted web server on port 80, suggesting that the risk may have been remediated.

Recommendations prioritize the immediate implementation of MFA and the development of a security awareness program to mitigate the most severe risks identified.

% --- 2. Organizational Information ---
\section{Organizational Information}
The following details were provided for the assessment scope.
\begin{itemize}
    \item \textbf{Organization Name:} Tidal Wave Sports
    \item \textbf{Email Domain:} \texttt{TidalWaveSports.net}
    \item \textbf{External IP Address:} \texttt{167.200.14.233}
\end{itemize}

% --- 3. Security Control Review ---
\section{Security Control Review}
An analysis of the security questionnaire reveals critical deficiencies in administrative and technical controls. The following table summarizes the organization's responses and provides an assessment of the associated risk.

\begin{table}[h!]
\centering
\caption{Security Controls Questionnaire Analysis}
\label{tab:controls}
\begin{tabular}{p{7cm} c p{5cm}}
\toprule
\textbf{Control Question} & \textbf{Response} & \textbf{Assessment} \\
\midrule
Do you require MFA to access email? & \ding{55} & \textbf{Critical Gap.} Lack of MFA on email is a primary vector for Business Email Compromise (BEC) and phishing attacks. \\
\addlinespace
Do you require MFA to log into computers? & \ding{55} & \textbf{Critical Gap.} Compromised credentials could lead to direct endpoint and internal network access. \\
\addlinespace
Do you require MFA to access sensitive data systems? & \ding{55} & \textbf{Critical Gap.} The organization's most sensitive data is protected only by a password, which is a single point of failure. \\
\addlinespace
Does your organization have an employee acceptable use policy? & \ding{55} & \textbf{High Risk.} Without a formal policy, there are no established rules for technology use, increasing insider threat and compliance risks. \\
\addlinespace
Does your organization do security awareness training for new employees? & \ding{55} & \textbf{High Risk.} New staff are not equipped with the knowledge to identify and avoid common threats like phishing. \\
\addlinespace
Does your organization do security awareness training for all employees at least once per year? & \ding{55} & \textbf{High Risk.} The lack of ongoing training means the human firewall is not maintained, and employees remain a key vulnerability. \\
\bottomrule
\end{tabular}
\end{table}

% --- 4. Technical Scan Results ---
\section{Technical Scan Results}
A network scan was performed to identify accessible services and potential vulnerabilities on the specified target.

\begin{itemize}
    \item \textbf{Target IP Address:} \seqsplit{\texttt{192.168.0.5}}
    \item \textbf{Scan Date:} \today
\end{itemize}

The scan revealed that the host was online, but no open ports were detected. This indicates a properly configured firewall or that no network services are currently exposed from this host.

\begin{table}[h!]
\centering
\caption{Nmap Scan Results for \texttt{192.168.0.5}}
\label{tab:nmap}
\begin{tabular}{llll}
\toprule
\textbf{Port} & \textbf{State} & \textbf{Service} & \textbf{Product / Version} \\
\midrule
80 & closed & http & N/A \\
\bottomrule
\end{tabular}
\end{table}

\subsection{Analysis of Technical Findings}
The technical scan results are positive, showing a minimal attack surface. However, this scan's finding that port 80 is closed directly contradicts a pre-existing risk documented in \texttt{Input\_3\_Current\_Risks\_JSON}, which stated "Port 80 is open." This suggests that the previously identified risk has been remediated. This discrepancy should be investigated to ensure the organization's risk register is accurate.

% --- 5. Consolidated Risk Assessment ---
\section{Consolidated Risk Assessment}
The following table synthesizes findings from the security control review, technical scan, and pre-existing risk data into a prioritized list.

\begin{table}[h!]
\centering
\caption{Summary of Identified Risks}
\label{tab:risks}
\begin{tabular}{p{2cm} p{4cm} p{6cm} l}
\toprule
\textbf{Risk ID} & \textbf{Risk Title} & \textbf{Description} & \textbf{Severity} \\
\midrule
RISK-001 & No Multi-Factor Authentication (MFA) & The absence of MFA for email, endpoints, and sensitive systems exposes the organization to severe risk of account takeover and data breach from stolen credentials. & \textbf{Critical} \\
\addlinespace
RISK-002 & Lack of Security Policy and Training & The absence of an Acceptable Use Policy and a security awareness training program leaves the organization vulnerable to insider threats and social engineering attacks. & \textbf{High} \\
\addlinespace
RISK-003 & Outdated Risk Register & A pre-existing risk noted an open port 80, but the current scan shows it is closed. An inaccurate risk register can lead to misallocation of security resources. & Medium \\
\bottomrule
\end{tabular}
\end{table}

% --- 6. Recommendations ---
\section{Recommendations}
Based on the analysis, the following actions are recommended to improve the cybersecurity posture of \textbf{Tidal Wave Sports}.

\subsection{Immediate Priority: Mitigate Critical Risks}
\begin{enumerate}
    \item \textbf{Implement Multi-Factor Authentication (MFA):}
    \begin{itemize}
        \item Immediately enforce MFA for all user accounts on the email platform (\texttt{TidalWaveSports.net}).
        \item Deploy MFA for remote access solutions (e.g., VPN) and all administrative accounts.
        \item Develop a phased rollout plan to require MFA for all computer logins and access to systems containing sensitive data.
    \end{itemize}
\end{enumerate}

\subsection{High Priority: Build Foundational Controls}
\begin{enumerate}
    \setcounter{enumi}{1}
    \item \textbf{Develop and Implement Security Policies:}
    \begin{itemize}
        \item Create a formal Acceptable Use Policy (AUP) that all employees must read and sign. This policy should outline the rules for using company technology and data.
    \end{itemize}
    \item \textbf{Establish a Security Awareness Training Program:}
    \begin{itemize}
        \item Institute mandatory security awareness training for all new hires as part of their onboarding process.
        \item Conduct annual, mandatory refresher training for all employees covering topics such as phishing, password security, and social engineering.
    \end{itemize}
\end{enumerate}

\subsection{Administrative Recommendation}
\begin{enumerate}
    \setcounter{enumi}{3}
    \item \textbf{Validate and Update Risk Register:}
    \begin{itemize}
        \item Formally investigate the status of the "Unencrypted Web Server" risk. If Port 80 was intentionally closed as a remediation step, update the risk register to mark the risk as "Closed" or "Remediated."
    \end{itemize}
\end{enumerate}

\end{document}
```