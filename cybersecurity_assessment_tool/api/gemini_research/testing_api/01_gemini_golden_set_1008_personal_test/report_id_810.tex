```latex
\documentclass[12pt]{article}

% ----------------------------------------------------------------------
% METADATA & PACKAGES
% ----------------------------------------------------------------------
\usepackage[margin=1in]{geometry}
\usepackage{pifont} % For checkmarks and crosses
\usepackage{booktabs} % For professional tables
\usepackage{hyperref} % For clickable links and metadata
\usepackage{url}      % For formatting URLs
\usepackage{seqsplit} % To split long strings in tt font
\usepackage[utf8]{inputenc}

\hypersetup{
    colorlinks=true,
    linkcolor=black,
    filecolor=magenta,      
    urlcolor=blue,
    pdftitle={Cybersecurity Posture Report},
    pdfauthor={Cybersecurity Analysis Division},
    pdfsubject={Security Assessment},
    pdfkeywords={Security, Risk, Analysis},
    bookmarks=true
}

\newcommand{\yes}{\ding{51}} % Green checkmark
\newcommand{\no}{\ding{55}}  % Red cross

% ----------------------------------------------------------------------
% DOCUMENT START
% ----------------------------------------------------------------------
\begin{document}

% ----------------------------------------------------------------------
% TITLE PAGE
% ----------------------------------------------------------------------
\title{
    Cybersecurity Posture Report \\
    \large A Synthesized Analysis of Technical and Organizational Controls \\
    \vspace{1cm}
    \textbf{Nexus Dynamics}
}
\author{Cybersecurity Analysis Division}
\date{\today}
\maketitle
\thispagestyle{empty}
\newpage

\tableofcontents
\newpage

% ----------------------------------------------------------------------
% SECTION 1: EXECUTIVE OVERVIEW
% ----------------------------------------------------------------------
\section{Executive Overview}
This report provides a consolidated analysis of the cybersecurity posture for \textbf{Nexus Dynamics}. The assessment is based on a synthesis of three data sources: a network vulnerability scan, a self-reported organizational security controls questionnaire, and a list of pre-existing known risks.

The overall security posture presents a mixed landscape. On one hand, the organization demonstrates a strong commitment to identity and access management, with Multi-Factor Authentication (MFA) widely implemented across critical systems. Furthermore, a technical scan of the target host \texttt{192.168.0.5} indicates that a previously identified risk related to an unencrypted web server has been successfully remediated.

However, a critical gap was identified in the employee onboarding process. The lack of mandatory security awareness training for new employees represents a high-risk exposure. This oversight can undermine other technical controls, as untrained staff are significantly more susceptible to social engineering attacks such as phishing.

This report details these findings and provides actionable recommendations to address the identified gap and maintain a robust security posture.

% ----------------------------------------------------------------------
% SECTION 2: ORGANIZATIONAL INFORMATION
% ----------------------------------------------------------------------
\section{Organizational Information}
The following details were provided for the assessment. This information helps establish the context for the technical and procedural findings.

\begin{itemize}
    \item \textbf{Organization Name:} Nexus Dynamics
    \item \textbf{Primary Email Domain:} \texttt{NexusDynamics.org}
    \item \textbf{Primary Website Domain:} \url{www.NexusDynamics.org}
    \item \textbf{External IP Address:} \texttt{177.133.67.53}
\end{itemize}

% ----------------------------------------------------------------------
% SECTION 3: SECURITY CONTROL REVIEW
% ----------------------------------------------------------------------
\section{Security Control Review}
This section evaluates the organization's security posture based on a self-reported questionnaire. "Yes" answers indicate that a control is in place, while "No" answers highlight potential gaps that require attention.

\subsection{Questionnaire Results}
\begin{table}[h!]
\centering
\begin{tabular}{p{0.75\textwidth} c}
\toprule
\textbf{Control Question} & \textbf{Response} \\
\midrule
Do you require MFA to access email? & \yes \\
Do you require MFA to log into computers? & \yes \\
Do you require MFA to access sensitive data systems? & \yes \\
Does your organization have an employee acceptable use policy? & \yes \\
Does your organization do security awareness training for new employees? & \no \\
Does your organization do security awareness training for all employees at least once per year? & \yes \\
\bottomrule
\end{tabular}
\caption{Organizational Security Control Status}
\label{tab:controls}
\end{table}

\subsection{Analysis}
The organization has implemented a robust MFA strategy, which is a critical defense against credential theft and unauthorized access. However, the lack of security awareness training for \textbf{new employees} is a significant vulnerability. The initial days and weeks of employment are a critical period where new staff are learning company procedures and may be more easily manipulated by social engineering tactics. This gap negates some of the benefits of the annual training program by leaving the organization exposed until the next cycle.

% ----------------------------------------------------------------------
% SECTION 4: TECHNICAL SCAN RESULTS
% ----------------------------------------------------------------------
\section{Technical Scan Results}
A network scan was performed on the specified target to identify open ports and potentially vulnerable services.

\begin{itemize}
    \item \textbf{Target IP Address:} \texttt{192.168.0.5}
    \item \textbf{Scan Date:} \today
\end{itemize}

\begin{table}[h!]
\centering
\begin{tabular}{cccc}
\toprule
\textbf{Port} & \textbf{State} & \textbf{Service} & \textbf{Product / Version} \\
\midrule
80/tcp & closed & http & Not Detected \\
\bottomrule
\end{tabular}
\caption{Nmap Scan Results for \texttt{192.168.0.5}}
\label{tab:scan}
\end{table}

\subsection{Analysis}
The scan results are positive. The target host shows that port 80 (HTTP) is closed. This finding is particularly important as it directly contradicts a pre-existing risk documented in the organization's risk register. This indicates that the previously identified vulnerability of an "Unencrypted Web Server" has been successfully remediated on this specific asset. No other open ports were detected during this scan.

% ----------------------------------------------------------------------
% SECTION 5: CONSOLIDATED RISK ASSESSMENT
% ----------------------------------------------------------------------
\section{Consolidated Risk Assessment}
This table synthesizes findings from all data sources to provide a unified view of the current risk landscape.

\begin{table}[h!]
\centering
\begin{tabular}{p{0.25\textwidth} p{0.1\textwidth} p{0.4\textwidth} p{0.15\textwidth}}
\toprule
\textbf{Risk Name} & \textbf{Severity} & \textbf{Overview} & \textbf{Status} \\
\midrule
\textbf{Lack of New Employee Security Training} & \textbf{High} & New hires are not provided with security awareness training upon joining, creating a high-risk window for social engineering and policy violations. & \textbf{Active} \\
\addlinespace
Unencrypted Web Server & Medium & Port 80 was previously identified as open, exposing the system to unencrypted communication. & \textbf{Remediated} \\
\bottomrule
\end{tabular}
\caption{Summary of Identified Risks}
\label{tab:risks}
\end{table}

% ----------------------------------------------------------------------
% SECTION 6: RECOMMENDATIONS
% ----------------------------------------------------------------------
\section{Recommendations}
Based on the consolidated risk assessment, the following actions are recommended to enhance the security posture of \textbf{Nexus Dynamics}.

\subsection{High Priority: Implement Onboarding Security Training}
\begin{itemize}
    \item \textbf{Immediate Action:} Develop and mandate a security awareness training module as a required step in the new employee onboarding process. This training should be completed within the first week of employment.
    \item \textbf{Content Focus:} The training should cover, at a minimum: phishing identification, password security, acceptable use of company assets, and procedures for reporting security incidents.
    \item \textbf{Long-Term Strategy:} Integrate the completion of this training into the formal HR onboarding checklist to ensure 100\% compliance.
\end{itemize}

\subsection{Informational: Validate Risk Register}
\begin{itemize}
    \item \textbf{Action:} The technical scan indicates that the risk of an "Unencrypted Web Server" on host \texttt{192.168.0.5} has been resolved. The security team should verify if this remediation applies to all relevant systems.
    \item \textbf{Recommendation:} Update the official risk register to reflect that this vulnerability is closed, and document the date of remediation. This ensures that security resources are focused on active threats.
\end{itemize}

\end{document}
```