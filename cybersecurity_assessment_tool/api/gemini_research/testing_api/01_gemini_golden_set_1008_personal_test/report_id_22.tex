```latex
\documentclass[12pt]{article}

% Preamble: Required Packages
\usepackage[margin=1in]{geometry}
\usepackage{pifont} % For checkmarks and crosses
\usepackage{booktabs} % For professional tables
\usepackage{hyperref} % For hyperlinks
\usepackage{url} % For URL formatting
\usepackage{seqsplit} % For splitting long strings in tt font
\usepackage{graphicx}
\usepackage{xcolor}

% Document Information
\title{Cybersecurity Posture Assessment Report}
\author{Cybersecurity Analysis Division}
\date{\today}

% Hyperlink Setup
\hypersetup{
    colorlinks=true,
    linkcolor=blue,
    filecolor=magenta,      
    urlcolor=cyan,
    pdftitle={Cybersecurity Posture Assessment Report},
    pdfpagemode=FullScreen,
}

% Begin Document
\begin{document}

\maketitle
\thispagestyle{empty}
\newpage

\tableofcontents
\newpage

% --- 1. Executive Overview ---
\section{Executive Overview}
This report provides a comprehensive cybersecurity assessment for \textbf{Vanguard Heritage}. The analysis is based on a correlation of organizational data, a security controls questionnaire, and an external network scan.

The assessment reveals critical deficiencies in fundamental security controls, primarily concerning identity and access management and employee security awareness. While the external network scan of the target IP \texttt{[Target IP]} did not identify any open ports, which is a positive technical finding, the organization's lack of Multi-Factor Authentication (MFA) for email and computer access represents a significant and immediate risk. Furthermore, the absence of an Acceptable Use Policy and a formal security awareness training program creates a high-risk environment susceptible to social engineering and insider threats.

Immediate remediation efforts should focus on implementing MFA across all critical systems, establishing foundational security policies, and deploying a comprehensive employee training program.

% --- 2. Organizational Information ---
\section{Organizational Information}
The following information was provided for the assessment.

\begin{tabular}{@{}ll}
\toprule
\textbf{Attribute} & \textbf{Value} \\
\midrule
Organization Name & \textbf{Vanguard Heritage} \\
Email Domain & \texttt{VanguardHeritage.org} \\
Website Domain & \url{www.VanguardHeritage.org} \\
External IP Address & \texttt{39.27.131.129} \\
\bottomrule
\end{tabular}

% --- 3. Security Control Review ---
\section{Security Control Review}
A review of the organization's security controls was conducted via a questionnaire. The findings below highlight significant gaps between current practices and established security best practices. A green checkmark (\textcolor{green}{\ding{51}}) indicates alignment with best practices, while a red cross (\textcolor{red}{\ding{55}}) indicates a critical gap.

\begin{tabular}{@{}p{0.5\textwidth}cp{0.3\textwidth}@{}}
\toprule
\textbf{Control Question} & \textbf{Status} & \textbf{Finding} \\
\midrule
Do you require MFA to access email? & \textcolor{red}{\ding{55}} & \textbf{Critical Gap:} Lack of MFA on email exposes the primary communication channel to account takeover attacks. \\
\addlinespace
Do you require MFA to log into computers? & \textcolor{red}{\ding{55}} & \textbf{Critical Gap:} Unprotected computer access allows for lateral movement and unauthorized access if credentials are stolen. \\
\addlinespace
Do you require MFA to access sensitive data systems? & \textcolor{green}{\ding{51}} & \textbf{Control in Place:} MFA is correctly applied to protect high-value data systems. \\
\addlinespace
Does your organization have an employee acceptable use policy? & \textcolor{red}{\ding{55}} & \textbf{High Risk:} Without a formal policy, there is no enforceable standard for employee behavior on company systems. \\
\addlinespace
Does your organization do security awareness training for new employees? & \textcolor{red}{\ding{55}} & \textbf{High Risk:} New employees are not equipped with the knowledge to identify and avoid common security threats. \\
\addlinespace
Does your organization do security awareness training for all employees at least once per year? & \textcolor{red}{\ding{55}} & \textbf{High Risk:} The lack of ongoing training allows security knowledge to become stale, increasing susceptibility to phishing and other attacks. \\
\bottomrule
\end{tabular}

% --- 4. Technical Scan Results ---
\section{Technical Scan Results}
An external network vulnerability scan was performed to identify open ports and exposed services.

\begin{itemize}
    \item \textbf{Target IP Address:} \texttt{[Target IP]}
    \item \textbf{Scan Date:} [Scan Date]
\end{itemize}

\subsection{Summary of Findings}
The network scan completed successfully but did not identify any open TCP ports on the target system. This indicates that the host is likely protected by a well-configured firewall that is properly denying external connection attempts. While this is a positive finding from a network security perspective, it does not mitigate the internal and identity-based risks identified in other sections of this report.

% --- 5. Risk Assessment Summary ---
\section{Risk Assessment Summary}
This section synthesizes findings from the security control review, technical scans, and pre-existing risk data. The following table prioritizes the identified risks based on their potential impact and likelihood.

\begin{tabular}{@{}p{0.4\textwidth}p{0.4\textwidth}l@{}}
\toprule
\textbf{Risk Name} & \textbf{Overview} & \textbf{Severity} \\
\midrule
\textbf{Email Account Compromise} & The absence of MFA on email accounts makes them highly vulnerable to credential stuffing and phishing attacks, which could lead to data breaches and financial fraud. & \textbf{Critical} \\
\addlinespace
\textbf{Unauthorized Workstation Access} & Stolen or weak credentials can be used to gain direct access to employee computers, enabling lateral movement, data theft, and ransomware deployment. & \textbf{Critical} \\
\addlinespace
\textbf{Lack of Security Governance} & The absence of an Acceptable Use Policy means there are no defined rules for technology use, creating legal and operational risks. & \textbf{High} \\
\addlinespace
\textbf{Untrained Workforce} & Without a security awareness program, employees are significantly more likely to fall victim to social engineering attacks, introducing malware or compromising sensitive data. & \textbf{High} \\
\bottomrule
\end{tabular}

% --- 6. Recommendations ---
\section{Recommendations}
Based on the analysis, we provide the following prioritized recommendations to mitigate the identified risks and improve the overall security posture of \textbf{Vanguard Heritage}.

\subsection{Priority 1: Immediate Actions (0-30 Days)}
\begin{enumerate}
    \item \textbf{Enforce MFA on All Email Accounts:} Immediately deploy and enforce MFA for all user access to the email system (\texttt{VanguardHeritage.org}). This is the single most effective control to prevent account takeovers.
    \item \textbf{Enforce MFA on All Workstation Logins:} Implement a policy requiring MFA for all employees to log into their company-issued computers. This adds a critical layer of defense against credential theft.
\end{enumerate}

\subsection{Priority 2: Foundational Improvements (30-90 Days)}
\begin{enumerate}
    \item \textbf{Develop and Implement an Acceptable Use Policy (AUP):} Create a formal AUP that clearly defines the rules and responsibilities for all employees when using company technology and data. Require all employees to read and acknowledge the policy.
    \item \textbf{Establish a Security Awareness Training Program:}
    \begin{itemize}
        \item Procure and deploy a training solution for all current employees, covering topics like phishing, password security, and data handling.
        \item Integrate this security training into the onboarding process for all new hires.
        \item Schedule mandatory annual refresher training for all staff.
    \end{itemize}
\end{enumerate}

\end{document}
```