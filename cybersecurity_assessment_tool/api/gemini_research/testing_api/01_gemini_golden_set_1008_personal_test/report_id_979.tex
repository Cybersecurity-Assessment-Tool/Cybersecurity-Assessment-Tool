```latex
\documentclass[12pt]{article}

% Preamble: Required Packages
\usepackage[margin=1in]{geometry}
\usepackage{pifont} % For \ding
\usepackage{booktabs} % For professional tables
\usepackage{hyperref} % For hyperlinks
\usepackage{url} % For URL formatting
\usepackage{seqsplit} % To split long strings in tt font
\usepackage{fancyhdr} % For headers and footers
\usepackage{xcolor} % For colors
\usepackage{graphicx} % For images (e.g., logo)
\usepackage{datetime} % For date formatting

% --- Document Setup ---

% Hyperref settings
\hypersetup{
    colorlinks=true,
    linkcolor=blue,
    filecolor=magenta,      
    urlcolor=cyan,
    pdftitle={Cybersecurity Assessment Report},
    pdfpagemode=FullScreen,
}

% Header and Footer
\pagestyle{fancy}
\fancyhf{} % Clear all header and footer fields
\fancyhead[L]{Cybersecurity Assessment Report}
\fancyhead[R]{Mainframe Managed}
\fancyfoot[C]{\thepage}
\renewcommand{\headrulewidth}{0.4pt}
\renewcommand{\footrulewidth}{0.4pt}

% --- Document Content ---

\begin{document}

% --- Title Page ---
\begin{titlepage}
    \centering
    \vspace*{1cm}
    
    {\Huge\bfseries Cybersecurity Assessment Report\par}
    \vspace{1.5cm}
    
    {\Large\bfseries For\par}
    \vspace{0.5cm}
    {\Huge\bfseries Mainframe Managed\par}
    
    \vfill
    
    {\large
    \textbf{Date of Report:} \today \par
    \vspace{0.5cm}
    \textbf{Author:} Cybersecurity Analyst \par
    }
    
    \vspace{1cm}
    \textit{This report contains sensitive information and should be handled with care.}
    
\end{titlepage}

\tableofcontents
\newpage

% --- Section 1: Executive Summary ---
\section{Executive Summary}

This report provides a cybersecurity assessment for Mainframe Managed, based on a combination of organizational data, a security controls questionnaire, and a network vulnerability scan. The assessment was conducted to identify key risks and provide actionable recommendations to improve the organization's security posture.

The analysis revealed several significant security gaps that require immediate attention. Most critically, the lack of mandatory Multi-Factor Authentication (MFA) for email and computer access represents a \textbf{Critical} risk. Email is a primary vector for phishing and account compromise, while unprotected workstations provide an entry point for attackers to gain internal network access.

Additionally, a technical scan of the external IPv6 address \seqsplit{\texttt{2001:db8::1}} identified an open Secure Shell (SSH) port (22). While necessary for remote administration, a publicly exposed SSH service is a common target for brute-force attacks and requires robust hardening.

The overall security posture is weakened by these fundamental control gaps. We strongly recommend prioritizing the implementation of MFA across all user-facing systems and securing the exposed SSH service as outlined in the recommendations section of this report.

% --- Section 2: Organizational Information ---
\section{Organizational Information}

The following details were provided for the assessment. This information helps establish the context and scope of the review.

\begin{tabular}{@{}ll}
    \toprule
    \textbf{Attribute} & \textbf{Value} \\
    \midrule
    Organization Name & Mainframe Managed \\
    Email Domain & \texttt{MainframeManaged.org} \\
    Website Domain & \url{www.MainframeManaged.org} \\
    External IP Address & \seqsplit{\texttt{71.202.116.226}} \\
    \bottomrule
\end{tabular}

% --- Section 3: Security Control Review ---
\section{Security Control Review}

The following table summarizes the organization's responses to a security controls questionnaire. This review helps identify gaps in administrative and policy-based security measures. Items marked with \ding{55} indicate a deviation from security best practices and are discussed further in the Risk Assessment section.

\begin{tabular}{@{}p{0.6\linewidth} c p{0.25\linewidth}@{}}
    \toprule
    \textbf{Control Question} & \textbf{Response} & \textbf{Assessment} \\
    \midrule
    Do you require MFA to access email? & \textcolor{red}{\ding{55}} & \textbf{Critical Gap.} Lack of MFA on email exposes the organization to phishing and account takeover. \\
    \addlinespace
    Do you require MFA to log into computers? & \textcolor{red}{\ding{55}} & \textbf{High Risk.} Unprotected workstations are vulnerable to credential theft and unauthorized access. \\
    \addlinespace
    Do you require MFA to access sensitive data systems? & \textcolor{green}{\ding{51}} & Best practice is being followed. \\
    \addlinespace
    Does your organization have an employee acceptable use policy? & \textcolor{green}{\ding{51}} & Foundational policy is in place. \\
    \addlinespace
    Does your organization do security awareness training for new employees? & \textcolor{green}{\ding{51}} & Good practice for onboarding. \\
    \addlinespace
    Does your organization do security awareness training for all employees at least once per year? & \textcolor{green}{\ding{51}} & Meets compliance and best practice standards. \\
    \bottomrule
\end{tabular}

% --- Section 4: Technical Scan Results ---
\section{Technical Scan Results}

A network scan was performed on the specified target to identify open ports and exposed services. This scan provides insight into the external attack surface of the organization.

\begin{itemize}
    \item \textbf{Target IP Address:} \seqsplit{\texttt{2001:db8::1}}
    \item \textbf{Scan Date:} \today
\end{itemize}

The following table details the open ports discovered during the scan.

\begin{tabular}{@{}llll@{}}
    \toprule
    \textbf{Port} & \textbf{State} & \textbf{Service (Inferred)} & \textbf{Notes} \\
    \midrule
    22/tcp & Open & SSH (Secure Shell) & The service is exposed to the public internet. This is a common target for automated brute-force attacks. Version information was not available from this scan. \\
    \bottomrule
\end{tabular}

% --- Section 5: Risk Assessment ---
\section{Risk Assessment}

This section synthesizes findings from the security control review and the technical scan into a prioritized list of risks. Each risk is assigned a severity level to guide remediation efforts.

\begin{tabular}{@{}p{0.1\linewidth} p{0.6\linewidth} l@{}}
    \toprule
    \textbf{Risk ID} & \textbf{Description} & \textbf{Severity} \\
    \midrule
    RISK-001 & \textbf{No MFA for Email Access:} User email accounts are protected only by passwords, making them highly susceptible to phishing, credential stuffing, and unauthorized access. A compromised email account can lead to data breaches and further internal compromise. & \textcolor{red}{\textbf{Critical}} \\
    \addlinespace
    RISK-002 & \textbf{No MFA for Computer Logins:} Employee workstations and laptops lack MFA, weakening endpoint security. If user credentials are stolen, an attacker can gain direct access to the internal network, company data, and potentially escalate privileges. & \textcolor{orange}{\textbf{High}} \\
    \addlinespace
    RISK-003 & \textbf{Publicly Exposed SSH Service:} The SSH management port (22) is open to the entire internet on \seqsplit{\texttt{2001:db8::1}}. This exposes the system to continuous brute-force login attempts and potential exploitation if a vulnerability exists in the SSH server software. & \textcolor{yellow!80!black}{\textbf{Medium}} \\
    \bottomrule
\end{tabular}

\textit{Note: No pre-existing vulnerabilities were provided for this assessment.}

% --- Section 6: Recommendations ---
\section{Recommendations}

Based on the identified risks, the following prioritized actions are recommended to enhance the security posture of Mainframe Managed.

\subsection*{Immediate Priority (Critical Risk)}

\begin{itemize}
    \item \textbf{RISK-001: Remediate Lack of Email MFA}
    \begin{itemize}
        \item Immediately enforce a mandatory MFA policy for all user accounts accessing the email system (e.g., Office 365, Google Workspace).
        \item Prioritize authentication methods that are resistant to phishing, such as FIDO2 security keys or authenticator apps over SMS.
    \end{itemize}
\end{itemize}

\subsection*{High Priority}

\begin{itemize}
    \item \textbf{RISK-002: Remediate Lack of Workstation MFA}
    \begin{itemize}
        \item Implement and enforce MFA for all Windows, macOS, and Linux workstation logins.
        \item This control is critical for all employees, especially those with privileged access and remote workers.
    \end{itemize}
\end{itemize}

\subsection*{Medium Priority}

\begin{itemize}
    \item \textbf{RISK-003: Harden Exposed SSH Service}
    \begin{itemize}
        \item \textbf{Restrict Access:} If remote SSH access is required, configure firewall rules to allow connections only from specific, trusted IP addresses (e.g., administrator home or office IPs).
        \item \textbf{Disable Password Authentication:} Configure the SSH server to disallow password-based logins and require public key authentication exclusively. This mitigates the risk of brute-force attacks.
        \item \textbf{Keep Updated:} Ensure the SSH server software is patched and running the latest stable version to protect against known vulnerabilities.
    \end{itemize}
\end{itemize}

\end{document}
```