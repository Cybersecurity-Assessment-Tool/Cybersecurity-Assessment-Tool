```latex
\documentclass[12pt, a4paper]{article}

% Preamble: Required Packages
\usepackage[margin=1in]{geometry}
\usepackage{pifont} % For checkmarks and crosses
\usepackage{booktabs} % For professional tables
\usepackage{hyperref} % For clickable links
\usepackage{url} % For URL formatting
\usepackage{seqsplit} % To split long strings in tt font
\usepackage{graphicx}
\usepackage{xcolor}
\usepackage{fancyhdr}
\usepackage{lastpage}

% --- Document Setup ---
% Define colors for risk levels
\definecolor{criticalred}{HTML}{D32F2F}
\definecolor{highorange}{HTML}{F57C00}
\definecolor{mediumyellow}{HTML}{FBC02D}
\definecolor{lowblue}{HTML}{1976D2}

% Hyperlink setup
\hypersetup{
    colorlinks=true,
    linkcolor=blue,
    filecolor=magenta,      
    urlcolor=cyan,
    pdftitle={Cybersecurity Assessment Report},
    pdfpagemode=FullScreen,
}

% Header and Footer
\pagestyle{fancy}
\fancyhf{} % Clear all header and footer fields
\fancyhead[L]{Cybersecurity Assessment Report}
\fancyhead[R]{Vertex Solutions}
\fancyfoot[C]{\textit{Confidential}}
\fancyfoot[R]{Page \thepage\ of \pageref{LastPage}}
\renewcommand{\headrulewidth}{0.4pt}
\renewcommand{\footrulewidth}{0.4pt}

% --- Document Body ---
\begin{document}

% --- Title Page ---
\begin{titlepage}
    \centering
    \vspace*{1cm}
    \includegraphics[width=0.4\textwidth]{example-image-a} % Placeholder for company logo
    
    \vspace{1.5cm}
    
    \Huge
    \textbf{Cybersecurity Assessment Report}
    
    \vspace{1.5cm}
    
    \Large
    Prepared for: \\
    \vspace{0.5cm}
    \textbf{Vertex Solutions}
    
    \vspace{2cm}
    
    \large
    Date of Report: \today \\
    Scan Date: 2025-11-22
    
    \vfill
    
    \normalsize
    \textit{This document contains sensitive information and is intended solely for the use of the designated recipient. Unauthorized distribution is strictly prohibited.}
    
\end{titlepage}

\tableofcontents
\newpage

% --- Section 1: Executive Summary ---
\section{Executive Summary}
This report provides a comprehensive cybersecurity assessment for \textbf{Vertex Solutions}, based on a combination of technical network scanning, a security controls questionnaire, and a review of pre-existing risks. The assessment was conducted to identify vulnerabilities, policy gaps, and misconfigurations that could expose the organization to cyber threats.

The analysis revealed several areas of significant concern requiring immediate attention. Key findings include:
\begin{itemize}
    \item \textbf{Critical Software Vulnerability:} The external-facing web server is running an outdated and unsupported version of Nginx (1.18.0), which has reached its end-of-life and is known to have multiple security vulnerabilities.
    \item \textbf{Critical Identity and Access Management Gaps:} Multi-Factor Authentication (MFA) is not enforced for accessing critical systems, including email and sensitive data repositories. This significantly increases the risk of account compromise and data breaches.
    \item \textbf{High-Risk Policy Gap:} The organization lacks a formal Acceptable Use Policy (AUP), creating ambiguity for employees regarding the secure use of company assets and data.
    \item \textbf{Medium-Risk Misconfiguration:} The SSL certificate on the web server does not match the organization's domain, which can lead to user trust issues and indicates a potential configuration oversight.
\end{itemize}

While the organization has implemented security awareness training, the identified vulnerabilities in software and access controls undermine these positive efforts. Immediate remediation of these critical and high-risk findings is strongly recommended to reduce the attack surface and enhance the overall security posture of \textbf{Vertex Solutions}.

% --- Section 2: Organizational Information ---
\section{Organizational Information}
The following details were provided for the assessment.

\begin{table}[h!]
\centering
\begin{tabular}{@{}ll@{}}
\toprule
\textbf{Attribute} & \textbf{Value} \\ \midrule
Organization Name & \textbf{Vertex Solutions} \\
Email Domain & \texttt{VertexSolutions.net} \\
Website Domain & \url{www.VertexSolutions.net} \\
External IP Address & \texttt{64.124.53.223} \\ \bottomrule
\end{tabular}
\caption{Client Organizational Data.}
\end{table}

% --- Section 3: Security Control Review ---
\section{Security Control Review}
A security questionnaire was completed to evaluate the current state of administrative and procedural controls. The responses are summarized below. "No" answers indicate significant gaps in the security framework.

\begin{table}[h!]
\centering
\begin{tabular}{@{}lc@{}}
\toprule
\textbf{Control Question} & \textbf{Response} \\ \midrule
Do you require MFA to access email? & \textcolor{criticalred}{\ding{55}} \\
Do you require MFA to log into computers? & \textcolor{green}{\ding{51}} \\
Do you require MFA to access sensitive data systems? & \textcolor{criticalred}{\ding{55}} \\
Does your organization have an employee acceptable use policy? & \textcolor{highorange}{\ding{55}} \\
Does your organization do security awareness training for new employees? & \textcolor{green}{\ding{51}} \\
Does your organization do security awareness training for all employees at least once per year? & \textcolor{green}{\ding{51}} \\ \bottomrule
\end{tabular}
\caption{Security Controls Questionnaire Results.}
\end{table}

\subsection*{Analysis of Controls}
The review highlights critical weaknesses in access control policies. The absence of MFA for email and sensitive data systems are major vulnerabilities. Email is a primary target for phishing attacks, and a compromised account can serve as a gateway to the entire organization. Similarly, sensitive data systems without MFA are highly susceptible to unauthorized access. The lack of an Acceptable Use Policy represents a foundational governance gap that can lead to inconsistent security practices among employees.

% --- Section 4: Technical Scan Results ---
\section{Technical Scan Results}
An external network scan was performed to identify open ports and exposed services on the public-facing infrastructure.

\begin{itemize}
    \item \textbf{Target IP Address:} \texttt{192.168.10.5}
    \item \textbf{Scan Date:} 2025-11-22T10:00:00Z
\end{itemize}

\begin{table}[h!]
\centering
\begin{tabular}{@{}llllll@{}}
\toprule
\textbf{Port} & \textbf{State} & \textbf{Service} & \textbf{Product} & \textbf{Version} & \textbf{Notes} \\ \midrule
443/tcp & open & https & nginx & 1.18.0 & SSL cert CN mismatch \\ \bottomrule
\end{tabular}
\caption{Open Ports and Services Detected.}
\end{table}

\subsection*{Analysis of Technical Findings}
The scan identified one open port, 443 (HTTPS), which is standard for a web server. However, two significant issues were discovered:
\begin{enumerate}
    \item \textbf{Outdated Nginx Version:} The server is running Nginx version 1.18.0. This version reached its official End-of-Life (EOL) in April 2022. It no longer receives security patches, making it vulnerable to numerous publicly known exploits. This is a critical risk.
    \item \textbf{SSL Certificate Mismatch:} The SSL certificate presented by the server has a Common Name (CN) of \texttt{www.acme-corp.com}, which does not match the organization's domain (\texttt{www.VertexSolutions.net}). This will cause browser trust warnings, may deter legitimate users, and could be an indicator of a more significant server misconfiguration.
\end{enumerate}

% --- Section 5: Risk Assessment ---
\section{Risk Assessment}
This section synthesizes findings from the security control review and technical scan to provide a consolidated list of identified risks.

\begin{table}[h!]
\centering
\resizebox{\textwidth}{!}{%
\begin{tabular}{@{}llll@{}}
\toprule
\textbf{ID} & \textbf{Risk Title} & \textbf{Severity} & \textbf{Description} \\ \midrule
\textbf{R-01} & Outdated Nginx Web Server & \textcolor{criticalred}{\textbf{Critical}} & The public-facing web server runs Nginx 1.18.0, which is past its end-of-life and contains known, unpatched vulnerabilities. \\
\textbf{R-02} & No MFA for Email Access & \textcolor{criticalred}{\textbf{Critical}} & Lack of MFA on email accounts makes them highly vulnerable to phishing, credential stuffing, and subsequent account takeover. \\
\textbf{R-03} & No MFA for Sensitive Data & \textcolor{criticalred}{\textbf{Critical}} & Sensitive data systems are protected only by username/password, exposing critical assets to unauthorized access if credentials are compromised. \\
\textbf{R-04} & No Acceptable Use Policy & \textcolor{highorange}{\textbf{High}} & The absence of a formal AUP creates a policy gap, leading to inconsistent security practices and lack of accountability for employees. \\
\textbf{R-05} & SSL Certificate Mismatch & \textcolor{mediumyellow}{\textbf{Medium}} & The SSL certificate's domain does not match the server's domain, eroding user trust and indicating a server misconfiguration. \\ \bottomrule
\end{tabular}%
}
\caption{Summary of Identified Risks.}
\end{table}

% --- Section 6: Recommendations ---
\section{Recommendations}
Based on the risks identified in this assessment, the following actions are recommended to improve the security posture of \textbf{Vertex Solutions}. Recommendations are prioritized based on risk severity.

\begin{enumerate}
    \item \textbf{Upgrade Nginx Server (Risk R-01):}
    \begin{itemize}
        \item \textbf{Action:} Immediately plan and execute an upgrade of the Nginx server from version 1.18.0 to a current, stable, and supported version.
        \item \textbf{Impact:} Mitigates numerous publicly known vulnerabilities, significantly reducing the risk of a server compromise.
    \end{itemize}

    \item \textbf{Enforce MFA on All Email Accounts (Risk R-02):}
    \begin{itemize}
        \item \textbf{Action:} Procure and implement an MFA solution for the organization's email system. Enforce this policy for all users, including administrative and service accounts.
        \item \textbf{Impact:} Drastically reduces the risk of email account takeover, which is a primary vector for broader network intrusions.
    \end{itemize}

    \item \textbf{Deploy MFA for Sensitive Systems (Risk R-03):}
    \begin{itemize}
        \item \textbf{Action:} Identify all systems containing sensitive or critical data and enforce MFA for all user access.
        \item \textbf{Impact:} Provides a critical layer of defense for the organization's most valuable data assets.
    \end{itemize}

    \item \textbf{Develop and Implement an Acceptable Use Policy (Risk R-04):}
    \begin{itemize}
        \item \textbf{Action:} Draft, approve, and disseminate a formal AUP that clearly defines the rules for using company IT assets, data handling, and internet usage. Integrate this policy into the employee onboarding and annual training programs.
        \item \textbf{Impact:} Establishes a clear security baseline for all employees and provides a framework for enforcing security standards.
    \end{itemize}

    \item \textbf{Correct SSL Certificate Configuration (Risk R-05):}
    \begin{itemize}
        \item \textbf{Action:} Replace the current SSL certificate with one that correctly matches the organization's domain (\texttt{www.VertexSolutions.net}). Ensure the server is configured to use the correct certificate.
        \item \textbf{Impact:} Resolves browser trust errors, protects brand reputation, and ensures encrypted traffic is properly validated.
    \end{itemize}
\end{enumerate}

\end{document}
```