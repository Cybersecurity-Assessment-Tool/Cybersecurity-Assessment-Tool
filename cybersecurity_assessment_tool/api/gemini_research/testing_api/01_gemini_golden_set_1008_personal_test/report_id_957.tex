```latex
\documentclass[12pt]{article}

% Preamble: Required Packages
\usepackage[margin=1in]{geometry}
\usepackage{pifont} % For checkmarks and crosses
\usepackage{booktabs} % For professional tables
\usepackage[hidelinks]{hyperref} % For clickable links
\usepackage{url} % For formatting URLs
\usepackage{seqsplit} % To split long strings in tt font
\usepackage{xcolor} % For colors in text

% Document Metadata
\title{Cybersecurity Posture Assessment Report \\ \large For: \textbf{Swift Current Labs}}
\author{Cybersecurity Analysis Division}
\date{\today}

\begin{document}

\maketitle
\tableofcontents
\newpage

\section{Executive Summary}

This report provides a comprehensive cybersecurity assessment for \textbf{Swift Current Labs}, synthesizing data from technical network scans, an organizational security questionnaire, and a review of pre-existing risk documentation.

The assessment has identified a \textbf{critical risk} that requires immediate attention. A network scan of the internal host at \texttt{10.5.5.5} revealed an open service on port \texttt{8080} with the title ``TOP SECRET DB''. This finding directly contradicts pre-existing documentation which classifies this port as secure. This discrepancy suggests either a recent, dangerous misconfiguration or a previously flawed assessment.

Furthermore, significant gaps were identified in foundational security controls. The lack of mandatory Multi-Factor Authentication (MFA) for computer logins, the absence of an employee Acceptable Use Policy (AUP), and the failure to conduct annual security awareness training for all staff represent high-impact risks. These policy and training deficiencies create an environment where technical vulnerabilities are more likely to be introduced and exploited.

Immediate remediation should focus on investigating and securing the exposed service on port \texttt{8080}, followed by the swift implementation of the missing administrative and policy-based controls outlined in this report.

\section{Organizational Information}

The following details were provided for the assessment scope.

\begin{description}
    \item[Organization Name:] \textbf{Swift Current Labs}
    \item[Email Domain:] \texttt{SwiftCurrentLabs.org}
    \item[Website Domain:] \url{www.SwiftCurrentLabs.org}
    \item[External IP Address:] \texttt{167.194.168.160}
\end{description}

\section{Security Control Review}

An analysis of the organization's security questionnaire reveals several key strengths and critical weaknesses in current administrative controls. The following table summarizes the responses provided.

\begin{center}
\begin{tabular}{p{0.8\textwidth}c}
\toprule
\textbf{Control Question} & \textbf{Status} \\
\midrule
Do you require MFA to access email? & \textcolor{green!70!black}{\ding{51}} \\
Do you require MFA to log into computers? & \textcolor{red!90!black}{\ding{55}} \\
Do you require MFA to access sensitive data systems? & \textcolor{green!70!black}{\ding{51}} \\
Does your organization have an employee acceptable use policy? & \textcolor{red!90!black}{\ding{55}} \\
Does your organization do security awareness training for new employees? & \textcolor{green!70!black}{\ding{51}} \\
Does your organization do security awareness training for all employees at least once per year? & \textcolor{red!90!black}{\ding{55}} \\
\bottomrule
\end{tabular}
\end{center}

\subsection*{Analysis of Control Gaps}
The responses marked with a \textcolor{red!90!black}{\ding{55}} indicate significant gaps in the security posture:
\begin{itemize}
    \item \textbf{No MFA for Computer Logins:} This is a critical vulnerability. If an employee's password is compromised, an attacker could gain direct access to the internal network, bypassing other controls.
    \item \textbf{No Acceptable Use Policy (AUP):} An AUP is a foundational document that sets clear expectations for employee behavior regarding company assets and data. Its absence can lead to inconsistent security practices and unintentional insider threats.
    \item \textbf{No Annual Security Training:} Security threats evolve constantly. Failing to provide annual refresher training for all employees means that their knowledge becomes outdated, making them more susceptible to phishing and social engineering attacks.
\end{itemize}

\section{Technical Scan Results}

A network scan was performed on the internal target \texttt{10.5.5.5}. The results indicate a critical finding that requires immediate investigation.

\begin{center}
\begin{tabular}{llll}
\toprule
\textbf{Port} & \textbf{State} & \textbf{Service / Script} & \textbf{Details} \\
\midrule
8080/tcp & OPEN & http-title & \textbf{Title: TOP SECRET DB} \\
\bottomrule
\end{tabular}
\end{center}

\subsection*{Analysis of Technical Findings}
The discovery of an open port (\texttt{8080}) with a service title of ``TOP SECRET DB'' is a finding of the highest severity. This strongly suggests that a sensitive, possibly unauthenticated, database or application interface is exposed on the internal network.

This finding is particularly alarming because it directly contradicts the information provided in the \texttt{Current\_Risks\_JSON} data, which stated: \textit{``Port 8080 is confirmed secure and false positive.''} This indicates a severe disconnect between the documented security posture and the actual state of the network. This discrepancy must be investigated to determine if this is a new, unauthorized service or if the original assessment was critically flawed.

\section{Consolidated Risk Assessment}

The following table synthesizes findings from all data sources into a prioritized list of risks. The risk from the pre-existing documentation regarding Port 8080 has been superseded by the new, high-fidelity technical evidence.

\begin{center}
\begin{tabular}{lp{0.8\textwidth}}
\toprule
\textbf{Risk ID} & \textbf{Risk Details} \\
\midrule
\textbf{RISK-001} & \textbf{Severity: CRITICAL} \\
& \textbf{Exposed Sensitive Data Interface on Internal Network} \\
& An open service on \texttt{10.5.5.5:8080} titled ``TOP SECRET DB'' suggests a highly sensitive system is accessible. This contradicts existing documentation and poses an immediate threat of data exposure or compromise. \\
\midrule
\textbf{RISK-002} & \textbf{Severity: CRITICAL} \\
& \textbf{Lack of MFA for Endpoint Computer Access} \\
& The absence of MFA on computer logins significantly increases the risk of unauthorized access to the internal network via compromised credentials. \\
\midrule
\textbf{RISK-003} & \textbf{Severity: HIGH} \\
& \textbf{Inadequate Security Awareness Program} \\
& Without mandatory annual training, employees are less equipped to defend against evolving threats like phishing, potentially leading to credential theft or malware infection. \\
\midrule
\textbf{RISK-004} & \textbf{Severity: HIGH} \\
& \textbf{Missing Acceptable Use Policy (AUP)} \\
& The lack of a formal AUP creates ambiguity regarding secure practices and the handling of company data, increasing the likelihood of unintentional insider threats and misconfigurations. \\
\bottomrule
\end{tabular}
\end{center}

\section{Recommendations}

The following actions are recommended to mitigate the identified risks, prioritized by severity.

\begin{description}
    \item[For RISK-001 (Exposed Service):]
        \begin{enumerate}
            \item \textbf{Immediate Action:} Investigate the service running on \texttt{10.5.5.5:8080} to identify its purpose and the data it contains.
            \item \textbf{Containment:} If the system is sensitive, immediately apply firewall rules to restrict access to only authorized personnel and systems.
            \item \textbf{Remediation:} Ensure robust authentication and encryption are implemented for the service. If the service is not required, it should be disabled and removed.
            \item \textbf{Review:} Update all risk documentation to reflect the true state of this service.
        \end{enumerate}

    \item[For RISK-002 (MFA Gap):]
        \begin{enumerate}
            \item \textbf{Policy:} Draft and ratify a policy that mandates the use of MFA for all computer and remote access logins.
            \item \textbf{Implementation:} Procure and deploy an MFA solution compatible with the organization's endpoints (e.g., Windows Hello for Business, Duo, etc.) within the next 30 days.
        \end{enumerate}

    \item[For RISK-003 (Training Gap):]
        \begin{enumerate}
            \item \textbf{Program Development:} Establish a formal, mandatory security awareness training program for all employees to be completed annually.
            \item \textbf{Execution:} Enroll all current employees in the training program, with a 90-day deadline for completion.
        \end{enumerate}

    \item[For RISK-004 (AUP Gap):]
        \begin{enumerate}
            \item \textbf{Development:} Draft a comprehensive Acceptable Use Policy covering topics such as data handling, internet usage, password security, and incident reporting.
            \item \textbf{Implementation:} Require all employees to read and formally acknowledge the new policy.
        \end{enumerate}
\end{description}

\end{document}
```