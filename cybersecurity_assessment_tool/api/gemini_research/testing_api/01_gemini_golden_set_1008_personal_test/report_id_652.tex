```latex
\documentclass[12pt]{article}

% Preamble: Required Packages
\usepackage[margin=1in]{geometry}
\usepackage{pifont} % For checkmarks and crosses
\usepackage{booktabs} % For professional tables
\usepackage{hyperref} % For clickable links
\usepackage{url} % For URL formatting
\usepackage{seqsplit} % To split long strings without breaking
\usepackage{xcolor} % For colors
\usepackage{graphicx} % For potential logos/images
\usepackage{lastpage} % To get the total number of pages
\usepackage{fancyhdr} % For headers and footers

% --- Document Setup ---
\hypersetup{
    colorlinks=true,
    linkcolor=blue,
    filecolor=magenta,      
    urlcolor=cyan,
    pdftitle={Cybersecurity Assessment Report},
    pdfauthor={Cybersecurity Analyst},
    pdfsubject={Security Assessment},
    pdfkeywords={Cybersecurity, Risk, Assessment},
    bookmarks=true
}

% --- Custom Colors ---
\definecolor{darkblue}{rgb}{0.0, 0.0, 0.55}
\definecolor{darkred}{rgb}{0.55, 0.0, 0.0}
\definecolor{darkgreen}{rgb}{0.0, 0.39, 0.0}

% --- Header and Footer ---
\pagestyle{fancy}
\fancyhf{} % Clear all header and footer fields
\fancyhead[L]{Cybersecurity Assessment Report}
\fancyhead[R]{Falcon Heavy}
\fancyfoot[C]{\thepage\ of \pageref{LastPage}}
\renewcommand{\headrulewidth}{0.4pt}
\renewcommand{\footrulewidth}{0.4pt}

% --- Document Body ---
\begin{document}

% --- Title Page ---
\begin{titlepage}
    \centering
    \vspace*{1cm}
    
    \Huge
    \textbf{Cybersecurity Assessment Report}
    
    \vspace{1.5cm}
    
    \Large
    Prepared for: \\
    \vspace{0.5cm}
    \textbf{Falcon Heavy}
    
    \vspace{2cm}
    
    \large
    \textbf{Date of Report:} \today \\
    \textbf{Author:} Cybersecurity Analyst
    
    \vfill
    
    \small
    \textit{This document contains sensitive information. Distribution should be limited to authorized personnel only.}
    
\end{titlepage}

\tableofcontents
\newpage

% --- Section 1: Executive Overview ---
\section{Executive Overview}
This report provides a comprehensive cybersecurity assessment for \textbf{Falcon Heavy}, based on a combination of network scanning, a security controls questionnaire, and a review of pre-existing risks.

The assessment reveals a mixed security posture. On one hand, the organization demonstrates a strong network-level defense for the scanned asset, with no exposed services detected. Additionally, a commendable security awareness training program is in place for all employees.

However, several critical gaps in internal security controls were identified that present a significant risk to the organization. The most pressing issues are the lack of multi-factor authentication (MFA) for computer logins and access to sensitive data systems. Furthermore, the absence of a formal Acceptable Use Policy (AUP) represents a foundational governance weakness that can lead to inconsistent security practices and insider threats.

Immediate remediation of these high-risk findings is strongly recommended to protect critical assets and reduce the likelihood of a security breach.

% --- Section 2: Organizational Information ---
\section{Organizational Information}
The following details were provided for the assessment.

\begin{tabular}{@{}ll}
    \toprule
    \textbf{Attribute} & \textbf{Value} \\
    \midrule
    Organization Name & \textbf{Falcon Heavy} \\
    Email Domain & \texttt{FalconHeavy.com} \\
    Website Domain & \seqsplit{\url{www.FalconHeavy.com}} \\
    External IP Address & \texttt{34.186.70.250} \\
    \bottomrule
\end{tabular}

% --- Section 3: Security Control Review ---
\section{Security Control Review}
The following table summarizes the organization's responses to the security controls questionnaire. Items marked with a red 'X' (\textcolor{darkred}{\ding{55}}) indicate significant gaps in the current security framework and are addressed in the Risk Assessment section.

\begin{tabular}{@{}p{0.6\textwidth}cc}
    \toprule
    \textbf{Control Question} & \textbf{Response} & \textbf{Assessment} \\
    \midrule
    Do you require MFA to access email? & \textcolor{darkgreen}{\ding{51}} & Good Practice \\
    Do you require MFA to log into computers? & \textcolor{darkred}{\ding{55}} & \textbf{Critical Gap} \\
    Do you require MFA to access sensitive data systems? & \textcolor{darkred}{\ding{55}} & \textbf{Critical Gap} \\
    Does your organization have an employee acceptable use policy? & \textcolor{darkred}{\ding{55}} & \textbf{High Risk} \\
    Does your organization do security awareness training for new employees? & \textcolor{darkgreen}{\ding{51}} & Good Practice \\
    Does your organization do security awareness training for all employees at least once per year? & \textcolor{darkgreen}{\ding{51}} & Good Practice \\
    \bottomrule
\end{tabular}

% --- Section 4: Technical Scan Results ---
\section{Technical Scan Results}
A network port scan was conducted to identify externally accessible services on the specified target.

\begin{itemize}
    \item \textbf{Target IP Address:} \texttt{192.168.1.100}
    \item \textbf{Scan Summary:} The scan revealed that the host is online, but no open TCP ports were discovered within the top 1000 scanned ports. All ports were in a \texttt{closed} state.
\end{itemize}

\subsection*{Analysis}
The absence of open ports is a positive security finding. It indicates that the target system is either not running any network services or is protected by a well-configured firewall that blocks incoming connections. This significantly reduces the external attack surface of this particular asset.

% --- Section 5: Consolidated Risk Assessment ---
\section{Consolidated Risk Assessment}
This section synthesizes findings from the security control review and technical scans. No pre-existing vulnerabilities were reported. The following new risks have been identified.

\begin{tabular}{@{}p{0.1\textwidth} p{0.25\textwidth} p{0.45\textwidth} p{0.1\textwidth}@{}}
    \toprule
    \textbf{Risk ID} & \textbf{Risk Name} & \textbf{Description} & \textbf{Severity} \\
    \midrule
    RISK-001 & Inadequate Access Control on Endpoints & The lack of MFA on computer logins means that a compromised password could grant an attacker direct access to an employee's workstation and potentially the internal network. & \textbf{High} \\
    \addlinespace
    RISK-002 & Inadequate Access Control for Sensitive Data & The absence of MFA for sensitive data systems is a critical failure. It removes a crucial layer of defense, making high-value data vulnerable to unauthorized access via stolen credentials. & \textbf{Critical} \\
    \addlinespace
    RISK-003 & Missing Governance Policy (AUP) & Without a formal Acceptable Use Policy, there are no clear guidelines for employees on the proper use of company assets. This increases the risk of insider threats, data leakage, and non-compliance. & \textbf{High} \\
    \bottomrule
\end{tabular}

% --- Section 6: Recommendations ---
\section{Recommendations}
Based on the identified risks, the following prioritized actions are recommended to improve the overall security posture of \textbf{Falcon Heavy}.

\subsection*{Immediate Priority (Critical Risk)}
\begin{description}
    \item[Remediates RISK-002:] \textbf{Implement MFA for Sensitive Systems.} Deploy and enforce multi-factor authentication for all user accounts (including administrative and service accounts) that have access to systems containing sensitive, confidential, or proprietary data.
\end{description}

\subsection*{High Priority}
\begin{description}
    \item[Remediates RISK-001:] \textbf{Enforce MFA for Endpoint Access.} Mandate the use of MFA for all employees to log into their company-issued computers (desktops and laptops). This strengthens initial access controls and contains the impact of credential theft.
    
    \item[Remediates RISK-003:] \textbf{Develop and Implement an Acceptable Use Policy (AUP).} Create a formal AUP that clearly defines the rules and responsibilities for all users of IT assets. This policy should be communicated to all employees and acknowledgement should be tracked. It should be reviewed and updated annually.
\end{description}

\end{document}
```