```latex
\documentclass[12pt, a4paper]{article}

% Preamble: Required Packages
\usepackage[margin=1in]{geometry}
\usepackage{pifont} % For checkmarks and crosses
\usepackage{booktabs} % For professional tables
\usepackage{hyperref} % For clickable links
\usepackage{url} % For formatting URLs
\usepackage{seqsplit} % For splitting long strings in texttt
\usepackage[T1]{fontenc}

% Document Metadata
\title{Cybersecurity Posture Assessment Report}
\author{Cybersecurity Analysis Division}
\date{\today}

\begin{document}

\maketitle
\thispagestyle{empty}
\newpage

\tableofcontents
\newpage

% --- 1. Executive Overview ---
\section{Executive Overview}
This report provides a comprehensive analysis of the cybersecurity posture for \textbf{Harbor Light Foundation}. The assessment is based on a correlation of network scan data, organizational security control questionnaires, and a review of pre-existing risk registers.

The analysis identified several critical and high-risk security gaps that require immediate attention. Key findings include the external exposure of an end-of-life database service (\texttt{MySQL 5.7}), a lack of mandatory multi-factor authentication (MFA) for email access, and an insufficient security awareness training program.

Collectively, these vulnerabilities create a significant risk of unauthorized access, data breach, and business disruption. This report details these findings and provides prioritized, actionable recommendations to mitigate the identified risks and strengthen the organization's overall security posture.

% --- 2. Organizational Information ---
\section{Organizational Information}
The following information was provided for the assessment.

\begin{tabular}{@{}ll}
\toprule
\textbf{Attribute} & \textbf{Value} \\
\midrule
Organization Name & Harbor Light Foundation \\
Email Domain & \seqsplit{\texttt{HarborLightFoundation.org}} \\
Website Domain & \seqsplit{\url{www.HarborLightFoundation.org}} \\
External IP Address & \seqsplit{\texttt{76.177.231.198}} \\
\bottomrule
\end{tabular}

% --- 3. Security Control Review ---
\section{Security Control Review}
A review of the organization's security controls was conducted via a questionnaire. The results below highlight current practices. Gaps identified with a \ding{55} (cross) indicate a deviation from security best practices and are addressed in the Risk Assessment section.

\begin{tabular}{@{}lcc}
\toprule
\textbf{Security Control Question} & \textbf{Status} & \textbf{Control Area} \\
\midrule
Do you require MFA to access email? & \ding{55} & Access Control \\
Do you require MFA to log into computers? & \ding{51} & Access Control \\
Do you require MFA to access sensitive data systems? & \ding{51} & Access Control \\
\addlinespace
Does your organization have an employee acceptable use policy? & \ding{51} & Policy \& Governance \\
\addlinespace
Does your organization do security awareness training for new employees? & \ding{51} & Training \& Awareness \\
Does your organization do training for all employees annually? & \ding{55} & Training \& Awareness \\
\bottomrule
\end{tabular}

\vspace{1em}
\noindent \textbf{Note:} \ding{51} indicates 'Yes' (Control in place), \ding{55} indicates 'No' (Control gap).

% --- 4. Technical Scan Results ---
\section{Technical Scan Results}
An external network scan was performed to identify open ports and exposed services.

\subsection{Scan Target}
The scan was conducted against the host at \texttt{172.16.50.20}.

\subsection{Open Ports and Services}
The following open port was discovered:

\begin{tabular}{@{}lllll}
\toprule
\textbf{Port} & \textbf{State} & \textbf{Service} & \textbf{Product} & \textbf{Version} \\
\midrule
3306/tcp & open & mysql & MySQL & 5.7.33 \\
\bottomrule
\end{tabular}

\subsection{Technical Analysis}
\begin{itemize}
    \item \textbf{Exposed Database Service:} The scan confirms that port \texttt{3306}, the default port for \texttt{MySQL}, is open to the network. Exposing database services directly is a critical security risk, as it provides a direct attack vector for threat actors to exploit vulnerabilities, attempt brute-force password attacks, or exfiltrate data.
    \item \textbf{End-of-Life (EOL) Software:} The detected version, \textbf{MySQL 5.7.33}, reached its official End of Life in \textbf{October 2023}. EOL software no longer receives security patches from the vendor, meaning any newly discovered vulnerabilities will remain unpatched. Running EOL software, especially on an internet-facing service, presents a critical and unacceptable risk.
\end{itemize}

% --- 5. Risk Assessment Summary ---
\section{Risk Assessment Summary}
This section synthesizes findings from the security control review, technical scan, and pre-existing risk data into a prioritized list of security risks.

\begin{tabular}{@{}p{0.3\linewidth}p{0.15\linewidth}p{0.5\linewidth}}
\toprule
\textbf{Risk Name} & \textbf{Severity} & \textbf{Overview} \\
\midrule
\textbf{Exposed End-of-Life Database} & \textbf{Critical} & A MySQL 5.7 database is publicly accessible on port 3306. This version is past its End of Life and no longer receives security updates, making it highly vulnerable to exploitation. This finding correlates with and elevates the pre-existing "Database Exposure" risk. \\
\addlinespace
\textbf{Lack of MFA for Email} & \textbf{Critical} & Email accounts are not protected by Multi-Factor Authentication. This creates a high risk of account compromise via phishing or credential stuffing, which can lead to business email compromise (BEC), data breaches, and further network intrusion. \\
\addlinespace
\textbf{Insufficient Security Awareness Training} & \textbf{High} & While new employees receive training, there is no recurring annual training for all staff. This leads to a degradation of security awareness over time, making the organization more susceptible to social engineering and phishing attacks. \\
\bottomrule
\end{tabular}

% --- 6. Recommendations ---
\section{Recommendations}
The following actions are recommended to mitigate the identified risks. Recommendations are prioritized based on severity.

\subsection{Critical Risk: Exposed End-of-Life Database}
\begin{itemize}
    \item \textbf{Immediate (Remediation):} Implement strict firewall rules to block all public access to TCP port \texttt{3306}. Access should be restricted to only trusted internal IP addresses, such as application servers that require database connectivity.
    \item \textbf{Short-Term (Hardening):} Develop and execute a plan to upgrade the MySQL 5.7 instance to a currently supported version (e.g., MySQL 8.x). This is essential to ensure the service receives ongoing security patches.
    \item \textbf{Long-Term (Architectural):} For any required external access, implement a secure access solution such as a VPN or a bastion host. Direct database access from the internet should be prohibited.
\end{itemize}

\subsection{Critical Risk: Lack of MFA for Email}
\begin{itemize}
    \item \textbf{Immediate (Remediation):} Procure and enforce the use of Multi-Factor Authentication for all email accounts immediately. This is one of the most effective controls to prevent unauthorized account access.
\end{itemize}

\subsection{High Risk: Insufficient Security Awareness Training}
\begin{itemize}
    \item \textbf{Short-Term (Implementation):} Establish a mandatory annual security awareness training program for all employees, contractors, and volunteers. The training should cover current threats such as phishing, social engineering, and proper data handling.
    \item \textbf{Long-Term (Maturity):} Enhance the training program with periodic phishing simulation campaigns to test and reinforce employee awareness.
\end{itemize}

\end{document}
```