```latex
\documentclass[12pt]{article}

% Required Packages
\usepackage[margin=1in]{geometry}
\usepackage{pifont} % For checkmarks and crosses
\usepackage{booktabs} % For professional tables
\usepackage{hyperref} % For clickable links
\usepackage{url} % For URL formatting
\usepackage{seqsplit} % For splitting long strings
\usepackage{graphicx}
\usepackage{fancyhdr}
\usepackage[utf8]{inputenc}

% Document Metadata and Hyperlink Setup
\hypersetup{
    colorlinks=true,
    linkcolor=black,
    filecolor=magenta,      
    urlcolor=blue,
    pdftitle={Cybersecurity Posture Assessment Report},
    pdfpagemode=FullScreen,
}

% Header and Footer
\pagestyle{fancy}
\fancyhf{}
\lhead{Cybersecurity Postuadfre Assessment}
\rhead{For: Calyx Botany}
\cfoot{\thepage}

% Document Start
\begin{document}

% Title Page
\begin{titlepage}
    \centering
    \vspace*{1cm}
    \Huge\textbf{Cybersecurity Posture Assessment Report}
    \vspace{1.5cm}
    \large
    \begin{center}
        \includegraphics[width=0.4\textwidth]{example-image-a} % Placeholder logo
    \end{center}
    \vspace{1.5cm}
    \textbf{Prepared for:}\\
    Calyx Botany
    \vspace{2cm}
    \textbf{Date of Report:}\\
    \today
    \vfill
    \textit{This report contains sensitive information and should be handled with care.}
\end{titlepage}

\tableofcontents
\newpage

% --- 1. Executive Overview ---
\section{Executive Overview}
This report details the findings of a cybersecurity assessment for Calyx Botany. The analysis combines a review of organizational security controls, an external network scan, and a review of existing risks.

The assessment identified several critical and high-risk security gaps that require immediate attention. The most significant findings are the absence of Multi-Factor Authentication (MFA) for email and computer access, and a lack of a formal security awareness training program for employees. These deficiencies expose the organization to a high likelihood of account compromise, phishing attacks, and ransomware.

Additionally, a technical scan of the internal network asset at \texttt{172.16.0.1} revealed an open port for unencrypted HTTP traffic, posing a risk of data interception on the local network.

Immediate remediation of the identified risks, particularly the implementation of MFA and employee training, is strongly recommended to significantly improve the organization's security posture.

% --- 2. Organizational Information ---
\section{Organizational Information}
The following details were provided for the assessment.

\begin{table}[h!]
\centering
\begin{tabular}{@{}ll@{}}
\toprule
\textbf{Attribute} & \textbf{Value} \\ \midrule
Organization Name & Calyx Botany \\
Email Domain & \texttt{CalyxBotany.com} \\
Website Domain & \url{www.CalyxBotany.com} \\
External IP Address & \texttt{76.2.249.182} \\ \bottomrule
\end{tabular}
\caption{Client Organizational Details}
\end{table}

% --- 3. Security Control Review ---
\section{Security Control Review}
A review of foundational security controls was conducted via a questionnaire. The responses indicate significant gaps in user access controls and security awareness. The symbol \ding{51} denotes a "Yes" response (control in place), while \ding{55} denotes a "No" response (control gap).

\begin{table}[h!]
\centering
\begin{tabular}{@{}lc@{}}
\toprule
\textbf{Control Question} & \textbf{Response} \\ \midrule
Do you require MFA to access email? & \ding{55} \\
Do you require MFA to log into computers? & \ding{55} \\
Do you require MFA to access sensitive data systems? & \ding{51} \\
Does your organization have an employee acceptable use policy? & \ding{51} \\
Does your organization do security awareness training for new employees? & \ding{55} \\
Does your organization do security awareness training annually? & \ding{55} \\ \bottomrule
\end{tabular}
\caption{Security Controls Questionnaire Results}
\end{table}

% --- 4. Technical Scan Results ---
\section{Technical Scan Results}
A network scan was performed on the specified target to identify open ports and exposed services.

\begin{itemize}
    \item \textbf{Target IP Address:} \texttt{172.16.0.1}
    \item \textbf{Scan Status:} Host is UP
\end{itemize}

The following table details the open ports discovered on the target system.

\begin{table}[h!]
\centering
\begin{tabular}{@{}llll@{}}
\toprule
\textbf{Port} & \textbf{State} & \textbf{Service} & \textbf{Product / Version} \\ \midrule
80/tcp & open & http & N/A (Not specified) \\ \bottomrule
\end{tabular}
\caption{Open Ports on \texttt{172.16.0.1}}
\end{table}

\textbf{Analysis:} The presence of an open port 80 (HTTP) indicates that an unencrypted web service is running. Data transmitted to and from this service is in cleartext and can be intercepted by an attacker on the same network.

% --- 5. Risk Assessment Summary ---
\section{Risk Assessment Summary}
The following risks were identified by correlating the security control gaps and technical findings. The prompt injection attempt in the input data was disregarded as invalid.

\begin{table}[h!]
\centering
\begin{tabular}{@{}p{0.1\linewidth}p{0.4\linewidth}p{0.3\linewidth}p{0.1\linewidth}@{}}
\toprule
\textbf{Risk ID} & \textbf{Risk Description} & \textbf{Affected Asset(s)} & \textbf{Severity} \\ \midrule
RISK-001 & Lack of MFA on email accounts significantly increases the risk of business email compromise (BEC) and phishing success. & Email System, User Accounts, Sensitive Data & Critical \\
\addlinespace
RISK-002 & Lack of MFA on workstations allows an attacker with stolen credentials to gain direct access to the corporate network. & Employee Computers, Internal Network & Critical \\
\addlinespace
RISK-003 & Inadequate security awareness training leaves employees vulnerable to phishing and social engineering attacks. & All Employees, Organizational Data & High \\
\addlinespace
RISK-004 & An unencrypted web service (HTTP) is exposed on the internal network, allowing for potential data interception. & Internal Server \texttt{172.16.0.1} & Moderate \\ \bottomrule
\end{tabular}
\caption{Identified Risks and Severity}
\end{table}

% --- 6. Recommendations ---
\section{Recommendations}
The following actions are recommended to mitigate the identified risks and improve the overall security posture of Calyx Botany.

\subsection{RISK-001 \& RISK-002: Implement Mandatory MFA}
\begin{itemize}
    \item \textbf{Immediate Action:} Enforce mandatory Multi-Factor Authentication (MFA) across all user accounts for both email access (e.g., Microsoft 365, Google Workspace) and workstation logins (e.g., Windows Hello, Duo).
    \item \textbf{Long-Term Strategy:} Develop a formal identity and access management policy that mandates MFA for all systems and applications, prioritizing those that handle sensitive data.
\end{itemize}

\subsection{RISK-003: Establish a Security Awareness Program}
\begin{itemize}
    \item \textbf{Immediate Action:} Enroll all current employees in a foundational security awareness training course. This should be a mandatory part of the onboarding process for all new hires.
    \item \textbf{Long-Term Strategy:} Implement a continuous security training program that includes annual refresher courses and regular phishing simulations to keep employees vigilant against evolving threats.
\end{itemize}

\subsection{RISK-004: Remediate Unencrypted Service}
\begin{itemize}
    \item \textbf{Immediate Action:} Identify the purpose of the web service running on port 80 on the server at \texttt{172.16.0.1}.
    \item \textbf{Long-Term Strategy:} If the service is necessary, reconfigure it to use HTTPS (port 443) with a valid TLS certificate to encrypt all traffic. If the service is not required, it should be disabled and the port closed to reduce the attack surface.
\end{itemize}

\end{document}
```