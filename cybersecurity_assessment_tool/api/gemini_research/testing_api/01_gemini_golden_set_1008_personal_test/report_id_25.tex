```latex
\documentclass[12pt]{article}

% Preamble: Required Packages
\usepackage[margin=1in]{geometry}
\usepackage{pifont} % For checkmarks and crosses
\usepackage{booktabs} % For professional tables
\usepackage{hyperref} % For clickable links
\usepackage{url} % For URL formatting
\usepackage{seqsplit} % To split long monospaced text
\usepackage{graphicx} % For potential logos (not used here, but good practice)
\usepackage{xcolor} % For colors in text

% Document Metadata
\title{Cybersecurity Assessment Report}
\author{Cybersecurity Analysis Division}
\date{November 22, 2025}

\begin{document}

\maketitle
\thispagestyle{empty}
\newpage
\tableofcontents
\newpage

% --- 1. Executive Summary ---
\section{Executive Summary}

This report details the findings of a cybersecurity assessment for \textbf{Verve \& Vigor}, conducted on November 22, 2025. The assessment combined a review of organizational security controls, an external network scan, and an analysis of known risks.

The analysis revealed several critical and high-risk security gaps that require immediate attention. Most notably, the lack of Multi-Factor Authentication (MFA) on the organization's email system (\texttt{VerveVigor.org}) presents a \textbf{critical risk} of business email compromise and unauthorized account access.

Furthermore, the organization lacks a formal security awareness training program and an acceptable use policy, significantly increasing its susceptibility to social engineering attacks like phishing.

Technical scanning identified an internet-facing web server running an outdated and vulnerable version of Nginx (1.18.0). This exposes the organization to potential exploitation from external threats. Actionable recommendations to mitigate these identified risks are provided in the final section of this report.

% --- 2. Organizational Information ---
\section{Organizational Information}

The following details were provided for the assessment. This information is used to establish the context and scope of the review.

\begin{tabular}{@{}ll}
\toprule
\textbf{Attribute} & \textbf{Value} \\
\midrule
Organization Name & \textbf{Verve \& Vigor} \\
Email Domain & \texttt{VerveVigor.org} \\
Website Domain & \url{www.VerveVigor.org} \\
External IP Address & \texttt{3.199.243.42} \\
\bottomrule
\end{tabular}

% --- 3. Security Control Review ---
\section{Security Control Review}

A review of administrative and organizational security controls was conducted via a questionnaire. The responses indicate significant gaps in foundational security practices. A "No" response highlights a missing control and a potential area of high risk.

\begin{table}[h!]
\centering
\begin{tabular}{@{}p{0.8\textwidth}c@{}}
\toprule
\textbf{Control Question} & \textbf{Response} \\
\midrule
Do you require MFA to log into computers? & \ding{51} \\ % Yes
Do you require MFA to access sensitive data systems? & \ding{51} \\ % Yes
\textcolor{red}{Do you require MFA to access email?} & \textcolor{red}{\ding{55}} \\ % No
\textcolor{red}{Does your organization have an employee acceptable use policy?} & \textcolor{red}{\ding{55}} \\ % No
\textcolor{red}{Does your organization do security awareness training for new employees?} & \textcolor{red}{\ding{55}} \\ % No
\textcolor{red}{Does your organization do security awareness training for all employees at least once per year?} & \textcolor{red}{\ding{55}} \\ % No
\bottomrule
\end{tabular}
\caption{Organizational Security Control Status. (\ding{51} = Yes, \ding{55} = No)}
\end{table}

The lack of MFA for email and the complete absence of a security awareness program are critical deficiencies that elevate the organization's risk profile.

% --- 4. Technical Scan Results ---
\section{Technical Scan Results}

A network scan was performed on \texttt{192.168.10.5} on November 22, 2025. The scan identified one open port with a service exposed to the network.

\subsection{Open Ports}
\begin{table}[h!]
\centering
\begin{tabular}{@{}llll@{}}
\toprule
\textbf{Port} & \textbf{State} & \textbf{Service} & \textbf{Product / Version} \\
\midrule
443/TCP & Open & HTTPS & Nginx / \textcolor{red}{1.18.0} \\
\bottomrule
\end{tabular}
\caption{Open Ports Detected on \texttt{192.168.10.5}.}
\end{table}

\subsection{Analysis of Findings}
\begin{itemize}
    \item \textbf{Outdated Web Server:} The server is running \textbf{Nginx version 1.18.0}, which was released in April 2020. This version is considered end-of-life and is known to be vulnerable to multiple security exploits (e.g., CVE-2021-23017). Running outdated software on an internet-facing service presents a high risk of compromise.
    \item \textbf{SSL Certificate Mismatch:} The SSL certificate presented by the service has a Common Name of \seqsplit{\texttt{www.acme-corp.com}}. This does not match the organization's domain (\texttt{www.VerveVigor.org}), indicating a server misconfiguration. This can lead to browser trust errors and may confuse users, potentially aiding phishing attacks.
\end{itemize}

% --- 5. Risk Assessment Summary ---
\section{Risk Assessment Summary}

The following table synthesizes findings from the security control review and technical scan into a prioritized list of risks. No pre-existing vulnerabilities were reported.

\begin{table}[h!]
\centering
\begin{tabular}{@{}lp{0.3\textwidth}p{0.4\textwidth}l@{}}
\toprule
\textbf{ID} & \textbf{Risk Name} & \textbf{Description} & \textbf{Severity} \\
\midrule
RISK-001 & Lack of MFA on Email & The absence of MFA on email accounts allows for account takeover with only a compromised password. & \textbf{Critical} \\
\addlinespace
RISK-002 & Outdated Web Server Software & The public-facing Nginx server is running an old, unsupported version with known vulnerabilities. & \textbf{High} \\
\addlinespace
RISK-003 & Lack of Security Awareness Program & Employees are not trained on security best practices, making them highly vulnerable to phishing and social engineering. & \textbf{High} \\
\addlinespace
RISK-004 & Missing Acceptable Use Policy & There is no formal policy governing the use of company assets, leading to potential misuse and insider threats. & Medium \\
\addlinespace
RISK-005 & SSL Certificate Mismatch & The web server's certificate does not match the company domain, indicating a misconfiguration. & Low \\
\bottomrule
\end{tabular}
\caption{Summary of Identified Risks.}
\end{table}

% --- 6. Recommendations ---
\section{Recommendations}

Based on the risks identified, the following actions are recommended to improve the cybersecurity posture of \textbf{Verve \& Vigor}.

\begin{enumerate}
    \item \textbf{[Critical] Implement MFA for Email:} Immediately enforce mandatory multi-factor authentication (MFA) for all user accounts on the \texttt{VerveVigor.org} email system. This is the single most effective control to mitigate RISK-001.

    \item \textbf{[High] Upgrade Web Server:} Plan and execute an upgrade for the Nginx web server at \texttt{192.168.10.5} from version 1.18.0 to the latest stable release. This will patch known vulnerabilities and mitigate RISK-002.

    \item \textbf{[High] Establish a Security Awareness Program:} Develop and implement a mandatory security awareness training program. This program should be required for all new hires and conducted annually for all staff to address RISK-003.

    \item \textbf{[Medium] Develop an Acceptable Use Policy (AUP):} Draft and enforce an AUP that clearly defines the rules for using company technology and data. All employees should be required to read and acknowledge this policy to mitigate RISK-004.

    \item \textbf{[Low] Correct SSL Certificate Configuration:} Investigate the web server configuration on \texttt{192.168.10.5} and install a valid SSL certificate that matches the organization's domain (\texttt{www.VerveVigor.org}) to resolve RISK-005.
\end{enumerate}

\end{document}
```