```latex
\documentclass[12pt]{article}

% Preamble: Required Packages
\usepackage[margin=1in]{geometry}
\usepackage{pifont} % For checkmarks and crosses
\usepackage{booktabs} % For professional tables
\usepackage{hyperref} % For clickable links
\usepackage{url} % For formatting URLs
\usepackage{seqsplit} % For splitting long text strings
\usepackage{graphicx}
\usepackage{xcolor}

% Document Metadata
\title{Cybersecurity Posture Assessment Report}
\author{Cybersecurity Analysis Division}
\date{\today}

% Hyperref Setup
\hypersetup{
    colorlinks=true,
    linkcolor=blue,
    filecolor=magenta,      
    urlcolor=cyan,
    pdftitle={Cybersecurity Posture Assessment Report},
    pdfpagemode=FullScreen,
}

\begin{document}

\maketitle
\thispagestyle{empty}
\newpage
\tableofcontents
\newpage

% --- 1. Executive Summary ---
\section{Executive Summary}

This report provides a comprehensive cybersecurity assessment for \textbf{True North Travel}, based on network scan data, an organizational security questionnaire, and a review of pre-existing risks. The analysis was conducted on \today.

The assessment reveals critical deficiencies in foundational security controls. The most significant concerns are the widespread lack of Multi-Factor Authentication (MFA) for computers and sensitive systems, a complete absence of security policies and employee training, and the exposure of an unencrypted HTTP service on the internal network.

These gaps create a high-risk environment where a single compromised credential could lead to significant data exposure and lateral movement within the network. Immediate and decisive action is required to implement the recommendations outlined in this report to mitigate these risks and establish a baseline of security for the organization.

% --- 2. Organizational Information ---
\section{Organizational Information}

The following details were provided for the assessment. This information is used to establish the context and scope of the analysis.

\begin{itemize}
    \item \textbf{Organization Name:} True North Travel
    \item \textbf{Primary Email Domain:} \texttt{TrueNorthTravel.net}
    \item \textbf{Primary Website Domain:} \url{www.TrueNorthTravel.net}
    \item \textbf{External IP Address:} \seqsplit{\texttt{16.164.160.248}}
\end{itemize}

% --- 3. Security Control Review (Questionnaire Analysis) ---
\section{Security Control Review}

A review of the organization's security practices was conducted via a questionnaire. The responses indicate major gaps in administrative and access controls. A "No" response (\ding{55}) highlights a deviation from security best practices and introduces significant risk.

\begin{table}[h!]
\centering
\caption{Security Controls Questionnaire Analysis}
\begin{tabular}{@{}p{0.6\linewidth} c p{0.25\linewidth}@{}}
\toprule
\textbf{Control Question} & \textbf{Response} & \textbf{Assessment} \\
\midrule
Do you require MFA to access email? & \ding{51} & \textbf{Good Practice.} Email is a primary target, and MFA is a critical defense. \\
\addlinespace
Do you require MFA to log into computers? & \textbf{\color{red}\ding{55}} & \textbf{Critical Gap.} Lack of MFA on endpoints allows for trivial lateral movement if credentials are stolen. \\
\addlinespace
Do you require MFA to access sensitive data systems? & \textbf{\color{red}\ding{55}} & \textbf{Critical Gap.} The organization's most valuable data is not adequately protected from unauthorized access. \\
\addlinespace
Does your organization have an employee acceptable use policy? & \textbf{\color{red}\ding{55}} & \textbf{High Risk.} Without a policy, there are no clear rules for employees, leading to inconsistent and insecure behavior. \\
\addlinespace
Does your organization do security awareness training for new employees? & \textbf{\color{red}\ding{55}} & \textbf{High Risk.} New staff are not equipped with the knowledge to identify and avoid common threats like phishing. \\
\addlinespace
Does your organization do security awareness training for all employees at least once per year? & \textbf{\color{red}\ding{55}} & \textbf{High Risk.} Security is a continuous process. Lack of recurring training allows security knowledge to become outdated. \\
\bottomrule
\end{tabular}
\end{table}

% --- 4. Technical Scan Results ---
\section{Technical Scan Results}

An Nmap scan was performed to identify open ports and services on the specified target. The scan identified one open port running an unencrypted service.

\subsection{Scan Details}
\begin{itemize}
    \item \textbf{Target IP Address:} \texttt{172.16.0.1}
    \item \textbf{Scan Type:} TCP Port Scan
\end{itemize}

\subsection{Open Ports Discovered}
The following table details the open ports found on the target system.

\begin{table}[h!]
\centering
\caption{Open Port Analysis for \texttt{172.16.0.1}}
\begin{tabular}{@{}llll@{}}
\toprule
\textbf{Port} & \textbf{State} & \textbf{Likely Service} & \textbf{Analysis} \\
\midrule
80/tcp & Open & HTTP & \textbf{Critical Finding.} This port is used for unencrypted web traffic. Any data, including potential login credentials or sensitive information, is sent in cleartext and can be easily intercepted by an attacker on the same network. \\
\bottomrule
\end{tabular}
\end{table}

\subsection{Pre-existing Risk Review}
The provided list of current risks was analyzed. The single entry was determined to be invalid test data and was excluded from this assessment. No other pre-existing vulnerabilities were provided for correlation.

% --- 5. Consolidated Risk Assessment ---
\section{Consolidated Risk Assessment}

This section synthesizes the findings from the security control review and the technical scan into a consolidated list of identified risks.

\begin{table}[h!]
\centering
\caption{Summary of Identified Risks}
\begin{tabular}{@{}p{0.1\linewidth} p{0.25\linewidth} p{0.45\linewidth} l@{}}
\toprule
\textbf{ID} & \textbf{Risk Title} & \textbf{Description} & \textbf{Severity} \\
\midrule
RISK-001 & Lack of Multi-Factor Authentication (MFA) & MFA is not enforced on computers or sensitive data systems. A compromised password directly leads to system access. & \textbf{Critical} \\
\addlinespace
RISK-002 & Unencrypted Internal Web Service & An HTTP service is exposed on the internal network, transmitting data in cleartext. This could expose credentials or sensitive data to network eavesdropping. & \textbf{High} \\
\addlinespace
RISK-003 & Inadequate Security Policies and Training & The absence of an acceptable use policy and any form of security awareness training leaves the organization vulnerable to human error, phishing, and insider threats. & \textbf{High} \\
\bottomrule
\end{tabular}
\end{table}

% --- 6. Recommendations ---
\section{Recommendations}

The following actions are recommended to mitigate the identified risks and improve the overall security posture of \textbf{True North Travel}. Recommendations are prioritized based on severity and impact.

\subsection{RISK-001: Remediate Lack of MFA (Priority: Immediate)}
\begin{itemize}
    \item \textbf{Action:} Procure and deploy an MFA solution for all employee computer logins.
    \item \textbf{Action:} Identify all systems containing sensitive data and enforce MFA for access.
    \item \textbf{Justification:} This is the single most effective control to prevent unauthorized access from compromised credentials. It should be the top priority.
\end{itemize}

\subsection{RISK-002: Secure Unencrypted Web Service (Priority: High)}
\begin{itemize}
    \item \textbf{Action:} Identify the business purpose of the service running on \texttt{172.16.0.1:80}.
    \item \textbf{Action:} If the service is necessary, reconfigure it to use HTTPS (Port 443) with a valid internal TLS certificate to encrypt all traffic.
    \item \textbf{Action:} If the service is not necessary, it should be disabled and the port firewalled to reduce the attack surface.
\end{itemize}

\subsection{RISK-003: Establish Security Policies and Training (Priority: High)}
\begin{itemize}
    \item \textbf{Action:} Develop and ratify a formal Employee Acceptable Use Policy that all employees must read and sign.
    \item \textbf{Action:} Implement a mandatory security awareness training program for all new hires as part of their onboarding process.
    \item \textbf{Action:} Schedule and conduct annual security awareness training for all staff to ensure their knowledge remains current on threats such as phishing, social engineering, and malware.
\end{itemize}

\end{document}
```