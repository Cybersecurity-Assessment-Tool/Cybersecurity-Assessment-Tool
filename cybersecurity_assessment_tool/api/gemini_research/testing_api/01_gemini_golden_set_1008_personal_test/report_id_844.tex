```latex
\documentclass[12pt]{article}

% --- PACKAGES ---
\usepackage[margin=1in]{geometry} % Set page margins
\usepackage{pifont}               % For checkmarks and crosses
\usepackage{booktabs}             % For professional-looking tables
\usepackage{hyperref}             % For hyperlinks
\usepackage{url}                  % For URL formatting
\usepackage{seqsplit}             % For splitting long strings without spaces
\usepackage{graphicx}             % For logos (optional)
\usepackage{xcolor}               % For color

% --- DOCUMENT SETUP ---
\hypersetup{
    colorlinks=true,
    linkcolor=blue,
    filecolor=magenta,      
    urlcolor=cyan,
    pdftitle={Cybersecurity Posture Assessment},
    pdfpagemode=FullScreen,
}

\newcommand{\yes}{\ding{51}}
\newcommand{\no}{\ding{55}}

% --- TITLE ---
\title{Cybersecurity Posture Assessment Report}
\author{Cybersecurity Analysis Division}
\date{\today}

% --- BEGIN DOCUMENT ---
\begin{document}

\maketitle
\thispagestyle{empty}
\newpage

\tableofcontents
\newpage

% ===================================================================
\section{Executive Summary}
% ===================================================================
This report provides a comprehensive cybersecurity assessment for \textbf{Skyward Bound}, conducted on \today. The analysis synthesizes data from a network vulnerability scan, a review of organizational security controls, and a list of pre-existing risks.

The assessment reveals a mixed security posture. On the technical front, the external network scan of the target host (\texttt{192.168.1.100}) showed a strong perimeter, with \textbf{no open ports detected}. This significantly reduces the external attack surface for the scanned asset.

However, the organizational security control review identified two high-impact gaps:
\begin{itemize}
    \item \textbf{Critical Risk:} Lack of Multi-Factor Authentication (MFA) for accessing sensitive data systems.
    \item \textbf{High Risk:} Absence of mandatory security awareness training for new employees during their onboarding process.
\end{itemize}

These procedural and policy-based weaknesses expose the organization to significant risks, including unauthorized data access and increased susceptibility to social engineering attacks. This report provides detailed findings and actionable recommendations to mitigate these identified risks and enhance the overall security posture of \textbf{Skyward Bound}.

% ===================================================================
\section{Organizational Information}
% ===================================================================
The following information was provided for the assessment.

\begin{tabular}{@{}ll}
\toprule
\textbf{Attribute} & \textbf{Value} \\
\midrule
Organization Name & \textbf{Skyward Bound} \\
Primary Email Domain & \texttt{SkywardBound.net} \\
Primary Website & \url{www.SkywardBound.net} \\
External IP Address & \texttt{55.176.67.51} \\
\bottomrule
\end{tabular}

% ===================================================================
\section{Security Control Review}
% ===================================================================
A review of organizational security controls was conducted based on a standardized questionnaire. The responses highlight areas of both strength and weakness in the current security policies and procedures. Gaps identified with a \no{} represent a deviation from security best practices and are addressed in the risk assessment section.

\begin{table}[h!]
\centering
\begin{tabular}{@{}p{0.7\linewidth} c c@{}}
\toprule
\textbf{Control Question} & \textbf{Response} & \textbf{Status} \\
\midrule
Do you require MFA to access email? & Yes & \yes \\
Do you require MFA to log into computers? & Yes & \yes \\
\textbf{Do you require MFA to access sensitive data systems?} & \textbf{No} & \textcolor{red}{\no} \\
Does your organization have an employee acceptable use policy? & Yes & \yes \\
\textbf{Does your organization do security awareness training for new employees?} & \textbf{No} & \textcolor{red}{\no} \\
Does your organization do security awareness training for all employees at least once per year? & Yes & \yes \\
\bottomrule
\end{tabular}
\caption{Organizational Security Control Questionnaire Results.}
\end{label{tab:controls}
\end{table}

% ===================================================================
\section{Technical Scan Results}
% ===================================================================
An external network scan was performed on the specified target to identify open ports, running services, and potential vulnerabilities.

\begin{itemize}
    \item \textbf{Target IP Address:} \texttt{192.168.1.100}
    \item \textbf{Scan Date:} \today
    \item \textbf{Summary of Findings:} The scan completed successfully and determined the host to be online. However, all scanned ports were found to be in a \textbf{`closed`} state. No open TCP or UDP ports were discovered.
\end{itemize}

\subsection{Analysis}
The absence of open ports on the scanned host is a positive security finding. It indicates a well-configured firewall or network access control list (ACL) that effectively blocks unsolicited inbound traffic, minimizing the external attack surface of this particular asset. While this is a strong defensive posture, it does not eliminate risks from other vectors such as phishing or vulnerabilities in web applications hosted elsewhere.

% ===================================================================
\section{Identified Risks and Vulnerabilities}
% ===================================================================
This section correlates findings from the organizational control review and the technical scan. The following risks have been identified and prioritized based on their potential impact on the organization.

\begin{table}[h!]
\centering
\begin{tabular}{@{}p{0.25\linewidth} p{0.55\linewidth} p{0.1\linewidth}@{}}
\toprule
\textbf{Risk Name} & \textbf{Overview} & \textbf{Severity} \\
\midrule
\textbf{Lack of MFA for Sensitive Data} & Sensitive data systems are accessible with only a username and password. This exposes critical data to unauthorized access in the event of credential compromise through phishing, brute-force attacks, or password reuse. & \textbf{Critical} \\
\addlinespace
\textbf{No Onboarding Security Training} & New employees are not provided with security awareness training upon joining. This creates a window of high vulnerability where new hires are more susceptible to social engineering and may mishandle sensitive data due to a lack of awareness of company policies. & \textbf{High} \\
\bottomrule
\end{tabular}
\caption{Summary of Identified Risks.}
\label{tab:risks}
\end{table}

\textit{Note: No pre-existing vulnerabilities were reported, and no technical vulnerabilities were discovered during the network scan of the target host.}

% ===================================================================
\section{Recommendations}
% ===================================================================
Based on the findings of this assessment, the following actions are recommended to mitigate the identified risks and strengthen the overall security posture of \textbf{Skyward Bound}.

\begin{enumerate}
    \item \textbf{Implement MFA on All Sensitive Systems (Critical Priority):}
    \begin{itemize}
        \item \textbf{Action:} Immediately begin the process of deploying and enforcing Multi-Factor Authentication (MFA) for all user accounts that have access to systems containing sensitive, proprietary, or regulated data.
        \item \textbf{Justification:} This is the single most effective control to prevent unauthorized access resulting from compromised credentials. It adds a critical layer of defense for the organization's most valuable data assets.
    \end{itemize}
    \vspace{0.5cm}
    \item \textbf{Establish a New Employee Security Training Program (High Priority):}
    \begin{itemize}
        \item \textbf{Action:} Develop and integrate a mandatory security awareness training module into the new employee onboarding process. This training should be completed within the first week of employment.
        \item \textbf{Justification:} A well-informed employee is the first line of defense. Training new hires immediately reduces the risk of early-tenure security incidents and establishes a culture of security from day one. Key topics should include phishing identification, password hygiene, and the acceptable use policy.
    \end{itemize}
    \vspace{0.5cm}
    \item \textbf{Maintain Strong Network Perimeter Security (Ongoing):}
    \begin{itemize}
        \item \textbf{Action:} The strong perimeter security observed on the scanned host should be maintained. Continue conducting regular, authenticated vulnerability scans across all external and internal assets to ensure this posture is consistent throughout the network.
        \item \textbf{Justification:} The principle of "least privilege" should apply to network services. Regularly verifying that no unnecessary ports are open helps prevent accidental exposure from misconfigurations or new service deployments.
    \end{itemize}
\end{enumerate}

\end{document}
```