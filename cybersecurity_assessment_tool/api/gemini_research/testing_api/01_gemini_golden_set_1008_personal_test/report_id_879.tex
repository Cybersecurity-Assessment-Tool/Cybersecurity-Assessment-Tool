```latex
\documentclass[12pt, a4paper]{article}

% Preamble: Required Packages
\usepackage[margin=1in]{geometry}
\usepackage{pifont} % For checkmarks and crosses (\ding)
\usepackage{booktabs} % For professional tables
\usepackage{hyperref} % For clickable links
\usepackage{url} % For formatting URLs
\usepackage{seqsplit} % For splitting long strings in tt font
\usepackage{graphicx}
\usepackage[table]{xcolor} % For coloring table cells
\usepackage{tocloft} % For table of contents customization

% --- Document Setup ---
% Define colors for severity levels
\definecolor{criticalred}{HTML}{D10000}
\definecolor{highorange}{HTML}{E57200}
\definecolor{mediumyellow}{HTML}{FFC72C}
\definecolor{lowblue}{HTML}{007398}
\definecolor{infogray}{HTML}{F2F2F2}

% Hyperref setup
\hypersetup{
    colorlinks=true,
    linkcolor=blue,
    filecolor=magenta,      
    urlcolor=cyan,
    pdftitle={Cybersecurity Posture Assessment Report},
    pdfpagemode=FullScreen,
}

% --- Document Start ---
\begin{document}

% --- Title Page ---
\begin{titlepage}
    \centering
    \vspace*{\stretch{1.0}}
    {\Huge \textbf{Cybersecurity Posture Assessment Report}\par}
    \vspace{1.5cm}
    {\Large Prepared for:\par}
    \vspace{0.5cm}
    {\huge \textbf{Aeon Pharmaceuticals}\par}
    \vspace{2cm}
    {\Large \today\par}
    \vspace*{\stretch{2.0}}
    \vfill
    {\large \textit{This report contains sensitive information and should be handled with care.}\par}
\end{titlepage}

% --- Table of Contents ---
\newpage
\tableofcontents
\newpage

% --- Section 1: Executive Summary ---
\section{Executive Summary}
This report provides a comprehensive analysis of the cybersecurity posture for \textbf{Aeon Pharmaceuticals}, based on network scans, a security controls questionnaire, and a review of pre-existing risks. The assessment was conducted to identify vulnerabilities, security control gaps, and areas of non-compliance with cybersecurity best practices.

The analysis revealed several critical and high-risk findings that require immediate attention. Key among these is the external exposure of a database service (\texttt{MySQL}) running an unsupported, end-of-life version. This presents a significant risk of data breach.

Furthermore, critical gaps were identified in internal security controls, including the lack of Multi-Factor Authentication (MFA) for computer logins and the absence of security awareness training for new employees. These weaknesses in the "human firewall" and endpoint security significantly increase the organization's susceptibility to credential theft and social engineering attacks.

This report details these findings and provides actionable recommendations to mitigate the identified risks and strengthen the overall security posture of the organization.

% --- Section 2: Organizational Information ---
\section{Organizational Information}
The following details were provided for the assessment. This information helps to establish the context and scope of the review.

\begin{tabular}{@{}ll}
\toprule
\textbf{Attribute} & \textbf{Value} \\
\midrule
Organization Name & \textbf{Aeon Pharmaceuticals} \\
Email Domain & \texttt{AeonPharmaceuticals.net} \\
Website Domain & \url{www.AeonPharmaceuticals.net} \\
External IP Address & \texttt{201.117.200.82} \\
\bottomrule
\end{tabular}

% --- Section 3: Security Control Review ---
\section{Security Control Review}
A security questionnaire was completed to evaluate the implementation of fundamental security controls. The results highlight critical gaps in endpoint security and employee training protocols. A "No" response indicates a deviation from security best practices.

\begin{table}[h!]
\centering
\caption{Security Controls Questionnaire Results}
\begin{tabular}{p{0.6\linewidth} c c}
\toprule
\textbf{Control Question} & \textbf{Response} & \textbf{Status} \\
\midrule
Do you require MFA to access email? & \ding{51} & Implemented \\
Do you require MFA to log into computers? & \ding{55} & \cellcolor{criticalred!25}Critical Gap \\
Do you require MFA to access sensitive data systems? & \ding{51} & Implemented \\
Does your organization have an employee acceptable use policy? & \ding{51} & Implemented \\
Does your organization do security awareness training for new employees? & \ding{55} & \cellcolor{highorange!25}High Risk Gap \\
Does your organization do security awareness training for all employees at least once per year? & \ding{51} & Implemented \\
\bottomrule
\end{tabular}
\end{table}

% --- Section 4: Technical Scan Results ---
\section{Technical Scan Results}
A network scan was performed on the target system to identify open ports and exposed services. The scan confirmed the presence of a publicly accessible database service.

\subsection{Scan Summary}
\begin{itemize}
    \item \textbf{Target IP:} \texttt{172.16.50.20}
    \item \textbf{Status:} Host is Up
    \item \textbf{Key Finding:} One critical service was found to be open and running an outdated version.
\end{itemize}

\subsection{Open Ports and Services}
The following table details the services discovered during the network scan.

\begin{table}[h!]
\centering
\caption{Discovered Network Services}
\begin{tabular}{l l l l l}
\toprule
\textbf{Port} & \textbf{State} & \textbf{Service} & \textbf{Version} & \textbf{Analyst Notes} \\
\midrule
3306/tcp & Open & MySQL & 5.7.33 & \cellcolor{criticalred!25}\textbf{End-of-Life (EOL)} \\
\bottomrule
\end{tabular}
\end{table}

\paragraph{Finding Detail:} The MySQL database service on port 3306 is exposed. The detected version, \textbf{MySQL 5.7.33}, reached its official End of Life (EOL) in October 2023. EOL software no longer receives security updates from the vendor, leaving it perpetually vulnerable to newly discovered exploits.

% --- Section 5: Correlated Risk Assessment ---
\section{Correlated Risk Assessment}
This section synthesizes findings from the security control review, technical scans, and pre-existing risk data to provide a holistic view of the organization's risk profile.

\begin{table}[h!]
\centering
\caption{Summary of Identified Risks}
\begin{tabular}{p{0.1\linewidth} p{0.25\linewidth} p{0.1\linewidth} p{0.45\linewidth}}
\toprule
\textbf{Risk ID} & \textbf{Risk Title} & \textbf{Severity} & \textbf{Description} \\
\midrule
\rowcolor{criticalred!25}
RISK-001 & Exposed End-of-Life Database Service & \textbf{Critical} & The network scan confirmed that a MySQL 5.7.33 database is exposed on port 3306. This version is no longer supported with security patches, making it a prime target for attackers. This directly validates the pre-existing risk "Database Exposure". \\
\addlinespace[3pt]
\rowcolor{criticalred!25}
RISK-002 & Lack of Endpoint Multi-Factor Authentication & \textbf{Critical} & The absence of MFA on computer logins creates a significant vulnerability. If an employee's password is stolen (e.g., via phishing), an attacker can gain direct access to the corporate network and internal resources, including the exposed database. \\
\addlinespace[3pt]
\rowcolor{highorange!25}
RISK-003 & Inadequate New-Hire Security Training & \textbf{High} & New employees are not receiving security awareness training upon being hired. This makes them highly susceptible to phishing and social engineering attacks, as they are unfamiliar with corporate security policies and common threats. This gap directly increases the likelihood of a credential compromise event (RISK-002). \\
\bottomrule
\end{tabular}
\end{table}

% --- Section 6: Recommendations ---
\section{Recommendations}
The following actions are recommended to mitigate the identified risks. Recommendations are prioritized based on severity and potential impact.

\subsection{RISK-001: Exposed End-of-Life Database Service (Critical)}
\begin{itemize}
    \item \textbf{Immediate Action (Containment):} Implement strict firewall rules to deny all access to TCP port 3306 from the internet. Access should only be permitted from specific, trusted internal IP addresses (e.g., application servers).
    \item \textbf{Short-Term (Remediation):} Develop and execute a plan to upgrade the MySQL 5.7 instance to a currently supported version (e.g., MySQL 8.0 or later). This is crucial to ensure the service receives ongoing security patches.
    \item \textbf{Long-Term (Strategic):} Review the application architecture to determine if the database requires direct network access. If administrative access is needed, it should be facilitated through a secure bastion host or a Virtual Private Network (VPN) rather than direct exposure.
\end{itemize}

\subsection{RISK-002: Lack of Endpoint MFA (Critical)}
\begin{itemize}
    \item \textbf{Immediate Action (Planning):} Initiate a project to deploy MFA for all employee computer logins. Evaluate solutions compatible with your existing identity provider and operating systems.
    \item \textbf{Short-Term (Implementation):} Begin a phased rollout of endpoint MFA, prioritizing users with access to sensitive data and systems administrators.
    \item \textbf{Long-Term (Maturity):} Enforce MFA for all user accounts, including standard users, service accounts, and contractors, to establish a strong baseline for endpoint security.
\end{itemize}

\subsection{RISK-003: Inadequate New-Hire Training (High)}
\begin{itemize}
    \item \textbf{Immediate Action (Integration):} Integrate mandatory security awareness training into the standard onboarding process for all new employees and contractors, to be completed within their first week.
    \item \textbf{Short-Term (Content):} Ensure the training curriculum covers critical topics such as phishing identification, strong password creation, acceptable use of company assets, and how to report a security incident.
\end{itemize}

% --- Document End ---
\end{document}
```