```latex
\documentclass[12pt]{article}

% Preamble: Required Packages
\usepackage[margin=1in]{geometry}
\usepackage{pifont} % For checkmarks and crosses
\usepackage{booktabs} % For professional tables
\usepackage{hyperref} % For hyperlinks
\usepackage{url} % For URL formatting
\usepackage{seqsplit} % To split long strings in tt font
\usepackage{graphicx}
\usepackage{xcolor}

% Document Information
\title{Cybersecurity Posture Assessment Report}
\author{Cybersecurity Analyst}
\date{\today}

% Hyperref Setup
\hypersetup{
    colorlinks=true,
    linkcolor=blue,
    filecolor=magenta,      
    urlcolor=cyan,
    pdftitle={Cybersecurity Posture Assessment Report},
    pdfpagemode=FullScreen,
}

\begin{document}

\maketitle
\thispagestyle{empty}
\newpage

\tableofcontents
\newpage

% --- 1. Executive Summary ---
\section{Executive Summary}

This report details the findings of a cybersecurity posture assessment for \textbf{True North Travel}. The analysis is based on a combination of an external network scan, a review of existing risks, and an organizational security controls questionnaire.

The assessment reveals a \textbf{critical risk posture}. Key findings indicate a complete absence of fundamental security controls, including Multi-Factor Authentication (MFA) across all systems and any form of employee security awareness training. These deficiencies create a high probability of security incidents, such as account compromise through phishing or credential stuffing attacks.

Furthermore, the technical scan identified an exposed Secure Shell (SSH) service on the network perimeter. When combined with the lack of MFA and user training, this exposed service presents an immediate and significant threat to the organization's infrastructure.

Urgent and decisive action is required to remediate these critical vulnerabilities. Recommendations are provided in this report to establish a baseline of security and mitigate the most severe risks.

% --- 2. Organizational Information ---
\section{Organizational Information}

The following information was provided for the assessment.

\begin{itemize}
    \item \textbf{Organization Name:} True North Travel
    \item \textbf{Email Domain:} \texttt{TrueNorthTravel.com}
    \item \textbf{Primary External IP:} \texttt{21.198.50.229}
    \item \textbf{Scanned Target IP:} \texttt{2001:db8::1}
\end{itemize}

% --- 3. Security Control Review ---
\section{Security Control Review}

A security questionnaire was completed to evaluate the implementation of essential administrative and technical controls. The results, detailed in Table \ref{tab:controls}, highlight critical gaps in the organization's security framework. Each "No" response represents a significant deviation from industry best practices.

\begin{table}[h!]
\centering
\caption{Security Controls Questionnaire Results}
\label{tab:controls}
\begin{tabular}{@{}lc@{}}
\toprule
\textbf{Control Question} & \textbf{Response} \\ \midrule
Do you require MFA to access email? & \ding{55} \\
Do you require MFA to log into computers? & \ding{55} \\
Do you require MFA to access sensitive data systems? & \ding{55} \\
Does your organization have an employee acceptable use policy? & \ding{55} \\
Does your organization do security awareness training for new employees? & \ding{55} \\
Does your organization do security awareness training for all employees annually? & \ding{55} \\ \bottomrule
\end{tabular}
\end{table}

\textbf{Analysis:} The universal "No" responses indicate a foundational weakness in the security program. The lack of MFA is the most severe issue, as it removes a critical layer of defense against account takeovers. The absence of security training and policies leaves the organization highly vulnerable to human-error-based attacks like phishing and social engineering.

% --- 4. Technical Scan Results ---
\section{Technical Scan Results}

An external network scan was performed against the target IP address \texttt{2001:db8::1}. The scan identified the following open port.

\begin{table}[h!]
\centering
\caption{Open Port Findings}
\label{tab:scan}
\begin{tabular}{@{}llll@{}}
\toprule
\textbf{Port} & \textbf{State} & \textbf{Service (Inferred)} & \textbf{Product / Version} \\ \midrule
22/tcp & open & SSH (Secure Shell) & N/A \\ \bottomrule
\end{tabular}
\end{table}

\textbf{Analysis:} The scan confirms that port 22, the standard port for SSH, is open to the public internet. SSH is a critical administrative protocol, and its direct exposure increases the network's attack surface. This service is a common target for automated brute-force attacks, where threat actors attempt to guess credentials. Given the lack of MFA, a successful password guess could lead to a complete system compromise.

% --- 5. Consolidated Risk Assessment ---
\section{Consolidated Risk Assessment}

The following table synthesizes findings from the security control review, technical scan, and pre-existing risk data. Since no pre-existing vulnerabilities were reported, all risks listed below are new findings from this assessment.

\begin{table}[h!]
\centering
\caption{Identified Risks}
\label{tab:risks}
\resizebox{\textwidth}{!}{%
\begin{tabular}{@{}llll@{}}
\toprule
\textbf{ID} & \textbf{Risk Name} & \textbf{Severity} & \textbf{Description} \\ \midrule
\textbf{RISK-001} & No Multi-Factor Authentication & \textbf{Critical} & The complete absence of MFA on email, computers, and sensitive \\
& & & systems makes accounts highly susceptible to takeover. \\
\addlinespace
\textbf{RISK-002} & Inadequate Security Awareness & \textbf{Critical} & No security training for new or existing employees. This creates \\
& & & a high likelihood of compromise via phishing or social engineering. \\
\addlinespace
\textbf{RISK-003} & Exposed SSH Management Port & \textbf{High} & Port 22 (SSH) is open to the internet, inviting brute-force and \\
& & & credential stuffing attacks against a critical management service. \\
\addlinespace
\textbf{RISK-004} & Lack of Acceptable Use Policy & \textbf{Medium} & No formal policy defining acceptable use of IT assets, creating \\
& & & ambiguity and risk of misuse by employees. \\ \bottomrule
\end{tabular}%
}
\end{table}

% --- 6. Recommendations ---
\section{Recommendations}

The following actions are recommended to address the identified risks. They are prioritized based on severity and potential impact.

\begin{description}
    \item[\textbf{Priority 1 (RISK-001): Implement MFA Immediately}] \\
    Deploy Multi-Factor Authentication across all systems, with the highest priority on:
    \begin{itemize}
        \item Email accounts (e.g., Office 365, Google Workspace).
        \item All remote access systems, including VPNs and the exposed SSH service.
        \item Access to systems containing sensitive or critical data.
    \end{itemize}

    \item[\textbf{Priority 2 (RISK-002): Establish a Security Awareness Program}] \\
    Immediately develop and deploy a security awareness training program that includes:
    \begin{itemize}
        \item Mandatory training for all new hires during onboarding.
        \item Annual mandatory training for all employees covering topics like phishing, password security, and data handling.
        \item Consider periodic phishing simulations to test and reinforce training.
    \end{itemize}

    \item[\textbf{Priority 3 (RISK-003): Secure the SSH Service}] \\
    Reduce the attack surface of the exposed SSH service by implementing the following controls:
    \begin{itemize}
        \item If possible, place the service behind a VPN and disable public access.
        \item If public access is required, use a firewall to restrict access to only known, trusted IP addresses.
        \item Disable password-based authentication and enforce the use of strong SSH keys.
    \end{itemize}
    
    \item[\textbf{Priority 4 (RISK-004): Develop an Acceptable Use Policy (AUP)}] \\
    Create a formal AUP that clearly outlines the rules and responsibilities for all employees when using company IT assets. Ensure the policy is distributed to all staff and that they provide written acknowledgment of its receipt and understanding.
\end{description}

\end{document}
```