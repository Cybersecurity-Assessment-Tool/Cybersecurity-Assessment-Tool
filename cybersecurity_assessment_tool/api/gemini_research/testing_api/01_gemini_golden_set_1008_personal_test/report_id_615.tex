```latex
\documentclass[12pt]{article}

% Required Packages
\usepackage[margin=1in]{geometry}
\usepackage{pifont} % For checkmarks and crosses
\usepackage{booktabs} % For professional tables
\usepackage{hyperref} % For clickable links
\usepackage{url} % For URL formatting
\usepackage{seqsplit} % For splitting long strings in texttt

% Document Metadata
\title{Cybersecurity Posture Assessment Report}
\author{Cybersecurity Analyst}
\date{\today}

\begin{document}

\maketitle
\thispagestyle{empty}
\newpage
\tableofcontents
\newpage

% --- 1. Executive Summary ---
\section{Executive Summary}
This report provides a comprehensive cybersecurity assessment for \textbf{Solaris Energy}, conducted on \today. The analysis synthesizes data from an external network scan, a security controls questionnaire, and a review of pre-existing risks.

The assessment reveals a \textbf{Critical} overall risk posture. The primary findings include a publicly exposed MySQL database (\texttt{172.16.50.20:3306}) running an End-of-Life (EOL) version (5.7.33), which no longer receives security updates. This technical vulnerability is critically compounded by organizational policy gaps, most notably the lack of Multi-Factor Authentication (MFA) for accessing sensitive data systems.

Furthermore, the absence of a formal Acceptable Use Policy and a mandatory annual security awareness training program for all employees indicates significant weaknesses in administrative controls. These gaps heighten the organization's susceptibility to both targeted attacks and opportunistic threats. Immediate remediation is required to address the exposed database and enforce stronger access controls across the organization.

% --- 2. Organizational Information ---
\section{Organizational Information}
The following details were provided for the assessment scope.

\begin{itemize}
    \item \textbf{Organization Name:} Solaris Energy
    \item \textbf{Email Domain:} \texttt{SolarisEnergy.net}
    \item \textbf{Website Domain:} \texttt{www.SolarisEnergy.net}
    \item \textbf{External IP Address:} \texttt{96.147.74.17}
\end{itemize}

% --- 3. Security Control Review ---
\section{Security Control Review}
A review of the organization's security controls was conducted via a questionnaire. The responses highlight critical gaps in administrative and access control policies. A summary of the findings is presented in Table \ref{tab:controls}.

\begin{table}[h!]
\centering
\caption{Security Controls Questionnaire Results}
\label{tab:controls}
\begin{tabular}{p{0.7\linewidth} c c}
\toprule
\textbf{Control Question} & \textbf{Response} & \textbf{Status} \\
\midrule
Do you require MFA to access email? & Yes & \ding{51} \\
Do you require MFA to log into computers? & Yes & \ding{51} \\
\textbf{Do you require MFA to access sensitive data systems?} & \textbf{No} & \textbf{\ding{55}} \\
\textbf{Does your organization have an employee acceptable use policy?} & \textbf{No} & \textbf{\ding{55}} \\
Does your organization do security awareness training for new employees? & Yes & \ding{51} \\
\textbf{Does your organization do security awareness training for all employees at least once per year?} & \textbf{No} & \textbf{\ding{55}} \\
\bottomrule
\end{tabular}
\end{table}

The items marked with \ding{55} represent significant control deficiencies that increase organizational risk. The lack of MFA for sensitive systems is the most critical finding in this section.

% --- 4. Technical Scan Results ---
\section{Technical Scan Results}
An external network scan was performed on the target IP address \texttt{172.16.50.20}. The scan identified one open port, which presents a significant security risk.

\begin{table}[h!]
\centering
\caption{Open Port Analysis}
\label{tab:nmap}
\begin{tabular}{l l l l}
\toprule
\textbf{Port} & \textbf{Service} & \textbf{Product} & \textbf{Version} \\
\midrule
3306/tcp & mysql & MySQL & 5.7.33 \\
\bottomrule
\end{tabular}
\end{table}

\subsection*{Analysis}
The scan confirms that a MySQL database is directly exposed to the network on port \texttt{3306}. This configuration is highly discouraged as it exposes the database to brute-force attacks, credential stuffing, and exploitation of known vulnerabilities.

\textbf{Critical Finding:} The identified MySQL version, \textbf{5.7.33}, reached its official End-of-Life (EOL) in October 2023. EOL software no longer receives security patches from the vendor, meaning any newly discovered vulnerabilities will remain unpatched. This elevates the risk of compromise from High to Critical.

% --- 5. Risk Assessment ---
\section{Risk Assessment}
The following table summarizes and correlates the identified risks from the security control review, technical scan, and pre-existing risk data. Risks are prioritized based on their potential impact and likelihood of exploitation.

\begin{table}[h!]
\centering
\caption{Consolidated Risk Summary}
\label{tab:risks}
\begin{tabular}{p{0.25\linewidth} p{0.55\linewidth} l}
\toprule
\textbf{Risk Name} & \textbf{Overview} & \textbf{Severity} \\
\midrule
\textbf{Exposed End-of-Life Database} & The MySQL database (v5.7.33) is publicly accessible and no longer receives security updates, making it a prime target for automated attacks and exploitation. & \textbf{Critical} \\
\textbf{Lack of MFA on Sensitive Systems} & The absence of MFA on sensitive systems, such as the exposed database, drastically lowers the barrier for unauthorized access if credentials are compromised. & \textbf{Critical} \\
\textbf{Inadequate Security Training Program} & Security training is not conducted annually for all employees, leading to a decline in security awareness and making staff more susceptible to social engineering attacks. & \textbf{High} \\
\textbf{Missing Acceptable Use Policy (AUP)} & Without a formal AUP, there are no clear, enforceable rules for employees regarding the use of company systems and data, increasing the risk of insider threat and misuse. & \textbf{High} \\
\bottomrule
\end{tabular}
\end{table}

% --- 6. Recommendations ---
\section{Recommendations}
The following actions are recommended to mitigate the identified risks. Recommendations are prioritized to address the most critical vulnerabilities first.

\subsection*{Immediate Actions (0-7 Days)}
\begin{enumerate}
    \item \textbf{Restrict Database Access:} Immediately implement strict firewall rules to block all public access to port \texttt{3306} on \texttt{172.16.50.20}. Access should only be permitted from trusted internal IP addresses or via a secure VPN.
    \item \textbf{Enforce MFA on Sensitive Systems:} Prioritize the deployment of MFA on all sensitive systems, with the exposed database system being the top priority. This is a critical compensating control while the database is being upgraded.
\end{enumerate}

\subsection*{Short-Term Actions (1-3 Months)}
\begin{enumerate}
    \item \textbf{Upgrade End-of-Life Software:} Plan and execute the migration of the MySQL 5.7.33 database to a currently supported version (e.g., MySQL 8.x). This is essential to ensure the system receives ongoing security patches.
    \item \textbf{Develop and Implement an Acceptable Use Policy (AUP):} Create a formal AUP that clearly defines the rules for using company technology and data. Ensure all employees read and acknowledge the policy.
\end{enumerate}

\subsection*{Long-Term Actions (3-6 Months)}
\begin{enumerate}
    \item \textbf{Establish Annual Security Awareness Training:} Implement a mandatory, annual security awareness training program for all employees. The training should cover topics such as phishing, password security, and data handling.
    \item \textbf{Implement a Vulnerability Management Program:} Establish a formal process for regularly scanning internal and external systems for vulnerabilities, tracking them, and ensuring timely remediation.
\end{enumerate}

\end{document}
```