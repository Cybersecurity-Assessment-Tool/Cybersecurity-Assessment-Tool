```latex
\documentclass[12pt]{article}

% --- PACKAGES ---
\usepackage[margin=1in]{geometry}
\usepackage{pifont} % For checkmarks and crosses
\usepackage{booktabs} % For professional tables
\usepackage{hyperref} % For clickable links
\usepackage{url} % For URL formatting
\usepackage{seqsplit} % For splitting long strings
\usepackage{xcolor} % For colors

% --- DOCUMENT METADATA ---
\title{Cybersecurity Assessment Report}
\author{Cybersecurity Analysis Division}
\date{\today}

% --- HYPERREF SETUP ---
\hypersetup{
    colorlinks=true,
    linkcolor=blue,
    filecolor=magenta,      
    urlcolor=cyan,
    pdftitle={Cybersecurity Assessment Report},
    pdfpagemode=FullScreen,
}

\begin{document}

\maketitle
\tableofcontents
\newpage

% ===================================================================
% SECTION 1: EXECUTIVE SUMMARY
% ===================================================================
\section{Executive Summary}

This report provides a comprehensive cybersecurity assessment for \textbf{Top Tier}, synthesizing data from technical network scans, a security controls questionnaire, and a review of pre-existing risks. The analysis reveals a high-risk security posture that requires immediate attention and remediation.

Key findings indicate critical vulnerabilities in both technical infrastructure and administrative controls. A network scan identified a host (\texttt{10.10.10.51}) with an exposed Remote Desktop Protocol (RDP) service. This finding is particularly alarming as it points to a systemic issue, echoing a previously documented risk on a different host. This pattern of RDP exposure presents a significant attack vector for ransomware and unauthorized access.

Furthermore, the security controls review identified two major gaps:
\begin{itemize}
    \item \textbf{Critical Risk:} The lack of Multi-Factor Authentication (MFA) on the primary email system (\texttt{TopTier.net}) leaves the organization highly susceptible to account takeovers through phishing and credential stuffing attacks.
    \item \textbf{High Risk:} The absence of mandatory annual security awareness training for all employees significantly increases the organization's vulnerability to social engineering tactics.
\end{itemize}

Immediate remediation of the exposed RDP services and the implementation of MFA for email are critical priorities. A long-term strategy must be developed to improve vulnerability management processes and establish a robust security awareness program.

% ===================================================================
% SECTION 2: ORGANIZATIONAL INFORMATION
% ===================================================================
\section{Organizational Information}

The following information was provided for the assessment.

\begin{tabular}{@{}ll}
    \toprule
    \textbf{Attribute} & \textbf{Value} \\
    \midrule
    Organization Name & \textbf{Top Tier} \\
    Email Domain & \texttt{TopTier.net} \\
    Website Domain & \url{www.TopTier.net} \\
    External IP Address & \texttt{56.83.209.179} \\
    \bottomrule
\end{tabular}

% ===================================================================
% SECTION 3: SECURITY CONTROL REVIEW
% ===================================================================
\section{Security Control Review}

A review of administrative and technical security controls was conducted via a questionnaire. The results highlight significant gaps in the organization's defense-in-depth strategy. "No" answers indicate a failure to meet baseline security best practices.

\begin{table}[h!]
\centering
\begin{tabular}{@{}p{0.7\linewidth}cc@{}}
    \toprule
    \textbf{Control Question} & \textbf{Response} & \textbf{Status} \\
    \midrule
    Do you require MFA to access email? & No & \textcolor{red}{\ding{55}} \\
    Do you require MFA to log into computers? & Yes & \textcolor{green}{\ding{51}} \\
    Do you require MFA to access sensitive data systems? & Yes & \textcolor{green}{\ding{51}} \\
    Does your organization have an employee acceptable use policy? & Yes & \textcolor{green}{\ding{51}} \\
    Does your organization do security awareness training for new employees? & Yes & \textcolor{green}{\ding{51}} \\
    Does your organization do security awareness training for all employees at least once per year? & No & \textcolor{red}{\ding{55}} \\
    \bottomrule
\end{tabular}
\caption{Security Controls Questionnaire Results}
\end{table}

% ===================================================================
% SECTION 4: TECHNICAL SCAN RESULTS
% ===================================================================
\section{Technical Scan Results}

A network scan was performed on the specified target to identify open ports and exposed services.

\begin{itemize}
    \item \textbf{Target IP:} \texttt{10.10.10.51}
    \item \textbf{Scan Date:} As per scan metadata
\end{itemize}

The scan revealed the following open port:

\begin{table}[h!]
\centering
\begin{tabular}{@{}lll@{}}
    \toprule
    \textbf{Port} & \textbf{State} & \textbf{Service Name} \\
    \midrule
    3389/tcp & open & ms-wbt-server (Remote Desktop Protocol) \\
    \bottomrule
\end{tabular}
\caption{Open Ports on Target \texttt{10.10.10.51}}
\end{table}

\subsection{Analysis of Findings}
The presence of an open RDP port (3389) is a critical security risk. RDP is a primary target for attackers who use brute-force, password spraying, or credential stuffing techniques to gain unauthorized remote access to internal systems. Once compromised, an RDP entry point is commonly used to deploy ransomware or exfiltrate sensitive data. This finding, combined with the pre-existing risk of RDP exposure on another host, indicates a systemic failure in network security configuration and vulnerability management.

% ===================================================================
% SECTION 5: CONSOLIDATED RISK ASSESSMENT
% ===================================================================
\section{Consolidated Risk Assessment}

The following table synthesizes findings from the technical scan, control review, and pre-existing risk data into a consolidated list of security risks facing the organization.

\begin{table}[h!]
\centering
\begin{tabular}{@{}p{0.25\linewidth}p{0.15\linewidth}p{0.5\linewidth}@{}}
    \toprule
    \textbf{Risk Name} & \textbf{Severity} & \textbf{Description \& Affected Assets} \\
    \midrule
    \textbf{Systemic RDP Exposure} & \textbf{Critical} & A new host (\texttt{10.10.10.51}) was found with RDP exposed. This adds to a previously identified risk on \texttt{10.10.10.50}, indicating a recurring and unmitigated vulnerability pattern. \\
    \addlinespace
    \textbf{Lack of MFA on Email} & \textbf{Critical} & The \texttt{TopTier.net} email domain lacks MFA, exposing all user accounts to a high risk of compromise via phishing or credential theft. Email is a gateway to further system compromise. \\
    \addlinespace
    \textbf{Insufficient Security Training} & \textbf{High} & Annual security awareness training is not conducted for all employees. This degrades security posture over time and increases the likelihood of human error leading to a breach. \\
    \bottomrule
\end{tabular}
\caption{Summary of Identified Security Risks}
\end{table}

% ===================================================================
% SECTION 6: RECOMMENDATIONS
% ===================================================================
\section{Recommendations}

The following actions are recommended to mitigate the identified risks. They are prioritized based on severity and potential impact.

\begin{enumerate}
    \item \textbf{Remediate All RDP Exposures (Priority: Critical)}
    \begin{itemize}
        \item \textbf{Immediate Action:} Implement firewall rules to block all inbound traffic to TCP port 3389 from the internet on all affected systems, including \texttt{10.10.10.51} and \texttt{10.10.10.50}.
        \item \textbf{Long-Term Solution:} If remote access is required, deploy a secure solution such as a Virtual Private Network (VPN) or a Remote Desktop Gateway that enforces strong authentication, including MFA.
    \end{itemize}
    \vspace{1em}
    \item \textbf{Implement MFA for Email (Priority: Critical)}
    \begin{itemize}
        \item \textbf{Immediate Action:} Begin a project to enable and enforce MFA for all user accounts on the \texttt{TopTier.net} email platform. This is the single most effective control to prevent email account takeovers.
    \end{itemize}
    \vspace{1em}
    \item \textbf{Establish Annual Security Awareness Training (Priority: High)}
    \begin{itemize}
        \item \textbf{Action:} Procure or develop a security awareness training program that covers key topics like phishing, password hygiene, and acceptable use.
        \item \textbf{Policy:} Mandate that all employees complete this training annually and track completion to ensure compliance.
    \end{itemize}
\end{enumerate}

\end{document}
```