```latex
\documentclass[12pt]{article}

% Preamble: Required Packages
\usepackage[margin=1in]{geometry}
\usepackage{pifont} % For checkmarks and crosses
\usepackage{booktabs} % For professional tables
\usepackage{hyperref} % For clickable links
\usepackage{url} % For formatting URLs
\usepackage{seqsplit} % For splitting long text strings to prevent overflow
\usepackage{graphicx}
\usepackage{xcolor}

% Hyperref Setup
\hypersetup{
    colorlinks=true,
    linkcolor=blue,
    filecolor=magenta,      
    urlcolor=cyan,
    pdftitle={Cybersecurity Assessment Report},
    pdfpagemode=FullScreen,
}

% Define checkmark and cross symbols for convenience
\newcommand{\cmark}{\ding{51}}%
\newcommand{\xmark}{\ding{55}}%

\begin{document}

% --- Title Page ---
\begin{titlepage}
    \centering
    \vspace*{1cm}
    \Huge
    \textbf{Cybersecurity Assessment Report}
    
    \vspace{1.5cm}
    \Large
    Prepared for: \textbf{Top Tier}
    
    \vspace{2cm}
    
    \includegraphics[width=0.4\textwidth]{example-image-a} % Placeholder for a logo
    
    \vfill
    
    \Large
    \textbf{Report Date:} \today \\
    \textbf{Analysis Period:} October 2023 \\
    \textbf{Author:} Cybersecurity Analyst
    
\end{titlepage}

\tableofcontents
\newpage

% --- Section 1: Executive Summary ---
\section{Executive Summary}

This report provides a comprehensive analysis of the security posture of \textbf{Top Tier}, based on network scan data, a security controls questionnaire, and a review of pre-existing risks. The assessment identified several critical-risk findings that require immediate attention to mitigate the threat of unauthorized access and potential compromise.

The two most significant findings are:
\begin{enumerate}
    \item \textbf{Lack of Multi-Factor Authentication (MFA):} The organization does not enforce MFA for accessing email or for logging into employee computers. This represents a critical control gap, significantly increasing the risk of account compromise via credential theft or phishing attacks.
    \item \textbf{Insecure Remote Desktop Protocol (RDP) Exposure:} A network scan identified an open RDP port (3389) on a new host (\texttt{10.10.10.51}). This finding, correlated with a pre-existing risk on another host, indicates a systemic issue of exposing a high-risk service.
\end{enumerate}

The combination of these two vulnerabilities presents an exceptionally high risk. An attacker who obtains a user's credentials could gain direct, unhindered remote access to the internal network. Immediate remediation of the RDP exposure and a phased rollout of MFA are strongly recommended.

% --- Section 2: Organizational Information ---
\section{Organizational Information}

The following details were provided for the assessment. This information is used to establish the context and scope of the review.

\begin{tabular}{@{}ll}
\toprule
\textbf{Attribute} & \textbf{Value} \\
\midrule
Organization Name & \textbf{Top Tier} \\
Email Domain & \seqsplit{\texttt{TopTier.net}} \\
Website Domain & \seqsplit{\url{www.TopTier.net}} \\
External IP Address & \seqsplit{\texttt{81.133.26.230}} \\
\bottomrule
\end{tabular}

% --- Section 3: Security Control Review ---
\section{Security Control Review}

A review of the organization's security controls was conducted via a questionnaire. The responses indicate a solid foundation in policy and training, but reveal critical gaps in technical access controls.

\begin{table}[h!]
\centering
\caption{Security Controls Questionnaire Results}
\begin{tabular}{@{}lc}
\toprule
\textbf{Control Question} & \textbf{Response} \\
\midrule
Does your organization have an employee acceptable use policy? & \textcolor{green}{\cmark} \\
Does your organization do security awareness training for new employees? & \textcolor{green}{\cmark} \\
Does your organization do security awareness training for all employees annually? & \textcolor{green}{\cmark} \\
Do you require MFA to access sensitive data systems? & \textcolor{green}{\cmark} \\
\midrule
\textbf{Do you require MFA to access email?} & \textcolor{red}{\xmark} \\
\textbf{Do you require MFA to log into computers?} & \textcolor{red}{\xmark} \\
\bottomrule
\end{tabular}
\end{table}

\subsection*{Analysis of Control Gaps}
The lack of MFA on email and computer logins are the most significant weaknesses identified in this review.
\begin{itemize}
    \item \textbf{Email Access:} Without MFA, email accounts are vulnerable to takeover from phishing attacks or credential stuffing. A compromised email account is a primary vector for internal phishing, data exfiltration, and business email compromise (BEC) fraud.
    \item \textbf{Computer Login:} The absence of MFA for computer access means that a single compromised password can grant an attacker full access to an employee's workstation and any network resources they are authorized to use. This risk is severely amplified when combined with exposed remote access services like RDP.
\end{itemize}

% --- Section 4: Technical Scan Results ---
\section{Technical Scan Results}

An Nmap scan was performed to identify open ports and services on the target system. The scan revealed a service commonly targeted by attackers.

\begin{itemize}
    \item \textbf{Target IP Address:} \texttt{10.10.10.51}
\end{itemize}

\begin{table}[h!]
\centering
\caption{Open Ports Detected on \texttt{10.10.10.51}}
\begin{tabular}{@{}lll}
\toprule
\textbf{Port} & \textbf{State} & \textbf{Service Name} \\
\midrule
3389/tcp & open & \texttt{ms-wbt-server} (Microsoft Remote Desktop Protocol) \\
\bottomrule
\end{tabular}
\end{table}

\subsection*{Analysis of Technical Findings}
The scan confirms that port 3389, used for Microsoft Remote Desktop Protocol (RDP), is open on the host \texttt{10.10.10.51}. RDP is a primary target for ransomware gangs and other malicious actors who scan the internet for exposed servers. When not secured behind a VPN or RDP Gateway and protected with strong passwords and MFA, it provides a direct entry point into an organization's internal network.

This finding is especially concerning as it expands upon a pre-existing risk identified on another host (\texttt{10.10.10.50}), suggesting a recurring configuration weakness.

% --- Section 5: Consolidated Risk Assessment ---
\section{Consolidated Risk Assessment}

The following table synthesizes findings from the security questionnaire, the technical scan, and pre-existing risk data. Risks are prioritized based on their potential impact and likelihood of exploitation.

\begin{table}[h!]
\centering
\caption{Summary of Identified Risks}
\begin{tabular}{@{}p{0.1\linewidth} p{0.25\linewidth} p{0.45\linewidth} p{0.1\linewidth}@{}}
\toprule
\textbf{Risk ID} & \textbf{Risk Name} & \textbf{Description} & \textbf{Severity} \\
\midrule
\textbf{RISK-001} & \textbf{Systemic RDP Exposure} & The RDP service (port 3389) is exposed on hosts \texttt{10.10.10.50} and \texttt{10.10.10.51}. This provides a direct vector for brute-force attacks and exploitation. The risk is compounded by the lack of MFA for computer logins. & \textbf{Critical} \\
\addlinespace
\textbf{RISK-002} & \textbf{Lack of Multi-Factor Authentication} & MFA is not enforced for email or computer logins. This exposes the organization to account takeover, data breaches, and phishing-based attacks. A single compromised password can lead to a significant breach. & \textbf{Critical} \\
\bottomrule
\end{tabular}
\end{table}

% --- Section 6: Recommendations ---
\section{Recommendations}

The following actions are recommended to address the identified risks. They are prioritized to ensure the most critical vulnerabilities are remediated first.

\subsection*{Immediate Priority (Within 72 Hours)}
\begin{enumerate}
    \item \textbf{Remediate RDP Exposure on All Hosts:}
    \begin{itemize}
        \item Immediately close port 3389 on the firewalls for hosts \texttt{10.10.10.50} and \texttt{10.10.10.51} and any other systems where it is exposed.
        \item If remote access is required, implement a secure solution such as a Virtual Private Network (VPN) or a dedicated RDP Gateway, which requires users to authenticate to the gateway before accessing internal resources.
    \end{itemize}
\end{enumerate}

\subsection*{High Priority (Within 30 Days)}
\begin{enumerate}
    \setcounter{enumi}{1} % Continue numbering
    \item \textbf{Deploy Multi-Factor Authentication (MFA):}
    \begin{itemize}
        \item Enable and enforce MFA for all users on the primary email system (e.g., Office 365, Google Workspace).
        \item Implement an MFA solution for all computer logins (both local and remote). Solutions like Duo, Okta, or Windows Hello for Business can be used.
    \end{itemize}
\end{enumerate}

\subsection*{Medium Priority (Within 90 Days)}
\begin{enumerate}
    \setcounter{enumi}{2} % Continue numbering
    \item \textbf{Conduct a Network Security Architecture Review:}
    \begin{itemize}
        \item Perform a comprehensive review of firewall rules and network segmentation to ensure that no other high-risk services are unnecessarily exposed.
        \item Validate that critical systems are properly isolated from user workstations.
    \end{itemize}
    \item \textbf{Establish a Formal Vulnerability Management Program:}
    \begin{itemize}
        \item Implement a recurring schedule of internal and external network scans to proactively identify and remediate vulnerabilities before they can be exploited.
    \end{itemize}
\end{enumerate}

\end{document}
```