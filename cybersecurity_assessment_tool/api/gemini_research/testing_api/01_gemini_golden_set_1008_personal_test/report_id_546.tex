```latex
\documentclass[12pt]{article}

% Preamble: Required Packages
\usepackage[a4paper, margin=1in]{geometry}
\usepackage{pifont} % For checkmarks and crosses
\usepackage{booktabs} % For professional tables
\usepackage{hyperref} % For clickable links
\usepackage{url} % For URL formatting
\usepackage{seqsplit} % To split long strings without breaking
\usepackage{graphicx}
\usepackage{xcolor}
\usepackage{fancyhdr}

% Document Information
\title{Cybersecurity Posture Assessment Report}
\author{Cybersecurity Analyst}
\date{\today}

% Hyperref Setup
\hypersetup{
    colorlinks=true,
    linkcolor=blue,
    filecolor=magenta,      
    urlcolor=cyan,
    pdftitle={Cybersecurity Posture Assessment Report},
    pdfpagemode=FullScreen,
}

% Header and Footer
\pagestyle{fancy}
\fancyhf{}
\fancyhead[L]{Astraeus Aerospace}
\fancyhead[R]{Confidential}
\fancyfoot[C]{\thepage}

\begin{document}

\maketitle
\thispagestyle{empty}
\newpage

\tableofcontents
\newpage

% --- 1. Executive Summary ---
\section{Executive Summary}
This report provides a comprehensive cybersecurity assessment for Astraeus Aerospace, conducted on \today. The analysis synthesizes data from a network vulnerability scan, a review of organizational security controls, and a list of pre-existing risks.

The assessment reveals a significant disparity between the organization's technical and administrative security postures. The targeted network host (\texttt{192.168.0.5}) was found to be securely configured, with no open ports detected. This finding contradicts a pre-existing risk concerning an unencrypted web server, suggesting that the issue may have been remediated.

However, the review of organizational security controls uncovered critical deficiencies. The widespread absence of Multi-Factor Authentication (MFA) across email, computers, and sensitive data systems presents a critical risk of account compromise. Furthermore, the lack of a formal security awareness training program for employees constitutes a high-risk vulnerability to social engineering and phishing attacks.

Immediate remediation should focus on implementing MFA and establishing a mandatory security awareness training program to mitigate the most severe risks to the organization.

% --- 2. Organizational Information ---
\section{Organizational Information}
The following details were provided for the assessment.

\begin{itemize}
    \item \textbf{Organization Name:} Astraeus Aerospace
    \item \textbf{Email Domain:} \seqsplit{\texttt{AstraeusAerospace.net}}
    \item \textbf{Website Domain:} \seqsplit{\url{www.AstraeusAerospace.net}}
    \item \textbf{External IP Address:} \seqsplit{\texttt{95.52.159.152}}
\end{itemize}

% --- 3. Security Control Review ---
\section{Security Control Review}
An assessment of administrative and policy-based security controls was conducted via a questionnaire. The responses highlight critical gaps in the organization's defense-in-depth strategy. A "No" response indicates a missing control that exposes the organization to significant risk.

\begin{table}[h!]
\centering
\caption{Organizational Security Control Assessment}
\begin{tabular}{p{8cm} c l}
\toprule
\textbf{Control Question} & \textbf{Response} & \textbf{Assessment} \\
\midrule
Do you require MFA to access email? & \ding{55} & \textcolor{red}{\textbf{Critical Gap}} \\
Do you require MFA to log into computers? & \ding{55} & \textcolor{red}{\textbf{Critical Gap}} \\
Do you require MFA to access sensitive data systems? & \ding{55} & \textcolor{red}{\textbf{Critical Gap}} \\
Does your organization have an employee acceptable use policy? & \ding{51} & Control in Place \\
Does your organization do security awareness training for new employees? & \ding{55} & \textcolor{orange}{High Risk} \\
Does your organization do security awareness training for all employees at least once per year? & \ding{55} & \textcolor{orange}{High Risk} \\
\bottomrule
\end{tabular}
\end{table}

% --- 4. Technical Scan Results ---
\section{Technical Scan Results}
A network scan was performed to identify active services and potential vulnerabilities on the specified target system.

\begin{itemize}
    \item \textbf{Scan Target:} \seqsplit{\texttt{192.168.0.5}}
    \item \textbf{Scan Date:} \today
\end{itemize}

\subsection{Port Scan Analysis}
The scan results indicate that the target host is online, but no open ports were discovered. This suggests a strong network-level security posture for this specific host, as it does not expose any services to the network segment from which it was scanned.

\begin{table}[h!]
\centering
\caption{Port Scan Results for \texttt{192.168.0.5}}
\begin{tabular}{l l l}
\toprule
\textbf{Port} & \textbf{State} & \textbf{Service} \\
\midrule
80/tcp & Closed & http \\
\bottomrule
\end{tabular}
\end{table}

\textbf{Note:} The scan result showing Port 80 as \textbf{closed} directly contradicts the pre-existing risk "Unencrypted Web Server." This indicates the risk has likely been remediated or was a false positive.

% --- 5. Risk Assessment Summary ---
\section{Risk Assessment Summary}
This section correlates findings from the security control review, the technical scan, and pre-existing risk data to provide a unified view of the current risk landscape.

\begin{table}[h!]
\centering
\caption{Consolidated Risk Register}
\begin{tabular}{p{4cm} p{6.5cm} l}
\toprule
\textbf{Risk Name} & \textbf{Overview} & \textbf{Severity} \\
\midrule
\textbf{Absence of Multi-Factor Authentication (MFA)} & The lack of MFA for email, computer, and sensitive data access makes user accounts highly vulnerable to compromise via stolen credentials. & \textcolor{red}{\textbf{Critical}} \\
\addlinespace
\textbf{Inadequate Security Awareness Training} & Without regular training, employees are ill-equipped to identify and resist phishing, social engineering, and other common cyberattacks. & \textcolor{orange}{\textbf{High}} \\
\addlinespace
\textbf{Unencrypted Web Server (Potentially Remediated)} & A pre-existing risk noted an open Port 80. Our \today\ scan found this port to be closed, suggesting remediation. This risk requires validation. & \textcolor{gray}{Informational} \\
\bottomrule
\end{tabular}
\end{table}

% --- 6. Recommendations ---
\section{Recommendations}
The following actions are recommended to mitigate the identified risks, prioritized by severity.

\begin{enumerate}
    \item \textbf{[Critical] Implement a Universal MFA Policy:}
    \begin{itemize}
        \item \textbf{Action:} Immediately begin deploying MFA across all critical systems.
        \item \textbf{Priority 1:} Secure all email accounts (e.g., Office 365, Google Workspace) with MFA.
        \item \textbf{Priority 2:} Enforce MFA for access to all systems containing sensitive or proprietary data.
        \item \textbf{Priority 3:} Roll out MFA for all employee computer logins.
    \end{itemize}
    \vspace{0.5cm}
    \item \textbf{[High] Establish a Security Awareness Training Program:}
    \begin{itemize}
        \item \textbf{Action:} Procure and implement a comprehensive security awareness training solution.
        \item \textbf{Details:} The program must be mandatory for all new hires upon onboarding and for all existing employees on at least an annual basis. Training should cover phishing, password security, and acceptable use policies.
    \end{itemize}
    \vspace{0.5cm}
    \item \textbf{[Informational] Validate and Update Risk Register:}
    \begin{itemize}
        \item \textbf{Action:} Formally review the "Unencrypted Web Server" risk.
        \item \textbf{Details:} Confirm that Port 80 is intentionally and correctly closed on the relevant server(s). If the control is verified, update the organization's risk register to formally close the finding.
    \end{itemize}
\end{enumerate}

\end{document}
```