```latex
\documentclass[12pt]{article}

% Preamble: Required Packages
\usepackage[margin=1in]{geometry}
\usepackage{pifont} % For checkmarks and crosses (\ding{51}, \ding{55})
\usepackage{booktabs} % For professional-looking tables
\usepackage{hyperref} % For clickable links
\usepackage{url} % For formatting URLs
\usepackage{seqsplit} % To split long strings without breaking words

% Document Metadata
\title{Cybersecurity Posture Assessment Report}
\author{Cybersecurity Analysis Division}
\date{\today}

\begin{document}

\maketitle
\thispagestyle{empty}
\newpage
\tableofcontents
\newpage

% --- 1. Executive Overview ---
\section{Executive Overview}

This report provides a comprehensive cybersecurity assessment for \textbf{Golden Gate Gaming}. The analysis is based on a review of organizational security controls, an external network scan, and pre-existing risk data.

The assessment reveals a mixed security posture. While the organization has implemented some foundational controls, such as an acceptable use policy and security training for new hires, there are critical deficiencies in access control. The absence of Multi-Factor Authentication (MFA) for email and computer access represents a significant and immediate risk of account compromise and unauthorized access. Furthermore, the lack of recurring annual security training for all staff members constitutes a high risk, as it diminishes the organization's resilience against evolving threats like phishing.

On a positive note, the external network scan of the target system did not identify any open ports, suggesting a well-configured perimeter firewall at the scanned address. Recommendations in this report focus on urgently addressing the identified MFA and training gaps to bolster the organization's defense-in-depth strategy.

% --- 2. Organizational Information ---
\section{Organizational Information}

This section details the organizational data provided for the assessment. This information is used to establish the context for the technical and procedural analysis.

\begin{tabular}{@{}ll}
\toprule
\textbf{Attribute} & \textbf{Value} \\
\midrule
Organization Name & \textbf{Golden Gate Gaming} \\
Email Domain & \texttt{GoldenGateGaming.com} \\
Website Domain & \href{http://www.GoldenGateGaming.com}{\texttt{www.GoldenGateGaming.com}} \\
External IP Address & \texttt{131.78.108.222} \\
\bottomrule
\end{tabular}

% --- 3. Security Control Review ---
\section{Security Control Review}

The following table summarizes the organization's responses to a security controls questionnaire. Items marked with a red 'X' (\ding{55}) indicate significant gaps in the current security framework and are discussed in the Risk Assessment section.

\begin{table}[h!]
\centering
\begin{tabular}{@{}lc}
\toprule
\textbf{Security Control Question} & \textbf{Status} \\
\midrule
Do you require MFA to access email? & \ding{55} \\
Do you require MFA to log into computers? & \ding{55} \\
Do you require MFA to access sensitive data systems? & \ding{51} \\
Does your organization have an employee acceptable use policy? & \ding{51} \\
Does your organization do security awareness training for new employees? & \ding{51} \\
Does your organization do security awareness training for all employees at least once per year? & \ding{55} \\
\bottomrule
\end{tabular}
\caption{Organizational Security Controls Questionnaire Results. (\ding{51} = Yes, \ding{55} = No)}
\label{tab:controls}
\end{table}

% --- 4. Technical Scan Results ---
\section{Technical Scan Results}

An external network vulnerability scan was conducted to identify potential exposures on the organization's perimeter.

\begin{itemize}
    \item \textbf{Target IP Address:} \texttt{[Target IP]}
    \item \textbf{Scan Date:} Not specified in scan data.
\end{itemize}

\subsection{Summary of Findings}
The network scan completed successfully. \textbf{No open TCP ports were detected on the target system.} This is a positive security finding, indicating that the external firewall is effectively blocking unsolicited inbound connections to this specific IP address, thereby reducing the external attack surface.

% --- 5. Risk Assessment ---
\section{Risk Assessment}

This section correlates findings from the security control review and technical scans to identify and prioritize risks. Although the external scan was clear, significant risks were identified through the questionnaire.

\begin{table}[h!]
\centering
\begin{tabular}{@{}p{0.25\linewidth}p{0.15\linewidth}p{0.5\linewidth}@{}}
\toprule
\textbf{Risk Name} & \textbf{Severity} & \textbf{Overview} \\
\midrule
\textbf{Lack of MFA on Email Accounts} & \textbf{Critical} & Without MFA, email accounts are highly vulnerable to takeover via phishing or credential stuffing. A compromised email account can lead to Business Email Compromise (BEC), data exfiltration, and a launchpad for further internal attacks. \\
\addlinespace
\textbf{Lack of MFA on Workstations} & \textbf{Critical} & If an attacker obtains valid user credentials, they can log directly into a company computer without a second authentication factor. This provides a strong foothold within the internal network to access data and escalate privileges. \\
\addlinespace
\textbf{Inadequate Security Awareness Training} & \textbf{High} & Security threats are constantly evolving. Without mandatory, annual security training, employees' ability to recognize and appropriately respond to threats like sophisticated phishing attacks diminishes over time, making them the weakest link in the security chain. \\
\bottomrule
\end{tabular}
\caption{Identified Security Risks and Severity.}
\label{tab:risks}
\end{table}

% --- 6. Recommendations ---
\section{Recommendations}

Based on the analysis, the following actions are recommended to mitigate the identified risks and improve the overall security posture of \textbf{Golden Gate Gaming}.

\begin{enumerate}
    \item \textbf{Urgent: Deploy MFA for Email and Workstations.}
    \begin{itemize}
        \item \textbf{Action:} Immediately prioritize the implementation of a robust MFA solution (e.g., authenticator app, hardware token) for all user accounts.
        \item \textbf{Priority:} Critical. This is the single most effective control to mitigate the risk of account takeover.
        \item \textbf{Impact:} Significantly reduces the risk of unauthorized access to email and the internal network.
    \end{itemize}
    \vspace{1em}
    \item \textbf{Establish a Mandatory Annual Security Training Program.}
    \begin{itemize}
        \item \textbf{Action:} Develop or procure a security awareness training program to be completed by all employees annually. The training should cover current threats, phishing identification, and company security policies.
        \item \textbf{Priority:} High.
        \item \textbf{Impact:} Creates a more security-conscious culture and strengthens the human firewall, reducing the likelihood of successful social engineering attacks.
    \end{itemize}
    \vspace{1em}
    \item \textbf{Conduct Regular Security Reviews.}
    \begin{itemize}
        \item \textbf{Action:} Perform periodic reviews of security controls and policies to ensure they remain effective and aligned with the current threat landscape. This includes regular network scans and policy updates.
        \item \textbf{Priority:} Medium.
        \item \textbf{Impact:} Proactively identifies and addresses new security gaps before they can be exploited.
    \end{itemize}
\end{enumerate}

\end{document}
```