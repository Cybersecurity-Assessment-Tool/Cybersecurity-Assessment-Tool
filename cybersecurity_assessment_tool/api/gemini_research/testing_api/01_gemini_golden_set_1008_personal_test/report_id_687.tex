```latex
\documentclass[12pt]{article}

% Preamble: Required Packages
\usepackage[margin=1in]{geometry}
\usepackage{pifont} % For checkmarks and crosses (\ding)
\usepackage{booktabs} % For professional tables
\usepackage{hyperref} % For clickable links
\usepackage{url} % For formatting URLs
\usepackage{seqsplit} % For splitting long strings in texttt
\usepackage[T1]{fontenc}

% Document Metadata
\title{Cybersecurity Posture Assessment Report}
\author{Cybersecurity Analysis Division}
\date{\today}

\begin{document}

\maketitle
\tableofcontents
\newpage

% --- 1. Executive Summary ---
\section{Executive Summary}
This report provides a comprehensive cybersecurity assessment for \textbf{Stellar Pathways}, based on a combination of network scanning, organizational policy review, and an analysis of pre-existing risk data. The assessment was conducted on \today.

The analysis reveals a mixed security posture. While the organization has implemented some essential controls, such as Multi-Factor Authentication (MFA) for email and sensitive systems, several critical vulnerabilities and policy gaps were identified that expose the organization to significant risk.

Key findings include an externally accessible database service (\texttt{MySQL}) running an outdated and unsupported version, which is a critical vulnerability. This technical finding confirms a previously identified risk and elevates its severity. Furthermore, significant gaps in internal security controls were noted, specifically the lack of mandatory MFA for computer logins and the absence of security awareness training for new employees.

This report details these findings and provides actionable recommendations to mitigate the identified risks and strengthen the overall security posture of \textbf{Stellar Pathways}.

% --- 2. Organizational Information ---
\section{Organizational Information}
The following information was provided for the assessment:
\begin{itemize}
    \item \textbf{Organization Name:} Stellar Pathways
    \item \textbf{Email Domain:} \texttt{StellarPathways.com}
    \item \textbf{Website Domain:} \url{www.StellarPathways.com}
    \item \textbf{External IP Address:} \texttt{100.191.26.59}
\end{itemize}

% --- 3. Security Control Review ---
\section{Security Control Review}
An organizational security questionnaire was reviewed to assess the current state of administrative and policy-based controls. The results are summarized in Table~\ref{tab:controls}. Answers marked with \ding{55} (No) indicate a deviation from security best practices and represent a potential gap in the organization's defenses.

\begin{table}[h!]
\centering
\caption{Organizational Security Controls Questionnaire}
\label{tab:controls}
\begin{tabular}{p{0.8\linewidth} c}
\toprule
\textbf{Control Question} & \textbf{Status} \\
\midrule
Do you require MFA to access email? & \ding{51} \\
Do you require MFA to log into computers? & \textbf{\color{red}\ding{55}} \\
Do you require MFA to access sensitive data systems? & \ding{51} \\
Does your organization have an employee acceptable use policy? & \ding{51} \\
Does your organization do security awareness training for new employees? & \textbf{\color{red}\ding{55}} \\
Does your organization do security awareness training for all employees at least once per year? & \ding{51} \\
\bottomrule
\end{tabular}
\end{table}

\subsection*{Analysis of Gaps}
\begin{itemize}
    \item \textbf{Lack of Endpoint MFA:} The absence of MFA for computer logins is a critical security gap. This weakness significantly increases the risk of unauthorized access resulting from compromised credentials (e.g., through phishing or password spraying).
    \item \textbf{No Security Training for New Hires:} Failing to provide security awareness training during onboarding leaves new employees vulnerable and unaware of organizational policies and common threats, making them prime targets for social engineering attacks.
\end{itemize}

% --- 4. Technical Network Scan Results ---
\section{Technical Network Scan Results}
A network scan was performed on the target system to identify open ports and exposed services. The findings are detailed in Table~\ref{tab:scan}.

\begin{table}[h!]
\centering
\caption{Network Scan Findings for Target: \texttt{172.16.50.20}}
\label{tab:scan}
\begin{tabular}{l l l l}
\toprule
\textbf{Port} & \textbf{State} & \textbf{Service} & \textbf{Version} \\
\midrule
3306/tcp & Open & MySQL & 5.7.33 \\
\bottomrule
\end{tabular}
\end{table}

\subsection*{Technical Analysis}
The scan identified a publicly accessible MySQL database on port \texttt{3306}. This configuration is highly discouraged as it exposes the database directly to the network, making it a target for brute-force attacks, credential stuffing, and exploitation of vulnerabilities.

\textbf{Critical Finding:} The detected MySQL version, \textbf{5.7.33}, reached its official End of Life (EOL) in October 2023. This means it no longer receives security patches or updates from the vendor, and likely contains known, unpatched vulnerabilities. Running EOL software, especially for a critical database service, poses an extreme risk to the confidentiality, integrity, and availability of the data it contains.

% --- 5. Consolidated Risk Assessment ---
\section{Consolidated Risk Assessment}
The following table synthesizes findings from the security questionnaire, technical scan, and pre-existing risk data into a consolidated list of organizational risks.

\begin{table}[h!]
\centering
\caption{Summary of Identified Risks}
\label{tab:risks}
\begin{tabular}{p{0.25\linewidth} p{0.55\linewidth} l}
\toprule
\textbf{Risk Name} & \textbf{Description} & \textbf{Severity} \\
\midrule
\textbf{Database Exposure} & The MySQL database port (\texttt{3306}) is open to the network, allowing direct connection attempts from potentially malicious actors. This confirms a pre-existing identified risk. & High (7.5) \\
\addlinespace
\textbf{Unsupported Database Software} & The MySQL service is running version 5.7.33, which is End of Life (EOL) and no longer receives security updates. This exposes the system to numerous known vulnerabilities. & Critical \\
\addlinespace
\textbf{Lack of Endpoint MFA} & User computers can be accessed without a second factor of authentication, making stolen or weak passwords a critical point of failure for endpoint security. & High \\
\addlinespace
\textbf{Inadequate Onboarding Training} & New employees are not provided with security awareness training, increasing the organization's susceptibility to phishing and other social engineering attacks. & Medium \\
\bottomrule
\end{tabular}
\end{table}

% --- 6. Recommendations ---
\section{Recommendations}
The following actions are recommended to mitigate the identified risks and improve the security posture of \textbf{Stellar Pathways}. Recommendations are prioritized based on severity.

\subsection*{Critical Priority}
\begin{itemize}
    \item \textbf{Upgrade Unsupported Database Software:}
    \begin{itemize}
        \item Immediately create a plan to migrate the MySQL 5.7.33 database to a fully supported version (e.g., MySQL 8.x).
        \item Test the application's compatibility with the new database version in a staging environment before deploying to production.
    \end{itemize}
\end{itemize}

\subsection*{High Priority}
\begin{itemize}
    \item \textbf{Remediate Database Exposure:}
    \begin{itemize}
        \item \textbf{Immediate Fix:} Implement strict firewall rules to restrict access to port \texttt{3306}. Access should only be allowed from specific, trusted IP addresses (e.g., application servers). Deny all other traffic by default.
        \item \textbf{Long-Term Fix:} For any administrative access, require connections to be made over a secure Virtual Private Network (VPN). Publicly exposing a database port is never a recommended practice.
    \end{itemize}
    \item \textbf{Implement Endpoint MFA:}
    \begin{itemize}
        \item Deploy a solution that enforces MFA for all user logins to company workstations and laptops.
        \item This can be achieved through modern identity providers (e.g., Azure AD, Okta) or dedicated endpoint security solutions.
    \end{itemize}
\end{itemize}

\subsection*{Medium Priority}
\begin{itemize}
    \item \textbf{Establish Security Onboarding Process:}
    \begin{itemize}
        \item Integrate mandatory security awareness training into the new-employee onboarding process.
        \item Training should cover key topics such as the acceptable use policy, phishing identification, password hygiene, and incident reporting procedures.
    \end{itemize}
\end{itemize}

\end{document}
```