```latex
\documentclass[12pt]{article}

% --- PACKAGES ---
\usepackage[margin=1in]{geometry}
\usepackage{pifont} % For checkmarks and crosses
\usepackage{booktabs} % For professional tables
\usepackage{hyperref} % For clickable links
\usepackage{url} % For URL formatting
\usepackage{seqsplit} % To split long strings in texttt
\usepackage{graphicx}
\usepackage{xcolor}
\usepackage{fancyhdr}
\usepackage{lastpage}

% --- DOCUMENT SETUP ---
\hypersetup{
    colorlinks=true,
    linkcolor=black,
    urlcolor=blue,
    pdftitle={Cybersecurity Posture Report},
    pdfauthor={Cybersecurity Analysis Unit}
}

% --- HEADER & FOOTER ---
\pagestyle{fancy}
\fancyhf{} % Clear all header and footer fields
\fancyhead[L]{Cybersecurity Posture Report}
\fancyhead[R]{Borealis Tech}
\fancyfoot[C]{\thepage\ of \pageref{LastPage}}
\renewcommand{\headrulewidth}{0.4pt}
\renewcommand{\footrulewidth}{0.4pt}

% --- COMMANDS ---
\newcommand{\yes}{\ding{51}}
\newcommand{\no}{\ding{55}}

\begin{document}

% --- TITLE PAGE ---
\begin{titlepage}
    \centering
    \vspace*{1cm}
    \Huge\textbf{Cybersecurity Posture Report}
    \vspace{1.5cm}
    \Large
    \textbf{Prepared for:} \\
    \vspace{0.5cm}
    \textbf{Borealis Tech} \\
    \vspace{2cm}
    \textbf{Date of Report:} \\
    \vspace{0.5cm}
    \today
    \vfill
    \large
    \textit{This report contains sensitive information and is intended solely for the use of the recipient organization. Distribution is strictly prohibited.}
\end{titlepage}

\tableofcontents
\newpage

% --- EXECUTIVE OVERVIEW ---
\section{Executive Overview}
This report provides a cybersecurity posture assessment for \textbf{Borealis Tech}, based on an analysis of organizational security controls provided via a questionnaire. 

It is critical to note that the technical network scan data (\texttt{Input\_1\_Network\_Scan\_JSON}) and the pre-existing risk data (\texttt{Input\_3\_Current\_Risks\_JSON}) were found to be corrupted and could not be processed for this report. Consequently, this assessment is based exclusively on the administrative controls self-reported by the organization.

The analysis indicates a mixed security posture. The organization has implemented strong Multi-Factor Authentication (MFA) controls across key systems, which significantly reduces the risk of unauthorized access. However, two critical gaps were identified in foundational security policies and practices:
\begin{itemize}
    \item \textbf{Lack of an Employee Acceptable Use Policy (AUP):} This is a critical deficiency that exposes the organization to insider threats, misuse of assets, and potential legal liabilities.
    \item \textbf{Absence of Annual Security Awareness Training:} While new hires receive training, the lack of a recurring, annual program for all staff leaves the organization highly vulnerable to evolving social engineering tactics like phishing.
\end{itemize}

The overall risk level is assessed as \textbf{Moderate}, with a strong recommendation to immediately address the identified policy and training gaps. A successful re-scan of the network perimeter is also a top priority to complete the technical portion of this assessment.

% --- ORGANIZATIONAL INFORMATION ---
\section{Organizational Information}
The following details were provided by the client for this assessment.
\begin{itemize}
    \item \textbf{Organization Name:} Borealis Tech
    \item \textbf{Email Domain:} \texttt{BorealisTech.com}
    \item \textbf{Website Domain:} \url{www.BorealisTech.com}
    \item \textbf{Primary External IP:} \texttt{114.211.251.84}
\end{itemize}

% --- SECURITY CONTROL REVIEW ---
\section{Security Control Review (Questionnaire Analysis)}
The following table summarizes the organization's self-reported security controls. "No" answers represent significant gaps that increase organizational risk.

\begin{table}[h!]
\centering
\caption{Security Controls Questionnaire Summary}
\begin{tabular}{p{0.7\textwidth} c c}
\toprule
\textbf{Control Question} & \textbf{Response} & \textbf{Status} \\
\midrule
Do you require MFA to access email? & Yes & \textcolor{green}{\yes} \\
Do you require MFA to log into computers? & Yes & \textcolor{green}{\yes} \\
Do you require MFA to access sensitive data systems? & Yes & \textcolor{green}{\yes} \\
Does your organization have an employee acceptable use policy? & No & \textcolor{red}{\no} \\
Does your organization do security awareness training for new employees? & Yes & \textcolor{green}{\yes} \\
Does your organization do security awareness training for all employees at least once per year? & No & \textcolor{red}{\no} \\
\bottomrule
\end{tabular}
\end{table}

\subsection*{Analysis of Gaps}
\begin{itemize}
    \item \textbf{Acceptable Use Policy (AUP):} The absence of a formal AUP is a critical administrative gap. This policy is the foundation for setting user expectations regarding the use of company technology and data. Without it, there is no formal basis for enforcing security standards or taking disciplinary action in case of misuse.
    \item \textbf{Annual Security Awareness Training:} The threat landscape evolves continuously. Training provided only to new hires becomes outdated quickly. A lack of mandatory, annual training for all employees significantly increases the likelihood that an employee will fall victim to a phishing email or other social engineering attack, potentially leading to a major breach.
\end{itemize}

% --- TECHNICAL SCAN RESULTS ---
\section{Technical Scan Results}
\textbf{Data Not Available:} The provided network scan data file (\texttt{Input\_1\_Network\_Scan\_JSON}) was determined to be corrupted or incomplete. As a result, no technical analysis of the external IP address (\texttt{114.211.251.84}) could be performed.

This prevents the identification of potential vulnerabilities, including:
\begin{itemize}
    \item Open ports and exposed services.
    \item Outdated software versions with known exploits (e.g., vulnerable web servers, SSH, or RDP).
    \item Insecure service configurations.
\end{itemize}
A re-scan is required to complete this crucial part of the assessment.

% --- RISK ASSESSMENT ---
\section{Risk Assessment}
This risk assessment is based solely on the findings from the Security Control Review. The pre-existing risk data (\texttt{Input\_3\_Current\_Risks\_JSON}) was unavailable. The severity is rated on a scale of Low, Medium, High, and Critical.

\begin{table}[h!]
\centering
\caption{Identified Risks}
\begin{tabular}{p{0.1\textwidth} p{0.25\textwidth} p{0.45\textwidth} p{0.1\textwidth}}
\toprule
\textbf{Risk ID} & \textbf{Risk Name} & \textbf{Description} & \textbf{Severity} \\
\midrule
RISK-001 & Lack of Employee Acceptable Use Policy (AUP) & Without a formal AUP, employees lack clear guidelines on the acceptable use of company systems, data, and networks. This can lead to unintentional misuse, security breaches, and legal/compliance issues. & \textbf{Critical} \\
\addlinespace
RISK-002 & No Annual Security Awareness Training & The absence of mandatory, recurring security training for all employees increases susceptibility to social engineering attacks like phishing, as security knowledge degrades over time. & \textbf{High} \\
\bottomrule
\end{tabular}
\end{table}

% --- RECOMMENDATIONS ---
\section{Recommendations}
The following actions are recommended to mitigate the identified risks and improve the overall security posture of \textbf{Borealis Tech}.

\subsection*{Immediate Actions (Next 30 Days)}
\begin{enumerate}
    \item \textbf{Develop and Implement an Acceptable Use Policy (RISK-001):}
    \begin{itemize}
        \item Draft a comprehensive AUP that covers data handling, internet usage, email security, use of personal devices, and consequences for non-compliance.
        \item Require all current employees to read and formally acknowledge the new policy.
        \item Integrate the AUP acknowledgement into the standard onboarding process for all future hires.
    \end{itemize}
    \item \textbf{Establish an Annual Security Training Program (RISK-002):}
    \begin{itemize}
        \item Procure and schedule a mandatory security awareness training module for all employees to be completed within the next quarter.
        \item The training must cover modern threats, including phishing, ransomware, business email compromise, and password hygiene.
        \item Plan for this training to be a recurring, annual requirement for all staff.
    \end{itemize}
    \item \textbf{Execute a New Network Scan:}
    \begin{itemize}
        \item Conduct a new, validated external vulnerability scan against the public IP address \texttt{114.211.251.84}.
        \item Provide the scan results to the security team for a comprehensive technical analysis and an updated report.
    \end{itemize}
\end{enumerate}

\subsection*{Strategic Actions (Next 6-12 Months)}
\begin{enumerate}
    \item \textbf{Implement Phishing Simulations:}
    \begin{itemize}
        \item Following the initial training, begin a program of regular (e.g., quarterly) phishing simulations.
        \item Use the results to identify individuals or departments that require additional, targeted training.
    \end{itemize}
    \item \textbf{Policy Review Cycle:}
    \begin{itemize}
        \item Establish a formal annual review cycle for the new AUP and any other security policies to ensure they remain relevant and effective against emerging threats.
    \end{itemize}
\end{enumerate}

\end{document}
```