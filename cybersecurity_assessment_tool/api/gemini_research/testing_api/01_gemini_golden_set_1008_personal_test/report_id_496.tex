```latex
\documentclass[12pt, a4paper]{article}

% Preamble: Required Packages and Document Setup
\usepackage[margin=1in]{geometry}
\usepackage{pifont}
\usepackage{booktabs}
\usepackage[hidelinks]{hyperref}
\usepackage{url}
\usepackage{seqsplit}
\usepackage{graphicx}
\usepackage{fancyhdr}
\usepackage{lastpage}
\usepackage{xcolor}
\usepackage{array}

% --- Customization ---
% Define colors for severity and status
\definecolor{darkred}{rgb}{0.76, 0.0, 0.0}
\definecolor{darkgreen}{rgb}{0.0, 0.5, 0.0}
\definecolor{darkorange}{rgb}{1.0, 0.55, 0.0}
\definecolor{darkblue}{rgb}{0.0, 0.0, 0.55}
\definecolor{lightgray}{rgb}{0.9, 0.9, 0.9}

% Define custom commands for checkmarks and crosses
\newcommand{\cmark}{\textcolor{darkgreen}{\ding{51}}}
\newcommand{\xmark}{\textcolor{darkred}{\ding{55}}}

% Configure hyperref
\hypersetup{
    colorlinks=true,
    linkcolor=darkblue,
    filecolor=magenta,      
    urlcolor=darkblue,
    pdftitle={Cybersecurity Posture Assessment Report},
    pdfauthor={Cybersecurity Analyst},
    pdfsubject={Security Assessment},
    pdfkeywords={Cybersecurity, Report, Analysis},
    pdftoolbar=true,
}

% Header and Footer Configuration
\pagestyle{fancy}
\fancyhf{} % Clear all header and footer fields
\fancyhead[L]{Cybersecurity Posture Assessment}
\fancyhead[R]{North Star Education}
\fancyfoot[C]{\thepage\ of \pageref{LastPage}}
\renewcommand{\headrulewidth}{0.4pt}
\renewcommand{\footrulewidth}{0.4pt}

% --- Document Start ---
\begin{document}

% --- Title Page ---
\begin{titlepage}
    \centering
    \vspace*{1cm}
    
    \Huge
    \textbf{Cybersecurity Posture Assessment Report}
    
    \vspace{1.5cm}
    
    \Large
    Prepared for: \\
    \vspace{0.5cm}
    \textbf{North Star Education}
    
    \vfill
    
    \large
    \textbf{Date of Report:} \today \\
    \textbf{Analysis Period:} Based on data provided on \today
    
    \vspace{1cm}
    
    \textit{This report is confidential and intended solely for the use of North Star Education. Distribution without prior written consent is prohibited.}
    
\end{titlepage}

% --- Table of Contents ---
\tableofcontents
\newpage

% --- Section 1: Executive Summary ---
\section{Executive Summary}
This report provides a comprehensive assessment of the cybersecurity posture for \textbf{North Star Education}, based on an analysis of network scan data, organizational security controls, and known risks.

The assessment reveals a mixed security posture. On a positive note, the external network scan of the target host \seqsplit{\texttt{192.168.1.100}} showed no open ports, indicating a strong perimeter defense and a minimal external attack surface for that specific asset. This is a commendable security configuration.

However, a review of organizational security controls identified several \textbf{critical gaps} that significantly elevate the risk of a security breach. The complete absence of Multi-Factor Authentication (MFA) across email, computer logins, and sensitive data systems exposes the organization to severe risks from credential theft and account takeover attacks. Furthermore, the lack of an annual security awareness training program for all employees and a formal Acceptable Use Policy (AUP) weakens the human element of the defense strategy.

Our primary recommendations focus on immediately addressing these critical control gaps. The highest priority is the phased implementation of MFA, starting with email and sensitive systems. This should be followed by the development of a comprehensive security awareness program and the formalization of key security policies.

% --- Section 2: Organizational Information ---
\section{Organizational Information}
The following details were provided for the assessment.

\begin{tabular}{@{}ll}
    \toprule
    \textbf{Attribute} & \textbf{Value} \\
    \midrule
    Organization Name & \textbf{North Star Education} \\
    Email Domain & \seqsplit{\texttt{NorthStarEducation.net}} \\
    External IP Address & \seqsplit{\texttt{122.176.159.66}} \\
    \bottomrule
\end{tabular}

% --- Section 3: Security Control Review ---
\section{Security Control Review}
The following table summarizes the organization's responses to a security controls questionnaire. Each response is assessed against industry best practices. "No" answers represent significant gaps in the security framework.

\begin{table}[h!]
\centering
\begin{tabular}{p{0.6\textwidth} >{\centering\arraybackslash}p{0.1\textwidth} l}
    \toprule
    \textbf{Control Question} & \textbf{Response} & \textbf{Assessment} \\
    \midrule
    Do you require MFA to access email? & \xmark & \textcolor{darkred}{\textbf{Critical Gap}} \\
    Do you require MFA to log into computers? & \xmark & \textcolor{darkorange}{High Risk} \\
    Do you require MFA to access sensitive data systems? & \xmark & \textcolor{darkred}{\textbf{Critical Gap}} \\
    Does your organization have an employee acceptable use policy? & \xmark & \textcolor{darkorange}{High Risk} \\
    Does your organization do security awareness training for new employees? & \cmark & Best Practice \\
    Does your organization do security awareness training for all employees at least once per year? & \xmark & \textcolor{darkorange}{High Risk} \\
    \bottomrule
\end{tabular}
\caption{Organizational Security Controls Assessment}
\end{table}

The lack of MFA is the most pressing issue. Email is a primary target for phishing, and without MFA, a single compromised password can lead to a full account takeover. Similarly, the absence of MFA on sensitive systems removes a critical layer of defense for the organization's most valuable data.

% --- Section 4: Technical Scan Results ---
\section{Technical Scan Results}
A network scan was performed to identify open ports and services exposed on the organization's network perimeter.

\begin{itemize}
    \item \textbf{Target IP Address:} \seqsplit{\texttt{192.168.1.100}}
    \item \textbf{Scan Date:} Scan data provided on \today
    \item \textbf{Host Status:} \textbf{Up}
\end{itemize}

\subsection{Scan Findings}
The scan results were positive and indicate a strong network security posture for the scanned host.
\begin{itemize}
    \item \textbf{Open Ports:} \textbf{None Detected.}
    \item \textbf{Filtered/Closed Ports:} All other scanned ports were found to be in a closed state.
\end{itemize}

\textbf{Analyst Interpretation:} The absence of open ports on the target system is an excellent security practice. It drastically reduces the attack surface available to external threats and suggests that a well-configured firewall is in place, properly implementing a policy of "default deny."

% --- Section 5: Consolidated Risk Assessment ---
\section{Consolidated Risk Assessment}
The following table synthesizes findings from the security control review and technical scan into a prioritized list of organizational risks. No pre-existing vulnerabilities were reported.

\begin{table}[h!]
\centering
\begin{tabular}{p{0.1\textwidth} p{0.25\textwidth} p{0.4\textwidth} >{\centering\arraybackslash}p{0.1\textwidth}}
    \toprule
    \textbf{ID} & \textbf{Risk Name} & \textbf{Description} & \textbf{Severity} \\
    \midrule
    \rowcolor{lightgray}
    RISK-001 & \textbf{Lack of Multi-Factor Authentication (MFA)} & The absence of MFA for email, computer, and sensitive system access makes the organization highly vulnerable to account takeovers via credential theft or phishing. & \textcolor{darkred}{\textbf{Critical}} \\
    RISK-002 & Insufficient Security Awareness Program & While new employees receive training, the lack of an annual refresher for all staff leads to knowledge decay and an inability to recognize new threats. & \textcolor{darkorange}{High} \\
    \rowcolor{lightgray}
    RISK-003 & Missing Acceptable Use Policy (AUP) & Without a formal AUP, there is no clear guidance for employees on the proper use of company assets, creating legal and operational risks. & \textcolor{darkorange}{High} \\
    \bottomrule
\end{tabular}
\caption{Summary of Identified Risks}
\end{table}

% --- Section 6: Recommendations ---
\section{Recommendations}
The following actionable recommendations are prioritized based on the risk assessment to help \textbf{North Star Education} improve its cybersecurity posture.

\subsection{Priority 1: Critical}
\begin{description}
    \item[Implement Multi-Factor Authentication (MFA):] This is the single most effective control to mitigate the risks identified.
    \begin{itemize}
        \item \textbf{Immediately:} Enforce MFA on all email accounts (e.g., via Office 365 or Google Workspace security settings).
        \item \textbf{Next:} Enforce MFA on all remote access systems (VPNs) and any applications that process or store sensitive data.
        \item \textbf{Finally:} Plan a phased rollout of MFA for all employee computer logins.
    \end{itemize}
\end{description}

\subsection{Priority 2: High}
\begin{description}
    \item[Establish a Continuous Security Awareness Program:] A one-time training for new hires is insufficient.
    \begin{itemize}
        \item Procure or develop a security awareness training module that is mandatory for all employees on an annual basis.
        \item Conduct regular phishing simulations to test employee awareness and provide targeted follow-up training.
    \end{itemize}
    \item[Develop and Enforce an Acceptable Use Policy (AUP):]
    \begin{itemize}
        \item Work with management, HR, and legal counsel to draft a clear AUP that governs the use of all company IT assets, including computers, networks, and data.
        \item Require all employees to read and formally acknowledge the policy as a condition of their employment.
    \end{itemize}
\end{description}

\subsection{Priority 3: Informational}
\begin{description}
    \item[Maintain Strong Network Security:]
    \begin{itemize}
        \item The clean network scan is a positive result. Continue this practice by conducting regular, authenticated vulnerability scans on both external and internal network assets.
        \item Regularly review firewall rules to ensure they remain aligned with business needs and security best practices.
    \end{itemize}
\end{description}

\end{document}
```