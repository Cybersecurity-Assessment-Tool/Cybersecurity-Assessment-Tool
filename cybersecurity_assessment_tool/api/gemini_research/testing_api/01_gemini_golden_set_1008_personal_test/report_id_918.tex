```latex
\documentclass[12pt]{article}

% Required Packages
\usepackage[a4paper, margin=1in]{geometry}
\usepackage{pifont} % For checkmarks and crosses
\usepackage{booktabs} % For professional tables
\usepackage{hyperref} % For clickable links
\usepackage{url} % For URL formatting
\usepackage{seqsplit} % For splitting long text strings
\usepackage{graphicx}
\usepackage[table]{xcolor} % For coloring table cells
\usepackage{fancyhdr} % For custom headers and footers

% --- Document Setup ---
\hypersetup{
    colorlinks=true,
    linkcolor=blue,
    filecolor=magenta,      
    urlcolor=cyan,
    pdftitle={Cybersecurity Posture Report},
    pdfpagemode=FullScreen,
}

% --- Custom Colors for Severity ---
\definecolor{severityhigh}{HTML}{D9534F}
\definecolor{severitymedium}{HTML}{F0AD4E}
\definecolor{severitylow}{HTML}{5CB85C}
\definecolor{severityinfo}{HTML}{5BC0DE}
\definecolor{tablehead}{gray}{0.9}

% --- Header and Footer ---
\pagestyle{fancy}
\fancyhf{} % Clear all header and footer fields
\fancyhead[L]{Cybersecurity Posture Report}
\fancyhead[R]{Willow Creek Health}
\fancyfoot[C]{\thepage}
\renewcommand{\headrulewidth}{0.4pt}
\renewcommand{\footrulewidth}{0.4pt}

\begin{document}

% --- Title Page ---
\begin{titlepage}
    \centering
    \vspace*{1cm}
    \Huge
    \textbf{Cybersecurity Posture Report}
    \vspace{1.5cm}
    \Large
    Prepared for: \\
    \vspace{0.5cm}
    \textbf{Willow Creek Health}
    \vspace{2cm}
    \large
    Report Date: \today
    \vfill
    \normalsize
    This report provides an analysis of the organization's security posture based on network scans, a security controls questionnaire, and a review of pre-existing risks.
\end{titlepage}

\tableofcontents
\newpage

% --- Section 1: Executive Summary ---
\section{Executive Summary}
This report synthesizes data from a network vulnerability scan, a security controls questionnaire, and a list of current risks to provide a comprehensive overview of the cybersecurity posture for \textbf{Willow Creek Health}.

The assessment identified one critical gap in the organization's security controls. While Multi-Factor Authentication (MFA) is commendably enforced for email and computer access, its absence on systems containing sensitive data presents a \textbf{High} risk. Unauthorized access to these systems could lead to a significant data breach.

On a positive note, a technical scan of the target host \texttt{192.168.0.5} revealed no open ports. This finding indicates that a previously identified risk, "Unencrypted Web Server" on Port 80, has been successfully \textbf{mitigated}. This demonstrates proactive risk management.

Key recommendations focus on prioritizing the deployment of MFA for all sensitive data systems to close the identified security gap and continuing with regular vulnerability scanning to maintain a strong security posture.

% --- Section 2: Organizational Information ---
\section{Organizational Information}
The following details were provided for the assessment.

\begin{tabular}{@{}ll}
\toprule
\textbf{Attribute} & \textbf{Value} \\
\midrule
Organization Name & \textbf{Willow Creek Health} \\
Email Domain & \texttt{WillowCreekHealth.com} \\
Website Domain & \url{www.WillowCreekHealth.com} \\
External IP Address & \texttt{29.81.135.149} \\
\bottomrule
\end{tabular}

% --- Section 3: Security Control Review ---
\section{Security Control Review}
The following table summarizes the organization's responses to the security controls questionnaire. A checkmark (\ding{51}) indicates a positive control is in place, while a cross (\ding{55}) indicates a potential security gap.

\rowcolors{2}{gray!10}{white}
\begin{tabular}{p{0.6\textwidth} c p{0.25\textwidth}}
\toprule
\rowcolor{tablehead}
\textbf{Control Question} & \textbf{Response} & \textbf{Analyst Notes} \\
\midrule
Do you require MFA to access email? & \textcolor{green}{\ding{51}} & Strong control. Protects against email account takeover. \\
Do you require MFA to log into computers? & \textcolor{green}{\ding{51}} & Excellent. Reduces risk from compromised credentials. \\
Do you require MFA to access sensitive data systems? & \textcolor{red}{\ding{55}} & \textbf{Critical Gap.} Lack of MFA on critical systems is a high-risk exposure. \\
Does your organization have an employee acceptable use policy? & \textcolor{green}{\ding{51}} & Foundational policy for security governance. \\
Does your organization do security awareness training for new employees? & \textcolor{green}{\ding{51}} & Good practice for onboarding. \\
Does your organization do security awareness training for all employees at least once per year? & \textcolor{green}{\ding{51}} & Meets compliance and best practice standards. \\
\bottomrule
\end{tabular}

\subsection{Analysis}
The questionnaire reveals a strong foundation in security policies and general access controls. However, the lack of MFA for sensitive data systems is a significant weakness that attackers could exploit if they obtain valid user credentials. This finding is the highest priority for remediation.

% --- Section 4: Technical Scan Results ---
\section{Technical Scan Results}
An external network scan was performed to identify open ports and exposed services.

\begin{itemize}
    \item \textbf{Scan Target:} \texttt{192.168.0.5}
    \item \textbf{Scan Date:} Scan performed prior to this report's generation date.
\end{itemize}

\subsection{Port Scan Findings}
The scan results for the target host are detailed below.
\rowcolors{2}{gray!10}{white}
\begin{tabular}{llll}
\toprule
\rowcolor{tablehead}
\textbf{Port} & \textbf{State} & \textbf{Service} & \textbf{Version} \\
\midrule
80/tcp & Closed & N/A & N/A \\
\bottomrule
\end{tabular}

\subsection{Analysis}
The scan indicates that the target host has a secure network posture from the perspective of the scanned ports. No open ports were discovered. This is a positive finding and contradicts a pre-existing risk entry (see Section 5), suggesting recent remediation efforts have been successful.

% --- Section 5: Risk Assessment ---
\section{Risk Assessment}
This section correlates findings from the security questionnaire, technical scans, and pre-existing risk data into a consolidated list.

\rowcolors{2}{gray!10}{white}
\begin{tabular}{p{0.5\textwidth} p{0.15\textwidth} p{0.2\textwidth}}
\toprule
\rowcolor{tablehead}
\textbf{Risk / Finding} & \textbf{Severity} & \textbf{Status} \\
\midrule
\textbf{Lack of MFA on Sensitive Systems} \newline \footnotesize{Identified from the security questionnaire. Failure to protect critical data stores with MFA leaves them vulnerable to credential-based attacks.} & \cellcolor{severityhigh!80}\textbf{High} & \textbf{Active} \\
\addlinespace
\textbf{Unencrypted Web Server (Port 80)} \newline \footnotesize{A pre-existing risk stated that Port 80 was open. The current network scan confirms this port is now \textbf{closed}, mitigating the risk.} & \cellcolor{severitymedium!80}\textbf{Medium} & \textbf{Mitigated} \\
\bottomrule
\end{tabular}

% --- Section 6: Recommendations ---
\section{Recommendations}
Based on the analysis, the following actions are recommended to enhance the security posture of \textbf{Willow Creek Health}.

\subsection{High Priority}
\begin{itemize}
    \item \textbf{Implement MFA on Sensitive Data Systems:}
    \begin{itemize}
        \item \textbf{Action:} Deploy a mandatory Multi-Factor Authentication solution for all user accounts (including administrative and service accounts) that have access to systems containing sensitive patient, financial, or organizational data.
        \item \textbf{Justification:} This is the single most effective control to prevent unauthorized access resulting from compromised credentials. It directly addresses the primary active risk identified in this report.
    \end{itemize}
\end{itemize}

\subsection{Informational / Validation}
\begin{itemize}
    \item \textbf{Acknowledge and Validate Remediation:}
    \begin{itemize}
        \item \textbf{Action:} Formally update internal risk registers to reflect that the risk associated with the open Port 80 is now mitigated.
        \item \textbf{Justification:} Acknowledging successful remediation efforts is crucial for accurate risk management and demonstrates a maturing security program.
    \end{itemize}
    \item \textbf{Continue Regular Scanning:}
    \begin{itemize}
        \item \textbf{Action:} Maintain a schedule of regular, automated vulnerability scanning for all internal and external assets.
        \item \textbf{Justification:} Continuous monitoring ensures that new vulnerabilities or misconfigurations are detected and addressed promptly, preventing the re-emergence of old risks or the introduction of new ones.
    \end{itemize}
\end{itemize}

\end{document}
```