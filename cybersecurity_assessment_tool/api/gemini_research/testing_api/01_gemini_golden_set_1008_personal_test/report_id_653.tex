```latex
\documentclass[12pt]{article}

% --- PACKAGES ---
\usepackage[margin=1in]{geometry} % Set page margins
\usepackage{pifont}               % For checkmarks and crosses (\ding)
\usepackage{booktabs}             % For professional-looking tables
\usepackage{graphicx}             % For potential logos (not used here, but good practice)
\usepackage{array}                % For table column formatting
\usepackage{xcolor}               % For colors
\usepackage{hyperref}             % For hyperlinks
\usepackage{url}                  % For URL formatting
\usepackage{seqsplit}             % To split long strings in texttt

% --- DOCUMENT METADATA ---
\title{Cybersecurity Posture Assessment Report}
\author{Cybersecurity Analyst}
\date{\today}

% --- HYPERREF SETUP ---
\hypersetup{
    colorlinks=true,
    linkcolor=blue,
    filecolor=magenta,      
    urlcolor=cyan,
    pdftitle={Cybersecurity Posture Assessment Report},
    pdfpagemode=FullScreen,
}

% --- CUSTOM COMMANDS ---
\newcommand{\yes}{\ding{51}} % Green checkmark
\newcommand{\no}{\ding{55}}  % Red cross

\begin{document}

\maketitle
\hrule
\vspace{1em}
\begin{center}
    \textbf{Organization:} Silent Spring \\
    \textbf{Report Date:} \today
\end{center}
\vspace{1em}
\hrule

\tableofcontents
\newpage

% ===================================================================
\section{Executive Summary}
% ===================================================================

This report provides a cybersecurity assessment for Silent Spring, based on a combination of technical network scanning, a review of organizational security controls, and an analysis of pre-existing risks.

The assessment has identified several critical and high-risk vulnerabilities that require immediate attention. The most pressing finding is the exposure of Remote Desktop Protocol (RDP) on an internal server (\texttt{10.10.10.51}). This issue is compounded by a previously identified RDP exposure on another system (\texttt{10.10.10.50}), indicating a systemic configuration weakness.

Furthermore, critical security gaps were identified in the organization's security policies. The lack of mandatory Multi-Factor Authentication (MFA) for computer logins and the absence of a formal security awareness training program for employees significantly increase the risk of a successful cyberattack, such as a ransomware incident originating from compromised credentials.

This report outlines these findings in detail and provides a prioritized list of actionable recommendations to mitigate the identified risks and strengthen the overall security posture of Silent Spring.

% ===================================================================
\section{Organizational Information}
% ===================================================================

The following information was provided for the assessment.

\begin{itemize}
    \item \textbf{Organization Name:} Silent Spring
    \item \textbf{Email Domain:} \seqsplit{\texttt{SilentSpring.org}}
    \item \textbf{Website Domain:} \seqsplit{\texttt{www.SilentSpring.org}}
    \item \textbf{External IP Address:} \seqsplit{\texttt{62.183.65.2}}
\end{itemize}

% ===================================================================
\section{Security Control Review}
% ===================================================================

A review of the organization's security controls was conducted via a questionnaire. The results are summarized below. "No" answers indicate significant gaps in the security framework.

\begin{table}[h!]
\centering
\caption{Security Controls Questionnaire Results}
\begin{tabular}{p{0.75\textwidth} c}
\toprule
\textbf{Control Question} & \textbf{Status} \\
\midrule
Do you require MFA to access email? & \yes \\
Do you require MFA to log into computers? & \no \\
Do you require MFA to access sensitive data systems? & \yes \\
Does your organization have an employee acceptable use policy? & \yes \\
Does your organization do security awareness training for new employees? & \no \\
Does your organization do security awareness training for all employees at least once per year? & \no \\
\bottomrule
\end{tabular}
\end{table}

\subsection*{Analysis of Gaps}
The review identified three critical gaps:
\begin{itemize}
    \item \textbf{No MFA for Computer Logins:} This is a high-risk gap. Without MFA, compromised user credentials (e.g., from a phishing attack) can be used directly to gain access to an employee's computer and, by extension, the internal network.
    \item \textbf{No Security Training for New Employees:} New hires are often targeted by attackers. Failing to provide immediate security training leaves a critical window of vulnerability.
    \item \textbf{No Annual Security Training:} The threat landscape is constantly evolving. A lack of recurring training means employees' awareness of current threats, like sophisticated phishing emails, will diminish over time.
\end{itemize}

% ===================================================================
\section{Technical Scan Results}
% ===================================================================

A network scan was performed on the specified target to identify open ports and exposed services.

\begin{itemize}
    \item \textbf{Target IP Address:} \seqsplit{\texttt{10.10.10.51}}
    \item \textbf{Scan Tool:} Nmap
\end{itemize}

\begin{table}[h!]
\centering
\caption{Open Ports on \texttt{10.10.10.51}}
\begin{tabular}{c c l l}
\toprule
\textbf{Port} & \textbf{State} & \textbf{Service} & \textbf{Notes} \\
\midrule
3389/tcp & open & \texttt{ms-wbt-server} & Remote Desktop Protocol (RDP) \\
\bottomrule
\end{tabular}
\end{table}

\subsection*{Analysis of Findings}
The scan revealed that port \textbf{3389 (RDP)} is open on the host \texttt{10.10.10.51}. RDP is a primary vector for ransomware attacks and unauthorized access. Exposing this service directly on the network without mitigating controls like a VPN or strict access lists is a critical vulnerability. This finding, correlated with the pre-existing risk of RDP exposure on \texttt{10.10.10.50}, suggests a pattern of insecure server configuration.

% ===================================================================
\section{Consolidated Risk Assessment}
% ===================================================================

The following table synthesizes findings from the security control review, technical scan, and pre-existing risk data into a consolidated list of key risks.

\begin{table}[h!]
\centering
\caption{Summary of Identified Risks}
\begin{tabular}{p{0.25\textwidth} p{0.45\textwidth} p{0.15\textwidth}}
\toprule
\textbf{Risk / Vulnerability} & \textbf{Description} & \textbf{Severity} \\
\midrule
\textbf{Systemic RDP Exposure} & RDP (Port 3389) is exposed on multiple internal systems (\texttt{10.10.10.50}, \texttt{10.10.10.51}). This is a common entry point for ransomware. & \textbf{Critical} \\
\addlinespace
\textbf{Lack of Endpoint MFA} & The absence of MFA on computer logins, combined with exposed RDP, creates a high-impact path for an attacker with stolen credentials to access the network. & \textbf{High} \\
\addlinespace
\textbf{Inadequate Security Awareness Training} & No security training for new or existing employees significantly increases susceptibility to phishing and other social engineering attacks, which are the primary source of credential theft. & \textbf{High} \\
\bottomrule
\end{tabular}
\end{table}

% ===================================================================
\section{Recommendations}
% ===================================================================

The following actions are recommended to mitigate the identified risks. They are prioritized based on severity and potential impact.

\subsection*{Immediate Priority (To Be Completed Within 72 Hours)}
\begin{enumerate}
    \item \textbf{Restrict RDP Access:} Immediately close port 3389 on \texttt{10.10.10.51} and any other systems where it is not strictly necessary. For required administrative access, place these systems behind a Virtual Private Network (VPN) that requires Multi-Factor Authentication.
\end{enumerate}

\subsection*{High Priority (To Be Completed Within 30 Days)}
\begin{enumerate}
    \setcounter{enumi}{1}
    \item \textbf{Implement MFA for All Computer Logins:} Deploy a mandatory MFA solution for all employee and privileged user accounts to log into their workstations and servers. This is the single most effective control to prevent credential abuse.
    \item \textbf{Establish a Security Awareness Program:}
    \begin{itemize}
        \item Develop and implement a mandatory security training module for all new employees as part of their onboarding process.
        \item Schedule and conduct annual security awareness training for all staff to keep them informed of current threats.
    \end{itemize}
\end{enumerate}

\subsection*{Medium Priority (To Be Completed Within 90 Days)}
\begin{enumerate}
    \setcounter{enumi}{3}
    \item \textbf{Conduct a Comprehensive Internal Vulnerability Scan:} Perform a credentialed scan of the entire internal network to identify other instances of RDP exposure, outdated software, or misconfigurations.
    \item \textbf{Develop Secure Configuration Baselines:} Create and enforce standardized, secure configuration templates for all new servers and workstations to prevent the recurrence of these vulnerabilities.
\end{enumerate}

\end{document}
```