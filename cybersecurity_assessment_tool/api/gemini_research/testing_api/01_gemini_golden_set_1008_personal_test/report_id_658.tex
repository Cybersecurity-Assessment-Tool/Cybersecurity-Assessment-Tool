```latex
\documentclass[12pt]{article}

% Preamble: Required Packages
\usepackage[margin=1in]{geometry}
\usepackage{pifont} % For using symbols like checkmarks (\ding{51}) and crosses (\ding{55})
\usepackage{booktabs} % For professional-looking tables
\usepackage{hyperref} % For clickable links
\usepackage{url}      % For formatting URLs
\usepackage{seqsplit} % To split long strings in \texttt
\usepackage{graphicx} % For potential logo inclusion
\usepackage{xcolor}   % For custom colors

% Document Information
\title{Cybersecurity Posture Assessment Report}
\author{Cybersecurity Analysis Division}
\date{\today}

% Hyperref Setup
\hypersetup{
    colorlinks=true,
    linkcolor=blue,
    filecolor=magenta,      
    urlcolor=cyan,
    pdftitle={Cybersecurity Posture Assessment Report},
    pdfpagemode=FullScreen,
}

\begin{document}

\maketitle
\thispagestyle{empty} % No page number on the title page

\newpage
\tableofcontents
\thispagestyle{empty} % No page number on the ToC page

\newpage
\pagestyle{headings} % Regular page numbering from here

% --- Section 1: Executive Summary ---
\section{Executive Summary}

This report details the findings of a cybersecurity posture assessment for \textbf{Solaris Energy}. The assessment combines an analysis of self-reported security controls, an external network vulnerability scan, and a review of previously identified risks.

The analysis revealed several \textbf{critical} security gaps related to identity and access management. Specifically, the lack of Multi-Factor Authentication (MFA) for email, computer logins, and access to sensitive data systems presents a significant and immediate risk of account compromise and unauthorized access. Additionally, a high-risk gap was identified in the employee onboarding process, as new hires do not receive security awareness training, leaving the organization vulnerable to social engineering and policy violations from the outset.

The external network scan conducted on the target IP address \texttt{[Target IP]} did not identify any open ports or exposed services. While this indicates a strong network perimeter at the time of the scan, it does not mitigate the severe internal and identity-related risks.

Immediate remediation should focus on the enterprise-wide implementation of MFA and the integration of mandatory security training into the employee onboarding process.

% --- Section 2: Organizational Information ---
\section{Organizational Information}

The following information was provided for the assessment.

\begin{tabular}{@{}ll}
\toprule
\textbf{Attribute} & \textbf{Value} \\
\midrule
Organization Name & \textbf{Solaris Energy} \\
Email Domain & \texttt{SolarisEnergy.net} \\
Website Domain & \url{www.SolarisEnergy.net} \\
External IP Address & \texttt{171.201.97.95} \\
\bottomrule
\end{tabular}

% --- Section 3: Security Control Review ---
\section{Security Control Review}

The following table summarizes the organization's responses to a security controls questionnaire. A checkmark (\ding{51}) indicates a positive control is in place, while a cross (\ding{55}) indicates a security gap.

\begin{table}[h!]
\centering
\begin{tabular}{@{}p{0.8\linewidth}c}
\toprule
\textbf{Control Question} & \textbf{Response} \\
\midrule
Do you require MFA to access email? & \ding{55} \\
Do you require MFA to log into computers? & \ding{55} \\
Do you require MFA to access sensitive data systems? & \ding{55} \\
Does your organization have an employee acceptable use policy? & \ding{51} \\
Does your organization do security awareness training for new employees? & \ding{55} \\
Does your organization do security awareness training for all employees at least once per year? & \ding{51} \\
\bottomrule
\end{tabular}
\caption{Security Controls Questionnaire Results.}
\end{table}

The responses highlight critical deficiencies in access control measures and significant weaknesses in the security training lifecycle.

% --- Section 4: Technical Scan Results ---
\section{Technical Scan Results}

An external network scan was performed to identify potential vulnerabilities on the public-facing infrastructure.

\begin{itemize}
    \item \textbf{Target IP Address:} \texttt{[Target IP]}
    \item \textbf{Scan Date:} The scan date was not provided in the input data.
    \item \textbf{Summary of Findings:} The scan completed successfully and \textbf{did not identify any open TCP/UDP ports or exposed services}. This suggests a well-configured firewall or network access control list (ACL) is in place for the scanned asset, which is a positive security control.
\end{itemize}

% --- Section 5: Risk Assessment ---
\section{Risk Assessment}

This section correlates the findings from the security control review, technical scan, and pre-existing risk data. The primary risks identified stem from policy and configuration gaps rather than externally exposed vulnerabilities. No pre-existing vulnerabilities were provided for this assessment.

\begin{table}[h!]
\centering
\begin{tabular}{@{}p{0.1\linewidth}p{0.35\linewidth}p{0.15\linewidth}p{0.3\linewidth}@{}}
\toprule
\textbf{ID} & \textbf{Risk Name} & \textbf{Severity} & \textbf{Details} \\
\midrule
RISK-001 & Lack of MFA for Email Access & \textbf{Critical} & Without MFA, email accounts are susceptible to takeover via phishing or credential stuffing, leading to data breaches and further internal compromise. \\
\addlinespace
RISK-002 & Lack of MFA for Endpoint Login & \textbf{Critical} & Compromised credentials could allow an attacker to gain direct access to employee workstations, corporate data, and the internal network. \\
\addlinespace
RISK-003 & Lack of MFA for Sensitive Systems & \textbf{Critical} & Failure to protect sensitive data systems with MFA removes a critical defense layer, making high-value data highly vulnerable to unauthorized access. \\
\addlinespace
RISK-004 & No Security Training for New Hires & \textbf{High} & New employees are not equipped to identify and respond to security threats, increasing the likelihood of successful phishing attacks and inadvertent policy violations. \\
\bottomrule
\end{tabular}
\caption{Summary of Identified Risks.}
\end{table}

% --- Section 6: Recommendations ---
\section{Recommendations}

Based on the identified risks, the following prioritized actions are recommended to improve the cybersecurity posture of \textbf{Solaris Energy}.

\begin{enumerate}
    \item \textbf{[Critical] Implement Multi-Factor Authentication (MFA) Enterprise-Wide:}
    \begin{itemize}
        \item \textbf{Priority 1 (Immediate):} Enforce MFA for all user access to the email system (e.g., Microsoft 365, Google Workspace). This is the most common vector for initial compromise.
        \item \textbf{Priority 2 (Near-Term):} Deploy MFA for all remote access solutions (VPN) and logins to company-managed computers (laptops and desktops).
        \item \textbf{Priority 3 (Near-Term):} Require MFA for all applications and systems that process or store sensitive or critical business data.
    \end{itemize}

    \item \textbf{[High] Integrate Security Training into Employee Onboarding:}
    \begin{itemize}
        \item Develop or procure a foundational security awareness training module.
        \item Make completion of this training a mandatory part of the formal onboarding process for all new employees and contractors before they are granted full access to corporate systems.
        \item The training should cover, at a minimum: phishing awareness, password hygiene, acceptable use of company assets, and how to report a security incident.
    \end{itemize}
    
    \item \textbf{[Informational] Maintain Strong Network Perimeter Security:}
    \begin{itemize}
        \item The lack of findings in the external scan is a positive result. Continue to operate on a principle of least privilege for all external-facing firewalls and network devices, ensuring that only necessary ports and services are exposed to the internet.
    \end{itemize}
\end{enumerate}

\end{document}
```