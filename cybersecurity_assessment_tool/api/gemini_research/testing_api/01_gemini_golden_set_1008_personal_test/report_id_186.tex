```latex
\documentclass[12pt]{article}

% Preamble: Required Packages
\usepackage[margin=1in]{geometry}
\usepackage{pifont} % For checkmarks and crosses
\usepackage{booktabs} % For professional tables
\usepackage{hyperref} % For clickable links and better PDF navigation
\usepackage{url} % For formatting URLs
\usepackage{seqsplit} % To split long strings in tt font
\usepackage{graphicx}
\usepackage{xcolor}
\usepackage{fancyhdr}

% --- Document Setup ---
\hypersetup{
    colorlinks=true,
    linkcolor=blue,
    filecolor=magenta,      
    urlcolor=cyan,
    pdftitle={Cybersecurity Assessment Report},
    pdfpagemode=FullScreen,
}

% --- Custom Commands & Settings ---
\pagestyle{fancy}
\fancyhf{}
\fancyhead[L]{\textbf{Cybersecurity Assessment Report}}
\fancyhead[R]{\textbf{Moxie Marketing}}
\fancyfoot[C]{\thepage}

\newcommand{\yes}{\ding{51}} % Green checkmark
\newcommand{\no}{\ding{55}}  % Red X

\begin{document}

% --- Title Page ---
\begin{titlepage}
    \centering
    \vspace*{1cm}
    \Huge\textbf{Cybersecurity Assessment Report}
    \vspace{0.5cm}
    \LARGE Prepared for:
    \vspace{0.5cm}
    \Large\textbf{Moxie Marketing}
    
    \vspace{1.5cm}
    
    \includegraphics[width=0.4\textwidth]{example-image-a} % Placeholder for company logo
    
    \vfill
    
    \Large
    Report Date: \today \\
    Analysis Period: \today
    
\end{titlepage}

\tableofcontents
\newpage

% --- Section 1: Executive Summary ---
\section{Executive Summary}
This report provides a comprehensive cybersecurity assessment for \textbf{Moxie Marketing}, synthesizing findings from a network vulnerability scan, a review of existing risks, and an analysis of organizational security controls.

The assessment identified a mixed security posture. On the positive side, the organization has implemented strong Multi-Factor Authentication (MFA) controls across email, computer logins, and sensitive data systems. Furthermore, a technical scan of the target host \texttt{192.168.0.5} did not reveal any open ports or immediate vulnerabilities, directly contradicting a previously documented risk concerning an unencrypted web server. This suggests the prior risk has been successfully remediated.

However, the analysis revealed two \textbf{critical gaps} in administrative and procedural controls. The absence of a formal Employee Acceptable Use Policy and the lack of mandatory security awareness training for new hires represent significant organizational risks. These gaps expose the organization to potential insider threats, social engineering attacks, and non-compliance issues.

Our primary recommendations focus on immediately addressing these policy and training deficiencies to build a more resilient security culture and formalize security expectations for all employees.

% --- Section 2: Organizational Information ---
\section{Organizational Information}
The following details were provided for the assessment.
\begin{center}
\begin{tabular}{ll}
\toprule
\textbf{Attribute} & \textbf{Value} \\
\midrule
Organization Name & \textbf{Moxie Marketing} \\
Email Domain & \texttt{MoxieMarketing.org} \\
Website Domain & \url{www.MoxieMarketing.org} \\
External IP Address & \texttt{108.46.233.178} \\
\bottomrule
\end{tabular}
\end{center}

% --- Section 3: Security Control Review ---
\section{Security Control Review}
The following table summarizes the organization's responses to a security controls questionnaire. Each response is assessed against industry best practices. "No" answers indicate significant control gaps that require immediate attention.

\begin{center}
\begin{tabular}{p{0.6\linewidth} c p{0.2\linewidth}}
\toprule
\textbf{Control Question} & \textbf{Response} & \textbf{Assessment} \\
\midrule
Do you require MFA to access email? & \yes & Strong Control \\
Do you require MFA to log into computers? & \yes & Strong Control \\
Do you require MFA to access sensitive data systems? & \yes & Strong Control \\
Does your organization have an employee acceptable use policy? & \no & \textbf{Critical Gap} \\
Does your organization do security awareness training for new employees? & \no & \textbf{Critical Gap} \\
Does your organization do security awareness training for all employees at least once per year? & \yes & Good Practice \\
\bottomrule
\end{tabular}
\end{center}

% --- Section 4: Technical Scan Results ---
\section{Technical Scan Results}
An external network scan was performed to identify potential vulnerabilities on publicly exposed services.

\begin{itemize}
    \item \textbf{Target IP Address:} \texttt{192.168.0.5}
    \item \textbf{Scan Date:} \today
\end{itemize}

\subsection{Scan Summary}
The scan of the target host revealed no open ports. This indicates a strong network perimeter configuration for this specific asset, as it does not expose any services to the network segment from which it was scanned. The finding directly contradicts a previously identified risk (see Section 5), suggesting that the risk has been remediated.

\begin{center}
\begin{tabular}{llll}
\toprule
\textbf{Port} & \textbf{State} & \textbf{Service} & \textbf{Version} \\
\midrule
80/tcp & closed & http & N/A \\
\bottomrule
\end{tabular}
\end{center}

% --- Section 5: Risk Assessment ---
\section{Risk Assessment}
This section correlates findings from the security control review, the technical scan, and pre-existing risk documentation. The primary risks facing \textbf{Moxie Marketing} are currently administrative rather than technical.

\begin{center}
\begin{tabular}{p{0.2\linewidth} p{0.4\linewidth} p{0.15\linewidth} p{0.15\linewidth}}
\toprule
\textbf{Risk Title} & \textbf{Description} & \textbf{Severity} & \textbf{Status / Comment} \\
\midrule
\textbf{Lack of Acceptable Use Policy} & No formal policy exists to govern the use of company assets, data, and networks by employees. This increases the risk of misuse and data leakage. & \textbf{High} & \textbf{Active Risk}. Identified via questionnaire. \\
\hline
\textbf{No New-Hire Security Training} & New employees are not provided with security awareness training upon joining, making them highly susceptible to phishing and social engineering attacks. & \textbf{High} & \textbf{Active Risk}. Identified via questionnaire. \\
\hline
\textbf{Unencrypted Web Server} & Existing risk documentation stated that Port 80 was open, exposing an unencrypted web service. & Medium & \textbf{Not Validated}. The current scan found Port 80 to be closed. This risk should be formally re-evaluated and likely closed. \\
\bottomrule
\end{tabular}
\end{center}

% --- Section 6: Recommendations ---
\section{Recommendations}
The following actionable recommendations are prioritized based on the risk and potential impact to the organization.

\subsection{Priority 1: High-Impact Recommendations}
\begin{enumerate}
    \item \textbf{Develop and Implement an Acceptable Use Policy (AUP):}
    \begin{itemize}
        \item \textbf{Action:} Draft a formal AUP that clearly defines the rules and expectations for employees when using company computers, networks, email, and data. The policy should cover topics such as data handling, password security, internet usage, and incident reporting.
        \item \textbf{Justification:} An AUP is a foundational administrative control that reduces insider risk, sets clear security expectations, and provides a basis for disciplinary action in case of policy violations.
    \end{itemize}
    
    \item \textbf{Establish a Mandatory New-Hire Security Training Program:}
    \begin{itemize}
        \item \textbf{Action:} Create a security awareness training module that is a mandatory part of the onboarding process for all new employees and contractors. This training should cover phishing, social engineering, password hygiene, and the new AUP.
        \item \textbf{Justification:} New hires are a primary target for attackers. Immediate training significantly reduces the organization's vulnerability to common cyber threats from day one of a new employee's tenure.
    \end{itemize}
\end{enumerate}

\subsection{Priority 2: Administrative and Validation Tasks}
\begin{enumerate}
    \setcounter{enumi}{2} % Continue numbering
    \item \textbf{Validate and Update the Risk Register:}
    \begin{itemize}
        \item \textbf{Action:} Formally review the "Unencrypted Web Server" risk. Based on the technical scan results showing Port 80 as closed, this risk should be marked as remediated or closed in the official risk register.
        \item \textbf{Justification:} Maintaining an accurate risk register is crucial for focusing security efforts and resources on genuine, active threats.
    \end{itemize}
\end{enumerate}

\end{document}
```