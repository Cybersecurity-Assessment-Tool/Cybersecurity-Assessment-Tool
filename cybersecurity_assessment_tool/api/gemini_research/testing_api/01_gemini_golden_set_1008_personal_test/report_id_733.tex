```latex
\documentclass[12pt]{article}

% Preamble: Required Packages
\usepackage[margin=1in]{geometry}
\usepackage{pifont} % For checkmarks and crosses
\usepackage{booktabs} % For professional tables
\usepackage{hyperref} % For clickable links
\usepackage{url} % For URL formatting
\usepackage{seqsplit} % For splitting long strings in tt font
\usepackage{graphicx}
\usepackage{xcolor}

% Document Metadata
\title{Cybersecurity Posture Assessment Report}
\author{Cybersecurity Analyst}
\date{November 22, 2025}

% Hyperref Setup
\hypersetup{
    colorlinks=true,
    linkcolor=blue,
    filecolor=magenta,      
    urlcolor=cyan,
    pdftitle={Cybersecurity Posture Assessment Report},
    pdfpagemode=FullScreen,
}

\begin{document}

\maketitle
\thispagestyle{empty}
\newpage

\tableofcontents
\newpage

% --- 1. Executive Summary ---
\section{Executive Summary}

This report details the findings of a cybersecurity assessment conducted for \textbf{Astraeus Aerospace} on November 22, 2025. The assessment combined a review of organizational security controls, an external network scan, and an analysis of pre-existing risks.

The overall security posture is determined to be critically weak, presenting a high likelihood of compromise. The primary areas of concern are systemic failures in identity and access management, coupled with a vulnerable internet-facing system.

\textbf{Key Findings:}
\begin{itemize}
    \item \textbf{Critical Control Gaps:} The organization has not implemented Multi-Factor Authentication (MFA) for email, computer logins, or access to sensitive data. This represents a severe deficiency that significantly increases the risk of account takeover and unauthorized access.
    \item \textbf{Vulnerable External Service:} The external web server is running an outdated and vulnerable version of Nginx (1.18.0). This service is a prime target for automated attacks and could serve as an initial entry point into the network.
    \item \textbf{Policy and Training Deficiencies:} The absence of an Acceptable Use Policy and a mandatory annual security awareness training program indicates a lack of foundational cybersecurity governance. This creates an environment where employees are more susceptible to social engineering attacks.
\end{itemize}

Immediate remediation of these issues is strongly recommended to reduce the organization's exposure to common and impactful cyber threats.

% --- 2. Organizational Information ---
\section{Organizational Information}

The following information was provided for the assessment.

\begin{tabular}{@{}ll}
\toprule
\textbf{Attribute} & \textbf{Value} \\
\midrule
Organization Name & \textbf{Astraeus Aerospace} \\
Email Domain & \texttt{AstraeusAerospace.com} \\
Website Domain & \seqsplit{\texttt{www.AstraeusAerospace.com}} \\
External IP Address & \texttt{224.160.119.253} \\
\bottomrule
\end{tabular}

% --- 3. Security Control Review ---
\section{Security Control Review}

A review of organizational security controls was conducted via a questionnaire. The responses reveal significant gaps in fundamental security practices. A "No" response indicates a missing control and a potential risk.

\begin{table}[h!]
\centering
\caption{Organizational Security Controls Questionnaire}
\begin{tabular}{@{}p{0.6\linewidth} c l}
\toprule
\textbf{Control Question} & \textbf{Response} & \textbf{Assessment} \\
\midrule
Do you require MFA to access email? & \ding{55} & \textcolor{red}{\textbf{Critical Gap}} \\
Do you require MFA to log into computers? & \ding{55} & \textcolor{red}{High Risk} \\
Do you require MFA to access sensitive data systems? & \ding{55} & \textcolor{red}{\textbf{Critical Gap}} \\
Does your organization have an employee acceptable use policy? & \ding{55} & \textcolor{red}{High Risk} \\
Does your organization do security awareness training for new employees? & \ding{51} & Meets Baseline \\
Does your organization do security awareness training for all employees at least once per year? & \ding{55} & \textcolor{red}{High Risk} \\
\bottomrule
\end{tabular}
\end{table}

% --- 4. Technical Scan Results ---
\section{Technical Scan Results}

An Nmap scan was performed to identify open ports and services on the organization's perimeter.

\begin{itemize}
    \item \textbf{Target IP:} \texttt{192.168.10.5}
    \item \textbf{Scan Date:} 2025-11-22T10:00:00Z
\end{itemize}

\begin{table}[h!]
\centering
\caption{Open Ports and Services}
\begin{tabular}{@{}lllll}
\toprule
\textbf{Port} & \textbf{State} & \textbf{Service} & \textbf{Product} & \textbf{Version} \\
\midrule
443/tcp & open & https & nginx & 1.18.0 \\
\bottomrule
\end{tabular}
\end{table}

\subsection{Analysis of Technical Findings}
The scan identified a single open port, 443 (HTTPS), running an Nginx web server. The detected version, \textbf{1.18.0}, was released in April 2020 and is now significantly outdated. This version is known to be vulnerable to multiple Common Vulnerabilities and Exposures (CVEs), including but not limited to request smuggling and denial-of-service attacks (e.g., CVE-2021-23017). Running unsupported and unpatched software on an internet-facing system constitutes a high-risk vulnerability that could be exploited by attackers to gain initial access to the network.

% --- 5. Risk Assessment Summary ---
\section{Risk Assessment Summary}

The following table synthesizes the findings from the control review and technical scan into a prioritized list of risks. No pre-existing vulnerabilities were reported.

\begin{table}[h!]
\centering
\caption{Identified Risks}
\begin{tabular}{@{}p{0.1\linewidth} p{0.3\linewidth} p{0.4\linewidth} l}
\toprule
\textbf{Risk ID} & \textbf{Risk Name} & \textbf{Description} & \textbf{Severity} \\
\midrule
RISK-001 & Lack of Multi-Factor Authentication (MFA) & The absence of MFA for email, endpoints, and sensitive systems makes the organization highly susceptible to account compromise via credential theft or phishing. & \textbf{Critical} \\
\addlinespace
RISK-002 & Outdated Web Server Software & The public-facing Nginx server is running an outdated version with known vulnerabilities, creating a direct vector for external compromise. & High \\
\addlinespace
RISK-003 & Deficient Security Policies and Training & The lack of an Acceptable Use Policy and recurring security training weakens the human firewall, increasing susceptibility to social engineering. & High \\
\bottomrule
\end{tabular}
\end{table}

% --- 6. Recommendations ---
\section{Recommendations}

To mitigate the identified risks and improve the overall security posture, the following actions are recommended with urgency.

\begin{enumerate}
    \item \textbf{Implement Mandatory MFA (RISK-001):}
    \begin{itemize}
        \item \textbf{Action:} Deploy a robust MFA solution across the organization.
        \item \textbf{Priority:}
            \begin{enumerate}
                \item Immediately enforce MFA for all email accounts (cloud and on-premise).
                \item Enforce MFA for all remote access (VPN) and administrative accounts.
                \item Phase in MFA for all computer logins and access to systems containing sensitive data.
            \end{enumerate}
    \end{itemize}
    \item \textbf{Remediate Vulnerable Web Server (RISK-002):}
    \begin{itemize}
        \item \textbf{Action:} Upgrade the Nginx server on host \texttt{192.168.10.5} to the latest stable version immediately.
        \item \textbf{Process Improvement:} Establish a formal patch management program for all internet-facing systems. This program should include regular vulnerability scanning and a defined schedule for applying security patches.
    \end{itemize}
    \item \textbf{Establish Security Governance (RISK-003):}
    \begin{itemize}
        \item \textbf{Action:} Develop and ratify a formal Acceptable Use Policy (AUP) that all employees must read and acknowledge.
        \item \textbf{Training:} Implement a mandatory, annual security awareness training program for all employees. This training should cover topics such as phishing, password hygiene, and social engineering.
    \end{itemize}
\end{enumerate}

\end{document}
```