```latex
\documentclass[12pt]{article}

% --- PACKAGES ---
\usepackage[margin=1in]{geometry}
\usepackage{pifont} % For checkmarks and crosses
\usepackage{booktabs} % For professional tables
\usepackage{hyperref} % For clickable links and TOC
\usepackage{url} % For formatting URLs
\usepackage{seqsplit} % For splitting long strings without spaces
\usepackage{graphicx} % For logo (placeholder)
\usepackage{xcolor} % For colors in text

% --- DOCUMENT METADATA ---
\hypersetup{
    colorlinks=true,
    linkcolor=blue,
    filecolor=magenta,      
    urlcolor=cyan,
    pdftitle={Cybersecurity Posture Report},
    pdfauthor={Cybersecurity Analysis Cell},
    pdfsubject={Security Assessment},
    pdfkeywords={Cybersecurity, Risk, Assessment},
    bookmarks=true
}

% --- COMMANDS ---
\newcommand{\yes}{\ding{51}}
\newcommand{\no}{\ding{55}}

% --- TITLE ---
\title{
    \vspace{-1.5cm}
    \includegraphics[width=0.4\textwidth]{example-image-a} \\ % Placeholder for client logo
    \vspace{1cm}
    \textbf{Cybersecurity Posture Report} \\
    \large Prepared for: Quantum Reach
}
\author{Cybersecurity Analysis Cell}
\date{November 22, 2025}

\begin{document}

\maketitle
\thispagestyle{empty}
\newpage

\tableofcontents
\newpage

% --- EXECUTIVE SUMMARY ---
\section{Executive Summary}

This report provides a comprehensive analysis of the cybersecurity posture of \textbf{Quantum Reach}, based on data gathered on November 22, 2025. The assessment combines a review of organizational security controls, an external network scan, and an evaluation of known risks.

The overall security posture is assessed as \textbf{Moderate}. The organization demonstrates a solid foundation in security awareness training and policy. However, two significant risks were identified that require immediate attention to prevent potential security breaches:

\begin{enumerate}
    \item \textbf{Critical Policy Gap:} The absence of Multi-Factor Authentication (MFA) for computer logins presents a high risk of unauthorized access through compromised credentials.
    \item \textbf{Vulnerable External Service:} The public-facing web server is running an outdated version of Nginx (1.18.0), which contains multiple known vulnerabilities. This exposes the organization to external attack and potential system compromise.
\end{enumerate}

This report details these findings and provides actionable recommendations to mitigate the identified risks and enhance the organization's overall defensive capabilities.

% --- ORGANIZATIONAL INFORMATION ---
\section{Organizational Information}

The following details were provided for the assessment.

\begin{tabular}{@{}ll}
    \toprule
    \textbf{Attribute} & \textbf{Value} \\
    \midrule
    Organization Name & Quantum Reach \\
    Email Domain & \texttt{QuantumReach.org} \\
    Website Domain & \url{www.QuantumReach.org} \\
    External IP Address & \texttt{199.11.177.102} \\
    \bottomrule
\end{tabular}

% --- SECURITY CONTROL REVIEW ---
\section{Security Control Review}

A review of the organization's security controls was conducted via a standardized questionnaire. The results indicate a strong commitment to security policies and training, but highlight a critical gap in endpoint access controls.

\begin{table}[h!]
    \centering
    \caption{Security Questionnaire Results}
    \begin{tabular}{@{}p{0.8\linewidth}c@{}}
        \toprule
        \textbf{Control Question} & \textbf{Status} \\
        \midrule
        Do you require MFA to access email? & \yes \\
        Do you require MFA to log into computers? & \textcolor{red}{\no} \\
        Do you require MFA to access sensitive data systems? & \yes \\
        Does your organization have an employee acceptable use policy? & \yes \\
        Does your organization do security awareness training for new employees? & \yes \\
        Does your organization do security awareness training for all employees at least once per year? & \yes \\
        \bottomrule
    \end{tabular}
\end{table}

\paragraph{Analysis:} The lack of MFA for computer logins is a significant weakness. If an employee's password is stolen, an attacker could gain direct access to their workstation and, subsequently, the internal network. This gap negates some of the protection offered by MFA on other systems, as an attacker on a trusted device may be able to bypass further controls.

% --- TECHNICAL SCAN RESULTS ---
\section{Technical Scan Results}

An external network scan was performed against the target host \texttt{192.168.10.5} to identify open ports and exposed services.

\begin{itemize}
    \item \textbf{Scan Target:} \texttt{192.168.10.5}
    \item \textbf{Scan Date:} 2025-11-22
\end{itemize}

\begin{table}[h!]
    \centering
    \caption{Open Ports and Services}
    \begin{tabular}{@{}llllll@{}}
        \toprule
        \textbf{Port} & \textbf{Protocol} & \textbf{State} & \textbf{Service} & \textbf{Product} & \textbf{Version} \\
        \midrule
        443 & TCP & open & https & nginx & 1.18.0 \\
        \bottomrule
    \end{tabular}
\end{table}

\paragraph{Analysis:} The scan identified a single open port, 443/TCP, which is serving HTTPS traffic via an \textbf{Nginx 1.18.0} web server. This version was released in April 2020 and is now considered outdated. It is known to be vulnerable to several security issues, including Request Smuggling (CVE-2023-44487) and a DNS resolver vulnerability (CVE-2021-23017), which could lead to information disclosure or denial of service. Running outdated software on public-facing systems is a critical risk.

% --- RISK ASSESSMENT ---
\section{Risk Assessment Summary}

The following table synthesizes findings from the security control review and the technical scan. No pre-existing vulnerabilities were reported.

\begin{table}[h!]
    \centering
    \caption{Identified Risks}
    \begin{tabular}{@{}p{0.1\linewidth}p{0.6\linewidth}l@{}}
        \toprule
        \textbf{Risk ID} & \textbf{Description} & \textbf{Severity} \\
        \midrule
        QR-RISK-001 & \textbf{Lack of MFA on Workstations:} User workstations do not require MFA for login, increasing the risk of unauthorized access via compromised credentials. & \textbf{\textcolor{red}{High}} \\
        \addlinespace
        QR-RISK-002 & \textbf{Outdated Web Server Software:} The public-facing Nginx server (v. 1.18.0) on host \texttt{192.168.10.5} is outdated and has known vulnerabilities. & \textbf{\textcolor{red}{High}} \\
        \bottomrule
    \end{tabular}
\end{table}

% --- RECOMMENDATIONS ---
\section{Recommendations}

To address the identified risks, the following actions are recommended for immediate implementation.

\begin{enumerate}
    \item \textbf{Implement MFA for Workstation Logins (Risk ID: QR-RISK-001):}
    \begin{itemize}
        \item \textbf{Action:} Deploy a mandatory Multi-Factor Authentication solution for all employee computer and server logins.
        \item \textbf{Justification:} This will create a critical layer of defense against credential theft, ensuring that a compromised password alone is not sufficient to grant an attacker access to a trusted endpoint on the internal network.
        \item \textbf{Examples:} Windows Hello for Business, Duo Security, or other FIDO2/WebAuthn compliant solutions.
    \end{itemize}
    \vspace{0.5cm}
    \item \textbf{Upgrade Nginx Web Server (Risk ID: QR-RISK-002):}
    \begin{itemize}
        \item \textbf{Action:} Plan and execute an upgrade of the Nginx server on host \texttt{192.168.10.5} from version 1.18.0 to the latest stable release.
        \item \textbf{Justification:} Upgrading will patch known vulnerabilities, protecting the organization's public-facing website and backend infrastructure from potential compromise.
        \item \textbf{Procedure:} A backup of the current configuration and site data must be performed before the upgrade. The new version should be tested in a staging environment to ensure compatibility with the web application.
    \end{itemize}
\end{enumerate}

\end{document}
```