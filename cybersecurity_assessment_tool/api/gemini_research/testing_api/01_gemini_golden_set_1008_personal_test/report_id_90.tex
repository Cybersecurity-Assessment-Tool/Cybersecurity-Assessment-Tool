```latex
\documentclass[12pt]{article}

% Preamble: Required Packages
\usepackage[margin=1in]{geometry}
\usepackage{pifont} % For checkmarks and crosses
\usepackage{booktabs} % For professional-looking tables
\usepackage{hyperref} % For clickable links and references
\usepackage{url} % For formatting URLs
\usepackage{seqsplit} % To split long monospaced strings
\usepackage{xcolor} % For colors

% Document Information
\title{Cybersecurity Posture Assessment Report}
\author{Cybersecurity Analyst}
\date{\today}

% Hyperref Setup
\hypersetup{
    colorlinks=true,
    linkcolor=blue,
    filecolor=magenta,      
    urlcolor=cyan,
    pdftitle={Cybersecurity Posture Assessment Report},
    pdfpagemode=FullScreen,
}

\begin{document}

\maketitle

\section{Executive Summary}
This report provides a comprehensive analysis of the cybersecurity posture for \textbf{New Era}. The assessment is based on a correlation of network scan data, a security controls questionnaire, and a review of pre-existing risks.

The analysis reveals a mixed security posture. Positive controls, such as the implementation of Multi-Factor Authentication (MFA) for email access, are in place. However, several critical gaps were identified that significantly increase the organization's risk profile. These include the absence of MFA for computer and sensitive data system access, the lack of a formal employee acceptable use policy, and an incomplete security awareness training program.

A technical network scan of the target IP \texttt{192.168.0.5} did not identify any open ports. This finding conflicts with a pre-existing documented risk concerning an unencrypted web server on Port 80. This discrepancy requires further investigation to validate the current status of external services.

Immediate remediation efforts should focus on expanding MFA coverage, formalizing security policies, and establishing a recurring security awareness training program for all employees.

\section{Organizational Information}
The following details were provided for the assessment.

\begin{tabular}{@{}ll}
\toprule
\textbf{Attribute} & \textbf{Value} \\
\midrule
Organization Name & New Era \\
Email Domain & \texttt{NewEra.org} \\
Website Domain & \seqsplit{\url{www.NewEra.org}} \\
External IP & \texttt{26.95.208.80} \\
\bottomrule
\end{tabular}

\section{Security Control Review}
The following table summarizes the organization's responses to the security controls questionnaire. Items marked with \ding{55} represent significant gaps in the current security framework and are addressed in the Risk Assessment section.

\begin{table}[h!]
\centering
\begin{tabular}{@{}lcc}
\toprule
\textbf{Control Question} & \textbf{Response} & \textbf{Assessment} \\
\midrule
Do you require MFA to access email? & \ding{51} & Best Practice Met \\
Do you require MFA to log into computers? & \ding{55} & \textcolor{red}{Critical Gap} \\
Do you require MFA to access sensitive data systems? & \ding{55} & \textcolor{red}{Critical Gap} \\
Does your organization have an employee acceptable use policy? & \ding{55} & \textcolor{orange}{High Risk} \\
Does your organization do security awareness training for new employees? & \ding{51} & Good Practice \\
Does your organization do security awareness training for all employees annually? & \ding{55} & \textcolor{orange}{High Risk} \\
\bottomrule
\end{tabular}
\caption{Security Controls Questionnaire Analysis}
\end{table}

\section{Technical Scan Results}
A network scan was performed to identify exposed services on the specified target.

\begin{itemize}
    \item \textbf{Target IP:} \texttt{192.168.0.5}
    \item \textbf{Scan Date:} \today
    \item \textbf{Scanner:} Nmap
\end{itemize}

The scan reported the target host as "up" but found no open ports. The status of scanned ports is detailed below.

\begin{table}[h!]
\centering
\begin{tabular}{@{}ll}
\toprule
\textbf{Port / Protocol} & \textbf{State} \\
\midrule
80/tcp & closed \\
\bottomrule
\end{tabular}
\caption{Nmap Scan Results for \texttt{192.168.0.5}}
\end{table}

\textbf{Note:} This result conflicts with the pre-existing risk data (Input 3), which states that Port 80 is open. This may indicate that the risk has been remediated, the scan targeted an incorrect or internal asset, or the service is protected by a firewall that blocked the scan. This discrepancy warrants further investigation.

\section{Consolidated Risk Assessment}
The following table synthesizes findings from the security control review, technical scan, and pre-existing risk data into a prioritized list of security risks.

\begin{table}[h!]
\centering
\begin{tabular}{@{}p{0.3\linewidth}p{0.5\linewidth}l}
\toprule
\textbf{Risk Name} & \textbf{Overview} & \textbf{Severity} \\
\midrule
\textbf{No MFA on Sensitive Systems} & The lack of MFA on systems containing sensitive data exposes the organization's most critical assets to compromise via stolen credentials. & \textcolor{red}{Critical} \\
\addlinespace
\textbf{No MFA on Endpoints} & User computers (endpoints) are not protected by MFA. A compromised password could grant an attacker direct access to a corporate device and internal network resources. & \textcolor{red}{Critical} \\
\addlinespace
\textbf{Lack of Acceptable Use Policy} & Without a formal policy, there is no enforceable standard for employee behavior, creating legal and security ambiguities regarding data handling and system usage. & \textcolor{orange}{High} \\
\addlinespace
\textbf{Inadequate Security Training} & Security awareness is not refreshed annually. This increases the likelihood that employees will fall victim to phishing, social engineering, and other common attacks. & \textcolor{orange}{High} \\
\addlinespace
\textbf{Unconfirmed Web Server Risk} & A pre-existing risk indicates an unencrypted web server on Port 80. While the current scan did not confirm this, the risk remains documented and must be validated across all external assets. & Medium \\
\bottomrule
\end{tabular}
\caption{Summary of Identified Risks}
\end{table}

\section{Recommendations}
The following actions are recommended to mitigate the identified risks and improve the overall security posture of \textbf{New Era}.

\begin{enumerate}
    \item \textbf{[Critical] Implement MFA for Sensitive Systems:} Immediately deploy a robust MFA solution for all applications and systems that store, process, or transmit sensitive organizational or customer data.
    
    \item \textbf{[Critical] Enforce MFA for Endpoint Logon:} Require MFA for all user logins to company-owned computers (desktops and laptops). This severely limits an attacker's ability to move laterally after compromising a user's password.
    
    \item \textbf{[High] Develop and Implement an Acceptable Use Policy (AUP):} Create a formal AUP that clearly defines the rules and expectations for using company assets, data, and networks. Ensure all employees read and acknowledge the policy.
    
    \item \textbf{[High] Establish an Annual Security Awareness Program:} Implement a mandatory, annual security awareness training program for all employees. The training should cover current threats such as phishing, ransomware, and proper data handling.
    
    \item \textbf{[Medium] Investigate and Validate Port 80 Status:} Conduct a comprehensive external vulnerability scan across all public IP addresses associated with \textbf{New Era} to confirm the status of Port 80. If an unencrypted web service is found, either decommission it or secure it with TLS/SSL (HTTPS).
\end{enumerate}

\end{document}
```