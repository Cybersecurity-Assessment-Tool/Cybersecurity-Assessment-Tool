```latex
\documentclass[12pt]{article}

% Preamble: Required Packages
\usepackage[a4paper, margin=1in]{geometry}
\usepackage{pifont} % For checkmarks and crosses
\usepackage{booktabs} % For professional tables
\usepackage{hyperref} % For clickable links
\usepackage{url}      % For URL formatting
\usepackage{seqsplit} % For splitting long strings without spaces
\usepackage{graphicx} % For potential logos
\usepackage{fancyhdr} % For headers/footers
\usepackage{xcolor}   % For colors

% --- Document Metadata ---
\title{Cybersecurity Posture Assessment Report}
\author{Cybersecurity Analysis Division}
\date{\today}

% --- Header and Footer Configuration ---
\pagestyle{fancy}
\fancyhf{} % Clear all header and footer fields
\fancyhead[L]{Cybersecurity Assessment Report}
\fancyhead[R]{\textbf{Nebula Creative}}
\fancyfoot[C]{\thepage}
\renewcommand{\headrulewidth}{0.4pt}
\renewcommand{\footrulewidth}{0.4pt}

% --- Hyperref Setup ---
\hypersetup{
    colorlinks=true,
    linkcolor=blue,
    filecolor=magenta,      
    urlcolor=cyan,
    pdftitle={Cybersecurity Posture Assessment Report},
    pdfpagemode=FullScreen,
}

\begin{document}

\maketitle
\thispagestyle{empty}
\newpage

\tableofcontents
\newpage

% --- Section 1: Executive Summary ---
\section{Executive Summary}

This report provides a comprehensive cybersecurity posture assessment for \textbf{Nebula Creative}. The analysis is based on a synthesis of technical network scanning, a review of organizational security controls via a questionnaire, and an evaluation of pre-existing risk data.

The assessment reveals a mixed security posture. On the technical front, the scanned host at \texttt{192.168.1.100} demonstrates a strong defensive configuration, with no open ports detected. This indicates a well-hardened perimeter for that specific asset, minimizing its external attack surface.

However, the organizational security control review identified two significant gaps that present a high level of risk to the organization:
\begin{itemize}
    \item \textbf{Critical Risk: Lack of Endpoint Multi-Factor Authentication (MFA).} The absence of mandatory MFA for computer logins exposes the organization to significant threats from credential theft and unauthorized access.
    * \textbf{High Risk: Inadequate Security Awareness Training.} The lack of annual, recurring security training for all employees increases susceptibility to phishing, social engineering, and other human-centric attacks.
\end{itemize}

While no pre-existing vulnerabilities were reported and the network scan was positive, these procedural and policy-based gaps must be addressed urgently. Recommendations are provided in this report to mitigate these risks and strengthen the overall security posture of \textbf{Nebula Creative}.

% --- Section 2: Organizational Information ---
\section{Organizational Information}

The following information was provided for the assessment.

\begin{table}[h!]
\centering
\begin{tabular}{@{}ll@{}}
\toprule
\textbf{Attribute} & \textbf{Value} \\ \midrule
Organization Name & \textbf{Nebula Creative} \\
Email Domain & \texttt{NebulaCreative.com} \\
Website Domain & \url{www.NebulaCreative.com} \\
External IP Address & \texttt{54.158.8.61} \\ \bottomrule
\end{tabular}
\caption{Client Organizational Details.}
\end{table}

% --- Section 3: Security Control Review ---
\section{Security Control Review}

A review of foundational security controls was conducted based on a questionnaire. The results highlight areas of strength and critical areas requiring improvement. A "No" response indicates a deviation from security best practices and represents a potential risk.

\begin{table}[h!]
\centering
\begin{tabular}{@{}p{0.7\linewidth}c@{}}
\toprule
\textbf{Control Question} & \textbf{Response} \\ \midrule
Do you require MFA to access email? & \textcolor{green}{\ding{51}} \\
Do you require MFA to log into computers? & \textcolor{red}{\ding{55}} \\
Do you require MFA to access sensitive data systems? & \textcolor{green}{\ding{51}} \\
Does your organization have an employee acceptable use policy? & \textcolor{green}{\ding{51}} \\
Does your organization do security awareness training for new employees? & \textcolor{green}{\ding{51}} \\
Does your organization do security awareness training for all employees at least once per year? & \textcolor{red}{\ding{55}} \\ \bottomrule
\end{tabular}
\caption{Security Controls Questionnaire Results (\ding{51} = Yes, \ding{55} = No).}
\end{table}

% --- Section 4: Technical Scan Results ---
\section{Technical Scan Results}

A network scan was performed to identify open ports and exposed services on the target system.

\begin{itemize}
    \item \textbf{Target IP Address:} \texttt{192.168.1.100}
    \item \textbf{Host Status:} Up
    \item \textbf{Scan Summary:} The scan confirmed that the host is online and responsive. However, no open TCP ports were discovered within the scanned range. All tested ports were found to be in a 'closed' state.
\end{itemize}

\textbf{Analysis:} This is a positive security finding. A host with no open ports presents a minimal attack surface to the network, suggesting it is either not running network-facing services or is protected by a well-configured firewall that denies all unsolicited incoming connections.

% --- Section 5: Risk Assessment ---
\section{Risk Assessment}

This section correlates the findings from the security control review and technical scans to identify and prioritize risks.

\begin{table}[h!]
\centering
\begin{tabular}{@{}p{0.2\linewidth}p{0.55\linewidth}p{0.15\linewidth}@{}}
\toprule
\textbf{Risk Name} & \textbf{Overview} & \textbf{Severity} \\ \midrule
\textbf{Lack of Endpoint MFA} & The absence of MFA on computer logins means that a single compromised password could grant an attacker full access to an employee's workstation and any local or network resources accessible from it. & \textbf{Critical} \\
\addlinespace
\textbf{Inadequate Security Awareness Training} & Without regular, annual training, employees' ability to recognize and appropriately respond to evolving threats like phishing and social engineering diminishes over time, making them the weakest link in the security chain. & \textbf{High} \\
\addlinespace
\textbf{No Pre-existing Risks Reported} & The provided data contained no previously identified vulnerabilities. The risks in this table are newly identified during this assessment. & Low \\
\bottomrule
\end{tabular}
\caption{Identified Risks and Severity.}
\end{table}

% --- Section 6: Recommendations ---
\section{Recommendations}

The following actions are recommended to mitigate the identified risks and improve the overall security posture.

\subsection{Endpoint Multi-Factor Authentication (Critical)}
\begin{itemize}
    \item \textbf{Action:} Implement and enforce a mandatory Multi-Factor Authentication (MFA) policy for all employee and contractor computer logins.
    \item \textbf{Justification:} This control acts as a critical barrier against unauthorized access resulting from stolen or weak credentials. Even if an attacker obtains a user's password, they will be unable to log in without the second factor (e.g., an authenticator app, hardware token, or biometric scan).
    \item \textbf{Tools:} Consider solutions integrated with your operating system (e.g., Windows Hello for Business) or third-party endpoint security solutions (e.g., Duo, Okta).
\end{itemize}

\subsection{Annual Security Awareness Training (High)}
\begin{itemize}
    \item \textbf{Action:} Establish a formal, mandatory security awareness training program that all employees must complete at least once per year.
    \item \textbf{Justification:} The threat landscape is constantly changing. Annual training ensures that all staff members are kept up-to-date on the latest threats, such as new phishing techniques, and are reminded of their responsibilities under the organization's security policies. This builds a strong human firewall.
    \item \textbf{Implementation:} The program should cover topics such as phishing identification, password hygiene, acceptable use, and incident reporting. Consider incorporating simulated phishing campaigns to test and reinforce the training.
\end{itemize}

% --- Section 7: Conclusion ---
\section{Conclusion}

The security assessment of \textbf{Nebula Creative} indicates a strong network perimeter for the scanned asset but reveals critical gaps in internal security controls. The lack of MFA on endpoints and the absence of annual security training for all staff are significant vulnerabilities that substantially increase organizational risk.

By implementing the recommendations outlined in this report, \textbf{Nebula Creative} can effectively mitigate these risks, reduce its attack surface, and foster a more resilient security culture. Prioritizing these initiatives is essential for protecting company data and systems against modern cyber threats.

\end{document}
```