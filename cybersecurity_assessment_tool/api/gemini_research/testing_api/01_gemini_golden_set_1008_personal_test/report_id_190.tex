```latex
\documentclass[12pt]{article}

% Preamble: Required Packages
\usepackage[margin=1in]{geometry}
\usepackage{pifont} % For checkmarks and crosses
\usepackage{booktabs} % For professional tables
\usepackage{hyperref} % For clickable links
\usepackage{url} % For URL formatting
\usepackage{seqsplit} % For splitting long strings without breaking
\usepackage{graphicx} % For potential logos
\usepackage{fancyhdr} % For headers/footers

% Document Metadata
\title{Cybersecurity Posture Assessment Report}
\author{Cybersecurity Analysis Division}
\date{\today}

% Header and Footer Configuration
\pagestyle{fancy}
\fancyhf{} % Clear all header and footer fields
\fancyhead[L]{Cybersecurity Assessment for \textbf{Copperhead Cables}}
\fancyfoot[C]{\thepage}
\renewcommand{\headrulewidth}{0.4pt}
\renewcommand{\footrulewidth}{0.4pt}

\begin{document}

\maketitle
\thispagestyle{empty}
\newpage

\tableofcontents
\newpage

\section{Executive Summary}

This report details the findings of a cybersecurity posture assessment conducted for \textbf{Copperhead Cables}. The assessment integrated an external network scan, a review of existing risks, and an analysis of organizational security controls based on a questionnaire.

The primary findings indicate significant gaps in foundational security controls. While the external network scan of the target IP address revealed no open ports—a positive indicator of a well-configured perimeter firewall—critical vulnerabilities exist within the organization's access control and employee security awareness policies.

The absence of Multi-Factor Authentication (MFA) for email and computer access represents a \textbf{Critical} risk, exposing the organization to credential theft, business email compromise, and unauthorized system access. Furthermore, the lack of a formal security awareness training program for new and existing employees constitutes a \textbf{High} risk, leaving the organization vulnerable to phishing and other social engineering attacks.

Immediate remediation efforts should focus on implementing MFA across all critical systems and establishing a comprehensive security awareness training program to mitigate these identified risks.

\section{Organizational Information}

The following details were provided for the assessment.
\begin{itemize}
    \item \textbf{Organization Name:} Copperhead Cables
    \item \textbf{Email Domain:} \texttt{CopperheadCables.org}
    \item \textbf{Website Domain:} \url{www.CopperheadCables.org}
    \item \textbf{Primary External IP:} \texttt{191.83.170.132}
\end{itemize}

\section{Security Control Review}

The following table summarizes the organization's responses to the security controls questionnaire. Items marked with \ding{55} (No) represent significant gaps in the security posture and are discussed in the Risk Assessment section.

\begin{table}[h!]
\centering
\caption{Security Controls Questionnaire Analysis}
\label{tab:controls}
\begin{tabular}{p{0.7\linewidth} c c}
\toprule
\textbf{Control Question} & \textbf{Response} & \textbf{Status} \\
\midrule
Do you require MFA to access email? & No & \ding{55} \\
Do you require MFA to log into computers? & No & \ding{55} \\
Do you require MFA to access sensitive data systems? & Yes & \ding{51} \\
Does your organization have an employee acceptable use policy? & Yes & \ding{51} \\
Does your organization do security awareness training for new employees? & No & \ding{55} \\
Does your organization do security awareness training for all employees at least once per year? & No & \ding{55} \\
\bottomrule
\end{tabular}
\end{table}

\subsection*{Analysis of Gaps}
The questionnaire reveals critical deficiencies in two key areas:
\begin{enumerate}
    \item \textbf{Access Control:} The failure to enforce MFA on email and computer logins is a severe weakness. Email is a primary target for attackers, and compromised accounts can lead to data breaches and financial fraud. Unprotected computer logins remove a crucial layer of defense against unauthorized access.
    \item \textbf{Security Awareness:} The complete absence of a security awareness training program means employees are likely unprepared to identify and respond to common threats like phishing, malware, and social engineering. This significantly increases the organization's "human" attack surface.
\end{enumerate}

\section{Technical Scan Results}

An external network scan was performed to identify exposed services and potential vulnerabilities.
\begin{itemize}
    \item \textbf{Target IP Address:} \texttt{[Target IP]}
    \item \textbf{Scan Date:} No date provided in scan data.
\end{itemize}

\subsection*{Findings}
The scan completed successfully and found \textbf{no open TCP or UDP ports} on the target system.

\subsection*{Interpretation}
This is a strong positive finding. It indicates that a perimeter firewall is in place and configured correctly to block unsolicited inbound traffic, adhering to the principle of least privilege. This significantly reduces the external attack surface and protects internal systems from direct network-based attacks from the internet. However, this does not mitigate risks from phishing or other attacks that originate from within a compromised account or system.

\section{Consolidated Risk Assessment}

This section correlates findings from the security control review, technical scan, and pre-existing risk data. No pre-existing vulnerabilities were provided. The primary risks identified stem from policy and procedural gaps.

\begin{table}[h!]
\centering
\caption{Identified Risks and Severity}
\label{tab:risks}
\begin{tabular}{p{0.25\linewidth} p{0.5\linewidth} c}
\toprule
\textbf{Risk Name} & \textbf{Overview} & \textbf{Severity} \\
\midrule
\textbf{Lack of MFA on Email} & Email accounts are protected only by passwords, making them highly susceptible to compromise via phishing or credential stuffing. This can lead to data breaches and financial fraud. & \textbf{Critical} \\
\addlinespace
\textbf{Lack of MFA on Workstations} & Employee computers are not protected by MFA, allowing an attacker with stolen credentials to gain direct access to the internal network and sensitive data. & \textbf{Critical} \\
\addlinespace
\textbf{No Security Awareness Training Program} & Employees are not trained to recognize or report security threats. This makes the organization highly vulnerable to phishing, malware, and social engineering attacks. & \textbf{High} \\
\bottomrule
\end{tabular}
\end{table}

\section{Recommendations}

The following actions are recommended to mitigate the identified risks and improve the overall security posture of \textbf{Copperhead Cables}.

\begin{enumerate}
    \item \textbf{Implement MFA for All Email Accounts (Priority: Immediate)} \\
    \textit{Rationale:} To address the critical risk of business email compromise. \\
    \textit{Action:} Enforce MFA for all user accounts on the \texttt{CopperheadCables.org} email platform. Provide clear instructions and support to all employees during the rollout.

    \item \textbf{Enforce MFA for Workstation and Remote Access (Priority: Immediate)} \\
    \textit{Rationale:} To prevent unauthorized access to the internal network and company devices. \\
    \textit{Action:} Deploy and mandate the use of MFA for all computer logins (Windows, macOS) and any remote access solutions (e.g., VPN).

    \item \textbf{Establish a Security Awareness Training Program (Priority: High)} \\
    \textit{Rationale:} To reduce the risk of human error and strengthen the "human firewall." \\
    \textit{Action:} Procure and implement a security awareness training solution. The program must include initial training for all new hires and mandatory annual refresher training for all staff. Conduct regular phishing simulations to measure effectiveness.
\end{enumerate}

\end{document}
```