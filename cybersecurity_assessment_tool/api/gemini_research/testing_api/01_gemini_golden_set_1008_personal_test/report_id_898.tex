```latex
\documentclass[12pt]{article}

% Required Packages
\usepackage[margin=1in]{geometry}
\usepackage{pifont} % For checkmarks and crosses
\usepackage{booktabs} % For professional tables
\usepackage{hyperref} % For clickable links and metadata
\usepackage{url} % For URL formatting
\usepackage{seqsplit} % For splitting long strings in tt font

% Document Metadata
\hypersetup{
    colorlinks=true,
    linkcolor=black,
    urlcolor=blue,
    pdftitle={Cybersecurity Posture Assessment Report},
    pdfauthor={Cybersecurity Analyst},
    pdfsubject={Security Assessment},
    pdfkeywords={Cybersecurity, Risk, Assessment, Scan}
}

% Title
\title{Cybersecurity Posture Assessment Report \\ \large For: \textbf{Firebrand Media}}
\author{Cybersecurity Analyst}
\date{\today}

\begin{document}

\maketitle
\thispagestyle{empty}
\newpage
\tableofcontents
\newpage

\section{Executive Summary}

This report provides a comprehensive cybersecurity posture assessment for \textbf{Firebrand Media}, based on a combination of self-reported organizational data, an external network scan, and a review of pre-existing risks. The assessment identifies the organization's security strengths, weaknesses, and a prioritized list of actionable recommendations to enhance its security posture.

The analysis revealed critical gaps in access control policies. Specifically, the absence of Multi-Factor Authentication (MFA) for computer logins and access to sensitive data systems represents a significant vulnerability. Furthermore, the lack of a formal security awareness training program for both new and existing employees elevates the risk of human-error-related incidents, such as phishing and social engineering attacks.

On a positive note, the external network scan of the target IP address \texttt{[Target IP]} did not reveal any open ports. This suggests a restrictive firewall policy is in place for the scanned asset, which effectively minimizes its external attack surface. The organization also enforces MFA for email access, a commendable security control.

The highest priority recommendations are to immediately implement a comprehensive MFA policy across all critical systems and to establish a recurring security awareness training program for all employees. Addressing these findings will substantially mitigate the risk of unauthorized access and strengthen the organization's overall defense against common cyber threats.

\section{Organizational Information}

The following details were provided for the assessment. This information establishes the context and scope of the review.

\begin{tabular}{@{}ll}
    \toprule
    \textbf{Attribute} & \textbf{Value} \\
    \midrule
    Organization Name & \textbf{Firebrand Media} \\
    Email Domain & \seqsplit{\texttt{FirebrandMedia.net}} \\
    Website Domain & \seqsplit{\url{www.FirebrandMedia.net}} \\
    External IP Address & \seqsplit{\texttt{45.30.37.196}} \\
    \bottomrule
\end{tabular}

\section{Security Control Review}

The following table summarizes the organization's responses to a security controls questionnaire. "No" answers indicate potential gaps in the security framework that may require remediation.

\begin{tabular}{@{}p{0.7\linewidth}c@{}}
    \toprule
    \textbf{Control Question} & \textbf{Response} \\
    \midrule
    Do you require MFA to access email? & \ding{51} \\ % Yes
    Do you require MFA to log into computers? & \ding{55} \\ % No
    Do you require MFA to access sensitive data systems? & \ding{55} \\ % No
    Does your organization have an employee acceptable use policy? & \ding{51} \\ % Yes
    Does your organization do security awareness training for new employees? & \ding{55} \\ % No
    Does your organization do security awareness training for all employees at least once per year? & \ding{55} \\ % No
    \bottomrule
\end{tabular}

\subsection*{Analysis of Controls}
The questionnaire highlights critical weaknesses in access control and employee security training. The lack of MFA for computer and sensitive data access significantly increases the risk of account compromise. The absence of a security awareness training program leaves the organization highly susceptible to phishing and other social engineering attacks.

\section{Technical Scan Results}

An external network vulnerability scan was conducted to identify open ports and services exposed to the internet.

\begin{itemize}
    \item \textbf{Target IP Address:} \texttt{[Target IP]}
    \item \textbf{Scan Summary:} The scan completed successfully but did not identify any open TCP or UDP ports on the target system.
\end{itemize}

\subsection*{Analysis of Scan Results}
The absence of open ports is a positive finding, indicating that the target system has a minimal external attack surface. This is likely due to a well-configured firewall that denies unsolicited inbound traffic. While this reduces the risk of external network-based attacks against this specific IP, it does not preclude vulnerabilities in web applications, client-side attacks, or threats from internal sources.

\section{Consolidated Risk Assessment}

This section synthesizes findings from the security control review and technical scan to present a consolidated list of identified risks. No pre-existing vulnerabilities were provided for this assessment.

\begin{tabular}{@{}p{0.1\linewidth}p{0.25\linewidth}p{0.45\linewidth}l@{}}
    \toprule
    \textbf{Risk ID} & \textbf{Risk Name} & \textbf{Description} & \textbf{Severity} \\
    \midrule
    R-01 & Lack of Multi-Factor Authentication (MFA) & MFA is not enforced for computer logins or access to sensitive data systems. This significantly increases the risk of unauthorized access resulting from compromised credentials (e.g., via phishing or password spraying). & \textbf{Critical} \\
    \addlinespace
    R-02 & Insufficient Security Awareness Training & The organization does not provide security awareness training to new or existing employees. This lack of training increases the organization's susceptibility to phishing, social engineering, malware infections, and policy violations. & \textbf{High} \\
    \bottomrule
\end{tabular}

\section{Recommendations}

The following prioritized recommendations are provided to address the identified risks and improve the overall security posture of \textbf{Firebrand Media}.

\begin{enumerate}
    \item \textbf{Implement Comprehensive Multi-Factor Authentication (Critical)} \\
    To mitigate the risk of unauthorized access (R-01), MFA should be implemented immediately.
    \begin{itemize}
        \item \textbf{Phase 1 (Immediate):} Enforce MFA on all systems containing sensitive or critical data. This includes databases, file servers, and administrative portals.
        \item \textbf{Phase 2 (Short-Term):} Roll out MFA for all employee computer logins (e.g., Windows, macOS).
        \item \textbf{Technology:} Utilize strong MFA methods such as authenticator applications (e.g., Google Authenticator, Microsoft Authenticator) or FIDO2-compliant hardware keys.
    \end{itemize}
    \vspace{1em}

    \item \textbf{Establish a Security Awareness Training Program (High)} \\
    To address the risks associated with human error (R-02), a formal and continuous security awareness program should be established.
    \begin{itemize}
        \item \textbf{Onboarding:} Integrate mandatory security awareness training into the onboarding process for all new employees.
        \item \textbf{Annual Training:} Conduct comprehensive training for all staff at least once per year to cover topics like phishing identification, password hygiene, and acceptable use policies.
        \item \textbf{Phishing Simulations:} Perform regular, simulated phishing campaigns to test employee awareness and provide targeted, just-in-time training to those who fall victim.
    \end{itemize}
\end{enumerate}

\end{document}
```