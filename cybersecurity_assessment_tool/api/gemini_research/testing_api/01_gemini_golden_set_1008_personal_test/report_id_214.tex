```latex
\documentclass[12pt]{article}

% Preamble: Required Packages
\usepackage[margin=1in]{geometry}
\usepackage{pifont} % For checkmarks and crosses
\usepackage{booktabs} % For professional tables
\usepackage{hyperref} % For clickable links
\usepackage{url} % For formatting URLs
\usepackage{seqsplit} % For splitting long strings in tt font
\usepackage{graphicx} % For logo (placeholder)
\usepackage{fancyhdr} % For header/footer

% --- Document Metadata ---
\hypersetup{
    colorlinks=true,
    linkcolor=blue,
    filecolor=magenta,      
    urlcolor=cyan,
    pdftitle={Cybersecurity Posture Report},
    pdfauthor={Cybersecurity Analysis Cell},
    pdfsubject={Security Assessment},
    pdfkeywords={Cybersecurity, Risk, Assessment},
}

% --- Header and Footer ---
\pagestyle{fancy}
\fancyhf{} % Clear all header and footer fields
\fancyhead[L]{Cybersecurity Posture Report}
\fancyhead[R]{Opal Sky Media}
\fancyfoot[C]{\thepage}
\renewcommand{\headrulewidth}{0.4pt}
\renewcommand{\footrulewidth}{0.4pt}

% --- Helper Commands ---
\newcommand{\yes}{\ding{51}}
\newcommand{\no}{\ding{55}}

\begin{document}

% --- Title Page ---
\begin{titlepage}
    \centering
    \vspace*{1cm}
    
    \Huge
    \textbf{Cybersecurity Posture Report}
    
    \vspace{1.5cm}
    
    \Large
    Prepared for: \\
    \vspace{0.5cm}
    \textbf{Opal Sky Media}
    
    \vspace{2cm}
    
    \large
    \textbf{Date of Report:} \today \\
    \textbf{Analysis Period:} Based on data provided on \today
    
    \vfill
    
    \large
    \textbf{CONFIDENTIAL} \\
    \textit{This document contains sensitive information and is intended solely for the use of the designated recipient.}
    
\end{titlepage}

\tableofcontents
\newpage

% --- Section 1: Executive Summary ---
\section{Executive Summary}

This report provides a comprehensive cybersecurity assessment for \textbf{Opal Sky Media}, synthesizing information from organizational questionnaires, external network scans, and a review of pre-existing risks. The analysis reveals a mixed security posture with significant strengths in policy but critical deficiencies in technical and procedural controls.

The external network scan of the target IP address revealed no open ports, suggesting a strong network perimeter defense or a well-configured firewall. This is a positive security finding.

However, the organizational data highlights several critical-risk gaps. The most pressing concern is the complete absence of Multi-Factor Authentication (MFA) for accessing email, computers, and sensitive data systems. This exposes the organization to a high risk of account compromise and unauthorized access. Additionally, the lack of mandatory security awareness training for new employees creates an immediate vulnerability, as new hires may be unaware of company policies and common cyber threats.

Immediate remediation should focus on implementing a robust MFA solution across all critical assets and integrating security training into the employee onboarding process.

% --- Section 2: Organizational Information ---
\section{Organizational Information}

The following details were provided for the assessment. This information forms the basis of the organizational context for the technical and procedural analysis.

\begin{itemize}
    \item \textbf{Organization Name:} Opal Sky Media
    \item \textbf{Primary Email Domain:} \texttt{OpalSkyMedia.com}
    \item \textbf{Primary Website:} \url{www.OpalSkyMedia.com}
    \item \textbf{External IP Address Scanned:} \texttt{214.126.19.114}
\end{itemize}

% --- Section 3: Security Control Review ---
\section{Security Control Review}

A review of the organization's security controls was conducted via a questionnaire. The responses indicate the current state of procedural and policy-based defenses. "No" answers represent significant gaps that increase organizational risk.

\begin{table}[h!]
\centering
\caption{Security Controls Questionnaire Results}
\begin{tabular}{p{0.7\linewidth} c c}
\toprule
\textbf{Control Question} & \textbf{Response} & \textbf{Status} \\
\midrule
Do you require MFA to access email? & No & \no \\
Do you require MFA to log into computers? & No & \no \\
Do you require MFA to access sensitive data systems? & No & \no \\
Does your organization have an employee acceptable use policy? & Yes & \yes \\
Does your organization do security awareness training for new employees? & No & \no \\
Does your organization do security awareness training for all employees at least once per year? & Yes & \yes \\
\bottomrule
\end{tabular}
\end{table}

\subsection*{Analysis of Control Gaps}
\begin{itemize}
    \item \textbf{Lack of Multi-Factor Authentication (MFA):} The absence of MFA for email, computer logins, and sensitive systems is a \textbf{Critical Risk}. Stolen or weak passwords are a primary vector for attackers, and MFA is the single most effective control to mitigate this threat.
    \item \textbf{No Onboarding Security Training:} Failing to train new employees on security best practices and company policies from day one is a \textbf{High Risk}. New hires are often targeted by social engineering attacks and may inadvertently cause security incidents.
\end{itemize}

% --- Section 4: Technical Scan Results ---
\section{Technical Scan Results}

An external network scan was performed to identify exposed services and potential vulnerabilities on the organization's public-facing infrastructure.

\begin{itemize}
    \item \textbf{Target IP Address:} \texttt{[Target IP]}
    \item \textbf{Scan Date:} \today
\end{itemize}

\subsection*{Findings}
The scan completed successfully but did not identify any open TCP or UDP ports on the target system. This indicates that the host is likely protected by a well-configured firewall that is effectively blocking unsolicited inbound traffic. This is a strong positive indicator of a hardened network perimeter. No vulnerabilities were identified.

% --- Section 5: Consolidated Risk Assessment ---
\section{Consolidated Risk Assessment}

This section synthesizes findings from the security control review, technical scans, and pre-existing risk data. Based on the provided inputs, no pre-existing vulnerabilities were reported. The primary risks identified are procedural and originate from the control gaps.

\begin{table}[h!]
\centering
\caption{Summary of Identified Risks}
\begin{tabular}{p{0.25\linewidth} p{0.5\linewidth} l}
\toprule
\textbf{Risk Name} & \textbf{Overview} & \textbf{Severity} \\
\midrule
\textbf{No Multi-Factor Authentication (MFA)} & The lack of MFA on critical systems like email and sensitive data platforms makes the organization highly susceptible to account takeover via credential theft or password spraying attacks. & \textbf{Critical} \\
\addlinespace
\textbf{Inadequate Employee Onboarding} & New employees are not receiving security awareness training upon being hired. This gap increases the likelihood of human error, such as falling for phishing attacks or mishandling sensitive data. & \textbf{High} \\
\bottomrule
\end{tabular}
\end{table}

% --- Section 6: Recommendations ---
\section{Recommendations}

The following actions are recommended to mitigate the identified risks and improve the overall security posture of \textbf{Opal Sky Media}.

\subsection*{Immediate Actions (0-30 Days)}
\begin{enumerate}
    \item \textbf{Implement MFA on Critical Systems (Risk: Critical):}
    \begin{itemize}
        \item \textbf{Action:} Enable and enforce MFA for all users on the primary email system (e.g., Office 365, Google Workspace).
        \item \textbf{Justification:} This is the most effective step to prevent unauthorized access to email, which is a gateway to other sensitive systems.
        \item \textbf{Priority:} Urgent.
    \end{itemize}
    \vspace{0.5cm}
    \item \textbf{Develop an Onboarding Security Module (Risk: High):}
    \begin{itemize}
        \item \textbf{Action:} Create a mandatory security awareness training module to be completed by all new employees within their first week of employment. This should cover the acceptable use policy, phishing identification, and password security.
        \item \textbf{Justification:} This ensures a baseline security understanding for all new personnel, reducing the risk of early-tenure security incidents.
        \item \textbf{Priority:} High.
    \end{itemize}
\end{enumerate}

\subsection*{Strategic Actions (30-90 Days)}
\begin{enumerate}
    \setcounter{enumi}{2} % Continue numbering from the previous list
    \item \textbf{Expand MFA Coverage (Risk: Critical):}
    \begin{itemize}
        \item \textbf{Action:} Roll out MFA requirements for all remote access solutions (VPN), privileged accounts, and systems containing sensitive or regulated data.
        \item \textbf{Justification:} Extends protection beyond email to secure the entire data ecosystem and administrative access points.
        \item \textbf{Priority:} High.
    \end{itemize}
\end{enumerate}

% --- Section 7: Conclusion ---
\section{Conclusion}

\textbf{Opal Sky Media} has established a strong network perimeter, as evidenced by the clean external scan. However, significant internal and procedural weaknesses present a clear and present danger to the organization. The absence of Multi-Factor Authentication is a critical vulnerability that must be addressed with the utmost urgency.

By implementing the recommendations outlined in this report, particularly the rapid deployment of MFA and the formalization of security training for new hires, the organization can substantially reduce its risk exposure and build a more resilient security posture.

\end{document}
```