```latex
\documentclass[12pt]{article}

% --- PACKAGES ---
\usepackage[margin=1in]{geometry}
\usepackage{pifont} % For checkmarks and crosses
\usepackage{booktabs} % For professional tables
\usepackage{hyperref} % For clickable links
\usepackage{url} % For URL formatting
\usepackage{seqsplit} % For splitting long strings
\usepackage{graphicx}
\usepackage{xcolor}

% --- DOCUMENT SETUP ---
\hypersetup{
    colorlinks=true,
    linkcolor=blue,
    filecolor=magenta,      
    urlcolor=cyan,
    pdftitle={Cybersecurity Posture Report},
    pdfpagemode=FullScreen,
}

\newcommand{\yes}{\ding{51}} % Green checkmark
\newcommand{\no}{\ding{55}}  % Red X

\title{Cybersecurity Posture Report \\ \large For: Open Door}
\author{Cybersecurity Analyst Group}
\date{\today}

% --- BEGIN DOCUMENT ---
\begin{document}

\maketitle
\thispagestyle{empty}
\newpage

\tableofcontents
\newpage

% --- EXECUTIVE SUMMARY ---
\section{Executive Summary}
This report provides a comprehensive analysis of the cybersecurity posture for \textbf{Open Door}, based on data collected from a network scan, an organizational security questionnaire, and a review of pre-existing risks.

The assessment reveals a generally strong security posture. The organization reports full implementation of key security controls, including multi-factor authentication (MFA) and regular security awareness training. The external network scan of the target host did not identify any open ports, which suggests a minimal attack surface and effective firewall configuration. No pre-existing vulnerabilities were provided for this assessment.

While the current findings are positive, continuous vigilance is recommended. Recommendations focus on maintaining this strong posture through ongoing monitoring and periodic, in-depth security testing.

% --- ORGANIZATIONAL INFORMATION ---
\section{Organizational Information}
The following details were provided by the organization and used as a baseline for this assessment.

\begin{tabular}{@{}ll}
\toprule
\textbf{Attribute} & \textbf{Value} \\
\midrule
Organization Name & \textbf{Open Door} \\
Email Domain & \texttt{OpenDoor.com} \\
External IP Address & \texttt{185.88.8.60} \\
\bottomrule
\end{tabular}

% --- SECURITY CONTROL REVIEW ---
\section{Security Control Review}
A review of the organization's self-reported security controls was conducted. The responses indicate a strong commitment to fundamental security practices. All reviewed controls are reportedly in place.

\begin{table}[h!]
\centering
\caption{Security Controls Questionnaire Results}
\begin{tabular}{@{}p{0.8\linewidth}c@{}}
\toprule
\textbf{Control Question} & \textbf{Response} \\
\midrule
Do you require MFA to access email? & \yes \\
Do you require MFA to log into computers? & \yes \\
Do you require MFA to access sensitive data systems? & \yes \\
Does your organization have an employee acceptable use policy? & \yes \\
Does your organization do security awareness training for new employees? & \yes \\
Does your organization do security awareness training for all employees at least once per year? & \yes \\
\bottomrule
\end{tabular}
\end{table}

% --- TECHNICAL SCAN RESULTS ---
\section{Technical Scan Results}
An external network port scan was performed to identify accessible services and potential vulnerabilities.

\begin{itemize}
    \item \textbf{Target Host:} \texttt{[Target IP]}
    \item \textbf{Scan Date:} Not specified in scan data.
\end{itemize}

\subsection{Findings}
The network scan against the target host \textbf{did not reveal any open ports}.

\subsubsection{Analysis}
This is a positive security finding. It indicates that the host is likely protected by a well-configured firewall that denies unsolicited inbound traffic. This significantly reduces the external attack surface, making it more difficult for an attacker to discover and exploit potential services.

% --- RISK ASSESSMENT ---
\section{Risk Assessment}
This section synthesizes findings from the security control review, technical scan, and pre-existing risk data. Based on the information provided for this assessment, no significant risks were identified. The combination of strong, self-reported policy controls and a lack of externally visible services points to a low-risk posture at this time.

\begin{table}[h!]
\centering
\caption{Identified Risk Summary}
\begin{tabular}{@{}p{0.3\linewidth}p{0.5\linewidth}c@{}}
\toprule
\textbf{Risk Name} & \textbf{Overview} & \textbf{Severity} \\
\midrule
\multicolumn{3}{c}{\textit{No significant risks identified from the provided data.}} \\
\bottomrule
\end{tabular}
\end{table}

% --- RECOMMENDATIONS ---
\section{Recommendations}
While the current posture is strong, security is an ongoing process. The following recommendations are provided to help \textbf{Open Door} maintain and enhance its defenses.

\begin{enumerate}
    \item \textbf{Implement Continuous Monitoring:}
    \begin{itemize}
        \item \textbf{Action:} Establish a program of regular, automated vulnerability scanning for all external and internal assets.
        \item \textbf{Justification:} Threats evolve and configurations can change. Continuous monitoring ensures that new vulnerabilities or misconfigurations are detected and remediated promptly, maintaining the secure baseline observed in this report.
    \end{itemize}
    \vspace{1em}
    \item \textbf{Conduct In-Depth Penetration Testing:}
    \begin{itemize}
        \item \textbf{Action:} Engage a third party to perform an annual penetration test that goes beyond port scanning to actively test the resilience of defenses, including web applications and firewall rule sets.
        \item \textbf{Justification:} A port scan confirms that no ports are open, but a penetration test validates that the firewall and other security controls are truly effective against a skilled attacker attempting to bypass them.
    \end{itemize}
    \vspace{1em}
    \item \textbf{Maintain and Review Policies:}
    \begin{itemize}
        \item \textbf{Action:} Schedule an annual review of all security policies (e.g., Acceptable Use, MFA) to ensure they remain relevant and aligned with current best practices and business needs.
        \item \textbf{Justification:} A strong policy framework is the foundation of a good security program. Regular reviews ensure this foundation does not become outdated.
    \end{itemize}
\end{enumerate}

\end{document}
```