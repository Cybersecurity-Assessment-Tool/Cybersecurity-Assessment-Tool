```latex
\documentclass[12pt, a4paper]{article}

% Preamble: Required Packages
\usepackage[margin=1in]{geometry}
\usepackage{pifont} % For checkmarks and crosses
\usepackage{booktabs} % For professional tables
\usepackage{hyperref} % For clickable links
\usepackage{url} % For formatting URLs
\usepackage{seqsplit} % To split long strings without breaking
\usepackage{graphicx}
\usepackage[table]{xcolor}
\usepackage{fancyhdr}

% --- Document Setup ---
\definecolor{critical}{HTML}{990000}
\definecolor{high}{HTML}{D14124}
\definecolor{medium}{HTML}{E89028}
\definecolor{low}{HTML}{F2C94C}
\definecolor{info}{HTML}{2D9CDB}
\definecolor{tablegray}{gray}{0.9}

\hypersetup{
    colorlinks=true,
    linkcolor=blue,
    filecolor=magenta,      
    urlcolor=cyan,
    pdftitle={Cybersecurity Posture Report},
    pdfpagemode=FullScreen,
}

\pagestyle{fancy}
\fancyhf{}
\fancyhead[L]{Cybersecurity Posture Report}
\fancyhead[R]{Nova Terra}
\fancyfoot[C]{\thepage}

% --- Document Start ---
\begin{document}

% --- Title Page ---
\begin{titlepage}
    \centering
    \vspace*{1cm}
    \Huge{\textbf{Cybersecurity Posture Report}}
    \vspace{0.5cm}
    \Large{Prepared for: Nova Terra}
    \vspace{1.5cm}
    \normalsize
    \textbf{Date of Report:} \today \\
    \textbf{Report ID:} CSR-2023-451
    \vfill
    \large
    \textit{This report contains sensitive and confidential information intended only for the designated recipients. Unauthorized distribution is strictly prohibited.}
\end{titlepage}

\tableofcontents
\newpage

% --- Section 1: Executive Summary ---
\section{Executive Summary}
This report provides a comprehensive analysis of the cybersecurity posture of Nova Terra, based on network scans, a security controls questionnaire, and a review of pre-existing risks.

The assessment has identified several critical and high-severity risks that require immediate attention. A key technical finding is the continued exposure of Remote Desktop Protocol (RDP) services on internal systems, including a newly discovered host at \texttt{10.10.10.51}. This finding expands upon a previously identified risk, indicating a systemic configuration issue.

Furthermore, analysis of organizational security controls reveals critical gaps in access management. The absence of Multi-Factor Authentication (MFA) for email and sensitive data systems significantly increases the risk of unauthorized access and potential data breach. These technical vulnerabilities are compounded by foundational policy weaknesses, including the lack of a formal Acceptable Use Policy and mandatory annual security training for all employees.

This combination of exposed services, weak access controls, and policy gaps creates a high-risk environment susceptible to credential theft, ransomware attacks, and unauthorized data exfiltration. We strongly recommend prioritizing the remediation actions outlined in Section \ref{sec:recommendations} to mitigate these threats and improve the overall security posture.

% --- Section 2: Organizational Information ---
\section{Organizational Information}
The following details were provided for the assessment.

\begin{tabular}{@{}ll}
    \toprule
    \textbf{Attribute} & \textbf{Value} \\
    \midrule
    Organization Name & Nova Terra \\
    Email Domain & \texttt{NovaTerra.com} \\
    Website Domain & \seqsplit{\url{www.NovaTerra.com}} \\
    External IP Address & \texttt{173.201.203.83} \\
    \bottomrule
\end{tabular}

% --- Section 3: Security Control Review ---
\section{Security Control Review}
The following table summarizes the organization's responses to a security controls questionnaire. Items marked with \ding{55} represent significant gaps in the security framework and are correlated with identified risks in Section \ref{sec:risk-assessment}.

\rowcolors{2}{tablegray!50!white}{white}
\begin{tabular}{p{0.6\textwidth} c c}
    \toprule
    \textbf{Control Question} & \textbf{Response} & \textbf{Status} \\
    \midrule
    Do you require MFA to access email? & No & \ding{55} \\
    Do you require MFA to log into computers? & Yes & \ding{51} \\
    Do you require MFA to access sensitive data systems? & No & \ding{55} \\
    Does your organization have an employee acceptable use policy? & No & \ding{55} \\
    Does your organization do security awareness training for new employees? & Yes & \ding{51} \\
    Does your organization do security awareness training for all employees at least once per year? & No & \ding{55} \\
    \bottomrule
\end{tabular}

% --- Section 4: Technical Scan Results ---
\section{Technical Scan Results}
A network scan was performed to identify active services and potential vulnerabilities on the target host.

\subsection{Nmap Scan: \texttt{10.10.10.51}}
The scan identified the following open port on the target system.

\begin{tabular}{@{}llll}
    \toprule
    \textbf{Port} & \textbf{State} & \textbf{Service Name} & \textbf{Analysis} \\
    \midrule
    3389/tcp & Open & \texttt{ms-wbt-server} & Remote Desktop Protocol (RDP) \\
    \bottomrule
\end{tabular}

\paragraph{Analysis:} The scan confirms that port 3389, used for Microsoft's Remote Desktop Protocol (RDP), is open on the host \texttt{10.10.10.51}. RDP is a primary target for attackers who use it to gain remote control over systems. Exposing RDP without compensating controls like a VPN or strict firewall rules is a critical security risk. This finding, combined with the pre-existing risk on \texttt{10.10.10.50}, suggests a pattern of insecure RDP deployment within the network.

% --- Section 5: Correlated Risk Assessment ---
\section{Correlated Risk Assessment}
\label{sec:risk-assessment}
The following table synthesizes findings from the technical scan, security control review, and pre-existing risk data into a prioritized list of current risks.

\begin{tabular}{p{0.15\textwidth} p{0.65\textwidth} p{0.15\textwidth}}
    \toprule
    \textbf{Risk ID} & \textbf{Description} & \textbf{Severity} \\
    \midrule
    \rowcolor{critical!20}
    RISK-001 & \textbf{Systemic RDP Exposure:} Remote Desktop Protocol is exposed on multiple internal systems (\texttt{10.10.10.50}, \texttt{10.10.10.51}). This provides a direct vector for attackers to gain unauthorized remote access, often leading to ransomware deployment or lateral movement. & \textcolor{critical}{\textbf{Critical}} \\
    
    \rowcolor{critical!20}
    RISK-002 & \textbf{Lack of MFA on Critical Systems:} MFA is not enforced for accessing email or sensitive data systems. A compromised password (e.g., via phishing) would be sufficient for an attacker to gain access to confidential communications and critical business data. & \textcolor{critical}{\textbf{Critical}} \\
    
    \rowcolor{high!20}
    RISK-003 & \textbf{Insufficient Policy and Training:} The absence of an Acceptable Use Policy and mandatory annual security training for all staff weakens the human element of security. Employees may be unaware of best practices, making them more susceptible to social engineering and phishing attacks. & \textcolor{high}{\textbf{High}} \\
    \bottomrule
\end{tabular}

% --- Section 6: Recommendations ---
\section{Recommendations}
\label{sec:recommendations}
The following prioritized recommendations are provided to address the identified risks.

\subsection{Immediate Priority (Remediate within 72 hours)}
\begin{itemize}
    \item \textbf{Remediate RDP Exposure (RISK-001):}
    \begin{itemize}
        \item Immediately restrict access to RDP (port 3389) on hosts \texttt{10.10.10.50} and \texttt{10.10.10.51} from all non-essential sources.
        \item For long-term secure access, implement a Virtual Private Network (VPN) or a secure gateway solution. All remote administrative access should occur exclusively through this managed channel.
    \end{itemize}
\end{itemize}

\subsection{High Priority (Remediate within 30 days)}
\begin{itemize}
    \item \textbf{Implement MFA (RISK-002):}
    \begin{itemize}
        \item Enforce mandatory MFA for all user accounts on the email platform (e.g., Microsoft 365, Google Workspace).
        \item Identify all systems classified as containing sensitive data and enforce MFA for all access, both privileged and standard.
    \end{itemize}
\end{itemize}

\subsection{Medium Priority (Remediate within 90 days)}
\begin{itemize}
    \item \textbf{Develop and Implement Security Policies (RISK-003):}
    \begin{itemize}
        \item Create a formal Acceptable Use Policy (AUP) that defines the rules for using company IT assets. Require all employees to read and acknowledge the policy.
    \end{itemize}
    \item \textbf{Establish Annual Security Training (RISK-003):}
    \begin{itemize}
        \item Procure and schedule mandatory annual security awareness training for all employees. The training should cover key topics such as phishing, password hygiene, and social engineering.
    \end{itemize}
\end{itemize}

\end{document}
```