```latex
\documentclass[12pt, a4paper]{article}

% Preamble: Required Packages
\usepackage[margin=1in]{geometry}
\usepackage{pifont} % For checkmarks and crosses
\usepackage{booktabs} % For professional tables
\usepackage{hyperref} % For clickable links
\usepackage{url} % For URL formatting
\usepackage{seqsplit} % To split long strings in texttt
\usepackage[T1]{fontenc}

% Document Metadata
\title{Cybersecurity Assessment Report}
\author{Cybersecurity Analysis Division}
\date{\today}

\begin{document}

\maketitle
\thispagestyle{empty}
\newpage
\tableofcontents
\newpage

% --- 1. Executive Summary ---
\section{Executive Summary}
This report details the findings of a cybersecurity assessment for \textbf{Catalyst Consulting}. The analysis is based on a network scan, a review of organizational security controls, and an evaluation of pre-existing risks.

The assessment identified several critical and high-risk vulnerabilities that require immediate attention. The most significant findings include:
\begin{itemize}
    \item \textbf{Systemic Insecure Service Exposure:} The network scan identified an open Remote Desktop Protocol (RDP) port on a new host (\texttt{10.10.10.51}). This finding, correlated with a pre-existing risk on another host, indicates a pattern of insecure RDP configuration across the network.
    \item \textbf{Critical Gaps in Identity and Access Management:} The organization does not enforce Multi-Factor Authentication (MFA) for email or computer logins. This significantly increases the risk of unauthorized access through credential compromise (e.g., phishing or password spraying).
\end{itemize}

The combination of exposed high-risk services and weak authentication controls creates a substantial risk of a security breach. We strongly recommend prioritizing the remediation actions outlined in Section \ref{sec:recommendations} to mitigate these threats and improve the organization's overall security posture.

% --- 2. Organizational Information ---
\section{Organizational Information}
The following details were provided for the assessment.

\begin{tabular}{@{}ll}
    \toprule
    \textbf{Attribute} & \textbf{Value} \\
    \midrule
    Organization Name & \textbf{Catalyst Consulting} \\
    Email Domain & \texttt{CatalystConsulting.org} \\
    Website Domain & \seqsplit{\url{www.CatalystConsulting.org}} \\
    External IP Address & \texttt{205.14.75.99} \\
    \bottomrule
\end{tabular}

% --- 3. Security Control Review ---
\section{Security Control Review}
A review of the organization's security controls was conducted via a questionnaire. The responses are summarized below. Gaps in security best practices are marked with \ding{55} and represent significant areas of risk.

\vspace{1em}
\textit{Key: \ding{51} = Yes (Control in Place) \quad \ding{55} = No (Control Gap)}
\vspace{1em}

\begin{tabular}{@{}p{0.8\linewidth}c}
    \toprule
    \textbf{Control Question} & \textbf{Response} \\
    \midrule
    Do you require MFA to access email? & \ding{55} \\
    Do you require MFA to log into computers? & \ding{55} \\
    Do you require MFA to access sensitive data systems? & \ding{51} \\
    Does your organization have an employee acceptable use policy? & \ding{51} \\
    Does your organization do security awareness training for new employees? & \ding{51} \\
    Does your organization do security awareness training for all employees at least once per year? & \ding{51} \\
    \bottomrule
\end{tabular}

\subsection*{Analysis of Control Gaps}
The lack of MFA for email and computer logins are critical security deficiencies.
\begin{itemize}
    \item \textbf{Email Access:} Without MFA, corporate email is vulnerable to phishing attacks and credential stuffing, which can lead to data breaches, business email compromise (BEC), and further internal network compromise.
    \item \textbf{Computer Logins:} The absence of MFA on endpoints allows an attacker with valid credentials (potentially stolen via a phishing attack on email) to gain direct access to the internal network, bypassing perimeter defenses.
\end{itemize}

% --- 4. Technical Scan Results ---
\section{Technical Scan Results}
An external network scan was performed to identify exposed services.

\begin{itemize}
    \item \textbf{Target IP Address:} \texttt{10.10.10.51}
    \item \textbf{Scan Tool:} Nmap
    \item \textbf{Scan Status:} Host is up.
\end{itemize}

The following open ports were discovered:

\begin{tabular}{@{}llll}
    \toprule
    \textbf{Port} & \textbf{State} & \textbf{Service Name} & \textbf{Analysis} \\
    \midrule
    3389/tcp & open & \texttt{ms-wbt-server} & High Risk \\
    \bottomrule
\end{tabular}

\subsection*{Analysis of Findings}
The scan identified that port \textbf{3389/tcp}, used by the Remote Desktop Protocol (RDP), is open on host \texttt{10.10.10.51}. RDP is a primary target for attackers who use brute-force password attacks or exploit known vulnerabilities (e.g., BlueKeep) to gain unauthorized remote control of systems.

This finding is especially critical when correlated with two other data points:
\begin{enumerate}
    \item The pre-existing risk of RDP exposure on another host (\texttt{10.10.10.50}).
    \item The lack of MFA for computer logins, which would otherwise protect RDP access.
\end{enumerate}
This indicates a systemic issue that significantly elevates the risk of a network breach.

% --- 5. Consolidated Risk Assessment ---
\section{Consolidated Risk Assessment}
The following table synthesizes findings from the security questionnaire, the technical scan, and pre-existing risk data into a consolidated list of security risks.

\begin{tabular}{@{}lp{0.3\linewidth}p{0.4\linewidth}l}
    \toprule
    \textbf{ID} & \textbf{Risk Name} & \textbf{Description} & \textbf{Severity} \\
    \midrule
    RISK-001 & Lack of MFA for Email & User email accounts are protected only by a password, making them highly susceptible to phishing and credential theft. & High \\
    \addlinespace
    RISK-002 & Lack of MFA for Endpoint Login & Employee computers can be accessed with a single factor (password), increasing the impact of credential compromise. & High \\
    \addlinespace
    RISK-003 & Systemic RDP Exposure & RDP is exposed on multiple internal hosts (\texttt{10.10.10.50}, \texttt{10.10.10.51}). This, combined with a lack of MFA, presents a critical vector for ransomware and network intrusion. & Critical \\
    \bottomrule
\end{tabular}

% --- 6. Recommendations ---
\section{Recommendations}
\label{sec:recommendations}
Based on the analysis, we provide the following prioritized recommendations to mitigate the identified risks.

\subsection*{Priority 1: Immediate Actions (0-7 Days)}
\begin{enumerate}
    \item \textbf{Remediate RDP Exposure:}
        \begin{itemize}
            \item Immediately close port 3389 on any host where RDP access is not required, starting with \texttt{10.10.10.51} and verifying \texttt{10.10.10.50}.
            \item For all systems requiring remote access, implement a Virtual Private Network (VPN) with MFA. RDP should only be accessible through the secure VPN tunnel, not directly from the network.
        \end{itemize}
\end{enumerate}

\subsection*{Priority 2: High-Impact Actions (1-4 Weeks)}
\begin{enumerate}
    \setcounter{enumi}{1}
    \item \textbf{Deploy Multi-Factor Authentication (MFA):}
        \begin{itemize}
            \item Enable MFA for all user accounts on the corporate email system (RISK-001).
            \item Enforce MFA for all computer/endpoint logins, especially for administrative accounts (RISK-002).
        \end{itemize}
\end{enumerate}

\subsection*{Priority 3: Strategic Improvements (1-3 Months)}
\begin{enumerate}
    \setcounter{enumi}{2}
    \item \textbf{Conduct Internal Vulnerability Assessment:}
        \begin{itemize}
            \item The pattern of RDP exposure suggests other internal systems may be misconfigured. Perform a comprehensive internal network scan to identify and remediate other insecurely configured services.
        \end{itemize}
    \item \textbf{Review and Update Security Policies:}
        \begin{itemize}
            \item Update the acceptable use policy to include mandatory MFA usage.
            \item Develop a formal policy for secure remote access that mandates the use of the corporate VPN.
        \end{itemize}
\end{enumerate}

\end{document}
```