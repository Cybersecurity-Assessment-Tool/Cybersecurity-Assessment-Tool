```latex
\documentclass[12pt]{article}

% Preamble: Required Packages
\usepackage[margin=1in]{geometry}
\usepackage{pifont} % For \ding symbols (checkmarks/crosses)
\usepackage{booktabs} % For professional-looking tables
\usepackage{hyperref} % For clickable links and table of contents
\usepackage{url}
\usepackage{seqsplit} % To break long strings in \texttt
\usepackage{xcolor} % For custom colors
\usepackage{graphicx}
\usepackage{fancyhdr}

% --- Document Setup ---
\hypersetup{
    colorlinks=true,
    linkcolor=blue,
    filecolor=magenta,
    urlcolor=cyan,
    pdftitle={Cybersecurity Posture Assessment Report},
    pdfauthor={Cybersecurity Analysis Division},
}

% Define colors for risk levels
\definecolor{critical}{HTML}{990000}
\definecolor{high}{HTML}{DD4B39}
\definecolor{medium}{HTML}{FF9900}
\definecolor{low}{HTML}{34A853}

% Define commands for Yes/No with color
\newcommand{\yes}{\textcolor{green}{\ding{51}}}
\newcommand{\no}{\textcolor{red}{\ding{55}}}

% --- Header and Footer ---
\pagestyle{fancy}
\fancyhf{} % Clear all header and footer fields
\fancyhead[L]{\textbf{Cybersecurity Posture Assessment}}
\fancyhead[R]{\textbf{Arcane Security}}
\fancyfoot[C]{\thepage}
\renewcommand{\headrulewidth}{0.4pt}
\renewcommand{\footrulewidth}{0.4pt}

% --- Document Begins ---
\begin{document}

\title{Cybersecurity Posture Assessment Report \\ \large For Arcane Security}
\author{Cybersecurity Analysis Division}
\date{\today}
\maketitle

\begin{abstract}
    This report provides a comprehensive analysis of the cybersecurity posture of Arcane Security. The assessment is based on a synthesis of organizational data, technical network scans, and a review of existing risks. The findings indicate several critical and high-risk vulnerabilities that require immediate attention. Key issues include an externally exposed, end-of-life database server, significant gaps in identity and access management controls, and deficiencies in foundational security policies and training. This document details these findings and provides actionable recommendations to mitigate the identified risks.
\end{abstract}

\newpage
\tableofcontents
\newpage

% ===================================================================
\section{Executive Summary}
% ===================================================================

The cybersecurity assessment of Arcane Security revealed a mixed but concerning security posture. While some security controls are in place, the analysis identified three primary areas of significant risk:

\begin{enumerate}
    \item \textbf{Critical Infrastructure Exposure:} A network scan identified a MySQL database server (version 5.7.33) on an internal IP address (\texttt{172.16.50.20}) with its primary port (3306) open. This version of MySQL reached its official End-of-Life (EOL) in October 2023 and no longer receives security updates, making it a prime target for exploitation. This finding correlates directly with a pre-existing identified risk of "Database Exposure".

    \item \textbf{Identity and Access Management Gaps:} The organization does not enforce Multi-Factor Authentication (MFA) for accessing email or for logging into employee computers. This represents a critical weakness, as compromised credentials could grant an attacker broad access to sensitive communications and internal systems without any secondary verification.

    \item \textbf{Deficient Security Governance:} The organization lacks a formal employee Acceptable Use Policy (AUP) and does not provide security awareness training for new employees. These foundational policy and training gaps increase the likelihood of human error leading to a security incident and expose the organization to compliance and legal risks.
\end{enumerate}

The combination of these findings presents a high-risk scenario. An attacker could leverage the lack of MFA to compromise an employee's account and then pivot to exploit the exposed, unpatched database server. Immediate remediation is strongly advised.

% ===================================================================
\section{Organizational Information}
% ===================================================================

The following information was provided for the assessment.

\begin{table}[h!]
\centering
\begin{tabular}{@{}ll@{}}
\toprule
\textbf{Attribute} & \textbf{Value} \\ \midrule
Organization Name & Arcane Security \\
Email Domain & \texttt{ArcaneSecurity.com} \\
Website Domain & \url{www.ArcaneSecurity.com} \\
External IP Address & \texttt{82.49.243.14} \\ \bottomrule
\end{tabular}
\caption{Client Organizational Data.}
\label{tab:org_data}
\end{table}

% ===================================================================
\section{Security Control Review (Questionnaire)}
% ===================================================================

A review of the security controls questionnaire highlights significant gaps in access control and employee security governance. The lack of MFA for email and computer access, coupled with missing foundational policies and training, are identified as high-priority risks.

\begin{table}[h!]
\centering
\begin{tabular}{@{}p{0.8\textwidth}c@{}}
\toprule
\textbf{Control Question} & \textbf{Response} \\ \midrule
Do you require MFA to access email? & \no \\
Do you require MFA to log into computers? & \no \\
Do you require MFA to access sensitive data systems? & \yes \\
Does your organization have an employee acceptable use policy? & \no \\
Does your organization do security awareness training for new employees? & \no \\
Does your organization do security awareness training for all employees at least once per year? & \yes \\ \bottomrule
\end{tabular}
\caption{Security Controls Questionnaire Analysis.}
\label{tab:controls}
\end{table}

% ===================================================================
\section{Technical Scan Results}
% ===================================================================

A network scan was conducted to identify open ports and exposed services on the target system.

\subsection{Target: \texttt{172.16.50.20}}
The scan revealed one open port on the target host, exposing a database service to the network.

\begin{table}[h!]
\centering
\begin{tabular}{@{}lllll@{}}
\toprule
\textbf{Port} & \textbf{State} & \textbf{Service} & \textbf{Product} & \textbf{Version} \\ \midrule
3306/tcp & open & mysql & MySQL & 5.7.33 \\ \bottomrule
\end{tabular}
\caption{Open Ports and Services Detected on \texttt{172.16.50.20}.}
\label{tab:scan_results}
\end{table}

\paragraph{Critical Finding:} The detected MySQL version \textbf{5.7.33} is \textbf{End-of-Life (EOL)} as of October 2023. EOL software no longer receives security patches from the vendor, leaving it perpetually vulnerable to newly discovered exploits. Running EOL software, especially on a network-accessible service, is a critical security risk. This finding validates and elevates the severity of the pre-existing risk "Database Exposure".

% ===================================================================
\section{Consolidated Risk Assessment}
% ===================================================================

This section synthesizes findings from the organizational review, technical scans, and pre-existing risk data into a consolidated list of key risks.

\begin{table}[h!]
\centering
\begin{tabular}{@{}p{0.25\linewidth}p{0.55\linewidth}p{0.1\linewidth}@{}}
\toprule
\textbf{Risk Title} & \textbf{Description} & \textbf{Severity} \\ \midrule
\textbf{Exposed End-of-Life Database} & The MySQL database on port 3306 is open to the network and is running an unsupported, unpatched version (5.7.33). This exposes the organization to data breach, ransomware, and system compromise. & \textcolor{critical}{\textbf{Critical}} \\
\addlinespace
\textbf{Insufficient Access Controls} & Lack of MFA on critical entry points like email and computer logins significantly increases the risk of account takeover via phishing or credential stuffing attacks. & \textcolor{critical}{\textbf{Critical}} \\
\addlinespace
\textbf{Deficient Security Governance} & The absence of an Acceptable Use Policy and security training for new hires creates an environment where employees are more likely to engage in risky behavior or fall victim to social engineering. & \textcolor{high}{\textbf{High}} \\
\bottomrule
\end{tabular}
\caption{Summary of Identified Security Risks.}
\label{tab:risk_summary}
\end{table}

% ===================================================================
\section{Recommendations}
% ===================================================================

The following actions are recommended to mitigate the identified risks. They are prioritized based on severity and ease of implementation.

\subsection{Immediate Actions (Next 72 Hours)}
\begin{itemize}
    \item \textbf{Restrict Database Access:} Immediately apply firewall rules to restrict access to TCP port 3306 on host \texttt{172.16.50.20}. Access should only be permitted from specific, authorized application servers. All other access, especially from the wider internal network or externally, must be blocked.
\end{itemize}

\subsection{Short-Term Actions (Next 30 Days)}
\begin{itemize}
    \item \textbf{Deploy MFA:} Prioritize the deployment of MFA for all users on the email system (\texttt{ArcaneSecurity.com}). This is the single most effective control to prevent account takeovers.
    \item \textbf{Develop Acceptable Use Policy (AUP):} Draft and implement a formal AUP that clearly defines the rules for using company IT assets, data, and internet access. Require all employees to read and acknowledge the policy.
    \item \textbf{Initiate Database Upgrade Project:} Begin planning the migration from MySQL 5.7 to a currently supported version (e.g., MySQL 8.x or a managed cloud equivalent). This project is critical for long-term security.
\end{itemize}

\subsection{Long-Term Actions (Next 90 Days)}
\begin{itemize}
    \item \textbf{Expand MFA Coverage:} Implement MFA for all remote access solutions (e.g., VPN) and for local computer logins, especially for privileged users and remote workers.
    \item \textbf{Implement New Hire Training:} Develop and mandate a security awareness training module for all new employees as part of the onboarding process.
    \item \textbf{Conduct Vulnerability Scanning:} Establish a regular, automated vulnerability scanning program for all internal and external network assets to proactively identify issues like EOL software and misconfigurations in the future.
\end{itemize}

\end{document}
```