```latex
\documentclass[12pt, a4paper]{article}

% Preamble: Required Packages
\usepackage[margin=1in]{geometry}
\usepackage{pifont} % For checkmarks and crosses
\usepackage{booktabs} % For professional tables
\usepackage{hyperref} % For clickable links
\usepackage{url} % For URL formatting
\usepackage{seqsplit} % To split long strings in texttt
\usepackage{graphicx} % For logos, etc.
\usepackage{fancyhdr} % For headers and footers
\usepackage{lastpage} % To get the total number of pages
\usepackage{xcolor} % For custom colors

% --- Document Setup ---

% Define custom colors for the report
\definecolor{ReportBlue}{RGB}{0, 82, 155}
\definecolor{GrayText}{RGB}{100, 100, 100}
\definecolor{SeverityCritical}{RGB}{192, 0, 0}
\definecolor{SeverityHigh}{RGB}{255, 102, 0}
\definecolor{SeverityMedium}{RGB}{255, 192, 0}
\definecolor{SeverityLow}{RGB}{0, 176, 80}

% Hyperlink setup
\hypersetup{
    colorlinks=true,
    linkcolor=ReportBlue,
    urlcolor=ReportBlue,
    pdftitle={Cybersecurity Assessment Report},
    pdfauthor={Cybersecurity Analyst},
    pdfsubject={Security Assessment},
    pdfkeywords={Cybersecurity, Risk, Assessment}
}

% --- Header and Footer Configuration ---
\pagestyle{fancy}
\fancyhf{} % Clear all header and footer fields
\fancyhead[L]{\textit{Cybersecurity Assessment Report}}
\fancyhead[R]{Cinder & Ash}
\fancyfoot[C]{\thepage\ of \pageref{LastPage}}
\renewcommand{\headrulewidth}{0.4pt}
\renewcommand{\footrulewidth}{0.4pt}

% --- Document Start ---
\begin{document}

% --- Title Page ---
\begin{titlepage}
    \centering
    \vspace*{1cm}
    
    {\Huge \textbf{Cybersecurity Assessment Report}\par}
    \vspace{1.5cm}
    
    {\Large \textbf{Prepared For:}\par}
    \vspace{0.5cm}
    {\Large Cinder & Ash\par}
    
    \vfill
    
    {\large \today\par}
    
    \vspace{1cm}
    
    \begin{center}
        \rule{0.8\textwidth}{0.4pt}
    \end{center}
    \vspace{0.5cm}
    
    \textit{This document contains sensitive information. Distribution should be limited to authorized personnel only.}
    
\end{titlepage}

\tableofcontents
\newpage

% --- Section 1: Executive Summary ---
\section{Executive Summary}

This report details the findings of a cybersecurity assessment conducted for Cinder & Ash. The assessment combined a review of organizational security controls via a questionnaire with a targeted technical network scan to provide a holistic view of the current security posture.

The assessment identified a significant disparity between the organization's technical and procedural security controls. The technical scan of the target host (\seqsplit{\texttt{192.168.1.100}}) revealed a strong network security posture, with no open ports detected. This indicates effective firewalling and a minimal attack surface for that specific asset.

However, the review of organizational controls uncovered several critical and high-risk gaps. The most severe finding is the lack of Multi-Factor Authentication (MFA) for email access, which exposes the organization to a high risk of Business Email Compromise (BEC), phishing, and account takeovers. Additional high-risk findings include the absence of an employee Acceptable Use Policy and a lack of security awareness training for new hires upon joining.

While no pre-existing vulnerabilities were documented and the technical scan was clean, these policy and procedural gaps represent the most significant threats to the organization. We strongly recommend prioritizing the remediation of these findings, starting with the immediate implementation of MFA for all email accounts.

\newpage

% --- Section 2: Organizational Information ---
\section{Organizational Information}

The following information was provided for the assessment.

\begin{table}[h!]
    \centering
    \begin{tabular}{@{}ll@{}}
        \toprule
        \textbf{Attribute} & \textbf{Value} \\
        \midrule
        Organization Name & Cinder & Ash \\
        Email Domain & \seqsplit{\texttt{CinderAsh.com}} \\
        External IP Address & \seqsplit{\texttt{188.143.211.149}} \\
        \bottomrule
    \end{tabular}
    \caption{Client Organizational Details.}
    \label{tab:org_info}
\end{table}

% --- Section 3: Security Control Review ---
\section{Security Control Review}

A review of key administrative and technical security controls was conducted based on a questionnaire. The results are summarized below. "Yes" responses are marked with \ding{51} and "No" responses with \ding{55}. Findings from "No" responses are classified as either a Critical Gap or a High Risk.

\begin{table}[h!]
    \centering
    \begin{tabular}{@{}p{0.55\textwidth} c p{0.2\textwidth}@{}}
        \toprule
        \textbf{Control Question} & \textbf{Response} & \textbf{Assessment} \\
        \midrule
        Do you require MFA to access email? & \ding{55} & \textcolor{SeverityCritical}{\textbf{Critical Gap}} \\
        \addlinespace
        Do you require MFA to log into computers? & \ding{51} & Satisfactory \\
        \addlinespace
        Do you require MFA to access sensitive data systems? & \ding{51} & Satisfactory \\
        \addlinespace
        Does your organization have an employee acceptable use policy? & \ding{55} & \textcolor{SeverityHigh}{\textbf{High Risk}} \\
        \addlinespace
        Does your organization do security awareness training for new employees? & \ding{55} & \textcolor{SeverityHigh}{\textbf{High Risk}} \\
        \addlinespace
        Does your organization do security awareness training for all employees at least once per year? & \ding{51} & Satisfactory \\
        \bottomrule
    \end{tabular}
    \caption{Security Control Questionnaire Results.}
    \label{tab:controls}
\end{table}

\newpage

% --- Section 4: Technical Scan Results ---
\section{Technical Scan Results}

A network port scan was performed on the specified target system to identify potentially vulnerable services.

\begin{itemize}
    \item \textbf{Target IP Address:} \seqsplit{\texttt{192.168.1.100}}
    \item \textbf{Scan Summary:} The scan reported the host as "up", but found no open TCP ports. All 1000 scanned ports were in a "closed" state.
\end{itemize}

\subsection{Interpretation}
The results indicate a very strong network security posture for the scanned host. The absence of open ports significantly reduces the network attack surface, suggesting that the system is either not hosting network services or is protected by a well-configured host-based or network firewall. No network-based vulnerabilities were identified on this system during the scan.

% --- Section 5: Overall Risk Assessment ---
\section{Overall Risk Assessment}

This section synthesizes findings from the security control review and technical scan. While no pre-existing risks were documented and no technical vulnerabilities were found, the procedural gaps represent a significant threat. The following new risks have been identified:

\begin{table}[h!]
    \centering
    \begin{tabular}{@{}p{0.1\textwidth} p{0.25\textwidth} p{0.45\textwidth} p{0.1\textwidth}@{}}
        \toprule
        \textbf{Risk ID} & \textbf{Risk Name} & \textbf{Description} & \textbf{Severity} \\
        \midrule
        ORG-001 & Lack of MFA on Email & The absence of MFA on email accounts allows an attacker with valid credentials (e.g., from a phishing attack) to gain full access, leading to data breaches or BEC. & \textcolor{SeverityCritical}{Critical} \\
        \addlinespace
        ORG-002 & No Acceptable Use Policy (AUP) & Without a formal AUP, there are no clear guidelines for employees on the acceptable use of company assets, leading to inconsistent security practices and lack of accountability. & \textcolor{SeverityHigh}{High} \\
        \addlinespace
        ORG-003 & Inadequate New Hire Security Training & New employees are not trained on security policies upon hiring, creating a window of vulnerability until the annual training cycle. They are often prime targets for social engineering. & \textcolor{SeverityHigh}{High} \\
        \bottomrule
    \end{tabular}
    \caption{Summary of Identified Risks.}
    \label{tab:risks}
\end{table}

\newpage

% --- Section 6: Recommendations ---
\section{Recommendations}

Based on the findings of this assessment, we provide the following actionable recommendations to mitigate the identified risks and improve the overall security posture of Cinder & Ash.

\begin{enumerate}
    \item \textbf{[Critical] Implement MFA for Email Access:}
    \begin{itemize}
        \item \textbf{Action:} Immediately enable and enforce Multi-Factor Authentication (MFA) for all user and administrative email accounts.
        \item \textbf{Justification:} This is the single most effective control to prevent unauthorized account access resulting from credential theft or phishing. It directly mitigates the risk of Business Email Compromise (BEC). This should be the highest priority remediation activity.
    \end{itemize}
    \vspace{0.5cm}
    
    \item \textbf{[High] Develop and Implement an Acceptable Use Policy (AUP):}
    \begin{itemize}
        \item \textbf{Action:} Create a formal AUP document that clearly defines the rules and expectations for the use of company networks, systems, and data. This policy should be distributed to all employees, who must formally acknowledge their understanding and agreement.
        \item \textbf{Justification:} An AUP establishes a baseline for secure user behavior, reduces insider threats (both malicious and accidental), and provides a framework for enforcing security standards.
    \end{itemize}
    \vspace{0.5cm}
    
    \item \textbf{[High] Integrate Security Training into Employee Onboarding:}
    \begin{itemize}
        \item \textbf{Action:} Develop a security awareness training module specifically for new hires and make it a mandatory part of the onboarding process. This training should occur within the first week of employment.
        \item \textbf{Justification:} New employees are often more susceptible to social engineering attacks. Immediate training closes the significant gap between their start date and the annual security training, reducing early-stage vulnerabilities.
    \end{itemize}
\end{enumerate}

\end{document}
```