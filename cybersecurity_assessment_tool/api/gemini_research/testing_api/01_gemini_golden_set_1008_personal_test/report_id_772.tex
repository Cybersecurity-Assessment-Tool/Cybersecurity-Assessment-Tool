```latex
\documentclass[12pt]{article}

% Preamble: Required Packages
\usepackage[margin=1in]{geometry}
\usepackage{pifont} % For checkmarks and crosses
\usepackage{booktabs} % For professional tables
\usepackage{hyperref} % For clickable links
\usepackage{url} % For URL formatting
\usepackage{seqsplit} % For splitting long strings to prevent overflow
\usepackage{graphicx}
\usepackage{xcolor}
\usepackage{fancyhdr}

% Document Styling
\hypersetup{
    colorlinks=true,
    linkcolor=blue,
    filecolor=magenta,      
    urlcolor=cyan,
    pdftitle={Cybersecurity Posture Assessment Report},
    pdfpagemode=FullScreen,
}

\pagestyle{fancy}
\fancyhf{}
\fancyhead[L]{\textbf{Cybersecurity Posture Assessment}}
\fancyhead[R]{Swift Current Labs}
\fancyfoot[C]{\thepage}
\renewcommand{\headrulewidth}{0.4pt}
\renewcommand{\footrulewidth}{0.4pt}

% --- DOCUMENT START ---
\begin{document}

% --- TITLE PAGE ---
\begin{titlepage}
    \centering
    \vspace*{2cm}
    
    \Huge{\textbf{Cybersecurity Posture Assessment Report}}
    
    \vspace{1.5cm}
    
    \Large{\textbf{Prepared for:}}
    
    \vspace{0.5cm}
    
    \Large{Swift Current Labs}
    
    \vspace{3cm}
    
    \includegraphics[width=0.3\textwidth]{example-image-a} % Placeholder for a logo
    
    \vfill
    
    \large{\textbf{Date of Report:}}
    
    \large{\today}
    
\end{titlepage}

\tableofcontents
\newpage

% --- EXECUTIVE SUMMARY ---
\section{Executive Summary}
This report provides a comprehensive analysis of the cybersecurity posture for Swift Current Labs, based on a combination of self-reported organizational controls, an external network scan, and a review of existing risks.

The assessment reveals a mixed security posture. On a positive note, the external network scan of the target host \texttt{192.168.1.100} found no open ports, indicating a strong perimeter defense for that specific asset. The organization also has some foundational policies in place, such as an acceptable use policy and annual security training for all employees.

However, several critical and high-risk security gaps were identified through the security controls questionnaire. The most significant concerns are the lack of multi-factor authentication (MFA) for email and computer access, and the absence of mandatory security awareness training for new employees. These gaps expose the organization to significant risks, including account takeovers, unauthorized access to internal systems, and increased susceptibility to social engineering attacks.

This document details these findings and provides prioritized, actionable recommendations to mitigate the identified risks and strengthen the overall security framework of Swift Current Labs.

% --- ORGANIZATIONAL INFORMATION ---
\section{Organizational Information}
The following details were provided for the assessment. This information is used to establish the context and scope of the review.

\begin{table}[h!]
\centering
\begin{tabular}{@{}ll@{}}
\toprule
\textbf{Attribute} & \textbf{Value} \\ \midrule
Organization Name & Swift Current Labs \\
Email Domain & \texttt{SwiftCurrentLabs.com} \\
Website Domain & \url{www.SwiftCurrentLabs.com} \\
External IP Address & \texttt{2.21.124.226} \\ \bottomrule
\end{tabular}
\caption{Provided Organizational Data.}
\end{table}

% --- SECURITY CONTROL REVIEW ---
\section{Security Control Review}
A review of self-reported security controls was conducted via a questionnaire. The responses highlight key areas of strength and weakness in the organization's current security policies and procedures. A (\textcolor{green}{\ding{51}}) indicates a positive control is in place, while a (\textcolor{red}{\ding{55}}) indicates a security gap.

\begin{table}[h!]
\centering
\begin{tabular}{@{}lc@{}}
\toprule
\textbf{Security Control Question} & \textbf{Status} \\ \midrule
Do you require MFA to access email? & \textcolor{red}{\ding{55}} \\
Do you require MFA to log into computers? & \textcolor{red}{\ding{55}} \\
Do you require MFA to access sensitive data systems? & \textcolor{green}{\ding{51}} \\
Does your organization have an employee acceptable use policy? & \textcolor{green}{\ding{51}} \\
Does your organization do security awareness training for new employees? & \textcolor{red}{\ding{55}} \\
Does your organization do security awareness training for all employees annually? & \textcolor{green}{\ding{51}} \\ \bottomrule
\end{tabular}
\caption{Security Controls Questionnaire Results.}
\end{table}

\subsection*{Analysis of Controls}
The "No" responses to critical questions regarding Multi-Factor Authentication (MFA) and new employee training represent significant vulnerabilities. Email and workstation access are primary vectors for attackers, and the lack of MFA greatly increases the risk of a successful breach via compromised credentials. Furthermore, failing to train new employees on security best practices from day one makes them prime targets for phishing and other social engineering attacks.

% --- TECHNICAL SCAN RESULTS ---
\section{Technical Scan Results}
An external network scan was performed to identify accessible services and potential vulnerabilities on the organization's perimeter.

\begin{itemize}
    \item \textbf{Scan Target:} \texttt{192.168.1.100}
    \item \textbf{Scan Date:} \today
\end{itemize}

\subsection*{Findings}
The scan results were positive, indicating a strong network security posture for the targeted host.
\begin{itemize}
    \item \textbf{Open Ports:} None detected.
    \item \textbf{Port State:} All 1000 scanned ports were reported as being in a \texttt{closed} state. This means that while the host is online, it is not accepting connections on any of the commonly used ports, significantly reducing its external attack surface.
\end{itemize}

% --- CONSOLIDATED RISK ASSESSMENT ---
\section{Consolidated Risk Assessment}
This section synthesizes findings from the security control review, technical scan, and pre-existing risk data. As no pre-existing risks were provided, the following risks are derived directly from this assessment.

\begin{table}[h!]
\centering
\begin{tabular}{@{}p{0.1\linewidth} p{0.3\linewidth} p{0.15\linewidth} p{0.35\linewidth}@{}}
\toprule
\textbf{Risk ID} & \textbf{Risk Name} & \textbf{Severity} & \textbf{Overview} \\ \midrule
\textbf{R-01} & Lack of MFA for Email Access & \textbf{\textcolor{red}{Critical}} & Email accounts are a primary target for phishing and account takeover. Without MFA, a compromised password provides an attacker with full access, which can be leveraged to reset other passwords, access sensitive data, and launch further attacks. \\
\addlinespace
\textbf{R-02} & Lack of MFA for Workstation Access & \textbf{\textcolor{orange}{High}} & If an employee's credentials are stolen, an attacker could log directly into their computer, gaining access to the internal network. This facilitates lateral movement, data exfiltration, and ransomware deployment. \\
\addlinespace
\textbf{R-03} & No Security Training for New Employees & \textbf{\textcolor{orange}{High}} & New hires are often unfamiliar with corporate security policies and are highly susceptible to social engineering. This gap creates a persistent weak link in the organization's human firewall. \\ \bottomrule
\end{tabular}
\caption{Summary of Identified Risks.}
\end{table}

% --- RECOMMENDATIONS ---
\section{Recommendations}
The following actions are recommended to address the identified risks. Recommendations are prioritized based on severity.

\begin{enumerate}
    \item \textbf{[Critical] Implement MFA for Email System:}
    \begin{itemize}
        \item \textbf{Action:} Immediately enforce MFA for all users accessing the \texttt{SwiftCurrentLabs.com} email system.
        \item \textbf{Justification:} This is the single most effective control to prevent email account takeovers, which are a common entry point for major security breaches. This directly mitigates risk \textbf{R-01}.
    \end{itemize}
    \vspace{0.5cm}
    
    \item \textbf{[High] Deploy MFA for Endpoint Authentication:}
    \begin{itemize}
        \item \textbf{Action:} Roll out a mandatory MFA requirement for all employee computer and laptop logins.
        \item \textbf{Justification:} This measure prevents unauthorized access to company devices and the internal network, even if user credentials are stolen. This directly mitigates risk \textbf{R-02}.
    \end{itemize}
    \vspace{0.5cm}
    
    \item \textbf{[High] Integrate Security into Employee Onboarding:}
    \begin{itemize}
        \item \textbf{Action:} Develop a mandatory security awareness training module for all new employees. This training should be completed within their first week of employment and cover topics such as phishing, password security, and the acceptable use policy.
        \item \textbf{Justification:} Educating employees from the start establishes a culture of security and reduces the likelihood of human error leading to a breach. This directly mitigates risk \textbf{R-03}.
    \end{itemize}
\end{enumerate}

% --- DOCUMENT END ---
\end{document}
```