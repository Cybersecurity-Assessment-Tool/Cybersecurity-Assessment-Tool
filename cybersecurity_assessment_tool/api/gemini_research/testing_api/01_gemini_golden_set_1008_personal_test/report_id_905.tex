```latex
\documentclass[12pt]{article}

% Preamble: Required Packages and Document Setup
\usepackage[margin=1in]{geometry}
\usepackage{pifont} % For checkmarks and crosses
\usepackage{booktabs} % For professional tables
\usepackage{hyperref} % For clickable links and TOC
\usepackage{url} % For URL formatting
\usepackage{seqsplit} % To split long strings in texttt
\usepackage{xcolor} % For colored text
\usepackage{graphicx} % For logo (placeholder)

% Define colors for risk levels
\definecolor{criticalred}{HTML}{D7263D}
\definecolor{highorange}{HTML}{F49D40}
\definecolor{mediumyellow}{HTML}{F4D440}
\definecolor{lowblue}{HTML}{5486F3}
\definecolor{infogray}{HTML}{808080}

% Hyperref Setup
\hypersetup{
    colorlinks=true,
    linkcolor=blue,
    filecolor=magenta,      
    urlcolor=cyan,
    pdftitle={Cybersecurity Risk Assessment Report},
    pdfpagemode=FullScreen,
}

% --- Document Start ---
\begin{document}

% --- Title Page ---
\begin{titlepage}
    \centering
    \vfill
    {\Huge\bfseries Cybersecurity Risk Assessment Report\par}
    \vspace{1.5cm}
    {\Large Prepared for:\par}
    \vspace{0.5cm}
    {\Huge\bfseries Grizzly Peak\par}
    \vspace{2cm}
    {\large \today\par}
    \vfill
    {\large Generated by: Cybersecurity Analysis Engine\par}
\end{titlepage}

\tableofcontents
\newpage

% --- Section 1: Executive Summary ---
\section{Executive Summary}
This report provides a comprehensive cybersecurity assessment for \textbf{Grizzly Peak}, synthesizing data from network scans, organizational questionnaires, and pre-existing risk registers. The analysis reveals several critical and high-risk vulnerabilities that require immediate attention to mitigate potential security breaches.

The most significant findings include a systemic issue of exposed Remote Desktop Protocol (RDP) services on internal servers, a critical lack of Multi-Factor Authentication (MFA) on employee email accounts, and a complete absence of a security awareness training program. The combination of these vulnerabilities creates a high-likelihood attack path where a compromised email account could lead directly to internal network access.

Immediate remediation should focus on securing remote access points, enforcing MFA across all critical services, and implementing a foundational security awareness program for all employees.

% --- Section 2: Organizational Information ---
\section{Organizational Information}
The following information was provided for the assessment scope.

\begin{tabular}{@{}ll}
    \toprule
    \textbf{Attribute} & \textbf{Value} \\
    \midrule
    Organization Name & \textbf{Grizzly Peak} \\
    Email Domain & \seqsplit{\texttt{GrizzlyPeak.com}} \\
    Website Domain & \seqsplit{\url{www.GrizzlyPeak.com}} \\
    External IP Address & \seqsplit{\texttt{179.173.165.152}} \\
    \bottomrule
\end{tabular}

% --- Section 3: Security Control Review ---
\section{Security Control Review}
The following table summarizes the organization's responses to a security controls questionnaire. Items marked with \textcolor{red}{\ding{55}} represent significant gaps in the current security posture.

\begin{tabular}{@{}p{0.6\linewidth}cp{0.2\linewidth}@{}}
    \toprule
    \textbf{Control Question} & \textbf{Response} & \textbf{Assessment} \\
    \midrule
    Do you require MFA to access email? & \textcolor{red}{\ding{55}} & \textcolor{criticalred}{\textbf{Critical Gap}} \\
    Do you require MFA to log into computers? & \textcolor{green}{\ding{51}} & Good Practice \\
    Do you require MFA to access sensitive data systems? & \textcolor{green}{\ding{51}} & Good Practice \\
    Does your organization have an employee acceptable use policy? & \textcolor{green}{\ding{51}} & Good Practice \\
    Does your organization do security awareness training for new employees? & \textcolor{red}{\ding{55}} & \textcolor{highorange}{\textbf{High Risk}} \\
    Does your organization do security awareness training for all employees at least once per year? & \textcolor{red}{\ding{55}} & \textcolor{highorange}{\textbf{High Risk}} \\
    \bottomrule
\end{tabular}

\subsection*{Analysis of Control Gaps}
\begin{itemize}
    \item \textbf{MFA for Email:} The absence of MFA on email is a critical vulnerability. Email accounts are primary targets for phishing attacks. A compromised email account can lead to data breaches, financial fraud, and further infiltration of the network.
    \item \textbf{Security Awareness Training:} The lack of any security awareness training program leaves the organization highly susceptible to social engineering and phishing attacks. Employees are the first line of defense, and without training, they are unprepared to identify and report threats.
\end{itemize}

% --- Section 4: Technical Scan Results ---
\section{Technical Scan Results}
An external network scan was performed to identify exposed services and potential vulnerabilities.

\begin{itemize}
    \item \textbf{Target IP Address:} \seqsplit{\texttt{10.10.10.51}}
\end{itemize}

\subsection*{Open Ports and Services}
The following table details the services discovered on the target system.

\begin{tabular}{@{}llll@{}}
    \toprule
    \textbf{Port} & \textbf{State} & \textbf{Service Name} & \textbf{Analyst Notes} \\
    \midrule
    3389/tcp & open & \texttt{ms-wbt-server} & \textcolor{criticalred}{\textbf{Critical Risk}}. This is the Remote Desktop Protocol (RDP). \\
    & & & Exposing RDP directly to the internet is extremely dangerous and makes \\
    & & & the server a prime target for brute-force attacks and exploitation of \\
    & & & known vulnerabilities (e.g., BlueKeep). \\
    \bottomrule
\end{tabular}

% --- Section 5: Correlated Risk Assessment ---
\section{Correlated Risk Assessment}
This section synthesizes findings from the security questionnaire, technical scans, and the existing risk register to provide a holistic view of the organization's risk posture.

\begin{tabular}{@{}p{0.1\linewidth}p{0.2\linewidth}p{0.4\linewidth}p{0.15\linewidth}@{}}
    \toprule
    \textbf{Risk ID} & \textbf{Risk Name} & \textbf{Description} & \textbf{Severity} \\
    \midrule
    RISK-001 & \textbf{Systemic RDP Exposure} & The pre-existing risk register noted RDP exposure on \texttt{10.10.10.50}. The new scan confirms another host, \texttt{10.10.10.51}, is also exposed. This indicates a systemic configuration issue, not an isolated incident. & \textcolor{criticalred}{\textbf{Critical}} \\
    \addlinespace
    RISK-002 & \textbf{Lack of MFA on Email} & Email accounts are not protected by Multi-Factor Authentication. This significantly increases the risk of account takeover via phishing or credential stuffing, which could serve as an entry point for attackers. & \textcolor{criticalred}{\textbf{Critical}} \\
    \addlinespace
    RISK-003 & \textbf{Inadequate Security Awareness Program} & The organization has no security awareness training for new or existing employees. This makes the success of phishing attacks (targeting email from RISK-002) much more likely. & \textcolor{highorange}{\textbf{High}} \\
    \bottomrule
\end{tabular}

% --- Section 6: Recommendations ---
\section{Recommendations}
The following actions are recommended to address the identified risks. Recommendations are prioritized based on severity.

\subsection*{RISK-001: Systemic RDP Exposure (Critical)}
\begin{itemize}
    \item \textbf{Immediate Action:} Implement firewall rules to block all inbound traffic to TCP port 3389 on all external-facing systems, including \texttt{10.10.10.50} and \texttt{10.10.10.51}.
    \item \textbf{Long-Term Solution:} For necessary remote access, deploy a Virtual Private Network (VPN) solution. Access to the internal network should only be granted after a user successfully authenticates to the VPN, which should itself be protected with Multi-Factor Authentication.
\end{itemize}

\subsection*{RISK-002: Lack of MFA on Email (Critical)}
\begin{itemize}
    \item \textbf{Immediate Action:} Enable and enforce MFA for all user accounts on the \texttt{GrizzlyPeak.com} email platform. This is the single most effective control to prevent email account takeovers.
    \item \textbf{Policy Update:} Update the information security policy to mandate MFA on all cloud services and critical systems.
\end{itemize}

\subsection*{RISK-003: Inadequate Security Awareness Program (High)}
\begin{itemize}
    \item \textbf{Immediate Action:} Enroll all employees in a foundational security awareness training course that covers phishing, password hygiene, and acceptable use.
    \item \textbf{Long-Term Solution:} Establish a continuous security training program. This should include mandatory annual training for all staff, targeted training for new hires during onboarding, and regular phishing simulation campaigns to test and reinforce learning.
\end{itemize}

\end{document}
```