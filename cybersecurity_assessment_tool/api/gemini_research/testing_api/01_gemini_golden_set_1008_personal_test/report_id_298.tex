```latex
\documentclass[12pt]{article}

% Preamble: Required Packages
\usepackage[margin=1in]{geometry}
\usepackage{pifont} % For checkmarks and crosses
\usepackage{booktabs} % For professional tables
\usepackage{hyperref} % For clickable links
\usepackage{url} % For URL formatting
\usepackage{seqsplit} % For splitting long strings in tt font
\usepackage{graphicx}
\usepackage{xcolor}

% --- Document Setup ---
\hypersetup{
    colorlinks=true,
    linkcolor=blue,
    filecolor=magenta,      
    urlcolor=cyan,
    pdftitle={Cybersecurity Posture Report},
    pdfpagemode=FullScreen,
}

\newcommand{\yes}{\ding{51}}
\newcommand{\no}{\ding{55}}

% --- Document Begins ---
\begin{document}

% --- Title Page ---
\begin{titlepage}
    \centering
    \vspace*{1cm}
    \Huge
    \textbf{Cybersecurity Posture Report}
    
    \vspace{1.5cm}
    \Large
    Prepared for: \textbf{Open Door}
    
    \vspace{2cm}
    \large
    Date of Report: \today
    
    \vfill
    
    \large
    \textit{This report contains sensitive information and should be handled with care. Distribution is restricted to authorized personnel only.}
    
\end{titlepage}

\tableofcontents
\newpage

% --- Section 1: Executive Summary ---
\section{Executive Summary}
This report provides a comprehensive analysis of the cybersecurity posture for \textbf{Open Door}, based on a review of organizational security controls, an external network scan, and an assessment of current risks. The assessment was conducted on \today.

The analysis reveals a mixed security posture. The organization has implemented some positive security controls, such as requiring multi-factor authentication (MFA) for computer logins and conducting annual security awareness training for all staff.

However, several critical gaps were identified that expose the organization to significant risk. These include the absence of MFA for email and sensitive data systems, which dramatically increases the risk of account compromise and data breaches. Furthermore, the lack of an employee acceptable use policy and security training for new hires creates foundational weaknesses in the security program.

The external network scan did not identify any exposed services on the target system, which suggests a well-configured firewall or a lack of public-facing services at that address. While this is a positive finding from a network perspective, the identified policy and access control gaps must be addressed urgently. This report outlines prioritized, actionable recommendations to mitigate these risks and strengthen the overall security posture.

\newpage

% --- Section 2: Organizational Information ---
\section{Organizational Information}
The following details were provided for the assessment.

\begin{tabular}{@{}ll}
    \toprule
    \textbf{Attribute} & \textbf{Value} \\
    \midrule
    Organization Name & \textbf{Open Door} \\
    Email Domain & \texttt{OpenDoor.net} \\
    Website Domain & \url{www.OpenDoor.net} \\
    External IP Address & \texttt{174.100.139.94} \\
    \bottomrule
\end{tabular}

% --- Section 3: Security Control Review ---
\section{Security Control Review}
A questionnaire was completed to evaluate the implementation of key administrative and technical security controls. The responses are summarized below. Gaps (\no) indicate a deviation from security best practices and represent a potential risk.

\begin{tabular}{@{}p{0.6\textwidth}p{0.15\textwidth}p{0.15\textwidth}@{}}
    \toprule
    \textbf{Control Question} & \textbf{Best Practice} & \textbf{Response} \\
    \midrule
    Do you require MFA to access email? & Yes & \textcolor{red}{\no} \\
    Do you require MFA to log into computers? & Yes & \textcolor{green}{\yes} \\
    Do you require MFA to access sensitive data systems? & Yes & \textcolor{red}{\no} \\
    Does your organization have an employee acceptable use policy? & Yes & \textcolor{red}{\no} \\
    Does your organization do security awareness training for new employees? & Yes & \textcolor{red}{\no} \\
    Does your organization do security awareness training for all employees at least once per year? & Yes & \textcolor{green}{\yes} \\
    \bottomrule
\end{tabular}

\subsection{Analysis of Gaps}
The review identified four significant control gaps:
\begin{itemize}
    \item \textbf{No MFA for Email:} Email is a primary target for phishing and account takeover attacks. The lack of MFA makes user credentials the single point of failure.
    \item \textbf{No MFA for Sensitive Data:} Critical business and client data are at high risk of unauthorized access and exfiltration without the protection of MFA.
    \item \textbf{No Acceptable Use Policy (AUP):} An AUP is a foundational document that sets expectations for employee behavior and the use of company assets, reducing insider risk.
    \item \textbf{No Security Training for New Hires:} New employees are often targeted by attackers. Failing to provide immediate security training leaves a critical window of vulnerability.
\end{itemize}

% --- Section 4: Technical Scan Results ---
\section{Technical Scan Results}
An external network scan was performed to identify exposed services and potential vulnerabilities.

\begin{tabular}{@{}ll}
    \toprule
    \textbf{Scan Parameter} & \textbf{Value} \\
    \midrule
    Target IP Address & \texttt{[Target IP]} \\
    Scan Date & \today \\
    \bottomrule
\end{tabular}

\subsection{Findings}
The scan completed successfully but did not detect any open TCP or UDP ports on the target host \texttt{[Target IP]}.

\textbf{Conclusion:} This result indicates that there are no publicly accessible services running on this IP address, or that a firewall is in place that effectively blocks all unsolicited inbound traffic. From an external network security perspective, this is a strong defensive posture, as it significantly reduces the attack surface. No further technical vulnerabilities were identified from this scan.

% --- Section 5: Risk Assessment ---
\section{Risk Assessment}
This section synthesizes findings from the security control review and the technical scan. While no pre-existing vulnerabilities were reported and the network scan was clean, the administrative and access control gaps identified in the questionnaire present a clear and immediate danger to the organization.

\begin{tabular}{@{}p{0.25\textwidth}p{0.55\textwidth}p{0.1\textwidth}@{}}
    \toprule
    \textbf{Risk Name} & \textbf{Overview} & \textbf{Severity} \\
    \midrule
    Email Account Compromise & The absence of MFA on email accounts (\texttt{OpenDoor.net}) makes them highly susceptible to takeover via phishing or credential stuffing. A compromised email account can lead to data breaches, financial fraud, and further attacks on partners and clients. & \textbf{Critical} \\
    \addlinespace
    Sensitive Data Breach & Critical systems lack MFA protection, allowing an attacker with stolen credentials to gain direct access to sensitive organizational or client data, potentially leading to severe financial and reputational damage. & \textbf{Critical} \\
    \addlinespace
    Lack of Foundational Policies & Without an Acceptable Use Policy, the organization lacks a formal mechanism to govern technology use, define prohibited activities, and enforce security standards. This increases the risk of insider threats and inconsistent security practices. & \textbf{High} \\
    \addlinespace
    Inadequate Employee Onboarding & New employees are not receiving security awareness training upon being hired. This makes them significantly more vulnerable to social engineering and phishing attacks during their initial, and most vulnerable, period of employment. & \textbf{High} \\
    \bottomrule
\end{tabular}

% --- Section 6: Recommendations ---
\section{Recommendations}
The following actions are recommended to mitigate the identified risks. They are prioritized based on severity and potential impact.

\begin{enumerate}
    \item \textbf{Implement Multi-Factor Authentication (Critical):}
    \begin{itemize}
        \item \textbf{Action:} Immediately enable and enforce MFA for all user access to the email system.
        \item \textbf{Action:} Identify all systems containing sensitive data and enforce MFA for all privileged and standard user access.
        \item \textbf{Justification:} This is the single most effective control to prevent unauthorized access and mitigate the risk of credential-based attacks.
    \end{itemize}
    
    \item \textbf{Develop and Implement an Acceptable Use Policy (AUP) (High):}
    \begin{itemize}
        \item \textbf{Action:} Draft an AUP that clearly defines the rules for using company networks, devices, and data. Require all employees to read and acknowledge the policy.
        \item \textbf{Justification:} An AUP establishes a baseline for secure behavior, reduces ambiguity, and provides a basis for disciplinary action if violated.
    \end{itemize}
    
    \item \textbf{Integrate Security Training into Onboarding (High):}
    \begin{itemize}
        \item \textbf{Action:} Create a mandatory security awareness training module for all new hires to be completed during their first week of employment.
        \item \textbf{Justification:} This ensures that all employees, from day one, understand their security responsibilities and can recognize common threats like phishing.
    \end{itemize}
\end{enumerate}

\end{document}
```