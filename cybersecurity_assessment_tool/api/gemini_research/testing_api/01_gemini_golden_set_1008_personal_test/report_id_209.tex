```latex
\documentclass[12pt]{article}

% --- PACKAGES ---
\usepackage[margin=1in]{geometry}
\usepackage{pifont} % For checkmarks and crosses
\usepackage{booktabs} % For professional tables
\usepackage{hyperref} % For clickable links
\usepackage{url} % For formatting URLs
\usepackage{seqsplit} % For splitting long text strings
\usepackage{graphicx} % For logo
\usepackage{xcolor} % For colors

% --- DOCUMENT METADATA ---
\title{Cybersecurity Posture Assessment Report}
\author{Cybersecurity Analysis Division}
\date{\today}

% --- HYPERREF SETUP ---
\hypersetup{
    colorlinks=true,
    linkcolor=blue,
    filecolor=magenta,      
    urlcolor=cyan,
    pdftitle={Cybersecurity Posture Assessment Report},
    pdfpagemode=FullScreen,
}

% --- CUSTOM COMMANDS ---
\newcommand{\yes}{\ding{51}}
\newcommand{\no}{\ding{55}}
\newcommand{\orgname}{Vanguard Heritage}
\newcommand{\orgdomain}{\texttt{VanguardHeritage.org}}
\newcommand{\orgip}{\texttt{181.181.43.137}}

\begin{document}

\maketitle
\thispagestyle{empty}
\newpage

\tableofcontents
\newpage

% ===================================================================
% 1. EXECUTIVE SUMMARY
% ===================================================================
\section{Executive Summary}

This report provides a cybersecurity assessment for \textbf{\orgname}, synthesizing data from technical network scans, a security controls questionnaire, and a review of pre-existing risks. The analysis reveals critical and high-risk security gaps that require immediate attention.

Two primary findings stand out:
\begin{enumerate}
    \item \textbf{Systemic Remote Desktop Protocol (RDP) Exposure:} A network scan identified an additional server (\texttt{10.10.10.51}) with RDP (port 3389) open to the internal network. This finding, correlated with a pre-existing risk on another host, points to a systemic configuration issue that significantly increases the attack surface for ransomware and unauthorized lateral movement.
    \item \textbf{Critical Gap in Multi-Factor Authentication (MFA):} The organization does not enforce MFA for accessing sensitive data systems. This policy gap, combined with the exposed RDP services, creates a high-risk scenario where a single compromised credential could lead to a significant data breach.
\end{enumerate}

Immediate remediation of the exposed RDP service and the prompt implementation of an MFA policy for all sensitive systems are strongly recommended to mitigate these risks.

% ===================================================================
% 2. ORGANIZATIONAL INFORMATION
% ===================================================================
\section{Organizational Information}

The following details were provided for the assessment. This information is used to establish the context and scope of the review.

\begin{tabular}{@{}ll}
    \toprule
    \textbf{Attribute} & \textbf{Value} \\
    \midrule
    Organization Name & \orgname \\
    Email Domain & \orgdomain \\
    Website Domain & \texttt{www.VanguardHeritage.org} \\
    External IP Address & \orgip \\
    \bottomrule
\end{tabular}

% ===================================================================
% 3. SECURITY CONTROL REVIEW
% ===================================================================
\section{Security Control Review}

A security controls questionnaire was completed to evaluate the organization's policies and procedures. The results are summarized below. A red cross (\no) indicates a potential security gap that requires further investigation and remediation.

\begin{table}[h!]
\centering
\begin{tabular}{@{}lc}
    \toprule
    \textbf{Security Control Question} & \textbf{Response} \\
    \midrule
    Do you require MFA to access email? & \yes \\
    Do you require MFA to log into computers? & \yes \\
    \color{red}Do you require MFA to access sensitive data systems? & \color{red}\no \\
    Does your organization have an employee acceptable use policy? & \yes \\
    Does your organization do security awareness training for new employees? & \yes \\
    Does your organization do security awareness training for all employees annually? & \yes \\
    \bottomrule
\end{tabular}
\caption{Security Controls Questionnaire Results.}
\label{tab:controls}
\end{table}

\subsection*{Analysis}
The organization has implemented several key security controls, including MFA for email and computer logins, and maintains a consistent security awareness training program. However, the lack of MFA for sensitive data systems is a \textbf{critical control failure}. This gap undermines the principle of least privilege and defense-in-depth, leaving high-value assets vulnerable to credential-based attacks.

% ===================================================================
% 4. TECHNICAL SCAN RESULTS
% ===================================================================
\section{Technical Scan Results}

An Nmap scan was conducted to identify open ports and running services on the target system.

\begin{itemize}
    \item \textbf{Target IP Address:} \texttt{10.10.10.51}
    \item \textbf{Scan Status:} Host is up.
\end{itemize}

\begin{table}[h!]
\centering
\begin{tabular}{@{}llll}
    \toprule
    \textbf{Port} & \textbf{State} & \textbf{Service Name} & \textbf{Description} \\
    \midrule
    3389/tcp & Open & \texttt{ms-wbt-server} & Microsoft Remote Desktop Protocol (RDP) \\
    \bottomrule
\end{tabular}
\caption{Open Ports Detected on \texttt{10.10.10.51}.}
\label{tab:scan}
\end{table}

\subsection*{Analysis}
The scan confirms that port 3389 is open, indicating that the Remote Desktop Protocol (RDP) service is active and accessible on the network. Unsecured or unnecessarily exposed RDP is a primary vector for ransomware attacks and unauthorized access. This finding is highly significant, especially when correlated with the pre-existing risk profile.

% ===================================================================
% 5. CONSOLIDATED RISK ASSESSMENT
% ===================================================================
\section{Consolidated Risk Assessment}

The following table synthesizes findings from the security questionnaire, technical scans, and pre-existing risk data to provide a holistic view of the current risk posture.

\begin{table}[h!]
\centering
\begin{tabular}{@{}p{0.25\linewidth}p{0.45\linewidth}p{0.1\linewidth}p{0.15\linewidth}}
    \toprule
    \textbf{Risk Name} & \textbf{Description} & \textbf{Severity} & \textbf{Affected Systems} \\
    \midrule
    \color{red}\textbf{Lack of MFA for Sensitive Systems} & The absence of a mandatory MFA policy for sensitive data systems allows an attacker with valid credentials to gain direct access to critical assets. & \textbf{High} & All sensitive data systems \\
    \addlinespace
    \color{red}\textbf{Systemic RDP Exposure} & RDP is exposed on multiple internal hosts (\texttt{10.10.10.50}, \texttt{10.10.10.51}), indicating a pattern of insecure configuration. This increases the risk of lateral movement and ransomware deployment. & \textbf{Critical (9.0)} & \texttt{10.10.10.50}, \texttt{10.10.10.51} \\
    \bottomrule
\end{tabular}
\caption{Summary of Identified Risks.}
\label{tab:risks}
\end{table}

% ===================================================================
% 6. RECOMMENDATIONS
% ===================================================================
\section{Recommendations}

Based on the analysis, the following actions are recommended to mitigate the identified risks. Recommendations are prioritized by urgency.

\subsection*{Immediate (0-7 Days)}
\begin{enumerate}
    \item \textbf{Remediate RDP on \texttt{10.10.10.51}:}
    \begin{itemize}
        \item If RDP access is not required, disable the service entirely.
        \item If access is required, restrict access to the port using a host-based or network firewall, allowing connections only from authorized jump boxes or administrative workstations.
    \end{itemize}
\end{enumerate}

\subsection*{High Priority (1-4 Weeks)}
\begin{enumerate}
    \item \textbf{Implement MFA for Sensitive Systems:}
    \begin{itemize}
        \item Immediately begin a project to enforce MFA on all systems classified as containing sensitive data. This is the single most effective control to mitigate the identified policy gap.
    \end{itemize}
    \item \textbf{Conduct Internal RDP Audit:}
    \begin{itemize}
        \item Perform a comprehensive scan of the internal network to identify all systems with RDP enabled.
        \item Document the business justification for each instance and remediate all non-essential exposures.
    \end{itemize}
\end{enumerate}

\subsection*{Strategic (1-3 Months)}
\begin{enumerate}
    \item \textbf{Implement Secure Remote Access Solution:}
    \begin{itemize}
        \item Transition all administrative remote access to a centralized, secure solution such as a VPN gateway or a privileged access management (PAM) system.
        \item This solution must enforce MFA for all connections, providing a robust, audited, and controlled access method.
    \end{itemize}
\end{enumerate}

\end{document}
```