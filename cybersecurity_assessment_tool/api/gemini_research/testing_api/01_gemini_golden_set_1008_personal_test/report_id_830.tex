```latex
\documentclass[12pt]{article}

% Preamble: Required Packages
\usepackage[margin=1in]{geometry}
\usepackage{pifont} % For checkmarks and crosses
\usepackage{booktabs} % For professional tables
\usepackage[hidelinks]{hyperref} % For clickable links without boxes
\usepackage{url} % For URL formatting
\usepackage{seqsplit} % For splitting long strings in tt font
\usepackage{graphicx}
\usepackage{xcolor}
\usepackage{longtable} % For tables that can span multiple pages

% Define colors for severity
\definecolor{sev_critical}{HTML}{990000}
\definecolor{sev_high}{HTML}{DD0000}
\definecolor{sev_medium}{HTML}{FF8C00}
\definecolor{sev_low}{HTML}{F8DE7E}
\definecolor{sev_info}{HTML}{87CEEB}

% Document Title and Author
\title{Cybersecurity Assessment Report \\ \large For: \textbf{Blue Horizon Initiative}}
\author{Cybersecurity Analysis Division}
\date{\today}

\begin{document}

\maketitle
\thispagestyle{empty}
\newpage

\tableofcontents
\newpage

\section{Executive Summary}

This report details the findings of a cybersecurity assessment conducted for \textbf{Blue Horizon Initiative}. The assessment incorporated a review of organizational policies, an analysis of pre-existing risks, and a technical network scan.

The overall security posture reveals several areas of significant concern that require immediate attention. A critical vulnerability was identified on an internal network host (\texttt{10.0.0.15}), which is running a dangerously outdated and misconfigured FTP service (vsftpd 2.3.4) with a known remote code execution vulnerability.

Furthermore, significant gaps exist in administrative and policy-based controls. The absence of mandatory Multi-Factor Authentication (MFA) for sensitive data systems, the lack of an employee acceptable use policy, and a complete absence of a security awareness training program represent high-risk deficiencies. These gaps create a permissive environment for security incidents, from insider threats to successful phishing attacks.

The pre-existing risk of outdated Windows 7 workstations further compounds the organization's exposure. Immediate and decisive action is required to remediate the critical technical vulnerability and to implement the foundational security controls outlined in this report.

\section{Organizational Information}

The following details were provided for the assessment.

\begin{tabular}{@{}ll}
\toprule
\textbf{Attribute} & \textbf{Value} \\
\midrule
Organization Name & \textbf{Blue Horizon Initiative} \\
Email Domain & \texttt{BlueHorizonInitiative.com} \\
Website Domain & \texttt{www.BlueHorizonInitiative.com} \\
External IP Address & \texttt{134.40.224.164} \\
\bottomrule
\end{tabular}

\section{Security Control Review}

A review of administrative security controls was conducted based on a standardized questionnaire. The responses indicate critical gaps in governance and employee-facing security policies. The symbol \ding{51} denotes a "Yes" response (control in place), while \ding{55} denotes a "No" response (control gap).

\begin{tabular}{@{}p{0.7\textwidth}c}
\toprule
\textbf{Control Question} & \textbf{Response} \\
\midrule
Do you require MFA to access email? & \ding{51} \\
Do you require MFA to log into computers? & \ding{51} \\
Do you require MFA to access sensitive data systems? & \textcolor{red}{\ding{55}} \\
Does your organization have an employee acceptable use policy? & \textcolor{red}{\ding{55}} \\
Does your organization do security awareness training for new employees? & \textcolor{red}{\ding{55}} \\
Does your organization do security awareness training for all employees at least once per year? & \textcolor{red}{\ding{55}} \\
\bottomrule
\end{tabular}

\subsection*{Analysis of Control Gaps}
The lack of MFA for sensitive data systems, coupled with the absence of an acceptable use policy and any form of security awareness training, constitutes a severe risk. This combination leaves the organization highly vulnerable to credential theft, insider threat, and social engineering attacks.

\section{Technical Scan Results}

A network scan was performed to identify active services and potential vulnerabilities on the specified target system.

\begin{itemize}
    \item \textbf{Target IP Address:} \texttt{10.0.0.15}
    \item \textbf{Scan Engine:} Nmap
\end{itemize}

\begin{longtable}{@{}p{0.1\textwidth}p{0.1\textwidth}p{0.15\textwidth}p{0.2\textwidth}p{0.35\textwidth}@{}}
\toprule
\textbf{Port} & \textbf{State} & \textbf{Service} & \textbf{Version} & \textbf{Analyst Notes} \\
\midrule
\endhead % End of header for subsequent pages
21/tcp & Open & ftp & vsftpd 2.3.4 & \textbf{Critical Finding.} Anonymous FTP login is allowed. This specific version is known to contain a critical backdoor vulnerability (CVE-2011-2523) that allows for remote command execution. \\
\bottomrule
\caption{Open Ports and Services on \texttt{10.0.0.15}}
\label{tab:scan_results}
\end{longtable}

\section{Consolidated Risk Assessment}

The following table synthesizes findings from the security control review, technical scan, and pre-existing risk data into a prioritized list.

\begin{longtable}{@{}p{0.15\textwidth}p{0.15\textwidth}p{0.7\textwidth}@{}}
\toprule
\textbf{Risk ID} & \textbf{Severity} & \textbf{Description} \\
\midrule
\endhead
RISK-001 & \textbf{\textcolor{sev_critical}{Critical}} & \textbf{Vulnerable FTP Server:} The host at \texttt{10.0.0.15} is running vsftpd 2.3.4, which is vulnerable to CVE-2011-2523, a backdoor allowing remote code execution. The service is further misconfigured to allow anonymous logins, removing the need for authentication to exploit it. \\
\midrule
RISK-002 & \textbf{\textcolor{sev_high}{High}} & \textbf{No MFA on Sensitive Systems:} Lack of Multi-Factor Authentication on systems containing sensitive data exposes critical assets to compromise via stolen credentials. \\
\midrule
RISK-003 & \textbf{\textcolor{sev_high}{High}} & \textbf{No Security Awareness Training:} The absence of a training program makes employees highly susceptible to phishing, social engineering, and other common attack vectors. This elevates the risk of initial compromise significantly. \\
\midrule
RISK-004 & \textbf{\textcolor{sev_high}{High}} & \textbf{No Acceptable Use Policy (AUP):} Without a formal AUP, there are no established rules for employee use of company assets, increasing the risk of insider threat and misuse of systems. It also creates legal and compliance challenges. \\
\midrule
RISK-005 & \textbf{\textcolor{sev_medium}{Medium}} & \textbf{Outdated Windows 7 Workstations:} The continued use of Windows 7, which is End-of-Life (EOL), means these workstations do not receive security updates. They are vulnerable to a wide range of known exploits. \\
\bottomrule
\caption{Summary of Identified Risks}
\label{tab:risk_summary}
\end{longtable}

\section{Recommendations}

The following actions are recommended to mitigate the identified risks, prioritized by severity.

\subsection{Critical Priority}
\begin{itemize}
    \item \textbf{Remediate Vulnerable FTP Server (RISK-001):}
        \begin{enumerate}
            \item \textbf{Immediate:} Take the FTP server on \texttt{10.0.0.15} offline immediately to prevent exploitation.
            \item \textbf{Short-Term:} If the FTP service is required, upgrade vsftpd to the latest stable version and disable anonymous access.
            \item \textbf{Long-Term:} Evaluate the business need for FTP. If possible, replace it with a more secure file transfer protocol like SFTP (SSH File Transfer Protocol).
        \end{enumerate}
\end{itemize}

\subsection{High Priority}
\begin{itemize}
    \item \textbf{Implement MFA for Sensitive Systems (RISK-002):}
    Deploy a mandatory MFA policy for all user accounts, especially privileged ones, that can access sensitive or critical data systems.
    
    \item \textbf{Establish Security Awareness Training (RISK-003):}
    Implement a comprehensive security awareness training program. This must include mandatory training for all new hires and annual refresher training for all staff, covering topics like phishing, password security, and data handling.

    \item \textbf{Develop an Acceptable Use Policy (RISK-004):}
    Draft and implement a formal AUP that all employees must read and sign. This policy should clearly define the rules for using company networks, devices, and data.
\end{itemize}

\subsection{Medium Priority}
\begin{itemize}
    \item \textbf{Upgrade Outdated Workstations (RISK-005):}
    Develop and execute a phased plan to upgrade or replace all workstations running Windows 7 with a modern, supported operating system such as Windows 11.
\end{itemize}

\end{document}
```