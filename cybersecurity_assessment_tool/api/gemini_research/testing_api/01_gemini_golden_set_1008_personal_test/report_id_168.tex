Here is the complete and professional LaTeX report, generated based on the provided inputs and your instructions.

\documentclass[12pt]{article}

\usepackage[margin=1in]{geometry}
\usepackage{pifont}
\usepackage{booktabs}
\usepackage{hyperref}
\usepackage{url}
\usepackage{seqsplit}
\usepackage{graphicx}
\usepackage{xcolor}

\hypersetup{
    colorlinks=true,
    linkcolor=blue,
    filecolor=magenta,      
    urlcolor=cyan,
}

\newcommand{\yes}{\ding{51}}
\newcommand{\no}{\ding{55}}

\title{Cybersecurity Posture Assessment Report}
\author{Expert Cybersecurity Analyst}
\date{\today}

\begin{document}

\maketitle
\hrule
\vspace{1em}

\section*{1. Executive Overview}

This report provides a cybersecurity posture assessment for \textbf{Hidden Gem}. The analysis is based on a combination of organizational data provided via a security questionnaire, a technical network scan, and a review of pre-existing risks.

During this assessment, critical data sources for the technical scan and pre-existing risks were found to be corrupted and could not be parsed. Consequently, this report focuses primarily on the significant policy and procedural gaps identified from the organizational security control review.

The analysis revealed three high-priority areas of concern:
\begin{itemize}
    \item \textbf{Lack of Multi-Factor Authentication (MFA) on Sensitive Systems:} This is a critical vulnerability that significantly increases the risk of a data breach.
    \item \textbf{Absence of an Employee Acceptable Use Policy (AUP):} This creates ambiguity and increases the risk of insider threats and misuse of company assets.
    \item \textbf{Inadequate Annual Security Awareness Training:} The lack of recurring training for all staff members leaves the organization vulnerable to evolving threats like phishing and social engineering.
\end{itemize}

Immediate action is required to address these administrative and policy-based control deficiencies. It is also imperative to conduct a new, successful technical scan to identify potential network-level vulnerabilities that are currently unknown due to the corrupted data.

\section{Organizational Information}

The following details were provided by the client organization.

\begin{table}[h!]
\centering
\begin{tabular}{@{}ll@{}}
\toprule
\textbf{Attribute} & \textbf{Value} \\ \midrule
Organization Name    & \textbf{Hidden Gem} \\
Email Domain         & \texttt{HiddenGem.net} \\
Website Domain       & \url{www.HiddenGem.net} \\
External IP Address  & \seqsplit{\texttt{109.187.94.53}} \\ \bottomrule
\end{tabular}
\caption{Client Organizational Details.}
\end{table}

\section{Security Control Review}

The following table summarizes the organization's responses to a security controls questionnaire. Items marked with a red 'X' (\no) indicate a potential gap in the security posture and are addressed in the Risk Assessment section.

\begin{table}[h!]
\centering
\begin{tabular}{@{}p{0.75\textwidth}c@{}}
\toprule
\textbf{Control Question} & \textbf{Response} \\ \midrule
Do you require MFA to access email? & \yes \\
Do you require MFA to log into computers? & \yes \\
Do you require MFA to access sensitive data systems? & \textcolor{red}{\no} \\
Does your organization have an employee acceptable use policy? & \textcolor{red}{\no} \\
Does your organization do security awareness training for new employees? & \yes \\
Does your organization do security awareness training for all employees at least once per year? & \textcolor{red}{\no} \\ \bottomrule
\end{tabular}
\caption{Security Controls Questionnaire Analysis.}
\end{table}

\section{Technical Scan Results}

The technical network scan data (\texttt{Input\_1\_Network\_Scan\_JSON}) provided for the target \texttt{[Target IP]} was found to be corrupted and could not be processed. 

\textbf{Status: Data Unusable.}

Due to this data integrity issue, it was not possible to perform an analysis of open ports, running services, or software versions. This represents a significant visibility gap in the current assessment. Without this data, the organization may be unknowingly exposed to network-level vulnerabilities, such as unpatched services or insecure configurations. A successful re-scan is a high-priority recommendation.

\section{Risk Assessment}

The pre-existing risk data (\texttt{Input\_3\_Current\_Risks\_JSON}) was also found to be corrupted. The risk assessment below is therefore based exclusively on the gaps identified in the Security Control Review (Section 3).

\begin{table}[h!]
\centering
\begin{tabular}{@{}p{0.25\textwidth}p{0.5\textwidth}l@{}}
\toprule
\textbf{Identified Risk} & \textbf{Overview} & \textbf{Severity} \\ \midrule
\textbf{Lack of MFA for Sensitive Systems} & Failure to protect critical data systems with multi-factor authentication creates a single point of failure (a compromised password) for an attacker to gain access. This exposes sensitive data to a high risk of unauthorized access and exfiltration. & \textbf{Critical} \\
\addlinespace
\textbf{No Employee Acceptable Use Policy} & The absence of a formal AUP leads to inconsistent and potentially insecure use of company technology and data. It increases the risk of insider threat, non-compliance with regulations, and data leakage. & \textbf{High} \\
\addlinespace
\textbf{Inadequate Security Awareness Training} & While new hires receive training, the lack of a mandatory annual refresher course for all employees allows security knowledge to become outdated. This makes the organization highly susceptible to evolving phishing, ransomware, and social engineering attacks. & \textbf{High} \\ \bottomrule
\end{tabular}
\caption{Summary of Identified Risks.}
\end{table}

\section{Recommendations}

The following actionable steps are recommended to mitigate the identified risks and improve the overall security posture of \textbf{Hidden Gem}.

\begin{itemize}
    \item \textbf{Remediation of Corrupted Data (Immediate Priority):}
    \begin{itemize}
        \item Re-run the external network scan against target \texttt{[Target IP]} and provide the complete, valid JSON output.
        \item Provide a valid, uncorrupted export of any existing risk register or vulnerability data.
    \end{itemize}
    \vspace{0.5em}
    \item \textbf{Mitigate Critical Risk - MFA Implementation (Immediate Priority):}
    \begin{itemize}
        \item \textbf{Action:} Immediately deploy and enforce MFA across all systems that store, process, or transmit sensitive data. This includes databases, financial applications, and administrative portals.
        \item \textbf{Impact:} Drastically reduces the risk of unauthorized access from compromised credentials.
    \end{itemize}
    \vspace{0.5em}
    \item \textbf{Mitigate High Risk - Policy Development (High Priority):}
    \begin{itemize}
        \item \textbf{Action:} Develop, approve, and implement a comprehensive Employee Acceptable Use Policy (AUP). This policy must be distributed to all employees, who must formally acknowledge their understanding and agreement.
        \item \textbf{Impact:} Establishes clear security expectations, reduces insider risk, and provides a foundation for disciplinary action in case of policy violation.
    \end{itemize}
    \vspace{0.5em}
    \item \textbf{Mitigate High Risk - Training Program (High Priority):}
    \begin{itemize}
        \item \textbf{Action:} Establish a mandatory, annual security awareness training program for all employees and contractors. The curriculum should be updated yearly to reflect the current threat landscape.
        \item \textbf{Impact:} Creates a more resilient human firewall, reducing the likelihood of successful phishing and social engineering attacks.
    \end{itemize}
\end{itemize}

\end{document}