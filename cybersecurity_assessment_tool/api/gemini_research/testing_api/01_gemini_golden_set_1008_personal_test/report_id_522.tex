```latex
\documentclass[12pt]{article}

% Preamble: Required Packages
\usepackage[margin=1in]{geometry}
\usepackage{pifont} % For checkmarks and crosses
\usepackage{booktabs} % For professional tables
\usepackage{hyperref} % For clickable links
\usepackage{url}      % For URL formatting
\usepackage{seqsplit} % For splitting long strings
\usepackage{graphicx} % For potential logos
\usepackage{xcolor}   % For colors

% Document Information
\title{Cybersecurity Posture Assessment Report}
\author{Cybersecurity Analyst}
\date{\today}

% Hyperref Setup
\hypersetup{
    colorlinks=true,
    linkcolor=blue,
    filecolor=magenta,      
    urlcolor=cyan,
    pdftitle={Cybersecurity Posture Assessment Report},
    pdfpagemode=FullScreen,
}

\begin{document}

\maketitle
\thispagestyle{empty}
\newpage

\tableofcontents
\newpage

% --- 1. Executive Summary ---
\section{Executive Summary}

This report provides a comprehensive cybersecurity assessment for \textbf{Clear Path}, conducted by correlating data from a network scan, an organizational security questionnaire, and a review of pre-existing risks.

The assessment reveals several critical and high-risk gaps in the organization's security controls, primarily related to identity and access management and security governance. The most pressing concerns are the lack of Multi-Factor Authentication (MFA) for email and sensitive data systems, which exposes the organization to significant risks of account compromise and data breach. Additionally, the absence of a formal Acceptable Use Policy and a mandatory annual security awareness training program indicates foundational weaknesses in security governance.

On a positive note, a technical network scan of the target host \texttt{192.168.0.5} found that port 80 (HTTP) was closed. This finding contradicts a previously identified risk of an "Unencrypted Web Server" on that port, suggesting that the vulnerability may have been recently remediated.

Immediate action is recommended to implement MFA across all critical systems, formalize security policies, and establish a recurring security training schedule to mitigate these identified risks and strengthen the overall security posture.

% --- 2. Organizational Information ---
\section{Organizational Information}

The following details were provided for the assessment.

\begin{tabular}{@{}ll}
\toprule
\textbf{Attribute} & \textbf{Value} \\
\midrule
Organization Name & \textbf{Clear Path} \\
Email Domain & \texttt{ClearPath.com} \\
Website Domain & \url{www.ClearPath.com} \\
External IP Address & \texttt{211.80.98.211} \\
\bottomrule
\end{tabular}

% --- 3. Security Control Review (Questionnaire) ---
\section{Security Control Review}

The following table summarizes the organization's responses to a security controls questionnaire. Items marked with \ding{55} represent significant gaps in the security framework and are discussed in the Risk Assessment section.

\begin{tabular}{@{}p{0.7\linewidth}lcc}
\toprule
\textbf{Control Question} & \textbf{Response} & \textbf{Status} \\
\midrule
Do you require MFA to access email? & No & \textcolor{red}{\ding{55}} \\
Do you require MFA to log into computers? & Yes & \textcolor{green}{\ding{51}} \\
Do you require MFA to access sensitive data systems? & No & \textcolor{red}{\ding{55}} \\
Does your organization have an employee acceptable use policy? & No & \textcolor{red}{\ding{55}} \\
Does your organization do security awareness training for new employees? & Yes & \textcolor{green}{\ding{51}} \\
Does your organization do security awareness training for all employees at least once per year? & No & \textcolor{red}{\ding{55}} \\
\bottomrule
\end{tabular}

% --- 4. Technical Scan Results ---
\section{Technical Scan Results}

A network scan was performed on the specified target to identify open ports and exposed services.

\begin{itemize}
    \item \textbf{Target IP Address:} \texttt{192.168.0.5}
    \item \textbf{Scan Tool:} Nmap
    \item \textbf{Target Status:} Up
\end{itemize}

\subsection{Port Scan Details}
The scan revealed the following port status. No open ports were detected, which is a secure configuration for the scanned ports.

\begin{tabular}{@{}lll}
\toprule
\textbf{Port} & \textbf{State} & \textbf{Service} \\
\midrule
80/tcp & closed & http \\
\bottomrule
\end{tabular}

\subsection{Analysis}
The scan indicates that port 80 (HTTP) is closed on the target host. This is a positive finding, as it prevents unencrypted web traffic. This result contradicts a pre-existing risk entry (see Section 5), suggesting that the vulnerability has been addressed. Verification is recommended to formally close the associated risk.

% --- 5. Risk Assessment ---
\section{Risk Assessment}

This section synthesizes findings from the security questionnaire, technical scan, and pre-existing risk data into a consolidated list of current risks.

\begin{tabular}{@{}p{0.3\linewidth}p{0.5\linewidth}l}
\toprule
\textbf{Risk Name} & \textbf{Overview} & \textbf{Severity} \\
\midrule
\textbf{Lack of MFA on Email} & The absence of MFA on email accounts significantly increases the risk of Business Email Compromise (BEC), phishing success, and unauthorized access to sensitive communications. & \textbf{Critical} \\
\addlinespace
\textbf{Lack of MFA on Sensitive Systems} & Sensitive data systems without MFA are highly vulnerable to credential theft and brute-force attacks, which could lead to a major data breach. & \textbf{Critical} \\
\addlinespace
\textbf{Inadequate Security Training} & Without mandatory annual training, employees' awareness of evolving threats diminishes, making them more susceptible to social engineering and phishing attacks. & \textbf{High} \\
\addlinespace
\textbf{Missing Acceptable Use Policy} & The lack of a formal policy creates ambiguity regarding the secure use of company assets, increases insider threat risk, and weakens the organization's legal standing in case of misuse. & \textbf{High} \\
\addlinespace
\textbf{Unencrypted Web Server (Potentially Remediated)} & A pre-existing risk noted that port 80 was open. However, our recent scan found this port to be \textbf{closed}. This risk may be remediated but requires formal verification. & Medium \\
\bottomrule
\end{tabular}

% --- 6. Recommendations ---
\section{Recommendations}

Based on the analysis, the following actions are recommended to mitigate the identified risks. Recommendations are prioritized from highest to lowest urgency.

\begin{enumerate}
    \item \textbf{Implement MFA for Critical Systems (Urgent):}
    \begin{itemize}
        \item Immediately enforce MFA for all user access to the email system (\texttt{ClearPath.com}).
        \item Deploy MFA for all systems identified as containing sensitive data. This is the single most effective control to prevent unauthorized access.
    \end{itemize}

    \item \textbf{Establish a Mandatory Annual Security Training Program:}
    \begin{itemize}
        \item Develop or procure a security awareness training module covering current threats such as phishing, ransomware, and social engineering.
        \item Mandate that all employees complete this training annually and track completion to ensure compliance.
    \end{itemize}
    
    \item \textbf{Develop and Implement an Acceptable Use Policy (AUP):}
    \begin{itemize}
        \item Create a formal AUP that clearly defines the rules for using company networks, devices, and data.
        \item Require all employees to read and acknowledge the policy upon hire and during annual reviews.
    \end{itemize}

    \item \textbf{Verify and Update Risk Register:}
    \begin{itemize}
        \item Formally verify that the service previously running on port 80 of \texttt{192.168.0.5} has been decommissioned or secured.
        \item Upon confirmation, update the internal risk register to reflect that the "Unencrypted Web Server" risk has been remediated.
    \end{itemize}
\end{enumerate}

\end{document}
```