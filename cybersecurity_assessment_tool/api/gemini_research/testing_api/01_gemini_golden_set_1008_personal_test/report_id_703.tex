```latex
\documentclass[12pt]{article}

% Preamble: Required Packages and Document Setup
\usepackage[a4paper, margin=1in]{geometry}
\usepackage{pifont} % For checkmarks and crosses
\usepackage{booktabs} % For professional tables
\usepackage{hyperref} % For clickable links
\usepackage{url} % For URL formatting
\usepackage{seqsplit} % To split long strings like IPs
\usepackage{xcolor} % For colors
\usepackage{graphicx} % For potential logos (not used here, but good practice)

% Hyperref Setup for PDF metadata
\hypersetup{
    colorlinks=true,
    linkcolor=blue,
    filecolor=magenta,      
    urlcolor=cyan,
    pdftitle={Cybersecurity Posture Assessment Report},
    pdfauthor={Cybersecurity Analyst},
    pdfsubject={Security Assessment},
    pdfkeywords={Cybersecurity, Risk, Assessment},
    bookmarks=true
}

% Custom Commands
\newcommand{\yes}{\textcolor{green}{\ding{51}}}
\newcommand{\no}{\textcolor{red}{\ding{55}}}

% --- Document Start ---
\begin{document}

% --- Title Page ---
\begin{titlepage}
    \centering
    \vspace*{1cm}
    \Huge\textbf{Cybersecurity Posture Assessment Report}
    \vspace{1.5cm}
    \Large
    \textbf{Prepared for:}\\
    Nova Terra
    \vspace{2cm}
    \large
    \textbf{Date of Report:}\\
    \today
    \vfill
    \textit{This report contains sensitive information and should be handled with care. Distribution is restricted to authorized personnel only.}
\end{titlepage}

\tableofcontents
\newpage

% --- Section 1: Executive Summary ---
\section{Executive Summary}
This report provides a comprehensive analysis of the cybersecurity posture for Nova Terra, based on a combination of network scanning, a security controls questionnaire, and a review of known risks. The assessment was conducted to identify vulnerabilities, policy gaps, and technical misconfigurations that could expose the organization to cyber threats.

The key findings indicate several critical and high-risk areas requiring immediate attention. The complete absence of Multi-Factor Authentication (MFA) across all critical access points—including email, user workstations, and sensitive data systems—represents a significant vulnerability. This is compounded by foundational gaps in security governance, such as the lack of an employee acceptable use policy and security training for new hires.

Technically, an exposed Secure Shell (SSH) service was identified on the network perimeter. While a common administrative tool, its exposure without compensating controls like MFA and IP address restrictions creates a direct pathway for unauthorized access.

Urgent remediation is recommended, focusing on the swift implementation of MFA, the development of core security policies, and the hardening of externally-facing services. Addressing these issues will substantially improve the organization's resilience against common cyberattacks.

% --- Section 2: Organizational Information ---
\section{Organizational Information}
The following details were provided for the assessment.
\begin{itemize}
    \item \textbf{Organization Name:} Nova Terra
    \item \textbf{Email Domain:} \texttt{NovaTerra.net}
    \item \textbf{Website Domain:} \url{www.NovaTerra.net}
    \item \textbf{Primary External IP:} \texttt{135.55.224.236}
\end{itemize}

% --- Section 3: Security Control Review ---
\section{Security Control Review}
The following table summarizes the organization's responses to a security controls questionnaire. "No" answers highlight significant gaps in the current security framework.

\begin{table}[h!]
\centering
\caption{Security Controls Questionnaire Analysis}
\label{tab:controls}
\begin{tabular}{p{8cm} c l}
\toprule
\textbf{Control Question} & \textbf{Response} & \textbf{Assessment} \\
\midrule
Do you require MFA to access email? & \no & \textbf{Critical Gap} \\
Do you require MFA to log into computers? & \no & \textbf{Critical Gap} \\
Do you require MFA to access sensitive data systems? & \no & \textbf{Critical Gap} \\
Does your organization have an employee acceptable use policy? & \no & High Risk \\
Does your organization do security awareness training for new employees? & \no & High Risk \\
Does your organization do security awareness training for all employees at least once per year? & \yes & Best Practice Met \\
\bottomrule
\end{tabular}
\end{table}

The analysis reveals a systemic failure to implement MFA, which is a foundational control for preventing account takeover attacks. Additionally, the absence of an acceptable use policy and new hire training indicates a lack of security governance and awareness culture.

% --- Section 4: Technical Scan Results ---
\section{Technical Scan Results}
An external network scan was performed to identify open ports and services exposed to the internet.

\begin{itemize}
    \item \textbf{Target IP Address:} \seqsplit{\texttt{2001:db8::1}}
\end{itemize}

The following table details the findings from the scan.

\begin{table}[h!]
\centering
\caption{Open Port Analysis}
\label{tab:ports}
\begin{tabular}{c c l p{6cm}}
\toprule
\textbf{Port} & \textbf{State} & \textbf{Service (Presumed)} & \textbf{Notes} \\
\midrule
22/TCP & Open & SSH (Secure Shell) & This port is commonly used for remote system administration. Its exposure to the public internet is a high risk, especially without MFA or IP whitelisting. It is a primary target for brute-force attacks. \\
\bottomrule
\end{tabular}
\end{table}

\textbf{Analysis:} The presence of an open SSH port is a significant finding. When correlated with the lack of MFA (Section 3), this creates a high-impact risk. An attacker who compromises a user's password could gain direct administrative access to a critical system.

% --- Section 5: Consolidated Risk Assessment ---
\section{Consolidated Risk Assessment}
This section synthesizes findings from the questionnaire, technical scans, and pre-existing risk data. No pre-existing vulnerabilities were reported. The following new risks have been identified.

\begin{table}[h!]
\centering
\caption{Summary of Identified Risks}
\label{tab:risks}
\begin{tabular}{p{4cm} p{7cm} l}
\toprule
\textbf{Risk Name} & \textbf{Description} & \textbf{Severity} \\
\midrule
\textbf{Lack of Multi-Factor Authentication (MFA)} & MFA is not enforced for email, computer logins, or access to sensitive data. This makes user accounts highly susceptible to takeover via phishing or password spraying. & \textbf{Critical} \\
\addlinespace
\textbf{Exposed Administrative Service} & The SSH service on port 22 is open to the internet. Combined with the lack of MFA, this creates a high-risk vector for unauthorized server access. & \textbf{High} \\
\addlinespace
\textbf{Insufficient Security Policies \& Training} & The absence of an Acceptable Use Policy and security training for new hires weakens the human firewall and increases the likelihood of security incidents caused by employee error. & \textbf{High} \\
\bottomrule
\end{tabular}
\end{table}

% --- Section 6: Recommendations ---
\section{Recommendations}
Based on the identified risks, the following prioritized actions are recommended to improve the security posture of Nova Terra.

\subsection{Immediate Actions (Critical Priority)}
\begin{enumerate}
    \item \textbf{Implement MFA Across All Critical Systems:} Deploy MFA immediately for all users. Prioritize the following services:
    \begin{itemize}
        \item Email (e.g., Office 365, Google Workspace).
        \item VPN and other remote access solutions.
        \item Access to servers, especially via SSH.
        \item All applications handling sensitive data.
    \end{itemize}
\end{enumerate}

\subsection{High Priority Actions}
\begin{enumerate}
    \setcounter{enumi}{1} % Continue numbering
    \item \textbf{Develop and Enforce an Acceptable Use Policy (AUP):} Create a formal AUP that defines rules for the use of company assets, including password complexity, data handling, and prohibited activities. Ensure all employees read and acknowledge the policy.
    \item \textbf{Establish a Security Training Program:} Implement a mandatory security awareness training module for all new employees during onboarding. This should cover topics like phishing, password hygiene, and social engineering.
    \item \textbf{Harden the Exposed SSH Service:}
    \begin{itemize}
        \item If possible, restrict access to the SSH port to trusted IP addresses only (e.g., office or administrator IPs).
        \item Disable password-based authentication and enforce the use of public key cryptography.
        \item Ensure the SSH server software is fully patched and configured according to security best practices (e.g., disabling root login).
    \end{itemize}
\end{enumerate}

\subsection{Ongoing Recommendations}
\begin{enumerate}
    \setcounter{enumi}{4} % Continue numbering
    \item \textbf{Conduct Regular Vulnerability Scanning:} Implement a program for regular, automated vulnerability scanning of all external and internal assets to proactively identify and remediate security flaws.
\end{enumerate}

\end{document}
% --- Document End ---
```