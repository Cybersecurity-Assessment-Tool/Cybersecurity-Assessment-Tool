```latex
\documentclass[12pt]{article}

% Preamble: Required Packages
\usepackage[margin=1in]{geometry}
\usepackage{pifont} % For checkmarks and crosses
\usepackage{booktabs} % For professional tables
\usepackage{hyperref} % For clickable links
\usepackage{url} % For URL formatting
\usepackage{seqsplit} % For splitting long text strings
\usepackage{graphicx}
\usepackage{xcolor}

% Document Metadata
\title{Cybersecurity Posture Assessment Report}
\author{Cybersecurity Analyst}
\date{\today}

% Hyperref Setup
\hypersetup{
    colorlinks=true,
    linkcolor=blue,
    filecolor=magenta,      
    urlcolor=cyan,
    pdftitle={Cybersecurity Posture Assessment Report},
    pdfpagemode=FullScreen,
}

\begin{document}

\maketitle
\thispagestyle{empty}
\newpage

\tableofcontents
\newpage

% --- 1. Executive Summary ---
\section*{1. Executive Summary}

This report provides a comprehensive cybersecurity assessment for \textbf{Common Ground}, based on an analysis of network scan data, organizational security controls, and existing risk documentation. The assessment reveals several critical and high-risk security gaps that require immediate attention.

A network scan identified an exposed service on an internal host (\texttt{10.5.5.5}) on port \texttt{8080}, which alarmingly identifies itself as a \textbf{"TOP SECRET DB"}. This finding directly contradicts previous risk documentation which incorrectly labeled this port as secure. This information disclosure represents a critical vulnerability.

Furthermore, analysis of the organization's security controls highlights a critical lack of Multi-Factor Authentication (MFA) for email and computer access. This, combined with deficiencies in security policies and annual employee training, significantly increases the risk of a successful cyberattack, such as a business email compromise or ransomware event.

Immediate remediation should focus on securing the exposed database service and implementing MFA across all user accounts. Strategic improvements in security policy and training are also strongly recommended to build a more resilient security posture.

% --- 2. Organizational Information ---
\section*{2. Organizational Information}

The following information was provided for the assessment.

\begin{tabular}{@{}ll}
\toprule
\textbf{Attribute} & \textbf{Value} \\
\midrule
Organization Name & \textbf{Common Ground} \\
Email Domain & \texttt{CommonGround.org} \\
Website Domain & \url{www.CommonGround.org} \\
External IP Address & \texttt{190.3.252.25} \\
\bottomrule
\end{tabular}

% --- 3. Security Control Review ---
\section*{3. Security Control Review}

A review of the organization's security controls was conducted via a questionnaire. The responses indicate significant gaps in foundational security practices. "No" answers represent areas of high risk that weaken the organization's defenses against common cyber threats.

\begin{tabular}{@{}p{0.6\linewidth}cp{0.25\linewidth}@{}}
\toprule
\textbf{Control Question} & \textbf{Response} & \textbf{Analyst Note} \\
\midrule
Do you require MFA to access email? & \ding{55} & \textcolor{red}{\textbf{Critical Gap.}} Lack of MFA on email is a primary vector for account takeover and phishing attacks. \\
\addlinespace
Do you require MFA to log into computers? & \ding{55} & \textcolor{red}{\textbf{Critical Gap.}} Compromised credentials could lead directly to device and network access. \\
\addlinespace
Do you require MFA to access sensitive data systems? & \ding{51} & Good. This control should be expanded to all systems. \\
\addlinespace
Does your organization have an employee acceptable use policy? & \ding{55} & \textbf{High Risk.} Without a formal policy, there is no enforceable standard for employee behavior on corporate assets. \\
\addlinespace
Does your organization do security awareness training for new employees? & \ding{51} & Good. This establishes a baseline for new hires. \\
\addlinespace
Does your organization do security awareness training for all employees at least once per year? & \ding{55} & \textbf{High Risk.} Security knowledge degrades over time. Annual training is essential to keep staff aware of evolving threats. \\
\bottomrule
\end{tabular}

% --- 4. Technical Scan Results ---
\section*{4. Technical Scan Results}

A network scan was performed to identify exposed services and potential vulnerabilities.

\subsection*{Nmap Scan Findings}
The scan targeted the internal IP address \texttt{10.5.5.5} and revealed the following open port.

\begin{tabular}{@{}llll@{}}
\toprule
\textbf{Port} & \textbf{State} & \textbf{Service Details} & \textbf{Finding} \\
\midrule
8080/tcp & Open & HTTP Service Title: \texttt{TOP SECRET DB} & \textcolor{red}{\textbf{CRITICAL.}} A service is exposed with a title that implies it contains highly sensitive data. This is a severe information disclosure vulnerability and makes the service a prime target for attackers. This finding invalidates previous assessments that marked this port as a false positive. \\
\bottomrule
\end{tabular}

% --- 5. Consolidated Risk Assessment ---
\section*{5. Consolidated Risk Assessment}

The following table synthesizes findings from the security control review, technical scan, and pre-existing risk data. Risks are prioritized based on their potential impact on the organization.

\begin{tabular}{@{}p{0.25\linewidth}p{0.5\linewidth}l@{}}
\toprule
\textbf{Risk Name} & \textbf{Overview} & \textbf{Severity} \\
\midrule
\textbf{Exposed Sensitive Database Interface} & The service on \texttt{10.5.5.5:8080} is open and advertises itself as a "TOP SECRET DB". This presents a clear and immediate danger of data exfiltration or system compromise. & \textcolor{red}{\textbf{Critical}} \\
\addlinespace
\textbf{Lack of Multi-Factor Authentication (MFA)} & The absence of MFA for email and computer access makes user accounts highly susceptible to takeover via credential stuffing or phishing. A single compromised password could grant an attacker significant access. & \textcolor{red}{\textbf{Critical}} \\
\addlinespace
\textbf{Insufficient Security Policies \& Training} & The lack of an Acceptable Use Policy and mandatory annual security training creates a weak human firewall. Employees are more likely to fall for social engineering or misuse company assets. & \textbf{High} \\
\addlinespace
\textbf{Outdated Risk Documentation} & Previous documentation incorrectly stated that port 8080 was secure. This indicates a flaw in the risk management process, where risks are closed without proper validation, leading to a false sense of security. & Informational \\
\bottomrule
\end{tabular}

% --- 6. Recommendations ---
\section*{6. Recommendations}

The following actionable recommendations are provided to mitigate the identified risks.

\subsection*{Immediate Actions (0-7 Days)}
\begin{itemize}
    \item \textbf{Secure Port 8080:} Immediately investigate the service running on \texttt{10.5.5.5:8080}. 
    \begin{itemize}
        \item If the service is not required for business, disable it.
        \item If required, restrict access to only authorized users/IPs via firewall rules.
        \item Change the service's public-facing title to remove any reference to sensitive data.
    \end{itemize}
\end{itemize}

\subsection*{High-Priority Actions (1-3 Months)}
\begin{itemize}
    \item \textbf{Deploy MFA:} Implement and enforce MFA for all employees on all critical systems, prioritizing:
    \begin{enumerate}
        \item Email access (e.g., Office 365, Google Workspace).
        \item VPN and other remote access solutions.
        \item All computer/endpoint logins.
    \end{enumerate}
    \item \textbf{Develop Acceptable Use Policy (AUP):} Create and distribute a formal AUP that defines the rules for using company technology and data. Require all employees to read and acknowledge the policy.
\end{itemize}

\subsection*{Strategic Actions (3-6 Months)}
\begin{itemize}
    \item \textbf{Implement Annual Security Training:} Establish a mandatory, annual security awareness training program for all employees. The training should cover phishing, password hygiene, and the new AUP.
    \item \textbf{Improve Risk Management Process:} Update the risk assessment and management lifecycle to include a mandatory validation step before any identified risk can be formally closed.
\end{itemize}

\end{document}
```