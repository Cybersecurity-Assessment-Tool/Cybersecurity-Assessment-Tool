```latex
\documentclass[12pt]{article}

% --- PACKAGES ---
\usepackage[margin=1in]{geometry}
\usepackage{pifont} % For checkmarks and crosses
\usepackage{booktabs} % For professional tables
\usepackage{hyperref} % For clickable links
\usepackage{url} % For URL formatting
\usepackage{seqsplit} % To split long text strings in tt font
\usepackage{xcolor} % For colors
\usepackage{graphicx} % For potential logos/images
\usepackage{fancyhdr} % For headers/footers

% --- DOCUMENT METADATA ---
\title{Cybersecurity Posture Assessment Report}
\author{Cybersecurity Analysis Division}
\date{\today}

% --- HYPERREF SETUP ---
\hypersetup{
    colorlinks=true,
    linkcolor=blue,
    filecolor=magenta,      
    urlcolor=cyan,
    pdftitle={Cybersecurity Posture Assessment Report},
    pdfpagemode=FullScreen,
}

% --- HEADER & FOOTER ---
\pagestyle{fancy}
\fancyhf{}
\lhead{Astraeus Aerospace}
\rhead{Confidential}
\cfoot{\thepage}

% --- BEGIN DOCUMENT ---
\begin{document}

\maketitle
\thispagestyle{empty}
\newpage

\tableofcontents
\newpage

% ==============================================================================
% SECTION 1: EXECUTIVE SUMMARY
% ==============================================================================
\section{Executive Summary}

This report provides a comprehensive cybersecurity assessment for \textbf{Astraeus Aerospace}, based on a technical network scan, a review of organizational security controls, and an analysis of pre-existing risk data. The assessment was conducted on \today.

The organization's overall security posture is mixed. Foundational controls such as Multi-Factor Authentication (MFA) for computer logins and an employee acceptable use policy are in place. However, several critical security gaps were identified that expose the organization to significant risk.

\textbf{Key Findings:}
\begin{itemize}
    \item \textbf{Critical MFA Gaps:} Multi-Factor Authentication is not enforced for accessing email or sensitive data systems. This represents a critical vulnerability, as compromised credentials could lead to unauthorized access to confidential communications and critical business data.
    \item \textbf{Insufficient Security Training:} While new employees receive security training, there is no mandatory annual training program for all staff. This gap can lead to a decline in security awareness over time, making the organization more susceptible to social engineering and phishing attacks.
    \item \textbf{Risk Mitigation Confirmed:} A previously identified risk, "Unencrypted Web Server" on port 80, was not present on the scanned target (\texttt{192.168.0.5}). The technical scan confirmed that port 80 is closed on this host, indicating successful mitigation or a false positive in prior reporting.
\end{itemize}

Immediate action is required to address the identified MFA and training deficiencies. Detailed recommendations are provided in Section \ref{sec:recommendations} to guide remediation efforts and strengthen the overall security posture of \textbf{Astraeus Aerospace}.

% ==============================================================================
% SECTION 2: ORGANIZATIONAL INFORMATION
% ==============================================================================
\section{Organizational Information}

The following details were provided for the assessment.

\begin{tabular}{@{}ll}
    \toprule
    \textbf{Attribute} & \textbf{Value} \\
    \midrule
    Organization Name & \textbf{Astraeus Aerospace} \\
    Email Domain & \seqsplit{\texttt{AstraeusAerospace.net}} \\
    Website Domain & \seqsplit{\url{www.AstraeusAerospace.net}} \\
    External IP Address & \seqsplit{\texttt{78.87.77.237}} \\
    \bottomrule
\end{tabular}

% ==============================================================================
% SECTION 3: SECURITY CONTROL REVIEW
% ==============================================================================
\section{Security Control Review}

A review of internal security controls was conducted via a questionnaire. The results highlight both strengths and critical weaknesses in the current security framework. Answers of "No" indicate significant gaps that increase organizational risk.

\begin{table}[h!]
\centering
\begin{tabular}{@{}p{0.7\textwidth}c}
    \toprule
    \textbf{Control Question} & \textbf{Response} \\
    \midrule
    Do you require MFA to access email? & \textcolor{red}{\ding{55}} \\
    Do you require MFA to log into computers? & \textcolor{green}{\ding{51}} \\
    Do you require MFA to access sensitive data systems? & \textcolor{red}{\ding{55}} \\
    Does your organization have an employee acceptable use policy? & \textcolor{green}{\ding{51}} \\
    Does your organization do security awareness training for new employees? & \textcolor{green}{\ding{51}} \\
    Does your organization do security awareness training for all employees at least once per year? & \textcolor{red}{\ding{55}} \\
    \bottomrule
\end{tabular}
\caption{Organizational Security Control Questionnaire Results.}
\label{tab:controls}
\end{table}

\subsection{Analysis of Control Gaps}
The lack of enforced MFA on email and sensitive data systems constitutes a critical risk. Email is a primary vector for phishing and business email compromise (BEC) attacks. Failure to protect sensitive data systems with MFA removes a crucial layer of defense against unauthorized access and data exfiltration. Furthermore, the absence of annual security awareness training for all employees allows security knowledge to become stale, increasing the likelihood of human error leading to a security incident.

% ==============================================================================
% SECTION 4: TECHNICAL SCAN RESULTS
% ==============================================================================
\section{Technical Scan Results}

A network scan was performed to identify open ports and exposed services on the specified target.

\begin{itemize}
    \item \textbf{Scan Target:} \texttt{192.168.0.5}
    \item \textbf{Scan Tool:} Nmap
\end{itemize}

\begin{table}[h!]
\centering
\begin{tabular}{@{}ccccc}
    \toprule
    \textbf{Port} & \textbf{State} & \textbf{Service} & \textbf{Product} & \textbf{Version} \\
    \midrule
    80/tcp & closed & http & N/A & N/A \\
    \bottomrule
\end{tabular}
\caption{Port Scan Results for \texttt{192.168.0.5}.}
\label{tab:scan}
\end{table}

\subsection{Scan Analysis}
The scan results are positive for the target host. The only port checked, port 80 (HTTP), was found to be \textbf{closed}. This indicates that the host is not running an unencrypted web server accessible on the standard port. This finding directly contradicts a pre-existing risk entry (see Section \ref{sec:risk}), suggesting that the risk has been remediated on this specific asset or was inaccurately reported. No other open ports or vulnerable services were discovered on this host during the scan.

% ==============================================================================
% SECTION 5: CONSOLIDATED RISK ASSESSMENT
% ==============================================================================
\section{Consolidated Risk Assessment}
\label{sec:risk}

This section synthesizes findings from the security control review, technical scan, and pre-existing risk data to provide a consolidated view of the current risk landscape.

\begin{table}[h!]
\centering
\begin{tabular}{@{}p{0.25\textwidth}p{0.5\textwidth}l}
    \toprule
    \textbf{Risk Name} & \textbf{Description} & \textbf{Severity} \\
    \midrule
    \textbf{Inadequate MFA for Email} & Lack of MFA on email accounts significantly increases the risk of account takeover via credential theft or phishing. & \textbf{Critical} \\
    \textbf{Inadequate MFA for Data} & Sensitive data systems are not protected by MFA, creating a high risk of unauthorized access and data breach. & \textbf{Critical} \\
    \textbf{Insufficient Security Training} & Without mandatory annual training, employees are more likely to fall victim to social engineering, phishing, and other attacks. & \textbf{High} \\
    \textit{Unencrypted Web Server} & \textit{Port 80 was reported as open. However, our technical scan confirmed Port 80 is closed on the target host.} & \textit{Mitigated} \\
    \bottomrule
\end{tabular}
\caption{Summary of Identified and Reviewed Risks.}
\label{tab:risks}
\end{table}

% ==============================================================================
% SECTION 6: RECOMMENDATIONS
% ==============================================================================
\section{Recommendations}
\label{sec:recommendations}

The following actions are recommended to address the identified risks and improve the security posture of \textbf{Astraeus Aerospace}. Recommendations are prioritized based on severity.

\subsection{Critical Priority}
\begin{enumerate}
    \item \textbf{Enforce MFA for Email Access:} Immediately implement and enforce a policy requiring Multi-Factor Authentication for all user access to the email system (\seqsplit{\texttt{AstraeusAerospace.net}}). This is the single most effective control to prevent business email compromise.
    \item \textbf{Enforce MFA for Sensitive Data Systems:} Identify all systems containing sensitive, confidential, or proprietary data and mandate the use of MFA for all user access, including administrative accounts.
\end{enumerate}

\subsection{High Priority}
\begin{enumerate}
    \item \textbf{Implement Annual Security Awareness Training:} Establish a mandatory security awareness training program for all employees, to be completed annually. The training should cover current threats such as phishing, social engineering, and proper data handling.
\end{enumerate}

\subsection{Verification and Review}
\begin{enumerate}
    \item \textbf{Validate Risk Mitigation:} Although the "Unencrypted Web Server" risk appears mitigated on the scanned host, it is recommended to perform a broader network scan to ensure no other systems are exposing an open Port 80. Formally update the organizational risk register to reflect the mitigated status.
\end{enumerate}

\end{document}
```