```latex
\documentclass[12pt]{article}

% Preamble: Required Packages
\usepackage[margin=1in]{geometry}
\usepackage{pifont} % For checkmarks and crosses (\ding)
\usepackage{booktabs} % For professional tables
\usepackage[hidelinks]{hyperref} % For clickable links without boxes
\usepackage{url}
\usepackage{seqsplit} % To split long text sequences like URLs or hashes

% Document Information
\title{Cybersecurity Posture and Risk Assessment Report}
\author{Cybersecurity Analyst}
\date{\today}

\begin{document}

\maketitle
\thispagestyle{empty}
\newpage

\tableofcontents
\newpage

% --- 1. Executive Summary ---
\section*{1. Executive Summary}

This report provides a comprehensive analysis of the cybersecurity posture for \textbf{White Label}. The assessment is based on a correlation of network scan data, a review of organizational security controls, and an evaluation of pre-existing risks.

The analysis has identified several \textbf{critical} and \textbf{high-risk} vulnerabilities that require immediate attention. The most severe findings include a publicly accessible and highly vulnerable FTP service (\texttt{vsftpd 2.3.4}) on an internal server (\texttt{10.0.0.15}), which is known to contain a backdoor. This technical flaw is compounded by critical gaps in organizational policy, namely the lack of Multi-Factor Authentication (MFA) for email and sensitive systems, and the absence of a formal security awareness training program.

These issues, combined with the existing risk of outdated Windows 7 workstations, create a significant risk of unauthorized access, data breach, and system compromise. This report outlines these findings in detail and provides prioritized, actionable recommendations to mitigate the identified risks and strengthen the overall security posture.

% --- 2. Organizational Information ---
\section*{2. Organizational Information}

The following information was provided for the assessment.

\begin{table}[h!]
\centering
\begin{tabular}{@{}ll@{}}
\toprule
\textbf{Attribute} & \textbf{Value} \\ \midrule
Organization Name & \textbf{White Label} \\
Email Domain & \texttt{WhiteLabel.net} \\
Website Domain & \url{www.WhiteLabel.net} \\
External IP Address & \texttt{39.80.133.83} \\ \bottomrule
\end{tabular}
\caption{Client Organizational Details}
\end{table}

% --- 3. Security Control Review ---
\section*{3. Security Control Review}

A review of administrative and policy-based security controls was conducted via a questionnaire. The results highlight significant gaps in access control and employee security awareness, which are detailed below.

\begin{table}[h!]
\centering
\begin{tabular}{@{}lcc@{}}
\toprule
\textbf{Control Question} & \textbf{Response} & \textbf{Risk Level} \\ \midrule
Do you require MFA to access email? & \ding{55} No & \textcolor{red}{Critical} \\
Do you require MFA to log into computers? & \ding{51} Yes & Low \\
Do you require MFA to access sensitive data systems? & \ding{55} No & \textcolor{red}{Critical} \\
Does your organization have an employee acceptable use policy? & \ding{51} Yes & Low \\
Does your organization do security awareness training for new employees? & \ding{55} No & \textcolor{orange}{High} \\
Does your organization do security awareness training for all employees annually? & \ding{55} No & \textcolor{orange}{High} \\ \bottomrule
\end{tabular}
\caption{Security Control Questionnaire Results}
\end{table}

The absence of MFA on email and sensitive systems represents a critical failure in access control. Stolen credentials could lead directly to a compromise. Furthermore, the lack of a structured security awareness training program leaves the organization highly susceptible to social engineering and phishing attacks.

% --- 4. Technical Scan Results ---
\section*{4. Technical Scan Results}

A network scan was performed on the internal network to identify active services and potential vulnerabilities.

\begin{itemize}
    \item \textbf{Target IP Address:} \texttt{10.0.0.15}
\end{itemize}

\begin{table}[h!]
\centering
\begin{tabular}{@{}llllll@{}}
\toprule
\textbf{Port} & \textbf{State} & \textbf{Service} & \textbf{Product} & \textbf{Version} \\ \midrule
21/tcp & Open & ftp & vsftpd & 2.3.4 \\ \bottomrule
\end{tabular}
\caption{Open Ports and Services on \texttt{10.0.0.15}}
\end{table}

\subsection*{Analysis of Findings}
The scan revealed a critically vulnerable FTP service running on the target system.
\begin{enumerate}
    \item \textbf{Vulnerable Software (CVE-2011-2523):} The identified version, \textbf{vsftpd 2.3.4}, is known to contain a critical backdoor vulnerability. Malicious code was inserted into the source code of this specific version, which allows an attacker to gain a remote command shell on the server by entering a specific string in the username field. This represents a direct path to system compromise.
    \item \textbf{Anonymous FTP Login:} The scan confirmed that \textbf{anonymous FTP login is allowed}. This misconfiguration permits unauthenticated users to access files on the server, which could lead to sensitive information disclosure or provide a platform for attackers to stage malicious files.
    \item \textbf{Insecure Protocol:} FTP is an inherently insecure protocol that transmits credentials and data in cleartext. Its use should be deprecated in favor of secure alternatives like SFTP (SSH File Transfer Protocol).
\end{enumerate}

% --- 5. Consolidated Risk Assessment ---
\section*{5. Consolidated Risk Assessment}

The following table synthesizes findings from the security control review, technical scan, and pre-existing risk data into a prioritized list.

\begin{table}[h!]
\centering
\begin{tabular}{@{}p{0.35\linewidth}p{0.45\linewidth}l@{}}
\toprule
\textbf{Risk / Vulnerability} & \textbf{Description} & \textbf{Severity} \\ \midrule
\textbf{Vulnerable FTP Service (vsftpd 2.3.4)} & A backdoor in the running FTP service allows for remote code execution and full system compromise. & \textcolor{red}{Critical} \\
\addlinespace
\textbf{Lack of MFA on Critical Systems} & Email and sensitive data systems are protected only by passwords, making them vulnerable to account takeover from stolen credentials. & \textcolor{red}{Critical} \\
\addlinespace
\textbf{Anonymous FTP Access} & Unauthenticated users can access the FTP server, risking data leakage and misuse of the server. & \textcolor{red}{Critical} \\
\addlinespace
\textbf{No Security Awareness Training} & Employees are not trained to recognize or report phishing and other social engineering attacks, increasing organizational risk. & \textcolor{orange}{High} \\
\addlinespace
\textbf{Outdated Windows Policy} & Workstations are running Windows 7, an unsupported operating system that no longer receives security updates. & Medium \\
\bottomrule
\end{tabular}
\caption{Summary of Identified Risks}
\end{table}

% --- 6. Recommendations ---
\section*{6. Recommendations}

The following actions are recommended to mitigate the identified risks, prioritized by severity.

\subsection*{Critical Priority}
\begin{enumerate}
    \item \textbf{Remediate Vulnerable FTP Server (Immediate):}
    \begin{itemize}
        \item Take the server at \texttt{10.0.0.15} offline immediately to prevent exploitation.
        \item Investigate the server for signs of existing compromise.
        \item Rebuild the server using a secure, updated operating system and disable or replace the vsftpd service. If file transfer is required, implement a secure alternative such as SFTP.
        \item Disable anonymous access on any file transfer service.
    \end{itemize}
    \item \textbf{Implement Multi-Factor Authentication (MFA):}
    \begin{itemize}
        \item Enforce mandatory MFA for all user accounts on the email system (\texttt{WhiteLabel.net}).
        \item Enforce mandatory MFA for all systems identified as containing or providing access to sensitive data.
    \end{itemize}
\end{enumerate}

\subsection*{High Priority}
\begin{enumerate}
    \item \textbf{Establish a Security Awareness Training Program:}
    \begin{itemize}
        \item Procure and implement a training solution that covers key topics like phishing, password security, and acceptable use.
        \item Make training mandatory for all new hires during their onboarding process.
        \item Mandate annual refresher training for all employees to ensure knowledge remains current.
    \end{itemize}
\end{enumerate}

\subsection*{Medium Priority}
\begin{enumerate}
    \item \textbf{Upgrade Outdated Windows Systems:}
    \begin{itemize}
        \item Continue with the existing plan to upgrade all Windows 7 workstations to a supported operating system (e.g., Windows 10/11) to ensure they receive critical security patches.
    \end{itemize}
\end{enumerate}

\end{document}
```