```latex
\documentclass[12pt, a4paper]{article}

% Preamble: Required Packages
\usepackage[margin=1in]{geometry}
\usepackage{pifont} % For checkmarks and crosses
\usepackage{booktabs} % For professional tables
\usepackage{hyperref} % For clickable links and references
\usepackage{url} % For formatting URLs
\usepackage{seqsplit} % To split long strings like IPs
\usepackage{graphicx}
\usepackage{xcolor}
\usepackage{fancyhdr}

% Document Metadata
\title{Cybersecurity Posture Assessment Report}
\author{Cybersecurity Analysis Division}
\date{\today}

% Hyperref Setup
\hypersetup{
    colorlinks=true,
    linkcolor=blue,
    filecolor=magenta,      
    urlcolor=cyan,
    pdftitle={Cybersecurity Posture Assessment Report},
    pdfpagemode=FullScreen,
}

% Header and Footer
\pagestyle{fancy}
\fancyhf{}
\fancyhead[L]{Cybersecurity Assessment Report}
\fancyhead[R]{\textbf{Sovereign Trust}}
\fancyfoot[C]{\thepage}

\begin{document}

\maketitle
\thispagestyle{empty}
\newpage

\tableofcontents
\newpage

% --- 1. Executive Summary ---
\section{Executive Summary}

This report provides a cybersecurity posture assessment for \textbf{Sovereign Trust}, conducted on \today. The analysis is based on a combination of network scanning, a review of organizational security controls, and an evaluation of pre-existing risks.

The assessment reveals a mixed security posture. On a technical level, the scanned network host (\seqsplit{\texttt{192.168.1.100}}) demonstrates a strong defensive configuration, with no open ports detected. This indicates a well-maintained firewall and a minimal external attack surface for that specific asset.

However, significant and critical vulnerabilities were identified in the organization's procedural and access control policies. The absence of Multi-Factor Authentication (MFA) for computer logins and access to sensitive data systems constitutes a critical risk. Furthermore, the lack of a comprehensive security awareness training program for new and existing employees leaves the organization highly susceptible to social engineering and phishing attacks, which are common vectors for initial compromise.

While no pre-existing vulnerabilities were reported, the newly identified policy gaps are severe. Immediate remediation efforts should be focused on implementing robust access controls and establishing a culture of security awareness through mandatory training.

\vspace{1cm}

\noindent \textbf{Overall Risk Level: \color{red}HIGH}

% --- 2. Organizational Information ---
\section{Organizational Information}

The following information was provided for the assessment.

\begin{table}[h!]
\centering
\begin{tabular}{@{}ll@{}}
\toprule
\textbf{Attribute} & \textbf{Value} \\ \midrule
Organization Name & \textbf{Sovereign Trust} \\
Email Domain & \seqsplit{\texttt{SovereignTrust.org}} \\
Website Domain & \seqsplit{\url{www.SovereignTrust.org}} \\
External IP Address & \seqsplit{\texttt{121.242.52.40}} \\ \bottomrule
\end{tabular}
\caption{Client Organizational Details}
\end{table}

% --- 3. Security Control Review ---
\section{Security Control Review}

A review of internal security controls was conducted via a questionnaire. The responses highlight critical gaps in the organization's defense-in-depth strategy. The results are summarized below.

\begin{table}[h!]
\centering
\begin{tabular}{@{}p{0.6\textwidth} c p{0.2\textwidth}@{}}
\toprule
\textbf{Control Question} & \textbf{Response} & \textbf{Assessment} \\ \midrule
Do you require MFA to access email? & \ding{51} Yes & Good Practice \\
Do you require MFA to log into computers? & {\color{red}\ding{55} No} & \textbf{\color{red}Critical Gap} \\
Do you require MFA to access sensitive data systems? & {\color{red}\ding{55} No} & \textbf{\color{red}Critical Gap} \\
Does your organization have an employee acceptable use policy? & \ding{51} Yes & Good Practice \\
Does your organization do security awareness training for new employees? & {\color{red}\ding{55} No} & \textbf{\color{red}High Risk} \\
Does your organization do security awareness training for all employees at least once per year? & {\color{red}\ding{55} No} & \textbf{\color{red}High Risk} \\ \bottomrule
\end{tabular}
\caption{Security Controls Questionnaire Results}
\end{table}

% --- 4. Technical Scan Results ---
\section{Technical Scan Results}

A network scan was performed to identify open ports and exposed services on the target system.

\subsection{Nmap Scan: \seqsplit{\texttt{192.168.1.100}}}
The scan results for the target host are summarized below.

\begin{itemize}
    \item \textbf{Target IP:} \seqsplit{\texttt{192.168.1.100}}
    \item \textbf{Host Status:} Up
    \item \textbf{Finding:} No open ports were detected. The scan reported that all 1000 scanned ports were in a 'closed' state.
\end{itemize}

\paragraph{Analysis:} This is a positive security finding. A host with no open ports presents a minimal attack surface to the network, suggesting it is either not hosting network services or is protected by a properly configured firewall that denies all inbound connections. This configuration significantly reduces the risk of direct technical exploitation.

% --- 5. Risk Assessment ---
\section{Risk Assessment}
This section synthesizes findings from the security control review and technical scans. While no pre-existing vulnerabilities were provided, the following new risks have been identified.

\begin{table}[h!]
\centering
\begin{tabular}{@{}p{0.1\textwidth} p{0.25\textwidth} p{0.45\textwidth} p{0.1\textwidth}@{}}
\toprule
\textbf{Risk ID} & \textbf{Risk Name} & \textbf{Description} & \textbf{Severity} \\ \midrule
RISK-001 & Lack of MFA on Endpoints and Sensitive Systems & The absence of MFA for computer logins and access to sensitive data systems allows an attacker with stolen credentials to gain unauthorized access and move laterally within the network. & \textbf{Critical} \\
\addlinespace
RISK-002 & Inadequate Security Awareness Training Program & Without initial or recurring training, employees are more likely to fall victim to phishing, social engineering, or other tactics, leading to credential theft or malware infection. & \textbf{High} \\ \bottomrule
\end{tabular}
\caption{Summary of Identified Risks}
\end{table}

% --- 6. Recommendations ---
\section{Recommendations}
Based on the identified risks, the following prioritized actions are recommended to improve the cybersecurity posture of \textbf{Sovereign Trust}.

\subsection{RISK-001: Lack of MFA (Priority: Critical)}
\begin{itemize}
    \item \textbf{Action:} Procure and deploy a Multi-Factor Authentication (MFA) solution for all user endpoints (desktops, laptops) and for all applications and systems that store or process sensitive organizational data.
    \item \textbf{Justification:} MFA is a foundational security control that mitigates the risk of account takeover via stolen credentials. It is one of the most effective measures to prevent unauthorized access.
\end{itemize}

\subsection{RISK-002: Inadequate Training (Priority: High)}
\begin{itemize}
    \item \textbf{Action:} Establish a formal, mandatory security awareness training program. This program must include:
    \begin{enumerate}
        \item Onboarding training for all new employees before they are granted system access.
        \item Annual refresher training for all staff covering topics like phishing, password hygiene, and acceptable use.
        \item Periodic phishing simulation campaigns to test and reinforce employee awareness.
    \end{enumerate}
    \item \textbf{Justification:} A well-trained workforce acts as a human firewall and is the first line of defense against common cyberattacks. This reduces the likelihood of an initial compromise that could bypass even strong technical controls.
\end{itemize}

\end{document}
```