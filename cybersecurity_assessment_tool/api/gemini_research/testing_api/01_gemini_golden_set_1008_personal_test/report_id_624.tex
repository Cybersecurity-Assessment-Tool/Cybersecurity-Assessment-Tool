```latex
\documentclass[12pt]{article}

% Preamble: Required Packages
\usepackage[margin=1in]{geometry}
\usepackage{pifont} % For checkmarks and crosses
\usepackage{booktabs} % For professional tables
\usepackage{hyperref} % For hyperlinks
\usepackage{url}      % For URL formatting
\usepackage{seqsplit} % For splitting long strings to prevent overflow
\usepackage{graphicx} % For potential logos
\usepackage{xcolor}   % For colors

% Hyperref Setup
\hypersetup{
    colorlinks=true,
    linkcolor=black,
    filecolor=magenta,      
    urlcolor=blue,
    pdftitle={Cybersecurity Posture Report},
    pdfpagemode=FullScreen,
}

% Define custom colors
\definecolor{severitycritical}{HTML}{990000}
\definecolor{severityhigh}{HTML}{D14302}
\definecolor{severitymedium}{HTML}{E5A500}
\definecolor{severitylow}{HTML}{339900}

% Checkmark and Cross definitions
\newcommand{\cmark}{\ding{51}} % Checkmark
\newcommand{\xmark}{\ding{55}} % Cross

\begin{document}

% --- Title Page ---
\begin{titlepage}
    \centering
    \vspace*{1cm}
    \Huge\textbf{Cybersecurity Posture Report}
    \vspace{1.5cm}
    \Large\textbf{Prepared for:}\\
    \vspace{0.5cm}
    \textbf{Skyward Bound}
    \vspace{2cm}
    \rule{\linewidth}{0.5mm}
    \vspace{0.5cm}
    \Large \textit{Analysis of Organizational Controls and Technical Findings}
    \vspace{0.5cm}
    \rule{\linewidth}{0.5mm}
    \vfill
    \large
    \textbf{Date of Report:} \today \\
    \textbf{Author:} Cybersecurity Analyst
\end{titlepage}

\tableofcontents
\newpage

% --- Executive Summary ---
\section{Executive Summary}
This report provides a cybersecurity assessment for \textbf{Skyward Bound}, synthesizing information from an organizational questionnaire, a technical network scan, and a review of pre-existing risks.

The analysis reveals a mixed security posture. The organization demonstrates strong identity and access management practices, with Multi-Factor Authentication (MFA) consistently enforced across email, computer logins, and sensitive data systems. This significantly reduces the risk of unauthorized access through compromised credentials.

However, critical gaps were identified in foundational governance and human-factor controls. The absence of an employee \textbf{Acceptable Use Policy (AUP)} and the lack of \textbf{security awareness training for new employees} represent significant vulnerabilities. These policy and training deficiencies create an environment where employees may be unaware of security best practices, making the organization more susceptible to social engineering and insider threats.

Compounding these issues, the provided technical network scan data and the list of current risks were found to be corrupted and could not be analyzed. This prevents a complete assessment of the external attack surface and the existing risk landscape. Recommendations in this report focus on immediately addressing the identified policy gaps and re-initiating technical assessments to gain a full picture of the organization's security posture.

% --- Organizational Information ---
\section{Organizational Information}
The following details were provided for the assessment. This information is used to establish the context and scope of the analysis.

\begin{table}[h!]
\centering
\begin{tabular}{@{}ll@{}}
\toprule
\textbf{Attribute} & \textbf{Value} \\ \midrule
Organization Name & \textbf{Skyward Bound} \\
Email Domain & \texttt{SkywardBound.org} \\
Website Domain & \seqsplit{\texttt{www.SkywardBound.org}} \\
External IP Address & \texttt{147.194.185.109} \\ \bottomrule
\end{tabular}
\caption{Client Organizational Details.}
\label{tab:org_info}
\end{table}

% --- Security Control Review ---
\section{Security Control Review (Questionnaire Analysis)}
An internal security questionnaire was reviewed to assess the administrative and policy-based controls currently in place. The responses are summarized below. "No" answers indicate significant gaps in the security program.

\begin{table}[h!]
\centering
\begin{tabular}{@{}p{0.7\linewidth}c@{}}
\toprule
\textbf{Control Question} & \textbf{Response} \\ \midrule
Do you require MFA to access email? & \cmark \\
Do you require MFA to log into computers? & \cmark \\
Do you require MFA to access sensitive data systems? & \cmark \\
Does your organization have an employee acceptable use policy? & \xmark \\
Does your organization do security awareness training for new employees? & \xmark \\
Does your organization do security awareness training for all employees at least once per year? & \cmark \\ \bottomrule
\end{tabular}
\caption{Security Controls Questionnaire Results.}
\label{tab:controls}
\end{table}

\subsection*{Analysis of Control Gaps}
\begin{itemize}
    \item \textbf{Acceptable Use Policy (AUP):} The lack of an AUP is a \textbf{critical} governance failure. An AUP is a foundational document that sets clear expectations for employees regarding the use of company assets, data handling, and online behavior. Without it, there is no formal basis for enforcing security standards or taking corrective action against policy violations.
    
    \item \textbf{New Employee Onboarding Training:} The absence of security training during onboarding is a \textbf{high-risk} gap. New hires are often prime targets for phishing and other social engineering attacks. Failing to provide immediate training leaves a window of vulnerability where new staff may inadvertently expose the organization to threats. While annual training is in place, the initial onboarding phase is a critical moment to establish a security-conscious mindset.
\end{itemize}

% --- Technical Scan Results ---
\section{Technical Scan Results}
An external network scan was intended to be performed against the organization's public-facing infrastructure to identify open ports, running services, and potential vulnerabilities.

\textbf{Status: Data Unavailable.} The provided network scan data file (\texttt{Input\_1\_Network\_Scan\_JSON}) was found to be corrupted or incomplete. As a result, no analysis of the external technical posture could be performed. The intended target for this scan was \texttt{[Target IP]}. A full assessment of the external attack surface is not possible without this data.

% --- Risk Assessment ---
\section{Risk Assessment}
This section synthesizes findings from all available data sources. Due to data corruption in the technical scan and existing risk inputs, this assessment is based primarily on the security control review.

\textbf{Note:} The list of pre-existing vulnerabilities (\texttt{Input\_3\_Current\_Risks\_JSON}) was also found to be broken and could not be incorporated into this analysis.

\begin{table}[h!]
\centering
\begin{tabular}{@{}p{0.25\linewidth}p{0.55\linewidth}l@{}}
\toprule
\textbf{Risk Name} & \textbf{Overview} & \textbf{Severity} \\ \midrule
\textbf{Lack of Acceptable Use Policy} & Without a formal AUP, employee actions are not governed by a clear security standard. This increases the risk of data misuse, unauthorized software installation, and unsafe online practices. & \textcolor{severitycritical}{\textbf{Critical}} \\
\addlinespace
\textbf{No Onboarding Security Training} & New employees are not equipped with security knowledge upon hiring, making them highly susceptible to social engineering attacks and accidental policy violations during their initial, most vulnerable period. & \textcolor{severityhigh}{\textbf{High}} \\
\addlinespace
\textbf{Unknown External Attack Surface} & Due to the corrupted network scan data, the organization has no current visibility into its external-facing vulnerabilities, including outdated services, misconfigurations, or exposed sensitive ports. & \textcolor{severityhigh}{\textbf{High}} \\ \bottomrule
\end{tabular}
\caption{Summary of Identified Risks.}
\label{tab:risks}
\end{table}

% --- Recommendations ---
\section{Recommendations}
Based on the analysis, the following actions are recommended to mitigate the identified risks and improve the overall security posture of \textbf{Skyward Bound}.

\begin{enumerate}
    \item \textbf{[Immediate] Develop and Implement an Acceptable Use Policy (AUP):}
    \begin{itemize}
        \item \textbf{Action:} Draft a comprehensive AUP that clearly defines rules for computer, network, email, and internet usage. It should cover topics such as data handling, password security, and prohibited activities.
        \item \textbf{Impact:} Establishes a baseline for security governance and provides a legal and administrative framework for enforcing security standards.
    \end{itemize}
    
    \item \textbf{[Immediate] Institute Mandatory Onboarding Security Training:}
    \begin{itemize}
        \item \textbf{Action:} Create a security awareness training module that is a mandatory part of the new employee onboarding process. This training should cover the new AUP, phishing identification, password hygiene, and incident reporting.
        \item \textbf{Impact:} Reduces the risk posed by new employees and fosters a security-first culture from day one of employment.
    \end{itemize}
    
    \item \textbf{[High Priority] Conduct a New External Vulnerability Scan:}
    \begin{itemize}
        \item \textbf{Action:} Commission a new, authenticated and unauthenticated vulnerability scan of the external IP address (\texttt{147.194.185.109}) and any other public-facing assets.
        \item \textbf{Impact:} Provides critical visibility into the technical attack surface, allowing for the identification and remediation of exploitable vulnerabilities.
    \end{itemize}
    
    \item \textbf{[Medium Priority] Formalize Risk Management Process:}
    \begin{itemize}
        \item \textbf{Action:} Since the existing risk data was unavailable, establish a formal process to identify, assess, and document organizational risks. This should be reviewed on at least an annual basis.
        \item \textbf{Impact:} Ensures that all significant risks (technical, administrative, and physical) are known and managed effectively.
    \end{itemize}
\end{enumerate}

\end{document}
```