```latex
\documentclass[12pt]{article}

% ----------------------------------------------------------------------
% PREAMBLE
% ----------------------------------------------------------------------
\usepackage[margin=1in]{geometry}
\usepackage{pifont} % For checkmarks and crosses
\usepackage{booktabs} % For professional tables
\usepackage[hidelinks]{hyperref} % For clickable links without boxes
\usepackage{url} % For URL formatting
\usepackage{seqsplit} % To split long strings in tt font
\usepackage[utf8]{inputenc}
\usepackage{graphicx}
\usepackage{xcolor}
\usepackage{fancyhdr}

% Define colors
\definecolor{primarycolor}{RGB}{0, 32, 96} % A deep blue

% Header and Footer
\pagestyle{fancy}
\fancyhf{}
\fancyhead[L]{\color{primarycolor}\textbf{Cybersecurity Assessment Report}}
\fancyhead[R]{\color{primarycolor}Hearth \& Home}
\fancyfoot[C]{\thepage}
\renewcommand{\headrulewidth}{0.4pt}
\renewcommand{\footrulewidth}{0.4pt}

% Title Page
\title{
    \vspace{2cm}
    \color{primarycolor}
    \textbf{Cybersecurity Posture Assessment Report} \\
    \large \textit{Prepared for Hearth \& Home}
    \vspace{1.5cm}
}
\author{Cybersecurity Analyst Group}
\date{\today}

% ----------------------------------------------------------------------
% DOCUMENT START
% ----------------------------------------------------------------------
\begin{document}

\maketitle
\thispagestyle{empty}
\newpage
\tableofcontents
\newpage

% ----------------------------------------------------------------------
% SECTION 1: EXECUTIVE SUMMARY
% ----------------------------------------------------------------------
\section{Executive Summary}

This report provides a comprehensive cybersecurity assessment for \textbf{Hearth \& Home}, based on an analysis of network scan data, organizational security controls, and pre-existing risk information. The objective is to identify security gaps, assess the current risk posture, and provide actionable recommendations for improvement.

\paragraph{Key Findings:} The assessment reveals a mixed security posture. While the external network perimeter appears hardened—with the network scan showing no open ports on the tested system—significant internal and policy-related vulnerabilities exist. Critical gaps were identified in access control, specifically the lack of Multi-Factor Authentication (MFA) for computer and sensitive data system access. Furthermore, foundational security policies, such as an Acceptable Use Policy (AUP) and security training for new hires, are absent. These deficiencies substantially increase the risk of unauthorized access, data breaches, and insider threats.

\paragraph{Overall Posture:} The organization has implemented some key controls, such as MFA for email and annual security training for all staff. However, the identified gaps in fundamental security hygiene present a high level of risk that outweighs the current strengths. The recommendations in this report are prioritized to address the most critical vulnerabilities first, with a focus on strengthening access controls and establishing a formal security governance framework.

% ----------------------------------------------------------------------
% SECTION 2: ORGANIZATIONAL INFORMATION
% ----------------------------------------------------------------------
\section{Organizational Information}

The following details were provided for the assessment. This information is used to establish the context and scope of the review.

\begin{tabular}{@{}ll}
\toprule
\textbf{Attribute} & \textbf{Value} \\
\midrule
Organization Name & \textbf{Hearth \& Home} \\
Email Domain & \texttt{HearthHome.net} \\
Website Domain & \url{www.HearthHome.net} \\
External IP Address & \texttt{8.158.82.94} \\
\bottomrule
\end{tabular}

% ----------------------------------------------------------------------
% SECTION 3: SECURITY CONTROL REVIEW
% ----------------------------------------------------------------------
\section{Security Control Review}

A review of the organization's security controls was conducted via a questionnaire. The responses are detailed below, highlighting areas of compliance with security best practices and identifying significant gaps.

\begin{table}[h!]
\centering
\begin{tabular}{@{}p{8cm}ccp{3cm}@{}}
\toprule
\textbf{Control Question} & \multicolumn{2}{c}{\textbf{Response}} & \textbf{Assessment} \\
\midrule
Do you require MFA to access email? & \ding{51} & (Yes) & Strong Control \\
Do you require MFA to log into computers? & \ding{55} & (No) & \textbf{Critical Gap} \\
Do you require MFA to access sensitive data systems? & \ding{55} & (No) & \textbf{Critical Gap} \\
Does your organization have an employee acceptable use policy? & \ding{55} & (No) & High Risk \\
Does your organization do security awareness training for new employees? & \ding{55} & (No) & High Risk \\
Does your organization do security awareness training for all employees at least once per year? & \ding{51} & (Yes) & Good Practice \\
\bottomrule
\end{tabular}
\caption{Security Control Questionnaire Analysis. (\ding{51} = Yes, \ding{55} = No)}
\end{table}

The analysis indicates a lack of enforcement for MFA on critical assets beyond email. The absence of an Acceptable Use Policy and new hire training creates an environment where security responsibilities are not clearly defined or communicated from the outset of employment.

% ----------------------------------------------------------------------
% SECTION 4: TECHNICAL SCAN RESULTS
% ----------------------------------------------------------------------
\section{Technical Scan Results}

An external network scan was performed to identify open ports and exposed services on the target system.

\begin{itemize}
    \item \textbf{Target IP Address:} \texttt{192.168.0.5}
    \item \textbf{Scan Date:} \today
\end{itemize}

The scan results are summarized in the table below.

\begin{table}[h!]
\centering
\begin{tabular}{@{}llll@{}}
\toprule
\textbf{Port} & \textbf{State} & \textbf{Service} & \textbf{Version} \\
\midrule
80 & closed & http & N/A \\
\bottomrule
\end{tabular}
\caption{Nmap Scan Results for Target \texttt{192.168.0.5}.}
\end{table}

\paragraph{Analysis:} The scan indicates that the target system has a secure external posture, with no open ports detected. The previously identified risk from other documentation regarding an "Unencrypted Web Server" on port 80 was not validated by this scan, as the port was found to be closed. This suggests the risk has been remediated or was related to a different asset.

% ----------------------------------------------------------------------
% SECTION 5: RISK ASSESSMENT SUMMARY
% ----------------------------------------------------------------------
\section{Risk Assessment Summary}

This section synthesizes findings from the security control review, technical scan, and pre-existing risk data into a consolidated list of current risks.

\begin{table}[h!]
\centering
\begin{tabular}{@{}lp{7.5cm}l@{}}
\toprule
\textbf{Risk ID} & \textbf{Risk Title \& Description} & \textbf{Severity} \\
\midrule
\textbf{RISK-001} & \textbf{Lack of MFA on Endpoints and Sensitive Systems} \newline Lack of MFA on computers and sensitive data systems exposes the organization to significant risk from credential theft, leading to unauthorized access and lateral movement. & \textbf{Critical} \\
\addlinespace
\textbf{RISK-002} & \textbf{Missing Employee Acceptable Use Policy (AUP)} \newline Without a formal AUP, there is no enforceable standard for employee behavior regarding company assets, data handling, and internet usage, increasing the risk of insider threats. & \textbf{High} \\
\addlinespace
\textbf{RISK-003} & \textbf{No Security Training for New Hires} \newline New employees are not equipped with security knowledge specific to the organization's policies and threat landscape, making them more susceptible to phishing and social engineering attacks. & \textbf{High} \\
\addlinespace
\textbf{RISK-004} & \textbf{Outdated Risk Register (Unencrypted Web Server)} \newline A previously documented risk of an open port 80 was not confirmed by the current scan. This indicates that the risk register may be outdated, leading to misallocated resources. & Informational \\
\bottomrule
\end{tabular}
\caption{Consolidated Risk Summary.}
\end{table}

% ----------------------------------------------------------------------
% SECTION 6: RECOMMENDATIONS
% ----------------------------------------------------------------------
\section{Recommendations}

The following actionable recommendations are provided to mitigate the identified risks and improve the overall security posture of \textbf{Hearth \& Home}.

\subsection{Prioritize and Implement MFA (RISK-001)}
\begin{itemize}
    \item \textbf{Immediate Action:} Begin a phased rollout of Multi-Factor Authentication (MFA) for all employees to access company computers and sensitive internal systems (e.g., databases, financial applications, file servers).
    \item \textbf{Long-Term Strategy:} Enforce MFA as a mandatory, non-negotiable control for access to any system containing sensitive or critical business data.
\end{itemize}

\subsection{Develop and Enforce an AUP (RISK-002)}
\begin{itemize}
    \item \textbf{Immediate Action:} Draft a formal Employee Acceptable Use Policy (AUP). This policy should clearly define rules for using company networks, devices, email, and internet access.
    \item \textbf{Long-Term Strategy:} Require all current and new employees to read and sign the AUP. Integrate the policy into the employee handbook and conduct periodic reviews to ensure it remains current.
\end{itemize}

\subsection{Integrate Security into Onboarding (RISK-003)}
\begin{itemize}
    \item \textbf{Immediate Action:} Create a security awareness training module specifically for new hires. This should cover the AUP, phishing identification, password policies, and how to report a security incident.
    \item \textbf{Long-Term Strategy:} Make this training a mandatory part of the onboarding process for all new employees and contractors before they are granted system access.
\end{itemize}

\subsection{Maintain Risk Register Hygiene (RISK-004)}
\begin{itemize}
    \item \textbf{Immediate Action:} Formally mark the "Unencrypted Web Server" risk as remediated or closed in the organization's risk tracking system.
    \item \textbf{Long-Term Strategy:} Implement a process for periodically reviewing and validating all open risks (e.g., quarterly) to ensure the risk register accurately reflects the current security landscape.
\end{itemize}

\end{document}
```