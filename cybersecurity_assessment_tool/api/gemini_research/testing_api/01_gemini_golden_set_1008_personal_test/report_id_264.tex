Here is the complete and professional LaTeX report, generated based on your instructions and the provided data inputs.

```latex
\documentclass[12pt]{article}

% --- PACKAGES ---
\usepackage[margin=1in]{geometry}
\usepackage{pifont} % For checkmarks and crosses
\usepackage{booktabs} % For professional tables
\usepackage{hyperref} % For clickable links
\usepackage{url}      % For URL formatting
\usepackage{seqsplit} % For splitting long strings
\usepackage{graphicx} % For logo (optional)
\usepackage{xcolor}   % For colors

% --- DOCUMENT METADATA ---
\title{Cybersecurity Posture Assessment Report \\ \large For: \textbf{Iron River Finance}}
\author{Cybersecurity Analysis Division}
\date{\today}

% --- HYPERREF SETUP ---
\hypersetup{
    colorlinks=true,
    linkcolor=blue,
    filecolor=magenta,      
    urlcolor=cyan,
    pdftitle={Cybersecurity Posture Assessment Report},
    pdfpagemode=FullScreen,
}

% --- DOCUMENT START ---
\begin{document}

\maketitle
\thispagestyle{empty}
\newpage

\tableofcontents
\newpage

% ===================================================================
% SECTION 1: EXECUTIVE SUMMARY
% ===================================================================
\section{Executive Summary}

This report provides a cybersecurity posture assessment for \textbf{Iron River Finance}. The analysis is based on a review of organizational security controls provided via a questionnaire. 

It is critical to note that the technical network scan data (\texttt{Input\_1\_Network\_Scan\_JSON}) and the list of pre-existing organizational risks (\texttt{Input\_3\_Current\_Risks\_JSON}) were corrupted and could not be processed for this report. Consequently, this assessment is based solely on the administrative and policy controls self-reported by the organization.

The primary findings from the available data reveal significant gaps in fundamental security controls. The two most critical risks identified are:
\begin{itemize}
    \item \textbf{Lack of Multi-Factor Authentication (MFA) for Email:} The absence of MFA on email accounts, a primary target for attackers, exposes the organization to a high risk of account compromise, data breaches, and business email compromise (BEC) attacks.
    \item \textbf{Absence of Security Awareness Training:} The organization does not conduct security awareness training for new or existing employees. This creates a workforce that is more susceptible to social engineering attacks, such as phishing, which is the leading cause of security incidents.
\end{itemize}

Immediate remediation of these issues is strongly recommended to reduce the organization's attack surface and improve its overall security resilience. A new technical scan is also required to identify potential network-level vulnerabilities.

% ===================================================================
% SECTION 2: ORGANIZATIONAL INFORMATION
% ===================================================================
\section{Organizational Information}

The following details were provided by the client and used as the basis for this assessment.

\begin{table}[h!]
\centering
\begin{tabular}{@{}ll@{}}
\toprule
\textbf{Attribute} & \textbf{Value} \\
\midrule
Organization Name & \textbf{Iron River Finance} \\
Email Domain      & \texttt{IronRiverFinance.org} \\
Website Domain    & \texttt{www.IronRiverFinance.org} \\
External IP Address & \texttt{17.66.112.37} \\
\bottomrule
\end{tabular}
\caption{Client-Provided Organizational Data.}
\end{table}

% ===================================================================
% SECTION 3: SECURITY CONTROL REVIEW
% ===================================================================
\section{Security Control Review}

A review of the organization's security controls was conducted based on a standardized questionnaire. The results below highlight implemented controls (Yes) and identified gaps (No). A "No" response indicates a missing control that typically represents a significant security risk.

\begin{table}[h!]
\centering
\begin{tabular}{@{}p{0.75\linewidth}c@{}}
\toprule
\textbf{Control Question} & \textbf{Response} \\
\midrule
Do you require MFA to access email? & \textcolor{red}{\ding{55}} \\
Do you require MFA to log into computers? & \textcolor{green}{\ding{51}} \\
Do you require MFA to access sensitive data systems? & \textcolor{green}{\ding{51}} \\
Does your organization have an employee acceptable use policy? & \textcolor{green}{\ding{51}} \\
Does your organization do security awareness training for new employees? & \textcolor{red}{\ding{55}} \\
Does your organization do security awareness training for all employees at least once per year? & \textcolor{red}{\ding{55}} \\
\bottomrule
\end{tabular}
\caption{Security Controls Questionnaire Results.}
\end{table}

\subsection*{Analysis of Control Gaps}
\begin{itemize}
    \item \textbf{MFA for Email:} The lack of MFA on email is a critical vulnerability. Email systems are a gateway to an organization's sensitive data and are frequently targeted by threat actors for phishing and account takeover attacks.
    \item \textbf{Security Awareness Training:} The complete absence of a security awareness training program for both new and existing employees is a high-risk gap. Employees represent the first line of defense, and without proper training, they are significantly more likely to fall victim to social engineering tactics, leading to security breaches.
\end{itemize}

% ===================================================================
% SECTION 4: TECHNICAL SCAN RESULTS
% ===================================================================
\section{Technical Scan Results}

The data file intended to contain the results of the network scan against the target IP address (\texttt{17.66.112.37}) was found to be corrupted and could not be parsed. 

\textbf{Status: No Data Available.}

As a result, no analysis of open ports, running services, software versions, or potential vulnerabilities could be performed. It is essential to conduct a new, successful network vulnerability scan to identify and assess technical risks present on the organization's external infrastructure.

% ===================================================================
% SECTION 5: RISK ASSESSMENT
% ===================================================================
\section{Risk Assessment}

The following table summarizes the key risks identified during this assessment. These risks are derived from the control gaps identified in Section 3. Please note that the list of pre-existing vulnerabilities (\texttt{Input\_3\_Current\_Risks\_JSON}) was also unavailable for review.

\begin{table}[h!]
\centering
\begin{tabular}{@{}p{0.1\linewidth}p{0.25\linewidth}p{0.4\linewidth}p{0.1\linewidth}@{}}
\toprule
\textbf{Risk ID} & \textbf{Risk Name} & \textbf{Description} & \textbf{Severity} \\
\midrule
R-01 & Email Account Compromise via Lacking MFA & Email accounts are not protected by MFA, making them vulnerable to takeover from phishing or credential stuffing. A compromised account can lead to data exfiltration, financial fraud, and further network intrusion. & \textbf{Critical} \\
\addlinespace
R-02 & High Susceptibility to Social Engineering & The absence of a security awareness training program leaves employees ill-equipped to identify and report phishing, malware, and other social engineering attacks, increasing the likelihood of a successful breach. & \textbf{High} \\
\addlinespace
R-03 & Unknown External Attack Surface & Due to the failed network scan, the organization has no current visibility into technical vulnerabilities on its internet-facing systems, which may be actively exploited by attackers. & \textbf{High} \\
\bottomrule
\end{tabular}
\caption{Summary of Identified Risks.}
\end{table}

% ===================================================================
% SECTION 6: RECOMMENDATIONS
% ===================================================================
\section{Recommendations}

The following actions are recommended to mitigate the identified risks and improve the overall security posture of \textbf{Iron River Finance}.

\begin{table}[h!]
\centering
\begin{tabular}{@{}p{0.1\linewidth}p{0.25\linewidth}p{0.5\linewidth}p{0.1\linewidth}@{}}
\toprule
\textbf{Ref. ID} & \textbf{Recommendation} & \textbf{Actionable Steps} & \textbf{Priority} \\
\midrule
R-01 & \textbf{Enforce MFA on All Email Accounts} & Immediately configure the \texttt{IronRiverFinance.org} email system to require MFA for all users. Prioritize authenticator apps or hardware tokens over SMS-based methods. & \textbf{Critical} \\
\addlinespace
R-02 & \textbf{Implement Security Awareness Training} & Procure and deploy a security awareness training solution. Mandate foundational training for all new hires and implement an annual refresher course and periodic phishing simulations for all staff. & \textbf{High} \\
\addlinespace
R-03 & \textbf{Conduct a New Technical Assessment} & Schedule and execute a comprehensive external network vulnerability scan as soon as possible. The results should be analyzed to identify and remediate any discovered technical vulnerabilities. & \textbf{High} \\
\bottomrule
\end{tabular}
\caption{Actionable Recommendations.}
\end{table}

\end{document}
```