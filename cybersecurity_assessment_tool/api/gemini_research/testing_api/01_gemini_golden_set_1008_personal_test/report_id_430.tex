```latex
\documentclass[12pt]{article}

% Preamble: Required Packages
\usepackage[margin=1in]{geometry}
\usepackage{pifont} % For checkmarks and crosses
\usepackage{booktabs} % For professional tables
\usepackage{hyperref} % For clickable links
\usepackage{url} % For URL formatting
\usepackage{seqsplit} % To split long strings in tt font
\usepackage{graphicx}
\usepackage{xcolor}

% Document Metadata
\title{Cybersecurity Posture Assessment Report}
\author{Cybersecurity Analysis Division}
\date{\today}

% Hyperref Setup
\hypersetup{
    colorlinks=true,
    linkcolor=blue,
    filecolor=magenta,      
    urlcolor=cyan,
    pdftitle={Cybersecurity Posture Assessment Report},
    pdfpagemode=FullScreen,
}

\begin{document}

\maketitle
\thispagestyle{empty}
\newpage

\tableofcontents
\newpage

% --- 1. Executive Summary ---
\section{Executive Summary}

This report provides a comprehensive analysis of the cybersecurity posture for \textbf{Titanium Core}. The assessment is based on a correlation of data from an external network scan, a security controls questionnaire, and a review of pre-existing risks.

The analysis reveals a mixed security posture. While the organization has implemented foundational policies such as an Acceptable Use Policy and security training for new hires, there are \textbf{critical deficiencies} in access control. The absence of mandatory Multi-Factor Authentication (MFA) for email and computer logins represents a significant and immediate risk of account compromise and unauthorized access. Furthermore, the lack of annual security awareness training for all staff leaves the organization vulnerable to evolving social engineering and phishing attacks.

The external network scan of the designated target IP address did not reveal any open ports or services, which is a positive finding. However, this does not eliminate the possibility of vulnerabilities on other systems or misconfigurations that were not detectable by this specific scan.

Immediate remediation should focus on implementing MFA across all critical systems and establishing a recurring, mandatory security training program.

% --- 2. Organizational Information ---
\section{Organizational Information}

The following details were provided for the assessment. This information is used to establish the context and scope of the analysis.

\begin{tabular}{@{}ll}
\toprule
\textbf{Attribute} & \textbf{Value} \\
\midrule
Organization Name & \textbf{Titanium Core} \\
Email Domain & \texttt{TitaniumCore.org} \\
Website Domain & \url{www.TitaniumCore.org} \\
External IP Address & \texttt{189.45.72.178} \\
\bottomrule
\end{tabular}

% --- 3. Security Control Review ---
\section{Security Control Review}

A security questionnaire was completed to evaluate the implementation of key administrative and technical controls. The responses are summarized below. Items marked with \ding{55} indicate significant gaps in the organization's security framework.

\begin{tabular}{@{}p{0.7\linewidth}c}
\toprule
\textbf{Control Question} & \textbf{Status} \\
\midrule
Do you require MFA to access email? & \ding{55} \\
Do you require MFA to log into computers? & \ding{55} \\
Do you require MFA to access sensitive data systems? & \ding{51} \\
Does your organization have an employee acceptable use policy? & \ding{51} \\
Does your organization do security awareness training for new employees? & \ding{51} \\
Does your organization do security awareness training for all employees at least once per year? & \ding{55} \\
\bottomrule
\end{tabular}

\vspace{1em}
\noindent
\textbf{Key Findings:}
\begin{itemize}
    \item \textbf{Positive Controls:} The organization has successfully implemented MFA for sensitive data systems, has an acceptable use policy, and provides initial security training to new employees. These are strong foundational elements.
    \item \textbf{Critical Gaps:} The lack of MFA for primary communication (email) and endpoint access (computers) creates a significant vulnerability. These are common entry points for attackers.
    \item \textbf{High-Risk Gap:} The absence of annual, recurring security training means that employees' knowledge of threats is not kept current, increasing susceptibility to phishing and other social engineering tactics.
\end{itemize}

% --- 4. Technical Scan Results ---
\section{Technical Scan Results}

An external network vulnerability scan was performed against the designated public IP address to identify open ports and exposed services.

\begin{itemize}
    \item \textbf{Target IP Address:} \texttt{[Target IP]}
    \item \textbf{Scan Date:} \today
\end{itemize}

\subsection{Scan Summary}
The scan completed successfully but did not identify any open TCP or UDP ports on the target host. All probes were either blocked or dropped, suggesting the presence of a well-configured firewall or that the host was not responsive at the time of the scan.

\textbf{Conclusion:} While no exposed services were discovered, which is a positive security indicator, this result does not guarantee the absence of vulnerabilities. It is crucial to ensure that this hardened perimeter is consistent across all of the organization's external-facing assets.

% --- 5. Risk Assessment ---
\section{Risk Assessment}

The following table synthesizes findings from the security control review and technical analysis into a prioritized list of risks. No pre-existing vulnerabilities were provided for this assessment.

\begin{tabular}{@{}p{0.25\linewidth}p{0.5\linewidth}p{0.15\linewidth}}
\toprule
\textbf{Risk Name} & \textbf{Overview} & \textbf{Severity} \\
\midrule
\textbf{Lack of MFA on Email and Endpoints} & The absence of MFA on email and computer logins exposes the organization to a high likelihood of account takeover via credential stuffing or phishing. A single compromised password could grant an attacker broad access. & \textcolor{red}{\textbf{Critical}} \\
\addlinespace
\textbf{Inadequate Security Awareness Training} & Without mandatory annual training, employees are less likely to recognize and report modern phishing, ransomware, and social engineering attempts. This makes them a primary target for initial compromise. & \textcolor{orange}{\textbf{High}} \\
\bottomrule
\end{tabular}

% --- 6. Recommendations ---
\section{Recommendations}

Based on the identified risks, the following actions are recommended to improve the cybersecurity posture of \textbf{Titanium Core}.

\begin{enumerate}
    \item \textbf{Implement Comprehensive MFA (Immediate Priority):}
    \begin{itemize}
        \item \textbf{Action:} Procure and deploy a robust MFA solution for all user accounts.
        \item \textbf{Scope:} This policy must be enforced for all access to company email (e.g., Microsoft 365, Google Workspace) and for all computer/endpoint logins (Windows, macOS).
        \item \textbf{Justification:} This is the single most effective control to prevent unauthorized access resulting from compromised credentials. It directly mitigates the \textbf{Critical} risk identified.
    \end{itemize}
    \vspace{1em}
    \item \textbf{Establish a Continuous Security Training Program (High Priority):}
    \begin{itemize}
        \item \textbf{Action:} Develop or subscribe to a security awareness training platform.
        \item \textbf{Scope:} Mandate that all employees, including executives and contractors, complete a training module at least once per year. Supplement this with regular phishing simulation campaigns to measure effectiveness.
        \item \textbf{Justification:} An educated workforce is a critical layer of defense. This action mitigates the \textbf{High} risk of human error leading to a breach.
    \end{itemize}
\end{enumerate}

% --- 7. Conclusion ---
\section{Conclusion}

\textbf{Titanium Core} has established a baseline of security controls but is currently exposed to significant risk due to gaps in modern identity and access management. The findings of this report, particularly the lack of MFA and recurring training, should be addressed with urgency.

By implementing the recommendations outlined in this report, the organization can substantially reduce its attack surface and build a more resilient and defensible security posture.

\end{document}
```