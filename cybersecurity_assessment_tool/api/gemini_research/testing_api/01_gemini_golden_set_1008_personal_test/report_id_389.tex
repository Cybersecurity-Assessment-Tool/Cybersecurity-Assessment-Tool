```latex
\documentclass[12pt]{article}

% --- PACKAGES ---
\usepackage[margin=1in]{geometry}
\usepackage{pifont} % For checkmarks and crosses
\usepackage{booktabs} % For professional tables
\usepackage{hyperref} % For clickable links
\usepackage{url} % For URL formatting
\usepackage{seqsplit} % To split long strings in tt font
\usepackage{graphicx}
\usepackage{fancyhdr}
\usepackage{xcolor}

% --- DOCUMENT METADATA ---
\title{Cybersecurity Assessment Report \\ \large For: \textbf{Copperhead Cables}}
\author{Cybersecurity Analysis Division}
\date{\today}

% --- HYPERREF SETUP ---
\hypersetup{
    colorlinks=true,
    linkcolor=blue,
    filecolor=magenta,      
    urlcolor=cyan,
    pdftitle={Cybersecurity Assessment Report},
    pdfpagemode=FullScreen,
}

% --- HEADER & FOOTER ---
\pagestyle{fancy}
\fancyhf{}
\lhead{Cybersecurity Assessment Report}
\rhead{\textbf{Copperhead Cables}}
\cfoot{\thepage}

\begin{document}

\maketitle
\thispagestyle{empty}
\newpage

\tableofcontents
\newpage

% --- EXECUTIVE SUMMARY ---
\section{Executive Summary}

This report provides a comprehensive cybersecurity assessment for \textbf{Copperhead Cables}, synthesizing data from technical network scans, a security controls questionnaire, and a review of pre-existing risks.

The assessment reveals a notable dichotomy in the organization's security posture. On one hand, \textbf{Copperhead Cables} demonstrates a strong commitment to administrative and policy-based controls. The security questionnaire indicates consistent enforcement of Multi-Factor Authentication (MFA), a formal acceptable use policy, and a robust security awareness training program. These are commendable foundational elements of a mature security program.

However, a critical weakness was identified on the technical front. A network scan discovered an exposed Remote Desktop Protocol (RDP) service (port 3389) on the host \texttt{10.10.10.51}. This finding is particularly concerning as it echoes a previously identified risk on a different host (\texttt{10.10.10.50}), indicating a potential systemic issue in network configuration or change management. Exposed RDP is a primary attack vector for ransomware and unauthorized access, representing a critical and immediate threat to the organization.

Immediate remediation of the exposed service is required, followed by a strategic review of network segmentation and remote access policies to prevent future occurrences.

% --- ORGANIZATIONAL INFORMATION ---
\section{Organizational Information}

The following details were provided for the assessment.

\begin{table}[h!]
\centering
\begin{tabular}{@{}ll@{}}
\toprule
\textbf{Attribute} & \textbf{Value} \\ \midrule
Organization Name    & \textbf{Copperhead Cables} \\
Email Domain         & \texttt{CopperheadCables.org} \\
Website Domain       & \seqsplit{\url{www.CopperheadCables.org}} \\
External IP Address  & \seqsplit{\texttt{209.153.170.28}} \\ \bottomrule
\end{tabular}
\caption{Client Organizational Data.}
\label{tab:org_data}
\end{table}

% --- SECURITY CONTROL REVIEW ---
\section{Security Control Review}

The following table summarizes the organization's responses to a security controls questionnaire. A green checkmark (\ding{51}) indicates a positive control is in place, while a red cross (\ding{55}) would indicate a gap.

\begin{table}[h!]
\centering
\begin{tabular}{@{}p{0.7\textwidth}c@{}}
\toprule
\textbf{Control Question} & \textbf{Status} \\ \midrule
Do you require MFA to access email? & \textcolor{green}{\ding{51}} \\
Do you require MFA to log into computers? & \textcolor{green}{\ding{51}} \\
Do you require MFA to access sensitive data systems? & \textcolor{green}{\ding{51}} \\
Does your organization have an employee acceptable use policy? & \textcolor{green}{\ding{51}} \\
Does your organization do security awareness training for new employees? & \textcolor{green}{\ding{51}} \\
Does your organization do security awareness training for all employees at least once per year? & \textcolor{green}{\ding{51}} \\ \bottomrule
\end{tabular}
\caption{Security Controls Questionnaire Results.}
\label{tab:controls}
\end{table}

\paragraph{Analyst's Note:} The organization reports a strong and consistent implementation of key administrative security controls. This significantly reduces risks related to phishing, credential theft, and insider threats.

% --- TECHNICAL SCAN RESULTS ---
\section{Technical Scan Results}

A network scan was performed on the target host to identify open ports and exposed services.

\begin{itemize}
    \item \textbf{Target IP Address:} \texttt{10.10.10.51}
    \item \textbf{Host Status:} Up
\end{itemize}

\begin{table}[h!]
\centering
\begin{tabular}{@{}llll@{}}
\toprule
\textbf{Port} & \textbf{State} & \textbf{Service Name} & \textbf{Analysis} \\ \midrule
3389/tcp & Open & \texttt{ms-wbt-server} & \begin{tabular}[c]{@{}l@{}}Remote Desktop Protocol (RDP). This service \\ allows for direct remote administration of the \\ Windows operating system. Exposing RDP \\ directly to a network is a critical risk.\end{tabular} \\ \bottomrule
\end{tabular}
\caption{Open Ports Detected on \texttt{10.10.10.51}.}
\label{tab:scan_results}
\end{table}

% --- RISK ASSESSMENT ---
\section{Consolidated Risk Assessment}

This section correlates findings from the technical scan with pre-existing risk data to provide a unified view of the current threat landscape.

\begin{table}[h!]
\centering
\begin{tabular}{@{}p{0.2\textwidth}p{0.2\textwidth}p{0.15\textwidth}p{0.35\textwidth}@{}}
\toprule
\textbf{Risk Name} & \textbf{Affected Systems} & \textbf{Severity} & \textbf{Description} \\ \midrule
\textbf{New RDP Exposure} & \texttt{10.10.10.51} & \textbf{Critical (9.0)} & The network scan identified an open RDP port (3389) on a new host. This service allows direct, unproxied remote access, posing a severe risk of brute-force attacks, credential theft, and ransomware deployment. \\
\addlinespace
\textbf{Systemic RDP Exposure} & \texttt{10.10.10.50} (Existing) \texttt{10.10.10.51} (New) & \textbf{Critical (9.0)} & A pre-existing risk identified RDP exposure on host \texttt{10.10.10.50}. The discovery of the same vulnerability on a different host indicates a systemic failure in network security configuration, firewall rule management, or the change control process. \\ \bottomrule
\end{tabular}
\caption{Summary of Identified Risks.}
\label{tab:risks}
\end{table}

% --- RECOMMENDATIONS ---
\section{Recommendations}

Based on the correlated findings, the following actions are recommended to mitigate the identified risks and improve the overall security posture.

\subsection{Immediate Actions (Remediation)}

\begin{enumerate}
    \item \textbf{Restrict RDP Access on \texttt{10.10.10.51}:} \textbf{(Priority: CRITICAL)}
    \begin{itemize}
        \item If remote access to this host is not required, the most effective solution is to disable the RDP service and block port 3389 via the host-based and network firewalls.
        \item If remote access is business-critical, immediately implement a firewall rule to restrict access to port 3389 to only authorized, internal administrative IP addresses.
    \end{itemize}
\end{enumerate}

\subsection{Strategic Actions (Long-Term Improvement)}

\begin{enumerate}
    \item \textbf{Implement a Secure Remote Access Solution:}
    \begin{itemize}
        \item Deploy a Virtual Private Network (VPN) or Zero Trust Network Access (ZTNA) solution for all administrative access to internal systems.
        \item This solution should be configured to require Multi-Factor Authentication (MFA), leveraging the organization's existing strong MFA policies. This eliminates the need to expose services like RDP directly.
    \end{itemize}
    \item \textbf{Conduct a Network Security Architecture Review:}
    \begin{itemize}
        \item Perform a comprehensive audit of all firewall rules and network segmentation policies to ensure a "deny-by-default" principle is enforced.
        \item Identify and remediate any other services that may be improperly exposed.
    \end{itemize}
    \item \textbf{Enhance Change Management and Vulnerability Scanning:}
    \begin{itemize}
        \item Integrate security checks into the IT change management process to prevent insecure configurations from being deployed.
        \item Establish a routine, authenticated vulnerability scanning program for the internal network to proactively identify and remediate such exposures before they become critical risks.
    \end{itemize}
\end{enumerate}

\end{document}
```