```latex
\documentclass[12pt]{article}

% Preamble: Required Packages
\usepackage[a4paper, margin=1in]{geometry}
\usepackage{pifont} % For checkmarks and crosses
\usepackage{booktabs} % For professional tables
\usepackage{hyperref} % For clickable links and metadata
\usepackage{url} % For formatting URLs
\usepackage{seqsplit} % To split long strings in tt font
\usepackage{graphicx}
\usepackage{xcolor}
\usepackage{fancyhdr}

% --- Document Metadata ---
\hypersetup{
    colorlinks=true,
    linkcolor=blue,
    filecolor=magenta,      
    urlcolor=cyan,
    pdftitle={Cybersecurity Posture Assessment Report},
    pdfauthor={Cybersecurity Analyst},
    pdfsubject={Security Analysis},
    pdfkeywords={Cybersecurity, Risk Assessment, Nmap, LaTeX},
    pdftoolbar=true,
}

% --- Customizations ---
\pagestyle{fancy}
\fancyhf{} % clear all header and footers
\fancyhead[L]{Cybersecurity Posture Assessment}
\fancyhead[R]{Ember Glow Hospitality}
\fancyfoot[C]{\thepage}
\renewcommand{\headrulewidth}{0.4pt}
\renewcommand{\footrulewidth}{0.4pt}

% Define colors for severity
\definecolor{critical}{HTML}{990000}
\definecolor{high}{HTML}{D1410C}
\definecolor{medium}{HTML}{E5A00D}
\definecolor{low}{HTML}{4CAF50}

% Helper command for severity labels
\newcommand{\severitylabel}[2]{\colorbox{#1}{\textcolor{white}{\textbf{\strut #2}}}}

\begin{document}

% --- Title Page ---
\begin{titlepage}
    \centering
    \vspace*{1cm}
    \includegraphics[width=0.4\textwidth]{example-image-a} % Placeholder for company logo
    
    \vspace{1.5cm}
    
    \Huge
    \textbf{Cybersecurity Posture Assessment Report}
    
    \vspace{1.5cm}
    
    \Large
    Prepared for: \textbf{Ember Glow Hospitality}
    
    \vspace{2cm}
    
    \normalsize
    \textbf{Date of Report:} \today \\
    \textbf{Analysis Period:} Based on data provided on \today
    
    \vfill
    
    \normalsize
    \textit{This report contains sensitive information and is intended solely for the use of the recipient. Distribution without prior written consent is prohibited.}
    
\end{titlepage}

\tableofcontents
\newpage

% --- Section 1: Executive Summary ---
\section{Executive Summary}
This report provides a comprehensive cybersecurity assessment for \textbf{Ember Glow Hospitality}, based on a correlation of network scan data, organizational security controls, and pre-existing risk information.

The analysis reveals a \textbf{high-risk security posture} characterized by critical deficiencies in fundamental security controls and a confirmed, severe technical vulnerability. The most pressing finding is a publicly exposed internal service (SSH on localhost), which represents a critical threat (CVSS 10.0) and requires immediate remediation.

Furthermore, significant gaps were identified in identity and access management and employee security awareness. The absence of Multi-Factor Authentication (MFA) for email and computer access, coupled with a lack of a formal security training program, exposes the organization to a high likelihood of successful phishing attacks, business email compromise (BEC), and ransomware incidents.

Immediate and decisive action is required to address these findings. This report outlines prioritized, actionable recommendations to mitigate the identified risks and strengthen the overall security posture of \textbf{Ember Glow Hospitality}.

% --- Section 2: Organizational Information ---
\section{Organizational Information}
The following details were provided for the assessment.
\begin{center}
\begin{tabular}{@{}ll@{}}
\toprule
\textbf{Attribute} & \textbf{Value} \\ \midrule
Organization Name & \textbf{Ember Glow Hospitality} \\
Email Domain & \seqsplit{\texttt{EmberGlowHospitality.net}} \\
Website Domain & \url{www.EmberGlowHospitality.net} \\
External IP Address & \seqsplit{\texttt{134.36.6.208}} \\ \bottomrule
\end{tabular}
\end{center}

% --- Section 3: Security Control Review ---
\section{Security Control Review}
A review of the organization's security controls was conducted via a questionnaire. The results highlight critical gaps in foundational security practices. A summary of the responses is provided in Table \ref{tab:controls}.

\begin{table}[h!]
\centering
\caption{Security Controls Questionnaire Results}
\label{tab:controls}
\begin{tabular}{@{}p{0.6\linewidth}c@{}}
\toprule
\textbf{Control Question} & \textbf{Status} \\ \midrule
Does your organization have an employee acceptable use policy? & \ding{51} \\
Do you require MFA to access sensitive data systems? & \ding{51} \\
\addlinespace
Do you require MFA to access email? & \textbf{\color{critical}\ding{55}} \\
Do you require MFA to log into computers? & \textbf{\color{critical}\ding{55}} \\
Does your organization do security awareness training for new employees? & \textbf{\color{critical}\ding{55}} \\
Does your organization do security awareness training for all employees at least once per year? & \textbf{\color{critical}\ding{55}} \\ \bottomrule
\end{tabular}
\end{table}

\subsection*{Analysis of Controls}
The responses indicate a severe deficiency in two key areas:
\begin{itemize}
    \item \textbf{Identity and Access Management:} While MFA is commendably used for sensitive data systems, its absence for email and general computer logins is a critical oversight. Email is the primary vector for phishing and business email compromise. Unprotected computer logins remove a vital layer of defense against unauthorized access.
    \item \textbf{Security Awareness:} The complete lack of a security awareness training program for both new and existing employees leaves the organization highly vulnerable to social engineering attacks. Employees are the first line of defense, and without training, they are unprepared to identify and report threats.
\end{itemize}

% --- Section 4: Technical Scan Results ---
\section{Technical Scan Results}
A network scan was performed to identify exposed services and potential vulnerabilities on the specified target.

\subsection*{Scan Details}
\begin{itemize}
    \item \textbf{Target IP:} \seqsplit{\texttt{127.0.0.1}}
    \item \textbf{Scan Date:} Data provided on \today
\end{itemize}

\subsection*{Open Ports and Services}
The scan identified the following open port.
\begin{table}[h!]
\centering
\caption{Open Port Findings for \texttt{127.0.0.1}}
\label{tab:nmap}
\begin{tabular}{@{}llll@{}}
\toprule
\textbf{Port} & \textbf{State} & \textbf{Common Service} & \textbf{Notes} \\ \midrule
22/tcp & Open & SSH (Secure Shell) & Service version not available in scan data. \\ \bottomrule
\end{tabular}
\end{table}

\subsection*{Analysis of Technical Findings}
The scan confirms that port 22 (SSH) is open on the target host \texttt{127.0.0.1}. This IP address is the "localhost" or "loopback" interface, which is intended for services to communicate with themselves on the same machine. 

\textbf{Exposing a service on the localhost interface to an external scan is a critical misconfiguration.} It indicates that a service intended only for internal use is accessible from the outside, bypassing perimeter defenses. This finding directly corroborates the pre-existing risk identified as "Localhost Exposed" and validates its critical severity.

% --- Section 5: Consolidated Risk Assessment ---
\section{Consolidated Risk Assessment}
By correlating the security control gaps and technical findings, we have compiled a summary of the most significant risks facing the organization.

\begin{table}[h!]
\centering
\caption{Summary of Identified Risks}
\label{tab:risks}
\begin{tabular}{@{}p{0.25\linewidth}p{0.5\linewidth}p{0.15\linewidth}@{}}
\toprule
\textbf{Risk Name} & \textbf{Description} & \textbf{Severity} \\ \midrule
\textbf{Localhost Exposed} & The SSH service on the loopback interface (\texttt{127.0.0.1}) is exposed externally. This allows attackers to directly target a service that should be internal, bypassing firewalls. & \severitylabel{critical}{Critical} \\
\addlinespace
\textbf{Lack of Multi-Factor Authentication (MFA)} & The absence of MFA on email and computer logins creates a high risk of account takeover via credential theft or phishing. This could lead to data breaches, financial fraud, and ransomware. & \severitylabel{high}{High} \\
\addlinespace
\textbf{Insufficient Security Awareness Training} & The lack of a formal training program makes employees highly susceptible to phishing and other social engineering tactics, turning them into an unintentional insider threat. & \severitylabel{high}{High} \\ \bottomrule
\end{tabular}
\end{table}

% --- Section 6: Recommendations ---
\section{Recommendations}
The following prioritized recommendations are provided to address the identified risks. Swift implementation is crucial to reducing the organization's exposure.

\subsection*{Priority 1: Remediate Critical Exposure (Immediate)}
\begin{description}
    \item[Risk Addressed:] Localhost Exposed
    \item[Action:] Immediately reconfigure the network firewall and the SSH service configuration on the affected host. The service must be bound \textbf{only} to the internal network interface or configured to deny all traffic that does not originate from a trusted internal source. The goal is to make the service completely inaccessible from the public internet.
\end{description}

\subsection*{Priority 2: Implement Comprehensive MFA (Urgent)}
\begin{description}
    \item[Risk Addressed:] Lack of Multi-Factor Authentication (MFA)
    \item[Action:] Procure and deploy an MFA solution across the organization.
    \begin{enumerate}
        \item \textbf{Phase 1:} Enforce MFA for all email accounts, starting with administrative and executive accounts.
        \item \textbf{Phase 2:} Enforce MFA for all remote access solutions (e.g., VPN).
        \item \textbf{Phase 3:} Enforce MFA for all computer/endpoint logins.
    \end{enumerate}
\end{description}

\subsection*{Priority 3: Establish Security Awareness Program (High)}
\begin{description}
    \item[Risk Addressed:] Insufficient Security Awareness Training
    \item[Action:] Develop and implement a mandatory security awareness training program.
    \begin{enumerate}
        \item Implement foundational training for all new hires as part of the onboarding process.
        \item Conduct annual, mandatory refresher training for all employees covering topics like phishing, password security, and acceptable use.
        \item Perform periodic phishing simulations to test and reinforce employee awareness.
    \end{enumerate}
\end{description}

\end{document}
```