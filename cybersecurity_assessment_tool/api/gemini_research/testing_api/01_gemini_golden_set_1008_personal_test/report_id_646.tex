```latex
\documentclass[12pt]{article}

% Preamble: Required packages for formatting and symbols
\usepackage[margin=1in]{geometry}
\usepackage{pifont} % For checkmark and cross symbols
\usepackage{booktabs} % For professional-looking tables
\usepackage{hyperref} % For clickable links
\usepackage{url} % For formatting URLs
\usepackage{seqsplit} % For splitting long strings without breaking words

% Document Metadata
\title{Cybersecurity Posture Assessment Report}
\author{Vivid Vision}
\date{\today}

\begin{document}

\maketitle
\thispagestyle{empty}
\newpage

\tableofcontents
\newpage

% --- Executive Summary ---
\section*{Executive Summary}

This report provides a cybersecurity posture assessment for \textbf{Vivid Vision}, conducted on \today. The analysis is based on a combination of self-reported organizational data, a network vulnerability scan, and a review of pre-existing risks.

The assessment reveals a mixed security posture. The organization has implemented foundational security controls, such as requiring Multi-Factor Authentication (MFA) for email and computer access. The external network scan of the target IP address did not identify any open ports or exposed services, which is a positive security finding, suggesting a strong network perimeter or that no services are publicly hosted at that address.

However, significant areas of concern were identified through the security controls questionnaire. Critical gaps exist in the protection of sensitive data systems, which currently lack an MFA requirement. Furthermore, the complete absence of a security awareness training program for both new and existing employees presents a high risk, as it leaves the organization vulnerable to social engineering attacks like phishing.

This report details these findings and provides actionable recommendations to mitigate the identified risks and strengthen the overall security posture of \textbf{Vivid Vision}.

% --- Organizational Information ---
\section*{Organizational Information}

The following information was provided for the assessment.

\begin{table}[h!]
\centering
\begin{tabular}{@{}ll@{}}
\toprule
\textbf{Attribute} & \textbf{Value} \\ \midrule
Organization Name & Vivid Vision \\
Email Domain & \texttt{VividVision.org} \\
Website Domain & \url{www.VividVision.org} \\
External IP Address & \texttt{158.226.44.63} \\ \bottomrule
\end{tabular}
\caption{Client Organizational Data}
\label{tab:org_data}
\end{table}

% --- Security Control Review ---
\section*{Security Control Review}

A review of the organization's security controls was conducted based on a standardized questionnaire. The results are summarized below. Answers marked with \ding{55} (No) indicate potential security gaps that require attention.

\begin{table}[h!]
\centering
\begin{tabular}{@{}lc@{}}
\toprule
\textbf{Security Control Question} & \textbf{Status} \\ \midrule
Do you require MFA to access email? & \ding{51} \\
Do you require MFA to log into computers? & \ding{51} \\
Do you require MFA to access sensitive data systems? & \textbf{\color{red}\ding{55}} \\
Does your organization have an employee acceptable use policy? & \ding{51} \\
Does your organization do security awareness training for new employees? & \textbf{\color{red}\ding{55}} \\
Does your organization do security awareness training for all employees at least once per year? & \textbf{\color{red}\ding{55}} \\ \bottomrule
\end{tabular}
\caption{Security Controls Questionnaire Results}
\label{tab:controls}
\end{table}

\subsection*{Analysis of Findings}
The questionnaire highlights two primary areas of weakness:
\begin{itemize}
    \item \textbf{Access Control:} While MFA is enforced for standard email and computer access, the failure to extend this critical control to sensitive data systems leaves the organization's most valuable information vulnerable to unauthorized access, should an attacker compromise user credentials.
    \item \textbf{Security Awareness:} The complete lack of a security awareness training program is a significant deficiency. Without training, employees are more likely to fall victim to phishing, malware, and other social engineering tactics, turning the workforce into an unintentional insider threat.
\end{itemize}

% --- Technical Scan Results ---
\section*{Technical Scan Results}

An external network scan was performed against the designated target IP address.

\begin{table}[h!]
\centering
\begin{tabular}{@{}ll@{}}
\toprule
\textbf{Scan Parameter} & \textbf{Value} \\ \midrule
Target IP Address & \texttt{[Target IP]} \\
Scan Date & N/A (No services found) \\ \bottomrule
\end{tabular}
\caption{Network Scan Metadata}
\label{tab:scan_meta}
\end{table}

\subsection*{Summary of Findings}
The network scan completed successfully but did not detect any open TCP/IP ports or running services on the target host \texttt{[Target IP]}.

\textbf{Conclusion:} This result indicates a strong network perimeter. It is likely that a firewall is in place and configured to block all unsolicited inbound traffic, or there are no services intended to be publicly accessible from this IP address. No vulnerabilities were identified from this scan.

% --- Consolidated Risk Assessment ---
\section*{Consolidated Risk Assessment}

This section correlates findings from the security control review, technical scan, and pre-existing risk data. As no pre-existing or technical risks were found, the assessment focuses on the policy and procedural gaps identified.

\begin{table}[h!]
\centering
\begin{tabular}{@{}p{0.1\linewidth} p{0.25\linewidth} p{0.4\linewidth} p{0.1\linewidth}@{}}
\toprule
\textbf{Risk ID} & \textbf{Risk Name} & \textbf{Overview} & \textbf{Severity} \\ \midrule
RISK-001 & Lack of MFA for Sensitive Data Systems & User accounts with access to sensitive or critical data are not protected by MFA. A single password compromise could lead to a significant data breach. & \textbf{Critical} \\
\addlinespace
RISK-002 & No Security Awareness Training Program & Employees are not trained to identify or respond to security threats like phishing or social engineering. This elevates the risk of human error leading to a security incident. & \textbf{High} \\ \bottomrule
\end{tabular}
\caption{Summary of Identified Risks}
\label{tab:risks}
\end{table}

% --- Recommendations ---
\section*{Recommendations}

Based on the consolidated risk assessment, the following actions are recommended to mitigate the identified risks and improve the overall security posture of \textbf{Vivid Vision}.

\begin{enumerate}
    \item \textbf{Implement Comprehensive MFA (RISK-001):}
    \begin{itemize}
        \item \textbf{Action:} Immediately prioritize and deploy MFA across all systems and applications that store, process, or transmit sensitive data (e.g., financial systems, customer databases, intellectual property repositories).
        \item \textbf{Impact:} Drastically reduces the risk of unauthorized access and data breaches resulting from compromised credentials.
        \item \textbf{Priority:} \textbf{Critical}
    \end{itemize}
    \vspace{1em}
    \item \textbf{Establish a Security Awareness Training Program (RISK-002):}
    \begin{itemize}
        \item \textbf{Action:} Develop and implement a formal security awareness training program. This program must include mandatory onboarding training for all new hires and annual refresher training for all existing employees.
        \item \textbf{Content:} Training should cover key topics such as phishing identification, password security, safe browsing habits, and corporate data handling policies.
        \item \textbf{Priority:} \textbf{High}
    \end{itemize}
\end{enumerate}

\end{document}
```