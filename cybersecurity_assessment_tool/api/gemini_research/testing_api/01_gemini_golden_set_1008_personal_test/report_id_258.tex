```latex
\documentclass[12pt]{article}

% --- PACKAGES ---
\usepackage[margin=1in]{geometry}
\usepackage{pifont} % For checkmarks and crosses
\usepackage{booktabs} % For professional tables
\usepackage{hyperref} % For clickable links
\usepackage{url} % For URL formatting
\usepackage{seqsplit} % To split long monospaced text
\usepackage{graphicx}
\usepackage{xcolor}

% --- DOCUMENT SETUP ---
\hypersetup{
    colorlinks=true,
    linkcolor=blue,
    filecolor=magenta,      
    urlcolor=cyan,
    pdftitle={Cybersecurity Assessment Report},
    pdfpagemode=FullScreen,
}

\newcommand{\yes}{\ding{51}} % Green checkmark
\newcommand{\no}{\ding{55}}  % Red X

\begin{document}

% --- TITLE PAGE ---
\begin{titlepage}
    \centering
    \vspace*{1cm}
    
    \Huge
    \textbf{Cybersecurity Assessment Report}
    
    \vspace{1.5cm}
    
    \Large
    Prepared for: \\
    \vspace{0.5cm}
    \textbf{Borealis Tech}
    
    \vspace{2cm}
    
    \large
    Date of Report: \today
    
    \vfill
    
    \large
    \textit{This report contains sensitive information and should be handled with care.}
    
\end{titlepage}

\tableofcontents
\newpage

% --- 1. EXECUTIVE SUMMARY ---
\section{Executive Summary}

This report details the findings of a cybersecurity assessment conducted for \textbf{Borealis Tech}. The assessment incorporated a review of organizational security controls, an external network scan, and a correlation with pre-existing risk data.

The primary finding of this assessment is the discovery of several critical gaps in administrative and access controls. While a previously identified technical risk concerning an unencrypted web server on host \texttt{192.168.0.5} appears to have been remediated, significant new risks have been identified through the security questionnaire.

The most critical risks stem from the lack of Multi-Factor Authentication (MFA) for accessing email and sensitive data systems. This exposes the organization to a high likelihood of account compromise and subsequent data breaches. Additionally, the absence of a formal Acceptable Use Policy (AUP) represents a significant governance and operational risk.

Immediate remediation efforts should be focused on implementing MFA across all critical systems and establishing a formal AUP to govern employee use of company assets.

% --- 2. ORGANIZATIONAL INFORMATION ---
\section{Organizational Information}

The following details were provided for the assessment scope.

\begin{itemize}
    \item \textbf{Organization Name:} Borealis Tech
    \item \textbf{Email Domain:} \texttt{BorealisTech.net}
    \item \textbf{Website Domain:} \seqsplit{\url{www.BorealisTech.net}}
    \item \textbf{External IP Address:} \texttt{218.64.78.112}
\end{itemize}

% --- 3. SECURITY CONTROL REVIEW ---
\section{Security Control Review}

A review of administrative and procedural controls was conducted via a security questionnaire. The responses indicate several areas requiring immediate attention. A "No" response highlights a significant gap in the organization's security posture.

\begin{table}[h!]
\centering
\caption{Security Control Questionnaire Responses}
\begin{tabular}{p{0.8\linewidth} c}
\toprule
\textbf{Control Question} & \textbf{Response} \\
\midrule
Do you require MFA to access email? & \no \\
Do you require MFA to log into computers? & \yes \\
Do you require MFA to access sensitive data systems? & \no \\
Does your organization have an employee acceptable use policy? & \no \\
Does your organization do security awareness training for new employees? & \yes \\
Does your organization do security awareness training for all employees at least once per year? & \yes \\
\bottomrule
\end{tabular}
\end{table}

% --- 4. TECHNICAL SCAN RESULTS ---
\section{Technical Scan Results}

A network scan was performed to identify exposed services and potential technical vulnerabilities.

\begin{itemize}
    \item \textbf{Target IP Address:} \texttt{192.168.0.5}
    \item \textbf{Scan Date:} \today
\end{itemize}

The scan revealed that the target host is online, but the specific ports scanned were not open. The results are detailed below.

\begin{table}[h!]
\centering
\caption{Nmap Scan Port Summary for \texttt{192.168.0.5}}
\begin{tabular}{lllll}
\toprule
\textbf{Port} & \textbf{State} & \textbf{Service} & \textbf{Product} & \textbf{Version} \\
\midrule
80/tcp & closed & http & N/A & N/A \\
\bottomrule
\end{tabular}
\end{table}

\paragraph{Analysis:} The scan indicates that port 80 (HTTP) is explicitly closed on the target host. This contradicts a pre-existing risk entry, suggesting that the vulnerability may have been recently remediated on this specific system. No other open ports were identified in this scan.

% --- 5. RISK ASSESSMENT SUMMARY ---
\section{Risk Assessment Summary}

The following table synthesizes findings from the security control review, the technical scan, and pre-existing risk data. Risks are prioritized based on their potential impact on the organization.

\begin{table}[h!]
\centering
\caption{Consolidated Risk Register}
\begin{tabular}{p{0.2\linewidth} p{0.55\linewidth} p{0.15\linewidth}}
\toprule
\textbf{Risk Name} & \textbf{Description} & \textbf{Severity} \\
\midrule
\textbf{Lack of MFA for Email} & Email accounts are protected by passwords only. This exposes the organization to a high risk of Business Email Compromise (BEC), phishing success, and unauthorized access to sensitive communications. & \textbf{Critical} \\
\addlinespace
\textbf{Lack of MFA for Sensitive Data} & Access to systems containing sensitive data is not protected by a second authentication factor. A single compromised credential could lead directly to a significant data breach. & \textbf{Critical} \\
\addlinespace
\textbf{Missing Acceptable Use Policy} & The absence of a formal AUP creates ambiguity for employees regarding the proper use of corporate assets and data. This complicates enforcement and increases insider threat risk. & \textbf{High} \\
\addlinespace
\textbf{Unencrypted Web Server (Remediated)} & A pre-existing risk noted that port 80 was open. Our scan of \texttt{192.168.0.5} found this port to be closed, indicating this specific risk has been remediated on the scanned host. & Informational \\
\bottomrule
\end{tabular}
\end{table}

% --- 6. RECOMMENDATIONS ---
\section{Recommendations}

Based on the identified risks, the following actions are recommended to improve the security posture of \textbf{Borealis Tech}.

\subsection{Critical Priority}
\begin{description}
    \item[Implement MFA Everywhere:] Immediately deploy and enforce Multi-Factor Authentication (MFA) for all users across all critical systems.
    \begin{itemize}
        \item \textbf{Email:} Enable MFA for the \texttt{BorealisTech.net} email domain. This is the single most effective control to prevent account takeovers.
        \item \textbf{Sensitive Systems:} Identify all systems that store or process sensitive data and enforce MFA for access.
    \end{itemize}
\end{description}

\subsection{High Priority}
\begin{description}
    \item[Develop and Implement an AUP:] Draft a formal Acceptable Use Policy that clearly defines the rules and expectations for using company networks, devices, and data. This policy should be communicated to all employees and acknowledgment of receipt should be tracked.
\end{description}

\subsection{Informational}
\begin{description}
    \item[Update Risk Register:] The internal risk register should be updated to reflect that the "Unencrypted Web Server" risk on host \texttt{192.168.0.5} is resolved.
    \item[Continuous Monitoring:] Continue to perform regular, authenticated vulnerability scans across the network to validate that controls remain effective and to identify new vulnerabilities.
\end{description}

\end{document}
```