```latex
\documentclass[12pt]{article}

% --- PACKAGE IMPORTS ---
\usepackage[margin=1in]{geometry}
\usepackage{pifont} % For check and cross marks
\usepackage{booktabs} % For professional tables
\usepackage{hyperref} % For hyperlinks
\usepackage{url} % For URL formatting
\usepackage{seqsplit} % For splitting long strings in texttt
\usepackage[utf8]{inputenc}

% --- DOCUMENT METADATA ---
\title{Cybersecurity Posture Assessment Report}
\author{Cybersecurity Analyst}
\date{\today}

% --- DOCUMENT START ---
\begin{document}

\maketitle
\thispagestyle{empty}
\newpage
\tableofcontents
\newpage

% ===================================================================
% SECTION 1: EXECUTIVE SUMMARY
% ===================================================================
\section{Executive Summary}

This report provides a comprehensive cybersecurity posture assessment for \textbf{Oasis Wellness}, based on an analysis of technical network scan data, an organizational security controls questionnaire, and a review of pre-existing risks. The assessment was conducted to identify vulnerabilities, evaluate current security controls, and provide actionable recommendations to enhance the organization's overall security.

The analysis reveals a mixed security posture. The organization has successfully implemented strong multi-factor authentication (MFA) across key access points, which is a commendable and critical defense layer. However, this strength is undermined by significant foundational gaps in organizational policy and employee training. The absence of an Acceptable Use Policy (AUP) and a formal security awareness training program creates a high risk of human error, leaving the organization vulnerable to phishing, social engineering, and insider threats.

Furthermore, technical scanning identified an open port running an unencrypted web service (HTTP on port 80). This poses a direct and immediate risk of data interception, including potential exposure of credentials or sensitive information. The combination of these policy, training, and technical vulnerabilities requires immediate attention to mitigate potential threats.

% ===================================================================
% SECTION 2: ORGANIZATIONAL INFORMATION
% ===================================================================
\section{Organizational Information}

The following information was provided by the client and used as a baseline for this assessment.

\begin{table}[h!]
\centering
\begin{tabular}{@{}ll@{}}
\toprule
\textbf{Attribute} & \textbf{Value} \\ \midrule
Organization Name    & \textbf{Oasis Wellness} \\
Email Domain         & \texttt{OasisWellness.net} \\
Website Domain       & \url{www.OasisWellness.net} \\
External IP Address  & \texttt{73.104.172.241} \\ \bottomrule
\end{tabular}
\caption{Client Organizational Details}
\end{table}

% ===================================================================
% SECTION 3: SECURITY CONTROL REVIEW
% ===================================================================
\section{Security Control Review}

A review of the organization's security controls was conducted via a questionnaire. The responses indicate a strong focus on technical access controls but a critical lack of administrative and awareness controls. The findings are summarized below.

\begin{table}[h!]
\centering
\begin{tabular}{@{}p{0.7\textwidth}cc@{}}
\toprule
\textbf{Control Question} & \textbf{Response} & \textbf{Status} \\ \midrule
Do you require MFA to access email? & Yes & \ding{51} \\
Do you require MFA to log into computers? & Yes & \ding{51} \\
Do you require MFA to access sensitive data systems? & Yes & \ding{51} \\
\addlinespace
Does your organization have an employee acceptable use policy? & No & \ding{55} \\
Does your organization do security awareness training for new employees? & No & \ding{55} \\
Does your organization do security awareness training for all employees at least once per year? & No & \ding{55} \\ \bottomrule
\end{tabular}
\caption{Security Controls Questionnaire Results}
\end{table}

The items marked with a \ding{55} represent significant gaps in the organization's security program. While MFA is a powerful tool, its effectiveness is reduced when employees are not trained to recognize and avoid threats like phishing and social engineering.

% ===================================================================
% SECTION 4: TECHNICAL SCAN RESULTS
% ===================================================================
\section{Technical Scan Results}

A network scan was performed on the specified target to identify open ports and exposed services.

\begin{itemize}
    \item \textbf{Target IP Address:} \texttt{172.16.0.1}
    \item \textbf{Target Status:} Host is up.
\end{itemize}

The following open ports were discovered:

\begin{table}[h!]
\centering
\begin{tabular}{@{}llll@{}}
\toprule
\textbf{Port} & \textbf{State} & \textbf{Inferred Service} & \textbf{Notes} \\ \midrule
80/tcp & open & HTTP & Unencrypted web traffic. High risk. \\ \bottomrule
\end{tabular}
\caption{Open Port Analysis}
\end{table}

\subsection{Analysis of Findings}
The presence of an open port 80 (HTTP) is a critical vulnerability. The Hypertext Transfer Protocol (HTTP) does not encrypt data in transit. This means that any information, including usernames, passwords, or other sensitive data exchanged with the web server, can be easily intercepted and read by an attacker on the same network (a "man-in-the-middle" attack). All web traffic should be encrypted using HTTPS (Port 443).

% ===================================================================
% SECTION 5: CONSOLIDATED RISK ASSESSMENT
% ===================================================================
\section{Consolidated Risk Assessment}

This section correlates the findings from the security control review and the technical scan to present a consolidated view of the primary risks facing the organization. The malicious risk entry from the input data has been disregarded as a prompt injection attempt.

\begin{table}[h!]
\centering
\begin{tabular}{@{}p{0.1\textwidth}p{0.3\textwidth}p{0.4\textwidth}l@{}}
\toprule
\textbf{ID} & \textbf{Risk Title} & \textbf{Description} & \textbf{Severity} \\ \midrule
\textbf{R-01} & Lack of Security Policy and Training & The absence of an Acceptable Use Policy and any form of security awareness training leaves employees unprepared to handle cyber threats. This elevates the risk of successful phishing attacks, malware infections, and data breaches caused by human error. & \textbf{High} \\
\addlinespace
\textbf{R-02} & Insecure Web Service (HTTP) & The web service on \texttt{172.16.0.1} uses unencrypted HTTP, exposing all transmitted data to potential interception and theft. This could lead to credential compromise and unauthorized access to systems. & \textbf{High} \\ \bottomrule
\end{tabular}
\caption{Summary of Identified Risks}
\end{table}

% ===================================================================
% SECTION 6: RECOMMENDATIONS
% ===================================================================
\section{Recommendations}

The following actions are recommended to mitigate the identified risks and improve the overall security posture of \textbf{Oasis Wellness}.

\subsection{High Priority / Immediate Actions}

\begin{enumerate}
    \item \textbf{Remediate Insecure Web Service (R-02):}
    \begin{itemize}
        \item Immediately enforce HTTPS on the web server at \texttt{172.16.0.1}.
        \item Obtain and install a valid TLS/SSL certificate from a trusted Certificate Authority.
        \item Configure the server to redirect all HTTP requests to HTTPS.
        \item Disable plain HTTP access on port 80 once HTTPS is confirmed to be working correctly.
    \end{itemize}
\end{enumerate}

\subsection{Medium Priority / Strategic Actions}

\begin{enumerate}
    \setcounter{enumi}{1} % Continue numbering
    \item \textbf{Develop and Implement an Acceptable Use Policy (AUP) (R-01):}
    \begin{itemize}
        \item Draft a formal AUP that clearly defines the rules for using company IT assets, data, and internet access.
        \item Require all current employees and new hires to read and formally acknowledge the policy.
        \item Establish a regular review cycle for the AUP to keep it current.
    \end{itemize}
    \item \textbf{Establish a Security Awareness Training Program (R-01):}
    \begin{itemize}
        \item Implement a mandatory security awareness training module for all new employees during their onboarding process.
        \item Conduct annual, mandatory security awareness training for all staff. This training should cover key topics such as phishing, password security, social engineering, and safe browsing habits.
        \item Consider periodic phishing simulations to test and reinforce employee awareness.
    \end{itemize}
\end{enumerate}

\end{document}
```