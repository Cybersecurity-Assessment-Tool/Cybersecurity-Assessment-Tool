```latex
\documentclass[12pt]{article}

% Preamble: Required Packages and Document Setup
\usepackage[margin=1in]{geometry}
\usepackage{pifont} % For checkmarks and crosses
\usepackage{booktabs} % For professional tables
\usepackage{hyperref} % For clickable links
\usepackage{url} % For URL formatting
\usepackage{seqsplit} % For splitting long strings without breaking
\usepackage{xcolor} % For colors

% Hyperref and Color Setup
\hypersetup{
    colorlinks=true,
    linkcolor=blue,
    filecolor=magenta,      
    urlcolor=cyan,
    pdftitle={Cybersecurity Posture Assessment Report},
    pdfpagemode=FullScreen,
}
\urlstyle{same}

% Document Information
\title{Cybersecurity Posture Assessment Report \\ \large For: \textbf{Falcon Heavy}}
\author{Cybersecurity Analysis Division}
\date{\today}

\begin{document}

\maketitle
\tableofcontents
\newpage

% ===================================================================
% Section 1: Executive Overview
% ===================================================================
\section{Executive Overview}

This report provides a comprehensive cybersecurity posture assessment for \textbf{Falcon Heavy}, based on an analysis of network scan data, organizational security controls, and pre-existing risk information. The assessment was conducted to identify vulnerabilities, security gaps, and areas of non-compliance with cybersecurity best practices.

The analysis revealed several high-risk findings that require immediate attention. A key vulnerability is the external exposure of a MySQL database (Version 5.7.33) on port 3306. This version has reached its official End-of-Life (EOL) as of October 2023 and no longer receives security patches, making it a prime target for exploitation.

This technical vulnerability is critically compounded by significant gaps in organizational security controls. The absence of Multi-Factor Authentication (MFA) for accessing email and sensitive data systems drastically lowers the barrier for unauthorized access. Furthermore, the lack of a formal employee Acceptable Use Policy (AUP) indicates a gap in security governance.

In summary, the combination of an exposed, unpatched database service and weak access controls places \textbf{Falcon Heavy} at a high risk of data breach. This report outlines these findings in detail and provides prioritized, actionable recommendations to mitigate the identified risks.

% ===================================================================
% Section 2: Organizational Information
% ===================================================================
\section{Organizational Information}

The following information was provided for the assessment. This data forms the baseline for understanding the organization's digital footprint and internal policies.

\begin{table}[h!]
\centering
\caption{Client Organizational Data}
\begin{tabular}{@{}ll@{}}
\toprule
\textbf{Attribute} & \textbf{Value} \\ \midrule
Organization Name & \textbf{Falcon Heavy} \\
Email Domain & \texttt{FalconHeavy.com} \\
Website Domain & \texttt{www.FalconHeavy.com} \\
External IP Address & \texttt{184.126.3.180} \\ \bottomrule
\end{tabular}
\end{table}

% ===================================================================
% Section 3: Security Control Review
% ===================================================================
\section{Security Control Review}

A review of the organization's security controls was conducted via a questionnaire. The responses highlight critical gaps in access control and policy enforcement. Answers of ``No'' (\ding{55}) represent significant deviations from security best practices and are considered high-risk findings.

\begin{table}[h!]
\centering
\caption{Security Controls Questionnaire Results}
\begin{tabular}{@{}lc@{}}
\toprule
\textbf{Control Question} & \textbf{Response} \\ \midrule
Do you require MFA to access email? & \ding{55} \\
Do you require MFA to log into computers? & \ding{51} \\
Do you require MFA to access sensitive data systems? & \ding{55} \\
Does your organization have an employee acceptable use policy? & \ding{55} \\
Does your organization do security awareness training for new employees? & \ding{51} \\
Does your organization do security awareness training for all employees annually? & \ding{51} \\ \bottomrule
\end{tabular}
\label{tab:controls}
\end{table}

\subsection*{Analysis of Control Gaps}
\begin{itemize}
    \item \textbf{MFA for Email and Sensitive Systems:} The lack of MFA on email and sensitive data systems is a critical vulnerability. Email is a primary vector for phishing and account takeover attacks. Without MFA, a compromised password directly leads to a breach.
    \item \textbf{Acceptable Use Policy (AUP):} The absence of an AUP means there are no formally documented rules for employees regarding the use of company assets. This creates ambiguity and increases the risk of insider threats, whether malicious or accidental.
\end{itemize}

% ===================================================================
% Section 4: Technical Scan Results
% ===================================================================
\section{Technical Scan Results}

An external network scan was performed against the target IP address \texttt{172.16.50.20}. The scan identified one open port exposing a critical database service to the network.

\begin{table}[h!]
\centering
\caption{Open Ports and Services Detected on \texttt{172.16.50.20}}
\begin{tabular}{@{}lllll@{}}
\toprule
\textbf{Port} & \textbf{State} & \textbf{Service} & \textbf{Product} & \textbf{Version} \\ \midrule
3306/tcp & open & mysql & MySQL & 5.7.33 \\ \bottomrule
\end{tabular}
\label{tab:scanresults}
\end{table}

\subsection*{Analysis of Technical Findings}
The scan confirms the pre-existing risk titled "Database Exposure." The key finding is:
\begin{itemize}
    \item \textbf{Exposed End-of-Life MySQL Database:} Port 3306 is open, exposing a MySQL 5.7.33 database. The MySQL 5.7 branch reached its End-of-Life (EOL) in October 2023. EOL software no longer receives security updates from the vendor, meaning any newly discovered vulnerabilities will remain unpatched. Publicly exposing an EOL database service is a critical security risk that can lead to remote code execution, data exfiltration, or a complete system compromise.
\end{itemize}

% ===================================================================
% Section 5: Correlated Risk Assessment
% ===================================================================
\section{Correlated Risk Assessment}

This section synthesizes the findings from the security control review, technical scan, and pre-existing risk data into a prioritized list of risks.

\begin{table}[h!]
\centering
\caption{Summary of Identified Risks}
\begin{tabular}{@{}p{0.1\linewidth} p{0.4\linewidth} p{0.15\linewidth} p{0.25\linewidth}@{}}
\toprule
\textbf{Risk ID} & \textbf{Description} & \textbf{Severity} & \textbf{Affected Systems \& Data} \\ \midrule
\textbf{RISK-001} & An End-of-Life MySQL database (v5.7.33) is publicly exposed on port 3306. This is correlated with the lack of MFA on sensitive systems, creating a direct path for attackers. & \textbf{Critical} & Database server at \texttt{172.16.50.20}; All data stored within the database. \\
\addlinespace
\textbf{RISK-002} & Lack of Multi-Factor Authentication (MFA) on critical systems, including email and sensitive data applications. A single compromised password could lead to a major breach. & \textbf{High} & Email system (\texttt{FalconHeavy.com}); All sensitive data systems. \\
\addlinespace
\textbf{RISK-003} & Absence of a formal Acceptable Use Policy (AUP). This governance gap increases the likelihood of misuse of company IT assets and weakens the overall security culture. & \textbf{Medium} & All employees and IT systems; Organizational governance. \\
\bottomrule
\end{tabular}
\label{tab:risks}
\end{table}

% ===================================================================
% Section 6: Recommendations
% ===================================================================
\section{Recommendations}

The following actions are recommended to mitigate the identified risks. They are prioritized based on severity and potential impact.

\subsection*{Immediate Actions (To be completed within 72 hours)}
\begin{enumerate}
    \item \textbf{Restrict Access to Port 3306:} Immediately apply firewall rules to block all public access to port 3306 on \texttt{172.16.50.20}. Access should only be permitted from trusted, internal IP addresses. If remote access is required, it must be done through a secure channel like a VPN.
\end{enumerate}

\subsection*{Short-Term Actions (To be completed within 30 days)}
\begin{enumerate}
    \item \textbf{Upgrade MySQL Database:} Plan and execute the migration of the MySQL 5.7.33 database to a currently supported version (e.g., MySQL 8.x). This will ensure the system receives critical security patches.
    \item \textbf{Enforce MFA Everywhere:} Implement and enforce mandatory MFA for all users on the email system (\texttt{FalconHeavy.com}) and all applications that handle sensitive data.
\end{enumerate}

\subsection*{Long-Term Actions (To be completed within 90 days)}
\begin{enumerate}
    \item \textbf{Develop and Implement an AUP:} Create a formal Acceptable Use Policy that clearly defines the rules and responsibilities for all employees when using company technology and data. Ensure all employees read and acknowledge the policy.
    \item \textbf{Establish a Vulnerability Management Program:} Implement a regular schedule of internal and external vulnerability scanning to proactively identify and remediate security weaknesses before they can be exploited.
\end{enumerate}

\end{document}
```