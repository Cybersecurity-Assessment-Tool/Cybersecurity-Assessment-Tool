```latex
\documentclass[12pt]{article}

% ----------------------------------------------------------------------
% PREAMBLE
% ----------------------------------------------------------------------
\usepackage[margin=1in]{geometry}
\usepackage{pifont} % For check and cross marks (\ding)
\usepackage{booktabs} % For professional tables
\usepackage{hyperref} % For hyperlinks
\usepackage{url}      % For URL formatting
\usepackage{seqsplit} % For splitting long strings in \texttt

% Document Metadata
\title{Cybersecurity Posture Assessment Report}
\author{Cybersecurity Analysis Division}
\date{\today}

% Hyperref Setup
\hypersetup{
    colorlinks=true,
    linkcolor=black,
    filecolor=magenta,      
    urlcolor=blue,
    pdftitle={Cybersecurity Posture Assessment Report},
    pdfpagemode=FullScreen,
}

% ----------------------------------------------------------------------
% DOCUMENT START
% ----------------------------------------------------------------------
\begin{document}

\maketitle
\hrule
\vspace{1em}

% ----------------------------------------------------------------------
% EXECUTIVE SUMMARY
% ----------------------------------------------------------------------
\section*{Executive Summary}
This report provides a comprehensive analysis of the cybersecurity posture for \textbf{Urban Jungle Planning}. The assessment is based on a correlation of network scan data, organizational security control questionnaires, and a review of pre-existing risk documentation.

The analysis has revealed several critical and high-risk findings that require immediate attention. A significant discrepancy was identified where a pre-existing risk assessment incorrectly classified port 8080 as secure. Our technical scan confirms this port is open on an internal system, exposing a service explicitly titled \textbf{"TOP SECRET DB"}. This finding, coupled with critical gaps in administrative controls—most notably the lack of Multi-Factor Authentication (MFA) for sensitive systems and the absence of foundational employee security policies—presents a severe and immediate threat to the organization's data confidentiality and integrity.

Urgent remediation is required to address the exposed service and implement the recommended security controls to mitigate these substantial risks.

% ----------------------------------------------------------------------
% ORGANIZATIONAL INFORMATION
% ----------------------------------------------------------------------
\section{Organizational Information}
The following details were provided for the assessment scope.

\begin{tabular}{@{}ll}
\toprule
\textbf{Attribute} & \textbf{Value} \\
\midrule
Organization Name & \textbf{Urban Jungle Planning} \\
Email Domain & \texttt{UrbanJunglePlanning.org} \\
Website Domain & \url{www.UrbanJunglePlanning.org} \\
External IP Address & \texttt{46.79.101.121} \\
\bottomrule
\end{tabular}

% ----------------------------------------------------------------------
% SECURITY CONTROL REVIEW
% ----------------------------------------------------------------------
\section{Security Control Review}
A review of the organization's security questionnaire highlights significant gaps in administrative and access controls. "No" answers indicate a deviation from security best practices and are flagged as risks.

\begin{tabular}{@{}p{0.75\linewidth}c}
\toprule
\textbf{Control Question} & \textbf{Response} \\
\midrule
Do you require MFA to access email? & \ding{51} \\ % Yes
Do you require MFA to log into computers? & \ding{51} \\ % Yes
\textbf{Do you require MFA to access sensitive data systems?} & \textbf{\color{red}\ding{55}} \\ % No
\textbf{Does your organization have an employee acceptable use policy?} & \textbf{\color{red}\ding{55}} \\ % No
\textbf{Does your organization do security awareness training for new employees?} & \textbf{\color{red}\ding{55}} \\ % No
Does your organization do security awareness training for all employees at least once per year? & \ding{51} \\ % Yes
\bottomrule
\end{tabular}

% ----------------------------------------------------------------------
% TECHNICAL SCAN RESULTS
% ----------------------------------------------------------------------
\section{Technical Scan Results}
An Nmap scan was conducted on the internal network to identify active services and potential exposures.

\begin{itemize}
    \item \textbf{Target IP Address:} \texttt{10.5.5.5}
\end{itemize}

The scan revealed the following critical finding:

\begin{tabular}{@{}llll}
\toprule
\textbf{Port} & \textbf{State} & \textbf{Service} & \textbf{Notes} \\
\midrule
8080/tcp & OPEN & http & The HTTP service title is: \textbf{\color{red}"TOP SECRET DB"} \\
\bottomrule
\end{tabular}

\paragraph{Analysis:} The presence of an open port with a title explicitly indicating a "TOP SECRET" database is a severe security misconfiguration. This service is exposed on the internal network, and its discovery directly contradicts the information provided in the existing risk documentation, which stated this port was secure. This suggests a critical failure in the organization's risk management and validation processes.

% ----------------------------------------------------------------------
% RISK ASSESSMENT
% ----------------------------------------------------------------------
\section{Risk Assessment}
The following table synthesizes findings from the security control review, technical scans, and existing risk data.

\begin{tabular}{@{}p{0.1\linewidth}p{0.25\linewidth}p{0.1\linewidth}p{0.45\linewidth}}
\toprule
\textbf{ID} & \textbf{Risk Name} & \textbf{Severity} & \textbf{Description \& Correlated Findings} \\
\midrule
\textbf{R-01} & Exposed Sensitive Data System & \textbf{Critical} & The network scan on \texttt{10.5.5.5} found port 8080 open with the title "TOP SECRET DB". This directly contradicts a pre-existing risk assessment (\textit{Input 3}) that incorrectly labeled this as a secure false positive. This represents a potential data breach vector. \\
\addlinespace
\textbf{R-02} & Insufficient Multi-Factor Authentication & \textbf{Critical} & The organization does not enforce MFA on sensitive data systems. This significantly increases the risk of unauthorized access via compromised credentials, a risk that is amplified by the existence of the exposed database identified in R-01. \\
\addlinespace
\textbf{R-03} & Lack of Foundational Security Policies & \textbf{High} & The organization lacks an Acceptable Use Policy (AUP) and does not provide security training to new hires. This creates an environment where employees are unaware of their security responsibilities, increasing the likelihood of insider threats and accidental data exposure. \\
\bottomrule
\end{tabular}

% ----------------------------------------------------------------------
% RECOMMENDATIONS
% ----------------------------------------------------------------------
\section{Recommendations}
The following actions are recommended to mitigate the identified risks. They are prioritized based on severity and potential impact.

\subsection*{Priority 1: Immediate Actions (Due within 24 hours)}
\begin{enumerate}
    \item \textbf{Contain Exposed System:} Immediately investigate the service running on \texttt{10.5.5.5:8080}.
    \begin{itemize}
        \item Determine the nature of the "TOP SECRET DB" and what data it contains.
        \item Preserve logs and initiate an incident response process to determine if a breach has occurred.
        \item Implement a firewall rule to block all non-essential access to this port and system.
    \end{itemize}
    \item \textbf{Review Risk Assessment Process:} The existing risk documentation was dangerously inaccurate. A full review of the risk management and validation process must be conducted to prevent future oversights.
\end{enumerate}

\subsection*{Priority 2: High-Priority Actions (Due within 30 days)}
\begin{enumerate}
    \item \textbf{Enforce MFA on Sensitive Systems:} Procure and deploy an MFA solution for all systems identified as containing sensitive data. This is a critical compensating control.
    \item \textbf{Develop and Implement an AUP:} Create a formal Acceptable Use Policy that defines rules for employee use of company assets and data. All employees must read and acknowledge this policy.
    \item \textbf{Mandate New Hire Security Training:} Integrate mandatory cybersecurity awareness training into the new employee onboarding process to ensure all staff are aware of security best practices from day one.
\end{enumerate}

\subsection*{Priority 3: Strategic Actions (Due within 90 days)}
\begin{enumerate}
    \item \textbf{Conduct Comprehensive Vulnerability Scanning:} Perform a credentialed vulnerability scan across the entire internal network to identify other potential misconfigurations or unpatched systems.
    \item \textbf{Implement a Data Classification Policy:} Develop and implement a policy to classify data based on sensitivity. This will help prioritize the application of security controls like MFA and access restrictions.
\end{enumerate}

\end{document}
```