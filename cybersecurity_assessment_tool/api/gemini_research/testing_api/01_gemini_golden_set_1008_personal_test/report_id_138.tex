```latex
\documentclass[12pt]{article}

% Preamble: Required Packages
\usepackage[margin=1in]{geometry}
\usepackage{pifont} % For checkmarks and crosses
\usepackage{booktabs} % For professional tables
\usepackage{hyperref} % For hyperlinks
\usepackage{url} % For URL formatting
\usepackage{seqsplit} % To split long strings in \texttt
\usepackage{graphicx}
\usepackage{xcolor}

% Hyperref Setup
\hypersetup{
    colorlinks=true,
    linkcolor=blue,
    filecolor=magenta,      
    urlcolor=cyan,
    pdftitle={Cybersecurity Assessment Report},
    pdfpagemode=FullScreen,
}

% Define a command for the checkmark and cross
\newcommand{\cmark}{\ding{51}}
\newcommand{\xmark}{\ding{55}}

\begin{document}

% --- Title Page ---
\title{
    \vspace{2cm}
    \textbf{Cybersecurity Assessment Report} \\
    \large \textit{Prepared for: Aetheric Systems}
    \vspace{1cm}
}
\author{Cybersecurity Analysis Division}
\date{\today}
\maketitle
\thispagestyle{empty}
\newpage

% --- Table of Contents ---
\tableofcontents
\newpage

% --- Section 1: Executive Summary ---
\section{Executive Summary}

This report provides a comprehensive cybersecurity assessment for \textbf{Aetheric Systems}, based on an analysis of network scan data, organizational security controls, and pre-existing risk information. The assessment synthesizes technical findings with procedural and policy-based controls to provide a holistic view of the organization's security posture.

The key findings indicate significant risks stemming from gaps in organizational security policies, despite a clean technical scan of the target host. The most critical vulnerabilities identified are:

\begin{itemize}
    \item \textbf{Critical Risk: Lack of MFA for Email.} The absence of Multi-Factor Authentication (MFA) on email accounts represents a severe vulnerability, exposing the organization to business email compromise, phishing, and account takeover attacks.
    \item \textbf{High Risk: Inadequate Security Training.} The lack of mandatory annual security awareness training for all employees increases the organization's susceptibility to social engineering and human error.
\end{itemize}

Interestingly, the network scan of target \texttt{192.168.0.5} found no open ports. This contradicts a pre-existing risk report indicating an open Port 80. This suggests that the previously identified vulnerability may have been remediated.

Recommendations focus on immediate implementation of MFA for email, establishing a recurring security training program, and formally verifying the status of previously identified risks.

% --- Section 2: Organizational Information ---
\section{Organizational Information}

The following details were provided for the assessment. This information is used to establish the context and scope of the review.

\begin{tabular}{@{}ll}
    \toprule
    \textbf{Attribute} & \textbf{Value} \\
    \midrule
    Organization Name & \textbf{Aetheric Systems} \\
    Email Domain & \seqsplit{\texttt{AethericSystems.org}} \\
    Website Domain & \seqsplit{\texttt{www.AethericSystems.org}} \\
    External IP Address & \seqsplit{\texttt{147.48.28.165}} \\
    \bottomrule
\end{tabular}

% --- Section 3: Security Control Review ---
\section{Security Control Review}

A review of the organization's security controls was conducted via a questionnaire. The responses highlight critical gaps in identity management and employee training protocols. A "No" response (\xmark) indicates a deviation from security best practices and a potential area of high risk.

\begin{table}[h!]
\centering
\caption{Organizational Security Control Questionnaire}
\begin{tabular}{@{}p{0.7\textwidth}c@{}}
    \toprule
    \textbf{Control Question} & \textbf{Response} \\
    \midrule
    Do you require MFA to access email? & \textcolor{red}{\xmark} \\
    Do you require MFA to log into computers? & \textcolor{green}{\cmark} \\
    Do you require MFA to access sensitive data systems? & \textcolor{green}{\cmark} \\
    Does your organization have an employee acceptable use policy? & \textcolor{green}{\cmark} \\
    Does your organization do security awareness training for new employees? & \textcolor{green}{\cmark} \\
    Does your organization do security awareness training for all employees at least once per year? & \textcolor{red}{\xmark} \\
    \bottomrule
\end{tabular}
\end{table}

% --- Section 4: Technical Scan Results ---
\section{Technical Scan Results}

A network scan was performed on the specified target to identify open ports and exposed services.

\begin{itemize}
    \item \textbf{Target IP Address:} \texttt{192.168.0.5}
    \item \textbf{Scan Date:} \today
\end{itemize}

The scan revealed a secure host with no discoverable open ports. This is a positive finding. Notably, Port 80, which was listed as a vulnerability in a previous risk assessment, was found to be \textbf{closed}. This indicates that the risk may have been remediated.

\begin{table}[h!]
\centering
\caption{Port Scan Results for \texttt{192.168.0.5}}
\begin{tabular}{@{}lll@{}}
    \toprule
    \textbf{Port} & \textbf{State} & \textbf{Service/Notes} \\
    \midrule
    80/tcp & Closed & HTTP. No service detected. \\
    \multicolumn{3}{l}{\textit{No other open ports were identified during the scan.}} \\
    \bottomrule
\end{tabular}
\end{table}

% --- Section 5: Correlated Risk Assessment ---
\section{Correlated Risk Assessment}

This section correlates findings from the security control review, the technical scan, and pre-existing risk data to provide a unified risk summary.

\begin{table}[h!]
\centering
\caption{Summary of Identified Risks}
\begin{tabular}{@{}p{0.25\textwidth}p{0.5\textwidth}l@{}}
    \toprule
    \textbf{Risk Name} & \textbf{Description} & \textbf{Severity} \\
    \midrule
    \textbf{Lack of MFA for Email Access} & Email accounts are protected only by passwords, making them highly vulnerable to phishing, credential stuffing, and brute-force attacks. A compromise could lead to data breaches and financial fraud. & \textbf{Critical} \\
    \addlinespace
    \textbf{Inadequate Annual Security Training} & Without regular, mandatory training, employees are more likely to fall victim to social engineering attacks, mishandle sensitive data, or engage in risky online behavior, posing an ongoing threat to the organization. & \textbf{High} \\
    \addlinespace
    \textbf{Unencrypted Web Server (Previously Identified)} & A prior risk stated that Port 80 was open, exposing unencrypted HTTP traffic. \textbf{Note:} The current scan found this port to be closed, suggesting this risk has been remediated. Verification is recommended. & Medium \\
    \bottomrule
\end{tabular}
\end{table}

% --- Section 6: Recommendations ---
\section{Recommendations}

Based on the correlated risk assessment, the following actions are recommended to mitigate the identified vulnerabilities and improve the overall security posture of \textbf{Aetheric Systems}.

\begin{enumerate}
    \item \textbf{Implement MFA for Email (Priority: Immediate)}
    \begin{itemize}
        \item \textbf{Action:} Enforce Multi-Factor Authentication (MFA) for all user accounts across the \texttt{AethericSystems.org} email domain.
        \item \textbf{Justification:} This is the single most effective control to prevent unauthorized account access and mitigate the risk of business email compromise. It addresses the \textbf{Critical} risk identified in this report.
    \end{itemize}
    \vspace{0.5cm}
    
    \item \textbf{Establish an Annual Security Awareness Program (Priority: High)}
    \begin{itemize}
        \item \textbf{Action:} Develop and implement a mandatory security awareness training program for all employees, to be completed annually. The training should cover phishing, password security, acceptable use, and incident reporting.
        \item \textbf{Justification:} A well-trained workforce is a critical layer of defense. This program directly mitigates the \textbf{High} risk of human error and susceptibility to social engineering.
    \end{itemize}
    \vspace{0.5cm}
    
    \item \textbf{Verify and Close Out Historical Risks (Priority: Medium)}
    \begin{itemize}
        \item \textbf{Action:} Formally review the finding that Port 80 is now closed. Document the change in the organization's risk register and confirm that the closure was intentional and aligns with business requirements.
        \item \textbf{Justification:} Maintaining an accurate and up-to-date risk register is crucial for effective security management. This ensures that resources are not wasted on remediated issues and provides a clear picture of the current threat landscape.
    \end{itemize}
\end{enumerate}

\end{document}
```