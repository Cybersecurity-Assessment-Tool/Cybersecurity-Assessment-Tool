```latex
\documentclass[12pt]{article}

% Preamble: Required Packages
\usepackage[margin=1in]{geometry}
\usepackage{pifont} % For checkmarks and crosses (\ding)
\usepackage{booktabs} % For professional tables
\usepackage{hyperref} % For clickable links
\usepackage{url} % For formatting URLs
\usepackage{seqsplit} % For splitting long strings in texttt
\usepackage[T1]{fontenc}
\usepackage{graphicx}
\usepackage{xcolor}

% Document Information
\title{Cybersecurity Posture Assessment Report}
\author{Cybersecurity Analysis Division}
\date{\today}

% Hyperref Setup
\hypersetup{
    colorlinks=true,
    linkcolor=blue,
    filecolor=magenta,      
    urlcolor=cyan,
    pdftitle={Cybersecurity Posture Assessment Report},
    pdfpagemode=FullScreen,
}

\begin{document}

\maketitle
\thispagestyle{empty}
\newpage

\tableofcontents
\thispagestyle{empty}
\newpage

\setcounter{page}{1}

% ==============================================================================
% SECTION 1: EXECUTIVE SUMMARY
% ==============================================================================
\section{Executive Summary}

This report provides a comprehensive cybersecurity assessment for \textbf{Blue Horizon Initiative}, based on an analysis of organizational security controls, an external network scan, and a review of known risks.

The assessment reveals a mixed security posture. On a positive note, the external network scan of the target host at \texttt{[Target IP]} identified \textbf{no open ports}. This indicates a strong perimeter defense and a well-configured firewall for the scanned asset, which significantly reduces the external attack surface.

However, a critical analysis of the organization's internal security controls, gathered via a questionnaire, highlights significant and urgent vulnerabilities. The most critical gaps are the \textbf{absence of Multi-Factor Authentication (MFA) for email and computer access} and the \textbf{lack of a formal security awareness training program}. These deficiencies expose the organization to a high risk of phishing attacks, business email compromise, ransomware, and unauthorized access to sensitive data.

While the network perimeter appears secure, the identified procedural and policy gaps represent the most probable path for a security breach. This report outlines these risks in detail and provides prioritized, actionable recommendations to mitigate them effectively. Immediate focus should be placed on implementing MFA across all critical systems and establishing a comprehensive security training program for all employees.

% ==============================================================================
% SECTION 2: ORGANIZATIONAL INFORMATION
% ==============================================================================
\section{Organizational Information}

The following details were provided for the assessment. This information is used to establish the context and scope of the review.

\begin{table}[h!]
\centering
\begin{tabular}{@{}ll@{}}
\toprule
\textbf{Attribute} & \textbf{Value} \\ \midrule
Organization Name & \textbf{Blue Horizon Initiative} \\
Email Domain & \texttt{BlueHorizonInitiative.org} \\
Website Domain & \url{www.BlueHorizonInitiative.org} \\
External IP Address & \texttt{73.166.85.5} \\ \bottomrule
\end{tabular}
\caption{Client Organizational Details}
\label{tab:org_info}
\end{table}

% ==============================================================================
% SECTION 3: SECURITY CONTROL REVIEW
% ==============================================================================
\section{Security Control Review (Questionnaire Analysis)}

An analysis of the security questionnaire reveals the current state of implemented administrative and technical controls. "No" answers indicate significant gaps in the security framework that require immediate attention.

\begin{table}[h!]
\centering
\begin{tabular}{@{}p{0.6\textwidth} c p{0.2\textwidth}@{}}
\toprule
\textbf{Control Question} & \textbf{Response} & \textbf{Assessment} \\ \midrule
Do you require MFA to access email? & \ding{55} & \textcolor{red}{\textbf{Critical Gap}} \\
Do you require MFA to log into computers? & \ding{55} & \textcolor{red}{\textbf{High Risk}} \\
Do you require MFA to access sensitive data systems? & \ding{51} & Best Practice Met \\
Does your organization have an employee acceptable use policy? & \ding{51} & Best Practice Met \\
Does your organization do security awareness training for new employees? & \ding{55} & \textcolor{red}{\textbf{High Risk}} \\
Does your organization do security awareness training for all employees at least once per year? & \ding{55} & \textcolor{red}{\textbf{High Risk}} \\ \bottomrule
\end{tabular}
\caption{Security Control Questionnaire Results}
\label{tab:controls}
\end{table}

% ==============================================================================
% SECTION 4: TECHNICAL SCAN RESULTS
% ==============================================================================
\section{Technical Scan Results}

An external network vulnerability scan was conducted to identify open ports, running services, and potential vulnerabilities visible from the public internet.

\subsection{Scan Details}
\begin{itemize}
    \item \textbf{Target IP Address:} \texttt{[Target IP]}
    \item \textbf{Scan Date:} Data Not Available in Scan File
    \item \textbf{Scan Type:} TCP Port Scan (Top 1000 Ports)
\end{itemize}

\subsection{Findings}
The scan completed successfully and found \textbf{zero open TCP ports} on the target host.

\subsubsection*{Assessment}
This is a positive security finding. A host with no open ports facing the public internet presents a minimal attack surface from a network perspective. This configuration, often referred to as "stealthed," is indicative of a properly configured firewall that denies all unsolicited inbound traffic. This greatly reduces the risk of network-based attacks against this specific asset.

% ==============================================================================
% SECTION 5: CONSOLIDATED RISK ASSESSMENT
% ==============================================================================
\section{Consolidated Risk Assessment}

This section synthesizes findings from the security control review and technical scan. While no pre-existing risks were reported and no technical vulnerabilities were found on the scanned host, the procedural gaps identified in Section 3 constitute significant organizational risks.

\begin{table}[h!]
\centering
\begin{tabular}{@{}p{0.1\textwidth} p{0.2\textwidth} p{0.5\textwidth} p{0.1\textwidth}@{}}
\toprule
\textbf{Risk ID} & \textbf{Risk Name} & \textbf{Overview} & \textbf{Severity} \\ \midrule
RISK-001 & Lack of MFA for Email Access & Without MFA, email accounts are vulnerable to takeover via stolen or weak passwords. This is a primary vector for Business Email Compromise (BEC) and phishing campaigns. & \textbf{Critical} \\
\addlinespace
RISK-002 & Lack of MFA for Endpoint Access & The absence of MFA on computer logins means that compromised credentials can lead directly to unauthorized access to endpoints and the internal network. & High \\
\addlinespace
RISK-003 & Inadequate Security Awareness Training & Without regular training, employees are more likely to fall victim to social engineering attacks like phishing, inadvertently install malware, or mishandle sensitive data. & High \\
\bottomrule
\end{tabular}
\caption{Identified Risks and Severity}
\label{tab:risks}
\end{table}

% ==============================================================================
% SECTION 6: RECOMMENDATIONS
% ==============================================================================
\section{Recommendations}

The following prioritized recommendations are provided to address the identified risks and strengthen the overall security posture of \textbf{Blue Horizon Initiative}.

\subsection{Priority 1: Critical Risk Mitigation}

\subsubsection*{RISK-001: Implement MFA for Email Access}
\begin{itemize}
    \item \textbf{Action:} Immediately enable and enforce Multi-Factor Authentication for all user accounts, including administrative and service accounts, that have access to the organization's email system (e.g., Microsoft 365, Google Workspace).
    \item \textbf{Justification:} This is the single most effective control to prevent unauthorized account access and mitigate the risk of Business Email Compromise. It protects against password spray attacks, credential stuffing, and phishing.
    \item \textbf{Implementation:} Prioritize the use of strong MFA methods such as authenticator apps (e.g., Google Authenticator, Microsoft Authenticator) or FIDO2 security keys over less secure methods like SMS.
\end{itemize}

\subsection{Priority 2: High Risk Mitigation}

\subsubsection*{RISK-002: Enforce MFA for Endpoint and System Access}
\begin{itemize}
    \item \textbf{Action:} Deploy an MFA solution for all employee computer logins (Windows, macOS). This should also be extended to remote access solutions (VPN) and other critical internal systems.
    \item \textbf{Justification:} This creates a critical layer of defense, ensuring that even if a user's password is stolen, an attacker cannot easily access their computer or the internal network.
\end{itemize}

\subsubsection*{RISK-003: Establish a Security Awareness Training Program}
\begin{itemize}
    \item \textbf{Action:} Develop and implement a formal, mandatory security awareness training program. The program must include:
    \begin{enumerate}
        \item \textbf{Onboarding Training:} A required module for all new employees before they are granted system access.
        \item \textbf{Annual Refresher Training:} A yearly course for all staff covering current threats, company policies, and best practices.
        \item \textbf{Phishing Simulations:} Regular, simulated phishing campaigns to test and improve employee resilience to social engineering.
    \end{enumerate}
    \item \textbf{Justification:} A well-trained workforce is a critical component of a defense-in-depth strategy. Training reduces the likelihood of human error, which is a factor in over 80\% of all data breaches.
\end{itemize}

\end{document}
```