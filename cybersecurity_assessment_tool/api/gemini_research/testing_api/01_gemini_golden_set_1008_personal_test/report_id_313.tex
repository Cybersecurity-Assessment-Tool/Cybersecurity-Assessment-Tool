```latex
\documentclass[12pt]{article}

% Required Packages
\usepackage[margin=1in]{geometry}
\usepackage{pifont} % For \ding
\usepackage{booktabs} % For professional tables
\usepackage{hyperref} % For hyperlinks
\usepackage{url} % For URL formatting
\usepackage{seqsplit} % For splitting long strings in texttt

% Hyperref Setup
\hypersetup{
    colorlinks=true,
    linkcolor=black,
    filecolor=magenta,      
    urlcolor=blue,
    pdftitle={Cybersecurity Posture Assessment Report},
    pdfpagemode=FullScreen,
}

% Document Start
\begin{document}

\title{Cybersecurity Posture Assessment Report}
\author{Cybersecurity Analysis Division}
\date{November 22, 2025}
\maketitle

\hrule
\vspace{1em}

% ===================================================================
% Section 1: Overview and Executive Summary
% ===================================================================
\section{Overview and Executive Summary}

This report details the findings of a cybersecurity posture assessment for \textbf{North Star Education}. The assessment was conducted by correlating data from a network vulnerability scan, a security controls questionnaire, and a review of pre-existing risks.

The analysis reveals several critical and high-risk gaps in the organization's current security posture. Key findings include:
\begin{itemize}
    \item \textbf{Critical Gaps in Access Control:} Multi-Factor Authentication (MFA) is not enforced for employee email accounts or computer logins. This exposes the organization to a significant risk of account compromise through phishing and credential theft.
    \item \textbf{Lack of Security Awareness:} The organization lacks a formal employee acceptable use policy and does not provide security awareness training. This increases the likelihood of human error leading to security incidents.
    \item \textbf{Outdated Web Server Software:} The external-facing web server is running an outdated version of Nginx (1.18.0), which may contain unpatched vulnerabilities.
\end{itemize}

These findings indicate an elevated risk profile. Immediate remediation is recommended to address the identified weaknesses and strengthen the overall security posture.

% ===================================================================
% Section 2: Organizational Information
% ===================================================================
\section{Organizational Information}

The following information was provided by the client and used as a baseline for this assessment.

\begin{tabular}{@{}ll}
    \toprule
    \textbf{Attribute} & \textbf{Value} \\
    \midrule
    Organization Name & \textbf{North Star Education} \\
    Email Domain & \texttt{NorthStarEducation.com} \\
    External IP Address & \texttt{3.252.133.142} \\
    \bottomrule
\end{tabular}

% ===================================================================
% Section 3: Security Control Review
% ===================================================================
\section{Security Control Review}

The following table summarizes the organization's responses to a security controls questionnaire. A checkmark (\ding{51}) indicates a positive control is in place, while an X (\ding{55}) indicates a control gap.

\begin{table}[h!]
\centering
\begin{tabular}{@{}p{0.8\linewidth}c@{}}
    \toprule
    \textbf{Control Question} & \textbf{Response} \\
    \midrule
    Do you require MFA to access email? & \ding{55} \\
    Do you require MFA to log into computers? & \ding{55} \\
    Do you require MFA to access sensitive data systems? & \ding{51} \\
    Does your organization have an employee acceptable use policy? & \ding{55} \\
    Does your organization do security awareness training for new employees? & \ding{55} \\
    Does your organization do security awareness training for all employees at least once per year? & \ding{55} \\
    \bottomrule
\end{tabular}
\caption{Security Controls Questionnaire Results}
\end{table}

The responses highlight significant deficiencies in identity and access management and security awareness programs, which are foundational pillars of a robust cybersecurity strategy.

% ===================================================================
% Section 4: Technical Scan Results
% ===================================================================
\section{Technical Scan Results}

An external network scan was performed on \textbf{2025-11-22} against the target IP address \texttt{192.168.10.5}. The scan identified the following open ports and services.

\begin{table}[h!]
\centering
\begin{tabular}{@{}llll@{}}
    \toprule
    \textbf{Port} & \textbf{State} & \textbf{Service} & \textbf{Product \& Version} \\
    \midrule
    443/tcp & open & https & nginx 1.18.0 \\
    \bottomrule
\end{tabular}
\caption{Open Port Scan Results}
\end{table}

\subsection{Analysis of Findings}
\begin{itemize}
    \item \textbf{Outdated Software:} The web server is running \textbf{Nginx version 1.18.0}, which was released in April 2020. This version is no longer the most current stable release and may be susceptible to publicly known vulnerabilities that have been patched in newer versions.
    \item \textbf{SSL Certificate Mismatch:} The SSL certificate presented by the server has a Common Name (\texttt{www.acme-corp.com}) that does not match the organization's domain (\texttt{www.NorthStarEducation.com}). This is a misconfiguration that can cause trust errors for users and may indicate improper server setup.
\end{itemize}

% ===================================================================
% Section 5: Risk Assessment Summary
% ===================================================================
\section{Risk Assessment Summary}

The following table synthesizes findings from the security control review and technical scan into a prioritized list of risks. No pre-existing risks were provided for this assessment.

\begin{table}[h!]
\centering
\begin{tabular}{@{}lp{0.45\linewidth}l@{}}
    \toprule
    \textbf{Risk Name} & \textbf{Overview} & \textbf{Severity} \\
    \midrule
    Lack of MFA on Critical Systems & The absence of MFA on email and computer logins creates a high risk of unauthorized access via stolen or weak credentials. & \textbf{Critical} \\
    \addlinespace
    Inadequate Security Awareness Program & Without an acceptable use policy or regular training, employees are more likely to fall victim to phishing or engage in risky behavior. & \textbf{High} \\
    \addlinespace
    Outdated Web Server Software & The Nginx 1.18.0 server is outdated and may contain unpatched vulnerabilities, exposing the organization to potential compromise. & \textbf{Medium} \\
    \bottomrule
\end{tabular}
\caption{Synthesized Risk Register}
\end{table}

% ===================================================================
% Section 6: Recommendations
% ===================================================================
\section{Recommendations}

Based on the identified risks, the following remediation actions are recommended to improve the security posture of \textbf{North Star Education}.

\begin{enumerate}
    \item \textbf{Implement Comprehensive MFA (Critical):}
    \begin{itemize}
        \item Immediately enforce MFA for all user accounts on the email system (\texttt{NorthStarEducation.com}).
        \item Deploy and require MFA for all employee computer/endpoint logins.
        \item Continue enforcing MFA on sensitive data systems as is currently practiced.
    \end{itemize}

    \item \textbf{Establish a Security Awareness Program (High):}
    \begin{itemize}
        \item Develop a formal Acceptable Use Policy (AUP) and require all employees to read and acknowledge it.
        \item Implement a mandatory security awareness training module for all new hires during their onboarding process.
        \item Conduct annual security awareness training for all staff, covering topics such as phishing, password hygiene, and data handling.
    \end{itemize}

    \item \textbf{Remediate Web Server Vulnerabilities (Medium):}
    \begin{itemize}
        \item Plan and execute an upgrade of the Nginx web server from version 1.18.0 to the latest stable version to mitigate known vulnerabilities.
        \item Replace the current SSL certificate with one that correctly matches the organization's domain (\texttt{www.NorthStarEducation.com}) to resolve trust errors and ensure proper configuration.
    \end{itemize}
\end{enumerate}

\end{document}
```