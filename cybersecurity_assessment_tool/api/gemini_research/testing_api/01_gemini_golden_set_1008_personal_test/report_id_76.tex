```latex
\documentclass[12pt]{article}

% --- PACKAGES ---
\usepackage[margin=1in]{geometry}
\usepackage{pifont} % For checkmarks and crosses
\usepackage{booktabs} % For professional tables
\usepackage{hyperref} % For clickable links
\usepackage{url} % For URL formatting
\usepackage{seqsplit} % For splitting long strings to prevent overflow

% --- DOCUMENT METADATA ---
\title{Cybersecurity Posture Assessment Report}
\author{True Grit}
\date{\today}

\begin{document}

\maketitle
\hrule
\vspace{1em}

% ==============================================================================
% SECTION 1: EXECUTIVE OVERVIEW
% ==============================================================================
\section*{Executive Overview}

This report provides a cybersecurity assessment for True Grit, based on a combination of organizational data, a security controls questionnaire, and a network vulnerability scan.

The assessment reveals a mixed security posture. On a technical level, the scanned host (\texttt{192.168.1.100}) demonstrated a strong network configuration with no open ports detected, indicating a well-hardened perimeter or effective firewall rules.

However, significant procedural and policy gaps were identified through the security controls review. The absence of Multi-Factor Authentication (MFA) for computer logins and access to sensitive data systems represents a \textbf{critical risk}. Furthermore, the lack of annual security awareness training for all employees constitutes a \textbf{high risk}, increasing the organization's susceptibility to social engineering and phishing attacks.

Immediate remediation should focus on implementing a comprehensive MFA strategy and establishing a recurring security training program to address these vulnerabilities.

% ==============================================================================
% SECTION 2: ORGANIZATIONAL INFORMATION
% ==============================================================================
\section*{Organizational Information}

The following details were provided for the assessment:
\begin{itemize}
    \item \textbf{Organization Name:} True Grit
    \item \textbf{Email Domain:} \texttt{TrueGrit.net}
    \item \textbf{Website Domain:} \url{www.TrueGrit.net}
    \item \textbf{External IP Address:} \texttt{105.248.100.200}
\end{itemize}

% ==============================================================================
% SECTION 3: SECURITY CONTROL REVIEW
% ==============================================================================
\section*{Security Control Review}

The following table summarizes the organization's responses to the security controls questionnaire. A checkmark (\ding{51}) indicates a positive control is in place, while a cross (\ding{55}) indicates a potential security gap.

\begin{table}[h!]
\centering
\begin{tabular}{p{0.8\linewidth} c}
\toprule
\textbf{Control Question} & \textbf{Response} \\
\midrule
Do you require MFA to access email? & \ding{51} \\
Do you require MFA to log into computers? & \ding{55} \\
Do you require MFA to access sensitive data systems? & \ding{55} \\
Does your organization have an employee acceptable use policy? & \ding{51} \\
Does your organization do security awareness training for new employees? & \ding{51} \\
Does your organization do security awareness training for all employees at least once per year? & \ding{55} \\
\bottomrule
\end{tabular}
\caption{Security Controls Questionnaire Results}
\end{table}

% ==============================================================================
% SECTION 4: TECHNICAL SCAN RESULTS
% ==============================================================================
\section*{Technical Scan Results}

A network scan was performed to identify open ports and services on the target system.

\begin{itemize}
    \item \textbf{Target IP Address:} \texttt{192.168.1.100}
    \item \textbf{Host Status:} Up
    \item \textbf{Findings:} The scan confirmed that the host is online. However, \textbf{no open ports were detected}. All 65,535 TCP ports were found to be in a 'closed' state. This is a positive security finding, suggesting that the host is either not running any network-facing services or is protected by a very restrictive firewall.
\end{itemize}

% ==============================================================================
% SECTION 5: RISK ASSESSMENT
% ==============================================================================
\section*{Risk Assessment}

The following table synthesizes findings from the security control review and technical scan into a prioritized list of risks.

\begin{table}[h!]
\centering
\begin{tabular}{p{0.25\linewidth} p{0.5\linewidth} p{0.15\linewidth}}
\toprule
\textbf{Risk Name} & \textbf{Overview} & \textbf{Severity} \\
\midrule
\textbf{Lack of MFA on Endpoints} & Users are not required to use MFA to log into computers. If an employee's credentials are compromised, an attacker could gain direct access to their workstation and the corporate network. & \textbf{Critical} \\
\addlinespace
\textbf{Lack of MFA on Sensitive Systems} & Access to systems containing sensitive data is not protected by MFA. This exposes the organization's most critical data assets to unauthorized access and exfiltration. & \textbf{Critical} \\
\addlinespace
\textbf{Inadequate Security Awareness Training} & Security training is not conducted annually for all staff. This leads to knowledge decay and increases the organization's vulnerability to evolving threats like phishing and social engineering. & \textbf{High} \\
\bottomrule
\end{tabular}
\caption{Identified Risks and Severity}
\end{table}

% ==============================================================================
% SECTION 6: RECOMMENDATIONS
% ==============================================================================
\section*{Recommendations}

Based on the analysis, the following actions are recommended to mitigate the identified risks and improve the overall security posture.

\begin{enumerate}
    \item \textbf{[Critical] Implement MFA for Endpoint Access:}
    Deploy a mandatory Multi-Factor Authentication solution for all user logins to company workstations and laptops. This action directly mitigates the risk of credential theft leading to unauthorized endpoint access.

    \item \textbf{[Critical] Enforce MFA on Sensitive Data Systems:}
    Immediately enforce MFA for all applications, databases, and administrative interfaces that process or store sensitive or critical company data. This creates a vital security layer protecting your most valuable assets.

    \item \textbf{[High] Establish an Annual Security Training Program:}
    Develop and implement a mandatory annual security awareness training program for all employees. The training should cover current threats, such as phishing, ransomware, and social engineering, and reinforce company security policies.
\end{enumerate}

\end{document}
```