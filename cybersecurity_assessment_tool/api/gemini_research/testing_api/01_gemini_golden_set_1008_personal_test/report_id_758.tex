```latex
\documentclass[12pt]{article}

% Preamble: Required Packages
\usepackage[margin=1in]{geometry}
\usepackage{pifont} % For checkmarks and crosses
\usepackage{booktabs} % For professional tables
\usepackage{longtable} % For tables that may span multiple pages
\usepackage{hyperref} % For clickable links
\usepackage{url} % For formatting URLs
\usepackage{seqsplit} % To split long strings without spaces
\usepackage{xcolor} % For colors

% Document Metadata
\title{Cybersecurity Assessment Report}
\author{Cybersecurity Analysis Division}
\date{\today}

% Hyperref Setup
\hypersetup{
    colorlinks=true,
    linkcolor=blue,
    filecolor=magenta,      
    urlcolor=cyan,
    pdftitle={Cybersecurity Assessment Report},
    pdfpagemode=FullScreen,
}

\begin{document}

\maketitle
\thispagestyle{empty}
\newpage

\tableofcontents
\thispagestyle{empty}
\newpage

\setcounter{page}{1}

% ==============================================================================
% Section 1: Executive Summary
% ==============================================================================
\section{Executive Summary}

This report provides a comprehensive cybersecurity assessment for \textbf{Phoenix Rising}, based on an analysis of network scan data, organizational security controls, and pre-existing risk information.

The assessment reveals a mixed security posture. The organization has implemented commendable multi-factor authentication (MFA) controls across key areas, significantly strengthening its identity and access management. However, critical vulnerabilities and administrative gaps expose the organization to substantial risk.

A critical-risk vulnerability was identified on an internal host (\texttt{10.0.0.15}). This host is running a dangerously outdated and misconfigured FTP service (\texttt{vsftpd 2.3.4}), which is susceptible to remote command execution and allows anonymous, unauthenticated access. This finding requires immediate remediation to prevent a potential system compromise.

Furthermore, significant gaps were noted in administrative controls. The absence of an employee acceptable use policy and the lack of mandatory annual security awareness training for all staff create a high-risk environment susceptible to insider threats and social engineering attacks. These findings, combined with the known risk of outdated Windows 7 workstations, underscore the need for a multi-faceted approach to risk mitigation.

Immediate actions should focus on securing the vulnerable FTP server, followed by the development and implementation of the missing security policies and training programs.

% ==============================================================================
% Section 2: Organizational Information
% ==============================================================================
\section{Organizational Information}

The following details were provided for the assessment.

\begin{tabular}{@{}ll}
\toprule
\textbf{Attribute} & \textbf{Value} \\
\midrule
Organization Name & \textbf{Phoenix Rising} \\
Email Domain & \seqsplit{\texttt{PhoenixRising.org}} \\
Website Domain & \seqsplit{\url{www.PhoenixRising.org}} \\
External IP Address & \seqsplit{\texttt{172.148.66.41}} \\
\bottomrule
\end{tabular}

% ==============================================================================
% Section 3: Security Control Review
% ==============================================================================
\section{Security Control Review}

A review of the organization's security controls was conducted via a questionnaire. The results are summarized below. "No" answers indicate significant control gaps that increase organizational risk.

\begin{table}[h!]
\centering
\begin{tabular}{@{}p{0.6\linewidth} c p{0.25\linewidth}@{}}
\toprule
\textbf{Control Question} & \textbf{Response} & \textbf{Assessment} \\
\midrule
Do you require MFA to access email? & \ding{51} & Strong control in place. \\
\addlinespace
Do you require MFA to log into computers? & \ding{51} & Strong control in place. \\
\addlinespace
Do you require MFA to access sensitive data systems? & \ding{51} & Strong control in place. \\
\addlinespace
Does your organization have an employee acceptable use policy? & \textcolor{red}{\ding{55}} & \textbf{High Risk.} Lack of a formal policy creates ambiguity and legal risk. \\
\addlinespace
Does your organization do security awareness training for new employees? & \ding{51} & Good baseline control. \\
\addlinespace
Does your organization do security awareness training for all employees at least once per year? & \textcolor{red}{\ding{55}} & \textbf{High Risk.} Absence of recurring training increases susceptibility to phishing and social engineering. \\
\bottomrule
\end{tabular}
\caption{Security Control Questionnaire Results}
\end{table}

% ==============================================================================
% Section 4: Technical Scan Results
% ==============================================================================
\section{Technical Scan Results}

An Nmap scan was performed on the specified target to identify open ports and running services.

\subsection{Host: \texttt{10.0.0.15}}
\begin{itemize}
    \item \textbf{Status:} Up
    \item \textbf{Summary:} This host exposes a highly vulnerable and misconfigured FTP service.
\end{itemize}

\begin{table}[h!]
\centering
\begin{tabular}{@{}lllll@{}}
\toprule
\textbf{Port} & \textbf{Service} & \textbf{Product} & \textbf{Version} & \textbf{Notes} \\
\midrule
21/tcp & ftp & vsftpd & 2.3.4 & \parbox[t]{0.4\linewidth}{\textbf{CRITICAL FINDING:}\\
1. \textbf{Anonymous FTP login allowed.} Unauthenticated access is permitted.\\
2. \textbf{Vulnerable Version.} This version is known to be vulnerable to a backdoor command execution flaw (CVE-2011-2523).} \\
\bottomrule
\end{tabular}
\caption{Open Ports and Services on \texttt{10.0.0.15}}
\end{table}

% ==============================================================================
% Section 5: Risk Assessment
% ==============================================================================
\section{Risk Assessment}

The following table synthesizes findings from the security control review, technical scans, and pre-existing risk data. Risks are prioritized by severity to guide remediation efforts.

\begin{longtable}{@{}p{0.1\linewidth} p{0.25\linewidth} p{0.45\linewidth} p{0.1\linewidth}@{}}
\toprule
\textbf{Risk ID} & \textbf{Risk Name} & \textbf{Description} & \textbf{Severity} \\
\midrule
\endfirsthead
\toprule
\textbf{Risk ID} & \textbf{Risk Name} & \textbf{Description} & \textbf{Severity} \\
\midrule
\endhead
\bottomrule
\endfoot
RISK-001 & Vulnerable FTP Service (CVE-2011-2523) & The FTP server on \texttt{10.0.0.15} is running vsftpd 2.3.4, a version containing a critical backdoor that allows an attacker to execute arbitrary commands with root privileges. & \textbf{Critical} \\
\addlinespace
RISK-002 & Anonymous FTP Access Enabled & The FTP server is configured to allow anonymous login. This permits any unauthenticated user on the network to access, download, or potentially upload files, leading to data breaches or malware propagation. & \textbf{Critical} \\
\addlinespace
RISK-003 & Lack of Annual Security Training & The absence of a mandatory, recurring security awareness training program for all employees significantly increases the likelihood of successful phishing, ransomware, and social engineering attacks. & \textbf{High} \\
\addlinespace
RISK-004 & No Employee Acceptable Use Policy & Without a formal policy defining acceptable use of company assets, there is no clear guidance for employees or basis for disciplinary action in case of misuse, increasing insider threat risk. & \textbf{High} \\
\addlinespace
RISK-005 & Outdated Windows Policy & (Pre-existing risk) Workstations are running Windows 7, an end-of-life operating system that no longer receives security updates, leaving them vulnerable to known exploits. & Medium \\
\end{longtable}

% ==============================================================================
% Section 6: Recommendations
% ==============================================================================
\section{Recommendations}

The following actions are recommended to mitigate the identified risks.

\subsection{Immediate Actions (Next 72 Hours)}

\subsubsection{Remediate Vulnerable FTP Service (RISK-001 \& RISK-002)}
\begin{itemize}
    \item \textbf{Action:} Immediately take the FTP service on \texttt{10.0.0.15} offline.
    \item \textbf{Investigation:} Determine the business purpose of this service and who uses it.
    \item \textbf{Long-Term Fix:} If the service is required, upgrade the vsftpd software to the latest patched version. If it is not required, decommission it permanently.
    \item \textbf{Configuration Fix:} Regardless of the path chosen, ensure anonymous login is disabled. If file transfer is needed, enforce encrypted protocols like SFTP or FTPS with strong, unique user credentials.
\end{itemize}

\subsection{Short-Term Actions (Next 30-90 Days)}

\subsubsection{Implement Security Policies and Training (RISK-003 \& RISK-004)}
\begin{itemize}
    \item \textbf{Acceptable Use Policy:} Develop and implement a comprehensive Employee Acceptable Use Policy (AUP). This policy should be reviewed by legal counsel, communicated to all employees, and formally acknowledged via signature.
    \item \textbf{Security Awareness Training:} Procure and deploy a security awareness training platform. Enroll all employees in a baseline training module and establish a formal program for mandatory, annual refresher training.
\end{itemize}

\subsection{Ongoing Actions}

\subsubsection{Address Outdated Operating Systems (RISK-005)}
\begin{itemize}
    \item \textbf{Action:} Execute the plan to upgrade all remaining Windows 7 workstations to a currently supported operating system, such as Windows 10 or Windows 11. Prioritize systems used by employees with access to sensitive data.
\end{itemize}

\end{document}
```