```latex
\documentclass[12pt]{article}

% Preamble: Required Packages
\usepackage[margin=1in]{geometry}
\usepackage{pifont} % For checkmarks and crosses
\usepackage{booktabs} % For professional tables
\usepackage{hyperref} % For clickable links
\usepackage{url} % For URL formatting
\usepackage{seqsplit} % To split long strings without breaking
\usepackage{xcolor} % For colors
\usepackage{graphicx} % For logos/images
\usepackage{fancyhdr} % For headers/footers

% --- Document Setup ---
\hypersetup{
    colorlinks=true,
    linkcolor=blue,
    filecolor=magenta,      
    urlcolor=cyan,
    pdftitle={Cybersecurity Posture Report},
    pdfauthor={Cybersecurity Analysis Division},
    pdfsubject={Security Assessment},
    pdfkeywords={Cybersecurity, Risk, Analysis},
}

% --- Header and Footer ---
\pagestyle{fancy}
\fancyhf{} % Clear all header and footer fields
\fancyhead[L]{Cybersecurity Posture Report}
\fancyhead[R]{Vivid Vision}
\fancyfoot[C]{\thepage}
\renewcommand{\headrulewidth}{0.4pt}
\renewcommand{\footrulewidth}{0.4pt}

% --- Title Page ---
\title{
    \vspace{2cm}
    \textbf{Cybersecurity Posture Report} \\
    \large \textit{Confidential Assessment for Vivid Vision} \\
    \vspace{1.5cm}
}
\author{Cybersecurity Analysis Division}
\date{\today}

% --- Document Body ---
\begin{document}

\maketitle
\thispagestyle{empty}
\newpage

\tableofcontents
\newpage

\section{Executive Overview}

This report provides a comprehensive cybersecurity assessment for \textbf{Vivid Vision}, based on an analysis of organizational security controls, an external network scan, and a review of known risks. The assessment was conducted to identify security gaps, evaluate the current risk posture, and provide actionable recommendations to enhance security.

The analysis reveals a mixed security posture. Positive controls include the enforcement of Multi-Factor Authentication (MFA) for email access and the implementation of a security awareness training program for all employees. However, several critical and high-risk gaps were identified that require immediate attention.

Key findings include:
\begin{itemize}
    \item \textbf{Critical Risk:} The absence of MFA for computer logins and access to sensitive data systems exposes the organization to significant risks of unauthorized access and potential data breaches.
    \item \textbf{High Risk:} The lack of a formal Employee Acceptable Use Policy creates ambiguity regarding security responsibilities and acceptable behavior, increasing the likelihood of insider threats and policy violations.
    \item \textbf{Positive Finding:} The external network scan of the target IP address did not identify any open ports, suggesting a strong external network perimeter configuration that limits the attack surface.
\end{itemize}

This report details these findings and provides prioritized recommendations to mitigate the identified risks and strengthen the overall security posture of \textbf{Vivid Vision}.

\section{Organizational Information}

The following details were provided for the assessment. This information establishes the context and scope of the review.

\begin{table}[h!]
\centering
\begin{tabular}{@{}ll@{}}
\toprule
\textbf{Attribute} & \textbf{Value} \\ \midrule
Organization Name & \textbf{Vivid Vision} \\
Email Domain      & \texttt{VividVision.org} \\
Website Domain    & \url{www.VividVision.org} \\
External IP Address & \texttt{176.180.42.184} \\ \bottomrule
\end{tabular}
\caption{Organizational Context Data}
\label{tab:org_data}
\end{table}

\section{Security Control Review}

A review of administrative and technical security controls was conducted based on a standardized questionnaire. The results highlight both strengths and weaknesses in the current security framework. "No" answers indicate significant gaps that directly translate to increased risk.

\begin{table}[h!]
\centering
\begin{tabular}{@{}p{0.7\linewidth}c@{}}
\toprule
\textbf{Security Control Question} & \textbf{Status} \\ \midrule
Do you require MFA to access email? & \textcolor{green!80!black}{\ding{51}} \\
Do you require MFA to log into computers? & \textcolor{red!80!black}{\ding{55}} \\
Do you require MFA to access sensitive data systems? & \textcolor{red!80!black}{\ding{55}} \\
Does your organization have an employee acceptable use policy? & \textcolor{red!80!black}{\ding{55}} \\
Does your organization do security awareness training for new employees? & \textcolor{green!80!black}{\ding{51}} \\
Does your organization do security awareness training for all employees at least once per year? & \textcolor{green!80!black}{\ding{51}} \\ \bottomrule
\end{tabular}
\caption{Security Controls Questionnaire Results}
\label{tab:controls}
\end{table}

\section{Technical Scan Results}

An external network vulnerability scan was performed to identify open ports and exposed services on the organization's perimeter.

\begin{itemize}
    \item \textbf{Target IP Address:} \texttt{[Target IP]}
    \item \textbf{Scan Date:} \today
\end{itemize}

\subsection{Findings}
The scan completed successfully. No open TCP or UDP ports were discovered on the target system. This is a positive security finding, as it indicates a well-configured firewall that minimizes the external attack surface and prevents unauthorized network access to internal systems.

\section{Risk Assessment}

This section correlates the findings from the security control review and the list of current risks. The risks identified below are based on the gaps discovered during this assessment. As no pre-existing vulnerabilities were provided, the table focuses solely on newly identified risks.

\begin{table}[h!]
\centering
\begin{tabular}{@{}p{0.25\linewidth}p{0.5\linewidth}l@{}}
\toprule
\textbf{Risk Name} & \textbf{Overview} & \textbf{Severity} \\ \midrule
\textbf{Lack of MFA for Endpoints} & The absence of MFA for computer logins allows an attacker with compromised credentials (e.g., from a phishing attack) to gain direct access to an employee's workstation and the corporate network. & \textcolor{red!80!black}{\textbf{Critical}} \\
\addlinespace
\textbf{Lack of MFA for Sensitive Systems} & Sensitive data systems are not protected by MFA. A single compromised password could lead to a significant data breach, regulatory fines, and reputational damage. & \textcolor{red!80!black}{\textbf{Critical}} \\
\addlinespace
\textbf{No Acceptable Use Policy (AUP)} & Without a formal AUP, employees may be unaware of security best practices and their responsibilities. This increases the risk of unintentional data exposure, malware infections, and misuse of company assets. & \textcolor{orange!90!black}{\textbf{High}} \\
\bottomrule
\end{tabular}
\caption{Summary of Identified Risks}
\label{tab:risks}
\end{table}

\section{Recommendations}

The following prioritized recommendations are provided to address the identified risks and improve the overall security posture of \textbf{Vivid Vision}.

\subsection{Critical Priority}
\begin{enumerate}
    \item \textbf{Implement MFA for All Computer and System Logins:}
    \begin{itemize}
        \item \textbf{Action:} Deploy a robust MFA solution (e.g., Microsoft Authenticator, Duo, Okta) for all employee computer logins (Windows, macOS) and for access to all systems containing sensitive or critical data.
        \item \textbf{Impact:} This is the single most effective control to prevent unauthorized access resulting from compromised credentials. It mitigates the risk of lateral movement and access to sensitive data.
    \end{itemize}
\end{enumerate}

\subsection{High Priority}
\begin{enumerate}
    \setcounter{enumi}{1} % Continue numbering from previous list
    \item \textbf{Develop and Enforce an Acceptable Use Policy (AUP):}
    \begin{itemize}
        \item \textbf{Action:} Create a comprehensive AUP that clearly defines rules for the use of company technology, data handling, internet usage, and security responsibilities. This policy must be formally communicated to all employees, who should be required to sign an acknowledgment of receipt and understanding.
        \item \textbf{Impact:} An AUP establishes a clear security baseline for all employees, reduces ambiguity, and provides a formal basis for disciplinary action in case of violations. It is a foundational component of a strong security culture.
    \end{itemize}
\end{enumerate}

\end{document}
```