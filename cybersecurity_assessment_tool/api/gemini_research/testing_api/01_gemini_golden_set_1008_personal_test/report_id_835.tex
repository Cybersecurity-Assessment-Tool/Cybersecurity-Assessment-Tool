```latex
\documentclass[12pt]{article}

% Preamble: Required Packages
\usepackage[margin=1in]{geometry}
\usepackage{pifont} % For checkmarks and crosses
\usepackage{booktabs} % For professional tables
\usepackage{hyperref} % For clickable links
\usepackage{url} % For URL formatting
\usepackage{seqsplit} % For splitting long strings like IPs
\usepackage{graphicx} % For potential logos/images
\usepackage{xcolor} % For colors

% Document Information
\title{Cybersecurity Posture Assessment Report}
\author{Cybersecurity Analysis Division}
\date{\today}

\begin{document}

\maketitle
\thispagestyle{empty}
\newpage
\tableofcontents
\newpage

% --- 1. Executive Summary ---
\section{Executive Summary}

This report provides a comprehensive cybersecurity assessment for \textbf{White Label}, based on a technical network scan, a review of organizational security controls, and an analysis of pre-existing risks. The assessment was conducted to identify vulnerabilities, evaluate the current security posture, and provide actionable recommendations to mitigate identified risks.

The analysis revealed several critical and high-risk gaps in the organization's security controls. Key findings include the absence of Multi-Factor Authentication (MFA) for email and computer access, the lack of a formal Acceptable Use Policy, and incomplete security awareness training. These policy and procedural weaknesses significantly elevate the risk of unauthorized access and data compromise.

A technical scan identified an exposed management port (SSH) on an external-facing IPv6 address. When correlated with the lack of MFA, this finding presents a tangible and immediate threat that could be exploited by malicious actors.

This report outlines specific, prioritized recommendations to address these findings, focusing on strengthening access controls, formalizing security policies, and enhancing employee security awareness.

% --- 2. Organizational Information ---
\section{Organizational Information}

The following information was provided for the assessment.

\begin{tabular}{@{}ll}
\toprule
\textbf{Attribute} & \textbf{Value} \\
\midrule
Organization Name & \textbf{White Label} \\
Email Domain & \texttt{WhiteLabel.org} \\
Website Domain & \url{www.WhiteLabel.org} \\
External IP Address & \texttt{29.241.223.67} \\
\bottomrule
\end{tabular}

% --- 3. Security Control Review ---
\section{Security Control Review}

A review of administrative and procedural security controls was conducted based on a standardized questionnaire. The results highlight significant gaps in foundational security practices. A "No" response indicates a deviation from best practices and a potential risk to the organization.

\begin{table}[h!]
\centering
\caption{Security Controls Questionnaire Analysis}
\begin{tabular}{@{}p{0.6\linewidth}cc@{}}
\toprule
\textbf{Control Question} & \textbf{Response} & \textbf{Status} \\
\midrule
Do you require MFA to access email? & \ding{55} & \textcolor{red}{\textbf{High Risk}} \\
Do you require MFA to log into computers? & \ding{55} & \textcolor{red}{\textbf{High Risk}} \\
Do you require MFA to access sensitive data systems? & \ding{51} & Implemented \\
Does your organization have an employee acceptable use policy? & \ding{55} & \textcolor{red}{\textbf{Critical Gap}} \\
Does your organization do security awareness training for new employees? & \ding{51} & Implemented \\
Does your organization do security awareness training for all employees at least once per year? & \ding{55} & \textcolor{red}{\textbf{High Risk}} \\
\bottomrule
\end{tabular}
\\
\vspace{0.1in}
\textit{\small Legend: \ding{51} = Yes (Control in place), \ding{55} = No (Control is missing)}
\end{table}

% --- 4. Technical Network Scan Results ---
\section{Technical Network Scan Results}

An external network scan was performed to identify exposed services on the organization's perimeter. The scan provides insight into the technical attack surface visible to external adversaries.

\subsection{Scan Summary}
\begin{itemize}
    \item \textbf{Target IP Address:} \seqsplit{\texttt{2001:db8::1}}
    \item \textbf{Host Status:} Up
    \item \textbf{Key Finding:} One open port was identified.
\end{itemize}

\subsection{Open Ports Details}
The following table details the service discovered during the scan.

\begin{table}[h!]
\centering
\caption{Discovered Open Ports}
\begin{tabular}{@{}cccl@{}}
\toprule
\textbf{Port} & \textbf{Protocol} & \textbf{State} & \textbf{Common Service} \\
\midrule
22 & TCP & Open & SSH (Secure Shell) \\
\bottomrule
\end{tabular}
\end{table}

\textbf{Analysis:} The presence of an open SSH port on an external-facing system represents a significant risk. SSH is a common vector for brute-force attacks, where attackers attempt to guess credentials to gain remote administrative access. This risk is severely amplified by the organizational lack of MFA for computer access.

% --- 5. Consolidated Risk Assessment ---
\section{Consolidated Risk Assessment}
This section synthesizes findings from the security control review and the technical scan. No pre-existing vulnerabilities were reported. The following new risks have been identified and prioritized.

\begin{table}[h!]
\centering
\caption{Summary of Identified Risks}
\begin{tabular}{@{}p{0.25\linewidth}p{0.5\linewidth}l@{}}
\toprule
\textbf{Risk Title} & \textbf{Description} & \textbf{Severity} \\
\midrule
\textbf{Lack of MFA on Critical Systems} & The absence of MFA for email and computer logins exposes the organization to account compromise through phishing, credential stuffing, and password spraying attacks. & \textcolor{red}{\textbf{High}} \\
\addlinespace
\textbf{Exposed SSH Service} & The SSH management port is open to the internet on \seqsplit{\texttt{2001:db8::1}}. This service is a primary target for automated brute-force attacks, especially without MFA protection. & \textcolor{orange}{\textbf{Medium}} \\
\addlinespace
\textbf{Inadequate Security Awareness Program} & While new hires receive training, the lack of a mandatory annual program for all employees leads to knowledge decay, making staff more susceptible to evolving social engineering tactics. & \textcolor{orange}{\textbf{Medium}} \\
\addlinespace
\textbf{Missing Acceptable Use Policy (AUP)} & Without a formal AUP, there are no clear, enforceable rules for employees regarding the use of company assets, data handling, and security responsibilities. This creates legal and operational risks. & \textcolor{orange}{\textbf{Medium}} \\
\bottomrule
\end{tabular}
\end{table}

% --- 6. Recommendations ---
\section{Recommendations}
The following actions are recommended to mitigate the identified risks and improve the overall security posture of \textbf{White Label}.

\begin{enumerate}
    \item \textbf{Implement Comprehensive MFA (Immediate Priority):}
    \begin{itemize}
        \item Enforce MFA for all user accounts across all systems, prioritizing email (e.g., Office 365, Google Workspace) and remote access solutions (e.g., VPN, SSH).
        \item This single control is the most effective measure to prevent account compromise and directly mitigates the highest-rated risk.
    \end{itemize}
    \item \textbf{Secure the Exposed SSH Service (High Priority):}
    \begin{itemize}
        \item If remote access via SSH is required, restrict access to a whitelist of trusted source IP addresses.
        \item Disable password-based authentication and enforce the use of public-key cryptography (SSH keys) for all SSH connections.
        \item If the service is not required for external access, it should be blocked at the network firewall.
    \end{itemize}
    \item \textbf{Develop and Implement an Acceptable Use Policy (High Priority):}
    \begin{itemize}
        \item Create a formal AUP that defines the rules for using company technology and data.
        \item The policy should be reviewed by legal counsel, communicated to all employees, and require a signed acknowledgment of receipt.
    \end{itemize}
    \item \textbf{Establish a Mandatory Annual Security Awareness Program (Medium Priority):}
    \begin{itemize}
        \item Institute a recurring, mandatory security awareness training program for all employees.
        \item Training should cover current threats such as phishing, ransomware, and social engineering, reinforcing concepts introduced during onboarding.
    \end{itemize}
\end{enumerate}

\end{document}
```