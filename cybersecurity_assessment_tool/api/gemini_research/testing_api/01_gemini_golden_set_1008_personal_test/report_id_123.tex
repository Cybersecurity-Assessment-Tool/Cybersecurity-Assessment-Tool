```latex
\documentclass[12pt, a4paper]{article}

% Preamble: Required Packages
\usepackage[margin=1in]{geometry}
\usepackage{pifont} % For checkmarks and crosses
\usepackage{booktabs} % For professional tables
\usepackage{hyperref} % For clickable links
\usepackage{url} % For URL formatting
\usepackage{seqsplit} % For splitting long strings in tt font
\usepackage{graphicx}
\usepackage{xcolor}

% Document Metadata
\title{Cybersecurity Posture Assessment Report}
\author{Cybersecurity Analysis Division}
\date{\today}

% Hyperref Setup
\hypersetup{
    colorlinks=true,
    linkcolor=blue,
    filecolor=magenta,      
    urlcolor=cyan,
    pdftitle={Cybersecurity Posture Assessment Report},
    pdfpagemode=FullScreen,
}

\begin{document}

\maketitle
\thispagestyle{empty}
\newpage

\tableofcontents
\newpage

\section{Executive Summary}

This report provides a comprehensive analysis of the cybersecurity posture for \textbf{Blue Marble}. The assessment is based on a correlation of network scan data, organizational security control questionnaires, and a review of pre-existing risks.

The analysis identified several critical and high-risk vulnerabilities that require immediate attention. The most significant findings include:

\begin{itemize}
    \item \textbf{Critical - Exposed End-of-Life Database:} A MySQL database (version 5.7.33) is directly exposed to the network on port 3306. This version reached its official End of Life in October 2023 and no longer receives security updates, making it an easy target for exploitation.
    \item \textbf{Critical - Lack of Security Training:} The organization does not provide security awareness training for new or existing employees. This represents a fundamental gap in security, as an untrained workforce is highly susceptible to phishing, social engineering, and malware-based attacks.
    \item \textbf{High - Insufficient MFA Enforcement:} Multi-Factor Authentication (MFA) is not required for logging into company computers. This significantly increases the risk of unauthorized access and lateral movement within the network should an employee's credentials be compromised.
\end{itemize}

The combination of an exposed, unpatched database and significant gaps in foundational security controls places the organization at a high risk of a security breach. Immediate remediation of the identified issues is strongly recommended.

\section{Organizational Information}

The following information was provided for the assessment.

\begin{tabular}{@{}ll}
\toprule
\textbf{Attribute} & \textbf{Value} \\
\midrule
Organization Name & \textbf{Blue Marble} \\
Primary Email Domain & \texttt{BlueMarble.org} \\
Primary Website & \url{www.BlueMarble.org} \\
External IP Address & \texttt{18.50.22.125} \\
\bottomrule
\end{tabular}

\section{Security Control Review}

A review of the organization's security controls was conducted via a questionnaire. The results highlight significant gaps in employee security practices and endpoint protection. Answers marked with \ding{55} indicate a deviation from security best practices and represent an area of risk.

\begin{table}[h!]
\centering
\begin{tabular}{@{}p{0.8\linewidth}c}
\toprule
\textbf{Security Control Question} & \textbf{Status} \\
\midrule
Do you require MFA to access email? & \ding{51} \\
\textbf{Do you require MFA to log into computers?} & \textcolor{red}{\ding{55}} \\
Do you require MFA to access sensitive data systems? & \ding{51} \\
Does your organization have an employee acceptable use policy? & \ding{51} \\
\textbf{Does your organization do security awareness training for new employees?} & \textcolor{red}{\ding{55}} \\
\textbf{Does your organization do security awareness training for all employees at least once per year?} & \textcolor{red}{\ding{55}} \\
\bottomrule
\end{tabular}
\caption{Organizational Security Control Status}
\label{tab:controls}
\end{table}

\subsection*{Analysis of Gaps}
The lack of mandatory security awareness training for both new and existing employees is a critical vulnerability. It creates a permissive environment for human error, which is a leading cause of security incidents. Furthermore, the absence of MFA on computer logins negates many of the security benefits gained from protecting email and sensitive systems, as a compromised endpoint provides a direct path into the corporate network.

\section{Technical Scan Results}

An external network scan was performed against the target IP address \texttt{172.16.50.20}. The scan identified one open port with a service that is running an outdated, End-of-Life software version.

\begin{table}[h!]
\centering
\begin{tabular}{@{}lllll}
\toprule
\textbf{Port} & \textbf{State} & \textbf{Service} & \textbf{Product} & \textbf{Version} \\
\midrule
3306/tcp & Open & mysql & MySQL & 5.7.33 \\
\bottomrule
\end{tabular}
\caption{Open Ports and Services}
\label{tab:scanresults}
\end{table}

\subsection*{Technical Findings}
\begin{itemize}
    \item \textbf{Exposed Database Service:} Port 3306 is the default port for MySQL. Exposing a database server directly to the network is a major security risk, as it allows attackers to directly target the service with brute-force attacks, credential stuffing, or exploits for known vulnerabilities.
    \item \textbf{End-of-Life Software:} MySQL version 5.7.33 is no longer supported by its developer as of October 2023. This means it does not receive security patches for newly discovered vulnerabilities, leaving it perpetually at risk.
\end{itemize}

\section{Consolidated Risk Assessment}
The following table synthesizes findings from the security control review, technical scan, and pre-existing risk data into a prioritized list.

\begin{table}[h!]
\centering
\begin{tabular}{@{}p{0.25\linewidth}p{0.1\linewidth}p{0.6\linewidth}}
\toprule
\textbf{Risk Title} & \textbf{Severity} & \textbf{Description \& Affected Elements} \\
\midrule
\textbf{Exposed \& Outdated Database} & \textbf{Critical} & A MySQL 5.7.33 database is open on port 3306. This version is End-of-Life and unpatched. This confirms and elevates the pre-existing "Database Exposure" risk. \newline \textit{Affected: \texttt{172.16.50.20:3306}} \\
\addlinespace
\textbf{Lack of Security Awareness Training} & \textbf{Critical} & The absence of a formal security training program for any employees makes the organization highly vulnerable to social engineering and phishing attacks. \newline \textit{Affected: All Employees} \\
\addlinespace
\textbf{Insufficient MFA on Endpoints} & \textbf{High} & Lack of MFA on computer logins allows for trivial account takeover and lateral movement if credentials are stolen. \newline \textit{Affected: All Employee Computers} \\
\bottomrule
\end{tabular}
\caption{Prioritized Risk Summary}
\label{tab:risks}
\end{table}

\section{Recommendations}

Based on the consolidated risk assessment, the following actions are recommended to improve the organization's security posture. Recommendations are prioritized by urgency.

\begin{enumerate}
    \item \textbf{[Immediate] Isolate the Database Server:} Immediately apply firewall rules to restrict access to port 3306 on host \texttt{172.16.50.20}. Access should only be permitted from trusted application servers. Public access must be disabled.

    \item \textbf{[High Priority] Implement Security Awareness Training:} Procure and deploy a security awareness training program. All new hires must complete the training as part of their onboarding, and all existing employees must complete it annually.

    \item \textbf{[High Priority] Enforce MFA on All Endpoints:} Enable and enforce Multi-Factor Authentication for all computer and remote access logins (e.g., VPN). This is a critical defense against credential theft.

    \item \textbf{[Medium Priority] Plan Database Upgrade:} Develop a plan to migrate the MySQL 5.7.33 database to a currently supported version (e.g., MySQL 8.x). This will ensure the system receives critical security patches. This should be executed after the immediate isolation of the server.
\end{enumerate}

\section{Conclusion}

The current security posture of \textbf{Blue Marble} presents a high level of risk due to a combination of a critically exposed and outdated system, and fundamental gaps in organizational security controls. The identified vulnerabilities, particularly the exposed database and lack of employee training, create a significant likelihood of a security incident.

By implementing the prioritized recommendations outlined in this report, the organization can substantially reduce its attack surface and build a more resilient security foundation.

\end{document}
```