```latex
\documentclass[12pt]{article}

% Preamble: Required Packages
\usepackage[a4paper, margin=1in]{geometry}
\usepackage{pifont} % For checkmarks and crosses
\usepackage{booktabs} % For professional tables
\usepackage{hyperref} % For clickable links
\usepackage{url}      % For URL formatting
\usepackage{seqsplit} % For splitting long strings in tt font

% Document Metadata
\title{Cybersecurity Posture Assessment Report}
\author{Cybersecurity Analysis Division}
\date{\today}

% Hyperref Setup
\hypersetup{
    colorlinks=true,
    linkcolor=black,
    urlcolor=blue,
    pdftitle={Cybersecurity Posture Assessment Report},
    pdfauthor={Cybersecurity Analysis Division},
}

\begin{document}

\maketitle
\thispagestyle{empty}
\newpage
\tableofcontents
\newpage

% --- 1. Executive Overview ---
\section{Executive Overview}
This report details the findings of a cybersecurity posture assessment conducted for \textbf{Blue Horizon Initiative}. The assessment incorporated an analysis of organizational security controls, a technical network scan of the external perimeter, and a review of known risks.

The overall security posture presents a mixed landscape. On one hand, the external network perimeter at the scanned IP address appears to be well-hardened, as the technical scan did not detect any open ports. This is a positive indicator of a strong firewall configuration.

However, the assessment identified two significant policy and control gaps that introduce substantial risk to the organization:
\begin{itemize}
    \item \textbf{Critical Risk:} The absence of Multi-Factor Authentication (MFA) on employee email accounts. Email is a primary vector for cyberattacks, and this gap leaves the organization highly vulnerable to account compromise and subsequent data breaches.
    \item \textbf{High Risk:} The lack of mandatory, annual security awareness training for all employees. An organization's security is only as strong as its least aware user, and without regular training, staff are more susceptible to social engineering and phishing attacks.
\end{itemize}

Immediate remediation of these control gaps is strongly recommended to mitigate the risk of a significant security incident.

% --- 2. Organizational Information ---
\section{Organizational Information}
The following details were provided for the assessment.

\begin{tabular}{@{}ll}
    \toprule
    \textbf{Attribute} & \textbf{Value} \\
    \midrule
    Organization Name & \textbf{Blue Horizon Initiative} \\
    Email Domain & \seqsplit{\texttt{BlueHorizonInitiative.net}} \\
    Website Domain & \seqsplit{\url{www.BlueHorizonInitiative.net}} \\
    External IP Address & \texttt{26.208.114.241} \\
    \bottomrule
\end{tabular}

% --- 3. Security Control Review ---
\section{Security Control Review}
A review of organizational security controls was conducted based on a standardized questionnaire. The responses indicate key areas of strength and weakness in the current security policy framework. A (\ding{51}) indicates an affirmative response (control in place), while a (\ding{55}) indicates a negative response (control gap).

\begin{table}[h!]
\centering
\begin{tabular}{@{}p{0.8\linewidth}c@{}}
    \toprule
    \textbf{Control Question} & \textbf{Response} \\
    \midrule
    Do you require MFA to access email? & \ding{55} \\
    Do you require MFA to log into computers? & \ding{51} \\
    Do you require MFA to access sensitive data systems? & \ding{51} \\
    Does your organization have an employee acceptable use policy? & \ding{51} \\
    Does your organization do security awareness training for new employees? & \ding{51} \\
    Does your organization do security awareness training for all employees at least once per year? & \ding{55} \\
    \bottomrule
\end{tabular}
\caption{Organizational Security Control Questionnaire Results.}
\end{label{tab:controls}
\end{table}

% --- 4. Technical Scan Results ---
\section{Technical Scan Results}
A network reconnaissance scan was performed on the organization's designated external IP address to identify exposed services.

\begin{itemize}
    \item \textbf{Target IP Address:} \texttt{[Target IP]}
    \item \textbf{Scan Date:} \today
\end{itemize}

\subsection{Summary of Findings}
The network scan did not identify any open TCP ports on the target system. This is a positive finding and suggests that a well-configured firewall is in place, effectively limiting the external attack surface. No vulnerable services were exposed to the public internet at the time of the scan.

% --- 5. Risk Assessment & Correlation ---
\section{Risk Assessment \& Correlation}
The following table synthesizes risks identified from the security control review and technical analysis. No pre-existing vulnerabilities were reported. The primary risks stem from internal policy and procedure gaps rather than technical vulnerabilities on the network perimeter.

\begin{table}[h!]
\centering
\begin{tabular}{@{}p{0.1\linewidth}p{0.3\linewidth}p{0.4\linewidth}l@{}}
    \toprule
    \textbf{Risk ID} & \textbf{Risk Name} & \textbf{Description} & \textbf{Severity} \\
    \midrule
    R-01 & Inadequate Email Account Protection & Email accounts lack MFA, making them highly susceptible to phishing, credential stuffing, and unauthorized access. A compromised email account can lead to data breaches and further network compromise. & \textbf{Critical} \\
    \addlinespace
    R-02 & Insufficient Security Training & The absence of mandatory annual security training for all employees increases the likelihood of human error, such as falling for phishing scams or mishandling sensitive data. & \textbf{High} \\
    \bottomrule
\end{tabular}
\caption{Consolidated Risk Register.}
\label{tab:risks}
\end{table}

% --- 6. Recommendations ---
\section{Recommendations}
Based on the findings of this assessment, the following actions are recommended to enhance the organization's cybersecurity posture. Recommendations are prioritized by severity.

\begin{itemize}
    \item[\textbf{1.}] \textbf{(Critical - R-01)} \textbf{Implement MFA for Email:} Immediately develop a plan to deploy and enforce mandatory Multi-Factor Authentication (MFA) for all user access to the email system. This is the single most effective control to prevent unauthorized account access and should be the top priority.
    
    \item[\textbf{2.}] \textbf{(High - R-02)} \textbf{Establish Annual Security Training:} Institute a formal, mandatory security awareness training program for all employees, to be completed annually. This program should cover current threats, including phishing, social engineering, password hygiene, and proper data handling procedures.
    
    \item[\textbf{3.}] \textbf{(Informational)} \textbf{Confirm Scan Target:} As a matter of due diligence, confirm that the scanned IP address (\texttt{[Target IP]}) corresponds to the correct primary external gateway for the organization's critical services. This ensures the validity of the technical scan results.
\end{itemize}

\end{document}
```