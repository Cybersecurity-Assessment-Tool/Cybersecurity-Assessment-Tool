```latex
\documentclass[12pt]{article}

% Required Packages
\usepackage[margin=1in]{geometry}
\usepackage{pifont}
\usepackage{booktabs}
\usepackage{graphicx}
\usepackage{hyperref}
\usepackage{url}
\usepackage{seqsplit}
\usepackage{xcolor}

% Document Formatting
\hypersetup{
    colorlinks=true,
    linkcolor=blue,
    filecolor=magenta,      
    urlcolor=cyan,
    pdftitle={Cybersecurity Posture Report},
    pdfpagemode=FullScreen,
}

\linespread{1.2}

% --- Document Start ---
\begin{document}

\begin{titlepage}
    \centering
    \vspace*{1cm}
    \Huge{\textbf{Cybersecurity Posture Report}}
    \vspace{0.5cm}
    \Large{Prepared for: \textbf{Digital Drift}}
    \vspace{1.5cm}
    \normalsize
    \textbf{Date of Report:} \today \\
    \textbf{Author:} Cybersecurity Analyst
    \vspace{2cm}
    \vfill
    \begin{abstract}
        This report provides a comprehensive analysis of the cybersecurity posture for Digital Drift. The assessment is based on a synthesis of data from a network vulnerability scan, a review of organizational security controls, and an analysis of pre-existing risks. The findings indicate a mixed security posture. While the external network scan of the target system revealed no open ports, suggesting a well-configured perimeter defense, significant gaps were identified in internal security policies. Critical deficiencies include the absence of multi-factor authentication for computer logins and the lack of mandatory annual security awareness training for all employees. These gaps expose the organization to substantial risks, including unauthorized access and social engineering attacks. This document details these findings and provides actionable recommendations to mitigate the identified risks and strengthen the overall security framework.
    \end{abstract}
\end{titlepage}

\tableofcontents
\newpage

% ------------------------------------------------------------------
% Section 1: Overview
% ------------------------------------------------------------------
\section{Organizational Information}

This section outlines the key organizational details provided for this assessment.

\begin{itemize}
    \item \textbf{Organization Name:} Digital Drift
    \item \textbf{Email Domain:} \texttt{DigitalDrift.org}
    \item \textbf{Website Domain:} \texttt{www.DigitalDrift.org}
    \item \textbf{External IP Address:} \texttt{136.15.238.20}
\end{itemize}

% ------------------------------------------------------------------
% Section 2: Security Control Review
% ------------------------------------------------------------------
\section{Security Control Review}

The following table summarizes the organization's responses to a security controls questionnaire. This review provides insight into the current policies and procedures governing the security environment. A checkmark (\ding{51}) indicates an affirmative response (control in place), while a cross mark (\ding{55}) indicates a negative response (control gap).

\begin{table}[h!]
\centering
\caption{Security Controls Questionnaire Results}
\begin{tabular}{p{0.75\linewidth} c}
\toprule
\textbf{Control Question} & \textbf{Response} \\
\midrule
Do you require MFA to access email? & \textcolor{green}{\ding{51}} \\
\textbf{Do you require MFA to log into computers?} & \textcolor{red}{\ding{55}} \\
Do you require MFA to access sensitive data systems? & \textcolor{green}{\ding{51}} \\
Does your organization have an employee acceptable use policy? & \textcolor{green}{\ding{51}} \\
Does your organization do security awareness training for new employees? & \textcolor{green}{\ding{51}} \\
\textbf{Does your organization do security awareness training for all employees at least once per year?} & \textcolor{red}{\ding{55}} \\
\bottomrule
\end{tabular}
\end{table}

\subsection*{Analysis of Control Gaps}
Two significant control gaps were identified:
\begin{itemize}
    \item \textbf{Lack of MFA for Computer Logins:} This is a critical vulnerability. Without MFA, a compromised password is all an attacker needs to gain full access to an employee's computer, potentially leading to data theft, malware deployment, or lateral movement within the network.
    \item \textbf{Lack of Annual Security Awareness Training:} The threat landscape evolves continuously. Failing to provide regular, updated training for all employees increases the organization's susceptibility to phishing, social engineering, and other human-centric attacks. Initial training for new hires is a good first step, but it is not sufficient for long-term defense.
\end{itemize}

% ------------------------------------------------------------------
% Section 3: Technical Scan Results
% ------------------------------------------------------------------
\section{Technical Scan Results}

A network scan was performed to identify open ports and exposed services on the specified target system.

\begin{itemize}
    \item \textbf{Target IP Address:} \texttt{192.168.1.100}
    \item \textbf{Host Status:} Up
    \item \textbf{Scan Date:} \today
\end{itemize}

\subsection*{Findings}
The scan results were positive from a network security perspective. The target host was responsive, but \textbf{no open ports were discovered}. All other scanned ports were reported as 'closed'. This indicates that the host is likely protected by a well-configured firewall that denies unsolicited incoming traffic, significantly reducing its external attack surface.

% ------------------------------------------------------------------
% Section 4: Consolidated Risk Assessment
% ------------------------------------------------------------------
\section{Consolidated Risk Assessment}

This section synthesizes findings from the security control review, technical scan, and any pre-existing risk data. The following table prioritizes the most significant risks identified during this assessment.

\begin{table}[h!]
\centering
\caption{Identified Security Risks}
\begin{tabular}{p{0.1\linewidth} p{0.3\linewidth} p{0.4\linewidth} p{0.1\linewidth}}
\toprule
\textbf{Risk ID} & \textbf{Risk Name} & \textbf{Description} & \textbf{Severity} \\
\midrule
RISK-001 & Lack of Endpoint MFA & The absence of MFA on computer logins means a single compromised password could grant an attacker full device access. & \textbf{Critical} \\
\addlinespace
RISK-002 & Insufficient Security Training & Without mandatory annual training, employees are more likely to fall victim to evolving phishing and social engineering tactics. & \textbf{High} \\
\addlinespace
RISK-003 & No Open Ports Detected & (Informational) The network scan found no vulnerabilities on the target host, indicating a strong perimeter defense for that system. & Info \\
\bottomrule
\end{tabular}
\end{table}

% ------------------------------------------------------------------
% Section 5: Recommendations
% ------------------------------------------------------------------
\section{Recommendations}

Based on the consolidated risk assessment, the following actions are recommended to mitigate the identified vulnerabilities and enhance the overall security posture of Digital Drift.

\subsection*{RISK-001: Lack of Endpoint MFA (Critical)}
\begin{itemize}
    \item \textbf{Action:} Implement and enforce Multi-Factor Authentication (MFA) for all employee computer logins, including both local and remote access.
    \item \textbf{Justification:} This will create a critical security layer that protects against unauthorized access even if user credentials are stolen. It is one of the most effective controls for preventing account takeover.
    \item \textbf{Priority:} Immediate.
\end{itemize}

\subsection*{RISK-002: Insufficient Security Training (High)}
\begin{itemize}
    \item \textbf{Action:} Establish a mandatory annual security awareness training program for all employees. The program should cover current threats such as phishing, ransomware, and proper data handling.
    \item \textbf{Justification:} A well-informed workforce is a critical component of a defense-in-depth strategy. Regular training ensures that security remains a top-of-mind concern and equips employees to recognize and report threats effectively.
    \item \textbf{Priority:} High.
\end{itemize}

\end{document}
```