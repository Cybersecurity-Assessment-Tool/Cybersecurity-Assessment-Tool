```latex
\documentclass[12pt]{article}

% Preamble: Required Packages
\usepackage[margin=1in]{geometry}
\usepackage{pifont} % For checkmarks and crosses
\usepackage{booktabs} % For professional tables
\usepackage{hyperref} % For clickable links
\usepackage{url} % For formatting URLs
\usepackage{seqsplit} % For splitting long strings
\usepackage{graphicx}
\usepackage{xcolor}

% Document Metadata
\title{Cybersecurity Posture Assessment Report}
\author{Cybersecurity Analyst Group}
\date{\today}

% Hyperref Setup
\hypersetup{
    colorlinks=true,
    linkcolor=blue,
    filecolor=magenta,      
    urlcolor=cyan,
    pdftitle={Cybersecurity Posture Assessment Report},
    pdfpagemode=FullScreen,
}

\begin{document}

\maketitle
\thispagestyle{empty}
\newpage

\tableofcontents
\newpage

% --- 1. Executive Overview ---
\section{Executive Overview}
This report details the findings of a cybersecurity posture assessment conducted for \textbf{True Grit}. The assessment combined a review of organizational security controls via a questionnaire, an external network vulnerability scan, and an analysis of pre-existing risks.

The external network scan against the target IP address \texttt{[Target IP]} revealed \textbf{no open ports or exposed services}. This is a positive finding and indicates a strong perimeter security posture, likely due to a well-configured firewall that denies unsolicited inbound traffic.

However, the organizational security control review identified several significant policy and procedural gaps that present a high level of risk. The most critical findings include:
\begin{itemize}
    \item \textbf{Critical Risk:} Multi-Factor Authentication (MFA) is not enforced for accessing sensitive data systems.
    \item \textbf{High Risk:} The organization lacks a formal employee Acceptable Use Policy (AUP).
    \item \textbf{High Risk:} New employees do not receive mandatory security awareness training during their onboarding process.
\end{itemize}

While the technical perimeter appears secure, these internal control deficiencies expose the organization to significant threats, including unauthorized data access, insider threats, and social engineering attacks. Immediate remediation of these policy-based risks is strongly recommended.

% --- 2. Organizational Information ---
\section{Organizational Information}
The following information was provided for the assessment.

\begin{tabular}{@{}ll}
    \toprule
    \textbf{Attribute} & \textbf{Value} \\
    \midrule
    Organization Name & True Grit \\
    Email Domain & \texttt{TrueGrit.net} \\
    Website Domain & \url{www.TrueGrit.net} \\
    Primary External IP & \seqsplit{\texttt{90.190.181.56}} \\
    \bottomrule
\end{tabular}

% --- 3. Security Control Review ---
\section{Security Control Review}
A review of internal security controls was conducted based on a standardized questionnaire. The responses indicate key areas where security practices are strong and where they are critically deficient. A \ding{51} indicates an affirmative response (control in place), while a \ding{55} indicates a negative response (control gap).

\begin{table}[h!]
\centering
\begin{tabular}{@{}lc}
    \toprule
    \textbf{Control Question} & \textbf{Response} \\
    \midrule
    Do you require MFA to access email? & \ding{51} \\
    Do you require MFA to log into computers? & \ding{51} \\
    Do you require MFA to access sensitive data systems? & \textcolor{red}{\ding{55}} \\
    Does your organization have an employee acceptable use policy? & \textcolor{red}{\ding{55}} \\
    Does your organization do security awareness training for new employees? & \textcolor{red}{\ding{55}} \\
    Does your organization do security awareness training for all employees at least once per year? & \ding{51} \\
    \bottomrule
\end{tabular}
\caption{Organizational Security Control Questionnaire Results}
\end{table}

The identified gaps are analyzed in the Risk Assessment section of this report.

% --- 4. Technical Scan Results ---
\section{Technical Scan Results}
An external network scan was performed to identify open ports and exposed services on the organization's perimeter.

\begin{itemize}
    \item \textbf{Target IP Address:} \texttt{[Target IP]}
    \item \textbf{Scan Date:} \today
\end{itemize}

\subsection{Summary of Findings}
The network scan completed successfully and found \textbf{no open TCP or UDP ports} on the target system. This result suggests that the external network perimeter is well-secured, with a firewall likely configured to drop or reject all unsolicited incoming traffic. This is a strong security practice that significantly reduces the external attack surface. No vulnerabilities were identified as no services were exposed.

% --- 5. Risk Assessment ---
\section{Risk Assessment}
This section synthesizes findings from the security control review, technical scan, and pre-existing risk data. No pre-existing vulnerabilities were reported. The primary risks identified are related to internal policies and procedures.

\begin{table}[h!]
\centering
\begin{tabular}{@{}p{0.25\linewidth}p{0.5\linewidth}p{0.15\linewidth}@{}}
    \toprule
    \textbf{Risk Name} & \textbf{Description} & \textbf{Severity} \\
    \midrule
    \textbf{Lack of MFA on Sensitive Systems} & Failure to enforce MFA on systems containing sensitive data exposes the organization to severe risk of unauthorized access and data breaches, especially if credentials are compromised. & \textbf{Critical} \\
    \addlinespace
    \textbf{No Employee Acceptable Use Policy (AUP)} & The absence of a formal AUP creates ambiguity regarding the proper use of company technology and data. This increases the risk of insider threats (both malicious and accidental) and potential legal or compliance issues. & \textbf{High} \\
    \addlinespace
    \textbf{Inadequate New Hire Security Training} & New employees are not receiving security awareness training upon joining. As new staff are often targeted by attackers, this gap makes them highly susceptible to phishing, social engineering, and other common attacks. & \textbf{High} \\
    \bottomrule
\end{tabular}
\caption{Summary of Identified Risks}
\end{table}

% --- 6. Recommendations ---
\section{Recommendations}
The following actions are recommended to mitigate the identified risks and improve the overall security posture of \textbf{True Grit}.

\subsection{Remediation for Lack of MFA (Critical)}
\begin{itemize}
    \item \textbf{Action:} Immediately develop a plan to implement and enforce MFA across all systems, databases, and applications that store, process, or transmit sensitive organizational or customer data.
    \item \textbf{Priority:} This should be the highest priority remediation effort.
    \item \textbf{Details:} Prioritize systems based on data criticality. Communicate the upcoming change to all affected users and provide clear instructions for enrollment.
\end{itemize}

\subsection{Remediation for Missing AUP (High)}
\begin{itemize}
    \item \textbf{Action:} Develop and ratify a comprehensive Employee Acceptable Use Policy (AUP).
    \item \textbf{Priority:} High. This is a foundational governance document.
    \item \textbf{Details:} The policy should clearly define rules for computer, network, email, and internet usage. It must be distributed to all current employees for review and acknowledgment, and integrated into the new hire onboarding process.
\end{itemize}

\subsection{Remediation for New Hire Training Gap (High)}
\begin{itemize}
    \item \textbf{Action:} Integrate a mandatory security awareness training module into the new employee onboarding process.
    \item \textbf{Priority:} High. This closes a critical window of vulnerability.
    \item \textbf{Details:} The training should cover essential topics such as phishing identification, password security, data handling, and the newly created AUP. This will ensure a baseline level of security awareness from day one.
\end{itemize}

\end{document}
```