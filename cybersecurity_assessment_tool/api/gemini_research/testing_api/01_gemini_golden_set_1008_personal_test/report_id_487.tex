```latex
\documentclass[12pt, a4paper]{article}

% Preamble: Required Packages
\usepackage[margin=1in]{geometry}
\usepackage{pifont} % For checkmarks and crosses
\usepackage{booktabs} % For professional tables
\usepackage{hyperref} % For clickable links
\usepackage{url} % For URL formatting
\usepackage{seqsplit} % To split long text strings like IPs
\usepackage{graphicx}
\usepackage{xcolor}

% Document Information
\title{Cybersecurity Assessment Report}
\author{Cybersecurity Analysis Division}
\date{\today}

% Hyperref Setup
\hypersetup{
    colorlinks=true,
    linkcolor=blue,
    filecolor=magenta,      
    urlcolor=cyan,
    pdftitle={Cybersecurity Assessment Report},
    pdfpagemode=FullScreen,
}

\begin{document}

\maketitle
\thispagestyle{empty}
\newpage

\tableofcontents
\newpage

% --- 1. Executive Summary ---
\section{Executive Summary}

This report provides a comprehensive cybersecurity assessment for \textbf{Obsidian Operatives}, based on network scan data, organizational security questionnaires, and a review of pre-existing risks. The analysis was conducted on \today.

The assessment reveals a mixed security posture. While the organization has implemented foundational controls such as Multi-Factor Authentication (MFA) for computer and sensitive system access, critical gaps exist that expose the organization to significant risk.

Key findings include:
\begin{itemize}
    \item \textbf{Critical Risk: Lack of MFA on Email.} The absence of MFA on the primary email system (\texttt{ObsidianOperatives.com}) is a critical vulnerability, leaving the organization highly susceptible to Business Email Compromise (BEC), phishing attacks, and account takeovers.
    \item \textbf{High Risk: Exposed Administrative Service.} An external network scan identified an open Secure Shell (SSH) port on the IPv6 address \seqsplit{\texttt{2001:db8::1}}. If not properly configured, this service can provide a direct path for attackers to gain administrative access to internal systems.
    \item \textbf{High Risk: Incomplete Security Training Program.} While new employees receive security training, the lack of a mandatory annual refresher course for all staff diminishes long-term security awareness and resilience against evolving social engineering tactics.
\end{itemize}

Immediate remediation of the email MFA gap is strongly recommended. Further actions should focus on securing the exposed SSH service and institutionalizing a comprehensive, recurring security awareness training program.

% --- 2. Organizational Information ---
\section{Organizational Information}

The following information was provided for the assessment.

\begin{tabular}{@{}ll}
\toprule
\textbf{Attribute} & \textbf{Value} \\
\midrule
Organization Name & \textbf{Obsidian Operatives} \\
Email Domain & \seqsplit{\texttt{ObsidianOperatives.com}} \\
Website Domain & \seqsplit{\url{www.ObsidianOperatives.com}} \\
Primary External IP & \seqsplit{\texttt{231.121.52.130}} \\
\bottomrule
\end{tabular}

% --- 3. Security Control Review ---
\section{Security Control Review}

The following table summarizes the organization's responses to a security controls questionnaire. An analysis of identified gaps is provided below.

\begin{table}[h!]
\centering
\caption{Security Controls Questionnaire Results}
\begin{tabular}{@{}p{0.8\linewidth}c@{}}
\toprule
\textbf{Control Question} & \textbf{Status} \\
\midrule
Do you require MFA to access email? & \ding{55} \\
Do you require MFA to log into computers? & \ding{51} \\
Do you require MFA to access sensitive data systems? & \ding{51} \\
Does your organization have an employee acceptable use policy? & \ding{51} \\
Does your organization do security awareness training for new employees? & \ding{51} \\
Does your organization do security awareness training for all employees at least once per year? & \ding{55} \\
\bottomrule
\end{tabular}
\end{table}

\subsection{Analysis of Control Gaps}
Two significant control gaps were identified from the questionnaire:

\begin{itemize}
    \item \textbf{No MFA for Email:} This is the most critical finding. Email is the number one vector for cyberattacks. Without MFA, a single compromised password can lead to a full account takeover, data breaches, financial fraud, and further infiltration of the network.
    \item \textbf{No Annual Security Awareness Training:} The threat landscape is constantly changing. A one-time training for new hires is insufficient. A lack of ongoing training leaves existing employees vulnerable to new phishing techniques and social engineering tactics, weakening the human element of the security program.
\end{itemize}

% --- 4. Technical Scan Results ---
\section{Technical Scan Results}

An external network scan was performed to identify exposed services.

\begin{itemize}
    \item \textbf{Target IP Address:} \seqsplit{\texttt{2001:db8::1}}
    \item \textbf{Scan Tool:} Nmap
\end{itemize}

\subsection{Open Ports}
The scan identified the following open port on the target system.

\begin{table}[h!]
\centering
\caption{Discovered Open Ports}
\begin{tabular}{@{}llll@{}}
\toprule
\textbf{Port} & \textbf{State} & \textbf{Service} & \textbf{Notes} \\
\midrule
22/tcp & open & SSH & Secure Shell - Administrative access. \\
\bottomrule
\end{tabular}
\end{table}

\subsection{Technical Analysis}
The discovery of an open SSH port (22) is a significant finding. SSH is a powerful protocol used for remote administration of servers and network devices. While essential for management, its exposure to the public internet creates a substantial attack surface. 

The scan data did not include service version information, which prevents an immediate assessment for known vulnerabilities (e.g., outdated OpenSSH versions). An attacker could exploit this service through brute-force password attacks, credential stuffing, or by leveraging a known software vulnerability.

% --- 5. Consolidated Risk Assessment ---
\section{Consolidated Risk Assessment}

The following table synthesizes findings from the security control review, technical scan, and pre-existing risk register.

\begin{table}[h!]
\centering
\caption{Summary of Identified Risks}
\begin{tabular}{@{}p{0.1\linewidth} p{0.3\linewidth} p{0.15\linewidth} p{0.35\linewidth}@{}}
\toprule
\textbf{ID} & \textbf{Risk Name} & \textbf{Severity} & \textbf{Description} \\
\midrule
RISK-001 & No MFA on Email Accounts & \textcolor{red}{\textbf{Critical}} & Lack of MFA allows for account takeover with only a compromised password, enabling BEC and data theft. \\
\addlinespace
RISK-002 & Exposed SSH Service & \textcolor{orange}{\textbf{High}} & An open administrative port provides a direct vector for attackers to attempt unauthorized system access. \\
\addlinespace
RISK-003 & Inadequate Annual Security Training & \textcolor{orange}{\textbf{High}} & The absence of a recurring training program increases susceptibility to phishing and social engineering attacks. \\
\bottomrule
\end{tabular}
\end{table}

% --- 6. Recommendations ---
\section{Recommendations}

The following actions are recommended to mitigate the identified risks, prioritized by severity.

\subsection{Priority 1: Critical}
\begin{description}
    \item[RISK-001: Remediate Email MFA Gap]
    \begin{itemize}
        \item \textbf{Action:} Immediately enable and enforce MFA for all user accounts on the \texttt{ObsidianOperatives.com} email platform.
        \item \textbf{Justification:} This is the single most effective control to prevent unauthorized access to email and mitigate the risk of Business Email Compromise.
    \end{itemize}
\end{description}

\subsection{Priority 2: High}
\begin{description}
    \item[RISK-002: Secure Exposed SSH Service]
    \begin{itemize}
        \item \textbf{Action:} Review the business need for external SSH access to \seqsplit{\texttt{2001:db8::1}}. If not required, block the port at the firewall.
        \item \textbf{Action:} If required, implement layered security controls:
        \begin{enumerate}
            \item Restrict access to specific, trusted source IP addresses (IP whitelisting).
            \item Disable password-based authentication and enforce the use of strong cryptographic keys (e.g., ED25519).
            \item Implement an intrusion prevention tool like Fail2Ban to block brute-force attempts.
        \end{enumerate}
        \item \textbf{Action:} Conduct a detailed vulnerability scan to identify the SSH server version and patch any known vulnerabilities.
    \end{itemize}
    \item[RISK-003: Implement Annual Security Training]
    \begin{itemize}
        \item \textbf{Action:} Procure and deploy a security awareness training platform.
        \item \textbf{Action:} Develop a formal policy requiring all employees and contractors to complete security awareness training on an annual basis. Track completion to ensure compliance.
    \end{itemize}
\end{description}

\end{document}
```