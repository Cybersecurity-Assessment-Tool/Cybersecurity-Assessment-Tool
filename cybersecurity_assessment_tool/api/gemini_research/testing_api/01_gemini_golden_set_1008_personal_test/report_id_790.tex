```latex
\documentclass[12pt]{article}

% Preamble: Required Packages
\usepackage[margin=1in]{geometry} % For setting page margins
\usepackage{pifont}               % For checkmarks and crosses (dingbats)
\usepackage{booktabs}             % For professional-looking tables
\usepackage{hyperref}             % For hyperlinks, metadata
\usepackage{url}                  % For formatting URLs
\usepackage{seqsplit}             % For splitting long strings in texttt
\usepackage{graphicx}             % For including logos (optional)
\usepackage{xcolor}               % For custom colors

% Document Metadata
\hypersetup{
    colorlinks=true,
    linkcolor=blue,
    filecolor=magenta,      
    urlcolor=cyan,
    pdftitle={Cybersecurity Posture Assessment Report},
    pdfauthor={Cybersecurity Analyst},
    pdfsubject={Security Assessment},
    pdfkeywords={Cybersecurity, Risk, Assessment, Scan},
    bookmarks=true
}

% Define checkmark and cross symbols for clarity
\newcommand{\cmark}{\ding{51}} % Checkmark
\newcommand{\xmark}{\ding{55}} % Cross

\begin{document}

% --- Title Page ---
\begin{titlepage}
    \centering
    \vspace*{2cm}
    
    {\Huge \textbf{Cybersecurity Posture Assessment Report}\par}
    \vspace{1.5cm}
    
    {\Large Prepared for:\par}
    \vspace{0.5cm}
    {\huge \textbf{Hearth \& Home}\par}
    
    \vspace{2cm}
    
    {\large \today\par}
    
    \vfill
    
    {\large \textbf{Generated by:}\par}
    {\large Cybersecurity Analyst\par}
    
    \vspace{1cm}
    \textit{This report is confidential and intended solely for the use of Hearth \& Home.}
\end{titlepage}

\tableofcontents
\newpage

% --- Executive Summary ---
\section*{Executive Summary}
This report provides a comprehensive analysis of the cybersecurity posture for \textbf{Hearth \& Home}, based on a review of organizational security controls, an external network scan, and pre-existing risk data.

The assessment reveals a mixed security posture. The organization has implemented several positive security controls, including mandatory Multi-Factor Authentication (MFA) for computer and sensitive system access, as well as a robust security awareness training program for all employees. These measures significantly strengthen the defense against common cyber threats.

However, two critical gaps were identified that require immediate attention. The absence of mandatory MFA for email exposes the organization to a high risk of account compromise, phishing, and subsequent data breaches. Additionally, the lack of a formal employee Acceptable Use Policy (AUP) creates ambiguity regarding security responsibilities and increases legal and operational risks.

The external network scan of the target IP address \texttt{[Target IP]} did not identify any open ports. While this suggests a strong firewall configuration, it does not rule out vulnerabilities that could be exploited through other vectors.

Our primary recommendations are to enforce MFA for all email accounts immediately and to develop and implement a comprehensive AUP for all employees. Addressing these findings will substantially improve the overall security resilience of Hearth \& Home.

% --- Organizational Information ---
\section*{Organizational Information}
The following details were provided for the assessment. This information is used to establish the context and scope of the review.

\begin{tabular}{@{}ll}
\toprule
\textbf{Attribute} & \textbf{Value} \\
\midrule
Organization Name & \textbf{Hearth \& Home} \\
Email Domain & \texttt{HearthHome.org} \\
Website Domain & \texttt{www.HearthHome.org} \\
External IP Address & \texttt{89.61.134.188} \\
\bottomrule
\end{tabular}

% --- Security Control Review ---
\section*{Security Control Review}
A review of organizational security controls was conducted based on a standardized questionnaire. The results highlight areas of both strength and weakness in the current security policy framework. "No" answers indicate significant gaps that elevate organizational risk.

\begin{table}[h!]
\centering
\caption{Organizational Security Control Status}
\begin{tabular}{@{}lc}
\toprule
\textbf{Control Question} & \textbf{Status} \\
\midrule
Do you require MFA to access email? & \xmark \\
Do you require MFA to log into computers? & \cmark \\
Do you require MFA to access sensitive data systems? & \cmark \\
Does your organization have an employee acceptable use policy? & \xmark \\
Does your organization do security awareness training for new employees? & \cmark \\
Does your organization do security awareness training for all employees at least once per year? & \cmark \\
\bottomrule
\end{tabular}
\end{table}

\subsection*{Analysis of Control Gaps}
\begin{itemize}
    \item \textbf{MFA for Email (Critical Gap):} The lack of MFA on email is a critical vulnerability. Email accounts are primary targets for attackers seeking to gain an initial foothold, conduct phishing campaigns, or perform business email compromise (BEC) attacks.
    \item \textbf{Acceptable Use Policy (High Risk Gap):} An AUP is a foundational policy that defines rules and responsibilities for employees when using company IT assets. Its absence can lead to inconsistent security practices, misuse of resources, and challenges in enforcing security standards.
\end{itemize}

% --- Technical Scan Results ---
\section*{Technical Scan Results}
An external, unauthenticated network scan was performed to identify exposed services and potential vulnerabilities visible from the public internet.

\begin{itemize}
    \item \textbf{Target IP Address:} \texttt{[Target IP]}
    \item \textbf{Scan Date:} Data Not Provided
\end{itemize}

\subsection*{Summary of Findings}
The network scan completed successfully but found \textbf{no open TCP or UDP ports} on the target system. 

\paragraph{Interpretation:}
This result typically indicates a well-configured perimeter firewall that is effectively blocking unsolicited inbound traffic. This is a positive security finding, as it significantly reduces the external attack surface. However, it is important to note that this scan does not assess for web application vulnerabilities, misconfigurations exploitable by authenticated users, or threats originating from inside the network.

% --- Consolidated Risk Assessment ---
\section*{Consolidated Risk Assessment}
This section synthesizes findings from the security control review, technical scan, and any pre-existing risk data to provide a unified view of the current risk landscape.

\begin{table}[h!]
\centering
\caption{Identified Risks and Severity}
\begin{tabular}{@{}p{0.3\linewidth}p{0.5\linewidth}p{0.15\linewidth}@{}}
\toprule
\textbf{Risk Name} & \textbf{Overview} & \textbf{Severity} \\
\midrule
\textbf{Email Account Compromise} & The absence of MFA on email accounts makes them highly susceptible to takeover via credential stuffing, phishing, or password spraying attacks. & \textbf{Critical} \\
\addlinespace
\textbf{Lack of Formal IT Policy} & Without a documented Acceptable Use Policy, the organization lacks a formal mechanism to enforce secure behavior, manage employee expectations, and mitigate insider risk. & \textbf{High} \\
\addlinespace
\textbf{Unidentified Pre-existing Risks} & No pre-existing vulnerabilities were provided for this assessment. It is crucial to maintain an active vulnerability management program. & Informational \\
\bottomrule
\end{tabular}
\end{table}

% --- Recommendations ---
\section*{Recommendations}
The following actionable recommendations are prioritized based on the severity of the identified risks. Implementing these measures will significantly enhance the security posture of Hearth \& Home.

\subsection*{Priority 1: Critical Risk Mitigation}
\begin{enumerate}
    \item \textbf{Enforce MFA for Email Immediately:}
    \begin{itemize}
        \item \textbf{Action:} Enable and enforce MFA for all user mailboxes without exception.
        \item \textbf{Justification:} This is the single most effective control to prevent email account takeovers, which are a primary vector for more severe cyberattacks. This action directly mitigates the highest-priority risk identified in this report.
    \end{itemize}
\end{enumerate}

\subsection*{Priority 2: High Risk Mitigation}
\begin{enumerate}
    \setcounter{enumi}{1}
    \item \textbf{Develop and Implement an Acceptable Use Policy (AUP):}
    \begin{itemize}
        \item \textbf{Action:} Create a formal AUP document that clearly outlines the rules for using company technology and data. This policy should be reviewed by legal counsel, communicated to all employees, and require a formal acknowledgment of receipt and understanding.
        \item \textbf{Justification:} An AUP establishes a baseline for secure employee behavior, supports compliance requirements, and provides a framework for disciplinary action in case of policy violations.
    \end{itemize}
\end{enumerate}

\subsection*{Priority 3: Proactive Security Enhancements}
\begin{enumerate}
    \setcounter{enumi}{2}
    \item \textbf{Implement Comprehensive Vulnerability Scanning:}
    \begin{itemize}
        \item \textbf{Action:} Augment external scans with regular, authenticated internal vulnerability scans and periodic web application security testing.
        \item \textbf{Justification:} While the external posture appears strong, a comprehensive scanning program is necessary to identify vulnerabilities that are not visible from the internet but could be exploited by an attacker who has gained internal access.
    \end{itemize}
\end{enumerate}

\end{document}
```