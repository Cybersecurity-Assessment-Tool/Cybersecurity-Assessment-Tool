```latex
\documentclass[12pt]{article}

% --- PACKAGES ---
\usepackage[margin=1in]{geometry}
\usepackage{pifont} % For checkmarks and crosses
\usepackage{booktabs} % For professional tables
\usepackage{hyperref} % For clickable links
\usepackage{url} % For URL formatting
\usepackage{seqsplit} % For splitting long text strings
\usepackage{graphicx}
\usepackage[utf8]{inputenc}

% --- DOCUMENT METADATA ---
\hypersetup{
    colorlinks=true,
    linkcolor=blue,
    filecolor=magenta,      
    urlcolor=cyan,
    pdftitle={Cybersecurity Assessment Report},
    pdfauthor={Cybersecurity Analyst},
    pdfsubject={Security Analysis},
    pdfkeywords={Security, Report, Analysis},
    bookmarks=true
}

\newcommand{\yes}{\ding{51}} % Checkmark
\newcommand{\no}{\ding{55}}  % Cross

% --- DOCUMENT START ---
\begin{document}

% --- TITLE PAGE ---
\title{Cybersecurity Assessment Report \\ \large For: \textbf{Astraeus Aerospace}}
\author{Cybersecurity Analyst}
\date{November 22, 2025}
\maketitle

\hrule
\vspace{1em}
\begin{abstract}
This report provides a comprehensive cybersecurity assessment for \textbf{Astraeus Aerospace}, based on an analysis of network scan data, organizational security controls, and known risks. The assessment identified several critical and high-risk vulnerabilities that require immediate attention. Key findings include the lack of multi-factor authentication (MFA) on sensitive data systems, the use of outdated web server software, and a gap in the security training program for new employees. This document details these findings and provides actionable recommendations to mitigate the identified risks and improve the organization's overall security posture.
\end{abstract}
\vspace{1em}
\hrule

\tableofcontents
\newpage

% --- SECTION 1: OVERVIEW & SCOPE ---
\section{Organizational Information}
This section provides a summary of the organizational data used as a basis for this assessment.

\begin{tabular}{@{}ll}
\toprule
\textbf{Attribute} & \textbf{Value} \\
\midrule
Organization Name & \textbf{Astraeus Aerospace} \\
Email Domain & \texttt{AstraeusAerospace.org} \\
Website Domain & \url{www.AstraeusAerospace.org} \\
External IP Address & \texttt{68.208.136.236} \\
\bottomrule
\end{tabular}

\section{Security Control Review}
The following table summarizes the organization's responses to a security controls questionnaire. This review highlights existing security practices and identifies procedural gaps. Items marked with \no\ represent significant weaknesses in the current security framework.

\begin{tabular}{@{}p{0.6\linewidth}cp{0.2\linewidth}@{}}
\toprule
\textbf{Control Question} & \textbf{Response} & \textbf{Assessment} \\
\midrule
Do you require MFA to access email? & \yes & Control in Place \\
Do you require MFA to log into computers? & \yes & Control in Place \\
Do you require MFA to access sensitive data systems? & \no & \textbf{Critical Gap} \\
Does your organization have an employee acceptable use policy? & \yes & Control in Place \\
Does your organization do security awareness training for new employees? & \no & \textbf{High-Risk Gap} \\
Does your organization do security awareness training for all employees at least once per year? & \yes & Control in Place \\
\bottomrule
\end{tabular}

\section{Technical Scan Results}
A network scan was performed on \textbf{November 22, 2025}, targeting the host at \texttt{192.168.10.5}. The scan identified the following open ports and services, which represent the external attack surface of the assessed system.

\subsection{Open Ports and Services}
\begin{tabular}{@{}llllll@{}}
\toprule
\textbf{Port} & \textbf{State} & \textbf{Service} & \textbf{Product} & \textbf{Version} & \textbf{Notes} \\
\midrule
443/tcp & open & https & nginx & 1.18.0 & 1. Outdated Version \\
        &      &       &       &          & 2. SSL Cert Mismatch \\
\bottomrule
\end{tabular}

\subsection{Technical Analysis}
\begin{itemize}
    \item \textbf{Outdated Nginx Version:} The web server is running Nginx version \texttt{1.18.0}, which was released in April 2020. This version is significantly outdated and is known to be vulnerable to multiple security exploits (e.g., CVE-2021-23017). Running unsupported software poses a high risk of compromise.
    \item \textbf{SSL Certificate Mismatch:} The SSL certificate presented by the server has a Common Name of \texttt{www.acme-corp.com}, which does not match the organization's domain (\texttt{www.AstraeusAerospace.org}). This misconfiguration can lead to browser trust errors and may be indicative of a deployment issue.
\end{itemize}

\section{Risk Assessment}
The following table synthesizes findings from the security control review and the technical scan into a prioritized list of identified risks. Each risk is assigned a severity level based on its potential impact and likelihood of exploitation.

\begin{tabular}{@{}p{0.1\linewidth}p{0.25\linewidth}p{0.4\linewidth}l@{}}
\toprule
\textbf{Risk ID} & \textbf{Risk Title} & \textbf{Description} & \textbf{Severity} \\
\midrule
RISK-001 & Lack of MFA on Sensitive Systems & The absence of MFA on systems containing sensitive data allows an attacker with stolen credentials to gain direct access to critical assets, potentially leading to a major data breach. & \textbf{Critical} \\
\addlinespace
RISK-002 & Outdated Web Server Software & The public-facing web server is running an outdated version of Nginx with known vulnerabilities. This could allow an attacker to compromise the server, leading to service disruption or data exfiltration. & \textbf{High} \\
\addlinespace
RISK-003 & No Security Training for New Hires & New employees are not receiving security awareness training, making them highly susceptible to phishing and social engineering attacks. This gap increases the risk of initial compromise. & \textbf{High} \\
\addlinespace
RISK-004 & SSL Certificate Mismatch & The web server presents a certificate for a different domain, which erodes user trust and could be exploited in man-in-the-middle scenarios. It also indicates a lack of configuration management. & \textbf{Medium} \\
\bottomrule
\end{tabular}

\newpage
\section{Recommendations}
Based on the risks identified in this assessment, the following actions are recommended to strengthen the security posture of \textbf{Astraeus Aerospace}.

\begin{itemize}
    \item[\textbf{1.}] \textbf{(Critical) Implement MFA on All Sensitive Systems (RISK-001):}
    \begin{itemize}
        \item Immediately prioritize the deployment of a robust multi-factor authentication solution across all applications and systems that store, process, or transmit sensitive data.
        \item This control is the single most effective measure to prevent unauthorized access resulting from credential compromise.
    \end{itemize}
    \vspace{1em}
    \item[\textbf{2.}] \textbf{(High) Upgrade Web Server and Institute Patch Management (RISK-002):}
    \begin{itemize}
        \item Plan and execute an immediate upgrade of the Nginx web server from version \texttt{1.18.0} to the latest stable version.
        \item Establish a formal patch management policy to ensure all public-facing systems are regularly updated to protect against newly discovered vulnerabilities.
    \end{itemize}
    \vspace{1em}
    \item[\textbf{3.}] \textbf{(High) Implement Mandatory Onboarding Security Training (RISK-003):}
    \begin{itemize}
        \item Develop a security awareness training module and integrate it into the mandatory onboarding process for all new employees and contractors.
        \item The training should cover, at a minimum, phishing identification, password hygiene, and the acceptable use policy.
    \end{itemize}
    \vspace{1em}
    \item[\textbf{4.}] \textbf{(Medium) Correct SSL Certificate Configuration (RISK-004):}
    \begin{itemize}
        \item Investigate the SSL certificate misconfiguration on the web server at \texttt{192.168.10.5}.
        \item Procure and deploy a valid SSL certificate that matches the organization's domain, \texttt{www.AstraeusAerospace.org}, to ensure encrypted connections are trusted by users and browsers.
    \end{itemize}
\end{itemize}

% --- DOCUMENT END ---
\end{document}
```