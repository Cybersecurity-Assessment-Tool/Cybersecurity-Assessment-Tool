```latex
\documentclass[12pt]{article}

% --- PACKAGES ---
\usepackage[margin=1in]{geometry}
\usepackage{pifont} % For checkmarks and crosses
\usepackage{booktabs} % For professional tables
\usepackage{hyperref} % For clickable links
\usepackage{url} % For URL formatting
\usepackage{seqsplit} % For splitting long strings in tt font
\usepackage{xcolor} % For colors
\usepackage{graphicx} % For potential logos
\usepackage{fancyhdr} % For headers/footers

% --- DOCUMENT METADATA & HYPERREF SETUP ---
\hypersetup{
    colorlinks=true,
    linkcolor=blue,
    filecolor=magenta,      
    urlcolor=cyan,
    pdftitle={Cybersecurity Posture Report},
    pdfpagemode=FullScreen,
}

% --- HEADER & FOOTER ---
\pagestyle{fancy}
\fancyhf{} % Clear all header and footer fields
\fancyhead[L]{Cybersecurity Posture Report}
\fancyhead[R]{\textbf{Sterling Silver}}
\fancyfoot[C]{\thepage}
\renewcommand{\headrulewidth}{0.4pt}
\renewcommand{\footrulewidth}{0.4pt}

% --- COMMAND DEFINITIONS ---
\newcommand{\yes}{\ding{51}}
\newcommand{\no}{\ding{55}}

\begin{document}

% --- TITLE PAGE ---
\begin{titlepage}
    \centering
    \vspace*{1cm}
    \includegraphics[width=0.4\textwidth]{example-image-a} % Placeholder for a logo
    
    \vspace{1.5cm}
    
    \Huge
    \textbf{Cybersecurity Posture Report}
    
    \vspace{1cm}
    
    \Large
    Prepared for: \\
    \textbf{Sterling Silver}
    
    \vspace{2cm}
    
    \normalsize
    Report Date: \today \\
    Analysis Period: October 2023
    
    \vfill
    
    \normalsize
    \textit{This report contains sensitive information and should be handled with care. Access is restricted to authorized personnel only.}
    
\end{titlepage}

\tableofcontents
\newpage

% --- EXECUTIVE SUMMARY ---
\section{Executive Summary}

This report provides a comprehensive analysis of the cybersecurity posture for \textbf{Sterling Silver}, based on a combination of technical network scanning, a review of existing risks, and an assessment of organizational security controls.

The analysis revealed several critical and high-risk vulnerabilities that require immediate attention. Key findings include:
\begin{itemize}
    \item \textbf{Critical FTP Vulnerability:} An internal server (\texttt{10.0.0.15}) is running a dangerously outdated version of vsftpd (2.3.4), which is known to contain a backdoor (CVE-2011-2523). Furthermore, it is configured to allow anonymous FTP logins, posing a severe and immediate threat to the internal network.
    \item \textbf{Insufficient Access Controls:} Multi-Factor Authentication (MFA) is not enforced for logging into company computers. This represents a significant gap in endpoint security, increasing the risk of unauthorized access from compromised credentials.
    \item \textbf{Inadequate Security Training:} While new employees receive security training, there is no mandatory, recurring annual training for all staff. This oversight leaves the organization more susceptible to evolving threats like phishing and social engineering.
    \item \textbf{Legacy Operating Systems:} The organization continues to use Windows 7 workstations, which are end-of-life and no longer receive security updates from Microsoft, exposing them to a wide range of known exploits.
\end{itemize}

Overall, the security posture is considered \textbf{High-Risk}. The combination of a directly exploitable service on the internal network and significant gaps in administrative controls creates a high likelihood of a security incident. We strongly recommend prioritizing the remediation actions outlined in Section 6 of this report.

% --- ORGANIZATIONAL INFORMATION ---
\section{Organizational Information}

The following details were provided for the assessment.

\begin{table}[h!]
\centering
\begin{tabular}{@{}ll@{}}
\toprule
\textbf{Attribute} & \textbf{Value} \\ \midrule
Organization Name & \textbf{Sterling Silver} \\
Email Domain & \texttt{SterlingSilver.net} \\
Website Domain & \url{www.SterlingSilver.net} \\
External IP Address & \texttt{150.117.65.28} \\ \bottomrule
\end{tabular}
\caption{Client Organizational Details}
\end{table}

% --- SECURITY CONTROL REVIEW ---
\section{Security Control Review}

An assessment of administrative and policy-based security controls was conducted via a questionnaire. The results highlight key areas of strength and weakness in the organization's security policies.

\begin{table}[h!]
\centering
\begin{tabular}{@{}p{0.7\textwidth}cc@{}}
\toprule
\textbf{Control Question} & \textbf{Response} & \textbf{Status} \\ \midrule
Do you require MFA to access email? & Yes & \textcolor{green}{\yes} \\
\textbf{Do you require MFA to log into computers?} & \textbf{No} & \textcolor{red}{\no} \\
Do you require MFA to access sensitive data systems? & Yes & \textcolor{green}{\yes} \\
Does your organization have an employee acceptable use policy? & Yes & \textcolor{green}{\yes} \\
Does your organization do security awareness training for new employees? & Yes & \textcolor{green}{\yes} \\
\textbf{Does your organization do security awareness training for all employees at least once per year?} & \textbf{No} & \textcolor{red}{\no} \\ \bottomrule
\end{tabular}
\caption{Security Controls Questionnaire Results}
\end{table}

\subsection*{Analysis of Control Gaps}
The responses marked with a \textcolor{red}{\no} are significant findings:
\begin{itemize}
    \item \textbf{Lack of Workstation MFA:} The absence of MFA on computer logins is a critical security gap. Should an employee's password be compromised, an attacker could gain direct access to their workstation and, potentially, the internal network.
    \item \textbf{Lack of Recurring Security Training:} The threat landscape evolves constantly. Without annual refresher training, employees are less prepared to identify and respond to modern phishing, ransomware, and social engineering attacks.
\end{itemize}

% --- TECHNICAL SCAN RESULTS ---
\section{Technical Scan Results}

A network scan was performed on the specified target to identify open ports and exposed services.

\subsection*{Target: \texttt{10.0.0.15}}
The scan revealed one host to be active with the following open port:

\begin{table}[h!]
\centering
\begin{tabular}{@{}lllll@{}}
\toprule
\textbf{Port} & \textbf{State} & \textbf{Service} & \textbf{Version} & \textbf{Notes} \\ \midrule
21/tcp & Open & ftp & vsftpd 2.3.4 & \textcolor{red}{\textbf{Critical: Anonymous FTP allowed}} \\ \bottomrule
\end{tabular}
\caption{Open Ports and Services on \texttt{10.0.0.15}}
\end{table}

\subsection*{Analysis of Technical Findings}
The finding on port 21 is of \textbf{critical severity}.
\begin{itemize}
    \item \textbf{Vulnerable Software:} The identified version, \textbf{vsftpd 2.3.4}, is notoriously vulnerable. A malicious backdoor was added to this specific version's source code, which allows for unauthenticated remote command execution (CVE-2011-2523). An attacker could easily gain a shell on this server.
    \item \textbf{Insecure Configuration:} The service is configured to allow \textbf{anonymous FTP login}. This allows any user on the network to access the FTP server without credentials, which could lead to data exfiltration or the introduction of malware onto the server.
\end{itemize}

% --- CONSOLIDATED RISK ASSESSMENT ---
\section{Consolidated Risk Assessment}

This section synthesizes findings from the security control review, technical scan, and pre-existing risk data into a unified list of identified risks.

\begin{table}[h!]
\centering
\begin{tabular}{@{}p{0.3\textwidth}p{0.5\textwidth}l@{}}
\toprule
\textbf{Risk Name} & \textbf{Description} & \textbf{Severity} \\ \midrule
\textbf{Vulnerable FTP Server} & A server is running vsftpd 2.3.4 with anonymous login enabled, allowing for remote code execution. & \textcolor{red}{\textbf{Critical}} \\
\addlinespace
\textbf{No MFA on Workstations} & Lack of MFA for computer logins exposes the organization to unauthorized access via compromised credentials. & \textcolor{orange}{\textbf{High}} \\
\addlinespace
\textbf{Outdated Windows Policy} & Workstations are running Windows 7, an unsupported OS that no longer receives security patches. & \textcolor{orange}{\textbf{High}} \\
\addlinespace
\textbf{No Annual Security Training} & The absence of recurring security training increases the risk of employees falling victim to social engineering. & \textcolor{yellow!80!black}{\textbf{Medium}} \\ \bottomrule
\end{tabular}
\caption{Summary of Identified Risks}
\end{table}

% --- RECOMMENDATIONS ---
\section{Recommendations}

Based on the consolidated risk assessment, the following actions are recommended to mitigate the identified vulnerabilities and improve the overall security posture.

\begin{enumerate}
    \item \textbf{Remediate Vulnerable FTP Server (Immediate Priority):}
    \begin{itemize}
        \item Immediately take the server at \texttt{10.0.0.15} offline.
        \item If the FTP service is required, upgrade vsftpd to the latest stable version.
        \item If the service is not required, disable and uninstall it completely.
        \item Under no circumstances should anonymous FTP login be enabled. Reconfigure to require strong, unique user credentials.
    \end{itemize}
    
    \item \textbf{Implement Workstation MFA (High Priority):}
    \begin{itemize}
        \item Procure and deploy a Multi-Factor Authentication solution for all employee workstation and laptop logins (Windows, macOS, etc.).
        \item Prioritize deployment for users with access to sensitive data, such as administrators, finance, and executive staff.
    \end{itemize}
    
    \item \textbf{Upgrade Legacy Operating Systems (High Priority):}
    \begin{itemize}
        \item Develop and execute a plan to upgrade all remaining Windows 7 workstations to a supported operating system, such as Windows 10 or 11.
        \item Isolate any systems that cannot be immediately upgraded from the main corporate network to limit their exposure.
    \end{itemize}

    \item \textbf{Establish Annual Security Awareness Training (Medium Priority):}
    \begin{itemize}
        \item Implement a mandatory security awareness training program for all employees to be completed on an annual basis.
        \item The training should cover modern threats, including phishing identification, password hygiene, and social engineering tactics.
    \end{itemize}
\end{enumerate}

\end{document}
```