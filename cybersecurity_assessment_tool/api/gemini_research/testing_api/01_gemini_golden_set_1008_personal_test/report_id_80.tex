```latex
\documentclass[12pt]{article}

% ----------------------------------------------------------------------
% PREAMBLE
% ----------------------------------------------------------------------
\usepackage[margin=1in]{geometry}
\usepackage{pifont} % For checkmarks and crosses
\usepackage{booktabs} % For professional tables
\usepackage{hyperref} % For clickable links
\usepackage{url} % For formatting URLs
\usepackage{seqsplit} % To split long strings in texttt
\usepackage[T1]{fontenc}

\hypersetup{
    colorlinks=true,
    linkcolor=black,
    filecolor=magenta,      
    urlcolor=blue,
    pdftitle={Cybersecurity Assessment Report},
    pdfpagemode=FullScreen,
}

% Custom commands for consistency
\newcommand{\yes}{\ding{51}}
\newcommand{\no}{\ding{55}}
\newcommand{\orgname}[1]{\textbf{#1}}
\newcommand{\techterm}[1]{\texttt{#1}}
\newcommand{\splittext}[1]{\seqsplit{\techterm{#1}}}

% ----------------------------------------------------------------------
% DOCUMENT START
% ----------------------------------------------------------------------
\begin{document}

\title{Cybersecurity Assessment Report \\ \large For: \orgname{Sovereign Trust}}
\author{Cybersecurity Analysis Division}
\date{\today}
\maketitle

\hrule
\vspace{1em}
\begin{abstract}
This report provides a comprehensive cybersecurity assessment for \orgname{Sovereign Trust}. The analysis is based on a synthesis of data from three sources: an external network scan, a security controls questionnaire, and a review of pre-existing risk data. The assessment identifies the organization's security posture, highlights key risks, and provides actionable recommendations to enhance security and mitigate threats. Key findings indicate strong identity and access management controls but reveal a critical gap in ongoing security awareness training and a technical vulnerability related to unencrypted web traffic.
\end{abstract}
\vspace{1em}
\hrule

\tableofcontents
\newpage

% ----------------------------------------------------------------------
% SECTION 1: OVERVIEW
% ----------------------------------------------------------------------
\section{Executive Summary}
The overall security posture of \orgname{Sovereign Trust} shows a commendable foundation in core access control principles, with consistent enforcement of Multi-Factor Authentication (MFA) across critical systems. However, significant areas for improvement were identified.

\begin{itemize}
    \item \textbf{Strengths:} The organization has successfully implemented MFA for email, computer logins, and access to sensitive data systems. An acceptable use policy is in place, and new employees receive security awareness training.
    
    \item \textbf{Critical Gaps:} A primary area of concern is the lack of mandatory, annual security awareness training for all employees. This oversight creates a significant vulnerability to social engineering and phishing attacks, as employee knowledge degrades over time without reinforcement.
    
    \item \textbf{Technical Findings:} The network scan identified an open port 80, which typically serves unencrypted HTTP traffic. This exposes any data transmitted over this service to interception and eavesdropping.
    
    \item \textbf{Pre-existing Risks:} The provided data on current risks was determined to be non-applicable for this assessment and was excluded from the analysis. A formal, internally-managed risk register is recommended for future tracking.
\end{itemize}

This report details these findings and concludes with prioritized, actionable recommendations to address the identified risks.

% ----------------------------------------------------------------------
% SECTION 2: ORGANIZATIONAL INFORMATION
% ----------------------------------------------------------------------
\section{Organizational Information}
The following details were provided for the assessment.

\begin{table}[h!]
\centering
\begin{tabular}{@{}ll@{}}
\toprule
\textbf{Attribute} & \textbf{Value} \\
\midrule
Organization Name & \orgname{Sovereign Trust} \\
Email Domain & \splittext{SovereignTrust.org} \\
Website Domain & \url{www.SovereignTrust.org} \\
External IP Address & \splittext{159.31.226.184} \\
\bottomrule
\end{tabular}
\caption{Client Organizational Details.}
\label{tab:org_info}
\end{table}

% ----------------------------------------------------------------------
% SECTION 3: SECURITY CONTROL REVIEW
% ----------------------------------------------------------------------
\section{Security Control Review (Questionnaire Analysis)}
An analysis of the security controls questionnaire was performed to evaluate administrative and policy-based safeguards. The results are summarized in Table \ref{tab:controls}.

\begin{table}[h!]
\centering
\begin{tabular}{@{}p{0.75\linewidth}c@{}}
\toprule
\textbf{Control Question} & \textbf{Response} \\
\midrule
Do you require MFA to access email? & \yes \\
Do you require MFA to log into computers? & \yes \\
Do you require MFA to access sensitive data systems? & \yes \\
Does your organization have an employee acceptable use policy? & \yes \\
Does your organization do security awareness training for new employees? & \yes \\
Does your organization do security awareness training for all employees at least once per year? & \no \\
\bottomrule
\end{tabular}
\caption{Security Controls Questionnaire Results.}
\label{tab:controls}
\end{table}

\paragraph{Analysis:} The consistent "Yes" responses for MFA demonstrate a strong commitment to identity and access management. However, the "No" response regarding annual security awareness training for all staff is a critical administrative gap. The threat landscape evolves continuously, and without regular training, employees become more susceptible to modern phishing, ransomware, and social engineering tactics.

% ----------------------------------------------------------------------
% SECTION 4: TECHNICAL SCAN RESULTS
% ----------------------------------------------------------------------
\section{Technical Scan Results}
A network scan was conducted to identify open ports and services exposed on the target system.

\begin{itemize}
    \item \textbf{Target IP Address:} \techterm{172.16.0.1}
    \item \textbf{Scan Date:} \today
\end{itemize}

The scan revealed the following open port, as detailed in Table \ref{tab:scan_results}.

\begin{table}[h!]
\centering
\begin{tabular}{@{}llll@{}}
\toprule
\textbf{Port} & \textbf{State} & \textbf{Service (Inferred)} & \textbf{Notes} \\
\midrule
80/tcp & Open & HTTP & Unencrypted web traffic. \\
\bottomrule
\end{tabular}
\caption{Open Ports Detected on Target System.}
\label{tab:scan_results}
\end{table}

\paragraph{Analysis:} The presence of an open port 80 indicates that a web server is likely running and configured to use the Hypertext Transfer Protocol (HTTP). HTTP is an unencrypted protocol, meaning that any data exchanged between a client and the server (including login credentials or sensitive information) can be intercepted and read by a malicious actor on the same network. This represents a medium-severity technical risk.

% ----------------------------------------------------------------------
% SECTION 5: RISK ASSESSMENT SUMMARY
% ----------------------------------------------------------------------
\section{Risk Assessment Summary}
The following table consolidates the key risks identified through the analysis of the questionnaire and technical scan data.

\begin{table}[h!]
\centering
\begin{tabular}{@{}llll@{}}
\toprule
\textbf{ID} & \textbf{Risk Description} & \textbf{Source} & \textbf{Severity} \\
\midrule
RISK-001 & Lack of Annual Security Awareness Training & Questionnaire & \textbf{High} \\
RISK-002 & Unencrypted Web Service (HTTP) & Network Scan & \textbf{Medium} \\
\bottomrule
\end{tabular}
\caption{Consolidated Risk Register.}
\label{tab:risk_summary}
\end{table}

% ----------------------------------------------------------------------
% SECTION 6: RECOMMENDATIONS
% ----------------------------------------------------------------------
\section{Recommendations}
The following actionable recommendations are provided to mitigate the identified risks and improve the overall security posture of \orgname{Sovereign Trust}.

\subsection{RISK-001: Implement Annual Security Training (High)}
\paragraph{Description:} The absence of a mandatory, recurring security awareness program for all employees leaves the organization vulnerable to human-centric attacks like phishing.
\paragraph{Recommendation:}
\begin{itemize}
    \item Institute a mandatory annual security awareness training program for all employees, contractors, and executives.
    \item The training curriculum should cover modern threats, including phishing identification, password security, social engineering tactics, and safe data handling practices.
    \item Conduct periodic, simulated phishing campaigns to test and reinforce the training concepts. Track metrics to measure improvement over time.
\end{itemize}

\subsection{RISK-002: Remediate Unencrypted Web Service (Medium)}
\paragraph{Description:} The service running on port 80 uses the unencrypted HTTP protocol, exposing transmitted data to potential interception.
\paragraph{Recommendation:}
\begin{itemize}
    \item \textbf{Primary Fix:} Migrate the web service from HTTP to HTTPS (Port 443). This involves acquiring and installing a valid TLS/SSL certificate from a trusted Certificate Authority.
    \item \textbf{Secondary Fix:} Configure the web server to automatically redirect all incoming HTTP traffic on port 80 to its secure HTTPS equivalent on port 443.
    \item If the service on port 80 is not essential, it should be disabled and the port should be blocked at the firewall level to reduce the attack surface.
\end{itemize}

\end{document}
% ----------------------------------------------------------------------
% DOCUMENT END
% ----------------------------------------------------------------------
```