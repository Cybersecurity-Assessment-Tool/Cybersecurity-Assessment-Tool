```latex
\documentclass[12pt]{article}

% ----------------------------------------------------------------------
% PREAMBLE
% ----------------------------------------------------------------------
\usepackage[margin=1in]{geometry}
\usepackage{pifont} % For checkmarks and crosses
\usepackage{booktabs} % For professional tables
\usepackage[hidelinks]{hyperref} % For clickable links
\usepackage{url} % For URL formatting
\usepackage{seqsplit} % To split long strings in tt font
\usepackage{graphicx}
\usepackage{fancyhdr}
\usepackage{lastpage}

% --- Header & Footer ---
\pagestyle{fancy}
\fancyhf{} % Clear all header and footer fields
\fancyhead[L]{Cybersecurity Posture Report}
\fancyhead[R]{Structure \& Form}
\fancyfoot[C]{\thepage\ of \pageref{LastPage}}
\renewcommand{\headrulewidth}{0.4pt}
\renewcommand{\footrulewidth}{0.4pt}

% --- Custom Commands ---
\newcommand{\yes}{\ding{51}}
\newcommand{\no}{\ding{55}}

% --- Document Metadata ---
\title{Cybersecurity Posture Report \\ \large For Structure \& Form}
\author{Cybersecurity Analysis Division}
\date{November 22, 2025}

% ----------------------------------------------------------------------
% DOCUMENT START
% ----------------------------------------------------------------------
\begin{document}

\maketitle
\thispagestyle{empty}
\newpage

\tableofcontents
\newpage

% ----------------------------------------------------------------------
% SECTION 1: EXECUTIVE SUMMARY
% ----------------------------------------------------------------------
\section{Executive Summary}
This report provides a comprehensive analysis of the cybersecurity posture for Structure \& Form, based on data collected on November 22, 2025. The assessment combines a review of organizational security controls, an external network scan, and an evaluation of known risks.

The analysis has identified several areas of significant concern that elevate the organization's risk profile. Key findings include:
\begin{itemize}
    \item \textbf{Critical Control Gap:} Multi-Factor Authentication (MFA) is not enforced for email access. This represents a critical vulnerability, as email accounts are a primary target for account takeover and subsequent phishing attacks.
    \item \textbf{High-Risk Technical Finding:} The external-facing web server is running an outdated version of Nginx (1.18.0), which is known to have multiple security vulnerabilities. This exposes the organization to potential compromise.
    \item \textbf{High-Risk Procedural Gap:} The organization lacks a formal security awareness training program for both new and existing employees. This increases susceptibility to social engineering attacks like phishing.
\end{itemize}

Immediate remediation of these identified risks is strongly recommended to reduce the likelihood of a security incident. Detailed findings and actionable recommendations are provided in the subsequent sections of this report.

% ----------------------------------------------------------------------
% SECTION 2: ORGANIZATIONAL INFORMATION
% ----------------------------------------------------------------------
\section{Organizational Information}
The following details were provided for the assessment.

\begin{tabular}{@{}ll}
    \toprule
    \textbf{Attribute} & \textbf{Value} \\
    \midrule
    Organization Name & Structure \& Form \\
    Email Domain & \texttt{StructureForm.com} \\
    Website Domain & \url{www.StructureForm.com} \\
    External IP Address & \seqsplit{\texttt{85.4.169.59}} \\
    \bottomrule
\end{tabular}

% ----------------------------------------------------------------------
% SECTION 3: SECURITY CONTROL REVIEW
% ----------------------------------------------------------------------
\section{Security Control Review}
A review of self-reported security controls was conducted via a questionnaire. The responses highlight critical gaps in the organization's security policies and procedures. A "No" response indicates a missing control and a potential area of high risk.

\begin{table}[h!]
\centering
\caption{Organizational Security Controls Questionnaire}
\begin{tabular}{@{}p{0.75\linewidth}c@{}}
    \toprule
    \textbf{Control Question} & \textbf{Response} \\
    \midrule
    Do you require MFA to access email? & \no \\
    Do you require MFA to log into computers? & \yes \\
    Do you require MFA to access sensitive data systems? & \yes \\
    Does your organization have an employee acceptable use policy? & \yes \\
    Does your organization do security awareness training for new employees? & \no \\
    Does your organization do security awareness training for all employees at least once per year? & \no \\
    \bottomrule
\end{tabular}
\end{table}

\subsection*{Analysis of Control Gaps}
\begin{itemize}
    \item \textbf{MFA for Email (Critical Gap):} The absence of MFA on email is a severe weakness. Email is a central communication hub and often the key to password resets for other services. A compromised email account can lead to a widespread breach.
    \item \textbf{Security Awareness Training (High Risk):} The lack of a training program for new and existing employees leaves the organization highly vulnerable to phishing, social engineering, and other human-targeted attacks.
\end{itemize}

% ----------------------------------------------------------------------
% SECTION 4: TECHNICAL SCAN RESULTS
% ----------------------------------------------------------------------
\section{Technical Scan Results}
An external network scan was performed to identify open ports and exposed services on the organization's public-facing infrastructure.

\begin{itemize}
    \item \textbf{Target IP:} \seqsplit{\texttt{192.168.10.5}}
    \item \textbf{Scan Date:} 2025-11-22
\end{itemize}

\begin{table}[h!]
\centering
\caption{Open Ports and Services}
\begin{tabular}{@{}lllll@{}}
    \toprule
    \textbf{Port} & \textbf{State} & \textbf{Service} & \textbf{Product} & \textbf{Version} \\
    \midrule
    443/tcp & open & https & nginx & 1.18.0 \\
    \bottomrule
\end{tabular}
\end{table}

\subsection*{Analysis of Technical Findings}
The scan identified an Nginx web server, version \textbf{1.18.0}, exposed to the internet. This version was released in April 2020 and is now significantly outdated. The current stable version of Nginx is much newer. Running outdated software poses a serious security risk, as numerous vulnerabilities have been discovered and patched in subsequent releases. This server is likely vulnerable to known exploits and should be updated as a matter of priority.

% ----------------------------------------------------------------------
% SECTION 5: CONSOLIDATED RISK ASSESSMENT
% ----------------------------------------------------------------------
\section{Consolidated Risk Assessment}
The following table synthesizes findings from the security control review and the technical scan. No pre-existing vulnerabilities were reported.

\begin{table}[h!]
\centering
\caption{Summary of Identified Risks}
\begin{tabular}{@{}p{0.1\linewidth}p{0.3\linewidth}p{0.15\linewidth}p{0.35\linewidth}@{}}
    \toprule
    \textbf{Risk ID} & \textbf{Risk Name} & \textbf{Severity} & \textbf{Description} \\
    \midrule
    RISK-001 & Lack of MFA on Email & \textbf{Critical} & The absence of MFA on email accounts makes them highly susceptible to unauthorized access through credential theft or phishing. \\
    \addlinespace
    RISK-002 & Outdated Web Server Software & \textbf{High} & The public-facing Nginx server (v1.18.0) is outdated and likely contains unpatched vulnerabilities, exposing it to remote compromise. \\
    \addlinespace
    RISK-003 & Inadequate Security Awareness Training & \textbf{High} & Without formal training, employees are more likely to fall victim to phishing and other social engineering tactics, undermining other security controls. \\
    \bottomrule
\end{tabular}
\end{table}

% ----------------------------------------------------------------------
% SECTION 6: RECOMMENDATIONS
% ----------------------------------------------------------------------
\section{Recommendations}
Based on the consolidated risk assessment, the following actions are recommended to improve the security posture of Structure \& Form.

\subsection*{RISK-001: Lack of MFA on Email (Critical)}
\begin{itemize}
    \item \textbf{Immediate Action:} Procure and implement an MFA solution for the organization's email system.
    \item \textbf{Policy:} Update IT policy to mandate the use of MFA for all email accounts, both for existing employees and during new employee onboarding.
    \item \textbf{Timeline:} This should be treated as the highest priority and be fully implemented within 30 days.
\end{itemize}

\subsection*{RISK-002: Outdated Web Server Software (High)}
\begin{itemize}
    \item \textbf{Immediate Action:} Develop a patch management and upgrade plan for the Nginx server.
    \item \textbf{Execution:} Test the upgrade in a non-production environment before deploying to production. Upgrade the server to the latest stable version of Nginx to mitigate known vulnerabilities.
    \item \textbf{Ongoing:} Incorporate this server into a regular patch management cycle to ensure it remains up-to-date.
    \item \textbf{Timeline:} Plan and execute the upgrade within 60 days.
\end{itemize}

\subsection*{RISK-003: Inadequate Security Awareness Training (High)}
\begin{itemize}
    \item \textbf{Immediate Action:} Select and implement a security awareness training platform or program.
    \item \textbf{Policy:} Establish a mandatory training program that includes:
        \begin{itemize}
            \item Initial training for all new hires as part of their onboarding process.
            \item Annual refresher training for all employees.
            \item Regular simulated phishing campaigns to test and reinforce learning.
        \end{itemize}
    \item \textbf{Timeline:} A program should be selected and initial training rolled out within 90 days.
\end{itemize}

\end{document}
```