```latex
\documentclass[12pt]{article}

% Preamble: Required Packages
\usepackage[margin=1in]{geometry}
\usepackage{pifont}                 % For \ding symbols (checkmarks/crosses)
\usepackage{booktabs}               % For professional-looking tables
\usepackage{hyperref}               % For hyperlinks and metadata
\usepackage{url}                    % For formatting URLs
\usepackage{seqsplit}               % To split long strings in \texttt
\usepackage{graphicx}
\usepackage{fancyhdr}
\usepackage[utf8]{inputenc}

% Document Metadata
\hypersetup{
    colorlinks=true,
    linkcolor=blue,
    filecolor=magenta,      
    urlcolor=cyan,
    pdftitle={Cybersecurity Posture Assessment Report},
    pdfauthor={Cybersecurity Analysis Division},
    pdfsubject={Security Assessment},
    pdfkeywords={Cybersecurity, Risk, Assessment, RDP},
}

% Header and Footer Configuration
\pagestyle{fancy}
\fancyhf{}
\lhead{Ember Glow Hospitality}
\rhead{Confidential Security Report}
\cfoot{\thepage}

\begin{document}

% --- Title Page ---
\title{Cybersecurity Posture Assessment Report \\ \large For Ember Glow Hospitality}
\author{Cybersecurity Analysis Division}
\date{\today}
\maketitle
\thispagestyle{empty}
\newpage

% --- Table of Contents ---
\tableofcontents
\newpage

% --- Section 1: Executive Overview ---
\section{Executive Overview}

This report details the findings of a cybersecurity posture assessment for \textbf{Ember Glow Hospitality}. The analysis correlates data from network scans, a security controls questionnaire, and a list of pre-existing risks to provide a holistic view of the organization's security standing.

The assessment identified two primary areas of concern that elevate the organization's risk profile:

\begin{enumerate}
    \item \textbf{Systemic Remote Desktop Protocol (RDP) Exposure:} A network scan discovered an open RDP port (3389) on a new internal system (\texttt{10.10.10.51}). This finding is highly critical as it correlates with a pre-existing, known vulnerability of the same nature on a different host (\texttt{10.10.10.50}). This pattern suggests a systemic configuration weakness or a lack of policy enforcement, significantly increasing the risk of unauthorized access and ransomware attacks.
    
    \item \textbf{Security Awareness Training Gap:} The organization does not conduct mandatory annual security awareness training for all employees. This is a critical administrative gap, as a well-trained workforce is the first line of defense against phishing and social engineering attacks, which are common precursors to major security breaches.
\end{enumerate}

While the organization has implemented strong controls regarding Multi-Factor Authentication (MFA), the combination of exposed services and a lack of ongoing employee training creates a significant attack vector. Immediate remediation of the technical vulnerability and implementation of a recurring training program are strongly recommended.

% --- Section 2: Organizational Information ---
\section{Organizational Information}

The following details were provided for the assessment.

\begin{tabular}{@{}ll}
    \toprule
    \textbf{Attribute} & \textbf{Value} \\
    \midrule
    Organization Name & \textbf{Ember Glow Hospitality} \\
    External IP Address & \texttt{201.104.163.29} \\
    Email Domain & \texttt{EmberGlowHospitality.net} \\
    Website Domain & \seqsplit{\url{www.EmberGlowHospitality.net}} \\
    \bottomrule
\end{tabular}

% --- Section 3: Security Control Review ---
\section{Security Control Review}

A review of administrative and technical security controls was conducted via a questionnaire. The results indicate a strong foundation in identity and access management but reveal a critical gap in ongoing employee education.

\begin{table}[h!]
\centering
\caption{Security Controls Questionnaire Results}
\begin{tabular}{@{}p{0.7\linewidth}lc@{}}
    \toprule
    \textbf{Control Question} & \textbf{Response} & \textbf{Status} \\
    \midrule
    Do you require MFA to access email? & Yes & \ding{51} \\
    Do you require MFA to log into computers? & Yes & \ding{51} \\
    Do you require MFA to access sensitive data systems? & Yes & \ding{51} \\
    Does your organization have an employee acceptable use policy? & Yes & \ding{51} \\
    Does your organization do security awareness training for new employees? & Yes & \ding{51} \\
    \textbf{Does your organization do security awareness training for all employees at least once per year?} & \textbf{No} & \textbf{\ding{55}} \\
    \bottomrule
\end{tabular}
\end{table}

The lack of annual security awareness training (\textbf{\ding{55}}) is a high-risk finding. Cyber threats evolve rapidly, and without regular reinforcement, employees are more likely to fall victim to sophisticated phishing or social engineering schemes.

% --- Section 4: Technical Scan Results ---
\section{Technical Scan Results}

An Nmap scan was performed on the internal network to identify open ports and exposed services. The scan identified a critical service exposed on the target host.

\begin{table}[h!]
\centering
\caption{Open Port Findings for Target: \texttt{10.10.10.51}}
\begin{tabular}{@{}llll@{}}
    \toprule
    \textbf{IP Address} & \textbf{Port/Protocol} & \textbf{State} & \textbf{Service Name} \\
    \midrule
    \texttt{10.10.10.51} & 3389/tcp & open & \texttt{ms-wbt-server} (RDP) \\
    \bottomrule
\end{tabular}
\end{table}

\textbf{Analysis:} The \texttt{ms-wbt-server} service on port 3389 is the Microsoft Remote Desktop Protocol (RDP). RDP is a primary target for attackers who use brute-force attacks or stolen credentials to gain remote control of systems. Exposing RDP without mitigating controls like a VPN or a Remote Desktop Gateway is a critical security risk.

% --- Section 5: Correlated Risk Assessment ---
\section{Correlated Risk Assessment}

This section synthesizes findings from the security control review, technical scan, and pre-existing risk data to provide a consolidated view of the primary risks facing the organization.

\begin{table}[h!]
\centering
\caption{Summary of Identified Risks}
\begin{tabular}{@{}p{0.15\linewidth}p{0.65\linewidth}l@{}}
    \toprule
    \textbf{Risk Title} & \textbf{Description} & \textbf{Severity} \\
    \midrule
    \textbf{Systemic RDP Exposure} & The scan identified RDP open on \texttt{10.10.10.51}. This compounds a known risk of RDP exposure on \texttt{10.10.10.50}. The recurrence of this vulnerability indicates a systemic failure in configuration management or policy enforcement. This is a direct path for ransomware deployment. & \textbf{Critical} \\
    \addlinespace
    \textbf{Insufficient Security Training} & The organization does not provide annual security awareness training. This gap makes employees more susceptible to phishing and social engineering, which could lead to credential compromise. Stolen credentials could then be used to access the exposed RDP services. & \textbf{High} \\
    \bottomrule
\end{tabular}
\end{table}

% --- Section 6: Recommendations ---
\section{Recommendations}

Based on the correlated risk assessment, the following actions are recommended to mitigate the identified vulnerabilities and improve the overall security posture of \textbf{Ember Glow Hospitality}.

\subsection{Immediate Actions (Remediate within 72 hours)}
\begin{enumerate}
    \item \textbf{Isolate Exposed RDP Service:} Immediately close port 3389 on host \texttt{10.10.10.51} from all untrusted networks. If remote access is required, it must be placed behind a secure gateway.
    \item \textbf{Audit Network for RDP:} Conduct a comprehensive internal and external network scan to identify any other systems with exposed RDP services. Remediate all findings according to the principle of least privilege.
\end{enumerate}

\subsection{Strategic Actions (Implement within 90 days)}
\begin{enumerate}
    \item \textbf{Deploy Secure Remote Access Solution:} If remote access is a business requirement, implement a centralized, secure solution such as a Virtual Private Network (VPN) or a Remote Desktop Gateway. All RDP traffic must be routed through this encrypted and authenticated channel.
    \item \textbf{Implement Annual Security Training:} Establish a mandatory security awareness training program for all employees, to be completed annually. The training should cover topics such as phishing identification, password hygiene, and acceptable use policies.
    \item \textbf{Develop Hardening Standards:} Create and enforce system configuration hardening standards for all deployed servers and workstations. These standards should explicitly forbid the exposure of management services like RDP to untrusted networks.
\end{enumerate}

\end{document}
```