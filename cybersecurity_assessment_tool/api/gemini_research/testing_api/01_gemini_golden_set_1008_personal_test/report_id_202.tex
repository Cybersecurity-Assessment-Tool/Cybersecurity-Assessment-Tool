```latex
\documentclass[12pt]{article}

% --- PACKAGE IMPORTS ---
\usepackage[margin=1in]{geometry}
\usepackage{pifont} % For checkmarks and crosses
\usepackage{booktabs} % For professional tables
\usepackage{hyperref} % For clickable links
\usepackage{url} % For URL formatting
\usepackage{seqsplit} % For splitting long strings in texttt
\usepackage[T1]{fontenc}

% --- DOCUMENT METADATA ---
\title{Cybersecurity Posture Assessment Report}
\author{Cybersecurity Analyst}
\date{\today}

% --- HYPERREF SETUP ---
\hypersetup{
    colorlinks=true,
    linkcolor=black,
    urlcolor=blue,
    pdftitle={Cybersecurity Posture Assessment Report},
    pdfauthor={Cybersecurity Analyst},
}

% --- DOCUMENT START ---
\begin{document}

\maketitle
\thispagestyle{empty}
\newpage

\tableofcontents
\newpage

% --- EXECUTIVE SUMMARY ---
\section{Executive Summary}
This report provides a comprehensive cybersecurity posture assessment for \textbf{Quantum Reach}. The analysis is based on a correlation of organizational security control data, an external network scan, and a review of pre-existing risks.

The assessment reveals a mixed security posture. On the technical front, the organization demonstrates strong network perimeter security, as the external scan of the target IP address revealed no open ports. This suggests a well-configured firewall and a commendable "deny-by-default" security stance.

However, significant administrative and procedural gaps were identified through the security controls questionnaire. The absence of a formal Employee Acceptable Use Policy (AUP) and the lack of mandatory annual security awareness training for all staff members represent high-risk vulnerabilities. These policy gaps expose the organization to increased threats from insider risk and social engineering attacks, such as phishing.

Immediate action is recommended to develop and implement the identified policy controls to mitigate these risks and strengthen the organization's overall defense-in-depth strategy.

% --- ORGANIZATIONAL INFORMATION ---
\section{Organizational Information}
The following details were provided for the assessment.

\begin{tabular}{@{}ll}
    \toprule
    \textbf{Attribute} & \textbf{Value} \\
    \midrule
    Organization Name & \textbf{Quantum Reach} \\
    Email Domain & \texttt{QuantumReach.net} \\
    Website Domain & \url{www.QuantumReach.net} \\
    External IP Address & \texttt{132.53.96.72} \\
    \bottomrule
\end{tabular}

% --- SECURITY CONTROL REVIEW ---
\section{Security Control Review}
A review of the organization's security controls was conducted via a questionnaire. The results are summarized below. "No" answers indicate potential security gaps that require attention.

\begin{tabular}{@{}p{0.6\linewidth}cp{0.25\linewidth}@{}}
    \toprule
    \textbf{Control Question} & \textbf{Response} & \textbf{Analyst Note} \\
    \midrule
    Do you require MFA to access email? & \ding{51} & Strong control. \\
    Do you require MFA to log into computers? & \ding{51} & Strong control. \\
    Do you require MFA to access sensitive data systems? & \ding{51} & Strong control. \\
    \addlinespace
    Does your organization have an employee acceptable use policy? & \color{red}\ding{55} & \textbf{Critical Gap.} Lack of a formal policy creates ambiguity and increases insider risk. \\
    \addlinespace
    Does your organization do security awareness training for new employees? & \ding{51} & Good practice for onboarding. \\
    \addlinespace
    Does your organization do security awareness training for all employees at least once per year? & \color{red}\ding{55} & \textbf{High Risk.} Without recurring training, employees are more susceptible to evolving threats like phishing. \\
    \bottomrule
\end{tabular}

% --- TECHNICAL SCAN RESULTS ---
\section{Technical Scan Results}
An external network vulnerability scan was performed to identify open ports and exposed services.

\begin{itemize}
    \item \textbf{Target IP Address:} \texttt{[Target IP]}
    \item \textbf{Scan Date:} Not available in scan data.
    \item \textbf{Findings:} The scan completed successfully and found \textbf{no open ports}.
\end{itemize}

\subsection{Analysis}
The absence of open ports on the scanned external asset is a positive security finding. It indicates that a robust firewall is in place, properly configured to block unsolicited inbound traffic. This significantly reduces the external attack surface and is a foundational element of strong network perimeter defense.

% --- RISK ASSESSMENT ---
\section{Risk Assessment}
The following table summarizes the key risks identified during this assessment, combining findings from the security control review and technical scans. No pre-existing vulnerabilities were provided for this assessment.

\begin{tabular}{@{}lp{0.25\linewidth}p{0.45\linewidth}l@{}}
    \toprule
    \textbf{ID} & \textbf{Risk Name} & \textbf{Overview} & \textbf{Severity} \\
    \midrule
    R-01 & Lack of Employee Acceptable Use Policy (AUP) & Without a formal AUP, employees may be unaware of their responsibilities regarding data protection and appropriate use of company assets. This increases the risk of accidental data breaches and intentional misuse. & \textbf{High} \\
    \addlinespace
    R-02 & Inadequate Security Awareness Training & While new hires receive training, the lack of an annual refresher program for all staff leaves the organization vulnerable. Threat landscapes evolve, and employee knowledge must be kept current to defend against modern phishing and social engineering tactics. & \textbf{High} \\
    \bottomrule
\end{tabular}

% --- RECOMMENDATIONS ---
\section{Recommendations}
Based on the identified risks, the following prioritized actions are recommended to enhance the cybersecurity posture of \textbf{Quantum Reach}.

\begin{enumerate}
    \item \textbf{Develop and Implement an Employee Acceptable Use Policy (AUP):}
    \begin{itemize}
        \item \textbf{Action:} Draft a comprehensive AUP that clearly defines the rules for using company networks, devices, software, and data.
        \item \textbf{Details:} The policy should include guidelines on internet usage, data handling, password security, and the consequences of non-compliance.
        \item \textbf{Priority:} High. This foundational policy is critical for governance and risk management.
    \end{itemize}
    \vspace{1em}
    \item \textbf{Establish a Mandatory Annual Security Awareness Training Program:}
    \begin{itemize}
        \item \textbf{Action:} Procure or develop a security awareness training module and mandate its completion by all employees on an annual basis.
        \item \textbf{Details:} Training should cover current threats such as phishing, ransomware, business email compromise (BEC), and best practices for creating strong passwords and reporting security incidents.
        \item \textbf{Priority:} High. This is one of the most effective measures to reduce human-related security risks.
    \end{itemize}
\end{enumerate}

% --- CONCLUSION ---
\section{Conclusion}
\textbf{Quantum Reach} has established a strong technical perimeter defense, which is a critical component of its security strategy. However, this strength is undermined by significant gaps in administrative controls. The identified risks, particularly the lack of a formal AUP and annual security training, must be addressed promptly. By implementing the recommendations in this report, the organization can significantly mature its security program, reduce its vulnerability to common cyber threats, and foster a more security-conscious culture.

\end{document}
```