```latex
\documentclass[12pt]{article}

% Preamble: Required Packages
\usepackage[margin=1in]{geometry}
\usepackage{pifont} % For checkmarks and crosses
\usepackage{booktabs} % For professional tables
\usepackage{hyperref} % For clickable links
\usepackage{url} % For URL formatting
\usepackage{seqsplit} % To split long strings in tt font
\usepackage{graphicx}
\usepackage{xcolor}

% Document Metadata
\title{Cybersecurity Assessment Report}
\author{Cybersecurity Analyst}
\date{\today}

% Hyperref Setup
\hypersetup{
    colorlinks=true,
    linkcolor=blue,
    filecolor=magenta,      
    urlcolor=cyan,
    pdftitle={Cybersecurity Assessment Report},
    pdfpagemode=FullScreen,
}

% Custom Commands
\newcommand{\yes}{\ding{51}}
\newcommand{\no}{\ding{55}}

\begin{document}

\maketitle
\thispagestyle{empty}
\newpage
\tableofcontents
\newpage

% ==============================================================================
\section{Executive Summary}
% ==============================================================================
This report provides a comprehensive cybersecurity assessment for \textbf{Granite Shield}, based on network scans, an organizational security questionnaire, and a review of known risks. The analysis was conducted on \today.

The assessment identified several critical and high-severity risks that require immediate attention. Key findings include a publicly accessible FTP server running a dangerously outdated and vulnerable version of vsftpd (\texttt{2.3.4}), which is susceptible to a known remote code execution backdoor (CVE-2011-2523). This server also permits anonymous, unauthenticated access.

Furthermore, significant gaps were identified in organizational security controls. The absence of Multi-Factor Authentication (MFA) for email and sensitive data systems, combined with a lack of security awareness training, exposes the organization to a high risk of account compromise and social engineering attacks.

This report outlines these findings in detail and provides a prioritized list of actionable recommendations to mitigate the identified risks and strengthen the overall security posture of the organization.

% ==============================================================================
\section{Organizational Information}
% ==============================================================================
The following information was provided for the assessment.

\begin{itemize}
    \item \textbf{Organization Name:} Granite Shield
    \item \textbf{Email Domain:} \texttt{GraniteShield.net}
    \item \textbf{Website Domain:} \url{www.GraniteShield.net}
    \item \textbf{External IP Address:} \texttt{204.87.172.70}
\end{itemize}

% ==============================================================================
\section{Security Control Review}
% ==============================================================================
A review of the organization's security controls was conducted via a questionnaire. The responses highlight critical gaps in access control and employee security awareness. A "No" response indicates a deviation from security best practices and represents a significant risk.

\begin{table}[h!]
\centering
\caption{Security Controls Questionnaire Results}
\begin{tabular}{p{0.8\linewidth} c}
\toprule
\textbf{Control Question} & \textbf{Response} \\
\midrule
Do you require MFA to access email? & \no \\
Do you require MFA to log into computers? & \yes \\
Do you require MFA to access sensitive data systems? & \no \\
Does your organization have an employee acceptable use policy? & \yes \\
Does your organization do security awareness training for new employees? & \no \\
Does your organization do security awareness training for all employees at least once per year? & \no \\
\bottomrule
\end{tabular}
\end{table}

% ==============================================================================
\section{Technical Scan Results}
% ==============================================================================
An external network scan was performed to identify open ports and exposed services on the target system.

\begin{itemize}
    \item \textbf{Target IP Address:} \texttt{10.0.0.15}
\end{itemize}

The scan revealed one open port with a critically vulnerable service.

\begin{table}[h!]
\centering
\caption{Open Port Analysis}
\begin{tabular}{l l l l p{0.3\linewidth}}
\toprule
\textbf{Port} & \textbf{State} & \textbf{Service} & \textbf{Product / Version} & \textbf{Notes} \\
\midrule
21/tcp & Open & ftp & vsftpd 2.3.4 & \textbf{Critical Finding:} Anonymous FTP login is allowed. This version is vulnerable to a backdoor (CVE-2011-2523). \\
\bottomrule
\end{tabular}
\end{table}

% ==============================================================================
\section{Consolidated Risk Assessment}
% ==============================================================================
The following table synthesizes findings from the security control review, technical scan, and pre-existing risk data into a consolidated list of security risks.

\begin{table}[h!]
\centering
\caption{Summary of Identified Risks}
\begin{tabular}{p{0.25\linewidth} p{0.5\linewidth} l}
\toprule
\textbf{Risk Name} & \textbf{Description} & \textbf{Severity} \\
\midrule
\textbf{Vulnerable FTP Server} & The FTP service (vsftpd 2.3.4) has a known backdoor vulnerability (CVE-2011-2523) that allows for remote code execution by an unauthenticated attacker. & \textbf{Critical} \\
\addlinespace
\textbf{Anonymous FTP Access} & The FTP server is configured to allow anonymous logins, permitting any external entity to access, upload, or download files without authentication. & \textbf{Critical} \\
\addlinespace
\textbf{Lack of MFA for Critical Systems} & Email and sensitive data systems are not protected by Multi-Factor Authentication, making them highly susceptible to compromise via stolen credentials. & \textbf{High} \\
\addlinespace
\textbf{Inadequate Security Awareness Training} & The absence of a formal security training program for new or existing employees significantly increases the organization's vulnerability to phishing and social engineering attacks. & \textbf{High} \\
\addlinespace
\textbf{Outdated Windows Policy} & (Pre-existing risk) Workstations are running Windows 7, which is End-of-Life and no longer receives security updates from Microsoft, leaving them vulnerable to exploitation. & Medium \\
\bottomrule
\end{tabular}
\end{table}

% ==============================================================================
\section{Recommendations}
% ==============================================================================
Based on the risk assessment, the following prioritized actions are recommended to mitigate the identified vulnerabilities.

\subsection{Immediate Priority (Critical Risks)}
\begin{enumerate}
    \item \textbf{Remediate Vulnerable FTP Server:} The vsftpd service on \texttt{10.0.0.15} must be taken offline immediately.
    \begin{itemize}
        \item \textbf{Short-Term:} Disable the FTP service entirely.
        \item \textbf{Long-Term:} If file transfer is required, replace FTP with a secure, modern alternative such as SFTP (SSH File Transfer Protocol).
    \end{itemize}
    \item \textbf{Disable Anonymous FTP Access:} If the FTP service cannot be immediately disabled, the anonymous login capability must be removed from its configuration at once.
\end{enumerate}

\subsection{High Priority}
\begin{enumerate}
    \item \textbf{Implement Mandatory MFA:} Enforce Multi-Factor Authentication across all critical systems, prioritizing the following:
    \begin{itemize}
        \item All user email accounts (\texttt{GraniteShield.net}).
        \item All systems containing sensitive organizational or client data.
        \item All remote access solutions (e.g., VPN).
    \end{itemize}
    \item \textbf{Establish Security Awareness Training Program:} Develop and implement a mandatory security awareness training program.
    \begin{itemize}
        \item The program must be part of the onboarding process for all new employees.
        \item An annual refresher course must be completed by all staff.
    \end{itemize}
\end{enumerate}

\subsection{Medium Priority}
\begin{enumerate}
    \item \textbf{Upgrade End-of-Life Systems:} Develop and execute a plan to upgrade all workstations and servers from Windows 7 to a currently supported operating system, such as Windows 10 or 11.
\end{enumerate}

% ==============================================================================
\section{Conclusion}
% ==============================================================================
The assessment for \textbf{Granite Shield} has revealed critical vulnerabilities that pose an immediate and significant threat to the organization's data and operational integrity. The combination of a publicly exposed, backdoored FTP server and weak internal security controls creates a high-risk environment.

We strongly urge the organization to address the recommendations outlined in this report, starting with the immediate remediation of the FTP server. Proactive implementation of the remaining high-priority recommendations will build a more resilient and defensible security posture.

\end{document}
```