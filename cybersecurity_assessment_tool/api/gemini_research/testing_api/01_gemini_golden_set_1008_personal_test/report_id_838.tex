```latex
\documentclass[12pt]{article}

% Preamble: Required Packages
\usepackage[margin=1in]{geometry}
\usepackage{pifont} % For checkmarks and crosses
\usepackage{booktabs} % For professional tables
\usepackage{hyperref} % For clickable links
\usepackage{url} % For formatting URLs
\usepackage{seqsplit} % For splitting long strings
\usepackage{graphicx}
\usepackage{xcolor}
\usepackage{fancyhdr}

% Document Metadata
\title{Cybersecurity Posture Assessment Report}
\author{Cybersecurity Analysis Division}
\date{\today}

% Hyperref Setup
\hypersetup{
    colorlinks=true,
    linkcolor=blue,
    filecolor=magenta,      
    urlcolor=cyan,
    pdftitle={Cybersecurity Posture Assessment Report},
    pdfpagemode=FullScreen,
}

% Header and Footer
\pagestyle{fancy}
\fancyhf{}
\lhead{Cybersecurity Assessment Report}
\rhead{Mainframe Managed}
\cfoot{\thepage}

\begin{document}

\maketitle
\thispagestyle{empty}
\newpage

\tableofcontents
\newpage

% --- 1. Executive Summary ---
\section{Executive Summary}

This report provides a comprehensive cybersecurity posture assessment for \textbf{Mainframe Managed}, conducted on \today. The analysis synthesizes data from a network perimeter scan, a security controls questionnaire, and a review of pre-existing risks.

The assessment identified several critical and high-risk security gaps related to organizational policies and access controls. While the organization has implemented foundational security measures, such as MFA for email and an acceptable use policy, significant weaknesses exist. The most pressing concerns are the lack of Multi-Factor Authentication (MFA) for computer logins and access to sensitive data systems. Furthermore, the absence of mandatory annual security awareness training for all employees exposes the organization to a heightened risk of social engineering and phishing attacks.

The external network scan of the target IP address \texttt{[Target IP]} did not detect any open ports. This is a positive indicator of a strong firewall configuration but should be verified with internal and authenticated scans to ensure the host was online and the scan was not blocked.

Immediate remediation should focus on implementing a comprehensive MFA strategy and establishing a recurring security awareness training program to mitigate the most significant threats to the organization's data and systems.

% --- 2. Organizational Information ---
\section{Organizational Information}

The following information was provided for the assessment.

\begin{itemize}
    \item \textbf{Organization Name:} Mainframe Managed
    \item \textbf{Email Domain:} \texttt{MainframeManaged.com}
    \item \textbf{Website Domain:} \url{www.MainframeManaged.com}
    \item \textbf{External IP Address:} \texttt{21.95.32.149}
\end{itemize}

% --- 3. Security Control Review ---
\section{Security Control Review}

A security controls questionnaire was completed to evaluate the organization's policies and procedures. The responses are summarized below. A green checkmark (\textcolor{green}{\ding{51}}) indicates a positive control in place, while a red cross (\textcolor{red}{\ding{55}}) indicates a security gap.

\begin{table}[h!]
\centering
\caption{Security Controls Questionnaire Results}
\begin{tabular}{p{0.7\textwidth} c}
\toprule
\textbf{Control Question} & \textbf{Response} \\
\midrule
Do you require MFA to access email? & \textcolor{green}{\ding{51}} \\
Do you require MFA to log into computers? & \textcolor{red}{\ding{55}} \\
Do you require MFA to access sensitive data systems? & \textcolor{red}{\ding{55}} \\
Does your organization have an employee acceptable use policy? & \textcolor{green}{\ding{51}} \\
Does your organization do security awareness training for new employees? & \textcolor{green}{\ding{51}} \\
Does your organization do security awareness training for all employees at least once per year? & \textcolor{red}{\ding{55}} \\
\bottomrule
\end{tabular}
\end{table}

\subsection{Analysis of Controls}
The questionnaire reveals critical gaps in the organization's access control and security training programs. The absence of MFA on computers and sensitive systems dramatically increases the risk of unauthorized access should an employee's credentials be compromised. Additionally, the lack of annual security training for all staff means that employees may not be equipped to recognize and respond to evolving cyber threats.

% --- 4. Technical Scan Results ---
\section{Technical Scan Results}

An external network scan was performed to identify exposed services and potential vulnerabilities on the organization's public-facing infrastructure.

\begin{itemize}
    \item \textbf{Target IP Address:} \texttt{[Target IP]}
    \item \textbf{Scan Date:} \today
    \item \textbf{Scan Summary:} No open ports were detected.
\end{itemize}

\subsection{Interpretation}
The scan results indicate that the target host at \texttt{[Target IP]} did not have any discoverable TCP or UDP ports open to the internet at the time of the scan. This is a positive security finding, as it suggests a properly configured firewall is in place, limiting the external attack surface. 

However, this result could also be due to the host being offline or ICMP (ping) requests being blocked, preventing the scanner from confirming the host was active. Further authenticated internal testing is recommended to validate this finding.

% --- 5. Risk Assessment ---
\section{Risk Assessment}

This section correlates findings from the security control review, technical scan, and any pre-existing risks. The following new risks were identified during this assessment. No pre-existing vulnerabilities were provided for review.

\begin{table}[h!]
\centering
\caption{Identified Risks}
\begin{tabular}{p{0.15\textwidth} p{0.55\textwidth} l}
\toprule
\textbf{Risk ID} & \textbf{Risk Description} & \textbf{Severity} \\
\midrule
RISK-001 & \textbf{Lack of MFA on Sensitive Systems:} Access to sensitive data systems is protected only by username and password. A credential compromise could lead to a significant data breach. & \textbf{\textcolor{red}{Critical}} \\
\midrule
RISK-002 & \textbf{Lack of MFA on Endpoints:} Employee computers (desktops/laptops) do not require MFA for login. This exposes the internal network to unauthorized access if an employee's credentials are stolen. & \textbf{\textcolor{orange}{High}} \\
\midrule
RISK-003 & \textbf{Insufficient Security Training:} The absence of mandatory, annual security awareness training for all employees increases susceptibility to phishing, social engineering, and other human-targeted attacks. & \textbf{\textcolor{orange}{High}} \\
\bottomrule
\end{tabular}
\end{table}

% --- 6. Recommendations ---
\section{Recommendations}

Based on the identified risks, the following actions are recommended to improve the cybersecurity posture of \textbf{Mainframe Managed}.

\subsection{Critical Priority}
\begin{itemize}
    \item \textbf{Implement MFA for Sensitive Systems (RISK-001):}
    \begin{itemize}
        \item \textbf{Action:} Immediately deploy a robust Multi-Factor Authentication solution for all systems containing sensitive or critical data. This includes databases, financial applications, and administrative portals.
        \item \textbf{Justification:} This is the most effective single control to prevent unauthorized access to critical assets in the event of a credential compromise.
    \end{itemize}
\end{itemize}

\subsection{High Priority}
\begin{itemize}
    \item \textbf{Enforce MFA on All Endpoints (RISK-002):}
    \begin{itemize}
        \item \textbf{Action:} Roll out mandatory MFA for all employee computer logins. Solutions compatible with Windows, macOS, and Linux operating systems are widely available (e.g., Duo, Okta, Microsoft Authenticator).
        \item \textbf{Justification:} Securing endpoints with MFA prevents attackers from easily moving laterally within the network after obtaining a user's password.
    \end{itemize}
    \item \textbf{Establish Annual Security Awareness Training (RISK-003):}
    \begin{itemize}
        \item \textbf{Action:} Develop and implement a mandatory security awareness training program for all employees, to be completed annually. The program should include modules on phishing, password security, social engineering, and acceptable use.
        \item \textbf{Justification:} A well-trained workforce is a critical layer of defense. Regular training ensures that security remains a top-of-mind concern and that employees are prepared to face modern threats.
    \end{itemize}
\end{itemize}

\subsection{General Recommendations}
\begin{itemize}
    \item \textbf{Conduct Internal Vulnerability Scanning:}
    \begin{itemize}
        \item \textbf{Action:} Schedule regular, authenticated vulnerability scans of the internal network.
        \item \textbf{Justification:} An internal scan can validate the findings of the external scan and uncover misconfigurations or vulnerabilities that are not visible from the internet.
    \end{itemize}
\end{itemize}

% --- 7. Conclusion ---
\section{Conclusion}

The assessment of \textbf{Mainframe Managed} reveals a mixed cybersecurity posture. While the organization demonstrates a commitment to security through its hardened network perimeter and foundational policies, critical deficiencies in access control and employee training present a significant risk. The identified vulnerabilities, particularly the lack of comprehensive MFA, could be exploited by threat actors to gain unauthorized access to sensitive data and systems.

We strongly urge the organization to prioritize the implementation of the recommendations outlined in this report, starting with the deployment of MFA on sensitive systems and endpoints. By addressing these gaps, \textbf{Mainframe Managed} can substantially reduce its risk exposure and build a more resilient security foundation.

\end{document}
```