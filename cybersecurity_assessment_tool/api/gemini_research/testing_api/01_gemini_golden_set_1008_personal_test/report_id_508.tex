```latex
\documentclass[12pt]{article}

% Preamble: Required Packages
\usepackage[margin=1in]{geometry} % Set page margins
\usepackage{pifont}               % For check and cross marks (\ding{51}, \ding{55})
\usepackage{booktabs}             % For professional-looking tables
\usepackage{hyperref}             % For clickable links and better PDF navigation
\usepackage{url}                  % For formatting URLs
\usepackage{seqsplit}             % For splitting long strings without breaking
\usepackage{graphicx}             % For including logos (optional)
\usepackage{xcolor}               % For custom colors

% Document Information
\title{Cybersecurity Posture Assessment Report}
\author{Cybersecurity Analysis Division}
\date{\today}

% Define custom colors for severity
\definecolor{criticalred}{HTML}{D7263D}
\definecolor{highorange}{HTML}{F49D42}
\definecolor{mediumyellow}{HTML}{F4D442}
\definecolor{lowblue}{HTML}{4287F4}

% Hyperref setup
\hypersetup{
    colorlinks=true,
    linkcolor=blue,
    filecolor=magenta,      
    urlcolor=cyan,
    pdftitle={Cybersecurity Posture Assessment Report},
    pdfpagemode=FullScreen,
}

\begin{document}

\maketitle
\thispagestyle{empty}
\newpage

\tableofcontents
\newpage

% --- 1. Executive Overview ---
\section{Executive Overview}
This report provides a comprehensive analysis of the cybersecurity posture for \textbf{Modern Myth}. The assessment is based on a synthesis of network scan data, a review of organizational security controls, and an evaluation of known risks.

The overall security posture presents a significant contrast between technical and procedural controls. The external network scan of the target host revealed a strong perimeter defense, with no open ports detected. This indicates a well-hardened system from a network accessibility standpoint.

However, the organizational security control review identified \textbf{critical deficiencies} in Identity and Access Management (IAM). The complete absence of Multi-Factor Authentication (MFA) for email, computer logins, and sensitive data systems exposes the organization to a high risk of account compromise, data breaches, and business email compromise (BEC) attacks. Furthermore, the lack of annual security awareness training for all employees constitutes a high-risk gap, as it diminishes the organization's resilience against evolving social engineering threats.

Immediate remediation efforts should focus on implementing MFA across all critical platforms and establishing a recurring security training program.

% --- 2. Organizational Information ---
\section{Organizational Information}
The following details were provided for the assessment.

\begin{tabular}{@{}ll}
\toprule
\textbf{Attribute} & \textbf{Value} \\
\midrule
Organization Name & \textbf{Modern Myth} \\
Email Domain & \texttt{ModernMyth.net} \\
Website Domain & \url{www.ModernMyth.net} \\
External IP Address & \texttt{170.4.133.16} \\
\bottomrule
\end{tabular}

% --- 3. Security Control Review ---
\section{Security Control Review}
A review of foundational security controls was conducted via a questionnaire. The responses highlight key areas of strength and weakness in the organization's security policies and procedures. "No" answers indicate significant gaps that increase organizational risk.

\begin{tabular}{@{}p{0.65\linewidth}cc}
\toprule
\textbf{Control Question} & \textbf{Response} & \textbf{Status} \\
\midrule
Do you require MFA to access email? & No & \ding{55} \\
Do you require MFA to log into computers? & No & \ding{55} \\
Do you require MFA to access sensitive data systems? & No & \ding{55} \\
Does your organization have an employee acceptable use policy? & Yes & \ding{51} \\
Does your organization do security awareness training for new employees? & Yes & \ding{51} \\
Does your organization do security awareness training for all employees at least once per year? & No & \ding{55} \\
\bottomrule
\end{tabular}

% --- 4. Technical Scan Results ---
\section{Technical Scan Results}
A network scan was performed to identify accessible services and potential vulnerabilities on the organization's external infrastructure.

\begin{itemize}
    \item \textbf{Target IP Address:} \texttt{192.168.1.100}
    \item \textbf{Scan Summary:} The scan completed successfully and found the host to be online.
    \item \textbf{Key Findings:} No open ports were detected on the target system. All 1000 scanned TCP ports were reported as being in a \textbf{`closed`} state. This is a positive security finding, suggesting a properly configured firewall or a host with no listening network services exposed to the scanner.
\end{itemize}

% --- 5. Risk Assessment ---
\section{Risk Assessment}
The following risks have been identified based on the correlation of the security control review and technical findings. These risks should be prioritized for remediation.

\begin{tabular}{@{}p{0.1\linewidth}p{0.25\linewidth}p{0.15\linewidth}p{0.4\linewidth}@{}}
\toprule
\textbf{Risk ID} & \textbf{Risk Name} & \textbf{Severity} & \textbf{Overview} \\
\midrule
RISK-001 & Lack of Multi-Factor Authentication (MFA) & \textcolor{criticalred}{\textbf{Critical}} & The absence of MFA for email, endpoints, and sensitive systems creates a critical vulnerability. A single compromised password could lead to unauthorized access, data exfiltration, or financial fraud. \\
\addlinespace
RISK-002 & Insufficient Security Awareness Training & \textcolor{highorange}{\textbf{High}} & While new hires receive training, the lack of an annual refresher for all staff means the workforce's ability to recognize and report modern threats like phishing and social engineering will degrade over time. \\
\addlinespace
RISK-003 & No Pre-existing Risks Logged & \textcolor{mediumyellow}{\textbf{Medium}} & The list of current risks was empty. This may indicate a lack of a formal risk management or tracking process, which is a procedural gap in itself. \\
\bottomrule
\end{tabular}

% --- 6. Recommendations ---
\section{Recommendations}
The following actions are recommended to mitigate the identified risks and improve the overall security posture of the organization.

\subsection{Risk-001: Lack of MFA (Critical)}
\begin{itemize}
    \item \textbf{Immediate (0-30 days):} Prioritize and enforce MFA for all user accounts on the primary email platform (e.g., Office 365, Google Workspace). This is the single most effective control to prevent business email compromise.
    \item \textbf{Short-Term (30-90 days):} Extend MFA enforcement to all remote access solutions (VPN), privileged user accounts (administrators), and critical financial or data systems.
    \item \textbf{Long-Term (90+ days):} Develop and implement a corporate policy that mandates MFA as a baseline requirement for any new system or application that handles sensitive data or provides remote access.
\end{itemize}

\subsection{Risk-002: Insufficient Training (High)}
\begin{itemize}
    \item \textbf{Immediate (0-30 days):} Procure and schedule mandatory annual security awareness training for all current employees. The training should cover phishing, password security, and acceptable use.
    \item \textbf{Ongoing:} Establish a continuous security awareness program that includes periodic phishing simulation campaigns to test and reinforce employee knowledge. Track metrics to identify areas for improvement.
\end{itemize}

\subsection{Risk-003: No Risk Management Process (Medium)}
\begin{itemize}
    \item \textbf{Short-Term (30-90 days):} Implement a formal risk register or tracking system to document, assess, and manage identified security risks. This report's findings should be the first entries.
    \item \textbf{Long-Term (90+ days):} Develop a formal risk management process that includes regular risk assessments and a defined process for tracking vulnerabilities from identification through to remediation.
\end{itemize}

\end{document}
```