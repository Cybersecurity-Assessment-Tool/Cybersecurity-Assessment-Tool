```latex
\documentclass[12pt]{article}

% Preamble: Required Packages
\usepackage[margin=1in]{geometry}
\usepackage{pifont} % For checkmarks and crosses
\usepackage{booktabs} % For professional tables
\usepackage{hyperref} % For clickable links and references
\usepackage{url} % For formatting URLs
\usepackage{seqsplit} % For splitting long text strings like URLs/IPs
\usepackage{graphicx} % For potential logos
\usepackage{xcolor} % For colors

% Hyperref Setup
\hypersetup{
    colorlinks=true,
    linkcolor=blue,
    filecolor=magenta,      
    urlcolor=cyan,
    pdftitle={Cybersecurity Assessment Report},
    pdfpagemode=FullScreen,
}

% Document Metadata
\title{Cybersecurity Assessment Report \\ \large For: Green Sprout Organic}
\author{Cybersecurity Analysis Division}
\date{\today}

\begin{document}

\maketitle
\thispagestyle{empty}
\newpage

\tableofcontents
\newpage

% --- 1. Executive Summary ---
\section{Executive Summary}

This report provides a comprehensive cybersecurity assessment for \textbf{Green Sprout Organic}, based on an analysis of network scan data, organizational security controls, and pre-existing risk information. The assessment was conducted to identify key vulnerabilities, security gaps, and to provide actionable recommendations for improving the organization's overall security posture.

The analysis revealed several critical and high-risk findings. Key concerns include significant gaps in Identity and Access Management (IAM), particularly the absence of Multi-Factor Authentication (MFA) for computer and sensitive data system access. Furthermore, critical governance controls, such as an employee acceptable use policy and mandatory annual security training, are not in place.

A technical scan confirmed a pre-identified critical risk, ``Localhost Exposed,'' with an open SSH port (22) detected on the local loopback interface. While the immediate threat vector of this specific finding requires deeper investigation, it confirms an active misconfiguration.

The overall security posture is considered weak due to these fundamental control deficiencies. Immediate remediation of the identified critical risks is strongly advised to prevent potential unauthorized access, data breaches, and other security incidents.

% --- 2. Organizational Information ---
\section{Organizational Information}

The following details were provided for the assessment. This information forms the basis of the analysis and defines the scope of the review.

\begin{tabular}{@{}ll}
\toprule
\textbf{Attribute} & \textbf{Value} \\
\midrule
Organization Name & \textbf{Green Sprout Organic} \\
Email Domain & \seqsplit{\texttt{GreenSproutOrganic.org}} \\
Website Domain & \seqsplit{\url{www.GreenSproutOrganic.org}} \\
External IP Address & \seqsplit{\texttt{171.196.21.211}} \\
\bottomrule
\end{tabular}

% --- 3. Security Control Review ---
\section{Security Control Review}

A review of the organization's self-reported security controls was conducted via a questionnaire. The responses highlight critical gaps in foundational security practices. A summary of the findings is presented in Table \ref{tab:controls}. The symbol \ding{51} indicates a positive control is in place, while \ding{55} indicates a control gap.

\begin{table}[h!]
\centering
\caption{Organizational Security Control Questionnaire}
\label{tab:controls}
\begin{tabular}{@{}lc@{}}
\toprule
\textbf{Control Question} & \textbf{Response} \\
\midrule
Do you require MFA to access email? & \ding{51} \\
Do you require MFA to log into computers? & \textbf{\color{red}\ding{55}} \\
Do you require MFA to access sensitive data systems? & \textbf{\color{red}\ding{55}} \\
Does your organization have an employee acceptable use policy? & \textbf{\color{red}\ding{55}} \\
Does your organization do security awareness training for new employees? & \ding{51} \\
Does your organization do security awareness training for all employees annually? & \textbf{\color{red}\ding{55}} \\
\bottomrule
\end{tabular}
\end{table}

\subsection*{Analysis}
The lack of enforced MFA on computer logins and sensitive data systems represents a critical vulnerability. If an employee's credentials are compromised, an attacker could potentially gain widespread access to the network and critical data. Additionally, the absence of an acceptable use policy and mandatory annual security training for all staff creates a high-risk environment where employees may be unaware of secure practices and organizational expectations, increasing the likelihood of human error leading to a security incident.

% --- 4. Technical Scan Results ---
\section{Technical Scan Results}

A network scan was performed on the target system to identify open ports and exposed services. The scan provides a technical snapshot of the system's external-facing posture.

\begin{itemize}
    \item \textbf{Target IP:} \texttt{127.0.0.1}
    \item \textbf{Scan Date:} Scan date not provided in source data.
\end{itemize}

\begin{table}[h!]
\centering
\caption{Open Ports Detected on \texttt{127.0.0.1}}
\label{tab:scan}
\begin{tabular}{@{}llll@{}}
\toprule
\textbf{Port} & \textbf{State} & \textbf{Service} & \textbf{Product / Version} \\
\midrule
22 & Open & SSH & \textit{Details not provided by scan} \\
\bottomrule
\end{tabular}
\end{table}

\subsection*{Analysis}
The scan identified that port 22, the standard port for the Secure Shell (SSH) protocol, is open on the localhost interface (\texttt{127.0.0.1}). SSH is a powerful administrative tool that provides encrypted remote access. An open SSH port on a loopback address is unusual and directly correlates with the pre-existing risk ``Localhost Exposed.'' This could indicate a misconfigured service intended for remote management that is improperly bound to the wrong network interface, or a local service that is unnecessarily exposed. This finding requires immediate investigation to determine its purpose and necessity.

% --- 5. Consolidated Risk Assessment ---
\section{Consolidated Risk Assessment}

The following table synthesizes findings from the security control review, technical scan, and pre-existing risk data into a consolidated list of identified risks.

\begin{table}[h!]
\centering
\caption{Summary of Identified Risks}
\label{tab:risks}
\begin{tabular}{@{}p{0.3\linewidth}p{0.5\linewidth}l@{}}
\toprule
\textbf{Risk Name} & \textbf{Description} & \textbf{Severity} \\
\midrule
\textbf{Localhost Exposed} & Port 22 (SSH) is open on the localhost interface, confirming a potentially critical service misconfiguration. & \textbf{Critical} \\
\addlinespace
\textbf{Lack of MFA on Endpoints and Systems} & The absence of MFA for computer logins and access to sensitive data systems severely weakens access controls and exposes the organization to account takeover attacks. & \textbf{Critical} \\
\addlinespace
\textbf{Missing Acceptable Use Policy} & Without a formal policy, there are no clear rules for employees regarding the use of company assets, increasing the risk of insider threat and accidental data exposure. & \textbf{High} \\
\addlinespace
\textbf{Inadequate Annual Security Training} & Failing to provide annual security training means employees' security knowledge becomes outdated, making them more susceptible to evolving threats like phishing. & \textbf{High} \\
\bottomrule
\end{tabular}
\end{table}

% --- 6. Recommendations ---
\section{Recommendations}

Based on the consolidated risk assessment, the following actions are recommended to mitigate the identified vulnerabilities and improve the overall security posture of \textbf{Green Sprout Organic}. Recommendations are prioritized by severity.

\subsection*{Immediate Priority (Critical Risks)}
\begin{enumerate}
    \item \textbf{Implement Comprehensive MFA:} Enforce MFA for all employees for logging into their computers and for accessing any systems identified as containing sensitive data. This is the single most effective control to mitigate the risk of credential compromise.
    
    \item \textbf{Remediate Exposed Service:} Investigate the open SSH port on \texttt{127.0.0.1}. Determine the purpose of the service. If it is not required, disable it immediately. If it is required, ensure it is properly configured, firewalled, and bound only to the necessary interfaces.
\end{enumerate}

\subsection*{Near-Term Priority (High Risks)}
\begin{enumerate}
    \setcounter{enumi}{2} % Continue numbering
    \item \textbf{Develop and Implement an Acceptable Use Policy (AUP):} Create a formal AUP that clearly defines the rules and expectations for employees when using company technology and data. Require all employees to read and formally acknowledge the policy.
    
    \item \textbf{Establish an Annual Security Awareness Program:} Institute a mandatory security awareness training program for all employees to be completed at least once per year. This program should cover current threats such as phishing, social engineering, and password hygiene.
\end{enumerate}

\end{document}
```