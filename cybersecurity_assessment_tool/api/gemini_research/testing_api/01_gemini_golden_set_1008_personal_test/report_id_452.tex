```latex
\documentclass[12pt]{article}

% Preamble: Required Packages
\usepackage[a4paper, margin=1in]{geometry}
\usepackage{pifont} % For checkmarks and crosses (\ding)
\usepackage{booktabs} % For professional tables
\usepackage{hyperref} % For clickable links and references
\usepackage{url} % For formatting URLs
\usepackage{seqsplit} % To split long strings without breaking words
\usepackage{graphicx}
\usepackage{xcolor}
\usepackage{fancyhdr}

% --- Document Setup ---
\hypersetup{
    colorlinks=true,
    linkcolor=blue,
    filecolor=magenta,      
    urlcolor=cyan,
    pdftitle={Cybersecurity Assessment Report},
    pdfauthor={Cybersecurity Analyst},
    pdfsubject={Security Analysis},
    pdfkeywords={Cybersecurity, Report, Analysis},
    bookmarks=true
}

\pagestyle{fancy}
\fancyhf{} % clear all header and footers
\fancyhead[L]{Cybersecurity Assessment Report}
\fancyhead[R]{Summit Peak Partners}
\fancyfoot[C]{\thepage}

% --- Document Start ---
\begin{document}

% --- Title Page ---
\begin{titlepage}
    \centering
    \vspace*{1cm}
    \Huge\textbf{Cybersecurity Assessment Report}
    \vspace{1.5cm}
    \Large
    \textbf{Prepared for:}\\
    \vspace{0.5cm}
    Summit Peak Partners
    \vspace{2cm}
    \large
    \textbf{Date of Report:}\\
    \vspace{0.5cm}
    \today
    \vfill
    \large
    \textbf{Generated by:}\\
    \vspace{0.5cm}
    Expert Cybersecurity Analyst
\end{titlepage}

\tableofcontents
\newpage

% --- Section 1: Executive Summary ---
\section{Executive Summary}
This report provides a comprehensive cybersecurity assessment for Summit Peak Partners, based on an analysis of network scan data, organizational security controls, and known risks. The evaluation reveals a mixed security posture with several effective controls in place but also identifies critical and high-risk vulnerabilities that require immediate attention.

Key findings indicate that while Multi-Factor Authentication (MFA) is enforced for email and computer access, its absence on sensitive data systems represents a \textbf{critical security gap}. Furthermore, the lack of mandatory security awareness training for new employees creates a significant vulnerability, as new hires are often prime targets for social engineering attacks.

The technical network scan identified a service running on an internal host (\texttt{172.16.0.1}) using the unencrypted HTTP protocol (Port 80). This exposes internal communications to potential interception and eavesdropping.

Immediate remediation should focus on implementing MFA for all sensitive systems, establishing a security training program for new hires, and migrating all web services from HTTP to the secure HTTPS protocol. Addressing these issues will substantially improve the organization's resilience against common cyber threats.

% --- Section 2: Organizational Information ---
\section{Organizational Information}
This section details the organizational data provided for the assessment.

\begin{itemize}
    \item \textbf{Organization Name:} Summit Peak Partners
    \item \textbf{Email Domain:} \texttt{SummitPeakPartners.net}
    \item \textbf{Website Domain:} \url{www.SummitPeakPartners.net}
    \item \textbf{External IP Address:} \texttt{189.175.220.219}
\end{itemize}

% --- Section 3: Security Control Review ---
\section{Security Control Review}
The following table summarizes the organization's responses to a security controls questionnaire. The status indicates whether the control aligns with cybersecurity best practices. Gaps identified here are correlated with technical findings to determine overall risk.

\begin{table}[h!]
\centering
\caption{Security Controls Questionnaire Analysis}
\label{tab:controls}
\begin{tabular}{p{8cm}cc}
\toprule
\textbf{Control Question} & \textbf{Response} & \textbf{Status} \\
\midrule
Do you require MFA to access email? & Yes & \textcolor{green}{\ding{51}} \\
Do you require MFA to log into computers? & Yes & \textcolor{green}{\ding{51}} \\
\textbf{Do you require MFA to access sensitive data systems?} & \textbf{No} & \textcolor{red}{\ding{55}} \\
Does your organization have an employee acceptable use policy? & Yes & \textcolor{green}{\ding{51}} \\
\textbf{Does your organization do security awareness training for new employees?} & \textbf{No} & \textcolor{red}{\ding{55}} \\
Does your organization do security awareness training for all employees at least once per year? & Yes & \textcolor{green}{\ding{51}} \\
\bottomrule
\end{tabular}
\end{table}

\subsection*{Analysis of Control Gaps}
Two significant control gaps were identified:
\begin{itemize}
    \item \textbf{Lack of MFA for Sensitive Data:} This is a critical vulnerability. Should an attacker compromise a user's credentials, they would have direct access to the organization's most valuable data without needing a second authentication factor.
    \item \textbf{No Security Training for New Employees:} New hires are not being equipped with the knowledge to identify and avoid common threats like phishing and social engineering from day one. This makes them, and the organization, highly vulnerable during their initial employment period.
\end{itemize}

% --- Section 4: Technical Scan Results ---
\section{Technical Scan Results}
A network scan was performed to identify open ports and services on the specified target system.

\begin{itemize}
    \item \textbf{Target IP Address:} \texttt{172.16.0.1}
    \item \textbf{Scan Date:} \textbf{[Scan Date]}
\end{itemize}

\begin{table}[h!]
\centering
\caption{Open Ports on Target: \texttt{172.16.0.1}}
\label{tab:ports}
\begin{tabular}{cccl}
\toprule
\textbf{Port} & \textbf{State} & \textbf{Service (Common)} & \textbf{Notes} \\
\midrule
80/tcp & open & HTTP & \parbox{6cm}{Unencrypted web traffic. Exposes data to interception. High risk.} \\
\bottomrule
\end{tabular}
\end{table}

\subsection*{Analysis of Technical Findings}
The scan revealed that port 80 (HTTP) is open. The Hypertext Transfer Protocol (HTTP) is unencrypted, meaning any data transmitted between a client and this server—including potential login credentials or sensitive information—can be easily intercepted and read by an attacker on the same network. This is a significant security risk that violates the principle of data confidentiality.

% --- Section 5: Consolidated Risk Assessment ---
\section{Consolidated Risk Assessment}
This section synthesizes findings from the security control review and the technical scan into a consolidated list of identified risks. The risk from Input 3 was determined to be a non-pertinent system instruction and has been disregarded in this professional analysis.

\begin{table}[h!]
\centering
\caption{Summary of Identified Risks}
\label{tab:risks}
\begin{tabular}{p{1.5cm}p{6cm}p{3cm}l}
\toprule
\textbf{Risk ID} & \textbf{Description} & \textbf{Source} & \textbf{Severity} \\
\midrule
RISK-001 & Lack of MFA on sensitive data systems allows for single-factor credential compromise. & Questionnaire & \textbf{Critical} \\
\addlinespace
RISK-002 & Unencrypted HTTP service exposes internal data to eavesdropping and modification. & Network Scan & \textbf{High} \\
\addlinespace
RISK-003 & No security training for new employees increases susceptibility to phishing and social engineering. & Questionnaire & \textbf{High} \\
\bottomrule
\end{tabular}
\end{table}

% --- Section 6: Recommendations ---
\section{Recommendations}
The following actions are recommended to mitigate the identified risks and improve the overall security posture of Summit Peak Partners. Recommendations are prioritized by severity.

\subsection*{Critical Priority}
\begin{itemize}
    \item \textbf{RISK-001: Implement MFA on Sensitive Systems}
    \begin{itemize}
        \item \textbf{Action:} Immediately deploy and enforce Multi-Factor Authentication (MFA) across all systems classified as containing sensitive or critical data.
        \item \textbf{Impact:} Prevents unauthorized access even if user credentials are stolen, significantly reducing the risk of a major data breach.
    \end{itemize}
\end{itemize}

\subsection*{High Priority}
\begin{itemize}
    \item \textbf{RISK-002: Remediate Unencrypted Web Service}
    \begin{itemize}
        \item \textbf{Action:} Identify the service running on \texttt{172.16.0.1:80}. Configure it to use HTTPS with strong TLS encryption (port 443) and disable the unencrypted HTTP port 80 entirely.
        \item \textbf{Impact:} Protects data in transit, ensuring confidentiality and integrity of communications with the internal service.
    \end{itemize}
    \vspace{0.5cm}
    \item \textbf{RISK-003: Institute New Employee Security Training}
    \begin{itemize}
        \item \textbf{Action:} Develop a mandatory security awareness training module as part of the new employee onboarding process. This training should cover acceptable use, phishing identification, password hygiene, and incident reporting.
        \item \textbf{Impact:} Reduces the likelihood of human error leading to a security incident and fosters a stronger security culture from the start of employment.
    \end{itemize}
\end{itemize}

% --- Document End ---
\end{document}
```