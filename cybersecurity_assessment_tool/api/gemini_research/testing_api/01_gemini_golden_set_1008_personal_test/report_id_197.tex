```latex
\documentclass[12pt]{article}

% Preamble: Required Packages
\usepackage[margin=1in]{geometry} % Set page margins
\usepackage{pifont}                 % For checkmarks and crosses (\ding)
\usepackage{booktabs}               % For professional tables
\usepackage{graphicx}               % For including logos (placeholder)
\usepackage[table]{xcolor}          % For coloring table cells
\usepackage{hyperref}               % For clickable links and ToC
\usepackage{url}                    % For formatting URLs
\usepackage{seqsplit}               % For splitting long strings in \texttt

% --- Document Setup ---
\hypersetup{
    colorlinks=true,
    linkcolor=blue,
    filecolor=magenta,      
    urlcolor=cyan,
    pdftitle={Cybersecurity Assessment Report},
    pdfpagemode=FullScreen,
}

% Custom Colors for Severity
\definecolor{criticalred}{HTML}{D10000}
\definecolor{highorange}{HTML}{E56717}
\definecolor{mediumyellow}{HTML}{F8DE7E}
\definecolor{lowgreen}{HTML}{87A96B}
\definecolor{infoblue}{HTML}{C5DFF8}

% --- Document Start ---
\begin{document}

% --- Title Page ---
\begin{titlepage}
    \centering
    \vspace*{1cm}
    
    \Huge
    \textbf{Cybersecurity Assessment Report}
    
    \vspace{1.5cm}
    
    \Large
    Prepared for: \\
    \vspace{0.5cm}
    \textbf{Catalyst Consulting}
    
    \vspace{2cm}
    
    \large
    Report Date: \today
    
    \vfill
    
    \large
    \textbf{Confidential}
    
    \vspace{0.5cm}
    
    \small
    This document contains sensitive information. Access is restricted to authorized personnel only. Unauthorized distribution is prohibited.
    
\end{titlepage}

% --- Table of Contents ---
\tableofcontents
\newpage

% --- Section 1: Executive Summary ---
\section{Executive Summary}
This report provides a comprehensive analysis of the cybersecurity posture of Catalyst Consulting, based on a combination of network scanning, a security controls questionnaire, and a review of pre-existing risks. The assessment identified several high-impact vulnerabilities that require immediate attention to mitigate the risk of unauthorized access, data breach, and operational disruption.

The most critical findings are:
\begin{itemize}
    \item \textbf{Critical Gaps in Multi-Factor Authentication (MFA):} The organization does not enforce MFA for logging into computers or accessing sensitive data systems. This significantly increases the risk of account compromise leading to unauthorized system access.
    \item \textbf{Systemic Insecure Service Exposure:} The technical scan identified a server with Remote Desktop Protocol (RDP) open to the internal network. When correlated with existing risk data, this points to a pattern of insecure RDP configuration, a common vector for ransomware and other attacks.
\end{itemize}

While the organization has foundational policies and security awareness training in place, the technical and procedural gaps identified in this report present a tangible threat. Recommendations focus on immediate remediation of these critical risks, including the rapid deployment of MFA and the implementation of network segmentation and access controls to secure remote administration protocols.

% --- Section 2: Organizational Information ---
\section{Organizational Information}
The following information was provided for the assessment.
\begin{center}
\begin{tabular}{@{}ll@{}}
\toprule
\textbf{Attribute} & \textbf{Value} \\
\midrule
Organization Name & Catalyst Consulting \\
Email Domain & \texttt{CatalystConsulting.net} \\
Website Domain & \url{www.CatalystConsulting.net} \\
External IP Address & \texttt{184.161.90.89} \\
\bottomrule
\end{tabular}
\end{center}

% --- Section 3: Security Control Review ---
\section{Security Control Review}
A security questionnaire was completed to evaluate existing administrative and procedural controls. The results below highlight both strengths and critical weaknesses in the current security framework. A green checkmark (\textcolor{green}{\ding{51}}) indicates a positive control, while a red cross (\textcolor{red}{\ding{55}}) indicates a security gap.

\begin{center}
\begin{tabular}{@{}lc@{}}
\toprule
\textbf{Security Control Question} & \textbf{Status} \\
\midrule
Do you require MFA to access email? & \textcolor{green}{\ding{51}} \\
Do you require MFA to log into computers? & \textcolor{red}{\ding{55}} \\
Do you require MFA to access sensitive data systems? & \textcolor{red}{\ding{55}} \\
Does your organization have an employee acceptable use policy? & \textcolor{green}{\ding{51}} \\
Does your organization do security awareness training for new employees? & \textcolor{green}{\ding{51}} \\
Does your organization do security awareness training for all employees annually? & \textcolor{green}{\ding{51}} \\
\bottomrule
\end{tabular}
\end{center}

\subsection*{Analysis of Controls}
The review identified two critical gaps in access control management. The absence of MFA on workstations and sensitive data systems drastically lowers the barrier for an attacker who has compromised a user's credentials. This weakness effectively nullifies many other security measures, as it allows a threat actor to move laterally within the network and access critical data with only a password.

% --- Section 4: Technical Scan Results ---
\section{Technical Scan Results}
An internal network scan was performed to identify active services and potential vulnerabilities.

\subsection*{Host Scan: \texttt{10.10.10.51}}
The scan revealed the following open port on the target host:
\begin{center}
\begin{tabular}{@{}llll@{}}
\toprule
\textbf{Port} & \textbf{State} & \textbf{Service Name} & \textbf{Description} \\
\midrule
3389/tcp & open & \texttt{ms-wbt-server} & Microsoft Remote Desktop Protocol (RDP) \\
\bottomrule
\end{tabular}
\end{center}

\subsection*{Analysis of Technical Findings}
The discovery of an open RDP port on host \texttt{10.10.10.51} is a significant finding. RDP is a primary target for attackers seeking to gain remote control over systems. When exposed, it is susceptible to brute-force password attacks, credential stuffing, and exploitation of known vulnerabilities (e.g., BlueKeep). This finding, when combined with the pre-existing risk on host \texttt{10.10.10.50}, indicates a systemic issue in network configuration and a lack of secure remote access policies.

% --- Section 5: Risk Assessment ---
\section{Risk Assessment}
The following table synthesizes findings from the security control review, technical scan, and pre-existing risk data into a prioritized list of security risks.

\rowcolors{2}{gray!10}{white}
\begin{tabular}{p{0.25\linewidth} p{0.5\linewidth} p{0.15\linewidth}}
\toprule
\textbf{Risk Name} & \textbf{Description} & \textbf{Severity} \\
\midrule
\textbf{Lack of MFA on Sensitive Systems} & User accounts with access to critical or sensitive data are not protected by MFA. A single compromised password could lead to a major data breach. & \cellcolor{criticalred} \textbf{Critical} \\

\textbf{Insecure RDP Exposure} & Remote Desktop Protocol is exposed on multiple internal servers (\texttt{10.10.10.50}, \texttt{10.10.10.51}). This allows attackers on the network to attempt to gain full control of these systems. & \cellcolor{criticalred} \textbf{Critical} \\

\textbf{Lack of MFA on Workstations} & Standard user workstations do not require MFA for login. This allows for easier lateral movement and privilege escalation for an attacker who has compromised a user's credentials. & \cellcolor{highorange} \textbf{High} \\
\bottomrule
\end{tabular}

% --- Section 6: Recommendations ---
\section{Recommendations}
The following actions are recommended to address the identified risks. Recommendations are prioritized based on severity and potential impact.

\subsection{Remediation for: Lack of MFA (Critical \& High)}
The absence of MFA is the most pressing procedural gap.
\begin{itemize}
    \item \textbf{Immediate Action:} Prioritize and deploy an MFA solution for all access to systems identified as storing or processing sensitive data. This should be completed within 30 days.
    \item \textbf{Short-Term Action:} Begin a phased rollout of MFA for all employee computer logins. This will significantly harden endpoints against unauthorized use.
    \item \textbf{Long-Term Strategy:} Integrate MFA into all applications and services where possible, adopting a principle of "MFA everywhere" for authentication.
\end{itemize}

\subsection{Remediation for: Insecure RDP Exposure (Critical)}
The pattern of exposed RDP must be addressed systemically.
\begin{itemize}
    \item \textbf{Immediate Action:} On hosts \texttt{10.10.10.50} and \texttt{10.10.10.51}, use host-based firewalls (like Windows Defender Firewall) or network access control lists (ACLs) to restrict access to TCP port 3389. Access should only be permitted from specific, authorized IT administration IP addresses.
    \item \textbf{Short-Term Action:} Conduct a comprehensive scan of the entire internal network to identify any other systems with exposed RDP or other insecure remote management services (e.g., VNC, Telnet). Remediate all findings.
    \item \textbf{Long-Term Strategy:} Implement a secure remote access solution, such as a Virtual Private Network (VPN) or a bastion host (jump box). All administrative access should be routed through this controlled gateway, which must enforce MFA.
\end{itemize}

\end{document}
```