```latex
\documentclass[12pt]{article}

% --- PACKAGES ---
\usepackage[margin=1in]{geometry}
\usepackage{pifont} % For checkmarks and crosses
\usepackage{booktabs} % For professional tables
\usepackage{hyperref} % For clickable links
\usepackage{url} % For URL formatting
\usepackage{seqsplit} % For splitting long strings in tt font

% --- DOCUMENT METADATA ---
\hypersetup{
    colorlinks=true,
    linkcolor=black,
    urlcolor=blue,
    pdftitle={Cybersecurity Assessment Report},
    pdfauthor={Cybersecurity Analyst},
    pdfsubject={Security Assessment}
}

\title{Cybersecurity Assessment Report}
\author{Cybersecurity Analyst}
\date{November 22, 2025}

% --- BEGIN DOCUMENT ---
\begin{document}

\maketitle
\thispagestyle{empty}
\newpage
\tableofcontents
\newpage

% ==============================================================================
\section{Executive Summary}
% ==============================================================================

This report details the findings of a cybersecurity assessment conducted for \textbf{Calyx Botany}. The assessment combined a review of organizational security controls, an external network scan, and an analysis of pre-existing risks.

The overall security posture reveals significant areas for improvement. The most critical findings are the absence of Multi-Factor Authentication (MFA) for email and computer access, which exposes the organization to substantial risks of account compromise and unauthorized access. Additionally, the external-facing web server is running an outdated version of Nginx, which is susceptible to publicly known vulnerabilities.

Immediate remediation of these high-impact issues is strongly recommended to reduce the organization's risk profile and strengthen its defenses against common cyber threats.

% ==============================================================================
\section{Organizational Information}
% ==============================================================================

The following information was provided for the assessment.

\begin{itemize}
    \item \textbf{Organization Name:} Calyx Botany
    \item \textbf{Email Domain:} \texttt{CalyxBotany.org}
    \item \textbf{Website Domain:} \url{www.CalyxBotany.org}
    \item \textbf{External IP Address:} \texttt{99.191.208.201}
\end{itemize}

% ==============================================================================
\section{Security Control Review}
% ==============================================================================

A review of administrative and organizational security controls was conducted via a questionnaire. The responses indicate critical gaps in identity and access management policies. A summary of the responses is provided in Table \ref{tab:controls}. A checkmark (\ding{51}) indicates an affirmative response, while a cross (\ding{55}) indicates a negative response and a potential security gap.

\begin{table}[h!]
\centering
\caption{Organizational Security Control Responses}
\label{tab:controls}
\begin{tabular}{p{0.8\textwidth} c}
\toprule
\textbf{Control Question} & \textbf{Response} \\
\midrule
Do you require MFA to access email? & \ding{55} \\
Do you require MFA to log into computers? & \ding{55} \\
Do you require MFA to access sensitive data systems? & \ding{51} \\
Does your organization have an employee acceptable use policy? & \ding{51} \\
Does your organization do security awareness training for new employees? & \ding{51} \\
Does your organization do security awareness training for all employees at least once per year? & \ding{51} \\
\bottomrule
\end{tabular}
\end{table}

% ==============================================================================
\section{Technical Scan Results}
% ==============================================================================

An Nmap scan was performed on \texttt{2025-11-22} against the target host \texttt{192.168.10.5}. The scan identified one open port, detailed in Table \ref{tab:scan}.

\begin{table}[h!]
\centering
\caption{Open Port Analysis}
\label{tab:scan}
\begin{tabular}{l l l l l}
\toprule
\textbf{Port} & \textbf{State} & \textbf{Service} & \textbf{Product} & \textbf{Version} \\
\midrule
443/tcp & open & https & nginx & 1.18.0 \\
\bottomrule
\end{tabular}
\end{table}

\subsection{Scan Observations}
\begin{itemize}
    \item \textbf{Outdated Software:} The Nginx web server (version 1.18.0) is outdated. This version, released in 2020, has several known vulnerabilities, including CVE-2021-23017, which could allow an attacker to cause a denial of service or potentially execute arbitrary code.
    \item \textbf{SSL Certificate Mismatch:} The SSL certificate presented by the server has a Common Name of \texttt{www.acme-corp.com}, which does not match the organization's domain (\texttt{www.CalyxBotany.org}). This misconfiguration can cause browser trust errors and may indicate a deployment issue.
\end{itemize}

% ==============================================================================
\section{Consolidated Risk Assessment}
% ==============================================================================

The following table synthesizes findings from the security control review and the technical scan. No pre-existing vulnerabilities were reported. Each risk has been assigned a severity level based on its potential impact and likelihood of exploitation.

\begin{table}[h!]
\centering
\caption{Identified Risks and Severity}
\label{tab:risks}
\begin{tabular}{p{0.1\textwidth} p{0.3\textwidth} p{0.4\textwidth} p{0.1\textwidth}}
\toprule
\textbf{ID} & \textbf{Risk Name} & \textbf{Description} & \textbf{Severity} \\
\midrule
RISK-001 & Lack of MFA on Email & The absence of MFA on email accounts significantly increases the risk of business email compromise (BEC) through phishing or credential stuffing attacks. & \textbf{Critical} \\
\addlinespace
RISK-002 & Outdated Nginx Web Server & The web server is running a version of Nginx with publicly known vulnerabilities. This could allow an attacker to compromise the server, leading to data theft or service disruption. & High \\
\addlinespace
RISK-003 & Lack of MFA on Workstations & Without MFA, a compromised user password is sufficient for an attacker to gain access to a company computer, enabling lateral movement within the network. & High \\
\addlinespace
RISK-004 & SSL Certificate Mismatch & The server is configured with an incorrect SSL certificate. This erodes user trust and could be exploited in sophisticated man-in-the-middle (MITM) attacks. & Medium \\
\bottomrule
\end{tabular}
\end{table}

% ==============================================================================
\section{Recommendations}
% ==============================================================================

The following actions are recommended to mitigate the identified risks. Recommendations are prioritized by severity.

\begin{enumerate}
    \item \textbf{[Critical] RISK-001: Implement MFA for Email Access} \\
    Immediately enforce MFA for all user accounts accessing the email system. This is the single most effective control to prevent unauthorized account access.
    
    \item \textbf{[High] RISK-002: Upgrade Nginx Server} \\
    Plan and execute an upgrade of the Nginx server on host \texttt{192.168.10.5} from version 1.18.0 to the latest stable release. Ensure a backup is performed and the upgrade is tested in a non-production environment first.
    
    \item \textbf{[High] RISK-003: Deploy MFA for Workstation Logins} \\
    Implement an MFA solution for all employee computer logins (desktops and laptops). This adds a critical layer of security to protect against credential theft.
    
    \item \textbf{[Medium] RISK-004: Correct SSL Certificate Configuration} \\
    Replace the incorrect SSL certificate on the Nginx server with a valid certificate where the Subject Name or Subject Alternative Name (SAN) matches the organization's official domain, \texttt{www.CalyxBotany.org}.
\end{enumerate}

% ==============================================================================
\section{Conclusion}
% ==============================================================================

The assessment identified critical and high-severity risks that require immediate attention. By focusing on the implementation of Multi-Factor Authentication and patching the vulnerable web server, \textbf{Calyx Botany} can significantly improve its resilience against prevalent cyber attacks. A continuous security improvement process, including regular patching and policy review, is essential for maintaining a strong long-term security posture.

\end{document}
```