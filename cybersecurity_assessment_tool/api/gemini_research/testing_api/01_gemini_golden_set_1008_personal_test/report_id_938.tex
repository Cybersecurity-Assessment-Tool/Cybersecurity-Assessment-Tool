```latex
\documentclass[12pt]{article}

% Preamble: Required Packages
\usepackage[margin=1in]{geometry}
\usepackage{pifont} % For \ding
\usepackage{booktabs} % For professional tables
\usepackage{hyperref} % For clickable links
\usepackage{url} % For URL formatting
\usepackage{seqsplit} % To split long strings in \texttt
\usepackage{graphicx}
\usepackage{xcolor}
\usepackage{fancyhdr}

% --- Document Setup ---
% Define colors for severity levels
\definecolor{criticalred}{HTML}{D73027}
\definecolor{highorange}{HTML}{F46D43}
\definecolor{mediumyellow}{HTML}{FEE08B}
\definecolor{lowblue}{HTML}{4575B4}

% Hyperref setup for better presentation
\hypersetup{
    colorlinks=true,
    linkcolor=blue,
    filecolor=magenta,      
    urlcolor=cyan,
    pdftitle={Cybersecurity Assessment Report},
    pdfpagemode=FullScreen,
}

% Header and Footer
\pagestyle{fancy}
\fancyhf{}
\fancyhead[L]{Cybersecurity Assessment Report}
\fancyhead[R]{Cinder \& Ash}
\fancyfoot[C]{\thepage}

% --- Document Start ---
\begin{document}

% --- Title Page ---
\begin{titlepage}
    \centering
    \vspace*{1cm}
    \Huge{\textbf{Cybersecurity Assessment Report}}
    \vspace{0.5cm}
    \Large{Prepared for: Cinder \& Ash}
    \vspace{1.5cm}
    \textbf{Date of Report: \today}
    \vfill
    \large
    \textbf{CONFIDENTIAL}
    \vspace{0.8cm}
    \normalsize
    This document contains sensitive information. Access is restricted to authorized personnel only. Do not distribute without explicit permission.
\end{titlepage}

\tableofcontents
\newpage

% --- Section 1: Executive Overview ---
\section{Executive Overview}
This report details the findings of a cybersecurity assessment conducted for Cinder \& Ash. The analysis synthesizes data from an external network scan, a security controls questionnaire, and a review of pre-existing risks.

The assessment reveals several \textbf{critical and high-severity vulnerabilities} that expose the organization to significant risk of data breach, financial loss, and operational disruption. Key findings include:
\begin{itemize}
    \item \textbf{Exposed Insecure Service:} A publicly accessible FTP server was identified running a dangerously outdated version (\texttt{vsftpd 2.3.4}) with a known remote code execution vulnerability. Furthermore, it is misconfigured to allow anonymous logins, presenting an immediate and severe threat.
    \item \textbf{Lack of Multi-Factor Authentication (MFA):} MFA is not enforced for email or computer logins. This represents a critical gap, as a single compromised password could grant an attacker widespread access to sensitive communications and internal systems.
    \item \textbf{Insufficient Security Training:} The organization does not provide security awareness training for new or existing employees. This significantly increases the likelihood of successful phishing attacks and other social engineering tactics.
    \item \textbf{Outdated Operating Systems:} A known risk of workstations running the end-of-life Windows 7 operating system persists, leaving them vulnerable to unpatched exploits.
\end{itemize}
Immediate remediation of these issues is strongly recommended to reduce the organization's attack surface and improve its overall security posture from its current high-risk state.

% --- Section 2: Organizational Information ---
\section{Organizational Information}
The following details were provided for the assessment.
\begin{table}[h!]
\centering
\begin{tabular}{@{}ll@{}}
\toprule
\textbf{Attribute} & \textbf{Value} \\ \midrule
Organization Name & Cinder \& Ash \\
Email Domain & \texttt{CinderAsh.org} \\
Website Domain & \url{www.CinderAsh.org} \\
External IP Address & \texttt{77.103.205.82} \\ \bottomrule
\end{tabular}
\caption{Client Organizational Data.}
\label{tab:org_data}
\end{table}

% --- Section 3: Security Control Review ---
\section{Security Control Review}
A review of organizational security controls was conducted via a questionnaire. The results highlight significant gaps in fundamental security practices, particularly concerning access control and employee awareness.
\begin{table}[h!]
\centering
\begin{tabular}{@{}p{0.6\textwidth}cc@{}}
\toprule
\textbf{Control Question} & \textbf{Response} & \textbf{Analyst Note} \\ \midrule
Do you require MFA to access email? & \ding{55} & \textcolor{criticalred}{\textbf{Critical Gap}} \\
Do you require MFA to log into computers? & \ding{55} & \textcolor{highorange}{\textbf{High Risk}} \\
Do you require MFA to access sensitive data systems? & \ding{51} & Mitigating Control \\
Does your organization have an employee acceptable use policy? & \ding{51} & Good \\
Does your organization do security awareness training for new employees? & \ding{55} & \textcolor{highorange}{\textbf{High Risk}} \\
Does your organization do security awareness training for all employees at least once per year? & \ding{55} & \textcolor{highorange}{\textbf{High Risk}} \\ \bottomrule
\end{tabular}
\caption{Security Controls Questionnaire Results. (\ding{51} = Yes, \ding{55} = No)}
\label{tab:controls_review}
\end{table}

% --- Section 4: Technical Scan Results ---
\section{Technical Scan Results}
An Nmap scan was performed on the target system to identify open ports and exposed services.
\begin{itemize}
    \item \textbf{Target IP Address:} \texttt{10.0.0.15}
    \item \textbf{Scan Date:} \today
\end{itemize}

The scan identified one open port with a critically vulnerable service.
\begin{table}[h!]
\centering
\begin{tabular}{@{}lllll@{}}
\toprule
\textbf{Port} & \textbf{Service} & \textbf{Product} & \textbf{Version} & \textbf{Finding} \\ \midrule
21/tcp & ftp & vsftpd & 2.3.4 & \begin{tabular}[c]{@{}l@{}}\textcolor{criticalred}{\textbf{CRITICAL: Vulnerable Version (CVE-2011-2523)}}\\ \textcolor{criticalred}{\textbf{CRITICAL: Anonymous FTP Login Allowed}}\end{tabular} \\ \bottomrule
\end{tabular}
\caption{Network Scan Findings.}
\label{tab:scan_results}
\end{table}

\subsection{Analysis of Technical Findings}
The FTP service identified on port 21 is running \textbf{vsftpd version 2.3.4}. This specific version, released in 2011, contains a critical backdoor vulnerability (\href{https://nvd.nist.gov/vuln/detail/CVE-2011-2523}{CVE-2011-2523}) that allows an unauthenticated remote attacker to execute arbitrary commands on the server.

Compounding this issue, the service is configured to allow \textbf{anonymous FTP login}. This misconfiguration permits anyone on the internet to connect to the server and potentially access, upload, or delete files, leading to data leakage or malware infection. The combination of these two findings presents an extreme and immediate risk to the organization.

% --- Section 5: Comprehensive Risk Assessment ---
\section{Comprehensive Risk Assessment}
The following table synthesizes findings from the technical scan, controls review, and pre-existing risk data into a unified risk register. Risks are prioritized by severity.

\begin{table}[h!]
\centering
\resizebox{\textwidth}{!}{%
\begin{tabular}{@{}cllll@{}}
\toprule
\textbf{ID} & \textbf{Risk Description} & \textbf{Severity} & \textbf{Affected Systems} & \textbf{Source} \\ \midrule
\textbf{R-01} & \begin{tabular}[c]{@{}l@{}}A vulnerable FTP server (vsftpd 2.3.4) is\\ exposed and allows anonymous login.\end{tabular} & \colorbox{criticalred}{\textcolor{white}{\textbf{CRITICAL}}} & Server at \texttt{10.0.0.15} & Network Scan \\
\addlinespace
\textbf{R-02} & \begin{tabular}[c]{@{}l@{}}Lack of MFA on email and computer logins\\ allows for account/system takeover.\end{tabular} & \colorbox{criticalred}{\textcolor{white}{\textbf{CRITICAL}}} & \begin{tabular}[c]{@{}l@{}}All user accounts,\\ Workstations\end{tabular} & Questionnaire \\
\addlinespace
\textbf{R-03} & \begin{tabular}[c]{@{}l@{}}Workstations are running Windows 7, an\\ unsupported OS lacking security updates.\end{tabular} & \colorbox{highorange}{\textcolor{white}{\textbf{HIGH}}} & Workstations & Current Risks \\
\addlinespace
\textbf{R-04} & \begin{tabular}[c]{@{}l@{}}Lack of security awareness training increases\\ susceptibility to social engineering.\end{tabular} & \colorbox{highorange}{\textcolor{white}{\textbf{HIGH}}} & All Employees & Questionnaire \\ \bottomrule
\end{tabular}%
}
\caption{Synthesized Risk Register.}
\label{tab:risk_register}
\end{table}

% --- Section 6: Recommendations ---
\section{Recommendations}
Based on the comprehensive risk assessment, the following prioritized actions are recommended to mitigate the identified vulnerabilities and strengthen the security posture of Cinder \& Ash.

\subsection{Immediate Priority (Remediate within 72 hours)}
\begin{enumerate}
    \item \textbf{Address Vulnerable FTP Server (R-01):}
    \begin{itemize}
        \item Immediately take the server at \texttt{10.0.0.15} offline or place it behind a firewall that blocks all access to port 21.
        \item If the FTP service is required for business, upgrade the \texttt{vsftpd} software to the latest stable version.
        \item Disable anonymous login access.
        \item If possible, migrate from FTP to a secure alternative like SFTP (SSH File Transfer Protocol) or FTPS (FTP over SSL/TLS).
    \end{itemize}
    
    \item \textbf{Implement Multi-Factor Authentication (R-02):}
    \begin{itemize}
        \item Enable and enforce MFA for all user accounts on the primary email system (e.g., Microsoft 365, Google Workspace).
        \item Begin the phased rollout of MFA for all computer and remote access logins.
    \end{itemize}
\end{enumerate}

\subsection{High Priority (Remediate within 30-60 days)}
\begin{enumerate}
    \setcounter{enumi}{2}
    \item \textbf{Upgrade End-of-Life Operating Systems (R-03):}
    \begin{itemize}
        \item Develop and execute a plan to upgrade all workstations running Windows 7 to a modern, supported operating system such as Windows 10 or Windows 11.
        \item Ensure all new hardware acquisitions meet the requirements for modern operating systems.
    \end{itemize}
    
    \item \textbf{Establish Security Awareness Training Program (R-04):}
    \begin{itemize}
        \item Procure and implement a security awareness training solution.
        \item Mandate foundational training for all new employees as part of their onboarding process.
        \item Schedule and require annual refresher training for all staff, including simulated phishing campaigns to measure effectiveness.
    \end{itemize}
\end{enumerate}

\end{document}
```