```latex
\documentclass[12pt, a4paper]{article}

% Preamble: Required Packages
\usepackage[margin=1in]{geometry}
\usepackage{pifont} % For checkmarks and crosses
\usepackage{booktabs} % For professional tables
\usepackage{hyperref} % For clickable links and ToC
\usepackage{url} % For formatting URLs
\usepackage{seqsplit} % To split long strings without breaking
\usepackage{graphicx}
\usepackage[table]{xcolor} % For coloring table cells
\usepackage{fancyhdr} % For headers and footers

% --- Document Metadata ---
\hypersetup{
    colorlinks=true,
    linkcolor=blue,
    filecolor=magenta,      
    urlcolor=cyan,
    pdftitle={Cybersecurity Posture Assessment Report},
    pdfauthor={Cybersecurity Analyst},
    pdfsubject={Security Analysis},
    pdfkeywords={Security, Report, Analysis},
    bookmarks=true
}

% --- Custom Commands & Settings ---
\pagestyle{fancy}
\fancyhf{} % Clear all header and footer fields
\fancyhead[L]{Cybersecurity Post-Assessment Report}
\fancyhead[R]{Structure \& Form}
\fancyfoot[C]{\thepage}
\renewcommand{\headrulewidth}{0.4pt}
\renewcommand{\footrulewidth}{0.4pt}

% Define severity colors
\definecolor{sev_critical}{HTML}{990000}
\definecolor{sev_high}{HTML}{D14100}
\definecolor{sev_medium}{HTML}{E0C000}
\definecolor{sev_low}{HTML}{339900}

% --- Document Start ---
\begin{document}

% --- Title Page ---
\begin{titlepage}
    \centering
    \vspace*{1cm}
    \includegraphics[width=0.4\textwidth]{example-image-a} % Placeholder logo
    \vfill
    \Huge{\textbf{Cybersecurity Posture Assessment Report}}
    \vspace{1.5cm}
    \Large{\textbf{Prepared for: Structure \& Form}}
    \vspace{2cm}
    \large{
        \begin{tabular}{ll}
            \textbf{Date of Report:} & \today \\
            \textbf{Date of Scan:} & Not Specified \\
            \textbf{Author:} & Cybersecurity Analyst \\
        \end{tabular}
    }
    \vfill
    \textit{This document contains sensitive information and is intended for the exclusive use of the recipient.}
\end{titlepage}

\tableofcontents
\newpage

% --- Section 1: Executive Summary ---
\section{Executive Summary}
This report details the findings of a cybersecurity assessment conducted for Structure \& Form. The analysis combined a review of organizational security controls, a technical network scan, and an evaluation of pre-existing risks.

The assessment revealed several critical-risk vulnerabilities that require immediate attention. A publicly accessible FTP server was discovered running a notoriously vulnerable version of \texttt{vsftpd} (2.3.4) with anonymous login enabled. This configuration poses a significant and immediate threat of unauthorized access and system compromise.

Furthermore, critical gaps were identified in the organization's access control policies. Multi-Factor Authentication (MFA) is not enforced for accessing email or other sensitive data systems, leaving these key assets vulnerable to account takeover attacks. A high-risk gap was also noted in the employee onboarding process, which lacks mandatory security awareness training for new hires.

These findings, combined with the existing medium-risk issue of outdated Windows 7 workstations, indicate a security posture with substantial areas for improvement. This report provides prioritized, actionable recommendations to mitigate these risks and strengthen the overall security of the organization.

% --- Section 2: Organizational Information ---
\section{Organizational Information}
The following details were provided for the assessment.
\begin{center}
\begin{tabular}{ll}
\toprule
\textbf{Attribute} & \textbf{Value} \\
\midrule
Organization Name & Structure \& Form \\
Email Domain & \texttt{StructureForm.org} \\
Website Domain & \url{www.StructureForm.org} \\
External IP Address & \texttt{84.113.126.195} \\
\bottomrule
\end{tabular}
\end{center}

% --- Section 3: Security Control Review ---
\section{Security Control Review}
A review of organizational security controls was conducted based on a standardized questionnaire. The responses indicate the current state of implemented policies and procedures. "No" answers represent significant gaps in the security framework.

\begin{center}
\begin{tabular}{p{0.7\linewidth} c}
\toprule
\textbf{Control Question} & \textbf{Status} \\
\midrule
Do you require MFA to access email? & \ding{55} \\
Do you require MFA to log into computers? & \ding{51} \\
Do you require MFA to access sensitive data systems? & \ding{55} \\
Does your organization have an employee acceptable use policy? & \ding{51} \\
Does your organization do security awareness training for new employees? & \ding{55} \\
Does your organization do security awareness training for all employees at least once per year? & \ding{51} \\
\bottomrule
\end{tabular}
\end{center}

\subsection*{Analysis of Control Gaps}
\begin{itemize}
    \item \textbf{MFA for Email \& Sensitive Data (Critical Gap):} The absence of MFA on email and sensitive systems is a critical vulnerability. Email is a primary target for phishing attacks, and a compromised account can lead to widespread system access and data breaches.
    \item \textbf{New Hire Security Training (High-Risk Gap):} Failing to train new employees on security best practices from day one leaves the organization vulnerable. New hires are often unaware of internal policies and are prime targets for social engineering attacks.
\end{itemize}

% --- Section 4: Technical Scan Results ---
\section{Technical Scan Results}
An external network scan was performed to identify open ports and exposed services.
\subsection*{Target: \texttt{10.0.0.15}}
The scan identified one open port with a highly vulnerable service.

\begin{center}
\rowcolors{2}{gray!10}{white}
\begin{tabular}{p{0.1\linewidth} p{0.15\linewidth} p{0.65\linewidth}}
\toprule
\textbf{Port} & \textbf{Service/Version} & \textbf{Finding} \\
\midrule
21/tcp & FTP / vsftpd 2.3.4 & \textbf{CRITICAL: Vulnerable Service Detected.} This version of \texttt{vsftpd} is known to be vulnerable to a backdoor command execution vulnerability (CVE-2011-2523). An attacker can gain a root shell on the server. \\
\addlinespace
 & & \textbf{HIGH: Anonymous Login Enabled.} The FTP server allows anonymous login, permitting unauthenticated users to access, upload, or download files. This can lead to data leakage or be used to stage further attacks. \\
\bottomrule
\end{tabular}
\end{center}

% --- Section 5: Consolidated Risk Assessment ---
\section{Consolidated Risk Assessment}
The following table synthesizes findings from the security control review, technical scan, and pre-existing risk data into a consolidated list of security risks.

\begin{center}
\begin{tabular}{p{0.25\linewidth} p{0.5\linewidth} p{0.15\linewidth}}
\toprule
\textbf{Risk Name} & \textbf{Overview} & \textbf{Severity} \\
\midrule
\rowcolor{sev_critical!30}
Vulnerable FTP Server & The public-facing FTP server is running \texttt{vsftpd 2.3.4}, which contains a critical backdoor vulnerability (CVE-2011-2523). Anonymous login is also enabled. & \textbf{Critical} \\
\addlinespace
\rowcolor{sev_critical!30}
Insufficient MFA Controls & Multi-Factor Authentication is not required for email or sensitive data systems, exposing them to account takeover via credential theft or phishing. & \textbf{Critical} \\
\addlinespace
\rowcolor{sev_high!30}
Inadequate New Hire Training & New employees do not receive security awareness training as part of their onboarding, increasing their susceptibility to social engineering. & \textbf{High} \\
\addlinespace
\rowcolor{sev_medium!30}
Outdated Windows Policy & Workstations are running Windows 7, an end-of-life operating system that no longer receives security updates, leaving them vulnerable to known exploits. & \textbf{Medium} \\
\bottomrule
\end{tabular}
\end{center}

% --- Section 6: Recommendations ---
\section{Recommendations}
The following actions are recommended to mitigate the identified risks. They are prioritized based on severity and potential impact.

\subsection*{Immediate Priority (Critical Risks)}
\begin{enumerate}
    \item \textbf{Remediate Vulnerable FTP Server:}
    \begin{itemize}
        \item Immediately take the FTP server at \texttt{10.0.0.15} offline.
        \item If the service is not required, decommission the server permanently.
        \item If required, replace it with a secure file transfer protocol like SFTP (SSH File Transfer Protocol). If FTP must be used, upgrade \texttt{vsftpd} to the latest stable version and disable anonymous access.
    \end{itemize}
    \item \textbf{Implement Comprehensive MFA:}
    \begin{itemize}
        \item Enforce MFA for all user accounts across all critical systems, starting with email (e.g., Office 365, Google Workspace) and any systems identified as containing sensitive data.
    \end{itemize}
\end{enumerate}

\subsection*{High Priority}
\begin{enumerate}
    \setcounter{enumi}{2}
    \item \textbf{Establish New Hire Security Training:}
    \begin{itemize}
        \item Develop a mandatory security awareness training module for all new employees.
        \item This training should be part of the standard onboarding process and cover topics such as phishing, acceptable use, password hygiene, and incident reporting.
    \end{itemize}
\end{enumerate}

\subsection*{Medium Priority}
\begin{enumerate}
    \setcounter{enumi}{3}
    \item \textbf{Plan Operating System Upgrades:}
    \begin{itemize}
        \item Create a phased plan to upgrade or replace all workstations running Windows 7 with a modern, supported operating system like Windows 10 or 11.
    \end{itemize}
\end{enumerate}

\end{document}
```