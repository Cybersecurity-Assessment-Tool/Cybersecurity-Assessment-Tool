```latex
\documentclass[12pt, a4paper]{article}

% Preamble: Required Packages
\usepackage[margin=1in]{geometry}
\usepackage{pifont} % For checkmarks and crosses
\usepackage{booktabs} % For professional tables
\usepackage{hyperref} % For clickable links
\usepackage{url}      % For URL formatting
\usepackage{seqsplit} % For splitting long strings in texttt
\usepackage{graphicx} % For logo (placeholder)
\usepackage{fancyhdr} % For header/footer

% --- Document Metadata ---
\title{Cybersecurity Posture Assessment Report}
\author{Cybersecurity Analysis Division}
\date{\today}

% --- Header and Footer Configuration ---
\pagestyle{fancy}
\fancyhf{} % Clear all header and footer fields
\fancyhead[L]{Nexus Dynamics Cybersecurity Report}
\fancyfoot[C]{\thepage}
\renewcommand{\headrulewidth}{0.4pt}
\renewcommand{\footrulewidth}{0.4pt}

\begin{document}

\maketitle
\thispagestyle{empty}
\newpage

\tableofcontents
\newpage

% --- Section 1: Executive Overview ---
\section{Executive Overview}

This report details the findings of a cybersecurity posture assessment conducted for \textbf{Nexus Dynamics}. The analysis synthesizes data from an external network scan, a security controls questionnaire, and a review of pre-existing risks.

The overall security posture is assessed as \textbf{critically weak}. The assessment identified significant, fundamental gaps in security controls that expose the organization to a high likelihood of compromise.

Key findings include:
\begin{itemize}
    \item \textbf{Systemic Lack of Multi-Factor Authentication (MFA):} MFA is not enforced for email, computer logins, or access to sensitive data. This absence of a critical compensating control dramatically increases the risk of unauthorized access via credential theft or brute-force attacks.
    \item \textbf{Exposed Remote Services:} The technical scan identified an open Remote Desktop Protocol (RDP) port on an internal host. This finding, combined with a pre-existing risk of the same nature on another host, points to a systemic pattern of insecure configuration. Exposed RDP is a primary vector for ransomware attacks.
    \item \textbf{Absence of Foundational Security Programs:} The organization lacks a formal Acceptable Use Policy and does not conduct any security awareness training. This indicates a low level of security maturity and increases the risk of security incidents caused by human error.
\end{itemize}

The combination of exposed high-risk services and the complete lack of MFA creates a direct and immediate path for an attacker to gain internal network access. Urgent remediation is required to mitigate these critical risks.

% --- Section 2: Organizational Information ---
\section{Organizational Information}

The following information was provided for the assessment.

\begin{tabular}{@{}ll}
    \toprule
    \textbf{Attribute} & \textbf{Value} \\
    \midrule
    Organization Name & \textbf{Nexus Dynamics} \\
    Email Domain & \texttt{NexusDynamics.com} \\
    External IP Address & \texttt{29.33.133.87} \\
    \bottomrule
\end{tabular}

% --- Section 3: Security Control Review ---
\section{Security Control Review}

A review of administrative security controls was conducted via a questionnaire. The responses indicate critical gaps in foundational security practices. A (\ding{51}) indicates an affirmative response (control in place), while a (\ding{55}) indicates a negative response (control gap).

\begin{table}[h!]
\centering
\begin{tabular}{@{}p{0.8\linewidth}c@{}}
    \toprule
    \textbf{Control Question} & \textbf{Response} \\
    \midrule
    Do you require MFA to access email? & \ding{55} \\
    Do you require MFA to log into computers? & \ding{55} \\
    Do you require MFA to access sensitive data systems? & \ding{55} \\
    Does your organization have an employee acceptable use policy? & \ding{55} \\
    Does your organization do security awareness training for new employees? & \ding{55} \\
    Does your organization do security awareness training for all employees at least once per year? & \ding{55} \\
    \bottomrule
\end{tabular}
\caption{Security Controls Questionnaire Results.}
\label{tab:controls}
\end{table}

% --- Section 4: Technical Scan Results ---
\section{Technical Scan Results}

An Nmap scan was performed to identify open ports and services on the target system.

\begin{itemize}
    \item \textbf{Target IP Address:} \texttt{10.10.10.51}
\end{itemize}

\begin{table}[h!]
\centering
\begin{tabular}{@{}llll@{}}
    \toprule
    \textbf{Port/Proto} & \textbf{State} & \textbf{Service} & \textbf{Notes} \\
    \midrule
    3389/tcp & Open & ms-wbt-server & Remote Desktop Protocol (RDP). \\
    & & & A high-risk service commonly targeted \\
    & & & by attackers for initial access. \\
    \bottomrule
\end{tabular}
\caption{Open Ports Detected on \texttt{10.10.10.51}.}
\label{tab:scanresults}
\end{table}

% --- Section 5: Consolidated Risk Assessment ---
\section{Consolidated Risk Assessment}

The following table correlates findings from the security control review, the technical scan, and pre-existing risk data to provide a consolidated view of the primary risks facing the organization.

\begin{table}[h!]
\centering
\begin{tabular}{@{}p{0.2\linewidth}p{0.55\linewidth}p{0.15\linewidth}@{}}
    \toprule
    \textbf{Risk Title} & \textbf{Description} & \textbf{Severity} \\
    \midrule
    \textbf{Critical MFA Gaps} & MFA is not enforced for any key systems (email, endpoints, sensitive data). This negates a primary defense against credential-based attacks. & \textbf{Critical} \\
    \addlinespace
    \textbf{Systemic RDP Exposure} & RDP (port 3389) is exposed on host \texttt{10.10.10.51} (new finding) and was previously identified on \texttt{10.10.10.50}. This pattern, combined with the lack of MFA, makes these systems highly vulnerable to compromise. & \textbf{Critical} \\
    \addlinespace
    \textbf{Lack of Security Policy \& Training} & The absence of an Acceptable Use Policy and security awareness training creates a culture where employees are unaware of security best practices, increasing susceptibility to phishing and social engineering. & \textbf{High} \\
    \bottomrule
\end{tabular}
\caption{Summary of Key Risks.}
\label{tab:risks}
\end{table}

% --- Section 6: Recommendations ---
\section{Recommendations}

Based on the analysis, the following actions are recommended to mitigate the identified risks. Recommendations are prioritized by severity.

\subsection{Immediate Actions (Critical Priority)}
\begin{itemize}
    \item \textbf{Remediate RDP Exposure:} Immediately close or firewall port 3389 on hosts \texttt{10.10.10.51} and \texttt{10.10.10.50} from all untrusted networks. If remote access is a business requirement, it \textbf{must} be placed behind a Virtual Private Network (VPN) that requires Multi-Factor Authentication.
    \item \textbf{Deploy MFA:} Begin an emergency deployment of MFA for all users. Prioritize the following systems in order:
    \begin{enumerate}
        \item All remote access solutions (VPN, etc.).
        \item All email accounts (e.g., Office 365, Google Workspace).
        \item All administrative and privileged accounts.
    \end{enumerate}
\end{itemize}

\subsection{High Priority Actions}
\begin{itemize}
    \item \textbf{Implement Security Awareness Training:} Procure and implement a security awareness training program. All current employees must complete the training, and it should be a mandatory part of the onboarding process for new hires. Annual refresher training should be enforced.
    \item \textbf{Develop and Enforce Policy:} Draft and formally adopt a corporate Acceptable Use Policy (AUP). This policy should be distributed to all employees, who must formally acknowledge their understanding and agreement to comply.
\end{itemize}

\subsection{Strategic Recommendations}
\begin{itemize}
    \item \textbf{Conduct Comprehensive Vulnerability Assessment:} Perform an authenticated vulnerability scan across the entire internal network to identify other misconfigurations, outdated software, and security weaknesses.
    \item \textbf{Review Network Architecture:} Conduct a formal review of the network architecture and segmentation strategy to ensure the principle of least privilege is applied and that critical assets are isolated from general-purpose workstations.
\end{itemize}

\end{document}
```