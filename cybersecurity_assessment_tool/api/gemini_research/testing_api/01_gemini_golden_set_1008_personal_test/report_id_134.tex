```latex
\documentclass[12pt]{article}

% Preamble: Required Packages
\usepackage[margin=1in]{geometry}
\usepackage{pifont} % For checkmarks and crosses
\usepackage{booktabs} % For professional tables
\usepackage{hyperref} % For clickable links
\usepackage{url} % For formatting URLs
\usepackage{seqsplit} % For splitting long strings without spaces
\usepackage{graphicx}
\usepackage{xcolor}

% Document Information
\title{Cybersecurity Posture Assessment Report}
\author{Cybersecurity Analyst}
\date{\today}

% Hyperref Setup
\hypersetup{
    colorlinks=true,
    linkcolor=blue,
    filecolor=magenta,      
    urlcolor=cyan,
    pdftitle={Cybersecurity Posture Assessment Report},
    pdfpagemode=FullScreen,
}

\begin{document}

\maketitle
\thispagestyle{empty}
\newpage

\tableofcontents
\newpage

% --- 1. Executive Summary ---
\section{Executive Summary}
This report provides a comprehensive cybersecurity assessment for \textbf{Silver Leaf Collective}, synthesizing findings from network scans, an organizational security questionnaire, and a review of pre-existing risks.

The analysis has uncovered several critical and high-risk vulnerabilities that require immediate attention. A key finding is an externally accessible FTP server running a severely outdated and vulnerable version of \texttt{vsftpd} (2.3.4), which is configured to allow anonymous logins. This represents a direct and immediate threat to the organization's data integrity and network security.

Furthermore, a systemic lack of Multi-Factor Authentication (MFA) across email, workstations, and sensitive data systems constitutes a critical control gap. This weakness, combined with the absence of annual security awareness training for all staff, significantly elevates the risk of a successful phishing or credential compromise attack.

The pre-existing issue of outdated Windows 7 workstations remains a medium-risk concern that should be addressed. Immediate remediation of the FTP server and the phased implementation of MFA are the highest priorities for improving the organization's security posture.

% --- 2. Organizational Information ---
\section{Organizational Information}
The following details were provided for the assessment.

\begin{itemize}
    \item \textbf{Organization Name:} Silver Leaf Collective
    \item \textbf{Email Domain:} \seqsplit{\texttt{SilverLeafCollective.net}}
    \item \textbf{Website Domain:} \seqsplit{\url{www.SilverLeafCollective.net}}
    \item \textbf{External IP Address:} \texttt{8.195.47.43}
\end{itemize}

% --- 3. Security Control Review (Questionnaire Analysis) ---
\section{Security Control Review}
An internal security questionnaire was completed to evaluate existing administrative and technical controls. The responses are summarized below, with non-compliant answers highlighted as significant gaps.

\begin{table}[h!]
\centering
\caption{Security Controls Questionnaire Results}
\begin{tabular}{@{}lc@{}}
\toprule
\textbf{Control Question} & \textbf{Response} \\ \midrule
Do you require MFA to access email? & \ding{55} \\
Do you require MFA to log into computers? & \ding{55} \\
Do you require MFA to access sensitive data systems? & \ding{55} \\
Does your organization have an employee acceptable use policy? & \ding{51} \\
Does your organization do security awareness training for new employees? & \ding{51} \\
Does your organization do security awareness training for all employees at least once per year? & \ding{55} \\ \bottomrule
\end{tabular}
\end{table}

\subsection*{Analysis of Control Gaps}
\begin{itemize}
    \item \textbf{Lack of Multi-Factor Authentication (MFA):} The absence of MFA across all critical access points (email, computers, sensitive data) is a \textbf{Critical Risk}. Stolen or weak credentials are a primary vector for attackers, and MFA is the single most effective control to mitigate this threat.
    \item \textbf{Lack of Annual Security Training:} Without regular, recurring security awareness training, employee knowledge of current threats (like phishing and social engineering) diminishes. This is a \textbf{High Risk} that leaves the organization vulnerable to human error.
\end{itemize}

% --- 4. Technical Scan Results ---
\section{Technical Scan Results}
An external network scan was performed against the target IP address \texttt{10.0.0.15}. The scan identified the following open ports and services.

\begin{table}[h!]
\centering
\caption{Open Port Scan Findings for Target \texttt{10.0.0.15}}
\begin{tabular}{@{}cllll@{}}
\toprule
\textbf{Port} & \textbf{State} & \textbf{Service} & \textbf{Version} & \textbf{Details} \\ \midrule
21/tcp & open & ftp & vsftpd 2.3.4 & Anonymous FTP login allowed \\ \bottomrule
\end{tabular}
\end{table}

\subsection*{Analysis of Technical Findings}
The technical scan revealed a \textbf{Critical Risk}.
\begin{itemize}
    \item \textbf{Vulnerable FTP Service:} The server is running \texttt{vsftpd version 2.3.4}. This specific version, released in 2011, is known to contain a critical backdoor vulnerability (\textbf{CVE-2011-2523}) that allows for remote command execution. An attacker can exploit this to gain complete control over the server.
    \item \textbf{Insecure Configuration:} The FTP service is configured to allow \textbf{anonymous login}. This allows any external entity to connect to the server without authentication, posing a severe risk for data exfiltration or the uploading of malicious files onto the organization's network.
\end{itemize}

% --- 5. Consolidated Risk Assessment ---
\section{Consolidated Risk Assessment}
The following table synthesizes findings from the security questionnaire, technical scans, and pre-existing risk data into a prioritized list.

\begin{table}[h!]
\centering
\caption{Summary of Identified Risks}
\begin{tabular}{@{}p{0.3\linewidth}p{0.5\linewidth}l@{}}
\toprule
\textbf{Risk Title} & \textbf{Description} & \textbf{Severity} \\ \midrule
\textbf{Vulnerable FTP Server} & An internet-facing FTP server is running a critically outdated version (\texttt{vsftpd 2.3.4}) with a known RCE vulnerability and allows anonymous login. & \textbf{Critical} \\
\addlinespace
\textbf{Lack of Multi-Factor Authentication} & No MFA is enforced for access to email, computers, or sensitive data systems, making credential-based attacks highly likely to succeed. & \textbf{Critical} \\
\addlinespace
\textbf{Inadequate Security Training} & Security awareness training is not conducted annually for all employees, increasing susceptibility to phishing and social engineering. & \textbf{High} \\
\addlinespace
\textbf{Outdated Windows Policy} & Workstations are running the unsupported Windows 7 operating system, which no longer receives security updates. & \textbf{Medium} \\ \bottomrule
\end{tabular}
\end{table}

% --- 6. Recommendations ---
\section{Recommendations}
Based on the consolidated risk assessment, the following remediation actions are recommended in order of priority.

\subsection*{Immediate Priority (Critical Risks)}
\begin{enumerate}
    \item \textbf{Remediate Vulnerable FTP Server:}
    \begin{itemize}
        \item Immediately take the FTP server at \texttt{10.0.0.15} offline or firewall it from external access.
        \item If the business function is essential, migrate to a secure file transfer protocol like SFTP or FTPS.
        \item If FTP must be used, upgrade the \texttt{vsftpd} software to the latest stable version and \textbf{disable anonymous access} immediately.
    \end{itemize}
    \item \textbf{Deploy Multi-Factor Authentication (MFA):}
    \begin{itemize}
        \item Begin a phased rollout of MFA across the organization.
        \item Prioritize implementation for email access (e.g., Office 365, G Suite) and any remote access solutions (e.g., VPN).
        \item Extend MFA to all systems containing sensitive data and, subsequently, to all computer logins.
    \end{itemize}
\end{enumerate}

\subsection*{High Priority}
\begin{enumerate}
    \setcounter{enumi}{2}
    \item \textbf{Establish Annual Security Training:}
    \begin{itemize}
        \item Procure and schedule mandatory annual security awareness training for all employees.
        \item Include modules on identifying phishing, password hygiene, and acceptable use.
    \end{itemize}
\end{enumerate}

\subsection*{Medium Priority}
\begin{enumerate}
    \setcounter{enumi}{3}
    \item \textbf{Upgrade End-of-Life Systems:}
    \begin{itemize}
        \item Develop and execute a plan to upgrade all workstations from Windows 7 to a supported operating system, such as Windows 10 or 11.
    \end{itemize}
\end{enumerate}

\end{document}
```