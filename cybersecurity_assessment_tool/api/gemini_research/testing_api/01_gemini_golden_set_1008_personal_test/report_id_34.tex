```latex
\documentclass[12pt]{article}

% Preamble: Required Packages
\usepackage[margin=1in]{geometry}
\usepackage{pifont}                 % For \ding symbols (checkmarks and crosses)
\usepackage{booktabs}               % For professional-looking tables
\usepackage{hyperref}               % For clickable links and references
\usepackage{url}                    % For typesetting URLs
\usepackage{seqsplit}               % To split long strings in \texttt
\usepackage{xcolor}                 % For custom colors
\usepackage{fancyhdr}               % For headers and footers
\usepackage{graphicx}               % To include images, like a logo

% --- Document Setup ---
\definecolor{darkblue}{rgb}{0.0, 0.0, 0.55}
\definecolor{darkred}{rgb}{0.55, 0.0, 0.0}

\hypersetup{
    colorlinks=true,
    linkcolor=darkblue,
    filecolor=darkblue,      
    urlcolor=darkblue,
    citecolor=darkblue,
}

\pagestyle{fancy}
\fancyhf{} % Clear all header and footer fields
\fancyhead[L]{Cybersecurity Assessment Report}
\fancyhead[R]{Confidential}
\fancyfoot[C]{\thepage}
\renewcommand{\headrulewidth}{0.4pt}
\renewcommand{\footrulewidth}{0.4pt}

% --- Document Start ---
\begin{document}

\title{
    \vspace{2cm}
    \textbf{Cybersecurity Assessment Report} \\
    \large \vspace{0.5cm}
    Prepared for: \textbf{Blue Horizon Initiative}
    \vspace{1cm}
}
\author{Cybersecurity Analysis Division}
\date{\today}
\maketitle
\thispagestyle{empty}

\newpage

\tableofcontents

\newpage

% ===================================================================
\section{Executive Summary}
% ===================================================================

This report details the findings of a cybersecurity assessment conducted for \textbf{Blue Horizon Initiative}. The evaluation combined a review of organizational security controls via a questionnaire, an external network vulnerability scan, and an analysis of pre-existing risks.

The assessment identified several \textbf{critical and high-risk gaps} in foundational security controls. The most pressing concerns are the absence of Multi-Factor Authentication (MFA) for email and sensitive data systems, which exposes the organization to significant risks of account compromise and data breach. Furthermore, the lack of a formal Acceptable Use Policy (AUP) and a structured security awareness training program indicates a need to mature the organization's overall security culture and governance.

On a positive note, the external network scan of the target IP address, \texttt{[Client IP]}, did not identify any open ports. This suggests a strong network perimeter configuration, which is a commendable security posture.

In summary, while the external network perimeter appears secure, critical deficiencies in internal security policies and identity management controls present an urgent threat. This report provides specific, actionable recommendations to mitigate these identified risks and strengthen the overall security posture of \textbf{Blue Horizon Initiative}.

% ===================================================================
\section{Organizational Information}
% ===================================================================

The following information was provided for the assessment.

\begin{table}[h!]
\centering
\begin{tabular}{@{}ll@{}}
\toprule
\textbf{Attribute} & \textbf{Value} \\
\midrule
Organization Name & Blue Horizon Initiative \\
Email Domain      & \texttt{BlueHorizonInitiative.net} \\
Website Domain    & \url{www.BlueHorizonInitiative.net} \\
External IP Address & \texttt{223.87.112.138} \\
\bottomrule
\end{tabular}
\caption{Client Organizational Details}
\end{table}

% ===================================================================
\section{Security Control Review}
% ===================================================================

A review of administrative and technical security controls was conducted based on a standardized questionnaire. The responses reveal significant gaps in critical areas of security management. A summary of the findings is presented below.

\begin{table}[h!]
\centering
\begin{tabular}{@{}p{0.65\textwidth}cc@{}}
\toprule
\textbf{Control Question} & \textbf{Response} & \textbf{Status} \\
\midrule
Do you require MFA to access email? & No & \textcolor{darkred}{\ding{55}} \\
Do you require MFA to log into computers? & Yes & \textcolor{green!50!black}{\ding{51}} \\
Do you require MFA to access sensitive data systems? & No & \textcolor{darkred}{\ding{55}} \\
Does your organization have an employee acceptable use policy? & No & \textcolor{darkred}{\ding{55}} \\
Does your organization do security awareness training for new employees? & No & \textcolor{darkred}{\ding{55}} \\
Does your organization do security awareness training for all employees at least once per year? & No & \textcolor{darkred}{\ding{55}} \\
\bottomrule
\end{tabular}
\caption{Security Controls Questionnaire Results}
\end{table}

% ===================================================================
\section{Technical Scan Results}
% ===================================================================

An external network scan was performed to identify potential vulnerabilities on the public-facing infrastructure.

\subsection{External Network Scan}
\begin{itemize}
    \item \textbf{Target IP Address:} \texttt{[Target IP]}
    \item \textbf{Scan Date:} The scan date was not provided in the input data.
    \item \textbf{Summary:} The network scan completed successfully. \textbf{No open TCP or UDP ports were discovered on the target system.}
\end{itemize}

\subsubsection{Interpretation}
The absence of open ports is a positive security finding. It indicates that the network firewall is properly configured to deny unsolicited inbound traffic, significantly reducing the external attack surface. This is a strong foundational control for perimeter security.

% ===================================================================
\section{Risk Assessment Summary}
% ===================================================================

The following table synthesizes findings from the security control review, technical scan, and pre-existing risk data. The primary risks identified are related to policy and identity management gaps. No pre-existing vulnerabilities were reported.

\begin{table}[h!]
\centering
\begin{tabular}{@{}p{0.1\textwidth}p{0.3\textwidth}p{0.4\textwidth}l@{}}
\toprule
\textbf{Risk ID} & \textbf{Risk Name} & \textbf{Overview} & \textbf{Severity} \\
\midrule
RISK-001 & Lack of MFA on Email & The absence of MFA on email accounts makes them highly susceptible to phishing attacks, credential stuffing, and unauthorized access, which can lead to data breaches and further internal compromise. & \textbf{Critical} \\
\addlinespace
RISK-002 & Lack of MFA on Sensitive Systems & Sensitive data systems without MFA are prime targets for attackers. A single compromised password could grant an adversary direct access to the organization's most valuable information assets. & \textbf{Critical} \\
\addlinespace
RISK-003 & No Security Awareness Training & Without regular training, employees are more likely to fall victim to social engineering attacks like phishing, inadvertently install malware, or mishandle sensitive data, creating a significant human-based risk. & \textbf{High} \\
\addlinespace
RISK-004 & No Acceptable Use Policy (AUP) & The lack of a formal AUP creates ambiguity regarding the proper use of company assets and data. It weakens the organization's ability to enforce security standards and hold individuals accountable for risky behavior. & \textbf{High} \\
\bottomrule
\end{tabular}
\caption{Summary of Identified Risks}
\end{table}

% ===================================================================
\section{Recommendations}
% ===================================================================

Based on the risks identified in the previous section, the following actions are recommended to improve the security posture of \textbf{Blue Horizon Initiative}.

\begin{enumerate}
    \item \textbf{Implement Multi-Factor Authentication (Critical Priority):}
    \begin{itemize}
        \item \textbf{Action:} Immediately deploy a mandatory MFA solution for all user accounts, prioritizing email (RISK-001) and all systems containing sensitive data (RISK-002).
        \item \textbf{Justification:} This is the single most effective control to prevent unauthorized account access and will drastically reduce the risk of a breach resulting from compromised credentials.
    \end{itemize}
    \vspace{0.5cm}
    \item \textbf{Establish a Security Awareness Training Program (High Priority):}
    \begin{itemize}
        \item \textbf{Action:} Develop and implement a mandatory security awareness training program for all employees (RISK-003). The program should include initial training for new hires and at least one annual refresher course for all staff.
        \item \textbf{Justification:} A well-trained workforce is a critical layer of defense. Training empowers employees to recognize and report threats like phishing, strengthening the organization's human firewall.
    \end{itemize}
    \vspace{0.5cm}
    \item \textbf{Develop and Enforce an Acceptable Use Policy (High Priority):}
    \begin{itemize}
        \item \textbf{Action:} Create a formal Acceptable Use Policy (AUP) that clearly defines the rules for using company IT assets, networks, and data (RISK-004). This policy should be distributed to all employees, who must formally acknowledge their understanding and agreement.
        \item \textbf{Justification:} An AUP establishes clear security expectations, reduces risky behavior, and provides a framework for enforcing security standards across the organization.
    \end{itemize}
\end{enumerate}

\end{document}
```