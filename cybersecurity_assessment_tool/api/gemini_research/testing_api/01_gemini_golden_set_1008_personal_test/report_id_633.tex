```latex
\documentclass[12pt, a4paper]{article}

% Preamble: Required Packages
\usepackage[margin=1in]{geometry}
\usepackage{pifont} % For checkmarks and crosses
\usepackage{booktabs} % For professional tables
\usepackage{hyperref} % For clickable links
\usepackage{url} % For formatting URLs
\usepackage{seqsplit} % For splitting long strings without breaking
\usepackage{graphicx} % For logos, etc.
\usepackage{xcolor} % For colors
\usepackage{fancyhdr} % For headers and footers

% --- Document Setup ---

% Define colors for severity
\definecolor{criticalred}{HTML}{D12727}
\definecolor{highorange}{HTML}{E46C0A}
\definecolor{mediumyellow}{HTML}{F7B500}
\definecolor{lowblue}{HTML}{00B0F0}
\definecolor{infogray}{HTML}{808080}

% Hyperlink setup
\hypersetup{
    colorlinks=true,
    linkcolor=blue,
    filecolor=magenta,      
    urlcolor=cyan,
    pdftitle={Cybersecurity Posture Report},
    pdfpagemode=FullScreen,
}

% Header and Footer
\pagestyle{fancy}
\fancyhf{} % clear all header and footer fields
\fancyhead[L]{Cybersecurity Posture Report}
\fancyhead[R]{\textbf{Nebula Creative}}
\fancyfoot[C]{\thepage}

% --- Document Start ---
\begin{document}

% --- Title Page ---
\begin{titlepage}
    \centering
    \vspace*{1cm}
    
    \Huge
    \textbf{Cybersecurity Posture Report}
    
    \vspace{1.5cm}
    
    \Large
    Prepared for: \\
    \vspace{0.5cm}
    \textbf{Nebula Creative}
    
    \vspace{2cm}
    
    \large
    Report Date: \today
    
    \vfill
    
    \large
    \textit{This report contains sensitive information and should be handled with care.}
    
\end{titlepage}

\tableofcontents
\newpage

% --- Section 1: Executive Summary ---
\section{Executive Summary}
This report provides a comprehensive analysis of the cybersecurity posture for \textbf{Nebula Creative}, based on network scans, organizational data, and a review of existing risk assessments. The assessment was conducted to identify vulnerabilities, security control gaps, and provide actionable recommendations to mitigate identified risks.

\paragraph{Key Findings:} A critical vulnerability was discovered that directly contradicts the existing risk documentation. An externally facing service on port \texttt{8080} was found with the title \textbf{"TOP SECRET DB"}, indicating a high probability of sensitive data exposure. This finding invalidates a previous assessment that had marked this port as a secure false positive.

Furthermore, significant gaps were identified in foundational security controls. The organization currently lacks an employee acceptable use policy and does not conduct any form of security awareness training. These deficiencies in administrative controls create a high-risk environment susceptible to human error, social engineering, and insider threats.

\paragraph{Overall Posture:} The overall security posture is assessed as \textbf{\textcolor{criticalred}{CRITICAL}}. The combination of a potentially exposed sensitive database and a lack of fundamental employee security policies and training presents an immediate and significant threat to the organization's data and operations. Urgent remediation is required.

\newpage

% --- Section 2: Organizational Information ---
\section{Organizational Information}
The following details were provided for the assessment.

\begin{tabular}{@{}ll}
    \toprule
    \textbf{Attribute} & \textbf{Value} \\
    \midrule
    Organization Name & \textbf{Nebula Creative} \\
    Email Domain & \texttt{NebulaCreative.com} \\
    Website Domain & \url{www.NebulaCreative.com} \\
    External IP Address & \texttt{15.130.32.24} \\
    \bottomrule
\end{tabular}

% --- Section 3: Security Control Review ---
\section{Security Control Review}
A review of administrative security controls was conducted based on a supplied questionnaire. The results highlight a concerning lack of foundational policies and training, despite good implementation of technical controls like Multi-Factor Authentication (MFA).

\begin{table}[h!]
\centering
\caption{Security Controls Questionnaire Analysis}
\begin{tabular}{@{}p{0.6\linewidth} c p{0.2\linewidth}@{}}
    \toprule
    \textbf{Control Question} & \textbf{Response} & \textbf{Assessment} \\
    \midrule
    Do you require MFA to access email? & \textcolor{green}{\ding{51}} & Good Practice \\
    Do you require MFA to log into computers? & \textcolor{green}{\ding{51}} & Good Practice \\
    Do you require MFA to access sensitive data systems? & \textcolor{green}{\ding{51}} & Good Practice \\
    \midrule
    Does your organization have an employee acceptable use policy? & \textcolor{criticalred}{\ding{55}} & \textbf{Critical Gap} \\
    Does your organization do security awareness training for new employees? & \textcolor{criticalred}{\ding{55}} & \textbf{Critical Gap} \\
    Does your organization do security awareness training for all employees at least once per year? & \textcolor{criticalred}{\ding{55}} & \textbf{Critical Gap} \\
    \bottomrule
\end{tabular}
\end{table}

\newpage

% --- Section 4: Technical Scan Results ---
\section{Technical Scan Results}
An external network scan was performed to identify open ports and exposed services. The scan revealed a critical finding that requires immediate attention.

\subsection{Nmap Scan Findings}
\begin{itemize}
    \item \textbf{Target IP:} \texttt{10.5.5.5}
    \item \textbf{Scan Date:} \today
    \item \textbf{Status:} Host is Up
\end{itemize}

The following table details the open ports discovered on the target system.

\begin{table}[h!]
\centering
\caption{Open Port Analysis for \texttt{10.5.5.5}}
\begin{tabular}{@{}llll@{}}
    \toprule
    \textbf{Port} & \textbf{State} & \textbf{Service} & \textbf{Evidence / Banner} \\
    \midrule
    8080/tcp & Open & http-proxy? & \textbf{HTTP Title: TOP SECRET DB} \\
    \bottomrule
\end{tabular}
\end{table}

\paragraph{Analysis:} The scan identified an open service on port \texttt{8080}. Banner grabbing revealed the HTTP page title to be \textbf{"TOP SECRET DB"}. This is a major security concern. The title strongly suggests that a sensitive, possibly internal, database is exposed. This finding directly contradicts the information in the \textit{Current Risks} document (Input 3), which incorrectly classified this port as a secure false positive. This represents a failure in the existing risk management process.

% --- Section 5: Consolidated Risk Assessment ---
\section{Consolidated Risk Assessment}
The following table synthesizes findings from the technical scan, security control review, and analysis of existing risk data. New risks have been generated to reflect the current, accurate posture.

\begin{table}[h!]
\centering
\caption{Summary of Identified Risks}
\begin{tabular}{@{}p{0.1\linewidth} p{0.25\linewidth} p{0.4\linewidth} p{0.15\linewidth}@{}}
    \toprule
    \textbf{Risk ID} & \textbf{Risk Name} & \textbf{Description} & \textbf{Severity} \\
    \midrule
    CR-001 & Exposed Sensitive Database & A service on port 8080 is exposed with a title suggesting it is a "TOP SECRET DB". This contradicts a previous assessment that marked it as a false positive. & \textbf{\textcolor{criticalred}{CRITICAL}} \\
    \addlinespace
    HR-001 & Lack of Acceptable Use Policy & The absence of a formal AUP means there are no defined rules for employee use of company assets, increasing the risk of data misuse and insecure practices. & \textbf{\textcolor{highorange}{HIGH}} \\
    \addlinespace
    HR-002 & No Security Awareness Training & Employees are not trained on security best practices, making them highly susceptible to phishing, social engineering, and accidental data breaches. & \textbf{\textcolor{highorange}{HIGH}} \\
    \bottomrule
\end{tabular}
\end{table}

\newpage

% --- Section 6: Recommendations ---
\section{Recommendations}
The following actions are recommended to mitigate the identified risks. Recommendations are prioritized based on severity.

\subsection{CR-001: Exposed Sensitive Database (Severity: CRITICAL)}
\begin{itemize}
    \item \textbf{Immediate Action:} Immediately investigate the service running on \texttt{10.5.5.5:8080}. If it is a sensitive system, restrict all access by implementing strict firewall rules to block public access. Access should only be permitted from trusted internal IP addresses.
    \item \textbf{Short-Term Action:} Conduct a data exposure assessment to determine if any sensitive information was accessed or exfiltrated. Implement robust authentication and authorization controls on the application itself.
    \item \textbf{Long-Term Action:} Review and improve the vulnerability management and risk assessment process to prevent future misclassification of critical vulnerabilities.
\end{itemize}

\subsection{HR-001: Lack of Acceptable Use Policy (Severity: HIGH)}
\begin{itemize}
    \item \textbf{Immediate Action:} Draft a comprehensive Acceptable Use Policy (AUP) that covers topics such as data handling, internet usage, password requirements, and use of personal devices.
    \item \textbf{Short-Term Action:} Have the AUP reviewed by legal and HR departments. Upon approval, distribute the policy to all current employees for review and acknowledgment. Incorporate this step into the onboarding process for all new hires.
\end{itemize}

\subsection{HR-002: No Security Awareness Training (Severity: HIGH)}
\begin{itemize}
    \item \textbf{Immediate Action:} Enroll all employees in a foundational security awareness training course. This should be mandatory and completed within the next 30 days.
    \item \textbf{Short-Term Action:} Establish a formal security awareness program. This must include mandatory training for all new hires during onboarding and annual refresher training for all staff.
    \item \textbf{Long-Term Action:} Implement a continuous training program that includes periodic simulated phishing campaigns to measure and improve employee resilience to social engineering attacks.
\end{itemize}

\end{document}
```