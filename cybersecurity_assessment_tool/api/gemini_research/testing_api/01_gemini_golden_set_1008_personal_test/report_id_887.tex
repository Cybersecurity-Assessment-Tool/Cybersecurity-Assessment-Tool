```latex
\documentclass[12pt]{article}

% Preamble: Required Packages
\usepackage[margin=1in]{geometry}
\usepackage{pifont} % For checkmarks and crosses
\usepackage{booktabs} % For professional tables
\usepackage[hidelinks]{hyperref} % For clickable links
\usepackage{url} % For URL formatting
\usepackage{seqsplit} % For splitting long text strings
\usepackage{graphicx} % For potential logos
\usepackage{xcolor} % For colors

% Document Information
\title{Cybersecurity Posture Assessment Report}
\author{Cybersecurity Analysis Division}
\date{\today}

\begin{document}

\maketitle
\thispagestyle{empty}
\newpage

\tableofcontents
\newpage

% --- Section 1: Executive Overview ---
\section{Executive Overview}

This report details the findings of a cybersecurity posture assessment conducted for \textbf{Hidden Gem}. The assessment synthesizes data from a network infrastructure scan, a security controls questionnaire, and a review of pre-existing documented risks.

The analysis reveals a mixed security posture. While foundational controls like Multi-Factor Authentication (MFA) are implemented for email and computer access, several critical gaps exist that significantly elevate the organization's risk profile.

\textbf{Key Findings Include:}
\begin{itemize}
    \item \textbf{Critical Control Gaps:} The organization does not enforce MFA for sensitive data systems, lacks a formal Acceptable Use Policy (AUP), and does not provide mandatory annual security awareness training for all employees. These gaps present a high risk of unauthorized access, insider threat, and susceptibility to social engineering.
    \item \textbf{Technical Vulnerability Confirmation:} The network scan confirmed a pre-existing critical risk related to a service exposed on the local loopback interface (\texttt{127.0.0.1}). While this service is not directly exposed to the internet, its presence and critical severity rating warrant immediate investigation.
\end{itemize}

Immediate remediation of the identified policy and MFA-related gaps is strongly recommended to reduce the attack surface and strengthen the overall security posture. A detailed breakdown of findings and actionable recommendations is provided in the subsequent sections.

% --- Section 2: Organizational Information ---
\section{Organizational Information}

The following information was provided for the assessment. This data forms the baseline for understanding the organization's digital footprint.

\begin{table}[h!]
\centering
\begin{tabular}{@{}ll@{}}
\toprule
\textbf{Attribute} & \textbf{Value} \\ \midrule
Organization Name & \textbf{Hidden Gem} \\
Email Domain & \texttt{HiddenGem.org} \\
Website Domain & \seqsplit{\texttt{www.HiddenGem.org}} \\
External IP Address & \texttt{170.60.5.216} \\ \bottomrule
\end{tabular}
\caption{Client Organizational Details}
\label{tab:org_info}
\end{table}

% --- Section 3: Security Control Review ---
\section{Security Control Review}

This section evaluates the organization's security policies and procedures based on the provided questionnaire. "No" answers indicate significant control gaps that require immediate attention.

\begin{table}[h!]
\centering
\begin{tabular}{@{}p{0.7\textwidth}c@{}}
\toprule
\textbf{Control Question} & \textbf{Response} \\ \midrule
Do you require MFA to access email? & \ding{51} Yes \\
Do you require MFA to log into computers? & \ding{51} Yes \\
\textbf{Do you require MFA to access sensitive data systems?} & \textcolor{red}{\ding{55} No} \\
\textbf{Does your organization have an employee acceptable use policy?} & \textcolor{red}{\ding{55} No} \\
Does your organization do security awareness training for new employees? & \ding{51} Yes \\
\textbf{Does your organization do security awareness training for all employees at least once per year?} & \textcolor{red}{\ding{55} No} \\ \bottomrule
\end{tabular}
\caption{Security Controls Questionnaire Analysis}
\label{tab:controls}
\end{table}

\subsection*{Analysis of Control Gaps}
The questionnaire reveals three primary areas of concern:
\begin{enumerate}
    \item \textbf{MFA on Sensitive Systems:} The absence of MFA on systems containing sensitive data is a critical vulnerability. Should an attacker compromise a user's credentials, they would have direct access to the organization's most valuable information.
    \item \textbf{Acceptable Use Policy (AUP):} Lacking an AUP creates ambiguity regarding how employees should securely use company assets. This can lead to unintentional data exposure and makes it difficult to enforce security standards.
    \item \textbf{Annual Security Training:} While new hires receive training, the lack of an annual refresher course for all staff means that awareness of evolving threats (like new phishing techniques) diminishes over time, increasing the likelihood of a successful social engineering attack.
\end{enumerate}

% --- Section 4: Technical Scan Results ---
\section{Technical Scan Results}

A network scan was performed to identify open ports and services on the specified target. The results corroborate a known risk.

\begin{itemize}
    \item \textbf{Target IP Address:} \texttt{127.0.0.1}
    \item \textbf{Scan Tool:} Nmap
\end{itemize}

\begin{table}[h!]
\centering
\begin{tabular}{@{}llll@{}}
\toprule
\textbf{Port} & \textbf{State} & \textbf{Service (Inferred)} & \textbf{Notes} \\ \midrule
22/tcp & Open & SSH & The service is running on the local loopback interface. \\ 
& & & No version information was retrieved. This finding \\
& & & confirms the presence of the "Localhost Exposed" risk. \\ \bottomrule
\end{tabular}
\caption{Open Port Analysis}
\label{tab:scan_results}
\end{table}

% --- Section 5: Consolidated Risk Assessment ---
\section{Consolidated Risk Assessment}

This table synthesizes findings from the security questionnaire, the technical scan, and pre-existing risk documentation into a prioritized list.

\begin{table}[h!]
\centering
\resizebox{\textwidth}{!}{%
\begin{tabular}{@{}p{0.1\textwidth}p{0.25\textwidth}p{0.5\textwidth}p{0.15\textwidth}@{}}
\toprule
\textbf{Risk ID} & \textbf{Risk Name} & \textbf{Description} & \textbf{Severity} \\ \midrule
RISK-001 & Localhost Exposed & A service (inferred as SSH on port 22) is exposed on the local loopback interface. This was documented as a pre-existing risk with a CVSS score of 10.0 and has been confirmed by the technical scan. & \textbf{Critical} \\ \addlinespace
RISK-002 & Lack of MFA on Sensitive Systems & Multi-Factor Authentication is not required for accessing sensitive data systems, creating a significant risk of unauthorized access via compromised credentials. & \textbf{Critical} \\ \addlinespace
RISK-003 & Inadequate Security Awareness Training & Security awareness training is not conducted annually for all employees, increasing susceptibility to social engineering attacks like phishing. & \textbf{High} \\ \addlinespace
RISK-004 & Missing Acceptable Use Policy & The organization lacks a formal AUP, leading to ambiguity in employee responsibilities and acceptable behavior regarding IT assets. & \textbf{High} \\ \bottomrule
\end{tabular}%
}
\caption{Summary of Identified Risks}
\label{tab:risk_summary}
\end{table}

% --- Section 6: Recommendations ---
\section{Recommendations}

The following actions are recommended to mitigate the identified risks and improve the overall security posture of \textbf{Hidden Gem}. Recommendations are prioritized based on severity.

\begin{enumerate}
    \item \textbf{[Critical] Implement MFA for Sensitive Systems (RISK-002):}
    \begin{itemize}
        \item Immediately prioritize and deploy a robust MFA solution across all applications and systems classified as containing sensitive data. This is the single most effective control to prevent unauthorized access and should be treated as the top priority.
    \end{itemize}
    
    \item \textbf{[Critical] Investigate Exposed Localhost Service (RISK-001):}
    \begin{itemize}
        \item Conduct an immediate investigation into the service running on \texttt{127.0.0.1:22}.
        \item Determine the business purpose of this service. If it is not required, disable it immediately.
        \item If the service is required, ensure it is configured securely according to best practices, uses strong authentication, and is not unintentionally accessible from other network segments.
    \end{itemize}

    \item \textbf{[High] Establish an Annual Security Training Program (RISK-003):}
    \begin{itemize}
        \item Develop and mandate an annual security awareness training program for all employees, contractors, and relevant third parties.
        \item Training modules should cover modern threats, including phishing, ransomware, social engineering, and proper data handling.
    \end{itemize}

    \item \textbf{[High] Develop and Implement an Acceptable Use Policy (RISK-004):}
    \begin{itemize}
        \item Draft a comprehensive AUP that clearly defines the rules and expectations for using company networks, devices, and data.
        \item Require all employees to read and formally acknowledge the policy upon its implementation and during onboarding. Integrate AUP training into the security awareness program.
    \end{itemize}
\end{enumerate}

\end{document}
```