```latex
\documentclass[12pt]{article}

% --- PREAMBLE ---
\usepackage[margin=1in]{geometry}
\usepackage{pifont} % For checkmarks and crosses
\usepackage{booktabs} % For professional tables
\usepackage{hyperref} % For clickable links
\usepackage{url} % For URL formatting
\usepackage{seqsplit} % To split long strings in tt font
\usepackage{graphicx} % For logo (placeholder)
\usepackage{fancyhdr} % For header/footer

% --- DOCUMENT METADATA ---
\title{Cybersecurity Posture Assessment Report}
\author{Cybersecurity Analysis Division}
\date{\today}

% --- STYLING ---
\hypersetup{
    colorlinks=true,
    linkcolor=black,
    urlcolor=blue,
}
\pagestyle{fancy}
\fancyhf{}
\lhead{Vertex Solutions // Confidential}
\rhead{\thepage}
\cfoot{Cybersecurity Posture Assessment}

% --- DOCUMENT START ---
\begin{document}

\begin{titlepage}
    \centering
    \vspace*{2cm}
    {\Huge \textbf{Cybersecurity Posture Assessment Report}\par}
    \vspace{1.5cm}
    {\Large Prepared for:\par}
    \vspace{0.5cm}
    {\Huge \textbf{Vertex Solutions}\par}
    \vspace{2cm}
    {\large \today\par}
    \vfill
    {\large This document is confidential and intended solely for the use of the individual or entity to whom it is addressed.\par}
\end{titlepage}

\tableofcontents
\newpage

% --- EXECUTIVE SUMMARY ---
\section*{Executive Summary}
This report provides a comprehensive assessment of the cybersecurity posture for \textbf{Vertex Solutions}. The analysis is based on a correlation of organizational data, a security controls questionnaire, and an external network scan.

The assessment reveals several critical and high-risk security gaps. The most significant concern is the widespread lack of Multi-Factor Authentication (MFA) across all key systems, including email, endpoints, and sensitive data repositories. This exposes the organization to a high risk of account compromise and unauthorized access.

Furthermore, the absence of a formal security awareness training program for both new and existing employees presents a high risk. This leaves the organization vulnerable to social engineering attacks such as phishing, which are a primary vector for initial compromise in many security incidents.

The external network scan identified an open SSH port. While necessary for remote administration, it must be rigorously secured to prevent it from becoming an entry point for attackers.

Immediate remediation is required to address the MFA and security training deficiencies. Recommendations for these and other findings are detailed in the final section of this report.

% --- ORGANIZATIONAL INFORMATION ---
\section*{Organizational Information}
The following details were provided for the assessment.

\begin{tabular}{@{}ll}
\toprule
\textbf{Attribute} & \textbf{Value} \\
\midrule
Organization Name & \textbf{Vertex Solutions} \\
Email Domain & \texttt{VertexSolutions.net} \\
Website Domain & \url{www.VertexSolutions.net} \\
Known External IP & \texttt{34.247.55.77} \\
\bottomrule
\end{tabular}

% --- SECURITY CONTROL REVIEW ---
\section*{Security Control Review}
A review of the organization's security controls was conducted via a questionnaire. The responses indicate significant gaps in foundational security practices. A "No" answer (\ding{55}) represents a deviation from best practices and a potential security risk.

\begin{table}[h!]
\centering
\caption{Security Controls Questionnaire Results}
\begin{tabular}{@{}p{0.7\linewidth}c}
\toprule
\textbf{Control Question} & \textbf{Response} \\
\midrule
Do you require MFA to access email? & \ding{55} \\
Do you require MFA to log into computers? & \ding{55} \\
Do you require MFA to access sensitive data systems? & \ding{55} \\
Does your organization have an employee acceptable use policy? & \ding{51} \\
Does your organization do security awareness training for new employees? & \ding{55} \\
Does your organization do security awareness training for all employees at least once per year? & \ding{55} \\
\bottomrule
\end{tabular}
\end{table}

\paragraph{Analysis:} The lack of MFA on email, computers, and sensitive data systems is a critical vulnerability. The absence of a security awareness training program is a high-risk finding, as employees are the first line of defense against many cyber threats.

% --- TECHNICAL SCAN RESULTS ---
\section*{Technical Scan Results}
An external network scan was performed on the target IP address \seqsplit{\texttt{2001:db8::1}}. The scan identified the following open port.

\begin{table}[h!]
\centering
\caption{Open Port Findings}
\begin{tabular}{@{}lllll@{}}
\toprule
\textbf{Port} & \textbf{State} & \textbf{Service} & \textbf{Product/Version} & \textbf{Notes} \\
\midrule
22/tcp & open & ssh & (Not provided) & Secure Shell (SSH) for remote access. \\
\bottomrule
\end{tabular}
\end{table}

\paragraph{Analysis:} Port 22 is commonly used for SSH, which allows for secure remote administration. However, an exposed SSH service can be a target for brute-force attacks and exploitation if it is not properly configured. It is crucial to ensure the service is running an up-to-date version, configured to disallow password-based authentication in favor of public key cryptography, and that root login is disabled.

% --- RISK ASSESSMENT ---
\section*{Risk Assessment}
The following table summarizes the key risks identified by correlating the security control gaps, technical findings, and any pre-existing known risks.

\begin{table}[h!]
\centering
\caption{Summary of Identified Risks}
\begin{tabular}{@{}p{0.25\linewidth}p{0.5\linewidth}l@{}}
\toprule
\textbf{Risk Name} & \textbf{Overview} & \textbf{Severity} \\
\midrule
\textbf{Widespread Lack of MFA} & No MFA is enforced for access to email, computers, or sensitive data systems. This dramatically increases the risk of unauthorized access via compromised credentials. & \textbf{Critical} \\
\addlinespace
\textbf{Inadequate Security Awareness Program} & The absence of security training for new and existing employees leaves the organization highly susceptible to phishing, social engineering, and other human-centric attacks. & \textbf{High} \\
\addlinespace
\textbf{Potentially Insecure Remote Access} & An open SSH port was detected. Without verification of secure configuration (e.g., strong ciphers, key-based auth, patched version), it represents a potential vector for intrusion. & \textbf{Medium} \\
\addlinespace
\textbf{Pre-existing Risks} & No pre-existing risks were provided for this assessment. & N/A \\
\bottomrule
\end{tabular}
\end{table}

% --- RECOMMENDATIONS ---
\section*{Recommendations}
Based on the findings of this assessment, the following actions are recommended to mitigate the identified risks and improve the overall security posture of \textbf{Vertex Solutions}.

\subsection*{Immediate Priority (Critical Risks)}
\begin{itemize}
    \item \textbf{Implement Multi-Factor Authentication (MFA):}
    \begin{itemize}
        \item Immediately enable MFA for all user accounts across all critical systems.
        \item Prioritize email (e.g., Office 365, Google Workspace), VPN access, and access to any systems containing sensitive or regulated data.
        \item Develop a plan to roll out MFA for all computer logins.
    \end{itemize}
\end{itemize}

\subsection*{High Priority}
\begin{itemize}
    \item \textbf{Establish a Security Awareness Training Program:}
    \begin{itemize}
        \item Procure and implement a training solution that includes modules on phishing, password security, and acceptable use.
        \item Mandate that all new employees complete this training as part of their onboarding process.
        \item Mandate that all existing employees complete the training annually.
        \item Conduct regular phishing simulation campaigns to test and reinforce employee awareness.
    \end{itemize}
\end{itemize}

\subsection*{Medium Priority}
\begin{itemize}
    \item \textbf{Audit and Harden SSH Configuration:}
    \begin{itemize}
        \item Conduct an audit of the SSH service running on \seqsplit{\texttt{2001:db8::1}}.
        \item Ensure the software is patched to the latest stable version.
        \item Disable password-based authentication and enforce the use of public key cryptography.
        \item Disable direct root login via SSH.
        \item Implement firewall rules to restrict SSH access to only trusted IP addresses.
    \end{itemize}
\end{itemize}

\end{document}
```