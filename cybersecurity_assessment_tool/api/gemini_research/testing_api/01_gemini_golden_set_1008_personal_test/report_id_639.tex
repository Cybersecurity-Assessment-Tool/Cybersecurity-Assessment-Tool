```latex
\documentclass[12pt]{article}

% --- PACKAGES ---
\usepackage[margin=1in]{geometry}
\usepackage{pifont} % For checkmarks and crosses
\usepackage{booktabs} % For professional tables
\usepackage{hyperref} % For hyperlinks
\usepackage{url}      % For URL formatting
\usepackage{seqsplit} % For splitting long strings in tt font
\usepackage{xcolor}   % For colors

% --- DOCUMENT METADATA & STYLING ---
\hypersetup{
    colorlinks=true,
    linkcolor=blue,
    filecolor=magenta,      
    urlcolor=cyan,
    pdftitle={Cybersecurity Posture Assessment},
    pdfpagemode=FullScreen,
}

\newcommand{\yes}{\ding{51}}
\newcommand{\no}{\ding{55}}
\newcommand{\placeholder}[1]{\textcolor{red}{#1}}

\begin{document}

% --- TITLE PAGE ---
\begin{titlepage}
    \centering
    \vspace*{\fill}
    \huge\textbf{Cybersecurity Posture Assessment Report}
    \vspace{1.5cm}
    \Large\textbf{Prepared for: Obsidian Operatives}
    \vspace{2cm}
    \large
    \begin{tabular}{ll}
        \textbf{Report Date:} & \today \\
        \textbf{Analysis Period:} & October 2023 \\
    \end{tabular}
    \vspace*{\fill}
    \small This document contains sensitive and confidential information. Distribution is restricted to authorized personnel only.
\end{titlepage}

\tableofcontents
\newpage

% --- EXECUTIVE SUMMARY ---
\section*{1. Executive Summary}

This report provides a comprehensive cybersecurity posture assessment for \textbf{Obsidian Operatives}, based on network scans, a security controls questionnaire, and a review of known risks. The analysis reveals several critical and high-severity risks that require immediate attention to mitigate the potential for significant security incidents.

The most critical finding is the external exposure of a MySQL database service (\texttt{172.16.50.20:3306}). This service is running an outdated, End-of-Life version of MySQL (5.7.33), which no longer receives security updates. This exposure presents a direct path for attackers to compromise sensitive data.

Furthermore, significant gaps were identified in organizational security policies. The lack of mandatory Multi-Factor Authentication (MFA) for computer logins, the absence of an employee Acceptable Use Policy (AUP), and the failure to provide security awareness training to new hires create substantial vulnerabilities related to human error and credential compromise.

This report outlines these findings in detail and provides a prioritized list of actionable recommendations to strengthen the organization's overall security posture.

% --- ORGANIZATIONAL INFORMATION ---
\section*{2. Organizational Information}

The following information was provided for the assessment.

\begin{itemize}
    \item \textbf{Organization Name:} Obsidian Operatives
    \item \textbf{Email Domain:} \texttt{ObsidianOperatives.com}
    \item \textbf{Website Domain:} \url{www.ObsidianOperatives.com}
    \item \textbf{External IP Address:} \texttt{196.67.217.200}
\end{itemize}

% --- SECURITY CONTROL REVIEW ---
\section*{3. Security Control Review}
An assessment of organizational security controls was conducted via a questionnaire. The responses indicate several areas of concern where security best practices are not being met. These gaps represent significant risks to the organization.

\begin{table}[h!]
\centering
\caption{Security Controls Questionnaire Results}
\begin{tabular}{p{0.75\linewidth} c}
\toprule
\textbf{Control Question} & \textbf{Status} \\
\midrule
Do you require MFA to access email? & \yes \\
Do you require MFA to log into computers? & \textcolor{red}{\no} \\
Do you require MFA to access sensitive data systems? & \yes \\
Does your organization have an employee acceptable use policy? & \textcolor{red}{\no} \\
Does your organization do security awareness training for new employees? & \textcolor{red}{\no} \\
Does your organization do security awareness training for all employees at least once per year? & \yes \\
\bottomrule
\end{tabular}
\end{table}

% --- TECHNICAL SCAN RESULTS ---
\section*{4. Technical Scan Results}
A network scan was performed to identify open ports and exposed services on the target system. The scan confirmed the presence of an externally accessible database service.

\begin{table}[h!]
\centering
\caption{Nmap Scan Findings for Target: \texttt{172.16.50.20}}
\begin{tabular}{l l l l}
\toprule
\textbf{Port} & \textbf{State} & \textbf{Service} & \textbf{Version} \\
\midrule
3306/tcp & Open & MySQL & 5.7.33 \\
\bottomrule
\end{tabular}
\end{table}

\paragraph{Critical Finding:} The identified MySQL version, \textbf{5.7.33}, reached its official End-of-Life (EOL) in October 2023. EOL software no longer receives security patches from the vendor, making it highly susceptible to exploitation from newly discovered vulnerabilities. The combination of public exposure and EOL status constitutes a critical risk.

% --- RISK ASSESSMENT SUMMARY ---
\section*{5. Risk Assessment Summary}
The following table synthesizes findings from the security control review, technical scans, and pre-existing risk data. Risks are prioritized based on their potential impact on the organization.

\begin{table}[h!]
\centering
\caption{Consolidated Risk Register}
\begin{tabular}{p{0.25\linewidth} p{0.15\linewidth} p{0.5\linewidth}}
\toprule
\textbf{Risk Name} & \textbf{Severity} & \textbf{Overview} \\
\midrule
\textbf{Exposed End-of-Life Database} & \textbf{Critical} & The MySQL server on port 3306 is open to the network and is running an unsupported version (5.7.33). This could lead to a severe data breach, ransomware, or system compromise. \\
\addlinespace
\textbf{No MFA for Endpoint Logins} & \textbf{Critical} & The absence of MFA on computer logins means a single compromised password could grant an attacker full access to an employee's system and potentially the internal network. \\
\addlinespace
\textbf{Database Exposure} & High & This pre-existing risk is confirmed by the technical scan. The MySQL port 3306 is open to the network, violating the principle of least privilege and creating an unnecessary attack surface. \\
\addlinespace
\textbf{Missing Acceptable Use Policy} & High & Without a formal AUP, there are no defined rules for employees regarding the use of company IT assets, increasing the likelihood of unintentional security incidents. \\
\addlinespace
\textbf{No Security Training for New Hires} & High & New employees are not receiving security awareness training, making them highly susceptible to phishing and other social engineering attacks from their first day. \\
\bottomrule
\end{tabular}
\end{table}

% --- RECOMMENDATIONS ---
\section*{6. Recommendations}
The following actions are recommended to address the identified risks. They are prioritized to focus on the most critical vulnerabilities first.

\begin{enumerate}
    \item \textbf{Immediately Restrict Database Access (Critical):}
    \begin{itemize}
        \item Apply firewall rules to block all public access to port 3306 on host \texttt{172.16.50.20}.
        \item Access should be restricted to a strict whitelist of internal application servers or through a secure VPN connection.
    \end{itemize}

    \item \textbf{Immediately Enforce Endpoint MFA (Critical):}
    \begin{itemize}
        \item Procure and deploy an MFA solution for all employee computer (desktop and laptop) logins.
        \item This is a critical defense against credential theft and lateral movement within the network.
    \end{itemize}

    \item \textbf{Plan Database Upgrade (High):}
    \begin{itemize}
        \item Develop a migration plan to upgrade the MySQL 5.7.33 instance to a currently supported version (e.g., MySQL 8.x).
        \item This will ensure the service receives timely security patches.
    \end{itemize}

    \item \textbf{Develop and Implement Security Policies (High):}
    \begin{itemize}
        \item Draft and ratify a formal Employee Acceptable Use Policy (AUP) that clearly outlines the rules for using company technology and data.
        \item Integrate mandatory security awareness training into the standard onboarding process for all new employees.
    \end{itemize}
\end{enumerate}

\end{document}
```