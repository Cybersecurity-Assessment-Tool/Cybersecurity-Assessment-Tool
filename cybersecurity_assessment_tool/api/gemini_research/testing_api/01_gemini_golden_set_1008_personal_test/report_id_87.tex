```latex
\documentclass[12pt]{article}

% --- PACKAGE IMPORTS ---
\usepackage[margin=1in]{geometry}
\usepackage{pifont} % For checkmarks and crosses
\usepackage{booktabs} % For professional tables
\usepackage{hyperref} % For clickable links
\usepackage{url}      % For proper URL formatting
\usepackage{seqsplit} % For splitting long text strings
\usepackage[utf8]{inputenc}

% --- DOCUMENT METADATA ---
\title{Cybersecurity Assessment Report \\ \large For: Evergreen Alliance}
\author{Cybersecurity Analysis Division}
\date{\today}

% --- DOCUMENT START ---
\begin{document}

\maketitle
\thispagestyle{empty}
\newpage

\tableofcontents
\newpage

% ===================================================================
\section{Executive Summary}
% ===================================================================

This report provides a comprehensive analysis of the security posture of Evergreen Alliance, based on network scans, a review of existing risks, and an organizational security controls questionnaire. The assessment identified several critical and high-risk vulnerabilities that require immediate attention.

The key findings indicate significant gaps in fundamental security controls. Specifically, the complete absence of Multi-Factor Authentication (MFA) across email, workstations, and sensitive systems exposes the organization to a high risk of account compromise and unauthorized access. This is compounded by the lack of foundational security policies, such as an Acceptable Use Policy, and a failure to conduct regular security awareness training for all staff.

From a technical perspective, a critical vulnerability was identified: an outdated and publicly exposed MySQL database service. This service is running a version that has reached its end-of-life, meaning it no longer receives security updates and is likely susceptible to numerous publicly known exploits. The combination of weak access controls and a vulnerable, internet-facing database presents a severe risk of a data breach.

Immediate remediation should focus on isolating the exposed database, followed by a swift implementation of MFA and a comprehensive plan to upgrade the vulnerable software.

% ===================================================================
\section{Organizational Information}
% ===================================================================

The following information was provided for the assessment.

\begin{table}[h!]
\centering
\begin{tabular}{@{}ll@{}}
\toprule
\textbf{Attribute} & \textbf{Value} \\
\midrule
Organization Name & \textbf{Evergreen Alliance} \\
Email Domain & \texttt{EvergreenAlliance.com} \\
Website Domain & \url{www.EvergreenAlliance.com} \\
External IP Address & \texttt{55.46.187.49} \\
\bottomrule
\end{tabular}
\caption{Client Organizational Details}
\end{table}

% ===================================================================
\section{Security Control Review}
% ===================================================================

A review of the organization's security controls was conducted via a questionnaire. The responses highlight critical deficiencies in access control and security governance. A checkmark (\ding{51}) indicates a positive control, while a cross (\ding{55}) indicates a gap.

\begin{table}[h!]
\centering
\begin{tabular}{@{}p{8cm}cc@{}}
\toprule
\textbf{Control Question} & \textbf{Response} & \textbf{Assessment} \\
\midrule
Do you require MFA to access email? & No & \ding{55} Critical Gap \\
Do you require MFA to log into computers? & No & \ding{55} Critical Gap \\
Do you require MFA to access sensitive data systems? & No & \ding{55} Critical Gap \\
Does your organization have an employee acceptable use policy? & No & \ding{55} High Risk \\
Does your organization do security awareness training for new employees? & Yes & \ding{51} Control in Place \\
Does your organization do security awareness training for all employees at least once per year? & No & \ding{55} High Risk \\
\bottomrule
\end{tabular}
\caption{Security Controls Questionnaire Analysis}
\end{table}

% ===================================================================
\section{Technical Scan Results}
% ===================================================================

An external network scan was performed against the target IP address \texttt{172.16.50.20}. The scan revealed an open port exposing a critical database service to the network.

\begin{table}[h!]
\centering
\begin{tabular}{@{}llllll@{}}
\toprule
\textbf{Port} & \textbf{State} & \textbf{Service} & \textbf{Product} & \textbf{Version} & \textbf{Analyst Notes} \\
\midrule
3306 & open & mysql & MySQL & 5.7.33 & \begin{tabular}[t]{@{}l@{}}\textbf{High Risk.} Exposed database.\\ Outdated version (EOL: Oct 2023).\end{tabular} \\
\bottomrule
\end{tabular}
\caption{Nmap Scan Results for Target: \texttt{172.16.50.20}}
\end{table}

The identified MySQL version 5.7.33 is no longer supported by its developer. This "End-of-Life" (EOL) status means it does not receive security patches, making it an easy target for attackers leveraging known vulnerabilities.

% ===================================================================
\section{Consolidated Risk Assessment}
% ===================================================================

The following table synthesizes findings from the security questionnaire, technical scans, and pre-existing risk data into a prioritized list of organizational risks.

\begin{table}[h!]
\centering
\begin{tabular}{@{}p{1.5cm}p{3.5cm}p{6.5cm}l@{}}
\toprule
\textbf{Risk ID} & \textbf{Risk Name} & \textbf{Description} & \textbf{Severity} \\
\midrule
RISK-001 & Exposed and Outdated Database Service & The MySQL database on port 3306 is open to the network and is running an EOL version (5.7.33). This exposes the organization to data theft, modification, or destruction. & \textbf{Critical} \\
\addlinespace
RISK-002 & Lack of Multi-Factor Authentication & MFA is not enforced for email, computer logins, or access to sensitive data. This significantly increases the risk of unauthorized access via stolen or weak credentials. & \textbf{Critical} \\
\addlinespace
RISK-003 & Insufficient Security Policies and Training & The absence of an Acceptable Use Policy and mandatory annual security training creates a high likelihood of unintentional security incidents caused by employees. & High \\
\bottomrule
\end{tabular}
\caption{Summary of Identified Risks}
\end{table}

% ===================================================================
\section{Recommendations}
% ===================================================================

Based on the consolidated risk assessment, the following actions are recommended to mitigate the identified vulnerabilities and improve the overall security posture of Evergreen Alliance.

\subsection*{Priority 1: Immediate Actions (0-7 Days)}
\begin{enumerate}
    \item \textbf{Isolate the Database Service:} Immediately implement firewall rules to restrict all public access to TCP port 3306. Access should be limited to trusted internal IP addresses or required only through a secure Virtual Private Network (VPN).
\end{enumerate}

\subsection*{Priority 2: High-Impact Actions (1-3 Months)}
\begin{enumerate}
    \setcounter{enumi}{1} % Continue numbering
    \item \textbf{Deploy Multi-Factor Authentication (MFA):} Prioritize the rollout of MFA for all users and systems. The recommended order of implementation is:
    \begin{itemize}
        \item Email and external-facing applications.
        \item Privileged/administrator accounts.
        \item All user workstations and sensitive data systems.
    \end{itemize}
    \item \textbf{Upgrade the MySQL Database:} Plan and execute the migration of the MySQL 5.7 database to a fully supported version (e.g., MySQL 8.x). This will ensure the service receives critical security patches. This should be done in a test environment before deploying to production.
\end{enumerate}

\subsection*{Priority 3: Foundational Improvements (3-6 Months)}
\begin{enumerate}
    \setcounter{enumi}{3} % Continue numbering
    \item \textbf{Develop and Enforce Security Policies:} Draft and implement a formal Acceptable Use Policy (AUP) that all employees must read and acknowledge. This policy should define the rules for using company IT assets.
    \item \textbf{Establish Annual Security Awareness Training:} Implement a mandatory, annual security awareness training program for all employees. This program should cover topics such as phishing, password security, and social engineering to reduce human-related risk.
\end{enumerate}

\end{document}
```