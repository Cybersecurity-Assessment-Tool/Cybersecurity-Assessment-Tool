```latex
\documentclass[12pt]{article}

% --- PACKAGES ---
\usepackage[utf8]{inputenc}
\usepackage[T1]{fontenc}
\usepackage[margin=1in]{geometry}
\usepackage{pifont} % For \ding
\usepackage{booktabs} % For professional tables
\usepackage{hyperref} % For clickable links
\usepackage{url} % For URL formatting
\usepackage{seqsplit} % For splitting long strings
\usepackage{graphicx}
\usepackage{xcolor}

% --- DOCUMENT SETUP ---
\hypersetup{
    colorlinks=true,
    linkcolor=blue,
    filecolor=magenta,      
    urlcolor=cyan,
}

% --- COMMANDS ---
\newcommand{\yes}{\ding{51}} % Checkmark
\newcommand{\no}{\ding{55}}  % X-mark
\newcommand{\orgName}{\textbf{White Label}}
\newcommand{\orgDomain}{\texttt{WhiteLabel.net}}
\newcommand{\orgIP}{\texttt{148.13.150.154}}
\newcommand{\targetIP}{\texttt{172.16.50.20}}

\begin{document}

% --- TITLE PAGE ---
\begin{titlepage}
    \centering
    \vspace*{\stretch{1.0}}
    \Huge{\textbf{Cybersecurity Risk Assessment Report}}
    \vspace{1.5cm}
    \Large{\textbf{Prepared for:}} \\
    \vspace{0.5cm}
    \huge{\orgName}
    \vspace*{\stretch{2.0}}
    \large{\textbf{Report Date:}} \\
    \vspace{0.2cm}
    \large{\today}
    \vspace{1cm}
    \large{\textbf{Generated By:}} \\
    \vspace{0.2cm}
    \large{Cybersecurity Analyst}
\end{titlepage}

\tableofcontents
\newpage

% --- EXECUTIVE SUMMARY ---
\section*{Executive Summary}

This report provides a comprehensive cybersecurity risk assessment for \orgName, based on an analysis of network scan data, organizational security controls, and known risks. The assessment identified several critical and high-risk issues that require immediate attention.

The most critical finding is a publicly accessible MySQL database service on an internal host (\targetIP) running an \textbf{End-of-Life (EOL) version (5.7.33)}. This EOL status means the software no longer receives security updates, exposing the organization to numerous known and future vulnerabilities. This technical vulnerability is compounded by significant gaps in organizational policy, specifically the lack of an employee acceptable use policy and the absence of security awareness training for new hires.

These combined factors create a high-risk environment where sensitive data is directly exposed and the human element of security is underdeveloped. We strongly recommend immediate remediation of the exposed database, a plan to upgrade the EOL software, and the prompt development of foundational security policies and training programs.

% --- ORGANIZATIONAL INFORMATION ---
\section*{Organizational Information}

The following details were provided for the assessment.

\begin{table}[h!]
\centering
\begin{tabular}{@{}ll@{}}
\toprule
\textbf{Attribute} & \textbf{Value} \\ \midrule
Organization Name & \orgName \\
Email Domain & \orgDomain \\
Website Domain & \url{www.WhiteLabel.net} \\
External IP Address & \orgIP \\ \bottomrule
\end{tabular}
\caption{Client Organizational Details}
\end{table}

% --- SECURITY CONTROL REVIEW ---
\section*{Security Control Review}

A review of the organization's security controls was conducted via a questionnaire. The results highlight foundational strengths in multi-factor authentication (MFA) but reveal critical gaps in employee policy and training.

\begin{table}[h!]
\centering
\begin{tabular}{@{}lc@{}}
\toprule
\textbf{Security Control Question} & \textbf{Status} \\ \midrule
Do you require MFA to access email? & \yes \\
Do you require MFA to log into computers? & \yes \\
Do you require MFA to access sensitive data systems? & \yes \\
Does your organization have an employee acceptable use policy? & \no \\
Does your organization do security awareness training for new employees? & \no \\
Does your organization do security awareness training for all employees annually? & \yes \\ \bottomrule
\end{tabular}
\caption{Security Control Questionnaire Results}
\end{table}

\subsection*{Analysis of Controls}
The consistent implementation of MFA across email, computers, and sensitive systems is a commendable security practice. However, the absence of an \textbf{Acceptable Use Policy (AUP)} and security training for \textbf{new employees} are significant weaknesses. New hires are often a primary target for social engineering attacks, and without initial training and a clear policy on acceptable behavior, the risk of a security incident increases substantially.

% --- TECHNICAL SCAN RESULTS ---
\section*{Technical Scan Results}

A network scan was performed on the target host \targetIP. The scan identified one open port, which presents a critical security risk.

\begin{table}[h!]
\centering
\begin{tabular}{@{}lllll@{}}
\toprule
\textbf{Port} & \textbf{State} & \textbf{Service} & \textbf{Product} & \textbf{Version} \\ \midrule
3306/tcp & open & mysql & MySQL & 5.7.33 \\ \bottomrule
\end{tabular}
\caption{Open Ports Detected on \targetIP}
\end{table}

\subsection*{Critical Finding: End-of-Life Software}
The scan revealed that the MySQL service is version \textbf{5.7.33}. The MySQL 5.7 branch reached its official \textbf{End-of-Life (EOL) in October 2023}. This means it no longer receives security patches, bug fixes, or updates from the vendor. Running EOL software, especially for a critical database service, exposes the system to a wide range of publicly known vulnerabilities that will not be fixed. This finding elevates the risk of the exposed port from High to Critical.

% --- RISK ASSESSMENT SUMMARY ---
\section*{Risk Assessment Summary}

The following table synthesizes findings from the technical scan, control review, and pre-existing risk data into a prioritized list.

\begin{table}[h!]
\centering
\begin{tabular}{@{}p{0.3\linewidth}p{0.15\linewidth}p{0.45\linewidth}@{}}
\toprule
\textbf{Risk / Vulnerability} & \textbf{Severity} & \textbf{Description} \\ \midrule
\textbf{Exposed End-of-Life Database} & \textbf{Critical} & Port 3306 (MySQL 5.7.33) is open to the network. The software is EOL and no longer receives security updates, making it a prime target for exploitation. \\
\textbf{Lack of New Hire Security Training} & \textbf{High} & New employees are not provided with security awareness training, making them more susceptible to phishing and other social engineering attacks. \\
\textbf{No Employee Acceptable Use Policy} & \textbf{High} & The absence of a formal AUP creates ambiguity regarding safe and acceptable use of company assets, increasing the risk of insider threats and accidental data exposure. \\ \bottomrule
\end{tabular}
\caption{Synthesized Risk Register}
\end{table}

% --- RECOMMENDATIONS ---
\section*{Recommendations}

Based on the analysis, the following actions are recommended to mitigate the identified risks and improve the overall security posture of \orgName.

\subsection*{Immediate Actions (Critical Priority)}
\begin{enumerate}
    \item \textbf{Restrict Access to MySQL Port:} Immediately implement firewall rules to block all public access to TCP port 3306 on host \targetIP. Access should be restricted to only trusted internal IP addresses or managed exclusively through a secure VPN connection.
    \item \textbf{Plan Database Upgrade:} Develop and execute a plan to migrate the MySQL 5.7 database to a currently supported version (e.g., MySQL 8.x or a managed cloud equivalent). This is essential to ensure the service receives ongoing security patches.
\end{enumerate}

\subsection*{High Priority Actions}
\begin{enumerate}
    \item \textbf{Develop and Implement an Acceptable Use Policy (AUP):} Create a formal AUP that clearly defines the rules and expectations for employees when using company networks, systems, and data. Require all employees to read and acknowledge the policy.
    \item \textbf{Integrate Security Training into Onboarding:} Implement a mandatory security awareness training module for all new employees as part of the onboarding process. This training should cover key topics such as phishing, password security, and data handling.
\end{enumerate}

\end{document}
```