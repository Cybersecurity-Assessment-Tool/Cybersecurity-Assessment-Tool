```latex
\documentclass[12pt]{article}

% --- PACKAGES ---
\usepackage[a4paper, margin=1in]{geometry} % Page layout
\usepackage{pifont}                      % For checkmarks and crosses (\ding)
\usepackage{booktabs}                    % For professional tables
\usepackage{hyperref}                    % For hyperlinks
\usepackage{url}                         % For URL formatting
\usepackage{seqsplit}                    % For splitting long strings in texttt
\usepackage[T1]{fontenc}                 % Font encoding

% --- DOCUMENT METADATA ---
\title{Cybersecurity Posture Assessment Report}
\author{Cybersecurity Analysis Division}
\date{\today}

% --- HYPERREF SETUP ---
\hypersetup{
    colorlinks=true,
    linkcolor=black,
    urlcolor=blue,
    pdftitle={Cybersecurity Posture Assessment Report},
    pdfauthor={Cybersecurity Analysis Division},
}

% --- BEGIN DOCUMENT ---
\begin{document}

\maketitle
\hrule
\vspace{1em}

% ==============================================================================
% SECTION 1: EXECUTIVE SUMMARY
% ==============================================================================
\section*{Executive Summary}

This report provides a comprehensive cybersecurity assessment for \textbf{Orchid Isle}, based on a correlation of network scan data, organizational security controls, and pre-existing risk information. The analysis reveals several critical and high-risk findings that require immediate attention to mitigate potential security breaches.

The primary technical finding is a publicly accessible MySQL database (\texttt{172.16.50.20:3306}) running an End-of-Life (EOL) version (5.7.33). This exposes the organization to known, unpatched vulnerabilities and potential data exfiltration.

Furthermore, the review of organizational security controls identified significant gaps. The lack of mandatory Multi-Factor Authentication (MFA) for computer logins and the absence of annual security awareness training for all staff represent high-risk procedural deficiencies. These weaknesses increase the organization's susceptibility to credential theft and social engineering attacks.

This report outlines the identified risks in detail and provides actionable recommendations to strengthen the organization's overall security posture.

\vspace{2em}

% ==============================================================================
% SECTION 2: ORGANIZATIONAL INFORMATION
% ==============================================================================
\section{Organizational Information}

The following details were provided for the assessment.

\begin{itemize}
    \item \textbf{Organization Name:} \textbf{Orchid Isle}
    \item \textbf{Email Domain:} \texttt{OrchidIsle.org}
    \item \textbf{Website Domain:} \url{www.OrchidIsle.org}
    \item \textbf{External IP Address:} \texttt{58.11.54.109}
\end{itemize}

\vspace{2em}

% ==============================================================================
% SECTION 3: SECURITY CONTROL REVIEW
% ==============================================================================
\section{Security Control Review}

A review of the organization's security controls was conducted via a questionnaire. The results are summarized below. Answers marked with \ding{55} indicate a deviation from security best practices and represent a potential risk.

\begin{table}[h!]
\centering
\caption{Security Controls Questionnaire Results}
\begin{tabular}{p{0.8\linewidth} c}
\toprule
\textbf{Control Question} & \textbf{Status} \\
\midrule
Do you require MFA to access email? & \ding{51} \\
\textbf{Do you require MFA to log into computers?} & \textbf{\ding{55}} \\
Do you require MFA to access sensitive data systems? & \ding{51} \\
Does your organization have an employee acceptable use policy? & \ding{51} \\
Does your organization do security awareness training for new employees? & \ding{51} \\
\textbf{Does your organization do security awareness training for all employees at least once per year?} & \textbf{\ding{55}} \\
\bottomrule
\end{tabular}
\end{table}

\subsection*{Analysis of Gaps}
\begin{itemize}
    \item \textbf{Lack of MFA on Endpoints:} The absence of MFA for computer logins is a critical vulnerability. If an employee's credentials are stolen (e.g., through a phishing attack), an attacker could gain direct access to the network without needing a second factor of authentication.
    \item \textbf{Lack of Annual Security Training:} While new employees receive training, the lack of a mandatory annual refresher for all staff allows security knowledge to become stale. This increases the likelihood of employees falling victim to evolving threats like sophisticated phishing and social engineering campaigns.
\end{itemize}

\vspace{2em}

% ==============================================================================
% SECTION 4: TECHNICAL SCAN RESULTS
% ==============================================================================
\section{Technical Scan Results}

An external network scan was performed on the target IP address \texttt{172.16.50.20}. The scan identified the following open port and service.

\begin{table}[h!]
\centering
\caption{Nmap Scan Findings for \texttt{172.16.50.20}}
\begin{tabular}{lllll}
\toprule
\textbf{Port} & \textbf{State} & \textbf{Service} & \textbf{Product} & \textbf{Version} \\
\midrule
3306/tcp & open & mysql & MySQL & 5.7.33 \\
\bottomrule
\end{tabular}
\end{table}

\subsection*{Analysis of Findings}
\begin{itemize}
    \item \textbf{Publicly Exposed Database:} The MySQL service on port 3306 is open to the network. Databases should never be directly exposed to the internet, as this makes them a primary target for brute-force attacks, credential stuffing, and exploitation of known vulnerabilities. This finding confirms the pre-existing risk titled "Database Exposure".
    \item \textbf{End-of-Life (EOL) Software:} The identified MySQL version, \textbf{5.7.33}, reached its official End-of-Life in October 2023. This means it no longer receives security patches from the vendor, and any vulnerabilities discovered since that date will remain unpatched, posing a critical risk to the data stored within.
\end{itemize}

\vspace{2em}

% ==============================================================================
% SECTION 5: CONSOLIDATED RISK ASSESSMENT
% ==============================================================================
\section{Consolidated Risk Assessment}

The following table synthesizes findings from the security control review, technical scan, and pre-existing risk data into a consolidated list of key risks facing the organization.

\begin{table}[h!]
\centering
\caption{Summary of Identified Risks}
\begin{tabular}{p{0.3\linewidth} p{0.15\linewidth} p{0.45\linewidth}}
\toprule
\textbf{Risk Name} & \textbf{Severity} & \textbf{Description} \\
\midrule
\textbf{End-of-Life Database Software} & \textbf{Critical} & The public-facing MySQL server is running version 5.7.33, which is no longer supported and does not receive security updates. \\
\addlinespace
\textbf{Public Database Exposure} & \textbf{High} & The MySQL database port (3306) is open to the network, inviting brute-force attacks and direct exploitation attempts. \\
\addlinespace
\textbf{Insufficient Endpoint Authentication} & \textbf{High} & MFA is not enforced for computer logins, exposing the network to unauthorized access if user credentials are compromised. \\
\addlinespace
\textbf{Inadequate Security Awareness Program} & \textbf{Medium} & The lack of mandatory annual security training for all employees increases susceptibility to phishing and other social engineering attacks. \\
\bottomrule
\end{tabular}
\end{table}

\vspace{2em}

% ==============================================================================
% SECTION 6: RECOMMENDATIONS
% ==============================================================================
\section{Recommendations}

The following actions are recommended to address the identified risks and improve the overall security posture of \textbf{Orchid Isle}.

\begin{enumerate}
    \item \textbf{Mitigate Database Exposure (Immediate):}
    \begin{itemize}
        \item \textbf{Short-Term:} Implement strict firewall rules to restrict access to port 3306 on \texttt{172.16.50.20}. Access should only be permitted from trusted internal IP addresses.
        \item \textbf{Long-Term:} Place the database server in a private network segment. All administrative access should be routed through a secure Bastion Host or require a VPN connection.
    \end{itemize}
    \vspace{1em}
    \item \textbf{Upgrade End-of-Life Software (Urgent):}
    \begin{itemize}
        \item Plan and execute an upgrade of the MySQL 5.7.33 instance to a currently supported version (e.g., MySQL 8.x). This will ensure the system receives critical security patches.
    \end{itemize}
    \vspace{1em}
    \item \textbf{Enforce Multi-Factor Authentication (High Priority):}
    \begin{itemize}
        \item Procure and deploy an MFA solution for all employee computer logins (e.g., Windows Hello for Business, Duo, Okta). This is one of the most effective controls to prevent unauthorized access from stolen credentials.
    \end{itemize}
    \vspace{1em}
    \item \textbf{Establish a Continuous Security Training Program (High Priority):}
    \begin{itemize}
        \item Implement a mandatory security awareness training program that all employees must complete at least once per year.
        \item Supplement annual training with regular phishing simulations to test and reinforce employee vigilance.
    \end{itemize}
\end{enumerate}

\end{document}
```