```latex
\documentclass[12pt]{article}

% ----------------------------------------------------------------------
% PREAMBLE
% ----------------------------------------------------------------------
\usepackage[a4paper, margin=1in]{geometry}
\usepackage{pifont} % For \ding{51} (check) and \ding{55} (cross)
\usepackage{booktabs} % For professional tables (\toprule, \midrule, \bottomrule)
\usepackage{hyperref} % For clickable links and ToC
\usepackage{url}      % For formatting URLs
\usepackage{seqsplit} % To split long strings in \texttt
\usepackage{xcolor}   % For colored text
\usepackage{graphicx} % For potential logos or diagrams
\usepackage{fancyhdr} % For headers and footers

% --- Hyperref Setup ---
\hypersetup{
    colorlinks=true,
    linkcolor=blue,
    filecolor=magenta,      
    urlcolor=cyan,
    pdftitle={Cybersecurity Assessment Report},
    pdfauthor={Cybersecurity Analyst},
    pdfsubject={Security Assessment},
    pdfkeywords={Security, Report, Analysis},
    bookmarks=true,
    bookmarksopen=true
}

% --- Custom Colors for Severity ---
\definecolor{sev_critical}{RGB}{192,0,0}
\definecolor{sev_high}{RGB}{255,128,0}
\definecolor{sev_medium}{RGB}{255,192,0}

% --- Header and Footer ---
\pagestyle{fancy}
\fancyhf{} % clear all header and footer fields
\fancyhead[L]{Cybersecurity Assessment Report}
\fancyhead[R]{Top Tier}
\fancyfoot[C]{\thepage}
\renewcommand{\headrulewidth}{0.4pt}
\renewcommand{\footrulewidth}{0.4pt}

% ----------------------------------------------------------------------
% DOCUMENT START
% ----------------------------------------------------------------------
\begin{document}

% --- Title Page ---
\begin{titlepage}
    \centering
    \vspace*{1cm}
    \Huge{\textbf{Cybersecurity Assessment Report}}
    \vspace{1.5cm}
    \Large{\textbf{Prepared for:}} \\
    \vspace{0.5cm}
    \Large{Top Tier}
    \vfill
    \large{\textbf{Date of Report:}} \\
    \large{\today}
    \vspace{1.5cm}
    \large{\textbf{CONFIDENTIAL}}
\end{titlepage}

\tableofcontents
\newpage

% ----------------------------------------------------------------------
% 1. EXECUTIVE SUMMARY
% ----------------------------------------------------------------------
\section{Executive Summary}

This report provides a comprehensive cybersecurity assessment for Top Tier, based on an analysis of network scan data, organizational security controls, and pre-existing risk information. The assessment was conducted to identify vulnerabilities, evaluate the current security posture, and provide actionable recommendations to mitigate identified risks.

The analysis revealed several critical and high-severity risks that require immediate attention. Key findings include:
\begin{itemize}
    \item \textbf{Exposed End-of-Life Database:} A MySQL database server (version 5.7.33) was identified on the internal network. This version has reached its End-of-Life (EOL) as of October 2023 and no longer receives security updates, making it highly susceptible to exploitation. Its exposure on port 3306 compounds this risk.
    \item \textbf{Inadequate Multi-Factor Authentication (MFA):} MFA is not enforced for logging into computers or accessing sensitive data systems. This represents a critical security gap, significantly increasing the risk of unauthorized access via compromised credentials.
    \item \textbf{Insufficient Onboarding Security Training:} New employees do not receive security awareness training. This oversight leaves the organization vulnerable to social engineering and phishing attacks, as new hires are often prime targets.
\end{itemize}

This report details these findings and provides a prioritized list of recommendations to strengthen the organization's security posture, focusing on immediate remediation of the most critical vulnerabilities.

% ----------------------------------------------------------------------
% 2. ORGANIZATIONAL INFORMATION
% ----------------------------------------------------------------------
\section{Organizational Information}

The following details were provided for the assessment. This information helps establish the context and scope of the review.

\begin{tabular}{@{}ll}
\toprule
\textbf{Attribute} & \textbf{Value} \\
\midrule
Organization Name & Top Tier \\
Email Domain & \texttt{TopTier.org} \\
Website Domain & \seqsplit{\url{www.TopTier.org}} \\
External IP Address & \seqsplit{\texttt{2.235.188.215}} \\
\bottomrule
\end{tabular}

% ----------------------------------------------------------------------
% 3. SECURITY CONTROL REVIEW
% ----------------------------------------------------------------------
\section{Security Control Review}

A review of the organization's security controls was conducted based on a questionnaire. The results highlight significant gaps in access control and employee security training. A "No" answer indicates a deviation from security best practices and a potential risk.

\begin{tabular}{@{}p{0.6\textwidth}cc}
\toprule
\textbf{Control Question} & \textbf{Response} & \textbf{Status} \\
\midrule
Do you require MFA to access email? & Yes (\ding{51}) & \textcolor{green}{Good} \\
Do you require MFA to log into computers? & No (\ding{55}) & \textcolor{sev_critical}{Critical Gap} \\
Do you require MFA to access sensitive data systems? & No (\ding{55}) & \textcolor{sev_critical}{Critical Gap} \\
Does your organization have an employee acceptable use policy? & Yes (\ding{51}) & \textcolor{green}{Good} \\
Does your organization do security awareness training for new employees? & No (\ding{55}) & \textcolor{sev_high}{High Risk} \\
Does your organization do security awareness training for all employees at least once per year? & Yes (\ding{51}) & \textcolor{green}{Good} \\
\bottomrule
\end{tabular}

% ----------------------------------------------------------------------
% 4. TECHNICAL SCAN RESULTS
% ----------------------------------------------------------------------
\section{Technical Scan Results}

A network scan was performed on the specified target to identify open ports and running services.

\subsection{Scan Target}
\begin{itemize}
    \item \textbf{IP Address:} \texttt{172.16.50.20}
\end{itemize}

\subsection{Open Ports and Services}
The scan identified one open port, which exposes a critical database service to the network.

\begin{tabular}{@{}lllll}
\toprule
\textbf{Port} & \textbf{State} & \textbf{Service} & \textbf{Product} & \textbf{Version} \\
\midrule
3306/tcp & Open & mysql & MySQL & 5.7.33 \\
\bottomrule
\end{tabular}

\subsection{Analysis of Findings}
\subsubsection*{Critical Finding: End-of-Life Software}
The discovered MySQL server is running version \textbf{5.7.33}. This version of MySQL 5.7 reached its official End-of-Life (EOL) in \textbf{October 2023}. 

Software that has reached EOL no longer receives security patches, bug fixes, or technical support from the vendor. Any vulnerabilities discovered after the EOL date will remain unpatched, leaving the system permanently exposed to exploitation. This elevates the risk of the open port from High to Critical.

% ----------------------------------------------------------------------
% 5. CONSOLIDATED RISK ASSESSMENT
% ----------------------------------------------------------------------
\section{Consolidated Risk Assessment}

This section consolidates findings from all data sources into a prioritized list of identified risks.

\begin{tabular}{@{}p{0.3\textwidth}p{0.5\textwidth}l}
\toprule
\textbf{Risk Name} & \textbf{Overview} & \textbf{Severity} \\
\midrule
\textbf{Exposed End-of-Life Database} & A MySQL database (v5.7.33) is exposed on port 3306. The software is past its End-of-Life and no longer receives security updates. & \textcolor{sev_critical}{\textbf{Critical}} \\
\addlinespace
\textbf{Inadequate Multi-Factor Authentication} & MFA is not required for computer logins or access to sensitive data systems, making accounts highly vulnerable to credential theft and misuse. & \textcolor{sev_critical}{\textbf{Critical}} \\
\addlinespace
\textbf{Insufficient Onboarding Security Training} & New employees are not provided with security awareness training, creating a significant vulnerability to phishing and social engineering from their first day. & \textcolor{sev_high}{\textbf{High}} \\
\bottomrule
\end{tabular}

% ----------------------------------------------------------------------
% 6. RECOMMENDATIONS
% ----------------------------------------------------------------------
\section{Recommendations}

The following actions are recommended to mitigate the identified risks. They are prioritized based on severity and potential impact.

\subsection{Immediate Actions (To Be Completed within 72 Hours)}
\begin{enumerate}
    \item \textbf{Isolate the MySQL Server:} Apply strict firewall rules to the server at \texttt{172.16.50.20} to ensure that port 3306 is only accessible from specific, authorized application servers. All other access should be denied. This is a critical compensating control until the server can be upgraded.
\end{enumerate}

\subsection{High-Priority Actions (To Be Completed within 30 Days)}
\begin{enumerate}
    \setcounter{enumi}{1} % Continue numbering
    \item \textbf{Upgrade the MySQL Server:} Plan and execute an upgrade of the MySQL 5.7 server to a currently supported version (e.g., MySQL 8.0 or higher). This is essential to ensure the system receives ongoing security patches.
    \item \textbf{Enforce MFA Everywhere:} Implement and enforce MFA for all employee computer logins and for access to all systems classified as containing sensitive data. This is the single most effective control to prevent unauthorized access from compromised credentials.
\end{enumerate}

\subsection{Medium-Priority Actions (To Be Completed within 90 Days)}
\begin{enumerate}
    \setcounter{enumi}{3} % Continue numbering
    \item \textbf{Integrate Security into Onboarding:} Develop and mandate a security awareness training module as part of the new employee onboarding process. This training should cover phishing, acceptable use, password hygiene, and how to report security incidents.
\end{enumerate}

% ----------------------------------------------------------------------
% DOCUMENT END
% ----------------------------------------------------------------------
\end{document}
```