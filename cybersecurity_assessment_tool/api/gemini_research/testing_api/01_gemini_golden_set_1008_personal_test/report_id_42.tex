```latex
\documentclass[12pt]{article}

% ----------------------------------------------------------------------
% PREAMBLE
% ----------------------------------------------------------------------
\usepackage[margin=1in]{geometry}
\usepackage{pifont} % For checkmarks and crosses
\usepackage{booktabs} % For professional tables
\usepackage{hyperref} % For clickable links and ToC
\usepackage{url} % For URL formatting
\usepackage{seqsplit} % To split long strings in texttt
\usepackage{xcolor} % For colored text
\usepackage{fancyhdr} % For headers and footers
\usepackage{graphicx}
\usepackage[utf8]{inputenc}

% --- Color Definitions for Severity ---
\definecolor{sev_critical}{HTML}{940000}
\definecolor{sev_high}{HTML}{D14100}
\definecolor{sev_medium}{HTML}{DFA500}
\definecolor{sev_low}{HTML}{4183D7}
\definecolor{sev_info}{HTML}{555555}

% --- Hyperref Setup ---
\hypersetup{
    colorlinks=true,
    linkcolor=blue,
    filecolor=magenta,      
    urlcolor=cyan,
    pdftitle={Cybersecurity Posture Assessment Report},
    pdfpagemode=FullScreen,
}

% --- Header and Footer Setup ---
\pagestyle{fancy}
\fancyhf{} % clear all header and footer fields
\fancyhead[L]{\textbf{Cybersecurity Posture Assessment}}
\fancyfoot[C]{\thepage}
\renewcommand{\headrulewidth}{0.4pt}
\renewcommand{\footrulewidth}{0.4pt}

% --- Document Information ---
\title{Cybersecurity Posture Assessment Report \\ \large For: Radiant Life}
\author{Cybersecurity Analysis Division}
\date{\today}

% ----------------------------------------------------------------------
% DOCUMENT START
% ----------------------------------------------------------------------
\begin{document}

\maketitle
\thispagestyle{empty}
\newpage

\tableofcontents
\newpage

% ----------------------------------------------------------------------
% 1. EXECUTIVE OVERVIEW
% ----------------------------------------------------------------------
\section{Executive Overview}

This report details the findings of a cybersecurity posture assessment for \textbf{Radiant Life}. The assessment combines an analysis of organizational security controls, a technical network scan, and a review of pre-existing risks to provide a holistic view of the current security landscape.

The assessment identified two significant areas of concern requiring immediate attention. The most critical finding is the lack of mandatory Multi-Factor Authentication (MFA) for accessing email accounts. This exposes the organization to a high risk of business email compromise, phishing, and subsequent account takeovers. A second high-risk finding is the absence of security awareness training for new employees, leaving a critical window of vulnerability during the onboarding process.

On a positive note, the technical scan of the target host \texttt{192.168.0.5} did not reveal any open ports. This contradicts a pre-existing risk entry regarding an unencrypted web server on port 80, suggesting that this specific risk may have been remediated. This discrepancy should be validated to ensure the risk register is accurate.

Overall, \textbf{Radiant Life} has implemented several positive security controls, such as MFA for computer and sensitive system access. However, the identified gaps in email security and employee onboarding present a clear and present danger that must be addressed to improve the organization's defensive posture.

% ----------------------------------------------------------------------
% 2. ORGANIZATIONAL INFORMATION
% ----------------------------------------------------------------------
\section{Organizational Information}

The following details were provided for the assessment.

\begin{tabular}{@{}ll}
\toprule
\textbf{Attribute} & \textbf{Value} \\
\midrule
Organization Name & \textbf{Radiant Life} \\
Email Domain & \texttt{RadiantLife.net} \\
External IP Address & \texttt{173.124.255.74} \\
\bottomrule
\end{tabular}

% ----------------------------------------------------------------------
% 3. SECURITY CONTROL REVIEW
% ----------------------------------------------------------------------
\section{Security Control Review}

The following table summarizes the organization's responses to a security controls questionnaire. Items marked with \ding{55} indicate a potential gap in security posture and are discussed in the Risk Assessment section.

\begin{table}[h!]
\centering
\begin{tabular}{@{}p{0.7\textwidth}cc@{}}
\toprule
\textbf{Control Question} & \textbf{Response} & \textbf{Status} \\
\midrule
Do you require MFA to access email? & No & \textcolor{red}{\ding{55}} \\
Do you require MFA to log into computers? & Yes & \textcolor{green}{\ding{51}} \\
Do you require MFA to access sensitive data systems? & Yes & \textcolor{green}{\ding{51}} \\
Does your organization have an employee acceptable use policy? & Yes & \textcolor{green}{\ding{51}} \\
Does your organization do security awareness training for new employees? & No & \textcolor{red}{\ding{55}} \\
Does your organization do security awareness training for all employees at least once per year? & Yes & \textcolor{green}{\ding{51}} \\
\bottomrule
\end{tabular}
\caption{Security Controls Questionnaire Results}
\end{table}

% ----------------------------------------------------------------------
% 4. TECHNICAL SCAN RESULTS
% ----------------------------------------------------------------------
\section{Technical Scan Results}

A network scan was performed on the specified target to identify open ports and exposed services.

\begin{itemize}
    \item \textbf{Target IP Address:} \texttt{192.168.0.5}
    \item \textbf{Scan Date:} \today
\end{itemize}

The scan results indicate that the host is online, but no open ports were discovered. The status of scanned ports is detailed below.

\begin{table}[h!]
\centering
\begin{tabular}{@{}lllll@{}}
\toprule
\textbf{Port} & \textbf{State} & \textbf{Service} & \textbf{Product} & \textbf{Version} \\
\midrule
80/tcp & closed & http & N/A & N/A \\
\bottomrule
\end{tabular}
\caption{Scan Results for \texttt{192.168.0.5}}
\end{table}

\textbf{Analysis:} The scan shows that port 80 is closed. This contradicts the pre-existing risk data which stated "Port 80 is open." This suggests the risk has been mitigated or was based on outdated information. No other vulnerabilities were identified from this scan.

% ----------------------------------------------------------------------
% 5. RISK ASSESSMENT
% ----------------------------------------------------------------------
\section{Risk Assessment}

The following table synthesizes findings from the security control review, technical scan, and pre-existing risk data. Each risk is assigned a severity level to guide prioritization.

\begin{table}[h!]
\centering
\begin{tabular}{@{}p{0.2\textwidth}p{0.15\textwidth}p{0.55\textwidth}@{}}
\toprule
\textbf{Risk Name} & \textbf{Severity} & \textbf{Description \& Impact} \\
\midrule
\textbf{No MFA on Email} & \textcolor{sev_critical}{\textbf{Critical}} & The lack of MFA on email accounts creates a severe risk. A compromised password could lead to an account takeover, enabling attackers to conduct phishing campaigns, access sensitive data, and pivot to other internal systems. \\
\addlinespace
\textbf{No Onboarding Security Training} & \textcolor{sev_high}{\textbf{High}} & New employees are not receiving security awareness training upon being hired. This makes them highly susceptible to social engineering and phishing attacks, as they are not yet familiar with corporate security policies and procedures. \\
\addlinespace
\textbf{Unencrypted Web Server (Port 80)} & \textcolor{sev_info}{\textbf{Informational}} & A pre-existing risk stated that port 80 was open. However, our technical scan on \texttt{192.168.0.5} found this port to be \textbf{closed}. This risk appears to be remediated, but internal validation is required to confirm and update the risk register. \\
\bottomrule
\end{tabular}
\caption{Synthesized Risk Summary}
\end{table}

% ----------------------------------------------------------------------
% 6. RECOMMENDATIONS
% ----------------------------------------------------------------------
\section{Recommendations}

Based on the risk assessment, the following actions are recommended to enhance the security posture of \textbf{Radiant Life}.

\subsection{Immediate Priority (Critical)}
\begin{enumerate}
    \item \textbf{Enforce MFA for Email Access:} Immediately enable and enforce MFA for all user accounts across the \texttt{RadiantLife.net} email domain. This is the single most effective control to prevent unauthorized account access.
\end{enumerate}

\subsection{High Priority}
\begin{enumerate}
    \item \textbf{Implement Onboarding Security Training:} Develop and integrate a mandatory security awareness training module into the new employee onboarding process. This training should cover, at a minimum: phishing identification, acceptable use policies, password hygiene, and incident reporting procedures.
\end{enumerate}

\subsection{Informational / Validation}
\begin{enumerate}
    \item \textbf{Validate Port 80 Status and Update Risk Register:} Confirm internally that the service previously running on port 80 was intentionally disabled or moved. Once confirmed, update the organizational risk register to reflect that this finding is closed or remediated.
\end{enumerate}

\end{document}
```