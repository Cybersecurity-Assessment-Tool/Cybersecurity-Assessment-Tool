```latex
\documentclass[12pt]{article}

% Preamble: Required Packages
\usepackage[margin=1in]{geometry}
\usepackage{pifont} % For checkmarks and crosses
\usepackage{booktabs} % For professional tables
\usepackage{hyperref} % For clickable links
\usepackage{url} % For formatting URLs
\usepackage{seqsplit} % For splitting long strings in texttt
\usepackage{graphicx}
\usepackage{xcolor}
\usepackage{fancyhdr}

% --- Document Setup ---
\hypersetup{
    colorlinks=true,
    linkcolor=blue,
    filecolor=magenta,      
    urlcolor=cyan,
    pdftitle={Cybersecurity Posture Report},
    pdfpagemode=FullScreen,
}

\pagestyle{fancy}
\fancyhf{}
\fancyhead[L]{Cybersecurity Posture Report}
\fancyhead[R]{Crestview Analytics}
\fancyfoot[C]{\thepage}

% --- Document Start ---
\begin{document}

% --- Title Page ---
\begin{titlepage}
    \centering
    \vspace*{1cm}
    \Huge
    \textbf{Cybersecurity Posture Report}
    
    \vspace{1.5cm}
    \Large
    Prepared for: \\
    \vspace{0.5cm}
    \textbf{Crestview Analytics}
    
    \vfill
    
    \large
    Date of Report: \today
    
\end{titlepage}

\tableofcontents
\newpage

% --- Section 1: Executive Overview ---
\section{Executive Overview}
This report provides a comprehensive analysis of the cybersecurity posture for \textbf{Crestview Analytics}. The assessment is based on a correlation of network scan data, a security controls questionnaire, and a review of pre-existing risks.

The overall security posture requires immediate attention. While the organization has implemented strong multi-factor authentication (MFA) and security awareness training programs, two significant risks were identified that substantially increase the likelihood of a security incident:

\begin{enumerate}
    \item \textbf{Systemic RDP Exposure:} A network scan identified an open Remote Desktop Protocol (RDP) port on a new system (\texttt{10.10.10.51}). When correlated with a pre-existing risk for the same issue on another host (\texttt{10.10.10.50}), this indicates a systemic pattern of insecure configuration. Exposed RDP is a primary vector for ransomware attacks.
    \item \textbf{Administrative Control Gap:} The organization lacks a formal Employee Acceptable Use Policy (AUP). This foundational policy is critical for setting clear security expectations for all personnel and establishing a basis for enforcing security standards.
\end{enumerate}

This report details these findings and provides actionable recommendations to mitigate the identified risks and strengthen the organization's overall defensive capabilities.

% --- Section 2: Organizational Information ---
\section{Organizational Information}
The following details were provided for the assessment.

\begin{table}[h!]
\centering
\begin{tabular}{@{}ll@{}}
\toprule
\textbf{Attribute} & \textbf{Value} \\ \midrule
Organization Name & Crestview Analytics \\
Email Domain & \texttt{CrestviewAnalytics.com} \\
Website Domain & \seqsplit{\url{www.CrestviewAnalytics.com}} \\
External IP Address & \texttt{9.131.85.211} \\ \bottomrule
\end{tabular}
\caption{Client Organizational Details}
\label{tab:org_info}
\end{table}

% --- Section 3: Security Control Review ---
\section{Security Control Review}
A review of administrative and technical security controls was conducted via a questionnaire. The results indicate a strong implementation of authentication and training controls, but a critical gap in policy documentation.

\begin{table}[h!]
\centering
\begin{tabular}{@{}lc@{}}
\toprule
\textbf{Control Question} & \textbf{Status} \\ \midrule
Do you require MFA to access email? & \ding{51} \\
Do you require MFA to log into computers? & \ding{51} \\
Do you require MFA to access sensitive data systems? & \ding{51} \\
Does your organization have an employee acceptable use policy? & \textcolor{red}{\ding{55}} \\
Does your organization do security awareness training for new employees? & \ding{51} \\
Does your organization do security awareness training for all employees annually? & \ding{51} \\ \bottomrule
\end{tabular}
\caption{Security Controls Questionnaire Results (\ding{51}=Yes, \ding{55}=No)}
\label{tab:controls}
\end{table}

\paragraph{Analysis:} The absence of an Employee Acceptable Use Policy is a \textbf{High Risk} finding. This policy is a cornerstone of an effective security program, defining rules for the use of company assets, data handling, and online conduct. Without it, there is no formal standard to hold employees accountable for insecure actions.

% --- Section 4: Technical Scan Results ---
\section{Technical Scan Results}
An external network scan was performed to identify exposed services.

\paragraph{Scan Target:} \texttt{10.10.10.51}

\begin{table}[h!]
\centering
\begin{tabular}{@{}llll@{}}
\toprule
\textbf{Port} & \textbf{State} & \textbf{Service Name} & \textbf{Notes} \\ \midrule
3389/tcp & open & ms-wbt-server & Microsoft Remote Desktop Protocol (RDP) \\ \bottomrule
\end{tabular}
\caption{Open Ports Detected on \texttt{10.10.10.51}}
\label{tab:scan_results}
\end{table}

\paragraph{Analysis:} The scan identified that port \textbf{3389/tcp} is open. This port is used for Microsoft's Remote Desktop Protocol (RDP), which allows for direct graphical remote control of a system. Exposing RDP directly to the internet is extremely dangerous and is a leading cause of ransomware infections and unauthorized access. Attackers constantly scan the internet for open RDP ports to exploit via brute-force password attacks or known vulnerabilities.

% --- Section 5: Consolidated Risk Assessment ---
\section{Consolidated Risk Assessment}
The following table synthesizes findings from the security questionnaire, the technical scan, and pre-existing risk data to provide a unified view of the current risk landscape.

\begin{table}[h!]
\centering
\resizebox{\textwidth}{!}{%
\begin{tabular}{@{}lllll@{}}
\toprule
\textbf{ID} & \textbf{Risk Name} & \textbf{Description} & \textbf{Severity} & \textbf{Affected Systems} \\ \midrule
\textbf{RISK-001} & \textbf{Systemic RDP Exposure} & RDP is exposed on multiple internal hosts. This represents a critical, recurring configuration weakness. & \textbf{\textcolor{red}{Critical}} & \texttt{10.10.10.51} (New), \texttt{10.10.10.50} (Existing) \\
\textbf{RISK-002} & \textbf{Missing Acceptable Use Policy} & The organization lacks a formal AUP, a foundational administrative control. & \textbf{\textcolor{orange}{High}} & Organization-wide \\ \bottomrule
\end{tabular}
}
\caption{Summary of Identified Risks}
\label{tab:risk_summary}
\end{table}

% --- Section 6: Recommendations ---
\section{Recommendations}
The following actions are recommended to mitigate the identified risks. Recommendations are prioritized based on severity.

\subsection{Immediate Priority: Remediate RDP Exposure (RISK-001)}
This critical risk requires immediate action to prevent a potential breach.
\begin{itemize}
    \item \textbf{Step 1 - Investigate:} Immediately determine the business requirement for RDP access to hosts \texttt{10.10.10.51} and \texttt{10.10.10.50}.
    \item \textbf{Step 2 - Remediate (Short-Term):}
    \begin{itemize}
        \item If remote access is not required, disable the RDP service and implement a host-based firewall rule to block all traffic to port 3389.
        \item If remote access is required, implement a network firewall rule to restrict access to port 3389 from only trusted, specific IP addresses. \textbf{Do not leave it open to the entire internet.}
    \end{itemize}
    \item \textbf{Step 3 - Remediate (Long-Term):} For all remote administration, implement a Virtual Private Network (VPN) or a secure remote access gateway (e.g., bastion host). This requires users to first authenticate to a secure perimeter device before gaining access to internal resources like RDP, which is the industry best practice.
\end{itemize}

\subsection{High Priority: Implement Administrative Controls (RISK-002)}
This foundational gap should be addressed to improve security governance.
\begin{itemize}
    \item \textbf{Step 1 - Develop Policy:} Draft a formal Employee Acceptable Use Policy (AUP). This policy should clearly define rules regarding data handling, internet usage, software installation, and the use of company equipment. Templates are widely available from sources like SANS.
    \item \textbf{Step 2 - Implement \& Train:} Distribute the AUP to all current employees and integrate it into the onboarding process for new hires. Require all employees to formally read and acknowledge the policy.
\end{itemize}

\subsection{Continuous Improvement}
\begin{itemize}
    \item \textbf{Vulnerability Management:} Implement a regular, authenticated vulnerability scanning program for all internal and external assets to proactively identify misconfigurations, missing patches, and outdated software.
    \item \textbf{Configuration Management:} Develop and enforce secure configuration baselines for all new systems to ensure that services like RDP are not enabled by default.
\end{itemize}

\end{document}
```