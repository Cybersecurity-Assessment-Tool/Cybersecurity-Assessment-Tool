```latex
\documentclass[12pt]{article}

% Preamble: Required Packages
\usepackage[margin=1in]{geometry}
\usepackage{pifont} % For checkmarks and crosses
\usepackage{booktabs} % For professional tables
\usepackage{hyperref} % For clickable links
\usepackage{url} % For URL formatting
\usepackage{seqsplit} % For splitting long text strings
\usepackage{graphicx}
\usepackage{xcolor}

% Document Information
\title{Cybersecurity Posture Assessment Report}
\author{Cybersecurity Analyst}
\date{November 22, 2025}

% Hyperref Setup
\hypersetup{
    colorlinks=true,
    linkcolor=blue,
    filecolor=magenta,      
    urlcolor=cyan,
    pdftitle={Cybersecurity Posture Assessment Report},
    pdfpagemode=FullScreen,
}

\begin{document}

\maketitle
\thispagestyle{empty}
\newpage

\tableofcontents
\newpage

% ------------------------------------------------------------------
% 1. Executive Summary
% ------------------------------------------------------------------
\section{Executive Summary}

This report provides a comprehensive cybersecurity assessment for \textbf{Maple Leaf Logistics}, based on an analysis of network scan data, organizational security controls, and known risks. The assessment was conducted on November 22, 2025.

The overall security posture is considered \textbf{critically weak} and requires immediate remediation. Key findings include a complete absence of Multi-Factor Authentication (MFA) for critical systems, a lack of a formal security awareness training program, and the use of an outdated and vulnerable web server software on the network.

These deficiencies expose the organization to significant risks, including unauthorized access, data breaches, and ransomware attacks. This report details these findings and provides prioritized, actionable recommendations to mitigate the identified risks and strengthen the organization's security defenses.

% ------------------------------------------------------------------
% 2. Organizational Information
% ------------------------------------------------------------------
\section{Organizational Information}

The following details were provided for the assessment.

\begin{tabular}{@{}ll}
\toprule
\textbf{Attribute} & \textbf{Value} \\
\midrule
Organization Name & \textbf{Maple Leaf Logistics} \\
Email Domain & \seqsplit{\texttt{MapleLeafLogistics.net}} \\
Website Domain & \seqsplit{\texttt{www.MapleLeafLogistics.net}} \\
External IP Address & \seqsplit{\texttt{191.72.63.50}} \\
\bottomrule
\end{tabular}

% ------------------------------------------------------------------
% 3. Security Control Review (Questionnaire Analysis)
% ------------------------------------------------------------------
\section{Security Control Review}

A review of the organization's security controls was conducted via a questionnaire. The responses indicate critical gaps in fundamental security practices. A summary of the responses is provided in Table \ref{tab:controls}. The symbol \ding{51} denotes a "Yes" response, while \ding{55} denotes a "No" response.

\begin{table}[h!]
\centering
\caption{Organizational Security Control Responses}
\label{tab:controls}
\begin{tabular}{@{}lc}
\toprule
\textbf{Security Control Question} & \textbf{Response} \\
\midrule
Do you require MFA to access email? & \ding{55} \\
Do you require MFA to log into computers? & \ding{55} \\
Do you require MFA to access sensitive data systems? & \ding{55} \\
Does your organization have an employee acceptable use policy? & \ding{51} \\
Does your organization do security awareness training for new employees? & \ding{55} \\
Does your organization do security awareness training for all employees annually? & \ding{55} \\
\bottomrule
\end{tabular}
\end{table}

\subsection*{Analysis of Controls}
The lack of Multi-Factor Authentication (MFA) across email, computers, and sensitive data systems is a \textbf{critical vulnerability}. This significantly increases the risk of account compromise from phishing or credential theft. Furthermore, the absence of a security awareness training program for new and existing employees creates a high-risk environment where staff are more susceptible to social engineering attacks.

% ------------------------------------------------------------------
% 4. Technical Scan Results
% ------------------------------------------------------------------
\section{Technical Scan Results}

An Nmap scan was performed on the target system to identify open ports and running services.

\begin{itemize}
    \item \textbf{Target IP:} \seqsplit{\texttt{192.168.10.5}}
    \item \textbf{Scan Date:} 2025-11-22
\end{itemize}

\begin{table}[h!]
\centering
\caption{Open Ports and Services Detected}
\label{tab:nmap}
\begin{tabular}{@{}lllll}
\toprule
\textbf{Port} & \textbf{State} & \textbf{Service} & \textbf{Product} & \textbf{Version} \\
\midrule
443/tcp & open & https & nginx & 1.18.0 \\
\bottomrule
\end{tabular}
\end{table}

\subsection*{Analysis of Technical Findings}
The scan identified an Nginx web server running on port 443 (HTTPS). Two significant issues were identified:

\begin{enumerate}
    \item \textbf{Outdated Software:} The detected Nginx version is \textbf{1.18.0}, which was released in April 2020. This version is severely outdated and is known to be vulnerable to multiple high-severity security flaws, including CVE-2021-23017. Running this version poses a high risk of server compromise.
    
    \item \textbf{SSL Certificate Mismatch:} The SSL certificate presented by the server is for the common name \texttt{www.acme-corp.com}, which does not match the organization's domain (\texttt{www.MapleLeafLogistics.net}). This misconfiguration can cause trust errors for users and may indicate a testing environment has been improperly exposed.
\end{enumerate}

% ------------------------------------------------------------------
% 5. Risk Assessment Summary
% ------------------------------------------------------------------
\section{Risk Assessment Summary}

The following table summarizes the key risks identified by correlating the security control gaps and technical findings. No pre-existing vulnerabilities were reported.

\begin{table}[h!]
\centering
\caption{Identified Risks}
\label{tab:risks}
\begin{tabular}{@{}p{0.3\textwidth}p{0.15\textwidth}p{0.45\textwidth}}
\toprule
\textbf{Risk Name} & \textbf{Severity} & \textbf{Overview} \\
\midrule
Widespread Lack of MFA & \textbf{Critical} & The absence of MFA for email, endpoints, and sensitive systems makes the organization highly susceptible to account takeovers via credential theft. \\
\addlinespace
Outdated Web Server & \textbf{High} & The Nginx server (v1.18.0) is outdated and has multiple publicly known vulnerabilities, creating a high likelihood of remote exploitation. \\
\addlinespace
No Security Awareness Training & \textbf{High} & Employees are not trained to recognize or report phishing and other social engineering attacks, making them a primary target for attackers. \\
\addlinespace
SSL Certificate Mismatch & \textbf{Medium} & The web server presents an invalid certificate, which erodes user trust and could be indicative of a larger server misconfiguration. \\
\bottomrule
\end{tabular}
\end{table}

% ------------------------------------------------------------------
% 6. Recommendations
% ------------------------------------------------------------------
\section{Recommendations}

The following actions are recommended to mitigate the identified risks. Recommendations are prioritized by severity.

\begin{enumerate}
    \item \textbf{[Critical] Implement Multi-Factor Authentication (MFA):}
    Immediately deploy a mandatory MFA policy for all employees. Prioritize implementation for:
    \begin{itemize}
        \item Email access (e.g., Office 365, Google Workspace).
        \item Remote access and VPN solutions.
        \item Access to all systems containing sensitive or critical data.
    \end{itemize}

    \item \textbf{[High] Upgrade Nginx Web Server:}
    Plan and execute an upgrade of the Nginx server at \seqsplit{\texttt{192.168.10.5}} from version 1.18.0 to the latest stable version. This will patch critical vulnerabilities and improve performance and stability.
    
    \item \textbf{[High] Establish a Security Awareness Training Program:}
    Develop and implement a mandatory security awareness training program. This program should include:
    \begin{itemize}
        \item Onboarding training for all new employees.
        \item Annual refresher training for all staff.
        \item Regular phishing simulations to test and reinforce learning.
    \end{itemize}

    \item \textbf{[Medium] Correct SSL Certificate Configuration:}
    Investigate the SSL certificate mismatch on the web server. Procure and install a valid SSL certificate that matches the organization's domain, \seqsplit{\texttt{www.MapleLeafLogistics.net}}, to ensure secure and trusted communication.
\end{enumerate}

% ------------------------------------------------------------------
% 7. Conclusion
% ------------------------------------------------------------------
\section{Conclusion}

The assessment reveals critical deficiencies in the security posture of \textbf{Maple Leaf Logistics}. The combination of missing foundational controls like MFA and security training, coupled with a vulnerable public-facing server, places the organization at an unacceptably high risk of a significant cyber incident.

We strongly urge management to allocate the necessary resources to implement the recommendations outlined in this report. Prompt action is essential to protect the organization's data, assets, and reputation.

\end{document}
```