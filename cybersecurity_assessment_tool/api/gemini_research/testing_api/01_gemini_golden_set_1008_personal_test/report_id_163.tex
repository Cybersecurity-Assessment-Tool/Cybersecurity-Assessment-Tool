```latex
\documentclass[12pt]{article}

% Preamble: Required Packages
\usepackage[margin=1in]{geometry}
\usepackage{pifont} % For checkmarks and crosses
\usepackage{booktabs} % For professional tables
\usepackage{hyperref} % For clickable links
\usepackage{url} % For formatting URLs
\usepackage{seqsplit} % For splitting long strings like IPs
\usepackage{graphicx}
\usepackage{fancyhdr}
\usepackage[utf8]{inputenc}

% Document Metadata
\title{Cybersecurity Posture Assessment Report}
\author{Cybersecurity Analysis Division}
\date{\today}

% Header and Footer
\pagestyle{fancy}
\fancyhf{}
\fancyhead[L]{Harbor Light Foundation // Confidential}
\fancyfoot[C]{\thepage}

\begin{document}

\maketitle
\thispagestyle{empty}
\newpage

\tableofcontents
\newpage

% --- 1. Executive Overview ---
\section{Executive Overview}
This report details the findings of a cybersecurity posture assessment for \textbf{Harbor Light Foundation}. The assessment combines an analysis of organizational security controls, a technical network scan, and a review of pre-existing risks.

The overall security posture is considered weak and requires immediate attention. The primary areas of concern are a complete lack of Multi-Factor Authentication (MFA) across all critical systems and the absence of a formal Acceptable Use Policy (AUP). These administrative gaps are compounded by a technical finding: an externally exposed Secure Shell (SSH) service on an IPv6 address. This combination presents a significant risk of unauthorized access and system compromise.

On a positive note, the organization has implemented a security awareness training program for both new and existing employees, which establishes a foundational level of security culture.

Key recommendations focus on the immediate implementation of MFA, securing the exposed SSH service, and developing foundational security policies.

% --- 2. Organizational Information ---
\section{Organizational Information}
The following information was provided for the assessment.

\begin{itemize}
    \item \textbf{Organization Name:} Harbor Light Foundation
    \item \textbf{Email Domain:} \texttt{HarborLightFoundation.net}
    \item \textbf{Website Domain:} \url{www.HarborLightFoundation.net}
    \item \textbf{Primary External IP:} \texttt{46.64.107.141}
\end{itemize}

% --- 3. Security Control Review ---
\section{Security Control Review}
A review of administrative security controls was conducted based on a standardized questionnaire. The results indicate critical gaps in access control and governance policies.

\begin{table}[h!]
\centering
\caption{Organizational Security Control Assessment}
\begin{tabular}{p{0.6\linewidth} c l}
\toprule
\textbf{Control Question} & \textbf{Response} & \textbf{Assessment} \\
\midrule
Do you require MFA to access email? & \ding{55} & \textbf{Critical Gap} \\
Do you require MFA to log into computers? & \ding{55} & \textbf{Critical Gap} \\
Do you require MFA to access sensitive data systems? & \ding{55} & \textbf{Critical Gap} \\
\addlinespace
Does your organization have an employee acceptable use policy? & \ding{55} & \textbf{High Risk} \\
\addlinespace
Does your organization do security awareness training for new employees? & \ding{51} & Best Practice Met \\
Does your organization do security awareness training for all employees at least once per year? & \ding{51} & Best Practice Met \\
\bottomrule
\end{tabular}
\end{table}

\noindent \textbf{Note:} A response of \ding{51} (Yes) indicates the control is in place, while \ding{55} (No) indicates a control gap that introduces risk.

% --- 4. Technical Scan Results ---
\section{Technical Scan Results}
A network scan was performed on the specified target to identify open ports and exposed services. The scan date was not provided in the source data.

\begin{itemize}
    \item \textbf{Target IP Address:} \seqsplit{\texttt{2001:db8::1}}
\end{itemize}

The following open ports were discovered on the target host:

\begin{table}[h!]
\centering
\caption{Open Port Analysis}
\begin{tabular}{c c p{0.4\linewidth} p{0.4\linewidth}}
\toprule
\textbf{Port} & \textbf{State} & \textbf{Inferred Service} & \textbf{Notes} \\
\midrule
22/TCP & Open & SSH (Secure Shell) & Direct exposure of a remote administration protocol. This is a primary target for automated brute-force and credential stuffing attacks. \\
\bottomrule
\end{tabular}
\end{table}

\noindent \textbf{Analysis:} The presence of an open SSH port is a significant finding. When combined with the lack of MFA, it creates a high-risk entry point for attackers. An adversary who compromises a single user's credentials could potentially gain direct administrative access to this system.

% --- 5. Correlated Risk Assessment ---
\section{Correlated Risk Assessment}
This section synthesizes the findings from the security control review and the technical scan into a prioritized list of risks. No pre-existing vulnerabilities were reported.

\begin{table}[h!]
\centering
\caption{Summary of Identified Risks}
\begin{tabular}{p{0.15\linewidth} p{0.65\linewidth} c}
\toprule
\textbf{Risk Name} & \textbf{Description} & \textbf{Severity} \\
\midrule
\textbf{Lack of MFA} & The absence of MFA for email, computer, and sensitive data access significantly increases the risk of unauthorized access from compromised credentials. This is exacerbated by the externally exposed SSH service. & \textbf{Critical} \\
\addlinespace
\textbf{Exposed SSH Service} & The SSH service on port 22 is open on an external IPv6 address. Without strong controls like MFA, IP whitelisting, and key-based authentication, this service is highly vulnerable to attack. & \textbf{High} \\
\addlinespace
\textbf{Missing Acceptable Use Policy} & The lack of a formal AUP means employees are not provided with clear guidelines on the secure use of company assets, increasing the risk of insider threats and unintentional security incidents. & \textbf{Medium} \\
\bottomrule
\end{tabular}
\end{table}

% --- 6. Recommendations ---
\section{Recommendations}
The following actionable recommendations are provided to mitigate the identified risks and improve the overall security posture of \textbf{Harbor Light Foundation}.

\subsection{Immediate Priority (Critical)}
\begin{enumerate}
    \item \textbf{Implement Multi-Factor Authentication (MFA):} Enforce MFA across all critical systems immediately.
    \begin{itemize}
        \item Prioritize email (e.g., Office 365, Google Workspace).
        \item Prioritize all remote access solutions, including the identified SSH service.
        \item Extend MFA to all systems handling sensitive data.
    \end{itemize}
\end{enumerate}

\subsection{High Priority}
\begin{enumerate}
    \setcounter{enumi}{1}
    \item \textbf{Secure the Exposed SSH Service:}
    \begin{itemize}
        \item If external access is required, restrict it to known, trusted IP addresses (IP whitelisting).
        \item Disable password-based authentication and enforce the use of public key cryptography.
        \item If external access is not required, block port 22 at the network firewall.
    \end{itemize}
    \item \textbf{Develop and Implement an Acceptable Use Policy (AUP):}
    \begin{itemize}
        \item Create a comprehensive AUP that defines rules for computer, network, and data usage.
        \item Ensure all new and existing employees read and formally acknowledge the policy.
    \end{itemize}
\end{enumerate}

\subsection{Medium Priority}
\begin{enumerate}
    \setcounter{enumi}{3}
    \item \textbf{Conduct a Comprehensive Vulnerability Assessment:}
    \begin{itemize}
        \item The initial scan was limited. A full, authenticated vulnerability scan should be conducted on all external and internal systems to identify outdated software and specific misconfigurations.
    \end{itemize}
\end{enumerate}

\end{document}
```