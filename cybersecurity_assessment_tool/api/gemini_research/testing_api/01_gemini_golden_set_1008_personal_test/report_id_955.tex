```latex
\documentclass[12pt]{article}

% Preamble: Required Packages
\usepackage[margin=1in]{geometry}
\usepackage{pifont} % For checkmarks and crosses
\usepackage{booktabs} % For professional tables
\usepackage{hyperref} % For clickable links
\usepackage{url} % For URL formatting
\usepackage{seqsplit} % For splitting long strings like IPs

% Document Metadata
\title{Cybersecurity Posture Assessment Report}
\author{Cybersecurity Analysis Division}
\date{\today}

% Hyperref Setup
\hypersetup{
    colorlinks=true,
    linkcolor=black,
    urlcolor=blue,
    pdftitle={Cybersecurity Posture Assessment Report},
    pdfauthor={Cybersecurity Analysis Division},
}

\begin{document}

\maketitle

\begin{abstract}
\noindent This report provides a comprehensive cybersecurity assessment for \textbf{Tidal Wave Sports}. The analysis is based on a synthesis of network scan data, an organizational security questionnaire, and a review of pre-existing risks. The assessment identifies the organization's security strengths and weaknesses, culminating in a prioritized list of actionable recommendations designed to enhance its overall security posture.
\end{abstract}

\tableofcontents
\newpage

\section{Overview and Executive Summary}

This assessment was conducted to evaluate the current cybersecurity posture of \textbf{Tidal Wave Sports}. The evaluation combined technical scanning of external infrastructure with a review of internal security controls and policies.

\paragraph{Key Findings:} The organization demonstrates a solid foundation in security awareness and endpoint protection, with established policies for acceptable use and mandatory Multi-Factor Authentication (MFA) for email and computer access. However, two primary areas of risk were identified:

\begin{itemize}
    \item \textbf{Critical Control Gap:} A significant gap exists in the application of MFA to sensitive data systems. This oversight exposes critical assets to increased risk of unauthorized access.
    \item \textbf{Network Exposure:} The external network scan identified an open Secure Shell (SSH) port. While necessary for remote administration, public exposure requires stringent hardening to prevent it from becoming an attack vector.
\end{itemize}

\paragraph{Conclusion:} While the organization has implemented several key security controls effectively, the identified risks should be addressed with priority to mitigate potential threats. Recommendations are provided in Section \ref{sec:recommendations} to guide remediation efforts.

\section{Organizational Information}

The following details were provided for the assessment.

\begin{itemize}
    \item \textbf{Organization Name:} Tidal Wave Sports
    \item \textbf{Primary Email Domain:} \texttt{TidalWaveSports.org}
    \item \textbf{Website Domain:} \url{www.TidalWaveSports.org}
    \item \textbf{External IP Address:} \texttt{84.159.176.220}
\end{itemize}

\section{Security Control Review}

The following table summarizes the organization's responses to a security controls questionnaire. This review provides insight into the documented policies and procedures currently in place. A checkmark (\ding{51}) indicates an affirmative response (control implemented), while a cross (\ding{55}) indicates a negative response (control gap).

\begin{table}[h!]
\centering
\caption{Security Controls Questionnaire Results}
\label{tab:controls}
\begin{tabular}{p{0.8\linewidth} c}
\toprule
\textbf{Control Question} & \textbf{Status} \\
\midrule
Do you require MFA to access email? & \ding{51} \\
Do you require MFA to log into computers? & \ding{51} \\
\textbf{Do you require MFA to access sensitive data systems?} & \textbf{\color{red}\ding{55}} \\
Does your organization have an employee acceptable use policy? & \ding{51} \\
Does your organization do security awareness training for new employees? & \ding{51} \\
Does your organization do security awareness training for all employees at least once per year? & \ding{51} \\
\bottomrule
\end{tabular}
\end{table}

\paragraph{Analysis:} The questionnaire reveals a significant control gap. The failure to enforce MFA on sensitive data systems constitutes a high-risk vulnerability. An attacker who compromises a user's credentials could gain direct access to the organization's most critical information assets without needing a second authentication factor.

\section{Technical Scan Results}

An external network scan was performed to identify open ports and exposed services on the organization's public-facing infrastructure.

\begin{itemize}
    \item \textbf{Target IP Address:} \seqsplit{\texttt{2001:db8::1}}
    \item \textbf{Scan Date:} \today
\end{itemize}

The following table details the services discovered during the scan.

\begin{table}[h!]
\centering
\caption{Open Ports Detected on Target IP}
\label{tab:scanresults}
\begin{tabular}{c c l}
\toprule
\textbf{Port} & \textbf{State} & \textbf{Inferred Service} \\
\midrule
22/TCP & Open & SSH (Secure Shell) \\
\bottomrule
\end{tabular}
\end{table}

\paragraph{Analysis:} The scan identified that port 22 is open, which is standard for the SSH protocol used for secure remote system administration. While a common and necessary service, its public exposure presents a security risk. It is a frequent target for automated brute-force attacks and can be vulnerable if not configured securely or if outdated software versions are in use. Detailed service and version information was not available from this scan.

\section{Consolidated Risk Assessment}

This section correlates findings from the security control review, technical scan, and pre-existing risk data. The following risks have been identified and prioritized based on their potential impact and likelihood.

\begin{table}[h!]
\centering
\caption{Identified Security Risks}
\label{tab:risks}
\begin{tabular}{p{0.1\linewidth} p{0.3\linewidth} p{0.15\linewidth} p{0.35\linewidth}}
\toprule
\textbf{ID} & \textbf{Risk Name} & \textbf{Severity} & \textbf{Description} \\
\midrule
RISK-001 & Lack of MFA on Sensitive Systems & \textbf{High} & The absence of mandatory MFA for sensitive data systems creates a critical vulnerability. Compromised credentials could lead directly to a significant data breach. \\
\addlinespace
RISK-002 & Publicly Exposed SSH Service & \textbf{Medium} & The SSH service is exposed to the internet. Without proper hardening, it is susceptible to brute-force attacks, credential stuffing, and exploitation of potential software vulnerabilities. \\
\bottomrule
\end{tabular}
\end{table}

\section{Recommendations}
\label{sec:recommendations}

The following actionable recommendations are provided to address the identified risks and improve the overall security posture of \textbf{Tidal Wave Sports}.

\subsection{RISK-001: Lack of MFA on Sensitive Systems (High)}

\begin{itemize}
    \item \textbf{Immediate Action:} Prioritize and implement a mandatory MFA policy for all user accounts (including administrative and service accounts) that have access to systems containing sensitive or critical data.
    \item \textbf{Strategic Action:} Conduct a data classification review to ensure all systems storing sensitive data are correctly identified and fall under the new MFA policy.
\end{itemize}

\subsection{RISK-002: Publicly Exposed SSH Service (Medium)}

\begin{itemize}
    \item \textbf{Immediate Action:} Enforce strong configuration settings for the SSH service. This includes:
    \begin{itemize}
        \item Disabling password-based authentication and requiring public key cryptography.
        \item Disabling direct root login (`PermitRootLogin no`).
        \item Implementing an intrusion detection/prevention tool like \texttt{fail2ban} to automatically block IPs that exhibit malicious behavior (e.g., repeated failed login attempts).
    \end{itemize}
    \item \textbf{Strategic Action:} Review the business necessity for exposing SSH to the public internet. If possible, restrict access to a trusted set of source IP addresses (e.g., via a firewall) or place the service behind a Virtual Private Network (VPN).
\end{itemize}

\end{document}
```