```latex
\documentclass[12pt]{article}

% 1. Required Packages
\usepackage[margin=1in]{geometry}
\usepackage{pifont} % For checkmarks and crosses
\usepackage{booktabs} % For professional tables
\usepackage{hyperref} % For clickable links
\usepackage{url} % For URL formatting
\usepackage{seqsplit} % For splitting long strings to prevent overflow
\usepackage{graphicx}
\usepackage{xcolor}

% 2. Document Setup
\hypersetup{
    colorlinks=true,
    linkcolor=blue,
    filecolor=magenta,      
    urlcolor=cyan,
    pdftitle={Cybersecurity Assessment Report},
    pdfpagemode=FullScreen,
}

\newcommand{\yes}{\ding{51}}
\newcommand{\no}{\ding{55}}

\begin{document}

% 3. Title Page
\begin{titlepage}
    \centering
    \vspace*{1cm}
    \Huge{\textbf{Cybersecurity Assessment Report}}
    \vspace{1.5cm}
    \Large{\textbf{Prepared for:}} \\
    \vspace{0.5cm}
    \Large{White Label}
    \vspace{2cm}
    \large{\textbf{Date of Report:}} \\
    \vspace{0.5cm}
    \large{\today}
    \vfill
    \large{\textbf{Generated by:}} \\
    \vspace{0.5cm}
    \large{Cybersecurity Analyst}
\end{titlepage}

\tableofcontents
\newpage

% 4. Executive Summary
\section{Executive Summary}
This report provides a cybersecurity assessment for White Label, based on an analysis of organizational data, a network scan, and a review of pre-existing risks. The assessment was conducted on \today.

The organization demonstrates a solid foundation in security policy and awareness, with an established acceptable use policy and regular security training for all employees. These are commendable proactive measures.

However, a critical security gap was identified: the complete absence of Multi-Factor Authentication (MFA) for accessing email, computers, and sensitive data systems. This represents a significant and immediate risk, as it leaves the organization highly vulnerable to account compromise through credential theft or phishing attacks.

On a positive note, a technical network scan of the target host \texttt{192.168.0.5} revealed that a previously identified risk, an unencrypted web server on port 80, has been mitigated. The scan confirmed that this port is now closed, effectively resolving the vulnerability.

Our primary recommendation is the immediate implementation of a robust MFA solution across all critical systems to address the most severe risk to the organization.

% 5. Organizational Information
\section{Organizational Information}
The following details were provided for the assessment.

\begin{tabular}{@{}ll}
\toprule
\textbf{Attribute} & \textbf{Value} \\
\midrule
Organization Name & White Label \\
Email Domain & \seqsplit{\texttt{WhiteLabel.net}} \\
Website Domain & \seqsplit{\url{www.WhiteLabel.net}} \\
External IP Address & \texttt{87.21.122.4} \\
\bottomrule
\end{tabular}

% 6. Security Control Review (from Questionnaire)
\section{Security Control Review}
The following table summarizes the organization's responses to a security controls questionnaire. "No" answers indicate significant gaps that increase organizational risk.

\begin{tabular}{@{}p{0.7\textwidth}c}
\toprule
\textbf{Control Question} & \textbf{Status} \\
\midrule
Do you require MFA to access email? & \textcolor{red}{\no} \\
Do you require MFA to log into computers? & \textcolor{red}{\no} \\
Do you require MFA to access sensitive data systems? & \textcolor{red}{\no} \\
Does your organization have an employee acceptable use policy? & \textcolor{green}{\yes} \\
Does your organization do security awareness training for new employees? & \textcolor{green}{\yes} \\
Does your organization do security awareness training for all employees at least once per year? & \textcolor{green}{\yes} \\
\bottomrule
\end{tabular}

\subsection*{Analysis}
The lack of MFA for email, computer logins, and sensitive data access is a critical vulnerability. This single point of failure for authentication exposes the organization to significant risks, including business email compromise, ransomware, and data breaches. The positive controls around policy and training are excellent but are not sufficient to mitigate the risks associated with weak authentication.

% 7. Technical Scan Results
\section{Technical Scan Results}
An external network scan was performed to identify open ports and exposed services.

\begin{itemize}
    \item \textbf{Target IP Address:} \texttt{192.168.0.5}
    \item \textbf{Scan Date:} \today
    \item \textbf{Scanner Used:} Nmap
\end{itemize}

The scan results for the target host are summarized below.
\begin{tabular}{@{}llll}
\toprule
\textbf{Port} & \textbf{State} & \textbf{Service} & \textbf{Notes} \\
\midrule
80/tcp & closed & http & The port is not accessible from the scanner's location. \\
\bottomrule
\end{tabular}

\subsection*{Analysis}
The technical scan revealed a very limited attack surface, with no open ports detected on the target system. Specifically, port 80 (HTTP) was found to be closed. This finding directly contradicts a previously identified risk, indicating that remediation has occurred.

% 8. Risk Assessment
\section{Risk Assessment}
This section correlates findings from the security control review, technical scan, and pre-existing risk data.

\begin{tabular}{@{}p{0.25\textwidth}p{0.5\textwidth}p{0.15\textwidth}}
\toprule
\textbf{Risk Name} & \textbf{Overview} & \textbf{Severity} \\
\midrule
\textbf{Lack of Multi-Factor Authentication} & User accounts for email, endpoints, and sensitive systems are protected only by passwords. This creates a high risk of unauthorized access via credential theft or brute-force attacks. & \textbf{Critical} \\
\addlinespace
\textbf{Unencrypted Web Server} & \textit{(From previous assessment)} Port 80 was believed to be open, exposing unencrypted web traffic. The current scan confirms this port is now \textbf{closed}. & \textbf{Mitigated} \\
\bottomrule
\end{tabular}

% 9. Recommendations
\section{Recommendations}
Based on the analysis, the following actions are recommended to improve the organization's security posture.

\subsection{Immediate Priority (Critical Risk)}
\begin{itemize}
    \item \textbf{Implement Multi-Factor Authentication (MFA):}
    \begin{itemize}
        \item \textbf{Action:} Procure and deploy an MFA solution across the organization.
        \item \textbf{Priority 1:} Enable MFA for all email accounts (e.g., via Microsoft 365, Google Workspace).
        \item \textbf{Priority 2:} Enforce MFA for all remote access systems (e.g., VPN) and administrative accounts.
        \item \textbf{Priority 3:} Roll out MFA for all employee computer logins.
        \item \textbf{Justification:} This is the single most effective control to prevent unauthorized account access and mitigate the risk of business email compromise and subsequent attacks.
    \end{itemize}
\end{itemize}

\subsection{Risk Management}
\begin{itemize}
    \item \textbf{Update Risk Register:}
    \begin{itemize}
        \item \textbf{Action:} Formally mark the "Unencrypted Web Server" risk as "Resolved" or "Mitigated" in the organization's risk register.
        \item \textbf{Justification:} The technical scan validates that the underlying vulnerability (open port 80) no longer exists. This ensures the risk register is accurate and reflects the current security posture.
    \end{itemize}
\end{itemize}

\subsection{Continuous Improvement}
\begin{itemize}
    \item \textbf{Maintain Security Awareness Program:}
    \begin{itemize}
        \item \textbf{Action:} Continue the existing program of security awareness training for new and existing employees.
        \item \textbf{Justification:} A well-informed user base remains a critical layer of defense against phishing and social engineering, complementing technical controls like MFA.
    \end{itemize}
\end{itemize}

\end{document}
```