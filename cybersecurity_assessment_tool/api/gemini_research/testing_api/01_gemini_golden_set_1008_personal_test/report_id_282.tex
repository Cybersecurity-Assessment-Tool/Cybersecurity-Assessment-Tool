```latex
\documentclass[12pt]{article}

% --- PACKAGES ---
\usepackage[margin=1in]{geometry}
\usepackage{pifont} % For checkmarks and crosses
\usepackage{booktabs} % For professional tables
\usepackage{hyperref} % For clickable links
\usepackage{url} % For URL formatting
\usepackage{seqsplit} % For splitting long strings in tt font
\usepackage[T1]{fontenc}

% --- DOCUMENT METADATA ---
\title{Cybersecurity Posture Assessment Report}
\author{Cybersecurity Analysis Division}
\date{\today}

\hypersetup{
    colorlinks=true,
    linkcolor=black,
    urlcolor=blue,
    pdftitle={Cybersecurity Posture Assessment Report},
    pdfauthor={Cybersecurity Analysis Division},
    pdfsubject={Security Assessment},
    pdfkeywords={Security, Risk, Assessment}
}

% --- BEGIN DOCUMENT ---
\begin{document}

\maketitle
\thispagestyle{empty}
\newpage

\tableofcontents
\newpage

% --- SECTION 1: EXECUTIVE OVERVIEW ---
\section{Executive Overview}
This report details the findings of a cybersecurity posture assessment conducted for \textbf{Iron River Finance}. The assessment combines an analysis of organizational security controls, a technical network scan, and a review of pre-existing risk data.

The overall security posture presents significant organizational and policy-based risks. The most critical deficiencies identified are the lack of Multi-Factor Authentication (MFA) for email and computer access, and the complete absence of an employee security awareness training program and an acceptable use policy. These gaps expose the organization to a high risk of phishing, business email compromise, and unauthorized access.

On a positive note, a technical scan of the target host \texttt{192.168.0.5} indicates that a previously identified risk, an unencrypted web server on port 80, appears to have been remediated, as the port was found to be closed.

Recommendations in this report are prioritized to address the most critical vulnerabilities first. Immediate focus should be placed on implementing MFA and establishing a foundational security awareness program to mitigate the most probable and impactful threats.

% --- SECTION 2: ORGANIZATIONAL INFORMATION ---
\section{Organizational Information}
The following information was provided for the assessment.

\begin{tabular}{@{}ll}
    \toprule
    \textbf{Attribute} & \textbf{Value} \\
    \midrule
    Organization Name & \textbf{Iron River Finance} \\
    Email Domain & \texttt{IronRiverFinance.net} \\
    Website Domain & \url{www.IronRiverFinance.net} \\
    External IP Address & \seqsplit{\texttt{170.216.241.140}} \\
    \bottomrule
\end{tabular}

% --- SECTION 3: SECURITY CONTROL REVIEW ---
\section{Security Control Review}
An analysis of the organization's security questionnaire reveals several critical gaps in foundational security controls. The following table summarizes the responses and provides a brief assessment of the associated risk.

\begin{table}[h!]
\centering
\begin{tabular}{@{}p{0.5\textwidth}cp{0.3\textwidth}@{}}
    \toprule
    \textbf{Control Question} & \textbf{Response} & \textbf{Assessment} \\
    \midrule
    Do you require MFA to access email? & \ding{55} & \textbf{Critical Gap.} High risk of account takeover and business email compromise. \\
    \addlinespace
    Do you require MFA to log into computers? & \ding{55} & \textbf{High Risk.} Weakens endpoint security and allows for easier lateral movement if credentials are stolen. \\
    \addlinespace
    Do you require MFA to access sensitive data systems? & \ding{51} & \textbf{Good.} A key mitigating control is in place for sensitive data. \\
    \addlinespace
    Does your organization have an employee acceptable use policy? & \ding{55} & \textbf{High Risk.} Lack of clear guidelines for employees creates legal and security ambiguity. \\
    \addlinespace
    Does your organization do security awareness training for new employees? & \ding{55} & \textbf{Critical Gap.} New hires are not equipped to identify or respond to common threats like phishing. \\
    \addlinespace
    Does your organization do security awareness training for all employees at least once per year? & \ding{55} & \textbf{Critical Gap.} The absence of ongoing training leaves the "human firewall" weak and susceptible to attack. \\
    \bottomrule
\end{tabular}
\caption{Security Controls Questionnaire Analysis}
\end{table}

% --- SECTION 4: TECHNICAL SCAN RESULTS ---
\section{Technical Scan Results}
A network scan was performed on the specified target to identify open ports and exposed services.

\begin{itemize}
    \item \textbf{Target IP:} \texttt{192.168.0.5}
    \item \textbf{Scan Date:} \today
\end{itemize}

The scan revealed a very limited attack surface, with no open ports detected. This is a strong security posture from a network perspective.

\begin{table}[h!]
\centering
\begin{tabular}{@{}llll@{}}
    \toprule
    \textbf{Port} & \textbf{State} & \textbf{Service} & \textbf{Product / Version} \\
    \midrule
    80 & closed & http & Not Applicable \\
    \bottomrule
\end{tabular}
\caption{Nmap Scan Results for \texttt{192.168.0.5}}
\end{table}

\subsection*{Correlation with Existing Risks}
The provided list of current risks included "Unencrypted Web Server" due to an open port 80. The current scan shows this port is \textbf{closed}, which indicates this specific vulnerability has likely been remediated. This is a positive development and should be confirmed internally.

% --- SECTION 5: RISK ASSESSMENT SUMMARY ---
\section{Risk Assessment Summary}
The following table synthesizes findings from the security questionnaire, technical scan, and pre-existing risk data to provide a consolidated view of the current risk landscape.

\begin{table}[h!]
\centering
\begin{tabular}{@{}p{0.3\textwidth}p{0.5\textwidth}l@{}}
    \toprule
    \textbf{Risk Name} & \textbf{Description} & \textbf{Severity} \\
    \midrule
    \textbf{Lack of Multi-Factor Authentication (MFA)} & The absence of MFA on email and computer logins makes user accounts highly vulnerable to compromise via stolen or weak credentials. & \textbf{Critical} \\
    \addlinespace
    \textbf{Absence of Security Awareness Program} & Without training, employees are significantly more likely to fall victim to phishing, social engineering, and malware attacks. & \textbf{Critical} \\
    \addlinespace
    \textbf{Missing Acceptable Use Policy (AUP)} & The lack of a formal AUP creates ambiguity regarding safe computing practices and the handling of company data, increasing insider threat risk. & \textbf{High} \\
    \addlinespace
    \textbf{Unencrypted Web Server (Remediated)} & A previously identified risk of an open port 80. The current scan indicates this port is now closed, suggesting successful remediation. & Informational \\
    \bottomrule
\end{tabular}
\caption{Consolidated Risk Summary}
\end{table}

% --- SECTION 6: RECOMMENDATIONS ---
\section{Recommendations}
Based on the analysis, the following actions are recommended to improve the cybersecurity posture of \textbf{Iron River Finance}. Recommendations are prioritized by severity.

\begin{enumerate}
    \item \textbf{[Critical] Implement MFA for Email:} Immediately enforce MFA for all user access to the email system (\texttt{IronRiverFinance.net}). This is the single most effective control to prevent business email compromise.

    \item \textbf{[Critical] Establish a Security Awareness Program:} Develop and implement a mandatory security awareness training program. This should include initial training for all new hires and at least one annual refresher for all staff, focusing on phishing, password hygiene, and social engineering.

    \item \textbf{[High] Develop and Enforce an Acceptable Use Policy (AUP):} Create a formal AUP that clearly defines the rules and expectations for using company assets, data, and networks. Ensure all employees read and acknowledge the policy.

    \item \textbf{[High] Implement MFA for Computer Logins:} Deploy MFA for all employee computer logins to strengthen endpoint security and protect against unauthorized local and remote access.

    \item \textbf{[Informational] Verify Risk Remediation:} Internally confirm that the closure of port 80 was an intentional act to remediate the "Unencrypted Web Server" risk. Implement a change management process to track such security improvements.
\end{enumerate}

\end{document}
```