```latex
\documentclass[12pt]{article}

% --- PACKAGES ---
\usepackage[margin=1in]{geometry}
\usepackage{pifont} % For checkmarks and crosses
\usepackage{booktabs} % For professional tables
\usepackage{hyperref} % For clickable links
\usepackage{url} % For URL formatting
\usepackage{seqsplit} % For splitting long strings to prevent overflow

% --- DOCUMENT METADATA ---
\title{Cybersecurity Posture Assessment Report}
\author{Cybersecurity Analysis Division}
\date{\today}

% --- HYPERREF SETUP ---
\hypersetup{
    colorlinks=true,
    linkcolor=black,
    urlcolor=blue,
    pdftitle={Cybersecurity Posture Assessment Report},
    pdfauthor={Cybersecurity Analysis Division},
}

\begin{document}

\maketitle
\thispagestyle{empty}
\newpage
\tableofcontents
\newpage

% ==============================================================================
\section{Executive Overview}
% ==============================================================================

This report provides a comprehensive cybersecurity assessment for \textbf{Echo Chamber Arts}, conducted on \today. The analysis is based on a synthesis of organizational security control data, a network vulnerability scan, and a review of pre-existing documented risks.

The assessment reveals a mixed security posture. The organization has successfully implemented critical controls such as Multi-Factor Authentication (MFA) for email and sensitive data systems. However, significant gaps exist that expose the organization to considerable risk.

Key findings include:
\begin{itemize}
    \item \textbf{Critical Policy Gap:} The absence of a formal Employee Acceptable Use Policy.
    \item \textbf{High-Risk Gaps:} A lack of mandatory MFA for computer logins and the absence of a recurring annual security awareness training program for all staff. These issues significantly increase the risk of unauthorized access and successful social engineering attacks.
    \item \textbf{Technical Finding:} A network scan of the internal asset at \texttt{192.168.0.5} showed no open ports, which is a positive finding. This result contradicts a pre-existing documented risk concerning an unencrypted web server, suggesting the risk may apply to a different asset or has been partially remediated.
\end{itemize}

Immediate action is recommended to address the identified policy and training deficiencies to bolster the organization's defense against common cyber threats.

% ==============================================================================
\section{Organizational Information}
% ==============================================================================

The following information was provided for the assessment.

\begin{tabular}{@{}ll}
    \toprule
    \textbf{Attribute} & \textbf{Value} \\
    \midrule
    Organization Name & \textbf{Echo Chamber Arts} \\
    Email Domain & \texttt{EchoChamberArts.com} \\
    External IP Address & \texttt{52.178.173.26} \\
    \bottomrule
\end{tabular}

% ==============================================================================
\section{Security Control Review}
% ==============================================================================

The following table summarizes the organization's responses to a security controls questionnaire. A green checkmark (\ding{51}) indicates a positive control is in place, while a red cross (\ding{55}) indicates a gap.

\begin{table}[h!]
\centering
\begin{tabular}{@{}lc}
    \toprule
    \textbf{Control Question} & \textbf{Response} \\
    \midrule
    Do you require MFA to access email? & \ding{51} \\
    Do you require MFA to log into computers? & \textbf{\color{red}\ding{55}} \\
    Do you require MFA to access sensitive data systems? & \ding{51} \\
    Does your organization have an employee acceptable use policy? & \textbf{\color{red}\ding{55}} \\
    Does your organization do security awareness training for new employees? & \ding{51} \\
    Does your organization do security awareness training for all employees at least once per year? & \textbf{\color{red}\ding{55}} \\
    \bottomrule
\end{tabular}
\caption{Organizational Security Controls}
\end{table}

\subsection{Analysis of Control Gaps}
The identified gaps represent significant weaknesses in the organization's security posture:
\begin{itemize}
    \item \textbf{No MFA for Computers:} Without MFA on workstations, a single compromised password could grant an attacker full access to an employee's computer, facilitating data theft or lateral movement within the network.
    \item \textbf{No Acceptable Use Policy (AUP):} An AUP is a foundational document that sets clear expectations for employee behavior regarding company assets and data. Its absence can lead to inconsistent security practices and ambiguity during a security incident.
    \item \textbf{No Annual Security Training:} Cyber threats evolve constantly. Training only new hires means that the knowledge of long-term employees becomes outdated, making them more susceptible to phishing and other social engineering tactics.
\end{itemize}

% ==============================================================================
\section{Technical Scan Results}
% ==============================================================================

A network scan was performed to identify open ports and services on the specified target.

\subsection{Scan Details}
\begin{itemize}
    \item \textbf{Target IP:} \texttt{192.168.0.5}
    \item \textbf{Scan Date:} \today
\end{itemize}

\subsection{Port Scan Findings}
The scan of the target host revealed no open ports. This indicates a strong baseline security posture for this specific device, as it exposes no network services to potential attackers.

\begin{table}[h!]
\centering
\begin{tabular}{@{}lllll}
    \toprule
    \textbf{Port} & \textbf{State} & \textbf{Service} & \textbf{Product} & \textbf{Version} \\
    \midrule
    80 & closed & http & N/A & N/A \\
    \bottomrule
\end{tabular}
\caption{Nmap Scan Results for \texttt{192.168.0.5}}
\end{table}

\subsection{Correlation with Existing Risks}
A pre-existing risk, "Unencrypted Web Server," noted that Port 80 was open. The scan of \texttt{192.168.0.5} found this port to be \textbf{closed}. This suggests the pre-existing risk may be outdated, remediated for this host, or may pertain to another asset, such as the public-facing IP address (\texttt{52.178.173.26}). Further investigation is required to validate the status of this risk across all company assets.

% ==============================================================================
\section{Consolidated Risk Assessment}
% ==============================================================================

The following table consolidates risks identified from the security control review and pre-existing documentation.

\begin{table}[h!]
\centering
\begin{tabular}{@{}p{0.3\linewidth}p{0.5\linewidth}l}
    \toprule
    \textbf{Risk Name} & \textbf{Description} & \textbf{Severity} \\
    \midrule
    \textbf{Lack of Workstation MFA} & The absence of MFA on computer logins allows for easier account takeover if credentials are stolen. & \textbf{High} \\
    \addlinespace
    \textbf{No Acceptable Use Policy} & Foundational policy gap that creates ambiguity in security responsibilities and acceptable employee behavior. & \textbf{High} \\
    \addlinespace
    \textbf{No Annual Security Training} & Lack of recurring training increases susceptibility to social engineering and phishing attacks over time. & \textbf{High} \\
    \addlinespace
    \textbf{Unencrypted Web Server} & An open Port 80 allows for unencrypted HTTP traffic, exposing data in transit. \textit{(Note: Not confirmed on the scanned internal host.)} & Medium \\
    \bottomrule
\end{tabular}
\caption{Summary of Identified Risks}
\end{table}

% ==============================================================================
\section{Recommendations}
% ==============================================================================

The following actions are recommended to mitigate the identified risks and improve the overall security posture of \textbf{Echo Chamber Arts}.

\begin{enumerate}
    \item \textbf{Implement MFA for All Computer Logins (High Priority):}
    \begin{itemize}
        \item \textbf{Action:} Deploy a Multi-Factor Authentication solution for all employee workstations and servers.
        \item \textbf{Impact:} Drastically reduces the risk of unauthorized access from compromised credentials.
    \end{itemize}
    \vspace{1em}
    \item \textbf{Develop and Enforce an Acceptable Use Policy (High Priority):}
    \begin{itemize}
        \item \textbf{Action:} Create a formal AUP that defines the rules for using company IT assets, networks, and data. Require all employees to read and formally acknowledge the policy.
        \item \textbf{Impact:} Establishes a clear security baseline for all employees and provides a framework for enforcing security standards.
    \end{itemize}
    \vspace{1em}
    \item \textbf{Establish a Mandatory Annual Security Training Program (High Priority):}
    \begin{itemize}
        \item \textbf{Action:} Implement a security awareness training program that is mandatory for all employees on an annual basis. The training should cover current threats such as phishing, ransomware, and password security.
        \item \textbf{Impact:} Maintains a high level of security awareness across the organization, creating a stronger human firewall.
    \end{itemize}
    \vspace{1em}
    \item \textbf{Investigate the "Unencrypted Web Server" Risk (Medium Priority):}
    \begin{itemize}
        \item \textbf{Action:} Perform a vulnerability scan on the organization's external IP address (\texttt{52.178.173.26}) and other public-facing assets to determine if Port 80 is open.
        \item \textbf{Impact:} If found, closing the port or enforcing redirection to HTTPS (SSL/TLS) will protect data in transit and improve user trust.
    \end{itemize}
\end{enumerate}

\end{document}
```