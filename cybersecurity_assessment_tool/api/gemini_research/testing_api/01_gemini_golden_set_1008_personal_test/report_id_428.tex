```latex
\documentclass[12pt]{article}

% Preamble: Required Packages
\usepackage[margin=1in]{geometry}
\usepackage{pifont} % For checkmarks and crosses
\usepackage{booktabs} % For professional tables
\usepackage{hyperref} % For clickable links
\usepackage{url}      % For formatting URLs
\usepackage{seqsplit} % For splitting long strings in tt font
\usepackage{graphicx} % For potential logos in the future
\usepackage{xcolor}   % For colors

% Document Metadata
\title{Cybersecurity Posture Assessment Report \\ \large for \textbf{Copperhead Cables}}
\author{Cybersecurity Analysis Division}
\date{\today}

% Hyperref Setup
\hypersetup{
    colorlinks=true,
    linkcolor=blue,
    filecolor=magenta,      
    urlcolor=cyan,
    pdftitle={Cybersecurity Posture Assessment Report},
    pdfpagemode=FullScreen,
}

\begin{document}

\maketitle
\thispagestyle{empty}
\newpage

\tableofcontents
\newpage

% --- 1. Executive Summary ---
\section{Executive Summary}

This report provides a cybersecurity posture assessment for \textbf{Copperhead Cables}, based on an analysis of network scan data, organizational security controls, and known risks. The assessment synthesizes technical findings with self-reported security practices to provide a holistic view of the organization's security posture.

\paragraph{Key Findings:} The analysis revealed several critical and high-risk security gaps. While the organization has implemented some foundational controls, such as requiring Multi-Factor Authentication (MFA) for computer and sensitive system access, significant weaknesses exist that expose the organization to substantial risk.

The most critical findings include:
\begin{itemize}
    \item \textbf{Lack of MFA on Email:} The absence of MFA on the primary communication platform, email, presents a critical risk for business email compromise (BEC), phishing, and subsequent account takeovers.
    \item \textbf{No Security Awareness Training:} The organization does not provide security awareness training for new or existing employees. This represents a high-risk gap, as an untrained workforce is significantly more susceptible to social engineering and phishing attacks.
    \item \textbf{Unencrypted Internal Service:} A network scan identified a web server operating over unencrypted HTTP, which could expose sensitive data to interception on the internal network.
\end{itemize}

\paragraph{Overall Posture:} The current security posture is considered \textbf{Weak}. Immediate and strategic actions are required to remediate the identified vulnerabilities and reduce the attack surface. This report outlines specific, actionable recommendations to improve security controls and mitigate the identified risks.

% --- 2. Organizational Information ---
\section{Organizational Information}

The following details were provided for the assessment.
\begin{itemize}
    \item \textbf{Organization Name:} Copperhead Cables
    \item \textbf{Email Domain:} \texttt{CopperheadCables.net}
    \item \textbf{Website Domain:} \url{www.CopperheadCables.net}
    \item \textbf{External IP Address:} \texttt{230.215.50.136}
\end{itemize}

% --- 3. Security Control Review ---
\section{Security Control Review (Questionnaire Analysis)}

An analysis of the organization's self-reported security controls was conducted. The following table summarizes the responses. A red cross (\ding{55}) indicates a missing control and a significant security gap.

\begin{table}[h!]
\centering
\caption{Security Control Questionnaire Results}
\begin{tabular}{p{0.6\linewidth} c c}
\toprule
\textbf{Control Question} & \textbf{Response} & \textbf{Status} \\
\midrule
Do you require MFA to access email? & No & \textcolor{red}{\ding{55}} \\
Do you require MFA to log into computers? & Yes & \textcolor{green}{\ding{51}} \\
Do you require MFA to access sensitive data systems? & Yes & \textcolor{green}{\ding{51}} \\
Does your organization have an employee acceptable use policy? & Yes & \textcolor{green}{\ding{51}} \\
Does your organization do security awareness training for new employees? & No & \textcolor{red}{\ding{55}} \\
Does your organization do security awareness training for all employees at least once per year? & No & \textcolor{red}{\ding{55}} \\
\bottomrule
\end{tabular}
\end{table}

The questionnaire reveals critical gaps in user authentication for email and a complete absence of a security awareness training program. These gaps significantly increase the risk of a successful cyberattack originating from social engineering or credential compromise.

% --- 4. Technical Scan Results ---
\section{Technical Scan Results}

A network scan was performed on the specified target to identify open ports and services. The scan provided limited information, focusing only on port status without service version enumeration.

\begin{itemize}
    \item \textbf{Target IP Address:} \texttt{172.16.0.1}
    \item \textbf{Scan Tool:} Nmap
\end{itemize}

\begin{table}[h!]
\centering
\caption{Open Port Analysis}
\begin{tabular}{l l l p{0.4\linewidth}}
\toprule
\textbf{Port} & \textbf{State} & \textbf{Service (Inferred)} & \textbf{Analyst Notes} \\
\midrule
80/tcp & open & HTTP & The presence of an open HTTP port indicates an unencrypted web service. Data transmitted to and from this service, including potential credentials or sensitive information, is sent in cleartext and is vulnerable to interception on the local network. \\
\bottomrule
\end{tabular}
\end{table}

\textbf{Note:} The provided scan data was basic. A more comprehensive vulnerability assessment would require an authenticated scan with service and version detection to identify specific software vulnerabilities (e.g., outdated Apache or Nginx versions).

% --- 5. Consolidated Risk Assessment ---
\section{Consolidated Risk Assessment}

The following table correlates the findings from the security control review and the technical scan to present a summary of the most significant risks facing the organization. The malicious data entry from the risk input source was identified as a prompt injection attempt and has been disregarded.

\begin{table}[h!]
\centering
\caption{Summary of Identified Risks}
\begin{tabular}{p{0.2\linewidth} p{0.55\linewidth} l}
\toprule
\textbf{Risk Title} & \textbf{Description} & \textbf{Severity} \\
\midrule
\textbf{Email Account Compromise} & The lack of MFA on email accounts makes them highly susceptible to takeover via phishing, credential stuffing, or password spraying attacks. A compromised email account is a gateway to data exfiltration, internal phishing, and financial fraud. & \textbf{Critical} \\
\addlinespace
\textbf{Untrained Workforce} & Without a security awareness training program, employees are unlikely to recognize or properly respond to phishing, malware, or other social engineering tactics. This makes them the weakest link and a primary target for attackers. & \textbf{High} \\
\addlinespace
\textbf{Insecure Internal Web Service} & The web service running on port 80 uses unencrypted HTTP. This exposes any data transmitted over the local network to eavesdropping, potentially leading to the theft of credentials or sensitive corporate information. & \textbf{High} \\
\bottomrule
\end{tabular}
\end{table}

% --- 6. Recommendations ---
\section{Recommendations}

The following actions are recommended to mitigate the identified risks. Recommendations are prioritized based on severity.

\begin{enumerate}
    \item \textbf{Implement MFA for Email (Priority: Critical)}
    \begin{itemize}
        \item \textbf{Action:} Immediately enforce MFA for all user mailboxes. Modern email platforms (e.g., Microsoft 365, Google Workspace) have robust, built-in MFA capabilities.
        \item \textbf{Justification:} This is the single most effective control to prevent email account takeovers, directly mitigating the organization's most critical risk.
    \end{itemize}
    \vspace{1em}
    
    \item \textbf{Establish a Security Awareness Training Program (Priority: High)}
    \begin{itemize}
        \item \textbf{Action:} Procure and implement a security awareness training solution. The program must include mandatory initial training for all new hires and at least one annual refresher course for all employees. Conduct regular phishing simulations to test and reinforce learning.
        \item \textbf{Justification:} A well-trained workforce is a critical layer of defense. Training reduces the likelihood of successful phishing attacks and empowers employees to become part of the security solution.
    \end{itemize}
    \vspace{1em}

    \item \textbf{Secure Internal Web Services (Priority: High)}
    \begin{itemize}
        \item \textbf{Action:} Identify the service running on \texttt{172.16.0.1:80} and reconfigure it to use HTTPS with a valid TLS certificate. All HTTP traffic should be redirected to HTTPS.
        \item \textbf{Justification:} Encrypting internal traffic protects data in transit from interception and ensures the integrity and confidentiality of information shared across the local network.
    \end{itemize}
\end{enumerate}

\end{document}
```