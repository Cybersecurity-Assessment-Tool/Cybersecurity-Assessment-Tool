```latex
\documentclass[12pt]{article}

% Preamble: Required Packages
\usepackage[margin=1in]{geometry}
\usepackage{pifont} % For checkmarks and crosses
\usepackage{booktabs} % For professional-looking tables
\usepackage{hyperref} % For clickable links
\usepackage{url} % For URL formatting
\usepackage{seqsplit} % To split long strings in tt font
\usepackage{graphicx}
\usepackage{xcolor}

% Document Information
\title{Cybersecurity Posture Assessment Report \\ \large For: \textbf{Prism Logic}}
\author{Cybersecurity Analysis Division}
\date{\today}

% Hyperref Setup
\hypersetup{
    colorlinks=true,
    linkcolor=blue,
    filecolor=magenta,      
    urlcolor=cyan,
    pdftitle={Cybersecurity Posture Assessment Report},
    pdfpagemode=FullScreen,
}

\begin{document}

\maketitle
\thispagestyle{empty}
\newpage

\tableofcontents
\newpage

% --- 1. Executive Overview ---
\section*{1. Executive Overview}

This report provides a comprehensive cybersecurity posture assessment for \textbf{Prism Logic}, based on an analysis of network scan data, organizational security controls, and pre-existing risk information. The assessment synthesizes technical findings with procedural and policy-based controls to provide a holistic view of the organization's security standing.

The analysis reveals several critical gaps in the organization's security controls, primarily related to identity and access management. The absence of Multi-Factor Authentication (MFA) for email and computer access represents a significant and immediate risk. Furthermore, the lack of mandatory security awareness training for new employees creates a window of vulnerability that could be exploited by threat actors.

On a positive note, a previously identified risk concerning an unencrypted web server on port 80 appears to have been remediated. Our recent network scan found this port to be closed on the assessed target, indicating successful mitigation of that specific vulnerability. Recommendations in this report focus on addressing the remaining high-priority risks to bolster the overall security posture.

% --- 2. Organizational & Scan Information ---
\section*{2. Organizational \& Scan Information}

This section details the organizational data provided for the assessment and the metadata associated with the technical network scan.

\subsection*{2.1. Organizational Details}
\begin{tabular}{@{}ll}
\toprule
\textbf{Attribute} & \textbf{Value} \\
\midrule
Organization Name & \textbf{Prism Logic} \\
Email Domain & \texttt{PrismLogic.net} \\
Website Domain & \url{www.PrismLogic.net} \\
External IP Address & \texttt{94.31.208.187} \\
\bottomrule
\end{tabular}

\subsection*{2.2. Scan Details}
\begin{tabular}{@{}ll}
\toprule
\textbf{Attribute} & \textbf{Value} \\
\midrule
Scan Date & \today \\
Target IP Address & \texttt{192.168.0.5} \\
Scan Type & Nmap Port Scan \\
\bottomrule
\end{tabular}

% --- 3. Security Control Review (Questionnaire) ---
\section*{3. Security Control Review}

The following table summarizes the organization's responses to a security controls questionnaire. Items marked with a red 'X' (\ding{55}) indicate a deviation from security best practices and represent a significant gap in the defensive posture.

\begin{table}[h!]
\centering
\begin{tabular}{@{}p{0.75\linewidth}c}
\toprule
\textbf{Control Question} & \textbf{Status} \\
\midrule
Does your organization have an employee acceptable use policy? & \textcolor{green}{\ding{51}} \\
Does your organization do security awareness training for all employees at least once per year? & \textcolor{green}{\ding{51}} \\
Do you require MFA to access sensitive data systems? & \textcolor{green}{\ding{51}} \\
\midrule
\textbf{Do you require MFA to access email?} & \textbf{\textcolor{red}{\ding{55}}} \\
\textbf{Do you require MFA to log into computers?} & \textbf{\textcolor{red}{\ding{55}}} \\
\textbf{Does your organization do security awareness training for new employees?} & \textbf{\textcolor{red}{\ding{55}}} \\
\bottomrule
\end{tabular}
\caption{Security Controls Questionnaire Results}
\end{table}

\subsection*{3.1. Analysis of Control Gaps}
The review identified three critical control gaps:
\begin{itemize}
    \item \textbf{Lack of MFA for Email and Endpoints:} The absence of MFA for primary communication (email) and system access (computers) exposes the organization to a high risk of account compromise through phishing, credential stuffing, and password spraying attacks.
    \item \textbf{No Onboarding Security Training:} New employees are often prime targets for social engineering. Failing to provide immediate security awareness training upon hiring leaves a significant vulnerability window until the annual training cycle occurs.
\end{itemize}

% --- 4. Technical Scan Results ---
\section*{4. Technical Scan Results}

A network scan was performed on the target system to identify open ports and exposed services. The results are detailed below.

\begin{table}[h!]
\centering
\begin{tabular}{@{}lllll}
\toprule
\textbf{Port} & \textbf{State} & \textbf{Service} & \textbf{Product} & \textbf{Version} \\
\midrule
80/tcp & closed & http & N/A & N/A \\
\bottomrule
\end{tabular}
\caption{Nmap Scan Results for \texttt{192.168.0.5}}
\end{table}

\subsection*{4.1. Technical Findings}
The scan of the target \texttt{192.168.0.5} revealed no open ports. Notably, port 80 (HTTP) was found to be \textbf{closed}. This finding contradicts a previously identified risk (see Section 5), suggesting that the vulnerability has been successfully remediated. The limited exposure of services on this host is a positive security finding.

% --- 5. Consolidated Risk Assessment ---
\section*{5. Consolidated Risk Assessment}

This section synthesizes findings from the security control review, technical scans, and pre-existing risk data into a consolidated list of organizational risks.

\begin{table}[h!]
\centering
\begin{tabular}{@{}p{0.3\linewidth}p{0.5\linewidth}l}
\toprule
\textbf{Risk Name} & \textbf{Overview} & \textbf{Severity} \\
\midrule
\textbf{Lack of Foundational MFA} & Email and computer accounts are protected only by passwords, making them highly susceptible to takeover. & \textbf{Critical} \\
\addlinespace
\textbf{Inadequate New Hire Training} & New employees are not equipped with security awareness from day one, increasing susceptibility to social engineering. & \textbf{High} \\
\addlinespace
Unencrypted Web Server & Port 80 was believed to be open, exposing unencrypted traffic. The recent scan shows this port is now closed. & \textcolor{green}{Remediated} \\
\bottomrule
\end{tabular}
\caption{Summary of Identified Risks}
\end{table}

% --- 6. Recommendations ---
\section*{6. Recommendations}

Based on the consolidated risk assessment, the following actions are recommended to mitigate the identified vulnerabilities and improve the overall security posture of \textbf{Prism Logic}.

\subsection*{6.1. Critical Priority}
\begin{itemize}
    \item \textbf{Implement Multi-Factor Authentication (MFA):}
    \begin{itemize}
        \item \textbf{Action:} Immediately deploy a mandatory MFA solution for all user accounts for both email access (e.g., Office 365, Google Workspace) and computer/endpoint logins (e.g., Windows Hello, Duo).
        \item \textbf{Justification:} This is the single most effective control to prevent unauthorized account access and will drastically reduce the risk of a breach resulting from compromised credentials.
    \end{itemize}
\end{itemize}

\subsection*{6.2. High Priority}
\begin{itemize}
    \item \textbf{Integrate Security Training into Employee Onboarding:}
    \begin{itemize}
        \item \textbf{Action:} Develop or procure a security awareness training module and make it a mandatory component of the new-hire onboarding process, to be completed within the first week of employment.
        \item \textbf{Justification:} This ensures that all employees, from their first day, understand their security responsibilities, recognize common threats like phishing, and know the organization's acceptable use policies. This closes a critical vulnerability window.
    \end{itemize}
\end{itemize}

\subsection*{6.3. Informational}
\begin{itemize}
    \item \textbf{Validate Risk Remediation:}
    \begin{itemize}
        \item \textbf{Action:} Formally update the internal risk register to mark the "Unencrypted Web Server" vulnerability as remediated, referencing the scan results from this report.
        \item \textbf{Justification:} Maintaining an accurate and up-to-date risk register is crucial for focusing security efforts on current and relevant threats.
    \end{itemize}
\end{itemize}

\end{document}
```