```latex
\documentclass[12pt]{article}

% Preamble: Required Packages
\usepackage[margin=1in]{geometry}
\usepackage{pifont} % For checkmarks and crosses
\usepackage{booktabs} % For professional tables
\usepackage[hidelinks]{hyperref} % For clickable links without boxes
\usepackage{url} % For formatting URLs
\usepackage{seqsplit} % For splitting long text strings to prevent overflow
\usepackage{xcolor} % For custom colors
\usepackage{fancyhdr} % For headers and footers
\usepackage{graphicx} % For images (e.g., logo)

% --- Document Setup ---
\pagestyle{fancy}
\fancyhf{} % Clear all header and footer fields
\fancyhead[L]{Cybersecurity Assessment Report}
\fancyhead[R]{\textbf{Urban Jungle Planning}}
\fancyfoot[C]{\thepage}
\renewcommand{\headrulewidth}{0.4pt}
\renewcommand{\footrulewidth}{0.4pt}

% Define a custom color for severity
\definecolor{criticalred}{RGB}{217, 83, 79}

\begin{document}

% --- Title Page ---
\begin{titlepage}
    \centering
    \vspace*{1cm}
    
    \Huge
    \textbf{Cybersecurity Assessment Report}
    
    \vspace{1.5cm}
    
    \Large
    Prepared for: \\
    \vspace{0.5cm}
    \textbf{Urban Jungle Planning}
    
    \vspace{2cm}
    
    \large
    Report Date: \today \\
    Analysis Period: Current Assessment
    
    \vfill
    
    \normalsize
    \textbf{CONFIDENTIAL} \\
    This document contains sensitive information. Distribution is restricted to authorized personnel only.
    
\end{titlepage}

\tableofcontents
\newpage

% --- Section 1: Executive Summary ---
\section{Executive Summary}

This report provides a cybersecurity assessment for \textbf{Urban Jungle Planning}, synthesizing data from a technical network scan, a security controls questionnaire, and a review of pre-existing risks.

The assessment reveals a notable disparity between administrative policies and technical security posture. The organization demonstrates strong administrative controls, with mandatory Multi-Factor Authentication (MFA) and consistent security awareness training reported across the board. These are commendable foundational elements of a robust security program.

However, a critical technical vulnerability was identified. The network scan discovered an open Remote Desktop Protocol (RDP) port (3389) on the internal host \texttt{10.10.10.51}. This finding is highly concerning as it mirrors a previously identified risk on another host (\texttt{10.10.10.50}), indicating a potential systemic or recurring misconfiguration in the network environment.

Exposed RDP is a primary target for attackers, who leverage it for brute-force attacks, credential theft, and the deployment of ransomware. The recurrence of this high-risk vulnerability significantly elevates the organization's risk profile, despite its strong administrative controls.

Immediate remediation of the exposed services is required, followed by a strategic initiative to implement secure remote access solutions like a VPN and enforce secure configuration standards for all systems.

% --- Section 2: Organizational Information ---
\section{Organizational Information}

The following details were provided for the assessment.

\begin{itemize}
    \item \textbf{Organization Name:} Urban Jungle Planning
    \item \textbf{Email Domain:} \texttt{UrbanJunglePlanning.net}
    \item \textbf{Website Domain:} \url{www.UrbanJunglePlanning.net}
    \item \textbf{External IP Address:} \texttt{111.133.62.27}
\end{itemize}

% --- Section 3: Security Control Review ---
\section{Security Control Review}

A security questionnaire was completed to evaluate the existing administrative and preventative controls. The results, summarized in Table \ref{tab:controls}, indicate a strong commitment to security policies and user-level protections.

\begin{table}[h!]
\centering
\caption{Security Controls Questionnaire Results}
\label{tab:controls}
\begin{tabular}{@{}lc@{}}
\toprule
\textbf{Control Question} & \textbf{Response} \\
\midrule
Do you require MFA to access email? & \ding{51} \\
Do you require MFA to log into computers? & \ding{51} \\
Do you require MFA to access sensitive data systems? & \ding{51} \\
Does your organization have an employee acceptable use policy? & \ding{51} \\
Does your organization do security awareness training for new employees? & \ding{51} \\
Does your organization do security awareness training for all employees annually? & \ding{51} \\
\bottomrule
\end{tabular}
\end{table}

\textbf{Analysis:} The self-reported data shows a mature approach to user authentication and security awareness. The consistent implementation of MFA is a critical defense against credential-based attacks. These positive controls reduce the likelihood of account compromise.

% --- Section 4: Technical Scan Results ---
\section{Technical Scan Results}

A network scan was performed to identify open ports and services on the target system.

\begin{itemize}
    \item \textbf{Target IP Address:} \texttt{10.10.10.51}
    \item \textbf{Scan Tool:} Nmap
\end{itemize}

The scan identified the following open port, as detailed in Table \ref{tab:scan}.

\begin{table}[h!]
\centering
\caption{Open Port Findings for \texttt{10.10.10.51}}
\label{tab:scan}
\begin{tabular}{@{}llll@{}}
\toprule
\textbf{Port} & \textbf{State} & \textbf{Service} & \textbf{Description} \\
\midrule
3389/tcp & open & \texttt{ms-wbt-server} & Microsoft Remote Desktop Protocol (RDP) \\
\bottomrule
\end{tabular}
\end{table}

\textbf{Analysis:} The presence of an open RDP port is a critical security risk. This service allows for direct remote administration of the Windows operating system. If exposed, it becomes a high-value target for attackers attempting to gain unauthorized access to the internal network.

% --- Section 5: Risk Assessment ---
\section{Risk Assessment}

This section correlates findings from the technical scan, the security questionnaire, and pre-existing risk data. The primary concern is the recurring nature of a critical-risk vulnerability.

\begin{table}[h!]
\centering
\caption{Synthesized Risk Summary}
\label{tab:risks}
\begin{tabular}{@{}p{0.25\linewidth}p{0.1\linewidth}p{0.2\linewidth}p{0.35\linewidth}@{}}
\toprule
\textbf{Risk Name} & \textbf{Severity} & \textbf{Affected Systems} & \textbf{Description} \\
\midrule
RDP Exposure (Pre-existing) & \colorbox{criticalred}{\color{white} \textbf{Critical}} & \texttt{10.10.10.50} & A previously identified risk where RDP is directly exposed, allowing for brute-force and exploit attempts. \\
\addlinespace
\textbf{Recurring RDP Exposure (New Finding)} & \colorbox{criticalred}{\color{white} \textbf{Critical}} & \texttt{10.10.10.51} & The current scan discovered another host with the same critical vulnerability. This points to a systemic issue in server deployment or network firewall configuration. \\
\bottomrule
\end{tabular}
\end{table}

% --- Section 6: Recommendations ---
\section{Recommendations}

Based on the correlated findings, the following tactical and strategic recommendations are provided to mitigate the identified risks and improve the overall security posture.

\subsection{Immediate Actions (Tactical)}

These steps should be taken within the next 24-48 hours to address the critical risks.

\begin{enumerate}
    \item \textbf{Restrict RDP Access on \texttt{10.10.10.51} and \texttt{10.10.10.50}:}
    \begin{itemize}
        \item Immediately implement firewall rules to block all inbound access to TCP port 3389 on both systems from any untrusted network.
        \item If remote access is required, create explicit "allow" rules for a limited set of trusted administrator IP addresses.
        \item If remote access is not required, disable the Remote Desktop service on these servers entirely.
    \end{itemize}
    \item \textbf{Conduct an Internal Vulnerability Scan:}
    \begin{itemize}
        \item Perform a comprehensive scan of the internal network range to identify any other systems with exposed RDP or other risky services.
    \end{itemize}
\end{enumerate}

\subsection{Strategic Initiatives (Long-Term)}

These actions will address the root cause of the recurring vulnerability and strengthen long-term security.

\begin{enumerate}
    \item \textbf{Implement a Secure Remote Access Solution:}
    \begin{itemize}
        \item Deploy a Virtual Private Network (VPN) with strong authentication (MFA) for all remote administrative access.
        \item Configure the network so that services like RDP are only accessible after a user has successfully connected to the VPN. This removes the direct exposure from the network.
    \end{itemize}
    \item \textbf{Develop and Enforce Secure Configuration Standards:}
    \begin{itemize}
        \item Create a documented, hardened server build standard (a "gold image") that disables unnecessary services and ports by default.
        \item All new systems deployed in the environment must be built from this approved standard.
    \end{itemize}
    \item \textbf{Implement a Change Control and Auditing Process:}
    \begin{itemize}
        \item Ensure that any firewall or server configuration changes go through a formal approval process.
        \item Regularly audit firewall rules and server configurations to ensure they align with security policy and have not been altered without authorization.
    \end{itemize}
\end{enumerate}

\end{document}
```