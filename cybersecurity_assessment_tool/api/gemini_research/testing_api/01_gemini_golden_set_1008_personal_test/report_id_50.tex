```latex
\documentclass[12pt]{article}

% Preamble: Required Packages
\usepackage[margin=1in]{geometry}
\usepackage{pifont} % For checkmarks and crosses
\usepackage{booktabs} % For professional tables
\usepackage{hyperref} % For clickable links
\usepackage{url} % For URL formatting
\usepackage{seqsplit} % To split long strings in texttt
\usepackage{graphicx}
\usepackage{xcolor}
\usepackage{fancyhdr}

% --- Document Setup ---
% Define colors for severity levels
\definecolor{criticalred}{HTML}{D73027}
\definecolor{highorange}{HTML}{F46D43}
\definecolor{mediumyellow}{HTML}{FEE08B}
\definecolor{lowgreen}{HTML}{66BD63}
\definecolor{infoblue}{HTML}{4575B4}

% Hyperlink setup
\hypersetup{
    colorlinks=true,
    linkcolor=blue,
    filecolor=magenta,      
    urlcolor=cyan,
    pdftitle={Cybersecurity Posture Report},
    pdfpagemode=FullScreen,
}

% Header and Footer
\pagestyle{fancy}
\fancyhf{}
\fancyhead[L]{Cybersecurity Posture Report}
\fancyhead[R]{Confidential}
\fancyfoot[C]{\thepage}

% --- Document Start ---
\begin{document}

% --- Title Page ---
\begin{titlepage}
    \centering
    \vspace*{2cm}
    
    {\Huge \textbf{Cybersecurity Posture Report}\par}
    \vspace{1.5cm}
    
    {\Large Prepared for:\par}
    \vspace{0.5cm}
    {\Huge \textbf{Midnight Oil Studios}\par}
    
    \vfill
    
    {\large \today\par}
    
    \vspace{1cm}
    \textit{This document contains sensitive information and is intended for internal use only.}
\end{titlepage}

\tableofcontents
\newpage

% --- Section 1: Executive Summary ---
\section{Executive Summary}

This report provides a comprehensive analysis of the cybersecurity posture for \textbf{Midnight Oil Studios}, based on a synthesis of network scan data, organizational security controls, and pre-existing risk assessments. The evaluation was conducted on \today.

The assessment has identified a \textbf{critical, immediate risk} related to an externally facing FTP server. This server is running a dangerously outdated version of \texttt{vsftpd} (2.3.4) with a known remote code execution vulnerability (CVE-2011-2523). The server is further misconfigured to allow anonymous access, compounding the risk of a severe data breach or system compromise.

Furthermore, significant gaps were identified in the organization's administrative controls. The absence of a formal Acceptable Use Policy and the lack of security awareness training for new employees represent high-risk deficiencies that weaken the human firewall and increase susceptibility to social engineering and insider threats.

While the organization has implemented strong Multi-Factor Authentication (MFA) controls, the identified technical and administrative vulnerabilities require immediate attention to mitigate the substantial risk of a security incident. Recommendations are prioritized to address the most critical findings first.

% --- Section 2: Organizational Information ---
\section{Organizational Information}

The following details were provided for the assessment.

\begin{tabular}{@{}ll}
    \toprule
    \textbf{Attribute} & \textbf{Value} \\
    \midrule
    Organization Name & \textbf{Midnight Oil Studios} \\
    Email Domain & \texttt{MidnightOilStudios.org} \\
    Website Domain & \url{www.MidnightOilStudios.org} \\
    External IP & \texttt{119.203.6.47} \\
    \bottomrule
\end{tabular}

% --- Section 3: Security Control Review ---
\section{Security Control Review}

A review of the organization's security controls was conducted via a questionnaire. The responses highlight strengths in authentication but reveal critical gaps in policy and training. "No" answers indicate areas of significant risk.

\begin{tabular}{@{}p{0.7\linewidth}c}
    \toprule
    \textbf{Control Question} & \textbf{Response} \\
    \midrule
    Do you require MFA to access email? & \textcolor{green}{\ding{51}} \\
    Do you require MFA to log into computers? & \textcolor{green}{\ding{51}} \\
    Do you require MFA to access sensitive data systems? & \textcolor{green}{\ding{51}} \\
    \addlinespace
    Does your organization have an employee acceptable use policy? & \textcolor{red}{\ding{55}} \\
    \textit{\footnotesize (Gap: Increases risk of misuse and insider threat.)} & \\
    \addlinespace
    Does your organization do security awareness training for new employees? & \textcolor{red}{\ding{55}} \\
    \textit{\footnotesize (Gap: New hires are a primary target for phishing.)} & \\
    \addlinespace
    Does your organization do security awareness training for all employees at least once per year? & \textcolor{green}{\ding{51}} \\
    \bottomrule
\end{tabular}

% --- Section 4: Technical Scan Results ---
\section{Technical Scan Results}

An Nmap scan was performed on the specified target to identify open ports and exposed services.

\begin{itemize}
    \item \textbf{Target IP:} \texttt{10.0.0.15}
    \item \textbf{Scan Date:} \today
\end{itemize}

The scan revealed one open port with a critically vulnerable service.

\begin{tabular}{@{}llllll}
    \toprule
    \textbf{Port} & \textbf{State} & \textbf{Service} & \textbf{Product} & \textbf{Version} & \textbf{Notes} \\
    \midrule
    21/tcp & OPEN & ftp & vsftpd & 2.3.4 & \parbox[t]{4cm}{\textbf{CRITICAL FINDING}:\\ 1. Anonymous FTP login allowed.\\ 2. This version is vulnerable to a backdoor (CVE-2011-2523).} \\
    \bottomrule
\end{tabular}

\subsection{Analysis of Technical Findings}
The presence of \textbf{vsftpd version 2.3.4} is a severe security risk. This specific version contains a well-known, malicious backdoor that was inserted into the source code. An attacker can gain a command shell on the server by sending a specific string as the username. Combined with the \textbf{Anonymous FTP login} misconfiguration, this system is trivial to compromise, providing a foothold for attackers to pivot into the internal network.

% --- Section 5: Consolidated Risk Assessment ---
\section{Consolidated Risk Assessment}

The following table summarizes and prioritizes the identified risks by correlating technical findings, control gaps, and pre-existing issues.

\begin{tabular}{@{}p{0.1\linewidth}p{0.25\linewidth}p{0.15\linewidth}p{0.4\linewidth}@{}}
    \toprule
    \textbf{Severity} & \textbf{Risk Name} & \textbf{Affected Systems} & \textbf{Description} \\
    \midrule
    \colorbox{criticalred}{\color{white}\textbf{CRITICAL}} & \textbf{Vulnerable FTP Server with Anonymous Access} & Server at \texttt{10.0.0.15} & The server is running vsftpd 2.3.4, which has a critical backdoor vulnerability (CVE-2011-2523). It is also configured to allow anonymous logins, making exploitation trivial. \\
    \addlinespace
    \colorbox{highorange}{\color{white}\textbf{HIGH}} & \textbf{Lack of Foundational Security Policies} & All Employees, Organizational Data & The absence of an Acceptable Use Policy and security training for new hires creates a significant risk from insider threats (accidental or malicious) and social engineering attacks. \\
    \addlinespace
    \colorbox{mediumyellow}{\textbf{MEDIUM}} & \textbf{Outdated Workstation Operating Systems} & Workstations & As per existing risk data, computers are running Windows 7. This OS is end-of-life and no longer receives security updates, leaving it vulnerable to known exploits. \\
    \bottomrule
\end{tabular}

% --- Section 6: Recommendations ---
\section{Recommendations}

The following actions are recommended to mitigate the identified risks. They are prioritized based on severity and potential impact.

\subsection{Immediate Priority (Critical Risk)}
\begin{enumerate}
    \item \textbf{Isolate and Remediate the FTP Server (10.0.0.15):}
    \begin{itemize}
        \item Immediately take the server offline or apply a firewall rule to block all access to port 21.
        \item Investigate the server for signs of existing compromise.
        \item Upgrade the \texttt{vsftpd} service to the latest stable version.
        \item Disable anonymous FTP access unless there is an explicit and documented business requirement.
        \item \textbf{Long-Term:} Replace FTP with a secure alternative, such as SFTP (SSH File Transfer Protocol), which encrypts both credentials and data in transit.
    \end{itemize}
\end{enumerate}

\subsection{High Priority}
\begin{enumerate}
    \setcounter{enumi}{1}
    \item \textbf{Develop and Implement an Acceptable Use Policy (AUP):}
    \begin{itemize}
        \item Draft a clear AUP that defines the rules for using company IT assets, data, and networks.
        \item Require all current and new employees to read and formally acknowledge the policy.
    \end{itemize}
    \item \textbf{Establish Onboarding Security Training:}
    \begin{itemize}
        \item Create a mandatory security awareness training module for all new employees.
        \item This training should cover key topics such as phishing, password security, data handling, and the new AUP.
    \end{itemize}
\end{enumerate}

\subsection{Medium Priority}
\begin{enumerate}
    \setcounter{enumi}{3}
    \item \textbf{Execute Operating System Upgrade Plan:}
    \begin{itemize}
        \item Prioritize and execute the existing recommendation to upgrade all Windows 7 workstations to a supported operating system (e.g., Windows 10/11) to ensure they receive critical security patches.
    \end{itemize}
\end{enumerate}

\end{document}
```