```latex
\documentclass[12pt]{article}

% Preamble: Required Packages
\usepackage[margin=1in]{geometry}
\usepackage{pifont} % For checkmarks and crosses (\ding)
\usepackage{booktabs} % For professional-looking tables
\usepackage{hyperref} % For clickable links
\usepackage{url} % For URL formatting
\usepackage{seqsplit} % To split long strings in tt font
\usepackage{xcolor} % For coloring text
\usepackage{graphicx} % For logo placeholder

% Hyperref Setup for better visuals
\hypersetup{
    colorlinks=true,
    linkcolor=blue,
    filecolor=magenta,      
    urlcolor=cyan,
    pdftitle={Cybersecurity Posture Assessment Report},
    pdfpagemode=FullScreen,
}

% Define colors for severity
\definecolor{sev_critical}{HTML}{940000}
\definecolor{sev_high}{HTML}{D14000}
\definecolor{sev_medium}{HTML}{E09100}

\begin{document}

% --- Title Page ---
\begin{titlepage}
    \centering
    \vspace*{1cm}
    
    \Huge\textbf{Cybersecurity Posture Assessment Report}
    
    \vspace{1.5cm}
    
    \Large\textbf{Prepared for:} \\
    \vspace{0.5cm}
    \textbf{Nomad Gear Co.}
    
    \vspace{2cm}
    
    \textbf{Date of Report:} \today
    
    \vfill
    
    \large
    \textbf{Analysis Conducted By:} \\
    Cybersecurity Analysis Division
    
    \vspace{0.8cm}
    
    \textit{This report contains sensitive information and should be handled with discretion. Distribution is restricted to authorized personnel only.}
    
\end{titlepage}

\tableofcontents
\newpage

% --- Section 1: Executive Overview ---
\section{Executive Overview}
This report provides a comprehensive assessment of the cybersecurity posture for \textbf{Nomad Gear Co.}. The analysis is based on a combination of network scanning, a review of existing risks, and a security controls questionnaire.

The overall security posture is assessed as \textbf{HIGH RISK}. Several critical deficiencies were identified that expose the organization to significant threats, including unauthorized access, data breaches, and service disruption.

Key findings include:
\begin{itemize}
    \item \textbf{Complete Absence of Multi-Factor Authentication (MFA):} The lack of MFA for email, computer logins, and sensitive data systems represents a critical vulnerability. This significantly increases the risk of account compromise via credential theft or brute-force attacks.
    \item \textbf{Exposed Network Services:} Technical scans confirmed a pre-existing risk of an exposed service on a critical system (\texttt{127.0.0.1}). The combination of this exposed SSH port (22) and the lack of MFA creates a direct and severe threat vector.
    \item \textbf{Policy and Training Gaps:} The absence of a formal Acceptable Use Policy and mandatory annual security awareness training indicates a weakness in security governance and employee preparedness against common cyber threats like phishing.
\end{itemize}

Immediate remediation of the identified critical risks is strongly recommended to reduce the organization's attack surface and mitigate the likelihood of a security incident.

% --- Section 2: Organizational Information ---
\section{Organizational Information}
The following details were provided for the assessment.

\begin{tabular}{@{}ll}
    \toprule
    \textbf{Attribute} & \textbf{Value} \\
    \midrule
    Organization Name & \textbf{Nomad Gear Co.} \\
    Primary Email Domain & \seqsplit{\texttt{NomadGearCo.com}} \\
    Primary Website & \url{www.NomadGearCo.com} \\
    External IP Address & \seqsplit{\texttt{117.8.205.222}} \\
    \bottomrule
\end{tabular}

% --- Section 3: Security Control Review ---
\section{Security Control Review}
A security questionnaire was completed to evaluate the implementation of fundamental security controls. The responses highlight significant gaps in the organization's defensive measures. A checkmark (\ding{51}) indicates a positive control, while a cross (\textcolor{red}{\ding{55}}) indicates a control gap.

\begin{table}[h!]
\centering
\caption{Security Controls Questionnaire Results}
\begin{tabular}{@{}lcc@{}}
\toprule
\textbf{Control Question} & \textbf{Response} & \textbf{Assessment} \\
\midrule
Do you require MFA to access email? & No & \textcolor{red}{\ding{55}} \\
Do you require MFA to log into computers? & No & \textcolor{red}{\ding{55}} \\
Do you require MFA to access sensitive data systems? & No & \textcolor{red}{\ding{55}} \\
Does your organization have an employee acceptable use policy? & No & \textcolor{red}{\ding{55}} \\
Does your organization do security awareness training for new employees? & Yes & \ding{51} \\
Does your organization do security awareness training for all employees at least once per year? & No & \textcolor{red}{\ding{55}} \\
\bottomrule
\end{tabular}
\end{table}

\subsection*{Analysis of Control Gaps}
The consistent "No" responses to all questions regarding Multi-Factor Authentication (MFA) are of the highest concern. MFA is an industry-standard control that provides a critical layer of defense against account takeover. Its absence across all key systems is a critical failure. Furthermore, the lack of an Acceptable Use Policy and annual security training weakens the human element of security, leaving the organization more susceptible to social engineering and insider threats.

% --- Section 4: Technical Scan Results ---
\section{Technical Scan Results}
A network scan was performed to identify open ports and services accessible on the target system.

\begin{itemize}
    \item \textbf{Target IP Address:} \seqsplit{\texttt{127.0.0.1}}
    \item \textbf{Scan Tool:} Nmap
\end{itemize}

\begin{table}[h!]
\centering
\caption{Open Ports Detected on \texttt{127.0.0.1}}
\begin{tabular}{@{}llll@{}}
\toprule
\textbf{Port} & \textbf{State} & \textbf{Service} & \textbf{Notes} \\
\midrule
22/tcp & open & SSH & Secure Shell access. No version information was available. \\
\bottomrule
\end{tabular}
\end{table}

\subsection*{Analysis of Technical Findings}
The scan confirms that port 22 (SSH) is open on the target system. While SSH is an encrypted protocol, its exposure is a significant risk, especially when secured only by a password. This finding validates the pre-existing risk documented in the following section and is exacerbated by the lack of MFA controls.

% --- Section 5: Consolidated Risk Assessment ---
\section{Consolidated Risk Assessment}
This section synthesizes findings from the security control review, technical scan, and pre-existing risk data into a consolidated list of key risks facing the organization.

\begin{table}[h!]
\centering
\caption{Summary of Identified Risks}
\begin{tabular}{@{}p{0.25\linewidth}p{0.45\linewidth}p{0.15\linewidth}@{}}
\toprule
\textbf{Risk Name} & \textbf{Description} & \textbf{Severity} \\
\midrule
\textbf{Exposed SSH Service without MFA} & The SSH service on \texttt{127.0.0.1} is exposed. Combined with the organization-wide lack of MFA, this creates a high likelihood of a successful brute-force or credential stuffing attack leading to system compromise. & \textcolor{sev_critical}{\textbf{Critical}} \\
\addlinespace
\textbf{Localhost Exposed} & A pre-existing documented risk confirming that the localhost interface is improperly exposed. This could allow attackers to bypass network-level security controls. & \textcolor{sev_critical}{\textbf{Critical}} \\
\addlinespace
\textbf{Lack of Multi-Factor Authentication} & A systemic failure to implement MFA for email, endpoints, and sensitive systems. This removes a critical security layer and makes user accounts highly vulnerable to takeover. & \textcolor{sev_critical}{\textbf{Critical}} \\
\addlinespace
\textbf{Inadequate Security Policy and Training} & The absence of an Acceptable Use Policy and annual security training for all staff increases the risk of human error, insider threat, and successful phishing attacks. & \textcolor{sev_high}{\textbf{High}} \\
\bottomrule
\end{tabular}
\end{table}

% --- Section 6: Recommendations ---
\section{Recommendations}
The following actions are recommended to mitigate the identified risks. Recommendations are prioritized based on severity.

\subsection*{Critical Priority}
\begin{enumerate}
    \item \textbf{Implement MFA Across All Critical Systems:} Immediately begin a phased rollout of MFA.
    \begin{itemize}
        \item \textbf{Phase 1 (Immediate):} Enable MFA for all remote access services (including SSH) and email accounts (e.g., Office 365, Google Workspace).
        \item \textbf{Phase 2 (Next 30 days):} Enable MFA for all sensitive data systems and administrative accounts.
        \item \textbf{Phase 3 (Next 90 days):} Enable MFA for all standard user computer logins.
    \end{itemize}
    
    \item \textbf{Restrict Network Access:} Review the network configuration for \texttt{127.0.0.1}. Ensure that this interface and its services (like SSH) are not accessible from any external or untrusted network. Access should be restricted by a firewall to only authorized administrative hosts.
\end{enumerate}

\subsection*{High Priority}
\begin{enumerate}
    \setcounter{enumi}{2} % Continue numbering from previous list
    \item \textbf{Develop and Implement Security Policies:} Draft and formally adopt a comprehensive set of security policies, starting with an \textbf{Acceptable Use Policy (AUP)}. Ensure all employees read and acknowledge the policy as a condition of using company resources.
    
    \item \textbf{Establish Annual Security Awareness Training:} Implement a mandatory security awareness training program for all employees to be completed annually. The training should cover key topics such as phishing, password security, and data handling.
\end{enumerate}

\end{document}
```