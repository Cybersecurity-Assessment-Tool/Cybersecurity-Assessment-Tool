```latex
\documentclass[12pt]{article}

% --- PACKAGE IMPORTS ---
\usepackage[margin=1in]{geometry}
\usepackage{pifont} % For checkmarks and crosses
\usepackage{booktabs} % For professional tables
\usepackage{hyperref} % For clickable links
\usepackage{url} % For URL formatting
\usepackage{seqsplit} % For splitting long strings in tt font
\usepackage{graphicx} % For logo (placeholder)
\usepackage{xcolor} % For colors in tables

% --- DOCUMENT METADATA ---
\title{Cybersecurity Posture Assessment Report}
\author{Cybersecurity Analysis Division}
\date{\today}

% --- HYPERREF SETUP ---
\hypersetup{
    colorlinks=true,
    linkcolor=blue,
    filecolor=magenta,      
    urlcolor=cyan,
    pdftitle={Cybersecurity Posture Assessment Report},
    pdfpagemode=FullScreen,
}

% --- CUSTOM COMMANDS ---
\newcommand{\severitycritical}[1]{\textcolor{red}{\textbf{#1}}}
\newcommand{\severityhigh}[1]{\textcolor{orange}{\textbf{#1}}}
\newcommand{\severitymedium}[1]{\textcolor{yellow!80!black}{\textbf{#1}}}
\newcommand{\severitylow}[1]{\textcolor{green}{\textbf{#1}}}

\begin{document}

\maketitle
\thispagestyle{empty}
\newpage

\tableofcontents
\newpage

% ===================================================================
% SECTION 1: EXECUTIVE SUMMARY
% ===================================================================
\section{Executive Summary}

This report provides a comprehensive analysis of the cybersecurity posture for \textbf{Solaris Energy}. The assessment is based on a correlation of network scan data, a review of existing security controls via a questionnaire, and an analysis of pre-identified risks.

The overall security posture reveals several critical and high-risk vulnerabilities that require immediate attention. The most significant finding is the direct network exposure of an outdated and unsupported MySQL database service. This technical vulnerability is severely compounded by critical gaps in organizational security controls, specifically the lack of Multi-Factor Authentication (MFA) for computer and sensitive system access, and the absence of a formal security awareness training program.

These weaknesses create a high-risk environment where a single compromised user credential could lead to a significant data breach. We strongly recommend prioritizing the remediation actions outlined in Section \ref{sec:recommendations} to mitigate these threats and strengthen the organization's defensive capabilities.

% ===================================================================
% SECTION 2: ORGANIZATIONAL INFORMATION
% ===================================================================
\section{Organizational Information}

The following information was provided for the assessment.

\begin{itemize}
    \item \textbf{Organization Name:} Solaris Energy
    \item \textbf{Primary Email Domain:} \texttt{SolarisEnergy.net}
    \item \textbf{Primary Website Domain:} \url{www.SolarisEnergy.net}
    \item \textbf{External IP Address:} \texttt{149.101.130.165}
\end{itemize}

% ===================================================================
% SECTION 3: SECURITY CONTROL REVIEW
% ===================================================================
\section{Security Control Review}

A questionnaire was used to evaluate the implementation of key administrative and technical security controls. The results are summarized in Table \ref{tab:controls}. Answers marked with \ding{55} indicate significant gaps in the security framework.

\begin{table}[h!]
\centering
\caption{Security Controls Questionnaire Results}
\label{tab:controls}
\begin{tabular}{p{0.8\linewidth} c}
\toprule
\textbf{Control Question} & \textbf{Status} \\
\midrule
Do you require MFA to access email? & \ding{51} \\
Do you require MFA to log into computers? & \severitycritical{\ding{55}} \\
Do you require MFA to access sensitive data systems? & \severitycritical{\ding{55}} \\
Does your organization have an employee acceptable use policy? & \ding{51} \\
Does your organization do security awareness training for new employees? & \severityhigh{\ding{55}} \\
Does your organization do security awareness training for all employees at least once per year? & \severityhigh{\ding{55}} \\
\bottomrule
\end{tabular}
\end{table}

\paragraph{Analysis:} The lack of MFA on computer and sensitive system logins represents a critical risk. Furthermore, the absence of a security awareness training program leaves the organization highly susceptible to phishing and other social engineering attacks, which are primary vectors for credential compromise.

% ===================================================================
% SECTION 4: TECHNICAL SCAN RESULTS
% ===================================================================
\section{Technical Scan Results}

A network scan was performed on the target host \texttt{172.16.50.20} to identify open ports and exposed services. The findings are detailed in Table \ref{tab:scan}.

\begin{table}[h!]
\centering
\caption{Open Port and Service Analysis}
\label{tab:scan}
\begin{tabular}{l l l l}
\toprule
\textbf{Port} & \textbf{Service} & \textbf{Product} & \textbf{Version} \\
\midrule
3306/tcp & mysql & MySQL & 5.7.33 \\
\bottomrule
\end{tabular}
\end{table}

\paragraph{Analysis:} The scan identified an open MySQL database port (3306). The running version, \textbf{MySQL 5.7.33}, reached its official End of Life (EOL) in October 2023. EOL software no longer receives security updates from the vendor, leaving it perpetually vulnerable to newly discovered exploits. Exposing an unsupported database service directly to the network is a critical security risk.

% ===================================================================
% SECTION 5: RISK ASSESSMENT & CORRELATION
% ===================================================================
\section{Risk Assessment \& Correlation}

This section synthesizes findings from the security control review, technical scan, and pre-existing risk data into a consolidated list of identified risks.

\begin{table}[h!]
\centering
\caption{Consolidated Risk Register}
\label{tab:risks}
\begin{tabular}{p{0.25\linewidth} p{0.55\linewidth} l}
\toprule
\textbf{Risk Title} & \textbf{Description} & \textbf{Severity} \\
\midrule
\textbf{Exposed \& Outdated Database} & A MySQL database (v5.7.33) is exposed to the network. This version is End-of-Life and no longer receives security patches, making it a prime target for attackers. & \severitycritical{Critical} \\
\addlinespace
\textbf{Insufficient MFA Implementation} & MFA is not enforced for computer logins or access to sensitive data systems. A compromised password would grant an attacker direct access to critical assets. & \severitycritical{Critical} \\
\addlinespace
\textbf{Inadequate Security Awareness Program} & The lack of employee security training significantly increases the likelihood of successful phishing attacks, malware infections, and credential theft. & \severityhigh{High} \\
\bottomrule
\end{tabular}
\end{table}

% ===================================================================
% SECTION 6: RECOMMENDATIONS
% ===================================================================
\section{Recommendations}
\label{sec:recommendations}

Based on the correlated findings, we recommend the following prioritized actions to mitigate the identified risks and improve the overall security posture.

\begin{enumerate}
    \item \textbf{Remediate Database Exposure (Immediate):}
    \begin{itemize}
        \item Implement strict firewall rules to restrict access to port 3306. Access should only be permitted from trusted application servers or specific administrative IP addresses.
        \item Public-facing access to databases is strongly discouraged. If remote access is required, it must be facilitated through a secure VPN with MFA.
    \end{itemize}

    \item \textbf{Deploy Comprehensive MFA (High Priority):}
    \begin{itemize}
        \item Enforce MFA for all user accounts for logging into company computers (desktops and laptops).
        \item Mandate MFA for access to all systems containing sensitive or critical data, including internal applications, cloud services, and administrative interfaces.
    \end{itemize}

    \item \textbf{Upgrade Unsupported Software (High Priority):}
    \begin{itemize}
        \item Develop a plan to migrate the MySQL 5.7.33 database to a currently supported version (e.g., MySQL 8.x). This is crucial for receiving ongoing security patches.
    \end{itemize}

    \item \textbf{Establish a Security Awareness Program (Medium Priority):}
    \begin{itemize}
        \item Implement a mandatory security awareness training program for all new and existing employees.
        \item Conduct training annually and supplement it with regular simulated phishing campaigns to measure effectiveness and maintain a high level of vigilance.
    \end{itemize}
\end{enumerate}

\end{document}
```