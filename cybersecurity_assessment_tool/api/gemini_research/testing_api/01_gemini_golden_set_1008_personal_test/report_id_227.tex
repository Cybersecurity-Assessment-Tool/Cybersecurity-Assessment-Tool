```latex
\documentclass[12pt]{article}

% Preamble: Required Packages
\usepackage[margin=1in]{geometry}
\usepackage{pifont} % For checkmarks and crosses (\ding)
\usepackage{booktabs} % For professional-looking tables
\usepackage{xcolor} % For coloring text
\usepackage[hidelinks]{hyperref} % For hyperlinks
\usepackage{url} % For properly formatting URLs
\usepackage{seqsplit} % To split long strings in tt font

% Custom Commands for convenience
\newcommand{\yes}{\textcolor{green}{\ding{51}}}
\newcommand{\no}{\textcolor{red}{\ding{55}}}
\newcommand{\riskcritical}{\textcolor{red!80!black}{\textbf{Critical}}}
\newcommand{\riskhigh}{\textcolor{orange!90!black}{\textbf{High}}}
\newcommand{\riskmedium}{\textcolor{yellow!80!black}{\textbf{Medium}}}

% Document Metadata
\title{Cybersecurity Posture Assessment Report \\ \large For: \textbf{Great Lakes}}
\author{Cybersecurity Analyst}
\date{\today}

\begin{document}

\maketitle
\thispagestyle{empty}
\newpage
\tableofcontents
\newpage

\section{Executive Summary}

This report provides a comprehensive cybersecurity assessment for \textbf{Great Lakes}, based on a combination of technical network scanning, a security controls questionnaire, and a review of pre-existing risks. The analysis reveals a mixed security posture with several critical vulnerabilities that require immediate attention.

While the organization has implemented Multi-Factor Authentication (MFA) for email and maintains an acceptable use policy, significant gaps exist. The most critical findings are the absence of MFA for computer logins and access to sensitive data systems. Furthermore, the lack of security awareness training for new employees creates a significant window of opportunity for social engineering attacks.

A technical scan confirmed a service listening on port 22 (commonly SSH) on the local loopback interface (\texttt{127.0.0.1}), which correlates with a pre-existing high-severity risk. The combination of these administrative and technical weaknesses elevates the organization's overall risk profile. This report outlines these findings in detail and provides a prioritized list of actionable recommendations to mitigate the identified risks and strengthen the overall security posture.

\section{Organizational Information}

The following information was provided for the assessment. This data is used to establish the context and scope of the review.

\begin{table}[h!]
\centering
\begin{tabular}{@{}ll@{}}
\toprule
\textbf{Attribute} & \textbf{Value} \\ \midrule
Organization Name & \textbf{Great Lakes} \\
Email Domain & \texttt{GreatLakes.org} \\
Website Domain & \url{www.GreatLakes.org} \\
External IP Address & \seqsplit{\texttt{26.128.71.49}} \\ \bottomrule
\end{tabular}
\caption{Client Organizational Data.}
\end{table}

\section{Security Control Review}

The following table summarizes the organization's responses to the security controls questionnaire. Each "No" response represents a potential gap in the security framework and is flagged as a significant finding.

\begin{table}[h!]
\centering
\begin{tabular}{@{}p{0.6\textwidth} c p{0.2\textwidth}@{}}
\toprule
\textbf{Control Question} & \textbf{Response} & \textbf{Assessment} \\ \midrule
Do you require MFA to access email? & \yes & Best Practice Met \\
Do you require MFA to log into computers? & \no & \riskcritical{} Gap \\
Do you require MFA to access sensitive data systems? & \no & \riskcritical{} Gap \\
Does your organization have an employee acceptable use policy? & \yes & Best Practice Met \\
Does your organization do security awareness training for new employees? & \no & \riskhigh{} Risk \\
Does your organization do security awareness training for all employees at least once per year? & \yes & Best Practice Met \\ \bottomrule
\end{tabular}
\caption{Security Controls Questionnaire Analysis.}
\end{table}

\section{Technical Scan Results}

A network scan was performed to identify open ports and services on the target system. The results provide insight into the external attack surface and potential points of entry.

\subsection{Scan Metadata}
\begin{itemize}
    \item \textbf{Target IP:} \texttt{127.0.0.1}
    \item \textbf{Scanner:} Nmap
    \item \textbf{Scan Date:} Not specified in scan data.
\end{itemize}

\subsection{Open Ports}
The scan identified the following open port. The service information is inferred from the standard port assignment, as the scan did not include service version detection.

\begin{table}[h!]
\centering
\begin{tabular}{@{}llll@{}}
\toprule
\textbf{Port} & \textbf{State} & \textbf{Inferred Service} & \textbf{Notes} \\ \midrule
22/tcp & open & SSH (Secure Shell) & Service confirmed listening. No version data available. \\ \bottomrule
\end{tabular}
\caption{Open Ports Detected on \texttt{127.0.0.1}.}
\end{table}

\section{Consolidated Risk Assessment}

This section correlates findings from the security control review, technical scan, and pre-existing risk data into a unified list. Each risk is assigned a severity level to aid in prioritization.

\begin{table}[h!]
\centering
\begin{tabular}{@{}p{0.25\textwidth} p{0.45\textwidth} l@{}}
\toprule
\textbf{Risk Name} & \textbf{Description} & \textbf{Severity} \\ \midrule
\textbf{Localhost Exposed} & A pre-existing risk notes a critical exposure on \texttt{127.0.0.1}. The technical scan confirms a service (likely SSH) is listening on this interface. & \riskcritical{} \\
\addlinespace
\textbf{Lack of MFA on Sensitive Systems} & The absence of MFA for systems containing sensitive data allows an attacker with stolen credentials to gain direct access. & \riskcritical{} \\
\addlinespace
\textbf{Lack of Endpoint MFA} & User computers are not protected by MFA. A compromised password could lead to a full endpoint compromise, serving as a foothold for lateral movement. & \riskcritical{} \\
\addlinespace
\textbf{Inadequate New Hire Training} & New employees do not receive security awareness training upon being hired, making them highly susceptible to phishing and social engineering attacks. & \riskhigh{} \\ \bottomrule
\end{tabular}
\caption{Summary of Identified Risks.}
\end{table}

\section{Recommendations}

Based on the consolidated risk assessment, the following prioritized actions are recommended to mitigate the identified vulnerabilities and improve the overall security posture of \textbf{Great Lakes}.

\subsection{Priority 1: Critical}
\begin{enumerate}
    \item \textbf{Implement MFA for Sensitive Systems:} Immediately enforce MFA for all user and administrative access to systems containing sensitive or critical business data. This is the single most effective control to prevent unauthorized data access.
    \item \textbf{Deploy Endpoint MFA:} Enforce MFA for all computer and laptop logins (e.g., Windows, macOS). This mitigates the risk of a compromised password leading to an endpoint compromise.
\end{enumerate}

\subsection{Priority 2: High}
\begin{enumerate}
    \item \textbf{Mandate Onboarding Security Training:} Integrate a mandatory security awareness training module into the new employee onboarding process. This should be completed before a new hire is granted access to company systems and data.
    \item \textbf{Review and Secure Exposed Services:} Conduct an urgent review of the service running on \texttt{127.0.0.1:22}. Confirm its business purpose, ensure it is not unintentionally exposed to external networks, and apply security hardening best practices (e.g., disable password authentication in favor of public key cryptography for SSH).
\end{enumerate}

\end{document}
```