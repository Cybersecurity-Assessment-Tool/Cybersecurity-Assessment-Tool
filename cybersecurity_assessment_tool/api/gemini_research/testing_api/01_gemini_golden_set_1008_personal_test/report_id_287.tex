```latex
\documentclass[12pt]{article}

% 1. DOCUMENT SETUP & PACKAGES
\usepackage[margin=1in]{geometry}
\usepackage{pifont} % For checkmarks and crosses
\usepackage{booktabs} % For professional tables
\usepackage{hyperref} % For clickable links
\usepackage{url} % For formatting URLs
\usepackage{seqsplit} % For splitting long text strings
\usepackage{graphicx}
\usepackage{xcolor}

% --- Document Metadata ---
\title{Cybersecurity Assessment Report}
\author{Cybersecurity Analysis Division}
\date{\today}

% --- Hyperref Setup ---
\hypersetup{
    colorlinks=true,
    linkcolor=blue,
    filecolor=magenta,      
    urlcolor=cyan,
    pdftitle={Cybersecurity Assessment Report},
    pdfpagemode=FullScreen,
}

\begin{document}

\maketitle
\thispagestyle{empty}
\newpage
\tableofcontents
\newpage

% 2. EXECUTIVE SUMMARY
\section{Executive Summary}

This report provides a comprehensive cybersecurity assessment for \textbf{Open Door}. The analysis is based on a combination of network scanning, a review of existing risk documentation, and an evaluation of organizational security controls via a questionnaire.

The assessment identified several high-risk areas requiring immediate attention. Critical gaps were found in administrative controls, including the absence of a mandatory Multi-Factor Authentication (MFA) policy for sensitive data systems, a lack of a formal Acceptable Use Policy, and no security awareness training program for employees. These deficiencies significantly increase the organization's susceptibility to social engineering, unauthorized access, and insider threats.

Technically, the scan confirmed a pre-existing critical risk: an exposed service on a local loopback interface (\texttt{127.0.0.1}). While this is typically a low-risk interface, its exposure in a production context can indicate severe misconfigurations, such as insecure port forwarding, which could allow an attacker to bypass perimeter defenses.

Urgent remediation is recommended, focusing on implementing foundational security controls (MFA, policies, training) and investigating the exposed service to mitigate the identified risks.

% 3. ORGANIZATIONAL INFORMATION
\section{Organizational Information}

The following details were provided for the assessment scope.

\begin{tabular}{@{}ll}
\toprule
\textbf{Attribute} & \textbf{Value} \\
\midrule
Organization Name & \textbf{Open Door} \\
Email Domain      & \texttt{OpenDoor.net} \\
Website Domain    & \url{www.OpenDoor.net} \\
External IP Address & \texttt{57.44.189.248} \\
\bottomrule
\end{tabular}

% 4. SECURITY CONTROL REVIEW (QUESTIONNAIRE)
\section{Security Control Review}

An administrative review was conducted based on a security questionnaire. The following table summarizes the organization's current security posture regarding key controls. Gaps identified with a \ding{55} represent significant risks.

\begin{table}[h!]
\centering
\begin{tabular}{@{}p{0.6\linewidth}cc@{}}
\toprule
\textbf{Control Question} & \textbf{Response} & \textbf{Status} \\
\midrule
Do you require MFA to access email? & Yes & \ding{51} \\
Do you require MFA to log into computers? & Yes & \ding{51} \\
\textbf{Do you require MFA to access sensitive data systems?} & \textbf{No} & \textcolor{red}{\ding{55}} \\
\textbf{Does your organization have an employee acceptable use policy?} & \textbf{No} & \textcolor{red}{\ding{55}} \\
\textbf{Does your organization do security awareness training for new employees?} & \textbf{No} & \textcolor{red}{\ding{55}} \\
\textbf{Does your organization do security awareness training for all employees at least once per year?} & \textbf{No} & \textcolor{red}{\ding{55}} \\
\bottomrule
\end{tabular}
\caption{Organizational Security Control Status}
\end{table}

The review reveals critical deficiencies in identity management for sensitive data and a complete absence of foundational policy and training programs.

% 5. TECHNICAL SCAN RESULTS
\section{Technical Scan Results}

A network scan was performed to identify open ports and exposed services on the target system. The scan confirmed the findings from the pre-existing risk documentation.

\begin{itemize}
    \item \textbf{Target IP Address:} \texttt{127.0.0.1}
    \item \textbf{Scan Date:} \today
\end{itemize}

\begin{table}[h!]
\centering
\begin{tabular}{@{}ccccc@{}}
\toprule
\textbf{Port} & \textbf{State} & \textbf{Service} & \textbf{Version} & \textbf{Notes} \\
\midrule
22/tcp & open & SSH (Inferred) & N/A & Service version detection was not performed. \\
\bottomrule
\end{tabular}
\caption{Open Ports Detected on \texttt{127.0.0.1}}
\end{table}

\textbf{Analysis:} The scan detected an open SSH port on the localhost interface. This directly corresponds to the "Localhost Exposed" risk identified in Input 3. Exposing SSH, even on localhost, can be a serious security risk if services are misconfigured (e.g., via remote port forwarding) or if an attacker gains local access. This finding requires immediate investigation to determine its business purpose and necessity.

% 6. CONSOLIDATED RISK ASSESSMENT
\section{Consolidated Risk Assessment}

The following table synthesizes findings from the security control review, technical scan, and pre-existing risk data into a consolidated list of identified risks.

\begin{table}[h!]
\centering
\begin{tabular}{@{}p{0.2\linewidth}p{0.5\linewidth}l@{}}
\toprule
\textbf{Risk Name} & \textbf{Description} & \textbf{Severity} \\
\midrule
\textbf{Localhost Exposed} & An SSH service is accessible on the local loopback interface (\texttt{127.0.0.1}), which may indicate a serious system misconfiguration. & \textbf{Critical (10.0)} \\
\addlinespace
\textbf{No MFA on Sensitive Systems} & The lack of Multi-Factor Authentication for sensitive data systems allows for potential unauthorized access via compromised credentials. & \textbf{High} \\
\addlinespace
\textbf{No Security Awareness Training} & Employees are not trained on security best practices, making the organization highly vulnerable to phishing, malware, and social engineering attacks. & \textbf{High} \\
\addlinespace
\textbf{No Acceptable Use Policy} & The absence of a formal policy defining acceptable use of company assets creates ambiguity and increases the risk of insider threat and misuse. & \textbf{High} \\
\bottomrule
\end{tabular}
\caption{Summary of Identified Risks}
\end{table}

% 7. RECOMMENDATIONS
\section{Recommendations}

The following actions are recommended to mitigate the identified risks. Recommendations are prioritized based on severity and potential impact.

\subsection{Immediate Priority (Remediate within 7 days)}

\begin{enumerate}
    \item \textbf{Investigate and Secure Exposed SSH Service:}
    \begin{itemize}
        \item Determine the business justification for the open SSH port on \texttt{127.0.0.1}.
        \item If the service is not required, disable it immediately.
        \item If required, ensure it is configured securely: restrict access to specific IP addresses, enforce key-based authentication, and disable password-based logins.
    \end{itemize}
    
    \item \textbf{Implement MFA on Sensitive Systems:}
    \begin{itemize}
        \item Immediately begin a project to enforce MFA on all systems classified as containing sensitive or critical data.
        \item Prioritize systems that are publicly accessible or hold personally identifiable information (PII).
    \end{itemize}
\end{enumerate}

\subsection{High Priority (Remediate within 30-60 days)}

\begin{enumerate}
    \item \textbf{Establish a Security Awareness Training Program:}
    \begin{itemize}
        \item Procure and implement a training solution for all employees.
        \item Ensure the program is mandatory for all new hires during onboarding.
        \item Conduct annual refresher training for all staff, including phishing simulations.
    \end{itemize}
    
    \item \textbf{Develop and Implement an Acceptable Use Policy (AUP):}
    \begin{itemize}
        \item Draft a formal AUP that clearly outlines the rules for using company networks, devices, and data.
        \item Require all employees to read and formally acknowledge the policy.
        \item Integrate the AUP into the new hire onboarding process.
    \end{itemize}
\end{enumerate}

\end{document}
```