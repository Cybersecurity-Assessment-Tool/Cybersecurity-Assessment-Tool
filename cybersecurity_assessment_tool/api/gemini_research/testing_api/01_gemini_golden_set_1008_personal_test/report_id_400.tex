```latex
\documentclass[12pt]{article}

% Preamble: Required Packages
\usepackage[margin=1in]{geometry}
\usepackage{pifont} % For checkmarks and crosses
\usepackage{booktabs} % For professional tables
\usepackage{hyperref} % For clickable links
\usepackage{url}      % For URL formatting
\usepackage{seqsplit} % For splitting long strings in texttt

% Document Metadata
\title{Cybersecurity Posture Assessment Report}
\author{Cybersecurity Analyst}
\date{\today}

\begin{document}

\maketitle
\thispagestyle{empty}
\newpage
\tableofcontents
\newpage

% --- 1. Executive Summary ---
\section{Executive Summary}
This report provides a cybersecurity posture assessment for \textbf{Velocity Ventures}, based on a synthesis of network scan data, organizational security control questionnaires, and a review of pre-existing risks. The assessment was conducted on \today.

The analysis reveals a mixed security posture. On a positive note, the targeted network host (\seqsplit{\texttt{192.168.1.100}}) presented a strong defensive stance with no open ports detected. The organization also demonstrates a commitment to security awareness through its training programs for both new and existing employees.

However, several critical and high-risk deficiencies were identified in foundational security controls. The complete absence of Multi-Factor Authentication (MFA) across email, computer logins, and sensitive data systems represents a \textbf{critical vulnerability}. This gap significantly increases the risk of account compromise through common attacks like phishing and credential stuffing. Furthermore, the lack of a formal Acceptable Use Policy (AUP) constitutes a high-risk gap in governance, creating ambiguity and increasing the potential for insider threats.

Urgent remediation is required to address the identified gaps in identity and access management and to establish essential security policies. Recommendations are detailed in Section \ref{sec:recommendations}.

% --- 2. Organizational Information ---
\section{Organizational Information}
The following information was provided for the assessment.

\begin{tabular}{@{}ll}
    \toprule
    \textbf{Attribute} & \textbf{Value} \\
    \midrule
    Organization Name & \textbf{Velocity Ventures} \\
    Email Domain & \texttt{VelocityVentures.com} \\
    Website Domain & \url{www.VelocityVentures.com} \\
    External IP Address & \seqsplit{\texttt{122.93.36.71}} \\
    \bottomrule
\end{tabular}

% --- 3. Security Control Review ---
\section{Security Control Review}
The following table summarizes the organization's responses to a security controls questionnaire. A checkmark (\ding{51}) indicates a positive control is in place, while a cross (\ding{55}) indicates a control gap.

\begin{table}[h!]
\centering
\begin{tabular}{@{}lc}
    \toprule
    \textbf{Control Question} & \textbf{Response} \\
    \midrule
    Do you require MFA to access email? & \ding{55} \\
    Do you require MFA to log into computers? & \ding{55} \\
    Do you require MFA to access sensitive data systems? & \ding{55} \\
    Does your organization have an employee acceptable use policy? & \ding{55} \\
    Does your organization do security awareness training for new employees? & \ding{51} \\
    Does your organization do security awareness training for all employees at least once per year? & \ding{51} \\
    \bottomrule
\end{tabular}
\caption{Organizational Security Control Status}
\label{tab:controls}
\end{table}

The review highlights critical deficiencies in Identity and Access Management (IAM) and a significant gap in security governance policy.

% --- 4. Technical Scan Results ---
\section{Technical Scan Results}
A network scan was performed to identify the external attack surface of the specified target.

\begin{itemize}
    \item \textbf{Target IP Address:} \seqsplit{\texttt{192.168.1.100}}
    \item \textbf{Scan Date:} \today
    \item \textbf{Status:} The host was found to be online.
    \item \textbf{Findings:} The scan confirmed that \textbf{no open TCP ports were detected}. All 1000 scanned ports were in a 'closed' state. This indicates a minimal network attack surface for this specific host, which is a strong security practice.
\end{itemize}

% --- 5. Risk Assessment ---
\section{Risk Assessment}
The following table synthesizes findings from the security control review, technical scans, and pre-existing risk data. The risks are prioritized based on their potential impact on the organization.

\begin{table}[h!]
\centering
\begin{tabular}{@{}p{0.1\textwidth} p{0.3\textwidth} p{0.15\textwidth} p{0.35\textwidth}@{}}
    \toprule
    \textbf{Risk ID} & \textbf{Risk Name} & \textbf{Severity} & \textbf{Description} \\
    \midrule
    RISK-001 & Lack of Multi-Factor Authentication (MFA) & \textbf{Critical} & The absence of MFA for email, computer, and sensitive system access exposes the organization to a high likelihood of account compromise. A single compromised password could grant an attacker full access. \\
    \addlinespace
    RISK-002 & Missing Acceptable Use Policy (AUP) & \textbf{High} & Without a formal AUP, employees lack clear guidance on the proper use of company assets. This increases the risk of accidental data exposure, malware infections, and insider threats. \\
    \bottomrule
\end{tabular}
\caption{Identified Risks and Severity}
\label{tab:risks}
\end{table}

% --- 6. Recommendations ---
\section{Recommendations}
\label{sec:recommendations}
Based on the risk assessment, the following actions are recommended to mitigate the identified vulnerabilities and strengthen the overall security posture.

\subsection{Remediation for RISK-001: Lack of MFA}
\begin{enumerate}
    \item \textbf{Immediate Priority:} Procure and implement an MFA solution for the corporate email system (e.g., Office 365, Google Workspace). This is the most critical step to prevent business email compromise.
    \item \textbf{High Priority:} Enforce MFA for all remote access solutions (e.g., VPN) and for all accounts with administrative privileges on any system.
    \item \textbf{Next Steps:} Develop a phased rollout plan to enable MFA for all employee computer logins and access to other systems containing sensitive data.
\end{enumerate}

\subsection{Remediation for RISK-002: Missing AUP}
\begin{enumerate}
    \item \textbf{Immediate Priority:} Develop a formal Acceptable Use Policy that clearly defines rules for employee use of networks, computers, email, internet, and mobile devices.
    \item \textbf{High Priority:} Communicate the new policy to all current employees and require them to formally acknowledge they have read and understood it.
    \item \textbf{Next Steps:} Integrate the AUP review and acknowledgment into the standard onboarding process for all new hires.
\end{enumerate}

\end{document}
```