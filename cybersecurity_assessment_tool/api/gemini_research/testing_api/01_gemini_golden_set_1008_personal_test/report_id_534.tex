```latex
\documentclass[12pt]{article}

% --- PACKAGE IMPORTS ---
\usepackage[margin=1in]{geometry}
\usepackage{pifont} % For checkmarks and crosses
\usepackage{booktabs} % For professional tables
\usepackage{hyperref} % For clickable links
\usepackage{url}      % For proper URL formatting
\usepackage{seqsplit} % To split long strings in texttt
\usepackage[utf8]{inputenc}

% --- DOCUMENT METADATA ---
\title{Cybersecurity Posture Assessment Report \\ \large For \textbf{Swift Current Labs}}
\author{Cybersecurity Analyst Group}
\date{\today}

% --- DOCUMENT START ---
\begin{document}

\maketitle
\thispagestyle{empty}
\newpage
\tableofcontents
\newpage

% ==============================================================================
% SECTION 1: EXECUTIVE OVERVIEW
% ==============================================================================
\section*{1. Executive Overview}

This report provides a comprehensive cybersecurity posture assessment for \textbf{Swift Current Labs}, conducted on \today. The analysis is based on a synthesis of technical network scanning, a review of organizational security controls, and an evaluation of pre-existing risk data.

The assessment identified several critical and high-risk gaps in administrative and identity management controls. Specifically, the lack of Multi-Factor Authentication (MFA) for email and computer access, coupled with deficiencies in employee security policies and onboarding training, presents a significant risk of unauthorized access and internal threats.

On a positive note, the technical network scan of the target system \texttt{192.168.0.5} revealed a secure configuration regarding web services. A previously documented risk of an unencrypted web server on Port 80 appears to have been remediated, as our scan confirmed this port is now closed.

Immediate action is required to address the identified policy and MFA-related vulnerabilities to mitigate the high likelihood of a security incident. Detailed findings and actionable recommendations are provided in the subsequent sections.

% ==============================================================================
% SECTION 2: ORGANIZATIONAL INFORMATION
% ==============================================================================
\section*{2. Organizational Information}

The following information was provided for the assessment.

\begin{table}[h!]
\centering
\begin{tabular}{@{}ll@{}}
\toprule
\textbf{Attribute} & \textbf{Value} \\ \midrule
Organization Name & \textbf{Swift Current Labs} \\
Email Domain & \texttt{SwiftCurrentLabs.org} \\
Website Domain & \url{www.SwiftCurrentLabs.org} \\
External IP Address & \texttt{77.20.109.14} \\ \bottomrule
\end{tabular}
\caption{Client Organizational Details.}
\end{table}

% ==============================================================================
% SECTION 3: SECURITY CONTROL REVIEW (QUESTIONNAIRE)
% ==============================================================================
\section*{3. Security Control Review}

A review of foundational security controls was conducted via a questionnaire. The responses indicate significant gaps in the organization's security posture. "No" answers represent a failure to implement a critical security control.

\begin{table}[h!]
\centering
\begin{tabular}{@{}lc@{}}
\toprule
\textbf{Security Control Question} & \textbf{Implemented?} \\ \midrule
Do you require MFA to access email? & \ding{55} \\
Do you require MFA to log into computers? & \ding{55} \\
Do you require MFA to access sensitive data systems? & \ding{51} \\
Does your organization have an employee acceptable use policy? & \ding{55} \\
Does your organization do security awareness training for new employees? & \ding{55} \\
Does your organization do security awareness training for all employees annually? & \ding{51} \\ \bottomrule
\end{tabular}
\caption{Organizational Security Control Status. (\ding{51}=Yes, \ding{55}=No)}
\end{table}

\subsection*{Analysis of Control Gaps}
\begin{itemize}
    \item \textbf{Lack of MFA:} The absence of MFA for email and computer logins is a critical vulnerability. Email is a primary target for phishing attacks, and compromised accounts can lead to widespread data breaches.
    \item \textbf{Policy and Training Deficiencies:} Without an Acceptable Use Policy, employees may be unaware of their security responsibilities. The lack of security training during onboarding leaves new hires particularly vulnerable to social engineering and other common attack vectors.
\end{itemize}

% ==============================================================================
% SECTION 4: TECHNICAL SCAN RESULTS
% ==============================================================================
\section*{4. Technical Scan Results}

A network scan was performed on the specified target to identify open ports and exposed services.

\begin{table}[h!]
\centering
\begin{tabular}{@{}lll@{}}
\toprule
\textbf{Target IP} & \textbf{Port / Protocol} & \textbf{State} \\ \midrule
\texttt{192.168.0.5} & 80 / TCP & Closed \\ \bottomrule
\end{tabular}
\caption{Nmap Scan Results.}
\end{table}

\subsection*{Scan Analysis}
The scan of target \texttt{192.168.0.5} found no open ports. Specifically, Port 80 (HTTP) was confirmed to be closed. This is a positive finding, as it indicates that the system is not exposing an unencrypted web service to the network. This result contradicts a previously identified risk, suggesting that remediation has occurred.

% ==============================================================================
% SECTION 5: CORRELATED RISK ASSESSMENT
% ==============================================================================
\section*{5. Correlated Risk Assessment}

This section synthesizes findings from the security control review, technical scan, and pre-existing risk data into a prioritized list of current risks.

\begin{table}[h!]
\centering
\begin{tabular}{@{}p{0.25\linewidth}p{0.5\linewidth}l@{}}
\toprule
\textbf{Risk Name} & \textbf{Description} & \textbf{Severity} \\ \midrule
\textbf{Account Compromise via Missing MFA} & Email and computer accounts lack MFA, making them highly susceptible to takeover through phishing or password guessing. A single compromised account could grant an attacker significant access. & \textbf{Critical} \\
\addlinespace
\textbf{Weakened Human Firewall} & The absence of an Acceptable Use Policy and security training for new hires creates an environment where employees are more likely to fall victim to social engineering or mishandle sensitive data. & \textbf{High} \\
\addlinespace
\textbf{Remediated Risk: Unencrypted Web Server} & A previously documented risk stated Port 80 was open. Our scan confirms this port is now closed on the target system. This risk should be marked as resolved in the risk register. & Informational \\ \bottomrule
\end{tabular}
\caption{Summary of Identified and Correlated Risks.}
\end{table}

% ==============================================================================
% SECTION 6: RECOMMENDATIONS
% ==============================================================================
\section*{6. Recommendations}

The following actions are recommended to mitigate the identified risks and improve the overall security posture of \textbf{Swift Current Labs}.

\subsection*{Immediate Priority (To Be Completed Within 30 Days)}
\begin{enumerate}
    \item \textbf{Enforce Multi-Factor Authentication (MFA):}
    \begin{itemize}
        \item Immediately enable and enforce MFA for all user accounts across the email system (e.g., \texttt{SwiftCurrentLabs.org}).
        \item Implement MFA for all remote and local computer login access.
    \end{itemize}
    \item \textbf{Develop and Implement an Acceptable Use Policy (AUP):}
    \begin{itemize}
        \item Draft a clear AUP that outlines the rules for using company technology and data.
        \item Require all current employees to read and formally acknowledge the policy. Integrate this process into the new hire onboarding flow.
    \end{itemize}
\end{enumerate}

\subsection*{High Priority (To Be Completed Within 90 Days)}
\begin{enumerate}
    \setcounter{enumi}{2} % Continue numbering
    \item \textbf{Establish a New Hire Security Training Program:}
    \begin{itemize}
        \item Create a mandatory security awareness training module that all new employees must complete during their first week of employment.
        \item This training should cover key topics such as phishing, password security, data handling, and the new AUP.
    \end{itemize}
    \item \textbf{Update Risk Register:}
    \begin{itemize}
        \item Formally close the risk item related to the "Unencrypted Web Server" on Port 80, documenting the successful remediation confirmed by this assessment.
    \end{itemize}
\end{enumerate}

\end{document}
```