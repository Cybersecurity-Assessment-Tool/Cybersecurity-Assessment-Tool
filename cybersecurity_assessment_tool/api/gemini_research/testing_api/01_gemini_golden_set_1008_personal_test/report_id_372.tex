An analysis of the provided data has been completed. The following is the synthesized, professional LaTeX report based on the inputs and your instructions.

```latex
\documentclass[12pt]{article}

% --- PACKAGES ---
\usepackage[margin=1in]{geometry}
\usepackage{pifont} % For checkmarks and crosses
\usepackage{booktabs} % For professional tables
\usepackage{hyperref} % For hyperlinks
\usepackage{url} % For URL formatting
\usepackage{seqsplit} % For splitting long strings in tt font
\usepackage{xcolor} % For colors

% --- HYPERREF SETUP ---
\hypersetup{
    colorlinks=true,
    linkcolor=blue,
    filecolor=magenta,      
    urlcolor=cyan,
    pdftitle={Cybersecurity Posture Assessment Report},
    pdfpagemode=FullScreen,
}

% --- DOCUMENT START ---
\begin{document}

% --- TITLE PAGE ---
\title{Cybersecurity Posture Assessment Report \\ \large For: \textbf{Nova Terra}}
\author{Cybersecurity Analysis Division}
\date{\today}
\maketitle
\thispagestyle{empty}
\newpage

% --- TABLE OF CONTENTS ---
\tableofcontents
\newpage

% --- SECTION 1: EXECUTIVE OVERVIEW ---
\section{Executive Overview}
This report details the findings of a cybersecurity posture assessment for \textbf{Nova Terra}. The assessment was conducted by analyzing organizational data provided via a security questionnaire. It is critical to note that the technical network scan data (\texttt{Input\_1\_Network\_Scan\_JSON}) and the list of pre-existing risks (\texttt{Input\_3\_Current\_Risks\_JSON}) were unavailable due to data corruption. Therefore, this assessment is based exclusively on the security control questionnaire.

The analysis of the questionnaire reveals several critical and high-risk gaps in the organization's security controls. The most pressing issues include the lack of Multi-Factor Authentication (MFA) for email and sensitive data systems. This exposes the organization to significant risks of account compromise and data breaches. 

Furthermore, the absence of a formal Acceptable Use Policy (AUP) and the lack of security awareness training for new employees create a high-risk environment where employees may be unaware of security best practices and organizational expectations, increasing the likelihood of human error leading to a security incident.

Immediate remediation of these identified gaps is strongly recommended to reduce the organization's attack surface and improve its overall security posture. A follow-up technical assessment is required to evaluate the external network infrastructure.

% --- SECTION 2: ORGANIZATIONAL INFORMATION ---
\section{Organizational Information}
The following details were provided for the assessment.

\begin{tabular}{@{}ll}
    \toprule
    \textbf{Attribute} & \textbf{Value} \\
    \midrule
    Organization Name & \textbf{Nova Terra} \\
    Email Domain & \texttt{NovaTerra.net} \\
    Website Domain & \texttt{www.NovaTerra.net} \\
    External IP Address & \seqsplit{\texttt{76.201.150.104}} \\
    \bottomrule
\end{tabular}

% --- SECTION 3: SECURITY CONTROL REVIEW ---
\section{Security Control Review (Questionnaire Analysis)}
The following table summarizes the organization's responses to the security controls questionnaire. Items marked with \ding{55} represent significant gaps in the security framework and are correlated to risks in Section 5.

\begin{table}[h!]
\centering
\begin{tabular}{@{}p{8cm}ccp{3cm}@{}}
    \toprule
    \textbf{Control Question} & \textbf{Response} & \textbf{Status} & \textbf{Assessment} \\
    \midrule
    Do you require MFA to access email? & No & \ding{55} & \textcolor{red}{\textbf{Critical Gap}} \\
    Do you require MFA to log into computers? & Yes & \ding{51} & Meets Best Practice \\
    Do you require MFA to access sensitive data systems? & No & \ding{55} & \textcolor{red}{\textbf{Critical Gap}} \\
    Does your organization have an employee acceptable use policy? & No & \ding{55} & \textcolor{orange}{High Risk} \\
    Does your organization do security awareness training for new employees? & No & \ding{55} & \textcolor{orange}{High Risk} \\
    Does your organization do security awareness training for all employees at least once per year? & Yes & \ding{51} & Meets Best Practice \\
    \bottomrule
\end{tabular}
\caption{Security Control Questionnaire Analysis}
\end{table}

% --- SECTION 4: TECHNICAL SCAN RESULTS ---
\section{Technical Scan Results}
\textbf{Note:} The input data for the network scan was corrupted and could not be parsed. Therefore, no technical findings regarding open ports or vulnerable services can be provided in this report. A new scan is required to assess the technical security of the external-facing infrastructure. 

The intended target for the scan was \texttt{[Target IP]}. A placeholder table is provided below for illustrative purposes.

\begin{table}[h!]
\centering
\begin{tabular}{@{}llll@{}}
    \toprule
    \textbf{Port} & \textbf{State} & \textbf{Service} & \textbf{Product / Version} \\
    \midrule
    N/A & N/A & Data Not Available & Data Not Available \\
    N/A & N/A & Data Not Available & Data Not Available \\
    N/A & N/A & Data Not Available & Data Not Available \\
    \bottomrule
\end{tabular}
\caption{External Network Scan Results (Data Unavailable)}
\end{table}

% --- SECTION 5: RISK ASSESSMENT ---
\section{Risk Assessment}
\textbf{Note:} The input data for pre-existing risks was corrupted. The risks listed below are derived solely from the Security Control Review in Section 3.

\begin{table}[h!]
\centering
\begin{tabular}{@{}lp{5cm}p{6cm}l@{}}
    \toprule
    \textbf{Risk ID} & \textbf{Risk Name} & \textbf{Description} & \textbf{Severity} \\
    \midrule
    RISK-001 & Lack of MFA on Critical Systems & Email and sensitive data systems are protected only by passwords, making them highly vulnerable to phishing, credential stuffing, and brute-force attacks. & \textcolor{red}{\textbf{Critical}} \\
    \addlinespace
    RISK-002 & Absence of Acceptable Use Policy (AUP) & Without a formal AUP, employees lack clear guidelines on the acceptable use of company assets, data handling, and security responsibilities, increasing insider threat risk. & \textcolor{orange}{\textbf{High}} \\
    \addlinespace
    RISK-003 & Inadequate Employee Onboarding Security & New employees are not provided with security awareness training, leaving a critical window where they are more susceptible to social engineering and policy violations. & \textcolor{orange}{\textbf{High}} \\
    \bottomrule
\end{tabular}
\caption{Identified Risks from Questionnaire Analysis}
\end{table}

% --- SECTION 6: RECOMMENDATIONS ---
\section{Recommendations}
Based on the findings of this assessment, the following actions are recommended to mitigate the identified risks and strengthen the security posture of \textbf{Nova Terra}. Recommendations are prioritized by severity.

\begin{enumerate}
    \item \textbf{[Critical] Implement Multi-Factor Authentication (MFA):}
    \begin{itemize}
        \item \textbf{Action:} Immediately enforce MFA for all user access to email services (e.g., Office 365, Google Workspace).
        \item \textbf{Action:} Enforce MFA for all systems that store, process, or transmit sensitive organizational or customer data.
        \item \textbf{Impact:} Drastically reduces the risk of account compromise and unauthorized access to critical data.
    \end{itemize}
    \vspace{0.5cm}
    
    \item \textbf{[High] Develop and Implement an Acceptable Use Policy (AUP):}
    \begin{itemize}
        \item \textbf{Action:} Create a formal AUP that clearly defines rules for computer, network, email, and internet usage. All employees must read and formally acknowledge the policy.
        \item \textbf{Impact:} Establishes a baseline for secure employee behavior and provides a framework for disciplinary action in case of violations.
    \end{itemize}
    \vspace{0.5cm}
    
    \item \textbf{[High] Integrate Security into Employee Onboarding:}
    \begin{itemize}
        \item \textbf{Action:} Develop a mandatory security awareness training module for all new hires. This module should cover the AUP, phishing identification, password security, and data handling procedures.
        \item \textbf{Impact:} Ensures that new employees are aware of their security responsibilities from day one, reducing the risk of early-tenure security incidents.
    \end{itemize}
    \vspace{0.5cm}
    
    \item \textbf{[Informational] Commission a New Technical Assessment:}
    \begin{itemize}
        \item \textbf{Action:} Schedule a new external network vulnerability scan and penetration test to identify technical vulnerabilities that could not be assessed due to the corrupted input data.
        \item \textbf{Impact:} Provides a complete picture of the organization's security posture by combining policy and technical assessments.
    \end{itemize}
\end{enumerate}

% --- DOCUMENT END ---
\end{document}
```