```latex
\documentclass[12pt]{article}

% Preamble: Required Packages
\usepackage[margin=1in]{geometry}
\usepackage{pifont} % For checkmarks and crosses
\usepackage{booktabs} % For professional tables
\usepackage{hyperref} % For clickable links
\usepackage{url} % For URL formatting
\usepackage{seqsplit} % To split long strings in tt font
\usepackage{xcolor} % For colors

% Document Information
\title{Cybersecurity Posture Assessment Report}
\author{Cybersecurity Analysis Division}
\date{November 22, 2025}

% Hyperref Setup
\hypersetup{
    colorlinks=true,
    linkcolor=blue,
    filecolor=magenta,      
    urlcolor=cyan,
    pdftitle={Cybersecurity Posture Assessment Report},
    pdfpagemode=FullScreen,
}

\begin{document}

\maketitle
\thispagestyle{empty}
\newpage

\tableofcontents
\newpage

% --- 1. Executive Summary ---
\section{Executive Summary}
This report provides a comprehensive cybersecurity assessment for \textbf{Nomad Gear Co.}, conducted on November 22, 2025. The analysis is based on a synthesis of external network scan data, a review of organizational security controls, and an evaluation of known risks.

The assessment reveals several critical and high-risk security gaps that require immediate attention. Key findings include:
\begin{itemize}
    \item \textbf{Critical Gaps in Access Control:} Multi-Factor Authentication (MFA) is not enforced for accessing email or other sensitive data systems, exposing the organization to significant risk of account compromise and data breach.
    \item \textbf{Vulnerable External Infrastructure:} The primary web server is running an outdated version of Nginx (1.18.0) with known vulnerabilities. Furthermore, a critical SSL certificate mismatch was identified, which erodes user trust and poses a security risk.
    \item \textbf{Deficient Security Policies:} The organization lacks a formal Acceptable Use Policy and does not provide security awareness training to new employees, creating a weak security culture and increasing the likelihood of human error.
\end{itemize}

Immediate remediation of these issues is crucial to mitigate the risk of a significant security incident. Detailed findings and actionable recommendations are provided in the subsequent sections of this report.

% --- 2. Organizational Information ---
\section{Organizational Information}
The following details were provided for the assessment. This information forms the baseline for understanding the organization's digital footprint.

\begin{tabular}{@{}ll}
\toprule
\textbf{Attribute} & \textbf{Value} \\
\midrule
Organization Name & \textbf{Nomad Gear Co.} \\
Email Domain & \texttt{NomadGearCo.org} \\
Website Domain & \url{www.NomadGearCo.org} \\
External IP Address & \texttt{92.17.241.186} \\
\bottomrule
\end{tabular}

% --- 3. Security Control Review ---
\section{Security Control Review}
A review of internal security controls was conducted via a standardized questionnaire. The responses highlight significant gaps in foundational security practices. A "No" response indicates a missing control and a potential area of high risk.

\begin{table}[h!]
\centering
\caption{Security Controls Questionnaire Results}
\begin{tabular}{@{}lc}
\toprule
\textbf{Control Question} & \textbf{Response} \\
\midrule
Do you require MFA to access email? & \ding{55} \\
Do you require MFA to log into computers? & \ding{51} \\
Do you require MFA to access sensitive data systems? & \ding{55} \\
Does your organization have an employee acceptable use policy? & \ding{55} \\
Does your organization do security awareness training for new employees? & \ding{55} \\
Does your organization do security awareness training for all employees annually? & \ding{51} \\
\bottomrule
\end{tabular}
\end{table}

\subsection*{Analysis}
The lack of MFA for email and sensitive systems (\ding{55}) represents a critical vulnerability. Email is a primary target for phishing and account takeover attacks, which can serve as a gateway to the entire organization. The absence of an Acceptable Use Policy and security training for new hires indicates a reactive, rather than proactive, approach to cybersecurity culture.

% --- 4. Technical Scan Results ---
\section{Technical Scan Results}
An external network scan was performed to identify open ports and exposed services on the organization's perimeter.

\begin{itemize}
    \item \textbf{Target IP Address:} \texttt{192.168.10.5}
    \item \textbf{Scan Date:} \texttt{2025-11-22T10:00:00Z}
\end{itemize}

\begin{table}[h!]
\centering
\caption{Open Ports and Services}
\begin{tabular}{@{}lllll}
\toprule
\textbf{Port} & \textbf{State} & \textbf{Service} & \textbf{Product} & \textbf{Version} \\
\midrule
443/tcp & open & https & nginx & 1.18.0 \\
\bottomrule
\end{tabular}
\end{table}

\subsection*{Analysis of Findings}
The scan identified one open port, 443 (HTTPS), which is expected for a web server. However, two significant issues were discovered:
\begin{enumerate}
    \item \textbf{Outdated Software:} The server is running Nginx version \textbf{1.18.0}, which was released in April 2020. This version is outdated and has multiple documented vulnerabilities (e.g., CVE-2021-23017) that could be exploited by attackers.
    \item \textbf{SSL Certificate Mismatch:} The SSL certificate presented by the server is for the common name \texttt{www.acme-corp.com}, which does not match the organization's domain (\texttt{www.NomadGearCo.org}). This is a critical misconfiguration that will cause browser trust warnings and could be exploited in man-in-the-middle attacks.
\end{enumerate}

% --- 5. Consolidated Risk Assessment ---
\section{Consolidated Risk Assessment}
The following table synthesizes findings from the security control review and the technical scan into a prioritized list of risks. As no pre-existing risks were provided, this assessment is based solely on the data gathered during this engagement.

\begin{table}[h!]
\centering
\caption{Summary of Identified Risks}
\begin{tabular}{@{}lp{6cm}ll}
\toprule
\textbf{Risk ID} & \textbf{Description} & \textbf{Severity} & \textbf{Source} \\
\midrule
RISK-001 & Lack of MFA on email and sensitive systems exposes the organization to account takeover and data breach. & \textbf{Critical} & Questionnaire \\
RISK-002 & The public-facing web server is running outdated software with known vulnerabilities. & \textbf{High} & Technical Scan \\
RISK-003 & An incorrect SSL certificate is configured, eroding user trust and creating security risks. & \textbf{High} & Technical Scan \\
RISK-004 & Absence of an Acceptable Use Policy leads to inconsistent security practices among employees. & \textbf{High} & Questionnaire \\
RISK-005 & New employees are not receiving security training, making them susceptible to social engineering from day one. & \textbf{Medium} & Questionnaire \\
\bottomrule
\end{tabular}
\end{table}

% --- 6. Recommendations ---
\section{Recommendations}
Based on the consolidated risk assessment, the following actions are recommended to improve the cybersecurity posture of \textbf{Nomad Gear Co.}. Recommendations are prioritized by severity.

\begin{description}
    \item[\textbf{For RISK-001 (Critical): Implement MFA Everywhere}]
    Immediately deploy a mandatory Multi-Factor Authentication (MFA) solution for all user accounts, prioritizing access to email (e.g., Office 365, Google Workspace) and any systems containing sensitive corporate or customer data.
    
    \item[\textbf{For RISK-002 (High): Remediate Vulnerable Web Server}]
    Upgrade the Nginx server on \texttt{92.17.241.186} from version 1.18.0 to the latest stable version. Implement a formal patch management policy to ensure all internet-facing systems are updated in a timely manner.
    
    \item[\textbf{For RISK-003 (High): Correct SSL Certificate}]
    Immediately replace the mismatched SSL certificate with a valid certificate issued for the domain \texttt{www.NomadGearCo.org}. This will resolve browser trust errors and secure communications.
    
    \item[\textbf{For RISK-004 (High): Establish Foundational Policies}]
    Develop and implement a formal Acceptable Use Policy (AUP) that clearly defines the rules for using company IT assets. Require all current employees to read and acknowledge the policy.
    
    \item[\textbf{For RISK-005 (Medium): Enhance Security Training Program}]
    Integrate mandatory security awareness training into the onboarding process for all new employees. This training should cover key topics such as phishing, password hygiene, and the new Acceptable Use Policy.
\end{description}

\end{document}
```