```latex
\documentclass[12pt]{article}

% Preamble: Required Packages
\usepackage[margin=1in]{geometry}
\usepackage{pifont} % For checkmarks and crosses
\usepackage{booktabs} % For professional tables
\usepackage{hyperref} % For clickable links
\usepackage{url} % For formatting URLs
\usepackage{seqsplit} % For splitting long strings without spaces
\usepackage{graphicx}
\usepackage{xcolor}

% Document Metadata
\title{Cybersecurity Posture Assessment Report}
\author{Cybersecurity Analysis Division}
\date{\today}

% Hyperref Setup
\hypersetup{
    colorlinks=true,
    linkcolor=blue,
    filecolor=magenta,      
    urlcolor=cyan,
    pdftitle={Cybersecurity Posture Assessment Report},
    pdfpagemode=FullScreen,
}

\begin{document}

\maketitle
\thispagestyle{empty}
\newpage

\tableofcontents
\newpage

% --- 1. Executive Summary ---
\section*{1. Executive Summary}

This report provides a cybersecurity assessment for \textbf{Neon Pulse Entertainment}, based on a correlation of network scan data, organizational security controls, and a review of pre-existing risks. The analysis was conducted on \today.

The assessment reveals critical deficiencies in foundational security controls. The most significant risks stem from a lack of Multi-Factor Authentication (MFA) for computer and sensitive data access, and the complete absence of an employee security awareness program. These gaps expose the organization to a high likelihood of account compromise, social engineering, and data breaches.

A notable discrepancy was identified between the current risk register and the live network scan. The risk register lists an "Unencrypted Web Server" on Port 80 as an active vulnerability; however, our technical scan confirmed that Port 80 is closed on the target system. This suggests a potential gap in the risk management lifecycle, where remediated risks are not being properly tracked or closed out.

Immediate remediation should focus on implementing MFA and establishing a security awareness training program to address the most severe risks to the organization.

% --- 2. Organizational Information ---
\section*{2. Organizational Information}

The following details were provided for the assessment. This information is used to establish the context and scope of the review.

\begin{tabular}{@{}ll}
\toprule
\textbf{Attribute} & \textbf{Value} \\
\midrule
Organization Name & \textbf{Neon Pulse Entertainment} \\
Email Domain & \texttt{NeonPulseEntertainment.net} \\
Website Domain & \url{www.NeonPulseEntertainment.net} \\
External IP Address & \texttt{159.207.172.53} \\
\bottomrule
\end{tabular}

% --- 3. Security Control Review ---
\section*{3. Security Control Review}

A review of administrative and technical security controls was conducted via a questionnaire. The responses indicate the current state of implemented security policies. "No" answers represent significant gaps in the organization's defense-in-depth strategy.

\begin{tabular}{@{}p{0.8\linewidth}c}
\toprule
\textbf{Control Question} & \textbf{Response} \\
\midrule
Do you require MFA to access email? & \ding{51} \\ % Yes
Do you require MFA to log into computers? & \textcolor{red}{\ding{55}} \\ % No
Do you require MFA to access sensitive data systems? & \textcolor{red}{\ding{55}} \\ % No
Does your organization have an employee acceptable use policy? & \textcolor{red}{\ding{55}} \\ % No
Does your organization do security awareness training for new employees? & \textcolor{red}{\ding{55}} \\ % No
Does your organization do security awareness training for all employees at least once per year? & \textcolor{red}{\ding{55}} \\ % No
\bottomrule
\end{tabular}

\vspace{1em}
\textbf{Analysis:} The lack of MFA on computer and sensitive data logins is a critical vulnerability. Furthermore, the absence of a security training program and an acceptable use policy indicates a significant weakness in the "human firewall," making the organization highly susceptible to phishing and other social engineering attacks.

% --- 4. Technical Scan Results ---
\section*{4. Technical Scan Results}

A network scan was performed to identify open ports and exposed services on the target system.

\begin{itemize}
    \item \textbf{Scan Target:} \texttt{192.168.0.5}
    \item \textbf{Scan Date:} \today
\end{itemize}

The scan results for the target host are summarized below.

\begin{tabular}{@{}llll}
\toprule
\textbf{Port} & \textbf{State} & \textbf{Service} & \textbf{Product / Version} \\
\midrule
80/tcp & closed & http & N/A (Port is closed) \\
\bottomrule
\end{tabular}

\vspace{1em}
\textbf{Analysis:} The scan indicates that the target system has a minimal attack surface, with no open ports detected in the common range. Notably, Port 80 (HTTP) was found to be \textbf{closed}. This finding directly contradicts the information in the current risk register (see Input 3), which lists an open Port 80 as an active risk. This discrepancy must be investigated to ensure the risk register is accurate.

% --- 5. Consolidated Risk Assessment ---
\section*{5. Consolidated Risk Assessment}

The following table synthesizes findings from the security control review, technical scan, and pre-existing risk data. Risks are prioritized based on their potential impact on the organization.

\begin{tabular}{@{}p{0.15\linewidth}p{0.55\linewidth}p{0.2\linewidth}}
\toprule
\textbf{Risk Name} & \textbf{Description} & \textbf{Severity} \\
\midrule
\textbf{Lack of MFA Enforcement} & MFA is not required for computer logins or access to sensitive data systems. A single compromised password could lead to widespread unauthorized access and data exfiltration. & \textbf{Critical} \\
\addlinespace
\textbf{No Security Awareness Program} & The absence of an acceptable use policy and security training leaves employees unprepared to identify and respond to threats like phishing, significantly increasing the risk of malware infection and credential theft. & \textbf{Critical} \\
\addlinespace
\textbf{Inconsistent Risk Register} & The risk register lists "Unencrypted Web Server" (Port 80) as an active risk, but technical scans show the port is closed. This indicates a flawed process for tracking and closing remediated vulnerabilities. & Medium \\
\bottomrule
\end{tabular}

% --- 6. Recommendations ---
\section*{6. Recommendations}

Based on the consolidated risk assessment, the following actions are recommended to improve the security posture of \textbf{Neon Pulse Entertainment}.

\subsection*{Immediate Priority (Critical Risks)}
\begin{enumerate}
    \item \textbf{Implement Multi-Factor Authentication (MFA):}
    \begin{itemize}
        \item Enforce MFA for all employee computer logins (e.g., via Windows Hello, Duo, or similar).
        \item Enforce MFA for access to all applications and repositories containing sensitive or critical business data.
    \end{itemize}
    \item \textbf{Establish a Security Awareness Program:}
    \begin{itemize}
        \item Develop and implement a mandatory security awareness training module for all new hires.
        \item Conduct mandatory annual security awareness training for all staff, covering topics such as phishing, password hygiene, and data handling.
    \end{itemize}
\end{enumerate}

\subsection*{High Priority}
\begin{enumerate}
    \setcounter{enumi}{2} % Continue numbering
    \item \textbf{Develop an Acceptable Use Policy (AUP):}
    \begin{itemize}
        \item Draft and ratify a formal AUP that clearly defines the rules for using company technology and data.
        \item Require all employees to read and acknowledge the policy as a condition of employment.
    \end{itemize}
\end{enumerate}

\subsection*{Medium Priority}
\begin{enumerate}
    \setcounter{enumi}{3} % Continue numbering
    \item \textbf{Validate and Update the Risk Register:}
    \begin{itemize}
        \item Conduct a full review of the existing risk register.
        \item Use technical validation tools (e.g., vulnerability scanners) to confirm whether listed risks are still present.
        \item Formally close any risks that have been remediated, such as the "Unencrypted Web Server" on Port 80, and document the remediation date and method.
    \end{itemize}
\end{enumerate}

\end{document}
```