```latex
\documentclass[12pt]{article}

% Preamble: Required Packages
\usepackage[margin=1in]{geometry}
\usepackage{pifont} % For checkmarks and crosses
\usepackage{booktabs} % For professional tables
\usepackage{hyperref} % For clickable links
\usepackage{url} % For formatting URLs
\usepackage{seqsplit} % For splitting long strings in texttt
\usepackage{graphicx}
\usepackage{xcolor}

% Document Metadata
\title{Cybersecurity Posture Assessment Report}
\author{Cybersecurity Analysis Division}
\date{\today}

% Hyperref Setup
\hypersetup{
    colorlinks=true,
    linkcolor=blue,
    filecolor=magenta,      
    urlcolor=cyan,
    pdftitle={Cybersecurity Posture Assessment Report},
    pdfpagemode=FullScreen,
}

\begin{document}

\maketitle
\thispagestyle{empty}
\newpage

\tableofcontents
\newpage

% --- 1. Executive Summary ---
\section{Executive Summary}

This report provides a cybersecurity posture assessment for \textbf{Nomad Gear Co.}. The analysis is based on a correlation of external network scan data, a security controls questionnaire, and a review of pre-existing risks.

The assessment reveals a mixed security posture. On the positive side, the external network scan of the target IP address \texttt{[Target IP]} did not identify any open ports, suggesting a strong firewall configuration and a minimal external attack surface. The organization also has several key security controls in place, including mandatory Multi-Factor Authentication (MFA) for email, computer logins, and access to sensitive systems.

However, a critical procedural gap was identified: the lack of mandatory annual security awareness training for all employees. This oversight significantly increases the organization's susceptibility to social engineering attacks, such as phishing, which remain one of the most common vectors for security breaches. This issue is categorized as a \textbf{High} severity risk.

The primary recommendation is to immediately implement a comprehensive, mandatory annual security awareness training program for all staff to mitigate this human-centric risk.

% --- 2. Organizational Information ---
\section{Organizational Information}

The following information was provided for the assessment.

\begin{tabular}{@{}ll}
\toprule
\textbf{Attribute} & \textbf{Value} \\
\midrule
Organization Name & \textbf{Nomad Gear Co.} \\
Primary Email Domain & \texttt{NomadGearCo.org} \\
Primary Website & \url{www.NomadGearCo.org} \\
External IP Scanned & \texttt{217.112.55.245} \\
\bottomrule
\end{tabular}

% --- 3. Security Control Review ---
\section{Security Control Review}

A review of the organization's security controls was conducted via a questionnaire. The responses are summarized below. A green checkmark (\textcolor{green}{\ding{51}}) indicates a positive control, while a red cross (\textcolor{red}{\ding{55}}) indicates a potential security gap.

\begin{table}[h!]
\centering
\begin{tabular}{@{}p{0.8\linewidth}c@{}}
\toprule
\textbf{Security Control Question} & \textbf{Response} \\
\midrule
Do you require MFA to access email? & \textcolor{green}{\ding{51}} \\
Do you require MFA to log into computers? & \textcolor{green}{\ding{51}} \\
Do you require MFA to access sensitive data systems? & \textcolor{green}{\ding{51}} \\
Does your organization have an employee acceptable use policy? & \textcolor{green}{\ding{51}} \\
Does your organization do security awareness training for new employees? & \textcolor{green}{\ding{51}} \\
Does your organization do security awareness training for all employees at least once per year? & \textcolor{red}{\ding{55}} \\
\bottomrule
\end{tabular}
\caption{Security Controls Questionnaire Results.}
\end{table}

The review indicates strong controls in identity and access management through the enforcement of MFA. However, the lack of recurring security training for all employees is a significant weakness.

% --- 4. Technical Scan Results ---
\section{Technical Scan Results}

An external network scan was performed to identify open ports and exposed services.

\begin{itemize}
    \item \textbf{Target IP Address:} \texttt{[Target IP]}
    \item \textbf{Scan Date:} Data Not Provided
\end{itemize}

\subsection{Scan Findings}
The network scan completed successfully but found \textbf{no open ports} on the target system. This is a strong security posture, indicating that a well-configured firewall is in place, effectively blocking unsolicited inbound traffic and minimizing the external attack surface. No vulnerabilities related to exposed services could be identified.

% --- 5. Risk Assessment ---
\section{Risk Assessment}

This section synthesizes findings from the security control review, technical scan, and pre-existing risk data. The primary risk identified during this assessment is procedural.

\begin{table}[h!]
\centering
\begin{tabular}{@{}p{0.3\linewidth}p{0.15\linewidth}p{0.45\linewidth}@{}}
\toprule
\textbf{Risk Name} & \textbf{Severity} & \textbf{Overview} \\
\midrule
\textbf{Lack of Annual Security Awareness Training} & \textbf{High} & While new employees receive training, the absence of an annual refresher course for all staff leaves the organization vulnerable. Attack techniques evolve, and employee awareness diminishes over time. This gap increases the likelihood of a successful phishing, business email compromise (BEC), or malware attack originating from human error. \\
\bottomrule
\end{tabular}
\caption{Identified Risks and Vulnerabilities.}
\end{table}

\textit{Note: No pre-existing vulnerabilities were provided for this assessment.}

% --- 6. Recommendations ---
\section{Recommendations}

Based on the analysis, the following actions are recommended to enhance the cybersecurity posture of \textbf{Nomad Gear Co.}.

\begin{enumerate}
    \item \textbf{Implement Annual Security Awareness Training (Critical):}
    \begin{itemize}
        \item \textbf{Action:} Develop and mandate an annual security awareness training program for all employees, including management.
        \item \textbf{Details:} The training should cover current threats such as phishing, ransomware, social engineering, and proper data handling. The program should include a confirmation of completion for audit and compliance purposes.
        \item \textbf{Impact:} Significantly reduces the risk of security incidents caused by human error and strengthens the organization's "human firewall."
    \end{itemize}
    \vspace{1em}
    \item \textbf{Maintain Strong Network Perimeter Security (Good Practice):}
    \begin{itemize}
        \item \textbf{Action:} Continue the current practice of maintaining a restrictive firewall policy.
        \item \textbf{Details:} Regularly review and audit firewall rules to ensure that only explicitly required ports and services are accessible from the internet.
        \item \textbf{Impact:} Preserves the minimal external attack surface, protecting internal systems from direct external threats.
    \end{itemize}
\end{enumerate}

% --- 7. Conclusion ---
\section{Conclusion}

\textbf{Nomad Gear Co.} has established a solid technical foundation for its external security, as evidenced by the lack of exposed services on its network perimeter. The enforcement of MFA across critical systems is also a commendable strength. However, the organization's defense is undermined by the absence of ongoing security education for its staff. By implementing the recommendation to establish a mandatory annual training program, \textbf{Nomad Gear Co.} can significantly mature its security posture and build a more resilient defense against modern cyber threats.

\end{document}
```