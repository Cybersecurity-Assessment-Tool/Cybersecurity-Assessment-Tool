```latex
\documentclass[12pt]{article}

% Preamble: Required Packages
\usepackage[margin=1in]{geometry}
\usepackage{pifont} % For checkmarks and crosses
\usepackage{booktabs} % For professional tables
\usepackage{hyperref} % For clickable links and references
\usepackage{url} % For formatting URLs
\usepackage{seqsplit} % To split long monospaced text
\usepackage{graphicx}
\usepackage{xcolor}

% Hyperref Setup
\hypersetup{
    colorlinks=true,
    linkcolor=black,
    filecolor=magenta,      
    urlcolor=blue,
    pdftitle={Cybersecurity Risk Assessment Report},
    pdfpagemode=FullScreen,
}

% Document Metadata
\title{Cybersecurity Risk Assessment Report \\ \large For: Midnight Oil Studios}
\author{Cybersecurity Analysis Division}
\date{\today}

\begin{document}

\maketitle
\thispagestyle{empty}
\newpage

\tableofcontents
\thispagestyle{empty}
\newpage

\pagestyle{headings}

% --- 1. Executive Summary ---
\section{Executive Summary}
This report provides a comprehensive cybersecurity risk assessment for Midnight Oil Studios, based on an analysis of network scan data, organizational security controls, and pre-existing risk information. The assessment was conducted on \today.

The analysis revealed several critical and high-risk findings that require immediate attention. The most significant concern is the systemic exposure of Remote Desktop Protocol (RDP) services on the internal network. This technical vulnerability is critically compounded by a policy gap: the absence of Multi-Factor Authentication (MFA) for computer logins. This combination creates a high-impact attack vector where a single compromised password could lead to a significant network breach.

Furthermore, a gap in the employee onboarding process, specifically the lack of initial security awareness training for new hires, elevates the risk of credential compromise through social engineering or phishing attacks.

Key findings include:
\begin{itemize}
    \item \textbf{Critical Risk:} Open RDP port discovered on a new host (\texttt{10.10.10.51}), indicating a recurring pattern of insecure configuration, as a similar issue was previously identified on another host.
    \item \textbf{Critical Risk:} Lack of mandatory MFA for logging into company computers, which nullifies a fundamental security layer for endpoint protection.
    \item \textbf{High Risk:} New employees do not receive security awareness training upon being hired, leaving a window of vulnerability.
\end{itemize}

This report details these findings and provides actionable recommendations to mitigate the identified risks and strengthen the overall security posture of Midnight Oil Studios.

% --- 2. Organizational Information ---
\section{Organizational Information}
The following information was provided for the assessment.

\begin{tabular}{@{}ll}
\toprule
\textbf{Attribute} & \textbf{Value} \\
\midrule
Organization Name & Midnight Oil Studios \\
Email Domain & \texttt{MidnightOilStudios.net} \\
Website Domain & \seqsplit{\url{www.MidnightOilStudios.net}} \\
External IP Address & \texttt{227.196.51.127} \\
\bottomrule
\end{tabular}

% --- 3. Security Control Review ---
\section{Security Control Review}
A review of the organization's security controls was conducted based on a standardized questionnaire. The responses highlight key areas of strength and weakness in the current security policy framework. Gaps identified in this review are directly correlated with increased risk.

\begin{table}[h!]
\centering
\caption{Security Control Questionnaire Analysis}
\begin{tabular}{@{}p{8cm}cc}
\toprule
\textbf{Control Question} & \textbf{Response} & \textbf{Assessment} \\
\midrule
Do you require MFA to access email? & \ding{51} Yes & Compliant \\
\textbf{Do you require MFA to log into computers?} & \textbf{\color{red}\ding{55} No} & \textbf{\color{red}Critical Gap} \\
Do you require MFA to access sensitive data systems? & \ding{51} Yes & Compliant \\
Does your organization have an employee acceptable use policy? & \ding{51} Yes & Compliant \\
\textbf{Does your organization do security awareness training for new employees?} & \textbf{\color{red}\ding{55} No} & \textbf{\color{red}High Risk} \\
Does your organization do security awareness training for all employees at least once per year? & \ding{51} Yes & Compliant \\
\bottomrule
\end{tabular}
\end{table}

% --- 4. Technical Scan Results ---
\section{Technical Scan Results}
A network scan was performed to identify active services on the target system. The results provide technical evidence of the system's configuration and potential exposures.

\begin{itemize}
    \item \textbf{Target IP Address:} \texttt{10.10.10.51}
    \item \textbf{Scan Date:} Data provided on \today
\end{itemize}

The following open port was identified:
\begin{table}[h!]
\centering
\caption{Open Port Analysis for \texttt{10.10.10.51}}
\begin{tabular}{@{}llll@{}}
\toprule
\textbf{Port} & \textbf{State} & \textbf{Service Name} & \textbf{Analysis} \\
\midrule
3389/tcp & Open & \texttt{ms-wbt-server} & Microsoft Remote Desktop Protocol (RDP). \\
\bottomrule
\end{tabular}
\end{table}

\subsection*{Finding Analysis}
The discovery of an open RDP port on host \texttt{10.10.10.51} is a significant finding. RDP is a primary target for attackers seeking to gain unauthorized remote access to a network. When combined with the pre-existing risk of RDP exposure on host \texttt{10.10.10.50}, this finding suggests a systemic issue in server deployment and hardening procedures rather than an isolated incident.

% --- 5. Correlated Risk Assessment ---
\section{Correlated Risk Assessment}
This section synthesizes the findings from the security control review, technical scan, and pre-existing risk data to provide a holistic view of the current risk landscape.

\begin{table}[h!]
\centering
\caption{Summary of Identified Risks}
\begin{tabular}{@{}p{2.5cm}p{7.5cm}l@{}}
\toprule
\textbf{Risk Name} & \textbf{Description} & \textbf{Severity} \\
\midrule
\textbf{Systemic RDP Exposure} & The technical scan identified RDP open on \texttt{10.10.10.51}. This correlates with a known risk on \texttt{10.10.10.50}, indicating a pattern of insecure remote access configuration. This service is a frequent target for brute-force and exploit attacks. & \textbf{Critical} \\
\addlinespace
\textbf{Lack of Endpoint MFA} & The policy of not requiring MFA for computer logins, combined with exposed RDP, creates a critical vulnerability. A compromised password is all an attacker needs to gain remote control of a corporate endpoint. & \textbf{Critical} \\
\addlinespace
\textbf{Inadequate Employee Onboarding Security} & New employees are not provided with security awareness training. This makes them highly susceptible to phishing and social engineering attacks, increasing the likelihood of credential theft that could be used to exploit other vulnerabilities. & \textbf{High} \\
\bottomrule
\end{tabular}
\end{table}

% --- 6. Recommendations ---
\section{Recommendations}
The following actions are recommended to mitigate the identified risks. Recommendations are prioritized based on severity and potential impact.

\subsection{Remediation for Systemic RDP Exposure (Critical)}
\begin{itemize}
    \item \textbf{Immediate Action:}
    \begin{enumerate}
        \item Immediately close port 3389 on \texttt{10.10.10.51} and any other internal systems where RDP access is not business-critical.
        \item For any system requiring RDP, implement strict firewall rules to restrict access to only authorized administrative IP addresses.
    \end{enumerate}
    \item \textbf{Long-Term Strategy:}
    \begin{enumerate}
        \item Implement a secure remote access solution, such as a Virtual Private Network (VPN) or a Zero Trust Network Access (ZTNA) gateway.
        \item Mandate that all administrative access, including RDP, must be routed through this secure gateway, which should enforce MFA.
    \end{enumerate}
\end{itemize}

\subsection{Remediation for Lack of Endpoint MFA (Critical)}
\begin{itemize}
    \item \textbf{Immediate Action:}
    \begin{enumerate}
        \item Prioritize and deploy a robust MFA solution for all employee computer and server logins.
        \item Solutions like Windows Hello for Business, Duo Security, or other FIDO2-compliant authenticators should be evaluated and implemented across the organization.
    \end{enumerate}
\end{itemize}

\subsection{Remediation for Inadequate Onboarding Security (High)}
\begin{itemize}
    \item \textbf{Immediate Action:}
    \begin{enumerate}
        \item Integrate a mandatory security awareness training module into the new employee onboarding process.
        \item This training must cover, at a minimum: phishing and social engineering awareness, strong password creation and management, and the company's acceptable use policy.
    \end{enumerate}
\end{itemize}

\end{document}
```