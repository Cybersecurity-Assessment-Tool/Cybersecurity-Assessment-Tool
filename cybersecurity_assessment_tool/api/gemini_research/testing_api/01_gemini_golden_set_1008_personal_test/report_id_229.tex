```latex
\documentclass[12pt]{article}

% --- PACKAGES ---
\usepackage[margin=1in]{geometry}
\usepackage{pifont} % For checkmarks and crosses
\usepackage{booktabs} % For professional-looking tables
\usepackage{hyperref} % For clickable links
\usepackage{url} % For formatting URLs
\usepackage{seqsplit} % For splitting long strings in tt font

% --- DOCUMENT METADATA ---
\title{Cybersecurity Posture Assessment Report}
\author{Cybersecurity Analysis Division}
\date{November 22, 2025}

% --- HYPERREF SETUP ---
\hypersetup{
    colorlinks=true,
    linkcolor=black,
    urlcolor=blue,
    pdftitle={Cybersecurity Posture Assessment Report},
    pdfauthor={Cybersecurity Analysis Division},
}

\begin{document}

\maketitle

% ===================================================================
% 1. EXECUTIVE OVERVIEW
% ===================================================================
\section*{Executive Overview}

This report provides a cybersecurity posture assessment for \textbf{True Grit}, conducted on November 22, 2025. The analysis combines a review of organizational security controls, external network scanning, and a review of pre-existing risks.

The assessment reveals a mixed security posture. The organization has implemented several key security controls, including mandatory Multi-Factor Authentication (MFA) for email and sensitive systems, as well as a robust security awareness training program. These are commendable foundational practices.

However, two significant risks were identified that require immediate attention. Firstly, there is a critical gap in endpoint security, as MFA is not required for computer logins. This exposes the organization to significant risk from compromised credentials. Secondly, the external-facing web server is running an outdated version of Nginx (1.18.0), which has publicly known vulnerabilities.

This report details these findings and provides actionable recommendations to mitigate the identified risks and strengthen the overall security posture.

% ===================================================================
% 2. ORGANIZATIONAL INFORMATION
% ===================================================================
\section{Organizational Information}

The following details were provided for the assessment.

\begin{itemize}
    \item \textbf{Organization Name:} True Grit
    \item \textbf{Email Domain:} \seqsplit{\texttt{TrueGrit.net}}
    \item \textbf{Website Domain:} \seqsplit{\texttt{www.TrueGrit.net}}
    \item \textbf{External IP Address:} \seqsplit{\texttt{52.226.227.87}}
\end{itemize}

% ===================================================================
% 3. SECURITY CONTROL REVIEW
% ===================================================================
\section{Security Control Review}

A review of the organization's security controls was conducted via a questionnaire. The responses indicate a strong foundation in policy and training but highlight a critical gap in endpoint access controls.

\begin{table}[h!]
\centering
\caption{Security Control Questionnaire Responses}
\begin{tabular}{p{0.75\linewidth} c}
\toprule
\textbf{Control Question} & \textbf{Response} \\
\midrule
Do you require MFA to access email? & \ding{51} \\
Do you require MFA to log into computers? & \textbf{\color{red}\ding{55}} \\
Do you require MFA to access sensitive data systems? & \ding{51} \\
Does your organization have an employee acceptable use policy? & \ding{51} \\
Does your organization do security awareness training for new employees? & \ding{51} \\
Does your organization do security awareness training for all employees at least once per year? & \ding{51} \\
\bottomrule
\end{tabular}
\end{table}

% ===================================================================
% 4. TECHNICAL SCAN RESULTS
% ===================================================================
\section{Technical Scan Results}

An external network scan was performed to identify open ports and exposed services on the public-facing infrastructure.

\begin{itemize}
    \item \textbf{Target IP:} \seqsplit{\texttt{192.168.10.5}}
    \item \textbf{Scan Date:} 2025-11-22T10:00:00Z
\end{itemize}

The scan identified the following open port:

\begin{table}[h!]
\centering
\caption{Open Ports and Services}
\begin{tabular}{l l l l l}
\toprule
\textbf{Port} & \textbf{State} & \textbf{Service} & \textbf{Product} & \textbf{Version} \\
\midrule
443/tcp & Open & https & nginx & \textbf{\color{red}1.18.0} \\
\bottomrule
\end{tabular}
\end{table}

\subsection*{Analysis}
The scan detected an Nginx web server, version 1.18.0, exposed to the internet. This version was released in April 2020 and is now considered outdated. It is known to be affected by several security vulnerabilities, including CVE-2021-23017, which could allow an attacker to cause a denial of service or potentially execute arbitrary code under specific conditions. Running outdated software on internet-facing systems presents a high level of risk.

% ===================================================================
% 5. RISK ASSESSMENT SUMMARY
% ===================================================================
\section{Risk Assessment Summary}

The following table synthesizes findings from the security control review and technical scan. No pre-existing vulnerabilities were documented.

\begin{table}[h!]
\centering
\caption{Identified Risks}
\begin{tabular}{p{0.1\linewidth} p{0.3\linewidth} p{0.4\linewidth} p{0.1\linewidth}}
\toprule
\textbf{Risk ID} & \textbf{Risk Name} & \textbf{Description} & \textbf{Severity} \\
\midrule
RISK-001 & Lack of Endpoint MFA & Workstation and server logins are protected only by passwords. Compromised credentials could lead to direct, unauthorized access to internal systems. & \textbf{High} \\
\addlinespace
RISK-002 & Outdated Web Server Software & The public-facing Nginx server (v1.18.0) is outdated and has known vulnerabilities, exposing the organization to web-based attacks and potential system compromise. & \textbf{High} \\
\bottomrule
\end{tabular}
\end{table}

% ===================================================================
% 6. RECOMMENDATIONS
% ===================================================================
\section{Recommendations}

Based on the risk assessment, the following actions are recommended to improve the security posture of \textbf{True Grit}.

\subsection*{RISK-001: Lack of Endpoint MFA}
\begin{itemize}
    \item \textbf{Immediate Action:} Develop a project plan to deploy Multi-Factor Authentication for all employee computer and server logins.
    \item \textbf{Implementation:} Evaluate and implement an endpoint MFA solution compatible with the existing IT environment (e.g., Windows Hello for Business, Duo Security, Okta).
    \item \textbf{Policy:} Update the organization's access control policy to mandate the use of MFA for all system logins.
\end{itemize}

\subsection*{RISK-002: Outdated Web Server Software}
\begin{itemize}
    \item \textbf{Immediate Action:} Upgrade the Nginx web server from version 1.18.0 to the latest stable version recommended by the vendor. This should be tested in a non-production environment before deployment.
    \item \textbf{Ongoing Process:} Implement a formal patch and vulnerability management program. This program should include regular scanning of all external and internal systems to identify and remediate outdated software in a timely manner.
\end{itemize}

\end{document}
```