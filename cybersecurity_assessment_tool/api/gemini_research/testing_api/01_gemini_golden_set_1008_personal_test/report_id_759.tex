```latex
\documentclass[12pt]{article}

% --- PACKAGES ---
\usepackage[margin=1in]{geometry}
\usepackage{pifont} % For checkmarks and crosses
\usepackage{booktabs} % For professional tables
\usepackage{hyperref} % For clickable links
\usepackage{url} % For URL formatting
\usepackage{seqsplit} % To split long strings without breaking
\usepackage{graphicx}
\usepackage{xcolor}

% --- DOCUMENT SETUP ---
\hypersetup{
    colorlinks=true,
    linkcolor=blue,
    filecolor=magenta,      
    urlcolor=cyan,
    pdftitle={Cybersecurity Posture Report},
    pdfpagemode=FullScreen,
}

\newcommand{\yes}{\ding{51}}
\newcommand{\no}{\ding{55}}

\begin{document}

% --- TITLE PAGE ---
\begin{titlepage}
    \centering
    \vspace*{1cm}
    \Huge
    \textbf{Cybersecurity Posture Report}
    \vspace{1.5cm}
    \Large
    Prepared for: \\
    \vspace{0.5cm}
    \textbf{Pacific Rim Exports}
    \vspace{2cm}
    \rule{\linewidth}{0.5mm}
    \vspace{0.5cm}
    \large
    \textit{This report contains a synthesized analysis of organizational security controls, technical network scans, and pre-existing risk data.}
    \vspace{0.5cm}
    \rule{\linewidth}{0.5mm}
    \vfill
    \large
    \today
\end{titlepage}

\tableofcontents
\newpage

% --- SECTION 1: EXECUTIVE SUMMARY ---
\section{Executive Summary}
This report provides a comprehensive cybersecurity assessment for \textbf{Pacific Rim Exports}, correlating organizational policies, technical vulnerabilities, and known risks. The analysis indicates a mixed security posture with notable strengths and critical weaknesses.

\textbf{Strengths:} The organization has successfully implemented Multi-Factor Authentication (MFA) across key areas, including email, computer logins, and access to sensitive systems. This significantly reduces the risk of unauthorized access via compromised credentials.

\textbf{Critical Weaknesses:} Two primary areas of critical concern were identified:
\begin{enumerate}
    \item \textbf{Exposed End-of-Life Database:} A network scan identified a MySQL database server (\texttt{172.16.50.20}) with port 3306 open to the network. The running version, MySQL 5.7.33, is past its End-of-Life (EOL) as of October 2023 and no longer receives security updates. This presents a severe and immediate risk of data breach.
    \item \textbf{Onboarding Security Gap:} The organization does not provide security awareness training for new employees during their onboarding process. This gap leaves the organization vulnerable, as new hires are often prime targets for social engineering attacks before they are fully integrated into the corporate security culture.
\end{enumerate}

Immediate action is required to remediate the exposed database and to implement a security training module for all new hires. Detailed findings and actionable recommendations are provided in the subsequent sections.

% --- SECTION 2: ORGANIZATIONAL INFORMATION ---
\section{Organizational Information}
The following details were provided for the assessment.

\begin{tabular}{@{}ll}
    \toprule
    \textbf{Attribute} & \textbf{Value} \\
    \midrule
    Organization Name & \textbf{Pacific Rim Exports} \\
    Email Domain & \texttt{PacificRimExports.net} \\
    Website Domain & \url{www.PacificRimExports.net} \\
    External IP Address & \texttt{147.254.148.20} \\
    \bottomrule
\end{tabular}

% --- SECTION 3: SECURITY CONTROL REVIEW ---
\section{Security Control Review}
A review of the organization's security controls was conducted via a questionnaire. The results indicate a strong focus on authentication controls but a significant gap in employee security onboarding.

\begin{table}[h!]
    \centering
    \caption{Security Questionnaire Results}
    \begin{tabular}{@{}lc@{}}
        \toprule
        \textbf{Security Control Question} & \textbf{Status} \\
        \midrule
        Do you require MFA to access email? & \yes \\
        Do you require MFA to log into computers? & \yes \\
        Do you require MFA to access sensitive data systems? & \yes \\
        Does your organization have an employee acceptable use policy? & \yes \\
        Does your organization do security awareness training for new employees? & \textcolor{red}{\no} \\
        Does your organization do security awareness training for all employees at least once per year? & \yes \\
        \bottomrule
    \end{tabular}
\end{table}

\textbf{Analysis:} The lack of security awareness training for new employees is a critical process gap. New hires are often unfamiliar with corporate policies and are more susceptible to phishing and other social engineering attacks. While annual training is in place, the initial period of employment represents a window of high risk that must be addressed.

% --- SECTION 4: TECHNICAL SCAN RESULTS ---
\section{Technical Scan Results}
A network scan was performed to identify open ports and services on the target system. The findings confirm the pre-existing risk of database exposure and add a critical vulnerability related to outdated software.

\begin{itemize}
    \item \textbf{Target IP Address:} \texttt{172.16.50.20}
    \item \textbf{Scan Status:} Host is up.
\end{itemize}

\begin{table}[h!]
    \centering
    \caption{Open Ports and Services Detected}
    \begin{tabular}{@{}lllll@{}}
        \toprule
        \textbf{Port} & \textbf{State} & \textbf{Service} & \textbf{Product} & \textbf{Version} \\
        \midrule
        3306/tcp & Open & mysql & MySQL & 5.7.33 \\
        \bottomrule
    \end{tabular}
\end{table}

\textbf{Analysis:}
\begin{itemize}
    \item \textbf{Exposed Service:} Port 3306 (MySQL) is open. Database ports should not be exposed to general network traffic and should be restricted by a firewall to allow access only from specific application servers.
    \item \textbf{End-of-Life (EOL) Software:} The detected version, \textbf{MySQL 5.7.33}, is no longer supported by its developer. This means it does not receive security patches for newly discovered vulnerabilities. Running EOL software, especially on a network-accessible service, poses a critical risk of exploitation.
\end{itemize}

% --- SECTION 5: RISK ASSESSMENT SUMMARY ---
\section{Risk Assessment Summary}
The following table synthesizes findings from the security control review, technical scan, and pre-existing risk data into a prioritized list.

\begin{table}[h!]
    \centering
    \caption{Synthesized Risk Register}
    \begin{tabular}{@{}p{0.3\linewidth}p{0.5\linewidth}l@{}}
        \toprule
        \textbf{Risk Name} & \textbf{Overview} & \textbf{Severity} \\
        \midrule
        \textbf{Exposed End-of-Life Database Service} & Port 3306 is open on the network for a MySQL 5.7.33 database. This version is past its end-of-life and is unpatched against new vulnerabilities. & \textbf{Critical} \\
        \addlinespace
        \textbf{Inadequate Employee Onboarding Security} & New employees do not receive security awareness training upon being hired, creating a significant vulnerability to social engineering. & \textbf{High} \\
        \bottomrule
    \end{tabular}
\end{table}

% --- SECTION 6: RECOMMENDATIONS ---
\section{Recommendations}
Based on the analysis, the following actions are recommended to mitigate the identified risks.

\subsection{Immediate Actions (Critical Priority)}
\begin{enumerate}
    \item \textbf{Restrict Database Access:} Immediately implement firewall rules to restrict all access to port 3306 on host \texttt{172.16.50.20}. Access should only be permitted from the IP addresses of trusted application servers that require a direct database connection.
    \item \textbf{Plan Database Upgrade:} Develop an urgent migration plan to upgrade the MySQL 5.7.33 database to a currently supported version (e.g., MySQL 8.x). This is essential to ensure the system receives ongoing security patches.
\end{enumerate}

\subsection{High Priority Actions}
\begin{enumerate}
    \item \textbf{Implement Onboarding Training:} Integrate a mandatory security awareness training module into the new employee onboarding process. This training should cover, at a minimum, phishing identification, acceptable use policies, and password security.
\end{enumerate}

\subsection{Long-Term Improvements}
\begin{enumerate}
    \item \textbf{Network Segmentation Review:} Conduct a review of the network architecture to ensure that critical infrastructure, such as databases, is located in isolated network segments, separate from user workstations and less-trusted zones.
    \item \textbf{Establish Patch Management Program:} Formalize a vulnerability and patch management program to proactively track software lifecycles, identify EOL systems, and apply security patches in a timely manner.
\end{enumerate}

\end{document}
```