```latex
\documentclass[12pt]{article}

% --- PACKAGES ---
\usepackage[margin=1in]{geometry}
\usepackage{pifont} % For \ding
\usepackage{booktabs} % For professional tables
\usepackage[hidelinks]{hyperref} % For clickable links without boxes
\usepackage{url}
\usepackage{seqsplit} % To split long text strings
\usepackage{fancyhdr} % For header/footer
\usepackage{graphicx} % For logo
\usepackage{xcolor} % For colors
\usepackage{datetime} % For report date

% --- DOCUMENT METADATA & SETUP ---
\title{Cybersecurity Posture Assessment Report}
\author{Cybersecurity Analysis Division}
\date{\today}

% --- CUSTOM COMMANDS & COLORS ---
\definecolor{SeverityCritical}{HTML}{990000}
\definecolor{SeverityHigh}{HTML}{D14302}
\definecolor{SeverityMedium}{HTML}{F5A623}
\newcommand{\yes}{\ding{51}} % Checkmark
\newcommand{\no}{\ding{55}}  % X mark

% --- HEADER & FOOTER ---
\pagestyle{fancy}
\fancyhf{} % Clear all header and footer fields
\fancyhead[L]{Cybersecurity Posture Assessment}
\fancyhead[R]{\textbf{Common Ground}}
\fancyfoot[C]{\thepage}
\renewcommand{\headrulewidth}{0.4pt}
\renewcommand{\footrulewidth}{0.4pt}

% --- BEGIN DOCUMENT ---
\begin{document}

\maketitle
\thispagestyle{empty}
\newpage

\tableofcontents
\newpage

% ==============================================================================
% SECTION 1: EXECUTIVE OVERVIEW
% ==============================================================================
\section{Executive Overview}

This report provides a comprehensive analysis of the cybersecurity posture for \textbf{Common Ground}, based on a correlation of technical network scans, a security controls questionnaire, and a review of pre-existing risks.

The assessment has identified several critical and high-risk deficiencies that expose the organization to significant threats, most notably ransomware and data breaches. The primary findings are:

\begin{itemize}
    \item \textbf{Critical Remote Service Exposure:} The technical scan confirmed an open Remote Desktop Protocol (RDP) port on a new host (\texttt{10.10.10.51}). This finding, combined with a pre-existing risk on another host, indicates a systemic issue of exposing a highly targeted service.
    \item \textbf{Systemic Lack of Multi-Factor Authentication (MFA):} The organization has not implemented MFA for email, computer logins, or access to sensitive data. This absence of a fundamental security control dramatically increases the risk of account compromise.
    \item \textbf{Foundational Policy Gaps:} Key security policies, such as an Acceptable Use Policy and security training for new hires, are not in place. This points to a reactive rather than proactive security culture, increasing the likelihood of human error.
\end{itemize}

The combination of exposed RDP and the lack of MFA creates a direct and immediate pathway for an attacker to gain unauthorized access to the internal network. Urgent remediation is required to address these findings. This report outlines specific, actionable recommendations to mitigate these risks and strengthen the organization's overall security posture.

% ==============================================================================
% SECTION 2: ORGANIZATIONAL INFORMATION
% ==============================================================================
\section{Organizational Information}

The following information was provided and used as the basis for this assessment.

\begin{table}[h!]
\centering
\begin{tabular}{@{}ll@{}}
\toprule
\textbf{Attribute} & \textbf{Value} \\ \midrule
Organization Name  & \textbf{Common Ground} \\
Email Domain       & \texttt{CommonGround.org} \\
Website Domain     & \url{www.CommonGround.org} \\
External IP Address & \texttt{151.55.33.131} \\ \bottomrule
\end{tabular}
\caption{Client Profile Information.}
\end{table}

% ==============================================================================
% SECTION 3: SECURITY CONTROL REVIEW
% ==============================================================================
\section{Security Control Review}

The following table summarizes the organization's responses to a security controls questionnaire. Responses marked with a \no{} indicate a significant gap in security controls that requires attention.

\begin{table}[h!]
\centering
\begin{tabular}{@{}p{0.55\linewidth} c p{0.25\linewidth}@{}}
\toprule
\textbf{Control Question} & \textbf{Response} & \textbf{Assessment} \\ \midrule
Do you require MFA to access email? & \no & \textcolor{SeverityCritical}{\textbf{Critical Gap}} \\
Do you require MFA to log into computers? & \no & \textcolor{SeverityCritical}{\textbf{Critical Gap}} \\
Do you require MFA to access sensitive data systems? & \no & \textcolor{SeverityCritical}{\textbf{Critical Gap}} \\
Does your organization have an employee acceptable use policy? & \no & \textcolor{SeverityHigh}{\textbf{High Risk}} \\
Does your organization do security awareness training for new employees? & \no & \textcolor{SeverityHigh}{\textbf{High Risk}} \\
Does your organization do security awareness training for all employees at least once per year? & \yes & Good Practice \\ \bottomrule
\end{tabular}
\caption{Security Controls Questionnaire Analysis.}
\end{table}

% ==============================================================================
% SECTION 4: TECHNICAL SCAN RESULTS
% ==============================================================================
\section{Technical Scan Results}

A network scan was performed on the target host \texttt{10.10.10.51}. The results below detail the open ports and services discovered.

\begin{table}[h!]
\centering
\begin{tabular}{@{}llll@{}}
\toprule
\textbf{Port} & \textbf{State} & \textbf{Service Name} & \textbf{Analyst Notes} \\ \midrule
3389/tcp & open & \texttt{ms-wbt-server} & Remote Desktop Protocol (RDP). This service is a \\
& & & primary target for ransomware attackers. Its \\
& & & exposure is a critical risk, especially without MFA. \\ \bottomrule
\end{tabular}
\caption{Open Port Findings for Target \texttt{10.10.10.51}.}
\end{table}

% ==============================================================================
% SECTION 5: CORRELATED RISK ASSESSMENT
% ==============================================================================
\section{Correlated Risk Assessment}

This section synthesizes the findings from the security control review, technical scans, and pre-existing risk data to provide a holistic view of the current risk landscape.

\begin{table}[h!]
\centering
\begin{tabular}{@{}p{0.25\linewidth} p{0.15\linewidth} p{0.5\linewidth}@{}}
\toprule
\textbf{Risk Title} & \textbf{Severity} & \textbf{Description \& Correlation} \\ \midrule
\textbf{Systemic RDP Exposure without MFA} & \textcolor{SeverityCritical}{\textbf{Critical}} & The scan identified open RDP on \texttt{10.10.10.51}, adding to a known exposure on \texttt{10.10.10.50}. The questionnaire confirmed a complete lack of MFA for computer logins. This combination makes the network highly susceptible to brute-force or credential-stuffing attacks, which could lead to a full network compromise. \\
\addlinespace
\textbf{Inadequate Identity and Access Management} & \textcolor{SeverityHigh}{\textbf{High}} & The organization does not enforce MFA on any critical systems (email, computers, data). This represents a fundamental failure in access control and leaves all user accounts vulnerable to takeover from a single compromised password. \\
\addlinespace
\textbf{Foundational Security Policy Gaps} & \textcolor{SeverityHigh}{\textbf{High}} & The absence of an Acceptable Use Policy and security training for new hires indicates a weak security culture. This increases the risk of unsafe employee behavior, such as weak password usage or falling for phishing attacks, which directly impacts the security of all systems. \\ \bottomrule
\end{tabular}
\caption{Summary of Identified and Correlated Risks.}
\end{table}

% ==============================================================================
% SECTION 6: RECOMMENDATIONS
% ==============================================================================
\section{Recommendations}

The following prioritized recommendations are provided to address the identified risks.

\subsection{Priority 1: Immediate Remediation (0-7 Days)}
\begin{enumerate}
    \item \textbf{Remediate RDP Exposure:} Immediately close port 3389 on any system where it is exposed to untrusted networks, including \texttt{10.10.10.51} and \texttt{10.10.10.50}.
    \item \textbf{Implement a Secure Remote Access Solution:} If remote desktop access is a business requirement, it \textbf{must} be placed behind a Virtual Private Network (VPN) that requires Multi-Factor Authentication for access.
\end{enumerate}

\subsection{Priority 2: High-Impact Hardening (1-3 Months)}
\begin{enumerate}
    \item \textbf{Deploy Multi-Factor Authentication (MFA):} Begin a phased rollout of MFA for all employees. The highest priorities are:
    \begin{itemize}
        \item Email (e.g., Office 365, Google Workspace).
        \item VPN and any other remote access systems.
        \item All administrative and privileged accounts.
    \end{itemize}
    \item \textbf{Develop Foundational Security Policies:} Create and implement an official Acceptable Use Policy (AUP) that all employees must read and sign. This policy should govern the use of company assets and data.
\end{enumerate}

\subsection{Priority 3: Long-Term Program Development (3-6 Months)}
\begin{enumerate}
    \item \textbf{Establish a Security Awareness Training Program:} Implement a mandatory security awareness training module for all new employees as part of their onboarding process. Continue to provide annual refresher training for all staff to maintain a high level of security consciousness.
\end{enumerate}

\end{document}
```