```latex
\documentclass[12pt]{article}

% Required Packages
\usepackage[margin=1in]{geometry}
\usepackage{pifont} % For checkmarks and crosses
\usepackage{booktabs} % For professional tables
\usepackage{hyperref} % For clickable links
\usepackage{url} % For formatting URLs
\usepackage{seqsplit} % For splitting long strings in tables
\usepackage{graphicx}
\usepackage{xcolor}

% Document Metadata
\title{Cybersecurity Posture Assessment Report}
\author{Cybersecurity Analyst}
\date{\today}

% Hyperref Setup
\hypersetup{
    colorlinks=true,
    linkcolor=blue,
    filecolor=magenta,      
    urlcolor=cyan,
    pdftitle={Cybersecurity Posture Assessment Report},
    pdfpagemode=FullScreen,
}

% Define a command for table cells to handle long text
\newcommand{\tablecell}[1]{\seqsplit{\texttt{#1}}}

\begin{document}

\maketitle

\begin{abstract}
This report provides a comprehensive cybersecurity assessment for Hearth \& Home. The analysis is based on a correlation of network scan data, an organizational security questionnaire, and a review of pre-existing risks. The assessment reveals several critical and high-risk security gaps, including the absence of multi-factor authentication (MFA) on key systems, inadequate security policies, and the presence of an insecure network service. This document outlines the identified risks and provides actionable recommendations to enhance the organization's security posture.
\end{abstract}

\newpage

\tableofcontents

\newpage

\section{Overview and Executive Summary}
This assessment was conducted to evaluate the current cybersecurity posture of Hearth \& Home. By synthesizing technical scan data with self-reported security controls, we have identified significant areas of risk that require immediate attention.

\textbf{Key Findings:}
\begin{itemize}
    \item \textbf{Critical MFA Gaps:} Multi-Factor Authentication is not enforced for computer logins or access to sensitive data systems. This represents a critical vulnerability, significantly increasing the risk of unauthorized access and lateral movement in the event of a credential compromise.
    \item \textbf{Insecure Network Services:} A network scan of the target host \texttt{172.16.0.1} identified an open Port 80 (HTTP). This service transmits data in cleartext, exposing any information, including potential credentials, to interception.
    \item \textbf{Policy and Training Deficiencies:} The organization lacks a formal employee acceptable use policy and does not provide security awareness training for new hires. These foundational elements are essential for establishing a strong security culture and minimizing human error.
\end{itemize}

The combination of these findings indicates a reactive security posture with significant exposure to common cyber threats. The recommendations provided in this report are prioritized to address the most critical risks first.

\section{Organizational Information}
The following information was provided for the assessment.

\begin{tabular}{@{}ll}
\toprule
\textbf{Attribute} & \textbf{Value} \\
\midrule
Organization Name & \textbf{Hearth \& Home} \\
Email Domain & \texttt{HearthHome.net} \\
Website Domain & \url{www.HearthHome.net} \\
External IP Address & \texttt{154.216.142.194} \\
\bottomrule
\end{tabular}

\section{Security Control Review}
The following table summarizes the organization's responses to a security controls questionnaire. "No" answers indicate significant gaps in the security framework.

\begin{table}[h!]
\centering
\caption{Security Controls Questionnaire Analysis}
\begin{tabular}{p{0.6\linewidth} c l}
\toprule
\textbf{Control Question} & \textbf{Response} & \textbf{Assessment} \\
\midrule
\seqsplit{Do you require MFA to access email?} & \ding{51} & Good Practice \\
\midrule
\seqsplit{Do you require MFA to log into computers?} & \textcolor{red}{\ding{55}} & \textbf{Critical Gap} \\
\midrule
\seqsplit{Do you require MFA to access sensitive data systems?} & \textcolor{red}{\ding{55}} & \textbf{Critical Gap} \\
\midrule
\seqsplit{Does your organization have an employee acceptable use policy?} & \textcolor{red}{\ding{55}} & \textbf{High Risk} \\
\midrule
\seqsplit{Does your organization do security awareness training for new employees?} & \textcolor{red}{\ding{55}} & \textbf{High Risk} \\
\midrule
\seqsplit{Does your organization do security awareness training for all employees at least once per year?} & \ding{51} & Good Practice \\
\bottomrule
\end{tabular}
\end{table}

\section{Technical Scan Results}
An Nmap scan was performed to identify open ports and services on the specified target.

\begin{itemize}
    \item \textbf{Target IP Address:} \texttt{172.16.0.1}
    \item \textbf{Host Status:} Up
\end{itemize}

\begin{table}[h!]
\centering
\caption{Open Port Analysis}
\begin{tabular}{l l l p{0.5\linewidth}}
\toprule
\textbf{Port} & \textbf{State} & \textbf{Service} & \textbf{Finding / Analysis} \\
\midrule
80/tcp & Open & HTTP & The presence of an open HTTP port is a significant security risk. This protocol does not encrypt traffic, meaning any data transmitted (including usernames, passwords, and session cookies) can be intercepted and read by an attacker on the same network. All web services should be served exclusively over HTTPS (Port 443). \\
\bottomrule
\end{tabular}
\end{table}

\section{Consolidated Risk Assessment}
The following table consolidates findings from the security control review and technical scan into a prioritized list of risks. Note: The pre-existing risk input was determined to be invalid and was excluded from this analysis.

\begin{table}[h!]
\centering
\caption{Summary of Identified Risks}
\begin{tabular}{l p{0.3\linewidth} p{0.4\linewidth} l}
\toprule
\textbf{ID} & \textbf{Risk Title} & \textbf{Description} & \textbf{Severity} \\
\midrule
RISK-001 & Lack of Multi-Factor Authentication & MFA is not enforced on computer logins or for access to sensitive data systems. This allows an attacker with stolen credentials to gain direct access to critical assets. & \textbf{Critical} \\
\addlinespace
RISK-002 & Insecure Web Service (HTTP) & An internal host (\texttt{172.16.0.1}) is running an unencrypted web server, exposing all transmitted data to sniffing attacks and potential credential theft. & \textbf{High} \\
\addlinespace
RISK-003 & Deficient Security Policies and Training & The lack of an Acceptable Use Policy and security training for new hires creates an environment where employees are more likely to engage in risky behavior or fall victim to social engineering. & \textbf{High} \\
\bottomrule
\end{tabular}
\end{table}

\section{Recommendations}
To mitigate the identified risks and improve the overall security posture, we recommend the following actions, prioritized by severity.

\begin{enumerate}
    \item \textbf{Implement Comprehensive MFA (RISK-001):}
    \begin{itemize}
        \item \textbf{Immediate Action:} Enable and enforce MFA for all employees and privileged users for logging into company computers and accessing all systems containing sensitive data.
        \item \textbf{Tools:} Utilize solutions integrated with your identity provider (e.g., Microsoft Entra ID, Okta, Duo).
    \end{itemize}

    \item \textbf{Remediate Insecure Web Service (RISK-002):}
    \begin{itemize}
        \item \textbf{Immediate Action:} Investigate the service running on \texttt{172.16.0.1:80}. If it is a necessary business application, reconfigure it to use HTTPS with a valid TLS certificate and disable the HTTP listener. If the service is not required, it should be disabled and firewalled.
    \end{itemize}

    \item \textbf{Develop Foundational Security Policies (RISK-003):}
    \begin{itemize}
        \item \textbf{Short-Term Action:} Draft and implement a formal Acceptable Use Policy (AUP) that clearly defines the rules for using company IT assets, data, and networks. Ensure all employees read and acknowledge the policy.
    \end{itemize}
    
    \item \textbf{Enhance Security Awareness Program (RISK-003):}
    \begin{itemize}
        \item \textbf{Short-Term Action:} Integrate a mandatory security awareness training module into the onboarding process for all new employees. This training should cover key topics such as phishing, password security, and the new AUP.
    \end{itemize}
\end{enumerate}

\end{document}
```