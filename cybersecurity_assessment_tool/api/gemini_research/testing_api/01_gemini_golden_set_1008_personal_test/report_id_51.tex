```latex
\documentclass[12pt, a4paper]{article}

% Preamble: Required Packages
\usepackage[utf8]{inputenc}
\usepackage[margin=1in]{geometry}
\usepackage{pifont} % For \ding
\usepackage{booktabs} % For professional tables (\toprule, \midrule, \bottomrule)
\usepackage{hyperref} % For hyperlinks
\usepackage{url} % For formatting URLs
\usepackage{seqsplit} % To split long strings in \texttt
\usepackage{graphicx}
\usepackage{xcolor}

% --- Document Setup ---
\hypersetup{
    colorlinks=true,
    linkcolor=blue,
    filecolor=magenta,      
    urlcolor=cyan,
    pdftitle={Cybersecurity Posture Assessment Report},
    pdfpagemode=FullScreen,
}

% Define check and cross marks for convenience
\newcommand{\cmark}{\ding{51}}%
\newcommand{\xmark}{\ding{55}}%

% --- Document Begins ---
\begin{document}

% --- Title Page ---
\begin{titlepage}
    \centering
    \vspace*{1cm}
    \Huge\textbf{Cybersecurity Posture Assessment Report}
    \vspace{1.5cm}
    \Large
    \textbf{Prepared for:}\\
    Skyward Bound
    \vspace{3cm}
    \includegraphics[width=0.4\textwidth]{example-image-a} % Placeholder for client logo
    \vfill
    \large
    \textbf{Date of Report:}\\
    \today
\end{titlepage}

\newpage
\tableofcontents
\newpage

% --- Section 1: Executive Overview ---
\section*{1. Executive Overview}

This report provides a comprehensive analysis of the cybersecurity posture for \textbf{Skyward Bound}, based on a combination of technical network scanning, a review of existing risk documentation, and an organizational security controls questionnaire.

The assessment identified several critical and high-risk findings that require immediate attention. The primary concerns are the external exposure of a database service running on an unsupported, End-of-Life (EOL) software version, and significant gaps in foundational security controls, most notably the lack of Multi-Factor Authentication (MFA) and a formal security awareness program.

The technical scan confirmed that a MySQL database on host \texttt{172.16.50.20} is accessible over the network. The running version, MySQL 5.7.33, is no longer supported by the vendor, meaning it does not receive security patches and is highly likely to contain unpatched, publicly known vulnerabilities.

This technical vulnerability is severely compounded by organizational weaknesses identified in the security questionnaire. The absence of MFA for computer and sensitive data system access, coupled with the lack of an employee security training program, creates a high-risk environment where a single compromised credential could lead to a significant data breach.

Immediate remediation should focus on isolating the exposed database, upgrading the EOL software, and implementing MFA across all critical systems.

% --- Section 2: Organizational Information ---
\section*{2. Organizational Information}

The following information was provided for the assessment.

\begin{itemize}
    \item \textbf{Organization Name:} Skyward Bound
    \item \textbf{Email Domain:} \texttt{SkywardBound.net}
    \item \textbf{Website Domain:} \url{www.SkywardBound.net}
    \item \textbf{External IP Address:} \texttt{39.186.91.10}
\end{itemize}

% --- Section 3: Security Control Review ---
\section*{3. Security Control Review}

A security controls questionnaire was completed to evaluate existing administrative and organizational safeguards. The results highlight critical gaps in identity and access management, policy, and employee security awareness. "No" answers indicate a failure to meet baseline security best practices and represent significant areas of risk.

\begin{table}[h!]
\centering
\caption{Security Controls Questionnaire Results}
\begin{tabular}{p{0.7\linewidth} c c}
\toprule
\textbf{Control Question} & \textbf{Response} & \textbf{Status} \\
\midrule
Do you require MFA to access email? & Yes & \textcolor{green}{\cmark} \\
Do you require MFA to log into computers? & No & \textcolor{red}{\xmark} \\
Do you require MFA to access sensitive data systems? & No & \textcolor{red}{\xmark} \\
Does your organization have an employee acceptable use policy? & No & \textcolor{red}{\xmark} \\
Does your organization do security awareness training for new employees? & No & \textcolor{red}{\xmark} \\
Does your organization do security awareness training for all employees at least once per year? & No & \textcolor{red}{\xmark} \\
\bottomrule
\end{tabular}
\end{table}

\subsection*{Analysis of Control Gaps}
\begin{itemize}
    \item \textbf{Lack of MFA:} The absence of MFA for computer and sensitive system access is a critical weakness. This significantly increases the risk of unauthorized access via stolen or weak credentials.
    \item \textbf{Lack of Policy:} Without an Acceptable Use Policy (AUP), there are no formal guidelines for employees regarding the protection of company assets, leading to inconsistent and insecure practices.
    \item \textbf{Lack of Training:} The complete absence of a security awareness training program leaves the organization highly vulnerable to social engineering attacks such as phishing, which is the leading cause of security breaches.
\end{itemize}

% --- Section 4: Technical Scan Results ---
\section*{4. Technical Scan Results}

A network scan was performed on the specified target to identify open ports and exposed services.

\begin{itemize}
    \item \textbf{Target IP Address:} \texttt{172.16.50.20}
\end{itemize}

\begin{table}[h!]
\centering
\caption{Open Ports and Services Detected}
\begin{tabular}{l l l l}
\toprule
\textbf{Port} & \textbf{Service} & \textbf{Product} & \textbf{Version} \\
\midrule
3306/tcp & mysql & MySQL & 5.7.33 \\
\bottomrule
\end{tabular}
\end{table}

\subsection*{Analysis of Technical Findings}
The scan revealed a single open port, 3306, which is the default port for the MySQL database service. 

\begin{itemize}
    \item \textbf{Critical Finding - End-of-Life Software:} The detected version, \textbf{MySQL 5.7.33}, is part of the MySQL 5.7 series which reached its official End-of-Life (EOL) in October 2023. EOL software no longer receives security updates from the vendor, making it a prime target for attackers who can exploit known, unpatched vulnerabilities.
    \item \textbf{High Risk - Direct Database Exposure:} Exposing a database port directly to the network is a significant security risk. It allows attackers to directly target the database with brute-force login attempts, exploit injection vulnerabilities, or leverage unpatched flaws in the database software itself. This configuration violates the principle of least privilege and defense-in-depth.
\end{itemize}

% --- Section 5: Correlated Risk Assessment ---
\section*{5. Correlated Risk Assessment}

This section synthesizes findings from the technical scan, control review, and pre-existing risk data into a consolidated list of key risks facing the organization.

\begin{table}[h!]
\centering
\caption{Summary of Identified Risks}
\begin{tabular}{p{0.2\linewidth} p{0.5\linewidth} p{0.15\linewidth}}
\toprule
\textbf{Risk Name} & \textbf{Description} & \textbf{Severity} \\
\midrule
\textbf{Database Exposure on EOL Software} & A MySQL database (v5.7.33) is exposed to the network. This version is End-of-Life and no longer receives security patches, making it highly vulnerable to exploitation. & \textbf{Critical (9.8)} \\
\addlinespace
\textbf{Lack of Multi-Factor Authentication} & No MFA is enforced for computer or sensitive system logins. A compromised password provides an attacker with direct access to critical assets. & \textbf{Critical (9.1)} \\
\addlinespace
\textbf{Insufficient Security Awareness Program} & The absence of employee security training makes the organization highly susceptible to phishing and other social engineering attacks, increasing the likelihood of credential compromise. & \textbf{High (8.2)} \\
\addlinespace
\textbf{Missing Acceptable Use Policy} & Lack of a formal policy creates an environment with no clear rules for secure employee behavior, leading to inconsistent security practices and increased insider risk. & \textbf{Medium (5.4)} \\
\bottomrule
\end{tabular}
\end{table}

% --- Section 6: Recommendations ---
\section*{6. Recommendations}

The following actions are recommended to mitigate the identified risks. Recommendations are prioritized based on severity and potential impact.

\subsection*{Immediate Priority (Within 72 Hours)}
\begin{enumerate}
    \item \textbf{Isolate the Database:} Implement strict firewall rules to block all access to TCP port 3306 on \texttt{172.16.50.20} from any source that does not explicitly require it. Access should be restricted to specific application servers or an administrative VPN.
\end{enumerate}

\subsection*{High Priority (Within 30 Days)}
\begin{enumerate}
    \item \textbf{Upgrade End-of-Life Software:} Plan and execute the migration of the MySQL 5.7 database to a currently supported version (e.g., MySQL 8.x). This is essential to ensure security patches are available and applied.
    \item \textbf{Implement Multi-Factor Authentication (MFA):}
        \begin{itemize}
            \item Deploy MFA for all users on sensitive data systems immediately.
            \item Roll out MFA for all computer logins (desktops, laptops).
        \end{itemize}
\end{enumerate}

\subsection*{Medium Priority (Within 90 Days)}
\begin{enumerate}
    \item \textbf{Establish Security Awareness Training:}
        \begin{itemize}
            \item Procure and implement a security awareness training program for all employees.
            \item Ensure all new hires complete the training as part of their onboarding process.
            \item Conduct annual refresher training and periodic phishing simulations.
        \end{itemize}
    \item \textbf{Develop and Implement an Acceptable Use Policy (AUP):} Create a formal AUP that clearly defines the rules and responsibilities for all employees when using company IT assets. Ensure all employees read and acknowledge the policy.
\end{enumerate}

\end{document}
```