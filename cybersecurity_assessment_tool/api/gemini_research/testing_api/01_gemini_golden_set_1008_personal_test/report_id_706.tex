```latex
\documentclass[12pt]{article}

% Preamble: Required Packages
\usepackage[a4paper, margin=1in]{geometry}
\usepackage{pifont} % For checkmarks and crosses
\usepackage{booktabs} % For professional tables
\usepackage{hyperref} % For hyperlinks
\usepackage{url} % For URL formatting
\usepackage{seqsplit} % To split long text strings in tt font
\usepackage{graphicx} % For logo (optional)
\usepackage{fancyhdr} % For headers/footers

% Document Metadata
\title{Cybersecurity Assessment Report}
\author{Cybersecurity Analyst}
\date{\today}

% Header and Footer Configuration
\pagestyle{fancy}
\fancyhf{}
\lhead{Blackwood Industries}
\rhead{Confidential}
\cfoot{\thepage}

\begin{document}

\maketitle
\thispagestyle{empty}
\newpage

\tableofcontents
\newpage

% --- 1. Executive Summary ---
\section{Executive Summary}

This report details the findings of a cybersecurity assessment for Blackwood Industries, conducted on \today. The analysis synthesizes data from a network perimeter scan, a security controls questionnaire, and a review of pre-existing risks.

The assessment identified several critical and high-risk gaps in the organization's security posture, primarily related to identity and access management and employee security policies. While the external network scan of the target \texttt{[Target IP]} did not reveal any open ports—a positive sign of a potentially well-configured firewall—the internal policy weaknesses present a significant risk.

Key findings include a lack of Multi-Factor Authentication (MFA) for email and computer access, the absence of a formal employee acceptable use policy, and no security training for new hires. These gaps expose the organization to substantial threats, including account compromise, phishing, and insider threats.

This report provides a detailed breakdown of these risks and offers prioritized, actionable recommendations to mitigate them and strengthen the overall security posture of Blackwood Industries.

% --- 2. Organizational Information ---
\section{Organizational Information}

The following details were provided for the assessment. This information is used to establish the context and scope of the review.

\begin{itemize}
    \item \textbf{Organization Name:} Blackwood Industries
    \item \textbf{Email Domain:} \seqsplit{\texttt{BlackwoodIndustries.net}}
    \item \textbf{Website Domain:} \url{www.BlackwoodIndustries.net}
    \item \textbf{External IP Address:} \seqsplit{\texttt{93.115.15.141}}
\end{itemize}

% --- 3. Security Control Review ---
\section{Security Control Review}

A review of internal security controls was conducted via a questionnaire. The responses indicate the current state of implemented policies and procedures. "No" answers often highlight significant gaps that require immediate attention.

\begin{table}[h!]
\centering
\caption{Security Controls Questionnaire Analysis}
\begin{tabular}{p{0.6\linewidth} c l}
\toprule
\textbf{Control Question} & \textbf{Response} & \textbf{Assessment} \\
\midrule
Do you require MFA to access email? & \ding{55} & \textbf{Critical Gap} \\
Do you require MFA to log into computers? & \ding{55} & \textbf{Critical Gap} \\
Do you require MFA to access sensitive data systems? & \ding{51} & Best Practice \\
Does your organization have an employee acceptable use policy? & \ding{55} & High Risk \\
Does your organization do security awareness training for new employees? & \ding{55} & High Risk \\
Does your organization do security awareness training for all employees at least once per year? & \ding{51} & Best Practice \\
\bottomrule
\end{tabular}
\end{table}

% --- 4. Technical Scan Results ---
\section{Technical Scan Results}

An external network scan was performed to identify exposed services and potential vulnerabilities on the public-facing infrastructure.

\begin{itemize}
    \item \textbf{Target IP Address:} \texttt{[Target IP]}
    \item \textbf{Scan Summary:} The scan completed successfully but did not detect any open TCP or UDP ports on the target host.
\end{itemize}

\textbf{Analysis:}
No open ports were discovered. This is a positive finding, suggesting that a firewall is in place and properly configured to block unsolicited inbound traffic (a "default deny" posture). However, this result could also indicate that the host was offline during the scan or that advanced filtering mechanisms are in place. Further authenticated testing is recommended to validate the internal security of the host.

% --- 5. Risk Assessment ---
\section{Risk Assessment}

This section correlates the findings from the security control review and the technical scan to provide a consolidated list of identified risks. The pre-existing risk list provided for this assessment was empty. The following risks were identified during this analysis.

\begin{table}[h!]
\centering
\caption{Identified Risks}
\begin{tabular}{p{0.25\linewidth} p{0.55\linewidth} l}
\toprule
\textbf{Risk Name} & \textbf{Overview} & \textbf{Severity} \\
\midrule
\textbf{Lack of MFA on Critical Systems} & The absence of MFA for email and computer logins makes user accounts highly susceptible to compromise via credential theft, phishing, or brute-force attacks. A single compromised password could grant an attacker full access. & \textbf{Critical} \\
\addlinespace
\textbf{No Acceptable Use Policy (AUP)} & Without a formal AUP, employees lack clear guidelines on the secure and acceptable use of company assets. This increases the risk of unintentional data exposure, malware infections, and insider threats. & High \\
\addlinespace
\textbf{Inadequate Onboarding Security Training} & New employees are not provided with security awareness training. This creates a window of high vulnerability where new hires are more likely to fall victim to social engineering or mishandle sensitive data due to a lack of awareness. & High \\
\bottomrule
\end{tabular}
\end{table}

% --- 6. Recommendations ---
\section{Recommendations}

Based on the identified risks, the following prioritized recommendations are provided to enhance the security posture of Blackwood Industries.

\begin{enumerate}
    \item \textbf{Implement Multi-Factor Authentication (Critical):}
    \begin{itemize}
        \item Immediately enforce MFA for all user access to the email system (e.g., Microsoft 365, Google Workspace).
        \item Deploy MFA for all computer and remote access (VPN) logins. This is a foundational control for preventing unauthorized access.
    \end{itemize}
    \vspace{1em}

    \item \textbf{Develop and Enforce an Acceptable Use Policy (High):}
    \begin{itemize}
        \item Create a comprehensive AUP that clearly defines rules for using company networks, devices, and data.
        \item The policy should be formally communicated to all employees, and acknowledgment should be tracked. This establishes a baseline for secure employee behavior.
    \end{itemize}
    \vspace{1em}

    \item \textbf{Establish Security Training for New Hires (High):}
    \begin{itemize}
        \item Integrate a mandatory security awareness training module into the new employee onboarding process.
        \item This training should cover key topics such as phishing, password security, data handling, and the new AUP.
    \end{itemize}
    \vspace{1em}
    
    \item \textbf{Conduct Comprehensive Vulnerability Scanning (Medium):}
    \begin{itemize}
        \item While the external scan showed no open ports, it is crucial to perform regular, authenticated vulnerability scans on internal and external assets to identify software vulnerabilities and misconfigurations that an unauthenticated scan would miss.
    \end{itemize}
\end{enumerate}

\end{document}
```