```latex
\documentclass[12pt]{article}

% Preamble: Required Packages
\usepackage[margin=1in]{geometry}
\usepackage{pifont} % For checkmarks and crosses
\usepackage{booktabs} % For professional tables
\usepackage{hyperref} % For clickable links
\usepackage{url}      % For proper URL formatting
\usepackage{seqsplit} % To split long strings in tt font

% Document Metadata
\title{Cybersecurity Assessment Report}
\author{Cybersecurity Analysis Division}
\date{\today}

\begin{document}

\maketitle
\thispagestyle{empty}
\newpage
\tableofcontents
\newpage

\section{Executive Overview}

This report details the findings of a cybersecurity assessment for \textbf{Fable \& Lore}. The analysis is based on a combination of technical network scanning, a review of self-reported security controls, and an evaluation of pre-existing risk documentation.

The assessment has identified several high-priority risks that require immediate attention. The most critical finding is the discovery of an exposed network service on port 8080 with the title \textbf{"TOP SECRET DB"}. This suggests a potentially unauthenticated interface to a highly sensitive database is accessible on the internal network.

This technical vulnerability is compounded by significant organizational security gaps. The lack of Multi-Factor Authentication (MFA) for sensitive data systems, the absence of an employee acceptable use policy, and the failure to conduct annual security awareness training for all staff create a high-risk environment. An attacker who compromises a single user's credentials could potentially gain access to the organization's most sensitive data without facing additional security controls.

Immediate remediation of the exposed service and the implementation of MFA are strongly recommended to mitigate the risk of a significant data breach.

\section{Organizational Information}

The following information was provided for the assessment.

\begin{itemize}
    \item \textbf{Organization Name:} Fable \& Lore
    \item \textbf{Email Domain:} \seqsplit{\texttt{FableLore.net}}
    \item \textbf{Website Domain:} \seqsplit{\texttt{www.FableLore.net}}
    \item \textbf{External IP Address:} \seqsplit{\texttt{111.227.91.236}}
\end{itemize}

\section{Security Control Review}

The following table summarizes the organization's responses to a security controls questionnaire. Items marked with a red 'X' (\ding{55}) indicate a deviation from security best practices and represent a significant gap in the organization's defensive posture.

\begin{table}[h!]
\centering
\caption{Security Controls Questionnaire Results}
\begin{tabular}{p{0.75\linewidth} c}
\toprule
\textbf{Security Control Question} & \textbf{Response} \\
\midrule
Do you require MFA to access email? & \ding{51} \\
Do you require MFA to log into computers? & \ding{51} \\
\textbf{Do you require MFA to access sensitive data systems?} & \textbf{\color{red}\ding{55}} \\
\textbf{Does your organization have an employee acceptable use policy?} & \textbf{\color{red}\ding{55}} \\
Does your organization do security awareness training for new employees? & \ding{51} \\
\textbf{Does your organization do security awareness training for all employees at least once per year?} & \textbf{\color{red}\ding{55}} \\
\bottomrule
\end{tabular}
\end{table}

\subsection*{Analysis}
The questionnaire reveals critical weaknesses in policy and identity management. The lack of MFA on sensitive systems is a primary concern, as it removes a crucial layer of defense for the organization's most valuable data. Furthermore, the absence of an acceptable use policy and annual security training increases the likelihood of human error leading to a security incident.

\section{Technical Scan Results}

A network scan was performed against the target system to identify open ports and exposed services.

\begin{itemize}
    \item \textbf{Target IP Address:} \seqsplit{\texttt{10.5.5.5}}
\end{itemize}

\begin{table}[h!]
\centering
\caption{Open Port Scan Findings}
\begin{tabular}{l l l}
\toprule
\textbf{Port} & \textbf{State} & \textbf{Service / Title} \\
\midrule
8080/tcp & Open & HTTP Title: "TOP SECRET DB" \\
\bottomrule
\end{tabular}
\end{table}

\subsection*{Analysis}
The scan identified a highly concerning open port. The title "TOP SECRET DB" strongly implies that the service running on port 8080 is an interface to a sensitive, mission-critical database. Its exposure on the network, potentially without authentication, presents a severe and immediate risk of data exfiltration or manipulation. This finding directly contradicts pre-existing risk documentation which stated this port was secure, indicating a regression in security posture or a previously inaccurate assessment.

\section{Consolidated Risk Assessment}

The following table synthesizes findings from the security control review and the technical scan into a prioritized list of risks.

\begin{table}[h!]
\centering
\caption{Summary of Identified Risks}
\begin{tabular}{p{0.1\linewidth} p{0.25\linewidth} p{0.45\linewidth} p{0.1\linewidth}}
\toprule
\textbf{ID} & \textbf{Risk Title} & \textbf{Description} & \textbf{Severity} \\
\midrule
\textbf{R-01} & Exposed Sensitive Data Interface & Port 8080 is open on \texttt{10.5.5.5} with the title "TOP SECRET DB", suggesting an exposed database or management console. & \textbf{Critical} \\
\addlinespace
\textbf{R-02} & No MFA on Sensitive Systems & Access to systems containing sensitive data is not protected by MFA, creating a single point of failure if credentials are compromised. & \textbf{Critical} \\
\addlinespace
\textbf{R-03} & Lack of Annual Security Training & Employees do not receive annual security awareness training, increasing susceptibility to phishing and social engineering attacks. & \textbf{High} \\
\addlinespace
\textbf{R-04} & Missing Acceptable Use Policy & The absence of a formal Acceptable Use Policy (AUP) creates ambiguity regarding safe technology use, increasing insider risk. & \textbf{High} \\
\bottomrule
\end{tabular}
\end{table}

\section{Recommendations}

The following actions are recommended to mitigate the identified risks. Recommendations are prioritized based on severity.

\subsection*{Immediate Actions (Critical Risks)}
\begin{enumerate}
    \item \textbf{Remediate R-01 (Exposed Interface):} Immediately investigate the service running on \texttt{10.5.5.5:8080}.
    \begin{itemize}
        \item Confirm the nature and sensitivity of the data it accesses.
        \item If the service is business-critical, ensure strong authentication and access controls are enforced.
        \item If it is not essential, disable the service.
        \item At a minimum, restrict access to this port via firewall rules to only authorized personnel and systems.
    \end{itemize}
    \item \textbf{Remediate R-02 (Implement MFA):} Deploy and enforce Multi-Factor Authentication across all applications and systems that store, process, or access sensitive data. The system identified in R-01 should be the highest priority for this rollout.
\end{enumerate}

\subsection*{Near-Term Actions (High Risks)}
\begin{enumerate}
    \setcounter{enumi}{2} % Continue numbering from previous list
    \item \textbf{Remediate R-03 (Deploy Annual Training):} Develop and mandate an annual security awareness training program for all employees. The training should cover current threats such as phishing, social engineering, and proper data handling.
    \item \textbf{Remediate R-04 (Establish Acceptable Use Policy):} Draft, approve, and implement a formal Employee Acceptable Use Policy (AUP). This policy should clearly define the rules for using company technology and data. Require all current and new employees to read and formally acknowledge the policy.
\end{enumerate}

\end{document}
```