```latex
\documentclass[12pt, a4paper]{article}

% Preamble: Required Packages
\usepackage[margin=1in]{geometry}
\usepackage{pifont} % For \ding{51} and \ding{55}
\usepackage{booktabs} % For professional tables
\usepackage{hyperref} % For hyperlinks
\usepackage{url} % For URL formatting
\usepackage{seqsplit} % To split long strings in \texttt
\usepackage{graphicx}
\usepackage{xcolor}
\usepackage{fancyhdr}

% Document Information
\title{Cybersecurity Posture Assessment Report}
\author{Cybersecurity Analysis Division}
\date{\today}

% Hyperref Setup
\hypersetup{
    colorlinks=true,
    linkcolor=blue,
    filecolor=magenta,      
    urlcolor=cyan,
    pdftitle={Cybersecurity Posture Assessment Report},
    pdfpagemode=FullScreen,
}

% Define colors for table cells
\definecolor{good}{HTML}{c8e6c9}
\definecolor{bad}{HTML}{ffcdd2}

% Header and Footer
\pagestyle{fancy}
\fancyhf{}
\fancyhead[L]{\textbf{Cybersecurity Assessment Report}}
\fancyhead[R]{\textbf{[Organization Name]}}
\fancyfoot[C]{\thepage}
\renewcommand{\headrulewidth}{0.4pt}
\renewcommand{\footrulewidth}{0.4pt}

\begin{document}

\maketitle
\thispagestyle{empty}
\newpage

\tableofcontents
\newpage

% --- Section 1: Executive Summary ---
\section*{1. Executive Summary}

This report provides a comprehensive cybersecurity assessment for \textbf{Wildfire Communications}, synthesizing data from technical network scans, organizational security control questionnaires, and a review of pre-existing risks.

The assessment reveals a mixed security posture. While foundational controls such as an Acceptable Use Policy and Multi-Factor Authentication (MFA) for computer logins are in place, several critical vulnerabilities exist that significantly increase the organization's risk profile. 

Key findings include:
\begin{itemize}
    \item \textbf{Critical Control Gaps:} Multi-Factor Authentication is not enforced for accessing email or other sensitive data systems. This represents a severe vulnerability to account takeover and data breach incidents.
    \item \textbf{High-Risk Procedural Gaps:} The organization lacks a security awareness training program for both new and existing employees. This deficiency makes personnel more susceptible to social engineering and phishing attacks.
    \item \textbf{Technical Posture Update:} A network scan of the internal asset at \texttt{192.168.0.5} found that port 80 (HTTP) is closed. This contradicts a pre-existing risk report indicating an unencrypted web server. This suggests that the previously identified risk may have been remediated, which is a positive development.
\end{itemize}

Immediate action is required to address the MFA and security training gaps. Recommendations are provided in Section 6 to mitigate these identified risks and strengthen the overall security posture of \textbf{Wildfire Communications}.

% --- Section 2: Organizational Information ---
\section*{2. Organizational Information}

This section details the organizational data provided for the assessment.

\begin{tabular}{@{}ll}
    \toprule
    \textbf{Attribute} & \textbf{Value} \\
    \midrule
    Organization Name & \textbf{Wildfire Communications} \\
    Email Domain & \texttt{WildfireCommunications.com} \\
    Website Domain & \url{www.WildfireCommunications.com} \\
    External IP Address & \texttt{132.253.137.142} \\
    \bottomrule
\end{tabular}

% --- Section 3: Security Control Review ---
\section*{3. Security Control Review}

The following table summarizes the organization's responses to the security controls questionnaire. Items marked with \ding{55} represent significant gaps in the security framework and are prioritized in the risk assessment.

\begin{table}[h!]
\centering
\begin{tabular}{@{}p{8cm}cc@{}}
    \toprule
    \textbf{Control Question} & \textbf{Response} & \textbf{Status} \\
    \midrule
    Do you require MFA to access email? & \ding{55} No & \textcolor{red}{\textbf{Critical Gap}} \\
    Do you require MFA to log into computers? & \ding{51} Yes & \textcolor{green}{Control Met} \\
    Do you require MFA to access sensitive data systems? & \ding{55} No & \textcolor{red}{\textbf{Critical Gap}} \\
    Does your organization have an employee acceptable use policy? & \ding{51} Yes & \textcolor{green}{Control Met} \\
    Does your organization do security awareness training for new employees? & \ding{55} No & \textcolor{orange}{\textbf{High Risk}} \\
    Does your organization do security awareness training for all employees at least once per year? & \ding{55} No & \textcolor{orange}{\textbf{High Risk}} \\
    \bottomrule
\end{tabular}
\caption{Organizational Security Control Status}
\end{table}

% --- Section 4: Technical Scan Results ---
\section*{4. Technical Scan Results}

A network scan was performed to identify active services and potential vulnerabilities on the specified target system.

\begin{itemize}
    \item \textbf{Target IP Address:} \texttt{192.168.0.5}
    \item \textbf{Target Status:} Host is Up
\end{itemize}

\subsection*{Port Scan Details}
The scan revealed the following port status:
\begin{table}[h!]
\centering
\begin{tabular}{@{}llll@{}}
    \toprule
    \textbf{Port} & \textbf{State} & \textbf{Service} & \textbf{Notes} \\
    \midrule
    80/tcp & Closed & http & Port is not listening. \\
    \bottomrule
\end{tabular}
\caption{Scan Results for \texttt{192.168.0.5}}
\end{table}

\subsection*{Analysis of Technical Findings}
The technical scan indicates a minimal attack surface on the target host, with no open ports detected in the scan scope. Notably, port 80 (HTTP) was found to be \textbf{closed}. This finding is significant as it contradicts the pre-existing risk data (Input 3), which listed an "Unencrypted Web Server" on port 80 as an active vulnerability. 

This discrepancy suggests that the previously identified risk has likely been remediated. Verification is recommended to confirm this change is intentional and documented.

% --- Section 5: Risk Assessment Summary ---
\section*{5. Risk Assessment Summary}

This section correlates findings from the security control review, technical scans, and pre-existing risk data into a prioritized list.

\begin{table}[h!]
\centering
\begin{tabular}{@{}p{4cm}p{6cm}l@{}}
    \toprule
    \textbf{Risk Name} & \textbf{Overview} & \textbf{Severity} \\
    \midrule
    \textbf{Lack of MFA on Critical Systems} & Email and sensitive data systems are accessible with only a password, making them highly vulnerable to credential stuffing, phishing, and brute-force attacks. & \textcolor{red}{\textbf{Critical}} \\
    \addlinespace
    \textbf{No Security Awareness Training} & Employees are not trained to recognize or respond to security threats like phishing or social engineering, making them the weakest link in the organization's defense. & \textcolor{orange}{\textbf{High}} \\
    \addlinespace
    \textbf{Unencrypted Web Server} & A previously identified risk indicated that an unencrypted web server was active on port 80. The recent scan shows this port is now closed. & \textcolor{green}{\textbf{Medium (Potentially Remediated)}} \\
    \bottomrule
\end{tabular}
\caption{Consolidated Risk Register}
\end{table}

% --- Section 6: Recommendations ---
\section*{6. Recommendations}

The following actions are recommended to mitigate the identified risks and improve the overall security posture of \textbf{Wildfire Communications}.

\subsection*{Immediate Priority (0-30 Days)}
\begin{enumerate}
    \item \textbf{Implement MFA for Email and Sensitive Data:}
    \begin{itemize}
        \item \textbf{Action:} Enforce MFA across all user accounts for email access (e.g., Office 365, Google Workspace) and any systems identified as containing sensitive data.
        \item \textbf{Justification:} This is the single most effective control to prevent unauthorized account access and subsequent data breaches. It directly mitigates the \textbf{Critical} risk identified.
    \end{itemize}
\end{enumerate}

\subsection*{High Priority (30-90 Days)}
\begin{enumerate}
    \item \textbf{Establish a Security Awareness Training Program:}
    \begin{itemize}
        \item \textbf{Action:} Procure and implement a security awareness training solution. All employees must complete an initial training module upon hiring, with mandatory annual refresher courses and periodic phishing simulations.
        \item \textbf{Justification:} An educated workforce is a critical layer of defense. This program will reduce the likelihood of human error leading to a security incident and mitigates the \textbf{High} risk.
    \end{itemize}
\end{enumerate}

\subsection*{Verification and Housekeeping}
\begin{enumerate}
    \item \textbf{Confirm Remediation of Port 80:}
    \begin{itemize}
        \item \textbf{Action:} Verify with the system administrator for \texttt{192.168.0.5} that the closure of port 80 was intentional. Update the organization's risk register to formally close the "Unencrypted Web Server" risk.
        \item \textbf{Justification:} Ensures that risk tracking is accurate and that the change was a deliberate security improvement, not an accidental service outage. If a web service is required on this host, it must be deployed securely using HTTPS on port 443.
    \end{itemize}
\end{enumerate}

\end{document}
```