```latex
\documentclass[12pt]{article}

% --- PACKAGES ---
\usepackage[margin=1in]{geometry} % Set page margins
\usepackage{pifont}                 % For checkmarks and crosses (\ding{51}, \ding{55})
\usepackage{booktabs}               % For professional-looking tables
\usepackage{graphicx}               % For including logos, etc.
\usepackage{xcolor}                 % For custom colors
\usepackage{hyperref}               % For hyperlinks
\usepackage{url}                    % For formatting URLs
\usepackage{seqsplit}               % For splitting long strings in texttt

% --- HYPERREF SETUP ---
\hypersetup{
    colorlinks=true,
    linkcolor=black,
    filecolor=black,      
    urlcolor=blue,
    pdftitle={Cybersecurity Posture Assessment Report},
    pdfpagemode=FullScreen,
}

% --- DOCUMENT TITLE ---
\title{
    \vspace{2cm}
    \textbf{Cybersecurity Posture Assessment Report} \\
    \large\vspace{0.5cm}
    Prepared for: \textbf{Titanium Core} \\
    \vspace{1.5cm}
}
\author{Cybersecurity Analyst}
\date{\today}

% --- BEGIN DOCUMENT ---
\begin{document}

\maketitle
\thispagestyle{empty}
\newpage

\tableofcontents
\newpage

% ==============================================================================
% SECTION 1: EXECUTIVE SUMMARY
% ==============================================================================
\section*{Executive Summary}

This report details the findings of a cybersecurity posture assessment conducted for \textbf{Titanium Core}. The assessment combined a review of organizational security controls, an analysis of pre-existing risks, and a technical network scan.

The analysis revealed several \textbf{critical deficiencies} that place the organization at a high and immediate risk of a significant security breach. The core findings indicate a systemic weakness in fundamental security practices.

Key identified risks include:
\begin{itemize}
    \item \textbf{Lack of Multi-Factor Authentication (MFA):} The complete absence of MFA across email, computer logins, and sensitive data systems represents a critical vulnerability. This allows for straightforward account takeovers if user credentials are compromised.
    \item \textbf{Direct Exposure of Remote Services:} The technical scan identified a server with Remote Desktop Protocol (RDP) directly exposed to the network. This finding, correlated with a pre-existing risk on another system, points to a pattern of insecure configuration. Exposed RDP is a primary vector for ransomware attacks.
    \item \textbf{Absence of Security Governance:} The organization lacks a formal employee acceptable use policy and does not conduct security awareness training. This fosters a high-risk user environment where employees are more susceptible to phishing, social engineering, and poor security hygiene.
\end{itemize}

The combination of these findings, particularly the lack of MFA coupled with exposed RDP, creates a direct and easily exploitable pathway for an attacker to gain internal network access. Immediate remediation of the identified issues is strongly recommended to reduce the organization's risk profile.

% ==============================================================================
% SECTION 2: ORGANIZATIONAL INFORMATION
% ==============================================================================
\section*{Organizational Information}

The following details were provided for the assessment.

\begin{tabular}{@{}ll}
    \toprule
    \textbf{Attribute} & \textbf{Value} \\
    \midrule
    Organization Name & \textbf{Titanium Core} \\
    Email Domain & \texttt{TitaniumCore.net} \\
    Website Domain & \url{www.TitaniumCore.net} \\
    External IP Address & \texttt{93.177.124.38} \\
    \bottomrule
\end{tabular}

% ==============================================================================
% SECTION 3: SECURITY CONTROL REVIEW
% ==============================================================================
\section*{Security Control Review}

A review of administrative and technical security controls was conducted via a questionnaire. The responses indicate critical gaps in foundational security measures. A summary of the findings is presented below.

\begin{table}[h!]
\centering
\begin{tabular}{@{}p{0.6\textwidth} c p{0.2\textwidth}@{}}
    \toprule
    \textbf{Control Question} & \textbf{Response} & \textbf{Assessment} \\
    \midrule
    Do you require MFA to access email? & \ding{55} & \textbf{Critical Gap} \\
    Do you require MFA to log into computers? & \ding{55} & \textbf{Critical Gap} \\
    Do you require MFA to access sensitive data systems? & \ding{55} & \textbf{Critical Gap} \\
    Does your organization have an employee acceptable use policy? & \ding{55} & \textbf{High Risk} \\
    Does your organization do security awareness training for new employees? & \ding{55} & \textbf{High Risk} \\
    Does your organization do security awareness training for all employees at least once per year? & \ding{55} & \textbf{High Risk} \\
    \bottomrule
\end{tabular}
\caption{Security Control Questionnaire Results. (\ding{55} indicates "No")}
\end{table}

The consistent "No" responses highlight a lack of basic security hygiene. The absence of MFA is the most severe finding, as it removes a critical layer of defense against credential theft. The lack of policies and training exacerbates this risk by increasing the likelihood of employees falling victim to phishing attacks or using weak, easily guessed passwords.

% ==============================================================================
% SECTION 4: TECHNICAL SCAN RESULTS
% ==============================================================================
\section*{Technical Scan Results}

A network scan was performed on the target system to identify open ports and exposed services.

\begin{itemize}
    \item \textbf{Target IP Address:} \texttt{10.10.10.51}
    \item \textbf{Scan Status:} Host is UP.
\end{itemize}

The following open ports were discovered:

\begin{table}[h!]
\centering
\begin{tabular}{@{}llll@{}}
    \toprule
    \textbf{Port} & \textbf{State} & \textbf{Service Name} & \textbf{Description} \\
    \midrule
    3389/tcp & open & \texttt{ms-wbt-server} & Microsoft Remote Desktop Protocol (RDP) \\
    \bottomrule
\end{tabular}
\caption{Open Ports Detected on \texttt{10.10.10.51}}
\end{table}

\subsection*{Analysis of Findings}
The scan confirms that port 3389 (RDP) is open. RDP is a common target for attackers who use brute-force or credential-stuffing attacks to gain unauthorized remote access to systems. This finding, combined with a pre-existing risk of RDP exposure on another host (\texttt{10.10.10.50}), indicates a systemic issue in network security configuration. When coupled with the lack of enforced MFA, this exposed service constitutes a critical and immediate threat.

% ==============================================================================
% SECTION 5: CONSOLIDATED RISK ASSESSMENT
% ==============================================================================
\section*{Consolidated Risk Assessment}

The following table synthesizes findings from the security control review, technical scan, and pre-existing risk data into a prioritized list of risks.

\begin{table}[h!]
\centering
\begin{tabular}{@{}p{0.15\textwidth} p{0.65\textwidth} p{0.1\textwidth}@{}}
    \toprule
    \textbf{Risk Title} & \textbf{Description} & \textbf{Severity} \\
    \midrule
    \textbf{Systemic RDP Exposure} & Multiple systems (\texttt{10.10.10.51} and \texttt{10.10.10.50}) have RDP services exposed. This provides a direct entry point for attackers to compromise the internal network, especially when MFA is not enforced. & \textbf{Critical} \\
    \addlinespace
    \textbf{No Multi-Factor Authentication} & MFA is not required for email, computer logins, or sensitive systems. This makes user accounts highly vulnerable to takeover via password guessing, phishing, or credential stuffing. & \textbf{Critical} \\
    \addlinespace
    \textbf{Lack of Security Policy and Training} & The absence of an Acceptable Use Policy and security awareness training program means employees are likely unaware of cyber threats and best practices, making them the weakest link in the organization's defense. & \textbf{High} \\
    \bottomrule
\end{tabular}
\caption{Summary of Identified Risks}
\end{table}

% ==============================================================================
% SECTION 6: RECOMMENDATIONS
% ==============================================================================
\section*{Recommendations}

The following actions are recommended to mitigate the identified risks. They are prioritized based on severity and potential impact.

\subsection*{Critical Priority Recommendations}

\begin{enumerate}
    \item \textbf{Remediate RDP Exposure Immediately:}
    \begin{itemize}
        \item \textbf{Short-Term:} Implement firewall rules to block all access to TCP port 3389 from untrusted networks on all affected systems, including \texttt{10.10.10.51} and \texttt{10.10.10.50}. If remote access is required, restrict it to specific, known IP addresses.
        \item \textbf{Long-Term:} De-provision direct RDP access. Implement a secure remote access solution, such as a Virtual Private Network (VPN) gateway that requires MFA for all connections.
    \end{itemize}

    \item \textbf{Deploy Multi-Factor Authentication (MFA):}
    \begin{itemize}
        \item \textbf{Short-Term:} Immediately enable and enforce MFA on the email system (\texttt{TitaniumCore.net}) for all users. This is the most critical first step to protect against business email compromise.
        \item \textbf{Long-Term:} Develop a phased rollout plan to enforce MFA for all computer logins, VPN access, and access to any systems storing sensitive organizational data.
    \end{itemize}
\end{enumerate}

\subsection*{High Priority Recommendations}

\begin{enumerate}
    \setcounter{enumi}{2} % Continue numbering from previous list
    \item \textbf{Establish a Security Governance Foundation:}
    \begin{itemize}
        \item \textbf{Short-Term:} Develop and distribute a baseline Acceptable Use Policy (AUP) that outlines rules for technology use, password security, and data handling.
        \item \textbf{Long-Term:} Implement a mandatory security awareness training program for all new and existing employees. This program should be conducted annually and cover topics such as phishing, password security, and incident reporting.
    \end{itemize}
\end{enumerate}

\end{document}
```