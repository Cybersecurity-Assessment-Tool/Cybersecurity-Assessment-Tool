```latex
\documentclass[12pt]{article}

% ----------------------------------------------------------------------
% PREAMBLE
% ----------------------------------------------------------------------
\usepackage[margin=1in]{geometry}
\usepackage{pifont} % For checkmarks and crosses
\usepackage{booktabs} % For professional tables
\usepackage{hyperref} % For clickable links
\usepackage{url} % For formatting URLs
\usepackage{seqsplit} % For splitting long strings like IPs
\usepackage{graphicx}
\usepackage{xcolor}
\usepackage{datetime}

% --- Document Metadata ---
\title{Cybersecurity Posture Assessment Report}
\author{Cybersecurity Analysis Division}
\date{\today}

% --- Hyperref Setup ---
\hypersetup{
    colorlinks=true,
    linkcolor=black,
    urlcolor=blue,
    pdftitle={Cybersecurity Posture Assessment Report},
    pdfauthor={Cybersecurity Analysis Division},
    pdfsubject={Security Assessment},
    pdfkeywords={Cybersecurity, Risk, Assessment, Nmap}
}

% --- Custom Commands ---
\newcommand{\yes}{\ding{51}} % Green checkmark
\newcommand{\no}{\ding{55}}  % Red cross

% ----------------------------------------------------------------------
% DOCUMENT START
% ----------------------------------------------------------------------
\begin{document}

\maketitle
\thispagestyle{empty}
\newpage

\tableofcontents
\thispagestyle{empty}
\newpage

% ----------------------------------------------------------------------
% 1. EXECUTIVE SUMMARY
% ----------------------------------------------------------------------
\section{Executive Summary}

This report details the findings of a cybersecurity posture assessment for \textbf{Crestview Analytics}. The evaluation combines a review of organizational security controls, an external network scan, and an analysis of pre-existing risks.

The assessment reveals a mixed security posture. The organization demonstrates strong foundational controls, particularly in the mandatory use of Multi-Factor Authentication (MFA) across critical systems. However, two key areas of concern were identified that require immediate attention:

\begin{enumerate}
    \item \textbf{High-Risk Policy Gap:} The organization does not conduct mandatory annual security awareness training for all employees. This represents a significant vulnerability, as it leaves the primary defense against social engineering and phishing attacks underdeveloped.
    \item \textbf{Medium-Risk Technical Exposure:} An external network scan identified an exposed Secure Shell (SSH) service (port 22) on a public-facing IPv6 address. Publicly accessible administrative services increase the attack surface and expose the organization to brute-force attacks and potential exploitation.
\end{enumerate}

While no pre-existing vulnerabilities were reported, these new findings present actionable risks. This report provides specific, prioritized recommendations to mitigate these risks and enhance the overall security posture of \textbf{Crestview Analytics}.

\newpage

% ----------------------------------------------------------------------
% 2. ORGANIZATIONAL INFORMATION
% ----------------------------------------------------------------------
\section{Organizational Information}

The following information was provided for the assessment.

\begin{tabular}{@{}ll}
    \toprule
    \textbf{Attribute} & \textbf{Value} \\
    \midrule
    Organization Name & \textbf{Crestview Analytics} \\
    Email Domain & \texttt{CrestviewAnalytics.org} \\
    Website Domain & \url{www.CrestviewAnalytics.org} \\
    External IP (IPv4) & \texttt{32.21.165.32} \\
    External IP (IPv6 Scanned) & \seqsplit{\texttt{2001:db8::1}} \\
    \bottomrule
\end{tabular}

% ----------------------------------------------------------------------
% 3. SECURITY CONTROL REVIEW
% ----------------------------------------------------------------------
\section{Security Control Review}

A review of self-reported security controls was conducted based on a standardized questionnaire. The results indicate a strong adoption of MFA but highlight a critical gap in ongoing employee security training.

\subsection{Questionnaire Results}

\begin{tabular}{@{}p{0.8\linewidth}c@{}}
    \toprule
    \textbf{Control Question} & \textbf{Response} \\
    \midrule
    Do you require MFA to access email? & \yes \\
    Do you require MFA to log into computers? & \yes \\
    Do you require MFA to access sensitive data systems? & \yes \\
    Does your organization have an employee acceptable use policy? & \yes \\
    Does your organization do security awareness training for new employees? & \yes \\
    \textbf{Does your organization do security awareness training for all employees at least once per year?} & \no \\
    \bottomrule
\end{tabular}

\subsection{Analysis}
The single "No" response is a significant finding. While providing security training to new hires is a good practice, the security landscape evolves rapidly. Without annual refresher training for all staff, the organization's human firewall weakens over time, making it more susceptible to phishing, business email compromise, and other social engineering attacks. This is classified as a \textbf{High} risk.

% ----------------------------------------------------------------------
% 4. TECHNICAL SCAN RESULTS
% ----------------------------------------------------------------------
\section{Technical Scan Results}

An external network scan was performed to identify open ports and exposed services on the organization's public-facing infrastructure.

\begin{itemize}
    \item \textbf{Target IP Address:} \seqsplit{\texttt{2001:db8::1}}
    \item \textbf{Scan Date:} \today
\end{itemize}

\subsection{Open Ports Discovered}

The scan identified one open port on the target system.

\begin{tabular}{@{}llll@{}}
    \toprule
    \textbf{Port} & \textbf{State} & \textbf{Service (Presumed)} & \textbf{Notes} \\
    \midrule
    22/tcp & open & SSH (Secure Shell) & No version information was available. \\
    \bottomrule
\end{tabular}

\subsection{Analysis}
Port 22 is used for SSH, a common protocol for remote system administration. While essential for management, exposing SSH directly to the public internet is a security risk. It allows attackers to perform brute-force password attacks, probe for username enumeration, and attempt to exploit any vulnerabilities in the SSH server software. This finding is classified as a \textbf{Medium} risk.

% ----------------------------------------------------------------------
% 5. RISK ASSESSMENT SUMMARY
% ----------------------------------------------------------------------
\section{Risk Assessment Summary}

This section synthesizes the findings from the security control review and the technical scan into a prioritized list of identified risks. No pre-existing risks were provided for this assessment.

\begin{tabular}{@{}lp{0.2\linewidth}lp{0.5\linewidth}@{}}
    \toprule
    \textbf{ID} & \textbf{Risk Name} & \textbf{Severity} & \textbf{Description} \\
    \midrule
    \textbf{R-01} & Lack of Annual Security Training & \textbf{High} & The absence of mandatory, recurring security awareness training for all employees weakens the organization's defense against phishing and social engineering, increasing the likelihood of a security breach originating from human error. \\
    \\
    \textbf{R-02} & Exposed SSH Service & \textbf{Medium} & The SSH management port (22) is open on a public IPv6 address. This exposure increases the attack surface, making the system a target for automated brute-force attacks and exploitation of potential software vulnerabilities. \\
    \bottomrule
\end{tabular}

% ----------------------------------------------------------------------
% 6. RECOMMENDATIONS
% ----------------------------------------------------------------------
\section{Recommendations}

The following actions are recommended to mitigate the identified risks and improve the overall security posture of \textbf{Crestview Analytics}.

\begin{enumerate}
    \item \textbf{[High Priority] Implement Annual Security Awareness Training (Mitigates R-01):}
    \begin{itemize}
        \item Establish a formal, mandatory security awareness training program to be completed by all employees annually.
        \item The curriculum should cover modern threats, including phishing identification, password hygiene, safe browsing, and the organization's acceptable use policy.
        \item Implement a tracking system to ensure 100\% completion and to identify individuals needing remedial training.
    \end{itemize}
    \vspace{1em}
    \item \textbf{[Medium Priority] Secure the Exposed SSH Service (Mitigates R-02):}
    \begin{itemize}
        \item \textbf{Restrict Access:} The preferred solution is to place the SSH service behind a Virtual Private Network (VPN). If this is not feasible, use a firewall to restrict access to a limited set of trusted source IP addresses.
        \item \textbf{Harden Configuration:} Regardless of access method, ensure the SSH service is securely configured by:
        \begin{itemize}
            \item Disabling password-based authentication and enforcing the use of public key authentication.
            \item Disabling root user login over SSH.
            \item Keeping the SSH server software updated to the latest patched version.
        \end{itemize}
    \end{itemize}
\end{enumerate}

\end{document}
```