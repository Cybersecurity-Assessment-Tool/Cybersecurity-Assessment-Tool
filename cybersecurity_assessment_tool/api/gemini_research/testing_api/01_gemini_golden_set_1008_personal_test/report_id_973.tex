```latex
\documentclass[12pt]{article}

% Preamble: Required Packages
\usepackage[margin=1in]{geometry}
\usepackage{pifont} % For checkmarks and crosses
\usepackage{booktabs} % For professional tables
\usepackage{hyperref} % For clickable links
\usepackage{url} % For URL formatting
\usepackage{seqsplit} % For splitting long strings in tt font

% Document Metadata
\title{Cybersecurity Posture Assessment Report}
\author{Cybersecurity Analysis Division}
\date{November 22, 2025}

\begin{document}

\maketitle
\thispagestyle{empty}
\newpage

\tableofcontents
\newpage

% --- Executive Summary ---
\section{Executive Summary}
This report provides a comprehensive analysis of the cybersecurity posture for \textbf{Phoenix Rising}, based on data collected on November 22, 2025. The assessment combines a review of organizational security controls, an external network scan, and an evaluation of pre-existing risks.

The organization demonstrates a solid foundation in security awareness, with established policies for acceptable use and regular employee training. However, this assessment has identified two \textbf{Critical} security gaps and one \textbf{High} risk technical vulnerability that require immediate attention.

The most significant weaknesses are the absence of Multi-Factor Authentication (MFA) for computer logins and access to sensitive data systems. These gaps expose the organization to significant risk from credential theft and unauthorized access. Additionally, the external-facing web server is running an outdated and vulnerable version of Nginx.

We strongly recommend prioritizing the implementation of MFA across all critical assets and upgrading the vulnerable web server software to mitigate these risks and improve the overall security posture.

% --- Organizational Information ---
\section{Organizational Information}
The following details were provided for the assessment.

\begin{itemize}
    \item \textbf{Organization Name:} Phoenix Rising
    \item \textbf{Email Domain:} \texttt{PhoenixRising.com}
    \item \textbf{Primary Website:} \url{www.PhoenixRising.com}
    \item \textbf{External IP Address:} \texttt{16.47.5.171}
\end{itemize}

% --- Security Control Review ---
\section{Security Control Review}
A review of the organization's security controls was conducted via a standardized questionnaire. The responses indicate key areas of strength and weakness in administrative controls. The absence of MFA on endpoints and for sensitive data access represents a critical gap in the organization's defense-in-depth strategy.

\begin{table}[h!]
\centering
\caption{Security Controls Questionnaire Results}
\begin{tabular}{p{0.75\linewidth} c}
\toprule
\textbf{Control Question} & \textbf{Response} \\
\midrule
Do you require MFA to access email? & \ding{51} \\
Do you require MFA to log into computers? & \textbf{\color{red}\ding{55}} \\
Do you require MFA to access sensitive data systems? & \textbf{\color{red}\ding{55}} \\
Does your organization have an employee acceptable use policy? & \ding{51} \\
Does your organization do security awareness training for new employees? & \ding{51} \\
Does your organization do security awareness training for all employees at least once per year? & \ding{51} \\
\bottomrule
\end{tabular}
\end{table}

% --- Technical Scan Results ---
\section{Technical Scan Results}
An external network scan was performed against the target IP address \texttt{192.168.10.5} to identify open ports and exposed services.

\subsection{Scan Details}
\begin{itemize}
    \item \textbf{Target IP:} \texttt{192.168.10.5}
    \item \textbf{Scan Date:} 2025-11-22T10:00:00Z
\end{itemize}

\subsection{Open Ports}
The following table details the services discovered during the scan.

\begin{table}[h!]
\centering
\caption{Discovered Open Ports and Services}
\begin{tabular}{l l l l l}
\toprule
\textbf{Port} & \textbf{State} & \textbf{Service} & \textbf{Product} & \textbf{Version} \\
\midrule
443/tcp & open & https & nginx & 1.18.0 \\
\bottomrule
\end{tabular}
\end{table}

\subsection{Technical Analysis}
The scan identified an Nginx web server, version \textbf{1.18.0}, exposed to the internet. This version was released in 2020 and is now considered end-of-life (EOL). It is no longer receiving security updates and is known to be vulnerable to multiple publicly disclosed exploits (e.g., CVE-2021-23017). Running EOL software on a public-facing server constitutes a high-risk vulnerability that could be exploited by attackers to compromise the server and potentially gain access to the internal network.

% --- Risk Assessment ---
\section{Risk Assessment}
This section synthesizes findings from the security control review and the technical scan. No pre-existing vulnerabilities were reported. The following new risks have been identified and prioritized.

\begin{table}[h!]
\centering
\caption{Summary of Identified Risks}
\begin{tabular}{p{0.1\linewidth} p{0.3\linewidth} p{0.4\linewidth} l}
\toprule
\textbf{ID} & \textbf{Risk Name} & \textbf{Description} & \textbf{Severity} \\
\midrule
RISK-001 & Lack of MFA on Endpoints & User computers are not protected by MFA, making them susceptible to takeover if an employee's password is compromised. & \textbf{Critical} \\
\addlinespace
RISK-002 & Lack of MFA on Sensitive Systems & Critical data systems can be accessed without MFA, drastically increasing the risk of a data breach via stolen credentials. & \textbf{Critical} \\
\addlinespace
RISK-003 & Vulnerable Web Server Software & The public-facing Nginx server is running an outdated, end-of-life version (1.18.0) with known vulnerabilities. & \textbf{High} \\
\bottomrule
\end{tabular}
\end{table}

% --- Recommendations ---
\section{Recommendations}
Based on the identified risks, we provide the following actionable recommendations to enhance the security posture of \textbf{Phoenix Rising}.

\begin{enumerate}
    \item \textbf{[Critical] Implement Comprehensive MFA:}
    Deploy a robust Multi-Factor Authentication (MFA) solution across the organization.
    \begin{itemize}
        \item \textbf{Priority 1:} Enforce MFA for all user logins to company computers and laptops (RISK-001).
        \item \textbf{Priority 2:} Enforce MFA for access to all systems containing sensitive or critical business data (RISK-002).
    \end{itemize}

    \item \textbf{[High] Remediate Vulnerable Web Server:}
    Immediately plan and execute an upgrade for the Nginx web server.
    \begin{itemize}
        \item Upgrade Nginx from version 1.18.0 to a currently supported and stable version to mitigate known vulnerabilities (RISK-003).
        \item Implement a formal patch management policy to ensure all internet-facing systems are regularly updated.
    \end{itemize}

    \item \textbf{[Strategic] Establish a Vulnerability Management Program:}
    Develop a formal program to proactively identify, assess, and remediate both technical vulnerabilities and policy gaps on an ongoing basis. This will help prevent similar issues from arising in the future.

\end{enumerate}

\end{document}
```