```latex
\documentclass[12pt]{article}

% --- PACKAGE IMPORTS ---
\usepackage[margin=1in]{geometry}
\usepackage{pifont} % For checkmarks and crosses
\usepackage{booktabs} % For professional tables
\usepackage{hyperref} % For clickable links
\usepackage{url} % For URL formatting
\usepackage{seqsplit} % To split long strings without breaking
\usepackage{graphicx}
\usepackage{xcolor}
\usepackage{fancyhdr}

% --- DOCUMENT METADATA & SETUP ---
\hypersetup{
    colorlinks=true,
    linkcolor=blue,
    filecolor=magenta,      
    urlcolor=cyan,
    pdftitle={Cybersecurity Posture Report},
    pdfpagemode=FullScreen,
}

\pagestyle{fancy}
\fancyhf{}
\lhead{Cybersecurity Posture Report}
\rhead{\textbf{Evergreen Alliance}}
\cfoot{\thepage}

\newcommand{\yes}{\ding{51}} % Green checkmark
\newcommand{\no}{\ding{55}}  % Red X

\begin{document}

% --- TITLE PAGE ---
\begin{titlepage}
    \centering
    \vspace*{1cm}
    \Huge\textbf{Cybersecurity Posture Report}
    \vspace{1.5cm}
    \Large
    Prepared for:\\
    \vspace{0.5cm}
    \textbf{Evergreen Alliance}
    \vspace{2cm}
    \includegraphics[width=0.4\textwidth]{example-image-a} % Placeholder for company logo
    \vfill
    \Large
    Report Date: \today
\end{titlepage}

\tableofcontents
\newpage

% --- SECTION 1: EXECUTIVE SUMMARY ---
\section{Executive Summary}

This report provides a comprehensive analysis of the cybersecurity posture for \textbf{Evergreen Alliance}, based on a technical network scan, a review of existing risks, and an organizational security controls questionnaire.

The overall security posture is largely positive. The organization demonstrates a strong commitment to security fundamentals, including the implementation of Multi-Factor Authentication (MFA) for email and computer access, and robust employee security training programs.

A technical scan of the target host \texttt{192.168.0.5} revealed no open ports, which is an excellent security configuration. Notably, a previously identified risk concerning an open web server port (Port 80) appears to have been successfully remediated, as the port was found to be closed during this assessment.

However, a critical gap was identified in the security controls: \textbf{MFA is not required for accessing sensitive data systems}. This represents the most significant risk to the organization, as it exposes critical assets to potential compromise via stolen credentials.

Our primary recommendation is the immediate planning and implementation of an MFA solution for all systems housing sensitive data.

% --- SECTION 2: ORGANIZATIONAL INFORMATION ---
\section{Organizational Information}

The following details were provided for the assessment.

\begin{itemize}
    \item \textbf{Organization Name:} Evergreen Alliance
    \item \textbf{Email Domain:} \texttt{EvergreenAlliance.com}
    \item \textbf{Website Domain:} \url{www.EvergreenAlliance.com}
    \item \textbf{External IP Address:} \texttt{95.13.14.128}
\end{itemize}

% --- SECTION 3: SECURITY CONTROL REVIEW ---
\section{Security Control Review}

The following table summarizes the organization's responses to a security controls questionnaire. These controls are foundational for establishing a secure operational environment.

\begin{table}[h!]
\centering
\caption{Security Controls Questionnaire Results}
\begin{tabular}{p{0.7\linewidth} c}
\toprule
\textbf{Control Question} & \textbf{Response} \\
\midrule
Do you require MFA to access email? & \yes \\
Do you require MFA to log into computers? & \yes \\
\textbf{Do you require MFA to access sensitive data systems?} & \textbf{\textcolor{red}{\no}} \\
Does your organization have an employee acceptable use policy? & \yes \\
Does your organization do security awareness training for new employees? & \yes \\
Does your organization do security awareness training for all employees at least once per year? & \yes \\
\bottomrule
\end{tabular}
\end{table}

\subsection{Analysis}
The organization has implemented a majority of the essential security controls reviewed. The presence of MFA for email and computer logins, combined with a comprehensive security awareness program, significantly reduces the risk of common cyber threats.

However, the absence of MFA for sensitive data systems is a critical vulnerability. Should an attacker compromise an employee's credentials, they could potentially gain direct access to the organization's most valuable data without needing a second authentication factor. This gap undermines the other strong security measures in place.

% --- SECTION 4: TECHNICAL SCAN RESULTS ---
\section{Technical Scan Results}

A network scan was performed to identify open ports and exposed services on the specified target system.

\begin{itemize}
    \item \textbf{Target IP Address:} \texttt{192.168.0.5}
    \item \textbf{Scan Status:} Host is up.
\end{itemize}

\begin{table}[h!]
\centering
\caption{Port Scan Details for \texttt{192.168.0.5}}
\begin{tabular}{ccccc}
\toprule
\textbf{Port} & \textbf{State} & \textbf{Service} & \textbf{Product} & \textbf{Version} \\
\midrule
80 & closed & http & N/A & N/A \\
\bottomrule
\end{tabular}
\end{table}

\subsection{Analysis}
The technical scan revealed no open ports on the target system. This indicates a very secure network configuration, minimizing the external attack surface. The finding that Port 80 is closed is particularly noteworthy, as it directly contradicts a previously documented risk. This suggests that the risk has been successfully remediated.

% --- SECTION 5: RISK ASSESSMENT SUMMARY ---
\section{Risk Assessment Summary}

The following table correlates findings from the security control review, the technical scan, and pre-existing risk data.

\begin{table}[h!]
\centering
\caption{Consolidated Risk Register}
\begin{tabular}{p{0.3\linewidth} p{0.5\linewidth} l}
\toprule
\textbf{Risk Name} & \textbf{Description} & \textbf{Severity} \\
\midrule
\textbf{Lack of MFA for Sensitive Data} & The absence of MFA on systems containing sensitive data exposes the organization to significant risk from credential theft and unauthorized access. & \textbf{\textcolor{red}{High}} \\
\addlinespace
Unencrypted Web Server & A pre-existing risk noted that Port 80 was open. The current scan confirms this port is now closed, indicating the risk is remediated. & \textbf{\textcolor{blue}{Informational / Closed}} \\
\bottomrule
\end{tabular}
\end{table}

% --- SECTION 6: RECOMMENDATIONS ---
\section{Recommendations}

Based on the analysis, the following actions are recommended to enhance the security posture of \textbf{Evergreen Alliance}.

\subsection{Priority 1: Implement MFA for Sensitive Data Systems (High)}
\begin{itemize}
    \item \textbf{Action:} Enforce Multi-Factor Authentication across all applications and systems that store, process, or transmit sensitive organizational or client data.
    \item \textbf{Justification:} This is the single most effective control to prevent unauthorized access resulting from compromised credentials. It acts as a critical barrier protecting the organization's most valuable assets.
    \item \textbf{Suggested Steps:}
    \begin{enumerate}
        \item Inventory all systems that handle sensitive data.
        \item Evaluate and select an MFA solution that integrates with your existing technology stack (e.g., Duo, Okta, Microsoft Authenticator).
        \item Develop a phased rollout plan, prioritizing the most critical systems first.
        \item Communicate the change to all affected employees and provide training on the new login process.
    \end{enumerate}
\end{itemize}

\subsection{Priority 2: Update Internal Risk Register (Informational)}
\begin{itemize}
    \item \textbf{Action:} Formally update the internal risk register to mark the "Unencrypted Web Server" risk as closed or remediated.
    \item \textbf{Justification:} Maintaining an accurate and up-to-date risk register is crucial for effective security management. This ensures that resources are focused on current, relevant threats rather than outdated issues.
    \item \textbf{Suggested Steps:}
    \begin{enumerate}
        \item Reference this report's technical scan results as evidence of remediation.
        \item Archive the old risk entry and associated documentation.
    \end{enumerate}
\end{itemize}

\end{document}
```