```latex
\documentclass[12pt, a4paper]{article}

% Preamble: Required Packages
\usepackage[margin=1in]{geometry}
\usepackage{pifont} % For checkmarks and crosses
\usepackage{booktabs} % For professional tables
\usepackage{hyperref} % For clickable links
\usepackage{url} % For formatting URLs
\usepackage{seqsplit} % To split long strings in tt font
\usepackage{graphicx}
\usepackage{xcolor}

% --- Document Setup ---
\hypersetup{
    colorlinks=true,
    linkcolor=blue,
    filecolor=magenta,      
    urlcolor=cyan,
    pdftitle={Cybersecurity Posture Report},
    pdfpagemode=FullScreen,
}

\newcommand{\yes}{\ding{51}}
\newcommand{\no}{\ding{55}}

% --- Document Body ---
\begin{document}

% --- Title Page ---
\begin{titlepage}
    \centering
    \vspace*{1cm}
    \Huge\textbf{Cybersecurity Posture Report}
    \vspace{1.5cm}
    \Large
    \textbf{Prepared for:} \\
    Top Tier
    \vspace{2cm}
    \large
    \textbf{Date of Report:} \\
    \today
    \vfill
    \large
    \textbf{CONFIDENTIAL} \\
    \textit{This document contains sensitive information and is intended for the exclusive use of the recipient.}
\end{titlepage}

\tableofcontents
\newpage

% --- Section 1: Executive Summary ---
\section{Executive Summary}
This report provides a comprehensive analysis of the cybersecurity posture for \textbf{Top Tier}, based on a review of organizational security controls, an external network scan, and pre-existing risk data. The assessment was conducted to identify key vulnerabilities, policy gaps, and areas for security enhancement.

Overall, \textbf{Top Tier} has implemented foundational security controls, such as requiring Multi-Factor Authentication (MFA) for email and computer access. The external network scan of the target system \texttt{[Target IP]} did not reveal any open ports, which is a positive security finding for that specific asset.

However, the assessment identified three critical gaps in the organization's security program that significantly elevate its risk profile:
\begin{itemize}
    \item \textbf{Lack of MFA for Sensitive Data Systems:} The absence of mandatory MFA for systems housing sensitive data presents a critical risk of unauthorized access and potential data breach.
    \item \textbf{Absence of an Acceptable Use Policy (AUP):} Without a formal AUP, there is no clear guidance for employees on the proper use of company assets, increasing the risk of insider threats and misuse of resources.
    \item \textbf{Inadequate Security Awareness Training:} While new employees receive training, the lack of a mandatory annual refresher course for all staff leaves the organization vulnerable to evolving threats like phishing and social engineering.
\end{itemize}

This report details these findings and provides prioritized, actionable recommendations to mitigate the identified risks and strengthen the organization's overall security posture.

% --- Section 2: Organizational Information ---
\section{Organizational Information}
The following details were provided for the assessment. This information helps to establish the context and scope of the review.

\begin{tabular}{@{}ll}
    \toprule
    \textbf{Attribute} & \textbf{Value} \\
    \midrule
    Organization Name & Top Tier \\
    Email Domain & \texttt{TopTier.net} \\
    Website Domain & \url{www.TopTier.net} \\
    External IP Address & \texttt{13.233.22.187} \\
    \bottomrule
\end{tabular}

% --- Section 3: Security Control Review ---
\section{Security Control Review}
A review of organizational security controls was conducted via a questionnaire. The responses are summarized below, highlighting alignment with security best practices. Gaps identified here form the basis for several high-priority risks.

\begin{tabular}{@{}p{0.6\textwidth} c p{0.2\textwidth}@{}}
    \toprule
    \textbf{Control Question} & \textbf{Response} & \textbf{Assessment} \\
    \midrule
    Do you require MFA to access email? & \yes & Best Practice Met \\
    Do you require MFA to log into computers? & \yes & Best Practice Met \\
    Do you require MFA to access sensitive data systems? & \textcolor{red}{\no} & \textbf{Critical Gap} \\
    Does your organization have an employee acceptable use policy? & \textcolor{red}{\no} & \textbf{High Risk Gap} \\
    Does your organization do security awareness training for new employees? & \yes & Good Practice \\
    Does your organization do security awareness training for all employees at least once per year? & \textcolor{red}{\no} & \textbf{High Risk Gap} \\
    \bottomrule
\end{tabular}

% --- Section 4: Technical Scan Results ---
\section{Technical Scan Results}
An external network vulnerability scan was performed to identify open ports and exposed services on the perimeter.

\begin{itemize}
    \item \textbf{Target IP Address:} \texttt{[Target IP]}
    \item \textbf{Scan Date:} Not provided in scan data.
\end{itemize}

\subsection{Scan Summary}
The scan completed successfully against the target system. \textbf{No open ports were detected.} This indicates a strong network firewall configuration for this specific host, as it does not expose any services to the public internet. While this is a positive finding, it is important to note that this assessment is limited to the single IP address provided.

% --- Section 5: Risk Assessment ---
\section{Risk Assessment}
This section synthesizes findings from the security control review and technical scan. No pre-existing vulnerabilities were provided in the input data. The following risks have been identified and prioritized based on their potential impact on the organization.

\begin{tabular}{@{}p{0.25\textwidth} p{0.5\textwidth} p{0.15\textwidth}@{}}
    \toprule
    \textbf{Risk Name} & \textbf{Overview} & \textbf{Severity} \\
    \midrule
    \textbf{Lack of MFA on Sensitive Systems} & The absence of MFA on critical systems creates a single point of failure (password compromise) for protecting the organization's most valuable data. This could lead to a severe data breach. & \textbf{Critical} \\
    \addlinespace
    \textbf{Missing Employee Acceptable Use Policy} & Without a formal AUP, employees lack clear rules on technology usage. This ambiguity can lead to unintentional data exposure, misuse of company assets, and a weakened legal standing in case of an insider incident. & \textbf{High} \\
    \addlinespace
    \textbf{Inadequate Annual Security Training} & Security knowledge degrades over time, and threat actor tactics evolve. Failing to provide annual refresher training makes employees more susceptible to phishing, ransomware, and other social engineering attacks. & \textbf{High} \\
    \bottomrule
\end{tabular}

% --- Section 6: Recommendations ---
\section{Recommendations}
The following recommendations are provided to address the identified risks. They are prioritized to ensure that the most critical vulnerabilities are remediated first.

\subsection{Priority 1: Critical}
\begin{enumerate}
    \item \textbf{Implement MFA for All Sensitive Data Systems:}
    \begin{itemize}
        \item \textbf{Action:} Immediately begin a project to identify all systems and applications containing sensitive or regulated data. Procure and deploy an MFA solution that integrates with these systems.
        \item \textbf{Justification:} This is the single most effective control to prevent unauthorized access resulting from compromised credentials. It directly mitigates the risk of a major data breach.
    \end{itemize}
\end{enumerate}

\subsection{Priority 2: High}
\begin{enumerate}
    \setcounter{enumi}{1}
    \item \textbf{Develop and Implement an Acceptable Use Policy (AUP):}
    \begin{itemize}
        \item \textbf{Action:} Draft a formal AUP that clearly defines the rules for using company networks, devices, software, and data. The policy should be reviewed by legal counsel, communicated to all employees, and require a signed acknowledgment of receipt.
        \item \textbf{Justification:} An AUP establishes a baseline for secure employee behavior, reduces organizational liability, and provides a framework for enforcing security standards.
    \end{itemize}
    \item \textbf{Establish a Mandatory Annual Security Awareness Program:}
    \begin{itemize}
        \item \textbf{Action:} Implement a recurring, mandatory security awareness training program for all employees. This program should cover current threats such as phishing, ransomware, and proper data handling. Track completion to ensure 100\% participation.
        \item \textbf{Justification:} A well-trained workforce is a critical layer of defense. Regular training reinforces good security habits and keeps employees vigilant against new and evolving cyber threats.
    \end{itemize}
\end{enumerate}

\end{document}
```