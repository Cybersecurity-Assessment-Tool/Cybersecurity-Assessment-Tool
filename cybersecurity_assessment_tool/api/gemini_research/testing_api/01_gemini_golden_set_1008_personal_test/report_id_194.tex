```latex
\documentclass[12pt]{article}

% Preamble: Required Packages
\usepackage[margin=1in]{geometry}
\usepackage{pifont} % For checkmarks and crosses
\usepackage{booktabs} % For professional tables
\usepackage{hyperref} % For clickable links and references
\usepackage{url} % For formatting URLs
\usepackage{seqsplit} % For splitting long strings to prevent overflow
\usepackage{xcolor} % For coloring text

% Document Metadata
\title{Cybersecurity Assessment Report}
\author{Cybersecurity Analysis Division}
\date{\today}

% Define severity colors
\definecolor{critical}{HTML}{990000}
\definecolor{high}{HTML}{D1410C}
\definecolor{medium}{HTML}{E89803}
\definecolor{low}{HTML}{3D9140}

\begin{document}

\maketitle
\tableofcontents
\newpage

% --- Section 1: Executive Summary ---
\section{Executive Summary}

This report provides a comprehensive cybersecurity assessment for \textbf{New Era}. The analysis is based on a synthesis of technical network scan data, a review of organizational security controls, and pre-existing risk information.

The assessment reveals several critical and high-risk vulnerabilities that expose the organization to significant threats, including unauthorized data access, account compromise, and malware infection. Key findings include:

\begin{itemize}
    \item \textbf{Critical FTP Vulnerability:} An internal server (\texttt{10.0.0.15}) is running a dangerously outdated version of \texttt{vsftpd} (2.3.4), which is known to contain a critical backdoor vulnerability (CVE-2011-2523). This service also permits anonymous, unauthenticated access.
    \item \textbf{Insufficient Access Controls:} Multi-Factor Authentication (MFA) is not enforced for employee email or computer logins, dramatically increasing the risk of account takeovers from phishing or password guessing attacks.
    \item \textbf{Systemic Policy Gaps:} The organization lacks a formal Acceptable Use Policy and does not conduct any security awareness training for its employees. This creates a culture where security best practices are not understood or enforced, making the organization highly susceptible to social engineering.
    \item \textbf{Outdated Operating Systems:} A known risk of using end-of-life Windows 7 workstations persists, leaving systems unpatched against modern threats.
\end{itemize}

Immediate remediation of the FTP server and the implementation of MFA are strongly recommended to mitigate the most severe risks. A strategic effort to address the policy and training gaps is necessary to build a more resilient long-term security posture.

% --- Section 2: Organizational Information ---
\section{Organizational Information}

The following details were provided for the assessment.

\begin{tabular}{@{}ll}
    \toprule
    \textbf{Attribute} & \textbf{Value} \\
    \midrule
    Organization Name & \textbf{New Era} \\
    Email Domain & \texttt{NewEra.net} \\
    Website Domain & \url{www.NewEra.net} \\
    External IP Address & \texttt{116.222.121.13} \\
    \bottomrule
\end{tabular}

% --- Section 3: Security Control Review ---
\section{Security Control Review}

A review of foundational security controls was conducted based on a questionnaire. The responses indicate significant gaps in security policy and access control enforcement. A checkmark (\ding{51}) indicates a positive control, while a cross (\ding{55}) indicates a gap.

\begin{table}[h!]
\centering
\begin{tabular}{@{}lc}
    \toprule
    \textbf{Control Question} & \textbf{Response} \\
    \midrule
    Do you require MFA to access email? & \ding{55} \\
    Do you require MFA to log into computers? & \ding{55} \\
    Do you require MFA to access sensitive data systems? & \ding{51} \\
    Does your organization have an employee acceptable use policy? & \ding{55} \\
    Does your organization do security awareness training for new employees? & \ding{55} \\
    Does your organization do security awareness training for all employees annually? & \ding{55} \\
    \bottomrule
\end{tabular}
\caption{Organizational Security Control Status}
\end{table}

% --- Section 4: Technical Scan Results ---
\section{Technical Scan Results}

An internal network scan was performed to identify active services and potential vulnerabilities.

\subsection{Host: \texttt{10.0.0.15}}
A single host was identified with an open port presenting a critical risk.

\begin{itemize}
    \item \textbf{Port 21/tcp (FTP):} Open
    \begin{itemize}
        \item \textbf{Service:} vsftpd version 2.3.4
        \item \textbf{Finding 1 (Critical):} The identified version of \texttt{vsftpd} (2.3.4) was released in 2011 and contains a well-documented critical backdoor vulnerability (\textbf{CVE-2011-2523}). An attacker can exploit this vulnerability to gain a command shell on the underlying server, leading to a full system compromise.
        \item \textbf{Finding 2 (High):} The FTP service is configured to allow \textbf{anonymous login}. This permits any user on the network to access, upload, or download files without authentication, posing a significant risk of data leakage or malware implantation.
    \end{itemize}
\end{itemize}

% --- Section 5: Consolidated Risk Assessment ---
\section{Consolidated Risk Assessment}

The following table summarizes the identified risks, combining technical findings, control gaps, and pre-existing issues. Risks are prioritized by severity.

\begin{table}[h!]
\centering
\begin{tabular}{@{}p{0.3\linewidth} p{0.5\linewidth} l}
    \toprule
    \textbf{Risk Name} & \textbf{Overview} & \textbf{Severity} \\
    \midrule
    \textbf{Vulnerable FTP Server} & The server at \texttt{10.0.0.15} runs \texttt{vsftpd 2.3.4}, which has a known remote code execution backdoor (CVE-2011-2523). & \textcolor{critical}{\textbf{Critical}} \\
    \addlinespace
    \textbf{Lack of MFA on Email and Endpoints} & No MFA is required for email or computer access, making user accounts highly vulnerable to compromise via phishing or password reuse. & \textcolor{critical}{\textbf{Critical}} \\
    \addlinespace
    \textbf{Anonymous FTP Access} & The FTP server allows unauthenticated access, creating a high risk of unauthorized data access or the introduction of malicious files. & \textcolor{high}{\textbf{High}} \\
    \addlinespace
    \textbf{No Security Awareness Training} & Employees are not trained on security topics, making them susceptible to social engineering attacks like phishing, which could lead to credential theft or malware. & \textcolor{high}{\textbf{High}} \\
    \addlinespace
    \textbf{Outdated Windows 7 Workstations} & The organization is using Windows 7, an end-of-life operating system that no longer receives security updates from Microsoft, leaving it vulnerable to exploitation. & \textcolor{medium}{\textbf{Medium}} \\
    \addlinespace
    \textbf{Absence of Acceptable Use Policy} & Lack of a formal AUP means there are no clear rules for employees regarding the safe and acceptable use of company assets, leading to inconsistent security practices. & \textcolor{medium}{\textbf{Medium}} \\
    \bottomrule
\end{tabular}
\caption{Summary of Identified Risks}
\end{table}

% --- Section 6: Recommendations ---
\section{Recommendations}

The following actions are recommended to mitigate the identified risks and improve the overall security posture of \textbf{New Era}.

\begin{enumerate}
    \item \textbf{[Immediate] Remediate Vulnerable FTP Server:}
    \begin{itemize}
        \item Immediately upgrade the \texttt{vsftpd} service on \texttt{10.0.0.15} to the latest stable version.
        \item If the FTP service is not essential for business operations, disable and remove it entirely.
        \item Disable anonymous FTP login immediately, regardless of the upgrade path. Access should be restricted to authenticated users only.
    \end{itemize}

    \item \textbf{[High Priority] Implement Multi-Factor Authentication (MFA):}
    \begin{itemize}
        \item Enforce MFA for all user accounts for accessing email (e.g., via Microsoft 365 or Google Workspace policies).
        \item Deploy an MFA solution for all computer and remote access (VPN) logins.
    \end{itemize}

    \item \textbf{[High Priority] Establish a Security Awareness Program:}
    \begin{itemize}
        \item Develop and deliver mandatory security awareness training for all new and existing employees.
        \item Conduct annual refresher training and periodic phishing simulations to reinforce learning and measure effectiveness.
    \end{itemize}

    \item \textbf{[Medium Priority] Develop and Enforce Security Policies:}
    \begin{itemize}
        \item Create a formal Acceptable Use Policy (AUP) that all employees must read and acknowledge. This policy should define rules for data handling, internet usage, and software installation.
    \end{itemize}

    \item \textbf{[Ongoing] Plan Operating System Upgrades:}
    \begin{itemize}
        \item Accelerate the project to upgrade all Windows 7 workstations to a modern, supported operating system like Windows 11 to ensure they receive critical security patches.
    \end{itemize}
\end{enumerate}

\end{document}
```