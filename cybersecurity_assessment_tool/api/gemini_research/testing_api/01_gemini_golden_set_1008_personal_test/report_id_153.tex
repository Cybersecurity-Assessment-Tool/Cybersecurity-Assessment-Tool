```latex
\documentclass[12pt]{article}

% --- PACKAGES ---
\usepackage[margin=1in]{geometry}
\usepackage{pifont} % For checkmarks and crosses
\usepackage{booktabs} % For professional tables
\usepackage{hyperref} % For clickable links
\usepackage{url} % For URL formatting
\usepackage{seqsplit} % To split long strings in texttt
\usepackage[utf8]{inputenc}
\usepackage{graphicx}
\usepackage{xcolor}
\usepackage{fancyhdr}
\usepackage{lastpage}

% --- DOCUMENT SETUP ---
\hypersetup{
    colorlinks=true,
    linkcolor=blue,
    filecolor=magenta,      
    urlcolor=cyan,
    pdftitle={Cybersecurity Posture Assessment Report},
    pdfauthor={Cybersecurity Analysis Division},
}

% --- HEADER & FOOTER ---
\pagestyle{fancy}
\fancyhf{} % Clear all header and footer fields
\fancyhead[L]{Cybersecurity Posture Assessment}
\fancyhead[R]{\textbf{Structure \& Form}}
\fancyfoot[C]{\thepage\ of \pageref{LastPage}}
\renewcommand{\headrulewidth}{0.4pt}
\renewcommand{\footrulewidth}{0.4pt}

% --- DOCUMENT START ---
\begin{document}

\title{Cybersecurity Posture Assessment Report \\ \large For: Structure \& Form}
\author{Cybersecurity Analysis Division}
\date{\today}
\maketitle

\begin{abstract}
\noindent This report provides a comprehensive analysis of the cybersecurity posture for Structure \& Form. The assessment is based on the synthesis of technical network scan data, a review of organizational security controls, and an evaluation of pre-existing risk documentation. The analysis has uncovered a critical-risk finding: a potentially sensitive database interface is exposed externally. This finding directly contradicts previous risk assessments and is exacerbated by identified gaps in security controls, specifically the lack of multi-factor authentication for sensitive systems. Immediate remediation is strongly advised.
\end{abstract}

\newpage
\tableofcontents
\newpage

% ===================================================================
\section{Executive Summary}
% ===================================================================

This assessment combined technical scanning of the network perimeter, a review of organizational security policies via a questionnaire, and an analysis of the current risk register. Our analysis revealed several areas of concern, with one critical vulnerability requiring immediate attention.

\begin{itemize}
    \item \textbf{Critical Exposure Discovered:} A network scan of host \texttt{10.5.5.5} identified an open service on port \texttt{8080} with the HTTP title "TOP SECRET DB". This strongly suggests a sensitive, and potentially unauthenticated, database or application is exposed.
    
    \item \textbf{Contradictory Risk Data:} The current risk register lists port \texttt{8080} as a secured false positive. Our technical findings prove this assessment is outdated or inaccurate, highlighting a potential failure in the risk management lifecycle.
    
    \item \textbf{Control Gaps Magnify Risk:} The organization does not require Multi-Factor Authentication (MFA) for sensitive data systems. This gap, combined with the exposed service, creates a significantly elevated risk of unauthorized data access.
    
    \item \textbf{Policy Weakness:} While foundational policies are in place, the lack of annual security awareness training for all staff increases susceptibility to social engineering and human error.
\end{itemize}

This report details these findings and provides actionable recommendations to mitigate the identified risks and strengthen the overall security posture of Structure \& Form.

% ===================================================================
\section{Organizational & Assessment Information}
% ===================================================================

The following information was used as the basis for this assessment.

\begin{table}[h!]
\centering
\begin{tabular}{@{}ll@{}}
\toprule
\textbf{Attribute} & \textbf{Value} \\
\midrule
Organization Name & Structure \& Form \\
Email Domain & \texttt{StructureForm.com} \\
Known External IP & \seqsplit{\texttt{130.86.143.138}} \\
Internal Scan Target & \seqsplit{\texttt{10.5.5.5}} \\
\bottomrule
\end{tabular}
\caption{Organizational and Assessment Scope.}
\end{table}

% ===================================================================
\section{Security Control Review}
% ===================================================================

A review of the organization's security controls was conducted via a standardized questionnaire. The responses indicate a solid foundation but also reveal critical gaps in authentication and training protocols.

\begin{table}[h!]
\centering
\begin{tabular}{@{}p{0.6\textwidth}cp{0.2\textwidth}@{}}
\toprule
\textbf{Control Question} & \textbf{Response} & \textbf{Assessment} \\
\midrule
Do you require MFA to access email? & \ding{51} Yes & Good Practice \\
Do you require MFA to log into computers? & \ding{51} Yes & Good Practice \\
\textbf{Do you require MFA to access sensitive data systems?} & \textbf{\color{red}\ding{55} No} & \textbf{\color{red}Critical Gap} \\
Does your organization have an employee acceptable use policy? & \ding{51} Yes & Foundational Control \\
Does your organization do security awareness training for new employees? & \ding{51} Yes & Good Onboarding \\
\textbf{Does your organization do security awareness training for all employees at least once per year?} & \textbf{\color{orange}\ding{55} No} & \textbf{\color{orange}High Risk} \\
\bottomrule
\end{tabular}
\caption{Security Controls Questionnaire Analysis.}
\end{table}

% ===================================================================
\section{Technical Scan Results}
% ===================================================================

A network scan was performed on the target host \texttt{10.5.5.5} to identify open ports and exposed services. The scan revealed a critical finding that requires immediate investigation.

\begin{table}[h!]
\centering
\begin{tabular}{@{}llll@{}}
\toprule
\textbf{Port} & \textbf{State} & \textbf{Service Details} & \textbf{Severity} \\
\midrule
8080 & Open & HTTP Title: \textbf{TOP SECRET DB} & \textbf{Critical} \\
\bottomrule
\end{tabular}
\caption{Open Port Analysis for Target: \texttt{10.5.5.5}.}
\end{table}

\subsection{Analysis of Findings}
The service on port \texttt{8080} is highly concerning. The title "TOP SECRET DB" implies that a database containing highly sensitive information is directly accessible from the network. This type of exposure can lead to a catastrophic data breach. This finding directly contradicts the information in the provided risk register, which stated this port was secure. This indicates a significant failure in the risk tracking process.

% ===================================================================
\section{Correlated Risk Assessment}
% ===================================================================

By correlating the technical findings with the security control gaps, we have identified the following key risks to the organization.

\begin{table}[h!]
\centering
\begin{tabular}{@{}p{0.25\textwidth}p{0.55\textwidth}l@{}}
\toprule
\textbf{Risk Title} & \textbf{Description} & \textbf{Severity} \\
\midrule
\textbf{Exposed Sensitive Database Interface} & An open service on port 8080 on host \texttt{10.5.5.5} is titled "TOP SECRET DB". This suggests a critical system is exposed. This risk is amplified by the lack of MFA on sensitive systems. & \textbf{Critical} \\
\addlinespace
\textbf{Lack of MFA on Sensitive Systems} & The organization does not enforce MFA for sensitive data systems. This removes a critical layer of defense against credential theft and unauthorized access. & \textbf{Critical} \\
\addlinespace
\textbf{Inadequate Security Awareness Training} & Annual security training is not provided to all employees. This leads to a decline in security vigilance and increases susceptibility to phishing and social engineering attacks. & \textbf{High} \\
\addlinespace
\textbf{Outdated Risk Register} & The current risk register incorrectly lists port 8080 as a secure false positive. This indicates the risk management process is not effective at tracking the current state of the environment. & \textbf{Medium} \\
\bottomrule
\end{tabular}
\caption{Summary of Identified Risks.}
\end{table}

% ===================================================================
\section{Recommendations}
% ===================================================================

The following actions are recommended to mitigate the identified risks. They are prioritized based on severity and potential impact.

\subsection{Immediate Priority (Remediate within 24 hours)}
\begin{enumerate}
    \item \textbf{Investigate and Isolate Port 8080:} Immediately investigate the service running on port \texttt{8080} of host \texttt{10.5.5.5}.
    \begin{itemize}
        \item Identify the system owner and the nature of the data it contains.
        \item Apply a restrictive firewall rule to block all access to this port from untrusted networks. Access should only be permitted from specific, authorized IP addresses.
    \end{itemize}
    \item \textbf{Validate Previous Assessment:} Determine why this port was previously classified as a "secure false positive" to identify and correct the failure in the risk assessment process.
\end{enumerate}

\subsection{High Priority (Remediate within 30 days)}
\begin{enumerate}
    \item \textbf{Enforce MFA on All Sensitive Systems:} Develop and execute a project to deploy mandatory MFA for all applications and systems that store, process, or transmit sensitive data. This is the single most effective control to mitigate the risk of unauthorized access.
    \item \textbf{Implement Annual Security Training:} Procure and roll out a mandatory annual security awareness training program for all employees. The training should cover phishing, password hygiene, and acceptable use policies.
\end{enumerate}

\subsection{Medium Priority (Remediate within 90 days)}
\begin{enumerate}
    \item \textbf{Conduct a Full Risk Register Review:} Perform a comprehensive audit of the entire risk register to identify and correct other outdated or inaccurate entries.
    \item \textbf{Enhance Network Monitoring:} Implement a more frequent or continuous vulnerability scanning schedule for all internal and external network assets to ensure that new exposures are detected and remediated in a timely manner.
\end{enumerate}

\end{document}
```