```latex
\documentclass[12pt]{article}

% Preamble: Required Packages
\usepackage[margin=1in]{geometry}
\usepackage{pifont} % For check and cross marks
\usepackage{booktabs} % For professional tables
\usepackage{hyperref} % For clickable links
\usepackage{url} % For formatting URLs
\usepackage{seqsplit} % For splitting long strings in tt font
\usepackage{graphicx}
\usepackage{xcolor}
\usepackage{datetime}

% Document Metadata
\title{Cybersecurity Posture Assessment Report}
\author{Cybersecurity Analysis Division}
\date{\today}

% Hyperref Setup
\hypersetup{
    colorlinks=true,
    linkcolor=blue,
    filecolor=magenta,      
    urlcolor=cyan,
    pdftitle={Cybersecurity Posture Assessment Report},
    pdfpagemode=FullScreen,
}

% Custom Commands
\newcommand{\yes}{\ding{51}}
\newcommand{\no}{\ding{55}}
\newcommand{\critical}[1]{\textcolor{red}{\textbf{#1}}}
\newcommand{\high}[1]{\textcolor{orange}{\textbf{#1}}}
\newcommand{\medium}[1]{\textcolor{yellow!80!black}{\textbf{#1}}}

\begin{document}

\maketitle
\thispagestyle{empty}
\newpage

\tableofcontents
\newpage

\section{Executive Summary}

This report provides a comprehensive cybersecurity posture assessment for \textbf{White Label}. The analysis is based on a correlation of network scan data, a security controls questionnaire, and a review of pre-existing risk registers.

The assessment identified several critical and high-risk vulnerabilities. Most notably, there is a lack of mandatory Multi-Factor Authentication (MFA) for sensitive data systems, which exposes the organization's most valuable assets to unauthorized access. Furthermore, the complete absence of a security awareness training program for both new and existing employees presents a significant organizational risk, making personnel highly susceptible to social engineering and phishing attacks.

Technical analysis confirmed an exposed SSH service on a local interface, which aligns with a pre-existing critical risk. Immediate and decisive action is required to remediate these findings. Recommendations outlined in this report focus on implementing foundational security controls, hardening systems, and establishing a culture of security awareness to mitigate these threats and improve the organization's overall defensive posture.

\section{Organizational Information}

The following information was provided for the assessment.

\begin{table}[h!]
\centering
\begin{tabular}{@{}ll@{}}
\toprule
\textbf{Attribute} & \textbf{Value} \\ \midrule
Organization Name & \textbf{White Label} \\
Email Domain & \texttt{WhiteLabel.com} \\
Website Domain & \seqsplit{\texttt{www.WhiteLabel.com}} \\
External IP Address & \texttt{27.220.225.151} \\ \bottomrule
\end{tabular}
\caption{Client Organizational Details}
\label{tab:org_info}
\end{table}

\section{Security Control Review}

A review of the organization's security controls was conducted via a questionnaire. The responses indicate significant gaps in security policies and procedures. The table below summarizes the findings, with non-compliant answers highlighted as identified gaps.

\begin{table}[h!]
\centering
\begin{tabular}{@{}lcc@{}}
\toprule
\textbf{Security Control Question} & \textbf{Response} & \textbf{Status} \\ \midrule
Do you require MFA to access email? & \yes & Compliant \\
Do you require MFA to log into computers? & \yes & Compliant \\
Do you require MFA to access sensitive data systems? & \no & \critical{Critical Gap} \\
Does your organization have an employee acceptable use policy? & \yes & Compliant \\
Does your organization do security awareness training for new employees? & \no & \high{High Risk Gap} \\
Does your organization do security awareness training for all employees annually? & \no & \high{High Risk Gap} \\ \bottomrule
\end{tabular}
\caption{Security Controls Questionnaire Results}
\label{tab:controls}
\end{table}

\section{Technical Scan Results}

An external network scan was performed to identify exposed services and potential vulnerabilities.

\subsection{Scan Metadata}
\begin{itemize}
    \item \textbf{Target IP:} \texttt{127.0.0.1}
    \item \textbf{Scan Date:} Not provided in scan data. Report generated on \today.
\end{itemize}

\subsection{Open Ports}
The scan identified the following open port(s) on the target system. An open port indicates a listening service that may be accessible from the network, representing a potential attack vector.

\begin{table}[h!]
\centering
\begin{tabular}{@{}lllll@{}}
\toprule
\textbf{Port} & \textbf{Protocol} & \textbf{State} & \textbf{Service} & \textbf{Version} \\ \midrule
22 & TCP & open & ssh & \textit{Not Detected} \\ \bottomrule
\end{tabular}
\caption{Open Ports Detected on \texttt{127.0.0.1}}
\label{tab:ports}
\end{table}

\section{Consolidated Risk Assessment}

This section synthesizes findings from the security control review, technical scans, and pre-existing risk data into a consolidated list of identified risks. Each risk is assigned a severity level based on its potential impact on the organization.

\begin{table}[h!]
\centering
\begin{tabular}{@{}p{0.3\linewidth}p{0.5\linewidth}p{0.15\linewidth}@{}}
\toprule
\textbf{Risk Name} & \textbf{Overview} & \textbf{Severity} \\ \midrule
\textbf{Lack of MFA on Sensitive Systems} & The absence of MFA on systems housing sensitive data allows an attacker with stolen credentials to gain direct access to critical assets, bypassing primary authentication controls. & \critical{Critical} \\
\addlinespace
\textbf{Exposed Localhost Service} & An SSH service is accessible on the local loopback interface (\texttt{127.0.0.1}). This confirms a pre-existing risk and indicates a potential misconfiguration that could be exploited by local processes or in a chained attack. & \critical{Critical} \\
\addlinespace
\textbf{No Security Awareness Training} & The organization lacks a formal security training program. This leaves employees unable to recognize and respond to common threats like phishing, significantly increasing the risk of human-error-related breaches. & \high{High} \\
\bottomrule
\end{tabular}
\caption{Summary of Identified Risks}
\label{tab:risks}
\end{table}

\section{Recommendations}

The following actions are recommended to mitigate the identified risks and strengthen the security posture of \textbf{White Label}.

\subsection{Critical Priority Recommendations}

\begin{enumerate}
    \item \textbf{Implement MFA for All Sensitive Systems:}
    \begin{itemize}
        \item \textbf{Action:} Immediately deploy and enforce a mandatory MFA policy for all user accounts (including administrative and service accounts) that have access to systems containing sensitive or critical data.
        \item \textbf{Impact:} Drastically reduces the risk of unauthorized access via compromised credentials.
    \end{itemize}
    \item \textbf{Remediate Exposed SSH Service:}
    \begin{itemize}
        \item \textbf{Action:} Investigate the SSH service running on port 22 of \texttt{127.0.0.1}. If this service is not essential for business operations, it should be disabled. If it is required, ensure it is properly configured, firewalled, and not intended to be exposed.
        \item \textbf{Impact:} Eliminates a direct attack vector and corrects a critical system misconfiguration.
    \end{itemize}
\end{enumerate}

\subsection{High Priority Recommendations}

\begin{enumerate}
    \setcounter{enumi}{2} % Continue numbering from previous list
    \item \textbf{Establish a Security Awareness Training Program:}
    \begin{itemize}
        \item \textbf{Action:} Develop and implement a comprehensive security awareness training program. This must include mandatory onboarding training for all new employees and annual refresher training for all staff. Topics should cover phishing, password hygiene, acceptable use, and incident reporting.
        \item \textbf{Impact:} Creates a "human firewall" by empowering employees to act as the first line of defense, reducing susceptibility to social engineering attacks.
    \end{itemize}
\end{enumerate}

\end{document}
```