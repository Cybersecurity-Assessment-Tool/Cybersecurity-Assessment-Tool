```latex
\documentclass[12pt]{article}

% Preamble: Required Packages and Document Setup
\usepackage[margin=1in]{geometry}
\usepackage{pifont} % For checkmarks and crosses
\usepackage{booktabs} % For professional tables
\usepackage{hyperref} % For clickable links
\usepackage{url} % For formatting URLs
\usepackage{seqsplit} % To split long strings without breaking
\usepackage{graphicx}
\usepackage{xcolor}
\usepackage{titling}

% --- Document Metadata ---
\title{Cybersecurity Posture Assessment Report \\ \large For: Pioneer Pulse}
\author{Cybersecurity Analyst Group}
\date{\today}

% --- Custom Commands & Colors ---
\newcommand{\yes}{\ding{51}}
\newcommand{\no}{\ding{55}}
\definecolor{critical}{HTML}{990000}
\definecolor{high}{HTML}{D1410C}
\definecolor{medium}{HTML}{E5A800}
\definecolor{low}{HTML}{3E8E41}

\hypersetup{
    colorlinks=true,
    linkcolor=blue,
    filecolor=magenta,      
    urlcolor=cyan,
    pdftitle={Cybersecurity Posture Assessment Report},
    pdfpagemode=FullScreen,
}

\begin{document}

\maketitle
\thispagestyle{empty}
\newpage
\tableofcontents
\newpage

% ===================================================================
% SECTION 1: EXECUTIVE SUMMARY
% ===================================================================
\section{Executive Summary}

This report provides a cybersecurity posture assessment for \textbf{Pioneer Pulse}, conducted on \today. The analysis is based on a network scan, a review of organizational security controls, and pre-existing risk data.

The assessment reveals several critical and high-risk vulnerabilities that require immediate attention. The most significant findings include:

\begin{itemize}
    \item \textbf{Systemic Remote Desktop Protocol (RDP) Exposure:} The network scan identified an open RDP port (3389) on host \texttt{10.10.10.51}. This finding, correlated with a pre-existing risk on another host (\texttt{10.10.10.50}), indicates a pattern of insecure remote access configurations. Exposed RDP is a primary vector for ransomware attacks.
    
    \item \textbf{Critical Gaps in Access Control:} The organization does not enforce Multi-Factor Authentication (MFA) for email or local computer access. This significantly increases the risk of account compromise and unauthorized access to sensitive information.
    
    \item \textbf{Inadequate Security Training:} While new employees receive security training, the lack of a mandatory annual refresher course for all staff leaves the organization vulnerable to evolving social engineering and phishing tactics.
\end{itemize}

The combination of these vulnerabilities places the organization at a high risk of a significant security incident. This report outlines these findings in detail and provides a prioritized list of actionable recommendations to mitigate the identified risks and strengthen the overall security posture.

% ===================================================================
% SECTION 2: ORGANIZATIONAL INFORMATION
% ===================================================================
\section{Organizational Information}

The following details were provided for the assessment. This information is used to establish the context and scope of the review.

\begin{tabular}{@{}ll}
    \toprule
    \textbf{Attribute} & \textbf{Value} \\
    \midrule
    Organization Name & Pioneer Pulse \\
    Email Domain & \texttt{PioneerPulse.org} \\
    Website Domain & \url{www.PioneerPulse.org} \\
    External IP Address & \texttt{139.30.228.26} \\
    \bottomrule
\end{tabular}

% ===================================================================
% SECTION 3: SECURITY CONTROL REVIEW
% ===================================================================
\section{Security Control Review (Questionnaire)}

A review of the organization's security controls was conducted via a questionnaire. The responses highlight critical gaps in identity and access management and employee security awareness.

\begin{tabular}{@{}p{0.6\linewidth}cp{0.25\linewidth}@{}}
    \toprule
    \textbf{Control Question} & \textbf{Response} & \textbf{Status / Comment} \\
    \midrule
    Do you require MFA to access email? & \no & \textcolor{critical}{\textbf{Critical Gap}} \\
    Do you require MFA to log into computers? & \no & \textcolor{high}{\textbf{High Risk}} \\
    Do you require MFA to access sensitive data systems? & \yes & Good Practice \\
    Does your organization have an employee acceptable use policy? & \yes & Good Practice \\
    Does your organization do security awareness training for new employees? & \yes & Good Foundation \\
    Does your organization do security awareness training for all employees at least once per year? & \no & \textcolor{high}{\textbf{High Risk}} \\
    \bottomrule
\end{tabular}

% ===================================================================
% SECTION 4: TECHNICAL SCAN RESULTS
% ===================================================================
\section{Technical Scan Results}

A network scan was performed to identify active services on the target system.

\subsection{Nmap Scan: \texttt{10.10.10.51}}
The scan revealed one open port on the target host. The details are as follows:

\begin{tabular}{@{}llll@{}}
    \toprule
    \textbf{Port} & \textbf{State} & \textbf{Service} & \textbf{Analysis} \\
    \midrule
    3389/tcp & Open & \texttt{ms-wbt-server} & This is the default port for Microsoft Remote \\
    & & & Desktop Protocol (RDP). Exposing RDP \\
    & & & directly to a network is a severe security risk. \\
    \bottomrule
\end{tabular}

\paragraph{Finding:} The presence of an open RDP port on \texttt{10.10.10.51} confirms a pattern of insecure remote access, as a similar issue was previously identified on host \texttt{10.10.10.50}. This service is a frequent target for brute-force attacks and exploitation by ransomware groups.

% ===================================================================
% SECTION 5: CONSOLIDATED RISK ASSESSMENT
% ===================================================================
\section{Consolidated Risk Assessment}

The following table synthesizes findings from the security control review, the technical scan, and pre-existing risk data into a consolidated list of security risks.

\begin{tabular}{@{}p{0.2\linewidth}p{0.4\linewidth}p{0.15\linewidth}p{0.15\linewidth}@{}}
    \toprule
    \textbf{Risk Name} & \textbf{Description} & \textbf{Severity} & \textbf{Affected Systems} \\
    \midrule
    \textbf{Systemic RDP Exposure} & The RDP service (port 3389) is exposed on multiple internal systems, creating a direct path for attackers to gain remote control. & \textcolor{critical}{\textbf{Critical}} & \texttt{10.10.10.50}, \texttt{10.10.10.51}, and potentially others. \\
    \addlinespace
    \textbf{No MFA for Email} & Lack of MFA on email accounts makes them highly susceptible to compromise via phishing or credential stuffing, leading to Business Email Compromise (BEC). & \textcolor{critical}{\textbf{Critical}} & All employee email accounts. \\
    \addlinespace
    \textbf{No MFA for Computers} & Lack of MFA for computer logins allows an attacker with stolen credentials to easily access workstations and move laterally within the network. & \textcolor{high}{\textbf{High}} & All workstations and servers. \\
    \addlinespace
    \textbf{Inadequate Security Awareness Training} & Without annual refresher training, employees are less likely to recognize and report modern phishing and social engineering attempts. & \textcolor{high}{\textbf{High}} & All employees. \\
    \bottomrule
\end{tabular}

% ===================================================================
% SECTION 6: RECOMMENDATIONS
% ===================================================================
\section{Recommendations}

The following actions are recommended to mitigate the identified risks. They are prioritized based on severity and potential impact.

\subsection{Immediate Priority (Critical Risks)}
\begin{enumerate}
    \item \textbf{Remediate RDP Exposure:}
    \begin{itemize}
        \item Immediately close port 3389 on hosts \texttt{10.10.10.51}, \texttt{10.10.10.50}, and any other systems where it is not strictly required and firewalled.
        \item Conduct a comprehensive internal network scan to identify all instances of exposed RDP.
        \item For necessary remote access, implement a secure solution such as a Virtual Private Network (VPN) with MFA.
    \end{itemize}

    \item \textbf{Implement MFA for Email:}
    \begin{itemize}
        \item Enable MFA for all user accounts on the \texttt{PioneerPulse.org} email platform immediately.
        \item This is the single most effective control to prevent unauthorized access to email.
    \end{itemize}
\end{enumerate}

\subsection{High Priority Recommendations}
\begin{enumerate}
    \setcounter{enumi}{2} % Continue numbering
    \item \textbf{Deploy MFA for Computer Logins:}
    \begin{itemize}
        \item Implement an MFA solution for all employee logins to company workstations and servers. This adds a crucial layer of defense against credential theft.
    \end{itemize}
    
    \item \textbf{Establish Annual Security Training:}
    \begin{itemize}
        \item Institute a mandatory security awareness training program for all employees, to be completed at least once per year.
        \item Training should cover phishing, password security, and incident reporting procedures.
    \end{itemize}
\end{enumerate}

\end{document}
```