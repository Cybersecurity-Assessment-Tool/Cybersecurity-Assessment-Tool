```latex
\documentclass[12pt]{article}

% Preamble: Required Packages
\usepackage[a4paper, margin=1in]{geometry}
\usepackage{pifont} % For checkmarks and crosses
\usepackage{booktabs} % For professional tables
\usepackage{hyperref} % For clickable links
\usepackage{url} % For URL formatting
\usepackage{seqsplit} % To split long strings in texttt
\usepackage{graphicx}
\usepackage{xcolor}
\usepackage{fancyhdr}

% Document Metadata and Styling
\hypersetup{
    colorlinks=true,
    linkcolor=blue,
    filecolor=magenta,      
    urlcolor=cyan,
    pdftitle={Cybersecurity Posture Assessment},
    pdfpagemode=FullScreen,
}

\pagestyle{fancy}
\fancyhf{}
\lhead{Cybersecurity Post-Assessment Report}
\rhead{Opal Sky Media}
\cfoot{\thepage}

\begin{document}

% --- Title Page ---
\begin{titlepage}
    \centering
    \vspace*{1cm}
    \Huge\textbf{Cybersecurity Posture Assessment Report}
    \vspace{1.5cm}
    \vfill
    \large
    \textbf{Prepared For:}\\
    Opal Sky Media
    \vspace{2cm}
    
    \textbf{Date of Report:}\\
    \today
    
\end{titlepage}

\tableofcontents
\newpage

% --- Section 1: Executive Summary ---
\section{Executive Summary}
This report provides a comprehensive analysis of the cybersecurity posture of Opal Sky Media, based on network scans, a security controls questionnaire, and a review of existing risk documentation. The assessment was conducted on \today.

The analysis revealed several \textbf{critical-risk findings} that require immediate attention. A network scan of the internal host \texttt{10.5.5.5} identified an open port (8080) exposing a service with the title \textbf{"TOP SECRET DB"}. This finding directly contradicts the existing risk documentation, which incorrectly classifies this port as a secure false positive. This discrepancy indicates a significant failure in the current risk management process.

Furthermore, organizational data reveals critical gaps in security controls. While Multi-Factor Authentication (MFA) is used for email and computer logins, it is \textbf{not enforced for sensitive data systems}. This gap, combined with the exposed database, creates a direct and severe pathway for a potential data breach. A high-risk gap was also identified in the employee onboarding process, which lacks mandatory initial security awareness training.

Immediate remediation is required to secure the exposed database, implement MFA across all sensitive systems, and correct the flawed risk register. Detailed findings and actionable recommendations are provided in the subsequent sections of this report.

% --- Section 2: Organizational Information ---
\section{Organizational Information}
The following information was provided for the assessment. This data is used to establish the context and scope of the review.

\begin{table}[h!]
\centering
\caption{Client Organizational Data}
\begin{tabular}{@{}ll@{}}
\toprule
\textbf{Attribute} & \textbf{Value} \\ \midrule
Organization Name & Opal Sky Media \\
Primary Email Domain & \texttt{OpalSkyMedia.com} \\
Primary Website & \url{www.OpalSkyMedia.com} \\
External IP Address & \texttt{105.206.45.225} \\ \bottomrule
\end{tabular}
\end{table}

% --- Section 3: Security Control Review ---
\section{Security Control Review}
A review of key security controls was conducted via a questionnaire. The results highlight the organization's current policies and practices. Gaps identified in this section are correlated with technical findings to determine overall risk.

\begin{table}[h!]
\centering
\caption{Security Controls Questionnaire Results}
\begin{tabular}{@{}lc@{}}
\toprule
\textbf{Control Question} & \textbf{Status} \\ \midrule
Do you require MFA to access email? & \ding{51} \\
Do you require MFA to log into computers? & \ding{51} \\
\textbf{Do you require MFA to access sensitive data systems?} & \textcolor{red}{\ding{55}} \\
Does your organization have an employee acceptable use policy? & \ding{51} \\
\textbf{Does your organization do security awareness training for new employees?} & \textcolor{red}{\ding{55}} \\
Does your organization do security awareness training for all employees annually? & \ding{51} \\ \bottomrule
\end{tabular}
\end{table}

\subsection*{Analysis of Control Gaps}
Two significant control gaps were identified:
\begin{itemize}
    \item \textbf{Critical Gap:} The absence of MFA on sensitive data systems is a critical vulnerability. Should an attacker compromise employee credentials, there is no secondary control to prevent access to the organization's most valuable data.
    \item \textbf{High-Risk Gap:} The lack of security awareness training during employee onboarding leaves the organization vulnerable. New employees are often prime targets for phishing and social engineering attacks, and this gap increases the likelihood of an initial compromise.
\end{itemize}

% --- Section 4: Technical Scan Results ---
\section{Technical Scan Results}
An Nmap scan was performed to identify open ports and services on the specified target.

\subsection*{Scan Details}
\begin{itemize}
    \item \textbf{Target IP Address:} \texttt{10.5.5.5}
    \item \textbf{Scan Date:} \today
\end{itemize}

\subsection*{Open Ports and Services}
The following table details the open ports and services discovered on the target host.

\begin{table}[h!]
\centering
\caption{Open Ports on \texttt{10.5.5.5}}
\begin{tabular}{@{}llll@{}}
\toprule
\textbf{Port} & \textbf{State} & \textbf{Service} & \textbf{Details} \\ \midrule
8080/tcp & open & http & HTTP Title: \textbf{TOP SECRET DB} \\ \bottomrule
\end{tabular}
\end{table}

\subsection*{Analysis of Technical Findings}
The scan revealed a single open port, 8080, running an HTTP service. The title of the web page, \textbf{"TOP SECRET DB"}, is an alarming indicator of a potentially exposed sensitive database or management interface. This finding is of \textbf{critical severity} for the following reasons:
\begin{enumerate}
    \item \textbf{Data Exposure:} The service name strongly suggests that sensitive, confidential, or proprietary data is accessible via this port.
    \item \textbf{Contradiction with Existing Risk Data:} The pre-existing risk documentation (Input 3) states that port 8080 is "confirmed secure and false positive." Our active scan proves this assessment is dangerously incorrect. This points to a systemic issue in the organization's vulnerability management and risk assessment processes.
    \item \textbf{Attack Vector:} An exposed database interface is a primary target for attackers seeking to exfiltrate data or establish a persistent foothold in the network.
\end{enumerate}

% --- Section 5: Consolidated Risk Assessment ---
\section{Consolidated Risk Assessment}
This section synthesizes the findings from the security control review and the technical scan to provide a consolidated view of the primary risks facing the organization.

\begin{table}[h!]
\centering
\caption{Summary of Identified Risks}
\begin{tabular}{@{}p{0.3\linewidth}p{0.15\linewidth}p{0.45\linewidth}@{}}
\toprule
\textbf{Risk Name} & \textbf{Severity} & \textbf{Description} \\ \midrule
\textbf{Exposed Sensitive Database Interface} & \textbf{Critical} & Port 8080 on host \texttt{10.5.5.5} exposes a service titled "TOP SECRET DB". This presents an immediate and severe risk of a data breach. \\
\addlinespace
\textbf{Lack of MFA on Sensitive Systems} & \textbf{Critical} & The absence of MFA on sensitive systems, combined with the exposed database, means a single compromised credential could lead to catastrophic data loss. \\
\addlinespace
\textbf{Inaccurate Risk Register} & \textbf{High} & The existing risk register incorrectly identifies port 8080 as a non-threat. This indicates the risk management program is unreliable and may be blind to other severe vulnerabilities. \\
\addlinespace
\textbf{Inadequate Employee Onboarding Security} & \textbf{High} & New employees are not provided with security training, making them susceptible to attacks that could lead to credential compromise and exploitation of the other identified risks. \\ \bottomrule
\end{tabular}
\end{table}

% --- Section 6: Recommendations ---
\section{Recommendations}
The following actions are recommended to mitigate the identified risks. They are prioritized based on severity and potential impact.

\subsection*{Priority 1: Immediate Actions (Due within 48 hours)}
\begin{enumerate}
    \item \textbf{Isolate and Secure Exposed Service:} Immediately investigate the service running on \texttt{10.5.5.5:8080}. If it is a database, restrict access to only authorized personnel and systems via strict firewall rules. The service should not be exposed to the general internal network.
    \item \textbf{Initiate Incident Response:} Validate the nature of the "TOP SECRET DB" and determine if an unauthorized access or data breach has already occurred. Review access logs immediately.
\end{enumerate}

\subsection*{Priority 2: High-Priority Actions (Due within 30 days)}
\begin{enumerate}
    \item \textbf{Implement MFA on All Sensitive Systems:} Enforce mandatory MFA for all systems classified as containing sensitive or critical data. This is the single most effective control to mitigate the risk of credential compromise.
    \item \textbf{Develop and Mandate Onboarding Security Training:} Create a mandatory security awareness training module for all new employees, to be completed during their first week of employment. This training should cover phishing, acceptable use, and password hygiene.
\end{enumerate}

\subsection*{Priority 3: Foundational Improvements (Due within 90 days)}
\begin{enumerate}
    \item \textbf{Overhaul the Risk Management Program:} Conduct a full review and update of the organization's risk register. All documented risks must be re-validated with active technical scanning. Implement a continuous monitoring process to ensure the register remains accurate.
    \item \textbf{Conduct a Comprehensive Vulnerability Assessment:} Perform a full internal and external vulnerability scan of the entire network infrastructure to identify any other unknown or undocumented risks.
\end{enumerate}

\end{document}
```