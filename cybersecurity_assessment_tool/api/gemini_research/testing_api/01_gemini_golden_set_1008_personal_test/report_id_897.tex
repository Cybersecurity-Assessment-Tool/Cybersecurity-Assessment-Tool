```latex
\documentclass[12pt]{article}

% --- PACKAGES ---
\usepackage[margin=1in]{geometry}
\usepackage{pifont} % For checkmarks and crosses
\usepackage{booktabs} % For professional tables
\usepackage{hyperref} % For clickable links
\usepackage{url} % For URL formatting
\usepackage{seqsplit} % For splitting long strings
\usepackage{graphicx}
\usepackage{xcolor}

% --- DOCUMENT SETUP ---
\hypersetup{
    colorlinks=true,
    linkcolor=blue,
    filecolor=magenta,      
    urlcolor=cyan,
    pdftitle={Cybersecurity Assessment Report},
    pdfpagemode=FullScreen,
}

\newcommand{\yes}{\ding{51}}
\newcommand{\no}{\ding{55}}

% --- TITLE ---
\title{Cybersecurity Assessment Report}
\author{Cybersecurity Analysis Division}
\date{\today}

\begin{document}

\maketitle
\thispagestyle{empty}
\newpage
\tableofcontents
\thispagestyle{empty}
\newpage
\setcounter{page}{1}

% --- EXECUTIVE OVERVIEW ---
\section{Executive Overview}
This report provides a comprehensive cybersecurity assessment for \textbf{Modern Myth}, based on a correlation of network scan data, organizational security controls, and a review of the current risk register.

The analysis reveals a mixed security posture. While the organization has implemented some essential controls, such as requiring Multi-Factor Authentication (MFA) for email and sensitive systems, several critical deficiencies were identified that pose a significant risk to the organization's data and operations.

\textbf{Key Findings:}
\begin{itemize}
    \item \textbf{Critical Risk - Exposed Sensitive Service:} A network scan of the internal host \texttt{10.5.5.5} discovered an open port (8080) with a service title of \textbf{"TOP SECRET DB"}. This finding directly contradicts the current risk register, which incorrectly lists this port as a secure false positive. This represents an immediate and severe threat of a data breach.
    \item \textbf{High Risk - Lack of Endpoint MFA:} The absence of mandatory MFA for computer logins exposes the organization to significant risk from credential theft, which could lead to unauthorized access and lateral movement within the network.
    \item \textbf{High Risk - Inadequate Security Training:} The lack of annual security awareness training for all employees increases the organization's susceptibility to social engineering and phishing attacks, which are primary vectors for initial compromise.
\end{itemize}

Immediate remediation of the exposed database service is paramount. Furthermore, addressing the identified gaps in endpoint security and employee training is crucial to improving the overall security resilience of \textbf{Modern Myth}.

% --- ORGANIZATIONAL INFORMATION ---
\section{Organizational Information}
The following details were provided for the assessment.
\begin{itemize}
    \item \textbf{Organization Name:} Modern Myth
    \item \textbf{Email Domain:} \texttt{ModernMyth.org}
    \item \textbf{Website Domain:} \url{www.ModernMyth.org}
    \item \textbf{External IP Address:} \texttt{26.74.62.149}
\end{itemize}

% --- SECURITY CONTROL REVIEW ---
\section{Security Control Review}
An analysis of the organization's security questionnaire highlights key areas of strength and weakness in existing administrative controls. "No" answers indicate significant gaps that require attention.

\begin{table}[h!]
\centering
\caption{Security Controls Questionnaire Analysis}
\begin{tabular}{p{0.6\linewidth} c l}
\toprule
\textbf{Control Question} & \textbf{Response} & \textbf{Assessment} \\
\midrule
Do you require MFA to access email? & \yes & Satisfactory \\
Do you require MFA to log into computers? & \no & \textbf{High Risk Gap} \\
Do you require MFA to access sensitive data systems? & \yes & Satisfactory \\
Does your organization have an employee acceptable use policy? & \yes & Satisfactory \\
Does your organization do security awareness training for new employees? & \yes & Satisfactory \\
Does your organization do security awareness training for all employees at least once per year? & \no & \textbf{High Risk Gap} \\
\bottomrule
\end{tabular}
\end{table}

% --- TECHNICAL SCAN RESULTS ---
\section{Technical Scan Results}
A network scan was performed on the internal network to identify active services and potential exposures.

\begin{itemize}
    \item \textbf{Target IP Address:} \texttt{10.5.5.5}
    \item \textbf{Host Status:} Up
\end{itemize}

\begin{table}[h!]
\centering
\caption{Open Port Analysis for \texttt{10.5.5.5}}
\begin{tabular}{l l p{0.5\linewidth}}
\toprule
\textbf{Port} & \textbf{State} & \textbf{Service Information} \\
\midrule
8080/tcp & Open & HTTP Service Title: \textbf{TOP SECRET DB} \\
\bottomrule
\end{tabular}
\end{table}

\subsection{Analysis of Technical Findings}
The scan identified a critically exposed service on port 8080. The service's title, "TOP SECRET DB," strongly suggests that a highly sensitive database is accessible over the network without adequate protection. 

This finding is of utmost concern because the existing risk register (from Input 3) explicitly states that port 8080 is "confirmed secure and false positive." \textbf{This indicates a severe failure in the risk assessment process.} The previous assessment was incorrect, and the organization has been operating with a significant, unmitigated vulnerability.

% --- CONSOLIDATED RISK ASSESSMENT ---
\section{Consolidated Risk Assessment}
The following table synthesizes findings from the technical scan, control review, and existing risk data into a prioritized list of risks.

\begin{table}[h!]
\centering
\caption{Summary of Identified Risks}
\begin{tabular}{p{0.1\linewidth} p{0.25\linewidth} p{0.45\linewidth} l}
\toprule
\textbf{ID} & \textbf{Risk Name} & \textbf{Description} & \textbf{Severity} \\
\midrule
RISK-001 & Exposed Sensitive Database Interface & Port 8080 is open on host \texttt{10.5.5.5}, exposing a service titled "TOP SECRET DB". This contradicts the current risk register and poses a direct threat of a major data breach. & \textbf{Critical} \\
\addlinespace
RISK-002 & Lack of Endpoint MFA & Workstations do not require MFA for login. A compromised password could grant an attacker full access to an employee's computer and network resources. & \textbf{High} \\
\addlinespace
RISK-003 & Inadequate Security Awareness Training & The absence of mandatory annual training for all employees increases the likelihood of a successful phishing or social engineering attack. & \textbf{High} \\
\bottomrule
\end{tabular}
\end{table}

% --- RECOMMENDATIONS ---
\section{Recommendations}
The following actions are recommended to mitigate the identified risks and improve the overall security posture of \textbf{Modern Myth}.

\subsection{RISK-001: Remediate Exposed Database (Critical Priority)}
\begin{itemize}
    \item \textbf{Immediate Action (0-24 hours):}
    \begin{enumerate}
        \item Immediately investigate the service running on \texttt{10.5.5.5:8080}.
        \item If the service is confirmed to be a sensitive database, apply network firewall rules to block all access immediately, or shut down the service.
        \item Launch an investigation to determine if the service has been accessed by unauthorized parties.
    \end{enumerate}
    \item \textbf{Long-Term Action:}
    \begin{enumerate}
        \item Ensure that all systems hosting sensitive data are located on properly segmented networks with strict, default-deny firewall policies.
        \item Conduct a comprehensive review of the risk register and asset inventory to identify and correct other potential inaccuracies.
    \end{enumerate}
\end{itemize}

\subsection{RISK-002: Implement Endpoint MFA (High Priority)}
\begin{itemize}
    \item \textbf{Short-Term Action (1-3 weeks):}
    \begin{enumerate}
        \item Develop a project plan and timeline for deploying MFA for all workstation and laptop logins (e.g., via Windows Hello for Business, Duo, or a similar solution).
        \item Begin a pilot program with IT staff and a small group of users.
    \end{enumerate}
    \item \textbf{Long-Term Action:}
    \begin{enumerate}
        \item Enforce MFA for all computer logins across the entire organization as a mandatory security policy.
    \end{enumerate}
\end{itemize}

\subsection{RISK-003: Establish Annual Security Training (High Priority)}
\begin{itemize}
    \item \textbf{Short-Term Action (1-4 weeks):}
    \begin{enumerate}
        \item Procure and schedule a security awareness training module for all current employees.
        \item Communicate the importance of this training to all staff.
    \end{enumerate}
    \item \textbf{Long-Term Action:}
    \begin{enumerate}
        \item Formalize a policy requiring all employees to complete security awareness training on an annual basis.
        \item Integrate phishing simulation exercises into the training program to measure effectiveness.
    \end{enumerate}
\end{itemize}

\end{document}
```