```latex
\documentclass[12pt]{article}

% Preamble: Required Packages
\usepackage[margin=1in]{geometry}
\usepackage{pifont} % For checkmarks and crosses
\usepackage{booktabs} % For professional-looking tables
\usepackage{xcolor} % For colors
\usepackage{hyperref} % For hyperlinks
\usepackage{url} % For URL formatting
\usepackage{seqsplit} % For splitting long strings in tt font

% Document Metadata
\title{Cybersecurity Posture Assessment Report}
\author{Cybersecurity Analysis Division}
\date{\today}

% Hyperref Setup
\hypersetup{
    colorlinks=true,
    linkcolor=blue,
    filecolor=magenta,      
    urlcolor=cyan,
    pdftitle={Cybersecurity Posture Assessment Report},
    pdfpagemode=FullScreen,
}

\begin{document}

\maketitle
\thispagestyle{empty}
\newpage

\tableofcontents
\newpage

% --- 1. Executive Summary ---
\section{Executive Summary}

This report provides a synthesized analysis of the cybersecurity posture for \textbf{Hidden Gem}, based on a recent network scan, a security controls questionnaire, and a review of pre-existing risk documentation.

The overall assessment indicates a maturing security program with several positive controls in place, including mandatory Multi-Factor Authentication (MFA) for email and computer access, and a robust security awareness training program. Furthermore, a previously identified risk concerning an unencrypted web server on port 80 has been successfully \textbf{mitigated}, as confirmed by the recent network scan which shows the port is now closed.

However, a \textbf{critical control gap} was identified: the absence of mandatory MFA for accessing sensitive data systems. This represents a significant risk to data confidentiality and integrity, as it exposes critical assets to potential compromise via stolen credentials.

This report details these findings and provides prioritized, actionable recommendations to address the identified gap and further strengthen the organization's security posture.

% --- 2. Organizational Information ---
\section{Organizational Information}

The following information was provided for the assessment.

\begin{itemize}
    \item \textbf{Organization Name:} Hidden Gem
    \item \textbf{Primary Email Domain:} \texttt{HiddenGem.org}
    \item \textbf{Primary Website Domain:} \texttt{www.HiddenGem.org}
    \item \textbf{External IP Address:} \texttt{235.92.7.79}
\end{itemize}

% --- 3. Security Control Review ---
\section{Security Control Review}

A review of the organization's security controls was conducted via a questionnaire. The responses indicate a strong foundation in user security and policy. However, a critical gap in data protection was noted.

\begin{table}[h!]
\centering
\caption{Security Controls Questionnaire Results}
\begin{tabular}{p{0.75\linewidth} c}
\toprule
\textbf{Control Question} & \textbf{Response} \\
\midrule
Do you require MFA to access email? & \textcolor{green!80!black}{\ding{51}} \\
Do you require MFA to log into computers? & \textcolor{green!80!black}{\ding{51}} \\
\textbf{Do you require MFA to access sensitive data systems?} & \textcolor{red!90!black}{\ding{55}} \\
Does your organization have an employee acceptable use policy? & \textcolor{green!80!black}{\ding{51}} \\
Does your organization do security awareness training for new employees? & \textcolor{green!80!black}{\ding{51}} \\
Does your organization do security awareness training for all employees at least once per year? & \textcolor{green!80!black}{\ding{51}} \\
\bottomrule
\end{tabular}
\end{table}

\subsection*{Analysis}
The lack of MFA for sensitive data systems is a \textbf{High-Risk} finding. While email and endpoint security are strong, the "crown jewels" of the organization are not protected by this essential security control. An attacker with valid (e.g., phished or stolen) credentials could gain direct access to critical data repositories.

% --- 4. Technical Scan Results ---
\section{Technical Scan Results}

An external network scan was performed to identify exposed services and potential vulnerabilities.

\begin{itemize}
    \item \textbf{Target IP Address:} \texttt{192.168.0.5}
    \item \textbf{Scan Tool:} Nmap
\end{itemize}

The scan results are summarized in the table below.

\begin{table}[h!]
\centering
\caption{Port Scan Findings for \texttt{192.168.0.5}}
\begin{tabular}{l l l l}
\toprule
\textbf{Port} & \textbf{State} & \textbf{Service} & \textbf{Product / Version} \\
\midrule
80/tcp & closed & http & N/A \\
\bottomrule
\end{tabular}
\end{table}

\subsection*{Analysis}
The scan results are positive. The target host is responsive, but no open ports were detected. Specifically, the finding that port 80 (HTTP) is \textbf{closed} is a significant security improvement. This directly contradicts a pre-existing risk on the risk register, indicating that remediation has occurred. This prevents unencrypted communication and reduces the attack surface of the host.

% --- 5. Correlated Risk Assessment ---
\section{Correlated Risk Assessment}

This section synthesizes findings from the security control review, the technical scan, and the pre-existing risk register.

\begin{table}[h!]
\centering
\caption{Summary of Identified Risks}
\begin{tabular}{p{0.1\linewidth} p{0.25\linewidth} p{0.35\linewidth} p{0.1\linewidth} p{0.1\linewidth}}
\toprule
\textbf{ID} & \textbf{Risk Name} & \textbf{Description} & \textbf{Severity} & \textbf{Status} \\
\midrule
\textbf{RISK-01} & \textbf{No MFA on Sensitive Systems} & The lack of MFA on critical data systems exposes the organization to data breaches via credential compromise. & \textbf{High} & \textbf{Active} \\
\midrule
RISK-02 & Unencrypted Web Server & The pre-existing risk register noted that port 80 was open, allowing for unencrypted HTTP traffic. & Medium & \textbf{Mitigated} \\
\bottomrule
\end{tabular}
\end{table}

\subsection*{Correlation Details}
\begin{itemize}
    \item \textbf{RISK-01 (Active):} This risk is derived directly from the "No" response in the security control questionnaire. It is the highest priority finding of this assessment.
    \item \textbf{RISK-02 (Mitigated):} The current risk register (Input 3) listed an open port 80 as an active vulnerability. The technical Nmap scan (Input 1) has confirmed that this port is now \textbf{closed}. This demonstrates positive security progress and indicates that the risk register requires updating to reflect this remediation.
\end{itemize}

% --- 6. Recommendations ---
\section{Recommendations}

Based on the correlated risk assessment, the following actions are recommended to enhance the security posture of \textbf{Hidden Gem}.

\subsection*{High Priority}
\begin{enumerate}
    \item \textbf{Implement MFA for All Sensitive Data Systems:}
    \begin{itemize}
        \item \textbf{Action:} Enforce MFA for all user accounts, especially privileged accounts, that have access to databases, file shares, or applications containing sensitive or critical organizational data.
        \item \textbf{Justification:} This is the single most effective control to prevent unauthorized access resulting from credential theft. It provides a critical layer of defense for the organization's most valuable assets.
    \end{itemize}
\end{enumerate}

\subsection*{Informational / Procedural}
\begin{enumerate}
    \setcounter{enumi}{1}
    \item \textbf{Update and Validate the Risk Register:}
    \begin{itemize}
        \item \textbf{Action:} Formally mark the "Unencrypted Web Server" risk (related to port 80) as resolved in the organization's risk register.
        \item \textbf{Justification:} Maintaining an accurate and up-to-date risk register is crucial for effective security management. This ensures that resources are focused on current, active threats and not on issues that have already been fixed. Regular validation of scan results against the register should be a standard operational procedure.
    \end{itemize}
\end{enumerate}

\end{document}
```