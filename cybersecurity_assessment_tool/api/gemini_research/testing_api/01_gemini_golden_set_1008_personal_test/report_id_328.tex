```latex
\documentclass[12pt, a4paper]{article}

% Preamble: Required Packages
\usepackage[margin=1in]{geometry}
\usepackage{pifont} % For checkmarks and crosses
\usepackage{booktabs} % For professional tables
\usepackage{hyperref} % For clickable links
\usepackage{url} % For formatting URLs
\usepackage{seqsplit} % For splitting long strings in tt font
\usepackage{graphicx}
\usepackage{xcolor}

% Document Metadata
\title{Cybersecurity Posture Assessment Report}
\author{Cybersecurity Analysis Division}
\date{\today}

% Hyperref Setup
\hypersetup{
    colorlinks=true,
    linkcolor=black,
    urlcolor=blue,
    pdftitle={Cybersecurity Posture Assessment Report},
    pdfauthor={Cybersecurity Analysis Division},
    pdfsubject={Security Assessment},
    pdfkeywords={Cybersecurity, Nmap, Risk, Analysis}
}

\begin{document}

\maketitle
\thispagestyle{empty}
\newpage

\tableofcontents
\thispagestyle{empty}
\newpage

\setcounter{page}{1}

% --- SECTION 1: EXECUTIVE SUMMARY ---
\section{Executive Summary}
This report details the findings of a cybersecurity posture assessment conducted for \textbf{Modern Myth}. The evaluation combined a technical network scan, a review of existing risks, and an analysis of organizational security controls based on a questionnaire.

The assessment revealed a mixed security posture. On a technical level, the organization demonstrates a strong perimeter defense. The external network scan of the target host \texttt{192.168.1.100} found \textbf{no open ports}, indicating a robust and well-configured firewall. This significantly reduces the external attack surface.

However, a critical administrative gap was identified in the employee onboarding process. The lack of mandatory security awareness training for new employees presents a \textbf{High} risk. New hires are often prime targets for social engineering attacks, and without proper initial training, the organization is more susceptible to incidents originating from human error, such as falling victim to phishing campaigns.

Immediate remediation should focus on implementing a mandatory security training module into the new employee onboarding process. While the technical controls are currently effective, strengthening the human element of the security program is essential for comprehensive risk reduction.

% --- SECTION 2: ORGANIZATIONAL INFORMATION ---
\section{Organizational Information}
The following details were provided for the assessment. This information establishes the scope and context for the findings presented in this report.

\begin{itemize}
    \item \textbf{Organization Name:} Modern Myth
    \item \textbf{Primary Email Domain:} \texttt{ModernMyth.com}
    \item \textbf{Primary Website Domain:} \url{www.ModernMyth.com}
    \item \textbf{Scoped External IP:} \texttt{38.33.34.129}
\end{itemize}

% --- SECTION 3: SECURITY CONTROL REVIEW ---
\section{Security Control Review}
The following table summarizes the organization's responses to a security controls questionnaire. A checkmark (\ding{51}) indicates a positive control is in place, while a cross (\ding{55}) indicates a potential gap.

\begin{table}[h!]
\centering
\caption{Security Controls Questionnaire Results}
\begin{tabular}{p{0.7\textwidth} c}
\toprule
\textbf{Control Question} & \textbf{Response} \\
\midrule
Do you require MFA to access email? & \ding{51} \\
Do you require MFA to log into computers? & \ding{51} \\
Do you require MFA to access sensitive data systems? & \ding{51} \\
Does your organization have an employee acceptable use policy? & \ding{51} \\
Does your organization do security awareness training for new employees? & \textcolor{red}{\ding{55}} \\
Does your organization do security awareness training for all employees at least once per year? & \ding{51} \\
\bottomrule
\end{tabular}
\end{table}

\paragraph{Analysis:} The organization has implemented strong identity and access management controls with consistent MFA enforcement. The presence of an acceptable use policy and annual security training are also positive indicators. The primary weakness identified is the failure to provide security awareness training specifically for new employees during their onboarding period.

% --- SECTION 4: TECHNICAL SCAN RESULTS ---
\section{Technical Scan Results}
An external network scan was performed to identify accessible services and potential vulnerabilities on the perimeter.

\begin{itemize}
    \item \textbf{Target IP Address:} \texttt{192.168.1.100}
    \item \textbf{Scan Date:} As of the date of this report.
    \item \textbf{Key Finding:} \textbf{No open ports were detected.} All probed ports were in a 'closed' state.
\end{itemize}

\paragraph{Analysis:} This result is highly favorable and indicates a very strong security posture for the scanned host. A lack of open ports means that there are no listening services exposed to the public internet, effectively minimizing the attack surface and preventing unauthorized access attempts against common services. This suggests a well-configured firewall is in place.

% --- SECTION 5: RISK ASSESSMENT ---
\section{Risk Assessment}
This section synthesizes findings from the security control review, technical scans, and pre-existing risk data. The primary risk identified during this assessment is detailed below.

\begin{table}[h!]
\centering
\caption{Identified Risks}
\begin{tabular}{p{0.15\textwidth} p{0.45\textwidth} p{0.15\textwidth} p{0.1\textwidth}}
\toprule
\textbf{Risk ID} & \textbf{Risk Description} & \textbf{Affected Asset(s)} & \textbf{Severity} \\
\midrule
RISK-001 & Lack of mandatory security awareness training for new employees during the onboarding process. & Personnel, Data, Reputation & \textbf{High} \\
\bottomrule
\end{tabular}
\end{table}

\paragraph{Risk-001 Justification:} New employees are statistically more likely to be targeted by and fall for social engineering attacks (e.g., phishing, pretexting). Without immediate training on organizational security policies and common threats, they represent a significant weak link. A successful attack could lead to credential compromise, malware infection, or a data breach.

% --- SECTION 6: RECOMMENDATIONS ---
\section{Recommendations}
The following actions are recommended to mitigate the identified risks and improve the overall security posture of \textbf{Modern Myth}.

\subsection*{Priority 1: High}
\begin{description}
    \item[Recommendation:] \textbf{Implement Mandatory Onboarding Security Training.}
    \item[Details:] Develop and integrate a mandatory security awareness training module into the formal onboarding process for all new hires, including contractors. This training should occur within the first week of employment and cover, at a minimum:
    \begin{itemize}
        \item The organization's Acceptable Use Policy.
        \item How to identify and report phishing attempts.
        \item Best practices for password management and data handling.
        \item The importance of MFA.
    \end{itemize}
    \item[Justification:] This action directly mitigates RISK-001 by ensuring that new personnel are equipped with fundamental security knowledge from day one, drastically reducing their susceptibility to common cyber threats.
\end{description}

\subsection*{Priority 2: Informational}
\begin{description}
    \item[Recommendation:] \textbf{Maintain and Validate Network Security Posture.}
    \item[Details:] The current network perimeter security is excellent. It is recommended to continue this practice through:
    \begin{itemize}
        \item Regular, automated vulnerability scanning of external assets.
        \item Periodic firewall rule set reviews to ensure continued adherence to the principle of least privilege.
    \end{itemize}
    \item[Justification:] Proactive maintenance and validation will ensure that the strong technical security posture observed during this assessment is not degraded over time by configuration changes or newly discovered vulnerabilities.
\end{description}

\end{document}
```