```latex
\documentclass[12pt, a4paper]{article}

% Preamble: Required Packages
\usepackage[margin=1in]{geometry} % Set page margins
\usepackage{pifont}               % For checkmarks and crosses (\ding)
\usepackage{booktabs}             % For professional-looking tables
\usepackage{graphicx}             % For including logos (optional)
\usepackage{hyperref}             % For hyperlinks and document metadata
\usepackage{url}                  % For formatting URLs
\usepackage{seqsplit}             % For splitting long strings in texttt
\usepackage[T1]{fontenc}          % Font encoding

% --- Document Metadata ---
\hypersetup{
    colorlinks=true,
    linkcolor=black,
    urlcolor=blue,
    pdftitle={Cybersecurity Assessment Report},
    pdfauthor={Cybersecurity Analysis Cell},
    pdfsubject={Security Posture Analysis},
    pdfkeywords={Cybersecurity, Risk Assessment, Nmap, Security Controls}
}

% --- Title ---
\title{Cybersecurity Assessment Report \\ \large For: Gilded Cage Design}
\author{Cybersecurity Analysis Cell}
\date{\today}

% ==============================================================================
% --- BEGIN DOCUMENT ---
% ==============================================================================
\begin{document}

\maketitle
\thispagestyle{empty}
\newpage

\tableofcontents
\thispagestyle{empty}
\newpage

\pagestyle{headings}

% ==============================================================================
\section{Executive Overview}
% ==============================================================================
This report details the findings of a cybersecurity assessment conducted for Gilded Cage Design. The assessment combined a review of organizational security controls, an external network scan, and an analysis of pre-existing risks.

The overall security posture is assessed as \textbf{High-Risk}. Several critical gaps in fundamental security controls were identified. The most significant risks stem from the lack of Multi-Factor Authentication (MFA) on employee computers and sensitive data systems. This is compounded by the absence of a formal security awareness training program, which significantly increases the organization's susceptibility to social engineering and phishing attacks.

Furthermore, a technical scan revealed an exposed Secure Shell (SSH) service on an IPv6 address. While a common administrative port, its exposure to the public internet without robust controls presents a direct vector for unauthorized access.

Immediate and decisive action is required to address these deficiencies. Recommendations in this report are prioritized to guide remediation efforts, focusing first on mitigating the most critical risks to the organization's data and operations.

% ==============================================================================
\section{Organizational Information}
% ==============================================================================
The following information was provided for the assessment.

\begin{tabular}{@{}ll}
\toprule
\textbf{Attribute} & \textbf{Value} \\
\midrule
Organization Name & \textbf{Gilded Cage Design} \\
Email Domain & \texttt{GildedCageDesign.com} \\
Website Domain & \url{www.GildedCageDesign.com} \\
Known External IP & \texttt{199.124.170.88} \\
\bottomrule
\end{tabular}

% ==============================================================================
\section{Security Control Review}
% ==============================================================================
A review of administrative and organizational security controls was conducted based on a standardized questionnaire. The responses indicate critical gaps in identity and access management and employee security training. "No" answers represent a failure to meet baseline security best practices and are a primary source of organizational risk.

\begin{table}[h!]
\centering
\caption{Organizational Security Control Status}
\begin{tabular}{@{}p{0.8\linewidth}c@{}}
\toprule
\textbf{Control Question} & \textbf{Response} \\
\midrule
Do you require MFA to access email? & \ding{51} \\
Do you require MFA to log into computers? & \textbf{\color{red}\ding{55}} \\
Do you require MFA to access sensitive data systems? & \textbf{\color{red}\ding{55}} \\
\addlinespace
Does your organization have an employee acceptable use policy? & \ding{51} \\
Does your organization do security awareness training for new employees? & \textbf{\color{red}\ding{55}} \\
Does your organization do security awareness training for all employees at least once per year? & \textbf{\color{red}\ding{55}} \\
\bottomrule
\end{tabular}
\end{table}

% ==============================================================================
\section{Technical Scan Results}
% ==============================================================================
An external network scan was performed to identify exposed services. The scan targeted the known external IPv6 address associated with the organization.

\subsection{Scan Target}
The following host was scanned:
\begin{itemize}
    \item \textbf{Target IP:} \seqsplit{\texttt{2001:db8::1}}
\end{itemize}

\subsection{Open Ports and Services}
The scan identified one open port on the target system. Details are provided in Table 2.

\begin{table}[h!]
\centering
\caption{Open Ports Detected on \seqsplit{\texttt{2001:db8::1}}}
\begin{tabular}{@{}llll@{}}
\toprule
\textbf{Port} & \textbf{State} & \textbf{Service} & \textbf{Notes} \\
\midrule
22/tcp & Open & ssh & Secure Shell access. Version information was not obtained. \\
\bottomrule
\end{tabular}
\end{table}

\subsection{Technical Analysis}
The presence of an open SSH port (22) indicates that a system is configured for remote administrative access from the internet. This is a common attack vector. Without robust controls such as IP address whitelisting, key-based authentication, and diligent patching, this service can be exploited by brute-force attacks or by leveraging stolen credentials. The risk associated with this finding is significantly elevated by the organizational control gaps identified in Section 3.

% ==============================================================================
\section{Risk Assessment}
% ==============================================================================
This section synthesizes the findings from the security control review and the technical scan. The following risks have been identified and prioritized based on their potential impact and likelihood.

\begin{table}[h!]
\centering
\caption{Identified Risk Summary}
\begin{tabular}{@{}lp{0.6\linewidth}l@{}}
\toprule
\textbf{ID} & \textbf{Risk Description} & \textbf{Severity} \\
\midrule
\textbf{RISK-001} & \textbf{Lack of MFA on Endpoints and Sensitive Systems:} The absence of MFA on computers and sensitive systems allows an attacker with valid credentials to gain unauthorized access, bypass primary security controls, and potentially access or exfiltrate critical data. & \textbf{Critical} \\
\addlinespace
\textbf{RISK-002} & \textbf{Inadequate Security Awareness Training:} Without initial and recurring training, employees are more likely to fall victim to phishing, malware, and social engineering attacks, which are the leading causes of security breaches. This undermines all other technical security investments. & \textbf{High} \\
\addlinespace
\textbf{RISK-003} & \textbf{Exposed SSH Management Interface:} The publicly accessible SSH service on \seqsplit{\texttt{2001:db8::1}} provides a direct target for attackers. This risk is amplified by the lack of MFA, as compromised employee credentials could be used to access this system. & \textbf{Medium} \\
\bottomrule
\end{tabular}
\end{table}

% ==============================================================================
\section{Recommendations}
% ==============================================================================
To mitigate the identified risks and improve the overall security posture, the following actions are recommended. They are prioritized to address the most critical vulnerabilities first.

\subsection{Immediate Priority (Critical)}
\begin{itemize}
    \item \textbf{Implement MFA (RISK-001):} Immediately enforce MFA for all employees for logging into their computers and accessing any system containing sensitive or critical data. This is the single most effective control to prevent unauthorized access.
\end{itemize}

\subsection{High Priority}
\begin{itemize}
    \item \textbf{Establish Security Awareness Training (RISK-002):} Develop and deploy a mandatory security awareness training program. The program must be completed by all new hires during onboarding and by all existing employees at least annually. Content should focus on identifying phishing, password hygiene, and acceptable use.
\end{itemize}

\subsection{Medium Priority}
\begin{itemize}
    \item \textbf{Secure the SSH Service (RISK-003):}
    \begin{enumerate}
        \item Review the business need for the SSH service on \seqsplit{\texttt{2001:db8::1}} to be accessible from the public internet.
        \item If access is required, restrict it to known, trusted IP addresses using firewall rules.
        \item Disable password-based authentication and enforce the use of strong cryptographic keys (e.g., ED25519) for all SSH access.
        \item Ensure the SSH service is kept up-to-date with the latest security patches.
    \end{enumerate}
\end{itemize}

\end{document}
% ==============================================================================
% --- END OF DOCUMENT ---
% ==============================================================================
```