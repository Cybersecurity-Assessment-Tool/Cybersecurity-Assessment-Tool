```latex
\documentclass[12pt]{article}

% ----------------------------------------------------------------------
% PREAMBLE
% ----------------------------------------------------------------------
\usepackage[margin=1in]{geometry}
\usepackage{pifont} % For \ding
\usepackage{booktabs} % For professional tables
\usepackage{hyperref} % For hyperlinks
\usepackage{url} % For URL formatting
\usepackage{seqsplit} % For splitting long strings in tt font
\usepackage{graphicx}
\usepackage{xcolor}
\usepackage{fancyhdr}
\usepackage{lastpage}

% --- Hyperref Setup ---
\hypersetup{
    colorlinks=true,
    linkcolor=blue,
    filecolor=magenta,      
    urlcolor=cyan,
    pdftitle={Cybersecurity Risk Assessment Report},
    pdfpagemode=FullScreen,
}

% --- Header and Footer ---
\pagestyle{fancy}
\fancyhf{} % Clear all header and footer fields
\fancyhead[L]{Cybersecurity Risk Assessment Report}
\fancyhead[R]{True Grit}
\fancyfoot[C]{\thepage\ of \pageref{LastPage}}
\renewcommand{\headrulewidth}{0.4pt}
\renewcommand{\footrulewidth}{0.4pt}

% --- Custom Commands ---
\newcommand{\yes}{\ding{51}}
\newcommand{\no}{\ding{55}}
\newcommand{\severity}[1]{\textbf{#1}}
\newcommand{\critical}{\textcolor{red}{\severity{Critical}}}
\newcommand{\high}{\textcolor{orange}{\severity{High}}}
\newcommand{\medium}{\textcolor{yellow!80!black}{\severity{Medium}}}
\newcommand{\low}{\textcolor{green}{\severity{Low}}}

% ----------------------------------------------------------------------
% DOCUMENT START
% ----------------------------------------------------------------------
\begin{document}

% --- Title Page ---
\title{
    \vspace{2cm}
    \textbf{Cybersecurity Risk Assessment Report} \\
    \large Prepared for: True Grit
}
\author{Cybersecurity Analyst}
\date{\today}
\maketitle
\thispagestyle{empty}
\newpage

\tableofcontents
\newpage

% ----------------------------------------------------------------------
% SECTION 1: EXECUTIVE SUMMARY
% ----------------------------------------------------------------------
\section{Executive Summary}

This report provides a comprehensive cybersecurity assessment for True Grit, based on network scan data, a security controls questionnaire, and a review of pre-existing risks. The analysis correlates technical findings with organizational policies to provide a holistic view of the current security posture.

The assessment identified several critical and high-risk findings that require immediate attention. Key issues include:

\begin{itemize}
    \item \textbf{Systemic Exposure of Remote Desktop Protocol (RDP):} Technical scans confirmed open RDP services on multiple internal systems, including a new finding on host \texttt{10.10.10.51}. This aligns with a previously identified risk, indicating a pattern of insecure configuration that significantly increases the risk of unauthorized access and ransomware attacks.
    
    \item \textbf{Critical Gaps in Access Control:} The organization does not require Multi-Factor Authentication (MFA) for computer logins. This is a critical security gap that nullifies a primary defense against credential theft. When combined with exposed RDP, this elevates the likelihood of a successful breach.
    
    \item \textbf{Inadequate Security Training Program:} The lack of mandatory, annual security awareness training for all employees leaves the organization vulnerable to phishing and social engineering attacks, which are the leading causes of initial compromise.
\end{itemize}

This report outlines these findings in detail and provides actionable recommendations to mitigate the identified risks and strengthen the overall security posture of True Grit.

% ----------------------------------------------------------------------
% SECTION 2: ORGANIZATIONAL INFORMATION
% ----------------------------------------------------------------------
\section{Organizational Information}

The following details were provided for the assessment. This information is used to establish the context and scope of the review.

\begin{tabular}{@{}ll}
    \toprule
    \textbf{Attribute} & \textbf{Value} \\
    \midrule
    Organization Name & True Grit \\
    Email Domain & \texttt{TrueGrit.net} \\
    Website Domain & \url{www.TrueGrit.net} \\
    External IP Address & \texttt{154.71.186.140} \\
    \bottomrule
\end{tabular}

% ----------------------------------------------------------------------
% SECTION 3: SECURITY CONTROL REVIEW
% ----------------------------------------------------------------------
\section{Security Control Review}

A review of the organization's security controls was conducted via a questionnaire. The responses are summarized below. Items marked with \no\ indicate significant gaps in the security framework.

\begin{table}[h!]
\centering
\begin{tabular}{@{}p{0.7\linewidth}cc@{}}
    \toprule
    \textbf{Control Question} & \textbf{Response} & \textbf{Status} \\
    \midrule
    Do you require MFA to access email? & Yes & \yes \\
    Do you require MFA to log into computers? & No & \no \\
    Do you require MFA to access sensitive data systems? & Yes & \yes \\
    Does your organization have an employee acceptable use policy? & Yes & \yes \\
    Does your organization do security awareness training for new employees? & Yes & \yes \\
    Does your organization do security awareness training for all employees at least once per year? & No & \no \\
    \bottomrule
\end{tabular}
\caption{Security Controls Questionnaire Results}
\end{table}

\subsection*{Analysis of Gaps}
\begin{itemize}
    \item \textbf{No MFA for Computer Logins:} This is a \high\ risk. Without MFA, compromised credentials (e.g., from a phishing attack) can be used directly to gain access to an employee's computer, providing a foothold for an attacker to move laterally within the network.
    \item \textbf{No Annual Security Awareness Training:} This is a \high\ risk. Security threats evolve, and employee awareness diminishes over time. The lack of recurring training increases the probability of successful phishing and social engineering attacks.
\end{itemize}

% ----------------------------------------------------------------------
% SECTION 4: TECHNICAL SCAN RESULTS
% ----------------------------------------------------------------------
\section{Technical Scan Results}

A network scan was performed to identify open ports and services on the target system.

\begin{itemize}
    \item \textbf{Target IP Address:} \texttt{10.10.10.51}
    \item \textbf{Scan Status:} Host is up.
\end{itemize}

The following open ports were discovered:

\begin{table}[h!]
\centering
\begin{tabular}{@{}llll@{}}
    \toprule
    \textbf{Port} & \textbf{State} & \textbf{Service Name} & \textbf{Analysis} \\
    \midrule
    3389/tcp & open & \texttt{ms-wbt-server} & Microsoft Remote Desktop Protocol (RDP). \\
    \bottomrule
\end{tabular}
\caption{Open Ports on \texttt{10.10.10.51}}
\end{table}

\subsection*{Analysis of Findings}
The discovery of open RDP on \texttt{10.10.10.51} is a significant finding. RDP is a primary target for attackers who use it for initial access via brute-force password attacks or by exploiting vulnerabilities. This finding, combined with the pre-existing risk of RDP exposure on \texttt{10.10.10.50}, suggests a systemic configuration issue across the network rather than an isolated incident.

% ----------------------------------------------------------------------
% SECTION 5: CORRELATED RISK ASSESSMENT
% ----------------------------------------------------------------------
\section{Correlated Risk Assessment}

The following table synthesizes findings from the security questionnaire, technical scans, and pre-existing risk data to provide a correlated view of the most pressing security risks.

\begin{table}[h!]
\centering
\begin{tabular}{@{}lp{0.6\linewidth}l@{}}
    \toprule
    \textbf{Risk Title} & \textbf{Description} & \textbf{Severity} \\
    \midrule
    \textbf{Systemic RDP Exposure} & RDP is exposed on multiple internal systems (\texttt{10.10.10.50}, \texttt{10.10.10.51}), indicating a lack of network hardening. This creates a large attack surface for ransomware and unauthorized access. & \critical \\
    \addlinespace
    \textbf{Lack of Endpoint MFA} & The absence of MFA for computer logins, combined with exposed RDP, drastically increases risk. Stolen credentials can be used to log into systems without a second authentication factor, facilitating lateral movement. & \high \\
    \addlinespace
    \textbf{Inadequate Training Cadence} & The lack of annual security training makes employees prime targets for phishing attacks aimed at stealing credentials. These stolen credentials can then be used against the exposed RDP services. & \high \\
    \bottomrule
\end{tabular}
\caption{Summary of Correlated Risks}
\end{table}

% ----------------------------------------------------------------------
% SECTION 6: RECOMMENDATIONS
% ----------------------------------------------------------------------
\section{Recommendations}

The following actions are recommended to mitigate the identified risks and improve the security posture of True Grit.

\subsection*{Immediate Actions (Mitigate Critical Risks)}
\begin{enumerate}
    \item \textbf{Restrict RDP Access Immediately:} For hosts \texttt{10.10.10.50} and \texttt{10.10.10.51}, implement host-based or network firewall rules to deny all access to TCP port 3389, except from a dedicated and secured management subnet or bastion host.
    \item \textbf{Review RDP Configurations:} Conduct an urgent audit of the entire network to identify and remediate any other systems with unnecessarily exposed RDP services.
\end{enumerate}

\subsection*{High-Priority Actions (Within 30-60 Days)}
\begin{enumerate}
    \item \textbf{Deploy MFA for All Computer Logins:} Implement a robust MFA solution for all user and administrator logins to workstations and servers. This is the single most effective control to prevent unauthorized access from compromised credentials.
    \item \textbf{Establish Annual Security Training Program:} Institute a mandatory, annual security awareness training program for all employees. The training should cover phishing, password hygiene, and acceptable use policies.
\end{enumerate}

\subsection*{Strategic Long-Term Improvements}
\begin{enumerate}
    \item \textbf{Implement a Secure Remote Access Solution:} Replace direct RDP access with a modern, secure solution such as a Virtual Private Network (VPN) with MFA or a Zero Trust Network Access (ZTNA) platform.
    \item \textbf{Develop a Vulnerability Management Program:} Establish a formal process for regularly scanning, identifying, prioritizing, and remediating vulnerabilities across all organizational assets.
\end{enumerate}

% ----------------------------------------------------------------------
% DOCUMENT END
% ----------------------------------------------------------------------
\end{document}
```