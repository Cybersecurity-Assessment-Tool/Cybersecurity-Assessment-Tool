```latex
\documentclass[12pt]{article}

% Preamble: Required Packages
\usepackage[margin=1in]{geometry}
\usepackage{pifont} % For checkmarks and crosses
\usepackage{booktabs} % For professional tables
\usepackage{hyperref} % For clickable links
\usepackage{url}      % For formatting URLs
\usepackage{seqsplit} % For splitting long strings in texttt
\usepackage{graphicx} % For logo (optional placeholder)
\usepackage{xcolor}   % For colors in table cells

% Document Information
\title{Cybersecurity Posture Assessment Report}
\author{Cybersecurity Analysis Division}
\date{November 22, 2025}

\begin{document}

\maketitle

\begin{abstract}
This report provides a comprehensive cybersecurity assessment for Phoenix Rising. The analysis is based on a synthesis of external network scanning, a review of organizational security controls via a questionnaire, and an evaluation of pre-existing risks. The objective is to identify vulnerabilities, security gaps, and misconfigurations, and to provide actionable recommendations to enhance the organization's security posture. Key findings include critical gaps in multi-factor authentication (MFA) controls, an outdated web server version, and a need for comprehensive annual security training.
\end{abstract}

\tableofcontents
\newpage

% --- Section 1: Overview ---
\section{Executive Summary}
The assessment reveals several areas of significant risk that require immediate attention. While the organization has implemented some foundational security controls, such as an acceptable use policy and security training for new hires, critical deficiencies exist.

The most severe findings are the absence of Multi-Factor Authentication (MFA) for email and computer access, which exposes the organization to account takeover and unauthorized access. Furthermore, the external-facing web server is running an outdated version of Nginx, which is known to have security vulnerabilities. These technical and procedural gaps, when combined, create a high-risk environment that could be exploited by malicious actors.

This report details these findings and provides a prioritized list of recommendations to mitigate the identified risks and strengthen the overall security posture.

% --- Section 2: Organizational Information ---
\section{Organizational Information}
The following information was provided for the assessment.

\begin{tabular}{@{}ll}
\toprule
\textbf{Attribute} & \textbf{Value} \\
\midrule
Organization Name & \textbf{Phoenix Rising} \\
Email Domain & \texttt{PhoenixRising.com} \\
Website Domain & \texttt{www.PhoenixRising.com} \\
External IP Address & \texttt{180.97.29.120} \\
\bottomrule
\end{tabular}

% --- Section 3: Security Control Review ---
\section{Security Control Review}
A review of the organization's security controls was conducted based on a standardized questionnaire. The results highlight critical gaps in access control and employee security awareness. A "No" answer indicates a deviation from security best practices and a potential area of risk.

\begin{tabular}{@{}p{0.75\linewidth}c}
\toprule
\textbf{Security Control Question} & \textbf{Response} \\
\midrule
Do you require MFA to access email? & \ding{55} \\
Do you require MFA to log into computers? & \ding{55} \\
Do you require MFA to access sensitive data systems? & \ding{51} \\
Does your organization have an employee acceptable use policy? & \ding{51} \\
Does your organization do security awareness training for new employees? & \ding{51} \\
Does your organization do security awareness training for all employees at least once per year? & \ding{55} \\
\bottomrule
\end{tabular}

\vspace{1em}
\noindent \textbf{Note:} \ding{51} = Yes (Control in place), \ding{55} = No (Control gap identified).

% --- Section 4: Technical Scan Results ---
\section{Technical Scan Results}
An external network scan was performed to identify open ports and exposed services on the organization's infrastructure.

\begin{itemize}
    \item \textbf{Scan Target:} \texttt{192.168.10.5}
    \item \textbf{Scan Date:} \texttt{2025-11-22T10:00:00Z}
\end{itemize}

\subsection{Open Ports and Services}
The following table details the services discovered during the scan.

\begin{tabular}{@{}lllll}
\toprule
\textbf{Port} & \textbf{State} & \textbf{Service} & \textbf{Product} & \textbf{Version} \\
\midrule
443/tcp & open & https & nginx & 1.18.0 \\
\bottomrule
\end{tabular}

\subsection{Detailed Findings}
\begin{itemize}
    \item \textbf{Outdated Web Server:} The Nginx server is running version \texttt{1.18.0}, which was released in 2020. This version is significantly outdated and is missing numerous security patches for known vulnerabilities. This poses a high risk of compromise.
    \item \textbf{SSL Certificate Mismatch:} The SSL certificate presented by the server has a Common Name (\texttt{www.acme-corp.com}) that does not match the organization's domain (\texttt{www.PhoenixRising.com}). This is a misconfiguration that can cause browser trust errors and may indicate improper certificate management.
\end{itemize}

% --- Section 5: Risk Assessment ---
\section{Risk Assessment}
This section synthesizes findings from the security control review and the technical scan. No pre-existing vulnerabilities were reported.

\begin{tabular}{@{}p{0.1\linewidth}p{0.3\linewidth}p{0.15\linewidth}p{0.4\linewidth}}
\toprule
\textbf{ID} & \textbf{Risk Name} & \textbf{Severity} & \textbf{Overview} \\
\midrule
RISK-001 & Lack of MFA for Email Access & \textbf{Critical} & The absence of MFA on email accounts is a primary enabler for Business Email Compromise (BEC) and phishing attacks, potentially leading to data breaches and financial loss. \\
\addlinespace
RISK-002 & Outdated Nginx Web Server & \textbf{High} & The web server is running Nginx 1.18.0, a version with multiple known vulnerabilities. An attacker could exploit these flaws to compromise the server. \\
\addlinespace
RISK-003 & Lack of MFA for Endpoint Logon & \textbf{High} & Without MFA for computer logins, a compromised password is all an attacker needs to gain access to an employee's workstation and the corporate network. \\
\addlinespace
RISK-004 & Inadequate Annual Security Training & \textbf{High} & Failing to provide annual security awareness training for all employees increases susceptibility to social engineering and phishing, as security knowledge degrades over time. \\
\addlinespace
RISK-005 & SSL Certificate Mismatch & \textbf{Medium} & The web server presents a certificate for a different domain, which erodes user trust and indicates a server misconfiguration that could be exploited in man-in-the-middle scenarios. \\
\bottomrule
\end{tabular}

% --- Section 6: Recommendations ---
\section{Recommendations}
The following actions are recommended to mitigate the identified risks and improve the overall security posture of Phoenix Rising.

\begin{enumerate}
    \item \textbf{[Critical] Implement MFA for Email:} Immediately enforce MFA for all user accounts on the \texttt{PhoenixRising.com} email system. This is the single most effective control to prevent account takeovers.
    
    \item \textbf{[High] Upgrade Web Server:} Plan and execute an upgrade of the Nginx server from version \texttt{1.18.0} to a current, stable, and patched version. Regularly schedule patch management for all internet-facing systems.
    
    \item \textbf{[High] Deploy MFA for Endpoints:} Implement MFA for all employee computer and laptop logons. This adds a critical layer of security to protect against unauthorized access from stolen credentials.
    
    \item \textbf{[High] Establish Annual Security Training:} Develop and enforce a mandatory annual security awareness training program for all employees. The training should cover current threats such as phishing, ransomware, and social engineering.
    
    \item \textbf{[Medium] Correct SSL Certificate:} Replace the incorrect SSL certificate on the web server at \texttt{192.168.10.5} with a valid certificate issued for \texttt{www.PhoenixRising.com}. Ensure automated certificate renewal and monitoring processes are in place.
\end{enumerate}

\end{document}
```