```latex
\documentclass[12pt]{article}

% Preamble: Required Packages
\usepackage[margin=1in]{geometry}
\usepackage{pifont} % For checkmarks and crosses
\usepackage{booktabs} % For professional tables
\usepackage{hyperref} % For hyperlinks
\usepackage{url} % For URL formatting
\usepackage{seqsplit} % To split long strings in tt font
\usepackage{graphicx} % For potential logos
\usepackage{xcolor} % For colors

% Document Information
\title{
    \vspace{-2cm}
    \includegraphics[width=0.3\textwidth]{cyber_logo.png} \\ % Placeholder for a logo
    \hrule
    \vspace{0.5cm}
    \textbf{Cybersecurity Posture Assessment Report} \\
    \vspace{0.2cm}
    \large{Prepared for: Terraform Global} \\
    \vspace{0.5cm}
    \hrule
}
\author{Cybersecurity Analysis Division}
\date{\today}

\begin{document}

\maketitle
\thispagestyle{empty}
\newpage

\tableofcontents
\newpage

\section{Executive Summary}

This report details the findings of a cybersecurity posture assessment conducted for Terraform Global. The analysis synthesizes data from an external network scan, a security controls questionnaire, and a review of pre-existing risks.

The assessment identified several high-impact security deficiencies that require immediate attention. The most critical finding is the systemic exposure of the Remote Desktop Protocol (RDP) on internal systems, including a newly discovered host at \texttt{10.10.10.51}. This technical vulnerability is significantly exacerbated by a critical policy gap: the lack of mandatory Multi-Factor Authentication (MFA) for computer logins. This combination creates a direct and high-risk pathway for attackers to gain unauthorized access to the internal network using compromised credentials.

Furthermore, gaps in foundational security policies, such as the absence of an employee Acceptable Use Policy and a lack of annual security awareness training for all staff, weaken the organization's human firewall and increase the likelihood of initial compromise through phishing or social engineering.

Immediate remediation should focus on securing RDP access, enforcing endpoint MFA, and formalizing security policies and training programs to mitigate these interconnected risks.

\section{Organizational Information}

The following information was provided for the assessment scope.

\begin{itemize}
    \item \textbf{Organization Name:} Terraform Global
    \item \textbf{Email Domain:} \texttt{TerraformGlobal.org}
    \item \textbf{Website Domain:} \url{www.TerraformGlobal.org}
    \item \textbf{External IP Address:} \texttt{215.108.73.196}
\end{itemize}

\section{Security Control Review}

A review of the organization's security controls was conducted via a questionnaire. The responses highlight critical gaps in endpoint security and security governance. A summary of the findings is presented in Table 1.

\begin{table}[h!]
\centering
\caption{Security Controls Questionnaire Results}
\begin{tabular}{p{0.75\linewidth} c}
\toprule
\textbf{Control Question} & \textbf{Response} \\
\midrule
Do you require MFA to access email? & \ding{51} \\
Do you require MFA to log into computers? & \textcolor{red}{\ding{55}} \\
Do you require MFA to access sensitive data systems? & \ding{51} \\
Does your organization have an employee acceptable use policy? & \textcolor{red}{\ding{55}} \\
Does your organization do security awareness training for new employees? & \ding{51} \\
Does your organization do security awareness training for all employees at least once per year? & \textcolor{red}{\ding{55}} \\
\bottomrule
\end{tabular}
\end{table}

\subsection*{Analysis of Control Gaps}
\begin{itemize}
    \item \textbf{No MFA for Computer Logins:} This is a critical weakness. In the event of a credential compromise, an attacker can gain direct access to an employee's workstation, and from there, pivot to other network resources.
    \item \textbf{No Acceptable Use Policy (AUP):} The absence of a formal AUP creates ambiguity regarding safe computing practices and the consequences of misuse, weakening the overall security culture.
    \item \textbf{No Annual Security Training:} Threats evolve constantly. Without annual refresher training for all employees, the organization is more susceptible to phishing, social engineering, and other human-targeted attacks.
\end{itemize}

\section{Technical Scan Results}

An Nmap scan was performed on the specified target to identify open ports and exposed services.

\begin{itemize}
    \item \textbf{Target IP Address:} \texttt{10.10.10.51}
    \item \textbf{Scan Date:} \today
\end{itemize}

The scan revealed the following open port, as detailed in Table 2.

\begin{table}[h!]
\centering
\caption{Open Port Findings for \texttt{10.10.10.51}}
\begin{tabular}{c c l}
\toprule
\textbf{Port} & \textbf{State} & \textbf{Service Name} \\
\midrule
3389/tcp & open & ms-wbt-server (Microsoft Remote Desktop Protocol) \\
\bottomrule
\end{tabular}
\end{table}

\subsection*{Analysis of Technical Findings}
The scan confirms that Port 3389, used for the Remote Desktop Protocol (RDP), is open on the target system. RDP is a primary target for attackers who use brute-force techniques or exploit vulnerabilities (such as BlueKeep) to gain remote control of systems. This finding, combined with the pre-existing risk of RDP exposure on another host (\texttt{10.10.10.50}), indicates a systemic issue in network configuration and access control.

\section{Correlated Risk Assessment}

By correlating the security control gaps, technical findings, and pre-existing vulnerabilities, we have compiled a prioritized list of risks facing the organization.

\begin{table}[h!]
\centering
\caption{Summary of Identified Risks}
\begin{tabular}{p{0.15\linewidth} p{0.55\linewidth} p{0.2\linewidth}}
\toprule
\textbf{Risk Name} & \textbf{Description} & \textbf{Severity} \\
\midrule
\textbf{Systemic RDP Exposure} & RDP is exposed on multiple internal systems (\texttt{10.10.10.50}, \texttt{10.10.10.51}). This service is a frequent target for ransomware and unauthorized access attacks. & \textbf{Critical} \\
\addlinespace
\textbf{Lack of Endpoint MFA} & The absence of MFA for computer logins directly enables attackers with stolen credentials to access endpoints and leverage exposed services like RDP. & \textbf{High} \\
\addlinespace
\textbf{Inadequate Security Policies \& Training} & The lack of an AUP and annual security training increases the likelihood of credential compromise via phishing, which is the primary precursor to attacks that would exploit the other identified risks. & \textbf{High} \\
\bottomrule
\end{tabular}
\end{table}

\section{Recommendations}

The following prioritized actions are recommended to mitigate the identified risks and improve the overall security posture of Terraform Global.

\subsection{Immediate Priority (Critical)}
\begin{enumerate}
    \item \textbf{Remediate RDP Exposure:}
    \begin{itemize}
        \item Immediately close port 3389 on all systems where it is not business-critical, including \texttt{10.10.10.50} and \texttt{10.10.10.51}.
        \item For systems requiring remote access, implement a secure solution such as a Virtual Private Network (VPN) or a Remote Desktop Gateway, both of which should require MFA.
    \end{itemize}
\end{enumerate}

\subsection{High Priority}
\begin{enumerate}
    \setcounter{enumi}{1}
    \item \textbf{Deploy Endpoint MFA:}
    \begin{itemize}
        \item Procure and deploy a solution to enforce MFA for all user logins to desktops and laptops. This is a critical compensating control that drastically reduces the risk of credential-based attacks.
    \end{itemize}
    \item \textbf{Develop and Implement Security Policies:}
    \begin{itemize}
        \item Create a formal Acceptable Use Policy (AUP) that all employees must read and acknowledge. The policy should clearly define rules for handling data, using company assets, and reporting security incidents.
    \end{itemize}
    \item \textbf{Enhance Security Awareness Program:}
    \begin{itemize}
        \item Institute a mandatory security awareness training program for all employees to be completed annually. This program should cover current threats such as phishing, ransomware, and social engineering.
    \end{itemize}
\end{enumerate}

\end{document}
```