```latex
\documentclass[12pt]{article}

% --- PACKAGE IMPORTS ---
\usepackage[margin=1in]{geometry} % Set page margins
\usepackage{pifont}                 % For checkmarks and crosses (\ding)
\usepackage{booktabs}               % For professional-looking tables
\usepackage{graphicx}               % For including logos (optional)
\usepackage{xcolor}                 % For color definitions
\usepackage{hyperref}               % For hyperlinks and PDF metadata
\usepackage{url}                    % For URL formatting
\usepackage{seqsplit}               % To split long strings in texttt

% --- DOCUMENT METADATA ---
\hypersetup{
    colorlinks=true,
    linkcolor=black,
    urlcolor=blue,
    pdftitle={Cybersecurity Assessment Report},
    pdfauthor={Cybersecurity Analysis Division},
    pdfsubject={Security Posture Analysis},
    pdfkeywords={Cybersecurity, Nmap, Risk Assessment, Security Controls}
}

% --- CUSTOM COMMANDS ---
\newcommand{\yes}{\ding{51}} % Green checkmark
\newcommand{\no}{\ding{55}}  % Red cross

% --- DOCUMENT START ---
\begin{document}

% --- TITLE PAGE ---
\begin{titlepage}
    \centering
    \vspace*{1cm}
    \Huge\textbf{Cybersecurity Assessment Report}
    \vspace{1.5cm}
    \Large
    \textbf{Prepared for:}\\
    \vspace{0.5cm}
    \textbf{Stone Arch Masonry}
    \vspace{2cm}
    \large
    \textbf{Report Date:}\\
    \vspace{0.5cm}
    \today
    \vfill
    \large
    \textbf{Analysis Division}\\
    Cybersecurity Analysis Group
\end{titlepage}

\tableofcontents
\newpage

% --- EXECUTIVE SUMMARY ---
\section{Executive Summary}
This report provides a comprehensive analysis of the cybersecurity posture of \textbf{Stone Arch Masonry}, based on a technical network scan, a review of organizational security controls, and an evaluation of pre-existing risk data.

The assessment has identified several \textbf{critical and high-severity risks} that require immediate attention. A technical scan of the internal network revealed an open service on port 8080 of host \texttt{10.5.5.5}, which presents itself as a \textbf{"TOP SECRET DB"}. This finding directly contradicts a previous risk assessment that had marked this port as secure, indicating a significant failure in the existing vulnerability management process.

Furthermore, the review of security controls highlights fundamental gaps in the organization's defenses. The lack of Multi-Factor Authentication (MFA) for email and computer access, combined with the absence of an employee Acceptable Use Policy and regular security awareness training, exposes the organization to a high likelihood of account compromise, phishing, and insider threats.

The overall security posture is considered \textbf{poor}. The recommendations outlined in this report are prioritized to address the most critical exposures first, with the immediate goal of securing the exposed database and strengthening foundational security controls.

% --- ORGANIZATIONAL INFORMATION ---
\section{Organizational Information}
The following details were provided for the assessment.
\begin{itemize}
    \item \textbf{Organization Name:} Stone Arch Masonry
    \item \textbf{Email Domain:} \texttt{StoneArchMasonry.com}
    \item \textbf{Website Domain:} \url{www.StoneArchMasonry.com}
    \item \textbf{External IP Address:} \texttt{6.226.156.93}
\end{itemize}

% --- SECURITY CONTROL REVIEW ---
\section{Security Control Review}
A review of administrative and technical security controls was conducted via a questionnaire. The responses indicate significant gaps in foundational security practices. "No" answers represent a lack of a critical control and are associated with a higher risk profile.

\begin{table}[h!]
\centering
\caption{Security Control Questionnaire Responses}
\begin{tabular}{p{0.7\linewidth} c}
\toprule
\textbf{Control Question} & \textbf{Response} \\
\midrule
Do you require MFA to access email? & \no \\
Do you require MFA to log into computers? & \no \\
Do you require MFA to access sensitive data systems? & \yes \\
Does your organization have an employee acceptable use policy? & \no \\
Does your organization do security awareness training for new employees? & \no \\
Does your organization do security awareness training for all employees at least once per year? & \no \\
\bottomrule
\end{tabular}
\end{table}

The absence of MFA on email and endpoints, coupled with a lack of security policies and training, creates a permissive environment for security incidents to occur and escalate.

% --- TECHNICAL SCAN RESULTS ---
\section{Technical Scan Results}
An internal network scan was performed to identify active services and potential vulnerabilities. The scan targeted the host at \texttt{10.5.5.5}.

\subsection{Scan Summary}
\begin{itemize}
    \item \textbf{Target IP:} \texttt{10.5.5.5}
    \item \textbf{Host Status:} Up
    \item \textbf{Critical Findings:} 1
\end{itemize}

\subsection{Open Ports and Services}
A critical service was identified running on port 8080. The service title suggests it contains highly sensitive information.

\begin{table}[h!]
\centering
\caption{Discovered Open Ports on \texttt{10.5.5.5}}
\begin{tabular}{l l l p{0.4\linewidth}}
\toprule
\textbf{Port} & \textbf{State} & \textbf{Service} & \textbf{Details} \\
\midrule
8080/tcp & open & http & The HTTP title script returned: \textbf{"TOP SECRET DB"}. This is a critical exposure. \\
\bottomrule
\end{tabular}
\end{table}

\textbf{Analysis:} The discovery of a service explicitly labeled as a "TOP SECRET DB" is a finding of the highest severity. This service appears to be an unauthenticated web interface for a database. This finding is particularly alarming as it \textbf{directly contradicts} the information from the existing risk register (Input 3), which incorrectly classified this port as a secure false positive.

% --- RISK ASSESSMENT ---
\section{Risk Assessment}
Based on the correlation of all data inputs, the following risks have been identified and prioritized. The severity level is assigned based on the potential impact and likelihood of exploitation.

\begin{table}[h!]
\centering
\caption{Summary of Identified Risks}
\begin{tabular}{p{0.15\linewidth} p{0.65\linewidth} l}
\toprule
\textbf{Risk Title} & \textbf{Description} & \textbf{Severity} \\
\midrule
\textbf{Exposed Sensitive Database Interface} & An open service on \texttt{10.5.5.5:8080} is titled "TOP SECRET DB". This suggests a direct, and likely unauthenticated, interface to critical data is exposed on the internal network. & \textbf{Critical} \\
\addlinespace
\textbf{Lack of Multi-Factor Authentication} & MFA is not enforced on email or computer logins. This dramatically increases the risk of unauthorized access via credential theft, password spraying, or phishing attacks. & High \\
\addlinespace
\textbf{Insufficient Security Policies \& Training} & The absence of an Acceptable Use Policy and security awareness training means employees are likely unaware of security best practices, making them highly susceptible to social engineering. & High \\
\addlinespace
\textbf{Flawed Vulnerability Management Process} & The critical exposure on port 8080 was previously misclassified as a secure false positive. This indicates a systemic failure in the process used to identify, assess, and track vulnerabilities. & Medium \\
\bottomrule
\end{tabular}
\end{table}

% --- RECOMMENDATIONS ---
\section{Recommendations}
The following actionable recommendations are provided to mitigate the identified risks. They are prioritized to address the most severe threats first.

\subsection{Immediate Priority (Within 24-48 Hours)}
\begin{enumerate}
    \item \textbf{Secure Exposed Database:} Immediately investigate the service running on \texttt{10.5.5.5:8080}.
    \begin{itemize}
        \item Identify the owner and purpose of the "TOP SECRET DB".
        \item Restrict all network access to the service until it can be properly secured with authentication, authorization, and encryption.
        \item If the service is not required, decommission it.
    \end{itemize}
    \item \textbf{Review Vulnerability Management Process:} Conduct a post-mortem to understand why the risk on port 8080 was previously dismissed. Implement corrective actions to ensure the process is effective.
\end{enumerate}

\subsection{High Priority (Within 30 Days)}
\begin{enumerate}
    \item \textbf{Implement Multi-Factor Authentication (MFA):}
    \begin{itemize}
        \item Enforce MFA for all users on email systems (e.g., Office 365, Google Workspace).
        \item Enforce MFA for all remote access (VPN) and local computer logins where possible.
    \end{itemize}
    \item \textbf{Develop and Implement Security Training:}
    \begin{itemize}
        \item Institute a mandatory security awareness training program for all current employees and as part of the onboarding process for new hires.
        \item Conduct annual refresher training and periodic phishing simulations.
    \end{itemize}
    \item \textbf{Establish an Acceptable Use Policy (AUP):}
    \begin{itemize}
        \item Draft a formal AUP that defines the rules for using company IT assets.
        \item Require all employees to read and acknowledge the policy.
    \end{itemize}
\end{enumerate}

\subsection{Medium Priority (Within 90 Days)}
\begin{enumerate}
    \item \textbf{Conduct a Comprehensive Security Assessment:} Engage a third party to perform a full internal and external penetration test to identify any other unknown vulnerabilities or misconfigurations.
\end{enumerate}

\end{document}
```