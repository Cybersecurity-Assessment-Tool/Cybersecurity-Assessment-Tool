```latex
\documentclass[12pt]{article}

% 1. DOCUMENT SETUP & PACKAGES
\usepackage[margin=1in]{geometry}
\usepackage{pifont} % For checkmarks and crosses
\usepackage{booktabs} % For professional tables
\usepackage{hyperref} % For clickable links
\usepackage{url} % For URL formatting
\usepackage{seqsplit} % For splitting long strings in texttt
\usepackage{graphicx}
\usepackage{xcolor}
\usepackage{titling}

% Custom commands for Yes/No indicators
\newcommand{\yes}{\textcolor{green!70!black}{\ding{51}}}
\newcommand{\no}{\textcolor{red}{\ding{55}}}

% Hyperref setup for metadata
\hypersetup{
    colorlinks=true,
    linkcolor=blue,
    filecolor=magenta,      
    urlcolor=cyan,
    pdftitle={Cybersecurity Posture Assessment Report},
    pdfauthor={Cybersecurity Analysis Division},
    pdfsubject={Security Analysis},
    pdfkeywords={Security, Report, Analysis},
}

% 2. TITLE & METADATA
\title{
    \vspace{-1.5cm}
    \includegraphics[width=0.3\textwidth]{example-image-a} \\ % Placeholder for a logo
    \vspace{0.5cm}
    \textbf{Cybersecurity Posture Assessment Report} \\
    \large For: \textbf{Prism Logic}
}
\author{Cybersecurity Analysis Division}
\date{\today}


% 3. DOCUMENT BODY
\begin{document}

\maketitle
\tableofcontents
\newpage

% 4. EXECUTIVE OVERVIEW
\section{Executive Overview}
This report details the findings of a cybersecurity posture assessment for \textbf{Prism Logic}. The analysis is based on a synthesis of network scan data, a self-reported organizational security questionnaire, and a review of pre-existing risk documentation.

The assessment reveals several high-priority risks that require immediate attention. A critical vulnerability, direct Remote Desktop Protocol (RDP) exposure, was identified on a new host (\texttt{10.10.10.51}). This finding confirms a systemic pattern of insecure remote access configurations, as a similar risk was previously documented for another system.

Furthermore, significant gaps in administrative and identity controls were identified. The lack of Multi-Factor Authentication (MFA) on the primary email system (\texttt{PrismLogic.com}) presents a critical risk of account compromise and business email compromise (BEC). This is compounded by the absence of a formal Acceptable Use Policy and a security training program for new employees.

Collectively, these findings indicate a high-risk posture. We strongly recommend immediate remediation of the technical vulnerabilities and the swift implementation of the foundational security controls outlined in this report.

% 5. ORGANIZATIONAL INFORMATION
\section{Organizational Information}
The following information was provided for the assessment.
\begin{table}[h!]
\centering
\begin{tabular}{@{}ll@{}}
\toprule
\textbf{Attribute} & \textbf{Value} \\ \midrule
Organization Name  & \textbf{Prism Logic} \\
Email Domain       & \texttt{PrismLogic.com} \\
Website Domain     & \url{www.PrismLogic.com} \\
External IP Address & \texttt{120.187.206.110} \\ \bottomrule
\end{tabular}
\caption{Client Organizational Data.}
\end{table}

% 6. SECURITY CONTROL REVIEW (QUESTIONNAIRE)
\section{Security Control Review}
The following table summarizes the organization's self-reported status on key security controls. Items marked with \no\ represent significant gaps in the security framework.

\begin{table}[h!]
\centering
\begin{tabular}{@{}p{0.5\linewidth} c p{0.3\linewidth}@{}}
\toprule
\textbf{Control Question} & \textbf{Response} & \textbf{Analyst Note} \\ \midrule
Do you require MFA to access email? & \no & \textbf{Critical Gap.} Increases risk of account takeover and phishing success. \\
\addlinespace
Do you require MFA to log into computers? & \yes & Meets best practice for endpoint security. \\
\addlinespace
Do you require MFA to access sensitive data systems? & \yes & Meets best practice for data protection. \\
\addlinespace
Does your organization have an employee acceptable use policy? & \no & \textbf{High Risk.} Lack of clear governance for employees. \\
\addlinespace
Does your organization do security awareness training for new employees? & \no & \textbf{High Risk.} New hires are a primary target for social engineering. \\
\addlinespace
Does your organization do security awareness training for all employees at least once per year? & \yes & Good practice for maintaining security awareness. \\ \bottomrule
\end{tabular}
\caption{Analysis of Security Questionnaire.}
\end{table}

% 7. TECHNICAL SCAN RESULTS
\section{Technical Scan Results}
An external network scan was performed on the specified target to identify open ports and exposed services.
\begin{itemize}
    \item \textbf{Target IP Address:} \texttt{10.10.10.51}
    \item \textbf{Scan Date:} \textit{Not Provided in Scan Data}
\end{itemize}

The following table details the services discovered to be accessible.

\begin{table}[h!]
\centering
\begin{tabular}{@{}llll@{}}
\toprule
\textbf{Port} & \textbf{State} & \textbf{Service Name} & \textbf{Analysis} \\ \midrule
3389/tcp & Open & \texttt{ms-wbt-server} & \textbf{Critical Finding.} This is the Remote Desktop Protocol (RDP). Direct exposure is a primary vector for brute-force attacks and ransomware. \\ \bottomrule
\end{tabular}
\caption{Open Ports and Services on \texttt{10.10.10.51}.}
\end{table}

% 8. CONSOLIDATED RISK ASSESSMENT
\section{Consolidated Risk Assessment}
This section correlates findings from the technical scan, control review, and pre-existing risk data into a consolidated list of identified risks.

\begin{table}[h!]
\centering
\begin{tabular}{@{}p{0.1\linewidth} p{0.2\linewidth} p{0.4\linewidth} p{0.15\linewidth}@{}}
\toprule
\textbf{Risk ID} & \textbf{Risk Title} & \textbf{Description} & \textbf{Severity} \\ \midrule
\textbf{RISK-001} & Systemic RDP Exposure & RDP is exposed on \texttt{10.10.10.51} (new finding) and was previously documented on \texttt{10.10.10.50}. This indicates a pattern of insecure remote access configuration. & \textbf{Critical (9.8)} \\
\addlinespace
\textbf{RISK-002} & No MFA on Email Infrastructure & The lack of MFA on the \texttt{PrismLogic.com} email domain makes it highly susceptible to credential theft, phishing, and Business Email Compromise (BEC). & \textbf{Critical (9.1)} \\
\addlinespace
\textbf{RISK-003} & Foundational Policy Gaps & The absence of an Acceptable Use Policy and security training for new hires creates an environment where employees are unaware of their security responsibilities, increasing human-related risk. & \textbf{High (7.5)} \\ \bottomrule
\end{tabular}
\caption{Summary of Identified Risks.}
\end{table}

% 9. RECOMMENDATIONS
\section{Recommendations}
Based on the analysis, we recommend the following actions, prioritized by urgency.

\subsection{Immediate Actions (To Be Completed Within 7 Days)}
\begin{enumerate}
    \item \textbf{Remediate RDP Exposure on \texttt{10.10.10.51}:} Immediately implement a firewall rule to block all inbound traffic to TCP port 3389 from the public internet. Access should only be permitted via a secure VPN.
    \item \textbf{Enforce MFA for All Email Accounts:} Enable and enforce MFA for all users on the \texttt{PrismLogic.com} email domain without exception.
\end{enumerate}

\subsection{Short-Term Actions (To Be Completed Within 90 Days)}
\begin{enumerate}
    \item \textbf{Conduct a Network-Wide Port Scan:} Perform a comprehensive audit of all external-facing IP addresses to identify and remediate any other instances of exposed RDP or other risky services.
    \item \textbf{Develop and Implement an Acceptable Use Policy (AUP):} Create a formal AUP document that clearly defines the rules for using company IT assets. Require all employees to read and acknowledge the policy.
    \item \textbf{Establish a Security Onboarding Program:} Develop a mandatory security awareness training module for all new employees to be completed during their first week of employment.
\end{enumerate}

\subsection{Strategic Recommendations (Long-Term)}
\begin{enumerate}
    \item \textbf{Deploy a Corporate VPN:} Implement a robust Virtual Private Network (VPN) solution as the standard method for all remote access to internal network resources. This eliminates the need for direct port exposure.
    \item \textbf{Establish a Formal Vulnerability Management Program:} Create a continuous process for identifying, assessing, and remediating vulnerabilities across the entire organization to move from a reactive to a proactive security posture.
\end{enumerate}

\end{document}
```