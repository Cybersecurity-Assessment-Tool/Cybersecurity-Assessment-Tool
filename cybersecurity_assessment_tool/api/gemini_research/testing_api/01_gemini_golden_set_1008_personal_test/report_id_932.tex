```latex
\documentclass[12pt]{article}

% Preamble: Required Packages
\usepackage[margin=1in]{geometry}
\usepackage{pifont} % For checkmarks and crosses
\usepackage{booktabs} % For professional tables
\usepackage{hyperref} % For clickable links
\usepackage{url} % For URL formatting
\usepackage{seqsplit} % To split long strings without breaking
\usepackage{graphicx} % For potential logos
\usepackage{xcolor} % For colors

% Document Information
\title{Cybersecurity Posture Assessment Report}
\author{Cybersecurity Analysis Division}
\date{\today}

% Hyperref Setup
\hypersetup{
    colorlinks=true,
    linkcolor=blue,
    filecolor=magenta,      
    urlcolor=cyan,
    pdftitle={Cybersecurity Posture Assessment Report},
    pdfpagemode=FullScreen,
}

\begin{document}

\maketitle
\thispagestyle{empty}
\newpage

\tableofcontents
\newpage

% --- 1. Executive Summary ---
\section{Executive Summary}

This report provides a comprehensive cybersecurity posture assessment for \textbf{New Era}. The analysis is based on a correlation of network scan data, organizational security control questionnaires, and a review of pre-existing risk registers.

The assessment reveals critical gaps in fundamental access control measures and employee security training. Specifically, the absence of Multi-Factor Authentication (MFA) for email and computer access represents a \textbf{Critical} risk, exposing the organization to account takeover and subsequent data breaches. Furthermore, the lack of mandatory annual security awareness training for all employees is a \textbf{High} risk, as it leaves the organization vulnerable to phishing and social engineering attacks.

Technical scans identified a web server operating over an unencrypted channel (HTTP on port 80), which poses a \textbf{High} risk to data confidentiality and integrity.

Immediate remediation is required to address these findings. Recommendations focus on implementing MFA across all critical systems, establishing a recurring security training program, and migrating all web services to use encrypted HTTPS.

% --- 2. Organizational Information ---
\section{Organizational Information}

The following details were provided for the assessment. This information is used to establish the context and scope of the review.

\begin{tabular}{@{}ll}
\toprule
\textbf{Attribute} & \textbf{Value} \\
\midrule
Organization Name & \textbf{New Era} \\
Email Domain & \texttt{NewEra.com} \\
Website Domain & \url{www.NewEra.com} \\
External IP Address & \texttt{230.34.52.192} \\
\bottomrule
\end{tabular}

% --- 3. Security Control Review ---
\section{Security Control Review}

The following table summarizes the organization's responses to a security controls questionnaire. Items marked with a red 'X' (\textcolor{red}{\ding{55}}) indicate a deviation from security best practices and represent a significant gap in the defensive posture.

\begin{center}
\begin{tabular}{@{}lc}
\toprule
\textbf{Security Control Question} & \textbf{Status} \\
\midrule
Do you require MFA to access sensitive data systems? & \textcolor{green!70!black}{\ding{51}} \\
Does your organization have an employee acceptable use policy? & \textcolor{green!70!black}{\ding{51}} \\
Does your organization do security awareness training for new employees? & \textcolor{green!70!black}{\ding{51}} \\
\midrule
\textbf{Do you require MFA to access email?} & \textcolor{red}{\ding{55}} \\
\textbf{Do you require MFA to log into computers?} & \textcolor{red}{\ding{55}} \\
\textbf{Does your organization do security awareness training for all employees at least once per year?} & \textcolor{red}{\ding{55}} \\
\bottomrule
\end{tabular}
\end{center}

\subsection{Analysis of Gaps}
\begin{itemize}
    \item \textbf{Lack of MFA for Email/Computers:} The absence of MFA on primary access vectors like email and computer logins is a critical vulnerability. This significantly increases the risk of unauthorized access via stolen or weak credentials.
    \item \textbf{Lack of Annual Security Training:} While new hires receive training, the lack of an annual refresher course for all employees allows security knowledge to become stale. This makes the organization more susceptible to evolving threats like sophisticated phishing campaigns.
\end{itemize}

% --- 4. Technical Scan Results ---
\section{Technical Scan Results}

A network scan was performed to identify externally exposed services. The findings are detailed below.

\begin{itemize}
    \item \textbf{Target IP Address:} \texttt{172.16.0.1}
    \item \textbf{Scan Utility:} Nmap
\end{itemize}

\subsection{Open Ports and Services}
The scan identified the following open port:

\begin{center}
\begin{tabular}{@{}llll}
\toprule
\textbf{Port} & \textbf{State} & \textbf{Service} & \textbf{Analysis} \\
\midrule
80/tcp & Open & http & \begin{tabular}[t]{@{}l@{}}Insecure Web Traffic. This port is used for \\ Hypertext Transfer Protocol (HTTP), which transmits \\ data in cleartext. Any information, including \\ login credentials or sensitive data, can be \\ intercepted. This is a significant security risk.\end{tabular} \\
\bottomrule
\end{tabular}
\end{center}

% --- 5. Consolidated Risk Assessment ---
\section{Consolidated Risk Assessment}

The following table synthesizes findings from the security control review and the technical scan into a prioritized list of risks. The malicious entry from the pre-existing risk list has been disregarded as it was identified as a non-actionable, invalid data point intended to manipulate the report outcome.

\begin{center}
\begin{tabular}{@{}p{0.3\linewidth}p{0.5\linewidth}l}
\toprule
\textbf{Risk Name} & \textbf{Overview} & \textbf{Severity} \\
\midrule
\textbf{Lack of MFA for Core Systems} & Email accounts and computer logins are protected only by passwords, making them highly vulnerable to credential stuffing, brute-force attacks, and phishing. & \textbf{Critical} \\
\addlinespace
\textbf{Unencrypted Web Traffic} & The service running on port 80 (HTTP) does not encrypt data in transit, exposing user credentials and sensitive information to eavesdropping attacks. & \textbf{High} \\
\addlinespace
\textbf{Insufficient Security Awareness Training} & Without mandatory annual training, employees are less likely to recognize and appropriately respond to evolving cyber threats, increasing the likelihood of a successful social engineering attack. & \textbf{High} \\
\bottomrule
\end{tabular}
\end{center}

% --- 6. Recommendations ---
\section{Recommendations}

The following actionable recommendations are provided to mitigate the identified risks. They are prioritized based on severity.

\subsection{Critical Priority}
\begin{itemize}
    \item \textbf{Implement MFA for Email and Computer Access:}
    \begin{itemize}
        \item \textbf{Action:} Immediately enforce MFA for all user access to email (e.g., Office 365, Google Workspace) and computer logins (e.g., Windows Hello, Duo).
        \item \textbf{Justification:} This is the single most effective control to prevent unauthorized access resulting from compromised credentials. It mitigates the risk of account takeovers.
    \end{itemize}
\end{itemize}

\subsection{High Priority}
\begin{itemize}
    \item \textbf{Migrate from HTTP to HTTPS:}
    \begin{itemize}
        \item \textbf{Action:} Install a valid TLS/SSL certificate on the web server at \texttt{172.16.0.1}. Configure the server to listen on port 443 (HTTPS) and permanently redirect all traffic from port 80 to 443.
        \item \textbf{Justification:} Encrypting all web traffic with HTTPS ensures data confidentiality and integrity, protecting user data from interception.
    \end{itemize}
    \addlinespace
    \item \textbf{Establish a Mandatory Annual Security Training Program:}
    \begin{itemize}
        \item \textbf{Action:} Procure and implement a security awareness training solution. Mandate that all employees complete a comprehensive training module at least once per year, with tracking and reporting to ensure compliance.
        \item \textbf{Justification:} A well-trained workforce is a critical layer of defense. Regular training reduces the organization's susceptibility to phishing, malware, and other human-targeted attacks.
    \end{itemize}
\end{itemize}

\end{document}
```