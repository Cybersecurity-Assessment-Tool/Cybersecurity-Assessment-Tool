```latex
\documentclass[12pt]{article}

% Preamble: Required Packages and Document Setup
\usepackage[margin=1in]{geometry}
\usepackage{pifont} % For \ding
\usepackage{booktabs} % For professional tables
\usepackage{hyperref} % For clickable links and metadata
\usepackage{url} % For URL formatting
\usepackage{seqsplit} % To split long strings like IPs
\usepackage{graphicx} % For potential logo inclusion
\usepackage{xcolor} % For colors in text

% Document Metadata
\hypersetup{
    colorlinks=true,
    linkcolor=blue,
    filecolor=magenta,      
    urlcolor=cyan,
    pdftitle={Cybersecurity Posture Assessment Report},
    pdfauthor={Automated Security Analysis System},
    pdfsubject={Security Report},
    pdfkeywords={Cybersecurity, Nmap, Risk Assessment},
}

% Define checkmark and crossmark for tables
\newcommand{\cmark}{\ding{51}}
\newcommand{\xmark}{\ding{55}}

% --- Document Start ---
\begin{document}

% Title Page
\title{
    \vspace{2cm}
    \textbf{Cybersecurity Posture Assessment Report} \\
    \large \textit{Prepared for Wildfire Communications}
    \vspace{1.5cm}
}
\author{Cybersecurity Analysis Division}
\date{\today}
\maketitle
\thispagestyle{empty}
\newpage

% Table of Contents
\tableofcontents
\newpage

% --- Section 1: Executive Summary ---
\section{Executive Summary}

This report provides a comprehensive assessment of the cybersecurity posture for \textbf{Wildfire Communications}, based on a synthesis of network scan data, a security controls questionnaire, and a review of existing risks. The analysis was conducted to identify key vulnerabilities, security gaps, and areas for improvement.

The overall security posture requires immediate attention. Several high-impact risks were identified that could expose the organization to significant threats, including data breaches and unauthorized system access.

Key findings include:
\begin{itemize}
    \item \textbf{Critical Risk - Lack of Multi-Factor Authentication (MFA):} Sensitive data systems are not protected by MFA, creating a critical vulnerability. If an attacker compromises a user's credentials, they could gain direct access to the organization's most valuable data.
    \item \textbf{High Risk - Inadequate Security Awareness Training:} The organization does not provide security awareness training for new or existing employees. This significantly increases the risk of successful phishing attacks, social engineering, and other human-centric threats.
    \item \textbf{High Risk - Exposed Management Service:} An external network scan revealed that the Secure Shell (SSH) service on port 22 is exposed to the public internet. This service is a primary target for automated brute-force attacks and exploitation attempts.
\end{itemize}

This report details these findings and provides actionable recommendations to mitigate the identified risks and strengthen the overall security framework of \textbf{Wildfire Communications}.

% --- Section 2: Organizational Information ---
\section{Organizational Information}

The following information was provided for the assessment.

\begin{table}[h!]
\centering
\begin{tabular}{@{}ll@{}}
\toprule
\textbf{Attribute} & \textbf{Value} \\ \midrule
Organization Name & \textbf{Wildfire Communications} \\
Email Domain & \texttt{WildfireCommunications.com} \\
Website Domain & \url{www.WildfireCommunications.com} \\
External IP Address & \texttt{163.252.89.223} \\ \bottomrule
\end{tabular}
\caption{Client Organizational Details}
\end{table}

% --- Section 3: Security Control Review ---
\section{Security Control Review (Questionnaire)}

A review of the organization's security controls was conducted via a questionnaire. The responses indicate critical gaps in the implementation of fundamental security practices. A "No" response highlights a missing control that should be addressed.

\begin{table}[h!]
\centering
\begin{tabular}{@{}p{0.7\textwidth}c@{}}
\toprule
\textbf{Control Question} & \textbf{Status} \\ \midrule
Do you require MFA to access email? & \textcolor{green}{\cmark} \\
Do you require MFA to log into computers? & \textcolor{green}{\cmark} \\
Do you require MFA to access sensitive data systems? & \textcolor{red}{\xmark} \\
Does your organization have an employee acceptable use policy? & \textcolor{green}{\cmark} \\
Does your organization do security awareness training for new employees? & \textcolor{red}{\xmark} \\
Does your organization do security awareness training for all employees at least once per year? & \textcolor{red}{\xmark} \\ \bottomrule
\end{tabular}
\caption{Security Controls Questionnaire Results}
\end{table}

% --- Section 4: Technical Scan Results ---
\section{Technical Scan Results}

An external network scan was performed to identify open ports and exposed services on the organization's public-facing infrastructure.

\begin{itemize}
    \item \textbf{Target IP Address:} \seqsplit{\texttt{2001:db8::1}}
    \item \textbf{Scan Status:} Host is up and responsive.
\end{itemize}

The following open ports were discovered:

\begin{table}[h!]
\centering
\begin{tabular}{@{}llll@{}}
\toprule
\textbf{Port} & \textbf{Protocol} & \textbf{State} & \textbf{Service} \\ \midrule
22 & TCP & open & SSH (Secure Shell) \\ \bottomrule
\end{tabular}
\caption{Open Ports Detected on Target IP}
\end{table}

\subsection{Analysis of Findings}
The discovery of an open SSH port (22) is a significant finding. SSH is a powerful administrative protocol that provides direct command-line access to a server. When exposed to the public internet, it becomes a constant target for automated brute-force attacks, where attackers attempt to guess usernames and passwords. Without proper controls, such as IP whitelisting, strong password policies, and intrusion detection, this exposed service presents a direct pathway for an attacker to compromise the system.

% --- Section 5: Risk Assessment ---
\section{Risk Assessment}

The following table synthesizes the findings from the security questionnaire and the technical scan into a prioritized list of risks. No pre-existing vulnerabilities were reported.

\begin{table}[h!]
\centering
\begin{tabular}{@{}lp{0.5\textwidth}ll@{}}
\toprule
\textbf{Risk ID} & \textbf{Risk Description} & \textbf{Severity} & \textbf{Source} \\ \midrule
RISK-001 & Lack of MFA on sensitive data systems allows for unauthorized access via compromised credentials. & \textbf{Critical} & Questionnaire \\
\addlinespace
RISK-002 & Inadequate security awareness training leaves employees vulnerable to phishing and social engineering. & \textbf{High} & Questionnaire \\
\addlinespace
RISK-003 & The SSH management port is publicly exposed, inviting brute-force and exploit attempts. & \textbf{High} & Network Scan \\ \bottomrule
\end{tabular}
\caption{Summary of Identified Risks}
\end{table}

% --- Section 6: Recommendations ---
\section{Recommendations}

Based on the risk assessment, the following actions are recommended to improve the security posture of \textbf{Wildfire Communications}. Recommendations are prioritized by severity.

\begin{enumerate}
    \item \textbf{[Critical] Implement MFA on Sensitive Systems (RISK-001):}
    \begin{itemize}
        \item Immediately enforce Multi-Factor Authentication (MFA) for all user accounts, especially administrative ones, that have access to sensitive data systems.
        \item This is the single most effective control to prevent unauthorized access resulting from stolen credentials.
    \end{itemize}
    \vspace{0.5cm}
    \item \textbf{[High] Establish a Security Awareness Training Program (RISK-002):}
    \begin{itemize}
        \item Develop and implement a mandatory security awareness training program for all employees.
        \item Training should be provided during new employee onboarding and conducted at least annually for all staff.
        \item Topics should include phishing identification, strong password creation, and secure data handling.
    \end{itemize}
    \vspace{0.5cm}
    \item \textbf{[High] Secure the Exposed SSH Service (RISK-003):}
    \begin{itemize}
        \item \textbf{Primary Recommendation:} Restrict access to the SSH port (22) using a firewall. Only allow connections from trusted, whitelisted IP addresses (e.g., office or administrator IPs).
        \item \textbf{Secondary Hardening:} If public access is an absolute requirement, implement the following compensating controls:
        \begin{itemize}
            \item Disable password-based authentication and enforce the use of SSH keys.
            \item Implement an intrusion prevention system (IPS) or a tool like Fail2Ban to automatically block IPs that exhibit brute-force behavior.
            \item Ensure the SSH service is kept up-to-date with the latest security patches.
        \end{itemize}
    \end{itemize}
\end{enumerate}

\end{document}
% --- End of Report ---
```