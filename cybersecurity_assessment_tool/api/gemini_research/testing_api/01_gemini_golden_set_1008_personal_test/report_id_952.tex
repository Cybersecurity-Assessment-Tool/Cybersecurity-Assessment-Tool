```latex
\documentclass[12pt]{article}

% Preamble: Required Packages
\usepackage[margin=1in]{geometry}
\usepackage{pifont} % For checkmarks (\ding{51}) and crosses (\ding{55})
\usepackage{booktabs} % For professional-looking tables
\usepackage{hyperref} % For clickable links and references
\usepackage{url}      % For formatting URLs
\usepackage{seqsplit} % To prevent long strings from overflowing
\usepackage[table]{xcolor} % For coloring table cells

% Define custom colors for severity levels
\definecolor{sev_critical}{HTML}{990000}
\definecolor{sev_high}{HTML}{DD4B39}
\definecolor{sev_medium}{HTML}{F4B400}
\definecolor{sev_low}{HTML}{4285F4}
\definecolor{lightgray}{gray}{0.9}

\hypersetup{
    colorlinks=true,
    linkcolor=blue,
    filecolor=magenta,      
    urlcolor=cyan,
}

\begin{document}

% --- Title Page ---
\title{Cybersecurity Posture Assessment Report \\ \large For: Bluebird Bio}
\author{Cybersecurity Analysis Division}
\date{\today}
\maketitle

\newpage

% --- Table of Contents ---
\tableofcontents
\newpage

% --- Section 1: Executive Overview ---
\section{Executive Overview}

This report details the findings of a cybersecurity posture assessment conducted for Bluebird Bio. The assessment combines an analysis of organizational security controls, an external network vulnerability scan, and a review of existing risks.

The overall security posture presents a mixed landscape. On a positive note, the external network scan of the target host \texttt{192.168.1.100} revealed no open ports, suggesting a strong perimeter defense or that no services are exposed on that specific host. The organization has also successfully implemented Multi-Factor Authentication (MFA) for email and computer access, which are foundational security controls.

However, several critical gaps were identified through the security controls questionnaire. The absence of MFA for sensitive data systems represents a significant risk, potentially exposing the organization's most valuable assets to unauthorized access. Furthermore, the complete lack of a security awareness training program for both new and existing employees creates a high susceptibility to social engineering attacks, such as phishing. These policy and procedural gaps are the most pressing concerns identified during this assessment and should be prioritized for remediation.

% --- Section 2: Organizational Information ---
\section{Organizational Information}

The following details were provided for the assessment.

\begin{table}[h!]
\centering
\begin{tabular}{@{}ll@{}}
\toprule
\textbf{Attribute}        & \textbf{Value}                  \\ \midrule
Organization Name         & Bluebird Bio                    \\
Email Domain              & \texttt{BluebirdBio.org}        \\
Website Domain            & \texttt{www.BluebirdBio.org}    \\
External IP Address       & \texttt{100.160.244.143}        \\ \bottomrule
\end{tabular}
\caption{Client Organizational Details}
\end{table}

% --- Section 3: Security Control Review ---
\section{Security Control Review}

A review of key organizational security controls was conducted based on a supplied questionnaire. The results indicate areas of both strength and weakness. Gaps identified with a 'No' response represent significant risks that require immediate attention.

\begin{table}[h!]
\centering
\rowcolors{2}{gray!10}{white}
\begin{tabular}{@{}p{0.7\textwidth}c@{}}
\toprule
\textbf{Control Question}                                                & \textbf{Response} \\ \midrule
Do you require MFA to access email?                                      & \ding{51} (Yes)   \\
Do you require MFA to log into computers?                                & \ding{51} (Yes)   \\
\rowcolor{red!15}
Do you require MFA to access sensitive data systems?                     & \ding{55} (No)    \\
Does your organization have an employee acceptable use policy?           & \ding{51} (Yes)   \\
\rowcolor{red!15}
Does your organization do security awareness training for new employees? & \ding{55} (No)    \\
\rowcolor{red!15}
Does your organization do security awareness training for all employees at least once per year? & \ding{55} (No)    \\ \bottomrule
\end{tabular}
\caption{Security Controls Questionnaire Analysis}
\end{table}

% --- Section 4: Technical Scan Results ---
\section{Technical Scan Results}

A network scan was performed to identify exposed services and potential vulnerabilities on the perimeter.

\subsection{Nmap Scan on \texttt{192.168.1.100}}
The scan results for the target host are summarized below.

\begin{itemize}
    \item \textbf{Target IP:} \texttt{192.168.1.100}
    \item \textbf{Host Status:} Up
    \item \textbf{Finding:} The scan confirmed that the host is online and responsive. However, all scanned ports were found to be in a 'closed' state. This indicates that there are no open, listening services (e.g., web, email, remote access) exposed to the scanner on this host. This is a positive security finding, suggesting effective firewall rules or a lack of externally facing services on this particular system.
\end{itemize}

% --- Section 5: Risk Assessment ---
\section{Risk Assessment}

This section synthesizes findings from the security control review and technical scans. Since no pre-existing vulnerabilities were reported and the network scan found no open ports, the primary risks identified are procedural and policy-based.

\begin{table}[h!]
\centering
\begin{tabular}{@{}p{0.15\textwidth}p{0.65\textwidth}p{0.15\textwidth}@{}}
\toprule
\textbf{Risk ID} & \textbf{Risk Description} & \textbf{Severity} \\ \midrule
\rowcolor{sev_critical!25}
RISK-001 & \textbf{Lack of MFA on Sensitive Systems:} The absence of MFA on systems containing sensitive data (e.g., financial, PII, intellectual property) significantly increases the risk of a data breach from compromised credentials. & \textcolor{sev_critical}{\textbf{Critical}} \\
\addlinespace
\rowcolor{sev_high!25}
RISK-002 & \textbf{No Security Awareness Training Program:} The lack of security training for new and existing employees leaves the organization highly vulnerable to phishing, social engineering, and other human-centric attacks. This is a critical gap in the human firewall. & \textcolor{sev_high}{\textbf{High}} \\
\bottomrule
\end{tabular}
\caption{Summary of Identified Risks}
\end{table}

% --- Section 6: Recommendations ---
\section{Recommendations}

Based on the risks identified in the previous section, the following prioritized recommendations are provided to improve the cybersecurity posture of Bluebird Bio.

\subsection{Remediation for RISK-001 (Critical)}
\begin{itemize}
    \item \textbf{Action:} Implement and enforce Multi-Factor Authentication (MFA) across all systems and applications that store, process, or transmit sensitive data.
    \item \textbf{Details:} Prioritize systems with access to financial records, customer Personally Identifiable Information (PII), and proprietary intellectual property. This single control is one of the most effective measures to prevent unauthorized access.
\end{itemize}

\subsection{Remediation for RISK-002 (High)}
\begin{itemize}
    \item \textbf{Action:} Develop and implement a mandatory security awareness training program for all personnel.
    \item \textbf{Details:} This program should be a mandatory part of the onboarding process for all new employees. It must cover essential topics such as phishing identification, password hygiene, acceptable use of company assets, and incident reporting procedures.
    \item \textbf{Action:} Establish a recurring, annual security awareness training and phishing simulation campaign for all employees.
    \item \textbf{Details:} Continuous education is vital. An annual refresher course, supplemented with periodic phishing tests, will help maintain a high level of security consciousness throughout the organization and measure the effectiveness of the training.
\end{itemize}

\end{document}
```