```latex
\documentclass[12pt]{article}

% Preamble: Required Packages
\usepackage[margin=1in]{geometry}
\usepackage{pifont} % For checkmarks and crosses
\usepackage{booktabs} % For professional tables
\usepackage{hyperref} % For clickable links
\usepackage{url} % For formatting URLs
\usepackage{seqsplit} % For splitting long strings in texttt
\usepackage{xcolor} % For colors

% Document Information
\title{Cybersecurity Posture Assessment Report}
\author{Cybersecurity Analysis Division}
\date{\today}

% Hyperref Setup
\hypersetup{
    colorlinks=true,
    linkcolor=blue,
    filecolor=magenta,      
    urlcolor=cyan,
    pdftitle={Cybersecurity Posture Assessment Report},
    pdfpagemode=FullScreen,
}

\begin{document}

\maketitle
\thispagestyle{empty}
\newpage

\tableofcontents
\thispagestyle{empty}
\newpage

\setcounter{page}{1}

% --- 1. Executive Summary ---
\section{Executive Summary}

This report provides a comprehensive cybersecurity assessment for \textbf{Iron Oak Furniture}, based on network scan data, organizational security questionnaires, and a review of pre-existing risks. The analysis reveals several critical vulnerabilities that expose the organization to significant threats, including ransomware attacks, data breaches, and unauthorized access.

The key findings indicate a reactive and underdeveloped security posture. Critical issues identified include:
\begin{itemize}
    \item \textbf{Systemic Network Exposure:} The technical scan discovered an open Remote Desktop Protocol (RDP) port on a new host (\seqsplit{\texttt{10.10.10.51}}), in addition to a previously known exposure. This indicates a pattern of insecure configuration and a lack of network hardening.
    \item \textbf{Critical Gaps in Access Control:} Multi-Factor Authentication (MFA) is not enforced for accessing email or sensitive data systems. This dramatically increases the risk of account compromise through phishing or credential stuffing attacks.
    \item \textbf{Lack of Foundational Security Policies:} The organization lacks a formal Acceptable Use Policy and does not conduct any security awareness training for its employees. This creates a high-risk environment where human error is likely to lead to a security incident.
\end{itemize}

Immediate remediation is required to address the exposed RDP services and implement MFA. A strategic, long-term effort is necessary to build a foundational cybersecurity program that includes employee training, policy development, and robust vulnerability management.

% --- 2. Organizational Information ---
\section{Organizational Information}

The following information was provided for the assessment.

\begin{tabular}{@{}ll}
\toprule
\textbf{Attribute} & \textbf{Value} \\
\midrule
Organization Name & \textbf{Iron Oak Furniture} \\
Email Domain & \seqsplit{\texttt{IronOakFurniture.com}} \\
Website Domain & \href{http://www.IronOakFurniture.com}{\seqsplit{\texttt{www.IronOakFurniture.com}}} \\
External IP Address & \seqsplit{\texttt{199.78.240.50}} \\
\bottomrule
\end{tabular}

% --- 3. Security Control Review (Questionnaire Analysis) ---
\section{Security Control Review (Questionnaire Analysis)}

A review of the organization's security controls was conducted via a questionnaire. The responses highlight significant gaps in fundamental security practices. A "No" response indicates a missing control that elevates organizational risk.

\begin{table}[h!]
\centering
\caption{Security Control Questionnaire Results}
\begin{tabular}{@{}p{8cm}cc@{}}
\toprule
\textbf{Control Question} & \textbf{Response} & \textbf{Status} \\
\midrule
Do you require MFA to log into computers? & Yes & \textcolor{green}{\ding{51}} \\
Do you require MFA to access email? & No & \textcolor{red}{\ding{55}} \\
Do you require MFA to access sensitive data systems? & No & \textcolor{red}{\ding{55}} \\
Does your organization have an employee acceptable use policy? & No & \textcolor{red}{\ding{55}} \\
Does your organization do security awareness training for new employees? & No & \textcolor{red}{\ding{55}} \\
Does your organization do security awareness training for all employees at least once per year? & No & \textcolor{red}{\ding{55}} \\
\bottomrule
\end{tabular}
\end{table}

\subsection*{Analysis of Gaps}
The lack of MFA on email and sensitive data systems is a critical vulnerability. Email accounts are a primary target for attackers seeking to launch internal phishing campaigns or gain access to confidential information. The absence of an acceptable use policy and security awareness training means that employees are not equipped with the knowledge or guidelines to protect company assets, making them highly susceptible to social engineering attacks.

% --- 4. Technical Scan Results ---
\section{Technical Scan Results}

An Nmap scan was performed on the target host \seqsplit{\texttt{10.10.10.51}} to identify open ports and exposed services.

\begin{table}[h!]
\centering
\caption{Open Port Findings for Host \seqsplit{\texttt{10.10.10.51}}}
\begin{tabular}{@{}llll@{}}
\toprule
\textbf{Port/Protocol} & \textbf{State} & \textbf{Service Name} & \textbf{Analysis} \\
\midrule
3389/tcp & open & \texttt{ms-wbt-server} & High Risk. This is the Microsoft Remote \\
& & & Desktop Protocol (RDP). Exposing RDP \\
& & & directly to a network is a common vector \\
& & & for ransomware and brute-force attacks. \\
\bottomrule
\end{tabular}
\end{table}

\subsection*{Correlation with Existing Risks}
This finding is particularly alarming as it corroborates a pre-existing risk of "RDP Exposure" on another host (\seqsplit{\texttt{10.10.10.50}}). Discovering a second instance on \seqsplit{\texttt{10.10.10.51}} indicates a systemic issue with network configuration and a lack of a consistent server hardening process.

% --- 5. Correlated Risk Assessment ---
\section{Correlated Risk Assessment}

The following table synthesizes findings from the security questionnaire, the technical scan, and pre-existing risk data to provide a holistic view of the organization's risk profile.

\begin{table}[h!]
\centering
\caption{Summary of Identified Risks}
\begin{tabular}{@{}lp{7cm}l@{}}
\toprule
\textbf{Risk Name} & \textbf{Description} & \textbf{Severity} \\
\midrule
\textbf{Systemic RDP Exposure} & Multiple systems (\seqsplit{\texttt{10.10.10.50}}, \seqsplit{\texttt{10.10.10.51}}) have RDP (port 3389) open, creating a direct path for attackers to gain remote access to the internal network. & \textbf{Critical} \\
\\
\textbf{Insufficient Access Controls} & Lack of MFA on critical systems like email and sensitive data repositories allows for account takeover via credential theft or brute-force attacks. & \textbf{Critical} \\
\\
\textbf{Lack of Security Culture} & The absence of an Acceptable Use Policy and any form of security awareness training leaves the organization highly vulnerable to human-centric threats like phishing and social engineering. & \textbf{High} \\
\bottomrule
\end{tabular}
\end{table}

% --- 6. Recommendations ---
\section{Recommendations}

The following actions are recommended to mitigate the identified risks and improve the overall security posture of \textbf{Iron Oak Furniture}. Recommendations are prioritized based on urgency and impact.

\subsection{Immediate Actions (To Be Completed in 0-7 Days)}
\begin{enumerate}
    \item \textbf{Remediate RDP Exposure:} Immediately close port 3389 on hosts \seqsplit{\texttt{10.10.10.51}} and \seqsplit{\texttt{10.10.10.50}}. If remote access is required, it must be placed behind a secure gateway like a VPN.
    \item \textbf{Network-Wide Audit:} Conduct an authenticated scan of the entire internal network to identify any other instances of exposed RDP or other high-risk services.
\end{enumerate}

\subsection{Short-Term Actions (To Be Completed in 1-3 Months)}
\begin{enumerate}
    \item \textbf{Implement MFA:} Prioritize the deployment of MFA for all users on the email system (e.g., Office 365, Google Workspace) and all systems housing sensitive data.
    \item \textbf{Develop Acceptable Use Policy (AUP):} Draft and implement a baseline AUP that clearly defines the rules for using company technology and data. All employees must read and acknowledge the policy.
    \item \textbf{Initial Security Awareness Training:} Enroll all employees in a foundational security awareness training program that covers, at a minimum, phishing, password hygiene, and data handling.
\end{enumerate}

\subsection{Long-Term Strategic Initiatives (To Be Completed in 3-12 Months)}
\begin{enumerate}
    \item \textbf{Establish Secure Remote Access:} Implement a robust and centrally managed remote access solution, such as a corporate VPN or a Zero Trust Network Access (ZTNA) platform, to replace all direct service exposure.
    \item \textbf{Formalize Vulnerability Management:} Develop a formal program to regularly scan for, identify, and remediate vulnerabilities across all organizational assets.
    \item \textbf{Annual Training Program:} Establish a recurring, mandatory security awareness training program for all employees to ensure continued vigilance.
\end{enumerate}

\end{document}
```