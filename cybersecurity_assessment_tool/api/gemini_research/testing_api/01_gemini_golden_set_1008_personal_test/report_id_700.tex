```latex
\documentclass[12pt]{article}

% Preamble: Required Packages
\usepackage[margin=1in]{geometry}
\usepackage{pifont} % For checkmarks and crosses
\usepackage{booktabs} % For professional tables
\usepackage{hyperref} % For clickable links
\usepackage{url} % For URL formatting
\usepackage{seqsplit} % To split long strings in tt font
\usepackage{graphicx}
\usepackage{fancyhdr}
\usepackage{lastpage}
\usepackage{xcolor}

% --- Document Setup ---
\hypersetup{
    colorlinks=true,
    linkcolor=blue,
    filecolor=magenta,      
    urlcolor=cyan,
    pdftitle={Cybersecurity Assessment Report},
    pdfpagemode=FullScreen,
}

% --- Header & Footer ---
\pagestyle{fancy}
\fancyhf{} % Clear all header and footer fields
\fancyhead[L]{Cybersecurity Assessment Report}
\fancyhead[R]{Radiant Life}
\fancyfoot[C]{\thepage\ of \pageref{LastPage}}
\renewcommand{\headrulewidth}{0.4pt}
\renewcommand{\footrulewidth}{0.4pt}

% --- Document Start ---
\begin{document}

% --- Title Page ---
\begin{titlepage}
    \centering
    \vspace*{1cm}
    \Huge
    \textbf{Cybersecurity Assessment Report}
    
    \vspace{1.5cm}
    \Large
    Prepared for: \\
    \vspace{0.5cm}
    \textbf{Radiant Life}
    
    \vfill
    
    \large
    Date of Report: \today
    
\end{titlepage}

\tableofcontents
\newpage

% --- Section 1: Executive Overview ---
\section{Executive Overview}
This report details the findings of a cybersecurity assessment conducted for Radiant Life. The evaluation combined a review of organizational security controls via a questionnaire, an external network scan of a designated target, and an analysis of pre-existing risks.

The assessment identified a mixed security posture. Radiant Life demonstrates a solid foundation in security awareness, with established policies for acceptable use and comprehensive training programs for all employees. Furthermore, the technical network scan of the target host \texttt{192.168.1.100} revealed no open ports, indicating effective network firewalling at the perimeter of that specific system.

However, two critical gaps were identified in the organization's access control policies. The absence of Multi-Factor Authentication (MFA) for computer logins and, more importantly, for accessing sensitive data systems, presents a significant risk. These gaps could allow an attacker with compromised credentials to gain unauthorized access to critical endpoints and data, bypassing primary security measures.

Immediate remediation should focus on implementing a robust MFA solution across all sensitive systems and user endpoints to mitigate the risk of credential-based attacks.

% --- Section 2: Organizational Information ---
\section{Organizational Information}
The following details were provided for the assessment.

\begin{tabular}{@{}ll}
    \toprule
    \textbf{Attribute} & \textbf{Value} \\
    \midrule
    Organization Name & Radiant Life \\
    Email Domain & \texttt{RadiantLife.org} \\
    Website Domain & \url{www.RadiantLife.org} \\
    External IP Address & \texttt{20.116.135.57} \\
    \bottomrule
\end{tabular}

% --- Section 3: Security Control Review ---
\section{Security Control Review}
A questionnaire was completed to assess the current state of administrative and policy-based security controls. The results are summarized below. "No" answers indicate potential gaps in the security framework.

\begin{table}[h!]
\centering
\begin{tabular}{p{0.6\linewidth} c p{0.25\linewidth}}
    \toprule
    \textbf{Control Question} & \textbf{Status} & \textbf{Analyst Notes} \\
    \midrule
    Do you require MFA to access email? & \ding{51} & Commendable. Protects primary communication channel. \\
    \addlinespace
    Do you require MFA to log into computers? & \textbf{\color{red}\ding{55}} & \textbf{High Risk.} Compromised credentials could lead to direct endpoint access. \\
    \addlinespace
    Do you require MFA to access sensitive data systems? & \textbf{\color{red}\ding{55}} & \textbf{Critical Gap.} Exposes crown jewel data to credential theft attacks. \\
    \addlinespace
    Does your organization have an employee acceptable use policy? & \ding{51} & Foundational policy is in place. \\
    \addlinespace
    Does your organization do security awareness training for new employees? & \ding{51} & Good practice for onboarding. \\
    \addlinespace
    Does your organization do security awareness training for all employees at least once per year? & \ding{51} & Excellent. Maintains security consciousness. \\
    \bottomrule
\end{tabular}
\caption{Security Control Questionnaire Results}
\end{table}

% --- Section 4: Technical Scan Results ---
\section{Technical Scan Results}
A network port scan was performed to identify accessible services on the specified target system.

\begin{itemize}
    \item \textbf{Target IP Address:} \texttt{192.168.1.100}
    \item \textbf{Scan Date:} \today
\end{itemize}

\subsection{Summary of Findings}
The scan concluded that the target host was online, but no open TCP ports were discovered. All 1000 scanned ports were reported as "closed".

\textbf{Conclusion:} This is a positive finding. The absence of open ports suggests that the host is protected by a well-configured firewall that denies unsolicited inbound connections from the scanner's network location. No vulnerabilities related to exposed services could be identified.

% --- Section 5: Risk Assessment ---
\section{Risk Assessment}
This section synthesizes findings from the security control review and technical scan. The primary risks identified are related to access control policies rather than technical vulnerabilities on the scanned host.

\begin{table}[h!]
\centering
\begin{tabular}{p{0.1\linewidth} p{0.25\linewidth} p{0.45\linewidth} p{0.1\linewidth}}
    \toprule
    \textbf{ID} & \textbf{Risk Name} & \textbf{Description} & \textbf{Severity} \\
    \midrule
    RISK-001 & Lack of MFA for Sensitive Data Systems & The absence of a second authentication factor for systems holding sensitive data means a single compromised password could lead to a major data breach. & \textbf{Critical} \\
    \addlinespace
    RISK-002 & Lack of MFA for Computer Logins & User endpoints (desktops/laptops) are not protected by MFA. An attacker with valid user credentials could gain full access to an employee's machine and potentially pivot to other network resources. & \textbf{High} \\
    \bottomrule
\end{tabular}
\caption{Summary of Identified Risks}
\end{table}

% --- Section 6: Recommendations ---
\section{Recommendations}
Based on the risks identified in the previous section, the following actions are recommended to improve the cybersecurity posture of Radiant Life. Recommendations are prioritized by severity.

\subsection{Recommendation 1: Implement MFA for Sensitive Systems (RISK-001)}
\begin{itemize}
    \item \textbf{Priority:} \textbf{CRITICAL}
    \item \textbf{Action:} Immediately prioritize the deployment of a robust Multi-Factor Authentication solution for all applications, databases, and administrative interfaces that process or store sensitive organizational data. This is the single most effective control to mitigate the risk of a data breach resulting from compromised credentials.
    \item \textbf{Suggested Solutions:} Consider modern MFA methods such as Time-based One-Time Passwords (TOTP) from authenticator apps, push notifications, or hardware security keys (FIDO2/WebAuthn).
\end{itemize}

\subsection{Recommendation 2: Enforce MFA for Endpoint Logon (RISK-002)}
\begin{itemize}
    \item \textbf{Priority:} \textbf{HIGH}
    \item \textbf{Action:} Roll out MFA for all employee computer logins (Windows, macOS, etc.). This creates a significant barrier for attackers attempting to gain a foothold in the network via stolen credentials and protects against unauthorized physical access.
    \item \textbf{Implementation Note:} This can often be integrated with existing identity providers (e.g., Azure AD, Okta) and endpoint management solutions.
\end{itemize}

\end{document}
```