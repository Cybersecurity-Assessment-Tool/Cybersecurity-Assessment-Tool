```latex
\documentclass[12pt]{article}

% --- PACKAGES ---
\usepackage[a4paper, margin=1in]{geometry}
\usepackage{pifont} % For checkmarks and crosses
\usepackage{booktabs} % For professional tables
\usepackage{hyperref} % For clickable links and ToC
\usepackage{url} % For formatting URLs
\usepackage{seqsplit} % To split long strings in tt font
\usepackage{graphicx}
\usepackage{xcolor}
\usepackage{fancyhdr}

% --- DOCUMENT SETUP ---
\hypersetup{
    colorlinks=true,
    linkcolor=blue,
    filecolor=magenta,      
    urlcolor=cyan,
    pdftitle={Cybersecurity Assessment Report},
    pdfpagemode=FullScreen,
}

\pagestyle{fancy}
\fancyhf{}
\fancyhead[L]{Cybersecurity Assessment Report}
\fancyhead[R]{Pacific Rim Exports}
\fancyfoot[C]{\thepage}

% --- COMMANDS ---
\newcommand{\yes}{\ding{51}} % Checkmark
\newcommand{\no}{\ding{55}}  % Cross

% Define severity colors
\definecolor{sev_critical}{HTML}{990000}
\definecolor{sev_high}{HTML}{D14302}
\definecolor{sev_medium}{HTML}{E8A600}
\definecolor{sev_low}{HTML}{0066CC}

\newcommand{\severity}[2]{%
  \ifstrequal{#1}{Critical}{\colorbox{sev_critical}{\textcolor{white}{\textbf{\strut#2}}}}%
  {\ifstrequal{#1}{High}{\colorbox{sev_high}{\textcolor{white}{\textbf{\strut#2}}}}%
  {\ifstrequal{#1}{Medium}{\colorbox{sev_medium}{\textcolor{black}{\textbf{\strut#2}}}}%
  {\ifstrequal{#1}{Low}{\colorbox{sev_low}{\textcolor{white}{\textbf{\strut#2}}}}%
  {\textbf{#2}}}}}%
}
\usepackage{etoolbox} % Required for \ifstrequal

% --- DOCUMENT START ---
\begin{document}

% --- TITLE PAGE ---
\begin{titlepage}
    \centering
    \vspace*{1cm}
    \Huge
    \textbf{Cybersecurity Assessment Report}
    
    \vspace{1.5cm}
    \Large
    Prepared for: \\
    \vspace{0.5cm}
    \textbf{Pacific Rim Exports}
    
    \vspace{2cm}
    \large
    Report Date: \today
    
    \vfill
    
    \large
    \textbf{Confidential}
    
    \vspace{0.8cm}
    \small
    This document contains sensitive information. Access, distribution, and use are restricted to authorized personnel only.
    
\end{titlepage}

\tableofcontents
\newpage

% --- EXECUTIVE SUMMARY ---
\section{Executive Summary}
This report details the findings of a cybersecurity assessment conducted for Pacific Rim Exports. The assessment combined a review of organizational security controls, an automated network scan of internal assets, and a correlation with pre-existing risk data.

The analysis revealed several critical and high-risk findings that require immediate attention. A key technical finding was the discovery of an open Remote Desktop Protocol (RDP) port on an internal server (\texttt{10.10.10.51}). This finding, when correlated with a pre-existing risk of RDP exposure on another host, indicates a systemic weakness in network hardening and access control.

Furthermore, the security control review identified significant gaps in foundational security practices. The lack of multi-factor authentication (MFA) for sensitive data systems, the absence of a formal Acceptable Use Policy, and the failure to conduct annual security awareness training for all employees represent major vulnerabilities. These policy and procedural gaps substantially increase the organization's susceptibility to compromise, particularly when combined with technical vulnerabilities like exposed RDP.

Urgent remediation is recommended to address these issues, focusing on securing remote access pathways, implementing mandatory MFA, and formalizing security policies and training programs.

% --- ORGANIZATIONAL INFORMATION ---
\section{Organizational Information}
The following details were provided for the assessment. This information establishes the context and scope of the review.

\begin{table}[h!]
\centering
\begin{tabular}{@{}ll@{}}
\toprule
\textbf{Attribute} & \textbf{Value} \\ \midrule
Organization Name & Pacific Rim Exports \\
Email Domain      & \texttt{PacificRimExports.org} \\
Website Domain    & \url{www.PacificRimExports.org} \\
External IP       & \texttt{89.196.220.78} \\ \bottomrule
\end{tabular}
\caption{Client Organizational Data}
\label{tab:org_data}
\end{table}

% --- SECURITY CONTROL REVIEW ---
\section{Security Control Review}
A questionnaire was used to evaluate the implementation of fundamental security controls. The responses are summarized below. Answers marked with a \no\ represent significant gaps in the organization's defensive posture.

\begin{table}[h!]
\centering
\begin{tabular}{@{}p{0.8\textwidth}c@{}}
\toprule
\textbf{Control Question} & \textbf{Status} \\ \midrule
Do you require MFA to access email? & \yes \\
Do you require MFA to log into computers? & \yes \\
Do you require MFA to access sensitive data systems? & \no \\
Does your organization have an employee acceptable use policy? & \no \\
Does your organization do security awareness training for new employees? & \yes \\
Does your organization do security awareness training for all employees at least once per year? & \no \\ \bottomrule
\end{tabular}
\caption{Security Controls Questionnaire Results}
\label{tab:controls}
\end{table}

\paragraph{Analysis:} The lack of MFA for sensitive data systems is a critical weakness, leaving high-value assets protected only by passwords. The absence of an Acceptable Use Policy and annual security training for all staff creates an environment where employees are more likely to engage in risky behavior, fall victim to social engineering, or be unaware of their security responsibilities.

% --- TECHNICAL SCAN RESULTS ---
\section{Technical Scan Results}
An Nmap scan was performed on the target system to identify open ports and running services.

\begin{table}[h!]
\centering
\begin{tabular}{@{}llll@{}}
\toprule
\textbf{Target IP} & \textbf{Port} & \textbf{State} & \textbf{Service} \\ \midrule
\texttt{10.10.10.51} & 3389 & open & ms-wbt-server (RDP) \\ \bottomrule
\end{tabular}
\caption{Nmap Scan Findings}
\label{tab:nmap}
\end{table}

\paragraph{Analysis:} The scan identified that port \textbf{3389/TCP} is open, which corresponds to the Microsoft Remote Desktop Protocol (RDP). Unrestricted RDP access is a highly common attack vector used by threat actors to gain initial access to a network, often leading to ransomware deployment or data exfiltration. This finding is especially concerning given the MFA gaps identified in the control review.

% --- CORRELATED RISK ASSESSMENT ---
\section{Correlated Risk Assessment}
The following table synthesizes findings from the security control review, the technical scan, and pre-existing risk data into a prioritized list of risks.

\begin{table}[h!]
\centering
\resizebox{\textwidth}{!}{%
\begin{tabular}{@{}lp{0.4\textwidth}ll@{}}
\toprule
\textbf{Risk ID} & \textbf{Risk Title \& Description} & \textbf{Severity} & \textbf{Affected Systems} \\ \midrule
\textbf{RISK-01} & \textbf{Systemic RDP Exposure} \newline A new instance of open RDP was found on \texttt{10.10.10.51}. This correlates with a known issue on \texttt{10.10.10.50}, indicating a pattern of insecure configuration for remote administration. & \severity{Critical}{Critical} & \texttt{10.10.10.51}, \texttt{10.10.10.50} \\
\addlinespace
\textbf{RISK-02} & \textbf{Lack of MFA on Sensitive Systems} \newline Critical data systems are not protected by MFA. This allows an attacker with valid credentials (e.g., obtained via phishing or brute-force) to gain direct access to high-value assets. & \severity{Critical}{Critical} & All sensitive data systems \\
\addlinespace
\textbf{RISK-03} & \textbf{Inadequate Security Training Program} \newline Security training is not conducted annually for all employees. This leads to a decline in security awareness, making the organization more vulnerable to phishing and social engineering attacks. & \severity{High}{High} & All Employees \\
\addlinespace
\textbf{RISK-04} & \textbf{Missing Acceptable Use Policy (AUP)} \newline The absence of a formal AUP creates ambiguity regarding secure practices and employee responsibilities, weakening the organization's legal and security stance. & \severity{High}{High} & Organization-wide Policy \\
\bottomrule
\end{tabular}%
}
\caption{Synthesized Risk Register}
\label{tab:risks}
\end{table}

% --- RECOMMENDATIONS ---
\section{Recommendations}
The following actions are recommended to mitigate the identified risks. Recommendations are prioritized based on severity.

\subsection{Remediation for RISK-01: Systemic RDP Exposure (Critical)}
\begin{itemize}
    \item \textbf{Immediate Action:} Implement firewall rules to block all access to TCP port 3389 on \texttt{10.10.10.51} and \texttt{10.10.10.50} from any untrusted network. If remote access is required, restrict it to specific, authorized IP addresses only.
    \item \textbf{Long-Term Strategy:} Deploy a Virtual Private Network (VPN) or a Zero Trust Network Access (ZTNA) solution for all remote administrative access. This ensures that all connections are authenticated and encrypted through a secure gateway, rather than exposing services directly.
\end{itemize}

\subsection{Remediation for RISK-02: Lack of MFA on Sensitive Systems (Critical)}
\begin{itemize}
    \item \textbf{Immediate Action:} Identify all systems classified as containing sensitive data. Procure and deploy an MFA solution for these systems as the highest priority.
    \item \textbf{Long-Term Strategy:} Develop a policy that mandates MFA for all remote access, privileged user accounts, and access to sensitive data as a baseline security requirement.
\end{itemize}

\subsection{Remediation for RISK-03: Inadequate Security Training (High)}
\begin{itemize}
    \item \textbf{Immediate Action:} Schedule mandatory annual security awareness training for all current employees. Ensure the training covers topics such as phishing, password security, and safe internet use.
    \item \textbf{Long-Term Strategy:} Establish a continuous security awareness program that includes annual training, regular phishing simulations, and periodic security newsletters or reminders.
\end{itemize}

\subsection{Remediation for RISK-04: Missing Acceptable Use Policy (High)}
\begin{itemize}
    \item \textbf{Immediate Action:} Draft an Acceptable Use Policy (AUP) that clearly defines the rules for using company IT assets, data handling, and internet access.
    \item \textbf{Long-Term Strategy:} Ratify the AUP and integrate it into the new employee onboarding process. Require all existing employees to read and formally acknowledge the policy. Review and update the policy annually.
\end{itemize}

\end{document}
```