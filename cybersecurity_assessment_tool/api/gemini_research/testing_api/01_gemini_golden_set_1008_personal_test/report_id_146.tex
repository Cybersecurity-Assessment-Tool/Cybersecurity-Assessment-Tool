```latex
\documentclass[12pt]{article}

% --- PACKAGES ---
\usepackage[margin=1in]{geometry}
\usepackage{pifont} % Required for \ding
\usepackage{booktabs} % For professional-looking tables
\usepackage{hyperref}
\usepackage{url}
\usepackage{seqsplit} % To break long strings in \texttt
\usepackage{xcolor} % For custom colors
\usepackage{graphicx}
\usepackage[T1]{fontenc}

% --- HYPERREF SETUP ---
\hypersetup{
    colorlinks=true,
    linkcolor=blue,
    filecolor=magenta,
    urlcolor=cyan,
    pdftitle={Cybersecurity Assessment Report},
    pdfauthor={Cybersecurity Analysis Division},
}

% --- CUSTOM COMMANDS ---
\newcommand{\yes}{\textcolor{green}{\ding{51}}} % Green checkmark
\newcommand{\no}{\textcolor{red}{\ding{55}}}   % Red X

% --- DOCUMENT START ---
\begin{document}

% --- TITLE PAGE ---
\begin{titlepage}
    \centering
    \vspace*{1cm}
    \Huge\textbf{Cybersecurity Assessment Report}
    \vspace{1.5cm}
    \Large\textbf{Prepared for:} \\
    \vspace{0.5cm}
    \huge Hidden Gem
    \vfill
    \large\textbf{Date of Report:} \\
    \vspace{0.5cm}
    \large \today
    \vspace{2cm}
    \large \textbf{Analysis Conducted By:} \\
    \vspace{0.5cm}
    \large Cybersecurity Analysis Division
\end{titlepage}

\tableofcontents
\newpage

% --- EXECUTIVE SUMMARY ---
\section*{Executive Summary}
This report details the findings of a cybersecurity assessment conducted for Hidden Gem. The assessment combined a review of organizational security controls, an external network vulnerability scan, and an analysis of pre-existing risks.

The analysis revealed several critical and high-risk vulnerabilities that require immediate attention. A key technical finding is a publicly accessible FTP server running \textbf{vsftpd version 2.3.4}, a version known to contain a critical backdoor vulnerability (CVE-2011-2523). Furthermore, this server permits anonymous logins, posing a significant data exfiltration and system compromise risk.

Procedurally, the assessment identified significant gaps in security controls. The lack of multi-factor authentication (MFA) on employee computers, coupled with the absence of an acceptable use policy and a formal security awareness training program, substantially weakens the organization's defense against common cyber threats like phishing and malware.

These new findings, when correlated with the existing risk of outdated Windows 7 workstations, paint a picture of a high-risk environment. Recommendations have been prioritized to address the most critical issues first, focusing on securing the network perimeter and strengthening internal security policies.

% --- ORGANIZATIONAL INFORMATION ---
\section{Organizational Information}
The following details were provided for the assessment. This information helps establish the context and scope of the review.

\begin{tabular}{@{}ll}
\toprule
\textbf{Attribute} & \textbf{Value} \\
\midrule
Organization Name & Hidden Gem \\
Email Domain & HiddenGem.org \\
Website Domain & www.HiddenGem.org \\
External IP Address & \texttt{12.170.10.53} \\
\bottomrule
\end{tabular}

% --- SECURITY CONTROL REVIEW ---
\section{Security Control Review}
A review of administrative and procedural security controls was conducted based on a questionnaire. The responses indicate critical gaps in foundational security practices.

\begin{table}[h!]
\centering
\caption{Security Controls Questionnaire Analysis}
\begin{tabular}{@{}p{0.6\linewidth}cp{0.2\linewidth}@{}}
\toprule
\textbf{Control Question} & \textbf{Response} & \textbf{Assessment} \\
\midrule
Do you require MFA to access email? & \yes & Meets best practice. \\
Do you require MFA to log into computers? & \no & \textbf{Critical Gap}. Lack of endpoint MFA increases risk of unauthorized access. \\
Do you require MFA to access sensitive data systems? & \yes & Meets best practice. \\
Does your organization have an employee acceptable use policy? & \no & \textbf{High Risk}. Lack of clear policy leads to inconsistent and unsafe user behavior. \\
Does your organization do security awareness training for new employees? & \no & \textbf{High Risk}. New staff are not equipped to identify or report security threats. \\
Does your organization do security awareness training for all employees at least once per year? & \no & \textbf{High Risk}. Security posture degrades over time without recurrent training. \\
\bottomrule
\end{tabular}
\end{table}

% --- TECHNICAL SCAN RESULTS ---
\section{Technical Scan Results}
A network scan was performed on the target IP address \texttt{10.0.0.15}. The scan identified one open port with a critically vulnerable service.

\begin{table}[h!]
\centering
\caption{Open Port Analysis}
\begin{tabular}{@{}lllll@{}}
\toprule
\textbf{Port} & \textbf{State} & \textbf{Service} & \textbf{Version} & \textbf{Details \& Analysis} \\
\midrule
21/tcp & Open & ftp & vsftpd 2.3.4 & \parbox[t]{0.4\linewidth}{\textbf{CRITICAL VULNERABILITY}. This version is affected by CVE-2011-2523, a well-known backdoor. Additionally, the scan confirmed that \textbf{Anonymous FTP login is allowed}, permitting unauthenticated access.} \\
\bottomrule
\end{tabular}
\end{table}

% --- CONSOLIDATED RISK ASSESSMENT ---
\section{Consolidated Risk Assessment}
This section synthesizes findings from the technical scan, control review, and pre-existing risk data into a unified risk register.

\begin{table}[h!]
\centering
\caption{Summary of Identified Risks}
\begin{tabular}{@{}p{0.3\linewidth}p{0.5\linewidth}l@{}}
\toprule
\textbf{Risk Name} & \textbf{Overview} & \textbf{Severity} \\
\midrule
\textbf{Vulnerable Public FTP Server} & An internet-facing FTP server is running a version with a known backdoor (CVE-2011-2523) and allows anonymous login. This could lead to a full system compromise. & \textbf{Critical} \\
\addlinespace
\textbf{Insufficient MFA Implementation} & MFA is not enforced for computer logins, leaving endpoints vulnerable to compromise via stolen credentials. & \textbf{High} \\
\addlinespace
\textbf{Lack of Security Policies \& Training} & The absence of an Acceptable Use Policy and any security awareness training program leaves the organization highly susceptible to phishing, social engineering, and insider threats. & \textbf{High} \\
\addlinespace
\textbf{Outdated Windows Policy} & (Pre-existing) Workstations are running Windows 7, an end-of-life operating system that no longer receives security updates, making them vulnerable to exploitation. & \textbf{Medium} \\
\bottomrule
\end{tabular}
\end{table}

% --- RECOMMENDATIONS ---
\section{Recommendations}
The following actions are recommended to mitigate the identified risks, prioritized by severity.

\begin{enumerate}
    \item \textbf{Remediate Vulnerable FTP Server (Priority: CRITICAL - Immediate Action Required)}
    \begin{itemize}
        \item Immediately take the FTP server offline.
        \item If the FTP service is business-critical, upgrade the vsftpd software to the latest stable version.
        \item Disable anonymous FTP access. Access should be restricted to authenticated users and logged.
        \item Consider replacing FTP with a more secure file transfer protocol like SFTP (SSH File Transfer Protocol).
    \end{itemize}
    \vspace{0.5cm}
    \item \textbf{Implement Comprehensive MFA (Priority: HIGH)}
    \begin{itemize}
        \item Enforce MFA for all employee computer logins (Windows, macOS, etc.) to protect against credential theft.
        \item Review all systems and ensure MFA is applied consistently for access to any sensitive data.
    \end{itemize}
    \vspace{0.5cm}
    \item \textbf{Establish Security Governance and Training (Priority: HIGH)}
    \begin{itemize}
        \item Develop and implement a formal Acceptable Use Policy (AUP) that all employees must read and sign.
        \item Institute a mandatory security awareness training program for all new hires.
        \item Conduct annual security awareness training and phishing simulations for all staff to ensure continuous education.
    \end{itemize}
    \vspace{0.5cm}
    \item \textbf{Address Outdated Systems (Priority: MEDIUM)}
    \begin{itemize}
        \item Continue with the existing plan to upgrade all Windows 7 workstations to a modern, supported operating system such as Windows 10 or 11.
    \end{itemize}
\end{enumerate}

\end{document}
```