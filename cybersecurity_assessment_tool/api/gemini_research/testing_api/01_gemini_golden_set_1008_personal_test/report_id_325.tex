```latex
\documentclass[12pt]{article}

% Preamble: Required Packages
\usepackage[margin=1in]{geometry}
\usepackage{pifont} % For checkmarks and crosses
\usepackage{booktabs} % For professional tables
\usepackage{hyperref} % For hyperlinks
\usepackage{url} % For formatting URLs
\usepackage{seqsplit} % For splitting long strings in texttt
\usepackage[T1]{fontenc}

% Document Metadata
\title{Cybersecurity Assessment Report}
\author{Cybersecurity Analysis Division}
\date{November 22, 2025}

\begin{document}

\maketitle
\thispagestyle{empty}
\newpage
\tableofcontents
\thispagestyle{empty}
\newpage
\setcounter{page}{1}

% ==============================================================================
% Section 1: Executive Summary
% ==============================================================================
\section{Executive Summary}

This report provides a comprehensive cybersecurity assessment for \textbf{Moxie Marketing}, conducted on November 22, 2025. The analysis is based on a synthesis of network scan data, an organizational security questionnaire, and a review of pre-existing risks.

The assessment reveals a mixed security posture. The organization has successfully implemented multi-factor authentication (MFA) across key systems, which is a significant strength. However, we have identified several high-risk vulnerabilities that require immediate attention.

Key findings include:
\begin{itemize}
    \item \textbf{High-Risk Technical Vulnerability:} The external-facing web server is running an outdated version of Nginx (1.18.0), which is known to have multiple security vulnerabilities. This exposes the organization to potential compromise.
    \item \textbf{Critical Policy Gaps:} The organization lacks a formal Acceptable Use Policy (AUP) for employees.
    \item \textbf{Critical Training Gaps:} New employees do not receive mandatory security awareness training, leaving the organization vulnerable to social engineering and phishing attacks from day one of a new hire's employment.
\end{itemize}

While no previously documented risks were provided, this assessment has identified three new high-risk findings. We strongly recommend prioritizing the remediation of these issues to reduce the organization's attack surface and strengthen its overall defensive capabilities. Detailed recommendations are provided in Section 6.

% ==============================================================================
% Section 2: Organizational Information
% ==============================================================================
\section{Organizational Information}

The following information was provided by the client and used as a baseline for this assessment.

\begin{table}[h!]
\centering
\begin{tabular}{@{}ll@{}}
\toprule
\textbf{Attribute} & \textbf{Value} \\
\midrule
Organization Name & \textbf{Moxie Marketing} \\
Email Domain & \texttt{MoxieMarketing.org} \\
Website Domain & \texttt{www.MoxieMarketing.org} \\
External IP Address & \texttt{134.247.40.4} \\
\bottomrule
\end{tabular}
\caption{Client Profile}
\label{tab:org_info}
\end{table}

% ==============================================================================
% Section 3: Security Control Review
% ==============================================================================
\section{Security Control Review}

A review of the organization's security controls was conducted via a standardized questionnaire. The results below highlight both implemented controls and existing gaps. The symbol \ding{51} indicates a "Yes" response (control in place), while \ding{55} indicates a "No" response (control gap).

\begin{table}[h!]
\centering
\begin{tabular}{@{}lc@{}}
\toprule
\textbf{Security Control Question} & \textbf{Status} \\
\midrule
Do you require MFA to access email? & \ding{51} \\
Do you require MFA to log into computers? & \ding{51} \\
Do you require MFA to access sensitive data systems? & \ding{51} \\
Does your organization have an employee acceptable use policy? & \textbf{\ding{55}} \\
Does your organization do security awareness training for new employees? & \textbf{\ding{55}} \\
Does your organization do security awareness training for all employees at least once per year? & \ding{51} \\
\bottomrule
\end{tabular}
\caption{Security Questionnaire Results}
\label{tab:controls}
\end{table}

\subsection*{Analysis of Control Gaps}
The questionnaire reveals two significant gaps in foundational security practices:
\begin{enumerate}
    \item \textbf{Lack of Acceptable Use Policy (AUP):} An AUP is a critical document that defines the rules and expectations for employee use of company technology and data. Its absence creates ambiguity and increases the risk of insider threats, whether malicious or accidental.
    \item \textbf{No Security Training for New Hires:} Failing to train new employees on security best practices from the start of their employment represents a major vulnerability. New staff are often targeted by phishing and social engineering attacks, and this gap leaves a critical window of exposure.
\end{enumerate}

% ==============================================================================
% Section 4: Technical Scan Results
% ==============================================================================
\section{Technical Scan Results}

An external network scan was performed to identify open ports and exposed services.

\begin{itemize}
    \item \textbf{Scan Target:} \texttt{192.168.10.5}
    \item \textbf{Scan Date:} 2025-11-22
\end{itemize}

\begin{table}[h!]
\centering
\begin{tabular}{@{}lllll@{}}
\toprule
\textbf{Port} & \textbf{State} & \textbf{Service} & \textbf{Product} & \textbf{Version} \\
\midrule
443/tcp & open & https & nginx & 1.18.0 \\
\bottomrule
\end{tabular}
\caption{Open Ports and Services}
\label{tab:scan_results}
\end{table}

\subsection*{Analysis of Findings}
\begin{enumerate}
    \item \textbf{Outdated Nginx Web Server (High Risk):} The scan identified an Nginx server running version \texttt{1.18.0}, which was released in April 2020. This version is significantly outdated and no longer receives security patches. It is known to be vulnerable to multiple Common Vulnerabilities and Exposures (CVEs), potentially allowing an attacker to cause a denial of service, bypass security restrictions, or execute arbitrary code.
    \item \textbf{SSL Certificate Mismatch (Informational):} Further inspection of the service on port 443 revealed an SSL certificate with the Common Name \texttt{www.acme-corp.com}. This does not match the organization's domain (\texttt{www.MoxieMarketing.org}). This misconfiguration can cause trust errors for users and may indicate an improper server setup. The full certificate subject string found was: \seqsplit{\texttt{Subject: commonName=www.acme-corp.com}}.
\end{enumerate}

% ==============================================================================
% Section 5: Consolidated Risk Assessment
% ==============================================================================
\section{Consolidated Risk Assessment}

The following table summarizes the risks identified through the analysis of organizational controls and technical scans. Since no pre-existing risks were provided, all findings listed below are new.

\begin{table}[h!]
\centering
\begin{tabular}{@{}lp{6cm}l@{}}
\toprule
\textbf{Risk ID} & \textbf{Risk Name \& Description} & \textbf{Severity} \\
\midrule
RISK-001 & \textbf{Outdated Web Server Software:} The public-facing Nginx server (v1.18.0) is outdated and vulnerable to numerous known exploits. & \textbf{High} \\
\addlinespace
RISK-002 & \textbf{Lack of Acceptable Use Policy:} No formal policy exists to govern employee use of corporate assets, increasing insider risk and legal exposure. & \textbf{High} \\
\addlinespace
RISK-003 & \textbf{No Security Training for New Hires:} New employees are not trained on security policies and threats, making them prime targets for social engineering. & \textbf{High} \\
\addlinespace
RISK-004 & \textbf{SSL Certificate Mismatch:} The server's SSL certificate name does not match the organization's domain, which can erode user trust and indicates a configuration error. & Informational \\
\bottomrule
\end{tabular}
\caption{Summary of Identified Risks}
\label{tab:risks}
\end{table}

% ==============================================================================
% Section 6: Recommendations
% ==============================================================================
\section{Recommendations}

Based on the findings of this assessment, we recommend the following actions, prioritized by severity, to improve the security posture of \textbf{Moxie Marketing}.

\begin{enumerate}
    \item \textbf{[High Priority] Upgrade Nginx Server:} Immediately plan and execute an upgrade of the Nginx server from version \texttt{1.18.0} to the latest stable version. This will patch critical vulnerabilities and significantly reduce the external attack surface.
    \item \textbf{[High Priority] Develop and Implement an Acceptable Use Policy (AUP):} Draft a formal AUP that clearly outlines the rules for using company networks, devices, and data. This policy should be distributed to all current employees for acknowledgement and integrated into the new hire onboarding process.
    \item \textbf{[High Priority] Institute Onboarding Security Training:} Create a mandatory security awareness training module for all new hires. This training should be completed within their first week of employment and cover topics such as phishing, password security, and the new AUP.
    \item \textbf{[Medium Priority] Correct SSL Certificate Configuration:} Investigate the SSL certificate mismatch on the web server. Procure and install a valid certificate that corresponds to the \texttt{www.MoxieMarketing.org} domain to resolve browser trust errors and ensure encrypted communications are properly configured.
\end{enumerate}

\end{document}
```