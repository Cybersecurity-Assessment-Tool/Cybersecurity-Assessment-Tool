```latex
\documentclass[12pt]{article}

% Preamble: Required Packages
\usepackage[margin=1in]{geometry}
\usepackage{pifont} % For \ding symbols (checkmark/cross)
\usepackage{booktabs} % For professional-looking tables
\usepackage{hyperref} % For clickable links and references
\usepackage{url}      % For formatting URLs
\usepackage{seqsplit} % For splitting long strings in \texttt
\usepackage{xcolor}   % For custom colors
\usepackage{fancyhdr} % For custom headers/footers
\usepackage{graphicx}

% --- Document Setup ---

% Define custom colors for severity
\definecolor{criticalred}{HTML}{D7263D}
\definecolor{highorange}{HTML}{F49D42}
\definecolor{mediumyellow}{HTML}{F4D03F}

% Hyperlink setup
\hypersetup{
    colorlinks=true,
    linkcolor=blue,
    filecolor=magenta,
    urlcolor=cyan,
}

% Header and Footer Configuration
\pagestyle{fancy}
\fancyhf{} % Clear all header and footer fields
\lhead{Cybersecurity Assessment Report}
\rhead{\textbf{CONFIDENTIAL}}
\cfoot{Page \thepage}
\renewcommand{\headrulewidth}{0.4pt}
\renewcommand{\footrulewidth}{0.4pt}

% --- Document Start ---

\begin{document}

\title{Cybersecurity Posture Assessment Report \\ \large For: \textbf{Green Sprout Organic}}
\author{Cybersecurity Analysis Division}
\date{\today}
\maketitle
\thispagestyle{fancy}

\begin{abstract}
    This report provides a comprehensive analysis of the cybersecurity posture of \textbf{Green Sprout Organic}. The assessment is based on a synthesis of network scan data, an organizational security controls questionnaire, and a review of pre-existing risks. The findings indicate several critical and high-risk vulnerabilities that require immediate attention. Key issues include an externally exposed and highly vulnerable FTP server, inadequate authentication controls for email, and gaps in foundational security policies. This document details these findings and provides actionable recommendations to mitigate the identified risks.
\end{abstract}

\newpage
\tableofcontents
\newpage

% ===================================================================
% SECTION 1: Organizational Information
% ===================================================================
\section{Organizational Information}

This section provides an overview of the organizational details used as a baseline for this assessment.

\begin{tabular}{@{}ll}
    \toprule
    \textbf{Attribute} & \textbf{Value} \\
    \midrule
    Organization Name & \textbf{Green Sprout Organic} \\
    Primary Email Domain & \texttt{GreenSproutOrganic.org} \\
    Primary Website & \url{www.GreenSproutOrganic.org} \\
    Known External IP & \texttt{166.138.11.98} \\
    \bottomrule
\end{tabular}

% ===================================================================
% SECTION 2: Security Control Review
% ===================================================================
\section{Security Control Review}

The following table summarizes the organization's responses to a security controls questionnaire. Items marked with a red cross (\textcolor{red}{\ding{55}}) represent significant gaps in the security framework and are discussed in the Risk Assessment section.

\begin{table}[h!]
\centering
\begin{tabular}{@{}p{0.7\linewidth}c}
    \toprule
    \textbf{Security Control Question} & \textbf{Status} \\
    \midrule
    Do you require MFA to access email? & \textcolor{red}{\ding{55}} \\
    Do you require MFA to log into computers? & \textcolor{green!70!black}{\ding{51}} \\
    Do you require MFA to access sensitive data systems? & \textcolor{green!70!black}{\ding{51}} \\
    Does your organization have an employee acceptable use policy? & \textcolor{red}{\ding{55}} \\
    Does your organization do security awareness training for new employees? & \textcolor{green!70!black}{\ding{51}} \\
    Does your organization do security awareness training for all employees at least once per year? & \textcolor{green!70!black}{\ding{51}} \\
    \bottomrule
\end{tabular}
\caption{Organizational Security Controls Questionnaire Results.}
\end{table}

\subsection*{Analysis of Gaps}
Two critical administrative and policy-based gaps were identified:
\begin{itemize}
    \item \textbf{No MFA for Email:} The lack of Multi-Factor Authentication for email is a high-risk vulnerability. Email is a primary vector for phishing attacks, and a compromised account can lead to further infiltration of the network, data breaches, and financial fraud.
    \item \textbf{No Acceptable Use Policy (AUP):} Without a formal AUP, employees may not be aware of their responsibilities regarding the secure use of company assets. This creates ambiguity and increases the risk of insider threats, whether malicious or accidental.
\end{itemize}

% ===================================================================
% SECTION 3: Technical Scan Results
% ===================================================================
\section{Technical Scan Results}

A network scan was performed on the target system to identify open ports and exposed services. The results reveal a critical misconfiguration.

\begin{itemize}
    \item \textbf{Target IP Address:} \texttt{10.0.0.15}
\end{itemize}

\begin{table}[h!]
\centering
\begin{tabular}{@{}llllll}
    \toprule
    \textbf{Port} & \textbf{State} & \textbf{Service} & \textbf{Product} & \textbf{Version} & \textbf{Notes} \\
    \midrule
    21/tcp & Open & ftp & vsftpd & 2.3.4 & \textbf{Critical:} Anonymous FTP login allowed. \\
    \bottomrule
\end{tabular}
\caption{Open Ports and Services Detected on \texttt{10.0.0.15}.}
\end{table}

\subsection*{Analysis of Technical Findings}
The scan identified an open FTP port running \textbf{vsftpd version 2.3.4}. This specific version is associated with a well-documented and severe backdoor vulnerability (\textbf{CVE-2011-2523}). An attacker can exploit this vulnerability to gain a command shell on the underlying server, leading to a full system compromise.

Furthermore, the server is configured to allow \textbf{anonymous FTP login}. This is a dangerous practice that permits any unauthenticated user on the internet to connect to the server and potentially upload or download files. This could be used to exfiltrate sensitive data or to host malicious content.

% ===================================================================
% SECTION 4: Consolidated Risk Assessment
% ===================================================================
\section{Consolidated Risk Assessment}

This section synthesizes findings from the security control review, technical scan, and pre-existing risk data into a consolidated list of prioritized risks.

\begin{table}[h!]
\centering
\begin{tabular}{@{}lp{0.55\linewidth}l}
    \toprule
    \textbf{Risk Name} & \textbf{Description} & \textbf{Severity} \\
    \midrule
    Vulnerable FTP Service & The server at \texttt{10.0.0.15} is running vsftpd 2.3.4, which contains a known remote code execution backdoor (CVE-2011-2523). & \colorbox{criticalred}{\color{white}\textbf{ CRITICAL }} \\
    \addlinespace
    Anonymous FTP Access & The FTP server is configured to allow anonymous login, enabling unauthorized data access, upload of malicious files, or data exfiltration. & \colorbox{criticalred}{\color{white}\textbf{ CRITICAL }} \\
    \addlinespace
    Lack of MFA for Email & Email accounts are protected only by passwords, making them highly susceptible to phishing, credential stuffing, and account takeover attacks. & \colorbox{highorange}{\color{white}\textbf{ HIGH }} \\
    \addlinespace
    Outdated Windows Policy & Workstations are running Windows 7, an end-of-life operating system that no longer receives security updates, exposing them to known exploits. & \colorbox{mediumyellow}{\color{black}\textbf{ MEDIUM }} \\
    \addlinespace
    Missing Acceptable Use Policy & The absence of a formal AUP creates governance and compliance risks, and increases the likelihood of insider threat incidents. & \colorbox{mediumyellow}{\color{black}\textbf{ MEDIUM }} \\
    \bottomrule
\end{tabular}
\caption{Prioritized List of Identified Risks.}
\end{table}

% ===================================================================
% SECTION 5: Recommendations
% ===================================================================
\section{Recommendations}

The following actionable recommendations are provided to mitigate the risks identified in this report. They are prioritized based on severity.

\subsection*{Critical Priority}
\begin{enumerate}
    \item \textbf{Remediate Vulnerable FTP Server (Risk: Vulnerable FTP Service \& Anonymous FTP Access):}
    \begin{itemize}
        \item \textbf{Immediate Action:} If the FTP service is not business-critical, disable and firewall port 21 on \texttt{10.0.0.15} immediately.
        \item \textbf{Long-Term Fix:} If FTP is required, upgrade the \texttt{vsftpd} software to the latest stable version. In the new configuration, explicitly disable anonymous access and enforce strong authentication. Consider using a more secure file transfer protocol like SFTP (SSH File Transfer Protocol).
    \end{itemize}
\end{enumerate}

\subsection*{High Priority}
\begin{enumerate}
    \setcounter{enumi}{1}
    \item \textbf{Implement MFA for Email (Risk: Lack of MFA for Email):}
    \begin{itemize}
        \item \textbf{Immediate Action:} Enforce mandatory Multi-Factor Authentication (MFA) for all user accounts on the \texttt{GreenSproutOrganic.org} email domain. This is the single most effective control to prevent account takeovers.
    \end{itemize}
\end{enumerate}

\subsection*{Medium Priority}
\begin{enumerate}
    \setcounter{enumi}{2}
    \item \textbf{Upgrade End-of-Life Systems (Risk: Outdated Windows Policy):}
    \begin{itemize}
        \item \textbf{Action:} Continue and prioritize the project to upgrade all Windows 7 workstations to a currently supported operating system, such as Windows 10 or Windows 11, to ensure they receive critical security patches.
    \end{itemize}
    \item \textbf{Develop and Implement an AUP (Risk: Missing Acceptable Use Policy):}
    \begin{itemize}
        \item \textbf{Action:} Draft, approve, and disseminate a formal Acceptable Use Policy (AUP) to all employees. Ensure all current and new employees read and acknowledge the policy as part of their onboarding and annual training.
    \end{itemize}
\end{enumerate}

\end{document}
```