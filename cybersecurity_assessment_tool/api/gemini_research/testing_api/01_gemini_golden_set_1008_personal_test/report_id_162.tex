```latex
\documentclass[12pt]{article}

% Preamble: Required Packages
\usepackage[margin=1in]{geometry} % Set page margins
\usepackage{pifont}               % For checkmarks and crosses (\ding)
\usepackage{booktabs}             % For professional-looking tables
\usepackage{hyperref}             % For hyperlinks and document metadata
\usepackage{url}                  % For formatting URLs
\usepackage{seqsplit}             % To split long monospaced strings
\usepackage{graphicx}             % For including logos (placeholder)
\usepackage{xcolor}               % For custom colors
\usepackage{fancyhdr}             % For custom headers and footers
\usepackage[T1]{fontenc}          % Font encoding

% --- Document Setup ---

% Define colors for the report
\definecolor{DarkBlue}{rgb}{0.0, 0.0, 0.55}
\definecolor{Grey}{rgb}{0.4, 0.4, 0.4}

% Setup hyperref
\hypersetup{
    colorlinks=true,
    linkcolor=DarkBlue,
    urlcolor=DarkBlue,
    citecolor=DarkBlue,
    pdftitle={Cybersecurity Posture Assessment Report},
    pdfauthor={Cybersecurity Analyst},
    pdfsubject={Security Analysis for Echo Chamber Arts}
}

% Custom commands for Yes/No symbols
\newcommand{\cmark}{\textcolor{green}{\ding{51}}}%
\newcommand{\xmark}{\textcolor{red}{\ding{55}}}%

% --- Header and Footer ---
\pagestyle{fancy}
\fancyhf{} % Clear all header and footer fields
\fancyhead[L]{Cybersecurity Posture Assessment}
\fancyhead[R]{For: Echo Chamber Arts}
\fancyfoot[C]{\thepage}
\renewcommand{\headrulewidth}{0.4pt}
\renewcommand{\footrulewidth}{0.4pt}

% --- Document Body ---
\begin{document}

% --- Title Page ---
\begin{titlepage}
    \centering
    \vspace*{2cm}
    
    {\Huge \textbf{Cybersecurity Posture Assessment Report}\par}
    \vspace{1.5cm}
    
    {\Large \textbf{Prepared For:}\par}
    \vspace{0.5cm}
    {\Large Echo Chamber Arts\par}
    
    \vfill
    
    {\large \textbf{Date of Report:}\par}
    {\large \today\par}
    
    \vspace{1cm}
    
    \begin{center}
        \rule{0.8\textwidth}{0.4pt}
    \end{center}
    \vspace{0.2cm}
    \textit{This report contains sensitive information and should be handled with care.}
    
\end{titlepage}

% --- Table of Contents ---
\tableofcontents
\newpage

% --- Section 1: Executive Overview ---
\section{Executive Overview}
This report provides a comprehensive analysis of the cybersecurity posture for \textbf{Echo Chamber Arts}, based on a synthesis of technical network scans, a security controls questionnaire, and a review of pre-existing risk data. The assessment was conducted to identify vulnerabilities, policy gaps, and areas for security enhancement.

\paragraph{Key Findings:} The assessment revealed a mixed security posture. While the organization has implemented Multi-Factor Authentication (MFA) for computer and sensitive data access, several critical and high-risk gaps were identified:
\begin{itemize}
    \item \textbf{Critical Risk - No MFA on Email:} The lack of mandatory MFA for email access represents the most significant threat, exposing the organization to account takeovers and business email compromise (BEC) attacks.
    \item \textbf{High Risk - Policy Gaps:} The absence of an employee Acceptable Use Policy (AUP) and a recurring annual security awareness training program creates significant organizational risk from both a compliance and operational security standpoint.
    \item \textbf{Risk Remediation Noted:} A previously identified risk concerning an unencrypted web server on Port 80 appears to have been remediated. Our latest network scan found Port 80 to be \texttt{closed} on the target system, contradicting the prior risk data.
\end{itemize}

\paragraph{Overall Assessment:} The organization's proactive stance on MFA for internal systems is commendable. However, the identified gaps in email security and foundational governance policies must be addressed urgently to mitigate substantial risks. This report provides prioritized, actionable recommendations to strengthen the overall security posture.

\newpage

% --- Section 2: Organizational Information ---
\section{Organizational Information}
The following details were provided for the assessment. This information helps to establish the context and scope of the review.

\begin{tabular}{@{}ll}
    \toprule
    \textbf{Attribute} & \textbf{Value} \\
    \midrule
    Organization Name & \textbf{Echo Chamber Arts} \\
    Email Domain & \seqsplit{\texttt{EchoChamberArts.net}} \\
    Website Domain & \seqsplit{\url{www.EchoChamberArts.net}} \\
    External IP Address & \seqsplit{\texttt{221.92.57.202}} \\
    \bottomrule
\endtabular}

% --- Section 3: Security Control Review ---
\section{Security Control Review}
The following table summarizes the organization's responses to a security controls questionnaire. The assessment column highlights areas that deviate from security best practices and represent potential risks.

\begin{tabular}{@{}p{0.6\linewidth} c p{0.25\linewidth}@{}}
    \toprule
    \textbf{Control Question} & \textbf{Response} & \textbf{Assessment} \\
    \midrule
    Do you require MFA to access email? & \xmark & \textbf{Critical Gap.} Email is a primary target for attackers. \\
    \addlinespace
    Do you require MFA to log into computers? & \cmark & Meets best practice. \\
    \addlinespace
    Do you require MFA to access sensitive data systems? & \cmark & Meets best practice. \\
    \addlinespace
    Does your organization have an employee acceptable use policy? & \xmark & \textbf{High Risk.} Lack of a formal policy creates legal and security ambiguity. \\
    \addlinespace
    Does your organization do security awareness training for new employees? & \cmark & Good foundational practice. \\
    \addlinespace
    Does your organization do security awareness training for all employees at least once per year? & \xmark & \textbf{High Risk.} Security is a continuous process; knowledge decays and threats evolve. \\
    \bottomrule
\end{tabular}

% --- Section 4: Technical Scan Results ---
\section{Technical Scan Results}
A network scan was performed to identify active services and potential technical vulnerabilities on the specified target system.

\subsection*{Nmap Scan of \seqsplit{\texttt{192.168.0.5}}}
\begin{itemize}
    \item \textbf{Scan Date:} \today
    \item \textbf{Target IP:} \seqsplit{\texttt{192.168.0.5}}
    \item \textbf{Host Status:} Up
\end{itemize}

The scan results indicate no open ports on the target host. This is a positive security finding, as it minimizes the external attack surface.

\begin{tabular}{@{}llll@{}}
    \toprule
    \textbf{Port} & \textbf{State} & \textbf{Service} & \textbf{Version} \\
    \midrule
    80 & closed & http & N/A \\
    \bottomrule
\end{tabular}

\paragraph{Analysis:} The scan shows that Port 80 (HTTP) is closed. This finding directly contradicts a pre-existing risk entry ("Unencrypted Web Server") which stated this port was open. This suggests that the risk has been successfully remediated or was a false positive in a previous assessment.

\newpage

% --- Section 5: Risk Assessment Summary ---
\section{Risk Assessment Summary}
This section synthesizes findings from the security control review, technical scans, and pre-existing risk data into a consolidated list of current risks.

\begin{tabular}{@{}p{0.1\linewidth} p{0.25\linewidth} p{0.45\linewidth} p{0.1\linewidth}@{}}
    \toprule
    \textbf{ID} & \textbf{Risk Title} & \textbf{Description} & \textbf{Severity} \\
    \midrule
    \textbf{R-01} & Lack of MFA on Email & The absence of MFA on the primary communication platform exposes the organization to a high likelihood of account compromise, data breaches, and financial fraud via BEC. & \textbf{Critical} \\
    \addlinespace
    \textbf{R-02} & No Acceptable Use Policy (AUP) & Without a formal AUP, employees may be unaware of their responsibilities regarding data handling and system usage, increasing the risk of insider threats and non-compliance. & High \\
    \addlinespace
    \textbf{R-03} & Insufficient Security Training & One-time training for new hires is inadequate. The lack of an annual refresher program for all staff means the workforce is unprepared for evolving phishing and social engineering tactics. & High \\
    \addlinespace
    \textbf{R-04} & Unencrypted Web Server & \textit{A pre-existing risk that appears to be resolved.} The current network scan confirmed Port 80 is closed, mitigating the risk of unencrypted web traffic from this host. & Resolved \\
    \bottomrule
\end{tabular}

% --- Section 6: Recommendations ---
\section{Recommendations}
The following actionable recommendations are prioritized based on the severity of the identified risks.

\subsection*{Priority 1 (Critical): Remediate R-01}
\paragraph{Action: Enforce MFA for All Email Accounts}
Immediately configure the organization's email service (e.g., Microsoft 365, Google Workspace) to require Multi-Factor Authentication for all user accounts. This single action provides the most significant improvement to the organization's security posture.
\begin{itemize}
    \item \textbf{Justification:} Protects against credential theft, phishing, and account takeovers, which are the root cause of most modern cyberattacks.
\end{itemize}

\subsection*{Priority 2 (High): Remediate R-02 \& R-03}
\paragraph{Action: Develop and Implement Foundational Security Policies}
\begin{enumerate}
    \item \textbf{Create an Acceptable Use Policy (AUP):} Draft and implement an AUP that clearly defines the rules for using company technology, data, and network resources. Require all employees to read and acknowledge the policy. Templates are available from sources like the SANS Institute.
    \item \textbf{Establish an Annual Security Awareness Program:} Implement a mandatory security awareness training program for all employees to be completed at least once per year. This ensures continuous education on current threats like phishing, ransomware, and social engineering.
\end{enumerate}

\subsection*{Priority 3 (Informational): Validate R-04}
\paragraph{Action: Update and Maintain Risk Register}
Formally mark the "Unencrypted Web Server" risk (R-04) as "Resolved" in the organization's risk register. Implement a process for periodically re-validating technical scan results against the risk register to ensure it remains accurate and up-to-date.

\end{document}
```