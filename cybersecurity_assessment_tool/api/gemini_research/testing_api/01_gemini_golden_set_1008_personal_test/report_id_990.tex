```latex
\documentclass[12pt]{article}

% Required Packages
\usepackage[margin=1in]{geometry} % Set page margins
\usepackage{pifont}                % For checkmarks and crosses (\ding)
\usepackage{booktabs}              % For professional tables (\toprule, \midrule, \bottomrule)
\usepackage{hyperref}              % For hyperlinks, references, and metadata
\usepackage{url}                   % For formatting URLs
\usepackage{seqsplit}              % To split long strings in \texttt
\usepackage{graphicx}              % For logos (optional)
\usepackage[utf8]{inputenc}        % For UTF-8 input

% Document Metadata
\hypersetup{
    colorlinks=true,
    linkcolor=blue,
    filecolor=magenta,      
    urlcolor=cyan,
    pdftitle={Cybersecurity Assessment Report},
    pdfauthor={Cybersecurity Analysis Team},
    pdfsubject={Security Assessment},
    pdfkeywords={Cybersecurity, Risk, Analysis},
    bookmarks=true
}

% Title Information
\title{Cybersecurity Assessment Report \\ \large For: Falcon Heavy}
\author{Cybersecurity Analysis Team}
\date{\today}

\begin{document}

\maketitle
\thispagestyle{empty}
\newpage

\tableofcontents
\thispagestyle{empty}
\newpage

\pagestyle{headings}

% --- 1. Executive Summary ---
\section{Executive Summary}
This report provides a comprehensive cybersecurity assessment for Falcon Heavy, based on network scan data, organizational security control information, and a review of pre-existing risks. The analysis correlates technical findings with procedural controls to provide a holistic view of the organization's security posture.

The assessment reveals a mixed security posture. On a technical level, the scanned host (\texttt{192.168.0.5}) appears secure, with no open ports detected. This finding contradicts a previously identified risk of an unencrypted web server on Port 80, suggesting that the issue may have been remediated.

However, significant procedural and policy-based risks were identified through the security questionnaire. The most critical gaps are the lack of multi-factor authentication (MFA) for sensitive data systems and the complete absence of a security awareness training program for employees. These weaknesses expose the organization to significant threats, including unauthorized data access, phishing, and social engineering attacks.

Immediate action should be focused on implementing MFA for all sensitive systems and establishing a mandatory security awareness training program.

% --- 2. Organizational Information ---
\section{Organizational Information}
The following information was provided for the assessment.

\begin{tabular}{@{}ll}
    \toprule
    \textbf{Attribute} & \textbf{Value} \\
    \midrule
    Organization Name & Falcon Heavy \\
    Email Domain & \texttt{FalconHeavy.net} \\
    Website Domain & \seqsplit{\url{www.FalconHeavy.net}} \\
    External IP Address & \texttt{138.107.238.156} \\
    \bottomrule
\end{tabular}

% --- 3. Security Control Review ---
\section{Security Control Review}
The following table summarizes the organization's responses to a security controls questionnaire. Items marked with \ding{55} (a cross) indicate significant gaps in the security framework and are discussed in the Risk Assessment section.

\begin{table}[h!]
\centering
\begin{tabular}{@{}p{0.75\linewidth}c@{}}
    \toprule
    \textbf{Control Question} & \textbf{Response} \\
    \midrule
    Do you require MFA to access email? & \ding{51} \\
    Do you require MFA to log into computers? & \ding{51} \\
    Do you require MFA to access sensitive data systems? & \textbf{\ding{55}} \\
    Does your organization have an employee acceptable use policy? & \ding{51} \\
    Does your organization do security awareness training for new employees? & \textbf{\ding{55}} \\
    Does your organization do security awareness training for all employees at least once per year? & \textbf{\ding{55}} \\
    \bottomrule
\end{tabular}
\caption{Security Controls Questionnaire Results}
\end{table}

% --- 4. Technical Scan Results ---
\section{Technical Scan Results}
A network scan was performed on the specified target to identify open ports and exposed services.

\begin{itemize}
    \item \textbf{Target IP Address:} \texttt{192.168.0.5}
    \item \textbf{Scan Status:} The host was found to be online.
\end{itemize}

The scan results below indicate that no open ports were discovered on the target system.

\begin{table}[h!]
\centering
\begin{tabular}{@{}llll@{}}
    \toprule
    \textbf{Port} & \textbf{State} & \textbf{Service} & \textbf{Version} \\
    \midrule
    80/tcp & closed & http & N/A \\
    \bottomrule
\end{tabular}
\caption{Port Scan Results for \texttt{192.168.0.5}}
\end{table}

\subsection*{Analysis}
The technical scan did not identify any immediate vulnerabilities on the host \texttt{192.168.0.5}. The finding that Port 80 is closed contradicts a pre-existing risk report (\textit{Unencrypted Web Server}). This suggests that the previously identified issue may have been resolved on this specific host or pertains to a different asset, such as the external IP \texttt{138.107.238.156}. Further investigation is recommended to confirm remediation across all public-facing assets.

% --- 5. Risk Assessment ---
\section{Risk Assessment}
This section synthesizes findings from the security control review, technical scan, and pre-existing risk data to provide a prioritized list of current risks.

\begin{table}[h!]
\centering
\begin{tabular}{@{}p{0.25\linewidth}p{0.5\linewidth}l@{}}
    \toprule
    \textbf{Risk Name} & \textbf{Description} & \textbf{Severity} \\
    \midrule
    \textbf{Lack of MFA for Sensitive Systems} & The absence of multi-factor authentication on systems containing sensitive data creates a high risk of unauthorized access and data breach from compromised credentials. & \textbf{Critical} \\
    \addlinespace
    \textbf{No Security Awareness Training Program} & Without training, employees are more susceptible to phishing, social engineering, and malware, making them a primary vector for initial compromise. This affects both new and existing staff. & \textbf{High} \\
    \addlinespace
    \textbf{Unencrypted Web Server (Potentially Remediated)} & A previously identified risk indicated an open Port 80. The current scan shows this port is closed on the scanned target, suggesting remediation. This should be verified across all external assets. & Medium \\
    \bottomrule
\end{tabular}
\caption{Summary of Identified Risks}
\end{table}

% --- 6. Recommendations ---
\section{Recommendations}
Based on the risk assessment, the following actions are recommended to improve the security posture of Falcon Heavy.

\subsection*{Immediate Priority (Critical Risk)}
\begin{enumerate}
    \item \textbf{Implement MFA on Sensitive Systems:} Immediately enforce mandatory multi-factor authentication for all user accounts (including administrative and service accounts) that have access to sensitive data storage, applications, and infrastructure.
\end{enumerate}

\subsection*{High Priority}
\begin{enumerate}
    \setcounter{enumi}{1}
    \item \textbf{Establish a Security Awareness Training Program:}
    \begin{itemize}
        \item Develop or procure a security awareness training module and make it a mandatory part of the onboarding process for all new employees.
        \item Schedule and require annual refresher security training for all staff. Topics should include phishing identification, password hygiene, and acceptable use policies.
    \end{itemize}
\end{enumerate}

\subsection*{Verification and Review}
\begin{enumerate}
    \setcounter{enumi}{2}
    \item \textbf{Verify Remediation of Port 80:} Conduct a thorough scan of all external-facing IP addresses, including \texttt{138.107.238.156}, to confirm that Port 80 is closed or redirects to a secure HTTPS (Port 443) service. Ensure no unencrypted web services are exposed to the internet.
\end{enumerate}

\end{document}
```