```latex
\documentclass[12pt]{article}

% --- PACKAGE IMPORTS ---
\usepackage[margin=1in]{geometry} % Set page margins
\usepackage{pifont}               % For checkmark and X symbols (\ding)
\usepackage{booktabs}             % For professional-looking tables
\usepackage{hyperref}             % For hyperlinks and document metadata
\usepackage{url}                  % For formatting URLs
\usepackage{seqsplit}             % For splitting long strings in texttt
\usepackage{graphicx}             % For logos (placeholder)
\usepackage{xcolor}               % For custom colors

% --- DOCUMENT METADATA ---
\hypersetup{
    colorlinks=true,
    linkcolor=blue,
    filecolor=magenta,      
    urlcolor=cyan,
    pdftitle={Cybersecurity Posture Assessment Report},
    pdfauthor={Cybersecurity Analyst},
    pdfsubject={Security Analysis},
    pdfkeywords={Cybersecurity, Risk Assessment, Network Scan},
}

% --- CUSTOM COMMANDS ---
\newcommand{\yes}{\ding{51}} % Green checkmark
\newcommand{\no}{\ding{55}}  % Red X

% --- TITLE PAGE SETUP ---
\title{
    \vspace{2cm}
    \textbf{Cybersecurity Posture Assessment Report}\\
    \large For: \textbf{Skyward Bound}
    \vspace{1cm}
}
\author{Cybersecurity Analyst}
\date{\today}

% --- DOCUMENT START ---
\begin{document}

\maketitle
\thispagestyle{empty}
\newpage

\tableofcontents
\newpage

% --- EXECUTIVE SUMMARY ---
\section{Executive Summary}
This report provides a comprehensive cybersecurity assessment for \textbf{Skyward Bound}, based on an analysis of network scan data, organizational security controls, and existing risk documentation. The assessment reveals several critical-level risks that require immediate attention to prevent potential data breaches and unauthorized access.

The most significant finding is an exposed internal service on port \texttt{8080} of host \texttt{10.5.5.5}, which identifies itself as a \textbf{"TOP SECRET DB"}. This directly contradicts existing documentation that dismisses this port as a false positive. This exposed database, combined with a complete lack of Multi-Factor Authentication (MFA) across all company systems (including email, computers, and sensitive data systems), creates a direct and severe pathway for data exfiltration.

Furthermore, foundational security practices, such as employee security training and an acceptable use policy, are absent. This indicates a low level of security maturity and elevates the risk of human-related security incidents.

Immediate remediation should focus on securing the exposed database and implementing MFA across all critical infrastructure.

% --- ORGANIZATIONAL INFORMATION ---
\section{Organizational Information}
The following information was provided for the assessment.
\begin{table}[h!]
\centering
\begin{tabular}{@{}ll@{}}
\toprule
\textbf{Attribute} & \textbf{Value} \\ \midrule
Organization Name  & \textbf{Skyward Bound} \\
Email Domain       & \seqsplit{\texttt{SkywardBound.com}} \\
Website Domain     & \seqsplit{\url{www.SkywardBound.com}} \\
External IP Address & \texttt{195.106.65.212} \\ \bottomrule
\end{tabular}
\caption{Client Organizational Data.}
\end{table}

% --- SECURITY CONTROL REVIEW ---
\section{Security Control Review}
A review of organizational security controls was conducted based on a standard questionnaire. The responses indicate significant gaps in fundamental security practices. Each "No" response represents a control failure that increases organizational risk.

\begin{table}[h!]
\centering
\begin{tabular}{@{}p{0.6\textwidth}cc@{}}
\toprule
\textbf{Control Question} & \textbf{Response} & \textbf{Assessment} \\ \midrule
Do you require MFA to access email? & \no & Critical Gap \\
Do you require MFA to log into computers? & \no & High Risk \\
Do you require MFA to access sensitive data systems? & \no & Critical Gap \\
Does your organization have an employee acceptable use policy? & \no & High Risk \\
Does your organization do security awareness training for new employees? & \no & High Risk \\
Does your organization do security awareness training for all employees at least once per year? & \no & High Risk \\ \bottomrule
\end{tabular}
\caption{Security Controls Questionnaire Analysis.}
\end{table}

% --- TECHNICAL SCAN RESULTS ---
\section{Technical Scan Results}
An internal network scan was performed to identify active services and potential vulnerabilities. The scan targeted the internal host \texttt{10.5.5.5}.

\subsection{Open Ports and Services}
A single open port was discovered, hosting a web service with a highly concerning title. This finding directly contradicts the pre-existing risk documentation (\textit{Input\_3\_Current\_Risks\_JSON}), which incorrectly labeled this port as a secure false positive.

\begin{table}[h!]
\centering
\begin{tabular}{@{}llll@{}}
\toprule
\textbf{Port} & \textbf{State} & \textbf{Service} & \textbf{Details} \\ \midrule
8080/tcp      & OPEN           & http-proxy       & Service title discovered: \textbf{"TOP SECRET DB"} \\ \bottomrule
\end{tabular}
\caption{Open Port Findings for Host \texttt{10.5.5.5}.}
\end{table}

% --- RISK ASSESSMENT ---
\section{Risk Assessment}
The following risks have been identified and prioritized based on the correlation of technical findings and organizational control gaps.

\begin{table}[h!]
\centering
\begin{tabular}{@{}p{0.2\textwidth}p{0.55\textwidth}l@{}}
\toprule
\textbf{Risk Title} & \textbf{Description} & \textbf{Severity} \\ \midrule
\textbf{Exposed Sensitive Database} & An internal service on port \texttt{8080} is publicly identifying itself as a "TOP SECRET DB". This poses an immediate and direct threat of a major data breach. & \textbf{Critical} \\
\addlinespace
\textbf{No Multi-Factor Authentication (MFA)} & The complete absence of MFA on email, endpoints, and sensitive systems makes the organization highly vulnerable to credential theft and account takeover attacks. & \textbf{Critical} \\
\addlinespace
\textbf{Inadequate Security Policies \& Training} & The lack of an Acceptable Use Policy and any form of security awareness training significantly increases the risk of insider threat and successful phishing attacks. & \textbf{High} \\
\addlinespace
\textbf{Flawed Risk Management Process} & The existing risk register incorrectly identifies the exposed database port as a "secure false positive," indicating a failure in the vulnerability validation and management process. & \textbf{High} \\
\bottomrule
\end{tabular}
\caption{Summary of Identified Risks.}
\end{table}

% --- RECOMMENDATIONS ---
\section{Recommendations}
The following actions are recommended to mitigate the identified risks. They are prioritized based on severity and potential impact.

\subsection{Immediate Priority (0-7 Days)}
\begin{enumerate}
    \item \textbf{Investigate and Secure Port 8080:} Immediately investigate the service running on port \texttt{8080} of host \texttt{10.5.5.5}.
    \begin{itemize}
        \item Identify the owner and purpose of the "TOP SECRET DB".
        \item Restrict access to this port using a host-based or network firewall.
        \item If the service is required, ensure it is placed behind a secure authentication layer (e.g., VPN, reverse proxy with MFA).
    \end{itemize}
    \item \textbf{Deploy MFA on Critical Systems:} Begin the phased rollout of mandatory MFA for all employees.
    \begin{itemize}
        \item \textbf{Phase 1:} Email and access to sensitive data systems.
        \item \textbf{Phase 2:} All remote access solutions (VPNs).
        \item \textbf{Phase 3:} All computer logins.
    \end{itemize}
\end{enumerate}

\subsection{High Priority (1-3 Months)}
\begin{enumerate}
    \item \textbf{Develop Foundational Security Policies:} Draft and implement an official Acceptable Use Policy (AUP) that all employees must read and acknowledge.
    \item \textbf{Implement Security Awareness Training:}
    \begin{itemize}
        \item Procure and deploy a security awareness training platform.
        \item Enroll all current employees in a baseline training module.
        \item Integrate this training into the onboarding process for all new hires.
    \end{itemize}
    \item \textbf{Review Risk Management Process:} Conduct a full review of the vulnerability management and risk assessment lifecycle to ensure that technical findings are accurately validated, tracked, and remediated.
\end{enumerate}

\end{document}
```