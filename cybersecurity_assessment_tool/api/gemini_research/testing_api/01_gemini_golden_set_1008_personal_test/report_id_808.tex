```latex
\documentclass[12pt]{article}

% Preamble: Required Packages
\usepackage[margin=1in]{geometry}
\usepackage{pifont} % For checkmarks and crosses
\usepackage{booktabs} % For professional tables
\usepackage{hyperref} % For clickable links
\usepackage{url} % For URL formatting
\usepackage{seqsplit} % For splitting long strings
\usepackage{graphicx} % For logo (placeholder)
\usepackage{fancyhdr} % For header/footer
\usepackage{lastpage} % To get total page count
\usepackage{xcolor} % For colors

% --- Document Setup ---
\hypersetup{
    colorlinks=true,
    linkcolor=blue,
    filecolor=magenta,      
    urlcolor=cyan,
    pdftitle={Cybersecurity Assessment Report},
    pdfauthor={Cybersecurity Analyst},
    pdfsubject={Security Analysis},
    pdfkeywords={Security, Report, Analysis},
}

% --- Header and Footer ---
\pagestyle{fancy}
\fancyhf{} % Clear all header and footer fields
\fancyhead[L]{Cybersecurity Assessment Report}
\fancyhead[R]{Aetheric Systems}
\fancyfoot[C]{\thepage\ of \pageref{LastPage}}
\renewcommand{\headrulewidth}{0.4pt}
\renewcommand{\footrulewidth}{0.4pt}

% --- Custom Commands ---
\newcommand{\yes}{\ding{51}}
\newcommand{\no}{\ding{55}}
\newcommand{\riskcritical}[1]{\textcolor{red}{\textbf{#1}}}
\newcommand{\riskhigh}[1]{\textcolor{orange}{\textbf{#1}}}
\newcommand{\riskmedium}[1]{\textcolor{yellow}{\textbf{#1}}}
\newcommand{\risklow}[1]{\textcolor{green}{\textbf{#1}}}

\begin{document}

% --- Title Page ---
\begin{titlepage}
    \centering
    \vspace*{1cm}
    
    \Huge
    \textbf{Cybersecurity Assessment Report}
    
    \vspace{1.5cm}
    
    \Large
    Prepared for: \\
    \vspace{0.5cm}
    \textbf{Aetheric Systems}
    
    \vfill
    
    \Large
    \today
    
    \vspace{1cm}
    
    \large
    Generated by: Cybersecurity Analyst
    
\end{titlepage}

\tableofcontents
\newpage

% --- Section 1: Executive Summary ---
\section{Executive Summary}
This report details the findings of a cybersecurity assessment conducted for Aetheric Systems. The analysis is based on a network vulnerability scan, a review of organizational security controls, and an evaluation of pre-existing risks.

The overall security posture of Aetheric Systems is mixed. The organization demonstrates a strong network perimeter, as evidenced by a network scan of the target host \texttt{192.168.1.100} which revealed no open ports. This is a commendable security practice.

However, significant gaps were identified in internal security policies and controls. The two most critical findings are:
\begin{itemize}
    \item \textbf{Lack of Multi-Factor Authentication (MFA) on Endpoints:} Employee computers are not protected by MFA, creating a critical vulnerability to credential theft and unauthorized access.
    \item \textbf{Insufficient Security Awareness Training:} The absence of mandatory, annual security awareness training for all employees constitutes a high risk, leaving the organization more susceptible to phishing and social engineering attacks.
\end{itemize}

This report provides a detailed breakdown of these findings and offers actionable recommendations to mitigate the identified risks and strengthen the organization's overall cybersecurity resilience.

% --- Section 2: Organizational Information ---
\section{Organizational Information}
The following details were provided for the assessment. This information forms the baseline for understanding the organization's digital footprint.

\begin{tabular}{@{}ll}
    \toprule
    \textbf{Attribute} & \textbf{Value} \\
    \midrule
    Organization Name & Aetheric Systems \\
    Email Domain & \texttt{AethericSystems.net} \\
    Website Domain & \url{www.AethericSystems.net} \\
    External IP Address & \texttt{163.223.215.221} \\
    \bottomrule
\end{tabular}

% --- Section 3: Security Control Review ---
\section{Security Control Review}
A review of organizational security controls was conducted based on a standardized questionnaire. The results highlight key areas of strength and weakness in the current security policy framework. "No" answers indicate a deviation from security best practices and represent potential risks.

\begin{tabular}{@{}p{0.75\linewidth}c}
    \toprule
    \textbf{Control Question} & \textbf{Response} \\
    \midrule
    Do you require MFA to access email? & \yes \\
    Do you require MFA to log into computers? & \no \\
    Do you require MFA to access sensitive data systems? & \yes \\
    Does your organization have an employee acceptable use policy? & \yes \\
    Does your organization do security awareness training for new employees? & \yes \\
    Does your organization do security awareness training for all employees at least once per year? & \no \\
    \bottomrule
\end{tabular}

% --- Section 4: Technical Scan Results ---
\section{Technical Scan Results}
An external network scan was performed to identify exposed services and potential vulnerabilities on the specified target system.

\begin{itemize}
    \item \textbf{Target IP Address:} \texttt{192.168.1.100}
    \item \textbf{Scan Date:} \today
    \item \textbf{Finding:} The scan reported that all 1000 scanned ports were in a \texttt{closed} state. No open ports were detected on the target host.
\end{itemize}

\subsection*{Analysis}
The absence of open ports is a positive security finding. It indicates that the target system is likely protected by a well-configured firewall that denies unsolicited inbound traffic. This significantly reduces the external attack surface of the host. No vulnerabilities were identified from this scan.

% --- Section 5: Consolidated Risk Assessment ---
\section{Consolidated Risk Assessment}
This section synthesizes findings from the security control review, technical scans, and pre-existing risk data. The following table prioritizes identified risks based on their potential impact on the organization.

\begin{tabular}{@{}p{0.1\linewidth}p{0.25\linewidth}p{0.4\linewidth}p{0.15\linewidth}}
    \toprule
    \textbf{ID} & \textbf{Risk Name} & \textbf{Overview} & \textbf{Severity} \\
    \midrule
    RISK-001 & Lack of Endpoint Multi-Factor Authentication & User computers do not require MFA for login. This exposes endpoints to unauthorized access if user credentials are stolen, lost, or cracked. & \riskcritical{Critical} \\
    \addlinespace
    RISK-002 & Insufficient Security Awareness Training & Security training is not conducted annually for all staff. This increases the organization's susceptibility to social engineering, phishing, and malware attacks. & \riskhigh{High} \\
    \bottomrule
\end{tabular}

\textit{Note: The list of pre-existing risks provided for this assessment was empty.}

% --- Section 6: Recommendations ---
\section{Recommendations}
The following actionable recommendations are provided to address the risks identified in this report.

\subsection*{RISK-001: Lack of Endpoint MFA (Critical)}
\begin{itemize}
    \item \textbf{Action:} Implement a mandatory Multi-Factor Authentication (MFA) solution for all employee computer (endpoint) logins. This should apply to both local and remote access.
    \item \textbf{Justification:} Enforcing MFA at the endpoint level provides a critical layer of defense. Even if an attacker obtains a user's password, they will be unable to log in without the second factor (e.g., a code from a mobile app, a hardware token, or a biometric scan).
    \item \textbf{Suggested Solutions:} Windows Hello for Business, Duo Security, Okta, or other enterprise-grade MFA providers.
\end{itemize}

\subsection*{RISK-002: Insufficient Security Awareness Training (High)}
\begin{itemize}
    \item \textbf{Action:} Establish a formal, mandatory security awareness training program that is required for all employees on an annual basis.
    \item \textbf{Justification:} Humans are often the first line of defense. Regular training ensures that staff are kept up-to-date on the latest threats, such as sophisticated phishing emails and ransomware tactics. This reduces the likelihood of human error leading to a security breach.
    \item \textbf{Suggested Program Elements:} The program should include modules on phishing identification, password security, acceptable use policies, and incident reporting. Track completion to ensure 100\% compliance.
\end{itemize}

\end{document}
```