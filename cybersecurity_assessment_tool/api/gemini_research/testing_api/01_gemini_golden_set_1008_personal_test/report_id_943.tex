```latex
\documentclass[12pt]{article}

% --- PACKAGES ---
\usepackage[margin=1in]{geometry}
\usepackage{pifont} % For checkmarks and crosses
\usepackage{booktabs} % For professional tables
\usepackage{hyperref} % For hyperlinks and metadata
\usepackage{url} % For formatting URLs
\usepackage{seqsplit} % For splitting long strings in tt font
\usepackage{graphicx}
\usepackage{xcolor}

% --- HYPERREF SETUP ---
\hypersetup{
    colorlinks=true,
    linkcolor=black,
    filecolor=magenta,      
    urlcolor=blue,
    pdftitle={Cybersecurity Posture Report},
    pdfauthor={Cybersecurity Analysis Team},
    pdfsubject={Security Assessment},
    pdfkeywords={Cybersecurity, Risk Assessment, Network Scan},
    bookmarks=true
}

% --- DOCUMENT START ---
\begin{document}

% --- TITLE PAGE ---
\begin{titlepage}
    \centering
    \vspace*{\fill}
    \Huge\textbf{Cybersecurity Posture Report}
    \vspace{1.5cm}
    \Large\textbf{Prepared for:}\\
    \vspace{0.5cm}
    \huge Digital Drift
    \vspace{2cm}
    \large\textbf{Date of Report:}\\
    \vspace{0.5cm}
    \large \today
    \vspace*{\fill}
    \textit{This report is confidential and intended solely for the use of Digital Drift.}
\end{titlepage}

\tableofcontents
\newpage

% --- SECTION 1: OVERVIEW AND EXECUTIVE SUMMARY ---
\section{Overview and Executive Summary}
This report provides a comprehensive analysis of the cybersecurity posture for Digital Drift. The assessment is based on a synthesis of organizational data, a review of existing security controls, and technical network scanning.

The overall security posture is moderately strong, with excellent adoption of security awareness training and acceptable use policies. The implementation of Multi-Factor Authentication (MFA) for email and sensitive systems is a significant strength.

However, two key risks were identified that require immediate attention:
\begin{enumerate}
    \item \textbf{Critical Gap in Endpoint Security:} The absence of mandatory MFA for computer logins presents a high-risk vulnerability. If an employee's credentials are stolen, an attacker could gain direct access to the internal network and company resources.
    \item \textbf{Exposed Management Service:} A publicly accessible Secure Shell (SSH) service was identified on the network. While necessary for remote administration, exposed management ports are primary targets for automated brute-force attacks and exploitation attempts.
\end{enumerate}

This report details these findings and provides actionable recommendations to mitigate the identified risks, enhancing the organization's overall resilience against cyber threats.

% --- SECTION 2: ORGANIZATIONAL INFORMATION ---
\section{Organizational Information}
The following details were provided for the assessment.
\begin{itemize}
    \item \textbf{Organization Name:} Digital Drift
    \item \textbf{Email Domain:} \texttt{DigitalDrift.net}
    \item \textbf{Website Domain:} \url{www.DigitalDrift.net}
    \item \textbf{Primary External IP:} \texttt{186.210.236.166}
\end{itemize}

% --- SECTION 3: SECURITY CONTROL REVIEW ---
\section{Security Control Review}
A review of administrative security controls was conducted based on the provided questionnaire. The results indicate a strong foundation in policy and training, but highlight a critical weakness in endpoint access controls.

\begin{table}[h!]
\centering
\caption{Security Controls Questionnaire Results}
\begin{tabular}{p{0.7\linewidth} c c}
\toprule
\textbf{Control Question} & \textbf{Response} & \textbf{Status} \\
\midrule
Do you require MFA to access email? & Yes & \ding{51} \\
Do you require MFA to log into computers? & No & \textcolor{red}{\ding{55}} \\
Do you require MFA to access sensitive data systems? & Yes & \ding{51} \\
Does your organization have an employee acceptable use policy? & Yes & \ding{51} \\
Does your organization do security awareness training for new employees? & Yes & \ding{51} \\
Does your organization do security awareness training for all employees at least once per year? & Yes & \ding{51} \\
\bottomrule
\end{tabular}
\end{table}

\subsection*{Analysis}
The lack of MFA for computer logins is the most significant finding from this review. Standard username and password combinations are highly susceptible to phishing, credential stuffing, and brute-force attacks. Without a second authentication factor, a compromised password directly translates to unauthorized access to an employee's workstation and, potentially, the entire corporate network.

% --- SECTION 4: TECHNICAL SCAN RESULTS ---
\section{Technical Scan Results}
An external network scan was performed to identify publicly exposed services.
\begin{itemize}
    \item \textbf{Target IP Address:} \seqsplit{\texttt{2001:db8::1}}
    \item \textbf{Scan Status:} Host is up and responsive.
\end{itemize}

The following open port was discovered:

\begin{table}[h!]
\centering
\caption{Open Ports Detected on Target Host}
\begin{tabular}{c c c p{0.5\linewidth}}
\toprule
\textbf{Port} & \textbf{State} & \textbf{Service} & \textbf{Notes} \\
\midrule
22 & Open & SSH & Secure Shell is a common protocol for remote server administration. If not properly secured, it can be a primary vector for unauthorized access. \\
\bottomrule
\end{tabular}
\end{table}

\subsection*{Analysis}
The presence of an open SSH port (22) confirms that a management interface is exposed to the public internet. While often necessary, this exposure increases the attack surface of the organization. The scan did not retrieve service version information, but any exposed service should be assumed to be a target. It must be hardened with strong authentication controls and monitored for anomalous activity.

% --- SECTION 5: RISK ASSESSMENT ---
\section{Risk Assessment}
Based on the analysis of security controls and technical scan data, the following risks have been identified. Note that no pre-existing vulnerabilities were reported for this assessment.

\begin{table}[h!]
\centering
\caption{Summary of Identified Risks}
\begin{tabular}{p{0.45\linewidth} p{0.35\linewidth} c}
\toprule
\textbf{Risk Description} & \textbf{Affected Asset(s)} & \textbf{Severity} \\
\midrule
\textbf{Lack of MFA on Workstation Logins:} Compromised credentials could lead to direct network access. & All employee workstations; Internal network integrity. & \textbf{High} \\
\addlinespace
\textbf{Exposed SSH Service:} Publicly accessible management port is a target for automated attacks. & Server at \seqsplit{\texttt{2001:db8::1}}. & \textbf{Medium} \\
\bottomrule
\end{tabular}
\end{table}

% --- SECTION 6: RECOMMENDATIONS ---
\section{Recommendations}
The following actionable recommendations are provided to address the identified risks and improve the overall security posture of Digital Drift.

\subsection{Implement MFA for All Computer Logins}
\begin{itemize}
    \item \textbf{Priority:} \textbf{Critical}
    \item \textbf{Action:} Deploy a mandatory Multi-Factor Authentication (MFA) solution for all employee computer and laptop logins (e.g., Windows, macOS). This is a fundamental security control that protects against unauthorized access even if user credentials are stolen.
    \item \textbf{Justification:} This measure drastically reduces the risk of lateral movement and ransomware deployment stemming from a single compromised account.
    \item \textbf{Suggested Solutions:} Windows Hello for Business, Duo Security, Okta, or other similar identity providers.
\end{itemize}

\subsection{Harden the Exposed SSH Service}
\begin{itemize}
    \item \textbf{Priority:} \textbf{High}
    \item \textbf{Action:} Immediately review the business requirement for the publicly exposed SSH service on \seqsplit{\texttt{2001:db8::1}}.
    \begin{enumerate}
        \item \textbf{If not required for public access:} Block port 22 at the network firewall and allow access only from internal or VPN-connected IP addresses.
        \item \textbf{If public access is required:} Implement the following hardening measures:
            \begin{itemize}
                \item \textbf{Disable Password Authentication:} Enforce the use of cryptographic keys for authentication, which are significantly more resistant to brute-force attacks.
                \item \textbf{Restrict Access by IP:} If possible, configure firewall rules to only allow SSH connections from a whitelist of trusted IP addresses.
                \item \textbf{Implement Intrusion Detection:} Use tools like \texttt{Fail2Ban} or an Intrusion Prevention System (IPS) to automatically block IPs that exhibit malicious behavior (e.g., repeated failed login attempts).
                \item \textbf{Ensure Regular Patching:} Keep the SSH server software and the underlying operating system fully patched to protect against known vulnerabilities.
            \end{itemize}
    \end{enumerate}
\end{itemize}

\end{document}
```