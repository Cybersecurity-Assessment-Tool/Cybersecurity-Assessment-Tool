```latex
\documentclass[12pt]{article}

% Preamble: Required Packages
\usepackage[margin=1in]{geometry}
\usepackage{pifont} % For checkmarks and crosses
\usepackage{booktabs} % For professional tables
\usepackage{hyperref} % For clickable links
\usepackage{url} % For URL formatting
\usepackage{seqsplit} % For splitting long strings to prevent overflow

% Document Metadata
\title{Cybersecurity Posture Assessment Report}
\author{Cybersecurity Analyst}
\date{\today}

\begin{document}

\maketitle
\thispagestyle{empty}
\newpage
\tableofcontents
\newpage

% --- Section 1: Executive Summary ---
\section*{Executive Summary}

This report provides a comprehensive cybersecurity assessment for \textbf{Hidden Gem}, based on an analysis of organizational security controls, technical network scan results, and a review of pre-existing risks. The assessment was conducted on \today.

The analysis reveals a mixed security posture. The organization has implemented several foundational security controls, including Multi-Factor Authentication (MFA) for email access, an acceptable use policy, and security training for new hires. These are commendable baseline practices.

However, several critical and high-risk gaps were identified that significantly increase the organization's risk exposure. The most severe findings are the absence of MFA for logging into computers and accessing sensitive data systems. These gaps create a direct path for an attacker with compromised credentials to gain deep access to the network and critical assets. Furthermore, the lack of mandatory, annual security awareness training for all employees weakens the human firewall, leaving the organization susceptible to social engineering and phishing attacks.

On a positive note, a technical scan of the host at \texttt{192.168.0.5} showed a minimal attack surface. Notably, Port 80, which was listed as an open risk in a previous assessment, was found to be closed. This suggests that remediation may have occurred, though formal verification is recommended.

This report concludes with a prioritized list of actionable recommendations to address the identified risks and strengthen the overall security posture of \textbf{Hidden Gem}.

% --- Section 2: Organizational Information ---
\section*{Organizational Information}

The following details were provided for the assessment. This information is used to establish the context and scope of the review.

\begin{table}[h!]
\centering
\begin{tabular}{@{}ll@{}}
\toprule
\textbf{Attribute} & \textbf{Value} \\
\midrule
Organization Name & \textbf{Hidden Gem} \\
Email Domain & \texttt{HiddenGem.com} \\
Website Domain & \url{www.HiddenGem.com} \\
External IP Address & \texttt{150.189.42.178} \\
\bottomrule
\end{tabular}
\caption{Client Organizational Details}
\end{table}

% --- Section 3: Security Control Review ---
\section*{Security Control Review}

A review of administrative and technical security controls was conducted via a questionnaire. The results below highlight the current state of implemented policies and procedures. "Yes" answers indicate a control is in place, while "No" answers represent a gap in the security framework.

\begin{table}[h!]
\centering
\begin{tabular}{@{}p{0.8\linewidth}c@{}}
\toprule
\textbf{Control Question} & \textbf{Status} \\
\midrule
Do you require MFA to access email? & \ding{51} \\
Do you require MFA to log into computers? & \textbf{\color{red}\ding{55}} \\
Do you require MFA to access sensitive data systems? & \textbf{\color{red}\ding{55}} \\
Does your organization have an employee acceptable use policy? & \ding{51} \\
Does your organization do security awareness training for new employees? & \ding{51} \\
Does your organization do security awareness training for all employees at least once per year? & \textbf{\color{red}\ding{55}} \\
\bottomrule
\end{tabular}
\caption{Security Controls Questionnaire Results}
\end{table}

\subsection*{Analysis of Control Gaps}
The questionnaire identified three significant control gaps:
\begin{itemize}
    \item \textbf{No MFA on Computers \& Sensitive Systems:} This is a critical vulnerability. If an employee's password is stolen, an attacker can log directly into their computer and potentially access sensitive systems without needing a second authentication factor. This dramatically increases the likelihood of a successful breach.
    \item \textbf{No Annual Security Awareness Training:} Security is a continuous process. Without regular, recurring training, employee awareness of new threats (like sophisticated phishing campaigns) diminishes, making them a primary and vulnerable target for attackers.
\end{itemize}

% --- Section 4: Technical Scan Results ---
\section*{Technical Scan Results}

A network scan was performed on the target host to identify open ports and exposed services.

\begin{itemize}
    \item \textbf{Target IP Address:} \texttt{192.168.0.5}
    \item \textbf{Scan Status:} The host was responsive (up) at the time of the scan.
\end{itemize}

\begin{table}[h!]
\centering
\begin{tabular}{@{}lll@{}}
\toprule
\textbf{Port} & \textbf{State} & \textbf{Notes} \\
\midrule
80/tcp & closed & No service detected. \\
\bottomrule
\end{tabular}
\caption{Nmap Scan Results for \texttt{192.168.0.5}}
\end{table}

\subsection*{Analysis of Technical Findings}
The scan revealed a minimal network attack surface for the specified target. The most significant finding is that Port 80 (HTTP) was found to be \textbf{closed}. This contradicts a pre-existing risk record which stated this port was open. This is a positive development and may indicate that the previously identified risk has been remediated. It is recommended to confirm this finding across all relevant systems.

% --- Section 5: Consolidated Risk Assessment ---
\section*{Consolidated Risk Assessment}

The following table synthesizes findings from the security control review, technical scan, and pre-existing risk data into a consolidated list of current risks.

\begin{table}[h!]
\centering
\begin{tabular}{@{}lp{0.25\linewidth}ll@{}}
\toprule
\textbf{Risk ID} & \textbf{Risk Name} & \textbf{Severity} & \textbf{Description} \\
\midrule
RISK-001 & No MFA on Workstations & \textbf{Critical} & Lack of MFA on computer logins allows for endpoint compromise via stolen credentials. \\
\addlinespace
RISK-002 & No MFA on Sensitive Systems & \textbf{Critical} & Sensitive data is exposed to unauthorized access from a single compromised account. \\
\addlinespace
RISK-003 & Lack of Annual Security Training & \textbf{High} & Employees are not kept up-to-date on evolving threats, increasing susceptibility to phishing. \\
\addlinespace
RISK-004 & Unencrypted Web Server & Medium & \textit{(Potentially Remediated)} Prior risk of unencrypted traffic on Port 80. Scan shows port is now closed. \\
\bottomrule
\end{tabular}
\caption{Summary of Identified Risks}
\end{table}

% --- Section 6: Recommendations ---
\section*{Recommendations}

The following actions are recommended to mitigate the identified risks and improve the overall security posture of the organization. Recommendations are prioritized based on severity and potential impact.

\subsection*{Priority 1: Critical}
\begin{enumerate}
    \item \textbf{Implement MFA on All Endpoints and Sensitive Systems:}
    \begin{itemize}
        \item \textbf{Action:} Deploy a robust MFA solution (e.g., authenticator app, hardware token) for all employee computer logins and for access to all systems containing sensitive or critical data.
        \item \textbf{Impact:} Drastically reduces the risk of unauthorized access from compromised credentials. This is the single most effective control to implement.
    \end{itemize}
\end{enumerate}

\subsection*{Priority 2: High}
\begin{enumerate}
    \setcounter{enumi}{1}
    \item \textbf{Establish a Mandatory Annual Security Awareness Program:}
    \begin{itemize}
        \item \textbf{Action:} Procure and deploy a security awareness training platform. Mandate that all employees complete a training module and phishing simulation at least once per year.
        \item \textbf{Impact:} Reinforces security best practices, reduces the likelihood of successful phishing attacks, and fosters a security-conscious culture.
    \end{itemize}
\end{enumerate}

\subsection*{Priority 3: Medium}
\begin{enumerate}
    \setcounter{enumi}{2}
    \item \textbf{Verify and Close Historical Risks:}
    \begin{itemize}
        \item \textbf{Action:} Formally verify that Port 80 is closed on all production web servers. If confirmed, update the risk register to mark the "Unencrypted Web Server" risk as remediated and closed.
        \item \textbf{Impact:} Ensures the risk register is accurate and reflects the current security posture.
    \end{itemize}
\end{enumerate}

\end{document}
```