```latex
\documentclass[12pt]{article}

% --- PACKAGE IMPORTS ---
\usepackage[margin=1in]{geometry}
\usepackage{pifont} % For checkmarks and crosses
\usepackage{booktabs} % For professional tables
\usepackage{hyperref} % For clickable links
\usepackage{url} % For URL formatting
\usepackage{seqsplit} % To split long monospaced text
\usepackage{graphicx}
\usepackage{xcolor}

% --- DOCUMENT SETUP ---
\hypersetup{
    colorlinks=true,
    linkcolor=blue,
    filecolor=magenta,      
    urlcolor=cyan,
    pdftitle={Cybersecurity Assessment Report},
    pdfpagemode=FullScreen,
}

\newcommand{\yes}{\ding{51}}
\newcommand{\no}{\ding{55}}

% --- DOCUMENT START ---
\begin{document}

% --- TITLE PAGE ---
\begin{titlepage}
    \centering
    \vspace*{1cm}
    \Huge{\textbf{Cybersecurity Assessment Report}}
    \vspace{0.5cm}
    \Large{Prepared for: Stone Arch Masonry}
    \vspace{1.5cm}
    \textbf{Date of Report: \today}
    \vfill
    \large
    \textbf{CONFIDENTIAL}
    \vspace{0.8cm}
    \rule{\linewidth}{0.5mm}
    \vspace{0.2cm}
    \textcopyright\ \the\year\ Cybersecurity Analysis Division. All Rights Reserved.
\end{titlepage}

\tableofcontents
\newpage

% --- EXECUTIVE SUMMARY ---
\section{Executive Summary}
This report provides a comprehensive cybersecurity assessment for Stone Arch Masonry, based on a correlation of network scan data, organizational security controls, and pre-existing risk information.

The analysis revealed a \textbf{critical-risk finding}. An internal network scan of host \texttt{10.5.5.5} discovered an open service on port \texttt{8080} with the title \texttt{"TOP SECRET DB"}. This finding directly contradicts a previous risk assessment which incorrectly labeled this port as a secure false positive. This discrepancy points to a severe potential data exposure and a significant gap in the vulnerability management lifecycle.

Furthermore, several high-risk procedural and policy gaps were identified through the security questionnaire. These include the lack of Multi-Factor Authentication (MFA) for computer logins, the absence of an employee acceptable use policy, and no security awareness training for new hires.

While the organization has implemented some positive controls, such as MFA for email, the identified critical and high-risk vulnerabilities require immediate attention to mitigate the risk of a significant security breach.

% --- ORGANIZATIONAL INFORMATION ---
\section{Organizational Information}
The following information was provided for the assessment.

\begin{table}[h!]
\centering
\begin{tabular}{@{}ll@{}}
\toprule
\textbf{Attribute} & \textbf{Value} \\
\midrule
Organization Name & Stone Arch Masonry \\
Email Domain & \seqsplit{\texttt{StoneArchMasonry.com}} \\
Website Domain & \seqsplit{\url{www.StoneArchMasonry.com}} \\
External IP Address & \seqsplit{\texttt{76.167.23.162}} \\
\bottomrule
\end{tabular}
\caption{Client Organizational Data}
\end{table}

% --- SECURITY CONTROL REVIEW ---
\section{Security Control Review}
The following table summarizes the organization's current security controls based on the provided questionnaire. "No" answers indicate significant gaps in the security posture.

\begin{table}[h!]
\centering
\begin{tabular}{@{}p{0.6\linewidth} c l@{}}
\toprule
\textbf{Control Question} & \textbf{Response} & \textbf{Assessment} \\
\midrule
Do you require MFA to access email? & \yes & Control In Place \\
Do you require MFA to log into computers? & \textcolor{red}{\no} & \textbf{High-Risk Gap} \\
Do you require MFA to access sensitive data systems? & \yes & Control In Place \\
Does your organization have an employee acceptable use policy? & \textcolor{red}{\no} & \textbf{High-Risk Gap} \\
Does your organization do security awareness training for new employees? & \textcolor{red}{\no} & \textbf{High-Risk Gap} \\
Does your organization do security awareness training for all employees at least once per year? & \yes & Control In Place \\
\bottomrule
\end{tabular}
\caption{Security Questionnaire Analysis}
\end{table}

% --- TECHNICAL SCAN RESULTS ---
\section{Technical Scan Results}
A network scan was performed to identify open ports and exposed services on the target system.

\subsection{Host: \texttt{10.5.5.5}}
The scan identified one host as "up" with the following open port:
\begin{itemize}
    \item \textbf{Port:} \texttt{8080/tcp}
    \item \textbf{State:} open
    \item \textbf{Service Details:} An HTTP service was identified. A script check revealed the title of the web page being served is \textbf{\texttt{"TOP SECRET DB"}}.
\end{itemize}
\textbf{Analysis:} This is a critical finding. The title strongly suggests that a sensitive, possibly unauthenticated, database interface is exposed on the internal network. This directly contradicts the existing risk documentation, which listed this port as a secure false positive. This indicates a severe and unmitigated data exposure risk.

% --- CONSOLIDATED RISK ASSESSMENT ---
\section{Consolidated Risk Assessment}
The following table synthesizes findings from the technical scan, control review, and existing risk data into a prioritized list of security risks.
\vspace{0.5cm}

\begin{tabular}{@{}p{0.25\linewidth} p{0.5\linewidth} p{0.15\linewidth}@{}}
\toprule
\textbf{Risk / Vulnerability} & \textbf{Description} & \textbf{Severity} \\
\midrule
\textbf{Exposed Sensitive Data Interface} & Port \texttt{8080} on host \texttt{10.5.5.5} is open and serves a page titled \texttt{"TOP SECRET DB"}. This represents a direct and severe risk of unauthorized access to sensitive information. & \textbf{Critical} \\
\addlinespace
\textbf{Lack of Endpoint MFA} & The absence of MFA for computer logins significantly increases the risk of unauthorized system access from compromised credentials (e.g., phishing, password reuse). & \textbf{High} \\
\addlinespace
\textbf{Missing Acceptable Use Policy (AUP)} & Without a formal AUP, there is no clear guidance for employees on the proper use of company assets, creating legal and operational risks. & \textbf{High} \\
\addlinespace
\textbf{No Onboarding Security Training} & New employees are not receiving security training, making them highly susceptible to social engineering attacks from their first day. & \textbf{High} \\
\addlinespace
\textbf{Invalidated Prior Risk Assessment} & The pre-existing risk stating Port 8080 was a "false positive" is incorrect. This points to a failure in the vulnerability validation and management process. & \textbf{High} \\
\bottomrule
\end{tabular}

% --- RECOMMENDATIONS ---
\section{Recommendations}
Based on the consolidated risk assessment, the following actions are recommended to improve the security posture of Stone Arch Masonry.

\subsection{Immediate Actions (Critical Priority)}
\begin{enumerate}
    \item \textbf{Isolate and Investigate Host \texttt{10.5.5.5}:} Immediately restrict all network access to the service on port \texttt{8080}. Conduct a forensic investigation to determine the nature of the "TOP SECRET DB", what data it contains, and if it has been accessed by unauthorized parties.
    \item \textbf{Overhaul Vulnerability Management Process:} Review the process that led to the critical exposure on port \texttt{8080} being misclassified as a "false positive". Implement a robust validation and remediation workflow to ensure findings are accurately assessed and tracked.
\end{enumerate}

\subsection{High Priority Actions}
\begin{enumerate}
    \item \textbf{Deploy Endpoint MFA:} Implement and enforce MFA for all employee computer and system logins. Solutions like Windows Hello for Business, Duo, or other third-party tools should be evaluated.
    \item \textbf{Develop and Implement an AUP:} Create a formal Employee Acceptable Use Policy that clearly defines rules for using company networks, devices, and data. Require all employees to read and acknowledge the policy.
    \item \textbf{Integrate Security into Onboarding:} Develop a mandatory security awareness training module for all new hires. This training should be a prerequisite for gaining access to company systems and should cover topics like phishing, password security, and the new AUP.
\end{enumerate}

\end{document}
```