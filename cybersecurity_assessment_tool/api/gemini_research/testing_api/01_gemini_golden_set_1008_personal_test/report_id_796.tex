```latex
\documentclass[12pt, a4paper]{article}

% Preamble: Required Packages
\usepackage[margin=1in]{geometry}
\usepackage{pifont} % For checkmarks and crosses
\usepackage{booktabs} % For professional tables
\usepackage{hyperref} % For clickable links
\usepackage{url} % For URL formatting
\usepackage{seqsplit} % To split long text strings
\usepackage{graphicx}
\usepackage{xcolor}
\usepackage{fancyhdr}
\usepackage{lastpage}

% --- Document Setup ---

% Define colors for risk levels
\definecolor{critical}{HTML}{990000}
\definecolor{high}{HTML}{D14343}
\definecolor{medium}{HTML}{EFAF4E}
\definecolor{low}{HTML}{5DADE2}
\definecolor{info}{HTML}{82E0AA}

% Hyperlink setup
\hypersetup{
    colorlinks=true,
    linkcolor=blue,
    filecolor=magenta,      
    urlcolor=cyan,
    pdftitle={Cybersecurity Posture Report},
    pdfpagemode=FullScreen,
}

% Header and Footer
\pagestyle{fancy}
\fancyhf{} % Clear all header and footer fields
\fancyhead[L]{Cybersecurity Posture Report}
\fancyhead[R]{Apex Legends Group}
\fancyfoot[C]{\thepage\ of \pageref{LastPage}}
\renewcommand{\headrulewidth}{0.4pt}
\renewcommand{\footrulewidth}{0.4pt}

% --- Document Body ---

\begin{document}

\begin{titlepage}
    \centering
    \vfill
    {\Huge\bfseries Cybersecurity Posture Report\par}
    \vspace{1cm}
    {\Large Prepared for: \textbf{Apex Legends Group}\par}
    \vspace{2cm}
    {\large \today\par}
    \vfill
    \hrule
    \vspace{0.5cm}
    \textit{This report contains sensitive information and should be handled with care. Distribution is restricted to authorized personnel only.}
\end{titlepage}

\tableofcontents
\newpage

\section{Executive Summary}

This report provides a comprehensive analysis of the cybersecurity posture of \textbf{Apex Legends Group}, based on a combination of network scanning, organizational data review, and an assessment of existing risks. The evaluation was conducted on \today.

The key findings indicate a mixed security posture. On a positive note, the technical network scan of the target host (\texttt{192.168.1.100}) revealed no open ports, suggesting a well-hardened system with a minimal external attack surface. This is a significant strength.

However, the organizational security control review identified two critical gaps that introduce substantial risk:
\begin{itemize}
    \item \textbf{High Risk:} The absence of Multi-Factor Authentication (MFA) for computer logins. This significantly increases the risk of unauthorized access via compromised credentials.
    \item \textbf{Medium Risk:} The lack of a formal Employee Acceptable Use Policy (AUP). This creates ambiguity regarding employee responsibilities for securing company assets and data.
\end{itemize}

No pre-existing vulnerabilities were reported. Recommendations in this report focus on addressing the identified policy and access control weaknesses to bolster the organization's overall defense-in-depth strategy.

\section{Organizational Information}

The following details were provided for the assessment.

\begin{table}[h!]
\centering
\caption{Client Organizational Data}
\label{tab:orgdata}
\begin{tabular}{@{}ll@{}}
\toprule
\textbf{Attribute} & \textbf{Value} \\ \midrule
Organization Name    & \textbf{Apex Legends Group} \\
Email Domain         & \texttt{ApexLegendsGroup.net} \\
Website Domain       & \url{www.ApexLegendsGroup.net} \\
External IP Address  & \texttt{94.214.246.71} \\ \bottomrule
\end{tabular}
\end{table}

\section{Security Control Review}

A review of internal security controls was conducted based on a standardized questionnaire. The results highlight areas of strength and identify significant gaps requiring immediate attention. A "No" response indicates a deviation from security best practices.

\begin{table}[h!]
\centering
\caption{Security Controls Questionnaire Results}
\label{tab:controls}
\begin{tabular}{@{}p{0.7\textwidth}c@{}}
\toprule
\textbf{Control Question} & \textbf{Status} \\ \midrule
Do you require MFA to access email? & \textcolor{green}{\ding{51}} \\
Do you require MFA to log into computers? & \textcolor{red}{\ding{55}} \\
Do you require MFA to access sensitive data systems? & \textcolor{green}{\ding{51}} \\
Does your organization have an employee acceptable use policy? & \textcolor{red}{\ding{55}} \\
Does your organization do security awareness training for new employees? & \textcolor{green}{\ding{51}} \\
Does your organization do security awareness training for all employees at least once per year? & \textcolor{green}{\ding{51}} \\ \bottomrule
\end{tabular}
\end{table}

\section{Technical Scan Results}

An external network scan was performed to identify open ports and exposed services on the target system.

\begin{itemize}
    \item \textbf{Scan Target:} \texttt{192.168.1.100}
    \item \textbf{Scan Date:} \today
    \item \textbf{Scanner Used:} Nmap
\end{itemize}

\subsection{Key Findings}
The scan confirmed that the host is online and responsive. However, \textbf{no open ports were detected}. All 1000 scanned TCP ports were found to be in a "closed" state.

\textbf{Conclusion:} This is a positive security finding. A host with no open ports presents a minimal attack surface to the network, significantly reducing the risk of remote exploitation. This indicates effective firewalling or host-based security configurations.

\section{Risk Assessment}

This section synthesizes findings from the security control review and technical scans to provide a consolidated list of identified risks.

\begin{table}[h!]
\centering
\caption{Summary of Identified Risks}
\label{tab:risks}
\begin{tabular}{@{}p{0.25\textwidth}p{0.1\textwidth}p{0.55\textwidth}@{}}
\toprule
\textbf{Risk Name} & \textbf{Severity} & \textbf{Overview} \\ \midrule
\addlinespace[0.3em]
Lack of Endpoint MFA & \colorbox{high}{\color{white}\textbf{High}} & The absence of MFA on computer logins means a compromised password is all an attacker needs for initial access to an employee's workstation and potentially the internal network. \\
\addlinespace[0.3em]
Missing Acceptable Use Policy (AUP) & \colorbox{medium}{\color{white}\textbf{Medium}} & Without a formal AUP, employees may be unaware of their security responsibilities. This increases the risk of unintentional data exposure, malware infections, and non-compliance with regulations. \\
\addlinespace[0.3em]
\bottomrule
\end{tabular}
\end{table}

\section{Recommendations}

The following actionable recommendations are provided to mitigate the identified risks and improve the overall security posture of \textbf{Apex Legends Group}.

\subsection{High Priority Recommendations}

\begin{itemize}
    \item \textbf{Risk Addressed:} Lack of Endpoint MFA
    \item \textbf{Recommendation:} Implement and enforce mandatory Multi-Factor Authentication for all user logins to company computers (desktops and laptops). This is one of the most effective controls for preventing unauthorized access.
    \item \textbf{Actionable Steps:}
    \begin{enumerate}
        \item Evaluate MFA solutions compatible with your existing operating systems (e.g., Windows Hello for Business, Duo Security, Okta).
        \item Develop a phased rollout plan, starting with privileged users (administrators) and expanding to all employees.
        \item Provide clear instructions and support to users during the transition.
    \end{enumerate}
\end{itemize}

\subsection{Medium Priority Recommendations}

\begin{itemize}
    \item \textbf{Risk Addressed:} Missing Acceptable Use Policy (AUP)
    \item \textbf{Recommendation:} Develop, approve, and implement a formal Acceptable Use Policy for all employees and contractors.
    \item \textbf{Actionable Steps:}
    \begin{enumerate}
        \item Draft an AUP that clearly defines rules for the use of company equipment, network access, email, and data handling.
        \item Include sections on prohibited activities, password security, and incident reporting procedures.
        \item Distribute the policy to all staff and require a signed acknowledgement of receipt and understanding.
        \item Review and update the policy annually or as significant organizational changes occur.
    \end{enumerate}
\end{itemize}

\end{document}
```