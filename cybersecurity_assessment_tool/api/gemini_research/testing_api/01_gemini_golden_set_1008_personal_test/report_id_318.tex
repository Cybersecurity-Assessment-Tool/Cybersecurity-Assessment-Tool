```latex
\documentclass[12pt]{article}

% Preamble: Required Packages
\usepackage[margin=1in]{geometry}
\usepackage{pifont} % For checkmarks and crosses
\usepackage{booktabs} % For professional tables
\usepackage{hyperref} % For clickable links
\usepackage{url}      % For formatting URLs
\usepackage{seqsplit} % For splitting long strings in tt font
\usepackage[utf8]{inputenc}

% Document Metadata
\title{Cybersecurity Posture Assessment Report}
\author{Cybersecurity Analyst}
\date{\today}

\hypersetup{
    colorlinks=true,
    linkcolor=black,
    urlcolor=blue,
    pdftitle={Cybersecurity Posture Assessment Report},
    pdfauthor={Cybersecurity Analyst},
}

\begin{document}

\maketitle
\thispagestyle{empty}
\newpage
\tableofcontents
\newpage

% --- 1. Executive Summary ---
\section{Executive Summary}
This report provides a comprehensive cybersecurity assessment for \textbf{Neon Pulse Entertainment}, synthesized from a network vulnerability scan, a review of organizational security controls, and an analysis of pre-existing risk data.

The assessment reveals a mixed security posture. On the technical front, the scanned host (\texttt{192.168.0.5}) shows a secure configuration, with no open, vulnerable ports detected. Notably, Port 80, previously identified as a risk, was found to be closed, indicating successful remediation or an outdated risk register.

However, significant and critical gaps were identified in the organization's administrative and policy-based security controls. The absence of Multi-Factor Authentication (MFA) for email and computer access represents a critical vulnerability, exposing the organization to high-impact threats like business email compromise and ransomware. Furthermore, the lack of a formal Acceptable Use Policy and a consistent security awareness training program creates a high-risk environment susceptible to insider threats and social engineering attacks.

Immediate remediation should focus on implementing MFA across all critical systems, followed by the development and enforcement of foundational security policies and training programs.

% --- 2. Organizational Information ---
\section{Organizational Information}
The following details were provided for the assessment scope.
\begin{itemize}
    \item \textbf{Organization Name:} Neon Pulse Entertainment
    \item \textbf{Email Domain:} \texttt{NeonPulseEntertainment.com}
    \item \textbf{Website Domain:} \url{www.NeonPulseEntertainment.com}
    \item \textbf{External IP Address:} \texttt{105.56.207.48}
\end{itemize}

% --- 3. Security Control Review ---
\section{Security Control Review}
A review of administrative security controls was conducted based on a standardized questionnaire. The results highlight critical deficiencies in identity management and employee security awareness. A "No" answer indicates a significant gap in the security framework.

\begin{table}[h!]
\centering
\caption{Organizational Security Controls Questionnaire}
\begin{tabular}{@{}lc@{}}
\toprule
\textbf{Control Question} & \textbf{Status} \\
\midrule
Do you require MFA to access email? & \ding{55} \\
Do you require MFA to log into computers? & \ding{55} \\
Do you require MFA to access sensitive data systems? & \ding{51} \\
Does your organization have an employee acceptable use policy? & \ding{55} \\
Does your organization do security awareness training for new employees? & \ding{55} \\
Does your organization do security awareness training for all employees annually? & \ding{55} \\
\bottomrule
\end{tabular}
\end{table}

\vspace{1em}
\noindent \textbf{Key Findings:}
\begin{itemize}
    \item \textbf{MFA Gaps:} The lack of enforced MFA for email and general computer access is a critical risk. Email is the primary vector for phishing and account takeover attacks.
    \item \textbf{Policy Gaps:} The absence of an Acceptable Use Policy means there are no formal rules governing how employees use company technology, increasing the risk of misuse and insider threats.
    \item \textbf{Training Gaps:} Without onboarding and annual security training, employees are significantly more likely to fall victim to social engineering attacks, such as phishing and pretexting.
\end{itemize}

% --- 4. Technical Scan Results ---
\section{Technical Scan Results}
An Nmap scan was performed to identify open ports and services on the target system. The scan provides a snapshot of the host's network exposure.

\begin{itemize}
    \item \textbf{Target IP Address:} \texttt{192.168.0.5}
    \item \textbf{Scan Date:} \today
\end{itemize}

\begin{table}[h!]
\centering
\caption{Nmap Port Scan Results for \texttt{192.168.0.5}}
\begin{tabular}{@{}llll@{}}
\toprule
\textbf{Port} & \textbf{State} & \textbf{Service} & \textbf{Version} \\
\midrule
80/tcp & closed & http & N/A \\
\bottomrule
\end{tabular}
\end{table}

\vspace{1em}
\noindent \textbf{Analysis:} The scan indicates a secure network posture for the target host. All scanned ports were found to be in a \texttt{closed} state, which is a positive security finding. This result contradicts a pre-existing risk entry (\textit{Unencrypted Web Server}) that stated Port 80 was open. This suggests the risk has been remediated and the risk register requires an update.

% --- 5. Consolidated Risk Assessment ---
\section{Consolidated Risk Assessment}
The following table summarizes the identified risks, combining findings from the security control review, technical scan, and pre-existing risk data. Risks are prioritized based on their potential impact on the organization.

\begin{table}[h!]
\centering
\caption{Summary of Identified Risks}
\begin{tabular}{@{}p{0.1\linewidth}p{0.3\linewidth}p{0.15\linewidth}p{0.35\linewidth}@{}}
\toprule
\textbf{ID} & \textbf{Risk Name} & \textbf{Severity} & \textbf{Description} \\
\midrule
RISK-001 & No MFA for Email and Endpoints & \textbf{Critical} & Lack of MFA on primary access points exposes the organization to account takeover, data breaches, and ransomware. \\
\addlinespace
RISK-002 & Inadequate Security Awareness Program & \textbf{High} & Without formal training, employees are highly susceptible to phishing and other social engineering attacks. \\
\addlinespace
RISK-003 & Missing Acceptable Use Policy & \textbf{High} & Absence of a clear policy creates ambiguity and increases the likelihood of insider threats and misuse of company assets. \\
\addlinespace
RISK-004 & Outdated Risk Register & Medium & The existing risk "Unencrypted Web Server" is inconsistent with current scan data, indicating a need for improved risk management processes. \\
\bottomrule
\end{tabular}
\end{table}

% --- 6. Recommendations ---
\section{Recommendations}
Based on the analysis, the following actions are recommended to mitigate the identified risks and improve the overall security posture of \textbf{Neon Pulse Entertainment}.

\subsection{Priority 1: Critical Risks}
\begin{itemize}
    \item \textbf{Implement MFA:} Immediately enable and enforce Multi-Factor Authentication (MFA) for all users across all critical platforms, starting with:
    \begin{itemize}
        \item Email systems (e.g., Microsoft 365, Google Workspace).
        \item VPN and remote access solutions.
        \item All computer/endpoint logins.
    \end{itemize}
\end{itemize}

\subsection{Priority 2: High Risks}
\begin{itemize}
    \item \textbf{Establish Security Awareness Training:}
    \begin{itemize}
        \item Develop a mandatory security awareness training module for all new employees as part of the onboarding process.
        \item Implement a recurring, annual security awareness training program for all staff to ensure continuous education on evolving threats.
    \end{itemize}
    \item \textbf{Develop an Acceptable Use Policy (AUP):}
    \begin{itemize}
        \item Create and distribute a formal AUP that clearly defines the rules for using company networks, devices, and data.
        \item Require all employees to read and acknowledge the policy.
    \end{itemize}
\end{itemize}

\subsection{Priority 3: Medium Risks}
\begin{itemize}
    \item \textbf{Update Risk Register:}
    \begin{itemize}
        \item Review the finding that Port 80 is closed on host \texttt{192.168.0.5}.
        \item Formally update the internal risk register to mark the "Unencrypted Web Server" risk as remediated or closed.
        \item Implement a process for periodically validating and updating the risk register.
    \end{itemize}
\end{itemize}

\end{document}
```