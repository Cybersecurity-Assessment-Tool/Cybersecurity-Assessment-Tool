```latex
\documentclass[12pt]{article}

% --- PACKAGE IMPORTS ---
\usepackage[margin=1in]{geometry}
\usepackage{pifont} % For checkmarks and crosses
\usepackage{booktabs} % For professional tables
\usepackage[hidelinks]{hyperref} % For clickable links without boxes
\usepackage{url} % For URL formatting
\usepackage{seqsplit} % For splitting long strings in texttt
\usepackage{graphicx}
\usepackage{fancyhdr}

% --- DOCUMENT METADATA ---
\title{Cybersecurity Posture Assessment Report}
\author{\textbf{Golden Gate Gaming}}
\date{\today}

% --- HEADER & FOOTER ---
\pagestyle{fancy}
\fancyhf{}
\lhead{Cybersecurity Postpractice Assessment}
\rhead{\textbf{Golden Gate Gaming}}
\cfoot{\thepage}

\begin{document}

\maketitle
\thispagestyle{empty}
\newpage

\tableofcontents
\newpage

% --- SECTION 1: EXECUTIVE OVERVIEW ---
\section{Executive Overview}
This report provides a comprehensive analysis of the cybersecurity posture for \textbf{Golden Gate Gaming}. The assessment is based on a synthesis of network scan data, a review of organizational security controls, and an evaluation of pre-existing risk information.

The analysis revealed several high-priority risks that require immediate attention. The most critical finding is an externally exposed MySQL database server. This server is running an End-of-Life (EOL) version of MySQL, which no longer receives security updates, significantly elevating its risk profile.

This technical vulnerability is compounded by critical gaps in administrative controls. Specifically, the lack of mandatory Multi-Factor Authentication (MFA) for sensitive data systems and the absence of a formal Employee Acceptable Use Policy create a high-risk environment. The combination of an exposed, outdated database and weak access controls presents a clear and present danger of a potential data breach.

We strongly recommend immediate remediation of the identified vulnerabilities, prioritizing the isolation of the database server from the public internet and the implementation of robust access controls.

% --- SECTION 2: ORGANIZATIONAL INFORMATION ---
\section{Organizational Information}
The following information was provided for the assessment.

\begin{tabular}{@{}ll}
\toprule
\textbf{Item} & \textbf{Detail} \\
\midrule
Organization Name & Golden Gate Gaming \\
Email Domain & \texttt{GoldenGateGaming.net} \\
Website Domain & \seqsplit{\url{www.GoldenGateGaming.net}} \\
External IP Address & \texttt{195.231.78.180} \\
\bottomrule
\end{tabular}

% --- SECTION 3: SECURITY CONTROL REVIEW ---
\section{Security Control Review}
A review of administrative and technical security controls was conducted based on a standardized questionnaire. The results indicate a solid foundation in some areas but also highlight critical gaps that weaken the organization's overall security posture.

\begin{table}[h!]
\centering
\caption{Security Controls Questionnaire Results}
\begin{tabular}{@{}p{0.75\linewidth}c@{}}
\toprule
\textbf{Control Question} & \textbf{Response} \\
\midrule
Do you require MFA to access email? & \ding{51} \\
Do you require MFA to log into computers? & \ding{51} \\
\textbf{Do you require MFA to access sensitive data systems?} & \textbf{\ding{55}} \\
\textbf{Does your organization have an employee acceptable use policy?} & \textbf{\ding{55}} \\
Does your organization do security awareness training for new employees? & \ding{51} \\
Does your organization do security awareness training for all employees at least once per year? & \ding{51} \\
\bottomrule
\end{tabular}
\end{table}

\subsection*{Analysis of Control Gaps}
The responses marked with a \ding{55} represent significant security weaknesses:
\begin{itemize}
    \item \textbf{No MFA for Sensitive Systems:} This is a critical vulnerability. Sensitive systems, such as databases and file servers, are high-value targets for attackers. The lack of MFA means that a compromised password is all an attacker needs to gain access.
    \item \textbf{No Acceptable Use Policy (AUP):} An AUP is a foundational administrative control that sets clear expectations for employee behavior when using company resources. Its absence can lead to unintentional security incidents and creates ambiguity regarding security responsibilities.
\end{itemize}

% --- SECTION 4: TECHNICAL SCAN RESULTS ---
\section{Technical Scan Results}
An external network scan was performed to identify open ports and exposed services on the target system.

\begin{itemize}
    \item \textbf{Target IP Address:} \texttt{172.16.50.20}
\end{itemize}

\begin{table}[h!]
\centering
\caption{Open Port Scan Findings}
\begin{tabular}{@{}lllll@{}}
\toprule
\textbf{Port} & \textbf{State} & \textbf{Service} & \textbf{Product} & \textbf{Version} \\
\midrule
3306/tcp & open & mysql & MySQL & 5.7.33 \\
\bottomrule
\end{tabular}
\end{table}

\subsection*{Analysis of Technical Findings}
The scan identified a single open port, 3306, which is the default port for the MySQL database service.
\begin{itemize}
    \item \textbf{Direct Database Exposure:} Exposing a database directly to the internet is highly discouraged as it provides a direct vector for attackers to attempt brute-force attacks, exploit vulnerabilities, or perform denial-of-service attacks.
    \item \textbf{End-of-Life (EOL) Software:} The detected version, \textbf{MySQL 5.7.33}, reached its official End-of-Life in October 2023. This means it no longer receives security patches from the vendor, and any newly discovered vulnerabilities will remain unpatched, making it an extremely high-risk asset.
\end{itemize}

% --- SECTION 5: CORRELATED RISK ASSESSMENT ---
\section{Correlated Risk Assessment}
By correlating the technical findings with the security control gaps and pre-existing risk data, we have identified the following key risks to the organization.

\begin{table}[h!]
\centering
\caption{Summary of Identified Risks}
\begin{tabular}{@{}p{0.2\linewidth}p{0.6\linewidth}l@{}}
\toprule
\textbf{Risk Name} & \textbf{Description} & \textbf{Severity} \\
\midrule
\textbf{Exposed \& Outdated Database} & The MySQL database (v5.7.33) is publicly accessible and is running End-of-Life software. This combines the risk of direct attack with the certainty of unpatchable vulnerabilities. & \textbf{Critical} \\
\addlinespace
\textbf{Insufficient Access Control} & The lack of MFA on sensitive systems, including the exposed database, means that a single compromised password could lead to a full database breach. & \textbf{High} \\
\addlinespace
\textbf{Lack of Formal Usage Policy} & The absence of an Acceptable Use Policy increases the likelihood of insider threats and unintentional misconfigurations by employees who are unaware of security best practices. & \textbf{Medium} \\
\bottomrule
\end{tabular}
\end{table}

% --- SECTION 6: RECOMMENDATIONS ---
\section{Recommendations}
The following prioritized actions are recommended to mitigate the identified risks and improve the overall security posture of \textbf{Golden Gate Gaming}.

\subsection*{Immediate Priority}
\begin{enumerate}
    \item \textbf{Isolate the Database Server:} Immediately implement firewall rules to deny all public access to port 3306 on the server at \texttt{172.16.50.20}. Access should only be permitted from trusted internal IP addresses or through a secure Virtual Private Network (VPN).
\end{enumerate}

\subsection*{High Priority}
\begin{enumerate}
    \setcounter{enumi}{1}
    \item \textbf{Upgrade End-of-Life Software:} Develop a plan to migrate the MySQL 5.7.33 database to a currently supported version (e.g., MySQL 8.x). This is essential to ensure the system receives critical security updates.
    \item \textbf{Enforce Multi-Factor Authentication:} Implement and enforce MFA across all sensitive data systems, with the database server being the top priority. This provides a critical layer of defense against credential theft.
\end{enumerate}

\subsection*{Medium Priority}
\begin{enumerate}
    \setcounter{enumi}{3}
    \item \textbf{Develop an Acceptable Use Policy (AUP):} Create a formal AUP that clearly defines the rules and responsibilities for all employees and contractors when using company IT assets. Ensure this policy is distributed and acknowledged by all personnel.
\end{enumerate}

\end{document}
```