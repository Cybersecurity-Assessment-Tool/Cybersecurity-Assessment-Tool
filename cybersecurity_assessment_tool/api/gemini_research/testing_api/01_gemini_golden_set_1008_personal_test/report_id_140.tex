```latex
\documentclass[12pt, a4paper]{article}

% Preamble: Required Packages
\usepackage[margin=1in]{geometry}
\usepackage{pifont} % For checkmarks and crosses
\usepackage{booktabs} % For professional tables
\usepackage{hyperref} % For clickable links
\usepackage{url} % For formatting URLs
\usepackage{seqsplit} % For splitting long strings in tt font
\usepackage{xcolor} % For custom colors
\usepackage{graphicx} % For potential logos/images
\usepackage{fancyhdr} % For headers and footers

% --- Document Setup ---

% Define custom colors for risk levels
\definecolor{mygreen}{RGB}{46, 139, 87}
\definecolor{myred}{RGB}{220, 20, 60}
\definecolor{myorange}{RGB}{255, 140, 0}
\definecolor{myyellow}{RGB}{255, 215, 0}

% Hyperref setup
\hypersetup{
    colorlinks=true,
    linkcolor=blue,
    filecolor=magenta,      
    urlcolor=cyan,
    pdftitle={Cybersecurity Posture Assessment Report},
    pdfpagemode=FullScreen,
}

% Header and Footer
\pagestyle{fancy}
\fancyhf{} % clear all header and footer fields
\fancyhead[L]{Cybersecurity Posture Assessment}
\fancyhead[R]{Confidential}
\fancyfoot[C]{\thepage}

% --- Document Start ---

\begin{document}

% --- Title Page ---
\begin{titlepage}
    \centering
    \vspace*{1cm}
    
    \Huge{\textbf{Cybersecurity Posture Assessment Report}}
    
    \vspace{1.5cm}
    
    \Large{\textbf{Prepared for:}} \\
    \vspace{0.5cm}
    \Large{Vivid Vision}
    
    \vspace{2cm}
    
    \large{\textbf{Date of Report:}} \\
    \large{\today}
    
    \vfill
    
    \large{\textit{This document contains sensitive and confidential information. Distribution is restricted to authorized personnel only.}}
    
\end{titlepage}

\tableofcontents
\newpage

% --- Section 1: Executive Overview ---
\section{Executive Overview}

This report provides a comprehensive cybersecurity posture assessment for \textbf{Vivid Vision}, based on an analysis of organizational security controls, a network vulnerability scan, and a review of pre-existing risk data. The assessment was conducted on \today.

The analysis reveals several critical and high-risk security gaps that require immediate attention. The most significant findings include the \textbf{lack of Multi-Factor Authentication (MFA) for email and computer access}, which exposes the organization to a high risk of account compromise and unauthorized access. Furthermore, a network scan identified a web server operating over an \textbf{unencrypted HTTP connection (Port 80)}, creating a significant risk for data interception and credential theft. The absence of mandatory security awareness training for new employees further compounds these risks by leaving the organization vulnerable to social engineering attacks.

This report outlines these findings in detail and provides a prioritized list of actionable recommendations to mitigate the identified risks and strengthen the overall security posture of Vivid Vision.

% --- Section 2: Organizational Information ---
\section{Organizational Information}

The following information was provided for the assessment. This data forms the basis for understanding the organization's digital footprint and context.

\begin{tabular}{@{}ll}
    \toprule
    \textbf{Attribute} & \textbf{Value} \\
    \midrule
    Organization Name & \textbf{Vivid Vision} \\
    Email Domain & \texttt{VividVision.com} \\
    Website Domain & \seqsplit{\url{www.VividVision.com}} \\
    External IP Address & \texttt{66.224.138.31} \\
    \bottomrule
\end{tabular}

% --- Section 3: Security Control Review ---
\section{Security Control Review}

A review of organizational security controls was conducted via a standardized questionnaire. The responses indicate the current state of implemented policies and technical measures. Gaps in these controls often represent significant organizational risk.

\begin{tabular}{@{}p{0.6\linewidth} c p{0.2\linewidth}@{}}
    \toprule
    \textbf{Control Question} & \textbf{Response} & \textbf{Analyst Assessment} \\
    \midrule
    Do you require MFA to access email? & \ding{55} & \textcolor{myred}{\textbf{Critical Gap}} \\
    Do you require MFA to log into computers? & \ding{55} & \textcolor{myred}{\textbf{High Risk}} \\
    Do you require MFA to access sensitive data systems? & \ding{51} & \textcolor{mygreen}{Satisfactory} \\
    Does your organization have an employee acceptable use policy? & \ding{51} & \textcolor{mygreen}{Satisfactory} \\
    Does your organization do security awareness training for new employees? & \ding{55} & \textcolor{myorange}{\textbf{High Risk}} \\
    Does your organization do security awareness training for all employees at least once per year? & \ding{51} & \textcolor{mygreen}{Satisfactory} \\
    \bottomrule
\end{tabular}

\vspace{0.5cm}
\noindent \textbf{Note:} Responses marked with \ding{51} (Yes) indicate a control is in place, while \ding{55} (No) indicates a control is absent and represents a potential security gap.

% --- Section 4: Technical Scan Results ---
\section{Technical Scan Results}

A network scan was performed to identify open ports and services exposed on the target system. This technical analysis helps validate policy controls and uncover potential vulnerabilities.

\begin{itemize}
    \item \textbf{Target IP Address:} \texttt{172.16.0.1}
    \item \textbf{Scan Date:} Scan data provided on \today
    \item \textbf{Scanner Used:} Nmap
\end{itemize}

\subsection{Open Ports and Services}
The scan revealed the following open port on the target host:

\begin{tabular}{@{}llll@{}}
    \toprule
    \textbf{Port} & \textbf{State} & \textbf{Inferred Service} & \textbf{Finding} \\
    \midrule
    80/tcp & Open & HTTP & \textcolor{myred}{\textbf{High Risk: Unencrypted Web Traffic}} \\
    \bottomrule
\end{tabular}

\subsubsection{Analysis of Findings}
The presence of an open Port 80 indicates that a web server is running and accessible. Traffic to and from this server via the Hypertext Transfer Protocol (HTTP) is \textbf{unencrypted}. This poses a significant security risk, as any data transmitted—including usernames, passwords, or other sensitive information—can be easily intercepted and read by an attacker on the same network (Man-in-the-Middle attack). All web services handling any form of data input should use HTTPS (Port 443) to ensure data confidentiality and integrity.

% --- Section 5: Consolidated Risk Assessment ---
\section{Consolidated Risk Assessment}

This section correlates the findings from the security control review and the technical scan to provide a consolidated view of the primary risks facing the organization.

\begin{tabular}{@{}p{0.1\linewidth} p{0.4\linewidth} p{0.2\linewidth} p{0.2\linewidth}@{}}
    \toprule
    \textbf{Risk ID} & \textbf{Risk Description} & \textbf{Severity} & \textbf{Affected Asset(s)} \\
    \midrule
    R-01 & \textbf{Lack of MFA on Email:} User email accounts are protected only by passwords, making them highly vulnerable to phishing, credential stuffing, and account takeover. & \textcolor{myred}{\textbf{Critical}} & Email System, All User Accounts \\
    \addlinespace
    R-02 & \textbf{Unencrypted Web Service (HTTP):} A web service on \texttt{172.16.0.1} transmits data in cleartext, exposing user credentials and sensitive data to interception. & \textcolor{myred}{\textbf{High}} & Internal Web Server, User Credentials \\
    \addlinespace
    R-03 & \textbf{Lack of MFA on Endpoints:} Employee computers are not protected by MFA, increasing the risk of unauthorized access if credentials are stolen. & \textcolor{myorange}{\textbf{High}} & Endpoints (Laptops, Desktops) \\
    \addlinespace
    R-04 & \textbf{No Security Training for New Hires:} New employees are not trained on security policies and threat identification, making them prime targets for social engineering. & \textcolor{myyellow}{\textbf{Medium}} & All New Employees, Organizational Data \\
    \bottomrule
\end{tabular}

% --- Section 6: Recommendations ---
\section{Recommendations}

The following actions are recommended to mitigate the identified risks and improve the overall security posture of \textbf{Vivid Vision}. Recommendations are prioritized based on risk severity.

\subsection{Priority 1: Critical}
\begin{enumerate}
    \item \textbf{Implement MFA for Email (R-01):} Immediately enforce mandatory Multi-Factor Authentication for all user access to the email system (e.g., Microsoft 365, Google Workspace). This is the single most effective control to prevent email account takeovers.
\end{enumerate}

\subsection{Priority 2: High}
\begin{enumerate}
    \setcounter{enumi}{1}
    \item \textbf{Secure the Web Server (R-02):} For the service on \texttt{172.16.0.1}:
        \begin{itemize}
            \item \textbf{Option A (Preferred):} Implement TLS/SSL to enforce HTTPS (Port 443). Redirect all HTTP traffic to HTTPS.
            \item \textbf{Option B:} If the service is not essential, decommission it and close port 80.
        \end{itemize}
    \item \textbf{Deploy MFA for Endpoint Logon (R-03):} Implement an MFA solution for logging into all company-managed computers. This adds a critical layer of defense against stolen credentials being used to access the internal network.
\end{enumerate}

\subsection{Priority 3: Medium}
\begin{enumerate}
    \setcounter{enumi}{3}
    \item \textbf{Establish New Hire Security Training (R-04):} Develop a mandatory security awareness training module that is integrated into the new employee onboarding process. This training should cover, at a minimum, acceptable use, phishing identification, and password security.
\end{enumerate}

\end{document}
```