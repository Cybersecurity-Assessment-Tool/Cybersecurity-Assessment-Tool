```latex
\documentclass[12pt]{article}

% --- PACKAGES ---
\usepackage[margin=1in]{geometry}
\usepackage{pifont}                 % For symbols like checkmarks and crosses
\usepackage{booktabs}               % For professional-looking tables
\usepackage{hyperref}               % For clickable links and references
\usepackage{url}                    % For formatting URLs
\usepackage{seqsplit}               % For splitting long strings in texttt
\usepackage[T1]{fontenc}            % For better font encoding

% --- HYPERREF SETUP ---
\hypersetup{
    colorlinks=true,
    linkcolor=black,
    filecolor=magenta,
    urlcolor=blue,
    pdftitle={Cybersecurity Posture Assessment Report},
    pdfauthor={Cybersecurity Analysis Division},
}

% --- CUSTOM COMMANDS ---
\newcommand{\yes}{\ding{51}} % Checkmark
\newcommand{\no}{\ding{55}}  % Cross mark

% --- DOCUMENT START ---
\begin{document}

% --- TITLE PAGE ---
\title{
    Cybersecurity Posture Assessment Report \\
    \large For: Radiant Life
}
\author{Cybersecurity Analysis Division}
\date{\today}
\maketitle

\newpage

% --- TABLE OF CONTENTS ---
\tableofcontents
\newpage

% --- EXECUTIVE SUMMARY ---
\section*{Executive Summary}

This report details the findings of a cybersecurity posture assessment conducted for Radiant Life. The analysis, based on a combination of network scanning, a security controls questionnaire, and a review of pre-existing risks, reveals a critical security posture requiring immediate attention.

Key findings indicate severe deficiencies in fundamental security controls. There is a complete absence of Multi-Factor Authentication (MFA) across all critical systems, including email and sensitive data access. This is compounded by a lack of foundational security policies and employee awareness training, leaving the organization highly susceptible to credential theft and social engineering attacks.

Furthermore, technical scanning identified a publicly accessible MySQL database service. This service is running an outdated and unsupported version (MySQL 5.7.33), which is known to have multiple security vulnerabilities. This exposed database, combined with the lack of MFA, represents a significant and immediate risk of a major data breach.

Urgent remediation is required to address these interconnected risks. Recommendations are prioritized to first contain the most immediate threats, followed by strategic improvements to build a resilient security foundation.

% --- ORGANIZATIONAL INFORMATION ---
\section{Organizational Information}

The following details were provided for the assessment.

\begin{tabular}{@{}ll}
    \toprule
    \textbf{Attribute} & \textbf{Value} \\
    \midrule
    Organization Name & Radiant Life \\
    Email Domain & \texttt{RadiantLife.net} \\
    Website Domain & \texttt{www.RadiantLife.net} \\
    External IP Address & \texttt{33.97.144.203} \\
    \bottomrule
\end{tabular}

% --- SECURITY CONTROL REVIEW ---
\section{Security Control Review}

A review of the organization's security controls was conducted via a questionnaire. The responses indicate critical gaps in essential cybersecurity practices. A red \no\ signifies a negative response that deviates from security best practices.

\begin{table}[h!]
\centering
\begin{tabular}{p{0.8\linewidth} c}
    \toprule
    \textbf{Control Question} & \textbf{Response} \\
    \midrule
    Do you require MFA to access email? & \no \\
    Do you require MFA to log into computers? & \no \\
    Do you require MFA to access sensitive data systems? & \no \\
    Does your organization have an employee acceptable use policy? & \no \\
    Does your organization do security awareness training for new employees? & \no \\
    Does your organization do security awareness training for all employees at least once per year? & \no \\
    \bottomrule
\end{tabular}
\caption{Security Controls Questionnaire Results.}
\end{table}

% --- TECHNICAL SCAN RESULTS ---
\section{Technical Scan Results}

An external network scan was performed to identify exposed services and potential vulnerabilities.

\begin{itemize}
    \item \textbf{Scan Target:} \texttt{172.16.50.20}
    \item \textbf{Scan Date:} Data provided corresponds to recent scan activity.
\end{itemize}

The following open ports and services were discovered:

\begin{table}[h!]
\centering
\begin{tabular}{l l l l l}
    \toprule
    \textbf{Port} & \textbf{State} & \textbf{Service} & \textbf{Product} & \textbf{Version} \\
    \midrule
    3306/tcp & open & mysql & MySQL & 5.7.33 \\
    \bottomrule
\end{tabular}
\caption{Open Ports Identified on Target Host.}
\end{table}

\subsection*{Analysis of Technical Findings}
The scan identified an open MySQL database port (3306). The running version, \textbf{MySQL 5.7.33}, reached its official End of Life (EOL) in October 2023. EOL software no longer receives security updates from the vendor, and this version is known to have multiple publicly disclosed vulnerabilities (CVEs). An exposed, unpatched database service presents a high-impact target for attackers. This finding directly corroborates the pre-existing risk titled "Database Exposure."

% --- SYNTHESIZED RISK ASSESSMENT ---
\section{Synthesized Risk Assessment}

This section correlates findings from the control review, technical scan, and existing risk data to provide a holistic view of the organization's risk profile.

\begin{table}[h!]
\centering
\begin{tabular}{p{0.25\linewidth} p{0.55\linewidth} p{0.1\linewidth}}
    \toprule
    \textbf{Risk Title} & \textbf{Description} & \textbf{Severity} \\
    \midrule
    \textbf{Exposed and Outdated Database Service} & An unsupported MySQL database is directly accessible from the network. The lack of MFA means a single compromised credential could grant an attacker direct access to the database, leading to a severe data breach. & \textbf{Critical} \\
    \addlinespace
    \textbf{Lack of Multi-Factor Authentication (MFA)} & The complete absence of MFA for email, computer logins, and sensitive systems dramatically increases the risk of unauthorized access from phishing, password spraying, or credential stuffing attacks. & \textbf{Critical} \\
    \addlinespace
    \textbf{Absence of Security Policies \& Training} & The lack of an acceptable use policy and security awareness training program indicates a low security maturity. This leaves the organization highly vulnerable to human-error incidents and social engineering. & \textbf{High} \\
    \bottomrule
\end{tabular}
\caption{Summary of Key Risks.}
\end{table}

% --- RECOMMENDATIONS ---
\section{Recommendations}

The following remediation steps are recommended, prioritized by urgency and impact.

\subsection*{Priority 1: Immediate Actions (Complete within 72 hours)}
\begin{enumerate}
    \item \textbf{Restrict Database Access:} Immediately implement firewall rules to block all public access to TCP port 3306 on host \texttt{172.16.50.20}. Access should be restricted to trusted internal IP addresses or through a secure Virtual Private Network (VPN).
\end{enumerate}

\subsection*{Priority 2: High-Priority Actions (Complete within 30 days)}
\begin{enumerate}
    \setcounter{enumi}{1} % Continue numbering
    \item \textbf{Implement Multi-Factor Authentication (MFA):} Deploy a robust MFA solution across all user accounts for, at a minimum:
    \begin{itemize}
        \item Email access (e.g., Office 365, Google Workspace).
        \item Remote access solutions (VPN).
        \item All administrative accounts and systems containing sensitive data.
    \end{itemize}
    \item \textbf{Upgrade Database Service:} Plan and execute the migration of the MySQL 5.7.33 database to a currently supported version (e.g., MySQL 8.x or a managed cloud equivalent). This is essential to ensure ongoing security patches are received.
\end{enumerate}

\subsection*{Priority 3: Foundational Improvements (Complete within 90 days)}
\begin{enumerate}
    \setcounter{enumi}{3} % Continue numbering
    \item \textbf{Develop Security Policies:} Establish and enforce a baseline set of information security policies, starting with an \textit{Employee Acceptable Use Policy} to define rules for technology usage.
    \item \textbf{Institute Security Awareness Training:} Implement a mandatory security awareness training program for all employees. This program should be conducted upon hiring and refreshed annually to educate staff on identifying and reporting threats like phishing.
\end{enumerate}

\end{document}
```