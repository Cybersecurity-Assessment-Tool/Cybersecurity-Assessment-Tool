```latex
\documentclass[12pt]{article}

% --- PACKAGES ---
\usepackage[margin=1in]{geometry}
\usepackage{pifont} % For checkmarks and crosses
\usepackage{booktabs} % For professional tables
\usepackage{hyperref} % For clickable links
\usepackage{url} % For URL formatting
\usepackage{seqsplit} % For splitting long strings
\usepackage{graphicx}
\usepackage{xcolor}

% --- DOCUMENT SETUP ---
\hypersetup{
    colorlinks=true,
    linkcolor=blue,
    filecolor=magenta,      
    urlcolor=cyan,
}

\newcommand{\yes}{\ding{51}}
\newcommand{\no}{\ding{55}}

% --- DOCUMENT START ---
\begin{document}

% --- TITLE PAGE ---
\begin{titlepage}
    \centering
    \vspace*{1cm}
    \Huge\textbf{Cybersecurity Posture Assessment Report}
    \vspace{1.5cm}
    \Large
    \textbf{Prepared for:}\\
    Green Sprout Organic
    \vspace{2cm}
    \large
    \textbf{Date of Report:}\\
    \today
    \vfill
    \large
    \textbf{Generated by:}\\
    Cybersecurity Analyst
\end{titlepage}

\tableofcontents
\newpage

% --- EXECUTIVE SUMMARY ---
\section{Executive Summary}
This report provides a comprehensive analysis of the cybersecurity posture of \textbf{Green Sprout Organic}, based on network scans, a security controls questionnaire, and a review of existing risks. The assessment reveals several critical vulnerabilities that expose the organization to significant threats, including data breaches, unauthorized access, and service disruption.

The most critical findings are the direct network exposure of a database service, the use of end-of-life software for this database, and a systemic lack of Multi-Factor Authentication (MFA) across all key systems. These issues, combined with gaps in security policies and employee training, create a high-risk environment.

Immediate remediation is required to address the exposed database and implement MFA. Strategic initiatives to update software and formalize security policies are strongly recommended to build a more resilient and defensible security posture.

% --- ORGANIZATIONAL INFORMATION ---
\section{Organizational Information}
The following information was provided for the assessment.
\begin{itemize}
    \item \textbf{Organization Name:} Green Sprout Organic
    \item \textbf{Email Domain:} \texttt{GreenSproutOrganic.net}
    \item \textbf{Website Domain:} \url{www.GreenSproutOrganic.net}
    \item \textbf{External IP Address:} \texttt{134.225.207.148}
\end{itemize}

% --- SECURITY CONTROL REVIEW ---
\section{Security Control Review}
A review of the organization's security controls was conducted via a questionnaire. The responses highlight significant gaps in fundamental security practices, particularly concerning access control and employee security awareness.

\begin{table}[h!]
\centering
\caption{Security Controls Questionnaire Analysis}
\begin{tabular}{@{}p{8cm}ccp{4cm}@{}}
\toprule
\textbf{Control Question} & \textbf{Response} & \textbf{Notes} \\
\midrule
Do you require MFA to access email? & \no & Critical Gap. Email is a primary target for account takeover. \\
Do you require MFA to log into computers? & \no & Critical Gap. Increases risk of unauthorized local access. \\
Do you require MFA to access sensitive data systems? & \no & Critical Gap. Leaves databases and other critical assets vulnerable. \\
Does your organization have an employee acceptable use policy? & \no & High Risk. Lack of formal guidelines for employees. \\
Does your organization do security awareness training for new employees? & \no & High Risk. New hires are a prime target for social engineering. \\
Does your organization do security awareness training for all employees at least once per year? & \yes & Good Practice. This should be continued and enhanced. \\
\bottomrule
\end{tabular}
\end{table}

% --- TECHNICAL SCAN RESULTS ---
\section{Technical Scan Results}
An external network scan was performed to identify open ports and exposed services on the target system.

\begin{itemize}
    \item \textbf{Target IP Address:} \texttt{172.16.50.20}
    \item \textbf{Scan Date:} Assessed on \today
\end{itemize}

\subsection{Open Ports and Services}
A single port was found to be open, which presents a significant risk.

\begin{table}[h!]
\centering
\caption{Discovered Open Ports}
\begin{tabular}{@{}lllll@{}}
\toprule
\textbf{Port} & \textbf{State} & \textbf{Service} & \textbf{Product} & \textbf{Version} \\
\midrule
3306/tcp & Open & mysql & MySQL & 5.7.33 \\
\bottomrule
\end{tabular}
\end{table}

\subsection{Technical Analysis}
The scan confirms that a MySQL database server is directly accessible from the network on port \texttt{3306}. Furthermore, the identified version, \textbf{MySQL 5.7.33}, reached its official End-of-Life (EOL) in October 2023. This means it no longer receives security patches from the vendor, and known vulnerabilities will remain unpatched. The combination of direct exposure and EOL software constitutes a critical security risk.

% --- RISK ASSESSMENT ---
\section{Risk Assessment}
The following table synthesizes findings from the security control review, technical scan, and pre-existing risk data. Risks are prioritized based on their potential impact and likelihood of exploitation.

\begin{table}[h!]
\centering
\caption{Summary of Identified Risks}
\begin{tabular}{@{}p{4cm}p{2cm}p{7cm}@{}}
\toprule
\textbf{Risk Name} & \textbf{Severity} & \textbf{Overview} \\
\midrule
\textbf{End-of-Life Database Software} & \textbf{Critical} & The MySQL 5.7.33 server is no longer supported and does not receive security updates, exposing it to numerous known vulnerabilities. \\
\addlinespace
\textbf{Lack of Multi-Factor Authentication} & \textbf{Critical} & The absence of MFA on email, computers, and sensitive data systems makes accounts highly susceptible to compromise via stolen or weak credentials. \\
\addlinespace
\textbf{Publicly Exposed Database Service} & \textbf{High} & The MySQL database port (3306) is open to the network, allowing attackers to directly attempt to connect, brute-force credentials, or exploit vulnerabilities. \\
\addlinespace
\textbf{Missing Security Policies \& Onboarding Training} & \textbf{High} & The lack of an Acceptable Use Policy and security training for new employees creates an environment where security incidents are more likely due to human error. \\
\bottomrule
\end{tabular}
\end{table}

% --- RECOMMENDATIONS ---
\section{Recommendations}
The following actions are recommended to mitigate the identified risks. They are prioritized to address the most critical vulnerabilities first.

\subsection{Immediate Priority (Implement within 72 hours)}
\begin{enumerate}
    \item \textbf{Restrict Access to Database Port 3306:} Immediately implement a firewall rule to block all public access to TCP port 3306. Access should only be permitted from trusted internal IP addresses or through a secure Virtual Private Network (VPN).
    \item \textbf{Enable Multi-Factor Authentication (MFA):} Begin immediate rollout of MFA for all users on critical systems, prioritizing:
    \begin{itemize}
        \item Email (e.g., Office 365, Google Workspace)
        \item Access to sensitive data systems (including the MySQL database)
        \item Remote access solutions (VPNs)
    \end{itemize}
\end{enumerate}

\subsection{High Priority (Implement within 30 days)}
\begin{enumerate}
    \item \textbf{Plan Database Upgrade:} Develop a migration plan to upgrade the MySQL 5.7 database to a currently supported version (e.g., MySQL 8.x). This is essential for receiving future security patches.
    \item \textbf{Develop an Acceptable Use Policy (AUP):} Create and implement a formal AUP that all employees must read and sign. This policy should outline the rules for using company technology and data.
    \item \textbf{Implement New Hire Security Training:} Integrate mandatory security awareness training into the onboarding process for all new employees.
\end{enumerate}

% --- CONCLUSION ---
\section{Conclusion}
The current security posture of \textbf{Green Sprout Organic} contains critical-level risks that require immediate attention. The exposed, end-of-life database and the lack of MFA create a direct path for potential attackers. By implementing the recommendations outlined in this report, the organization can significantly reduce its risk exposure and build a stronger foundation for its cybersecurity program.

\end{document}
```