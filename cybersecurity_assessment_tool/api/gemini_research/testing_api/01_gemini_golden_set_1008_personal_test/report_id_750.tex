```latex
\documentclass[12pt]{article}

% Preamble: Required Packages
\usepackage[margin=1in]{geometry}
\usepackage{pifont} % For checkmarks and crosses
\usepackage{booktabs} % For professional tables
\usepackage{hyperref} % For clickable links
\usepackage{url} % For URL formatting
\usepackage{seqsplit} % To split long strings in tt font
\usepackage{graphicx}
\usepackage{xcolor}
\usepackage{fancyhdr}

% --- Document Setup ---
\hypersetup{
    colorlinks=true,
    linkcolor=blue,
    filecolor=magenta,      
    urlcolor=cyan,
    pdftitle={Cybersecurity Posture Assessment Report},
    pdfpagemode=FullScreen,
}

% Define colors for severity
\definecolor{sevhigh}{RGB}{217, 83, 79}
\definecolor{sevmedium}{RGB}{240, 173, 78}
\definecolor{sevlow}{RGB}{92, 184, 92}

% Header and Footer
\pagestyle{fancy}
\fancyhf{}
\fancyhead[L]{Cybersecurity Posture Assessment}
\fancyhead[R]{\textbf{Silent Spring}}
\fancyfoot[C]{\thepage}

% --- Document Start ---
\begin{document}

% --- Title Page ---
\begin{titlepage}
    \centering
    \vspace*{1cm}
    
    \Huge
    \textbf{Cybersecurity Posture Assessment Report}
    
    \vspace{1.5cm}
    
    \Large
    Prepared for: \\
    \vspace{0.5cm}
    \textbf{Silent Spring}
    
    \vspace{2cm}
    
    \large
    Report Date: \today
    
    \vfill
    
    \large
    \textit{This report contains sensitive information and should be handled with care. Distribution is restricted to authorized personnel only.}
    
\end{titlepage}

\tableofcontents
\newpage

% --- Section 1: Executive Summary ---
\section{Executive Summary}
This report provides a comprehensive cybersecurity assessment for \textbf{Silent Spring}, synthesizing data from technical network scans, a security controls questionnaire, and a review of pre-existing risks.

The overall security posture is moderately strong, with excellent implementation of Multi-Factor Authentication (MFA) across key systems. Technical scans of the target host \texttt{192.168.0.5} revealed a secure configuration with no exposed services of concern.

However, a critical process gap was identified: \textbf{the lack of mandatory security awareness training for new employees during their onboarding process}. This oversight introduces a significant risk, as new hires are often targeted by social engineering attacks and may be unaware of internal security policies.

Furthermore, a conflict was observed between the technical scan results and the current risk register. The scan confirmed that port 80 (HTTP) was closed on the assessed host, which contradicts a pre-existing risk item ("Unencrypted Web Server"). This suggests the risk may be remediated for this asset or requires further validation across the network.

Key recommendations focus on immediately implementing a security training program for new hires and performing a comprehensive review of the risk register to ensure its accuracy.

% --- Section 2: Organizational Information ---
\section{Organizational Information}
The following details were provided for the assessment scope.
\begin{itemize}
    \item \textbf{Organization Name:} Silent Spring
    \item \textbf{Email Domain:} \texttt{SilentSpring.com}
    \item \textbf{Website Domain:} \url{www.SilentSpring.com}
    \item \textbf{External IP Address:} \texttt{235.126.52.211}
\end{itemize}

% --- Section 3: Security Control Review ---
\section{Security Control Review}
A review of organizational security controls was conducted via a questionnaire. The responses indicate a strong commitment to identity and access management but reveal a critical weakness in the employee onboarding process.

\begin{table}[h!]
\centering
\caption{Security Controls Questionnaire Results}
\begin{tabular}{p{0.7\linewidth} c c}
\toprule
\textbf{Control Question} & \textbf{Response} & \textbf{Status} \\
\midrule
Do you require MFA to access email? & Yes & \ding{51} \\
Do you require MFA to log into computers? & Yes & \ding{51} \\
Do you require MFA to access sensitive data systems? & Yes & \ding{51} \\
Does your organization have an employee acceptable use policy? & Yes & \ding{51} \\
\textbf{Does your organization do security awareness training for new employees?} & \textbf{No} & \textbf{\color{red}\ding{55}} \\
Does your organization do security awareness training for all employees at least once per year? & Yes & \ding{51} \\
\bottomrule
\end{tabular}
\end{table}

\subsection*{Analysis of Findings}
The single "No" response represents a \textbf{High-Risk Gap}. New employees are a primary target for phishing and social engineering attacks. Without immediate training on acceptable use, data handling, and threat identification, they represent a significant vulnerability to the organization from their first day of employment. While annual training is in place, the initial period of employment is the most critical for establishing a strong security mindset.

% --- Section 4: Technical Scan Results ---
\section{Technical Scan Results}
A network scan was performed to identify open ports and exposed services on the specified target.

\begin{itemize}
    \item \textbf{Target IP Address:} \texttt{192.168.0.5}
    \item \textbf{Scan Tool:} Nmap
\end{itemize}

\begin{table}[h!]
\centering
\caption{Nmap Scan Results for \texttt{192.168.0.5}}
\begin{tabular}{l l l l}
\toprule
\textbf{Port} & \textbf{State} & \textbf{Service} & \textbf{Product / Version} \\
\midrule
80/tcp & closed & http & N/A \\
\bottomrule
\end{tabular}
\end{table}

\subsection*{Analysis of Findings}
The scan of host \texttt{192.168.0.5} shows a secure configuration. The finding that port 80 (HTTP) is \textbf{closed} is a positive security measure, as it prevents unencrypted web traffic. This result directly conflicts with a pre-existing risk documented in the risk register (see Section 5), indicating that for this specific host, the risk has been successfully mitigated.

% --- Section 5: Correlated Risk Assessment ---
\section{Correlated Risk Assessment}
This section synthesizes findings from the security questionnaire, technical scans, and the pre-existing risk register.

\begin{table}[h!]
\centering
\caption{Summary of Identified and Reviewed Risks}
\begin{tabular}{p{0.25\linewidth} p{0.45\linewidth} l l}
\toprule
\textbf{Risk Name} & \textbf{Description} & \textbf{Severity} & \textbf{Status} \\
\midrule
\textbf{Lack of Onboarding Security Training} & New employees do not receive security training upon hiring, creating a window of vulnerability to social engineering and policy violation. & \colorbox{sevhigh}{\color{white} \textbf{High}} & Active \\
\addlinespace
\textbf{Unencrypted Web Server} & \textit{(Pre-existing risk)} An open port 80 was documented, potentially exposing web traffic to interception and modification. & \colorbox{sevmedium}{\color{white} \textbf{Medium}} & \textbf{Not Validated} \\
\bottomrule
\end{tabular}
\end{table}

\subsection*{Risk Correlation Details}
\begin{itemize}
    \item \textbf{High - Lack of Onboarding Security Training:} This is the primary finding of this assessment. It is an active and unmitigated risk derived directly from the security controls questionnaire. It requires immediate attention.
    
    \item \textbf{Medium - Unencrypted Web Server:} This risk was sourced from Input 3. However, our technical scan of host \texttt{192.168.0.5} found port 80 to be closed. This indicates that either the risk has been remediated on this host, it pertains to a different asset not in the scan scope, or the original finding was a false positive. The status is therefore marked as "Not Validated" pending a broader investigation.
\end{itemize}

% --- Section 6: Recommendations ---
\section{Recommendations}
The following actionable recommendations are provided to address the identified risks and improve the overall security posture of \textbf{Silent Spring}.

\subsection*{Priority 1: Implement Onboarding Security Training (High)}
\begin{itemize}
    \item \textbf{Action:} Develop and mandate a security awareness training module for all new employees. This training must be completed within the first week of employment, before full access to sensitive systems is granted.
    \item \textbf{Justification:} This closes a critical gap where new hires, who may have privileged access, are unaware of organizational security policies. It is the most effective way to mitigate the risk of human error and social engineering attacks targeting new staff.
\end{itemize}

\subsection*{Priority 2: Validate and Update Risk Register (Medium)}
\begin{itemize}
    \item \textbf{Action:} Conduct a comprehensive internal and external vulnerability scan across all organizational assets to validate the "Unencrypted Web Server" risk. Update the risk register to accurately reflect the current state of all assets.
    \item \textbf{Justification:} An accurate risk register is essential for effective security management. The discrepancy found during this assessment highlights the need for regular validation to ensure that resources are focused on genuine, active threats.
\end{itemize}

\end{document}
```