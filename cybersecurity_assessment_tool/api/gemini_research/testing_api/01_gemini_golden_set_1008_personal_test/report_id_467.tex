```latex
\documentclass[12pt]{article}

% Preamble: Required Packages
\usepackage[margin=1in]{geometry}
\usepackage{pifont} % For checkmarks and crosses
\usepackage{booktabs} % For professional tables
\usepackage{hyperref} % For clickable links
\usepackage{url}      % For URL formatting
\usepackage{seqsplit} % For splitting long strings
\usepackage{graphicx} % For logo (placeholder)
\usepackage{fancyhdr} % For headers and footers

% Document Metadata
\title{Cybersecurity Assessment Report}
\author{Cybersecurity Analysis Division}
\date{\today}

% Header and Footer Configuration
\pagestyle{fancy}
\fancyhf{} % Clear all header and footer fields
\fancyhead[L]{Security Assessment Report}
\fancyhead[R]{Grizzly Peak}
\fancyfoot[C]{\thepage}
\renewcommand{\headrulewidth}{0.4pt}
\renewcommand{\footrulewidth}{0.4pt}

\begin{document}

\maketitle
\thispagestyle{empty}
\newpage

\tableofcontents
\newpage

\section{Executive Summary}

This report details the findings of a cybersecurity assessment conducted for \textbf{Grizzly Peak}. The analysis synthesized data from an external network scan, a security controls questionnaire, and a review of pre-existing risks.

The assessment identified a \textbf{critical technical vulnerability} and several \textbf{significant procedural gaps} that elevate the organization's overall risk profile. The most severe finding is the exposure of a service on the localhost interface (\texttt{127.0.0.1}), which correlates with a known critical risk (CVSS 10.0). This represents an immediate and severe threat that requires urgent remediation.

Furthermore, significant weaknesses were identified in access control and employee security governance. The absence of Multi-Factor Authentication (MFA) for email and sensitive data systems, coupled with the lack of an acceptable use policy and annual security training, creates substantial opportunities for threat actors to compromise accounts and data.

Immediate and decisive action is required to address the identified risks. Recommendations are provided in Section \ref{sec:recommendations} to guide remediation efforts, prioritizing the most critical vulnerabilities.

\section{Organizational Information}

The following details were provided for the assessment:

\begin{itemize}
    \item \textbf{Organization Name:} Grizzly Peak
    \item \textbf{Email Domain:} \texttt{GrizzlyPeak.org}
    \item \textbf{Website Domain:} \url{www.GrizzlyPeak.org}
    \item \textbf{External IP Address:} \seqsplit{\texttt{13.192.154.118}}
\end{itemize}

\section{Security Control Review}

The following table summarizes the organization's responses to the security controls questionnaire. Items marked with \ding{55} indicate a deviation from security best practices and represent a gap in the defensive posture.

\begin{table}[h!]
\centering
\caption{Security Controls Questionnaire Analysis}
\label{tab:controls}
\begin{tabular}{@{}lcc@{}}
\toprule
\textbf{Control Question} & \textbf{Response} & \textbf{Assessment} \\ \midrule
Do you require MFA to log into computers? & \ding{51} & Compliant \\
Does your organization do security awareness training for new employees? & \ding{51} & Compliant \\
\addlinespace
Do you require MFA to access email? & \ding{55} & \textbf{Critical Gap} \\
Do you require MFA to access sensitive data systems? & \ding{55} & \textbf{Critical Gap} \\
Does your organization have an employee acceptable use policy? & \ding{55} & High Risk \\
Does your organization do security awareness training for all employees at least once per year? & \ding{55} & High Risk \\ \bottomrule
\end{tabular}
\end{table}

\subsection{Analysis of Gaps}
The lack of MFA on email and sensitive systems are critical vulnerabilities. Email is a primary target for account takeover attacks, which can lead to business email compromise (BEC), data breaches, and further internal network intrusion. The absence of an acceptable use policy and annual security training indicates a weakness in the human layer of security, making the organization more susceptible to phishing and social engineering attacks.

\section{Technical Scan Results}

An external network scan was performed to identify exposed services and potential vulnerabilities.

\subsection{Host: \texttt{127.0.0.1}}
The scan identified the following open port on the target system.
\begin{table}[h!]
\centering
\caption{Open Ports for \texttt{127.0.0.1}}
\label{tab:ports}
\begin{tabular}{@{}cccc@{}}
\toprule
\textbf{Port} & \textbf{State} & \textbf{Service (Probable)} & \textbf{Notes} \\ \midrule
22/tcp & open & SSH & Version information not available. \\ \bottomrule
\end{tabular}
\end{table}

\subsection{Technical Analysis}
The scan result is highly anomalous and indicates a severe misconfiguration. The target IP address, \texttt{127.0.0.1}, is the universal localhost (loopback) address, which should never be accessible from an external network. The presence of an open SSH port on this interface suggests a critical network or system configuration error. This finding directly corroborates the pre-existing risk "Localhost Exposed" and confirms its validity. An attacker able to reach this service could potentially gain unauthorized access to the system.

\section{Consolidated Risk Assessment}

The following table synthesizes findings from the security questionnaire, technical scan, and pre-existing risk data into a consolidated list of identified risks.

\begin{table}[h!]
\centering
\caption{Summary of Identified Risks}
\label{tab:risks}
\begin{tabular}{@{}p{0.3\linewidth}p{0.5\linewidth}l@{}}
\toprule
\textbf{Risk Name} & \textbf{Description} & \textbf{Severity} \\ \midrule
\textbf{Exposed Localhost Service} & The SSH service on the localhost interface (\texttt{127.0.0.1}) is exposed, indicating a severe network misconfiguration. This aligns with a known risk with a CVSS score of 10.0. & \textbf{Critical} \\
\addlinespace
\textbf{Lack of MFA on Email} & User email accounts are protected only by passwords, making them highly vulnerable to phishing, credential stuffing, and account takeover. & \textbf{Critical} \\
\addlinespace
\textbf{Lack of MFA on Sensitive Systems} & Critical data systems lack a fundamental access control, significantly increasing the risk of a data breach from compromised credentials. & \textbf{Critical} \\
\addlinespace
\textbf{Inadequate Security Training Program} & While new hires are trained, the lack of annual refresher training for all employees allows security knowledge to decay, increasing susceptibility to social engineering. & High \\
\addlinespace
\textbf{Missing Acceptable Use Policy (AUP)} & The absence of a formal AUP creates ambiguity regarding secure and acceptable use of company assets, increasing the risk of insider threat and unintentional misuse. & High \\ \bottomrule
\end{tabular}
\end{table}

\section{Recommendations}
\label{sec:recommendations}

The following actions are recommended to mitigate the identified risks. They are prioritized based on severity and potential impact.

\begin{enumerate}
    \item \textbf{[Immediate Priority] Remediate Exposed Localhost Service:}
    \begin{itemize}
        \item Immediately investigate the network architecture to determine how traffic is being routed to the \texttt{127.0.0.1} interface from an external source.
        \item Reconfigure firewall rules, routing tables, or system network settings to ensure the localhost interface is not exposed to any external network.
        \item Conduct a follow-up scan to verify that the port is no longer accessible.
    \end{itemize}

    \item \textbf{[High Priority] Implement Comprehensive MFA:}
    \begin{itemize}
        \item Enable MFA for all user accounts on the email platform (\texttt{GrizzlyPeak.org}).
        \item Enforce MFA for access to all systems identified as containing sensitive data.
        \item Prioritize phishing-resistant MFA methods (e.g., FIDO2/WebAuthn) where possible.
    \end{itemize}

    \item \textbf{[Medium Priority] Develop and Implement Governance Policies:}
    \begin{itemize}
        \item Draft and formally adopt an Employee Acceptable Use Policy (AUP) that clearly defines rules for using company networks, systems, and data.
        \item Require all employees to read and acknowledge the AUP.
    \end{itemize}

    \item \textbf{[Medium Priority] Establish a Recurring Security Training Program:}
    \begin{itemize}
        \item Institute a mandatory annual security awareness training program for all employees.
        \item The training should cover current threats, such as phishing, social engineering, and proper data handling, reinforcing the principles outlined in the new AUP.
    \end{itemize}
\end{enumerate}

\end{document}
```