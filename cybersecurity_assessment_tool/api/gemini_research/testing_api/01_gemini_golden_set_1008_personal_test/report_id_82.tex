```latex
\documentclass[12pt]{article}

% Preamble: Required Packages
\usepackage[margin=1in]{geometry}
\usepackage{pifont} % For checkmarks and crosses
\usepackage{booktabs} % For professional tables
\usepackage{hyperref} % For hyperlinks
\usepackage{url} % For URL formatting
\usepackage{seqsplit} % For splitting long strings like URLs or IPs
\usepackage{graphicx}
\usepackage{xcolor}
\usepackage{fancyhdr}

% --- Document Setup ---
\hypersetup{
    colorlinks=true,
    linkcolor=blue,
    filecolor=magenta,      
    urlcolor=cyan,
    pdftitle={Cybersecurity Posture Report},
    pdfpagemode=FullScreen,
}

% Define colors for table rows
\definecolor{tableheadcolor}{rgb}{0.1, 0.2, 0.4}
\definecolor{tablerowcolor}{gray}{0.95}

% Header and Footer
\pagestyle{fancy}
\fancyhf{}
\fancyhead[L]{Cybersecurity Posture Report}
\fancyhead[R]{Orchid Isle}
\fancyfoot[C]{\thepage}

% --- Document Start ---
\begin{document}

% --- Title Page ---
\begin{titlepage}
    \centering
    \vspace*{1cm}
    
    \Huge
    \textbf{Cybersecurity Posture Report}
    
    \vspace{1.5cm}
    
    \Large
    Prepared for:
    
    \vspace{0.5cm}
    
    \textbf{Orchid Isle}
    
    \vspace{2cm}
    
    \large
    \textbf{Date of Report:} \today
    
    \vfill
    
    \large
    \textit{This report contains sensitive information and is intended solely for the use of the designated recipient.}
    
\end{titlepage}

\tableofcontents
\newpage

% --- Section 1: Executive Summary ---
\section{Executive Summary}

This report provides a comprehensive analysis of the cybersecurity posture for \textbf{Orchid Isle}, based on a review of organizational security controls, an external network scan, and pre-existing risk data.

The assessment reveals a mixed security posture. The organization has implemented strong Multi-Factor Authentication (MFA) controls across email, computer logins, and sensitive data systems, which significantly reduces the risk of unauthorized access. An acceptable use policy is also in place.

However, a critical gap was identified in the employee onboarding process: \textbf{new employees do not receive security awareness training}. This oversight introduces a high risk, as new hires are often prime targets for social engineering and phishing attacks. While annual training is conducted, the initial onboarding period is a crucial window for establishing a security-conscious mindset.

The external network scan conducted on the target IP address \texttt{[Target IP]} did not identify any open ports or services. While this could indicate a securely configured firewall or a lack of exposed services, it could also result from the host being offline or scan-blocking technologies. This finding is currently inconclusive without further verification.

Recommendations in this report focus on immediately addressing the new hire training gap and verifying the external network posture to ensure the scan results reflect an intentional security configuration.

% --- Section 2: Organizational Information ---
\section{Organizational Information}
The following details were provided for the assessment.

\begin{table}[h!]
\centering
\begin{tabular}{@{}ll@{}}
\toprule
\rowcolor{tableheadcolor}
\textcolor{white}{\textbf{Attribute}} & \textcolor{white}{\textbf{Value}} \\
\midrule
Organization Name & Orchid Isle \\
Email Domain & \seqsplit{\texttt{OrchidIsle.com}} \\
Website Domain & \seqsplit{\url{www.OrchidIsle.com}} \\
External IP Address & \seqsplit{\texttt{60.220.119.127}} \\
\bottomrule
\end{tabular}
\caption{Client Organizational Data}
\label{tab:org_info}
\end{table}

% --- Section 3: Security Control Review ---
\section{Security Control Review (Questionnaire Analysis)}
A review of the organization's security controls was conducted via a standardized questionnaire. The results are summarized below. "Yes" answers indicate a control is in place, while "No" answers represent a potential security gap.

\begin{table}[h!]
\centering
\begin{tabular}{@{}lc@{}}
\toprule
\rowcolor{tableheadcolor}
\textcolor{white}{\textbf{Control Question}} & \textcolor{white}{\textbf{Status}} \\
\midrule
\rowcolor{tablerowcolor}
Do you require MFA to access email? & \ding{51} \\ % Yes
Do you require MFA to log into computers? & \ding{51} \\ % Yes
\rowcolor{tablerowcolor}
Do you require MFA to access sensitive data systems? & \ding{51} \\ % Yes
Does your organization have an employee acceptable use policy? & \ding{51} \\ % Yes
\rowcolor{tablerowcolor}
\textbf{Does your organization do security awareness training for new employees?} & \textbf{\color{red}\ding{55}} \\ % No
Does your organization do security awareness training for all employees at least once per year? & \ding{51} \\ % Yes
\bottomrule
\end{tabular}
\caption{Security Control Questionnaire Results}
\label{tab:controls}
\end{table}

\subsection*{Analysis of Findings}
The questionnaire reveals a significant gap in the security program. The lack of mandatory security awareness training for new employees (\textbf{highlighted in red}) is a high-risk finding. New hires are often unfamiliar with corporate policies and are more susceptible to social engineering attacks like phishing. Failing to provide this training during the onboarding process misses a critical opportunity to instill a culture of security from day one.

% --- Section 4: Technical Scan Results ---
\section{Technical Scan Results}
An external network vulnerability scan was performed to identify exposed services and potential vulnerabilities.

\begin{itemize}
    \item \textbf{Target IP Address:} \texttt{[Target IP]}
    \item \textbf{Scan Date:} \textbf{[Scan Date]}
\end{itemize}

\subsection*{Scan Summary}
The scan of the target system completed without discovering any open TCP or UDP ports. No network services were identified as being accessible from the public internet at the time of the scan.

\subsection*{Interpretation}
This result can be interpreted in several ways:
\begin{enumerate}
    \item \textbf{Secure Configuration:} The target system is properly firewalled, and there are no services intentionally exposed to the internet. This is the ideal state.
    \item \textbf{Host Offline:} The target system may have been offline or unreachable during the scan window.
    \item \textbf{Active Blocking:} An Intrusion Prevention System (IPS) or other network security appliance may have detected and blocked the scan traffic.
\end{enumerate}
While the absence of open ports is often a positive sign, it is recommended that the organization internally verify that this result aligns with their intended security architecture.

% --- Section 5: Consolidated Risk Assessment ---
\section{Consolidated Risk Assessment}
This section synthesizes findings from the security control review, technical scans, and any pre-existing risk data provided. No pre-existing vulnerabilities were reported. The primary risk identified during this assessment is detailed below.

\begin{table}[h!]
\centering
\begin{tabular}{@{}p{0.1\textwidth}p{0.3\textwidth}p{0.4\textwidth}p{0.1\textwidth}@{}}
\toprule
\rowcolor{tableheadcolor}
\textcolor{white}{\textbf{Risk ID}} & \textcolor{white}{\textbf{Risk Name}} & \textcolor{white}{\textbf{Overview}} & \textcolor{white}{\textbf{Severity}} \\
\midrule
\rowcolor{tablerowcolor}
RISK-001 & Lack of Security Awareness Training for New Hires & New employees are not provided with security training during onboarding. This makes them highly vulnerable to phishing and social engineering, as they are not yet familiar with company policies, communication patterns, or security best practices. An attacker could exploit this gap to gain an initial foothold in the network. & \textbf{High} \\
\bottomrule
\end{tabular}
\caption{Identified Risks}
\label{tab:risks}
\end{table}

% --- Section 6: Recommendations ---
\section{Recommendations}
The following actionable recommendations are provided to mitigate the identified risks and improve the overall security posture of \textbf{Orchid Isle}.

\subsection*{RISK-001 (High Priority): Implement Onboarding Security Training}
\begin{itemize}
    \item \textbf{Action:} Develop and implement a mandatory security awareness training module as a standard part of the new employee onboarding process.
    \item \textbf{Details:} This training should be completed within the first week of employment and cover, at a minimum:
    \begin{itemize}
        \item Phishing and social engineering awareness.
        \item The organization's acceptable use policy.
        \item Password security and MFA requirements.
        \item Procedures for reporting security incidents.
    \end{itemize}
    \item \textbf{Justification:} This measure will significantly reduce the organization's susceptibility to attacks targeting new employees and will help build a strong, security-first culture from the outset.
\end{itemize}

\subsection*{Informational: Verify External Network Posture}
\begin{itemize}
    \item \textbf{Action:} Internally validate that the external scan results (no open ports on \texttt{[Target IP]}) are accurate and intentional.
    \item \textbf{Details:} Review the firewall rules and network architecture for the specified IP address to confirm that no services are meant to be publicly accessible. If services should be available, investigate why they were not detected.
    \item \textbf{Justification:} This provides assurance that the external security posture is known and managed, rather than being the result of a misconfiguration or network issue.
\end{itemize}

\end{document}
```