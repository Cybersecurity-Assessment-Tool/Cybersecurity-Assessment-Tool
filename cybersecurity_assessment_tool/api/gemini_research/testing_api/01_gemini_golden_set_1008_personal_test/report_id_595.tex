```latex
\documentclass[12pt]{article}

% --- PACKAGE IMPORTS ---
\usepackage[margin=1in]{geometry}
\usepackage{pifont} % For checkmarks and crosses
\usepackage{booktabs} % For professional tables
\usepackage{hyperref} % For clickable links
\usepackage{url}      % For proper URL formatting
\usepackage{seqsplit} % For splitting long strings like IP addresses

% --- DOCUMENT METADATA ---
\title{Cybersecurity Posture Assessment Report \\ \large For: Hearth \& Home}
\author{Cybersecurity Analyst}
\date{\today}

% --- DOCUMENT START ---
\begin{document}

\maketitle
\thispagestyle{empty}
\newpage

\tableofcontents
\newpage

% --- EXECUTIVE SUMMARY ---
\section*{1.0 Executive Summary}

This report details the findings of a cybersecurity posture assessment for Hearth \& Home. The analysis combines a review of organizational security controls, a technical network scan, and an evaluation of pre-existing risks.

The assessment identified several high-risk gaps in both administrative and technical controls. Key findings include the lack of mandatory multi-factor authentication (MFA) for computer logins, the absence of a formal employee acceptable use policy, and incomplete annual security awareness training. These policy gaps create significant vulnerabilities to insider threats and social engineering attacks.

Furthermore, a technical scan of the organization's external infrastructure revealed an exposed Secure Shell (SSH) service on port 22. This service presents a direct and critical attack vector for brute-force and credential-based attacks.

Immediate remediation is strongly recommended to address these critical vulnerabilities and reduce the organization's overall risk exposure. A detailed breakdown of findings and actionable recommendations is provided in the subsequent sections.

% --- ORGANIZATIONAL INFORMATION ---
\section*{2.0 Organizational Information}

The following information was provided for the assessment.

\begin{table}[h!]
\centering
\begin{tabular}{@{}ll@{}}
\toprule
\textbf{Attribute} & \textbf{Value} \\
\midrule
Organization Name & Hearth \& Home \\
Email Domain & \texttt{HearthHome.net} \\
Website Domain & \url{www.HearthHome.net} \\
External IP Address & \texttt{163.106.179.244} \\
\bottomrule
\end{tabular}
\caption{Client Organizational Details.}
\end{table}

% --- SECURITY CONTROL REVIEW ---
\section*{3.0 Security Control Review}

A review of administrative security controls was conducted based on a standardized questionnaire. The results highlight critical gaps in the organization's security policies and procedures. "No" answers indicate a failure to meet baseline security best practices.

\begin{table}[h!]
\centering
\begin{tabular}{@{}p{0.7\textwidth}c@{}}
\toprule
\textbf{Control Question} & \textbf{Status} \\
\midrule
Do you require MFA to access email? & \ding{51} \\ % Yes
Do you require MFA to log into computers? & \textbf{\color{red}\ding{55}} \\ % No
Do you require MFA to access sensitive data systems? & \ding{51} \\ % Yes
Does your organization have an employee acceptable use policy? & \textbf{\color{red}\ding{55}} \\ % No
Does your organization do security awareness training for new employees? & \ding{51} \\ % Yes
Does your organization do security awareness training for all employees at least once per year? & \textbf{\color{red}\ding{55}} \\ % No
\bottomrule
\end{tabular}
\caption{Security Controls Questionnaire Results.}
\end{table}

% --- TECHNICAL SCAN RESULTS ---
\section*{4.0 Technical Scan Results}

An external network scan was performed to identify open ports and exposed services on the client's public-facing infrastructure.

\begin{itemize}
    \item \textbf{Target IP Address:} \seqsplit{\texttt{2001:db8::1}}
    \item \textbf{Scan Summary:} The scan identified one open port.
\end{itemize}

\begin{table}[h!]
\centering
\begin{tabular}{@{}llll@{}}
\toprule
\textbf{Port} & \textbf{State} & \textbf{Service} & \textbf{Analysis} \\
\midrule
22/TCP & Open & SSH & \begin{tabular}[t]{@{}l@{}}High Risk. The Secure Shell service is exposed \\ to the public internet. This makes the system a \\ prime target for automated brute-force attacks, \\ credential stuffing, and exploitation of any \\ potential service vulnerabilities. Service version \\ information was not available during the scan.\end{tabular} \\
\bottomrule
\end{tabular}
\caption{Open Port Analysis.}
\end{table}

% --- RISK ASSESSMENT ---
\section*{5.0 Risk Assessment Summary}

This section synthesizes findings from the security control review, technical scan, and pre-existing risk data. The following risks have been identified and prioritized based on their potential impact on the organization.

\begin{table}[h!]
\centering
\begin{tabular}{@{}p{0.25\textwidth}p{0.15\textwidth}p{0.5\textwidth}@{}}
\toprule
\textbf{Risk Title} & \textbf{Severity} & \textbf{Overview} \\
\midrule
\textbf{Exposed SSH Service} & \textbf{High} & Port 22 (SSH) is open to the internet on host \seqsplit{\texttt{2001:db8::1}}, creating a direct vector for unauthorized access attempts and system compromise. \\
\addlinespace
\textbf{Lack of MFA on Endpoints} & \textbf{High} & The absence of MFA for computer logins means a single compromised password could grant an attacker full access to an employee's workstation and any connected resources. \\
\addlinespace
\textbf{No Acceptable Use Policy (AUP)} & \textbf{Medium} & Without a formal AUP, there is no enforceable standard for employee behavior regarding company assets, data handling, or internet usage, increasing the risk of insider threats and data leakage. \\
\addlinespace
\textbf{Incomplete Security Awareness Training} & \textbf{Medium} & Failing to provide annual security training for all employees leaves the organization highly susceptible to evolving phishing, social engineering, and malware attacks. \\
\addlinespace
\textbf{No Pre-existing Risks} & \textbf{Informational} & The provided data indicated no previously tracked vulnerabilities. The risks in this table are derived from the current assessment. \\
\bottomrule
\end{tabular}
\caption{Consolidated Risk Register.}
\end{table}

% --- RECOMMENDATIONS ---
\section*{6.0 Recommendations}

The following actions are recommended to mitigate the identified risks and improve the overall security posture of Hearth \& Home.

\begin{enumerate}
    \item \textbf{Remediate Exposed SSH Service (High Priority):}
        \begin{itemize}
            \item \textbf{Immediate Action:} If public access is not essential, block all inbound traffic to port 22 on the external firewall.
            \item \textbf{If Access is Required:} Restrict access to a whitelist of trusted source IP addresses. Furthermore, enforce public key authentication and disable password-based logins entirely for SSH.
        \end{itemize}

    \item \textbf{Implement Endpoint MFA (High Priority):}
        \begin{itemize}
            \item Deploy and enforce a multi-factor authentication solution for all employee computer and laptop logins. This is a critical defense against credential theft and unauthorized access.
        \end{itemize}

    \item \textbf{Develop and Implement an Acceptable Use Policy (Medium Priority):}
        \begin{itemize}
            \item Create a formal AUP document that clearly defines rules for computer, network, and data usage.
            \item Require all employees to read and formally acknowledge the policy upon hiring and annually thereafter.
        \end{itemize}

    \item \textbf{Establish a Continuous Security Training Program (Medium Priority):}
        \begin{itemize}
            \item Institute a mandatory security awareness training program for all employees to be completed at least once per year.
            \item The training should cover current threats such as phishing, ransomware, and social engineering.
        \end{itemize}
\end{enumerate}

\end{document}
```