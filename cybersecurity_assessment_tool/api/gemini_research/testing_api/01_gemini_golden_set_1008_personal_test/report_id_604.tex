```latex
\documentclass[12pt]{article}

% Preamble: Required Packages
\usepackage[margin=1in]{geometry}
\usepackage{pifont} % For checkmarks and crosses
\usepackage{booktabs} % For professional tables
\usepackage{hyperref} % For clickable links
\usepackage{url} % For formatting URLs
\usepackage{seqsplit} % For splitting long strings
\usepackage{graphicx}
\usepackage{xcolor}
\usepackage{fancyhdr}

% Document Metadata and Styling
\hypersetup{
    colorlinks=true,
    linkcolor=blue,
    filecolor=magenta,      
    urlcolor=cyan,
    pdftitle={Cybersecurity Assessment Report},
    pdfpagemode=FullScreen,
}

\pagestyle{fancy}
\fancyhf{}
\fancyhead[L]{Cybersecurity Assessment Report}
\fancyhead[R]{Obsidian Operatives}
\fancyfoot[C]{\thepage}

% --- DOCUMENT START ---
\begin{document}

\begin{titlepage}
    \centering
    \vspace*{1cm}
    \Huge\textbf{Cybersecurity Assessment Report}
    \vspace{1.5cm}
    \Large
    \begin{tabular}{ll}
        \textbf{Client:} & Obsidian Operatives \\
        \textbf{Date of Report:} & \today \\
        \textbf{Report ID:} & CSR-2023-001 \\
    \end{tabular}
    \vfill
    \large
    \textit{This report contains sensitive and confidential information intended for the exclusive use of the client. Distribution without prior written consent is prohibited.}
\end{titlepage}

\tableofcontents
\newpage

% --- 1. EXECUTIVE SUMMARY ---
\section{Executive Summary}

This report details the findings of a cybersecurity assessment conducted for \textbf{Obsidian Operatives}. The assessment combined a technical network scan, a review of existing risks, and an analysis of organizational security controls based on a questionnaire.

The overall security posture is mixed. On a positive note, the technical network scan of the target system revealed no exposed services, indicating a strong network perimeter configuration. This significantly reduces the external attack surface.

However, critical administrative and procedural gaps were identified. The most severe finding is the \textbf{lack of Multi-Factor Authentication (MFA) for email access}. This exposes the organization to a high risk of Business Email Compromise (BEC), phishing, and account takeovers. Additionally, the absence of a formal \textbf{Acceptable Use Policy (AUP)} creates ambiguity regarding the secure use of company assets and increases the risk of insider threats and data misuse.

This report outlines these findings in detail and provides actionable recommendations to mitigate the identified risks and strengthen the organization's overall security posture.

% --- 2. ORGANIZATIONAL INFORMATION ---
\section{Organizational Information}
The following details were provided for the assessment scope.

\begin{tabular}{@{}ll}
    \toprule
    \textbf{Attribute} & \textbf{Value} \\
    \midrule
    Organization Name & \textbf{Obsidian Operatives} \\
    Email Domain & \texttt{ObsidianOperatives.com} \\
    Website Domain & \url{www.ObsidianOperatives.com} \\
    External IP Address & \texttt{13.29.240.66} \\
    \bottomrule
\end{tabular}

% --- 3. SECURITY CONTROL REVIEW ---
\section{Security Control Review}
The following table summarizes the organization's responses to the security controls questionnaire. "No" answers represent significant gaps that increase security risk.

\begin{table}[h!]
\centering
\begin{tabular}{@{}lcc@{}}
    \toprule
    \textbf{Control Question} & \textbf{Response} & \textbf{Assessment} \\
    \midrule
    Do you require MFA to access email? & \ding{55} & \textcolor{red}{\textbf{Critical Gap}} \\
    Do you require MFA to log into computers? & \ding{51} & Satisfactory \\
    Do you require MFA to access sensitive data systems? & \ding{51} & Satisfactory \\
    Does your organization have an employee acceptable use policy? & \ding{55} & \textcolor{orange}{\textbf{High Risk}} \\
    Does your organization do security awareness training for new employees? & \ding{51} & Satisfactory \\
    Does your organization do security awareness training for all employees annually? & \ding{51} & Satisfactory \\
    \bottomrule
\end{tabular}
\caption{Analysis of Security Control Questionnaire.}
\end{table}

The two primary deficiencies identified are the lack of MFA on the primary communication channel (email) and the absence of a foundational governance document (AUP). While the security awareness and other MFA controls are commendable, the email security gap undermines these positive efforts.

% --- 4. TECHNICAL SCAN RESULTS ---
\section{Technical Scan Results}
A network scan was performed to identify open ports and exposed services on the target system.

\begin{itemize}
    \item \textbf{Target IP Address:} \texttt{192.168.1.100}
    \item \textbf{Scan Result:} The scan completed successfully and found \textbf{zero open ports}. All 65,535 TCP ports were reported as being in a 'closed' state.
\end{itemize}

\subsection*{Analysis}
This is a strong positive finding. It indicates that the target host is either not running any network-facing services or is properly secured behind a firewall that denies all unsolicited inbound connections. This configuration minimizes the external attack surface and is a best practice for network security.

% --- 5. RISK ASSESSMENT SUMMARY ---
\section{Risk Assessment Summary}
This section synthesizes findings from the security control review and the technical scan. No pre-existing vulnerabilities were reported. The following new risks have been identified:

\begin{table}[h!]
\centering
\begin{tabular}{@{}lp{6cm}l@{}}
    \toprule
    \textbf{Risk Name} & \textbf{Overview} & \textbf{Severity} \\
    \midrule
    \textbf{Lack of MFA on Email} & Email accounts are protected only by passwords, making them highly vulnerable to phishing, credential stuffing, and brute-force attacks. A compromise could lead to Business Email Compromise (BEC), data breaches, and further internal network compromise. & \textcolor{red}{\textbf{Critical}} \\
    \addlinespace
    \textbf{Absence of Acceptable Use Policy} & Without a formal AUP, employees lack clear guidelines on the proper use of corporate assets, data handling, and security responsibilities. This increases the likelihood of accidental data exposure, intentional misuse, and non-compliance with regulations. & \textcolor{orange}{\textbf{High}} \\
    \bottomrule
\end{tabular}
\caption{Summary of Identified Risks.}
\end{table}

% --- 6. RECOMMENDATIONS ---
\section{Recommendations}
The following actions are recommended to mitigate the identified risks and improve the overall security posture.

\subsection{Critical Priority}
\begin{itemize}
    \item \textbf{Enforce MFA on All Email Accounts:}
    \begin{itemize}
        \item \textbf{Action:} Immediately deploy and enforce Multi-Factor Authentication (MFA) for all user access to the email system (\texttt{ObsidianOperatives.com}).
        \item \textbf{Justification:} This is the single most effective control to prevent unauthorized access to email accounts, mitigating the risk of BEC and phishing-related breaches.
    \end{itemize}
\end{itemize}

\subsection{High Priority}
\begin{itemize}
    \item \textbf{Develop and Implement an Acceptable Use Policy (AUP):}
    \begin{itemize}
        \item \textbf{Action:} Create a formal AUP document that clearly defines the rules and responsibilities for all employees when using company technology and data. This policy should be reviewed by management, distributed to all staff, and formally acknowledged via signature.
        \item \textbf{Justification:} An AUP establishes a baseline for secure behavior, reduces ambiguity, and provides a framework for enforcing security standards.
    \end{itemize}
\end{itemize}

\subsection{Informational}
\begin{itemize}
    \item \textbf{Maintain Network Security Posture:}
    \begin{itemize}
        \item \textbf{Action:} Continue the current practice of minimizing the external network footprint. Conduct regular, automated vulnerability scans against external IP addresses (\texttt{13.29.240.66}) to ensure no new services are inadvertently exposed.
        \item \textbf{Justification:} Proactive monitoring ensures that the current strong perimeter defense is maintained over time.
    \end{itemize}
\end{itemize}

% --- DOCUMENT END ---
\end{document}
```