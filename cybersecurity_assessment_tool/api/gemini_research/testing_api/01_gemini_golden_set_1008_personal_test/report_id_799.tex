```latex
\documentclass[12pt]{article}

% Preamble: Required Packages
\usepackage[margin=1in]{geometry}
\usepackage{pifont} % Required for check marks and crosses
\usepackage{booktabs} % For professional-looking tables
\usepackage{hyperref} % For clickable links and table of contents
\usepackage{url} % For formatting URLs
\usepackage{seqsplit} % For breaking long, unbreakable strings like IPs

% Document Metadata and Hyperlink Setup
\hypersetup{
    colorlinks=true,
    linkcolor=blue,
    filecolor=magenta,
    urlcolor=cyan,
    pdftitle={Cybersecurity Posture Assessment Report},
    pdfauthor={Cybersecurity Analysis Division},
}

\title{Cybersecurity Posture Assessment Report \\ \large For: \textbf{Ember Glow Hospitality}}
\author{Cybersecurity Analysis Division}
\date{\today}

\begin{document}

\maketitle
\tableofcontents
\newpage

\section{Executive Summary}
This report provides a comprehensive cybersecurity posture assessment for \textbf{Ember Glow Hospitality}. The analysis is based on a review of organizational security controls, an external network scan, and an evaluation of pre-existing risk data. The assessment identified several critical and high-risk security gaps that require immediate attention.

Key findings include a lack of Multi-Factor Authentication (MFA) for critical access points such as email and computer logins, the absence of a mandatory annual security awareness training program for all staff, and an externally exposed Secure Shell (SSH) service. These vulnerabilities, when combined, significantly increase the organization's risk of unauthorized access, data breach, and successful social engineering attacks. This report outlines these findings in detail and provides actionable recommendations to mitigate the identified risks and strengthen the overall security posture.

\section{Organizational Information}
The following information was provided as the basis for this assessment.
\begin{itemize}
    \item \textbf{Organization Name:} Ember Glow Hospitality
    \item \textbf{Email Domain:} \texttt{EmberGlowHospitality.org}
    \item \textbf{Website Domain:} \url{www.EmberGlowHospitality.org}
    \item \textbf{Known External IP:} \texttt{206.226.223.197}
\end{itemize}

\section{Security Control Review}
A review of foundational security controls was conducted via a questionnaire. The responses indicate several significant gaps in the organization's security posture. Responses marked with \ding{55} represent a deviation from security best practices and introduce substantial risk.

\begin{table}[h!]
\centering
\caption{Security Controls Questionnaire Results}
\label{tab:controls}
\begin{tabular}{p{0.6\textwidth}cc}
\toprule
\textbf{Control Question} & \textbf{Response} & \textbf{Status} \\
\midrule
Do you require MFA to access email? & No & \ding{55} \\
Do you require MFA to log into computers? & No & \ding{55} \\
Do you require MFA to access sensitive data systems? & Yes & \ding{51} \\
Does your organization have an employee acceptable use policy? & Yes & \ding{51} \\
Does your organization do security awareness training for new employees? & Yes & \ding{51} \\
Does your organization do security awareness training for all employees at least once per year? & No & \ding{55} \\
\bottomrule
\end{tabular}
\end{table}

\subsection*{Analysis}
The lack of MFA for email and computer access are critical vulnerabilities. Furthermore, the absence of annual security training for all employees means that the workforce's ability to recognize and respond to threats like phishing diminishes over time, making them a primary target for attackers.

\section{Technical Scan Results}
An external network scan was performed on the specified target to identify open ports and exposed services accessible from the public internet.
\begin{itemize}
    \item \textbf{Target IP Address:} \seqsplit{\texttt{2001:db8::1}}
    \item \textbf{Host Status:} Up
\end{itemize}

The following ports were found to be open:

\begin{table}[h!]
\centering
\caption{Open Port Scan Findings}
\label{tab:scan}
\begin{tabular}{ccl}
\toprule
\textbf{Port} & \textbf{State} & \textbf{Inferred Service} \\
\midrule
22 & Open & SSH (Secure Shell) \\
\bottomrule
\end{tabular}
\end{table}

\subsection*{Analysis}
The presence of an open SSH port (22) on an external-facing system is a critical security risk. This service is a primary target for automated brute-force attacks and credential stuffing, where attackers systematically attempt to guess credentials to gain unauthorized administrative access to the server.

\section{Risk Assessment}
The following table synthesizes findings from the security control review and the technical scan. No pre-existing vulnerabilities were reported in the input data. Each identified risk has been assigned a severity level based on its potential impact and likelihood of exploitation.

\begin{table}[h!]
\centering
\caption{Summary of Identified Risks}
\label{tab:risks}
\begin{tabular}{p{0.1\textwidth}p{0.25\textwidth}p{0.45\textwidth}l}
\toprule
\textbf{Risk ID} & \textbf{Risk Name} & \textbf{Description} & \textbf{Severity} \\
\midrule
RISK-001 & Lack of MFA for Email Access & Without MFA, a compromised password is all an attacker needs to gain full access to an employee's mailbox, leading to data exfiltration, business email compromise, and further internal phishing attacks. & Critical \\
\addlinespace
RISK-002 & Exposed SSH Service & The SSH service on \seqsplit{\texttt{2001:db8::1}} is publicly accessible, making it a prime target for automated brute-force attacks that could lead to a full system compromise and unauthorized network access. & Critical \\
\addlinespace
RISK-003 & Lack of MFA for Endpoint Login & The absence of MFA on computer logins allows an attacker with stolen credentials (physical or remote) to easily access the corporate network, user data, and potentially escalate privileges. & High \\
\addlinespace
RISK-004 & Inadequate Annual Security Training & Without regular, recurring training, employees are more likely to fall victim to phishing, malware, and other social engineering tactics, making them the weakest link in the organization's defense. & High \\
\bottomrule
\end{tabular}
\end{table}

\section{Recommendations}
Based on the risk assessment, the following actions are recommended to improve the security posture of \textbf{Ember Glow Hospitality}. These recommendations are prioritized to address the most critical risks first.

\begin{enumerate}
    \item \textbf{Implement Comprehensive MFA (Addresses: RISK-001, RISK-003):} Immediately begin a project to enforce mandatory Multi-Factor Authentication for all user accounts across all critical systems.
    \begin{itemize}
        \item \textbf{Priority 1:} Enforce MFA on all email accounts (e.g., Office 365, Google Workspace).
        \item \textbf{Priority 2:} Enforce MFA for all computer (endpoint) logins.
        \item This is the single most effective control to prevent unauthorized access resulting from compromised credentials.
    \end{itemize}

    \item \textbf{Secure Remote Access Protocols (Addresses: RISK-002):} The SSH service on host \seqsplit{\texttt{2001:db8::1}} must be secured immediately.
    \begin{itemize}
        \item \textbf{Immediate Action:} If this service is not required for business operations, it should be disabled and blocked at the firewall.
        \item \textbf{If Required:} If remote access is necessary, restrict access to a list of approved, static IP addresses (an allowlist) and require access via a Virtual Private Network (VPN) with MFA.
    \end{itemize}

    \item \textbf{Establish an Annual Security Training Program (Addresses: RISK-004):} Develop and implement a mandatory security awareness training program for all employees, to be completed annually. The training should be tracked for completion and cover key topics such as phishing identification, password hygiene, and the company's acceptable use policy.
\end{enumerate}

\end{document}
```