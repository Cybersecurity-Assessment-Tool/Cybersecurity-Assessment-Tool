```latex
\documentclass[12pt]{article}

% Preamble: Required Packages
\usepackage[margin=1in]{geometry}
\usepackage{pifont} % For \ding symbols
\usepackage{booktabs} % For professional tables
\usepackage{hyperref} % For hyperlinks
\usepackage{url} % For URL formatting
\usepackage{seqsplit} % For splitting long strings in tt font
\usepackage{graphicx}
\usepackage[table]{xcolor} % For coloring table cells

% --- Document Setup ---
\hypersetup{
    colorlinks=true,
    linkcolor=blue,
    filecolor=magenta,      
    urlcolor=cyan,
    pdftitle={Cybersecurity Assessment Report},
    pdfpagemode=FullScreen,
}

% Define colors for severity levels
\definecolor{critical}{HTML}{990000}
\definecolor{high}{HTML}{D14302}
\definecolor{medium}{HTML}{E5A500}
\definecolor{low}{HTML}{339900}

% --- Document Start ---
\begin{document}

% --- Title Page ---
\begin{titlepage}
    \centering
    \vspace*{1cm}
    \Huge
    \textbf{Cybersecurity Assessment Report}
    \vspace{1.5cm}
    \Large
    Prepared for: \textbf{Blue Marble}
    \vfill
    \Large
    Report Date: \today \\
    Generated by: Cybersecurity Analyst
\end{titlepage}

\tableofcontents
\newpage

% --- Section 1: Executive Summary ---
\section{Executive Summary}
This report provides a cybersecurity assessment for \textbf{Blue Marble}, based on a review of organizational security controls, an external network scan, and pre-existing risk data. The analysis was conducted on \today.

The assessment identified several critical and high-risk security gaps stemming from organizational policies. The most significant finding is the complete absence of Multi-Factor Authentication (MFA) for email, computer logins, and access to sensitive data systems. This represents a critical vulnerability, as compromised credentials could lead directly to a significant data breach.

Another high-risk finding is the lack of mandatory security awareness training for new employees. This gap leaves the organization vulnerable to social engineering and other human-centric attacks during the crucial initial employment period.

The external network scan performed against the target IP address \texttt{[Target IP]} did not identify any open ports. While this could indicate a strong firewall configuration, it also means the external attack surface could not be fully evaluated through this method. No previously documented vulnerabilities were provided for this assessment.

Immediate remediation should focus on implementing a robust MFA solution across all critical assets and integrating security training into the employee onboarding process.

% --- Section 2: Organizational Information ---
\section{Organizational Information}
The following details were provided by the client for this assessment.

\begin{table}[h!]
\centering
\begin{tabular}{@{}ll@{}}
\toprule
\textbf{Attribute} & \textbf{Value} \\ \midrule
Organization Name & \textbf{Blue Marble} \\
Email Domain      & \texttt{BlueMarble.com} \\
Website Domain    & \url{www.BlueMarble.com} \\
External IP Address & \texttt{78.13.101.34} \\ \bottomrule
\end{tabular}
\caption{Client-Provided Organizational Data}
\end{table}

% --- Section 3: Security Control Review ---
\section{Security Control Review}
A review of self-reported security controls was conducted via a questionnaire. The results highlight key areas of strength and weakness in the organization's security posture. A (\ding{51}) indicates an affirmative response (control in place), while a (\ding{55}) indicates a negative response (control gap).

\begin{table}[h!]
\centering
\begin{tabular}{@{}lc@{}}
\toprule
\textbf{Security Control Question} & \textbf{Response} \\ \midrule
Do you require MFA to access email? & \ding{55} \\
Do you require MFA to log into computers? & \ding{55} \\
Do you require MFA to access sensitive data systems? & \ding{55} \\
Does your organization have an employee acceptable use policy? & \ding{51} \\
Does your organization do security awareness training for new employees? & \ding{55} \\
Does your organization do security awareness training for all employees at least once per year? & \ding{51} \\ \bottomrule
\end{tabular}
\caption{Security Controls Questionnaire Results}
\end{table}

% --- Section 4: Technical Scan Results ---
\section{Technical Scan Results}
An external network vulnerability scan was conducted to identify open ports and services exposed to the internet.

\begin{itemize}
    \item \textbf{Target IP Address:} \texttt{[Target IP]}
    \item \textbf{Scan Date:} \today
\end{itemize}

\subsection{Findings}
\textbf{No open ports were discovered during the scan.}

\subsubsection{Interpretation}
The absence of open ports suggests that a firewall or other network security device is effectively blocking external probes. While this is a positive sign for perimeter security, it limits the visibility of this assessment into potential vulnerabilities on the services running behind the firewall. An internal, authenticated scan would be required for a more comprehensive technical analysis.

% --- Section 5: Risk Assessment ---
\section{Risk Assessment}
The following table synthesizes findings from the security control review and technical scan. Risks are prioritized by severity to guide remediation efforts. No pre-existing risks were provided for this assessment.

\begin{table}[h!]
\centering
\renewcommand{\arraystretch}{1.5}
\begin{tabular}{@{}p{0.1\linewidth} p{0.25\linewidth} p{0.45\linewidth} p{0.1\linewidth}@{}}
\toprule
\textbf{Risk ID} & \textbf{Risk Name} & \textbf{Overview} & \textbf{Severity} \\ \midrule
RISK-001 & \textbf{Lack of Multi-Factor Authentication (MFA)} & The organization does not enforce MFA for email, computer logins, or access to sensitive data. This allows an attacker with stolen credentials to gain unauthorized access without additional verification. & \cellcolor{critical!25}Critical \\
\addlinespace
RISK-002 & \textbf{Inadequate New Employee Onboarding} & New employees do not receive security awareness training upon being hired. This creates a window of vulnerability where new staff are more susceptible to phishing and other social engineering attacks. & \cellcolor{high!25}High \\
\bottomrule
\end{tabular}
\caption{Identified Risks and Severity}
\end{table}

% --- Section 6: Recommendations ---
\section{Recommendations}
Based on the findings of this assessment, the following actions are recommended to mitigate the identified risks and improve the overall security posture of \textbf{Blue Marble}.

\subsection{High Priority Recommendations}
\begin{description}
    \item[RISK-001: Implement MFA]
    \textbf{Action:} Deploy a mandatory Multi-Factor Authentication (MFA) solution for all employees and contractors. Prioritize implementation in the following order:
    \begin{enumerate}
        \item Remote access systems (VPN, RDP).
        \item Email (e.g., Office 365, Google Workspace).
        \item Access to systems containing sensitive or critical data.
        \item General computer and system logins.
    \end{enumerate}
    \textbf{Impact:} Drastically reduces the risk of account compromise via stolen credentials.

    \item[RISK-002: Enhance Onboarding Process]
    \textbf{Action:} Integrate mandatory cybersecurity awareness training into the standard new employee onboarding process. This training should be completed within the first week of employment and cover topics such as phishing, password hygiene, and the acceptable use policy.
    \textbf{Impact:} Reduces the likelihood of new employees falling victim to common cyber attacks and establishes a security-conscious culture from day one.
\end{description}

\subsection{Medium Priority Recommendations}
\begin{description}
    \item[Internal Vulnerability Assessment]
    \textbf{Action:} Conduct an internal, credentialed vulnerability scan of the network. The inconclusive external scan highlights the need for deeper visibility into the internal environment to identify unpatched systems, misconfigurations, and other vulnerabilities not visible from the internet.
    \textbf{Impact:} Provides a comprehensive view of internal security weaknesses, allowing for proactive remediation.
\end{description}

\end{document}
```