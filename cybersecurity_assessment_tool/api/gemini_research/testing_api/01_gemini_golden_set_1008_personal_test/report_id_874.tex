```latex
\documentclass[12pt]{article}

% Preamble: Required Packages
\usepackage[margin=1in]{geometry}
\usepackage{pifont} % For checkmarks and crosses
\usepackage{booktabs} % For professional tables
\usepackage{hyperref} % For clickable links
\usepackage{url}      % For URL formatting
\usepackage{seqsplit} % For splitting long strings without breaking
\usepackage{graphicx}
\usepackage{xcolor}

% Document Metadata
\title{Cybersecurity Posture Assessment Report}
\author{Cybersecurity Analysis Division}
\date{\today}

% Hyperref Setup
\hypersetup{
    colorlinks=true,
    linkcolor=blue,
    filecolor=magenta,      
    urlcolor=cyan,
    pdftitle={Cybersecurity Posture Assessment Report},
    pdfpagemode=FullScreen,
}

\begin{document}

\maketitle
\thispagestyle{empty}
\newpage
\tableofcontents
\newpage

% ------------------------------------------------------------------
% Section 1: Executive Summary
% ------------------------------------------------------------------
\section*{Executive Summary}

This report provides a comprehensive cybersecurity assessment for \textbf{Brimstone Manufacturing}, based on a review of organizational security controls, an external network scan, and an analysis of pre-existing risks. The assessment was conducted on \today.

Overall, \textbf{Brimstone Manufacturing} demonstrates a mixed security posture. Positive controls are in place, such as the mandatory use of Multi-Factor Authentication (MFA) for email and computer access, and the implementation of annual security awareness training for all staff. The external network scan of the target IP address (\texttt{[Target IP]}) revealed no open ports, which indicates a strong perimeter firewall configuration that effectively limits the external attack surface.

However, two significant gaps were identified through the security questionnaire, introducing critical and high-level risks to the organization:
\begin{itemize}
    \item \textbf{Critical Risk:} The absence of MFA for accessing sensitive data systems. This gap exposes the organization's most valuable data to compromise in the event of a credential theft incident.
    \item \textbf{High Risk:} The lack of mandatory security awareness training for new employees during their onboarding process. This leaves a critical window of vulnerability where new hires are more susceptible to social engineering and phishing attacks.
\end{itemize}

This report details these findings and provides actionable recommendations to mitigate the identified risks and enhance the organization's overall security resilience.

% ------------------------------------------------------------------
% Section 2: Organizational Information
% ------------------------------------------------------------------
\section*{Organizational Information}

The following details were provided for the assessment.

\begin{itemize}
    \item \textbf{Organization Name:} Brimstone Manufacturing
    \item \textbf{Email Domain:} \texttt{BrimstoneManufacturing.net}
    \item \textbf{Website Domain:} \url{www.BrimstoneManufacturing.net}
    \item \textbf{Primary External IP:} \texttt{169.111.226.96}
\end{itemize}

% ------------------------------------------------------------------
% Section 3: Security Control Review
% ------------------------------------------------------------------
\section*{Security Control Review}

A review of administrative and technical security controls was conducted via a questionnaire. The results below highlight implemented controls and identify significant gaps. A green checkmark (\textcolor{green}{\ding{51}}) indicates a positive control, while a red 'X' (\textcolor{red}{\ding{55}}) indicates a control gap that introduces risk.

\begin{table}[h!]
\centering
\caption{Organizational Security Controls Questionnaire}
\begin{tabular}{p{0.8\linewidth} c}
\toprule
\textbf{Control Question} & \textbf{Status} \\
\midrule
Do you require MFA to access email? & \textcolor{green}{\ding{51}} \\
Do you require MFA to log into computers? & \textcolor{green}{\ding{51}} \\
\textbf{Do you require MFA to access sensitive data systems?} & \textcolor{red}{\ding{55}} \\
Does your organization have an employee acceptable use policy? & \textcolor{green}{\ding{51}} \\
\textbf{Does your organization do security awareness training for new employees?} & \textcolor{red}{\ding{55}} \\
Does your organization do security awareness training for all employees at least once per year? & \textcolor{green}{\ding{51}} \\
\bottomrule
\end{tabular}
\end{table}

The two identified gaps are considered significant weaknesses in the organization's defense-in-depth strategy.

% ------------------------------------------------------------------
% Section 4: Technical Scan Results
% ------------------------------------------------------------------
\section*{Technical Scan Results}

An external network vulnerability scan was performed against the designated target IP address.

\begin{itemize}
    \item \textbf{Target IP Address:} \texttt{[Target IP]}
    \item \textbf{Scan Date:} Not available in scan data.
\end{itemize}

\subsection*{Summary of Findings}
The network scan completed successfully but did not identify any open TCP ports on the target host. This is a positive security finding, suggesting that the external firewall is properly configured to deny unsolicited inbound traffic, effectively minimizing the external attack surface. No services were exposed to the public internet on the scanned IP address.

% ------------------------------------------------------------------
% Section 5: Risk Assessment
% ------------------------------------------------------------------
\section*{Risk Assessment}

This section correlates findings from the security control review, technical scan, and pre-existing risk data. The primary risks identified stem from gaps in administrative and access controls. No pre-existing vulnerabilities were provided for this assessment.

\begin{table}[h!]
\centering
\caption{Summary of Identified Risks}
\begin{tabular}{p{0.25\linewidth} p{0.15\linewidth} p{0.5\linewidth}}
\toprule
\textbf{Risk Name} & \textbf{Severity} & \textbf{Overview} \\
\midrule
Lack of MFA for Sensitive Data Systems & \textbf{Critical} & The absence of MFA on systems holding sensitive corporate or customer data means a single compromised password could lead to a major data breach. This is the most critical vulnerability identified. \\
\noalign{\medskip}
No Security Training for New Employees & \textbf{High} & New hires are often targeted by attackers as they are unfamiliar with company policies and security procedures. Failing to train them upon hiring creates an immediate and ongoing vulnerability to phishing and social engineering. \\
\bottomrule
\end{tabular}
\end{table}

% ------------------------------------------------------------------
% Section 6: Recommendations
% ------------------------------------------------------------------
\section*{Recommendations}

The following actionable recommendations are provided to address the identified risks and improve the overall security posture of \textbf{Brimstone Manufacturing}.

\subsection*{Immediate Priorities (Critical Risk)}
\begin{enumerate}
    \item \textbf{Implement MFA for All Sensitive Systems:}
    \begin{itemize}
        \item \textbf{Action:} Immediately begin a project to identify all systems and applications containing sensitive data (e.g., financial records, intellectual property, PII) and enforce MFA for all user access.
        \item \textbf{Justification:} This directly mitigates the risk of a data breach resulting from stolen credentials, which is a primary attack vector. This is the single most effective control to implement to protect critical data.
    \end{itemize}
\end{enumerate}

\subsection*{Near-Term Priorities (High Risk)}
\begin{enumerate}
    \setcounter{enumi}{1} % Continue numbering
    \item \textbf{Integrate Security Training into Employee Onboarding:}
    \begin{itemize}
        \item \textbf{Action:} Develop a mandatory security awareness training module for all new hires. This training must be completed as part of the formal onboarding process before system access is granted.
        \item \textbf{Justification:} This closes a significant window of vulnerability by ensuring that all employees, from their first day, understand their security responsibilities, recognize common threats like phishing, and are familiar with the acceptable use policy.
    \end{itemize}
\end{enumerate}

\end{document}
```