An analysis of the provided data has been completed. The following is the synthesized Cybersecurity Posture Assessment Report in the required LaTeX format.

```latex
\documentclass[12pt]{article}

% Required Packages
\usepackage[margin=1in]{geometry}
\usepackage{pifont} % For checkmarks and crosses
\usepackage{booktabs} % For professional tables
\usepackage[hidelinks]{hyperref}
\usepackage{url}
\usepackage{seqsplit} % For splitting long strings in tt font
\usepackage{graphicx}
\usepackage{fancyhdr}
\usepackage{lastpage}
\usepackage{xcolor}

% Define colors for severity
\definecolor{criticalred}{HTML}{D10000}
\definecolor{highorange}{HTML}{E25F00}
\definecolor{mediumyellow}{HTML}{F0C200}
\definecolor{lowblue}{HTML}{0073E6}

% Header and Footer
\pagestyle{fancy}
\fancyhf{} % Clear all header and footer fields
\fancyhead[L]{Cybersecurity Posture Assessment}
\fancyhead[R]{Solaris Energy}
\fancyfoot[C]{\thepage\ of \pageref{LastPage}}
\renewcommand{\headrulewidth}{0.4pt}
\renewcommand{\footrulewidth}{0.4pt}

\begin{document}

\title{
    \vspace{2cm}
    \textbf{Cybersecurity Posture Assessment Report} \\
    \large \textit{Generated: \today}
}
\author{Cybersecurity Analysis Division}
\date{}

\maketitle
\thispagestyle{empty}
\newpage

\tableofcontents
\newpage

% --- Section 1: Executive Overview ---
\section{Executive Overview}

This report provides a cybersecurity posture assessment for \textbf{Solaris Energy}, based on an analysis of self-reported organizational data. The assessment's primary goal is to identify security gaps, evaluate existing controls, and provide actionable recommendations to enhance the organization's overall security resilience.

The analysis was conducted using a security controls questionnaire. It is critical to note that the provided technical network scan data (\texttt{Input\_1\_Network\_Scan\_JSON}) and the list of current known risks (\texttt{Input\_3\_Current\_Risks\_JSON}) were corrupted and could not be parsed. Therefore, this assessment is based solely on the organizational data provided.

The review identified several high-priority risks requiring immediate attention:
\begin{itemize}
    \item \textbf{Critical Risk - Lack of MFA on Email:} The absence of Multi-Factor Authentication (MFA) on email accounts exposes the organization to a significant risk of Business Email Compromise (BEC), phishing, and unauthorized access to sensitive communications.
    \item \textbf{High Risk - No Acceptable Use Policy:} The lack of a formal Acceptable Use Policy (AUP) creates ambiguity regarding the proper use of company assets, increasing the risk of insider threats and unintentional policy violations.
    \item \textbf{High Risk - Inadequate Security Training:} While new employees receive training, the absence of a mandatory annual security awareness program for all staff means the organization is not reinforcing security best practices against evolving threats.
\end{itemize}

Immediate remediation of these findings is strongly recommended to reduce the attack surface and strengthen the organization's defense against common cyber threats. A follow-up technical assessment is also crucial once scan data is available.

% --- Section 2: Organizational Information ---
\section{Organizational Information}

The following details were provided by the client and used as the basis for this assessment.

\begin{tabular}{@{}ll}
    \toprule
    \textbf{Attribute} & \textbf{Value} \\
    \midrule
    Organization Name & \textbf{Solaris Energy} \\
    Email Domain & \texttt{SolarisEnergy.net} \\
    Website Domain & \url{www.SolarisEnergy.net} \\
    External IP Address & \texttt{85.94.246.106} \\
    \bottomrule
\end{tabular}

% --- Section 3: Security Control Review ---
\section{Security Control Review (Questionnaire Analysis)}

The following table summarizes the organization's responses to the security controls questionnaire. Each response has been assessed against industry best practices. Items marked with \ding{55} (No) indicate a potential security gap that requires further investigation and remediation.

\begin{table}[h!]
\centering
\begin{tabular}{p{8cm} c l}
    \toprule
    \textbf{Control Question} & \textbf{Response} & \textbf{Assessment} \\
    \midrule
    Do you require MFA to access email? & \ding{55} & \textcolor{criticalred}{\textbf{Critical Gap}} \\
    Do you require MFA to log into computers? & \ding{51} & Control in Place \\
    Do you require MFA to access sensitive data systems? & \ding{51} & Control in Place \\
    Does your organization have an employee acceptable use policy? & \ding{55} & \textcolor{highorange}{\textbf{High Risk}} \\
    Does your organization do security awareness training for new employees? & \ding{51} & Good Practice \\
    Does your organization do security awareness training for all employees at least once per year? & \ding{55} & \textcolor{highorange}{\textbf{High Risk}} \\
    \bottomrule
\end{tabular}
\caption{Security Controls Questionnaire Results}
\label{tab:controls}
\end{table}

% --- Section 4: Technical Scan Results ---
\section{Technical Scan Results}

\subsection{External Host Scan (\texttt{85.94.246.106})}
The network scan data provided in \texttt{Input\_1\_Network\_Scan\_JSON} was found to be corrupted and could not be processed. Consequently, no technical analysis of open ports, running services, or potential vulnerabilities on the external IP address could be performed.

\textbf{Recommendation:} It is imperative to conduct a new, successful network vulnerability scan against the external perimeter to identify and remediate any technical vulnerabilities that could be exploited by attackers.

% --- Section 5: Risk Assessment ---
\section{Risk Assessment}

This risk assessment is based on the findings from the Security Control Review. The list of pre-existing vulnerabilities (\texttt{Input\_3\_Current\_Risks\_JSON}) was unavailable due to data corruption. The identified risks are prioritized by severity to guide remediation efforts.

\begin{table}[h!]
\centering
\begin{tabular}{p{2cm} p{7cm} l}
    \toprule
    \textbf{Risk ID} & \textbf{Risk Description} & \textbf{Severity} \\
    \midrule
    RISK-001 & Lack of MFA on email accounts significantly increases the risk of account takeover, data breaches, and successful phishing attacks. & \textcolor{criticalred}{\textbf{Critical}} \\
    \addlinespace
    RISK-002 & The absence of a formal Acceptable Use Policy (AUP) leads to inconsistent security practices and a lack of enforceable guidelines for employees. & \textcolor{highorange}{\textbf{High}} \\
    \addlinespace
    RISK-003 & Without mandatory, annual security awareness training for all staff, the workforce's ability to recognize and respond to new threats diminishes over time. & \textcolor{highorange}{\textbf{High}} \\
    \bottomrule
\end{tabular}
\caption{Identified Risks and Severity}
\label{tab:risks}
\end{table}

% --- Section 6: Recommendations ---
\section{Recommendations}

The following actions are recommended to address the identified risks and improve the overall security posture of \textbf{Solaris Energy}.

\begin{enumerate}
    \item \textbf{Implement MFA for Email (RISK-001 - Critical):}
    \begin{itemize}
        \item \textbf{Action:} Immediately enforce mandatory Multi-Factor Authentication (MFA) for all user and administrative accounts with access to the \texttt{SolarisEnergy.net} email system.
        \item \textbf{Justification:} This is the single most effective control to prevent Business Email Compromise (BEC) and unauthorized account access, which are primary vectors for financial fraud and data exfiltration.
    \end{itemize}
    \vspace{0.5cm}
    \item \textbf{Develop and Implement an Acceptable Use Policy (RISK-002 - High):}
    \begin{itemize}
        \item \textbf{Action:} Create a formal AUP that clearly defines the rules and responsibilities for all employees when using company networks, devices, and data. This policy should be reviewed by legal counsel, communicated to all staff, and formally acknowledged via signature.
        \item \textbf{Justification:} An AUP establishes a baseline for secure behavior, reduces insider risk, and provides a framework for disciplinary action in case of violations.
    \end{itemize}
    \vspace{0.5cm}
    \item \textbf{Establish a Mandatory Annual Security Training Program (RISK-003 - High):}
    \begin{itemize}
        \item \textbf{Action:} Institute a mandatory security awareness training program for all employees, to be completed annually. The training should cover current threats such as phishing, social engineering, ransomware, and proper data handling.
        \item \textbf{Justification:} Regular training reinforces security concepts and keeps employees vigilant against the latest attack techniques, turning the workforce into a human firewall.
    \end{itemize}
    \vspace{0.5cm}
    \item \textbf{Conduct a Technical Vulnerability Assessment:}
    \begin{itemize}
        \item \textbf{Action:} Re-run the external network scan against \texttt{85.94.246.106} and provide the complete results for analysis. Additionally, provide the current risk register data.
        \item \textbf{Justification:} A comprehensive security posture assessment requires both policy/procedural review and technical validation. Without technical data, a significant portion of the organization's risk profile remains unknown.
    \end{itemize}
\end{enumerate}

\end{document}
```