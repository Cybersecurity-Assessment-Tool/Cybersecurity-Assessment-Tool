```latex
\documentclass[12pt]{article}

% --- PACKAGES ---
\usepackage[margin=1in]{geometry}
\usepackage{pifont} % For checkmarks and crosses
\usepackage{booktabs} % For professional tables
\usepackage{hyperref} % For clickable links
\usepackage{url} % For URL formatting
\usepackage{seqsplit} % To split long strings in tt font
\usepackage[utf8]{inputenc}
\usepackage{graphicx}
\usepackage{xcolor}
\usepackage{datetime}

% --- DOCUMENT METADATA ---
\title{Cybersecurity Posture Assessment Report \\ \large For: \textbf{Vivid Vision}}
\author{Cybersecurity Analyst Group}
\date{\today}

% --- HYPERREF SETUP ---
\hypersetup{
    colorlinks=true,
    linkcolor=black,
    urlcolor=blue,
    pdftitle={Cybersecurity Posture Assessment Report},
    pdfauthor={Cybersecurity Analyst Group},
    pdfsubject={Security Assessment},
    pdfkeywords={Cybersecurity, Risk, Assessment}
}

\begin{document}

\maketitle
\thispagestyle{empty}
\newpage
\tableofcontents
\newpage

% --- EXECUTIVE OVERVIEW ---
\section{Executive Overview}

This report details the findings of a cybersecurity posture assessment for \textbf{Vivid Vision}. The assessment incorporated a review of organizational security controls via a questionnaire, an external network vulnerability scan, and an analysis of pre-existing risks.

\paragraph{Key Findings:} The organization demonstrates a strong commitment to identity and access management, with Multi-Factor Authentication (MFA) consistently enforced across email, computer logins, and sensitive data systems. The external network scan of the target IP address, \texttt{[Target IP]}, revealed no open ports or exposed services, indicating a robust network perimeter.

\paragraph{Identified Gaps:} Despite these strengths, two significant administrative control gaps were identified. The absence of a formal Employee Acceptable Use Policy (AUP) and the lack of mandatory security awareness training for new hires present high-risk vulnerabilities. These gaps expose the organization to insider threats, social engineering, and potential non-compliance with regulatory standards.

\paragraph{Overall Posture:} \textbf{Vivid Vision} has a solid technical security foundation but requires immediate attention to formalize its administrative and procedural controls. Addressing the identified policy and training gaps is critical to mitigating human-centric risks and maturing the overall security program.

% --- ORGANIZATIONAL INFORMATION ---
\section{Organizational Information}

The following information was provided by the client and used as a baseline for this assessment.

\begin{table}[h!]
\centering
\begin{tabular}{@{}ll@{}}
\toprule
\textbf{Attribute} & \textbf{Value} \\ \midrule
Organization Name & \textbf{Vivid Vision} \\
Email Domain & \texttt{VividVision.com} \\
Website Domain & \url{www.VividVision.com} \\
External IP Address & \texttt{99.87.45.98} \\ \bottomrule
\end{tabular}
\caption{Client Profile}
\label{tab:org_info}
\end{table}

% --- SECURITY CONTROL REVIEW ---
\section{Security Control Review}

A review of internal security controls was conducted based on a questionnaire. The responses highlight areas of both strength and weakness in the current security posture. "No" answers indicate critical gaps that require remediation.

\begin{table}[h!]
\centering
\begin{tabular}{@{}p{0.75\textwidth}c@{}}
\toprule
\textbf{Control Question} & \textbf{Response} \\ \midrule
Do you require MFA to access email? & \ding{51} \\
Do you require MFA to log into computers? & \ding{51} \\
Do you require MFA to access sensitive data systems? & \ding{51} \\
Does your organization have an employee acceptable use policy? & \textcolor{red}{\ding{55}} \\
Does your organization do security awareness training for new employees? & \textcolor{red}{\ding{55}} \\
Does your organization do security awareness training for all employees at least once per year? & \ding{51} \\ \bottomrule
\end{tabular}
\caption{Security Controls Questionnaire Analysis}
\label{tab:controls}
\end{table}

\paragraph{Analysis:} The consistent implementation of MFA is commendable and significantly reduces the risk of unauthorized access. However, the lack of an Acceptable Use Policy and security training for new hires are critical deficiencies. New employees are a primary target for attackers, and without immediate training, they represent a significant vulnerability from their first day.

% --- TECHNICAL SCAN RESULTS ---
\section{Technical Scan Results}

An external network scan was performed to identify potential vulnerabilities in internet-facing systems.

\begin{itemize}
    \item \textbf{Target IP Address:} \texttt{[Target IP]}
    \item \textbf{Scan Date:} \today
    \item \textbf{Summary:} The scan completed successfully and did not identify any open TCP or UDP ports on the target system.
\end{itemize}

\paragraph{Conclusion:} This result is highly positive. It suggests that the organization's firewall and network perimeter security controls are effectively configured to prevent unauthorized external access, adhering to the principle of least privilege by not exposing any unnecessary services to the internet.

% --- RISK ASSESSMENT ---
\section{Risk Assessment}

This section consolidates risks identified from the security control review, technical scan, and any pre-existing vulnerability data. No pre-existing risks were reported, and no technical vulnerabilities were discovered during the scan. The primary risks are administrative in nature.

\begin{table}[h!]
\centering
\begin{tabular}{@{}p{0.1\textwidth}p{0.25\textwidth}p{0.45\textwidth}l@{}}
\toprule
\textbf{ID} & \textbf{Risk Name} & \textbf{Description} & \textbf{Severity} \\ \midrule
RISK-001 & Lack of Employee Acceptable Use Policy (AUP) & Without a formal AUP, employees lack clear guidance on the proper use of company assets, data handling, and security responsibilities. This increases the risk of insider threat, data leakage, and legal liability. & \textbf{High} \\
\addlinespace
RISK-002 & No Security Training for New Employees & New hires are not provided with security awareness training during onboarding. This makes them highly susceptible to phishing, social engineering, and unintentional policy violations, creating a critical window of vulnerability. & \textbf{High} \\ \bottomrule
\end{tabular}
\caption{Identified Risk Summary}
\label{tab:risks}
\end{table}

% --- RECOMMENDATIONS ---
\section{Recommendations}

The following actionable recommendations are provided to address the identified risks and strengthen the organization's overall security posture.

\subsection{Develop and Implement an Acceptable Use Policy (AUP)}
\begin{itemize}
    \item \textbf{Action:} Draft a comprehensive AUP that clearly defines the rules and responsibilities for all employees when using company technology and data.
    \item \textbf{Details:} The policy should cover topics such as data classification and handling, password security, internet and email usage, use of personal devices (BYOD), and procedures for reporting security incidents.
    \item \textbf{Implementation:} The AUP should be formally approved by management, distributed to all current employees for review and acknowledgment, and integrated into the new hire onboarding process.
    \item \textbf{Risk Mitigated:} RISK-001
\end{itemize}

\subsection{Establish a New Hire Security Onboarding Program}
\begin{itemize}
    \item \textbf{Action:} Create a mandatory security awareness training module that all new employees must complete as part of their onboarding process.
    \item \textbf{Details:} This initial training should provide a foundational understanding of critical security topics, including phishing identification, social engineering tactics, MFA importance, and a thorough review of the newly created AUP.
    \item \textbf{Implementation:} This program should be a prerequisite for gaining full access to corporate systems. It will supplement the existing annual training program by ensuring a secure baseline for all personnel from day one.
    \item \textbf{Risk Mitigated:} RISK-002
\end{itemize}

\end{document}
```