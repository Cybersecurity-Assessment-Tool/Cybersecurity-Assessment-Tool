```latex
\documentclass[12pt]{article}

% --- PACKAGE IMPORTS ---
\usepackage[margin=1in]{geometry} % Set page margins
\usepackage{pifont}               % For checkmarks and crosses (\ding)
\usepackage{booktabs}             % For professional-looking tables
\usepackage{graphicx}             % For logos, etc. (optional)
\usepackage{hyperref}             % For hyperlinks
\usepackage{url}                  % For URL formatting
\usepackage{seqsplit}             % To split long strings in texttt
\usepackage{array}                % For table column formatting

% --- DOCUMENT SETUP ---
\hypersetup{
    colorlinks=true,
    linkcolor=black,
    filecolor=magenta,      
    urlcolor=blue,
    pdftitle={Cybersecurity Posture Report},
    pdfpagemode=FullScreen,
}

\newcommand{\yes}{\ding{51}} % Green checkmark
\newcommand{\no}{\ding{55}}  % Red X

% --- DOCUMENT START ---
\begin{document}

% --- TITLE PAGE ---
\begin{titlepage}
    \centering
    \vspace*{1cm}
    \Huge \textbf{Cybersecurity Posture Report}
    \vspace{1.5cm}
    \Large \textbf{Prepared for:} \\
    \vspace{0.5cm}
    \huge Grizzly Peak
    \vfill
    \large \textbf{Date of Report:} \\
    \vspace{0.5cm}
    \today
\end{titlepage}

\tableofcontents
\newpage

% --- EXECUTIVE SUMMARY ---
\section{Executive Summary}
This report provides a comprehensive analysis of the cybersecurity posture of Grizzly Peak, based on a combination of organizational data, technical network scanning, and a review of pre-existing risks. The assessment reveals a mixed security landscape. While foundational controls such as Multi-Factor Authentication (MFA) are in place for email and computer access, critical gaps exist that expose the organization to significant risk.

Key findings include a lack of MFA for sensitive data systems and a complete absence of a security awareness training program. These organizational gaps are compounded by a technical finding of an exposed service on the localhost interface, correlating with a known critical risk.

Immediate remediation is required to address these high-impact vulnerabilities. Recommendations focus on implementing mandatory MFA, establishing a robust security awareness training program, and securing the identified network service. Proactive measures will significantly strengthen the organization's defense against common cyber threats.

% --- ORGANIZATIONAL INFORMATION ---
\section{Organizational Information}
The following details were provided for the assessment. This information is used to establish the context and scope of the review.

\begin{table}[h!]
\centering
\begin{tabular}{@{}ll@{}}
\toprule
\textbf{Attribute} & \textbf{Value} \\
\midrule
Organization Name & Grizzly Peak \\
Email Domain & \texttt{GrizzlyPeak.com} \\
Website Domain & \url{www.GrizzlyPeak.com} \\
External IP Address & \texttt{115.99.213.3} \\
\bottomrule
\end{tabular}
\caption{Client Organizational Data}
\label{tab:org_data}
\end{table}

% --- SECURITY CONTROL REVIEW ---
\section{Security Control Review}
A review of organizational security controls was conducted via a questionnaire. The responses indicate several areas of concern where security best practices are not being met. These gaps represent a significant increase in organizational risk.

\begin{table}[h!]
\centering
\begin{tabular}{>{\raggedright\arraybackslash}p{9cm} c c}
\toprule
\textbf{Control Question} & \textbf{Response} & \textbf{Status} \\
\midrule
Do you require MFA to access email? & Yes & \yes \\
Do you require MFA to log into computers? & Yes & \yes \\
\textbf{Do you require MFA to access sensitive data systems?} & \textbf{No} & \no \\
Does your organization have an employee acceptable use policy? & Yes & \yes \\
\textbf{Does your organization do security awareness training for new employees?} & \textbf{No} & \no \\
\textbf{Does your organization do security awareness training for all employees at least once per year?} & \textbf{No} & \no \\
\bottomrule
\end{tabular}
\caption{Security Controls Questionnaire Results}
\label{tab:controls}
\end{table}

\subsection*{Analysis of Control Gaps}
\begin{itemize}
    \item \textbf{MFA on Sensitive Systems:} The absence of MFA for sensitive data systems is a critical vulnerability. Should an attacker compromise a user's credentials, they would have direct access to the organization's most valuable data.
    \item \textbf{Security Awareness Training:} The lack of any security awareness training program leaves the organization highly susceptible to social engineering attacks, such as phishing. Employees are the first line of defense, and without proper training, they are unprepared to identify and report threats.
\end{itemize}

% --- TECHNICAL SCAN RESULTS ---
\section{Technical Scan Results}
A network scan was performed to identify exposed services and potential vulnerabilities on the specified target.

\begin{itemize}
    \item \textbf{Target IP:} \texttt{127.0.0.1}
    \item \textbf{Scan Date:} Not Specified
    \item \textbf{Scanner Used:} Nmap
\end{itemize}

The scan revealed the following open port:

\begin{table}[h!]
\centering
\begin{tabular}{@{}cccc@{}}
\toprule
\textbf{Port} & \textbf{Protocol} & \textbf{State} & \textbf{Inferred Service} \\
\midrule
22 & TCP & open & SSH (Secure Shell) \\
\bottomrule
\end{tabular}
\caption{Open Ports Detected on \texttt{127.0.0.1}}
\label{tab:scan_results}
\end{table}

\subsection*{Analysis of Technical Findings}
The scan identified that port 22 (commonly used for SSH) is open on the localhost interface (\texttt{127.0.0.1}). This finding directly correlates with the pre-existing risk identified in Input 3, "Localhost Exposed." While localhost is typically not accessible from external networks, this configuration could be exploited by malware already on the system or through misconfigurations, potentially leading to privilege escalation or lateral movement within the network. The version of the SSH service was not determined, and it is crucial to ensure it is up-to-date and securely configured.

% --- RISK ASSESSMENT SUMMARY ---
\section{Risk Assessment Summary}
The following table synthesizes findings from the security control review, technical scan, and pre-existing risk data into a prioritized list.

\begin{table}[h!]
\centering
\begin{tabular}{@{}p{5cm} p{7cm} l@{}}
\toprule
\textbf{Risk / Vulnerability} & \textbf{Description} & \textbf{Severity} \\
\midrule
\textbf{Exposed Service on Localhost} & Port 22/SSH is open on the localhost interface, posing a risk of local privilege escalation or unauthorized access. & \textbf{Critical} \\
\addlinespace
\textbf{No MFA for Sensitive Data} & Lack of multi-factor authentication on critical systems allows for data compromise via stolen credentials. & \textbf{Critical} \\
\addlinespace
\textbf{No Security Awareness Training} & Employees are not trained to recognize or respond to phishing and other social engineering attacks, increasing the likelihood of a breach. & \textbf{High} \\
\bottomrule
\end{tabular}
\caption{Consolidated Risk Register}
\label{tab:risk_summary}
\end{table}

% --- RECOMMENDATIONS ---
\section{Recommendations}
Based on the analysis, the following actions are recommended to mitigate the identified risks and improve the overall security posture of Grizzly Peak.

\subsection{Immediate Actions (0-30 Days)}
\begin{enumerate}
    \item \textbf{Implement MFA on Sensitive Systems:}
    \begin{itemize}
        \item \textbf{Action:} Enforce MFA on all applications, databases, and administrative interfaces that contain or provide access to sensitive data (e.g., PII, financial records, intellectual property).
        \item \textbf{Impact:} Drastically reduces the risk of unauthorized access and data breaches resulting from compromised credentials.
    \end{itemize}
    \item \textbf{Investigate and Secure Exposed Service:}
    \begin{itemize}
        \item \textbf{Action:} Determine the business need for the SSH service running on \texttt{127.0.0.1}. If it is not required, disable the service. If it is required, ensure it is configured securely (e.g., disable root login, use key-based authentication) and that host-based firewall rules prevent unintended access.
        \item \textbf{Impact:} Mitigates the "Localhost Exposed" risk and prevents potential local exploitation.
    \end{itemize}
\end{enumerate}

\subsection{Strategic Actions (30-90 Days)}
\begin{enumerate}
    \item \textbf{Establish a Security Awareness Training Program:}
    \begin{itemize}
        \item \textbf{Action:} Procure and implement a security awareness training solution. Enroll all new hires as part of their onboarding process and schedule mandatory annual training for all existing employees. Conduct regular phishing simulations to reinforce learning.
        \item \textbf{Impact:} Creates a security-conscious culture and reduces the organization's susceptibility to human-targeted attacks.
    \end{itemize}
\end{enumerate}

\end{document}
```