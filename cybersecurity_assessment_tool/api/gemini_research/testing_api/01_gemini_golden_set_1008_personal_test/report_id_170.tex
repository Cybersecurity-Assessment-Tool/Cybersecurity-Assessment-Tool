```latex
\documentclass[12pt]{article}

% Preamble: Required Packages
\usepackage[margin=1in]{geometry}
\usepackage{pifont}         % For checkmarks and crosses (\ding{51}, \ding{55})
\usepackage{booktabs}       % For professional-looking tables
\usepackage{hyperref}       % For clickable links and table of contents
\usepackage{url}            % For formatting URLs
\usepackage{seqsplit}       % To split long, unbreakable strings
\usepackage[utf8]{inputenc} % For text encoding

% Custom Commands
\newcommand{\yes}{\ding{51}}
\newcommand{\no}{\ding{55}}

% Hyperlink Setup
\hypersetup{
    colorlinks=true,
    linkcolor=blue,
    filecolor=magenta,
    urlcolor=cyan,
    pdftitle={Cybersecurity Posture Assessment Report},
    pdfpagemode=FullScreen,
}

% Document Title Information
\title{Cybersecurity Posture Assessment Report \\ \large For: Urban Jungle Planning}
\author{Cybersecurity Analysis Division}
\date{\today}

\begin{document}

\maketitle
\thispagestyle{empty}
\newpage

\tableofcontents
\newpage

\section{Executive Summary}

This report provides a comprehensive cybersecurity assessment for Urban Jungle Planning, based on an analysis of network scan data, organizational security controls, and pre-existing risk information. The assessment reveals several critical and high-risk vulnerabilities that require immediate attention to mitigate potential security breaches, data loss, and operational disruption.

Key findings indicate significant weaknesses in both technical and administrative controls. A critical vulnerability was identified on an internal network host (\texttt{10.0.0.15}), which is running an outdated and notoriously insecure FTP service (\texttt{vsftpd 2.3.4}) with anonymous login enabled. This configuration exposes the organization to severe risks, including unauthorized data access and potential system compromise.

Furthermore, the review of organizational policies highlights a critical lack of Multi-Factor Authentication (MFA) for accessing email and other sensitive data systems. Compounded by the absence of a formal employee acceptable use policy, these gaps create a high-risk environment susceptible to phishing, account takeover, and insider threats.

This report outlines the identified risks and provides prioritized, actionable recommendations to strengthen the organization's security posture.

\section{Organizational Information}

The following information was provided for the assessment.

\begin{tabular}{@{}ll}
\toprule
\textbf{Attribute} & \textbf{Value} \\
\midrule
Organization Name & Urban Jungle Planning \\
Email Domain & \texttt{UrbanJunglePlanning.org} \\
Website Domain & \url{www.UrbanJunglePlanning.org} \\
External IP Address & \texttt{3.40.115.45} \\
\bottomrule
\end{tabular}

\section{Security Control Review}

A review of the organization's administrative and access controls was conducted via a security questionnaire. The results below highlight significant gaps in security best practices. "No" answers indicate a failure to meet a baseline security standard and are considered high-risk findings.

\begin{tabular}{@{}p{0.7\linewidth}c}
\toprule
\textbf{Control Question} & \textbf{Status} \\
\midrule
Do you require MFA to access email? & \no \\
Do you require MFA to log into computers? & \yes \\
Do you require MFA to access sensitive data systems? & \no \\
Does your organization have an employee acceptable use policy? & \no \\
Does your organization do security awareness training for new employees? & \yes \\
Does your organization do security awareness training for all employees at least once per year? & \yes \\
\bottomrule
\end{tabular}

\section{Technical Scan Results}

An Nmap scan was performed on the internal network host \texttt{10.0.0.15}. The scan identified one open port with a highly vulnerable service.

\subsection{Open Port Analysis}

\begin{tabular}{@{}lllll}
\toprule
\textbf{Port} & \textbf{Service} & \textbf{Product} & \textbf{Version} & \textbf{Notes} \\
\midrule
21/tcp & ftp & vsftpd & 2.3.4 & \textbf{CRITICAL:} Anonymous FTP login allowed. \\
\bottomrule
\end{tabular}

\subsection{Vulnerability Details: vsftpd 2.3.4}
The version of vsftpd detected (\texttt{2.3.4}) is extremely outdated and contains a well-known critical backdoor vulnerability (\textbf{CVE-2011-2523}). This specific version was intentionally compromised in its source code, allowing an attacker to execute arbitrary commands with root privileges by sending a specific sequence of characters as a username. The presence of this service, combined with anonymous login, represents a severe and immediate threat to the network.

\section{Consolidated Risk Assessment}

The following table synthesizes findings from the security control review, technical scan, and pre-existing risk data into a consolidated list of security risks.

\begin{tabular}{@{}p{0.25\linewidth}p{0.5\linewidth}l}
\toprule
\textbf{Risk Name} & \textbf{Overview} & \textbf{Severity} \\
\midrule
\textbf{Vulnerable FTP Server} & An internal server is running vsftpd 2.3.4, which has a known backdoor (CVE-2011-2523). Anonymous login is enabled, allowing unauthenticated access. & \textbf{Critical} \\
\addlinespace
\textbf{No MFA on Critical Systems} & Multi-Factor Authentication is not enforced for email or sensitive data systems, leaving them vulnerable to credential theft and phishing attacks. & \textbf{Critical} \\
\addlinespace
\textbf{No Acceptable Use Policy} & The absence of a formal policy defining acceptable use of company assets creates ambiguity and increases the risk of insider threats and misuse. & \textbf{High} \\
\addlinespace
\textbf{Outdated Windows Policy} & (Pre-existing risk) Workstations are running Windows 7, which is an unsupported operating system and does not receive security updates. & \textbf{Medium} \\
\bottomrule
\end{tabular}

\section{Recommendations}

The following prioritized recommendations are provided to address the identified risks and improve the overall security posture of Urban Jungle Planning.

\subsection{Immediate Actions (To Be Completed in < 7 Days)}
\begin{enumerate}
    \item \textbf{Remediate Vulnerable FTP Server:}
    \begin{itemize}
        \item Immediately take the FTP service on \texttt{10.0.0.15} offline.
        \item If the service is business-critical, upgrade to the latest stable version of vsftpd or replace it with a secure file transfer protocol like SFTP (SSH File Transfer Protocol).
        \item If FTP must be used, disable anonymous login immediately and enforce strong password policies.
    \end{itemize}
    \item \textbf{Implement MFA on Email:}
    \begin{itemize}
        \item Enforce MFA for all user accounts on the \texttt{UrbanJunglePlanning.org} email domain. This is the single most effective control to prevent account takeovers via phishing.
    \end{itemize}
\end{enumerate}

\subsection{Short-Term Actions (To Be Completed in < 30 Days)}
\begin{enumerate}
    \item \textbf{Implement MFA on Sensitive Systems:}
    \begin{itemize}
        \item Identify all systems containing sensitive data and enforce MFA for access.
    \end{itemize}
    \item \textbf{Develop and Implement an Acceptable Use Policy (AUP):}
    \begin{itemize}
        \item Draft a formal AUP that clearly outlines the rules and responsibilities for all employees when using company technology and data.
        \item Require all employees to read and acknowledge the policy.
    \end{itemize}
\end{enumerate}

\subsection{Long-Term Actions (Ongoing / Project-Based)}
\begin{enumerate}
    \item \textbf{Upgrade Outdated Windows Systems:}
    \begin{itemize}
        \item Continue and prioritize the project to upgrade all Windows 7 workstations to a supported operating system, such as Windows 10 or 11, to ensure they receive critical security patches.
    \end{itemize}
    \item \textbf{Implement a Vulnerability Management Program:}
    \begin{itemize}
        \item Establish a regular schedule for internal and external network scanning to proactively identify and remediate vulnerabilities before they can be exploited.
    \end{itemize}
\end{enumerate}

\end{document}
```