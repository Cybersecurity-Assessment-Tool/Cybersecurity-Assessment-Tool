```latex
\documentclass[12pt]{article}

% Preamble: Required Packages
\usepackage[margin=1in]{geometry}
\usepackage{pifont} % For checkmarks and crosses
\usepackage{booktabs} % For professional tables
\usepackage{hyperref} % For clickable links and metadata
\usepackage{url}
\usepackage{seqsplit} % To break long strings in \texttt
\usepackage{xcolor} % For colors
\usepackage{graphicx} % For potential logos

% Define colors for severity levels
\definecolor{sev_critical}{HTML}{990000}
\definecolor{sev_high}{HTML}{D14302}
\definecolor{sev_medium}{HTML}{E8B400}

% Custom commands for Yes/No symbols
\newcommand{\yes}{\ding{51}}
\newcommand{\no}{\ding{55}}

% Hyperref Setup
\hypersetup{
    colorlinks=true,
    linkcolor=blue,
    filecolor=magenta,
    urlcolor=cyan,
    pdftitle={Cybersecurity Posture Assessment Report},
    pdfauthor={Cybersecurity Analysis Division},
    pdfsubject={Security Assessment},
    pdfkeywords={Cybersecurity, Nmap, Risk, Assessment}
}

\begin{document}

% --- Title Page ---
\begin{titlepage}
    \centering
    \vspace*{1cm}
    \Huge\textbf{Cybersecurity Posture Assessment Report}
    \vspace{1.5cm}
    \Large\textbf{Prepared for:} \\
    \vspace{0.5cm}
    \huge Wildfire Communications
    \vfill
    \large\textbf{Date of Assessment:} November 22, 2025 \\
    \large\textbf{Report ID:} RPT-20251122-001
    \vspace{1.5cm}
    \small This document contains sensitive information and is intended for the exclusive use of the recipient.
\end{titlepage}

% --- Table of Contents ---
\tableofcontents
\newpage

% --- Executive Summary ---
\section*{Executive Summary}
This report details the findings of a cybersecurity posture assessment conducted on November 22, 2025, for Wildfire Communications. The assessment combined a review of organizational security controls, an external network scan, and an analysis of pre-existing risks.

The overall security posture reveals several critical and high-risk areas that require immediate attention. Key findings include:
\begin{itemize}
    \item \textbf{Critical Control Gaps:} The lack of mandatory Multi-Factor Authentication (MFA) for email access represents a critical vulnerability, exposing the organization to significant risks of business email compromise and account takeover.
    \item \textbf{Outdated Web Server Software:} The external-facing web server is running an outdated version of nginx (1.18.0), which is no longer supported and has multiple known vulnerabilities.
    \item \textbf{Deficient Security Policies and Training:} The organization lacks a formal employee acceptable use policy and a structured security awareness training program. This combination creates a high-risk environment where employees are more susceptible to social engineering attacks.
\end{itemize}

This report provides a detailed breakdown of these findings and offers actionable recommendations to mitigate the identified risks and strengthen the overall security posture of Wildfire Communications.

% --- Organizational Information ---
\section{Organizational Information}
The following information was provided for the assessment.
\begin{description}
    \item[Organization Name:] Wildfire Communications
    \item[Email Domain:] \texttt{WildfireCommunications.net}
    \item[Website Domain:] \url{www.WildfireCommunications.net}
    \item[External IP Address:] \texttt{125.178.188.186}
\end{description}

% --- Security Control Review ---
\section{Security Control Review}
A review of administrative and organizational security controls was conducted based on a standardized questionnaire. The responses highlight significant gaps in foundational security practices.

\begin{table}[h!]
\centering
\caption{Organizational Security Controls Questionnaire}
\begin{tabular}{p{8cm} c l}
\toprule
\textbf{Control Question} & \textbf{Response} & \textbf{Assessment} \\
\midrule
Do you require MFA to access email? & \no & \textcolor{sev_critical}{\textbf{Critical Gap}} \\
Do you require MFA to log into computers? & \yes & Satisfactory \\
Do you require MFA to access sensitive data systems? & \yes & Satisfactory \\
Does your organization have an employee acceptable use policy? & \no & \textcolor{sev_high}{High Risk} \\
Does your organization do security awareness training for new employees? & \no & \textcolor{sev_high}{High Risk} \\
Does your organization do security awareness training for all employees at least once per year? & \no & \textcolor{sev_high}{High Risk} \\
\bottomrule
\end{tabular}
\end{table}

\subsection*{Analysis of Control Gaps}
The absence of MFA on email is the most severe finding. Email is a primary target for attackers, and without MFA, a single compromised password can lead to a full account takeover. The lack of an acceptable use policy and security awareness training program compounds this risk by failing to establish a security-conscious culture, making employees more likely to fall for phishing attacks.

% --- Technical Scan Results ---
\section{Technical Scan Results}
An Nmap scan was performed on the target host \texttt{192.168.10.5} on November 22, 2025. The scan identified one open port with a service running outdated software.

\begin{table}[h!]
\centering
\caption{Open Port Scan Details for \texttt{192.168.10.5}}
\begin{tabular}{l l l l l}
\toprule
\textbf{Port} & \textbf{State} & \textbf{Service} & \textbf{Product} & \textbf{Version} \\
\midrule
443/tcp & open & https & nginx & 1.18.0 \\
\bottomrule
\end{tabular}
\end{table}

\subsection*{Analysis of Technical Findings}
The scan revealed that the web server is running \textbf{nginx version 1.18.0}. This version was released in April 2020 and is now considered outdated and unsupported. It is susceptible to multiple publicly disclosed vulnerabilities (e.g., CVE-2021-23017). Running outdated software on an internet-facing service presents a high risk of compromise, as attackers can exploit known flaws to gain unauthorized access to the server and potentially the internal network.

% --- Integrated Risk Assessment ---
\section{Integrated Risk Assessment}
The following table synthesizes the findings from the security control review and the technical scan. No pre-existing vulnerabilities were reported. The risks are prioritized by severity.

\begin{table}[h!]
\centering
\caption{Summary of Identified Risks}
\begin{tabular}{p{1.5cm} p{6cm} l p{4cm}}
\toprule
\textbf{Risk ID} & \textbf{Risk Description} & \textbf{Severity} & \textbf{Affected Asset(s)} \\
\midrule
RISK-001 & Lack of MFA on email allows for account takeover via compromised credentials. & \textcolor{sev_critical}{Critical} & Email System, User Accounts, Sensitive Data \\
\addlinespace
RISK-002 & Outdated nginx web server is vulnerable to remote exploitation. & \textcolor{sev_high}{High} & Public Website, Web Server, Underlying Infrastructure \\
\addlinespace
RISK-003 & Absence of security awareness training increases susceptibility to phishing and social engineering. & \textcolor{sev_high}{High} & All Employees, All Systems \\
\addlinespace
RISK-004 & Lack of a formal Acceptable Use Policy leads to inconsistent security practices. & \textcolor{sev_medium}{Medium} & Organizational Governance, Employee Conduct \\
\bottomrule
\end{tabular}
\end{table}

% --- Recommendations ---
\section{Recommendations}
The following actions are recommended to mitigate the identified risks and improve the security posture of Wildfire Communications.

\subsection*{RISK-001: Remediate Lack of MFA on Email (Critical)}
\begin{enumerate}
    \item \textbf{Immediate Action:} Enable and enforce MFA for all user accounts on the email system within the next 7 days.
    \item \textbf{Policy:} Update IT policy to mandate MFA for all externally accessible services that handle company data.
    \item \textbf{Audit:} Conduct a review of all email accounts for signs of unauthorized access or suspicious forwarding rules.
\end{enumerate}

\subsection*{RISK-002: Upgrade Outdated Nginx Server (High)}
\begin{enumerate}
    \item \textbf{Immediate Action:} Develop a patch management plan to upgrade the nginx server from version 1.18.0 to the latest stable version.
    \item \textbf{Implementation:} Schedule a maintenance window to perform the upgrade. Before deployment, test the new version in a staging environment to ensure compatibility with the web application.
    \item \textbf{Ongoing:} Implement a formal vulnerability management program that includes regular scanning and a defined patching cadence for all internet-facing systems.
\end{enumerate}

\subsection*{RISK-003: Establish Security Awareness Training (High)}
\begin{enumerate}
    \item \textbf{Immediate Action:} Procure and implement a security awareness training solution for all employees.
    \item \textbf{Onboarding:} Integrate mandatory security training into the new employee onboarding process.
    \item \textbf{Annual Training:} Schedule and require all employees to complete a security refresher course at least once per year. Conduct periodic phishing simulations to measure effectiveness.
\end{enumerate}

\subsection*{RISK-004: Develop an Acceptable Use Policy (Medium)}
\begin{enumerate}
    \item \textbf{Immediate Action:} Draft a comprehensive Acceptable Use Policy (AUP) that clearly defines the rules for using company IT assets, data, and internet access.
    \item \textbf{Distribution:} Require all current employees to read and acknowledge the policy. Include this as a mandatory step in the new employee onboarding process.
    \item \textbf{Review:} Schedule an annual review of the AUP to ensure it remains current with evolving technologies and threats.
\end{enumerate}

\end{document}
```