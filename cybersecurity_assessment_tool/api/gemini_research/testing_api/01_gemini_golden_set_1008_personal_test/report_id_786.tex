```latex
\documentclass[12pt]{article}

% Preamble: Required Packages
\usepackage[margin=1in]{geometry}
\usepackage{pifont} % For checkmarks and crosses
\usepackage{booktabs} % For professional tables
\usepackage{hyperref} % For clickable links
\usepackage{url} % For URL formatting
\usepackage{seqsplit} % For splitting long strings in texttt
\usepackage{graphicx} % For potential logos
\usepackage{xcolor} % For colors in text

% Document Metadata
\title{Cybersecurity Posture Assessment Report}
\author{Cybersecurity Analysis Division}
\date{\today}

\begin{document}

\maketitle
\thispagestyle{empty}
\newpage

\tableofcontents
\newpage

% --- Section 1: Executive Summary ---
\section{Executive Summary}

This report provides a cybersecurity posture assessment for \textbf{Radiant Life}, synthesizing data from a network vulnerability scan, an organizational security questionnaire, and a review of pre-existing risk documentation.

The assessment reveals a mixed security posture. A positive finding from the recent network scan of target \texttt{192.168.0.5} indicates that a previously identified risk, an "Unencrypted Web Server" on port 80, appears to have been remediated, as the port is now closed. This demonstrates progress in technical vulnerability management.

However, the security control review has identified several \textbf{critical gaps} in foundational security practices. The lack of mandatory Multi-Factor Authentication (MFA) for email and sensitive data systems exposes the organization to significant risks, including business email compromise, unauthorized data access, and credential theft. Furthermore, the absence of a structured security awareness training program for new and existing employees leaves the organization highly susceptible to social engineering and phishing attacks.

Immediate and decisive action is required to address these policy and procedural deficiencies to build a more resilient security foundation.

% --- Section 2: Organizational Information ---
\section{Organizational Information}

The following details were provided for the assessment.

\begin{table}[h!]
\centering
\begin{tabular}{@{}ll@{}}
\toprule
\textbf{Attribute} & \textbf{Value} \\ \midrule
Organization Name & Radiant Life \\
Email Domain & \texttt{RadiantLife.org} \\
Website Domain & \url{www.RadiantLife.org} \\
External IP Address & \texttt{74.195.200.110} \\ \bottomrule
\end{tabular}
\caption{Client Organizational Details.}
\label{tab:org_info}
\end{table}

% --- Section 3: Security Control Review ---
\section{Security Control Review}

A review of the organization's security controls was conducted via a questionnaire. The responses highlight critical areas requiring immediate attention. A "No" response indicates a significant gap in security best practices.

\begin{table}[h!]
\centering
\begin{tabular}{@{}p{0.6\textwidth}ccp{0.15\textwidth}@{}}
\toprule
\textbf{Security Control Question} & \textbf{Response} & \textbf{Status} & \textbf{Assessment} \\ \midrule
Do you require MFA to access email? & No & \ding{55} & \textcolor{red}{\textbf{Critical Gap}} \\
Do you require MFA to log into computers? & Yes & \ding{51} & Control in Place \\
Do you require MFA to access sensitive data systems? & No & \ding{55} & \textcolor{red}{\textbf{Critical Gap}} \\
Does your organization have an employee acceptable use policy? & Yes & \ding{51} & Control in Place \\
Does your organization do security awareness training for new employees? & No & \ding{55} & \textcolor{orange}{\textbf{High Risk}} \\
Does your organization do security awareness training for all employees at least once per year? & No & \ding{55} & \textcolor{orange}{\textbf{High Risk}} \\ \bottomrule
\end{tabular}
\caption{Security Controls Questionnaire Analysis.}
\label{tab:controls}
\end{table}

% --- Section 4: Technical Scan Results ---
\section{Technical Scan Results}

A network scan was performed on the specified target to identify open ports and exposed services.

\begin{itemize}
    \item \textbf{Scan Target:} \texttt{192.168.0.5}
    \item \textbf{Scan Tool:} Nmap
    \item \textbf{Scan Status:} Host is up.
\end{itemize}

\subsection{Port and Service Analysis}
The scan results were minimal, indicating a tightly controlled host or a very specific scan scope.

\begin{table}[h!]
\centering
\begin{tabular}{@{}llll@{}}
\toprule
\textbf{Port} & \textbf{State} & \textbf{Service} & \textbf{Finding} \\ \midrule
80/tcp & closed & http & No exposure detected. \\ \bottomrule
\end{tabular}
\caption{Scan results for target \texttt{192.168.0.5}.}
\label{tab:scan_results}
\end{table}

\subsection{Correlation with Existing Risks}
The pre-existing risk register (Input 3) listed a vulnerability named "Unencrypted Web Server" related to an open port 80. This technical scan shows that port 80 is now \textbf{closed} on the target system. This suggests that the previously identified risk has been successfully remediated. This is a positive security development.

% --- Section 5: Overall Risk Assessment ---
\section{Overall Risk Assessment}

The following table summarizes the most significant risks identified during this assessment, combining findings from the security control review and technical scans.

\begin{table}[h!]
\centering
\begin{tabular}{@{}p{0.3\textwidth}p{0.5\textwidth}l@{}}
\toprule
\textbf{Risk Name} & \textbf{Overview} & \textbf{Severity} \\ \midrule
\textbf{Lack of MFA on Critical Systems} & Email and sensitive data systems are secured with single-factor (password-only) authentication. This creates a high risk of account takeover and data breach via credential stuffing or phishing. & \textcolor{red}{\textbf{Critical}} \\
\textbf{Insufficient Security Awareness Training} & The absence of a formal training program leaves employees unable to recognize and respond to common cyber threats like phishing, significantly increasing the likelihood of a security incident caused by human error. & \textcolor{orange}{\textbf{High}} \\
\textbf{Unencrypted Web Server (Remediated)} & A previously documented risk indicated that port 80 was open. The recent scan confirms this port is now closed, and the risk is considered remediated. & \textcolor{green}{Remediated} \\
\bottomrule
\end{tabular}
\caption{Summary of Identified Risks.}
\label{tab:risk_summary}
\end{table}

% --- Section 6: Recommendations ---
\section{Recommendations}

Based on the analysis, we provide the following prioritized recommendations to strengthen the cybersecurity posture of \textbf{Radiant Life}.

\subsection{Immediate Priority (0-30 Days)}
\begin{enumerate}
    \item \textbf{Enforce MFA on Email:} Immediately enable and enforce MFA for all user accounts accessing the \texttt{RadiantLife.org} email system. This is the single most effective control to prevent business email compromise.
    \item \textbf{Enforce MFA on Sensitive Data Systems:} Identify all systems containing sensitive or critical data and immediately enforce MFA for all user access.
\end{enumerate}

\subsection{High Priority (30-90 Days)}
\begin{enumerate}
    \item \textbf{Implement Security Awareness Training Program:}
    \begin{itemize}
        \item Develop or procure a security awareness training module for all new employees as part of their onboarding process.
        \item Schedule and conduct mandatory annual security awareness training for all staff, covering topics such as phishing, password hygiene, and acceptable use.
    \end{itemize}
\end{enumerate}

\subsection{Verification and Documentation}
\begin{enumerate}
    \item \textbf{Update Risk Register:} Formally document the remediation of the "Unencrypted Web Server" vulnerability in the internal risk register and close the associated finding.
\end{enumerate}

\end{document}
```