```latex
\documentclass[12pt]{article}

% Preamble: Required Packages
\usepackage[margin=1in]{geometry}
\usepackage{pifont} % For checkmarks and crosses
\usepackage{booktabs} % For professional tables
\usepackage{hyperref} % For clickable links
\usepackage{url} % For URL formatting
\usepackage{seqsplit} % For splitting long strings in tt font
\usepackage{graphicx} % For potential logos
\usepackage{xcolor} % For colors in text

% Document Metadata
\title{Cybersecurity Posture Assessment Report}
\author{Cybersecurity Analyst}
\date{\today}

% Hyperref Setup
\hypersetup{
    colorlinks=true,
    linkcolor=blue,
    filecolor=magenta,      
    urlcolor=cyan,
    pdftitle={Cybersecurity Posture Assessment Report},
    pdfpagemode=FullScreen,
}

\begin{document}

\maketitle
\hrule
\vspace{1em}

% --- 1. Executive Overview ---
\section*{Executive Overview}
This report provides a cybersecurity posture assessment for \textbf{Green Sprout Organic}, conducted on \today. The analysis synthesizes data from a network vulnerability scan, an organizational security questionnaire, and a review of pre-existing risks.

The assessment reveals a mixed security posture. The organization has implemented foundational access controls, such as Multi-Factor Authentication (MFA) for email and computer access. Furthermore, the external network scan of the target IP address \texttt{[Target IP]} did not identify any open ports, suggesting a strong firewall configuration at the network perimeter.

However, several critical and high-risk gaps were identified in procedural and policy-based controls. The most significant finding is the lack of MFA for accessing sensitive data systems, which exposes critical assets to unauthorized access. Additionally, the absence of a formal security awareness training program and an acceptable use policy leaves the organization vulnerable to human error, phishing, and insider threats.

Immediate remediation should focus on closing these identified gaps to strengthen the organization's defense-in-depth strategy and reduce its overall risk profile.

% --- 2. Organizational Information ---
\section*{Organizational Information}
The following details were provided for the assessment.

\begin{table}[h!]
\centering
\begin{tabular}{@{}ll@{}}
\toprule
\textbf{Attribute} & \textbf{Value} \\ \midrule
Organization Name & \textbf{Green Sprout Organic} \\
Email Domain & \texttt{GreenSproutOrganic.com} \\
Website Domain & \url{www.GreenSproutOrganic.com} \\
External IP Address & \texttt{22.137.14.37} \\ \bottomrule
\end{tabular}
\caption{Client Organizational Data}
\end{table}

% --- 3. Security Control Review (Questionnaire Analysis) ---
\section*{Security Control Review}
A review of the organization's security controls was conducted via a questionnaire. The responses indicate key areas of strength and weakness. "No" answers represent significant gaps in the security framework.

\begin{table}[h!]
\centering
\begin{tabular}{@{}p{0.6\linewidth} c l@{}}
\toprule
\textbf{Control Question} & \textbf{Response} & \textbf{Assessment} \\ \midrule
Do you require MFA to access email? & \ding{51} Yes & Good Practice \\
Do you require MFA to log into computers? & \ding{51} Yes & Good Practice \\
Do you require MFA to access sensitive data systems? & \textcolor{red}{\ding{55} No} & \textcolor{red}{Critical Gap} \\
Does your organization have an employee acceptable use policy? & \textcolor{orange}{\ding{55} No} & \textcolor{orange}{High Risk} \\
Does your organization do security awareness training for new employees? & \textcolor{orange}{\ding{55} No} & \textcolor{orange}{High Risk} \\
Does your organization do security awareness training for all employees at least once per year? & \textcolor{orange}{\ding{55} No} & \textcolor{orange}{High Risk} \\ \bottomrule
\end{tabular}
\caption{Security Controls Questionnaire Analysis}
\end{table}

% --- 4. Technical Scan Results ---
\section*{Technical Scan Results}
An external network scan was performed on the target IP address to identify open ports and exposed services.

\begin{itemize}
    \item \textbf{Target IP Address:} \texttt{[Target IP]}
    \item \textbf{Scan Date:} Not available in scan data.
\end{itemize}

\subsection*{Findings}
\textbf{No open ports or services were detected on the target system.}

This is a positive security finding, suggesting a well-configured firewall that enforces a default-deny policy for inbound traffic. It effectively reduces the external attack surface of the scanned asset. It is recommended to confirm that the scan targeted the correct public-facing IP address for critical services.

% --- 5. Consolidated Risk Assessment ---
\section*{Consolidated Risk Assessment}
The following table summarizes the key risks identified during this assessment, combining findings from the security control review and technical scan. No pre-existing vulnerabilities were provided for this analysis.

\begin{table}[h!]
\centering
\begin{tabular}{@{}p{0.25\linewidth} p{0.5\linewidth} l@{}}
\toprule
\textbf{Risk Name} & \textbf{Description} & \textbf{Severity} \\ \midrule
\textbf{Lack of MFA for Sensitive Systems} & The absence of MFA on systems containing sensitive data significantly increases the risk of a data breach from compromised credentials. & \textcolor{red}{Critical} \\
\textbf{No Security Awareness Training Program} & Without regular training, employees are more susceptible to social engineering and phishing attacks, making them a weak link in the organization's defenses. & \textcolor{orange}{High} \\
\textbf{Absence of Acceptable Use Policy (AUP)} & The lack of a formal AUP creates ambiguity regarding the proper use of company assets and data, increasing the risk of insider threat and non-compliance. & \textcolor{orange}{High} \\
\bottomrule
\end{tabular}
\caption{Summary of Identified Risks}
\end{table}

% --- 6. Recommendations ---
\section*{Recommendations}
Based on the findings, the following prioritized actions are recommended to mitigate the identified risks and improve the overall security posture of \textbf{Green Sprout Organic}.

\begin{enumerate}
    \item \textbf{[Critical] Implement MFA for Sensitive Systems:} Immediately enforce MFA across all applications, databases, and systems that store or process sensitive company or customer data. This is the single most effective control to prevent unauthorized access.

    \item \textbf{[High] Develop a Security Awareness Training Program:} Establish a mandatory security awareness training program for all employees. The program should be conducted upon hiring and at least annually thereafter. Topics should include phishing awareness, password hygiene, and data handling policies.

    \item \textbf{[High] Create and Enforce an Acceptable Use Policy (AUP):} Draft a formal AUP that clearly defines the rules and responsibilities for employees when using company IT assets. This policy should be distributed to all employees, and acknowledgement of receipt should be tracked.

    \item \textbf{[Informational] Validate Scan Scope:} Confirm that the network scan target (\texttt{[Target IP]}) covers the full scope of the organization's public-facing infrastructure. Regular, comprehensive vulnerability scanning of all external assets is recommended.
\end{enumerate}

\end{document}
```