```latex
\documentclass[12pt]{article}

% === PACKAGES ===
\usepackage[margin=1in]{geometry}
\usepackage{pifont} % For checkmarks and crosses
\usepackage{booktabs} % For professional tables
\usepackage{hyperref} % For clickable links and ToC
\usepackage{url} % For URL formatting
\usepackage{seqsplit} % For splitting long text strings
\usepackage{graphicx}
\usepackage{xcolor}
\usepackage{tocbibind} % To add ToC to the ToC

% === DOCUMENT SETUP ===
\hypersetup{
    colorlinks=true,
    linkcolor=blue,
    filecolor=magenta,      
    urlcolor=cyan,
    pdftitle={Cybersecurity Posture Assessment Report},
    pdfpagemode=FullScreen,
}

% Define severity colors
\definecolor{sev_critical}{HTML}{990000}
\definecolor{sev_high}{HTML}{DD0000}
\definecolor{sev_medium}{HTML}{FFA500}
\definecolor{sev_low}{HTML}{32CD32}

% === DOCUMENT START ===
\begin{document}

% === TITLE PAGE ===
\begin{titlepage}
    \centering
    \vspace*{1cm}
    \Huge
    \textbf{Cybersecurity Posture Assessment Report}
    \vspace{1.5cm}
    \Large
    Prepared for: \textbf{Symmetry Architecture}\\
    \vspace{1cm}
    \normalsize
    Date of Report: \today\\
    \vfill
    \large
    \textit{This report contains sensitive information and should be handled with care.}
    \vspace{1cm}
    \hrule
    \vspace{0.5cm}
    \normalsize
    Analysis by: Cybersecurity Analysis Division
\end{titlepage}

\tableofcontents
\newpage

% === EXECUTIVE SUMMARY ===
\section{Executive Summary}
This report provides a comprehensive cybersecurity posture assessment for \textbf{Symmetry Architecture}, based on network scan data, organizational security questionnaires, and a review of pre-existing risks. The analysis correlates technical findings with administrative controls to provide a holistic view of the organization's security landscape.

Overall, \textbf{Symmetry Architecture} demonstrates a strong commitment to identity and access management, with multi-factor authentication (MFA) widely implemented across critical systems. However, this strength is contrasted by significant weaknesses in foundational security policies and a critical network misconfiguration.

Key findings include:
\begin{itemize}
    \item \textbf{Critical Risk:} A network service (SSH on port 22) was found exposed on the localhost interface (\seqsplit{\texttt{127.0.0.1}}), which appears to be incorrectly routed and accessible externally. This issue corresponds to a pre-existing risk, "Localhost Exposed," with a CVSS score of 10.0 (Critical). This requires immediate investigation and remediation.
    \item \textbf{High-Risk Gaps:} The organization lacks a formal Acceptable Use Policy (AUP) and does not provide security awareness training to new employees during onboarding. These administrative gaps expose the organization to significant insider threats and compliance risks.
\end{itemize}

Recommendations in this report are prioritized to address the most severe risks first, focusing on immediate remediation of the network exposure, followed by the development of essential security policies and training programs.

% === ORGANIZATIONAL INFORMATION ===
\section{Organizational Information}
The following details were provided for the assessment.
\begin{center}
\begin{tabular}{ll}
\toprule
\textbf{Attribute} & \textbf{Value} \\
\midrule
Organization Name & \textbf{Symmetry Architecture} \\
Email Domain & \seqsplit{\texttt{SymmetryArchitecture.com}} \\
Website Domain & \seqsplit{\texttt{www.SymmetryArchitecture.com}} \\
External IP Address & \seqsplit{\texttt{81.234.120.115}} \\
\bottomrule
\end{tabular}
\end{center}

% === SECURITY CONTROL REVIEW ===
\section{Security Control Review}
The following table summarizes the organization's responses to a security controls questionnaire. The assessment column highlights areas of concern where responses deviate from established best practices.
\begin{center}
\begin{tabular}{p{0.6\textwidth} c l}
\toprule
\textbf{Control Question} & \textbf{Response} & \textbf{Assessment} \\
\midrule
Do you require MFA to access email? & \ding{51} Yes & Strong Control \\
Do you require MFA to log into computers? & \ding{51} Yes & Strong Control \\
Do you require MFA to access sensitive data systems? & \ding{51} Yes & Strong Control \\
\addlinespace
Does your organization have an employee acceptable use policy? & \textbf{\color{red}\ding{55} No} & \textbf{Critical Policy Gap} \\
\addlinespace
Does your organization do security awareness training for new employees? & \textbf{\color{red}\ding{55} No} & \textbf{High Risk} \\
\addlinespace
Does your organization do security awareness training for all employees at least once per year? & \ding{51} Yes & Good Practice \\
\bottomrule
\end{tabular}
\end{center}

\subsection{Analysis of Controls}
The consistent implementation of MFA is commendable and significantly reduces the risk of account compromise. However, the absence of an Acceptable Use Policy and security training for new hires represents a major weakness in the administrative security posture. New employees are often a primary target for social engineering, and without formal guidance and training, the risk of a security incident is elevated.

% === TECHNICAL SCAN RESULTS ===
\section{Technical Scan Results}
A network scan was conducted to identify open ports and services on the target system.
\begin{itemize}
    \item \textbf{Scan Target:} \seqsplit{\texttt{127.0.0.1}}
    \item \textbf{Scan Date:} \today
\end{itemize}

\begin{center}
\begin{tabular}{l l l l}
\toprule
\textbf{Port} & \textbf{State} & \textbf{Service} & \textbf{Product / Version} \\
\midrule
22/tcp & open & ssh & (Not Fingerprinted) \\
\bottomrule
\end{tabular}
\end{center}

\subsection{Analysis of Scan Findings}
The scan identified that port 22, commonly used for the Secure Shell (SSH) protocol, is open. SSH is used for secure remote administration.

The most alarming finding is the target IP address itself: \seqsplit{\texttt{127.0.0.1}}. This is the universal loopback address (localhost), which should not be reachable from an external network scan. This discovery strongly suggests a severe network misconfiguration (e.g., a NAT or proxy issue) that is exposing an internal-only service to the public internet. This finding directly corroborates the pre-existing risk detailed in the next section and elevates its urgency. The lack of version information from the scan prevents an analysis for specific software vulnerabilities, highlighting the need for more in-depth, authenticated scanning.

% === CONSOLIDATED RISK ASSESSMENT ===
\section{Consolidated Risk Assessment}
The following table synthesizes findings from the security questionnaire, technical scan, and pre-existing risk data into a prioritized list.

\begin{center}
\begin{tabular}{p{0.3\textwidth} p{0.5\textwidth} p{0.15\textwidth}}
\toprule
\textbf{Risk Name} & \textbf{Description} & \textbf{Severity} \\
\midrule
\textbf{Localhost Exposed} & Port 22 (SSH) is open on the loopback interface (\seqsplit{\texttt{127.0.0.1}}) and is accessible from an external source. This confirms a critical network misconfiguration. & \textbf{\color{sev_critical}Critical (10.0)} \\
\addlinespace
\textbf{Lack of Acceptable Use Policy (AUP)} & The absence of a formal AUP creates ambiguity for employees regarding secure behavior and exposes the organization to insider threats and legal risks. & \textbf{\color{sev_high}High} \\
\addlinespace
\textbf{No Security Training for New Hires} & New employees are not receiving mandatory security awareness training upon joining, leaving a critical window of vulnerability before the annual training cycle. & \textbf{\color{sev_high}High} \\
\bottomrule
\end{tabular}
\end{center}

% === RECOMMENDATIONS ===
\section{Recommendations}
Based on the consolidated risk assessment, the following actions are recommended to improve the cybersecurity posture of \textbf{Symmetry Architecture}.

\subsection{Immediate Priority (Remediate within 72 hours)}
\begin{enumerate}
    \item \textbf{Remediate "Localhost Exposed" Risk:}
    \begin{itemize}
        \item Immediately investigate the network path that allows external scans to reach the \seqsplit{\texttt{127.0.0.1}} interface.
        \item Apply firewall rules to block all external access to this service. The principle of least privilege dictates that administrative services like SSH should only be accessible from trusted internal IP addresses.
        \item Conduct a full audit of network routing and NAT configurations to identify and correct the root cause.
    \end{itemize}
\end{enumerate}

\subsection{High Priority (Remediate within 30 days)}
\begin{enumerate}
    \setcounter{enumi}{1}
    \item \textbf{Develop and Implement an Acceptable Use Policy (AUP):}
    \begin{itemize}
        \item Draft a formal AUP that clearly defines rules for employee use of company assets, including data handling, internet usage, email, and software installation.
        \item Require all current and new employees to read and formally acknowledge the policy.
    \end{itemize}
    \item \textbf{Institute Onboarding Security Training:}
    \begin{itemize}
        \item Develop or procure a mandatory security awareness training module for all new employees.
        \item This training should be completed within the first week of employment and cover topics such as phishing, password security, MFA, and the new AUP.
    \end{itemize}
\end{enumerate}

\subsection{Medium Priority (Remediate within 90 days)}
\begin{enumerate}
    \setcounter{enumi}{3}
    \item \textbf{Enhance Vulnerability Scanning:}
    \begin{itemize}
        \item Implement a regular, authenticated vulnerability scanning program. Authenticated scans provide much deeper insight than unauthenticated port scans, allowing for the identification of outdated software versions, missing patches, and detailed configuration weaknesses.
    \end{itemize}
\end{enumerate}

\end{document}
```