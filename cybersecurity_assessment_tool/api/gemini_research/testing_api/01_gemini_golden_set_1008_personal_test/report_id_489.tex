```latex
\documentclass[12pt]{article}

% --- PACKAGES ---
\usepackage[margin=1in]{geometry}
\usepackage{pifont} % For checkmarks and crosses
\usepackage{booktabs} % For professional tables
\usepackage{hyperref} % For clickable links
\usepackage{url} % For URL formatting
\usepackage{seqsplit} % For splitting long text sequences
\usepackage{xcolor} % For colors

% --- DOCUMENT METADATA ---
\title{Cybersecurity Posture Assessment Report}
\author{Cybersecurity Analysis Division}
\date{\today}

% --- HYPERREF SETUP ---
\hypersetup{
    colorlinks=true,
    linkcolor=blue,
    filecolor=magenta,      
    urlcolor=cyan,
    pdftitle={Cybersecurity Posture Assessment Report},
    pdfpagemode=FullScreen,
}

\begin{document}

\maketitle

% ===================================================================
% SECTION 1: EXECUTIVE SUMMARY
% ===================================================================
\section{Executive Summary}

This report provides a comprehensive cybersecurity assessment for \textbf{Aventine Research}, based on network scan data, organizational questionnaires, and a review of existing risk documentation.

The assessment reveals a mixed security posture. The organization has successfully implemented multi-factor authentication (MFA) across critical systems, including email, computer logins, and sensitive data access. This is a commendable and significant security control that greatly reduces the risk of unauthorized access.

However, two critical areas of concern have been identified that require immediate attention:

\begin{enumerate}
    \item \textbf{Exposed Sensitive Service \& Outdated Risk Data:} A network scan identified an open service on port 8080 with the HTTP title ``TOP SECRET DB'' on the internal host \texttt{10.5.5.5}. This finding directly contradicts the current risk register, which incorrectly lists this port as a secure false positive. This indicates both a severe information disclosure vulnerability and a potential failure in the risk management lifecycle.
    
    \item \textbf{Absence of Security Awareness Training:} The organization does not conduct security awareness training for new or existing employees. This is a critical gap that leaves the organization highly vulnerable to social engineering, phishing attacks, and other human-centric threats.
\end{enumerate}

While foundational controls like MFA are in place, the combination of an exposed sensitive service and an untrained workforce creates a high-risk environment. Immediate remediation of these findings is strongly recommended to protect sensitive organizational data.

% ===================================================================
% SECTION 2: ORGANIZATIONAL INFORMATION
% ===================================================================
\section{Organizational Information}

The following information was provided for the assessment.

\begin{tabular}{@{}ll}
\toprule
\textbf{Attribute} & \textbf{Value} \\
\midrule
Organization Name & \textbf{Aventine Research} \\
Email Domain & \texttt{AventineResearch.org} \\
Website Domain & \seqsplit{\url{www.AventineResearch.org}} \\
External IP Address & \texttt{7.159.150.172} \\
\bottomrule
\end{tabular}

% ===================================================================
% SECTION 3: SECURITY CONTROL REVIEW
% ===================================================================
\section{Security Control Review}

The following table summarizes the organization's responses to a security controls questionnaire. A green checkmark (\ding{51}) indicates a positive control is in place, while a red cross (\ding{55}) indicates a control gap.

\begin{table}[h!]
\centering
\begin{tabular}{@{}lc@{}}
\toprule
\textbf{Control Question} & \textbf{Response} \\
\midrule
Do you require MFA to access email? & \textcolor{green}{\ding{51}} \\
Do you require MFA to log into computers? & \textcolor{green}{\ding{51}} \\
Do you require MFA to access sensitive data systems? & \textcolor{green}{\ding{51}} \\
Does your organization have an employee acceptable use policy? & \textcolor{green}{\ding{51}} \\
\midrule
\textit{Identified Gaps} & \\
\midrule
Does your organization do security awareness training for new employees? & \textcolor{red}{\ding{55}} \\
Does your organization do security awareness training for all employees annually? & \textcolor{red}{\ding{55}} \\
\bottomrule
\end{tabular}
\caption{Security Controls Questionnaire Results}
\end{table}

\subsection*{Analysis}
The consistent implementation of MFA is a significant strength. However, the complete lack of a security awareness training program is a critical weakness. This gap undermines the effectiveness of technical controls, as employees are not equipped to recognize or respond to threats like phishing, which are primary vectors for bypassing security measures.

% ===================================================================
% SECTION 4: TECHNICAL SCAN RESULTS
% ===================================================================
\section{Technical Scan Results}

A network scan was performed on the internal network to identify open ports and services.

\begin{itemize}
    \item \textbf{Target IP Address:} \texttt{10.5.5.5}
\end{itemize}

\begin{table}[h!]
\centering
\begin{tabular}{@{}llll@{}}
\toprule
\textbf{Port} & \textbf{State} & \textbf{Service Info} \\
\midrule
8080 & Open & HTTP Title: \textbf{TOP SECRET DB} \\
\bottomrule
\end{tabular}
\caption{Open Ports Detected on \texttt{10.5.5.5}}
\end{table}

\subsection*{Analysis}
The scan identified a single open port, \texttt{8080}, on the target host. The service running on this port presents a web page with the title ``TOP SECRET DB''. This constitutes a critical information disclosure vulnerability. The title explicitly suggests the nature of the underlying system, making it a high-value target for attackers. Furthermore, this finding directly conflicts with the existing risk documentation (\textit{Input\_3\_Current\_Risks\_JSON}), which states that this port is a secure false positive. This discrepancy points to a serious flaw in the organization's risk assessment and management process.

% ===================================================================
% SECTION 5: CORRELATED RISK ASSESSMENT
% ===================================================================
\section{Correlated Risk Assessment}

The following table synthesizes findings from the security questionnaire, technical scans, and existing risk data to provide a holistic view of the primary risks facing the organization.

\begin{table}[h!]
\centering
\begin{tabular}{@{}p{0.1\linewidth} p{0.25\linewidth} p{0.4\linewidth} p{0.15\linewidth}@{}}
\toprule
\textbf{Risk ID} & \textbf{Risk Name} & \textbf{Description} & \textbf{Severity} \\
\midrule
\textbf{R-01} & Exposed Sensitive Service \& Outdated Risk Data & A service on port 8080 is titled "TOP SECRET DB", indicating a potential database exposure. This contradicts the risk register, which lists it as a false positive, revealing a failure in the risk management process. & \textbf{Critical} \\
\addlinespace
\textbf{R-02} & Lack of Security Awareness Training & The absence of any security training for employees significantly increases susceptibility to phishing, social engineering, and insider threats, undermining existing technical controls. & \textbf{High} \\
\bottomrule
\end{tabular}
\caption{Summary of Identified Risks}
\end{table}

% ===================================================================
% SECTION 6: RECOMMENDATIONS
% ===================================================================
\section{Recommendations}

The following actions are recommended to mitigate the identified risks.

\subsection*{R-01: Exposed Sensitive Service \& Outdated Risk Data}
\begin{itemize}
    \item \textbf{Immediate Action:} Immediately investigate the service on \texttt{10.5.5.5:8080}. Determine the nature of the data and restrict all access via firewall rules or by moving the system to an isolated network segment. Remove or change the revealing HTTP title.
    \item \textbf{Short-Term Fix:} Conduct a full review of the existing risk register. The discrepancy found on port 8080 suggests other entries may be inaccurate. Establish a formal process for validating and updating risk assessments on a regular basis (e.g., quarterly) or after any significant network change.
\end{itemize}

\subsection*{R-02: Lack of Security Awareness Training}
\begin{itemize}
    \item \textbf{Immediate Action:} Procure and enroll all current employees in a foundational security awareness training program. This should cover key topics such as phishing, password hygiene, and acceptable use policies.
    \item \textbf{Long-Term Fix:} Integrate mandatory security awareness training into the onboarding process for all new hires. Implement an annual refresher training program for all staff and consider periodic phishing simulations to test and reinforce learning.
\end{itemize}

\end{document}
```