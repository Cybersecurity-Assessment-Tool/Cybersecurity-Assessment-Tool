```latex
\documentclass[12pt]{article}

% Preamble: Required Packages
\usepackage[margin=1in]{geometry}
\usepackage{pifont} % For checkmarks and crosses
\usepackage{booktabs} % For professional tables
\usepackage{hyperref} % For clickable links
\usepackage{url} % For URL formatting
\usepackage{seqsplit} % For splitting long strings
\usepackage{graphicx}
\usepackage{xcolor}

% Document Information
\title{Cybersecurity Posture Assessment Report}
\author{Cybersecurity Analysis Division}
\date{November 22, 2025}

% Hyperref Setup
\hypersetup{
    colorlinks=true,
    linkcolor=blue,
    filecolor=magenta,      
    urlcolor=cyan,
    pdftitle={Cybersecurity Posture Assessment Report},
    pdfpagemode=FullScreen,
}

\begin{document}

\maketitle
\thispagestyle{empty}
\newpage

\tableofcontents
\newpage

% --- 1. Executive Summary ---
\section{Executive Summary}

This report provides a comprehensive cybersecurity assessment for \textbf{Pioneer Pulse}, conducted on November 22, 2025. The analysis correlates data from organizational security questionnaires, external network scans, and a review of pre-existing risks.

The assessment identified several critical and high-risk areas requiring immediate attention. Key findings include significant gaps in foundational security controls, specifically the absence of Multi-Factor Authentication (MFA) for computer logins and a lack of a formal security awareness training program. These procedural weaknesses substantially increase the risk of credential compromise and unauthorized access.

Furthermore, technical scanning of the asset at \texttt{192.168.10.5} revealed an outdated version of the nginx web server (\texttt{1.18.0}). This software version is several years old and is known to have published vulnerabilities, exposing the organization to potential exploitation.

The overall security posture is considered weak due to these fundamental control gaps. This report outlines specific, actionable recommendations to mitigate the identified risks and strengthen the organization's defenses against common cyber threats.

% --- 2. Organizational Information ---
\section{Organizational Information}

The following details were provided for the assessment.

\begin{tabular}{@{}ll}
\toprule
\textbf{Attribute} & \textbf{Value} \\
\midrule
Organization Name & \textbf{Pioneer Pulse} \\
Email Domain & \texttt{PioneerPulse.com} \\
Website Domain & \url{www.PioneerPulse.com} \\
External IP Address & \texttt{167.191.126.202} \\
\bottomrule
\end{tabular}

% --- 3. Security Control Review ---
\section{Security Control Review}

A review of the organization's security controls was conducted via a standardized questionnaire. The results highlight critical gaps in access control and employee security training. A "No" answer indicates a deviation from security best practices and represents a significant risk.

\begin{table}[h!]
\centering
\caption{Security Control Questionnaire Results}
\begin{tabular}{@{}p{0.6\linewidth}cc}
\toprule
\textbf{Control Question} & \textbf{Response} & \textbf{Status} \\
\midrule
Do you require MFA to access email? & Yes & \ding{51} \\
Do you require MFA to log into computers? & No & \textcolor{red}{\ding{55}} \\
Do you require MFA to access sensitive data systems? & Yes & \ding{51} \\
Does your organization have an employee acceptable use policy? & Yes & \ding{51} \\
Does your organization do security awareness training for new employees? & No & \textcolor{red}{\ding{55}} \\
Does your organization do security awareness training for all employees at least once per year? & No & \textcolor{red}{\ding{55}} \\
\bottomrule
\end{tabular}
\end{table}

% --- 4. Technical Scan Results ---
\section{Technical Scan Results}

An Nmap scan was performed on the specified target to identify open ports and running services.

\begin{itemize}
    \item \textbf{Target IP:} \texttt{192.168.10.5}
    \item \textbf{Scan Date:} 2025-11-22T10:00:00Z
\end{itemize}

The scan revealed one open port, which is detailed below.

\begin{table}[h!]
\centering
\caption{Open Ports and Services on \texttt{192.168.10.5}}
\begin{tabular}{@{}lllll@{}}
\toprule
\textbf{Port} & \textbf{State} & \textbf{Service} & \textbf{Product} & \textbf{Version} \\
\midrule
443/tcp & open & https & nginx & \textbf{1.18.0} \\
\bottomrule
\end{tabular}
\end{table}

\subsection*{Analysis of Technical Findings}
The web server is running \textbf{nginx version 1.18.0}, which was released in April 2020. This version is significantly outdated and no longer receives security patches. It is susceptible to multiple known vulnerabilities, including but not limited to CVE-2021-23017. Running unsupported software on a publicly accessible service presents a high risk of compromise.

% --- 5. Risk Assessment ---
\section{Risk Assessment}

This section synthesizes the findings from the security control review and technical scans into a prioritized list of risks. No pre-existing vulnerabilities were reported.

\begin{table}[h!]
\centering
\caption{Summary of Identified Risks}
\begin{tabular}{@{}p{0.25\linewidth}p{0.55\linewidth}l@{}}
\toprule
\textbf{Risk Name} & \textbf{Overview} & \textbf{Severity} \\
\midrule
\textbf{Lack of MFA on Endpoints} & The absence of MFA for computer logins means that a single compromised password could grant an attacker full access to an employee's workstation, enabling lateral movement and data theft. & \textbf{High} \\
\addlinespace
\textbf{Inadequate Security Awareness Training} & Without mandatory training for new and existing employees, staff are more susceptible to phishing, social engineering, and other common attacks, making them the weakest link in the security chain. & \textbf{High} \\
\addlinespace
\textbf{Outdated Web Server Software} & The nginx server is running a vulnerable, end-of-life version. This exposes the organization to remote code execution, denial-of-service, and other exploits that could compromise the server and its data. & \textbf{High} \\
\bottomrule
\end{tabular}
\end{table}

% --- 6. Recommendations ---
\section{Recommendations}

The following actions are recommended to mitigate the identified risks and improve the overall security posture of \textbf{Pioneer Pulse}.

\begin{enumerate}
    \item \textbf{Implement Mandatory MFA for Endpoint Access:}
    \begin{itemize}
        \item \textbf{Action:} Enforce Multi-Factor Authentication for all employee computer and laptop logins. Solutions like Windows Hello for Business, Duo, or other third-party integrations should be deployed.
        \item \textbf{Impact:} Drastically reduces the risk of unauthorized access from stolen or weak credentials.
        \item \textbf{Priority:} Critical
    \end{itemize}
    \vspace{1em}
    \item \textbf{Establish a Comprehensive Security Awareness Program:}
    \begin{itemize}
        \item \textbf{Action:} Develop and implement a formal security awareness training program. This must include mandatory training for all new hires during onboarding and annual refresher training for all employees. Topics should cover phishing, password hygiene, and acceptable use policies.
        \item \textbf{Impact:} Creates a security-conscious culture and reduces the likelihood of human error leading to a breach.
        \item \textbf{Priority:} High
    \end{itemize}
    \vspace{1em}
    \item \textbf{Upgrade and Patch Web Server Software:}
    \begin{itemize}
        \item \textbf{Action:} Immediately plan and execute an upgrade of the nginx server on \texttt{192.168.10.5} from version \texttt{1.18.0} to the latest stable version. Implement a patch management policy to ensure all internet-facing systems are kept up-to-date.
        \item \textbf{Impact:} Closes known vulnerability exploits, protecting the server and organizational data from compromise.
        \item \textbf{Priority:} Critical
    \end{itemize}
\end{enumerate}

\end{document}
```