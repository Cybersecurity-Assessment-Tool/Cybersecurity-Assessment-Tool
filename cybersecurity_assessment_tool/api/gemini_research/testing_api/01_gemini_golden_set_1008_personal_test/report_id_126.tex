```latex
\documentclass[12pt, a4paper]{article}

% Preamble: Required Packages
\usepackage[margin=1in]{geometry}
\usepackage{pifont} % For checkmarks and crosses
\usepackage{booktabs} % For professional tables
\usepackage{hyperref} % For clickable links
\usepackage{url} % For URL formatting
\usepackage{seqsplit} % To split long strings without breaking
\usepackage{graphicx}
\usepackage{xcolor}
\usepackage{fancyhdr}

% Document Information
\title{Cybersecurity Posture Assessment Report}
\author{Cybersecurity Analysis Division}
\date{\today}

% Hyperref Setup
\hypersetup{
    colorlinks=true,
    linkcolor=blue,
    filecolor=magenta,      
    urlcolor=cyan,
    pdftitle={Cybersecurity Posture Assessment Report},
    pdfpagemode=FullScreen,
}

% Header and Footer
\pagestyle{fancy}
\fancyhf{}
\lhead{North Star Education}
\rhead{Confidential}
\cfoot{\thepage}

\begin{document}

\maketitle
\thispagestyle{empty}
\newpage

\tableofcontents
\newpage

% --- 1. Executive Summary ---
\section{Executive Summary}

This report provides a comprehensive cybersecurity assessment for North Star Education, based on an analysis of organizational data, network scan results, and a review of pre-existing risks. The assessment identified significant gaps in administrative and access controls, which present a critical risk to the organization's data and systems.

The most pressing concerns are the absence of Multi-Factor Authentication (MFA) for email and sensitive data systems. This exposes the organization to a high likelihood of account compromise through phishing or credential theft. Furthermore, the lack of a formal Acceptable Use Policy and a structured security awareness training program for employees exacerbates this risk, as personnel may be unaware of security best practices.

On a positive note, a technical network scan of the target host \texttt{192.168.0.5} revealed that port 80 (HTTP) is closed. This finding contradicts a pre-existing risk entry ("Unencrypted Web Server") and suggests that this specific vulnerability may have been previously remediated.

Immediate action is required to implement MFA across all critical platforms. Concurrently, the development and enforcement of security policies and training programs should be prioritized to build a more resilient security culture.

% --- 2. Organizational Information ---
\section{Organizational Information}

The following details were provided for the assessment. This information is used to establish the context and scope of the review.

\begin{table}[h!]
\centering
\begin{tabular}{@{}ll@{}}
\toprule
\textbf{Attribute} & \textbf{Value} \\ \midrule
Organization Name & North Star Education \\
Email Domain & \texttt{NorthStarEducation.net} \\
Website Domain & \url{www.NorthStarEducation.net} \\
External IP Address & \texttt{155.215.187.251} \\ \bottomrule
\end{tabular}
\caption{Client Organizational Details.}
\label{tab:org_info}
\end{table}

% --- 3. Security Control Review ---
\section{Security Control Review}

A review of the organization's security controls was conducted via a questionnaire. The responses highlight critical deficiencies in identity and access management and employee security awareness. A summary of the findings is presented in Table \ref{tab:controls}.

\begin{table}[h!]
\centering
\renewcommand{\arraystretch}{1.2}
\begin{tabular}{@{}p{0.7\textwidth}c@{}}
\toprule
\textbf{Control Question} & \textbf{Status} \\ \midrule
Do you require MFA to access email? & \textcolor{red}{\ding{55}} \\
Do you require MFA to log into computers? & \textcolor{green}{\ding{51}} \\
Do you require MFA to access sensitive data systems? & \textcolor{red}{\ding{55}} \\
Does your organization have an employee acceptable use policy? & \textcolor{red}{\ding{55}} \\
Does your organization do security awareness training for new employees? & \textcolor{red}{\ding{55}} \\
Does your organization do security awareness training for all employees at least once per year? & \textcolor{red}{\ding{55}} \\ \bottomrule
\end{tabular}
\caption{Security Control Questionnaire Results (\ding{51} = Yes, \ding{55} = No).}
\label{tab:controls}
\end{table}

The "No" responses indicate significant gaps that elevate the organization's risk profile. The lack of MFA on email and sensitive systems is particularly alarming and is a primary vector for modern cyberattacks.

% --- 4. Technical Scan Results ---
\section{Technical Scan Results}

A network scan was performed to identify open ports and services on the specified target system. The scan provides a technical snapshot of the host's external exposure.

\begin{itemize}
    \item \textbf{Target IP Address:} \texttt{192.168.0.5}
    \item \textbf{Scan Tool:} Nmap
\end{itemize}

The scan results, detailed in Table \ref{tab:scan_results}, were minimal.

\begin{table}[h!]
\centering
\begin{tabular}{@{}cccc@{}}
\toprule
\textbf{Port} & \textbf{State} & \textbf{Service} & \textbf{Notes} \\ \midrule
80/tcp & closed & http & N/A \\ \bottomrule
\end{tabular}
\caption{Nmap Scan Results for \texttt{192.168.0.5}.}
\label{tab:scan_results}
\end{table}

\textbf{Analysis:} The scan confirmed that port 80 (HTTP) is closed on the target host. This is a positive security finding, as it prevents unencrypted web traffic. This result contradicts the pre-existing risk titled "Unencrypted Web Server," which stated the port was open. It is likely this risk has been remediated.

% --- 5. Correlated Risk Assessment ---
\section{Correlated Risk Assessment}

This section synthesizes findings from the security control review, technical scan, and pre-existing risk data into a prioritized list of current risks.

\begin{table}[h!]
\centering
\renewcommand{\arraystretch}{1.3}
\begin{tabular}{@{}p{0.2\textwidth}p{0.5\textwidth}p{0.15\textwidth}@{}}
\toprule
\textbf{Risk Name} & \textbf{Overview} & \textbf{Severity} \\ \midrule
\textbf{Lack of MFA on Critical Systems} & Email and sensitive data systems are accessible with only a password, making them highly vulnerable to credential theft, phishing, and unauthorized access. & \textbf{Critical} \\
\textbf{Inadequate Security Policies \& Training} & The absence of an Acceptable Use Policy and security awareness training means employees are likely unaware of security risks and best practices, increasing the chance of human error leading to a breach. & \textbf{High} \\
\textbf{Unencrypted Web Server} & \textit{(Previously identified risk)} The vulnerability scanner reported port 80 was open. However, our recent technical scan confirms this port is now closed. & \textbf{Mitigated} \\ \bottomrule
\end{tabular}
\caption{Summary of Identified and Correlated Risks.}
\label{tab:risk_summary}
\end{table}

% --- 6. Recommendations ---
\section{Recommendations}

Based on the correlated risk assessment, the following actions are recommended to improve the cybersecurity posture of North Star Education.

\subsection{Immediate Priority (0-30 Days)}

\begin{enumerate}
    \item \textbf{Enforce Multi-Factor Authentication (MFA):}
    \begin{itemize}
        \item Immediately enable and enforce MFA for all user accounts across the email system (\texttt{NorthStarEducation.net}).
        \item Identify all systems classified as containing sensitive data and enforce MFA for access.
        \item This single action will provide the most significant and immediate reduction in risk.
    \end{itemize}
\end{enumerate}

\subsection{Short-Term Priority (30-90 Days)}

\begin{enumerate}
    \setcounter{enumi}{1} % Continue numbering
    \item \textbf{Develop and Implement Security Policies:}
    \begin{itemize}
        \item Draft and formally adopt an Employee Acceptable Use Policy (AUP). This policy should clearly define the rules for using company assets, data handling, and internet usage.
        \item Ensure all employees read and formally acknowledge the policy.
    \end{itemize}
    \item \textbf{Establish a Security Awareness Training Program:}
    \begin{itemize}
        \item Implement a mandatory security awareness training module for all new hires as part of their onboarding process.
        \item Schedule and conduct annual security awareness training for all existing employees. Topics should include phishing identification, password hygiene, and data protection responsibilities.
    \end{itemize}
\end{enumerate}

\subsection{Verification and Housekeeping}

\begin{enumerate}
    \setcounter{enumi}{3} % Continue numbering
    \item \textbf{Close Mitigated Risk:}
    \begin{itemize}
        \item Formally update the organization's risk register to mark the "Unencrypted Web Server" risk as closed, referencing the recent scan results that confirm port 80 is no longer open.
    \end{itemize}
\end{enumerate}

\end{document}
```