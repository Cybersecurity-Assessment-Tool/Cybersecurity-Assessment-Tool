```latex
\documentclass[12pt]{article}

% --- PACKAGES ---
\usepackage[margin=1in]{geometry}
\usepackage{pifont} % For checkmarks and crosses
\usepackage{booktabs} % For professional tables
\usepackage{hyperref} % For clickable links and metadata
\usepackage{url} % For URL formatting
\usepackage{seqsplit} % To split long strings in tt font

% --- DOCUMENT METADATA ---
\hypersetup{
    colorlinks=true,
    linkcolor=blue,
    filecolor=magenta,      
    urlcolor=cyan,
    pdftitle={Cybersecurity Assessment Report},
    pdfauthor={Cybersecurity Analyst},
    pdfsubject={Security Assessment},
    pdfkeywords={Cybersecurity, Risk, Assessment},
    pdftoolbar=true,
}

% --- CUSTOM COMMANDS ---
\newcommand{\cmark}{\ding{51}}%
\newcommand{\xmark}{\ding{55}}%

% --- TITLE ---
\title{Cybersecurity Assessment Report \\ \large For: \textbf{Prism Logic}}
\author{Cybersecurity Analyst}
\date{\today}

% --- DOCUMENT START ---
\begin{document}

\maketitle
\tableofcontents
\newpage

% --- EXECUTIVE OVERVIEW ---
\section{Executive Overview}
This report details the findings of a cybersecurity assessment conducted for \textbf{Prism Logic}. The assessment included a review of organizational security controls, an external network vulnerability scan, and a correlation of these findings with existing risk data.

The external network scan of the target IP address \texttt{[Target IP]} revealed no open ports. This is a positive finding, suggesting a strong network perimeter and a minimal external attack surface.

However, the security control review identified several critical and high-risk gaps in the organization's internal security posture. Key findings include:
\begin{itemize}
    \item \textbf{Critical Risk:} Lack of a formal Employee Acceptable Use Policy (AUP).
    \item \textbf{Critical Risk:} Absence of security awareness training for new employees during onboarding.
    \item \textbf{High Risk:} No requirement for Multi-Factor Authentication (MFA) to log into company computers.
\end{itemize}
While the organization's network perimeter appears secure, these internal policy and endpoint security deficiencies present a significant risk. Immediate remediation is recommended to address these gaps and mitigate potential threats from credential compromise and insider risk.

% --- ORGANIZATIONAL INFORMATION ---
\section{Organizational Information}
The following details were provided for the assessment.
\begin{itemize}
    \item \textbf{Organization Name:} Prism Logic
    \item \textbf{Email Domain:} \seqsplit{\texttt{PrismLogic.net}}
    \item \textbf{Primary External IP:} \seqsplit{\texttt{178.153.156.2}}
\end{itemize}

% --- SECURITY CONTROL REVIEW ---
\section{Security Control Review}
A review of administrative and technical security controls was conducted via a questionnaire. The responses reveal critical areas for improvement in policy and endpoint security.

\begin{table}[h!]
\centering
\caption{Security Control Questionnaire Analysis}
\begin{tabular}{p{0.6\linewidth} c l}
\toprule
\textbf{Control Question} & \textbf{Response} & \textbf{Assessment} \\
\midrule
Do you require MFA to access email? & \cmark & Control in Place \\
\addlinespace
Do you require MFA to log into computers? & \xmark & \textbf{High Risk Gap} \\
\addlinespace
Do you require MFA to access sensitive data systems? & \cmark & Control in Place \\
\addlinespace
Does your organization have an employee acceptable use policy? & \xmark & \textbf{Critical Policy Gap} \\
\addlinespace
Does your organization do security awareness training for new employees? & \xmark & \textbf{Critical Process Gap} \\
\addlinespace
Does your organization do security awareness training for all employees at least once per year? & \cmark & Control in Place \\
\bottomrule
\end{tabular}
\end{table}

% --- TECHNICAL SCAN RESULTS ---
\section{Technical Scan Results}
An external network scan was performed on the target IP address provided by the client.

\begin{itemize}
    \item \textbf{Target IP:} \texttt{[Target IP]}
    \item \textbf{Scan Summary:} The network scan did not identify any open TCP or UDP ports on the target host.
\end{itemize}

\paragraph{Analysis:} This result indicates a well-configured perimeter firewall that is correctly blocking unsolicited inbound traffic. From an external attacker's perspective, this significantly reduces the visible attack surface and is a strong security posture for the network edge. No vulnerabilities were discovered.

% --- PRE-EXISTING RISKS ---
\section{Pre-existing Risks}
The provided data on current risks was reviewed.
\begin{itemize}
    \item No pre-existing vulnerabilities or risks were reported for this assessment period.
\end{itemize}

% --- RISK ASSESSMENT SUMMARY ---
\section{Risk Assessment Summary}
The following table summarizes the new risks identified during this assessment, derived from the security control review. These risks represent the most significant threats to the organization's security posture.

\begin{table}[h!]
\centering
\caption{Identified Risks}
\begin{tabular}{p{0.25\linewidth} p{0.5\linewidth} l}
\toprule
\textbf{Risk Name} & \textbf{Description} & \textbf{Severity} \\
\midrule
Inadequate Employee Onboarding & New employees do not receive security awareness training, leaving them vulnerable to social engineering and policy violations from their first day. & \textbf{Critical} \\
\addlinespace
Missing Acceptable Use Policy & The absence of a formal AUP creates ambiguity for employees regarding safe technology use, increasing insider threat and compliance risks. & \textbf{High} \\
\addlinespace
Lack of Endpoint MFA & User computers do not require MFA for login. This exposes the organization to unauthorized access via compromised credentials, potentially leading to data breaches. & \textbf{High} \\
\bottomrule
\end{tabular}
\end{table}

% --- RECOMMENDATIONS ---
\section{Recommendations}
Based on the findings, the following actions are recommended to mitigate the identified risks and improve the overall security posture of \textbf{Prism Logic}.

\begin{enumerate}
    \item \textbf{Develop an Acceptable Use Policy (AUP):}
    \begin{itemize}
        \item \textbf{Action:} Create a formal AUP that clearly defines the rules and expectations for the use of company technology, data, and network resources.
        \item \textbf{Impact:} Establishes a baseline for secure employee behavior and provides a basis for enforcement actions. This is a foundational governance control.
    \end{itemize}
    \vspace{1em}
    
    \item \textbf{Integrate Security Training into Onboarding:}
    \begin{itemize}
        \item \textbf{Action:} Implement a mandatory security awareness training module as part of the new employee onboarding process. This should cover phishing, password security, data handling, and the new AUP.
        \item \textbf{Impact:} Reduces the risk posed by new employees who may be unaware of organizational security policies and common cyber threats.
    \end{itemize}
    \vspace{1em}
    
    \item \textbf{Enforce MFA for Endpoint Logins:}
    \begin{itemize}
        \item \textbf{Action:} Deploy and mandate the use of Multi-Factor Authentication for all employee computer (desktop and laptop) logins.
        \item \textbf{Impact:} Drastically reduces the risk of unauthorized access from stolen or weak credentials, protecting endpoints which are common entry points for attackers.
    \end{itemize}
\end{enumerate}

\end{document}
```