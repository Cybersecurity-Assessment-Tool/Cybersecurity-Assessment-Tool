```latex
\documentclass[12pt]{article}

% --- PACKAGES ---
\usepackage[margin=1in]{geometry}
\usepackage{pifont} % For checkmarks and crosses
\usepackage{booktabs} % For professional tables
\usepackage{xcolor} % For colors
\usepackage{hyperref} % For hyperlinks
\usepackage{url} % For URL formatting
\usepackage{seqsplit} % To split long strings in table cells
\usepackage{graphicx} % For potential logos or graphics

% --- DOCUMENT SETUP ---
\hypersetup{
    colorlinks=true,
    linkcolor=blue,
    filecolor=magenta,      
    urlcolor=cyan,
    pdftitle={Cybersecurity Posture Assessment Report},
    pdfpagemode=FullScreen,
}

% --- CUSTOM COMMANDS ---
\newcommand{\yes}{\ding{51}} % Green checkmark
\newcommand{\no}{\ding{55}}  % Red X

\begin{document}

% --- TITLE PAGE ---
\begin{titlepage}
    \centering
    \vspace*{1cm}
    \Huge\textbf{Cybersecurity Posture Assessment Report}
    \vspace{1.5cm}
    \Large
    \textbf{Prepared for:} \\
    \vspace{0.5cm}
    \textbf{Hearth \& Home}
    \vspace{2cm}
    \large
    \textbf{Date of Report:} \today \\
    \vspace{0.5cm}
    \textbf{Report ID:} CYBER-2023-014
    \vfill
    \large
    \textit{This report contains sensitive information and is intended solely for the use of the recipient organization. Distribution is strictly prohibited.}
\end{titlepage}

\tableofcontents
\newpage

% --- EXECUTIVE OVERVIEW ---
\section{Executive Overview}
This report provides a comprehensive assessment of the cybersecurity posture for \textbf{Hearth \& Home}. The analysis combines a review of organizational security controls, technical network scanning, and a review of pre-existing risk documentation.

The assessment identified several critical and high-risk findings that require immediate attention. A key discovery was an exposed web service on port \texttt{8080} of an internal system (\texttt{10.5.5.5}), which presents a title suggesting it is a \textbf{"TOP SECRET DB"}. This finding directly contradicts previous risk assessments which had marked this port as a secure false positive. This indicates a significant regression in security or a flawed initial assessment.

This technical vulnerability is compounded by critical gaps in organizational controls. The lack of multi-factor authentication (MFA) on sensitive data systems, combined with the absence of a formal employee acceptable use policy and a security awareness training program, creates a high-risk environment. An attacker who compromises this exposed database would face fewer barriers to accessing sensitive information.

Immediate remediation should focus on securing the exposed service on port \texttt{8080} and implementing MFA across all sensitive systems. Strategic initiatives must be launched to develop and enforce foundational security policies and training programs.

% --- ORGANIZATIONAL INFORMATION ---
\section{Organizational Information}
The following information was provided for the assessment.
\begin{center}
\begin{tabular}{ll}
\toprule
\textbf{Attribute} & \textbf{Value} \\
\midrule
Organization Name & \textbf{Hearth \& Home} \\
Email Domain & \texttt{HearthHome.org} \\
Website Domain & \url{www.HearthHome.org} \\
External IP Address & \texttt{125.17.102.22} \\
\bottomrule
\end{tabular}
\end{center}

% --- SECURITY CONTROL REVIEW ---
\section{Security Control Review}
A review of the organization's security controls was conducted via a questionnaire. The responses reveal significant gaps in policy and procedural safeguards. "No" answers indicate a lack of a critical security control and are flagged as high-risk findings.

\begin{center}
\begin{tabular}{p{0.6\textwidth} c c}
\toprule
\textbf{Control Question} & \textbf{Response} & \textbf{Assessment} \\
\midrule
\seqsplit{Do you require MFA to access email?} & Yes & \yes \\
\seqsplit{Do you require MFA to log into computers?} & Yes & \yes \\
\seqsplit{Do you require MFA to access sensitive data systems?} & No & \no \\
\seqsplit{Does your organization have an employee acceptable use policy?} & No & \no \\
\seqsplit{Does your organization do security awareness training for new employees?} & No & \no \\
\seqsplit{Does your organization do security awareness training for all employees at least once per year?} & No & \no \\
\bottomrule
\end{tabular}
\end{center}

% --- TECHNICAL SCAN RESULTS ---
\section{Technical Scan Results}
An internal network scan was performed to identify active services and potential vulnerabilities.

\subsection*{Scan Target: \texttt{10.5.5.5}}
The scan identified one host as active and responsive. The following open port was discovered:
\begin{center}
\begin{tabular}{l l l p{0.4\textwidth}}
\toprule
\textbf{Port} & \textbf{State} & \textbf{Service} & \textbf{Details} \\
\midrule
8080/tcp & OPEN & http & The HTTP service running on this port returned a title: \textbf{"TOP SECRET DB"}. This strongly suggests a sensitive, and likely unauthorized, database interface is exposed on the network. \\
\bottomrule
\end{tabular}
\end{center}

% --- RISK ASSESSMENT ---
\section{Risk Assessment}
The following table summarizes the correlated risks identified through the questionnaire, technical scanning, and review of existing documentation.

\begin{center}
\begin{tabular}{p{0.2\textwidth} p{0.5\textwidth} l}
\toprule
\textbf{Risk Name} & \textbf{Description} & \textbf{Severity} \\
\midrule
\textbf{Sensitive Database Exposure} & An open port (\texttt{8080}) on an internal server reveals a web service titled "TOP SECRET DB". This indicates a high-value data asset is exposed without proper access controls. This contradicts previous risk assessments. & \textcolor{red}{\textbf{Critical}} \\
\addlinespace
\textbf{Lack of MFA on Sensitive Systems} & The organization does not enforce MFA for accessing sensitive data systems. This removes a critical layer of defense, making credential theft or misuse significantly more impactful. & \textcolor{red}{\textbf{Critical}} \\
\addlinespace
\textbf{Absence of Security Awareness Training} & Employees are not trained on security best practices. This increases the likelihood of human error, such as falling for phishing attacks or mishandling sensitive data, which could lead to a breach. & \textcolor{orange}{\textbf{High}} \\
\addlinespace
\textbf{Missing Acceptable Use Policy (AUP)} & Without a formal AUP, there are no clear rules for employees regarding the use of company assets. This can lead to insecure practices and creates ambiguity in enforcing security standards. & \textcolor{orange}{\textbf{High}} \\
\addlinespace
\textbf{Outdated Risk Documentation} & The current risk register states that port 8080 is secure, which has been proven false by this scan. This indicates that the risk management process is not being kept up-to-date, leading to a false sense of security. & \textcolor{yellow!80!black}{\textbf{Medium}} \\
\bottomrule
\end{tabular}
\end{center}

% --- RECOMMENDATIONS ---
\section{Recommendations}
The following actions are recommended to mitigate the identified risks. Recommendations are prioritized based on severity.

\subsection*{Critical Risk Recommendations}
\begin{description}
    \item[Remediate Exposed Database (Port 8080):]
        \begin{itemize}
            \item \textbf{Immediate Action:} Investigate the service on \texttt{10.5.5.5:8080} immediately. Identify the system owner and the data it contains. If sensitive, take the service offline or place it behind a firewall with a restrictive Access Control List (ACL) limited to authorized personnel only.
            \item \textbf{Long-Term Fix:} Implement network segmentation to isolate database servers from user workstations. Enforce a policy of "default deny" for all network traffic.
        \end{itemize}
    \item[Implement MFA on Sensitive Systems:]
        \begin{itemize}
            \item \textbf{Immediate Action:} Identify all systems classified as containing sensitive data and prioritize the rollout of an MFA solution for all user access.
            \item \textbf{Long-Term Fix:} Develop and enforce a data classification policy and an access control policy that mandates MFA for all systems handling confidential or sensitive information.
        \end{itemize}
\end{description}

\subsection*{High Risk Recommendations}
\begin{description}
    \item[Establish Security Policy and Training:]
        \begin{itemize}
            \item \textbf{Immediate Action:} Develop and ratify a formal Acceptable Use Policy (AUP). Concurrently, procure and enroll all employees in a foundational security awareness training program covering topics like phishing, password hygiene, and data handling.
            \item \textbf{Long-Term Fix:} Schedule mandatory annual security awareness training for all staff and establish a process to review and update the AUP and other security policies on a yearly basis.
        \end{itemize}
\end{description}

\subsection*{Medium Risk Recommendations}
\begin{description}
    \item[Improve Risk Management Processes:]
        \begin{itemize}
            \item \textbf{Immediate Action:} Update the risk register to reflect the findings of this report. Mark the previous assessment of port 8080 as obsolete.
            \item \textbf{Long-Term Fix:} Implement a recurring, scheduled process for technical vulnerability scanning and risk assessment reviews (e.g., quarterly) to ensure that documentation accurately reflects the current security posture.
        \end{itemize}
\end{description}

\end{document}
```