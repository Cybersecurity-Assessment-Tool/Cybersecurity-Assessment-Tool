```latex
\documentclass[12pt]{article}

% Preamble: Required Packages
\usepackage[margin=1in]{geometry}
\usepackage{pifont} % For checkmarks and crosses (\ding)
\usepackage{booktabs} % For professional tables (\toprule, \midrule, \bottomrule)
\usepackage[hidelinks]{hyperref} % For clickable links without boxes
\usepackage{url}
\usepackage{seqsplit} % To split long monospaced strings
\usepackage{fancyhdr} % For headers and footers
\usepackage{lastpage} % To get the total number of pages
\usepackage{xcolor} % For color definitions

% --- Document Setup ---

% Define colors for severity
\definecolor{sev_critical}{HTML}{990000}
\definecolor{sev_high}{HTML}{D14302}

% Header and Footer Configuration
\pagestyle{fancy}
\fancyhf{} % Clear all header and footer fields
\fancyhead[L]{\textbf{Cybersecurity Posture Assessment Report}}
\fancyhead[R]{\textbf{Quantum Reach}}
\fancyfoot[C]{\thepage\ of \pageref{LastPage}}
\renewcommand{\headrulewidth}{0.4pt}
\renewcommand{\footrulewidth}{0.4pt}

\begin{document}

% --- Title Page ---
\begin{titlepage}
    \centering
    \vspace*{2cm}
    
    {\Huge \textbf{Cybersecurity Posture Assessment Report}\par}
    \vspace{1.5cm}
    
    {\Large Prepared for:\par}
    \vspace{0.5cm}
    {\Huge \textbf{Quantum Reach}\par}
    
    \vfill
    
    {\large \today\par}
    \vspace{1cm}
    
    {\large Confidential Document\par}
\end{titlepage}

\tableofcontents
\newpage

% --- Executive Summary ---
\section*{Executive Summary}

This report provides a comprehensive cybersecurity assessment for \textbf{Quantum Reach}, synthesizing data from technical network scans, a security controls questionnaire, and a review of pre-existing risks. The analysis reveals several critical and high-risk issues that require immediate attention to mitigate potential threats to the organization's data and infrastructure.

A critical vulnerability, \textbf{Localhost Exposed}, was identified and confirmed by the technical scan, which detected an open SSH port (22) on an internal loopback address (\texttt{127.0.0.1}). This configuration carries a CVSS score of 10.0 and represents a severe and immediate threat.

Furthermore, significant gaps were identified in the organization's security controls. The lack of mandatory Multi-Factor Authentication (MFA) for email and sensitive data systems, combined with the absence of a formal Acceptable Use Policy and annual security awareness training, indicates a weakened security posture. These procedural and policy-based deficiencies substantially increase the risk of unauthorized access, data breaches, and successful social engineering attacks.

This report outlines these findings in detail and provides a prioritized list of actionable recommendations to strengthen the overall security posture of \textbf{Quantum Reach}.

% --- Organizational Information ---
\section*{1. Organizational Information}

The following details were provided for the assessment.

\begin{tabular}{@{}ll}
    \toprule
    \textbf{Attribute} & \textbf{Value} \\
    \midrule
    Organization Name & \textbf{Quantum Reach} \\
    Primary Email Domain & \texttt{QuantumReach.com} \\
    Primary Website Domain & \seqsplit{\url{www.QuantumReach.com}} \\
    Known External IP & \texttt{23.142.125.246} \\
    \bottomrule
\end{tabular}

% --- Security Control Review ---
\section*{2. Security Control Review}

An analysis of the security controls questionnaire highlights key areas of strength and weakness. "No" responses indicate significant gaps in security posture and are flagged as risks. The checkmark (\ding{51}) indicates a positive control is in place, while the cross mark (\ding{55}) indicates a control gap.

\begin{tabular}{@{}p{0.6\linewidth} c l}
    \toprule
    \textbf{Control Question} & \textbf{Response} & \textbf{Assessment} \\
    \midrule
    Do you require MFA to log into computers? & \ding{51} & Good Practice \\
    Does your organization do security awareness training for new employees? & \ding{51} & Good Practice \\
    \addlinespace
    Do you require MFA to access email? & \ding{55} & \textbf{High Risk Gap} \\
    Do you require MFA to access sensitive data systems? & \ding{55} & \textbf{High Risk Gap} \\
    Does your organization have an employee acceptable use policy? & \ding{55} & \textbf{High Risk Gap} \\
    Does your organization do security awareness training for all employees at least once per year? & \ding{55} & \textbf{High Risk Gap} \\
    \bottomrule
\end{tabular}

% --- Technical Scan Results ---
\section*{3. Technical Scan Results}

A network scan was performed to identify open ports and exposed services on the specified target. The scan confirmed the presence of an externally accessible service on an internal address, corroborating the pre-existing risk data.

\subsection*{Scan Target}
\begin{itemize}
    \item \textbf{IP Address:} \texttt{127.0.0.1}
\end{itemize}

\subsection*{Open Ports Discovered}
\begin{tabular}{@{}llll}
    \toprule
    \textbf{Port} & \textbf{State} & \textbf{Inferred Service} & \textbf{Analysis} \\
    \midrule
    22/tcp & Open & SSH (Secure Shell) & Exposure of SSH on a localhost address is highly \\
           &      &                     & irregular and indicates a critical misconfiguration. \\
           &      &                     & This service could be a gateway for unauthorized \\
           &      &                     & access to the host system. \\
    \bottomrule
\end{tabular}

% --- Risk Assessment ---
\section*{4. Risk Assessment}

The following table summarizes the correlated risks identified through the analysis of all provided data inputs. Risks are categorized by severity to guide prioritization of remediation efforts.

\begin{tabular}{@{}p{0.2\linewidth} p{0.55\linewidth} l}
    \toprule
    \textbf{Risk Title} & \textbf{Description} & \textbf{Severity} \\
    \midrule
    \textbf{Localhost Exposed} & The technical scan confirmed an open SSH port on the localhost interface (\texttt{127.0.0.1}), which is listed as a known critical vulnerability. This suggests a severe network or service misconfiguration. & \textcolor{sev_critical}{\textbf{Critical}} \\
    \addlinespace
    \textbf{Insufficient MFA Coverage} & MFA is not enforced for accessing email or other sensitive data systems. This significantly increases the risk of account compromise via stolen or weak credentials. & \textcolor{sev_high}{\textbf{High}} \\
    \addlinespace
    \textbf{Lack of Acceptable Use Policy} & The absence of a formal policy defining acceptable use of company assets creates ambiguity and increases the risk of insider threat, whether malicious or accidental. & \textcolor{sev_high}{\textbf{High}} \\
    \addlinespace
    \textbf{Inadequate Security Training} & While new hires receive training, the lack of a mandatory annual refresher for all employees allows security knowledge to decay, making staff more susceptible to phishing and social engineering. & \textcolor{sev_high}{\textbf{High}} \\
    \bottomrule
\end{tabular}

% --- Recommendations ---
\section*{5. Recommendations}

The following prioritized recommendations are provided to address the identified risks and improve the overall security posture of \textbf{Quantum Reach}.

\subsection*{Critical Priority}
\begin{enumerate}
    \item \textbf{Remediate Exposed Localhost Service:} Immediately investigate the system at \texttt{127.0.0.1} to understand why port 22 (SSH) is exposed. This service should be firewalled from external access unless there is an explicit and documented business need. If the service is not required, it should be disabled.
\end{enumerate}

\subsection*{High Priority}
\begin{enumerate}
    \setcounter{enumi}{1} % Continue numbering
    \item \textbf{Implement Comprehensive MFA:} Enforce MFA across all critical systems, prioritizing email (e.g., Office 365, Google Workspace) and any applications that store or process sensitive data. This is one of the most effective controls against account takeover attacks.
    \item \textbf{Develop and Enforce an Acceptable Use Policy (AUP):} Create a formal AUP document that clearly outlines the rules and responsibilities for all employees when using company technology and data. Ensure all employees read and acknowledge the policy.
\end{enumerate}

\subsection*{Medium Priority}
\begin{enumerate}
    \setcounter{enumi}{3} % Continue numbering
    \item \textbf{Establish Annual Security Awareness Training:} Implement a mandatory, recurring security awareness training program for all employees. This program should be conducted at least annually and cover topics such as phishing, password hygiene, and social engineering.
\end{enumerate}

\end{document}
```