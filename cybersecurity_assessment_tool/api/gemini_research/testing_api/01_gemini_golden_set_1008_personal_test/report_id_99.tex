```latex
\documentclass[12pt]{article}

% Preamble: Required Packages and Document Setup
\usepackage[margin=1in]{geometry}
\usepackage{pifont} % For checkmarks and crosses
\usepackage{booktabs} % For professional-looking tables
\usepackage{hyperref} % For clickable links
\usepackage{url} % For formatting URLs
\usepackage{seqsplit} % For splitting long strings in texttt
\usepackage[table]{xcolor} % For table colors
\usepackage{graphicx}
\usepackage{tabularx}

% --- Document Metadata ---
\title{Cybersecurity Posture Assessment Report \\ \large For Oasis Wellness}
\author{Cybersecurity Analysis Division}
\date{\today}

% --- Hyperref Setup ---
\hypersetup{
    colorlinks=true,
    linkcolor=blue,
    filecolor=magenta,      
    urlcolor=cyan,
    pdftitle={Cybersecurity Posture Assessment Report},
    pdfpagemode=FullScreen,
}

% --- Custom Commands ---
\newcommand{\yes}{\ding{51}}
\newcommand{\no}{\ding{55}}
\newcommand{\cellgray}{\cellcolor{gray!20}}

\begin{document}

\maketitle
\thispagestyle{empty}
\newpage

\tableofcontents
\newpage

% ==============================================================================
% 1. Executive Summary
% ==============================================================================
\section{Executive Summary}

This report details the findings of a cybersecurity posture assessment conducted for \textbf{Oasis Wellness}. The analysis correlates results from a network vulnerability scan, a review of organizational security controls, and a list of pre-existing risks.

The assessment reveals several \textbf{critical} and \textbf{high-risk} security vulnerabilities that require immediate attention. Key findings include:
\begin{itemize}
    \item \textbf{Critical - End-of-Life Software:} The organization's primary database is running on MySQL version 5.7.33, which reached its official End of Life (EOL) in October 2023. This software no longer receives security patches and is highly susceptible to exploitation.
    \item \textbf{High - Public Database Exposure:} The technical scan confirmed that the MySQL database service on port 3306 is directly exposed to the network, a finding consistent with pre-existing risk data. This configuration significantly increases the risk of unauthorized access and data breach.
    \item \textbf{High - Gaps in Access Control:} Multi-Factor Authentication (MFA) is not enforced for accessing email or for logging into employee computers. This represents a major weakness in identity and access management, leaving the organization vulnerable to phishing and credential theft attacks.
    \item \textbf{High - Deficiencies in Security Governance:} The organization lacks a formal Acceptable Use Policy and does not provide recurring, annual security awareness training for all employees. These gaps in administrative controls weaken the overall security culture and increase the likelihood of human error leading to a security incident.
\end{itemize}

The combination of these findings places the organization at a significant risk of a security breach. This report provides a detailed breakdown of each risk and offers prioritized, actionable recommendations for remediation.

% ==============================================================================
% 2. Organizational Information
% ==============================================================================
\section{Organizational Information}

The following details were provided for the assessment.

\begin{tabular}{@{}ll}
\toprule
\textbf{Attribute} & \textbf{Value} \\
\midrule
Organization Name & \textbf{Oasis Wellness} \\
Email Domain & \texttt{OasisWellness.net} \\
Website Domain & \texttt{www.OasisWellness.net} \\
External IP Address & \texttt{70.114.150} \\
\bottomrule
\end{tabular}

% ==============================================================================
% 3. Security Control Review (Questionnaire Analysis)
% ==============================================================================
\section{Security Control Review (Questionnaire Analysis)}

A review of the organization's administrative and technical security controls was conducted via a questionnaire. The responses reveal significant gaps in foundational security practices. A "No" response indicates a deviation from security best practices and is flagged as a risk.

\begin{tabularx}{\textwidth}{@{}X c p{0.5\textwidth}@{}}
\toprule
\textbf{Control Question} & \textbf{Response} & \textbf{Assessment} \\
\midrule
Do you require MFA to access email? & \no & \textbf{High Risk.} Email is a primary vector for phishing and account takeover. Without MFA, a single compromised password grants an attacker full access to an employee's mailbox. \\
\addlinespace
Do you require MFA to log into computers? & \no & \textbf{High Risk.} Lack of MFA on endpoints weakens protection against unauthorized access if credentials are stolen or a device is lost. \\
\addlinespace
Do you require MFA to access sensitive data systems? & \yes & \textbf{Good Practice.} This control reduces the risk of unauthorized access to critical systems and should be maintained. \\
\addlinespace
Does your organization have an employee acceptable use policy? & \no & \textbf{High Risk.} The absence of a formal policy creates ambiguity regarding the secure and acceptable use of company assets, increasing insider threat and potential legal liabilities. \\
\addlinespace
Does your organization do security awareness training for new employees? & \yes & \textbf{Good Practice.} Onboarding training is a crucial first step in establishing a security-conscious culture. \\
\addlinespace
Does your organization do security awareness training for all employees at least once per year? & \no & \textbf{High Risk.} Security threats evolve constantly. Without recurring annual training, employee awareness diminishes, and they are less prepared to identify and respond to new threats like sophisticated phishing attacks. \\
\bottomrule
\end{tabularx}

% ==============================================================================
% 4. Technical Scan Results
% ==============================================================================
\section{Technical Scan Results}

An Nmap scan was performed on the target system to identify open ports and running services. The scan confirmed the public exposure of a critical database service.

\begin{itemize}
    \item \textbf{Target IP Address:} \texttt{172.16.50.20}
    \item \textbf{Host Status:} Up
\end{itemize}

\begin{tabularx}{\textwidth}{@{}c c l l X@{}}
\toprule
\textbf{Port} & \textbf{State} & \textbf{Service} & \textbf{Version} & \textbf{Analysis} \\
\midrule
\cellgray 3306/tcp & \cellgray Open & \cellgray mysql & \cellgray MySQL 5.7.33 & \cellgray \textbf{Critical Finding.} This version of MySQL is \textbf{End-of-Life (EOL)} as of October 2023. It no longer receives security updates from the vendor, making it an easy target for attackers exploiting known vulnerabilities. Exposing any database directly to the network is a severe security risk. \\
\bottomrule
\end{tabularx}

% ==============================================================================
% 5. Consolidated Risk Assessment
% ==============================================================================
\section{Consolidated Risk Assessment}

The following table summarizes and prioritizes the risks identified through the correlation of the security questionnaire, technical scan, and pre-existing risk data.

\begin{tabularx}{\textwidth}{@{}l l X@{}}
\toprule
\textbf{Severity} & \textbf{Risk Title} & \textbf{Description} \\
\midrule
\rowcolor{red!25}
\textbf{Critical} & End-of-Life Database Software & The MySQL server is running version 5.7.33, which is unsupported and unpatched. Any new vulnerabilities discovered for this version will not be fixed, leaving the system perpetually vulnerable. \\
\addlinespace
\rowcolor{orange!25}
\textbf{High} & Public Exposure of Database Service & Port 3306 (MySQL) is open to the network, allowing attackers to directly attempt brute-force attacks, exploit vulnerabilities, or exfiltrate data. This confirms the pre-existing "Database Exposure" risk. \\
\addlinespace
\rowcolor{orange!25}
\textbf{High} & Lack of Multi-Factor Authentication & The absence of MFA on email and computer logins creates a single point of failure (the password) for access control, making the organization highly vulnerable to credential-based attacks. \\
\addlinespace
\rowcolor{orange!25}
\textbf{High} & Inadequate Security Policies \& Training & The lack of a formal Acceptable Use Policy and mandatory annual security training for all staff indicates a reactive, rather than proactive, security posture, increasing the risk of human-related security incidents. \\
\bottomrule
\end{tabularx}

% ==============================================================================
% 6. Recommendations
% ==============================================================================
\section{Recommendations}

The following actions are recommended to mitigate the identified risks. They are prioritized based on severity and potential impact.

\subsection{Immediate Actions (Critical Priority)}
\begin{enumerate}
    \item \textbf{Remediate Database Exposure:} Immediately implement firewall rules to restrict all access to TCP port 3306 on the server at \texttt{172.16.50.20}. Access should only be permitted from specific, trusted internal IP addresses (e.g., application servers) that require it. Public access must be disabled.
    \item \textbf{Plan Database Upgrade:} Initiate an urgent project to migrate the MySQL 5.7.33 database to a fully supported version (e.g., MySQL 8.x) or a managed cloud database service. This is the only way to mitigate the risks associated with running End-of-Life software.
\end{enumerate}

\subsection{High Priority Actions}
\begin{enumerate}
    \item \textbf{Deploy Multi-Factor Authentication (MFA):}
    \begin{itemize}
        \item Enable MFA for all user accounts on the email system (\texttt{OasisWellness.net}).
        \item Implement an MFA solution for all remote and on-site computer logins, prioritizing accounts with administrative or sensitive data access.
    \end{itemize}
    \item \textbf{Develop and Implement Security Policies:} Draft, approve, and communicate a formal Employee Acceptable Use Policy (AUP). This policy should clearly define the rules for using company technology and data.
    \item \textbf{Establish a Security Training Program:} Implement a mandatory, annual security awareness training program for all employees. The training should cover current threats such as phishing, social engineering, and password security best practices.
\end{enumerate}

\end{document}
```