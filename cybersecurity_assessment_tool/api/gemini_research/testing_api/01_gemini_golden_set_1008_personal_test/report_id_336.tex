Here is the complete and professional LaTeX report, generated based on the provided data inputs and instructions.

```latex
\documentclass[12pt]{article}

% Preamble: Required Packages
\usepackage[utf8]{inputenc}
\usepackage[margin=1in]{geometry}
\usepackage{pifont} % For checkmarks and crosses
\usepackage{booktabs} % For professional tables
\usepackage{xcolor} % For colors in tables
\usepackage{graphicx}
\usepackage{hyperref} % For clickable links
\usepackage{url} % For URL formatting
\usepackage{seqsplit} % To split long strings in texttt

% Hyperref Setup
\hypersetup{
    colorlinks=true,
    linkcolor=blue,
    filecolor=magenta,      
    urlcolor=cyan,
    pdftitle={Cybersecurity Posture Assessment Report},
    pdfauthor={Cybersecurity Analysis Division},
    pdfkeywords={cybersecurity, assessment, report},
}

% Custom Commands for Table Readability
\newcommand{\yes}{\ding{51}}
\newcommand{\no}{\ding{55}}

% Document Start
\begin{document}

% Title Page
\title{
    Cybersecurity Posture Assessment Report \\
    \large For: Astraeus Aerospace
}
\author{Cybersecurity Analysis Division}
\date{\today}
\maketitle
\thispagestyle{empty}

\newpage

% Table of Contents
\tableofcontents
\thispagestyle{empty}

\newpage
\setcounter{page}{1}

% --- Executive Overview ---
\section{Executive Overview}

This report provides a cybersecurity posture assessment for Astraeus Aerospace. The analysis is based on a review of organizational security controls provided via a questionnaire. It is critical to note that the technical network scan data (\texttt{Input\_1\_Network\_Scan\_JSON}) and the list of pre-existing risks (\texttt{Input\_3\_Current\_Risks\_JSON}) were unavailable or corrupted at the time of this assessment. Consequently, this report focuses primarily on procedural and policy-based controls.

The assessment reveals several critical and high-risk security gaps that significantly increase the organization's vulnerability to cyber threats. The most severe findings include the lack of Multi-Factor Authentication (MFA) for email and computer access, which exposes the organization to account takeover and unauthorized access.

Furthermore, the absence of a formal Acceptable Use Policy and a security training program for new employees indicates foundational weaknesses in the organization's security culture and governance. While some controls are in place, such as MFA for sensitive systems and annual training for existing employees, the identified gaps require immediate attention to establish a baseline security posture.

\section{Organizational Information}

The following details were provided for the assessment.

\begin{tabular}{@{}ll}
    \toprule
    \textbf{Attribute} & \textbf{Value} \\
    \midrule
    Organization Name & Astraeus Aerospace \\
    Email Domain & \seqsplit{\texttt{AstraeusAerospace.com}} \\
    Website Domain & \seqsplit{\texttt{www.AstraeusAerospace.com}} \\
    External IP Address & \seqsplit{\texttt{143.235.198.188}} \\
    \bottomrule
\end{tabular}

\section{Security Control Review}

The following table summarizes the responses from the security questionnaire. A green checkmark (\yes) indicates a positive control, while a red cross (\no) indicates a potential security gap that requires remediation.

\begin{table}[h!]
\centering
\begin{tabular}{p{8cm} c l}
    \toprule
    \textbf{Control Question} & \textbf{Response} & \textbf{Assessment} \\
    \midrule
    Do you require MFA to access email? & \textcolor{red}{\no} & \textbf{Critical Gap} \\
    Do you require MFA to log into computers? & \textcolor{red}{\no} & \textbf{Critical Gap} \\
    Do you require MFA to access sensitive data systems? & \textcolor{green}{\yes} & Positive Control \\
    Does your organization have an employee acceptable use policy? & \textcolor{red}{\no} & \textbf{High Risk} \\
    Does your organization do security awareness training for new employees? & \textcolor{red}{\no} & \textbf{High Risk} \\
    Does your organization do security awareness training for all employees at least once per year? & \textcolor{green}{\yes} & Positive Control \\
    \bottomrule
\end{tabular}
\caption{Security Questionnaire Analysis}
\end{table}

\section{Technical Scan Results}

The data source for the technical network scan (\texttt{Input\_1\_Network\_Scan\_JSON}) was found to be broken or incomplete. Therefore, no analysis of open ports, running services, or potential vulnerabilities on the target host \texttt{[Target IP]} could be performed. 

A comprehensive external vulnerability scan is essential for identifying technical weaknesses that could be exploited by attackers. It is strongly recommended to conduct a new scan to obtain this critical visibility.

\section{Risk Assessment}

The risk assessment is based on the findings from the Security Control Review, as the pre-existing risk data (\texttt{Input\_3\_Current\_Risks\_JSON}) was also unavailable. The identified risks are documented and prioritized below.

\begin{table}[h!]
\centering
\begin{tabular}{p{2.5cm} p{4cm} p{6cm}}
    \toprule
    \textbf{Risk ID} & \textbf{Risk Name} & \textbf{Description} \\
    \midrule
    \textbf{RISK-001} & \textcolor{red}{\textbf{Critical:}} Lack of MFA on Email Accounts & Without MFA, email accounts are highly susceptible to compromise via phishing or password spraying. A compromised email account can lead to data breaches, financial fraud, and further internal network compromise. \\
    \addlinespace
    \textbf{RISK-002} & \textcolor{red}{\textbf{Critical:}} Lack of MFA on Workstations & The absence of MFA on computer logins allows an attacker with stolen credentials to gain direct access to an employee's workstation and potentially the internal network. \\
    \addlinespace
    \textbf{RISK-003} & \textcolor{orange}{\textbf{High:}} No Employee Acceptable Use Policy (AUP) & Without a formal AUP, employees may be unaware of their responsibilities regarding company assets and data, increasing the risk of insider threat, misuse of resources, and non-compliance. \\
    \addlinespace
    \textbf{RISK-004} & \textcolor{orange}{\textbf{High:}} No Security Training for New Hires & New employees are not equipped with the necessary security knowledge from day one, making them more likely to fall victim to social engineering attacks and mishandle sensitive data. \\
    \bottomrule
\end{tabular}
\caption{Summary of Identified Risks}
\end{table}

\section{Recommendations}

The following prioritized recommendations are provided to address the identified risks and improve the overall security posture of Astraeus Aerospace.

\begin{enumerate}
    \item \textbf{[Critical] Implement Multi-Factor Authentication (MFA):}
    \begin{itemize}
        \item \textbf{Immediately} enforce MFA for all users accessing email, whether on-premises or cloud-based (e.g., Office 365, Google Workspace).
        \item \textbf{Immediately} deploy and enforce MFA for all workstation and laptop logins, especially for users with administrative or sensitive data access.
    \end{itemize}
    \vspace{0.5cm}
    \item \textbf{[High] Develop Foundational Security Policies:}
    \begin{itemize}
        \item Develop, approve, and communicate a comprehensive Acceptable Use Policy (AUP) to all employees. This policy should clearly define the rules for using company IT assets, data handling, and internet usage.
    \end{itemize}
    \vspace{0.5cm}
    \item \textbf{[High] Enhance Security Awareness Program:}
    \begin{itemize}
        \item Integrate mandatory security awareness training into the new employee onboarding process. This ensures that security principles are established from the beginning of employment.
    \end{itemize}
    \vspace{0.5cm}
    \item \textbf{[Informational] Re-initiate Technical Assessment:}
    \begin{itemize}
        \item Schedule and conduct a new, authenticated external network vulnerability scan against the public-facing IP address \seqsplit{\texttt{143.235.198.188}} and other known assets.
        \item Review and consolidate any existing risk management documentation to ensure a complete and accurate picture of the organization's risk landscape.
    \end{itemize}
\end{enumerate}

\end{document}
```