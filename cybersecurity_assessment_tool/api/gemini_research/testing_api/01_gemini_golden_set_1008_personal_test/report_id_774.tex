```latex
\documentclass[12pt]{article}

% Preamble: Required Packages
\usepackage[margin=1in]{geometry}
\usepackage{pifont} % For checkmarks and crosses
\usepackage{booktabs} % For professional tables
\usepackage{hyperref} % For clickable links and metadata
\usepackage{url}      % For formatting URLs
\usepackage{seqsplit} % For splitting long strings in tt font
\usepackage{graphicx} % For potential logos
\usepackage[utf8]{inputenc}

% Document Metadata and Styling
\hypersetup{
    colorlinks=true,
    linkcolor=black,
    urlcolor=blue,
    pdftitle={Cybersecurity Assessment Report},
    pdfauthor={Cybersecurity Analysis Division},
    pdfsubject={Security Posture Analysis},
    pdfkeywords={Cybersecurity, Risk Assessment, Network Scan}
}

% Define checkmark and cross symbols for clarity
\newcommand{\cmark}{\ding{51}}%
\newcommand{\xmark}{\ding{55}}%

\begin{document}

% --- Title Page ---
\begin{titlepage}
    \centering
    \vspace*{\fill}
    {\Huge\bfseries Cybersecurity Assessment Report\par}
    \vspace{1.5cm}
    {\Large\bfseries For: Solid State\par}
    \vspace{2cm}
    {\large Report Date: \today\par}
    \vfill
    {\large Confidential\par}
    \vspace{1cm}
    {\it This report contains sensitive information and is intended solely for the designated recipient.\par}
\end{titlepage}

\tableofcontents
\newpage

% --- Section 1: Executive Overview ---
\section{Executive Overview}
This report provides a comprehensive cybersecurity assessment for \textbf{Solid State}, synthesizing findings from a technical network scan, a review of existing risks, and an analysis of organizational security controls.

The overall security posture presents a mixed landscape. The organization demonstrates a solid foundation in policy and employee training. A significant positive finding is the successful remediation of a previously identified risk, "Unencrypted Web Server," confirmed by our network scan which found port 80 to be closed.

However, critical gaps were identified in access control measures. The absence of Multi-Factor Authentication (MFA) for logging into computers and accessing sensitive data systems constitutes a high-priority risk. These gaps expose the organization to significant threats, including unauthorized access, credential theft, and potential data breaches.

Immediate action is recommended to implement and enforce MFA across all critical systems to mitigate these risks and strengthen the organization's defense-in-depth strategy.

% --- Section 2: Organizational Information ---
\section{Organizational Information}
The following details were provided for the assessment.

\begin{tabular}{@{}ll}
\toprule
\textbf{Attribute} & \textbf{Value} \\
\midrule
Organization Name & \textbf{Solid State} \\
Email Domain & \texttt{SolidState.net} \\
Website Domain & \url{www.SolidState.net} \\
External IP Address & \texttt{97.39.228.225} \\
\bottomrule
\end{tabular}

% --- Section 3: Security Control Review ---
\section{Security Control Review}
An assessment of administrative and technical security controls was conducted based on a standardized questionnaire. The responses indicate a strong policy framework but highlight critical deficiencies in access control enforcement.

\begin{table}[h!]
\centering
\caption{Security Controls Questionnaire Analysis}
\begin{tabular}{@{}p{0.8\linewidth}c@{}}
\toprule
\textbf{Control Question} & \textbf{Response} \\
\midrule
Do you require MFA to access email? & \cmark \\
Do you require MFA to log into computers? & \xmark \\
Do you require MFA to access sensitive data systems? & \xmark \\
Does your organization have an employee acceptable use policy? & \cmark \\
Does your organization do security awareness training for new employees? & \cmark \\
Does your organization do security awareness training for all employees at least once per year? & \cmark \\
\bottomrule
\end{tabular}
\end{table}

\paragraph{Analysis:} The two controls marked with an \xmark{} represent significant security gaps. The lack of MFA on workstations and sensitive data systems drastically increases the risk of a successful breach via compromised credentials. While strong policies and training are in place, they are not a substitute for robust technical access controls.

% --- Section 4: Technical Scan Results ---
\section{Technical Scan Results}
A network scan was performed to identify open ports and exposed services on the target system.

\begin{itemize}
    \item \textbf{Target IP Address:} \texttt{192.168.0.5}
    \item \textbf{Scan Date:} \today
\end{itemize}

The scan revealed no open ports on the target host, which is a positive security finding. The detailed results are presented below.

\begin{table}[h!]
\centering
\caption{Nmap Scan Port Summary}
\begin{tabular}{@{}cccc@{}}
\toprule
\textbf{Port} & \textbf{State} & \textbf{Service} & \textbf{Product / Version} \\
\midrule
80/tcp & closed & http & N/A \\
\bottomrule
\end{tabular}
\end{table}

\paragraph{Analysis:} The scan confirms that the target system has a minimal attack surface from a network perspective. Crucially, the finding that port 80 is \textbf{closed} contradicts a pre-existing risk entry ("Unencrypted Web Server"). This indicates that the previously identified vulnerability has been successfully remediated.

% --- Section 5: Consolidated Risk Assessment ---
\section{Consolidated Risk Assessment}
This section correlates findings from all data sources to provide a unified view of the current risk landscape.

\begin{table}[h!]
\centering
\caption{Summary of Identified Risks}
\begin{tabular}{@{}p{0.2\linewidth}p{0.2\linewidth}p{0.5\linewidth}@{}}
\toprule
\textbf{Risk Name} & \textbf{Severity} & \textbf{Description} \\
\midrule
\textbf{Lack of MFA on Sensitive Systems} & \textbf{Critical} & Failure to enforce MFA on systems containing sensitive data creates a direct path for attackers to access high-value assets, potentially leading to a major data breach. \\
\addlinespace
\textbf{Lack of MFA on Workstations} & \textbf{High} & Without MFA, compromised employee credentials can be used to gain initial access to the internal network, establishing a foothold for further attacks like ransomware deployment. \\
\addlinespace
Unencrypted Web Server & Remediated & A previously documented risk related to an open port 80 is now resolved. The port is confirmed to be closed, mitigating the threat of unencrypted data transmission. \\
\bottomrule
\end{tabular}
\end{table}

% --- Section 6: Recommendations ---
\section{Recommendations}
Based on the consolidated risk assessment, the following actions are recommended to improve the security posture of \textbf{Solid State}. Recommendations are prioritized by severity.

\subsection*{Priority 1 (Critical): Implement MFA on Sensitive Systems}
\begin{itemize}
    \item \textbf{Action:} Immediately begin the process of deploying a mandatory MFA solution for all systems identified as containing or providing access to sensitive data.
    \item \textbf{Justification:} This is the most critical vulnerability. Protecting sensitive data is paramount to preventing significant financial and reputational damage from a data breach.
    \item \textbf{Suggested Solutions:} Consider solutions like Time-based One-Time Passwords (TOTP), hardware security keys (FIDO2/WebAuthn), or push-based authentication apps.
\end{itemize}

\subsection*{Priority 2 (High): Enforce MFA for Workstation Logins}
\begin{itemize}
    \item \textbf{Action:} Roll out mandatory MFA for all employee and privileged user logins to company workstations and laptops.
    \item \textbf{Justification:} Workstations are the primary entry point for many cyberattacks. Securing them with MFA prevents attackers from leveraging stolen credentials to gain initial access to the corporate network.
    \item \textbf{Suggested Solutions:} Utilize built-in platform features (e.g., Windows Hello for Business) or third-party identity providers that integrate with endpoint operating systems.
\end{itemize}

\subsection*{Priority 3 (Informational): Update Risk Register}
\begin{itemize}
    \item \textbf{Action:} Formally update the internal risk register to mark the "Unencrypted Web Server" risk as "Remediated" or "Closed".
    \item \textbf{Justification:} Acknowledging and documenting successful remediation efforts is a key component of a mature risk management program. This ensures that resources are focused on current, active threats.
\end{itemize}

\end{document}
```