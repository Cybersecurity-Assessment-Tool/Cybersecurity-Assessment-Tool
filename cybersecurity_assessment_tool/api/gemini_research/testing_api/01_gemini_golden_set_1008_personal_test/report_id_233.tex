```latex
\documentclass[12pt]{article}

% --- PACKAGES ---
\usepackage[margin=1in]{geometry}
\usepackage{pifont} % For checkmarks and crosses
\usepackage{booktabs} % For professional tables
\usepackage{hyperref} % For clickable links
\usepackage{url} % For URL formatting
\usepackage{seqsplit} % To split long strings like IPs/hashes
\usepackage{graphicx}
\usepackage{xcolor}

% --- DOCUMENT SETUP ---
\hypersetup{
    colorlinks=true,
    linkcolor=blue,
    filecolor=magenta,      
    urlcolor=cyan,
    pdftitle={Cybersecurity Assessment Report},
    pdfpagemode=FullScreen,
}

% --- TITLE ---
\title{Cybersecurity Assessment Report \\ \large For: \textbf{Radiant Life}}
\author{Cybersecurity Analysis Division}
\date{\today}

% --- BEGIN DOCUMENT ---
\begin{document}

\maketitle
\thispagestyle{empty}
\newpage
\tableofcontents
\newpage

% ==============================================================================
\section*{1. Executive Overview}
% ==============================================================================

This report details the findings of a cybersecurity assessment conducted for \textbf{Radiant Life}. The analysis correlates results from an external network scan, a review of organizational security controls, and a list of pre-existing risks.

The assessment identified a critical and immediate risk: the exposure of a Remote Desktop Protocol (RDP) service on host \seqsplit{\texttt{10.10.10.51}}. This finding is particularly alarming as it indicates a systemic issue, echoing a previously identified risk on a different host. This technical vulnerability is severely compounded by critical gaps in organizational security controls. Specifically, the lack of mandatory Multi-Factor Authentication (MFA) for accessing email and sensitive data systems creates a significant opportunity for unauthorized access should credentials be compromised.

Furthermore, foundational security practices, such as a formal Acceptable Use Policy and security awareness training for new employees, are absent. This combination of technical exposure and policy deficiencies places the organization at a high risk of a security breach, including ransomware attacks and data exfiltration. Immediate remediation is strongly advised.

% ==============================================================================
\section*{2. Organizational Information}
% ==============================================================================

The following information was provided for the assessment.

\begin{itemize}
    \item \textbf{Organization Name:} Radiant Life
    \item \textbf{Email Domain:} \texttt{RadiantLife.com}
    \item \textbf{Website Domain:} \url{www.RadiantLife.com}
    \item \textbf{External IP Address:} \seqsplit{\texttt{239.183.211.152}}
\end{itemize}

% ==============================================================================
\section*{3. Security Control Review}
% ==============================================================================

A review of the organization's security controls was conducted via a questionnaire. The responses highlight critical areas for improvement, particularly concerning access control and employee security awareness. A "No" response indicates a significant gap in the organization's defense-in-depth strategy.

\begin{table}[h!]
\centering
\caption{Organizational Security Control Questionnaire}
\begin{tabular}{p{0.75\linewidth} c}
\toprule
\textbf{Control Question} & \textbf{Response} \\
\midrule
Do you require MFA to access email? & \ding{55} \\
Do you require MFA to log into computers? & \ding{51} \\
Do you require MFA to access sensitive data systems? & \ding{55} \\
Does your organization have an employee acceptable use policy? & \ding{55} \\
Does your organization do security awareness training for new employees? & \ding{55} \\
Does your organization do security awareness training for all employees at least once per year? & \ding{51} \\
\bottomrule
\end{tabular}
\\
\vspace{0.2cm}
\textit{Key: \ding{51} = Yes (Control in place), \ding{55} = No (Control gap)}
\end{table}

% ==============================================================================
\section*{4. Technical Scan Results}
% ==============================================================================

An Nmap scan was performed on the target host to identify open ports and exposed services. The scan revealed a service that is a common target for threat actors.

\begin{itemize}
    \item \textbf{Target IP Address:} \seqsplit{\texttt{10.10.10.51}}
    \item \textbf{Host Status:} Up
\end{itemize}

\begin{table}[h!]
\centering
\caption{Open Ports and Services on \seqsplit{\texttt{10.10.10.51}}}
\begin{tabular}{l l l l}
\toprule
\textbf{Port} & \textbf{State} & \textbf{Service Name} & \textbf{Analysis} \\
\midrule
3389/tcp & Open & \texttt{ms-wbt-server} & Critical Risk. This is the port for RDP. \\
& & & Exposing RDP directly to the internet is a \\
& & & primary vector for ransomware attacks. \\
\bottomrule
\end{tabular}
\end{table}

% ==============================================================================
\section*{5. Consolidated Risk Assessment}
% ==============================================================================

This section synthesizes findings from the technical scan, the control review, and pre-existing risk data. The correlation of these data points reveals a heightened risk posture.

\begin{table}[h!]
\centering
\caption{Summary of Identified Risks}
\begin{tabular}{p{0.2\linewidth} p{0.55\linewidth} p{0.15\linewidth}}
\toprule
\textbf{Risk/Finding} & \textbf{Description} & \textbf{Severity} \\
\midrule
\textbf{Systemic RDP Exposure} & The scan identified RDP open on \seqsplit{\texttt{10.10.10.51}}. This adds to a known, pre-existing RDP exposure on \seqsplit{\texttt{10.10.10.50}}. This pattern indicates a systemic failure in network security management. & \textbf{Critical} \\
\addlinespace
\textbf{Lack of MFA on Critical Systems} & MFA is not enforced on email or sensitive data systems. This allows an attacker with compromised credentials (e.g., from a phishing attack) to gain direct access to key organizational assets. & \textbf{Critical} \\
\addlinespace
\textbf{Inadequate Security Policies \& Training} & The absence of an Acceptable Use Policy and security training for new hires means employees are not equipped with the foundational knowledge to protect company assets, increasing the likelihood of security incidents. & \textbf{High} \\
\bottomrule
\end{tabular}
\end{table}

% ==============================================================================
\section*{6. Recommendations}
% ==============================================================================

Based on the findings, the following actions are recommended to mitigate the identified risks and improve the overall security posture of \textbf{Radiant Life}. Recommendations are prioritized by urgency.

\begin{enumerate}
    \item \textbf{[Immediate] Remediate All RDP Exposures:}
    \begin{itemize}
        \item Immediately close port 3389 on all systems where it is exposed externally, including \seqsplit{\texttt{10.10.10.51}} and \seqsplit{\texttt{10.10.10.50}}.
        \item For necessary remote access, implement a secure solution such as a Virtual Private Network (VPN) or a dedicated RDP Gateway that requires MFA.
    \end{itemize}

    \item \textbf{[High Priority] Implement Comprehensive MFA:}
    \begin{itemize}
        \item Enforce MFA for all user accounts on the email system (\texttt{RadiantLife.com}).
        \item Enforce MFA for access to all systems identified as containing sensitive data. This is the single most effective control to prevent unauthorized access via stolen credentials.
    \end{itemize}

    \item \textbf{[Medium Priority] Develop Foundational Security Policies:}
    \begin{itemize}
        \item Create and implement a formal Employee Acceptable Use Policy (AUP). This policy should clearly define the rules for using company IT assets.
        \item Ensure all current and new employees read and formally acknowledge the AUP.
    \end{itemize}

    \item \textbf{[Medium Priority] Enhance Security Awareness Program:}
    \begin{itemize}
        \item Integrate mandatory security awareness training into the onboarding process for all new employees.
        \item Continue the annual refresher training, but ensure it covers current threats such as phishing, social engineering, and the importance of MFA.
    \end{itemize}
\end{enumerate}

\end{document}
```