```latex
\documentclass[12pt]{article}

% Preamble: Required Packages
\usepackage[margin=1in]{geometry}
\usepackage{pifont} % For checkmarks and crosses
\usepackage{booktabs} % For professional tables
\usepackage{graphicx}
\usepackage{xcolor}
\usepackage{hyperref} % For clickable links
\usepackage{url} % For formatting URLs
\usepackage{seqsplit} % To split long strings in tt font

% Document Metadata
\title{Cybersecurity Posture Assessment Report \\ \large For: \textbf{Neon Pulse Entertainment}}
\author{Cybersecurity Analysis Division}
\date{\today}

% Hyperref Setup
\hypersetup{
    colorlinks=true,
    linkcolor=blue,
    filecolor=magenta,      
    urlcolor=cyan,
    pdftitle={Cybersecurity Posture Assessment Report},
    pdfpagemode=FullScreen,
}

\begin{document}

\maketitle
\thispagestyle{empty}
\newpage

\tableofcontents
\newpage

% --- Section 1: Executive Summary ---
\section{Executive Summary}

This report provides a comprehensive analysis of the cybersecurity posture for \textbf{Neon Pulse Entertainment}, based on network scan data, organizational security controls, and pre-existing risk information. The assessment reveals several critical and high-risk vulnerabilities that require immediate attention to mitigate the threat of unauthorized access, data breach, and ransomware attacks.

The key findings indicate a pattern of insecure remote access configurations combined with significant gaps in endpoint and personnel security controls. Specifically, the analysis identified:

\begin{itemize}
    \item \textbf{Systemic RDP Exposure:} The technical scan confirmed an open Remote Desktop Protocol (RDP) port on a new system (\texttt{10.10.10.51}), corroborating a pre-existing risk of RDP exposure on another host (\texttt{10.10.10.50}). This indicates a systemic issue rather than an isolated misconfiguration.
    \item \textbf{Critical MFA Gap:} Multi-Factor Authentication (MFA) is not required for computer logins. This dramatically increases the risk of a successful breach if user credentials are compromised, as it provides a direct path for lateral movement within the network.
    \item \textbf{Inadequate Security Onboarding:} New employees do not receive security awareness training upon being hired. This makes them highly susceptible to phishing and social engineering attacks, which are primary vectors for initial credential theft.
\end{itemize}

The combination of exposed RDP, lack of endpoint MFA, and insufficient employee training creates a high-likelihood, high-impact attack path. An adversary could exploit these weaknesses to gain initial access, escalate privileges, and deploy ransomware. Recommendations in this report prioritize the immediate remediation of these interconnected risks.

% --- Section 2: Organizational Information ---
\section{Organizational Information}

The following information was provided for the assessment.

\begin{tabular}{@{}ll}
    \toprule
    \textbf{Attribute} & \textbf{Value} \\
    \midrule
    Organization Name & \textbf{Neon Pulse Entertainment} \\
    Email Domain & \seqsplit{\texttt{NeonPulseEntertainment.org}} \\
    Website Domain & \href{http://www.NeonPulseEntertainment.org}{\seqsplit{\texttt{www.NeonPulseEntertainment.org}}} \\
    External IP Address & \texttt{232.193.177.62} \\
    \bottomrule
\end{tabular}

% --- Section 3: Security Control Review ---
\section{Security Control Review}

A review of the organization's security controls was conducted based on a standardized questionnaire. The responses highlight critical gaps in endpoint security and employee training protocols. A summary of the findings is presented in Table \ref{tab:controls}.

\begin{table}[h!]
    \centering
    \caption{Organizational Security Control Questionnaire}
    \label{tab:controls}
    \begin{tabular}{@{}lc}
        \toprule
        \textbf{Control Question} & \textbf{Response} \\
        \midrule
        Do you require MFA to access email? & \textcolor{green}{\ding{51}} \\
        Do you require MFA to log into computers? & \textcolor{red}{\ding{55}} \\
        Do you require MFA to access sensitive data systems? & \textcolor{green}{\ding{51}} \\
        Does your organization have an employee acceptable use policy? & \textcolor{green}{\ding{51}} \\
        Does your organization do security awareness training for new employees? & \textcolor{red}{\ding{55}} \\
        Does your organization do security awareness training for all employees annually? & \textcolor{green}{\ding{51}} \\
        \bottomrule
    \end{tabular}
\end{table}

\subsection*{Analysis of Control Gaps}
\begin{itemize}
    \item \textbf{No MFA for Computer Logins (Critical Risk):} The absence of MFA on endpoints is a severe security flaw. If an attacker obtains valid user credentials through phishing or other means, they can log into a company computer without a second authentication factor, gaining a significant foothold on the internal network.
    \item \textbf{No Security Training for New Employees (High Risk):} New hires are often prime targets for social engineering attacks as they are unfamiliar with company policies and personnel. Failing to provide immediate security training leaves a critical window of vulnerability.
\end{itemize}

% --- Section 4: Technical Scan Results ---
\section{Technical Scan Results}

A network scan was performed to identify open ports and exposed services on the target system.

\begin{itemize}
    \item \textbf{Target IP Address:} \texttt{10.10.10.51}
\end{itemize}

The scan revealed the following open port, as detailed in Table \ref{tab:scan}.

\begin{table}[h!]
    \centering
    \caption{Open Port Findings for \texttt{10.10.10.51}}
    \label{tab:scan}
    \begin{tabular}{@{}llll}
        \toprule
        \textbf{Port} & \textbf{State} & \textbf{Service Name} & \textbf{Analysis} \\
        \midrule
        3389/tcp & open & \texttt{ms-wbt-server} & Microsoft Remote Desktop Protocol (RDP). \\
        \bottomrule
    \end{tabular}
\end{table}

\subsection*{Analysis of Technical Findings}
The discovery of an open RDP port on \texttt{10.10.10.51} is a significant finding. RDP is a primary target for attackers seeking to gain remote access to a network. When exposed, it is vulnerable to brute-force password attacks, credential stuffing, and exploitation of known vulnerabilities (e.g., BlueKeep). This finding, combined with the pre-existing risk on host \texttt{10.10.10.50}, points to a systemic lack of network hardening.

% --- Section 5: Correlated Risk Assessment ---
\section{Correlated Risk Assessment}

This section synthesizes findings from the security control review, technical scan, and pre-existing risk data to provide a holistic view of the primary threats.

\begin{table}[h!]
    \centering
    \caption{Summary of Key Correlated Risks}
    \label{tab:risks}
    \begin{tabular}{@{}p{0.2\textwidth}p{0.55\textwidth}p{0.15\textwidth}}
        \toprule
        \textbf{Risk Title} & \textbf{Description} & \textbf{Severity} \\
        \midrule
        \textbf{Systemic RDP Exposure} & RDP (port 3389) is confirmed to be open on multiple internal systems (\texttt{10.10.10.50}, \texttt{10.10.10.51}). This configuration is a common vector for ransomware and unauthorized access. & \textbf{Critical} \\
        \addlinespace
        \textbf{Compromised Credentials via Unenforced MFA} & The lack of MFA on computer logins, coupled with exposed RDP, creates a direct path for an attacker with stolen credentials to gain remote control of an internal workstation and move laterally. & \textbf{Critical} \\
        \addlinespace
        \textbf{Vulnerable Onboarding Process} & The absence of security awareness training for new employees makes them highly susceptible to phishing attacks aimed at stealing the very credentials needed to exploit the RDP and MFA weaknesses. & \textbf{High} \\
        \bottomrule
    \end{tabular}
\end{table}

% --- Section 6: Recommendations ---
\section{Recommendations}

The following actionable recommendations are provided to address the identified risks. They are prioritized based on severity and impact.

\subsection{Immediate Actions (Critical Priority)}
\begin{enumerate}
    \item \textbf{Remediate RDP Exposure:}
    \begin{itemize}
        \item Immediately close TCP port 3389 on \texttt{10.10.10.51} and any other systems where it is not strictly required for business operations.
        \item For necessary remote access, implement a secure solution such as a Virtual Private Network (VPN) or a Zero Trust Network Access (ZTNA) gateway. All RDP traffic should be routed through this secure channel, never exposed directly.
    \end{itemize}
    \item \textbf{Enforce MFA on All Endpoints:}
    \begin{itemize}
        \item Deploy and mandate the use of MFA for all computer (workstation and server) login attempts.
        \item Prioritize deployment for accounts with administrative privileges and remote access capabilities.
    \end{itemize}
\end{enumerate}

\subsection{High Priority Actions}
\begin{enumerate}
    \setcounter{enumi}{2} % Continue numbering from previous list
    \item \textbf{Integrate Security into Employee Onboarding:}
    \begin{itemize}
        \item Develop and mandate a security awareness training module as a required part of the new employee onboarding process.
        \item This training should cover phishing, acceptable use, password hygiene, and how to report security incidents.
    \end{itemize}
\end{enumerate}

\subsection{Strategic Recommendations}
\begin{enumerate}
    \setcounter{enumi}{3}
    \item \textbf{Conduct Comprehensive Vulnerability Scanning:}
    \begin{itemize}
        \item Perform a full internal and external vulnerability scan across the entire network to identify and catalog any other exposed services or unpatched systems.
    \end{itemize}
    \item \textbf{Review and Formalize Remote Access Policies:}
    \begin{itemize}
        \item Update the acceptable use policy to include strict rules for remote access, explicitly prohibiting direct RDP exposure and mandating the use of the approved secure access solution (e.g., VPN).
    \end{itemize}
\end{enumerate}

\end{document}
```