```latex
\documentclass[12pt]{article}

% Preamble: Required Packages
\usepackage[margin=1in]{geometry}
\usepackage{pifont} % For checkmarks and crosses
\usepackage{booktabs} % For professional tables
\usepackage{hyperref} % For clickable links
\usepackage{url} % For formatting URLs
\usepackage{seqsplit} % For splitting long strings without spaces
\usepackage{xcolor} % For coloring text
\usepackage{graphicx} % For potential logos/images
\usepackage{fancyhdr} % For headers and footers

% --- Document Setup ---
\hypersetup{
    colorlinks=true,
    linkcolor=blue,
    filecolor=magenta,      
    urlcolor=cyan,
    pdftitle={Cybersecurity Risk Assessment Report},
    pdfpagemode=FullScreen,
}

% Define colors for risk levels
\definecolor{critical}{RGB}{192,0,0}
\definecolor{high}{RGB}{255,128,0}
\definecolor{medium}{RGB}{255,192,0}
\definecolor{low}{RGB}{0,176,80}

% --- Header and Footer ---
\pagestyle{fancy}
\fancyhf{}
\lhead{Cybersecurity Risk Assessment}
\rhead{Terraform Global}
\cfoot{\thepage}

% --- Document Start ---
\begin{document}

% --- Title Page ---
\begin{titlepage}
    \centering
    \vspace*{1cm}
    \Huge\textbf{Cybersecurity Risk Assessment Report}
    \vspace{1.5cm}
    \Large
    \textbf{Prepared for:} \\
    Terraform Global
    \vspace{2cm}
    \large
    \textbf{Date of Report:} \\
    \today
    \vspace{2cm}
    \textbf{Generated by:} \\
    Expert-Level Cybersecurity Analyst
    \vfill
    \textit{This report contains sensitive information and should be handled with the utmost confidentiality.}
\end{titlepage}

\tableofcontents
\newpage

% --- Section 1: Executive Summary ---
\section{Executive Summary}
This report provides a comprehensive cybersecurity assessment for Terraform Global, based on network scans, organizational data, and a review of existing risk documentation. The analysis reveals a \textbf{\textcolor{critical}{CRITICAL}} overall risk posture, driven by significant deficiencies in fundamental security controls and a newly discovered, high-impact technical vulnerability.

Key findings indicate a complete absence of Multi-Factor Authentication (MFA) across all critical systems, including email, computer logins, and sensitive data repositories. This is compounded by a lack of a formal security awareness training program for employees.

Most alarmingly, a network scan identified an open service on port \texttt{8080} of host \texttt{10.5.5.5}, which presents itself as a \textbf{"TOP SECRET DB"}. This finding directly contradicts the existing risk register, which incorrectly lists this port as secure. This suggests a potentially exposed sensitive database and a flawed risk management process.

Immediate remediation is required to address the exposed service and to implement foundational security controls like MFA and employee training to mitigate the severe risk of a data breach.

% --- Section 2: Organizational Information ---
\section{Organizational Information}
The following details were provided for the assessment.

\begin{tabular}{@{}ll}
\toprule
\textbf{Attribute} & \textbf{Value} \\
\midrule
Organization Name & Terraform Global \\
Email Domain & \texttt{TerraformGlobal.net} \\
Website Domain & \url{www.TerraformGlobal.net} \\
External IP Address & \texttt{172.53.173.137} \\
\bottomrule
\end{tabular}

% --- Section 3: Security Control Review ---
\section{Security Control Review}
An analysis of the organization's security questionnaire reveals critical gaps in administrative and access controls. The following table summarizes the responses and provides an initial assessment of the associated risk.

\begin{tabular}{@{}p{0.6\linewidth} c p{0.2\linewidth}@{}}
\toprule
\textbf{Control Question} & \textbf{Response} & \textbf{Assessment} \\
\midrule
Do you require MFA to access email? & \ding{55} & \textcolor{critical}{\textbf{Critical Gap}} \\
Do you require MFA to log into computers? & \ding{55} & \textcolor{critical}{\textbf{Critical Gap}} \\
Do you require MFA to access sensitive data systems? & \ding{55} & \textcolor{critical}{\textbf{Critical Gap}} \\
Does your organization have an employee acceptable use policy? & \ding{51} & \textcolor{low}{Control in Place} \\
Does your organization do security awareness training for new employees? & \ding{55} & \textcolor{high}{\textbf{High Risk}} \\
Does your organization do security awareness training for all employees at least once per year? & \ding{55} & \textcolor{high}{\textbf{High Risk}} \\
\bottomrule
\end{tabular}

\vspace{1em}
The lack of MFA is a severe deficiency, as it removes a critical layer of defense against credential theft and unauthorized access. The absence of security awareness training leaves the organization highly vulnerable to phishing and social engineering attacks.

% --- Section 4: Technical Scan Results ---
\section{Technical Scan Results}
A network scan was performed to identify open ports and exposed services on the target system.

\begin{itemize}
    \item \textbf{Target IP Address:} \texttt{10.5.5.5}
    \item \textbf{Scan Date:} \today
\end{itemize}

\begin{tabular}{@{}llll@{}}
\toprule
\textbf{Port} & \textbf{State} & \textbf{Service/Product} & \textbf{Notes} \\
\midrule
8080/tcp & Open & http & \textbf{Service Title: "TOP SECRET DB"} \\
\bottomrule
\end{tabular}

\subsection{Analysis of Findings}
The scan identified a single open port, \texttt{8080}, running an HTTP service. The title of the web page served on this port is \textbf{"TOP SECRET DB"}. This is a critical finding that strongly indicates a sensitive, and likely unencrypted, database is directly exposed on the network.

This technical finding directly contradicts the information in the current risk register (\textit{Input\_3\_Current\_Risks\_JSON}), which states that port 8080 is secure and a false positive. This discrepancy highlights a significant failure in the existing risk validation and management process.

% --- Section 5: Consolidated Risk Assessment ---
\section{Consolidated Risk Assessment}
The following table synthesizes findings from the security questionnaire, technical scans, and existing risk data into a prioritized list of identified risks.

\begin{tabular}{@{}p{0.1\linewidth} p{0.4\linewidth} p{0.2\linewidth} p{0.2\linewidth}@{}}
\toprule
\textbf{Risk ID} & \textbf{Description} & \textbf{Severity} & \textbf{Source} \\
\midrule
RISK-001 & \textbf{Potentially Exposed Sensitive Database.} An open service on port 8080 is titled "TOP SECRET DB," suggesting an unsecured database is accessible on the network. & \textcolor{critical}{\textbf{Critical}} & Network Scan \\
\addlinespace
RISK-002 & \textbf{No Multi-Factor Authentication (MFA).} Lack of MFA for email, computer, and sensitive data access drastically increases the risk of account compromise and unauthorized access. & \textcolor{critical}{\textbf{Critical}} & Questionnaire \\
\addlinespace
RISK-003 & \textbf{Inadequate Security Awareness Training.} The absence of a training program makes employees susceptible to phishing, malware, and social engineering attacks. & \textcolor{high}{\textbf{High}} & Questionnaire \\
\addlinespace
RISK-004 & \textbf{Flawed Risk Management Process.} The existing risk register incorrectly identified port 8080 as secure, proving the current process for tracking and validating risks is unreliable. & \textcolor{high}{\textbf{High}} & Data Correlation \\
\bottomrule
\end{tabular}

% --- Section 6: Recommendations ---
\section{Recommendations}
Based on the consolidated risk assessment, the following actions are recommended to improve the security posture of Terraform Global. Recommendations are prioritized by severity.

\subsection{Immediate Actions (Priority: Critical)}
\begin{enumerate}
    \item \textbf{Secure Exposed Database (RISK-001):}
    \begin{itemize}
        \item Immediately investigate the service running on \texttt{10.5.5.5:8080}.
        \item If a sensitive database is confirmed, take the service offline or restrict access via firewall rules to only authorized personnel.
        \item Develop a remediation plan to properly secure the database with authentication, encryption, and network segmentation.
    \end{itemize}
    \item \textbf{Implement Multi-Factor Authentication (RISK-002):}
    \begin{itemize}
        \item Procure and deploy an MFA solution immediately.
        \item Prioritize enforcement on all external-facing services, administrator accounts, and access to sensitive data systems.
        \item Create a roadmap to enforce MFA for all employee email and computer logins.
    \end{itemize}
\end{enumerate}

\subsection{Short-Term Actions (Priority: High)}
\begin{enumerate}
    \item \textbf{Establish Security Awareness Training (RISK-003):}
    \begin{itemize}
        \item Implement a mandatory security awareness training program for all new and existing employees.
        \item Conduct annual refresher training and periodic phishing simulations to maintain a high level of security consciousness.
    \end{itemize}
    \item \textbf{Review and Validate Risk Register (RISK-004):}
    \begin{itemize}
        \item Conduct a full review of all entries in the existing risk register.
        \item Validate all technical assumptions with fresh scans and manual verification.
        \item Implement a formal process for periodic risk review and validation.
    \end{itemize}
\end{enumerate}

\end{document}
```