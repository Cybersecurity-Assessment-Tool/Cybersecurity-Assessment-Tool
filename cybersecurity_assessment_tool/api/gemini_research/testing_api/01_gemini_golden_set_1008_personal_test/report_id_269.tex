```latex
\documentclass[12pt]{article}

% ----------------------------------------------------------------------
% METADATA & PACKAGES
% ----------------------------------------------------------------------
\usepackage[margin=1in]{geometry}
\usepackage{pifont} % For checkmarks and crosses
\usepackage{booktabs} % For professional tables
\usepackage{hyperref} % For clickable links
\usepackage{url} % For URL formatting
\usepackage{seqsplit} % To split long strings in texttt
\usepackage{graphicx}
\usepackage[table]{xcolor} % For coloring table cells

% Define severity colors
\definecolor{sev_critical}{HTML}{940000}
\definecolor{sev_high}{HTML}{D14000}
\definecolor{sev_medium}{HTML}{E09100}
\definecolor{sev_low}{HTML}{3A7D00}

% Hyperref setup
\hypersetup{
    colorlinks=true,
    linkcolor=blue,
    filecolor=magenta,      
    urlcolor=cyan,
    pdftitle={Cybersecurity Risk Assessment Report},
    pdfauthor={Cybersecurity Analyst},
    pdfsubject={Risk Assessment},
    pdfkeywords={Security, Risk, Analysis},
    bookmarks=true
}

% ----------------------------------------------------------------------
% DOCUMENT START
% ----------------------------------------------------------------------
\begin{document}

% ----------------------------------------------------------------------
% TITLE PAGE
% ----------------------------------------------------------------------
\title{
    \vspace{2cm}
    \textbf{Cybersecurity Risk Assessment Report} \\
    \large \textit{Analysis of Technical and Organizational Controls} \\
    \vspace{1.5cm}
    \textbf{Prepared for: Firebrand Media}
}
\author{Cybersecurity Analyst}
\date{\today}
\maketitle
\thispagestyle{empty}
\newpage

\tableofcontents
\newpage

% ----------------------------------------------------------------------
% SECTION 1: EXECUTIVE SUMMARY
% ----------------------------------------------------------------------
\section{Executive Summary}

This report provides a comprehensive cybersecurity risk assessment for \textbf{Firebrand Media}, based on an analysis of network scan data, organizational security controls, and pre-existing risk information. The assessment synthesizes technical vulnerabilities with procedural gaps to present a holistic view of the organization's current security posture.

\paragraph{Key Findings:} A network scan identified a server with an exposed Remote Desktop Protocol (RDP) service on port 3389. When correlated with a pre-existing risk of a similar nature on another host, this finding indicates a \textbf{systemic configuration weakness} rather than an isolated incident. This technical vulnerability is significantly amplified by critical gaps in organizational controls, most notably:
\begin{itemize}
    \item The absence of Multi-Factor Authentication (MFA) for computer logins.
    \item A lack of a formal employee Acceptable Use Policy (AUP).
    - The omission of security awareness training for new employees.
\end{itemize}

\paragraph{Overall Posture:} The combination of an externally accessible, high-value service (RDP) with weak authentication and policy controls places \textbf{Firebrand Media} at a \textbf{High Risk} of credential theft, unauthorized access, and potential ransomware attacks. Immediate remediation is required to address the identified critical vulnerabilities. Recommendations are provided in Section \ref{sec:recommendations} to mitigate these risks effectively.

% ----------------------------------------------------------------------
% SECTION 2: ORGANIZATIONAL INFORMATION
% ----------------------------------------------------------------------
\section{Organizational Information}

The following information was provided for the assessment. This data serves as the baseline for identifying the organization's digital footprint and internal policies.

\begin{table}[h!]
\centering
\caption{Client Profile}
\label{tab:org_info}
\begin{tabular}{@{}ll@{}}
\toprule
\textbf{Attribute} & \textbf{Value} \\ \midrule
Organization Name & \textbf{Firebrand Media} \\
Email Domain      & \seqsplit{\texttt{FirebrandMedia.org}} \\
Website Domain    & \seqsplit{\texttt{www.FirebrandMedia.org}} \\
External IP Address & \seqsplit{\texttt{131.101.46.55}} \\ \bottomrule
\end{tabular}
\end{table}

% ----------------------------------------------------------------------
% SECTION 3: SECURITY CONTROL REVIEW
% ----------------------------------------------------------------------
\section{Security Control Review}

A review of the organization's security controls was conducted via a questionnaire. The responses highlight key areas of strength and weakness in the current security framework. Answers marked with \ding{55} (No) are considered significant gaps that increase organizational risk.

\begin{table}[h!]
\centering
\caption{Questionnaire Analysis}
\label{tab:questionnaire}
\renewcommand{\arraystretch}{1.2}
\begin{tabular}{@{}p{8cm}cc@{}}
\toprule
\textbf{Control Question} & \textbf{Response} & \textbf{Assessment} \\ \midrule
Do you require MFA to access email? & Yes (\ding{51}) & Good Practice \\
\rowcolor{red!15}
Do you require MFA to log into computers? & No (\ding{55}) & \textbf{Critical Gap} \\
Do you require MFA to access sensitive data systems? & Yes (\ding{51}) & Good Practice \\
\rowcolor{red!15}
Does your organization have an employee acceptable use policy? & No (\ding{55}) & \textbf{High Risk} \\
\rowcolor{red!15}
Does your organization do security awareness training for new employees? & No (\ding{55}) & \textbf{High Risk} \\
Does your organization do security awareness training for all employees at least once per year? & Yes (\ding{51}) & Good Practice \\ \bottomrule
\end{tabular}
\end{table}

% ----------------------------------------------------------------------
% SECTION 4: TECHNICAL SCAN RESULTS
% ----------------------------------------------------------------------
\section{Technical Scan Results}

An external network scan was performed to identify open ports and exposed services on the target system.

\paragraph{Target Host:} \texttt{10.10.10.51}

\begin{table}[h!]
\centering
\caption{Open Port Analysis for \texttt{10.10.10.51}}
\label{tab:nmap_results}
\begin{tabular}{@{}llll@{}}
\toprule
\textbf{Port} & \textbf{State} & \textbf{Service Name} & \textbf{Analysis} \\ \midrule
\rowcolor{red!15}
3389/tcp & open & \texttt{ms-wbt-server} & \begin{tabular}[c]{@{}l@{}}Microsoft Remote Desktop Protocol (RDP). \\ Exposing RDP directly to the internet is a \\ major security risk and a common vector \\ for ransomware attacks.\end{tabular} \\ \bottomrule
\end{tabular}
\end{table}

% ----------------------------------------------------------------------
% SECTION 5: CONSOLIDATED RISK ASSESSMENT
% ----------------------------------------------------------------------
\section{Consolidated Risk Assessment}

This section synthesizes findings from the security control review, technical scan, and pre-existing risk data. Each identified risk is assigned a severity level to guide prioritization.

\begin{table}[h!]
\centering
\caption{Summary of Identified Risks}
\label{tab:risk_summary}
\renewcommand{\arraystretch}{1.5}
\begin{tabular}{@{}p{1.5cm}p{3cm}p{5.5cm}p{3cm}@{}}
\toprule
\textbf{Severity} & \textbf{Risk Name} & \textbf{Description} & \textbf{Affected Assets} \\ \midrule
\rowcolor{sev_critical!25}
\textbf{\color{sev_critical}CRITICAL} & Systemic RDP Exposure & RDP is exposed on multiple hosts (\texttt{10.10.10.50} and \texttt{10.10.10.51}). This indicates a systemic lack of network hardening, not an isolated misconfiguration. & Internal Servers, Domain Controller, Sensitive Data \\
\rowcolor{sev_high!25}
\textbf{\color{sev_high}HIGH} & Lack of Endpoint MFA & The absence of MFA for computer logins allows an attacker with compromised credentials to gain direct access to endpoints, bypassing a critical security layer. & Employee Workstations, Laptops, User Accounts \\
\rowcolor{sev_medium!25}
\textbf{\color{sev_medium}MEDIUM} & Missing Foundational Policies & The lack of an Acceptable Use Policy and security training for new hires creates an environment where employees are more likely to engage in risky behavior. & All Employees, Organizational Security Culture \\ \bottomrule
\end{tabular}
\end{table}

% ----------------------------------------------------------------------
% SECTION 6: RECOMMENDATIONS
% ----------------------------------------------------------------------
\section{Recommendations}
\label{sec:recommendations}

The following actions are recommended to mitigate the identified risks. They are prioritized based on severity and potential impact.

\subsection{Risk: Systemic RDP Exposure (Critical)}
\begin{itemize}
    \item \textbf{Immediate Action:} Implement firewall rules to block all inbound traffic to TCP port 3389 on hosts \texttt{10.10.10.50}, \texttt{10.10.10.51}, and any other internet-facing systems. Conduct an emergency network-wide scan to identify any other unintentionally exposed services.
    \item \textbf{Long-Term Solution:} Decommission direct RDP access. Implement a secure remote access solution, such as a Virtual Private Network (VPN) or a Zero Trust Network Access (ZTNA) gateway, that requires Multi-Factor Authentication.
\end{itemize}

\subsection{Risk: Lack of Endpoint MFA (High)}
\begin{itemize}
    \item \textbf{Immediate Action:} Procure and deploy an MFA solution for all computer logins. Prioritize deployment for system administrators, privileged users, and employees with remote access capabilities.
    \item \textbf{Long-Term Solution:} Integrate MFA into the standard build process for all new workstations and servers.
\end{itemize}

\subsection{Risk: Missing Foundational Policies (Medium)}
\begin{itemize}
    \item \textbf{Immediate Action:} Develop and implement a formal Acceptable Use Policy (AUP) that all employees must read and acknowledge. This policy should outline the rules for using company IT assets.
    \item \textbf{Long-Term Solution:} Integrate mandatory cybersecurity awareness training into the onboarding process for all new hires. Ensure this training covers key topics such as phishing, password hygiene, and the new AUP.
\end{itemize}

% ----------------------------------------------------------------------
% DOCUMENT END
% ----------------------------------------------------------------------
\end{document}
```