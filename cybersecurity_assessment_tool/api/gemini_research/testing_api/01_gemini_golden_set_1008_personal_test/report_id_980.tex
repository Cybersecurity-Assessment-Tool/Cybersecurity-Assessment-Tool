```latex
\documentclass[12pt]{article}

% Preamble: Required Packages
\usepackage[margin=1in]{geometry}
\usepackage{pifont} % For check marks and crosses
\usepackage{booktabs} % For professional tables
\usepackage{hyperref} % For clickable links
\usepackage{url}      % For proper URL formatting
\usepackage{seqsplit} % For splitting long text strings
\usepackage{graphicx} % For logo (placeholder)
\usepackage{fancyhdr} % For header/footer

% --- Document Metadata ---
\title{Cybersecurity Posture Assessment Report}
\author{Cybersecurity Analysis Division}
\date{\today}

% --- Header and Footer ---
\pagestyle{fancy}
\fancyhf{}
\lhead{Ironclad Logistics}
\rhead{Confidential}
\cfoot{\thepage}

\begin{document}

\begin{center}
    \vspace*{1cm}
    {\Huge \textbf{Cybersecurity Posture Assessment Report}\par}
    \vspace{1.5cm}
    {\Large \textbf{Prepared for:}\\ Ironclad Logistics\par}
    \vspace{2cm}
    {\large \today\par}
    \vfill
    {\large This document is confidential and intended solely for the use of the individual or entity to whom it is addressed.\par}
\end{center}

\newpage
\tableofcontents
\newpage

% --- Section 1: Executive Summary ---
\section*{Executive Summary}

This report provides a comprehensive analysis of the cybersecurity posture for \textbf{Ironclad Logistics}, based on a review of organizational security controls, a technical network scan, and pre-existing risk data.

The assessment reveals several critical and high-risk security gaps that require immediate attention. The most significant findings include a complete absence of Multi-Factor Authentication (MFA) across all critical systems, including email, computer logins, and sensitive data repositories. This exposes the organization to a high likelihood of account compromise through common attacks like phishing and credential stuffing.

Furthermore, foundational security policies, such as an Acceptable Use Policy, are not in place, and mandatory annual security training is not conducted for all employees. These policy and training deficiencies increase the risk of insider threats and human error.

Technically, the network scan identified a web server operating over an unencrypted channel (HTTP on port 80). This practice is obsolete and exposes any data transmitted to or from the server to eavesdropping and modification.

Immediate remediation should focus on implementing MFA, developing core security policies, and securing web communications with encryption. Addressing these vulnerabilities is crucial to safeguarding company data, protecting client information, and reducing the overall risk of a significant cyber incident.

% --- Section 2: Organizational Information ---
\section*{Organizational Information}

The following details were provided for the assessment. This information is used to establish the context and scope of the review.

\begin{table}[h!]
\centering
\begin{tabular}{@{}ll@{}}
\toprule
\textbf{Attribute} & \textbf{Value} \\ \midrule
Organization Name & \textbf{Ironclad Logistics} \\
Email Domain & \texttt{IroncladLogistics.com} \\
Website Domain & \url{www.IroncladLogistics.com} \\
External IP Address & \texttt{226.237.31.175} \\ \bottomrule
\end{tabular}
\caption{Client Organizational Details}
\end{table}

% --- Section 3: Security Control Review ---
\section*{Security Control Review}

A review of administrative and technical security controls was conducted via a standardized questionnaire. The responses indicate significant gaps in fundamental security practices. A checkmark (\ding{51}) indicates a positive control is in place, while a cross (\ding{55}) indicates a control gap.

\begin{table}[h!]
\centering
\begin{tabular}{@{}p{0.6\textwidth} c l@{}}
\toprule
\textbf{Control Question} & \textbf{Response} & \textbf{Assessment} \\ \midrule
Do you require MFA to access email? & \ding{55} & \textbf{Critical Gap} \\
Do you require MFA to log into computers? & \ding{55} & \textbf{Critical Gap} \\
Do you require MFA to access sensitive data systems? & \ding{55} & \textbf{Critical Gap} \\
Does your organization have an employee acceptable use policy? & \ding{55} & \textbf{High Risk} \\
Does your organization do security awareness training for new employees? & \ding{51} & Best Practice Met \\
Does your organization do security awareness training for all employees at least once per year? & \ding{55} & \textbf{High Risk} \\ \bottomrule
\end{tabular}
\caption{Analysis of Security Control Questionnaire}
\end{table}

% --- Section 4: Technical Scan Results ---
\section*{Technical Scan Results}

An external network scan was performed to identify open ports and exposed services. The scan provides insight into the organization's external attack surface.

\begin{itemize}
    \item \textbf{Target IP Address:} \texttt{172.16.0.1}
    \item \textbf{Scan Date:} \textit{Not Specified in Scan Data}
\end{itemize}

The following table details the findings from the network scan.

\begin{table}[h!]
\centering
\begin{tabular}{@{}llll@{}}
\toprule
\textbf{Port} & \textbf{State} & \textbf{Service (Inferred)} & \textbf{Notes} \\ \midrule
80/tcp & Open & HTTP & Unencrypted web traffic. This is a significant security risk, \\
& & & exposing data to interception and man-in-the-middle attacks. \\
& & & All web traffic should be encrypted using HTTPS (port 443). \\ \bottomrule
\end{tabular}
\caption{Open Ports and Services Detected}
\end{table}

\textit{Note: The provided risk data in Input 3 contained a non-actionable, invalid entry and was disregarded as it did not represent a legitimate security risk.}

% --- Section 5: Consolidated Risk Assessment ---
\section*{Consolidated Risk Assessment}

The following table synthesizes findings from the security control review and technical scan into a prioritized list of risks.

\begin{table}[h!]
\centering
\begin{tabular}{@{}p{0.25\textwidth} p{0.55\textwidth} l@{}}
\toprule
\textbf{Risk Title} & \textbf{Description} & \textbf{Severity} \\ \midrule
\textbf{Widespread Lack of MFA} & The absence of MFA for email, computer, and sensitive data access creates a single point of failure for authentication. A compromised password directly leads to an account takeover. & \textbf{Critical} \\
\addlinespace
\textbf{Inadequate Security Policies \& Training} & The lack of an Acceptable Use Policy and mandatory annual security training for all staff leads to inconsistent security practices and a workforce that is more susceptible to social engineering and phishing attacks. & \textbf{High} \\
\addlinespace
\textbf{Unencrypted Web Communications} & The use of HTTP on port 80 transmits all data, including potential login credentials or sensitive information, in cleartext. This data can be easily intercepted by an attacker on the same network. & \textbf{Medium} \\ \bottomrule
\end{tabular}
\caption{Summary of Identified Risks}
\end{table}

% --- Section 6: Recommendations ---
\section*{Recommendations}

Based on the analysis, we recommend the following actions, prioritized by severity, to mitigate the identified risks and improve the overall security posture of \textbf{Ironclad Logistics}.

\subsection*{Priority 1: Critical Risk Mitigation}
\begin{itemize}
    \item \textbf{Implement MFA Immediately:} Deploy MFA across all user accounts and systems. Prioritize the following order:
    \begin{enumerate}
        \item Email systems (e.g., Office 365, Google Workspace).
        \item Access to sensitive data systems and applications.
        \item Workstation and server logins.
    \end{enumerate}
\end{itemize}

\subsection*{Priority 2: High Risk Mitigation}
\begin{itemize}
    \item \textbf{Develop and Enforce an Acceptable Use Policy (AUP):} Create a formal AUP that clearly defines the rules for using company IT assets, data, and internet access. Require all employees to read and acknowledge the policy.
    \item \textbf{Establish Annual Security Awareness Training:} Implement a mandatory, annual security awareness training program for all employees. This program should cover key topics such as phishing, password security, and social engineering.
\end{itemize}

\subsection*{Priority 3: Medium Risk Mitigation}
\begin{itemize}
    \item \textbf{Secure Web Server with HTTPS:} Migrate the service running on port 80 to use HTTPS (port 443). This involves obtaining and installing an SSL/TLS certificate on the web server.
    \item \textbf{Reconfigure Firewall:} Once HTTPS is confirmed to be working correctly, reconfigure the network firewall to block all incoming traffic on port 80, or implement a permanent redirect from HTTP to HTTPS.
\end{itemize}

\end{document}
```