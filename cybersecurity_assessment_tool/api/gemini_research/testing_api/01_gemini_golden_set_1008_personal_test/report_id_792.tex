```latex
\documentclass[12pt]{article}

% --- Required Packages ---
\usepackage[margin=1in]{geometry}
\usepackage{pifont}         % For checkmarks and crosses (\ding)
\usepackage{booktabs}       % For professional-looking tables
\usepackage{hyperref}       % For clickable links and document metadata
\usepackage{url}            % For formatting URLs
\usepackage{seqsplit}       % For splitting long strings in texttt
\usepackage{graphicx}
\usepackage[table]{xcolor}  % For coloring table cells and rows

% --- Document & Hyperlink Setup ---
\hypersetup{
    colorlinks=true,
    linkcolor=blue,
    filecolor=magenta,
    urlcolor=cyan,
    pdftitle={Cybersecurity Posture Assessment Report},
    pdfpagemode=FullScreen,
}

% --- Color Definitions for Risk Levels ---
\definecolor{criticalred}{RGB}{217, 83, 79}
\definecolor{highorange}{RGB}{240, 173, 78}
\definecolor{mediumyellow}{RGB}{255, 217, 102}
\definecolor{lowgreen}{RGB}{92, 184, 92}
\definecolor{tablehead}{gray}{0.9}

\begin{document}

% --- Title Page ---
\begin{titlepage}
    \centering
    \vspace*{1cm}
    \Huge\textbf{Cybersecurity Posture Assessment Report}
    \vspace{1.5cm}
    \Large\textbf{Prepared for: Opal Sky Media}
    \vspace{2cm}
    \normalsize
    \textbf{Report Date:} \today \\
    \textbf{Analysis Period:} \today
    \vfill
    \large
    \textit{This report contains sensitive information and should be handled with care.}
\end{titlepage}

\tableofcontents
\newpage

% --- Section 1: Executive Summary ---
\section{Executive Summary}
This report provides a cybersecurity posture assessment for Opal Sky Media, based on an analysis of organizational security controls. The assessment reveals critical deficiencies in fundamental security practices, placing the organization at a high risk of compromise from common cyber threats such as phishing, credential theft, and ransomware.

The analysis was significantly limited by corrupted input data for the technical network scan and pre-existing vulnerabilities. Consequently, this report focuses primarily on the gaps identified through the security controls questionnaire.

\textbf{Key Findings:}
\begin{itemize}
    \item \textbf{Critical Lack of Multi-Factor Authentication (MFA):} The absence of MFA for email and computer access represents a severe security gap. This significantly increases the risk of unauthorized access and account takeovers.
    \item \textbf{Absence of Security Policies and Training:} The organization lacks a formal Acceptable Use Policy (AUP) and does not conduct security awareness training for employees. This creates a high-risk environment where employees are more likely to fall victim to social engineering attacks.
    \item \textbf{Incomplete Technical Visibility:} Due to corrupted scan data, there is no visibility into potential vulnerabilities on the organization's external-facing infrastructure.
\end{itemize}

Immediate action is required to address the identified control gaps. The recommendations provided in this report are prioritized to mitigate the most significant risks first.

% --- Section 2: Organizational Information ---
\section{Organizational Information}
The following details were provided for the assessment.

\begin{table}[h!]
\centering
\begin{tabular}{ll}
\toprule
\textbf{Attribute} & \textbf{Value} \\
\midrule
Organization Name & Opal Sky Media \\
Email Domain & \seqsplit{\texttt{OpalSkyMedia.com}} \\
Website Domain & \seqsplit{\url{www.OpalSkyMedia.com}} \\
Known External IP & \seqsplit{\texttt{228.222.193.2}} \\
\bottomrule
\end{tabular}
\caption{Client Organizational Data}
\label{tab:org_info}
\end{table}

% --- Section 3: Security Control Review ---
\section{Security Control Review}
The following table summarizes the responses from the organizational security questionnaire. "No" answers indicate significant gaps in the security posture and are highlighted for immediate attention.

\begin{table}[h!]
\centering
\rowcolors{2}{gray!10}{white}
\begin{tabular}{p{0.6\textwidth} c p{0.2\textwidth}}
\toprule
\rowcolor{tablehead}
\textbf{Control Question} & \textbf{Response} & \textbf{Assessment} \\
\midrule
Do you require MFA to access sensitive data systems? & \ding{51} (Yes) & Meets Best Practice \\
Do you require MFA to access email? & \ding{55} (No) & \textbf{Critical Gap} \\
Do you require MFA to log into computers? & \ding{55} (No) & \textbf{Critical Gap} \\
Does your organization have an employee acceptable use policy? & \ding{55} (No) & High Risk \\
Does your organization do security awareness training for new employees? & \ding{55} (No) & High Risk \\
Does your organization do security awareness training for all employees at least once per year? & \ding{55} (No) & High Risk \\
\bottomrule
\end{tabular}
\caption{Security Questionnaire Analysis}
\label{tab:questionnaire}
\end{table}

% --- Section 4: Technical Scan Results ---
\section{Technical Scan Results}
\subsection{Data Integrity Issue}
The provided network scan data (\texttt{Input\_1\_Network\_Scan\_JSON}) for target \texttt{[Target IP]} was found to be corrupted and could not be parsed. As a result, a technical analysis of open ports, running services, and potential software vulnerabilities could not be performed.

This represents a significant gap in visibility into the organization's external attack surface. Without this data, it is impossible to assess risks related to unpatched software, insecure service configurations, or unnecessary open ports.

\subsection{Recommendation for Rescan}
It is strongly recommended to conduct a new network vulnerability scan against the known external IP address (\seqsplit{\texttt{228.222.193.2}}) to gather the necessary technical data for a comprehensive assessment.

% --- Section 5: Risk Assessment ---
\section{Risk Assessment}
The pre-existing risk data (\texttt{Input\_3\_Current\_Risks\_JSON}) was also unavailable for this assessment. Therefore, the following table details the risks identified solely from the Security Control Review. The severity level is determined by the potential impact on the organization's confidentiality, integrity, and availability.

\begin{table}[h!]
\centering
\begin{tabular}{p{0.1\textwidth} p{0.25\textwidth} p{0.15\textwidth} p{0.4\textwidth}}
\toprule
\rowcolor{tablehead}
\textbf{Risk ID} & \textbf{Risk Name} & \textbf{Severity} & \textbf{Description} \\
\midrule
RISK-001 & Lack of MFA on Email & \cellcolor{criticalred!80}\textbf{Critical} & Without MFA, email accounts are vulnerable to takeover via stolen or weak passwords, leading to data breaches and business email compromise (BEC). \\
\addlinespace
RISK-002 & Lack of MFA on Endpoints & \cellcolor{criticalred!80}\textbf{Critical} & Compromised user credentials could allow an attacker to log directly into company computers, enabling lateral movement and ransomware deployment. \\
\addlinespace
RISK-003 & Absence of Acceptable Use Policy (AUP) & \cellcolor{highorange!80}\textbf{High} & Without a formal AUP, there are no clear guidelines for employees on safe technology use, increasing the likelihood of risky behavior and insider threats. \\
\addlinespace
RISK-004 & Insufficient Security Awareness Training & \cellcolor{highorange!80}\textbf{High} & Employees are not trained to recognize or respond to cyber threats like phishing, making them the weakest link and a primary target for attackers. \\
\bottomrule
\end{tabular}
\caption{Summary of Identified Risks}
\label{tab:risks}
\end{table}

% --- Section 6: Recommendations ---
\section{Recommendations}
The following actionable recommendations are prioritized based on the severity of the identified risks. Implementing these controls will significantly improve the security posture of Opal Sky Media.

\subsection{Priority 1: Critical Risks}
\begin{enumerate}
    \item \textbf{Implement Multi-Factor Authentication (MFA) Immediately:}
    \begin{itemize}
        \item \textbf{Action:} Enforce MFA for all user accounts across all critical systems, starting with email (e.g., Office 365, Google Workspace) and endpoint logins (e.g., Windows Hello, Duo).
        \item \textbf{Justification:} This is the single most effective control to prevent unauthorized access resulting from compromised credentials. It directly mitigates RISK-001 and RISK-002.
    \end{itemize}
\end{enumerate}

\subsection{Priority 2: High Risks}
\begin{enumerate}
    \setcounter{enumi}{1}
    \item \textbf{Develop and Enforce an Acceptable Use Policy (AUP):}
    \begin{itemize}
        \item \textbf{Action:} Create a formal AUP that clearly defines rules for the use of company assets, data handling, internet usage, and password management. All employees must read and sign this policy.
        \item \textbf{Justification:} An AUP establishes a baseline for secure behavior and provides a framework for enforcing security standards, mitigating RISK-003.
    \end{itemize}
    \item \textbf{Establish a Security Awareness Training Program:}
    \begin{itemize}
        \item \textbf{Action:} Implement a mandatory security awareness training program for all employees. This should include initial training for new hires and at least one annual refresher course. Training should cover phishing, social engineering, password security, and the new AUP.
        \item \textbf{Justification:} A well-trained workforce is a critical layer of defense. This program directly mitigates RISK-004 by empowering employees to identify and report threats.
    \end{itemize}
\end{enumerate}

\subsection{Priority 3: Foundational Improvements}
\begin{enumerate}
    \setcounter{enumi}{3}
    \item \textbf{Conduct a Comprehensive Technical Vulnerability Scan:}
    \begin{itemize}
        \item \textbf{Action:} Perform a new network vulnerability scan of all external-facing assets, including the IP address \seqsplit{\texttt{228.222.193.2}}.
        \item \textbf{Justification:} This action is necessary to gain visibility into technical vulnerabilities that were missed due to the corrupted scan data.
    \end{itemize}
\end{enumerate}

\end{document}
```