```latex
\documentclass[12pt]{article}

% --- PACKAGES ---
\usepackage[margin=1in]{geometry}
\usepackage{pifont} % For checkmarks and crosses
\usepackage{booktabs} % For professional tables
\usepackage{hyperref} % For clickable links
\usepackage{url} % For formatting URLs
\usepackage{seqsplit} % For splitting long strings in tt font
\usepackage{graphicx}
\usepackage{xcolor}

% --- DOCUMENT SETUP ---
\hypersetup{
    colorlinks=true,
    linkcolor=blue,
    filecolor=magenta,      
    urlcolor=cyan,
    pdftitle={Cybersecurity Assessment Report},
    pdfpagemode=FullScreen,
}

\newcommand{\yes}{\ding{51}}
\newcommand{\no}{\ding{55}}

% --- DOCUMENT START ---
\begin{document}

% --- TITLE PAGE ---
\begin{titlepage}
    \centering
    \vspace*{1cm}
    
    \Huge
    \textbf{Cybersecurity Assessment Report}
    
    \vspace{1.5cm}
    
    \Large
    \textbf{Prepared for:} \\
    \vspace{0.5cm}
    Iron River Finance
    
    \vspace{2cm}
    
    \Large
    \textbf{Date of Report:} \\
    \vspace{0.5cm}
    \today
    
    \vfill
    
    \large
    \textit{This report contains sensitive information and is intended solely for the use of the recipient.}
    
\end{titlepage}

\tableofcontents
\newpage

% --- EXECUTIVE SUMMARY ---
\section{Executive Summary}

This report provides a cybersecurity assessment for Iron River Finance, based on a security controls questionnaire, a network vulnerability scan, and a review of pre-existing risks. The assessment identifies the organization's current security posture, highlights key areas of risk, and offers actionable recommendations for improvement.

The primary findings indicate a mixed security posture. While the organization has implemented important controls such as Multi-Factor Authentication (MFA) for computer and sensitive system access, two significant gaps were identified that expose the organization to critical and high levels of risk:

\begin{itemize}
    \item \textbf{Critical Risk - No MFA for Email:} The absence of MFA on email accounts is a critical vulnerability. It significantly increases the risk of Business Email Compromise (BEC), phishing success, and unauthorized access to sensitive data communicated via email.
    \item \textbf{High Risk - Lack of Annual Security Training:} While new employees receive training, the lack of a mandatory annual security awareness program for all staff can lead to knowledge decay, making the organization more susceptible to evolving social engineering and phishing tactics.
\end{itemize}

On a positive note, the external network scan of the target IP address, \texttt{[Target IP]}, did not reveal any open ports. This suggests a strong firewall configuration that effectively limits the external attack surface. No pre-existing vulnerabilities were reported.

Recommendations are prioritized to address the most severe risks first, focusing on the immediate implementation of MFA for email and the establishment of a recurring security awareness training program.

% --- ORGANIZATIONAL INFORMATION ---
\section{Organizational Information}

The following information was provided for the assessment.
\begin{itemize}
    \item \textbf{Organization Name:} Iron River Finance
    \item \textbf{Email Domain:} \texttt{IronRiverFinance.net}
    \item \textbf{Website Domain:} \url{http://www.IronRiverFinance.net}
    \item \textbf{Primary External IP:} \texttt{123.8.139.126}
\end{itemize}

% --- SECURITY CONTROL REVIEW ---
\section{Security Control Review}

The following table summarizes the organization's responses to the security controls questionnaire. A green checkmark (\yes) indicates a positive control is in place, while a red cross (\no) indicates a potential security gap.

\begin{table}[h!]
\centering
\caption{Security Controls Questionnaire Results}
\begin{tabular}{p{0.75\textwidth} c}
\toprule
\textbf{Control Question} & \textbf{Response} \\
\midrule
Do you require MFA to access email? & \textcolor{red}{\no} \\
Do you require MFA to log into computers? & \textcolor{green}{\yes} \\
Do you require MFA to access sensitive data systems? & \textcolor{green}{\yes} \\
Does your organization have an employee acceptable use policy? & \textcolor{green}{\yes} \\
Does your organization do security awareness training for new employees? & \textcolor{green}{\yes} \\
Does your organization do security awareness training for all employees at least once per year? & \textcolor{red}{\no} \\
\bottomrule
\end{tabular}
\end{table}

The review highlights two key areas for improvement: email security and ongoing employee training, which are detailed in the Risk Assessment section.

% --- TECHNICAL SCAN RESULTS ---
\section{Technical Scan Results}

An external network scan was conducted to identify potential vulnerabilities on the organization's perimeter.

\begin{itemize}
    \item \textbf{Target IP Address:} \texttt{[Target IP]}
    \item \textbf{Scan Date:} Not provided in scan data.
\end{itemize}

\subsection{Summary of Findings}
The network scan did not identify any open TCP or UDP ports on the target system. This is a positive security finding, indicating that a firewall or other network security device is likely in place and configured to deny unsolicited inbound traffic. This significantly reduces the external attack surface available to potential adversaries.

\textit{Note: This result assumes the target host was online and reachable during the assessment period.}

% --- RISK ASSESSMENT ---
\section{Risk Assessment}

This section correlates the findings from the security control review and technical scan. The following risks have been identified and prioritized based on their potential impact on the organization.

\begin{table}[h!]
\centering
\caption{Identified Risks}
\begin{tabular}{p{0.25\textwidth} p{0.15\textwidth} p{0.5\textwidth}}
\toprule
\textbf{Risk Name} & \textbf{Severity} & \textbf{Overview} \\
\midrule
\textbf{Lack of MFA for Email Access} & \textbf{Critical} & The absence of MFA on email accounts makes them highly vulnerable to compromise via stolen or weak passwords. A compromised email account can lead to data breaches, financial fraud (BEC), and further infiltration of the network. \\
\addlinespace
\textbf{Inadequate Security Awareness Training} & \textbf{High} & Without mandatory, annual security training, employees' ability to recognize and respond to modern threats like phishing and social engineering diminishes over time. This increases the likelihood of human error leading to a security incident. \\
\bottomrule
\end{tabular}
\end{table}

% --- RECOMMENDATIONS ---
\section{Recommendations}

The following recommendations are provided to mitigate the identified risks and strengthen the overall security posture of Iron River Finance. They are prioritized based on the severity of the associated risk.

\subsection{Priority 1: Implement MFA for Email (Critical)}
\begin{itemize}
    \item \textbf{Action:} Immediately begin the process of enabling and mandating Multi-Factor Authentication (MFA) for all user and administrative email accounts.
    \item \textbf{Justification:} This is the single most effective control to prevent unauthorized access to email. It provides a critical layer of defense against password-based attacks, even if user credentials are stolen.
    \item \textbf{Implementation Guidance:} Prioritize authenticator applications (e.g., Google Authenticator, Microsoft Authenticator) or hardware tokens (e.g., YubiKey) over SMS-based MFA, which is susceptible to SIM-swapping attacks.
\end{itemize}

\subsection{Priority 2: Establish Annual Security Awareness Training (High)}
\begin{itemize}
    \item \textbf{Action:} Develop and implement a formal, mandatory security awareness training program for all employees to be completed at least once per year.
    \item \textbf{Justification:} Regular training reinforces security best practices and keeps employees informed about the latest threat trends. This creates a more resilient "human firewall" and reduces the risk of incidents caused by human error.
    \item \textbf{Implementation Guidance:} The training program should cover key topics such as phishing identification, password security, acceptable use of company assets, and incident reporting procedures. Consider incorporating periodic phishing simulation exercises to test and reinforce the training.
\end{itemize}

\end{document}
```