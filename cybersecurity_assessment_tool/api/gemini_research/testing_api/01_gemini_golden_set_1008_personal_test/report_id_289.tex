```latex
\documentclass[12pt, a4paper]{article}

% Preamble: Required Packages
\usepackage[margin=1in]{geometry}
\usepackage{pifont} % For checkmarks and crosses
\usepackage{booktabs} % For professional tables
\usepackage{hyperref} % For clickable links
\usepackage{url} % For URL formatting
\usepackage{seqsplit} % To split long strings without breaking
\usepackage{graphicx}
\usepackage[table]{xcolor}
\usepackage{fancyhdr}

% --- Document Setup ---
\definecolor{darkblue}{rgb}{0.0, 0.0, 0.55}
\definecolor{darkred}{rgb}{0.55, 0.0, 0.0}
\definecolor{tablehead}{gray}{0.9}

\hypersetup{
    colorlinks=true,
    linkcolor=darkblue,
    filecolor=darkblue,      
    urlcolor=darkblue,
    citecolor=darkblue,
}

\pagestyle{fancy}
\fancyhf{}
\fancyhead[L]{\textbf{Cybersecurity Assessment Report}}
\fancyhead[R]{Digital Drift}
\fancyfoot[C]{\thepage}
\renewcommand{\headrulewidth}{0.4pt}
\renewcommand{\footrulewidth}{0.4pt}

% --- Document Content ---
\begin{document}

% --- Title Page ---
\begin{titlepage}
    \centering
    \vspace*{2cm}
    
    \Huge\textbf{Cybersecurity Assessment Report}
    
    \vspace{1.5cm}
    
    \Large\textbf{Prepared for:} \\
    \vspace{0.5cm}
    \huge{Digital Drift}
    
    \vfill
    
    \large
    \textbf{Date of Assessment:} November 22, 2025 \\
    \textbf{Report ID:} CYBER-20251122-001
    
    \vspace{2cm}
    
    \rule{\textwidth}{0.4pt}
    \par
    \textit{This document contains sensitive information. Distribution is restricted to authorized personnel only.}
    
\end{titlepage}

\tableofcontents
\newpage

% --- Section 1: Executive Summary ---
\section{Executive Summary}
This report details the findings of a cybersecurity assessment conducted for Digital Drift on November 22, 2025. The assessment combined a review of organizational security controls, an external network scan, and an analysis of known risks.

The overall security posture requires immediate attention. Two critical-risk findings were identified that significantly increase the organization's exposure to common cyber threats.

\begin{itemize}
    \item \textbf{Critical MFA Gaps:} Multi-Factor Authentication (MFA) is not enforced for accessing email or sensitive data systems. This represents a critical vulnerability, as it leaves the organization highly susceptible to Business Email Compromise (BEC), phishing attacks, and unauthorized data access.
    \item \textbf{Outdated Web Server:} The public-facing web server is running an outdated version of Nginx (1.18.0), which dates back to early 2020. This software is no longer supported and is likely vulnerable to numerous publicly disclosed exploits.
\end{itemize}

While the organization demonstrates a solid foundation in employee security awareness training and policy, the identified technical and procedural gaps must be remediated urgently to reduce the risk of a significant security incident. Recommendations for remediation are detailed in Section \ref{sec:recommendations}.

\newpage

% --- Section 2: Organizational Information ---
\section{Organizational Information}
The following information was provided for the assessment.

\begin{table}[h!]
\centering
\rowcolors{2}{gray!10}{white}
\begin{tabular}{ll}
\toprule
\textbf{Attribute} & \textbf{Value} \\
\midrule
Organization Name & Digital Drift \\
Email Domain & \texttt{DigitalDrift.com} \\
Website Domain & \url{www.DigitalDrift.com} \\
External IP Address & \texttt{2.172.33.100} \\
\bottomrule
\end{tabular}
\caption{Client Organizational Details}
\end{table}

% --- Section 3: Security Control Review ---
\section{Security Control Review}
A review of administrative and procedural security controls was conducted based on a standardized questionnaire. The results indicate gaps in critical access control measures.

\begin{table}[h!]
\centering
\rowcolors{2}{gray!10}{white}
\begin{tabular}{p{0.7\textwidth}c}
\toprule
\textbf{Control Question} & \textbf{Response} \\
\midrule
Do you require MFA to access email? & \textcolor{darkred}{\ding{55}} \\
Do you require MFA to log into computers? & \textcolor{green!50!black}{\ding{51}} \\
Do you require MFA to access sensitive data systems? & \textcolor{darkred}{\ding{55}} \\
Does your organization have an employee acceptable use policy? & \textcolor{green!50!black}{\ding{51}} \\
Does your organization do security awareness training for new employees? & \textcolor{green!50!black}{\ding{51}} \\
Does your organization do security awareness training for all employees at least once per year? & \textcolor{green!50!black}{\ding{51}} \\
\bottomrule
\end{tabular}
\caption{Security Controls Questionnaire Results}
\end{table}

\subsection*{Analysis}
The lack of MFA for email and sensitive data systems is a critical weakness. Email is a primary target for attackers seeking to gain an initial foothold in an organization through phishing and credential theft. Failure to protect sensitive data systems with MFA removes a crucial layer of defense against unauthorized access and potential data exfiltration.

% --- Section 4: Technical Scan Results ---
\section{Technical Scan Results}
An Nmap scan was performed against the target host \texttt{192.168.10.5} on November 22, 2025. The scan identified one open port with an outdated service.

\begin{table}[h!]
\centering
\rowcolors{2}{gray!10}{white}
\begin{tabular}{lllll}
\toprule
\textbf{Port} & \textbf{State} & \textbf{Service} & \textbf{Product} & \textbf{Version} \\
\midrule
443/tcp & open & https & nginx & 1.18.0 \\
\bottomrule
\end{tabular}
\caption{Open Ports and Services on \texttt{192.168.10.5}}
\end{table}

\subsection*{Analysis}
The scan revealed that the web server is running \textbf{Nginx version 1.18.0}. This version was released in April 2020 and is now considered obsolete. Running outdated, internet-facing software poses a high risk, as it is likely susceptible to numerous publicly known vulnerabilities that have been patched in newer versions. Attackers frequently scan for and exploit such outdated services to compromise servers.

% --- Section 5: Risk Assessment ---
\section{Risk Assessment}
The following table synthesizes findings from the security control review and technical scan into a prioritized list of risks. No pre-existing vulnerabilities were reported.

\begin{table}[h!]
\centering
\begin{tabular}{p{0.25\textwidth}p{0.5\textwidth}l}
\toprule
\textbf{Risk Name} & \textbf{Description} & \textbf{Severity} \\
\midrule
\rowcolor{red!20}
Inadequate MFA Implementation & MFA is not enforced for email or sensitive data systems. This exposes the organization to a high likelihood of account compromise, data breaches, and financial fraud via BEC. & \textbf{Critical} \\
\addlinespace
\rowcolor{orange!20}
Outdated Web Server Software & The public-facing Nginx server (v1.18.0) is significantly outdated and unpatched, making it an easy target for automated exploits targeting known vulnerabilities. & \textbf{High} \\
\bottomrule
\end{tabular}
\caption{Summary of Identified Risks}
\end{table}

% --- Section 6: Recommendations ---
\section{Recommendations}
\label{sec:recommendations}
The following actions are recommended to mitigate the identified risks. Recommendations are prioritized by severity.

\subsection{Priority 1: Implement Comprehensive MFA (Critical)}
\begin{itemize}
    \item \textbf{Action:} Procure and deploy a robust MFA solution. Enforce its use for all user access to email (e.g., Office 365, Google Workspace) and any system identified as containing sensitive corporate or customer data.
    \item \textbf{Impact:} Drastically reduces the risk of account takeover from stolen credentials, mitigating threats like Business Email Compromise and unauthorized data access.
    \item \textbf{Timeline:} Immediate (1-2 Weeks)
\end{itemize}

\subsection{Priority 2: Remediate Outdated Software (High)}
\begin{itemize}
    \item \textbf{Action:} Plan and execute an upgrade of the Nginx web server on host \texttt{192.168.10.5} to the latest stable version. Prior to the upgrade, perform testing in a non-production environment to ensure application compatibility.
    \item \textbf{Impact:} Closes known security vulnerabilities, hardening the server against automated attacks and preventing potential system compromise.
    \item \textbf{Timeline:} Urgent (2-4 Weeks)
\end{itemize}

\subsection{Priority 3: Establish a Patch Management Policy (Medium)}
\begin{itemize}
    \item \textbf{Action:} Develop and implement a formal patch management policy. The policy should define a process for regularly identifying, testing, and deploying security patches for all operating systems and applications, especially those on internet-facing systems.
    \item \textbf{Impact:} Proactively reduces the organization's attack surface over the long term by ensuring systems remain up-to-date against emerging threats.
    \item \textbf{Timeline:} 1-2 Months
\end{itemize}

\end{document}
```