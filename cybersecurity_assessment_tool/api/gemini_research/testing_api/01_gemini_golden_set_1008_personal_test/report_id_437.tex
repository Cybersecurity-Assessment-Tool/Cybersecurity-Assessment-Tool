```latex
\documentclass[12pt, a4paper]{article}

% Preamble: Required Packages
\usepackage[margin=1in]{geometry}
\usepackage{pifont} % For checkmarks and crosses
\usepackage{booktabs} % For professional tables
\usepackage{hyperref} % For clickable links
\usepackage{url} % For URL formatting
\usepackage{seqsplit} % To split long strings in tt font
\usepackage{graphicx}
\usepackage{xcolor}
\usepackage{tocloft} % For table of contents formatting

% Define colors for severity
\definecolor{sev_critical}{HTML}{940000}
\definecolor{sev_high}{HTML}{D14000}
\definecolor{sev_medium}{HTML}{E09100}

% Hyperref setup
\hypersetup{
    colorlinks=true,
    linkcolor=blue,
    filecolor=magenta,      
    urlcolor=cyan,
    pdftitle={Cybersecurity Assessment Report},
    pdfpagemode=FullScreen,
}

% Document Title Block
\title{Cybersecurity Assessment Report \\ \large For Blue Horizon Initiative}
\author{Cybersecurity Analysis Division}
\date{\today}

\begin{document}

\maketitle
\thispagestyle{empty}
\newpage

\tableofcontents
\thispagestyle{empty}
\newpage

% --- Section 1: Executive Summary ---
\section{Executive Summary}
This report presents a cybersecurity assessment for Blue Horizon Initiative, synthesizing data from network scans, organizational questionnaires, and pre-existing risk registers. The analysis provides a consolidated view of the organization's current security posture, highlighting key vulnerabilities and offering actionable recommendations for remediation.

The assessment identified several areas of significant concern requiring immediate attention. The key findings are:
\begin{itemize}
    \item \textbf{Critical - Lack of Multi-Factor Authentication (MFA) for Email:} The absence of MFA on the primary email system (\texttt{BlueHorizonInitiative.org}) represents a critical vulnerability. This exposes the organization to a high risk of business email compromise (BEC), phishing attacks, and subsequent account takeovers.
    \item \textbf{Critical - Systemic Insecure RDP Exposure:} A network scan identified a new host (\seqsplit{\texttt{10.10.10.51}}) with an open Remote Desktop Protocol (RDP) port (3389). This finding, correlated with a pre-existing risk for a similar issue on another host, points to a systemic problem in managing remote access controls, significantly increasing the risk of unauthorized network access.
    \item \textbf{High - Gaps in Employee Security Training:} The organization lacks a mandatory security awareness training program for new employees. This creates a window of vulnerability where new hires, who are often targeted, are not equipped to recognize and respond to social engineering or phishing threats.
\end{itemize}

The overall security posture has critical gaps that undermine existing controls. We strongly advise prioritizing the remediation of these findings to reduce the likelihood of a significant security incident.

% --- Section 2: Organizational Information ---
\section{Organizational Information}
The following information was provided for the assessment scope.

\begin{tabular}{@{}ll}
\toprule
\textbf{Attribute} & \textbf{Value} \\
\midrule
Organization Name & Blue Horizon Initiative \\
Email Domain & \seqsplit{\texttt{BlueHorizonInitiative.org}} \\
Website Domain & \href{http://www.BlueHorizonInitiative.org}{\seqsplit{\texttt{www.BlueHorizonInitiative.org}}} \\
External IP Address & \seqsplit{\texttt{84.108.232.190}} \\
\bottomrule
\end{tabular}

% --- Section 3: Security Control Review ---
\section{Security Control Review}
An analysis of the organization's security questionnaire revealed significant gaps in foundational security controls. The table below summarizes the responses provided.

\begin{table}[h!]
\centering
\caption{Security Controls Questionnaire Summary}
\begin{tabular}{@{}p{0.7\linewidth} c c@{}}
\toprule
\textbf{Control Question} & \textbf{Response} & \textbf{Status} \\
\midrule
Do you require MFA to access email? & No & \textcolor{red}{\ding{55}} \\
Do you require MFA to log into computers? & Yes & \textcolor{green}{\ding{51}} \\
Do you require MFA to access sensitive data systems? & Yes & \textcolor{green}{\ding{51}} \\
Does your organization have an employee acceptable use policy? & Yes & \textcolor{green}{\ding{51}} \\
Does your organization do security awareness training for new employees? & No & \textcolor{red}{\ding{55}} \\
Does your organization do security awareness training for all employees at least once per year? & Yes & \textcolor{green}{\ding{51}} \\
\bottomrule
\end{tabular}
\end{table}

\subsection*{Analysis of Control Gaps}
The "No" responses indicate critical weaknesses:
\begin{itemize}
    \item \textbf{No MFA for Email:} Email is the primary entry point for many cyberattacks. Without MFA, a compromised password is all an attacker needs to gain access to an employee's mailbox, potentially leading to data theft, financial fraud, and a launchpad for further attacks on the internal network.
    \item \textbf{No Security Training for New Employees:} New hires are particularly vulnerable to social engineering attacks as they are unfamiliar with company policies and personnel. Failing to provide immediate security training upon hiring creates an unnecessary risk that could be easily mitigated.
\end{itemize}

% --- Section 4: Technical Scan Results ---
\section{Technical Scan Results}
An internal network scan was performed to identify active services and potential vulnerabilities.

\begin{itemize}
    \item \textbf{Target IP Address:} \seqsplit{\texttt{10.10.10.51}}
    \item \textbf{Scan Status:} Host is UP
\end{itemize}

\begin{table}[h!]
\centering
\caption{Open Ports Identified on \seqsplit{\texttt{10.10.10.51}}}
\begin{tabular}{@{}llll@{}}
\toprule
\textbf{Port} & \textbf{State} & \textbf{Service} & \textbf{Notes} \\
\midrule
3389/tcp & open & ms-wbt-server & Microsoft Remote Desktop Protocol (RDP). \\
 & & & A common target for brute-force and \\
 & & & exploit-based attacks. \\
\bottomrule
\end{tabular}
\end{table}

% --- Section 5: Consolidated Risk Assessment ---
\section{Consolidated Risk Assessment}
The following table correlates findings from the security questionnaire, the technical scan, and the pre-existing risk register. This provides a unified view of the most pressing security risks facing the organization.

\begin{table}[h!]
\centering
\caption{Summary of Identified Risks}
\begin{tabular}{@{}p{0.25\linewidth} p{0.45\linewidth} p{0.2\linewidth}@{}}
\toprule
\textbf{Risk Name} & \textbf{Description} & \textbf{Severity} \\
\midrule
\textbf{Systemic RDP Exposure} & The technical scan found open RDP (port 3389) on \texttt{10.10.10.51}. This correlates with a known risk on \texttt{10.10.10.50}, indicating a widespread issue with insecure remote access configurations. RDP is a primary vector for ransomware attacks. & \textbf{\textcolor{sev_critical}{Critical}} \\
\addlinespace
\textbf{Lack of MFA for Email} & The primary email domain lacks MFA, exposing all accounts to takeover via password compromise (e.g., phishing, credential stuffing). This can lead to significant data breaches and financial loss. & \textbf{\textcolor{sev_critical}{Critical}} \\
\addlinespace
\textbf{Inadequate New Employee Onboarding} & The absence of security training for new hires leaves the organization vulnerable to social engineering, as new employees are prime targets and lack awareness of organizational security policies from day one. & \textbf{\textcolor{sev_high}{High}} \\
\bottomrule
\end{tabular}
\end{table}

% --- Section 6: Recommendations ---
\section{Recommendations}
The following actions are recommended to mitigate the identified risks. Recommendations are prioritized based on severity and potential impact.

\subsection*{Priority 1 (Critical): Remediate Email and RDP Risks}
\begin{enumerate}
    \item \textbf{Enforce MFA for All Email Accounts:}
    \begin{itemize}
        \item \textbf{Action:} Immediately enable and enforce MFA for all user accounts accessing the \texttt{BlueHorizonInitiative.org} email system.
        \item \textbf{Impact:} Drastically reduces the risk of email account takeover, even if passwords are stolen. This is the single most effective control to implement.
    \end{itemize}
    \item \textbf{Secure Remote Desktop Protocol (RDP) Access:}
    \begin{itemize}
        \item \textbf{Immediate Action:} Create firewall rules to block all external access to TCP port 3389. For internal access, restrict RDP on \texttt{10.10.10.51} and \texttt{10.10.10.50} to a limited set of administrative jump hosts.
        \item \textbf{Long-Term Action:} Implement a secure remote access solution, such as a Virtual Private Network (VPN) with MFA or a Zero Trust Network Access (ZTNA) gateway. This ensures all remote administrative access is authenticated and encrypted.
        \item \textbf{Audit:} Conduct a full internal network scan to identify and remediate any other systems with exposed RDP or other risky remote management services.
    \end{itemize}
\end{enumerate}

\subsection*{Priority 2 (High): Improve Security Training}
\begin{enumerate}
    \setcounter{enumi}{2} % Start numbering from 3
    \item \textbf{Implement Onboarding Security Training:}
    \begin{itemize}
        \item \textbf{Action:} Integrate a mandatory security awareness training module into the new employee onboarding process. This training should cover phishing, acceptable use, password hygiene, and how to report security incidents.
        \item \textbf{Impact:} Reduces the likelihood of new employees falling victim to common cyberattacks and establishes a security-conscious mindset from their first day.
    \end{itemize}
\end{enumerate}

\end{document}
```