```latex
\documentclass[12pt]{article}

% Preamble: Required Packages
\usepackage[margin=1in]{geometry}
\usepackage{pifont} % For checkmarks and crosses (\ding{51}, \ding{55})
\usepackage{booktabs} % For professional tables (\toprule, \midrule, \bottomrule)
\usepackage{hyperref} % For clickable links
\usepackage{url} % For formatting URLs
\usepackage{seqsplit} % For splitting long strings in tt font
\usepackage{xcolor} % For colors

% Document Information
\title{Cybersecurity Assessment Report}
\author{Cybersecurity Analysis Division}
\date{\today}

% Hyperref Setup
\hypersetup{
    colorlinks=true,
    linkcolor=blue,
    filecolor=magenta,      
    urlcolor=cyan,
    pdftitle={Cybersecurity Assessment Report},
    pdfpagemode=FullScreen,
}

\begin{document}

\maketitle
\thispagestyle{empty}
\newpage

\tableofcontents
\newpage

% --- Section 1: Executive Overview ---
\section{Executive Overview}
This report provides a comprehensive cybersecurity assessment for \textbf{Copperhead Cables}. The analysis is based on a correlation of external network scan data, a self-reported security controls questionnaire, and a list of pre-existing risks.

The external network scan revealed a strong perimeter security posture, with no open ports detected on the target system. This indicates a well-configured firewall and is a significant positive finding.

However, the security controls review identified several critical administrative and policy-related gaps. The most severe risks stem from the lack of Multi-Factor Authentication (MFA) on email and the complete absence of a formal security awareness training program and an Acceptable Use Policy (AUP). These deficiencies expose the organization to a high risk of social engineering, particularly phishing attacks and Business Email Compromise (BEC), which could bypass the strong network defenses.

Immediate remediation should focus on implementing MFA for email and establishing foundational security policies and training programs to mitigate human-factor risks.

% --- Section 2: Organizational Information ---
\section{Organizational Information}
This section details the information provided by the client organization.

\begin{table}[h!]
\centering
\begin{tabular}{@{}ll@{}}
\toprule
\textbf{Attribute} & \textbf{Value} \\ \midrule
Organization Name & \textbf{Copperhead Cables} \\
Email Domain & \texttt{CopperheadCables.net} \\
Website Domain & \url{www.CopperheadCables.net} \\
External IP Address & \texttt{72.139.202.185} \\ \bottomrule
\end{tabular}
\caption{Client Organizational Details}
\label{tab:org_info}
\end{table}

% --- Section 3: Security Control Review ---
\section{Security Control Review}
The following table summarizes the organization's responses to a security controls questionnaire. Items marked with a red cross (\textcolor{red}{\ding{55}}) indicate a deviation from security best practices and represent a potential risk.

\begin{table}[h!]
\centering
\begin{tabular}{@{}lc@{}}
\toprule
\textbf{Security Control Question} & \textbf{Status} \\ \midrule
Do you require MFA to access email? & \textcolor{red}{\ding{55}} \\
Do you require MFA to log into computers? & \textcolor{green}{\ding{51}} \\
Do you require MFA to access sensitive data systems? & \textcolor{green}{\ding{51}} \\
Does your organization have an employee acceptable use policy? & \textcolor{red}{\ding{55}} \\
Does your organization do security awareness training for new employees? & \textcolor{red}{\ding{55}} \\
Does your organization do security awareness training for all employees at least once per year? & \textcolor{red}{\ding{55}} \\ \bottomrule
\end{tabular}
\caption{Security Controls Questionnaire Results}
\label{tab:controls_review}
\end{table}

% --- Section 4: Technical Scan Results ---
\section{Technical Scan Results}
An external network scan was performed to identify open ports and exposed services on the public-facing infrastructure.

\begin{itemize}
    \item \textbf{Target IP Address:} \texttt{192.168.1.100}
    \item \textbf{Scan Date:} \today
    \item \textbf{Status:} Host is up.
\end{itemize}

\subsection{Findings}
The scan concluded that there were \textbf{no open ports} detected on the target system. All 1000 scanned ports were reported as "closed". This is a positive security finding, suggesting that a firewall is effectively blocking unsolicited inbound traffic from the internet.

% --- Section 5: Risk Assessment ---
\section{Risk Assessment}
This section synthesizes findings from the security control review, technical scans, and pre-existing vulnerabilities to provide a consolidated list of identified risks.

\begin{table}[h!]
\centering
\begin{tabular}{@{}p{0.2\textwidth}p{0.6\textwidth}p{0.15\textwidth}@{}}
\toprule
\textbf{Risk Name} & \textbf{Overview} & \textbf{Severity} \\ \midrule
\textbf{Lack of MFA on Email} & Without MFA, email accounts are highly vulnerable to compromise via credential stuffing or phishing. A compromised email account can lead to Business Email Compromise (BEC), data exfiltration, and a launchpad for further internal attacks. & \textbf{Critical} \\
\addlinespace
\textbf{No Security Awareness Training} & Employees have not been trained to identify or respond to social engineering attacks like phishing, malware, or pretexting. This makes them a primary target for attackers seeking to gain initial access to the network. & \textbf{High} \\
\addlinespace
\textbf{No Acceptable Use Policy (AUP)} & The absence of a formal AUP creates ambiguity regarding the proper use of company IT assets. This increases the risk of insider threats, accidental data leakage, and installation of unauthorized software, while also creating potential legal and compliance issues. & \textbf{High} \\ \bottomrule
\end{tabular}
\caption{Consolidated Risk Summary}
\label{tab:risk_summary}
\end{table}

% --- Section 6: Recommendations ---
\section{Recommendations}
Based on the analysis, the following actions are recommended to mitigate the identified risks and improve the overall security posture of \textbf{Copperhead Cables}.

\begin{enumerate}
    \item \textbf{[Critical] Implement MFA for Email Access:}
    \begin{itemize}
        \item \textbf{Action:} Immediately enable and enforce Multi-Factor Authentication (MFA) for all user accounts on the \texttt{CopperheadCables.net} email platform.
        \item \textbf{Justification:} This is the single most effective control to prevent unauthorized access to email accounts and mitigate the risk of Business Email Compromise.
    \end{itemize}
    \vspace{1em}

    \item \textbf{[High] Establish a Security Awareness Training Program:}
    \begin{itemize}
        \item \textbf{Action:} Develop and implement a mandatory security awareness training program. The program should include initial training for all new hires and annual refresher courses for all employees.
        \item \textbf{Justification:} Training empowers employees to become a human firewall, enabling them to recognize and report suspicious activities, thereby reducing the likelihood of a successful social engineering attack.
    \end{itemize}
    \vspace{1em}
    
    \item \textbf{[High] Develop and Enforce an Acceptable Use Policy (AUP):}
    \begin{itemize}
        \item \textbf{Action:} Draft a formal AUP that clearly outlines the rules and responsibilities for using company computers, networks, and data. Require all employees to read and acknowledge the policy.
        \item \textbf{Justification:} An AUP establishes clear security expectations, reduces risky behavior, and provides a basis for disciplinary action in case of policy violations, protecting the organization from insider threats and legal liability.
    \end{itemize}
\end{enumerate}

\end{document}
```