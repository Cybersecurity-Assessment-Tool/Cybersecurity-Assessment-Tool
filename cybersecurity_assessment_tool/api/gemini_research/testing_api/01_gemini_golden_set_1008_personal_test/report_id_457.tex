```latex
\documentclass[12pt]{article}

% Preamble: Required Packages
\usepackage[margin=1in]{geometry}
\usepackage{pifont} % For checkmarks and crosses
\usepackage{booktabs} % For professional tables
\usepackage{hyperref} % For clickable links and better PDF navigation
\usepackage{url} % For formatting URLs
\usepackage{seqsplit} % To split long monospaced strings
\usepackage{graphicx} % For potential logos
\usepackage{xcolor} % For colors

% Document Information
\title{Cybersecurity Posture Assessment Report}
\author{Cybersecurity Analysis Division}
\date{November 22, 2025}

% Hyperref Setup
\hypersetup{
    colorlinks=true,
    linkcolor=blue,
    filecolor=magenta,      
    urlcolor=cyan,
    pdftitle={Cybersecurity Posture Assessment Report},
    pdfpagemode=FullScreen,
}

\begin{document}

\maketitle
\thispagestyle{empty}
\newpage

\tableofcontents
\newpage

% --- 1. Executive Summary ---
\section*{1. Executive Summary}

This report provides a comprehensive analysis of the cybersecurity posture for \textbf{Harbor Light Foundation}, conducted on November 22, 2025. The assessment synthesizes findings from a technical network scan, a review of organizational security controls, and an evaluation of pre-existing risks.

Overall, Harbor Light Foundation has implemented several positive security controls, including Multi-Factor Authentication (MFA) for email and sensitive data systems. However, this assessment has identified \textbf{three significant risks} that require immediate attention. 

Key findings include an outdated and potentially vulnerable web server (\texttt{nginx 1.18.0}), a critical gap in endpoint security due to the lack of MFA for computer logins, and an incomplete employee security training program that omits new hires. These issues expose the organization to potential threats such as unauthorized access, data breaches, and system compromise. Actionable recommendations are provided to mitigate these risks and strengthen the organization's overall security resilience.

% --- 2. Organizational Information ---
\section*{2. Organizational Information}

The following information was provided for the assessment. This data establishes the context and scope of the review.

\begin{table}[h!]
\centering
\begin{tabular}{@{}ll@{}}
\toprule
\textbf{Attribute} & \textbf{Value} \\ \midrule
Organization Name & Harbor Light Foundation \\
Email Domain & \texttt{HarborLightFoundation.net} \\
Website Domain & \url{www.HarborLightFoundation.net} \\
External IP Address & \texttt{1.122.20.117} \\
Assessment Date & 2025-11-22 \\ \bottomrule
\end{tabular}
\caption{Client Organizational Details}
\end{table}

% --- 3. Security Control Review ---
\section*{3. Security Control Review}

A review of administrative and policy-based security controls was conducted via a questionnaire. The responses highlight both strengths and critical weaknesses in the current security framework. Answers marked with a cross (\ding{55}) indicate a deviation from security best practices and are addressed in the Risk Assessment section.

\begin{table}[h!]
\centering
\begin{tabular}{@{}p{0.7\textwidth}c@{}}
\toprule
\textbf{Security Control Question} & \textbf{Status} \\ \midrule
Do you require MFA to access email? & \ding{51} \\
Do you require MFA to log into computers? & \textcolor{red}{\ding{55}} \\
Do you require MFA to access sensitive data systems? & \ding{51} \\
Does your organization have an employee acceptable use policy? & \ding{51} \\
Does your organization do security awareness training for new employees? & \textcolor{red}{\ding{55}} \\
Does your organization do security awareness training for all employees at least once per year? & \ding{51} \\ \bottomrule
\end{tabular}
\caption{Organizational Security Controls Questionnaire}
\end{table}

% --- 4. Technical Scan Results ---
\section*{4. Technical Scan Results}

An external network scan was performed to identify open ports and exposed services. The scan targeted the host at \texttt{192.168.10.5}, revealing the following key information.

\begin{table}[h!]
\centering
\begin{tabular}{@{}lllll@{}}
\toprule
\textbf{Port} & \textbf{State} & \textbf{Service} & \textbf{Product} & \textbf{Version} \\ \midrule
443/tcp & Open & https & nginx & 1.18.0 \\ \bottomrule
\end{tabular}
\caption{Open Ports and Services on \texttt{192.168.10.5}}
\end{table}

\subsection*{Analysis of Technical Findings}
The scan identified an \texttt{nginx} web server, version \textbf{1.18.0}, accessible on port 443 (HTTPS). This version was released in April 2020 and is now significantly outdated. Current stable versions of nginx have received numerous security patches and feature updates. Running this legacy version exposes the server to a wide range of publicly known vulnerabilities (CVEs) that could be exploited by attackers to compromise the system.

% --- 5. Risk Assessment & Findings ---
\section*{5. Risk Assessment \& Findings}

This section correlates the findings from the security control review and the technical scan. Three new risks were identified during this assessment. The pre-existing risk register was empty.

\begin{table}[h!]
\centering
\begin{tabular}{@{}p{0.15\textwidth}p{0.55\textwidth}l@{}}
\toprule
\textbf{Risk ID} & \textbf{Description} & \textbf{Severity} \\ \midrule
\textbf{RISK-001} & The public-facing web server runs an outdated version of nginx (1.18.0), which is susceptible to multiple known vulnerabilities. & \textbf{High} \\
\textbf{RISK-002} & Workstation logins are not protected by Multi-Factor Authentication (MFA), creating a significant risk of unauthorized access via compromised credentials. & \textbf{High} \\
\textbf{RISK-003} & New employees do not receive security awareness training as part of their onboarding, leaving them vulnerable to social engineering attacks until the annual training cycle. & \textbf{Medium} \\ \bottomrule
\end{tabular}
\caption{Summary of Identified Risks}
\end{table}

% --- 6. Recommendations ---
\section*{6. Recommendations}

The following actions are recommended to mitigate the identified risks and improve the overall security posture of Harbor Light Foundation.

\subsection*{RISK-001: Outdated Nginx Web Server (High)}
\begin{itemize}
    \item \textbf{Immediate Action:} Plan and execute an upgrade of the nginx server from version 1.18.0 to the latest stable version. This will patch known security vulnerabilities.
    \item \textbf{Strategic Action:} Implement a formal patch management policy and program. This should include regular scanning for outdated software and a defined process for testing and deploying security patches in a timely manner.
\end{itemize}

\subsection*{RISK-002: Lack of MFA on Workstation Logins (High)}
\begin{itemize}
    \item \textbf{Immediate Action:} Procure and deploy an MFA solution for all employee computer and laptop logins. Prioritize deployment for privileged users (e.g., administrators) and executives.
    \item \textbf{Strategic Action:} Enforce MFA as a mandatory security requirement for all endpoint access. Integrate this control into the organization's acceptable use and remote access policies.
\end{itemize}

\subsection*{RISK-003: Missing Security Training for New Hires (Medium)}
\begin{itemize}
    \item \textbf{Immediate Action:} Develop or procure a security awareness training module and integrate it into the mandatory onboarding process for all new employees, to be completed within their first week of employment.
    \item \textbf{Strategic Action:} Ensure the onboarding training covers key topics such as phishing, password security, and the organization's acceptable use policy. This initial training will complement the existing annual refresher course.
\end{itemize}

% --- 7. Conclusion ---
\section*{7. Conclusion}
Harbor Light Foundation has a foundational security program but faces immediate risks from outdated software and gaps in access control and employee training. By implementing the recommendations outlined in this report—specifically by upgrading critical software, enforcing MFA on all endpoints, and strengthening the employee training lifecycle—the organization can significantly reduce its attack surface and enhance its defense against common cyber threats.

\end{document}
```