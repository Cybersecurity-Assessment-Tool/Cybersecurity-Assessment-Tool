```latex
\documentclass[12pt]{article}

% Preamble: Required Packages and Document Setup
\usepackage[margin=1in]{geometry}
\usepackage{pifont} % For checkmarks and crosses
\usepackage{booktabs} % For professional tables
\usepackage{hyperref} % For clickable links
\usepackage{url} % For URL formatting
\usepackage{seqsplit} % To split long strings like URLs or IPs
\usepackage{graphicx} % For potential logo inclusion

% Document Metadata
\title{Cybersecurity Posture Assessment Report for \\ Maple Leaf Logistics}
\author{Cybersecurity Analysis Division}
\date{\today}

\begin{document}

\maketitle
\thispagestyle{empty}
\newpage

\tableofcontents
\newpage

% --- Section 1: Executive Overview ---
\section{Executive Overview}

This report presents a cybersecurity posture assessment for \textbf{Maple Leaf Logistics}, conducted on \today. The analysis synthesizes data from an external network scan, a security controls questionnaire, and a list of pre-existing risks.

The assessment reveals several critical and high-risk security gaps that require immediate attention. While the organization has implemented Multi-Factor Authentication (MFA) for key systems like email, its absence on employee computers represents a significant vulnerability. This gap, combined with a lack of a formal security awareness training program, substantially increases the risk of a successful cyberattack originating from credential compromise or social engineering.

Furthermore, technical scanning identified an exposed service (SSH on port 22) on a critical system. This finding correlates directly with a pre-identified risk, rated as `Critical` with a CVSS score of 10.0.

Immediate remediation should focus on implementing mandatory MFA for all workstation logins, establishing a comprehensive security awareness training program, and securing the exposed network service. Addressing these findings will markedly improve the organization's defensive posture against common and impactful cyber threats.

% --- Section 2: Organizational Information ---
\section{Organizational Information}

The following information was provided for the assessment. This data defines the scope and context of the analysis.

\begin{tabular}{@{}ll}
    \toprule
    \textbf{Attribute} & \textbf{Value} \\
    \midrule
    Organization Name & \textbf{Maple Leaf Logistics} \\
    Email Domain & \seqsplit{\texttt{MapleLeafLogistics.org}} \\
    Website Domain & \seqsplit{\url{www.MapleLeafLogistics.org}} \\
    External IP Address & \seqsplit{\texttt{208.133.162.121}} \\
    \bottomrule
\end{tabular}

% --- Section 3: Security Control Review ---
\section{Security Control Review}

A review of the organization's security controls was conducted via a questionnaire. The responses highlight key areas of strength and weakness in the current security policy framework. Gaps identified with a \ding{55} represent a failure to meet baseline security best practices and are correlated with specific risks in Section 5.

\begin{table}[h!]
\centering
\caption{Security Controls Questionnaire Analysis}
\begin{tabular}{@{}p{0.6\linewidth} c p{0.2\linewidth}@{}}
    \toprule
    \textbf{Control Question} & \textbf{Response} & \textbf{Assessment} \\
    \midrule
    Do you require MFA to access email? & \ding{51} & Compliant \\
    \addlinespace
    Do you require MFA to log into computers? & \ding{55} & \textbf{Critical Gap} \\
    \addlinespace
    Do you require MFA to access sensitive data systems? & \ding{51} & Compliant \\
    \addlinespace
    Does your organization have an employee acceptable use policy? & \ding{51} & Compliant \\
    \addlinespace
    Does your organization do security awareness training for new employees? & \ding{55} & \textbf{High Risk} \\
    \addlinespace
    Does your organization do security awareness training for all employees at least once per year? & \ding{55} & \textbf{High Risk} \\
    \bottomrule
\end{tabular}
\end{table}

% --- Section 4: Technical Scan Results ---
\section{Technical Scan Results}

An Nmap scan was performed on the target system to identify open ports and exposed services. The results are detailed below.

\begin{itemize}
    \item \textbf{Target IP Address:} \texttt{127.0.0.1}
    \item \textbf{Scan Status:} Host is up.
\end{itemize}

\begin{table}[h!]
\centering
\caption{Open Port Analysis}
\begin{tabular}{@{}lllll@{}}
    \toprule
    \textbf{Port} & \textbf{State} & \textbf{Service} & \textbf{Version} & \textbf{Notes} \\
    \midrule
    22/tcp & Open & SSH (Inferred) & N/A & The SSH service is exposed. If not securely \\
           &      &                &     & configured (e.g., weak passwords, no IP \\
           &      &                &     & filtering), it presents a primary vector for \\
           &      &                &     & unauthorized access. \\
    \bottomrule
\end{tabular}
\end{table}

% --- Section 5: Consolidated Risk Assessment ---
\section{Consolidated Risk Assessment}

This section consolidates findings from the security control review, technical scan, and pre-existing risk data into a unified list. Each risk is assigned a severity level based on its potential impact and likelihood.

\begin{table}[h!]
\centering
\caption{Summary of Identified Risks}
\begin{tabular}{@{}p{0.25\linewidth} p{0.5\linewidth} l@{}}
    \toprule
    \textbf{Risk Title} & \textbf{Description} & \textbf{Severity} \\
    \midrule
    \textbf{Localhost Exposed} & The network scan confirms an open SSH port on host \texttt{127.0.0.1}, which aligns with a known critical vulnerability. This allows a potential entry point for attackers to gain shell access to the system. & \textbf{Critical} \\
    \addlinespace
    \textbf{Lack of MFA on Workstations} & The absence of MFA for computer logins means that a single compromised password is sufficient for an attacker to gain access to an employee's workstation and potentially move laterally within the network. & \textbf{High} \\
    \addlinespace
    \textbf{Inadequate Security Awareness Training} & The organization does not provide security training to new or existing employees. This makes staff highly susceptible to phishing, social engineering, and other attacks that rely on human error. & \textbf{High} \\
    \bottomrule
\end{tabular}
\end{table}

% --- Section 6: Recommendations ---
\section{Recommendations}

The following actions are recommended to mitigate the identified risks and strengthen the overall security posture of \textbf{Maple Leaf Logistics}.

\begin{enumerate}
    \item \textbf{Remediate Exposed SSH Service (Critical Risk):}
    \begin{itemize}
        \item \textbf{Immediate Action:} Investigate the business need for the exposed SSH service on \texttt{127.0.0.1}. If it is not essential, the port should be closed immediately.
        \item \textbf{Long-Term Fix:} If the service is required, ensure it is protected by a firewall that restricts access to only trusted IP addresses. Furthermore, enforce strong authentication by disabling password-based logins in favor of public key authentication (SSH keys).
    \end{itemize}
    \vspace{1em}
    \item \textbf{Implement Mandatory MFA for Computer Logins (High Risk):}
    \begin{itemize}
        \item \textbf{Immediate Action:} Procure and deploy an MFA solution for all employee workstations and laptops (e.g., Windows Hello, Duo, or similar).
        \item \textbf{Policy Change:} Update the acceptable use and security policies to mandate the use of MFA for all computer access, without exception.
    \end{itemize}
    \vspace{1em}
    \item \textbf{Establish a Security Awareness Training Program (High Risk):}
    \begin{itemize}
        \item \textbf{Immediate Action:} Develop or subscribe to a security awareness training platform. The program must include, at a minimum, modules on phishing, password security, and acceptable use.
        \item \textbf{Long-Term Fix:} Integrate mandatory security training into the new employee onboarding process. Additionally, schedule and require annual refresher training for all staff to ensure their knowledge remains current.
    \end{itemize}
\end{enumerate}

\end{document}
```