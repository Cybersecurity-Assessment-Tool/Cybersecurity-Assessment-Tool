```latex
\documentclass[12pt]{article}

% --- PACKAGES ---
\usepackage[margin=1in]{geometry}
\usepackage{pifont} % For checkmarks and crosses
\usepackage{booktabs} % For professional tables
\usepackage{hyperref} % For clickable links
\usepackage{url} % For URL formatting
\usepackage{seqsplit} % For splitting long strings
\usepackage{graphicx}
\usepackage[utf8]{inputenc}
\usepackage{xcolor}

% --- DOCUMENT SETUP ---
\hypersetup{
    colorlinks=true,
    linkcolor=blue,
    filecolor=magenta,      
    urlcolor=cyan,
    pdftitle={Cybersecurity Posture Assessment},
    pdfpagemode=FullScreen,
}

\newcommand{\yes}{\ding{51}}
\newcommand{\no}{\ding{55}}

% --- DOCUMENT START ---
\begin{document}

% --- TITLE PAGE ---
\begin{titlepage}
    \centering
    \vspace*{1cm}
    \Huge\textbf{Cybersecurity Posture Assessment Report}
    \vspace{1.5cm}
    \Large
    \begin{tabular}{ll}
        \textbf{Organization:} & Velocity Ventures \\
        \textbf{Report Date:} & \today \\
        \textbf{Assessment Date:} & 2025-11-22 \\
    \end{tabular}
    \vfill
    \large
    Prepared by: \\
    \textbf{Expert Cybersecurity Analyst}
    \vspace{0.8cm}
    \textit{This report contains sensitive information and should be handled with care.}
\end{titlepage}

\tableofcontents
\newpage

% --- EXECUTIVE SUMMARY ---
\section{Executive Summary}
This report provides a comprehensive cybersecurity assessment for Velocity Ventures, based on an analysis of organizational security controls, an external network scan, and a review of existing risks. The assessment was conducted on 2025-11-22.

While the organization demonstrates a solid foundation in security policy and employee awareness training, several critical and high-risk vulnerabilities were identified that require immediate attention. 

Key findings include:
\begin{itemize}
    \item \textbf{Critical Gaps in Multi-Factor Authentication (MFA):} MFA is not enforced for logging into company computers or accessing sensitive data systems. This significantly increases the risk of unauthorized access via compromised credentials.
    \item \textbf{Outdated Web Server Software:} The external-facing web server is running an outdated version of Nginx (1.18.0), which is known to have multiple publicly disclosed vulnerabilities. This exposes the organization to potential web-based attacks.
    \item \textbf{Positive Security Culture:} The organization has implemented an acceptable use policy and provides security awareness training for all employees, which are commendable baseline controls.
\end{itemize}

Immediate remediation of the identified MFA and software vulnerabilities is strongly recommended to reduce the organization's attack surface and protect critical assets.

% --- ORGANIZATIONAL INFORMATION ---
\section{Organizational Information}
The following details were provided for the assessment.

\begin{tabular}{@{}ll}
    \toprule
    \textbf{Attribute} & \textbf{Value} \\
    \midrule
    Organization Name & Velocity Ventures \\
    Email Domain & \texttt{VelocityVentures.com} \\
    Website Domain & \url{www.VelocityVentures.com} \\
    External IP Address & \texttt{216.98.185.2} \\
    \bottomrule
\end{tabular}

% --- SECURITY CONTROL REVIEW ---
\section{Security Control Review}
The following table summarizes the organization's responses to a security controls questionnaire. "No" answers indicate significant gaps in the security posture.

\begin{table}[h!]
\centering
\begin{tabular}{@{}p{9cm}ccp{3cm}@{}}
    \toprule
    \textbf{Control Question} & \textbf{Response} & \textbf{Status} & \textbf{Assessment} \\
    \midrule
    Do you require MFA to access email? & Yes & \yes & Good \\
    Do you require MFA to log into computers? & No & \no & \textbf{Critical Gap} \\
    Do you require MFA to access sensitive data systems? & No & \no & \textbf{Critical Gap} \\
    Does your organization have an employee acceptable use policy? & Yes & \yes & Good \\
    Does your organization do security awareness training for new employees? & Yes & \yes & Good \\
    Does your organization do security awareness training for all employees at least once per year? & Yes & \yes & Good \\
    \bottomrule
\end{tabular}
\caption{Security Controls Questionnaire Analysis}
\end{table}

\subsection{Analysis of Control Gaps}
The lack of MFA for computer logins and access to sensitive data systems represents a critical vulnerability. A single compromised password could grant an attacker direct access to an employee's workstation and, subsequently, to the organization's most valuable data. This bypasses perimeter defenses and greatly elevates the risk of a data breach or ransomware event.

% --- TECHNICAL SCAN RESULTS ---
\section{Technical Scan Results}
An Nmap scan was performed against the target IP address \texttt{192.168.10.5} on 2025-11-22. The scan identified the following open ports and services.

\begin{table}[h!]
\centering
\begin{tabular}{@{}lllll@{}}
    \toprule
    \textbf{Port} & \textbf{State} & \textbf{Service} & \textbf{Product} & \textbf{Version} \\
    \midrule
    443/tcp & open & https & nginx & 1.18.0 \\
    \bottomrule
\end{tabular}
\caption{Open Ports and Services}
\end{table}

\subsection{Analysis of Technical Findings}
The scan revealed a web server running \textbf{Nginx version 1.18.0}. This version was released in April 2020 and is now significantly outdated. The current stable version of Nginx is much newer. Outdated software is a primary target for attackers, as it often contains publicly known and exploitable vulnerabilities (CVEs). Running this version presents a high risk to the organization's web presence and underlying server infrastructure.

% --- RISK ASSESSMENT ---
\section{Risk Assessment Summary}
The following table synthesizes findings from the security control review and technical scan into a prioritized list of risks.

\begin{table}[h!]
\centering
\begin{tabular}{@{}p{1.5cm}p{4cm}p{6cm}p{2cm}@{}}
    \toprule
    \textbf{Risk ID} & \textbf{Risk Name} & \textbf{Overview} & \textbf{Severity} \\
    \midrule
    RISK-001 & Lack of MFA on Endpoints & User computers are not protected by MFA, allowing a compromised password to grant full access to an endpoint. & \textbf{Critical} \\
    \addlinespace
    RISK-002 & Lack of MFA on Sensitive Systems & Critical data systems lack MFA protection, exposing them to unauthorized access if credentials are stolen. & \textbf{Critical} \\
    \addlinespace
    RISK-003 & Outdated Web Server Software & The public-facing web server runs an outdated version of Nginx (1.18.0) with known vulnerabilities. & \textbf{High} \\
    \bottomrule
\end{tabular}
\caption{Identified Risks}
\end{table}

% --- RECOMMENDATIONS ---
\section{Recommendations}
Based on the analysis, the following actions are recommended to mitigate the identified risks and improve the overall security posture of Velocity Ventures.

\begin{enumerate}
    \item \textbf{Enforce MFA on All Endpoints (RISK-001):}
    \begin{itemize}
        \item \textbf{Action:} Deploy and enforce a mandatory Multi-Factor Authentication solution for all employee computer logins (e.g., Windows Hello for Business, Duo, or similar).
        \item \textbf{Priority:} Immediate. This is the most effective way to prevent unauthorized access to corporate endpoints.
    \end{itemize}
    \vspace{0.5cm}
    \item \textbf{Enforce MFA on Sensitive Systems (RISK-002):}
    \begin{itemize}
        \item \textbf{Action:} Identify all systems containing sensitive or critical data and immediately enforce MFA for all user access. This includes databases, file shares, and administrative portals.
        \item \textbf{Priority:} Immediate. This control is essential for protecting the organization's "crown jewels."
    \end{itemize}
    \vspace{0.5cm}
    \item \textbf{Upgrade Web Server Software (RISK-003):}
    \begin{itemize}
        \item \textbf{Action:} Plan and execute an upgrade of the Nginx web server on host \texttt{192.168.10.5} from version 1.18.0 to the latest stable version.
        \item \textbf{Priority:} High. Before upgrading, test the new version in a staging environment to ensure compatibility with existing web applications. Establish a regular patch management cycle for all internet-facing systems.
    \end{itemize}
\end{enumerate}

\end{document}
```