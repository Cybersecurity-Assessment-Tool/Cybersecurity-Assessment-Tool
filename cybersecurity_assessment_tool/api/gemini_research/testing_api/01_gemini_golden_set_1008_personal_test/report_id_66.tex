```latex
\documentclass[12pt]{article}

% Preamble: Required Packages
\usepackage[margin=1in]{geometry}
\usepackage{pifont} % For checkmarks and crosses
\usepackage{booktabs} % For professional tables
\usepackage{hyperref} % For clickable links
\usepackage{url} % For URL formatting
\usepackage{seqsplit} % To split long strings in texttt
\usepackage{graphicx} % For potential logos in the future
\usepackage{xcolor} % For colors

% Document Information
\title{Cybersecurity Assessment Report}
\author{Cybersecurity Analysis Division}
\date{\today}

% Hyperref Setup
\hypersetup{
    colorlinks=true,
    linkcolor=blue,
    filecolor=magenta,      
    urlcolor=cyan,
    pdftitle={Cybersecurity Assessment Report},
    pdfpagemode=FullScreen,
}

\begin{document}

\maketitle
\thispagestyle{empty}
\newpage

\tableofcontents
\newpage

% --- 1. Executive Overview ---
\section{Executive Overview}
This report details the findings of a cybersecurity assessment conducted for \textbf{True North Travel}. The assessment incorporated a review of organizational security controls, an external network scan, and a correlation with pre-existing risk data.

The analysis revealed several critical and high-risk gaps in the organization's security posture, primarily related to administrative and access controls. Key findings include:
\begin{itemize}
    \item \textbf{Critical Gaps in Access Control:} Multi-Factor Authentication (MFA) is not enforced for accessing email or other sensitive data systems. This significantly increases the risk of account compromise and unauthorized data access.
    \item \textbf{High-Risk Policy Deficiencies:} The organization lacks a formal employee Acceptable Use Policy (AUP) and does not provide mandatory annual security awareness training for all staff. These gaps weaken the human element of security, leaving the organization vulnerable to social engineering and insider threats.
    \item \textbf{Technical Scan Anomaly:} The network scan of the target host \texttt{192.168.0.5} did not identify any open ports. This contradicts a pre-existing risk report indicating an unencrypted web server on port 80. This discrepancy requires further investigation to determine if the risk has been remediated or applies to a different asset.
\end{itemize}

Immediate remediation efforts should focus on implementing MFA across all critical systems. Concurrently, the development and enforcement of foundational security policies and training programs are essential to establishing a robust and resilient security posture.

% --- 2. Organizational Information ---
\section{Organizational Information}
The following details were provided for the assessment. This information is used to establish the context and scope of the review.

\begin{tabular}{@{}ll}
\toprule
\textbf{Attribute} & \textbf{Value} \\
\midrule
Organization Name & \textbf{True North Travel} \\
Email Domain & \texttt{TrueNorthTravel.net} \\
Website Domain & \seqsplit{\url{www.TrueNorthTravel.net}} \\
External IP Address & \texttt{173.121.46.191} \\
\bottomrule
\end{tabular}

% --- 3. Security Control Review ---
\section{Security Control Review}
A review of administrative security controls was conducted based on a standardized questionnaire. The responses indicate significant gaps in access control and security governance.

\begin{tabular}{@{}p{0.6\linewidth}cc}
\toprule
\textbf{Control Question} & \textbf{Response} & \textbf{Assessment} \\
\midrule
Do you require MFA to access email? & \textcolor{red}{\ding{55}} & \textbf{Critical Gap} \\
Do you require MFA to log into computers? & \textcolor{green}{\ding{51}} & Best Practice Met \\
Do you require MFA to access sensitive data systems? & \textcolor{red}{\ding{55}} & \textbf{Critical Gap} \\
Does your organization have an employee acceptable use policy? & \textcolor{red}{\ding{55}} & \textbf{High Risk} \\
Does your organization do security awareness training for new employees? & \textcolor{green}{\ding{51}} & Best Practice Met \\
Does your organization do security awareness training for all employees at least once per year? & \textcolor{red}{\ding{55}} & \textbf{High Risk} \\
\bottomrule
\end{tabular}

% --- 4. Technical Scan Results ---
\section{Technical Scan Results}
A network scan was performed to identify open ports and exposed services on the specified target system.

\begin{itemize}
    \item \textbf{Scan Target:} \texttt{192.168.0.5}
    \item \textbf{Scan Tool:} Nmap
\end{itemize}

The scan reported that the host was online but did not discover any open ports. The state of all scanned ports was reported as `closed`.

\begin{tabular}{@{}lll}
\toprule
\textbf{Port} & \textbf{State} & \textbf{Service/Product/Version} \\
\midrule
80/tcp & closed & http \\
\bottomrule
\end{tabular}

\textbf{Analyst Note:} A `closed` port indicates that the host responded to the scan probe but that no application was listening on that port. This result does not confirm the pre-existing risk of an "Unencrypted Web Server" on this specific target.

% --- 5. Correlated Risk Assessment ---
\section{Correlated Risk Assessment}
The following table synthesizes findings from the security control review, technical scan, and pre-existing risk data to provide a consolidated view of the current risk posture.

\begin{tabular}{@{}p{0.3\linewidth}p{0.5\linewidth}l}
\toprule
\textbf{Risk Name} & \textbf{Overview} & \textbf{Severity} \\
\midrule
\textbf{Inadequate MFA Coverage} & Lack of MFA for email and sensitive data systems exposes the organization to severe risks of business email compromise, data breaches, and ransomware attacks. & \textbf{Critical} \\
\addlinespace
\textbf{Lack of Foundational Security Policies} & The absence of an Acceptable Use Policy (AUP) creates ambiguity regarding employee responsibilities for protecting company assets and data, increasing insider risk. & \textbf{High} \\
\addlinespace
\textbf{Insufficient Security Awareness Training} & Without mandatory annual training, employees may not recognize or properly respond to evolving threats like phishing and social engineering. & \textbf{High} \\
\addlinespace
\textbf{Unencrypted Web Server} \textit{(Unconfirmed)} & A pre-existing risk indicated that Port 80 was open. The current scan on \texttt{192.168.0.5} found this port to be closed. The status of this risk is inconclusive and requires further investigation. & Medium \\
\bottomrule
\end{tabular}

% --- 6. Recommendations ---
\section{Recommendations}
Based on the correlated risk assessment, the following actions are recommended to mitigate the identified vulnerabilities and improve the overall security posture of \textbf{True North Travel}.

\subsection{Immediate Priority (0-30 Days)}
\begin{enumerate}
    \item \textbf{Enforce MFA on Critical Systems:} Immediately enable and enforce MFA for all users on the email platform (\texttt{TrueNorthTravel.net}) and all systems identified as containing sensitive data. This is the single most effective control to prevent unauthorized access.
\end{enumerate}

\subsection{High Priority (30-90 Days)}
\begin{enumerate}
    \item \textbf{Develop and Implement an Acceptable Use Policy (AUP):} Create a formal AUP that clearly defines the rules for using company networks, systems, and data. Require all employees to read and acknowledge the policy.
    \item \textbf{Establish Annual Security Awareness Training:} Implement a mandatory security awareness training program for all employees to be completed annually. The training should cover phishing, password security, social engineering, and the new AUP.
\end{enumerate}

\subsection{Investigation and Follow-up}
\begin{enumerate}
    \item \textbf{Verify Web Server Risk Status:} Investigate the "Unencrypted Web Server" finding. Determine if the risk applies to a different asset, if it was recently remediated, or if the initial report was inaccurate. All web traffic should be encrypted using TLS (HTTPS).
\end{enumerate}

\end{document}
```