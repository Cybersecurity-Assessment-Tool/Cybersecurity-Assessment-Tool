```latex
\documentclass[12pt]{article}

% Preamble: Required Packages
\usepackage[margin=1in]{geometry}
\usepackage{pifont} % For \ding
\usepackage{booktabs} % For professional tables (\toprule, \midrule, \bottomrule)
\usepackage{hyperref} % For clickable links and metadata
\usepackage{url} % For formatting URLs
\usepackage{seqsplit} % To split long strings in tt font
\usepackage{graphicx}
\usepackage[table]{xcolor} % For coloring table cells

% --- Document Metadata ---
\hypersetup{
    colorlinks=true,
    linkcolor=blue,
    filecolor=magenta,      
    urlcolor=cyan,
    pdftitle={Cybersecurity Assessment Report},
    pdfauthor={Cybersecurity Analyst Group},
    pdfsubject={Security Assessment},
    pdfkeywords={Cybersecurity, Pentest, Report},
}

% --- Custom Commands & Colors ---
\newcommand{\yes}{\ding{51}} % Checkmark
\newcommand{\no}{\ding{55}}  % X-mark
\definecolor{sev_critical}{HTML}{D10000}
\definecolor{sev_high}{HTML}{FF8C00}
\definecolor{tablegray}{gray}{0.9}

% --- Document Start ---
\begin{document}

% --- Title Page ---
\title{Cybersecurity Assessment Report \\ \large For Nexus Dynamics}
\author{Cybersecurity Analyst Group}
\date{\today}
\maketitle

\thispagestyle{empty}
\newpage

% --- Table of Contents ---
\tableofcontents
\newpage

% --- Section 1: Executive Summary ---
\section{Executive Summary}
This report details the findings of a cybersecurity assessment conducted for Nexus Dynamics. The assessment combined a review of organizational security controls, an analysis of pre-existing risks, and a technical network scan to provide a holistic view of the organization's security posture.

While the organization demonstrates a solid foundation in security awareness training and has implemented Multi-Factor Authentication (MFA) for email and computer access, two significant risks were identified that require immediate attention. 

\begin{itemize}
    \item \textbf{Critical Risk - Lack of MFA on Sensitive Systems:} The absence of mandatory MFA for accessing sensitive data systems represents a critical gap in access control. This significantly increases the risk of a data breach if user credentials are compromised.
    \item \textbf{Critical Risk - Exposed Localhost Service:} The technical scan confirmed a pre-existing risk related to an exposed service on the local loopback interface (\texttt{127.0.0.1}). An open SSH port (22) was detected, indicating a potential misconfiguration that could be exploited if the host's networking is ever improperly configured.
\end{itemize}

Overall, the security posture is mixed. Foundational controls are in place, but critical vulnerabilities in access control and network configuration undermine these strengths. Recommendations provided in this report focus on mitigating these high-impact risks to bolster the organization's defenses against common cyber threats.

% --- Section 2: Organizational Information ---
\section{Organizational Information}
The following information was provided for the assessment.
\begin{center}
\begin{tabular}{ll}
\toprule
\textbf{Attribute} & \textbf{Value} \\
\midrule
Organization Name & \textbf{Nexus Dynamics} \\
Email Domain & \texttt{NexusDynamics.net} \\
Website Domain & \texttt{www.NexusDynamics.net} \\
External IP Address & \texttt{142.120.164.2} \\
\bottomrule
\end{tabular}
\end{center}

% --- Section 3: Security Control Review ---
\section{Security Control Review}
A review of organizational security controls was conducted based on a standard questionnaire. The results highlight a critical gap in the application of Multi-Factor Authentication (MFA).

\begin{center}
\rowcolors{2}{tablegray}{white}
\begin{tabular}{p{0.6\textwidth} c p{0.2\textwidth}}
\toprule
\textbf{Control Question} & \textbf{Response} & \textbf{Analyst Assessment} \\
\midrule
Do you require MFA to access email? & \yes & Best Practice Met \\
Do you require MFA to log into computers? & \yes & Best Practice Met \\
Do you require MFA to access sensitive data systems? & \no & \textbf{Critical Gap} \\
Does your organization have an employee acceptable use policy? & \yes & Best Practice Met \\
Does your organization do security awareness training for new employees? & \yes & Best Practice Met \\
Does your organization do security awareness training for all employees at least once per year? & \yes & Best Practice Met \\
\bottomrule
\end{tabular}
\end{center}

% --- Section 4: Technical Scan Results ---
\section{Technical Scan Results}
A network scan was performed to identify open ports and exposed services on the target system.

\begin{itemize}
    \item \textbf{Scan Target:} \texttt{127.0.0.1}
    \item \textbf{Scan Date:} \today
\end{itemize}

\subsection{Open Ports}
The scan identified the following open port. The service information is inferred from the standard port assignment, as the scan did not return detailed service version data.

\begin{center}
\begin{tabular}{llll}
\toprule
\textbf{Port} & \textbf{State} & \textbf{Service (Inferred)} & \textbf{Product / Version} \\
\midrule
22/tcp & open & SSH & Not Available \\
\bottomrule
\end{tabular}
\end{center}

\subsection{Analysis}
The presence of an open SSH port on the loopback interface (\texttt{127.0.0.1}) confirms the pre-existing risk "Localhost Exposed". While this service is not directly accessible from the internet, it represents a significant misconfiguration. If this host were ever to have its network interfaces bridged or its firewall rules misconfigured, this service could become unintentionally exposed. This finding validates the concern and elevates the need for remediation.

% --- Section 5: Consolidated Risk Assessment ---
\section{Consolidated Risk Assessment}
The following table synthesizes findings from the security control review, technical scan, and pre-existing risk data.

\begin{center}
\begin{tabular}{p{0.1\textwidth} p{0.2\textwidth} p{0.15\textwidth} p{0.45\textwidth}}
\toprule
\textbf{Risk ID} & \textbf{Risk Name} & \textbf{Severity} & \textbf{Description} \\
\midrule
RISK-001 & Localhost Exposed & \cellcolor{sev_critical!80}\color{white}\textbf{Critical (10.0)} & An SSH service is running and exposed on the local loopback interface (\texttt{127.0.0.1}). This is a significant misconfiguration that could lead to unauthorized access if firewall or network settings are altered. This finding was confirmed by the technical scan. \\
\addlinespace
RISK-002 & Lack of MFA on Sensitive Systems & \cellcolor{sev_high!80}\color{white}\textbf{High} & The organization does not enforce MFA for access to systems containing sensitive data. This control gap dramatically increases the risk of a data breach from credential theft, phishing, or password spraying attacks. \\
\bottomrule
\end{tabular}
\end{center}

% --- Section 6: Recommendations ---
\section{Recommendations}
Based on the consolidated risk assessment, the following actions are recommended to improve the security posture of Nexus Dynamics.

\subsection{RISK-001: Localhost Exposed (Critical)}
\begin{itemize}
    \item \textbf{Immediate Action:} Review the configuration of the service running on port 22 of \texttt{127.0.0.1}. If this service is not required, it should be disabled immediately.
    \item \textbf{Long-Term Fix:} If the service is required for local administrative purposes, ensure it is bound explicitly to the loopback interface and that host-based firewall rules (e.g., `iptables`, `ufw`, Windows Firewall) are configured to block any non-local access to this port.
\end{itemize}

\subsection{RISK-002: Lack of MFA on Sensitive Systems (High)}
\begin{itemize}
    \item \textbf{Immediate Action:} Prioritize the rollout of MFA for all systems classified as containing sensitive or critical data. This includes databases, file shares, administrative portals, and cloud infrastructure consoles.
    \item \textbf{Policy Update:} Update the organization's access control policy to mandate the use of MFA for all new systems that will store or process sensitive information.
\end{itemize}

\end{document}
```