```latex
\documentclass[12pt]{article}

% --- PACKAGES ---
\usepackage[margin=1in]{geometry}
\usepackage{pifont} % For checkmarks and crosses
\usepackage{booktabs} % For professional tables
\usepackage{hyperref} % For clickable links
\usepackage{url} % For URL formatting
\usepackage{seqsplit} % To split long strings in tt font
\usepackage[utf8]{inputenc}

% --- DOCUMENT METADATA ---
\title{Cybersecurity Posture Assessment Report}
\author{Cybersecurity Analysis Division}
\date{\today}

\begin{document}

\maketitle
\thispagestyle{empty}
\newpage
\tableofcontents
\thispagestyle{empty}
\newpage
\setcounter{page}{1}

% --- SECTION 1: EXECUTIVE SUMMARY ---
\section{Executive Summary}
This report provides a comprehensive cybersecurity assessment for \textbf{Radiant Life}, based on a technical network scan, a review of existing risks, and an analysis of organizational security controls.

The assessment reveals a mixed security posture. The organization demonstrates positive security practices in key areas, including the mandatory use of Multi-Factor Authentication (MFA) for email and computer access, and a consistent security awareness training program. Furthermore, a technical scan confirmed that a previously identified risk involving an open, unencrypted web server port (Port 80) has been successfully remediated.

However, critical gaps were identified that expose the organization to significant risk. The two most pressing issues are:
\begin{enumerate}
    \item \textbf{Lack of MFA for Sensitive Data Systems:} The absence of mandatory MFA for systems containing sensitive data is a critical vulnerability, significantly increasing the risk of a data breach through compromised credentials.
    \item \textbf{Absence of an Acceptable Use Policy (AUP):} The lack of a formal AUP creates ambiguity regarding the proper use of company assets, increasing the potential for insider threats and non-compliance.
\end{enumerate}

Immediate action is recommended to address these high-risk findings to strengthen the organization's overall defense against common cyber threats.

% --- SECTION 2: ORGANIZATIONAL INFORMATION ---
\section{Organizational Information}
The following details were provided for the assessment.

\begin{tabular}{@{}ll}
    \toprule
    \textbf{Attribute} & \textbf{Value} \\
    \midrule
    Organization Name & \textbf{Radiant Life} \\
    Email Domain & \seqsplit{\texttt{RadiantLife.net}} \\
    Website Domain & \seqsplit{\texttt{www.RadiantLife.net}} \\
    External IP Address & \seqsplit{\texttt{153.56.73.52}} \\
    \bottomrule
\end{tabular}

% --- SECTION 3: SECURITY CONTROL REVIEW ---
\section{Security Control Review}
An analysis of the organization's self-reported security controls was conducted. The findings below highlight both strengths and critical areas for improvement. A green checkmark (\ding{51}) indicates an implemented control, while a red 'X' (\ding{55}) indicates a gap.

\begin{tabular}{@{}p{0.6\textwidth}cp{0.2\textwidth}@{}}
    \toprule
    \textbf{Control Question} & \textbf{Response} & \textbf{Assessment} \\
    \midrule
    Do you require MFA to access email? & \ding{51} Yes & Best Practice Met \\
    Do you require MFA to log into computers? & \ding{51} Yes & Best Practice Met \\
    Do you require MFA to access sensitive data systems? & \ding{55} No & \textbf{Critical Gap} \\
    Does your organization have an employee acceptable use policy? & \ding{55} No & \textbf{High Risk} \\
    Does your organization do security awareness training for new employees? & \ding{51} Yes & Best Practice Met \\
    Does your organization do security awareness training for all employees at least once per year? & \ding{51} Yes & Best Practice Met \\
    \bottomrule
\end{tabular}

% --- SECTION 4: TECHNICAL SCAN RESULTS ---
\section{Technical Scan Results}
A network scan was performed to identify externally exposed services and potential vulnerabilities.

\begin{itemize}
    \item \textbf{Target IP Address:} \seqsplit{\texttt{192.168.0.5}}
    \item \textbf{Scan Date:} \today
\end{itemize}

The scan results are summarized in the table below.

\begin{tabular}{@{}llll@{}}
    \toprule
    \textbf{Port} & \textbf{State} & \textbf{Service} & \textbf{Notes} \\
    \midrule
    80 & Closed & http & No unencrypted web service detected. \\
    \bottomrule
\end{tabular}

\subsection*{Analysis}
The technical scan found no open ports on the target system. Specifically, Port 80 (HTTP) was found to be closed. This is a positive finding, as it mitigates risks associated with unencrypted web traffic. This result indicates that the previously documented risk, "Unencrypted Web Server," has been successfully remediated.

% --- SECTION 5: CONSOLIDATED RISK ASSESSMENT ---
\section{Consolidated Risk Assessment}
The following table synthesizes findings from the security control review, technical scan, and pre-existing risk data into a consolidated list.

\begin{tabular}{@{}p{0.3\textwidth}p{0.5\textwidth}l@{}}
    \toprule
    \textbf{Risk Name} & \textbf{Description} & \textbf{Severity} \\
    \midrule
    \textbf{Lack of MFA for Sensitive Data Systems} & The absence of a second authentication factor for critical systems makes them highly vulnerable to unauthorized access via stolen credentials. & \textbf{High} \\
    \textbf{No Employee Acceptable Use Policy (AUP)} & Without a formal policy, there is no clear guidance for employees on safe and acceptable use of corporate resources, increasing insider risk. & \textbf{Medium} \\
    Unencrypted Web Server & Port 80 was previously identified as open, exposing the organization to unencrypted communications. & \textit{Remediated} \\
    \bottomrule
\end{tabular}

% --- SECTION 6: RECOMMENDATIONS ---
\section{Recommendations}
Based on the consolidated risk assessment, the following actions are recommended to improve the security posture of \textbf{Radiant Life}.

\subsection*{High Priority Recommendations}
\begin{itemize}
    \item \textbf{Implement MFA for Sensitive Data Systems:}
    \begin{itemize}
        \item \textbf{Immediate Action:} Inventory all applications and systems that store, process, or transmit sensitive data (e.g., financial, customer, employee data).
        \item \textbf{Short-Term Goal:} Prioritize the inventory and deploy MFA on the most critical systems within the next 90 days.
        \item \textbf{Long-Term Strategy:} Establish a policy requiring MFA as a default security control for any new system handling sensitive information.
    \end{itemize}
\end{itemize}

\subsection*{Medium Priority Recommendations}
\begin{itemize}
    \item \textbf{Develop and Implement an Acceptable Use Policy (AUP):}
    \begin{itemize}
        \item \textbf{Immediate Action:} Draft a formal AUP that clearly defines rules for employee use of company networks, devices, email, and internet access.
        \item \textbf{Short-Term Goal:} Have the draft AUP reviewed by management and legal/HR. Upon approval, distribute the policy to all employees and obtain signed acknowledgments.
        \item \textbf{Long-Term Strategy:} Integrate the AUP review and acknowledgment into the new employee onboarding process and conduct annual refresher training for all staff.
    \end{itemize}
\end{itemize}

\end{document}
```