```latex
\documentclass[12pt]{article}

% --- PACKAGES ---
\usepackage[margin=1in]{geometry}
\usepackage{pifont} % For checkmarks and crosses
\usepackage{booktabs} % For professional tables
\usepackage{hyperref} % For hyperlinks
\usepackage{url} % For URL formatting
\usepackage{seqsplit} % For splitting long text strings
\usepackage{xcolor} % For colors

% --- DOCUMENT METADATA ---
\title{Cybersecurity Posture and Risk Analysis Report}
\author{Cybersecurity Analysis Division}
\date{\today}

% --- HYPERREF SETUP ---
\hypersetup{
    colorlinks=true,
    linkcolor=blue,
    filecolor=magenta,      
    urlcolor=cyan,
    pdftitle={Cybersecurity Posture and Risk Analysis Report},
    pdfpagemode=FullScreen,
}

\begin{document}

\maketitle
\tableofcontents
\newpage

% ==============================================================================
% SECTION 1: EXECUTIVE SUMMARY
% ==============================================================================
\section{Executive Summary}

This report provides a comprehensive analysis of the cybersecurity posture for \textbf{Nebula Creative}. The assessment is based on a synthesis of technical network scans, a review of organizational security controls, and an evaluation of pre-existing risk documentation.

While the organization demonstrates a solid foundation in several key areas—including the enforcement of Multi-Factor Authentication (MFA) for email and computer access, and a consistent security awareness training program—a critical vulnerability has been identified that requires immediate attention.

A technical scan revealed an open port (8080) on an internal system (\texttt{10.5.5.5}) exposing a service with the title \textbf{"TOP SECRET DB"}. This finding directly correlates with a gap identified in the security controls questionnaire, where the organization confirmed it does not require MFA for accessing sensitive data systems.

Critically, this active finding contradicts a pre-existing risk assessment entry which incorrectly labeled Port 8080 as secure and a "false positive." This discrepancy highlights not only a severe technical vulnerability but also a potential flaw in the ongoing risk management process.

Immediate remediation is required to restrict access to this exposed system. Recommendations are detailed in Section 6 of this report.

% ==============================================================================
% SECTION 2: ORGANIZATIONAL INFORMATION
% ==============================================================================
\section{Organizational Information}

The following information was provided for the assessment.

\begin{tabular}{@{}ll}
\toprule
\textbf{Attribute} & \textbf{Value} \\
\midrule
Organization Name & \textbf{Nebula Creative} \\
Email Domain & \texttt{NebulaCreative.org} \\
Website Domain & \url{www.NebulaCreative.org} \\
External IP Address & \texttt{228.95.176.44} \\
\bottomrule
\end{tabular}

% ==============================================================================
% SECTION 3: SECURITY CONTROL REVIEW
% ==============================================================================
\section{Security Control Review (Questionnaire Analysis)}

The following table summarizes the organization's self-reported security controls. "Yes" answers indicate a control is in place, while "No" answers represent a potential gap in the security framework.

\begin{tabular}{@{}p{0.6\linewidth} c p{0.25\linewidth}@{}}
\toprule
\textbf{Control Question} & \textbf{Response} & \textbf{Analyst Notes} \\
\midrule
Do you require MFA to access email? & \textcolor{green}{\ding{51}} & Strong control. \\
Do you require MFA to log into computers? & \textcolor{green}{\ding{51}} & Strong control. \\
\textbf{Do you require MFA to access sensitive data systems?} & \textcolor{red}{\ding{55}} & \textbf{Critical Gap.} Lack of MFA on sensitive systems is a high-risk exposure. \\
Does your organization have an employee acceptable use policy? & \textcolor{green}{\ding{51}} & Foundational policy is in place. \\
Does your organization do security awareness training for new employees? & \textcolor{green}{\ding{51}} & Good practice for onboarding. \\
Does your organization do security awareness training for all employees at least once per year? & \textcolor{green}{\ding{51}} & Meets compliance standards. \\
\bottomrule
\end{tabular}

\subsection*{Analysis}
The questionnaire reveals a mature approach to endpoint and email security. However, the lack of mandatory MFA for sensitive data systems is a significant weakness. This administrative gap directly enables the type of technical vulnerability discovered during the network scan.

% ==============================================================================
% SECTION 4: TECHNICAL SCAN RESULTS
% ==============================================================================
\section{Technical Scan Results}

A network scan was conducted to identify open ports and services on the target system.

\begin{itemize}
    \item \textbf{Target IP Address:} \texttt{10.5.5.5}
    \item \textbf{Scan Date:} \today
\end{itemize}

\begin{tabular}{@{}llll@{}}
\toprule
\textbf{Port} & \textbf{State} & \textbf{Service/Product} & \textbf{Details \& Analyst Notes} \\
\midrule
8080/tcp & OPEN & http-proxy & \textbf{Critical Finding:} An HTTP service title was discovered: \\
& & & \textbf{\texttt{"TOP SECRET DB"}}. This strongly suggests an \\
& & & exposed database or administrative interface containing \\
& & & highly sensitive information. \\
\bottomrule
\end{tabular}

\subsection*{Analysis}
The discovery of an open port exposing a service explicitly named "TOP SECRET DB" is a severe security issue. This finding, combined with the lack of MFA controls for sensitive systems, creates a high-impact risk scenario where an unauthorized user could potentially gain access to critical organizational data. This technical evidence directly contradicts the pre-existing risk documentation provided.

% ==============================================================================
% SECTION 5: CORRELATED RISK ASSESSMENT
% ==============================================================================
\section{Correlated Risk Assessment}

The following risks have been identified by correlating the security control gaps with the technical scan results. The pre-existing risk assessment, which listed Port 8080 as secure, is now considered inaccurate and has been superseded by the findings below.

\begin{tabular}{@{}p{0.1\linewidth} p{0.3\linewidth} p{0.1\linewidth} p{0.4\linewidth}@{}}
\toprule
\textbf{Risk ID} & \textbf{Risk Title} & \textbf{Severity} & \textbf{Description} \\
\midrule
RISK-001 & Unprotected Sensitive Data System Exposure & \textbf{Critical} & A system with a title indicating it contains "TOP SECRET" data is accessible via port 8080. This exposure is exacerbated by the organizational policy of not requiring MFA for sensitive systems. \\
\addlinespace
RISK-002 & Inaccurate Prior Risk Assessment & High & The existing risk documentation incorrectly classified port 8080 as a "false positive." This indicates a flaw in the risk validation process, creating a false sense of security and preventing remediation of a critical vulnerability. \\
\bottomrule
\end{tabular}

% ==============================================================================
% SECTION 6: RECOMMENDATIONS
% ==============================================================================
\section{Recommendations}

The following actions are recommended to mitigate the identified risks.

\subsection*{Immediate Actions (Within 24 Hours)}
\begin{enumerate}
    \item \textbf{Restrict Network Access:} Immediately apply firewall rules to block all access to port 8080 on host \texttt{10.5.5.5} from any network except for a limited set of administrative IP addresses.
    \item \textbf{Investigate the System:} Identify the owner and purpose of the application running on port 8080. Determine the nature of the data it contains and review access logs for any signs of unauthorized activity.
\end{enumerate}

\subsection*{Short-Term Actions (1-2 Weeks)}
\begin{enumerate}
    \item \textbf{Enforce MFA:} Implement mandatory Multi-Factor Authentication for the application running on port 8080. If the application does not natively support MFA, use an authenticating proxy or alternative access control solution.
    \item \textbf{Update Policy:} Revise the security policy to mandate MFA for all systems that store, process, or transmit sensitive organizational data.
\end{enumerate}

\subsection*{Long-Term Actions (1-3 Months)}
\begin{enumerate}
    \item \textbf{Review Risk Management Process:} Conduct a thorough review of the risk assessment and validation process. Ensure that technical validation (e.g., vulnerability scanning) is required before any high-risk finding can be closed or accepted as a false positive.
    \item \textbf{Conduct Architecture Review:} Perform a comprehensive security architecture review of all systems classified as sensitive to ensure they are properly segmented, monitored, and protected with appropriate access controls.
\end{enumerate}

\end{document}
```