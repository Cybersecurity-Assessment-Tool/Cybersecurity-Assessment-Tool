```latex
\documentclass[12pt]{article}

% -----------------------------------------------------------------------------
% PREAMBLE
% -----------------------------------------------------------------------------
\usepackage[margin=1in]{geometry}
\usepackage{pifont} % For checkmarks and crosses
\usepackage{booktabs} % For professional tables
\usepackage{hyperref} % For clickable links
\usepackage{url} % For URL formatting
\usepackage{seqsplit} % For splitting long strings like IPs
\usepackage{fancyhdr} % For headers and footers
\usepackage[utf8]{inputenc}

% --- Document Metadata ---
\title{Cybersecurity Posture Assessment Report}
\author{Cybersecurity Analysis Division}
\date{\today}

% --- Hyperref Setup ---
\hypersetup{
    colorlinks=true,
    linkcolor=black,
    urlcolor=blue,
    pdftitle={Cybersecurity Posture Assessment Report},
    pdfauthor={Cybersecurity Analysis Division},
}

% --- Header and Footer Setup ---
\pagestyle{fancy}
\fancyhf{} % Clear all header and footer fields
\fancyhead[L]{Nexus Dynamics // Confidential}
\fancyfoot[C]{\thepage}
\renewcommand{\headrulewidth}{0.4pt}
\renewcommand{\footrulewidth}{0.4pt}

% -----------------------------------------------------------------------------
% DOCUMENT START
% -----------------------------------------------------------------------------
\begin{document}

\maketitle
\thispagestyle{empty}
\newpage

\tableofcontents
\newpage

% -----------------------------------------------------------------------------
% SECTION 1: EXECUTIVE OVERVIEW
% -----------------------------------------------------------------------------
\section{Executive Overview}

This report provides a comprehensive analysis of the cybersecurity posture for \textbf{Nexus Dynamics}, based on a review of organizational security controls, an external network scan, and pre-existing risk data. The assessment synthesizes these data points to identify key vulnerabilities, evaluate their potential impact, and provide actionable recommendations for remediation.

The overall security posture reveals a mix of implemented controls and critical gaps. While the organization has successfully deployed Multi-Factor Authentication (MFA) for computer and sensitive system access, several foundational controls are absent.

Key findings of this assessment include:
\begin{itemize}
    \item \textbf{Critical Risk: Lack of MFA on Email.} The absence of MFA on the primary communication platform (\texttt{NexusDynamics.org}) exposes the organization to significant risk from phishing, business email compromise (BEC), and account takeover attacks.
    \item \textbf{High Risk: Foundational Policy Gaps.} The organization lacks a formal Acceptable Use Policy and does not provide security awareness training to new employees. These gaps increase the likelihood of insider threats and successful social engineering attacks.
    \item \textbf{High Risk: Exposed Administrative Service.} An external network scan identified an open SSH port (22) on a public-facing IPv6 address. This service is a common target for brute-force and credential stuffing attacks.
\end{itemize}

Immediate remediation of these findings is strongly recommended to reduce the organization's attack surface and mitigate the risk of a significant security incident.

% -----------------------------------------------------------------------------
% SECTION 2: ORGANIZATIONAL INFORMATION
% -----------------------------------------------------------------------------
\section{Organizational Information}

The following details were provided for the assessment scope.

\begin{tabular}{@{}ll}
\toprule
\textbf{Attribute} & \textbf{Value} \\
\midrule
Organization Name & \textbf{Nexus Dynamics} \\
Email Domain & \texttt{NexusDynamics.org} \\
Website Domain & \url{www.NexusDynamics.org} \\
Known External IP & \texttt{164.170.224.144} \\
\bottomrule
\end{tabular}

% -----------------------------------------------------------------------------
% SECTION 3: SECURITY CONTROL REVIEW
% -----------------------------------------------------------------------------
\section{Security Control Review}

A review of self-reported security controls was conducted via a questionnaire. The responses highlight areas of both strength and weakness in the organization's security program. "No" answers indicate significant gaps that require immediate attention.

\begin{table}[h!]
\centering
\caption{Security Questionnaire Analysis}
\begin{tabular}{@{}p{0.6\linewidth} c l@{}}
\toprule
\textbf{Control Question} & \textbf{Response} & \textbf{Assessment} \\
\midrule
Do you require MFA to access email? & \ding{55} & \textbf{Critical Gap} \\
Do you require MFA to log into computers? & \ding{51} & Control in Place \\
Do you require MFA to access sensitive data systems? & \ding{51} & Control in Place \\
Does your organization have an employee acceptable use policy? & \ding{55} & \textbf{High Risk Gap} \\
Does your organization do security awareness training for new employees? & \ding{55} & \textbf{High Risk Gap} \\
Does your organization do security awareness training for all employees at least once per year? & \ding{51} & Control in Place \\
\bottomrule
\end{tabular}
\end{table}

% -----------------------------------------------------------------------------
% SECTION 4: TECHNICAL SCAN RESULTS
% -----------------------------------------------------------------------------
\section{Technical Scan Results}

An external network scan was performed to identify open ports and exposed services on the organization's perimeter.

\begin{itemize}
    \item \textbf{Target IP Address:} \seqsplit{\texttt{2001:db8::1}}
    \item \textbf{Scan Tool:} Nmap
    \item \textbf{Host Status:} Up
\end{itemize}

The scan revealed the following open port, which is accessible from the public internet.

\begin{table}[h!]
\centering
\caption{Open Port Analysis}
\begin{tabular}{@{}llll@{}}
\toprule
\textbf{Port} & \textbf{State} & \textbf{Service (Inferred)} & \textbf{Notes} \\
\midrule
22/tcp & open & SSH & Secure Shell is used for remote administration. \\
& & & Exposing this service to the internet makes it a \\
& & & prime target for brute-force attacks. \\
\bottomrule
\end{tabular}
\end{table}

\textit{Note: The scan did not include service version detection. Outdated versions of SSH may contain known vulnerabilities.}

% -----------------------------------------------------------------------------
% SECTION 5: CONSOLIDATED RISK ASSESSMENT
% -----------------------------------------------------------------------------
\section{Consolidated Risk Assessment}

This section correlates findings from the security control review and the technical scan. No pre-existing vulnerabilities were provided for this assessment.

\begin{table}[h!]
\centering
\caption{Identified Risks}
\begin{tabular}{@{}lp{0.4\linewidth}p{0.35\linewidth}l@{}}
\toprule
\textbf{ID} & \textbf{Risk Name} & \textbf{Description} & \textbf{Severity} \\
\midrule
RISK-001 & Lack of MFA on Email & Email accounts are protected only by passwords, making them highly susceptible to phishing, credential stuffing, and account takeover. & \textbf{Critical} \\
\addlinespace
RISK-002 & Exposed SSH Service & The SSH administrative port is open to the public internet, inviting automated brute-force attacks and targeted exploitation attempts. & \textbf{High} \\
\addlinespace
RISK-003 & Foundational Policy Gaps & The absence of an Acceptable Use Policy and security training for new hires creates an environment of high risk from insider threats and social engineering. & \textbf{High} \\
\bottomrule
\end{tabular}
\end{table}

% -----------------------------------------------------------------------------
% SECTION 6: RECOMMENDATIONS
% -----------------------------------------------------------------------------
\section{Recommendations}

The following actions are recommended to mitigate the identified risks and improve the overall security posture of \textbf{Nexus Dynamics}.

\begin{enumerate}
    \item \textbf{Remediate RISK-001 (Lack of MFA on Email):}
    \begin{itemize}
        \item \textbf{Immediate Action:} Procure and implement a Multi-Factor Authentication solution for the \texttt{NexusDynamics.org} email system.
        \item \textbf{Policy:} Enforce MFA for all users, including administrative and service accounts, without exception.
        \item \textbf{Justification:} This is the single most effective control to prevent unauthorized access to email, which is a gateway to further network compromise.
    \end{itemize}
    \vspace{1em}
    \item \textbf{Remediate RISK-002 (Exposed SSH Service):}
    \begin{itemize}
        \item \textbf{Immediate Action:} Implement a firewall rule to restrict access to TCP port 22 on \seqsplit{\texttt{2001:db8::1}}. Access should be limited to specific, trusted IP addresses (e.g., corporate VPN or administrator locations).
        \item \textbf{Hardening:} If public access is unavoidable, enforce public key authentication and completely disable password-based logins for SSH.
        \item \textbf{Justification:} This action significantly reduces the attack surface by preventing automated scanners and attackers from attempting to compromise the server.
    \end{itemize}
    \vspace{1em}
    \item \textbf{Remediate RISK-003 (Foundational Policy Gaps):}
    \begin{itemize}
        \item \textbf{Immediate Action:} Develop and implement a mandatory security awareness training module as part of the onboarding process for all new employees.
        \item \textbf{Policy:} Draft and ratify a formal Acceptable Use Policy (AUP) that clearly defines the rules for using company assets and data. Require all employees to read and acknowledge the policy annually.
        \item \textbf{Justification:} These foundational controls establish a security-aware culture, clarify employee responsibilities, and provide a legal and operational framework for enforcing security standards.
    \end{itemize}
\end{enumerate}

\end{document}
```