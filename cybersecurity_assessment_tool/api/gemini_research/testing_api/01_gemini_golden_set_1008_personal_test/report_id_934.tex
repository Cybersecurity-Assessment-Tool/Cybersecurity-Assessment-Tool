```latex
\documentclass[12pt]{article}

% Preamble: Required Packages
\usepackage[margin=1in]{geometry}
\usepackage{pifont} % For checkmarks and crosses
\usepackage{booktabs} % For professional tables
\usepackage{hyperref} % For clickable links
\usepackage{url} % For formatting URLs
\usepackage{seqsplit} % For splitting long strings in tt font
\usepackage{graphicx}
\usepackage{xcolor}

% Document Metadata
\title{Cybersecurity Posture Report for Crestview Analytics}
\author{Cybersecurity Analyst}
\date{\today}

% Hyperref Setup
\hypersetup{
    colorlinks=true,
    linkcolor=blue,
    filecolor=magenta,      
    urlcolor=cyan,
    pdftitle={Cybersecurity Posture Report},
    pdfpagemode=FullScreen,
}

\begin{document}

\maketitle
\tableofcontents
\newpage

% --- Section 1: Executive Summary ---
\section{Executive Summary}
This report provides a comprehensive analysis of the cybersecurity posture of Crestview Analytics, based on a review of organizational security controls, a technical network scan, and pre-existing risk data.

The assessment reveals a mixed security posture. On the technical front, the external scan of the target IP address \texttt{[Target IP]} showed no open ports, which indicates a strong firewall configuration for that specific host. This is a positive finding that reduces the immediate external attack surface.

However, the organizational security control review identified several critical and high-risk gaps. The most significant concerns are the lack of Multi-Factor Authentication (MFA) for computer logins and access to sensitive data systems. This exposes the organization to significant risk from credential theft and unauthorized access. Furthermore, the absence of mandatory annual security awareness training for all employees creates a vulnerability to social engineering and phishing attacks.

Overall, while the perimeter security of the scanned asset appears robust, critical internal and procedural controls are lacking. Immediate remediation of the identified MFA and training gaps is strongly recommended to mitigate these risks and improve the organization's resilience against common cyber threats.

% --- Section 2: Organizational Information ---
\section{Organizational Information}
The following information was provided for the assessment.
\begin{itemize}
    \item \textbf{Organization Name:} Crestview Analytics
    \item \textbf{Email Domain:} \texttt{CrestviewAnalytics.net}
    \item \textbf{Website Domain:} \url{www.CrestviewAnalytics.net}
    \item \textbf{External IP Assessed:} \texttt{68.232.251.170}
\end{itemize}

% --- Section 3: Security Control Review ---
\section{Security Control Review}
An assessment of key organizational security controls was conducted via a questionnaire. The responses are summarized below. Gaps identified with a \ding{55} represent significant weaknesses in the current security posture.

\begin{table}[h!]
\centering
\caption{Organizational Security Control Questionnaire}
\label{tab:controls}
\begin{tabular}{p{0.75\linewidth} c}
\toprule
\textbf{Control Question} & \textbf{Response} \\
\midrule
Do you require MFA to access email? & \ding{51} \\
Do you require MFA to log into computers? & \textcolor{red}{\ding{55}} \\
Do you require MFA to access sensitive data systems? & \textcolor{red}{\ding{55}} \\
Does your organization have an employee acceptable use policy? & \ding{51} \\
Does your organization do security awareness training for new employees? & \ding{51} \\
Does your organization do security awareness training for all employees at least once per year? & \textcolor{red}{\ding{55}} \\
\bottomrule
\end{tabular}
\end{table}

% --- Section 4: Technical Scan Results ---
\section{Technical Scan Results}
A network port scan was conducted to identify externally exposed services.

\begin{itemize}
    \item \textbf{Target IP Address:} \texttt{[Target IP]}
    \item \textbf{Scan Date:} \today
    \item \textbf{Scan Summary:} The scan completed successfully.
\end{itemize}

\subsection{Findings}
The network scan of the target IP address \textbf{found no open ports}.

\subsection{Analysis}
This result indicates a strong firewall configuration for the scanned host. No network services were detected as being exposed to the public internet at the time of the scan. While this is a positive security control, it is important to note that this assessment is a point-in-time snapshot of a single IP address and does not represent the security of the entire organization's network range.

% --- Section 5: Risk Assessment ---
\section{Risk Assessment}
This section synthesizes findings from the security control review, technical scan, and any pre-existing vulnerability data. The primary risks identified stem from gaps in administrative and procedural controls. No pre-existing vulnerabilities were provided for this assessment.

\begin{table}[h!]
\centering
\caption{Identified Risks and Severity}
\label{tab:risks}
\begin{tabular}{p{0.25\linewidth} p{0.5\linewidth} p{0.15\linewidth}}
\toprule
\textbf{Risk Name} & \textbf{Overview} & \textbf{Severity} \\
\midrule
\textbf{Lack of MFA for Endpoint and System Access} & The absence of MFA on computer logins and sensitive data systems makes the organization highly vulnerable to credential compromise. An attacker with valid credentials (e.g., from a phishing attack) could gain direct access to internal systems and data. & \textbf{Critical} \\
\addlinespace
\textbf{Inadequate Annual Security Awareness Training} & Without mandatory, recurring security training, employees are more likely to fall victim to phishing, social engineering, and other common attack vectors. This gap undermines the human element of the security program. & \textbf{High} \\
\bottomrule
\end{tabular}
\end{table}

% --- Section 6: Recommendations ---
\section{Recommendations}
Based on the findings of this assessment, the following actions are recommended to mitigate the identified risks and strengthen the overall security posture of Crestview Analytics.

\subsection{Critical Priority}
\begin{itemize}
    \item \textbf{Implement Comprehensive MFA:} Mandate and deploy Multi-Factor Authentication (MFA) across all employee computer logins (endpoints) and for all access to sensitive data systems. Prioritize remote access points (e.g., VPN) and systems containing financial, customer, or proprietary data.
\end{itemize}

\subsection{High Priority}
\begin{itemize}
    \item \textbf{Establish Annual Security Training:} Develop and implement a mandatory annual security awareness training program for all employees. The curriculum should be updated regularly to address current threats and should include topics such as phishing identification, password security, and secure data handling practices.
\end{itemize}

\subsection{Informational}
\begin{itemize}
    \item \textbf{Perform Continuous Monitoring:} While the initial scan was positive, security is not static. It is recommended to implement periodic, automated vulnerability scanning across the organization's entire external IP range to detect any new exposures or misconfigurations in a timely manner.
\end{itemize}

\end{document}
```