```latex
\documentclass[12pt]{article}

% --- PACKAGES ---
\usepackage[margin=1in]{geometry}
\usepackage{pifont} % For checkmarks and crosses
\usepackage{booktabs} % For professional tables
\usepackage{hyperref} % For clickable links
\usepackage{url}      % For URL formatting
\usepackage{seqsplit} % For splitting long strings to prevent overflow

% --- DOCUMENT METADATA ---
\title{Cybersecurity Posture Assessment Report}
\author{Cybersecurity Analysis Division}
\date{\today}

% --- DOCUMENT START ---
\begin{document}

\maketitle
\tableofcontents
\newpage

% ===================================================================
% SECTION 1: EXECUTIVE OVERVIEW
% ===================================================================
\section{Executive Overview}

This report provides a cybersecurity assessment for \textbf{Top Tier}, based on a technical network scan, a review of existing risks, and an analysis of organizational security controls.

The organization demonstrates a solid foundation in security policy and awareness training, with established acceptable use policies and regular training schedules. Multi-Factor Authentication (MFA) is commendably enforced for email and access to sensitive data systems.

However, two critical-risk findings require immediate attention. Firstly, the lack of mandatory MFA for computer logins represents a significant security gap. This weakness allows a compromised password to grant an attacker direct access to an endpoint. Secondly, a technical scan identified an open Remote Desktop Protocol (RDP) port on an internal system (\texttt{10.10.10.51}). This finding expands upon a previously identified risk of RDP exposure within the network.

The combination of these two findings creates a high-impact threat scenario where a single stolen credential could lead to unauthorized remote access and lateral movement within the internal network. Prioritized recommendations are provided to mitigate these risks and enhance the overall security posture.

% ===================================================================
% SECTION 2: ORGANIZATIONAL INFORMATION
% ===================================================================
\section{Organizational Information}

The following details were provided for the assessment.

\begin{tabular}{@{}ll}
\toprule
\textbf{Attribute} & \textbf{Value} \\
\midrule
Organization Name & \textbf{Top Tier} \\
Email Domain      & \texttt{TopTier.net} \\
Website Domain    & \texttt{www.TopTier.net} \\
External IP       & \texttt{117.31.178.130} \\
\bottomrule
\end{tabular}

% ===================================================================
% SECTION 3: SECURITY CONTROL REVIEW
% ===================================================================
\section{Security Control Review}

An assessment of administrative and technical security controls was conducted based on a standardized questionnaire. The results indicate a strong policy framework but highlight a critical gap in endpoint access controls.

\begin{tabular}{@{}p{0.8\linewidth}c@{}}
\toprule
\textbf{Control Question} & \textbf{Response} \\
\midrule
Do you require MFA to access email? & \ding{51} \\ % Yes
Do you require MFA to log into computers? & \textbf{\color{red}\ding{55}} \\ % No
Do you require MFA to access sensitive data systems? & \ding{51} \\ % Yes
Does your organization have an employee acceptable use policy? & \ding{51} \\ % Yes
Does your organization do security awareness training for new employees? & \ding{51} \\ % Yes
Does your organization do security awareness training for all employees at least once per year? & \ding{51} \\ % Yes
\bottomrule
\end{tabular}

\vspace{1em}
\noindent \textbf{Analysis:} The lack of MFA for computer logins is a critical weakness. While other controls are in place, a compromised user credential could allow an attacker to log into a corporate workstation without a second authentication factor, establishing a significant foothold within the network.

% ===================================================================
% SECTION 4: TECHNICAL SCAN RESULTS
% ===================================================================
\section{Technical Scan Results}

A network scan was performed on the specified target to identify open ports and exposed services.

\begin{itemize}
    \item \textbf{Target IP:} \texttt{10.10.10.51}
\end{itemize}

The following table details the services discovered during the scan.

\begin{tabular}{@{}llll@{}}
\toprule
\textbf{Port} & \textbf{State} & \textbf{Service Name} & \textbf{Analysis} \\
\midrule
3389/tcp & open & ms-wbt-server & Microsoft Remote Desktop Protocol (RDP) \\
\bottomrule
\end{tabular}

\vspace{1em}
\noindent \textbf{Analysis:} The scan confirmed that port 3389 is open, which is used for RDP. RDP is a frequent target for attackers attempting to gain unauthorized access to systems. When exposed, it can be vulnerable to brute-force password attacks, credential stuffing, and exploitation of vulnerabilities like BlueKeep. This finding confirms the presence of another system with RDP enabled, compounding the risk identified in previous assessments.

% ===================================================================
% SECTION 5: RISK ASSESSMENT
% ===================================================================
\section{Risk Assessment}

This section correlates the findings from the security control review, the technical scan, and pre-existing risk data.

\begin{tabular}{@{}p{0.2\linewidth}p{0.45\linewidth}p{0.15\linewidth}p{0.1\linewidth}@{}}
\toprule
\textbf{Risk Name} & \textbf{Description} & \textbf{Affected Systems} & \textbf{Severity} \\
\midrule
\textbf{Uncontrolled RDP Exposure} & The Remote Desktop Protocol is exposed on internal systems. This finding expands on a known issue, with a new host identified. This service is a primary target for ransomware and lateral movement. & \texttt{10.10.10.50} (Existing) \texttt{10.10.10.51} (New) & \textbf{Critical} \\
\addlinespace
\textbf{Lack of Endpoint MFA} & User workstations and laptops do not require Multi-Factor Authentication for login. A compromised password provides an attacker with direct, unimpeded access to the endpoint and any connected network resources. & All User Endpoints & \textbf{High} \\
\bottomrule
\end{tabular}

\vspace{1em}
\noindent \textbf{Correlated Threat:} The risk of \textbf{Uncontrolled RDP Exposure} is significantly amplified by the \textbf{Lack of Endpoint MFA}. An attacker who obtains a user's password through phishing or other means could use it to successfully connect to an exposed RDP service without being challenged by a second factor, leading to a complete system compromise.

% ===================================================================
% SECTION 6: RECOMMENDATIONS
% ===================================================================
\section{Recommendations}

The following actions are recommended to mitigate the identified risks. They are prioritized based on severity and potential impact.

\begin{enumerate}
    \item \textbf{[CRITICAL] Remediate RDP Exposure on All Systems:}
    \begin{itemize}
        \item Immediately review the business need for RDP on \texttt{10.10.10.51} and \texttt{10.10.10.50}.
        \item If RDP is not required, disable the service and block port 3389 via host-based or network firewalls.
        \item If RDP is required, place the systems behind a VPN or a secure remote access gateway that requires MFA. Direct access should never be permitted.
    \end{itemize}
    \vspace{1em}
    \item \textbf{[HIGH] Implement Mandatory MFA for Computer Logins:}
    \begin{itemize}
        \item Deploy an MFA solution for all Windows, macOS, and Linux endpoints (e.g., Windows Hello for Business, Duo, etc.).
        \item This control acts as a critical compensating measure, protecting against password compromise and preventing unauthorized endpoint access.
        \item Prioritize implementation for privileged users (administrators) and executives before rolling out to the entire organization.
    \end{itemize}
\end{enumerate}

\end{document}
```