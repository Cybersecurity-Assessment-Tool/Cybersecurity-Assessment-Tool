```latex
\documentclass[12pt]{article}

% --- PACKAGES ---
\usepackage[margin=1in]{geometry}
\usepackage{pifont} % For checkmarks and crosses
\usepackage{booktabs} % For professional tables
\usepackage{hyperref} % For clickable links
\usepackage{url} % For URL formatting
\usepackage{seqsplit} % To split long strings in texttt
\usepackage[T1]{fontenc}

% --- DOCUMENT METADATA ---
\hypersetup{
    colorlinks=true,
    linkcolor=black,
    urlcolor=blue,
    pdftitle={Cybersecurity Posture Assessment Report},
    pdfauthor={Cybersecurity Analyst},
    pdfsubject={Security Analysis},
    pdfkeywords={Cybersecurity, Nmap, Risk Assessment}
}

\newcommand{\yes}{\ding{51}}
\newcommand{\no}{\ding{55}}

\begin{document}

% --- TITLE PAGE ---
\title{
    Cybersecurity Posture Assessment Report \\
    \large Prepared for: \textbf{Pacific Rim Exports}
}
\author{Cybersecurity Analyst}
\date{November 22, 2025}
\maketitle

\hrule
\vspace{1cm}

% --- TABLE OF CONTENTS ---
\tableofcontents

\newpage

% --- EXECUTIVE SUMMARY ---
\section{Executive Summary}

This report provides a comprehensive cybersecurity assessment for \textbf{Pacific Rim Exports}, conducted on November 22, 2025. The analysis correlates findings from an external network scan, a review of internal security controls via a questionnaire, and a check of pre-existing risks.

The assessment identified several critical and high-risk security gaps that require immediate attention. Most notably, the lack of Multi-Factor Authentication (MFA) on email and sensitive data systems exposes the organization to significant risks of account compromise and data breach. Furthermore, the external-facing web server is running an outdated version of Nginx, which is known to have multiple security vulnerabilities.

While the organization has implemented some positive security measures, such as MFA for computer logins and security awareness training, the identified gaps undermine the overall security posture. Prioritized remediation of the findings detailed in this report is strongly recommended to mitigate potential threats.

% --- ORGANIZATIONAL INFORMATION ---
\section{Organizational Information}

The following information was provided for the assessment.

\begin{table}[h!]
\centering
\begin{tabular}{@{}ll@{}}
\toprule
\textbf{Attribute} & \textbf{Value} \\
\midrule
Organization Name & \textbf{Pacific Rim Exports} \\
Email Domain & \texttt{PacificRimExports.com} \\
Website Domain & \seqsplit{\url{www.PacificRimExports.com}} \\
External IP Address & \texttt{5.213.14.81} \\
\bottomrule
\end{tabular}
\caption{Client Organizational Details}
\end{label{tab:org_info}
\end{table}

% --- SECURITY CONTROL REVIEW ---
\section{Security Control Review}

A review of administrative and technical security controls was conducted based on a questionnaire. The results indicate critical gaps in access control and policy enforcement. "No" answers represent deviations from security best practices and are flagged as significant risks.

\begin{table}[h!]
\centering
\begin{tabular}{@{}p{0.7\textwidth}c@{}}
\toprule
\textbf{Security Control Question} & \textbf{Status} \\
\midrule
Do you require MFA to access email? & \no \\
Do you require MFA to log into computers? & \yes \\
Do you require MFA to access sensitive data systems? & \no \\
Does your organization have an employee acceptable use policy? & \no \\
Does your organization do security awareness training for new employees? & \yes \\
Does your organization do security awareness training for all employees at least once per year? & \yes \\
\bottomrule
\end{tabular}
\caption{Security Controls Questionnaire Results}
\label{tab:controls}
\end{table}

% --- TECHNICAL SCAN RESULTS ---
\section{Technical Scan Results}

An external network scan was performed against the target IP address \texttt{192.168.10.5} to identify open ports and exposed services.

\begin{itemize}
    \item \textbf{Scan Date:} 2025-11-22T10:00:00Z
    \item \textbf{Target IP:} \texttt{192.168.10.5}
    \item \textbf{Status:} Host is Up
\end{itemize}

The scan revealed one open port, which is detailed below.

\begin{table}[h!]
\centering
\begin{tabular}{@{}lllll@{}}
\toprule
\textbf{Port} & \textbf{State} & \textbf{Service} & \textbf{Product} & \textbf{Version} \\
\midrule
443/tcp & Open & https & nginx & 1.18.0 \\
\bottomrule
\end{tabular}
\caption{Open Ports and Services}
\label{tab:scan_results}
\end{table}

\paragraph{Analysis:} The scan identified an Nginx web server, version \textbf{1.18.0}, exposed to the internet. This version was released in April 2020 and is now significantly outdated. It is known to be vulnerable to multiple Common Vulnerabilities and Exposures (CVEs), making it a high-risk target for external attackers.

% --- PRE-EXISTING RISKS ---
\section{Pre-existing Risk Review}
A review of the provided list of current vulnerabilities was conducted. No pre-existing risks were documented for this assessment. All findings in this report are newly identified.

% --- RISK ASSESSMENT SUMMARY ---
\section{Risk Assessment Summary}

The following table synthesizes the findings from the security control review and the technical scan into a prioritized list of risks.

\begin{table}[h!]
\centering
\begin{tabular}{@{}p{0.3\textwidth}p{0.15\textwidth}p{0.45\textwidth}@{}}
\toprule
\textbf{Risk Name} & \textbf{Severity} & \textbf{Overview} \\
\midrule
\textbf{No MFA for Email Access} & \textbf{Critical} & Lack of MFA on email accounts makes them highly susceptible to phishing and credential stuffing attacks, which can lead to a full organizational compromise. \\
\addlinespace
\textbf{No MFA for Sensitive Data} & \textbf{Critical} & Sensitive corporate and customer data is not protected by strong authentication, creating a high risk of unauthorized access and data exfiltration. \\
\addlinespace
\textbf{Outdated Nginx Web Server} & High & The public-facing web server (v1.18.0) is outdated and has known vulnerabilities that can be exploited by attackers to compromise the server and gain a foothold in the network. \\
\addlinespace
\textbf{No Acceptable Use Policy} & Medium & The absence of a formal policy defining acceptable use of company assets creates ambiguity and increases the risk of insider threats and misuse of resources. \\
\bottomrule
\end{tabular}
\caption{Identified Security Risks}
\label{tab:risk_summary}
\end{table}

% --- RECOMMENDATIONS ---
\section{Recommendations}

Based on the identified risks, the following prioritized actions are recommended to improve the security posture of \textbf{Pacific Rim Exports}.

\begin{enumerate}
    \item \textbf{[Critical] Implement MFA for All Email and Sensitive Systems:}
    \begin{itemize}
        \item Immediately enforce mandatory MFA for all user accounts accessing the email system (e.g., \texttt{PacificRimExports.com}).
        \item Deploy MFA on all systems classified as containing sensitive data to protect critical assets from unauthorized access.
    \end{itemize}
    
    \item \textbf{[High] Upgrade the Nginx Web Server:}
    \begin{itemize}
        \item Develop a patch management plan to upgrade the Nginx server from version 1.18.0 to the latest stable version.
        \item Regularly scan external-facing systems for outdated software and vulnerabilities.
    \end{itemize}
    
    \item \textbf{[Medium] Develop and Implement an Acceptable Use Policy (AUP):}
    \begin{itemize}
        \item Create a formal AUP document that clearly outlines the rules and responsibilities for all employees when using company technology and data.
        \item Integrate the AUP into the new employee onboarding process and require all current employees to read and acknowledge the policy.
    \end{itemize}
\end{enumerate}

\end{document}
```