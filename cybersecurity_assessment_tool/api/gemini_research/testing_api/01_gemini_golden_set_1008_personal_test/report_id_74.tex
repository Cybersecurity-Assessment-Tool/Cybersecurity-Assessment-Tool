```latex
\documentclass[12pt]{article}

% Preamble: Required Packages
\usepackage[margin=1in]{geometry}
\usepackage{pifont} % For checkmarks and crosses
\usepackage{booktabs} % For professional tables
\usepackage{hyperref} % For clickable links
\usepackage{url} % For URL formatting
\usepackage{seqsplit} % For splitting long strings
\usepackage{graphicx} % For logo
\usepackage{xcolor} % For colors

% Document Information
\title{Cybersecurity Posture Assessment Report}
\author{Cybersecurity Analysis Division}
\date{\today}

% Hyperref Setup
\hypersetup{
    colorlinks=true,
    linkcolor=blue,
    filecolor=magenta,      
    urlcolor=cyan,
    pdftitle={Cybersecurity Posture Assessment Report},
    pdfpagemode=FullScreen,
}

\begin{document}

\maketitle
\thispagestyle{empty}
\newpage

\tableofcontents
\thispagestyle{empty}
\newpage

\setcounter{page}{1}

% ==============================================================================
% Section 1: Executive Summary
% ==============================================================================
\section{Executive Summary}

This report provides a comprehensive cybersecurity assessment for \textbf{Grizzly Peak}, based on an analysis of network scan data, organizational security controls, and pre-existing risk information. The assessment reveals several critical and high-risk vulnerabilities that require immediate attention to mitigate the potential for significant security breaches.

The key findings indicate a concerning security posture characterized by both technical and procedural weaknesses. A critical vulnerability was discovered on an internal server (\texttt{10.0.0.15}), which is running an outdated and notoriously vulnerable FTP service (\texttt{vsftpd 2.3.4}) configured to allow anonymous access. This presents a direct and easily exploitable vector for an attacker to compromise the network.

Furthermore, organizational policies exhibit significant gaps. The absence of multi-factor authentication (MFA) for sensitive data systems, the lack of a formal employee acceptable use policy, and the failure to conduct annual security awareness training for all staff collectively create a high-risk environment. These procedural failings, combined with the technical vulnerabilities, substantially increase the organization's susceptibility to cyberattacks, including data breaches, ransomware, and unauthorized access.

Immediate remediation of the vulnerable FTP server is paramount. Concurrently, the organization must prioritize the implementation of MFA and the development of foundational security policies and training programs to build a more resilient and secure operational environment.

% ==============================================================================
% Section 2: Organizational Information
% ==============================================================================
\section{Organizational Information}

The following details were provided for the assessment. This information helps establish the context and scope of the review.

\begin{tabular}{@{}ll}
\toprule
\textbf{Attribute} & \textbf{Value} \\
\midrule
Organization Name & \textbf{Grizzly Peak} \\
Email Domain & \texttt{GrizzlyPeak.org} \\
Website Domain & \url{www.GrizzlyPeak.org} \\
External IP Address & \texttt{191.76.55.190} \\
\bottomrule
\end{tabular}

% ==============================================================================
% Section 3: Security Control Review (Questionnaire Analysis)
% ==============================================================================
\section{Security Control Review (Questionnaire Analysis)}

A review of the organization's security controls was conducted via a standardized questionnaire. The responses highlight critical gaps in security policy and procedure. A "No" answer indicates a deviation from security best practices and represents a significant area of risk.

\begin{tabular}{@{}p{0.7\linewidth}c}
\toprule
\textbf{Control Question} & \textbf{Status} \\
\midrule
Do you require MFA to access email? & \textcolor{green}{\ding{51}} \\
Do you require MFA to log into computers? & \textcolor{green}{\ding{51}} \\
Do you require MFA to access sensitive data systems? & \textcolor{red}{\ding{55}} \\
Does your organization have an employee acceptable use policy? & \textcolor{red}{\ding{55}} \\
Does your organization do security awareness training for new employees? & \textcolor{green}{\ding{51}} \\
Does your organization do security awareness training for all employees at least once per year? & \textcolor{red}{\ding{55}} \\
\bottomrule
\end{tabular}

\subsection*{Analysis of Gaps}
\begin{itemize}
    \item \textbf{No MFA for Sensitive Systems:} This is a critical deficiency. Without MFA, sensitive data is protected only by a single factor (passwords), which can be easily compromised through phishing, brute-force attacks, or credential stuffing.
    \item \textbf{No Acceptable Use Policy (AUP):} An AUP is a foundational policy that sets clear expectations for employee behavior when using company resources. Its absence can lead to unintentional misuse of systems and data.
    \item \textbf{No Annual Security Training:} Cyber threats evolve constantly. Failing to provide annual refresher training leaves employees ill-equipped to recognize and respond to modern threats like sophisticated phishing and social engineering attacks.
\end{itemize}

% ==============================================================================
% Section 4: Technical Vulnerability Assessment
% ==============================================================================
\section{Technical Vulnerability Assessment}

A network scan was performed to identify technical vulnerabilities on the target host.

\subsection*{Scan Target: \texttt{10.0.0.15}}
The scan identified one open port with a service containing a critical vulnerability.

\begin{tabular}{@{}llllll}
\toprule
\textbf{Port} & \textbf{State} & \textbf{Service} & \textbf{Product} & \textbf{Version} & \textbf{Notes} \\
\midrule
21/tcp & open & ftp & vsftpd & 2.3.4 & \parbox[t]{4cm}{\textbf{Critical Vulnerability.} This version contains a known backdoor (CVE-2011-2523). Anonymous FTP login is also allowed.} \\
\bottomrule
\end{tabular}

\subsection*{Analysis of Technical Findings}
The presence of \textbf{vsftpd version 2.3.4} is a severe security risk. This specific version was compromised in 2011, and a backdoor was inserted into the source code. An attacker can gain a command shell on the server by sending a specific sequence of characters as a username. Compounding this issue, the service is configured to allow \textbf{anonymous FTP login}, which means any unauthenticated user on the network can access the FTP server, significantly lowering the bar for exploitation. This configuration poses an immediate and direct threat to the integrity and confidentiality of the server and potentially the entire network.

% ==============================================================================
% Section 5: Consolidated Risk Register
% ==============================================================================
\section{Consolidated Risk Register}

The following table synthesizes findings from the questionnaire, technical scan, and pre-existing risk data into a prioritized list.

\begin{tabular}{@{}p{0.5\linewidth}p{0.2\linewidth}p{0.2\linewidth}}
\toprule
\textbf{Risk Description} & \textbf{Severity} & \textbf{Affected Elements} \\
\midrule
\textbf{Vulnerable FTP Server (vsftpd 2.3.4)} \newline An outdated and backdoored FTP service is exposed on the internal network, allowing for remote code execution. & \textbf{Critical} & Server \texttt{10.0.0.15}, Internal Network \\
\addlinespace
\textbf{No MFA for Sensitive Data Systems} \newline Lack of multi-factor authentication on critical systems leaves sensitive data vulnerable to credential-based attacks. & \textbf{Critical} & Sensitive Data, Core Business Systems \\
\addlinespace
\textbf{Anonymous FTP Login Enabled} \newline The FTP server allows unauthenticated access, increasing the risk of unauthorized data access and facilitating exploitation of the service. & \textbf{High} & Server \texttt{10.0.0.15} \\
\addlinespace
\textbf{Inadequate Security Policies \& Training} \newline The absence of an AUP and annual security training increases the likelihood of human error leading to a security incident. & \textbf{High} & All Employees, Organizational Security Posture \\
\addlinespace
\textbf{Outdated Windows Policy} \newline Pre-existing risk: Computers are running Windows 7, which is an unsupported operating system and does not receive security updates. & \textbf{Medium} & Workstations \\
\bottomrule
\end{tabular}

% ==============================================================================
% Section 6: Recommendations & Remediation Plan
% ==============================================================================
\section{Recommendations \& Remediation Plan}
The following actions are recommended to address the identified risks, prioritized by severity and urgency.

\subsection*{Immediate Actions (To be completed within 72 hours)}
\begin{enumerate}
    \item \textbf{Isolate or Disable Vulnerable FTP Service:} Immediately take the server at \texttt{10.0.0.15} offline or apply a firewall rule to block all access to port 21. This is the most critical step to prevent imminent compromise.
\end{enumerate}

\subsection*{Short-Term Actions (To be completed within 30-90 days)}
\begin{enumerate}
    \item \textbf{Implement MFA on Sensitive Systems:} Prioritize the rollout of MFA for all systems containing sensitive or critical data. This is the single most effective control to mitigate the risk of unauthorized access.
    \item \textbf{Decommission or Replace FTP Server:} If the FTP service is required for business operations, it must be upgraded to a modern, patched version. Anonymous access must be disabled. It is strongly recommended to replace FTP with a secure alternative like SFTP (SSH File Transfer Protocol).
    \item \textbf{Develop and Implement an Acceptable Use Policy (AUP):} Draft and enforce a formal AUP that clearly defines the rules for using company technology and data. All employees must read and acknowledge this policy.
    \item \textbf{Schedule Annual Security Awareness Training:} Implement a mandatory security awareness training program for all employees, to be conducted at least annually.
\end{enumerate}

\subsection*{Long-Term Actions (Ongoing / 6-12 months)}
\begin{enumerate}
    \item \textbf{Execute Windows Upgrade Plan:} Continue with the existing recommendation to upgrade all workstations from Windows 7 to a modern, supported operating system (e.g., Windows 10/11) to ensure they receive critical security patches.
    \item \textbf{Establish a Vulnerability Management Program:} Implement a formal process for regularly scanning internal and external assets, identifying vulnerabilities, and tracking them to remediation.
\end{enumerate}

\end{document}
```