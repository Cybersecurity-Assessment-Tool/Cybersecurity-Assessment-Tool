```latex
\documentclass[12pt, a4paper]{article}

% Preamble: Required Packages
\usepackage[margin=1in]{geometry}
\usepackage{pifont} % For checkmarks and crosses
\usepackage{booktabs} % For professional tables
\usepackage{hyperref} % For clickable links
\usepackage{url} % For formatting URLs
\usepackage{seqsplit} % To split long strings without breaking
\usepackage{graphicx}
\usepackage{xcolor}
\usepackage{fancyhdr} % For headers and footers
\usepackage{lastpage} % To get the total number of pages

% --- Document Metadata and Hyperref Setup ---
\hypersetup{
    colorlinks=true,
    linkcolor=blue,
    filecolor=magenta,      
    urlcolor=cyan,
    pdftitle={Cybersecurity Assessment Report},
    pdfauthor={Cybersecurity Analyst},
    pdfsubject={Security Assessment},
    pdfkeywords={Security, Assessment, Report},
    bookmarks=true
}

% --- Custom Colors ---
\definecolor{darkred}{rgb}{0.55, 0.0, 0.0}
\definecolor{darkorange}{rgb}{0.8, 0.3, 0.0}
\definecolor{darkgreen}{rgb}{0.0, 0.3, 0.0}

% --- Header and Footer Configuration ---
\pagestyle{fancy}
\fancyhf{} % Clear all header and footer fields
\fancyhead[L]{\textbf{Cybersecurity Assessment Report}}
\fancyhead[R]{\textbf{Nexus Dynamics}}
\fancyfoot[C]{\thepage\ of \pageref{LastPage}}
\renewcommand{\headrulewidth}{0.4pt}
\renewcommand{\footrulewidth}{0.4pt}

% --- Document Start ---
\begin{document}

% --- Title Page ---
\begin{titlepage}
    \centering
    \vspace*{1cm}
    \Huge\textbf{Cybersecurity Assessment Report}
    \vspace{1.5cm}
    \Large
    \textbf{Prepared for:} \\
    \vspace{0.5cm}
    \seqsplit{\textbf{Nexus Dynamics}}
    \vspace{2cm}
    \large
    \textbf{Date of Report:} \\
    \vspace{0.5cm}
    \today
    \vfill
    \large
    \textit{This report contains sensitive information and should be handled with care. Distribution is restricted to authorized personnel only.}
\end{titlepage}

\tableofcontents
\newpage

% --- Section 1: Executive Summary ---
\section{Executive Summary}
This report details the findings of a cybersecurity assessment conducted for \textbf{Nexus Dynamics}. The evaluation combined a technical network scan, a review of existing risks, and an analysis of organizational security controls provided via a questionnaire.

The assessment identified several critical and high-risk vulnerabilities that require immediate attention. The most severe finding is a publicly exposed MySQL database (\seqsplit{\texttt{172.16.50.20:3306}}) running an \textbf{End-of-Life (EOL) version (5.7.33)}, which no longer receives security updates. This presents a significant and direct risk of data breach.

Furthermore, critical gaps were identified in administrative and access controls. The lack of mandatory Multi-Factor Authentication (MFA) for computer logins, the absence of a formal Acceptable Use Policy (AUP), and incomplete annual security awareness training for all employees substantially weaken the organization's defense against common cyber threats like phishing and credential theft.

This combination of technical vulnerabilities and procedural weaknesses creates a high-risk environment. We strongly recommend prioritizing the remediation actions outlined in Section 6 to mitigate these risks and improve the overall security posture of \textbf{Nexus Dynamics}.

% --- Section 2: Organizational Information ---
\section{Organizational Information}
The following details were provided for the assessment.
\begin{itemize}
    \item \textbf{Organization Name:} \seqsplit{Nexus Dynamics}
    \item \textbf{Email Domain:} \seqsplit{\texttt{NexusDynamics.com}}
    \item \textbf{Website Domain:} \seqsplit{\texttt{www.NexusDynamics.com}}
    \item \textbf{External IP Address:} \seqsplit{\texttt{220.133.10.183}}
\end{itemize}

% --- Section 3: Security Control Review ---
\section{Security Control Review}
The following table summarizes the organization's responses to the security controls questionnaire. A green checkmark (\textcolor{darkgreen}{\ding{51}}) indicates a positive control in place, while a red cross (\textcolor{darkred}{\ding{55}}) highlights a potential security gap.

\begin{table}[h!]
\centering
\caption{Security Controls Questionnaire Analysis}
\label{tab:controls}
\begin{tabular}{@{}llc@{}}
\toprule
\textbf{Control Question} & \textbf{Response} & \textbf{Status} \\ \midrule
Do you require MFA to access email? & Yes & \textcolor{darkgreen}{\ding{51}} \\
\textbf{Do you require MFA to log into computers?} & \textbf{No} & \textcolor{darkred}{\ding{55}} \\
Do you require MFA to access sensitive data systems? & Yes & \textcolor{darkgreen}{\ding{51}} \\
\textbf{Does your organization have an employee acceptable use policy?} & \textbf{No} & \textcolor{darkred}{\ding{55}} \\
Does your organization do security awareness training for new employees? & Yes & \textcolor{darkgreen}{\ding{51}} \\
\textbf{Does your organization do security training for all employees annually?} & \textbf{No} & \textcolor{darkred}{\ding{55}} \\ \bottomrule
\end{tabular}
\end{table}

\subsection*{Analysis of Gaps}
The questionnaire reveals three significant gaps in the organization's administrative and access controls:
\begin{itemize}
    \item \textbf{No MFA for Computer Logins:} This is a critical weakness. If an employee's credentials are stolen (e.g., through phishing), an attacker can gain direct access to their workstation and potentially the internal network without needing a second authentication factor.
    \item \textbf{No Acceptable Use Policy (AUP):} An AUP is a foundational policy that defines rules for employee use of company assets. Its absence can lead to inconsistent security practices and a lack of clear recourse for policy violations.
    \item \textbf{No Annual Security Training for All Employees:} The threat landscape is constantly evolving. Failing to provide regular, updated training for all staff members increases the likelihood of employees falling victim to social engineering attacks.
\end{itemize}

% --- Section 4: Technical Scan Results ---
\section{Technical Scan Results}
An external network scan was performed to identify open ports and exposed services.

\begin{itemize}
    \item \textbf{Target IP Address:} \seqsplit{\texttt{172.16.50.20}}
    \item \textbf{Scan Date:} \today
\end{itemize}

\begin{table}[h!]
\centering
\caption{Open Ports and Services Detected}
\label{tab:scan}
\begin{tabular}{@{}lllll@{}}
\toprule
\textbf{Port} & \textbf{State} & \textbf{Service} & \textbf{Product} & \textbf{Version} \\ \midrule
3306/tcp & open & mysql & MySQL & 5.7.33 \\ \bottomrule
\end{tabular}
\end{table}

\subsection*{Analysis of Findings}
The scan identified that port \textbf{3306}, the default port for the MySQL database service, is open to the network. This directly confirms the pre-existing risk "Database Exposure".

\textbf{CRITICAL FINDING:} The detected version, \textbf{MySQL 5.7.33}, reached its official \textbf{End-of-Life (EOL) in October 2023}. Software that is EOL no longer receives security patches from the vendor, even for newly discovered critical vulnerabilities. Running EOL software, especially on a publicly accessible service, exposes the organization to a high likelihood of compromise from known exploits.

% --- Section 5: Correlated Risk Assessment ---
\section{Correlated Risk Assessment}
This section synthesizes findings from the questionnaire, technical scan, and pre-existing risk data into a prioritized list of risks.

\begin{table}[h!]
\centering
\caption{Summary of Identified Risks}
\label{tab:risks}
\begin{tabular}{@{}p{0.1\linewidth}p{0.25\linewidth}p{0.45\linewidth}p{0.1\linewidth}@{}}
\toprule
\textbf{ID} & \textbf{Risk Name} & \textbf{Description} & \textbf{Severity} \\ \midrule
\textbf{RISK-001} & \textbf{Exposed End-of-Life Database Service} & Port 3306 is open, exposing a MySQL 5.7.33 instance. This version is End-of-Life and unpatched, making it highly susceptible to exploitation. & \textbf{Critical} \\
\addlinespace
\textbf{RISK-002} & \textbf{Insufficient Endpoint Access Control} & The absence of MFA on computer logins allows for trivial account takeover if user credentials are compromised. & \textbf{High} \\
\addlinespace
\textbf{RISK-003} & \textbf{Inadequate Administrative Security Controls} & The lack of an Acceptable Use Policy and mandatory annual security training for all staff weakens security culture and increases susceptibility to human error and social engineering. & \textbf{High} \\ \bottomrule
\end{tabular}
\end{table}

% --- Section 6: Recommendations ---
\section{Recommendations}
The following actions are recommended to mitigate the identified risks. Recommendations are prioritized based on severity.

\subsection*{RISK-001: Exposed End-of-Life Database Service (Critical)}
\begin{itemize}
    \item \textbf{Immediate Action:} Apply firewall rules to restrict all access to port 3306 from the public internet. Access should only be permitted from specific, trusted IP addresses, such as application servers or a management VPN.
    \item \textbf{Short-Term Plan (1-3 Months):} Plan and execute the migration of the database from MySQL 5.7 to a currently supported version (e.g., MySQL 8.x). This is essential to ensure security patches are available.
    \item \textbf{Long-Term Strategy:} Implement a formal patch and vulnerability management program to regularly scan for and remediate outdated and EOL software across the entire infrastructure.
\end{itemize}

\subsection*{RISK-002: Insufficient Endpoint Access Control (High)}
\begin{itemize}
    \item \textbf{Immediate Action:} Procure and enforce a Multi-Factor Authentication (MFA) solution for all employee computer logins (e.g., Windows Hello, Duo, YubiKey).
    \item \textbf{Long-Term Strategy:} Conduct a review of all information systems and enforce MFA as a default security control for access to any sensitive data or administrative function.
\end{itemize}

\subsection*{RISK-003: Inadequate Administrative Security Controls (High)}
\begin{itemize}
    \item \textbf{Immediate Action:} Develop and formally ratify an employee Acceptable Use Policy (AUP). Ensure all current employees read and acknowledge the policy, and incorporate it into the new employee onboarding process.
    \item \textbf{Short-Term Plan (1-3 Months):} Schedule and conduct mandatory security awareness training for all employees. This training should be repeated on an annual basis to address new and evolving threats.
\end{itemize}

\end{document}
```