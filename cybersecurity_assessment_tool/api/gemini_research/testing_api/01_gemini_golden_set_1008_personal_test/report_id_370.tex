```latex
\documentclass[12pt]{article}

% Required Packages
\usepackage[a4paper, margin=1in]{geometry}
\usepackage{pifont} % For checkmarks and crosses
\usepackage{booktabs} % For professional tables
\usepackage{hyperref} % For hyperlinks
\usepackage{url}      % For URL formatting
\usepackage{seqsplit} % For splitting long strings in tt font
\usepackage{graphicx} % For logo (optional placeholder)
\usepackage{fancyhdr} % For headers/footers
\usepackage{lastpage} % To get total page count

% --- Document Setup ---
\hypersetup{
    colorlinks=true,
    linkcolor=blue,
    filecolor=magenta,      
    urlcolor=cyan,
    pdftitle={Cybersecurity Assessment Report},
    pdfpagemode=FullScreen,
}

\pagestyle{fancy}
\fancyhf{} % Clear all header and footer fields
\fancyhead[L]{Cybersecurity Assessment Report}
\fancyhead[R]{Catalyst Consulting}
\fancyfoot[C]{\thepage\ of \pageref{LastPage}}
\renewcommand{\headrulewidth}{0.4pt}
\renewcommand{\footrulewidth}{0.4pt}

% --- Document Body ---
\begin{document}

% --- Title Page ---
\begin{titlepage}
    \centering
    \vspace*{2cm}
    
    \Huge
    \textbf{Cybersecurity Assessment Report}
    
    \vspace{1.5cm}
    
    \Large
    Prepared for: \\
    \vspace{0.5cm}
    \textbf{Catalyst Consulting}
    
    \vspace{2cm}
    
    \large
    \today
    
    \vfill
    
    \normalsize
    \textit{This report contains sensitive information and is intended solely for the use of Catalyst Consulting. Distribution is strictly prohibited.}
    
\end{titlepage}

\tableofcontents
\newpage

% --- Section 1: Executive Summary ---
\section{Executive Summary}

This report details the findings of a cybersecurity assessment conducted for \textbf{Catalyst Consulting}. The assessment combined a review of organizational security controls via a questionnaire, an external network vulnerability scan, and an analysis of pre-existing risks.

The primary finding of this assessment is the existence of several \textbf{critical} procedural and policy-based security gaps. The most significant risk is the widespread lack of Multi-Factor Authentication (MFA) for email, computer logins, and access to sensitive data systems. This deficiency dramatically increases the organization's vulnerability to account takeover attacks, which often stem from phishing or credential theft.

Further high-risk gaps were identified, including the absence of a formal Employee Acceptable Use Policy (AUP) and the lack of mandatory, annual security awareness training for all staff members. These gaps weaken the organization's human firewall and create an environment where security best practices are not consistently enforced or reinforced.

On a positive note, the external network scan of the target IP address \texttt{[Target IP]} did not identify any open ports. This suggests a well-configured perimeter firewall, which is a foundational element of network security. However, this strong external posture does not mitigate the severe internal risks identified.

Urgent action is recommended to address the identified control gaps, with the immediate implementation of MFA being the highest priority.

% --- Section 2: Organizational Information ---
\section{Organizational Information}

The following details were provided for the assessment.

\begin{tabular}{@{}ll}
    \toprule
    \textbf{Attribute} & \textbf{Value} \\
    \midrule
    Organization Name & Catalyst Consulting \\
    Email Domain & \texttt{CatalystConsulting.net} \\
    Website Domain & \seqsplit{\url{www.CatalystConsulting.net}} \\
    Primary External IP & \texttt{221.141.106.173} \\
    \bottomrule
\end{tabular}

% --- Section 3: Security Control Review ---
\section{Security Control Review}

The following table summarizes the responses from the organizational security questionnaire. "No" answers indicate significant gaps in the security framework and are flagged as risks.

\begin{table}[h!]
\centering
\begin{tabular}{p{8cm} c l}
    \toprule
    \textbf{Control Question} & \textbf{Response} & \textbf{Assessment} \\
    \midrule
    Do you require MFA to access email? & \ding{55} & \textbf{Critical Gap} \\
    Do you require MFA to log into computers? & \ding{55} & \textbf{Critical Gap} \\
    Do you require MFA to access sensitive data systems? & \ding{55} & \textbf{Critical Gap} \\
    Does your organization have an employee acceptable use policy? & \ding{55} & High Risk \\
    Does your organization do security awareness training for new employees? & \ding{51} & Best Practice Met \\
    Does your organization do security awareness training for all employees at least once per year? & \ding{55} & High Risk \\
    \bottomrule
\end{tabular}
\caption{Security Controls Questionnaire Results (\ding{51}=Yes, \ding{55}=No)}
\end{table}

% --- Section 4: Technical Scan Results ---
\section{Technical Scan Results}

An external network scan was performed to identify potential vulnerabilities visible from the public internet.

\begin{itemize}
    \item \textbf{Target IP Address:} \texttt{[Target IP]}
    \item \textbf{Scan Date:} [Scan Date Not Provided]
\end{itemize}

\subsection{Summary of Findings}
The network scan did not identify any open TCP or UDP ports on the target system. This is a positive finding, suggesting that a perimeter firewall is effectively blocking unsolicited inbound connection attempts. No services were exposed to the public internet at the time of the scan.

\textit{Note: While no vulnerabilities were found, this does not guarantee the absence of vulnerabilities on the internal network or in web applications that may be hosted elsewhere.}

% --- Section 5: Risk Assessment ---
\section{Risk Assessment}

This section correlates the findings from the security control review and the technical scan. The risks below are synthesized from the identified control gaps. No pre-existing vulnerabilities were provided for this assessment.

\begin{table}[h!]
\centering
\begin{tabular}{p{1.5cm} p{4cm} p{2cm} p{5.5cm}}
    \toprule
    \textbf{Risk ID} & \textbf{Risk Name} & \textbf{Severity} & \textbf{Description} \\
    \midrule
    RISK-001 & Lack of Multi-Factor Authentication (MFA) & \textbf{Critical} & The absence of MFA for email, endpoints, and sensitive systems exposes the organization to a high likelihood of account compromise via stolen credentials. A single compromised password could grant an attacker significant access. \\
    \addlinespace
    RISK-002 & Inadequate Security Awareness Training & High & While new hires receive training, the lack of a mandatory annual program for all employees leads to knowledge decay. Staff may not recognize or properly respond to modern phishing, social engineering, and malware threats. \\
    \addlinespace
    RISK-003 & Absence of Acceptable Use Policy (AUP) & High & Without a formal AUP, there are no clear, enforceable rules for employees regarding the safe and appropriate use of company assets, data, and network resources. This creates ambiguity and increases the risk of insider threat and accidental data exposure. \\
    \bottomrule
\end{tabular}
\caption{Synthesized Risk Register}
\end{table}

% --- Section 6: Recommendations ---
\section{Recommendations}

Based on the findings of this assessment, the following actions are recommended to mitigate the identified risks and improve the overall security posture of \textbf{Catalyst Consulting}. Recommendations are prioritized based on severity.

\begin{description}
    \item[Priority 1 (Critical) - Remediate RISK-001:] \textbf{Implement Multi-Factor Authentication (MFA).}
    \begin{itemize}
        \item Immediately enforce MFA for all user accounts on the primary email system (e.g., Microsoft 365, Google Workspace).
        \item Deploy MFA for all remote access solutions (e.g., VPNs, VDI).
        \item Phase in MFA for logins to all company computers and for access to all systems storing sensitive or critical data.
    \end{itemize}

    \item[Priority 2 (High) - Remediate RISK-003:] \textbf{Develop and Enforce an Acceptable Use Policy (AUP).}
    \begin{itemize}
        \item Create a formal AUP document that clearly defines the rules for using company technology, including internet usage, data handling, and software installation.
        \item Require all current and new employees to read and formally acknowledge the policy.
        \item Periodically review and update the AUP to reflect new technologies and threats.
    \end{itemize}

    \item[Priority 3 (High) - Remediate RISK-002:] \textbf{Establish a Continuous Security Training Program.}
    \begin{itemize}
        \item Implement a mandatory security awareness training program that all employees must complete at least once per year.
        \item Supplement annual training with regular phishing simulations and security newsletters to keep staff vigilant against current threats.
    \end{itemize}
\end{description}

\end{document}
```