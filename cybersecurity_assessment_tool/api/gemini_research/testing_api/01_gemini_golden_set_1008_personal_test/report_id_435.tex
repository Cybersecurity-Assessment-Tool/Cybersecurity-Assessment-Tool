```latex
\documentclass[12pt]{article}

% ----------------------------------------------------------------------
% PREAMBLE
% ----------------------------------------------------------------------
\usepackage[a4paper, margin=1in]{geometry}
\usepackage{pifont} % For checkmarks and crosses
\usepackage{booktabs} % For professional tables
\usepackage{hyperref} % For hyperlinks and document metadata
\usepackage{url} % For URL formatting
\usepackage{seqsplit} % For breaking long strings in tt font
\usepackage{graphicx}
\usepackage{xcolor}

% --- Hyperref Setup ---
\hypersetup{
    colorlinks=true,
    linkcolor=blue,
    filecolor=magenta,      
    urlcolor=cyan,
    pdftitle={Cybersecurity Posture Assessment Report},
    pdfauthor={Cybersecurity Analyst},
    pdfsubject={Security Analysis},
    pdfkeywords={Cybersecurity, Nmap, Risk Assessment},
    bookmarks=true
}

% --- Custom Commands ---
\newcommand{\yes}{\ding{51}} % Green checkmark
\newcommand{\no}{\ding{55}}  % Red cross

% ----------------------------------------------------------------------
% DOCUMENT START
% ----------------------------------------------------------------------
\begin{document}

% ----------------------------------------------------------------------
% TITLE PAGE
% ----------------------------------------------------------------------
\begin{titlepage}
    \centering
    \vspace*{1cm}
    \Huge \textbf{Cybersecurity Posture Assessment Report}
    \vspace{1.5cm}
    \Large
    \textbf{Prepared for:}\\
    \vspace{0.5cm}
    \textbf{Stellar Pathways}
    \vspace{2cm}
    \large
    \textbf{Date of Report:}\\
    \today
    \vfill
    \textit{This report contains sensitive information and should be handled with care. Distribution is restricted to authorized personnel only.}
\end{titlepage}

\tableofcontents
\newpage

% ----------------------------------------------------------------------
% 1. EXECUTIVE SUMMARY
% ----------------------------------------------------------------------
\section{Executive Summary}
This report provides a cybersecurity posture assessment for Stellar Pathways, based on a combination of network scanning, a security controls questionnaire, and a review of pre-existing risks.

The assessment reveals a mixed security posture. On one hand, the organization demonstrates a strong commitment to foundational security controls, as evidenced by the universal implementation of Multi-Factor Authentication (MFA), employee security training, and acceptable use policies. These administrative controls are commendable and significantly reduce risks related to account compromise and insider threats.

However, a critical technical vulnerability was identified. The external network scan revealed an exposed MySQL database service (port 3306) running on version 5.7.33. This version is officially End-of-Life (EOL) as of October 2023 and no longer receives security updates from the vendor. An exposed, unpatchable database presents a high-impact risk of data breach, denial of service, or system compromise. This finding directly corroborates the pre-existing identified risk of ``Database Exposure''.

Immediate action is required to mitigate this critical risk by restricting network access to the database. Subsequently, a plan must be developed to upgrade the database to a vendor-supported version.

% ----------------------------------------------------------------------
% 2. ORGANIZATIONAL INFORMATION
% ----------------------------------------------------------------------
\section{Organizational Information}
The following details were provided for the assessment.

\begin{table}[h!]
\centering
\begin{tabular}{@{}ll@{}}
\toprule
\textbf{Attribute} & \textbf{Value} \\ \midrule
Organization Name  & Stellar Pathways \\
Email Domain       & \seqsplit{\texttt{StellarPathways.org}} \\
Website Domain     & \seqsplit{\url{www.StellarPathways.org}} \\
External IP Address & \seqsplit{\texttt{175.210.110.142}} \\ \bottomrule
\end{tabular}
\caption{Client Organizational Details}
\end{table}

% ----------------------------------------------------------------------
% 3. SECURITY CONTROL REVIEW
% ----------------------------------------------------------------------
\section{Security Control Review}
A review of the organization's administrative and user-level security controls was conducted via a questionnaire. The results indicate a strong baseline for security policies and user access management.

\begin{table}[h!]
\centering
\begin{tabular}{@{}p{0.8\textwidth}c@{}}
\toprule
\textbf{Control Question} & \textbf{Status} \\ \midrule
Do you require MFA to access email? & \yes \\
Do you require MFA to log into computers? & \yes \\
Do you require MFA to access sensitive data systems? & \yes \\
Does your organization have an employee acceptable use policy? & \yes \\
Does your organization do security awareness training for new employees? & \yes \\
Does your organization do security awareness training for all employees at least once per year? & \yes \\ \bottomrule
\end{tabular}
\caption{Security Controls Questionnaire Results}
\end{table}

% ----------------------------------------------------------------------
% 4. TECHNICAL SCAN RESULTS
% ----------------------------------------------------------------------
\section{Technical Scan Results}
An external network scan was performed on the target IP address \seqsplit{\texttt{172.16.50.20}}. The scan identified one open port, which indicates a publicly accessible service.

\subsection{Open Ports and Services}
The following service was found to be exposed to the network:

\begin{table}[h!]
\centering
\begin{tabular}{@{}lllll@{}}
\toprule
\textbf{Port} & \textbf{State} & \textbf{Service} & \textbf{Product} & \textbf{Version} \\ \midrule
3306/tcp      & open           & mysql            & MySQL            & 5.7.33           \\ \bottomrule
\end{tabular}
\caption{Network Scan Findings for \seqsplit{\texttt{172.16.50.20}}}
\end{table}

\subsection{Analysis of Findings}
The scan identified a MySQL database server running on port 3306. The detected version, \textbf{MySQL 5.7.33}, reached its official End-of-Life (EOL) in October 2023. Systems running EOL software are a significant security risk because they no longer receive security patches, leaving them perpetually vulnerable to newly discovered exploits. Exposing an EOL database directly to the internet is a critical security flaw.

% ----------------------------------------------------------------------
% 5. RISK ASSESSMENT
% ----------------------------------------------------------------------
\section{Risk Assessment}
This section synthesizes the findings from the security control review, technical scan, and pre-existing risk data.

\begin{table}[h!]
\centering
\begin{tabular}{@{}p{0.2\textwidth}p{0.6\textwidth}c@{}}
\toprule
\textbf{Risk Name} & \textbf{Overview} & \textbf{Severity} \\ \midrule
\textbf{Database Exposure} & The MySQL database port (3306) is open to the public network, allowing unauthorized connection attempts and exposing it to enumeration and brute-force attacks. & \textbf{High (7.5)} \\
\addlinespace
\textbf{End-of-Life Software} & The exposed MySQL service is version 5.7.33, which is past its end-of-life. It no longer receives security updates, making it an easy target for known and future vulnerabilities. & \textbf{Critical (9.1)} \\ \bottomrule
\end{tabular}
\caption{Summary of Identified Risks}
\end{table}

% ----------------------------------------------------------------------
% 6. RECOMMENDATIONS
% ----------------------------------------------------------------------
\section{Recommendations}
Based on the risk assessment, the following actions are recommended to improve the security posture of Stellar Pathways. Recommendations are prioritized by urgency.

\begin{itemize}
    \item[\textbf{1. Immediate:}] \textbf{Restrict Network Access.}
    \begin{itemize}
        \item \textbf{Action:} Implement strict firewall rules to block all public access to TCP port 3306 on host \seqsplit{\texttt{172.16.50.20}}. Access should only be permitted from trusted internal IP addresses or via a secure VPN connection.
        \item \textbf{Impact:} Immediately mitigates the risk of external attack against the database.
    \end{itemize}
    \vspace{0.5cm}
    \item[\textbf{2. High Priority:}] \textbf{Upgrade End-of-Life Software.}
    \begin{itemize}
        \item \textbf{Action:} Develop and execute a plan to upgrade the MySQL 5.7 database to a currently supported version (e.g., MySQL 8.x). This involves testing application compatibility, backing up data, performing the upgrade, and validating functionality.
        \item \textbf{Impact:} Eliminates vulnerabilities associated with the EOL software and ensures ongoing security patch eligibility.
    \end{itemize}
    \vspace{0.5cm}
    \item[\textbf{3. Strategic:}] \textbf{Implement Secure Access Architecture.}
    \begin{itemize}
        \item \textbf{Action:} For long-term security, ensure that all administrative and data-sensitive services are placed in a private network segment. Enforce a policy where access to these systems is only possible through a secure, authenticated, and logged channel, such as a corporate VPN or a Zero Trust Network Access (ZTNA) solution.
        \item \textbf{Impact:} Proactively reduces the attack surface and strengthens the overall network security architecture against future threats.
    \end{itemize}
\end{itemize}

% ----------------------------------------------------------------------
% DOCUMENT END
% ----------------------------------------------------------------------
\end{document}
```