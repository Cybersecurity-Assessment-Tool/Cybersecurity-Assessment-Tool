```latex
\documentclass[12pt]{article}

% Preamble: Required Packages
\usepackage[a4paper, margin=1in]{geometry}
\usepackage{pifont} % For checkmarks and crosses
\usepackage{booktabs} % For professional-looking tables
\usepackage{hyperref} % For clickable links and better PDF navigation
\usepackage{url} % For formatting URLs
\usepackage{seqsplit} % For splitting long strings in \texttt
\usepackage{graphicx} % For potential logos
\usepackage{xcolor} % For colors

% Document Information
\title{Cybersecurity Posture Assessment Report}
\author{Cybersecurity Analysis Division}
\date{\today}

% Hyperref Setup
\hypersetup{
    colorlinks=true,
    linkcolor=black,
    urlcolor=blue,
    pdftitle={Cybersecurity Posture Assessment Report},
    pdfauthor={Cybersecurity Analysis Division},
}

\begin{document}

\maketitle
\thispagestyle{empty}
\newpage

\tableofcontents
\newpage

% --- 1. Executive Summary ---
\section{Executive Summary}

This report details the findings of a cybersecurity assessment conducted for \textbf{Iron Oak Furniture}. The analysis correlates data from a network infrastructure scan, a security controls questionnaire, and a review of pre-existing risks.

The assessment identified several high-impact vulnerabilities that require immediate attention. The most critical finding is an externally exposed MySQL database service on an internal server (\texttt{172.16.50.20}). This service is running MySQL version 5.7.33, which is an **End-of-Life (EOL)** product and no longer receives security updates, posing a significant risk of compromise.

Furthermore, significant gaps were identified in the organization's security policies. The lack of mandatory Multi-Factor Authentication (MFA) for email access and the absence of an annual security awareness training program for all employees represent high-risk deficiencies. These policy gaps increase the organization's susceptibility to phishing, business email compromise, and other social engineering attacks.

This report provides a detailed breakdown of these findings and offers actionable recommendations to mitigate the identified risks and improve the overall security posture of \textbf{Iron Oak Furniture}.

% --- 2. Organizational Information ---
\section{Organizational Information}

The following details were provided for the assessment.

\begin{tabular}{@{}ll}
    \toprule
    \textbf{Attribute} & \textbf{Value} \\
    \midrule
    Organization Name & \textbf{Iron Oak Furniture} \\
    Primary Email Domain & \texttt{IronOakFurniture.com} \\
    Primary Website & \url{www.IronOakFurniture.com} \\
    Known External IP & \texttt{120.89.239.139} \\
    \bottomrule
\end{tabular}

% --- 3. Security Control Review ---
\section{Security Control Review}

A review of the organization's security controls was conducted via a questionnaire. The responses are summarized below. Items marked with \ding{55} indicate a deviation from security best practices and represent a potential risk.

\begin{table}[h!]
\centering
\begin{tabular}{p{0.8\textwidth}c}
    \toprule
    \textbf{Control Question} & \textbf{Response} \\
    \midrule
    Do you require MFA to access email? & \textcolor{red}{\ding{55}} \\
    Do you require MFA to log into computers? & \textcolor{green}{\ding{51}} \\
    Do you require MFA to access sensitive data systems? & \textcolor{green}{\ding{51}} \\
    Does your organization have an employee acceptable use policy? & \textcolor{green}{\ding{51}} \\
    Does your organization do security awareness training for new employees? & \textcolor{green}{\ding{51}} \\
    Does your organization do security awareness training for all employees at least once per year? & \textcolor{red}{\ding{55}} \\
    \bottomrule
\end{tabular}
\caption{Security Controls Questionnaire Results}
\end{label{tab:controls}
\end{table}

\subsection*{Analysis of Control Gaps}
\begin{itemize}
    \item \textbf{MFA for Email (Critical Gap):} The absence of MFA on email is a critical vulnerability. Email accounts are a primary target for attackers seeking to perform credential theft, business email compromise (BEC), and launch further attacks against the organization.
    \item \textbf{Annual Security Training (High Risk):} Without regular, recurring security awareness training, employees are more likely to fall victim to evolving phishing and social engineering tactics. A one-time training for new hires is insufficient to maintain a high level of security consciousness.
\end{itemize}

% --- 4. Technical Scan Results ---
\section{Technical Scan Results}

A network scan was performed to identify open ports and exposed services on the target system.

\subsection*{Host Scanned: \texttt{172.16.50.20}}
The scan revealed the following open port and service:

\begin{table}[h!]
\centering
\begin{tabular}{l l l l p{0.3\textwidth}}
    \toprule
    \textbf{Port} & \textbf{State} & \textbf{Service} & \textbf{Version} & \textbf{Notes} \\
    \midrule
    3306/tcp & Open & MySQL & 5.7.33 & \textbf{Critical Finding:} This version is End-of-Life (EOL) as of October 2023 and is no longer supported with security patches. Direct exposure is highly discouraged. \\
    \bottomrule
\end{tabular}
\caption{Open Ports on \texttt{172.16.50.20}}
\label{tab:scanresults}
\end{table}

\subsection*{Analysis of Technical Findings}
The scan confirmed the pre-existing risk of an exposed database. The discovery that the MySQL service is an outdated and unsupported version significantly elevates this risk. EOL software is a prime target for attackers, as known vulnerabilities are not patched by the vendor, providing a reliable method of exploitation.

% --- 5. Consolidated Risk Assessment ---
\section{Consolidated Risk Assessment}

The following table synthesizes findings from the security control review, technical scan, and pre-existing risk data into a prioritized list.

\begin{table}[h!]
\centering
\begin{tabular}{l p{0.6\textwidth} l}
    \toprule
    \textbf{Risk ID} & \textbf{Finding} & \textbf{Severity} \\
    \midrule
    RISK-001 & \textbf{Exposed and End-of-Life Database Service:} MySQL port 3306 is open to the network, and the running version (5.7.33) is unsupported. & \textbf{Critical} \\
    \addlinespace
    RISK-002 & \textbf{Lack of Multi-Factor Authentication (MFA) on Email:} The primary communication system lacks a critical security control, exposing it to account takeover. & \textbf{High} \\
    \addlinespace
    RISK-003 & \textbf{Inadequate Security Awareness Training Program:} The absence of mandatory annual training for all staff increases susceptibility to social engineering. & \textbf{High} \\
    \bottomrule
\end{tabular}
\caption{Summary of Identified Risks}
\label{tab:risks}
\end{table}

% --- 6. Recommendations ---
\section{Recommendations}

The following actions are recommended to mitigate the identified risks.

\subsection*{RISK-001: Exposed and End-of-Life Database Service}
\begin{itemize}
    \item \textbf{Immediate Action:} Implement strict firewall rules to block all external access to TCP port 3306 on host \texttt{172.16.50.20}. Access should only be permitted from trusted internal application servers.
    \item \textbf{Short-Term Action:} Develop and execute a plan to upgrade the MySQL database from the EOL version 5.7 to a currently supported version (e.g., MySQL 8.x).
    \item \textbf{Long-Term Strategy:} Adopt a policy of network segmentation where database servers are placed in a secure zone, inaccessible from the public internet. All administrative access should require a Virtual Private Network (VPN) connection.
\end{itemize}

\subsection*{RISK-002: Lack of Multi-Factor Authentication (MFA) on Email}
\begin{itemize}
    \item \textbf{Immediate Action:} Procure and enforce MFA for all user accounts on the \texttt{IronOakFurniture.com} email platform. This is the single most effective control to prevent unauthorized account access.
\end{itemize}

\subsection*{RISK-003: Inadequate Security Awareness Training Program}
\begin{itemize}
    \item \textbf{Short-Term Action:} Implement a mandatory annual security awareness training program for all employees. This program should cover key topics such as phishing identification, password hygiene, and acceptable use of company assets.
\end{itemize}

\end{document}
```