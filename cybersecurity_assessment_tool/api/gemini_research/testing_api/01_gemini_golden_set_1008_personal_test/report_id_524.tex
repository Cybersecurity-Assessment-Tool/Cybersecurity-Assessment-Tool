```latex
\documentclass[12pt]{article}

% 1. DOCUMENT SETUP & PACKAGES
\usepackage[margin=1in]{geometry}
\usepackage{pifont} % For checkmarks and crosses
\usepackage{booktabs} % For professional tables
\usepackage{hyperref} % For hyperlinks and metadata
\usepackage{url} % For formatting URLs
\usepackage{seqsplit} % For splitting long text strings
\usepackage{graphicx}
\usepackage{xcolor}

% --- Hyperref Setup ---
\hypersetup{
    colorlinks=true,
    linkcolor=black,
    urlcolor=blue,
    pdftitle={Cybersecurity Posture Report},
    pdfauthor={Cybersecurity Analyst},
    pdfsubject={Security Assessment},
    pdfkeywords={Cybersecurity, Risk Assessment, Nmap, Security Controls}
}

% --- Document Title ---
\title{Cybersecurity Posture Report \\ \large For: \textbf{Symmetry Architecture}}
\author{Cybersecurity Analyst}
\date{\today}

\begin{document}

\maketitle
\thispagestyle{empty}
\newpage

\tableofcontents
\newpage

% 2. EXECUTIVE SUMMARY / OVERVIEW
\section{Overview}
This report provides a comprehensive analysis of the cybersecurity posture for \textbf{Symmetry Architecture}. The assessment is based on a synthesis of three data sources: a technical network scan, a security controls questionnaire, and a review of pre-existing risk data.

The organization demonstrates a foundational level of security by implementing Multi-Factor Authentication (MFA) for email and computer access. However, significant and critical gaps were identified that expose the organization to substantial risk. 

Key findings include:
\begin{itemize}
    \item \textbf{Critical Control Gaps:} The lack of MFA for sensitive data systems represents a critical vulnerability.
    \item \textbf{Policy Deficiencies:} The absence of a formal Acceptable Use Policy (AUP) and mandatory annual security awareness training for all staff creates significant human-factor risks.
    \item \textbf{Technical Vulnerabilities:} The network scan revealed an open port serving unencrypted HTTP traffic, posing a risk to data confidentiality and integrity.
\end{itemize}

This report details these findings and provides prioritized, actionable recommendations to mitigate the identified risks and strengthen the overall security posture.

% 3. ORGANIZATIONAL INFORMATION
\section{Organizational Information}
The following details were provided for the assessment.
\begin{center}
\begin{tabular}{ll}
\toprule
\textbf{Attribute} & \textbf{Value} \\
\midrule
Organization Name & \textbf{Symmetry Architecture} \\
Email Domain & \texttt{SymmetryArchitecture.org} \\
Website Domain & \url{www.SymmetryArchitecture.org} \\
External IP Address & \texttt{78.109.60.101} \\
\bottomrule
\end{tabular}
\end{center}

% 4. SECURITY CONTROL REVIEW (FROM QUESTIONNAIRE)
\section{Security Control Review}
A review of the organization's security controls was conducted via a questionnaire. The responses indicate several areas requiring immediate attention. "No" answers highlight significant gaps in the current security framework.

\begin{center}
\begin{tabular}{p{0.75\linewidth} c}
\toprule
\textbf{Control Question} & \textbf{Response} \\
\midrule
Do you require MFA to access email? & \textcolor{green}{\ding{51}} \\
Do you require MFA to log into computers? & \textcolor{green}{\ding{51}} \\
Do you require MFA to access sensitive data systems? & \textcolor{red}{\ding{55}} \\
Does your organization have an employee acceptable use policy? & \textcolor{red}{\ding{55}} \\
Does your organization do security awareness training for new employees? & \textcolor{green}{\ding{51}} \\
Does your organization do security awareness training for all employees at least once per year? & \textcolor{red}{\ding{55}} \\
\bottomrule
\end{tabular}
\end{center}

\subsection*{Analysis of Controls}
The "No" responses to critical controls are major contributors to the organization's risk profile. The lack of MFA on sensitive systems is a primary concern, as it leaves high-value data vulnerable to compromise through stolen credentials. Additionally, the absence of an Acceptable Use Policy and annual security training significantly increases the risk of insider threats, whether malicious or unintentional.

% 5. TECHNICAL SCAN RESULTS
\section{Technical Scan Results}
A network scan was performed on the specified target to identify open ports and services.

\begin{itemize}
    \item \textbf{Target IP Address:} \texttt{172.16.0.1}
    \item \textbf{Scan Utility:} Nmap
\end{itemize}

\subsection*{Open Ports Discovered}
The following table details the open ports found on the target host.

\begin{center}
\begin{tabular}{llll}
\toprule
\textbf{Port} & \textbf{State} & \textbf{Service} & \textbf{Notes} \\
\midrule
80/tcp & open & HTTP & Unencrypted web traffic. This service is vulnerable to \\
& & & eavesdropping and man-in-the-middle attacks. \\
\bottomrule
\end{tabular}
\end{center}

\subsection*{Technical Analysis}
The scan identified that Port 80 (HTTP) is open. Transmitting data over HTTP is inherently insecure as the traffic is not encrypted. This allows any adversary on the network path to intercept, read, or modify the data being exchanged. It is standard industry practice to use HTTPS (Port 443) to encrypt web traffic. The scan data did not provide specific service, product, or version information, which limits a deeper vulnerability analysis but highlights the immediate risk of using an unencrypted protocol.

% 6. RISK ASSESSMENT SUMMARY
\section{Risk Assessment}
This section correlates the findings from the security control review and the technical scan into a prioritized list of identified risks. The provided pre-existing risk data was determined to be invalid and was excluded from this analysis.

\begin{center}
\begin{tabular}{p{0.2\linewidth} p{0.55\linewidth} l}
\toprule
\textbf{Risk Title} & \textbf{Description} & \textbf{Severity} \\
\midrule
\textbf{No MFA on Sensitive Systems} & Failure to protect systems containing sensitive data with MFA greatly increases the risk of a data breach from compromised credentials. & \textbf{\textcolor{red}{Critical}} \\
\addlinespace
\textbf{Lack of AUP} & The absence of a formal Acceptable Use Policy creates ambiguity for employees and limits the organization's ability to enforce secure behavior. & \textbf{\textcolor{orange}{High}} \\
\addlinespace
\textbf{Inadequate Security Training} & Without mandatory annual refresher training, employees' awareness of evolving threats like phishing diminishes, making them more likely to fall victim to attacks. & \textbf{\textcolor{orange}{High}} \\
\addlinespace
\textbf{Unencrypted Web Traffic} & The use of HTTP on Port 80 exposes website visitors and internal data to interception and manipulation, damaging trust and data integrity. & \textbf{\textcolor{yellow!80!black}{Medium}} \\
\bottomrule
\end{tabular}
\end{center}

% 7. RECOMMENDATIONS
\section{Recommendations}
The following actions are recommended to mitigate the identified risks. They are prioritized based on severity and potential impact.

\begin{enumerate}
    \item \textbf{[Critical Priority] Implement MFA for Sensitive Systems:} Immediately deploy mandatory Multi-Factor Authentication on all applications, databases, and systems that store or process sensitive or confidential information. This is the single most effective control to prevent unauthorized access.

    \item \textbf{[High Priority] Develop and Enforce an Acceptable Use Policy (AUP):} Create a formal AUP that clearly defines the rules and expectations for the use of company technology, data, and network resources. All employees must be required to read and formally acknowledge the policy.

    \item \textbf{[High Priority] Establish Annual Security Awareness Training:} Institute a mandatory security awareness training program for all employees to be completed at least once per year. The training should cover current threats such as phishing, social engineering, and proper data handling.

    \item \textbf{[Medium Priority] Migrate to HTTPS:} Decommission the HTTP service on Port 80. Configure the web server to use HTTPS exclusively (on Port 443) with a valid TLS certificate and strong cipher suites to encrypt all web traffic.
\end{enumerate}

\end{document}
```