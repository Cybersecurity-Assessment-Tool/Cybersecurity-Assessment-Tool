```latex
\documentclass[12pt]{article}

% --- PACKAGES ---
\usepackage[margin=1in]{geometry}
\usepackage{pifont} % For checkmarks and crosses
\usepackage{booktabs} % For professional tables
\usepackage{hyperref} % For clickable links
\usepackage{url} % For URL formatting
\usepackage{seqsplit} % For splitting long strings in texttt
\usepackage{xcolor} % For colors
\usepackage{fancyhdr} % For headers and footers

% --- DOCUMENT SETUP ---
\hypersetup{
    colorlinks=true,
    linkcolor=blue,
    filecolor=magenta,      
    urlcolor=cyan,
    pdftitle={Cybersecurity Posture Assessment Report},
    pdfpagemode=FullScreen,
}

% --- HEADER & FOOTER ---
\pagestyle{fancy}
\fancyhf{}
\fancyhead[L]{Cybersecurity Posture Assessment}
\fancyhead[R]{\textbf{Nova Terra}}
\fancyfoot[C]{\thepage}
\renewcommand{\headrulewidth}{0.4pt}
\renewcommand{\footrulewidth}{0.4pt}

% --- COMMANDS ---
\newcommand{\yes}{\ding{51}}
\newcommand{\no}{\ding{55}}

\begin{document}

% --- TITLE PAGE ---
\begin{titlepage}
    \centering
    \vspace*{1cm}
    \Huge
    \textbf{Cybersecurity Posture Assessment Report}
    \vspace{1.5cm}
    \Large
    Prepared for: \textbf{Nova Terra}
    \vspace{2cm}
    \normalsize
    Report Date: \today \\
    \vspace{0.5cm}
    Analysis conducted by: \\
    \textbf{Cybersecurity Analysis Division}
    \vfill
    \textit{This report contains sensitive information and is intended for the exclusive use of the recipient organization. Unauthorized distribution is prohibited.}
\end{titlepage}

\tableofcontents
\newpage

% --- EXECUTIVE OVERVIEW ---
\section{Executive Overview}
This report provides a comprehensive assessment of the cybersecurity posture for \textbf{Nova Terra}, based on a correlation of organizational data, a technical network scan, and a review of pre-existing risks. The analysis was conducted on \today.

The organization demonstrates a solid foundation in certain areas, particularly with the mandatory use of Multi-Factor Authentication (MFA) for email access and the implementation of security awareness training for all employees. These are commendable controls that significantly reduce the risk of phishing and account compromise.

However, the assessment identified several critical and high-risk gaps that require immediate attention. The most significant concerns are the absence of MFA for computer logins and access to sensitive data systems. This exposes the organization to substantial risk from credential theft. Furthermore, the lack of a formal employee Acceptable Use Policy (AUP) creates ambiguity regarding security responsibilities and data handling procedures.

On a positive note, the technical scan of the target host \texttt{192.168.0.5} showed no open ports. This contradicts a pre-existing risk concerning an unencrypted web server on Port 80, suggesting that the vulnerability may have been successfully remediated.

This report outlines these findings in detail and provides actionable recommendations to mitigate the identified risks and strengthen the overall security posture.

% --- ORGANIZATIONAL INFORMATION ---
\section{Organizational Information}
The following details were provided for the assessment:
\begin{itemize}
    \item \textbf{Organization Name:} Nova Terra
    \item \textbf{Email Domain:} \seqsplit{\texttt{NovaTerra.net}}
    \item \textbf{Website Domain:} \seqsplit{\texttt{www.NovaTerra.net}}
    \item \textbf{External IP Address:} \texttt{71.6.111.74}
\end{itemize}

% --- SECURITY CONTROL REVIEW ---
\section{Security Control Review (Questionnaire)}
An analysis of the organization's security questionnaire responses reveals the status of key administrative and technical controls. While some controls are well-implemented, critical gaps were identified.

\begin{table}[h!]
\centering
\caption{Security Control Questionnaire Analysis}
\begin{tabular}{p{0.6\linewidth} c p{0.25\linewidth}}
\toprule
\textbf{Control Question} & \textbf{Status} & \textbf{Assessment} \\
\midrule
Do you require MFA to access email? & \textcolor{green}{\yes} & Strong control. \\
\addlinespace
Do you require MFA to log into computers? & \textcolor{red}{\no} & \textbf{Critical Gap.} Lack of MFA on endpoints significantly increases risk of unauthorized access. \\
\addlinespace
Do you require MFA to access sensitive data systems? & \textcolor{red}{\no} & \textbf{Critical Gap.} Sensitive data is not adequately protected from credential compromise. \\
\addlinespace
Does your organization have an employee acceptable use policy? & \textcolor{red}{\no} & \textbf{High Risk.} Absence of a formal policy leads to inconsistent security practices. \\
\addlinespace
Does your organization do security awareness training for new employees? & \textcolor{green}{\yes} & Good practice. \\
\addlinespace
Does your organization do security awareness training for all employees at least once per year? & \textcolor{green}{\yes} & Excellent. Reinforces security culture. \\
\bottomrule
\end{tabular}
\end{table}

% --- TECHNICAL SCAN RESULTS ---
\section{Technical Scan Results}
A network scan was performed to identify open ports and exposed services on the specified target system.

\begin{itemize}
    \item \textbf{Target IP Address:} \texttt{192.168.0.5}
    \item \textbf{Scan Date:} \today
\end{itemize}

The scan results indicate that the target host is not exposing any services to the scanner's network segment. This is a positive security finding.

\begin{table}[h!]
\centering
\caption{Port Scan Results for \texttt{192.168.0.5}}
\begin{tabular}{l l l l}
\toprule
\textbf{Port} & \textbf{State} & \textbf{Service} & \textbf{Version} \\
\midrule
80 & closed & http & N/A \\
\bottomrule
\end{tabular}
\end{table}

\subsection{Analysis of Technical Findings}
The scan revealed that port 80 (HTTP) is \textbf{closed}. This is a significant finding as it directly contradicts the pre-existing risk titled "Unencrypted Web Server," which was based on the assumption that this port was open. This suggests that the previously identified risk has been remediated or was a false positive. No other open ports were detected, indicating a hardened host from this scan's perspective.

% --- RISK ASSESSMENT SUMMARY ---
\section{Risk Assessment Summary}
The following table synthesizes findings from the security control review, technical scan, and pre-existing risk data into a prioritized list.

\begin{table}[h!]
\centering
\caption{Synthesized Risk Register}
\begin{tabular}{p{0.1\linewidth} p{0.4\linewidth} p{0.15\linewidth} p{0.2\linewidth}}
\toprule
\textbf{ID} & \textbf{Risk Description} & \textbf{Severity} & \textbf{Source} \\
\midrule
RISK-001 & \textbf{Lack of MFA on Endpoints and Systems:} User computers and sensitive data systems are accessible with only a password, exposing them to credential stuffing and phishing attacks. & \textbf{Critical} & Questionnaire \\
\addlinespace
RISK-002 & \textbf{No Employee Acceptable Use Policy (AUP):} Without a formal AUP, there are no defined rules for employee use of company assets, data handling, or security responsibilities. & \textbf{High} & Questionnaire \\
\addlinespace
RISK-003 & \textbf{Unencrypted Web Server Risk Status:} A pre-existing risk for an open Port 80 appears to be resolved, as current scans show the port is closed. The risk requires validation and closure. & Informational & Scan / Risk Correlation \\
\bottomrule
\end{tabular}
\end{table}

% --- RECOMMENDATIONS ---
\section{Recommendations}
The following actions are recommended to address the identified risks and improve the overall security posture of \textbf{Nova Terra}.

\subsection{RISK-001: Implement Comprehensive MFA (Critical)}
\begin{itemize}
    \item \textbf{Immediate Action:} Prioritize and enforce MFA on all systems containing sensitive data. This includes databases, financial applications, and administrative portals.
    \item \textbf{Short-Term Action:} Develop a roadmap for deploying MFA for all employee computer logins (e.g., Windows Hello, Duo, etc.). Begin with a pilot group of IT staff and executives.
    \item \textbf{Long-Term Goal:} Achieve 100\% MFA enforcement for all user accounts across all critical systems and endpoints.
\end{itemize}

\subsection{RISK-002: Develop and Implement an AUP (High)}
\begin{itemize}
    \item \textbf{Immediate Action:} Draft a formal Acceptable Use Policy. This policy should clearly define rules for data handling, internet usage, personal device usage (BYOD), and password management.
    \item \textbf{Short-Term Action:} Have the AUP reviewed by legal and HR departments. Upon approval, distribute it to all employees and require a signed acknowledgment. Integrate the AUP into the new employee onboarding process.
    \item \textbf{Long-Term Goal:} Schedule annual reviews of the AUP to ensure it remains current with evolving technology and threats.
\end{itemize}

\subsection{RISK-003: Validate and Close Remediated Risk (Informational)}
\begin{itemize}
    \item \textbf{Immediate Action:} Confirm internally that the closure of Port 80 on host \texttt{192.168.0.5} and other relevant web servers was intentional and is aligned with business requirements.
    \item \textbf{Short-Term Action:} If the remediation is confirmed, formally update the organization's risk register to mark the "Unencrypted Web Server" vulnerability as "Closed" or "Remediated." This ensures that security resources are focused on current, active risks.
\end{itemize}

\end{document}
```