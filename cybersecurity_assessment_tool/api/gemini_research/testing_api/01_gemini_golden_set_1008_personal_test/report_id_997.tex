```latex
\documentclass[12pt]{article}

% ----------------------------------------------------------------------
% PREAMBLE
% ----------------------------------------------------------------------
\usepackage[margin=1in]{geometry}
\usepackage{pifont} % For \ding{51} (checkmark) and \ding{55} (cross)
\usepackage{booktabs} % For professional-looking tables
\usepackage{hyperref} % For clickable links and references
\usepackage{url}      % For formatting URLs
\usepackage{seqsplit} % To split long monospaced strings without breaking
\usepackage{graphicx}
\usepackage{xcolor}
\usepackage{fancyhdr}
\usepackage{array}

% --- Customization ---
\definecolor{darkblue}{rgb}{0.0, 0.0, 0.55}
\definecolor{darkred}{rgb}{0.55, 0.0, 0.0}
\definecolor{darkgreen}{rgb}{0.0, 0.35, 0.0}

\hypersetup{
    colorlinks=true,
    linkcolor=darkblue,
    filecolor=darkblue,
    urlcolor=darkblue,
    citecolor=darkblue,
}

% --- Header and Footer ---
\pagestyle{fancy}
\fancyhf{} % Clear all header and footer fields
\lhead{Cybersecurity Assessment Report}
\rhead{Green Sprout Organic}
\cfoot{\thepage}
\renewcommand{\headrulewidth}{0.4pt}
\renewcommand{\footrulewidth}{0.4pt}

% --- Table Column Definitions ---
\newcolumntype{L}[1]{>{\raggedright\let\newline\\\arraybackslash\hspace{0pt}}m{#1}}
\newcolumntype{C}[1]{>{\centering\let\newline\\\arraybackslash\hspace{0pt}}m{#1}}

% ----------------------------------------------------------------------
% DOCUMENT START
% ----------------------------------------------------------------------
\begin{document}

% --- Title Page ---
\begin{titlepage}
    \centering
    \vspace*{2cm}
    
    {\Huge \textbf{Cybersecurity Posture Assessment Report}}
    
    \vspace{1.5cm}
    
    {\Large \textbf{Prepared for:}} \\
    \vspace{0.5cm}
    {\Large Green Sprout Organic}
    
    \vspace{2cm}
    
    {\large \textbf{Date of Report:}} \\
    \vspace{0.2cm}
    {\large November 22, 2025}
    
    \vspace{2cm}
    
    {\large \textbf{Generated By:}} \\
    \vspace{0.2cm}
    {\large Cybersecurity Analysis Division}
    
    \vfill
    
    \textit{This document contains sensitive information. Distribution is restricted.}
    
\end{titlepage}

\tableofcontents
\newpage

% ----------------------------------------------------------------------
% SECTION 1: EXECUTIVE OVERVIEW
% ----------------------------------------------------------------------
\section{Executive Overview}

This report details the findings of a cybersecurity posture assessment conducted for \textbf{Green Sprout Organic}. The assessment combined a review of organizational security controls, an external network scan, and an analysis of pre-existing risks.

The assessment identified several critical and high-risk security gaps that require immediate attention. The most urgent issues are the absence of Multi-Factor Authentication (MFA) for both email and sensitive data systems. These gaps expose the organization to significant risks of account compromise and data breaches.

Furthermore, technical analysis revealed that the primary external web server is running an outdated version of Nginx (1.18.0), which is no longer receiving security patches and is susceptible to known vulnerabilities. Finally, a procedural gap was noted in the security awareness training program, which is not conducted annually for all employees, increasing the organization's susceptibility to social engineering attacks.

This report provides a detailed breakdown of these findings and offers actionable recommendations to mitigate the identified risks and strengthen the overall security posture of \textbf{Green Sprout Organic}.

% ----------------------------------------------------------------------
% SECTION 2: ORGANIZATIONAL INFORMATION
% ----------------------------------------------------------------------
\section{Organizational Information}

The following details were provided for the assessment:

\begin{itemize}
    \item \textbf{Organization Name:} Green Sprout Organic
    \item \textbf{Email Domain:} \texttt{GreenSproutOrganic.net}
    \item \textbf{Website Domain:} \url{www.GreenSproutOrganic.net}
    \item \textbf{External IP Address:} \texttt{163.187.70.53}
\end{itemize}

% ----------------------------------------------------------------------
% SECTION 3: SECURITY CONTROL REVIEW
% ----------------------------------------------------------------------
\section{Security Control Review}

A review of internal security controls was conducted based on a standardized questionnaire. The results highlight critical gaps in access control and employee training policies. A "No" answer indicates a deviation from security best practices.

\begin{table}[h!]
\centering
\caption{Security Controls Questionnaire Results}
\label{tab:controls}
\begin{tabular}{L{12cm} C{2cm}}
\toprule
\textbf{Control Question} & \textbf{Status} \\
\midrule
Do you require MFA to access email? & \textcolor{darkred}{\ding{55}} \\
Do you require MFA to log into computers? & \textcolor{darkgreen}{\ding{51}} \\
Do you require MFA to access sensitive data systems? & \textcolor{darkred}{\ding{55}} \\
Does your organization have an employee acceptable use policy? & \textcolor{darkgreen}{\ding{51}} \\
Does your organization do security awareness training for new employees? & \textcolor{darkgreen}{\ding{51}} \\
Does your organization do security awareness training for all employees at least once per year? & \textcolor{darkred}{\ding{55}} \\
\bottomrule
\end{tabular}
\end{table}

% ----------------------------------------------------------------------
% SECTION 4: TECHNICAL SCAN RESULTS
% ----------------------------------------------------------------------
\section{Technical Scan Results}

An external network scan was performed to identify open ports and exposed services on the organization's perimeter.

\begin{itemize}
    \item \textbf{Scan Target:} \texttt{192.168.10.5}
    \item \textbf{Scan Date:} 2025-11-22T10:00:00Z
\end{itemize}

\subsection{Open Ports}
The following table details the open ports and services discovered during the scan.

\begin{table}[h!]
\centering
\caption{Discovered Open Ports and Services}
\label{tab:ports}
\begin{tabular}{lllll}
\toprule
\textbf{Port} & \textbf{State} & \textbf{Service} & \textbf{Product} & \textbf{Version} \\
\midrule
443/tcp & open & https & nginx & 1.18.0 \\
\bottomrule
\end{tabular}
\end{table}

\subsection{Analysis and Observations}
The scan identified one primary service exposed to the internet: an Nginx web server on port 443 (HTTPS).

\begin{itemize}
    \item \textbf{Outdated Software:} The detected Nginx version is \textbf{1.18.0}, which was released in April 2020. This version is significantly outdated and no longer receives security updates from the developer. It is highly likely to be vulnerable to numerous publicly disclosed exploits. This represents a high-risk finding.
    \item \textbf{Certificate Mismatch:} The SSL certificate's Common Name is listed as \texttt{www.acme-corp.com}, which does not match the organization's domain. This could indicate a misconfiguration, a test environment exposed to the internet, or a shared hosting environment.
\end{itemize}

% ----------------------------------------------------------------------
% SECTION 5: CONSOLIDATED RISK ASSESSMENT
% ----------------------------------------------------------------------
\section{Consolidated Risk Assessment}

The following table synthesizes findings from the security control review and technical scan into a prioritized list of risks.

\begin{table}[h!]
\centering
\caption{Summary of Identified Risks}
\label{tab:risks}
\begin{tabular}{p{1.5cm} p{9cm} p{2.5cm}}
\toprule
\textbf{Risk ID} & \textbf{Risk Description} & \textbf{Severity} \\
\midrule
\textbf{RISK-001} & \textbf{Lack of MFA for Email Access:} User email accounts are protected only by passwords, making them highly vulnerable to phishing, credential stuffing, and brute-force attacks. A compromised email account is a primary vector for further network intrusion and data exfiltration. & \textbf{Critical} \\
\addlinespace
\textbf{RISK-002} & \textbf{Lack of MFA for Sensitive Data Systems:} Critical systems containing sensitive data lack the protection of MFA. This significantly increases the risk of unauthorized access and a major data breach if credentials are compromised. & \textbf{Critical} \\
\addlinespace
\textbf{RISK-003} & \textbf{Outdated Web Server Software:} The public-facing Nginx web server is running version 1.18.0, which is end-of-life and contains known vulnerabilities. This exposes the server to automated attacks that could lead to a system compromise. & \textbf{High} \\
\addlinespace
\textbf{RISK-004} & \textbf{Inadequate Annual Security Training:} The absence of mandatory annual security awareness training for all employees increases the likelihood of successful social engineering and phishing attacks, as staff may not be aware of current threats and best practices. & \textbf{High} \\
\bottomrule
\end{tabular}
\end{table}

% ----------------------------------------------------------------------
% SECTION 6: RECOMMENDATIONS
% ----------------------------------------------------------------------
\section{Recommendations}

The following actions are recommended to mitigate the identified risks and improve the organization's security posture.

\subsection{Remediate RISK-001: Enforce MFA for Email}
\begin{itemize}
    \item \textbf{Action:} Immediately enable and enforce Multi-Factor Authentication (MFA) for all user accounts on the organization's email platform (e.g., Microsoft 365, Google Workspace).
    \item \textbf{Priority:} \textbf{Critical}. This should be addressed within 7 days.
\end{itemize}

\subsection{Remediate RISK-002: Enforce MFA for Sensitive Systems}
\begin{itemize}
    \item \textbf{Action:} Conduct an inventory of all systems containing sensitive data. Prioritize these systems and implement MFA for all user and administrative access.
    \item \textbf{Priority:} \textbf{Critical}. This should be addressed within 30 days.
\end{itemize}

\subsection{Remediate RISK-003: Upgrade Web Server Software}
\begin{itemize}
    \item \textbf{Action:} Plan and execute an upgrade of the Nginx server from version 1.18.0 to the latest stable version. Before deployment, test the new version in a staging environment to ensure application compatibility.
    \item \textbf{Priority:} \textbf{High}. This should be addressed within 30 days.
\end{itemize}

\subsection{Remediate RISK-004: Implement Annual Security Training}
\begin{itemize}
    \item \textbf{Action:} Establish a mandatory annual security awareness training program for all employees. The training should cover current threats such as phishing, ransomware, and proper data handling.
    \item \textbf{Priority:} \textbf{High}. A program should be selected and scheduled within 60 days.
\end{itemize}

\end{document}
```