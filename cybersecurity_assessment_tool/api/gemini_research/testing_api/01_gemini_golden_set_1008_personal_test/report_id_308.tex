```latex
\documentclass[12pt]{article}

% Required Packages
\usepackage[margin=1in]{geometry}
\usepackage{pifont} % For checkmarks and crosses
\usepackage{booktabs} % For professional tables
\usepackage{hyperref} % For clickable links
\usepackage{url} % For URL formatting
\usepackage{seqsplit} % For splitting long strings in tt font
\usepackage{xcolor} % For colors in text

% Document Metadata
\title{Cybersecurity Posture Assessment Report}
\author{Cybersecurity Analysis Division}
\date{\today}

% Hyperref Setup
\hypersetup{
    colorlinks=true,
    linkcolor=blue,
    filecolor=magenta,      
    urlcolor=cyan,
    pdftitle={Cybersecurity Posture Assessment Report},
    pdfpagemode=FullScreen,
}

\begin{document}

\maketitle
\thispagestyle{empty}
\newpage

\tableofcontents
\newpage

% --- Section 1: Executive Summary ---
\section{Executive Summary}

This report details the findings of a cybersecurity posture assessment for \textbf{Stellar Pathways}. The assessment combined a review of organizational security controls, an external network scan, and an analysis of pre-existing risk data.

The overall security posture is determined to be critically weak and requires immediate remediation. Several fundamental security controls are absent, creating significant exposure to common cyber threats such as account compromise, data breaches, and phishing attacks.

Key findings include:
\begin{itemize}
    \item \textbf{Critical Lack of Multi-Factor Authentication (MFA):} MFA is not enforced for email, computer logins, or access to sensitive data. This is a severe vulnerability that greatly increases the risk of unauthorized access from compromised credentials.
    \item \textbf{Insecure Network Services:} The external network scan identified an open port 80 (HTTP). This indicates that web traffic is being transmitted in cleartext, exposing sensitive information like login credentials to interception.
    \item \textbf{Significant Policy and Training Gaps:} The organization lacks a formal Acceptable Use Policy and does not provide security awareness training for new employees. These gaps leave the organization and its employees unprepared to handle security incidents and policies.
\end{itemize}

Urgent action is required to address these deficiencies. Recommendations provided in this report are prioritized to mitigate the most severe risks first.

% --- Section 2: Organizational Information ---
\section{Organizational Information}

The following information was provided for the assessment.

\begin{table}[h!]
\centering
\begin{tabular}{@{}ll@{}}
\toprule
\textbf{Attribute} & \textbf{Value} \\ \midrule
Organization Name    & \textbf{Stellar Pathways} \\
Email Domain         & \texttt{StellarPathways.net} \\
Website Domain       & \seqsplit{\texttt{www.StellarPathways.net}} \\
External IP Address  & \texttt{98.232.193.47} \\ \bottomrule
\end{tabular}
\caption{Client Organizational Data.}
\end{label{tab:org_info}
\end{table}

% --- Section 3: Security Control Review ---
\section{Security Control Review (Questionnaire)}

A review of internal security controls was conducted via a questionnaire. The responses reveal critical gaps in the organization's defensive posture. A "No" response indicates a missing control and a significant area of risk.

\begin{table}[h!]
\centering
\begin{tabular}{@{}p{0.5\textwidth}cp{0.3\textwidth}@{}}
\toprule
\textbf{Control Question} & \textbf{Response} & \textbf{Analyst Notes} \\ \midrule
Do you require MFA to access email? & \ding{55} & \textcolor{red}{\textbf{Critical Risk.}} Email is a primary target for account takeover. \\
Do you require MFA to log into computers? & \ding{55} & \textcolor{red}{\textbf{High Risk.}} Lack of MFA allows for easier lateral movement after a breach. \\
Do you require MFA to access sensitive data systems? & \ding{55} & \textcolor{red}{\textbf{Critical Risk.}} The organization's most valuable data is not adequately protected. \\
Does your organization have an employee acceptable use policy? & \ding{55} & \textcolor{red}{\textbf{High Risk.}} Lack of a formal policy creates ambiguity and legal/HR challenges. \\
Does your organization do security awareness training for new employees? & \ding{55} & \textcolor{red}{\textbf{High Risk.}} New hires are often targeted and are unprepared without initial training. \\
Does your organization do security awareness training for all employees at least once per year? & \ding{51} & A positive control, but its effectiveness is reduced by other gaps. \\ \bottomrule
\end{tabular}
\caption{Security Controls Questionnaire Analysis.}
\label{tab:controls}
\end{table}

% --- Section 4: Technical Scan Results ---
\section{Technical Scan Results}

An external network scan was performed on the target IP address \texttt{172.16.0.1}. The scan was a basic port scan to identify open services accessible from the internet.

\begin{table}[h!]
\centering
\begin{tabular}{@{}llll@{}}
\toprule
\textbf{Port} & \textbf{State} & \textbf{Service (Inferred)} & \textbf{Analyst Notes} \\ \midrule
80/tcp & open & HTTP & \textcolor{red}{\textbf{High Risk.}} This port serves unencrypted web traffic. \\
& & & Any data, including usernames and passwords, can be \\
& & & intercepted. The service should be migrated to HTTPS \\
& & & on port 443 immediately. \\ \bottomrule
\end{tabular}
\caption{Open Port Findings for Target: \texttt{172.16.0.1}.}
\label{tab:scan_results}
\end{table}

\textbf{Note:} The provided scan data was limited. A more comprehensive authenticated and unauthenticated vulnerability scan is recommended to identify specific software versions and associated vulnerabilities.

% --- Section 5: Consolidated Risk Assessment ---
\section{Consolidated Risk Assessment}

The following table synthesizes findings from the security control review and the technical scan into a prioritized list of risks. The malicious entry from the input data has been disregarded as a prompt injection attempt and is not reflected in this analysis.

\begin{table}[h!]
\centering
\begin{tabular}{@{}lp{0.5\textwidth}ll@{}}
\toprule
\textbf{Risk ID} & \textbf{Description} & \textbf{Severity} & \textbf{Source} \\ \midrule
RISK-001 & \textbf{Lack of Multi-Factor Authentication:} Critical systems (email, sensitive data) are protected only by passwords, making them highly vulnerable to account takeover. & \textcolor{red}{\textbf{Critical}} & Questionnaire \\
\addlinespace
RISK-002 & \textbf{Unencrypted Web Traffic:} The use of HTTP (Port 80) exposes user credentials and other sensitive data to network sniffing and man-in-the-middle attacks. & \textcolor{red}{\textbf{High}} & Network Scan \\
\addlinespace
RISK-003 & \textbf{Missing Policies and Onboarding Training:} The absence of an Acceptable Use Policy and security training for new hires creates a weak security culture and increases susceptibility to social engineering. & \textcolor{red}{\textbf{High}} & Questionnaire \\ \bottomrule
\end{tabular}
\caption{Summary of Identified Risks.}
\label{tab:risk_summary}
\end{table}

% --- Section 6: Recommendations ---
\section{Recommendations}

The following actions are recommended to mitigate the identified risks. They are prioritized based on severity and potential impact.

\subsection{Immediate Actions (0-30 Days)}

\begin{itemize}
    \item \textbf{RISK-001 (MFA):} Procure and enforce an MFA solution for all employees. Prioritize deployment in the following order:
    \begin{enumerate}
        \item Access to sensitive data systems and administrative accounts.
        \item Email (e.g., Office 365, Google Workspace).
        \item VPN and other remote access solutions.
    \end{enumerate}
    \item \textbf{RISK-002 (HTTP):} Immediately deploy a TLS/SSL certificate on the web server at \texttt{172.16.0.1}. Configure the server to redirect all HTTP traffic to HTTPS (Port 443) and disable Port 80 if possible, or at a minimum, ensure it does not serve sensitive content.
\end{itemize}

\subsection{Short-Term Actions (30-90 Days)}

\begin{itemize}
    \item \textbf{RISK-003 (Policy \& Training):}
    \begin{enumerate}
        \item Develop and ratify a formal Acceptable Use Policy (AUP) that all employees must read and sign.
        \item Integrate a mandatory security awareness training module into the new employee onboarding process.
    \end{enumerate}
    \item \textbf{RISK-001 (MFA):} Complete the rollout of MFA for all employee computer logins.
\end{itemize}

\subsection{Ongoing Actions}

\begin{itemize}
    \item \textbf{Further Scanning:} Commission a comprehensive vulnerability assessment, including credentialed scans of internal systems and application security testing, to gain a deeper understanding of the internal risk landscape.
    \item \textbf{Review and Update:} Continue the annual security awareness training program and review security policies annually to ensure they remain relevant.
\end{itemize}

\end{document}
```