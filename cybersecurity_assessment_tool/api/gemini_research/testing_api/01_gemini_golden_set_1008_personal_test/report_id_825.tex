```latex
\documentclass[12pt]{article}

% Preamble: Required Packages
\usepackage[margin=1in]{geometry}
\usepackage{pifont} % For \ding
\usepackage{booktabs} % For professional tables (\toprule, \midrule, \bottomrule)
\usepackage[hidelinks]{hyperref} % For clickable links without boxes
\usepackage{url} % For typesetting URLs
\usepackage{seqsplit} % For splitting long strings in texttt
\usepackage[table]{xcolor} % For table cell colors
\usepackage{graphicx} % For potential logos/images

% --- Document Metadata ---
\title{Cybersecurity Posture Assessment Report}
\author{Cybersecurity Analysis Division}
\date{\today}

\begin{document}

\maketitle
\thispagestyle{empty}
\newpage
\tableofcontents
\newpage
\pagestyle{headings}

% --- Section 1: Executive Summary ---
\section*{Executive Summary}

This report provides a cybersecurity posture assessment for \textbf{Oasis Wellness}, conducted on \today. The analysis is based on a synthesis of network scan data, organizational security control questionnaires, and a review of pre-existing risk documentation.

The assessment identified a \textbf{Critical} risk finding. An internal network scan of host \texttt{10.5.5.5} revealed an open service on port \texttt{8080} with the title ``TOP SECRET DB''. This suggests a potentially exposed and highly sensitive database. This finding directly contradicts a pre-existing risk assessment which incorrectly labeled the port as secure.

Furthermore, a \textbf{High} risk gap was identified in the organization's security program: the absence of mandatory security awareness training for new employees during their onboarding process. This gap can lead to unintentional security breaches and misconfigurations, potentially contributing to situations like the one observed with the exposed database.

Immediate remediation is required to investigate and secure the service on port \texttt{8080}. Recommendations are also provided to strengthen the employee security training program and improve the accuracy of the ongoing risk management process.

% --- Section 2: Organizational Information ---
\section*{Organizational Information}

The following information was provided for the assessment.

\begin{tabular}{@{}ll}
\toprule
\textbf{Attribute} & \textbf{Value} \\
\midrule
Organization Name & Oasis Wellness \\
Email Domain & \texttt{OasisWellness.com} \\
Website Domain & \url{www.OasisWellness.com} \\
External IP Address & \texttt{104.180.198.82} \\
\bottomrule
\end{tabular}

% --- Section 3: Security Control Review ---
\section*{Security Control Review}

An internal review of administrative security controls was conducted via a questionnaire. The results below highlight the current state of implemented policies and procedures. A checkmark (\ding{51}) indicates a positive control is in place, while an X (\ding{55}) indicates a control gap.

\begin{table}[h!]
\centering
\begin{tabular}{@{}p{0.8\linewidth}c@{}}
\toprule
\textbf{Control Question} & \textbf{Status} \\
\midrule
Do you require MFA to access email? & \ding{51} \\
Do you require MFA to log into computers? & \ding{51} \\
Do you require MFA to access sensitive data systems? & \ding{51} \\
Does your organization have an employee acceptable use policy? & \ding{51} \\
Does your organization do security awareness training for all employees at least once per year? & \ding{51} \\
\rowcolor{red!15} Does your organization do security awareness training for new employees? & \ding{55} \\
\bottomrule
\end{tabular}
\caption{Security Control Questionnaire Results}
\end{table}

\subsection*{Analysis of Control Gaps}
The review identified one significant control gap:
\begin{itemize}
    \item \textbf{Lack of Onboarding Security Training:} New employees are not receiving security awareness training upon joining the organization. This is a critical period where employees are unfamiliar with corporate policies and are more susceptible to social engineering or making configuration errors. This gap is classified as a \textbf{High} risk.
\end{itemize}

% --- Section 4: Technical Scan Results ---
\section*{Technical Scan Results}

An Nmap scan was performed on the internal network to identify open ports and exposed services.

\subsection*{Host: \texttt{10.5.5.5}}
The scan revealed the following open port and service information.

\begin{table}[h!]
\centering
\begin{tabular}{@{}llll@{}}
\toprule
\textbf{Port} & \textbf{State} & \textbf{Service} & \textbf{Details} \\
\midrule
\rowcolor{red!15} 8080/tcp & Open & http-proxy & HTTP Title: \textbf{TOP SECRET DB} \\
\bottomrule
\end{tabular}
\caption{Open Ports on Host \texttt{10.5.5.5}}
\end{table}

\subsection*{Analysis of Technical Findings}
The scan identified an extremely concerning service.
\begin{itemize}
    \item \textbf{Exposed Sensitive Database:} The service on port \texttt{8080} identifies itself as ``TOP SECRET DB''. This strongly implies that a sensitive, and possibly critical, database is exposed on the network without adequate access controls. This finding directly contradicts information from the current risk register (\textit{Input\_3\_Current\_Risks\_JSON}), which stated this port was a secure false positive. This finding is classified as a \textbf{Critical} risk.
\end{itemize}

% --- Section 5: Consolidated Risk Assessment ---
\section*{Consolidated Risk Assessment}

The following table summarizes and correlates the risks identified from the security control review, technical scan, and analysis of existing documentation.

\begin{table}[h!]
\centering
\begin{tabular}{@{}p{0.25\linewidth}p{0.15\linewidth}p{0.5\linewidth}@{}}
\toprule
\textbf{Risk Name} & \textbf{Severity} & \textbf{Overview} \\
\midrule
\rowcolor{red!25}
Exposed Sensitive Database & \textbf{Critical} & A service on internal host \texttt{10.5.5.5:8080} is titled "TOP SECRET DB", indicating a high-value asset is accessible. The previous risk assessment for this port was incorrect. \\
\addlinespace
\rowcolor{orange!25}
Inadequate Employee Onboarding Security & \textbf{High} & New employees do not receive security awareness training, increasing the likelihood of human error, policy violations, and insecure system configurations. \\
\addlinespace
\rowcolor{yellow!25}
Outdated Risk Register & \textbf{Medium} & The existing risk register incorrectly listed Port 8080 as a secured false positive. This indicates a flaw in the risk validation and management lifecycle, reducing trust in its accuracy. \\
\bottomrule
\end{tabular}
\caption{Summary of Identified Risks}
\end{table}

% --- Section 6: Recommendations ---
\section*{Recommendations}

Based on the correlated findings, the following actions are recommended, prioritized by severity.

\subsection*{Immediate Actions (Critical)}
\begin{enumerate}
    \item \textbf{Secure Exposed Database on \texttt{10.5.5.5:8080}:}
    \begin{itemize}
        \item Immediately investigate the service running on port \texttt{8080} to identify the data owner and system purpose.
        \item If the service is not essential, take it offline.
        \item If essential, implement strict firewall rules and require strong, MFA-based authentication for all access.
    \end{itemize}
\end{enumerate}

\subsection*{Short-Term Actions (High)}
\begin{enumerate}
    \item \textbf{Implement Onboarding Security Training:}
    \begin{itemize}
        \item Develop or procure a mandatory security awareness training module for all new hires.
        \item Integrate this training into the formal employee onboarding process, to be completed within the first week of employment.
    \end{itemize}
\end{enumerate}

\subsection*{Long-Term Actions (Medium)}
\begin{enumerate}
    \item \textbf{Improve Risk Management Process:}
    \begin{itemize}
        \item Conduct a full review of the current risk register to validate all existing entries against current technical data.
        \item Establish a formal process for periodic review and re-validation of all identified risks to ensure the register remains an accurate and reliable tool.
    \end{itemize}
\end{enumerate}

\end{document}
```