Here is the complete and professional LaTeX report, generated based on the provided data inputs and instructions.

```latex
\documentclass[12pt]{article}

% Preamble: Required Packages
\usepackage[margin=1in]{geometry}
\usepackage{pifont} % For checkmarks and crosses
\usepackage{booktabs} % For professional tables
\usepackage{hyperref} % For hyperlinks
\usepackage{url} % For URL formatting
\usepackage{seqsplit} % For splitting long strings in tt font
\usepackage{graphicx}
\usepackage{xcolor}

% Document Information
\title{Cybersecurity Posture Assessment Report}
\author{Cybersecurity Analysis Division}
\date{\today}

% Hyperref Setup
\hypersetup{
    colorlinks=true,
    linkcolor=blue,
    filecolor=magenta,      
    urlcolor=cyan,
    pdftitle={Cybersecurity Posture Assessment Report},
    pdfpagemode=FullScreen,
}

\begin{document}

\maketitle
\thispagestyle{empty}
\newpage

\tableofcontents
\newpage

% --- 1. Executive Overview ---
\section{Executive Overview}

This report details the findings of a cybersecurity posture assessment for \textbf{True North Travel}. The analysis is based on a review of organizational security controls provided via a questionnaire. 

It is critical to note that the provided technical network scan data (\texttt{Input\_1\_Network\_Scan\_JSON}) and the list of current organizational risks (\texttt{Input\_3\_Current\_Risks\_JSON}) were found to be corrupted and could not be processed. Consequently, this assessment focuses primarily on the administrative and policy-based controls self-reported by the organization.

The review identified several critical and high-risk security gaps. The most significant findings include:
\begin{itemize}
    \item \textbf{Lack of Multi-Factor Authentication (MFA):} MFA is not enforced for accessing email or for logging into company computers. This represents a critical vulnerability, as compromised credentials could lead directly to unauthorized access to sensitive communications and internal systems.
    \item \textbf{Absence of an Acceptable Use Policy (AUP):} The organization lacks a formal AUP, creating ambiguity regarding the proper use of company assets and data. This increases the risk of insider threats, whether malicious or accidental.
    \item \textbf{Incomplete Security Awareness Training:} While new employees receive training, there is no mandatory annual security training for all staff. This leaves the organization susceptible to evolving social engineering tactics, such as sophisticated phishing and business email compromise (BEC) attacks.
\end{itemize}

These deficiencies expose \textbf{True North Travel} to a heightened risk of data breaches, financial loss, and reputational damage. This report provides a detailed breakdown of these risks and offers prioritized, actionable recommendations for remediation.

% --- 2. Organizational Information ---
\section{Organizational Information}

The following details were provided for the assessment.

\begin{tabular}{@{}ll}
\toprule
\textbf{Attribute} & \textbf{Value} \\
\midrule
Organization Name & \textbf{True North Travel} \\
Email Domain & \texttt{TrueNorthTravel.net} \\
Website Domain & \url{www.TrueNorthTravel.net} \\
External IP Address & \texttt{114.57.210.0} \\
\bottomrule
\end{tabular}

% --- 3. Security Control Review ---
\section{Security Control Review (Questionnaire Analysis)}

An analysis of the security questionnaire reveals significant gaps in fundamental security controls. The table below summarizes the organization's self-reported responses against cybersecurity best practices. A green checkmark (\ding{51}) indicates an aligned practice, while a red cross (\ding{55}) indicates a security gap that requires immediate attention.

\begin{table}[h!]
\centering
\caption{Security Control Questionnaire Results}
\begin{tabular}{p{0.6\linewidth} c c}
\toprule
\textbf{Control Question} & \textbf{Response} & \textbf{Status} \\
\midrule
Do you require MFA to access email? & No & \textcolor{red}{\ding{55}} \\
Do you require MFA to log into computers? & No & \textcolor{red}{\ding{55}} \\
Do you require MFA to access sensitive data systems? & Yes & \textcolor{green}{\ding{51}} \\
Does your organization have an employee acceptable use policy? & No & \textcolor{red}{\ding{55}} \\
Does your organization do security awareness training for new employees? & Yes & \textcolor{green}{\ding{51}} \\
Does your organization do security awareness training for all employees at least once per year? & No & \textcolor{red}{\ding{55}} \\
\bottomrule
\end{tabular}
\end{table}

% --- 4. Technical Scan Results ---
\section{Technical Scan Results}

The technical network scan data provided in \texttt{Input\_1\_Network\_Scan\_JSON} was corrupted and could not be parsed for analysis. Therefore, a technical assessment of open ports, running services, and potential software vulnerabilities on the target IP address (\texttt{[Target IP]}) could not be performed.

It is strongly recommended that a new external network vulnerability scan be conducted as a high-priority action to identify and remediate any technical vulnerabilities that may be present on the organization's perimeter.

% --- 5. Risk Assessment ---
\section{Risk Assessment}

Similar to the technical scan data, the provided list of current organizational risks (\texttt{Input\_3\_Current\_Risks\_JSON}) was corrupted. The following risk assessment is therefore based exclusively on the gaps identified in the Security Control Review.

\begin{table}[h!]
\centering
\caption{Identified Risks from Questionnaire Analysis}
\begin{tabular}{p{0.1\linewidth} p{0.25\linewidth} p{0.4\linewidth} p{0.1\linewidth}}
\toprule
\textbf{Risk ID} & \textbf{Risk Name} & \textbf{Description} & \textbf{Severity} \\
\midrule
TNT-R-001 & Lack of MFA on Email and Endpoints & The absence of MFA on primary access vectors like email and computer logins drastically increases the likelihood of a successful account takeover via credential theft (e.g., phishing). & \textbf{Critical} \\
\addlinespace
TNT-R-002 & Missing Acceptable Use Policy (AUP) & Without a formal AUP, there are no enforceable rules for employee use of IT assets, increasing the risk of data exfiltration, malware introduction, and other insider threats. & \textbf{High} \\
\addlinespace
TNT-R-003 & Inadequate Security Awareness Training & Failing to provide recurring annual security training leaves employees unprepared for modern, evolving cyber threats, making them prime targets for social engineering. & \textbf{High} \\
\bottomrule
\end{tabular}
\end{table}

% --- 6. Recommendations ---
\section{Recommendations}

Based on the analysis, the following prioritized recommendations are provided to mitigate the identified risks and improve the overall security posture of \textbf{True North Travel}.

\begin{enumerate}
    \item \textbf{Priority 1 (Critical): Implement Comprehensive MFA.}
    \begin{itemize}
        \item \textbf{Action:} Deploy a robust Multi-Factor Authentication solution across the organization.
        \item \textbf{Details:} Prioritize the immediate enforcement of MFA for all user access to email systems (e.g., Office 365, Google Workspace) and for all remote and on-premise computer logins (e.g., via Windows Hello for Business, Duo, or a similar solution). This is the single most effective control to prevent unauthorized access.
    \end{itemize}
    \vspace{1em}

    \item \textbf{Priority 2 (High): Develop and Implement an Acceptable Use Policy (AUP).}
    \begin{itemize}
        \item \textbf{Action:} Draft, approve, and disseminate a formal AUP.
        \item \textbf{Details:} The policy should clearly define the rules for using company networks, devices, software, and data. All current and new employees must be required to read and formally acknowledge the policy as a condition of being granted system access.
    \end{itemize}
    \vspace{1em}
    
    \item \textbf{Priority 3 (High): Establish a Mandatory Annual Security Training Program.}
    \begin{itemize}
        \item \textbf{Action:} Institute a recurring, mandatory security awareness training program for all employees.
        \item \textbf{Details:} The training should be completed annually and cover current threats, including phishing, ransomware, social engineering, and the company's AUP. Track completion to ensure 100\% compliance.
    \end{itemize}
    \vspace{1em}

    \item \textbf{Priority 4 (Informational): Reschedule Technical Security Assessments.}
    \begin{itemize}
        \item \textbf{Action:} Commission a new external network vulnerability scan.
        \item \textbf{Details:} A new scan is necessary to gain a complete understanding of the organization's external attack surface and to identify technical vulnerabilities that require remediation. This will provide a more holistic view of the organization's risk profile.
    \end{itemize}
\end{enumerate}

\end{document}
```