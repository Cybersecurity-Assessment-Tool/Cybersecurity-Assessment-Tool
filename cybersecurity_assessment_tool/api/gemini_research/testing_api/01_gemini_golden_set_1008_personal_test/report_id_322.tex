```latex
\documentclass[12pt]{article}

% Preamble: Required Packages
\usepackage[margin=1in]{geometry}
\usepackage{pifont} % For checkmarks and crosses (\ding)
\usepackage{booktabs} % For professional tables
\usepackage{hyperref} % For hyperlinks
\usepackage{url} % For formatting URLs
\usepackage{seqsplit} % To split long strings in texttt
\usepackage[utf8]{inputenc}
\usepackage{graphicx}

% Document Information
\hypersetup{
    colorlinks=true,
    linkcolor=black,
    urlcolor=blue,
    pdftitle={Cybersecurity Posture Report},
    pdfauthor={Cybersecurity Analyst},
    pdfsubject={Security Assessment},
    pdfkeywords={Cybersecurity, Risk, Assessment}
}

\begin{document}

% --- Title Page ---
\begin{titlepage}
    \centering
    \vspace*{\stretch{1.0}}
    {\Huge\bfseries Cybersecurity Posture Report\par}
    \vspace{1.5cm}
    {\Large\bfseries Prepared for:\par}
    \vspace{0.5cm}
    {\Huge Clear Path\par}
    \vspace{2cm}
    {\large \today\par}
    \vspace*{\stretch{2.0}}
    \vfill
    \textit{This report contains sensitive information and is intended solely for the recipient.}
\end{titlepage}

\newpage
\tableofcontents
\newpage

% --- Section 1: Executive Summary ---
\section*{Executive Summary}

This report provides a comprehensive analysis of the cybersecurity posture for \textbf{Clear Path}, based on a review of organizational security controls, an external network vulnerability scan, and an assessment of known risks. The evaluation was conducted on \today.

The assessment reveals a mixed security posture. On a positive note, the external network perimeter appears well-hardened, as the vulnerability scan detected no open ports or exposed services. This indicates effective firewall configuration and a reduced external attack surface.

However, the review of internal security controls identified several critical and high-risk gaps. The most significant weaknesses are the absence of Multi-Factor Authentication (MFA) for sensitive data systems, the lack of a formal employee Acceptable Use Policy (AUP), and the complete absence of a security awareness training program. These deficiencies expose the organization to significant risks from credential theft, insider threats, and social engineering attacks like phishing.

Immediate remediation efforts should focus on implementing foundational security policies and technical controls to address these internal vulnerabilities. Key recommendations include mandating MFA across all sensitive applications, developing and enforcing a comprehensive AUP, and establishing a recurring security awareness training program for all employees.

% --- Section 2: Organizational Information ---
\section*{Organizational Information}

The following details were provided for the assessment.

\begin{tabular}{@{}ll}
    \toprule
    \textbf{Attribute} & \textbf{Value} \\
    \midrule
    Organization Name & Clear Path \\
    Email Domain & \texttt{ClearPath.org} \\
    Website Domain & \url{www.ClearPath.org} \\
    External IP Address & \texttt{236.143.227.45} \\
    \bottomrule
\end{tabular}

% --- Section 3: Security Control Review ---
\section*{Security Control Review}

A review of administrative and technical security controls was conducted via a standardized questionnaire. The responses highlight critical gaps in the organization's security policies and procedures. "No" answers indicate a deviation from security best practices and represent a potential risk.

\begin{tabular}{@{}p{0.8\linewidth}c}
    \toprule
    \textbf{Control Question} & \textbf{Response} \\
    \midrule
    Do you require MFA to access email? & \ding{51} \\
    Do you require MFA to log into computers? & \ding{51} \\
    Do you require MFA to access sensitive data systems? & \textcolor{red}{\ding{55}} \\
    Does your organization have an employee acceptable use policy? & \textcolor{red}{\ding{55}} \\
    Does your organization do security awareness training for new employees? & \textcolor{red}{\ding{55}} \\
    Does your organization do security awareness training for all employees at least once per year? & \textcolor{red}{\ding{55}} \\
    \bottomrule
\end{tabular}

% --- Section 4: Technical Scan Results ---
\section*{Technical Scan Results}

An external network vulnerability scan was performed against the organization's perimeter.
\begin{itemize}
    \item \textbf{Target IP Address:} \texttt{[Target IP]}
    \item \textbf{Scan Date:} \today
\end{itemize}

\subsection*{Summary of Findings}
The network scan completed successfully and did not identify any open TCP or UDP ports on the target host. This is a positive security finding, suggesting that a well-configured firewall is in place, effectively limiting the external attack surface and preventing unauthorized network access. No vulnerabilities were detected as no services were exposed.

% --- Section 5: Risk Assessment ---
\section*{Risk Assessment}

This section synthesizes findings from the security control review and technical scan to identify and prioritize risks to the organization. Although no pre-existing vulnerabilities were reported and the network scan was clean, significant risks were identified from the control gaps.

\begin{tabular}{@{}p{0.1\linewidth} p{0.25\linewidth} p{0.45\linewidth} p{0.1\linewidth}@{}}
    \toprule
    \textbf{Risk ID} & \textbf{Risk Name} & \textbf{Description} & \textbf{Severity} \\
    \midrule
    R-01 & No MFA on Sensitive Systems & The absence of MFA on systems containing sensitive data drastically increases the risk of a data breach resulting from compromised credentials. An attacker with a valid username and password can gain direct access. & Critical \\
    \addlinespace
    R-02 & Lack of Security Awareness Training & Without initial and ongoing training, employees are highly susceptible to phishing, social engineering, and other common attacks. This makes personnel the weakest link in the organization's defense. & High \\
    \addlinespace
    R-03 & No Employee Acceptable Use Policy (AUP) & The lack of a formal AUP creates ambiguity regarding the proper use of company assets and data. This exposes the organization to insider threats, misuse of resources, and potential legal or compliance issues. & High \\
    \bottomrule
\end{tabular}

% --- Section 6: Recommendations ---
\section*{Recommendations}

The following actions are recommended to mitigate the identified risks and improve the overall security posture of \textbf{Clear Path}.

\subsection*{Immediate Priorities (0-3 Months)}
\begin{itemize}
    \item \textbf{Remediates R-01: Implement MFA for Sensitive Systems} \\
    Mandate the use of MFA for all applications and systems that store, process, or transmit sensitive information (e.g., financial data, customer PII, intellectual property). This should be the highest priority remediation activity.

    \item \textbf{Remediates R-03: Develop and Enforce an Acceptable Use Policy} \\
    Draft a formal AUP that clearly outlines the rules for using company IT assets, data handling responsibilities, internet usage guidelines, and the consequences of non-compliance. Require all current and new employees to formally acknowledge this policy.

    \item \textbf{Remediates R-02: Establish a Security Awareness Training Program} \\
    Implement a mandatory security awareness training program for all new hires upon onboarding. Subsequently, conduct annual refresher training for all staff covering essential topics such as phishing identification, password hygiene, and data protection best practices.
\end{itemize}

\end{document}
```