```latex
\documentclass[12pt]{article}

% --- PACKAGES ---
\usepackage[margin=1in]{geometry}
\usepackage{pifont} % For checkmarks and crosses
\usepackage{booktabs} % For professional tables
\usepackage{hyperref} % For clickable links
\usepackage{url} % For URL formatting
\usepackage{seqsplit} % To split long strings like IPs

% --- DOCUMENT METADATA ---
\title{Cybersecurity Posture Assessment Report}
\author{Cybersecurity Analysis Division}
\date{\today}

% --- HYPERREF SETUP ---
\hypersetup{
    colorlinks=true,
    linkcolor=black,
    urlcolor=blue,
    pdftitle={Cybersecurity Posture Assessment Report},
    pdfauthor={Cybersecurity Analysis Division},
}

% --- DOCUMENT START ---
\begin{document}

\maketitle
\thispagestyle{empty}
\newpage

\tableofcontents
\newpage

% ==============================================================================
% SECTION 1: EXECUTIVE OVERVIEW
% ==============================================================================
\section{Executive Overview}

This report provides a comprehensive cybersecurity assessment for \textbf{Astraeus Aerospace}. The analysis is based on a correlation of self-reported organizational data, an external network scan, and a review of pre-existing risks.

The organization demonstrates a solid foundation in security awareness and endpoint protection, with established policies for acceptable use, security training for all employees, and mandatory Multi-Factor Authentication (MFA) for email and computer access.

However, two key areas of risk were identified that require immediate attention. A critical gap exists in the access control for sensitive data systems, which currently do not require MFA. This significantly increases the risk of unauthorized access to critical information. Additionally, a technical scan revealed an exposed Secure Shell (SSH) service on an IPv6 address, which presents a potential vector for external attack if not properly hardened and monitored.

This report details these findings and provides actionable recommendations to mitigate the identified risks and enhance the overall security posture of the organization.

% ==============================================================================
% SECTION 2: ORGANIZATIONAL INFORMATION
% ==============================================================================
\section{Organizational Information}

The following information was provided by the client and used as a baseline for this assessment.

\begin{tabular}{@{}ll}
\toprule
\textbf{Attribute} & \textbf{Value} \\
\midrule
Organization Name & Astraeus Aerospace \\
Email Domain & \texttt{AstraeusAerospace.org} \\
Website Domain & \url{www.AstraeusAerospace.org} \\
External IP (IPv4) & \texttt{4.91.136.119} \\
Scanned IP (IPv6) & \seqsplit{\texttt{2001:db8::1}} \\
\bottomrule
\end{tabular}

% ==============================================================================
% SECTION 3: SECURITY CONTROL REVIEW
% ==============================================================================
\section{Security Control Review}

The following table summarizes the organization's responses to a security controls questionnaire. The "Status" column indicates alignment with security best practices.

\begin{table}[h!]
\centering
\begin{tabular}{@{}p{0.8\linewidth}c@{}}
\toprule
\textbf{Security Control Question} & \textbf{Status} \\
\midrule
Do you require MFA to access email? & \ding{51} \\
Do you require MFA to log into computers? & \ding{51} \\
\textbf{Do you require MFA to access sensitive data systems?} & \textbf{\ding{55}} \\
Does your organization have an employee acceptable use policy? & \ding{51} \\
Does your organization do security awareness training for new employees? & \ding{51} \\
Does your organization do security awareness training for all employees at least once per year? & \ding{51} \\
\bottomrule
\end{tabular}
\caption{Security Controls Questionnaire Results (\ding{51}=Yes, \ding{55}=No)}
\end{table}

\subsection*{Analysis of Findings}
The review reveals a significant security gap. The lack of mandatory MFA for accessing sensitive data systems is a \textbf{High Risk}. Should an attacker compromise a user's credentials, they could gain direct access to the organization's most valuable data without needing a second authentication factor. This single point of failure undermines the security provided by MFA on other systems.

% ==============================================================================
% SECTION 4: TECHNICAL SCAN RESULTS
% ==============================================================================
\section{Technical Scan Results}

An external network scan was performed to identify exposed services and potential vulnerabilities.

\begin{itemize}
    \item \textbf{Target IP Address:} \seqsplit{\texttt{2001:db8::1}}
    \item \textbf{Scan Tool:} Nmap
    \item \textbf{Scan Date:} \today
\end{itemize}

The scan identified the following open port:

\begin{table}[h!]
\centering
\begin{tabular}{@{}llll@{}}
\toprule
\textbf{Port} & \textbf{State} & \textbf{Service (Inferred)} & \textbf{Notes} \\
\midrule
22/tcp & open & SSH & The service is exposed to the internet. \\
& & & Version information was not available from this scan. \\
\bottomrule
\end{tabular}
\caption{Open Ports Detected on \seqsplit{\texttt{2001:db8::1}}}
\end{table}

\subsection*{Analysis of Findings}
The presence of an open SSH port (22) indicates that a system is configured for remote administration. While often necessary, an internet-facing SSH service is a common target for brute-force attacks and exploitation of software vulnerabilities. Without detailed version information, it is not possible to determine if the running SSH server is vulnerable to known exploits. However, its exposure constitutes a \textbf{Medium Risk} that must be managed.

% ==============================================================================
% SECTION 5: RISK ASSESSMENT SUMMARY
% ==============================================================================
\section{Risk Assessment Summary}

This section correlates the findings from the security control review and the technical scan. No pre-existing vulnerabilities were reported.

\begin{table}[h!]
\centering
\begin{tabular}{@{}p{0.1\linewidth}p{0.25\linewidth}p{0.45\linewidth}p{0.1\linewidth}@{}}
\toprule
\textbf{Risk ID} & \textbf{Risk Name} & \textbf{Overview} & \textbf{Severity} \\
\midrule
\textbf{RISK-001} & Lack of MFA on Sensitive Systems & User credentials are the only barrier to accessing sensitive data. A single credential compromise could lead to a significant data breach. & \textbf{High} \\
\addlinespace
\textbf{RISK-002} & Exposed SSH Service & An internet-facing SSH port provides a direct vector for attackers to attempt unauthorized access via brute-force or exploitation of unpatched vulnerabilities. & Medium \\
\bottomrule
\end{tabular}
\caption{Summary of Identified Risks}
\end{table}

% ==============================================================================
% SECTION 6: RECOMMENDATIONS
% ==============================================================================
\section{Recommendations}

The following actions are recommended to mitigate the identified risks and strengthen the organization's security posture.

\subsection*{RISK-001: Lack of MFA on Sensitive Systems (High)}
\begin{itemize}
    \item \textbf{Immediate Action:} Prioritize and implement a mandatory MFA solution for all systems classified as containing sensitive data. This control should apply to all users, including administrators and service accounts where feasible.
    \item \textbf{Strategic Action:} Conduct a data classification review to ensure all systems storing, processing, or transmitting sensitive information are correctly identified and protected with appropriate access controls.
\end{itemize}

\subsection*{RISK-002: Exposed SSH Service (Medium)}
\begin{itemize}
    \item \textbf{Immediate Action:} Audit the SSH service running on \seqsplit{\texttt{2001:db8::1}}. Ensure it is updated to the latest stable version to patch any known vulnerabilities.
    \item \textbf{Hardening Actions:}
        \begin{enumerate}
            \item \textbf{Disable Password Authentication:} Configure the SSH server to only allow public key-based authentication. This is significantly more resistant to brute-force attacks.
            \item \textbf{Implement an Access Control List (ACL):} If remote access is only required from specific IP addresses (e.g., administrator workstations or a corporate VPN), configure firewall rules to deny all other connections to port 22.
            \item \textbf{Monitor Activity:} Implement logging and alerting for SSH login attempts to detect and respond to suspicious activity.
        \end{enumerate}
\end{itemize}

\end{document}
```