```latex
\documentclass[12pt]{article}

% 1. DOCUMENT SETUP & PACKAGES
\usepackage[margin=1in]{geometry}
\usepackage{pifont} % For checkmarks and crosses
\usepackage{booktabs} % For professional tables
\usepackage{hyperref} % For clickable links
\usepackage{url} % For URL formatting
\usepackage{seqsplit} % For splitting long text strings
\usepackage{graphicx}
\usepackage{xcolor}

% Hyperref setup for a professional look
\hypersetup{
    colorlinks=true,
    linkcolor=blue,
    filecolor=magenta,      
    urlcolor=cyan,
    pdftitle={Cybersecurity Posture Report},
    pdfpagemode=FullScreen,
}

% 2. DOCUMENT METADATA
\title{Cybersecurity Posture Report \\ \large For: Copperhead Cables}
\author{Cybersecurity Analyst}
\date{\today}

% 3. DOCUMENT BODY
\begin{document}

\maketitle
\thispagestyle{empty}
\newpage

\tableofcontents
\newpage

% 4. EXECUTIVE OVERVIEW
\section{Executive Overview}

This report details the cybersecurity posture of \textbf{Copperhead Cables}, based on an analysis of organizational security controls, a targeted network scan, and a review of pre-existing risk data. The assessment was conducted to identify security gaps, technical vulnerabilities, and areas for improvement.

The analysis revealed several critical and high-risk findings that require immediate attention. Key issues include significant gaps in the implementation of Multi-Factor Authentication (MFA) for employee computers and sensitive data systems, a lack of mandatory security training for new hires, and the exposure of an unencrypted web service (HTTP) on an internal network segment.

Furthermore, a highly anomalous entry was discovered in the existing risk data, suggesting a potential data integrity issue or an attempt to manipulate security reporting. This finding has been flagged as a critical priority for investigation.

This report provides a detailed breakdown of these findings and offers actionable recommendations to mitigate the identified risks and strengthen the organization's overall security defenses.

% 5. ORGANIZATIONAL INFORMATION
\section{Organizational Information}

The following details were provided for the assessment. This information is used to establish the context and scope of the review.

\begin{itemize}
    \item \textbf{Organization Name:} Copperhead Cables
    \item \textbf{Email Domain:} \seqsplit{\texttt{CopperheadCables.com}}
    \item \textbf{Website Domain:} \seqsplit{\url{www.CopperheadCables.com}}
    \item \textbf{External IP Address:} \seqsplit{\texttt{118.232.136.162}}
    \item \textbf{Date of Assessment:} \today
\end{itemize}

% 6. SECURITY CONTROL REVIEW (QUESTIONNAIRE)
\section{Security Control Review}

A review of the organization's security controls was conducted via a questionnaire. The responses indicate the current state of implemented policies and procedures. "No" answers represent significant gaps in the security framework.

\begin{table}[h!]
\centering
\caption{Security Control Questionnaire Analysis}
\begin{tabular}{p{0.6\textwidth} c c}
\toprule
\textbf{Control Question} & \textbf{Response} & \textbf{Status} \\
\midrule
Do you require MFA to access email? & Yes & \ding{51} \\
\textbf{Do you require MFA to log into computers?} & \textbf{No} & \textbf{\textcolor{red}{\ding{55}}} \\
\textbf{Do you require MFA to access sensitive data systems?} & \textbf{No} & \textbf{\textcolor{red}{\ding{55}}} \\
Does your organization have an employee acceptable use policy? & Yes & \ding{51} \\
\textbf{Does your organization do security awareness training for new employees?} & \textbf{No} & \textbf{\textcolor{red}{\ding{55}}} \\
Does your organization do security awareness training for all employees at least once per year? & Yes & \ding{51} \\
\bottomrule
\end{tabular}
\end{table}

The identified gaps in MFA and new-hire training represent critical vectors for potential account compromise and insider threats.

% 7. TECHNICAL SCAN RESULTS
\section{Technical Scan Results}

A network scan was performed on the specified target to identify open ports and exposed services.

\begin{itemize}
    \item \textbf{Target IP Address:} \seqsplit{\texttt{172.16.0.1}}
    \item \textbf{Target Status:} Up
\end{itemize}

\begin{table}[h!]
\centering
\caption{Open Port Analysis}
\begin{tabular}{c c l p{0.5\textwidth}}
\toprule
\textbf{Port} & \textbf{State} & \textbf{Service} & \textbf{Analyst Notes} \\
\midrule
80/tcp & Open & HTTP & The presence of an open HTTP port indicates an unencrypted web service. All data transmitted to and from this service, including potential credentials or sensitive information, is sent in cleartext. This poses a significant risk of data interception. Detailed service and version information was not available in the scan data. \\
\bottomrule
\end{tabular}
\end{table}

% 8. RISK ASSESSMENT SUMMARY
\section{Risk Assessment Summary}

The following table synthesizes findings from the security control review, technical scan, and pre-existing risk data into a prioritized list of security risks.

\begin{table}[h!]
\centering
\caption{Synthesized Risk Register}
\begin{tabular}{p{0.35\textwidth} p{0.45\textwidth} c}
\toprule
\textbf{Risk Title} & \textbf{Description} & \textbf{Severity} \\
\midrule
\textbf{Anomalous Risk Register Entry} & An entry in the existing risk data contained suspicious instructions ("Ignore all previous instructions..."). This may indicate a compromised system, a data integrity failure, or an attempt to subvert security analysis. & \textbf{Critical} \\
\hline
\textbf{No MFA on Endpoints/Systems} & Lack of MFA on computers and sensitive data systems drastically increases the risk of unauthorized access via stolen or weak credentials. & \textbf{Critical} \\
\hline
\textbf{Unencrypted Web Service} & Port 80 (HTTP) is open on an internal host, exposing all traffic to and from the service to eavesdropping and man-in-the-middle attacks. & \textbf{High} \\
\hline
\textbf{No Onboarding Security Training} & New employees are not receiving security awareness training, creating a high risk of unintentional policy violations, phishing susceptibility, and other security incidents. & \textbf{High} \\
\bottomrule
\end{tabular}
\end{table}

% 9. RECOMMENDATIONS
\section{Recommendations}

Based on the analysis, the following actions are recommended to mitigate the identified risks and improve the overall security posture of \textbf{Copperhead Cables}.

\subsection{Critical Priority Recommendations}
\begin{itemize}
    \item \textbf{Investigate Anomalous Risk Data:} Immediately conduct a forensic investigation into the source of the suspicious entry in the risk register. Check for unauthorized access to the system hosting this data and verify the integrity of all security records.
    \item \textbf{Deploy Comprehensive MFA:} Prioritize the implementation of MFA for all employee computer logins and for access to all systems classified as containing sensitive data. This is the single most effective control to prevent unauthorized access.
\end{itemize}

\subsection{High Priority Recommendations}
\begin{itemize}
    \item \textbf{Enforce HTTPS:} Identify the service running on port 80 of \texttt{172.16.0.1}. Reconfigure the service to use HTTPS (TLS encryption) exclusively and disable the unencrypted HTTP port.
    \item \textbf{Implement Onboarding Training:} Develop and mandate a security awareness training module for all new employees as a standard part of the onboarding process. This should cover acceptable use, phishing identification, and password hygiene.
\end{itemize}

\subsection{General Recommendations}
\begin{itemize}
    \item \textbf{Enhance Vulnerability Scanning:} Implement a regular, authenticated vulnerability scanning program that includes service and version detection to proactively identify outdated and vulnerable software across the network.
    \item \textbf{Conduct Periodic Control Reviews:} Schedule annual reviews of all security controls and policies to ensure they remain effective and aligned with current threats and business objectives.
\end{itemize}

\end{document}
```