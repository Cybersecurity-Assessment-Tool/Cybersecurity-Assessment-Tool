```latex
\documentclass[12pt]{article}

% ----------------------------------------------------------------------
% PREAMBLE
% ----------------------------------------------------------------------
\usepackage[margin=1in]{geometry}
\usepackage{pifont} % For check and cross marks
\usepackage{booktabs} % For professional tables
\usepackage{hyperref} % For clickable links
\usepackage{url} % For URL formatting
\usepackage{seqsplit} % To split long strings without breaking
\usepackage{graphicx}
\usepackage{xcolor}

% Define custom colors
\definecolor{darkblue}{rgb}{0.0, 0.0, 0.55}
\definecolor{darkred}{rgb}{0.55, 0.0, 0.0}

% Hyperref setup
\hypersetup{
    colorlinks=true,
    linkcolor=darkblue,
    filecolor=magenta,      
    urlcolor=darkblue,
    citecolor=darkblue,
}

% Define check and cross marks for convenience
\newcommand{\cmark}{\ding{51}}
\newcommand{\xmark}{\ding{55}}

% ----------------------------------------------------------------------
% DOCUMENT START
% ----------------------------------------------------------------------
\begin{document}

% ----------------------------------------------------------------------
% TITLE PAGE
% ----------------------------------------------------------------------
\begin{titlepage}
    \centering
    \vspace*{1cm}
    \Huge\textbf{Cybersecurity Posture Assessment Report}
    \vspace{1.5cm}
    \Large
    \textbf{Prepared for:} \\
    \vspace{0.5cm}
    \textbf{Iron Bridge Legal}
    \vspace{2.5cm}
    \large
    \textbf{Date of Report:} \today \\
    \vspace{0.5cm}
    \textbf{Author:} Cybersecurity Analyst
    \vfill
    \textit{This report contains sensitive information and should be handled with care. Distribution is restricted to authorized personnel only.}
\end{titlepage}

\tableofcontents
\newpage

% ----------------------------------------------------------------------
% 1. EXECUTIVE SUMMARY
% ----------------------------------------------------------------------
\section{Executive Summary}

This report provides a cybersecurity assessment for \textbf{Iron Bridge Legal}, synthesizing information from a network vulnerability scan, a security controls questionnaire, and a review of pre-existing risks.

The assessment reveals a mixed security posture. On a technical level, the scanned network host (\texttt{192.168.1.100}) demonstrates a strong defensive configuration, with no open ports detected. This indicates effective firewalling and a minimized attack surface for that specific asset.

However, significant and critical gaps were identified in the organization's procedural and policy-based controls. The most critical finding is the lack of Multi-Factor Authentication (MFA) for email access, which exposes the organization to a high risk of Business Email Compromise (BEC) and unauthorized account access. Additional high-risk gaps include the absence of an employee Acceptable Use Policy and the failure to conduct annual security awareness training for all staff.

Immediate remediation should focus on implementing MFA for email, developing foundational security policies, and establishing a recurring employee training program to mitigate these human-centric risks.

% ----------------------------------------------------------------------
% 2. ORGANIZATIONAL INFORMATION
% ----------------------------------------------------------------------
\section{Organizational Information}

The following details were provided for the assessment.

\begin{tabular}{@{}ll}
\toprule
\textbf{Attribute} & \textbf{Value} \\
\midrule
Organization Name & \textbf{Iron Bridge Legal} \\
Email Domain & \texttt{IronBridgeLegal.net} \\
Website Domain & \seqsplit{\texttt{www.IronBridgeLegal.net}} \\
External IP Address & \texttt{123.0.198.98} \\
\bottomrule
\end{tabular}

% ----------------------------------------------------------------------
% 3. SECURITY CONTROL REVIEW (QUESTIONNAIRE)
% ----------------------------------------------------------------------
\section{Security Control Review}

The following table summarizes the responses to the security controls questionnaire. "No" answers indicate potential gaps in the security framework and are highlighted for review.

\begin{table}[h!]
\centering
\begin{tabular}{@{}p{8cm} c l@{}}
\toprule
\textbf{Control Question} & \textbf{Response} & \textbf{Assessment} \\
\midrule
Do you require MFA to access email? & \xmark & \textcolor{darkred}{\textbf{Critical Gap}} \\
Do you require MFA to log into computers? & \cmark & Good Practice \\
Do you require MFA to access sensitive data systems? & \cmark & Good Practice \\
Does your organization have an employee acceptable use policy? & \xmark & \textcolor{darkred}{\textbf{High Risk}} \\
Does your organization do security awareness training for new employees? & \cmark & Good Practice \\
Does your organization do security awareness training for all employees at least once per year? & \xmark & \textcolor{darkred}{\textbf{High Risk}} \\
\bottomrule
\end{tabular}
\caption{Security Controls Questionnaire Analysis}
\end{table}

The analysis of the questionnaire highlights three major areas of concern:
\begin{itemize}
    \item \textbf{Email Security:} The absence of MFA on email is the most severe finding. Email is a primary target for attackers seeking to conduct phishing, social engineering, and Business Email Compromise (BEC).
    \item \textbf{Security Policy:} Lacking an Acceptable Use Policy means there are no formal, enforceable rules for how employees should use company technology and data, leading to inconsistent and potentially risky behavior.
    \item \textbf{Employee Training:} While new hires receive training, the lack of an annual refresher for all employees means that knowledge of evolving cyber threats becomes stale, increasing susceptibility to attacks like phishing.
\end{itemize}

% ----------------------------------------------------------------------
% 4. TECHNICAL SCAN RESULTS
% ----------------------------------------------------------------------
\section{Technical Scan Results}

A network scan was performed to identify open ports and services on the specified target.

\begin{itemize}
    \item \textbf{Target IP Address:} \texttt{192.168.1.100}
    \item \textbf{Scan Date:} Scan data provided on \today
    \item \textbf{Host Status:} Up
\end{itemize}

\subsection{Scan Findings}
The scan concluded that the host was online and responsive. However, \textbf{no open ports were discovered}. All 1000 scanned TCP ports were reported as being in a `closed` state.

\subsection{Analysis}
This is a positive security finding. A host with no open ports presents a minimal attack surface to the network, making it significantly more difficult for an attacker to compromise. This result suggests a properly configured host-based or network firewall is in place for this specific asset, denying all unsolicited inbound connections.

% ----------------------------------------------------------------------
% 5. CONSOLIDATED RISK ASSESSMENT
% ----------------------------------------------------------------------
\section{Consolidated Risk Assessment}

The following table consolidates risks identified from the security control review. No pre-existing or technical vulnerabilities were identified in the provided data.

\begin{table}[h!]
\centering
\begin{tabular}{@{}p{1.5cm} p{4cm} p{2cm} p{6.5cm}@{}}
\toprule
\textbf{Risk ID} & \textbf{Risk Name} & \textbf{Severity} & \textbf{Description} \\
\midrule
RISK-001 & Lack of MFA on Email & \textbf{Critical} & The absence of MFA on email accounts allows an attacker with valid credentials (e.g., from a phishing attack or password spray) to gain full access, leading to data breaches and BEC. \\
\addlinespace
RISK-002 & No Employee Acceptable Use Policy (AUP) & \textbf{High} & Without a formal AUP, there is no clear guidance for employees on safe computing practices. This increases the risk of inadvertent data exposure, malware infections, and misuse of company assets. \\
\addlinespace
RISK-003 & Insufficient Security Awareness Training & \textbf{High} & One-time training for new hires is not sufficient. Without annual refreshers, employees are less likely to recognize and appropriately respond to new and evolving threats like sophisticated phishing campaigns. \\
\bottomrule
\end{tabular}
\caption{Identified Risks}
\end{table}

% ----------------------------------------------------------------------
% 6. RECOMMENDATIONS
% ----------------------------------------------------------------------
\section{Recommendations}

The following actions are recommended to address the identified risks. Recommendations are prioritized based on severity.

\subsection{RISK-001: Lack of MFA on Email (Critical)}
\begin{itemize}
    \item \textbf{Action:} Immediately procure and implement a Multi-Factor Authentication solution for the \texttt{IronBridgeLegal.net} email domain.
    \item \textbf{Details:} Enforce this policy for all users without exception, including executive and administrative accounts. This is the single most effective control to prevent unauthorized access to email.
    \item \textbf{Priority:} \textbf{Immediate}
\end{itemize}

\subsection{RISK-002: No Employee Acceptable Use Policy (High)}
\begin{itemize}
    \item \textbf{Action:} Develop, approve, and implement a formal Acceptable Use Policy (AUP).
    \item \textbf{Details:} The policy should cover topics such as data handling, internet usage, use of personal devices, and password requirements. All employees must be required to read and formally acknowledge the policy upon its release and during onboarding.
    \item \textbf{Priority:} High
\end{itemize}

\subsection{RISK-003: Insufficient Security Awareness Training (High)}
\begin{itemize}
    \item \textbf{Action:} Establish a mandatory, annual security awareness training program for all employees.
    \item \textbf{Details:} The training curriculum should be updated yearly to reflect the current threat landscape. Consider including periodic phishing simulations to test and reinforce the training. Maintain records of completion for all staff.
    \item \textbf{Priority:} High
\end{itemize}

% ----------------------------------------------------------------------
% DOCUMENT END
% ----------------------------------------------------------------------
\end{document}
```