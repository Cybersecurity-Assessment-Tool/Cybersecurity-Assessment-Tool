```latex
\documentclass[12pt]{article}

% Preamble: Required Packages
\usepackage[margin=1in]{geometry}
\usepackage{pifont} % For checkmarks and crosses
\usepackage{booktabs} % For professional tables
\usepackage{hyperref} % For clickable links
\usepackage{url} % For URL formatting
\usepackage{seqsplit} % For splitting long strings in tt font
\usepackage[utf8]{inputenc} % For UTF-8 input
\usepackage{graphicx}
\usepackage{xcolor}

% Document Metadata
\title{Cybersecurity Assessment Report \\ \large Prepared for: Iron Oak Furniture}
\author{Cybersecurity Analysis Division}
\date{\today}

\begin{document}

\maketitle
\tableofcontents
\newpage

% --- 1. Executive Summary ---
\section{Executive Summary}

This report details the findings of a cybersecurity assessment conducted for Iron Oak Furniture. The assessment combined a review of organizational security controls, an external network scan, and an analysis of pre-existing risks.

The overall security posture presents a mixed landscape. On a positive note, the technical network scan of the target host \texttt{192.168.1.100} revealed no open ports, suggesting a strong firewall configuration and a minimal attack surface for that specific asset. This is a commendable security practice.

However, significant and high-impact risks were identified in the organization's security policies and procedures. The most critical findings stem from the lack of Multi-Factor Authentication (MFA) for accessing sensitive data systems and for computer logins. Furthermore, the absence of mandatory annual security awareness training for all staff leaves the organization vulnerable to evolving social engineering tactics like phishing.

These gaps substantially increase the risk of unauthorized access, credential compromise, and data breaches. This report outlines these risks in detail and provides prioritized, actionable recommendations to mitigate them and strengthen the organization's overall defensive capabilities.

% --- 2. Organizational Information ---
\section{Organizational Information}

The following details were provided for the assessment.

\begin{itemize}
    \item \textbf{Organization Name:} Iron Oak Furniture
    \item \textbf{Email Domain:} \texttt{IronOakFurniture.net}
    \item \textbf{Monitored External IP:} \texttt{124.180.182.174}
\end{itemize}

% --- 3. Security Control Review ---
\section{Security Control Review}

A review of internal security controls was conducted via a standardized questionnaire. The responses indicate foundational policies are in place, but critical gaps exist in identity and access management and ongoing employee training. The table below summarizes the findings, with a \textcolor{red}{\ding{55}} indicating a potential security gap.

\begin{table}[h!]
\centering
\caption{Organizational Security Control Status}
\label{tab:controls}
\begin{tabular}{p{0.7\linewidth} c}
\toprule
\textbf{Control Question} & \textbf{Status} \\
\midrule
Do you require MFA to access email? & \textcolor{green!70!black}{\ding{51}} \\
Do you require MFA to log into computers? & \textcolor{red}{\ding{55}} \\
Do you require MFA to access sensitive data systems? & \textcolor{red}{\ding{55}} \\
Does your organization have an employee acceptable use policy? & \textcolor{green!70!black}{\ding{51}} \\
Does your organization do security awareness training for new employees? & \textcolor{green!70!black}{\ding{51}} \\
Does your organization do security awareness training for all employees at least once per year? & \textcolor{red}{\ding{55}} \\
\bottomrule
\end{tabular}
\end{table}

% --- 4. Technical Scan Results ---
\section{Technical Scan Results}

A network scan was performed to identify exposed services and potential vulnerabilities on the perimeter.

\begin{itemize}
    \item \textbf{Scan Target:} \texttt{192.168.1.100}
    \item \textbf{Scan Date:} \today
\end{itemize}

\subsection{Summary of Findings}
The host at \texttt{192.168.1.100} was found to be online and responsive. The scan confirmed that all 1000 scanned TCP ports were in a \textbf{closed} state.

\textbf{Conclusion:} No open ports or exposed services were detected on the target host. This is a positive security finding, indicating that the device is likely protected by a well-configured firewall that denies unsolicited incoming traffic, significantly reducing its external attack surface.

% --- 5. Risk Assessment ---
\section{Risk Assessment}

This section correlates the findings from the security control review and technical scans. While no technical vulnerabilities were discovered, critical gaps in administrative controls present a significant risk to the organization. The following table summarizes the identified risks, prioritized by severity.

\begin{table}[h!]
\centering
\caption{Summary of Identified Risks}
\label{tab:risks}
\begin{tabular}{p{0.2\linewidth} p{0.5\linewidth} l}
\toprule
\textbf{Risk ID} & \textbf{Finding} & \textbf{Severity} \\
\midrule
RISK-001 & \textbf{Lack of MFA for Sensitive Data Systems:} Failure to protect systems containing sensitive corporate or customer data with MFA. & \textbf{Critical} \\
\addlinespace
RISK-002 & \textbf{Lack of MFA for Computer Logins:} User endpoints (desktops/laptops) are accessible with only a password, increasing risk from compromised credentials. & \textbf{High} \\
\addlinespace
RISK-003 & \textbf{Lack of Annual Security Training:} Employees do not receive regular, updated security awareness training, increasing susceptibility to phishing and social engineering. & \textbf{High} \\
\bottomrule
\end{tabular}
\end{table}

% --- 6. Recommendations ---
\section{Recommendations}

Based on the risk assessment, the following actions are recommended to enhance the security posture of Iron Oak Furniture. Recommendations are prioritized to address the most critical risks first.

\subsection{Immediate Priority (Critical Risks)}

\begin{itemize}
    \item \textbf{Implement MFA for All Sensitive Systems (RISK-001):}
    \begin{itemize}
        \item \textbf{Action:} Immediately deploy a mandatory Multi-Factor Authentication (MFA) solution for all user accounts (including administrative and service accounts) that can access sensitive data repositories, financial systems, or customer databases.
        \item \textbf{Justification:} This is the single most effective control to prevent unauthorized access to critical data in the event of password compromise. It serves as a crucial compensating control for other potential weaknesses.
    \end{itemize}
\end{itemize}

\subsection{High Priority}

\begin{itemize}
    \item \textbf{Deploy MFA for Endpoint Logon (RISK-002):}
    \begin{itemize}
        \item \textbf{Action:} Enforce MFA for all users logging into their company-issued computers. Solutions like Windows Hello for Business, Duo, or other third-party integrations can be used.
        \item \textbf{Justification:} Protecting endpoints with MFA prevents an attacker with stolen credentials from gaining an initial foothold on the internal network, thereby disrupting attack chains and preventing lateral movement.
    \end{itemize}
    \vspace{1em} % Add some space
    \item \textbf{Establish a Mandatory Annual Security Training Program (RISK-003):}
    \begin{itemize}
        \item \textbf{Action:} Develop and implement a formal security awareness training program that is mandatory for all employees on an annual basis. The training should cover current threats such as phishing, ransomware, and proper data handling.
        \item \textbf{Justification:} The threat landscape is constantly changing. Annual training ensures that the "human firewall" is kept up-to-date and resilient against modern social engineering attacks, reducing the likelihood of a breach caused by human error.
    \end{itemize}
\end{itemize}

\end{document}
```