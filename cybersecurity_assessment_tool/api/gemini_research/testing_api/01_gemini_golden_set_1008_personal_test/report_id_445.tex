```latex
\documentclass[12pt]{article}

% Preamble: Required Packages
\usepackage[margin=1in]{geometry}
\usepackage{pifont} % For checkmarks and crosses
\usepackage{booktabs} % For professional tables
\usepackage{hyperref} % For clickable links
\usepackage{url} % For URL formatting
\usepackage{seqsplit} % For splitting long strings in texttt
\usepackage[T1]{fontenc}

% Document Metadata
\title{Cybersecurity Assessment Report}
\author{Cybersecurity Analysis Division}
\date{November 22, 2025}

\begin{document}

\maketitle
\thispagestyle{empty}
\newpage

\tableofcontents
\newpage

% --- 1. Executive Summary ---
\section{Executive Summary}

This report details the findings of a cybersecurity assessment conducted for \textbf{Blue Horizon Initiative}. The assessment combined a review of organizational security controls, an external network scan, and an analysis of pre-existing risks.

The overall security posture presents several areas of significant concern that require immediate attention. Key findings include critical gaps in access control, outdated internet-facing software, and deficiencies in security governance policies.

While the organization has implemented Multi-Factor Authentication (MFA) for email access, its absence on computer logins and sensitive data systems constitutes a critical vulnerability. Furthermore, a public-facing web server was found to be running an outdated and potentially vulnerable version of Nginx. These technical vulnerabilities, compounded by the lack of an acceptable use policy and mandatory annual security training, create a high-risk environment susceptible to credential compromise, unauthorized access, and lateral movement within the network.

This report provides a detailed breakdown of these risks and offers actionable recommendations to mitigate them and strengthen the organization's overall defensive capabilities.

% --- 2. Organizational Information ---
\section{Organizational Information}

The following details were provided for the assessment.

\begin{itemize}
    \item \textbf{Organization Name:} Blue Horizon Initiative
    \item \textbf{Email Domain:} \texttt{BlueHorizonInitiative.net}
    \item \textbf{Website Domain:} \texttt{www.BlueHorizonInitiative.net}
    \item \textbf{External IP Address:} \texttt{194.212.228.36}
\end{itemize}

% --- 3. Security Control Review ---
\section{Security Control Review}

A review of administrative and organizational security controls was conducted via a questionnaire. The responses reveal critical gaps in security governance and access control policies. The results are summarized in Table \ref{tab:controls}.

\begin{table}[h!]
\centering
\caption{Organizational Security Control Questionnaire}
\label{tab:controls}
\begin{tabular}{@{}p{0.6\linewidth} c p{0.2\linewidth}@{}}
\toprule
\textbf{Control Question} & \textbf{Response} & \textbf{Assessment} \\
\midrule
Do you require MFA to access email? & \ding{51} Yes & Good Practice \\
Do you require MFA to log into computers? & \ding{55} No & \textbf{High Risk} \\
Do you require MFA to access sensitive data systems? & \ding{55} No & \textbf{Critical Risk} \\
Does your organization have an employee acceptable use policy? & \ding{55} No & High Risk \\
Does your organization do security awareness training for new employees? & \ding{51} Yes & Good Practice \\
Does your organization do security awareness training for all employees at least once per year? & \ding{55} No & High Risk \\
\bottomrule
\end{tabular}
\end{table}

% --- 4. Technical Scan Results ---
\section{Technical Scan Results}

A network scan was performed on \textbf{November 22, 2025}, targeting the host at \texttt{192.168.10.5}. The scan identified one open port with a service running an outdated software version.

\subsection{Open Ports}
The following ports were found to be open on the target system.

\begin{table}[h!]
\centering
\caption{Open Port Scan Results for \texttt{192.168.10.5}}
\label{tab:ports}
\begin{tabular}{@{}llll@{}}
\toprule
\textbf{Port} & \textbf{State} & \textbf{Service} & \textbf{Product \& Version} \\
\midrule
443/tcp & open & https & Nginx 1.18.0 \\
\bottomrule
\end{tabular}
\end{table}

\subsection{Technical Analysis}
The scan identified an Nginx web server, version \textbf{1.18.0}, accessible on port 443. This version was released in April 2020 and is now significantly outdated. It is known to be affected by multiple security vulnerabilities, including but not limited to issues like request smuggling (CVE-2021-23017). Running outdated software on an internet-facing system presents a high risk of compromise, as publicly known exploits can be used by attackers to gain initial access to the network.

% --- 5. Risk Assessment Summary ---
\section{Risk Assessment Summary}

The following table synthesizes findings from the security control review and the technical scan. No pre-existing vulnerabilities were documented.

\begin{table}[h!]
\centering
\caption{Identified Risks and Severity}
\label{tab:risks}
\begin{tabular}{@{}p{0.1\linewidth} p{0.45\linewidth} p{0.2\linewidth} p{0.1\linewidth}@{}}
\toprule
\textbf{Risk ID} & \textbf{Description} & \textbf{Source} & \textbf{Severity} \\
\midrule
RISK-001 & Lack of MFA on sensitive data systems exposes critical assets to unauthorized access. & Questionnaire & \textbf{Critical} \\
\addlinespace
RISK-002 & Outdated Nginx web server (v1.18.0) with known vulnerabilities is exposed on the network. & Technical Scan & High \\
\addlinespace
RISK-003 & Lack of MFA for computer logins increases the risk of lateral movement after a credential compromise. & Questionnaire & High \\
\addlinespace
RISK-004 & Absence of mandatory annual security awareness training for all staff increases susceptibility to phishing and social engineering. & Questionnaire & High \\
\addlinespace
RISK-005 & No formal Acceptable Use Policy, leading to a lack of accountability and unclear security expectations for employees. & Questionnaire & Medium \\
\bottomrule
\end{tabular}
\end{table}

% --- 6. Recommendations ---
\section{Recommendations}

To address the identified risks and improve the overall security posture of \textbf{Blue Horizon Initiative}, we strongly recommend implementing the following corrective actions. Recommendations are prioritized based on risk severity.

\begin{enumerate}
    \item \textbf{Implement MFA for All Critical Systems (RISK-001 \& RISK-003):}
    \begin{itemize}
        \item \textbf{Immediate Priority:} Enforce mandatory Multi-Factor Authentication (MFA) for all user access to systems containing sensitive data.
        \item \textbf{High Priority:} Deploy MFA for all employee computer and remote access logins (e.g., VPN, RDP). This significantly reduces the impact of stolen credentials.
    \end{itemize}
    
    \item \textbf{Upgrade Outdated Web Server Software (RISK-002):}
    \begin{itemize}
        \item \textbf{Immediate Priority:} Upgrade the Nginx server running on \texttt{192.168.10.5} from version 1.18.0 to the latest stable version.
        \item Establish a formal patch management process to ensure all internet-facing systems are updated in a timely manner.
    \end{itemize}
    
    \item \textbf{Establish a Comprehensive Security Training Program (RISK-004):}
    \begin{itemize}
        \item Develop and implement a mandatory annual security awareness training program for all employees.
        \item Training should cover current threats such as phishing, ransomware, and proper data handling procedures.
    \end{itemize}
    
    \item \textbf{Develop and Enforce an Acceptable Use Policy (RISK-005):}
    \begin{itemize}
        \item Create a formal Acceptable Use Policy (AUP) that clearly defines the rules and responsibilities for employees when using company IT assets.
        \item Ensure all employees read and acknowledge the policy upon hire and during annual reviews.
    \end{itemize}
\end{enumerate}

\end{document}
```