```latex
\documentclass[12pt]{article}

% Preamble: Required Packages
\usepackage[margin=1in]{geometry}
\usepackage{pifont} % For checkmarks and crosses
\usepackage{booktabs} % For professional tables
\usepackage{hyperref} % For clickable links
\usepackage{url} % For formatting URLs
\usepackage{seqsplit} % For splitting long strings to prevent overflow
\usepackage{graphicx}
\usepackage{xcolor}
\usepackage{fancyhdr}

% Document Styling
\hypersetup{
    colorlinks=true,
    linkcolor=blue,
    filecolor=magenta,      
    urlcolor=cyan,
    pdftitle={Cybersecurity Posture Assessment Report},
    pdfpagemode=FullScreen,
}

\pagestyle{fancy}
\fancyhf{}
\fancyhead[L]{Cybersecurity Posture Assessment}
\fancyhead[R]{\textbf{Harbor Light Foundation}}
\fancyfoot[C]{\thepage}

% --- DOCUMENT START ---
\begin{document}

\title{
    \vspace{2cm}
    \textbf{Cybersecurity Posture Assessment Report} \\
    \large For: \textbf{Harbor Light Foundation}
    \vspace{1cm}
}

\author{Cybersecurity Analysis Division}
\date{\today}

\maketitle
\thispagestyle{empty}
\newpage

\tableofcontents
\newpage

% --- SECTION 1: EXECUTIVE OVERVIEW ---
\section{Executive Overview}
This report provides a cybersecurity posture assessment for \textbf{Harbor Light Foundation}, based on a review of organizational security controls, an external network scan, and an analysis of known risks.

The assessment reveals several critical and high-risk security gaps that require immediate attention. The most significant weaknesses are procedural and policy-based. Specifically, the lack of Multi-Factor Authentication (MFA) on email and sensitive data systems exposes the organization to a high risk of account compromise, data breach, and ransomware attacks. Furthermore, the complete absence of a security awareness training program leaves the organization and its employees highly vulnerable to phishing and social engineering attacks.

The external network scan of the designated target IP address did not detect any open ports. While this can indicate a strong firewall configuration, it does not preclude vulnerabilities on internal systems or web applications.

Immediate remediation should focus on implementing MFA across all critical platforms and establishing a comprehensive security awareness training program to mitigate the most severe threats identified in this report.

% --- SECTION 2: ORGANIZATIONAL INFORMATION ---
\section{Organizational Information}
The following details were provided for the assessment.

\begin{tabular}{@{}ll}
    \toprule
    \textbf{Attribute} & \textbf{Value} \\
    \midrule
    Organization Name & \textbf{Harbor Light Foundation} \\
    Email Domain & \texttt{HarborLightFoundation.com} \\
    Website Domain & \url{www.HarborLightFoundation.com} \\
    External IP Address & \texttt{152.197.110.119} \\
    \bottomrule
\end{tabular}

% --- SECTION 3: SECURITY CONTROL REVIEW ---
\section{Security Control Review}
A review of the organization's security controls was conducted via a standardized questionnaire. The responses indicate key areas of strength and weakness in the current security posture. A \textcolor{green}{\ding{51}} indicates a positive control is in place, while a \textcolor{red}{\ding{55}} indicates a security gap.

\begin{table}[h!]
\centering
\begin{tabular}{@{}p{0.8\textwidth}c@{}}
    \toprule
    \textbf{Control Question} & \textbf{Response} \\
    \midrule
    Do you require MFA to access email? & \textcolor{red}{\ding{55}} \\
    Do you require MFA to log into computers? & \textcolor{green}{\ding{51}} \\
    Do you require MFA to access sensitive data systems? & \textcolor{red}{\ding{55}} \\
    Does your organization have an employee acceptable use policy? & \textcolor{green}{\ding{51}} \\
    Does your organization do security awareness training for new employees? & \textcolor{red}{\ding{55}} \\
    Does your organization do security awareness training for all employees at least once per year? & \textcolor{red}{\ding{55}} \\
    \bottomrule
\end{tabular}
\caption{Organizational Security Control Status}
\end{table}

% --- SECTION 4: TECHNICAL SCAN RESULTS ---
\section{Technical Scan Results}
An external network vulnerability scan was performed to identify exposed services and potential weaknesses on the organization's perimeter.

\begin{itemize}
    \item \textbf{Scan Target:} \texttt{[Target IP]}
    \item \textbf{Scan Date:} \today
    \item \textbf{Findings:} The scan completed successfully but did not identify any open TCP or UDP ports on the target host.
\end{itemize}

\subsection{Analysis}
No open ports were detected. This finding suggests that a firewall or other network security device is effectively blocking unsolicited inbound traffic to the scanned IP address. While this is a positive indicator for perimeter security, it does not provide visibility into vulnerabilities that may exist on internal systems or web applications hosted behind the firewall.

% --- SECTION 5: RISK ASSESSMENT ---
\section{Risk Assessment}
The following table synthesizes findings from the security control review and technical scan. No pre-existing vulnerabilities were provided for this assessment. The identified risks are based entirely on the data collected.

\begin{table}[h!]
\centering
\begin{tabular}{@{}p{0.25\textwidth}p{0.5\textwidth}p{0.15\textwidth}@{}}
    \toprule
    \textbf{Risk Name} & \textbf{Overview} & \textbf{Severity} \\
    \midrule
    \textbf{Lack of MFA on Critical Systems} & Email and sensitive data systems are protected only by username and password. This creates a single point of failure and makes accounts highly susceptible to compromise via phishing, credential stuffing, or password spraying attacks. A breach of these systems could lead to significant data loss and operational disruption. & \textbf{\textcolor{red}{Critical}} \\
    \addlinespace
    \textbf{No Security Awareness Training} & Employees are not trained to identify or respond to common cyber threats like phishing, malware, or social engineering. This significantly increases the likelihood of an employee inadvertently causing a security incident, which is the root cause of most successful cyberattacks. & \textbf{\textcolor{orange}{High}} \\
    \bottomrule
\end{tabular}
\caption{Summary of Identified Risks}
\end{table}

% --- SECTION 6: RECOMMENDATIONS ---
\section{Recommendations}
Based on the analysis, the following actions are recommended to mitigate the identified risks and improve the overall security posture of \textbf{Harbor Light Foundation}.

\subsection{Immediate Priority (Critical Risks)}
\begin{enumerate}
    \item \textbf{Implement MFA for Email:} Immediately enforce MFA for all user access to the email system (\texttt{HarborLightFoundation.com}). This is the single most effective control to prevent unauthorized account access.
    \item \textbf{Implement MFA for Sensitive Data:} Identify all systems containing sensitive organizational or client data and enforce MFA for all user access.
\end{enumerate}

\subsection{High Priority (High Risks)}
\begin{enumerate}
    \item \textbf{Establish Security Awareness Training Program:} Develop and implement a mandatory security awareness training program for all employees. The program should be required for new hires during onboarding and annually for all staff.
    \item \textbf{Conduct Phishing Simulations:} Periodically conduct simulated phishing campaigns to test employee awareness and reinforce training concepts. Use the results to provide targeted follow-up training.
\end{enumerate}

\subsection{General Recommendations}
\begin{enumerate}
    \item \textbf{Comprehensive Vulnerability Assessment:} Schedule a comprehensive, authenticated vulnerability assessment of internal and external systems. This will provide deeper insight into potential software vulnerabilities that an unauthenticated external scan cannot detect.
\end{enumerate}

\end{document}
```