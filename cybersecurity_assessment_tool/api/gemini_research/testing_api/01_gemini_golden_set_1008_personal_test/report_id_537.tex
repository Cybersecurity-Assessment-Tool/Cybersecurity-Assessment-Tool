```latex
\documentclass[12pt]{article}

% Preamble: Required Packages and Document Setup
\usepackage[margin=1in]{geometry}
\usepackage{pifont} % Required for checkmarks (\ding{51}) and crosses (\ding{55})
\usepackage{booktabs} % For professional-looking tables (\toprule, \midrule, \bottomrule)
\usepackage{hyperref} % For clickable links and document references
\usepackage{url} % For formatting URLs
\usepackage{seqsplit} % To split long strings in \texttt environments without breaking
\usepackage[T1]{fontenc}

% Custom commands for convenience
\newcommand{\yes}{\ding{51}}
\newcommand{\no}{\ding{55}}

% Hyperref setup for better appearance
\hypersetup{
    colorlinks=true,
    linkcolor=black,
    filecolor=magenta,      
    urlcolor=blue,
    pdftitle={Cybersecurity Posture Assessment Report},
    pdfpagemode=FullScreen,
}

\begin{document}

% --- Title Page ---
\title{Cybersecurity Posture Assessment Report \\ \large For: Gilded Cage Design}
\author{Cybersecurity Analysis Division}
\date{\today}
\maketitle
\thispagestyle{empty}
\newpage

% --- Table of Contents ---
\tableofcontents
\newpage

% --- Section 1: Executive Summary ---
\section{Executive Summary}
This report provides a comprehensive cybersecurity assessment for Gilded Cage Design, based on an analysis of network scan data, organizational security controls, and pre-existing risk documentation.

The assessment reveals a critical security posture. While the organization has implemented foundational controls such as Multi-Factor Authentication (MFA) for email and computer access, significant and high-risk gaps were identified.

\textbf{The most critical finding is an exposed network service on port 8080 with the title "TOP SECRET DB" on an internal host (\texttt{10.5.5.5})}. This suggests a sensitive database is directly accessible from the network, representing a severe information disclosure and potential data breach risk. This finding directly contradicts a previous risk assessment which incorrectly labeled this port as secure.

Furthermore, critical procedural gaps were identified, including the lack of MFA for sensitive data systems and the absence of mandatory annual security awareness training for all employees. These weaknesses, combined with the technical vulnerability, create a high-impact risk scenario that requires immediate remediation. This report outlines prioritized, actionable recommendations to mitigate these risks and strengthen the organization's overall security posture.

% --- Section 2: Organizational Information ---
\section{Organizational Information}
The following details were provided for the assessment. This information helps establish the context for the technical and procedural findings.

\begin{tabular}{@{}ll}
\toprule
\textbf{Attribute} & \textbf{Value} \\
\midrule
Organization Name & \textbf{Gilded Cage Design} \\
Email Domain & \texttt{GildedCageDesign.org} \\
Website Domain & \url{www.GildedCageDesign.org} \\
External IP Address & \texttt{210.218.63.81} \\
\bottomrule
\end{tabular}

% --- Section 3: Security Control Review ---
\section{Security Control Review}
A review of the organization's security controls was conducted via a questionnaire. The responses indicate areas of strength and significant weakness. "No" answers represent gaps in the security framework that increase organizational risk.

\begin{tabular}{@{}p{0.75\textwidth}c}
\toprule
\textbf{Control Question} & \textbf{Status} \\
\midrule
Do you require MFA to access email? & \yes \\
Do you require MFA to log into computers? & \yes \\
\textbf{Do you require MFA to access sensitive data systems?} & \textbf{\no} \\
Does your organization have an employee acceptable use policy? & \yes \\
Does your organization do security awareness training for new employees? & \yes \\
\textbf{Does your organization do security awareness training for all employees at least once per year?} & \textbf{\no} \\
\bottomrule
\end{tabular}

% --- Section 4: Technical Scan Results ---
\section{Technical Scan Results}
A network scan was performed to identify active services and potential vulnerabilities on the target host.

\subsection{Scan Details}
\begin{itemize}
    \item \textbf{Target IP Address:} \texttt{10.5.5.5}
    \item \textbf{Scan Date:} Not specified in scan data.
\end{itemize}

\subsection{Open Ports and Services}
A single open port was discovered on the target host. The details are highly concerning.

\begin{tabular}{@{}llll}
\toprule
\textbf{Port} & \textbf{State} & \textbf{Service/Title} & \textbf{Analysis} \\
\midrule
8080/tcp & Open & HTTP Title: \textbf{TOP SECRET DB} & \parbox[t]{0.5\textwidth}{This finding is critical. The service title strongly implies that a highly sensitive database is exposed on this port. Such information disclosure can guide an attacker and may indicate a lack of proper authentication controls on a critical internal system.} \\
\bottomrule
\end{tabular}

% --- Section 5: Correlated Risk Assessment ---
\section{Correlated Risk Assessment}
This section synthesizes findings from the security control review, technical scan, and pre-existing risk data. The correlation reveals a risk profile that is more severe than previously understood. The existing risk assessment for Port 8080 was found to be inaccurate and is superseded by these findings.

\begin{tabular}{@{}p{0.15\textwidth}p{0.5\textwidth}p{0.2\textwidth}}
\toprule
\textbf{Risk ID} & \textbf{Risk Description} & \textbf{Severity} \\
\midrule
\textbf{RISK-001} & \textbf{Exposed Sensitive Database Interface.} An internal service on port 8080 is accessible and self-identifies as "TOP SECRET DB". This presents a critical risk of unauthorized access and data exfiltration. & \textbf{Critical} \\
\addlinespace
\textbf{RISK-002} & \textbf{Inadequate Access Controls for Sensitive Data.} The lack of mandatory MFA for sensitive data systems, combined with the exposed database (RISK-001), significantly increases the likelihood of a successful breach. & \textbf{High} \\
\addlinespace
\textbf{RISK-003} & \textbf{Insufficient Security Awareness Program.} The absence of annual security training for all employees heightens the risk of human error, such as credential compromise or improper data handling, which could be exploited to access sensitive systems. & \textbf{High} \\
\addlinespace
\textbf{RISK-004} & \textbf{Outdated and Inaccurate Risk Assessment.} The pre-existing risk data incorrectly states Port 8080 is secure. This indicates a flawed risk management process, preventing the organization from addressing a known critical vulnerability. & \textbf{Medium} \\
\bottomrule
\end{tabular}

% --- Section 6: Recommendations ---
\section{Recommendations}
The following actions are recommended to mitigate the identified risks. They are prioritized based on severity and potential impact.

\subsection{Immediate Actions (0-7 Days)}
\begin{enumerate}
    \item \textbf{Investigate and Secure Port 8080 on \texttt{10.5.5.5}:}
    \begin{itemize}
        \item Immediately identify the owner and purpose of the service running on port 8080.
        \item If the service is not essential, disable it.
        \item If essential, restrict access to authorized personnel only via firewall rules and implement strong, mandatory authentication (e.g., username/password and MFA).
        \item Remove or change the "TOP SECRET DB" title to prevent information disclosure.
    \end{itemize}
\end{enumerate}

\subsection{High-Priority Actions (1-3 Months)}
\begin{enumerate}
    \setcounter{enumi}{1}
    \item \textbf{Implement Mandatory MFA for All Sensitive Systems:} Enforce MFA for all systems identified as containing sensitive or critical data. This is a crucial compensating control to protect against credential theft.
    \item \textbf{Establish Annual Security Awareness Training:} Develop and deploy a mandatory security awareness training program for all employees, to be completed annually. The training should cover phishing, password hygiene, and acceptable use policies.
\end{enumerate}

\subsection{Process Improvement (Ongoing)}
\begin{enumerate}
    \setcounter{enumi}{3}
    \item \textbf{Review and Enhance Risk Management Processes:} Update the risk assessment and management lifecycle. Ensure that technical scan results are regularly reviewed and that the risk register is a living document that accurately reflects the current threat landscape.
\end{enumerate}

\end{document}
```