```latex
\documentclass[12pt]{article}

% --- PACKAGES ---
\usepackage[margin=1in]{geometry}
\usepackage{pifont} % For \ding
\usepackage{booktabs} % For professional tables
\usepackage{hyperref} % For clickable links
\usepackage{url} % For URL formatting
\usepackage{seqsplit} % To split long strings in tt font
\usepackage{graphicx}
\usepackage{xcolor}

% --- DOCUMENT SETUP ---
\hypersetup{
    colorlinks=true,
    linkcolor=blue,
    filecolor=magenta,      
    urlcolor=cyan,
    pdftitle={Cybersecurity Assessment Report},
    pdfpagemode=FullScreen,
}

\linespread{1.1}

% --- TITLE ---
\title{
    \vspace{-2cm}
    \rule{\textwidth}{2pt} \\ [0.5cm]
    \textbf{Cybersecurity Assessment Report} \\ [0.2cm]
    \large \textbf{For: Fable \& Lore} \\ [0.5cm]
    \rule{\textwidth}{2pt}
}
\author{Cybersecurity Analyst}
\date{\today}

% --- BEGIN DOCUMENT ---
\begin{document}

\maketitle
\thispagestyle{empty}
\newpage

\tableofcontents
\newpage

% --- EXECUTIVE SUMMARY ---
\section{Overview and Executive Summary}
This report details the findings of a cybersecurity assessment conducted for \textbf{Fable \& Lore}. The analysis synthesizes data from a network infrastructure scan, a review of organizational security controls, and an evaluation of pre-existing risk documentation.

The assessment has identified a \textbf{critical risk}: a network service on port \texttt{8080} of the internal host \texttt{10.5.5.5} is publicly exposing a resource titled \textbf{"TOP SECRET DB"}. This finding directly contradicts previous risk assessments which had incorrectly classified this port as a secure false positive. This discrepancy points to a significant flaw in the risk validation process.

Furthermore, critical gaps were identified in the organization's security posture through a questionnaire. The lack of Multi-Factor Authentication (MFA) on computers and sensitive data systems, combined with the absence of annual security awareness training for all employees, creates a high-risk environment. These control deficiencies significantly increase the likelihood of unauthorized access and potential data breaches.

Immediate remediation is required to address the exposed database. Strategic initiatives must be launched to implement MFA and bolster the security training program to mitigate these high-priority risks.

% --- ORGANIZATIONAL INFORMATION ---
\section{Organizational Information}
The following details were provided for the assessment.
\begin{itemize}
    \item \textbf{Organization Name:} Fable \& Lore
    \item \textbf{Primary Email Domain:} \texttt{FableLore.net}
    \item \textbf{Primary Website:} \url{www.FableLore.net}
    \item \textbf{Known External IP:} \texttt{116.15.151.127}
\end{itemize}

% --- SECURITY CONTROL REVIEW ---
\section{Security Control Review}
The following table summarizes the organization's responses to a security controls questionnaire. "No" answers indicate significant gaps in the security framework and are flagged for immediate attention.

\begin{table}[h!]
\centering
\caption{Security Controls Questionnaire Analysis}
\label{tab:controls}
\begin{tabular}{p{8cm} c p{4cm}}
\toprule
\textbf{Control Question} & \textbf{Response} & \textbf{Analyst Assessment} \\
\midrule
Do you require MFA to access email? & \ding{51} & Good Practice \\
\addlinespace
Do you require MFA to log into computers? & \textbf{\color{red}\ding{55}} & \textbf{High Risk}. Lack of MFA on endpoints increases the risk of unauthorized access via compromised credentials. \\
\addlinespace
Do you require MFA to access sensitive data systems? & \textbf{\color{red}\ding{55}} & \textbf{Critical Gap}. This is a primary defense against unauthorized access to critical data, directly related to the findings in Section 4. \\
\addlinespace
Does your organization have an employee acceptable use policy? & \ding{51} & Good Practice \\
\addlinespace
Does your organization do security awareness training for new employees? & \ding{51} & Good Practice \\
\addlinespace
Does your organization do security awareness training for all employees at least once per year? & \textbf{\color{red}\ding{55}} & \textbf{High Risk}. Without recurring training, employee awareness of evolving threats diminishes, increasing susceptibility to phishing and social engineering. \\
\bottomrule
\end{tabular}
\end{table}

% --- TECHNICAL SCAN RESULTS ---
\section{Technical Scan Results}
A network scan was performed on the specified target to identify open ports and exposed services.

\begin{itemize}
    \item \textbf{Target IP Address:} \texttt{10.5.5.5}
\end{itemize}

The scan revealed the following open port:

\begin{table}[h!]
\centering
\caption{Open Port Analysis for Target: 10.5.5.5}
\label{tab:scan}
\begin{tabular}{l l p{8cm}}
\toprule
\textbf{Port} & \textbf{State} & \textbf{Service Information} \\
\midrule
8080/tcp & OPEN & The HTTP service running on this port returned the title: \textbf{"TOP SECRET DB"}. \\
\bottomrule
\end{tabular}
\end{table}

\subsection{Analysis of Technical Findings}
The discovery of an open port \texttt{8080} with a title explicitly indicating a "TOP SECRET DB" is a finding of \textbf{critical severity}. This suggests that a sensitive, and potentially unauthenticated, database interface is accessible on the network. This finding is especially alarming as the pre-existing risk documentation (Input 3) incorrectly states this port is secure and a false positive. This indicates a severe failure in the risk assessment and validation process.

% --- RISK ASSESSMENT ---
\section{Risk Assessment}
The following table synthesizes the findings from the security control review and technical scan into a prioritized list of risks.

\begin{table}[h!]
\centering
\caption{Summary of Identified Risks}
\label{tab:risks}
\begin{tabular}{p{4cm} p{6cm} l}
\toprule
\textbf{Risk Title} & \textbf{Description} & \textbf{Severity} \\
\midrule
\textbf{Exposed Sensitive Database Interface} & An HTTP service on \texttt{10.5.5.5:8080} is titled "TOP SECRET DB", suggesting unauthorized access to critical data is possible. & \textbf{Critical} \\
\addlinespace
\textbf{Lack of MFA on Sensitive Systems} & The absence of MFA on sensitive systems and computers allows for potential unauthorized access using only a single factor (e.g., a stolen password). & \textbf{High} \\
\addlinespace
\textbf{Inaccurate Risk Management Data} & Previous assessments incorrectly labeled port 8080 as a secure false positive. This indicates a flawed validation process, meaning other critical risks may be undocumented. & \textbf{High} \\
\addlinespace
\textbf{Insufficient Security Awareness} & Lack of annual training for all employees increases the risk of security incidents caused by human error, such as falling victim to phishing attacks. & \textbf{High} \\
\bottomrule
\end{tabular}
\end{table}

% --- RECOMMENDATIONS ---
\section{Recommendations}
Based on the identified risks, the following remediation actions are recommended. They are prioritized to address the most critical threats first.

\subsection{Immediate Actions (To Be Completed within 72 Hours)}
\begin{enumerate}
    \item \textbf{Secure Exposed Database:} Immediately investigate the service running on \texttt{10.5.5.5:8080}.
    \begin{itemize}
        \item If the service is not required for business operations, shut it down.
        \item If it is required, restrict access to authorized personnel only via firewall rules and implement mandatory authentication.
        \item Conduct a forensic analysis to determine if the service has been accessed by unauthorized parties.
    \end{itemize}
\end{enumerate}

\subsection{High-Priority Actions (To Be Completed within 90 Days)}
\begin{enumerate}
    \setcounter{enumi}{1} % Continue numbering
    \item \textbf{Implement Comprehensive MFA:} Deploy a robust Multi-Factor Authentication solution for all employees for:
    \begin{itemize}
        \item Logging into company computers and laptops.
        \item Accessing all systems identified as containing sensitive data.
    \end{itemize}
    \item \textbf{Revise Risk Assessment Procedures:} Conduct a full review of the risk management and validation process. Re-validate all previously identified risks, especially those closed as "false positives" or "low severity".
    \item \textbf{Establish Annual Security Training:} Implement a mandatory security awareness training program for all employees, to be conducted at least once per year. This program should cover current threats such as phishing, ransomware, and proper data handling.
\end{enumerate}

\end{document}
```