```latex
\documentclass[12pt]{article}

% Preamble: Required Packages
\usepackage[margin=1in]{geometry}
\usepackage{pifont} % For using dingbats like checkmarks and crosses
\usepackage{booktabs} % For professional-looking tables
\usepackage{hyperref} % For creating hyperlinks within the document
\usepackage{url}      % For formatting URLs
\usepackage{seqsplit} % To split long strings without spaces
\usepackage{xcolor}   % For defining colors

% Hyperlink Setup
\hypersetup{
    colorlinks=true,
    linkcolor=blue,
    filecolor=magenta,      
    urlcolor=cyan,
    pdftitle={Cybersecurity Posture Assessment Report},
    pdfauthor={Cybersecurity Analysis Division},
}

% Custom Commands for convenience
\newcommand{\yes}{\textcolor{green}{\ding{51}}} % Green checkmark
\newcommand{\no}{\textcolor{red}{\ding{55}}}   % Red cross

% Document Metadata
\title{Cybersecurity Posture Assessment Report}
\author{Cybersecurity Analysis Division}
\date{\today}

\begin{document}

\maketitle
\tableofcontents
\newpage

% ===================================================================
% SECTION 1: EXECUTIVE SUMMARY
% ===================================================================
\section*{Executive Summary}

This report provides a comprehensive cybersecurity assessment for \textbf{Silent Spring}. The analysis is based on a correlation of network scan data, an organizational security questionnaire, and a review of pre-existing risks.

The assessment reveals a critical security posture characterized by the absence of fundamental security controls. The most significant findings include:
\begin{itemize}
    \item \textbf{Lack of Multi-Factor Authentication (MFA):} MFA is not enforced for email, computer logins, or access to sensitive data systems. This represents a critical vulnerability, as a single compromised password could lead to a widespread breach.
    \item \textbf{Exposed Remote Services:} The technical scan identified a new host (\texttt{10.10.10.51}) with an open Remote Desktop Protocol (RDP) port (\texttt{3389}). This, combined with a pre-existing RDP exposure on another host, presents a high-impact vector for ransomware and unauthorized access.
    \item \textbf{Deficient Security Governance:} The organization lacks a formal employee acceptable use policy and does not conduct security awareness training. This elevates the risk of human error, such as falling for phishing attacks, which could directly lead to the compromise of credentials.
\end{itemize}

Immediate and decisive action is required to remediate these findings. Recommendations focus on implementing foundational security controls to significantly reduce the organization's risk profile.

% ===================================================================
% SECTION 2: ORGANIZATIONAL INFORMATION
% ===================================================================
\section{Organizational Information}

The following details were provided for the assessment.

\begin{itemize}
    \item \textbf{Organization Name:} Silent Spring
    \item \textbf{Email Domain:} \texttt{SilentSpring.net}
    \item \textbf{Website Domain:} \url{www.SilentSpring.net}
    \item \textbf{External IP Address:} \texttt{79.99.217.128}
\end{itemize}

% ===================================================================
% SECTION 3: SECURITY CONTROL REVIEW (QUESTIONNAIRE)
% ===================================================================
\section{Security Control Review}

The following table summarizes the organization's responses to a security controls questionnaire. A red \no\ indicates a significant gap in security posture.

\begin{table}[h!]
\centering
\caption{Security Controls Questionnaire Results}
\begin{tabular}{p{0.8\linewidth} c}
\toprule
\textbf{Control Question} & \textbf{Response} \\
\midrule
Do you require MFA to access email? & \no \\
Do you require MFA to log into computers? & \no \\
Do you require MFA to access sensitive data systems? & \no \\
Does your organization have an employee acceptable use policy? & \no \\
Does your organization do security awareness training for new employees? & \no \\
Does your organization do security awareness training for all employees at least once per year? & \no \\
\bottomrule
\end{tabular}
\end{table}

\subsection*{Analysis}
The complete absence of "Yes" responses indicates a lack of foundational security practices. The lack of MFA is the most critical technical gap, while the absence of policies and training points to a systemic governance issue.

% ===================================================================
% SECTION 4: TECHNICAL SCAN RESULTS
% ===================================================================
\section{Technical Scan Results}

An Nmap scan was performed on the target host \texttt{10.10.10.51}. The results identified the following open port.

\begin{table}[h!]
\centering
\caption{Open Ports Detected on \texttt{10.10.10.51}}
\begin{tabular}{llll}
\toprule
\textbf{Port} & \textbf{State} & \textbf{Service} & \textbf{Notes} \\
\midrule
3389/tcp & open & ms-wbt-server & Microsoft Remote Desktop Protocol (RDP) \\
\bottomrule
\end{tabular}
\end{table}

\subsection*{Analysis}
The discovery of an open RDP port is a high-risk finding. RDP is a primary target for attackers seeking to gain remote access to internal networks. When combined with the lack of MFA, this service is protected only by a username and password, making it highly susceptible to brute-force or credential-stuffing attacks.

% ===================================================================
% SECTION 5: CONSOLIDATED RISK ASSESSMENT
% ===================================================================
\section{Consolidated Risk Assessment}

The following table synthesizes findings from the questionnaire, technical scan, and pre-existing risk data into a prioritized list.

\begin{table}[h!]
\centering
\caption{Summary of Identified Risks}
\begin{tabular}{p{0.1\linewidth} p{0.45\linewidth} p{0.2\linewidth} p{0.1\linewidth}}
\toprule
\textbf{Risk ID} & \textbf{Description} & \textbf{Affected Asset(s)} & \textbf{Severity} \\
\midrule
RISK-001 & \textbf{Lack of MFA.} A single compromised password could grant an attacker full access to email, workstations, and sensitive data. & All Users \& Systems & \textbf{Critical} \\
\addlinespace
RISK-002 & \textbf{Newly Discovered RDP Exposure.} Port 3389 is open, providing a direct vector for remote network compromise. & \texttt{10.10.10.51} & \textbf{Critical} \\
\addlinespace
RISK-003 & \textbf{Pre-existing RDP Exposure.} A known RDP exposure that remains unresolved. & \texttt{10.10.10.50} & \textbf{Critical} \\
\addlinespace
RISK-004 & \textbf{Lack of Security Policies \& Training.} No acceptable use policy or security training increases the likelihood of human-related security incidents (e.g., phishing). & All Employees & \textbf{High} \\
\bottomrule
\end{tabular}
\end{table}

% ===================================================================
% SECTION 6: RECOMMENDATIONS
% ===================================================================
\section{Recommendations}

Based on the consolidated risk assessment, the following actions are recommended to improve the organization's security posture.

\subsection*{Immediate Actions (Remediate within 72 hours)}
\begin{enumerate}
    \item \textbf{Remediate RDP Exposures (RISK-002, RISK-003):} For hosts \texttt{10.10.10.51} and \texttt{10.10.10.50}, immediately implement one of the following:
    \begin{itemize}
        \item If RDP is not required, disable the service and block port 3389 at the firewall.
        \item If RDP is required, place it behind a Virtual Private Network (VPN) that requires MFA for access.
        \item At a minimum, restrict access to port 3389 to specific, trusted source IP addresses via firewall rules.
    \end{itemize}
\end{enumerate}

\subsection*{High-Priority Actions (Remediate within 30 days)}
\begin{enumerate}
    \setcounter{enumi}{1} % Continue numbering
    \item \textbf{Implement Multi-Factor Authentication (RISK-001):} Deploy MFA across the organization with the following priority:
    \begin{itemize}
        \item \textbf{Phase 1:} All external-facing services, privileged accounts, and email (e.g., Office 365, G Suite).
        \item \textbf{Phase 2:} All sensitive data systems.
        \item \textbf{Phase 3:} All employee computer logins.
    \end{itemize}
\end{enumerate}

\subsection*{Medium-Priority Actions (Remediate within 90 days)}
\begin{enumerate}
    \setcounter{enumi}{2} % Continue numbering
    \item \textbf{Develop Security Policies (RISK-004):} Create and formally adopt a baseline set of security policies, starting with an \textit{Employee Acceptable Use Policy}. This policy should be signed by all employees.
    \item \textbf{Institute Security Awareness Training (RISK-004):} Implement a mandatory security awareness training program for all employees. This should be required for new hires during onboarding and annually for all staff. Training should cover topics such as phishing, password security, and data handling.
\end{enumerate}

\end{document}
```