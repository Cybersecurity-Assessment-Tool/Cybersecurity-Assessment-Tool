```latex
\documentclass[12pt]{article}

% Preamble: Required Packages
\usepackage[margin=1in]{geometry}
\usepackage{pifont} % For checkmarks and crosses
\usepackage{booktabs} % For professional tables
\usepackage{hyperref} % For clickable links and references
\usepackage{url} % For URL formatting
\usepackage{seqsplit} % For splitting long strings without breaking
\usepackage{graphicx} % For potential logos
\usepackage{fancyhdr} % For headers and footers
\usepackage{xcolor} % For colors in text

% --- Document Metadata ---
\title{Cybersecurity Posture Assessment Report}
\author{Cybersecurity Analysis Division}
\date{\today}

% --- Header and Footer Configuration ---
\pagestyle{fancy}
\fancyhf{}
\lhead{Summit Peak Partners}
\rhead{Confidential}
\cfoot{\thepage}
\renewcommand{\headrulewidth}{0.4pt}
\renewcommand{\footrulewidth}{0.4pt}

% --- Hyperref Setup ---
\hypersetup{
    colorlinks=true,
    linkcolor=blue,
    filecolor=magenta,      
    urlcolor=cyan,
    pdftitle={Cybersecurity Posture Assessment Report},
    pdfpagemode=FullScreen,
}

\begin{document}

\maketitle
\thispagestyle{empty}
\newpage

\tableofcontents
\newpage

% --- Section 1: Executive Summary ---
\section{Executive Summary}

This report details the findings of a cybersecurity posture assessment for \textbf{Summit Peak Partners}. The assessment incorporated an analysis of organizational security controls via a questionnaire, a technical network scan of a key internal asset, and a review of pre-existing documented risks.

The overall security posture presents a significant contrast between technical and administrative controls. On a positive note, the technical scan of the target host at \texttt{192.168.1.100} revealed a strong network security configuration, with no open ports detected. This indicates effective firewalling and network hardening for that specific asset.

However, the review of organizational security controls identified several \textbf{critical gaps} that expose the organization to substantial risk. The most severe findings are the lack of Multi-Factor Authentication (MFA) for email and sensitive data systems. These gaps dramatically increase the likelihood of account compromise, data breaches, and successful phishing attacks. Furthermore, the absence of a formal Acceptable Use Policy and security training for new employees weakens the organization's human firewall and creates policy enforcement challenges.

Immediate remediation should focus on implementing MFA across all critical platforms and establishing foundational security policies and training programs as outlined in the Recommendations section.

% --- Section 2: Organizational Information ---
\section{Organizational Information}

The following information was provided for the assessment.

\begin{tabular}{@{}ll}
\toprule
\textbf{Attribute} & \textbf{Value} \\
\midrule
Organization Name & \textbf{Summit Peak Partners} \\
Email Domain & \texttt{SummitPeakPartners.net} \\
Website Domain & \url{www.SummitPeakPartners.net} \\
External IP Address & \texttt{96.179.141.255} \\
\bottomrule
\end{tabular}

% --- Section 3: Security Control Review ---
\section{Security Control Review}

The following table summarizes the organization's responses to the security controls questionnaire. Each "No" response represents a potential security gap that requires attention.

\begin{tabular}{@{}p{0.6\linewidth}cp{0.25\linewidth}@{}}
\toprule
\textbf{Control Question} & \textbf{Status} & \textbf{Assessment} \\
\midrule
Do you require MFA to access email? & \ding{55} & \textcolor{red}{\textbf{Critical Gap.}} Increases risk of Business Email Compromise (BEC). \\
\addlinespace
Do you require MFA to log into computers? & \ding{51} & \textbf{Good Practice.} Reduces risk of unauthorized local access. \\
\addlinespace
Do you require MFA to access sensitive data systems? & \ding{55} & \textcolor{red}{\textbf{Critical Gap.}} Exposes critical data to unauthorized access. \\
\addlinespace
Does your organization have an employee acceptable use policy? & \ding{55} & \textcolor{orange}{\textbf{High Risk.}} Lack of clear policy for employees creates ambiguity and legal risk. \\
\addlinespace
Does your organization do security awareness training for new employees? & \ding{55} & \textcolor{orange}{\textbf{High Risk.}} New hires are a prime target for social engineering attacks. \\
\addlinespace
Does your organization do security awareness training for all employees at least once per year? & \ding{51} & \textbf{Good Practice.} Helps maintain a security-conscious culture. \\
\bottomrule
\end{tabular}

% --- Section 4: Technical Scan Results ---
\section{Technical Scan Results}

A network scan was performed to identify open ports and services on the specified target system.

\begin{itemize}
    \item \textbf{Target IP Address:} \texttt{192.168.1.100}
    \item \textbf{Scan Tool:} Nmap
\end{itemize}

\subsection{Findings}
The scan determined that the host is online and responsive. However, \textbf{no open TCP or UDP ports were discovered}. All 1000 scanned TCP ports were in a 'closed' state.

\subsection{Analysis}
This is a \textbf{positive security finding}. It indicates that the target system is well-hardened from a network perspective, likely protected by a properly configured host-based or network firewall. This configuration significantly reduces the external attack surface of the asset.

% --- Section 5: Risk Assessment Summary ---
\section{Risk Assessment Summary}

The following table synthesizes findings from the security control review and technical scan into a prioritized list of risks. No pre-existing vulnerabilities were documented.

\begin{tabular}{@{}lp{0.25\linewidth}p{0.45\linewidth}l@{}}
\toprule
\textbf{ID} & \textbf{Risk Name} & \textbf{Description} & \textbf{Severity} \\
\midrule
RISK-001 & Inadequate Email Account Security & The absence of MFA on email accounts allows an attacker with stolen credentials to gain full access, leading to BEC, data theft, and further internal compromise. & \textbf{Critical} \\
\addlinespace
RISK-002 & Unprotected Access to Sensitive Data & Lack of MFA on sensitive data systems means a single compromised password could lead to a major data breach, violating regulatory and client trust obligations. & \textbf{Critical} \\
\addlinespace
RISK-003 & Lack of Acceptable Use Policy (AUP) & Without a formal AUP, employees may misuse company assets unintentionally or maliciously. It also complicates disciplinary action for policy violations. & \textbf{High} \\
\addlinespace
RISK-004 & Inadequate New Hire Security Onboarding & New employees are not trained on security policies and threats upon joining, making them highly susceptible to phishing and social engineering attacks. & \textbf{High} \\
\bottomrule
\end{tabular}

% --- Section 6: Recommendations ---
\section{Recommendations}

Based on the identified risks, the following prioritized actions are recommended to improve the cybersecurity posture of \textbf{Summit Peak Partners}.

\begin{enumerate}
    \item \textbf{(Critical) Implement MFA for Email:} Immediately enforce MFA for all user accounts accessing the \texttt{SummitPeakPartners.net} email system. This is the single most effective control to prevent account takeovers and Business Email Compromise.
    
    \item \textbf{(Critical) Enforce MFA on Sensitive Systems:} Identify all systems containing sensitive client, financial, or proprietary data. Mandate the use of MFA for all access, both privileged and standard user.
    
    \item \textbf{(High) Develop and Implement an Acceptable Use Policy (AUP):} Create a formal AUP that clearly defines the rules for using company IT assets, data handling, and internet usage. This policy should be reviewed by legal/HR and signed by all employees.
    
    \item \textbf{(High) Integrate Security Training into Employee Onboarding:} Develop a mandatory security awareness training module for all new hires. This training should be completed within their first week and cover key topics like phishing, password security, and the new AUP.
    
    \item \textbf{(Informational) Maintain Network Hardening:} Continue the excellent practice of network hardening and firewall management observed on the scanned host. Regularly audit firewall rules and scan for unintended open ports across the network.
\end{enumerate}

\end{document}
```