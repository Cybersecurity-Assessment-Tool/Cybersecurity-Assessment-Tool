```latex
\documentclass[12pt]{article}

% Preamble: Required Packages
\usepackage[margin=1in]{geometry}
\usepackage{pifont} % For checkmarks and crosses
\usepackage{booktabs} % For professional tables
\usepackage{hyperref} % For clickable links
\usepackage{url} % For formatting URLs
\usepackage{seqsplit} % To split long strings like IPs
\usepackage{graphicx}
\usepackage{xcolor}

% Document Metadata
\title{Cybersecurity Posture Assessment Report}
\author{Cybersecurity Analysis Division}
\date{\today}

% Hyperref Setup
\hypersetup{
    colorlinks=true,
    linkcolor=black,
    filecolor=magenta,      
    urlcolor=blue,
    pdftitle={Cybersecurity Posture Assessment Report},
    pdfpagemode=FullScreen,
}

\begin{document}

\maketitle
\thispagestyle{empty}
\newpage

\tableofcontents
\newpage

% --- Section 1: Executive Summary ---
\section{Executive Summary}
This report details the findings of a cybersecurity posture assessment for \textbf{Ironclad Logistics}. The assessment combined a review of organizational security controls, an external network scan, and an analysis of pre-existing risks.

The overall security posture is considered \textbf{High-Risk}. Critical deficiencies were identified in fundamental security controls, most notably the absence of Multi-Factor Authentication (MFA) for email and sensitive data systems. Furthermore, the lack of a formal employee security training program and an acceptable use policy creates a significant vulnerability to social engineering and insider threats.

Technical analysis revealed an externally accessible Secure Shell (SSH) service on the IPv6 address \seqsplit{\texttt{2001:db8::1}}. While SSH is a standard administrative protocol, its exposure to the public internet, combined with the identified policy and MFA gaps, presents a direct and viable attack vector for unauthorized access.

Immediate remediation is required to address the MFA and policy gaps to mitigate the most severe risks to the organization.

% --- Section 2: Organizational Information ---
\section{Organizational Information}
The following details were provided for the assessment.

\begin{itemize}
    \item \textbf{Organization Name:} Ironclad Logistics
    \item \textbf{Primary Email Domain:} \texttt{IroncladLogistics.com}
    \item \textbf{Primary Website:} \url{www.IroncladLogistics.com}
    \item \textbf{Scanned External IP:} \texttt{239.135.232.197}
\end{itemize}

% --- Section 3: Security Control Review ---
\section{Security Control Review}
A questionnaire was completed to evaluate the implementation of key administrative and technical security controls. The responses indicate critical gaps in the organization's security framework. A "No" response (\ding{55}) highlights a missing control that increases organizational risk.

\begin{table}[h!]
\centering
\caption{Security Controls Questionnaire Results}
\begin{tabular}{p{0.8\textwidth} c}
\toprule
\textbf{Control Question} & \textbf{Response} \\
\midrule
Do you require MFA to log into computers? & \ding{51} \\
\addlinespace
Do you require MFA to access email? & \textcolor{red}{\ding{55}} \\
\addlinespace
Do you require MFA to access sensitive data systems? & \textcolor{red}{\ding{55}} \\
\addlinespace
Does your organization have an employee acceptable use policy? & \textcolor{red}{\ding{55}} \\
\addlinespace
Does your organization do security awareness training for new employees? & \textcolor{red}{\ding{55}} \\
\addlinespace
Does your organization do security awareness training for all employees at least once per year? & \textcolor{red}{\ding{55}} \\
\bottomrule
\end{tabular}
\end{table}

% --- Section 4: Technical Scan Results ---
\section{Technical Scan Results}
An external network scan was performed against the organization's infrastructure to identify open ports and exposed services.

\begin{itemize}
    \item \textbf{Scan Target:} \seqsplit{\texttt{2001:db8::1}}
    \item \textbf{Scan Date:} Data not provided in scan metadata.
    \item \textbf{Target Status:} Host is up and responsive.
\end{itemize}

The following table details the open ports discovered during the scan.

\begin{table}[h!]
\centering
\caption{Open Ports Detected on \seqsplit{\texttt{2001:db8::1}}}
\begin{tabular}{llll}
\toprule
\textbf{Port} & \textbf{State} & \textbf{Service} & \textbf{Notes} \\
\midrule
22/tcp & open & SSH & Secure Shell (for remote administration). \\
\bottomrule
\end{tabular}
\end{table}

\subsection{Analysis of Technical Findings}
The scan identified that port 22 (SSH) is open to the public internet. SSH is a common protocol for remote system administration. However, its public exposure is a significant security risk if not properly configured. Potential vulnerabilities include:
\begin{itemize}
    \item \textbf{Brute-force attacks:} Automated attempts to guess usernames and passwords.
    \item \textbf{Credential stuffing:} Using credentials stolen from other breaches.
    \item \textbf{Exploitation of software vulnerabilities:} If the SSH server software is outdated.
\end{itemize}
The risk from this exposed service is amplified by the lack of MFA and security awareness training, which increases the likelihood of weak or compromised user credentials.

% --- Section 5: Risk Assessment ---
\section{Risk Assessment}
This section synthesizes findings from the security control review, technical scan, and pre-existing risk data. No pre-existing vulnerabilities were provided for this assessment.

\begin{table}[h!]
\centering
\caption{Summary of Identified Risks}
\begin{tabular}{p{0.1\textwidth} p{0.5\textwidth} p{0.15\textwidth} p{0.15\textwidth}}
\toprule
\textbf{ID} & \textbf{Risk Description} & \textbf{Severity} & \textbf{Source} \\
\midrule
RISK-001 & Lack of MFA on email exposes the organization to account takeover, phishing, and data breaches. & \textbf{Critical} & Questionnaire \\
\addlinespace
RISK-002 & Lack of MFA on sensitive systems allows unauthorized access to critical data with only a single factor (password). & \textbf{Critical} & Questionnaire \\
\addlinespace
RISK-003 & Absence of security policies and training leads to a high probability of human error, such as falling for phishing attacks. & \textbf{High} & Questionnaire \\
\addlinespace
RISK-004 & Publicly exposed SSH service provides a direct vector for network intrusion if credentials are weak or the service is vulnerable. & \textbf{High} & Network Scan \\
\bottomrule
\end{tabular}
\end{table}

% --- Section 6: Recommendations ---
\section{Recommendations}
The following actionable recommendations are provided to mitigate the identified risks. They are prioritized based on severity and potential impact.

\subsection{Immediate Priority (Critical Risks)}
\begin{enumerate}
    \item \textbf{Implement MFA Across All Critical Systems:} Enforce MFA for all user accounts on email (RISK-001) and any system containing sensitive data (RISK-002). This is the single most effective control to prevent unauthorized access.
\end{enumerate}

\subsection{High Priority}
\begin{enumerate}
    \item \textbf{Develop and Implement Security Policies:} Create and enforce a baseline set of security policies, starting with an Acceptable Use Policy (AUP) that all employees must read and sign (RISK-003).
    \item \textbf{Establish a Security Awareness Program:} Implement a mandatory security awareness training program for all new hires and conduct annual refresher training for all staff. This program should cover phishing, password hygiene, and social engineering (RISK-003).
    \item \textbf{Secure the Exposed SSH Service (RISK-004):}
        \begin{itemize}
            \item \textbf{Restrict Access:} If SSH access is not required from the public internet, block port 22 at the firewall. If it is required, restrict access to specific, known IP addresses (e.g., administrator's office).
            \item \textbf{Enforce Strong Configuration:} Disable password-based authentication and require public key authentication. Disable root login over SSH.
            \item \textbf{Patch Management:} Ensure the SSH server software is kept up-to-date with the latest security patches.
        \end{itemize}
\end{enumerate}

\end{document}
```