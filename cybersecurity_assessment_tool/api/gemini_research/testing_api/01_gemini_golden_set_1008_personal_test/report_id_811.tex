```latex
\documentclass[12pt]{article}

% Preamble: Required Packages
\usepackage[a4paper, margin=1in]{geometry}
\usepackage{pifont} % For checkmarks and crosses
\usepackage{booktabs} % For professional tables
\usepackage{hyperref} % For clickable links and references
\usepackage{url}      % For formatting URLs
\usepackage{seqsplit} % For splitting long strings in tt font

% Document Metadata
\title{Cybersecurity Posture Assessment Report}
\author{Cybersecurity Analysis Division}
\date{\today}

% Hyperref Setup
\hypersetup{
    colorlinks=true,
    linkcolor=black,
    urlcolor=blue,
    pdftitle={Cybersecurity Posture Assessment Report},
    pdfauthor={Cybersecurity Analysis Division},
    pdfsubject={Security Assessment},
    pdfkeywords={Cybersecurity, Risk, Assessment, Nmap, Controls}
}

\begin{document}

\maketitle
\thispagestyle{empty}
\newpage
\tableofcontents
\newpage

% --- 1. Executive Summary ---
\section{Executive Summary}

This report provides a comprehensive analysis of the cybersecurity posture for \textbf{Hidden Gem}. The assessment is based on a correlation of organizational data, a security controls questionnaire, and a network vulnerability scan.

The overall security posture has significant room for improvement. While several foundational controls are in place, such as multi-factor authentication (MFA) for computer and sensitive system access, two critical gaps were identified that expose the organization to substantial risk:

\begin{itemize}
    \item \textbf{Critical Risk: Lack of MFA on Email.} The absence of MFA on the primary email system (\texttt{HiddenGem.net}) presents a severe risk of account compromise, business email compromise (BEC), and subsequent data breaches.
    \item \textbf{High Risk: Inadequate Security Training.} The lack of mandatory, annual security awareness training for all employees increases susceptibility to phishing and other social engineering attacks.
\end{itemize}

A technical scan of the external network revealed an open Secure Shell (SSH) port on the IPv6 address \seqsplit{\texttt{2001:db8::1}}. While a common administrative service, its exposure to the public internet requires stringent security configurations to prevent unauthorized access.

Immediate remediation should focus on implementing MFA for all email accounts and establishing a recurring security awareness training program.

% --- 2. Organizational Information ---
\section{Organizational Information}

The following details were provided for the assessment. This information is used to establish the context and scope of the review.

\begin{tabular}{@{}ll}
\toprule
\textbf{Attribute} & \textbf{Value} \\
\midrule
Organization Name & \textbf{Hidden Gem} \\
Primary Email Domain & \texttt{HiddenGem.net} \\
Primary Website & \url{www.HiddenGem.net} \\
External IPv4 Address & \texttt{193.98.33.47} \\
Scanned IPv6 Address & \seqsplit{\texttt{2001:db8::1}} \\
\bottomrule
\end{tabular}

% --- 3. Security Control Review ---
\section{Security Control Review}

A review of internal security controls was conducted via a standardized questionnaire. The responses indicate the current state of implemented policies and procedures. Gaps identified in this section often represent significant organizational risks.

\begin{table}[h!]
\centering
\caption{Security Controls Questionnaire Results}
\begin{tabular}{@{}p{0.75\linewidth}c@{}}
\toprule
\textbf{Control Question} & \textbf{Response} \\
\midrule
Do you require MFA to access email? & \ding{55} \\
Do you require MFA to log into computers? & \ding{51} \\
Do you require MFA to access sensitive data systems? & \ding{51} \\
Does your organization have an employee acceptable use policy? & \ding{51} \\
Does your organization do security awareness training for new employees? & \ding{51} \\
Does your organization do security awareness training for all employees at least once per year? & \ding{55} \\
\bottomrule
\end{tabular}
\end{table}

\subsection*{Analysis of Control Gaps}
The questionnaire revealed two primary control deficiencies:
\begin{itemize}
    \item \textbf{MFA for Email:} The "No" response to requiring MFA for email is a critical vulnerability. Email accounts are high-value targets for attackers seeking to conduct phishing, spread malware, or gain a foothold within the organization's network.
    \item \textbf{Annual Security Training:} The lack of recurring, annual security awareness training for all staff is a high-risk gap. Threat landscapes evolve, and employee awareness is a key defense against social engineering. A one-time training for new hires is insufficient to maintain a strong security culture.
\end{itemize}

% --- 4. Technical Scan Results ---
\section{Technical Scan Results}

An external network scan was performed to identify exposed services and potential vulnerabilities.

\begin{itemize}
    \item \textbf{Target Host:} \seqsplit{\texttt{2001:db8::1}}
    \item \textbf{Scan Tool:} Nmap
    \item \textbf{Host Status:} Up
\end{itemize}

The following table details the open ports discovered on the target system.

\begin{table}[h!]
\centering
\caption{Open Ports on \seqsplit{\texttt{2001:db8::1}}}
\begin{tabular}{@{}llll@{}}
\toprule
\textbf{Port} & \textbf{State} & \textbf{Service} & \textbf{Notes} \\
\midrule
22/tcp & open & SSH & Secure Shell is a remote administration protocol. \\
& & & If exposed, it is a common target for brute-force \\
& & & attacks. No version information was available. \\
\bottomrule
\end{tabular}
\end{table}

\subsection*{Analysis of Technical Findings}
The presence of an open SSH port (22) on an external-facing system constitutes a medium risk. Without proper controls, this service can be exploited by attackers to gain unauthorized access to the server. Key hardening measures include using key-based authentication, disabling password-based logins, and restricting access to trusted IP addresses via firewall rules.

% --- 5. Consolidated Risk Assessment ---
\section{Consolidated Risk Assessment}
This section synthesizes findings from the security control review and the technical scan into a prioritized list of risks. The organization had no previously documented risks provided for this assessment.

\begin{table}[h!]
\centering
\caption{Summary of Identified Risks}
\begin{tabular}{@{}p{0.1\linewidth}p{0.4\linewidth}p{0.15\linewidth}p{0.25\linewidth}@{}}
\toprule
\textbf{Risk ID} & \textbf{Description} & \textbf{Severity} & \textbf{Affected Asset(s)} \\
\midrule
RISK-001 & Lack of Multi-Factor Authentication (MFA) on the primary email system. & \textbf{Critical} & Email System, User Accounts, Corporate Data \\
\addlinespace
RISK-002 & Inadequate security awareness training program (not performed annually for all employees). & \textbf{High} & All Employees, Organizational Security Culture \\
\addlinespace
RISK-003 & Exposed SSH management port (22/tcp) on an external-facing server. & \textbf{Medium} & Server at \seqsplit{\texttt{2001:db8::1}} \\
\bottomrule
\end{tabular}
\end{table}

% --- 6. Recommendations ---
\section{Recommendations}
The following actions are recommended to mitigate the identified risks and improve the overall security posture of \textbf{Hidden Gem}. Recommendations are prioritized based on risk severity.

\subsection*{Priority 1: Critical}
\begin{description}
    \item[RISK-001:] \textbf{Implement MFA on Email System.}
    \begin{itemize}
        \item \textbf{Action:} Immediately enable and enforce MFA for all user accounts on the \texttt{HiddenGem.net} email platform.
        \item \textbf{Justification:} This is the single most effective control to prevent unauthorized account access and mitigate the risk of business email compromise.
    \end{itemize}
\end{description}

\subsection*{Priority 2: High}
\begin{description}
    \item[RISK-002:] \textbf{Establish Annual Security Awareness Training.}
    \begin{itemize}
        \item \textbf{Action:} Develop and implement a mandatory security awareness training program that all employees must complete at least once per year. The training should cover current threats such as phishing, ransomware, and social engineering.
        \item \textbf{Justification:} A well-informed workforce is a critical layer of defense. Regular training ensures that security remains a top-of-mind concern and that employees are equipped to recognize and report threats.
    \end{itemize}
\end{description}

\subsection*{Priority 3: Medium}
\begin{description}
    \item[RISK-003:] \textbf{Harden Exposed SSH Service.}
    \begin{itemize}
        \item \textbf{Action:} Review the business need for the exposed SSH service on \seqsplit{\texttt{2001:db8::1}}. If it is required, implement the following controls:
        \begin{enumerate}
            \item Restrict access via firewall rules to only trusted source IP addresses.
            \item Disable password-based authentication and enforce the use of strong SSH keys.
            \item Ensure the SSH server software is up-to-date and patched against known vulnerabilities.
        \end{enumerate}
        \item \textbf{Justification:} Hardening the SSH configuration significantly reduces the attack surface and protects the underlying server from common brute-force and credential-stuffing attacks.
    \end{itemize}
\end{description}

\end{document}
```