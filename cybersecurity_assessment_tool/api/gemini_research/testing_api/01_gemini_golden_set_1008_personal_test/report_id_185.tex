```latex
\documentclass[12pt, a4paper]{article}

% Required Packages
\usepackage[margin=1in]{geometry}
\usepackage{pifont} % For checkmarks and crosses
\usepackage{booktabs} % For professional tables
\usepackage{hyperref} % For clickable links
\usepackage{url} % For URL formatting
\usepackage{seqsplit} % To split long text sequences like URLs or IPs
\usepackage{graphicx}
\usepackage[table]{xcolor} % For coloring table cells
\usepackage{fancyhdr}

% --- Document Setup ---
\definecolor{criticalred}{HTML}{D12B2B}
\definecolor{highorange}{HTML}{F28C28}
\definecolor{mediumyellow}{HTML}{FFC72C}
\definecolor{lowblue}{HTML}{2B88D1}
\definecolor{infogray}{HTML}{808080}
\definecolor{tablehead}{gray}{0.9}

\hypersetup{
    colorlinks=true,
    linkcolor=blue,
    filecolor=magenta,      
    urlcolor=cyan,
    pdftitle={Cybersecurity Posture Assessment Report},
    pdfauthor={Cybersecurity Analyst},
    pdfsubject={Security Analysis},
    pdfkeywords={Security, Report, Analysis},
}

\pagestyle{fancy}
\fancyhf{}
\fancyhead[L]{Cybersecurity Posture Assessment}
\fancyhead[R]{\textbf{North Star Education}}
\fancyfoot[C]{\thepage}

\newcommand{\yes}{\ding{51}}
\newcommand{\no}{\ding{55}}

% --- Document Start ---
\begin{document}

\begin{titlepage}
    \centering
    \vspace*{1cm}
    \Huge{\textbf{Cybersecurity Posture Assessment Report}}
    \vspace{1.5cm}
    \Large{\textbf{Prepared for:}} \\
    \vspace{0.5cm}
    \Large{North Star Education}
    \vspace{2cm}
    \large{\textbf{Date of Report:}} \\
    \vspace{0.5cm}
    \large{\today}
    \vfill
    \large{This report contains a comprehensive analysis of the organization's security controls, technical vulnerabilities, and overall risk posture based on the data provided. The findings and recommendations are intended to help improve security and mitigate identified risks.}
\end{titlepage}

\tableofcontents
\newpage

% --- Executive Summary ---
\section{Executive Summary}
This report details the findings of a cybersecurity assessment for \textbf{North Star Education}. The analysis combines a review of organizational security controls, a technical network scan, and pre-existing risk data.

The assessment identified several critical and high-risk issues that require immediate attention. The most significant finding is the systemic exposure of Remote Desktop Protocol (RDP) on internal systems, including a newly discovered instance on host \texttt{10.10.10.51}. This finding corroborates a known, pre-existing critical risk.

This technical vulnerability is severely compounded by critical gaps in organizational security controls, namely the \textbf{lack of Multi-Factor Authentication (MFA) for computer logons and access to sensitive data systems}. Furthermore, the absence of security awareness training for new employees creates a significant vulnerability to social engineering and phishing attacks, which are common precursors to ransomware incidents that exploit exposed RDP.

Collectively, these issues create a high-probability attack path for unauthorized access, lateral movement, and a potential ransomware event. This report provides prioritized, actionable recommendations to mitigate these risks effectively.

% --- Organizational Information ---
\section{Organizational Information}
The following information was provided for the assessment.
\begin{center}
\begin{tabular}{ll}
\toprule
\rowcolor{tablehead}
\textbf{Attribute} & \textbf{Value} \\
\midrule
Organization Name & \textbf{North Star Education} \\
Email Domain & \texttt{NorthStarEducation.com} \\
Website Domain & \seqsplit{\texttt{www.NorthStarEducation.com}} \\
External IP Address & \texttt{98.164.227.245} \\
\bottomrule
\end{tabular}
\end{center}

% --- Security Control Review ---
\section{Security Control Review}
A review of the organization's security questionnaire responses reveals critical gaps in foundational security controls. "No" answers indicate a lack of a necessary control and are highlighted as significant risks.

\begin{center}
\begin{tabular}{p{8cm} c p{4cm}}
\toprule
\rowcolor{tablehead}
\textbf{Control Question} & \textbf{Response} & \textbf{Analyst Note} \\
\midrule
Do you require MFA to access email? & \yes & Good Practice. \\
\rowcolor{criticalred!20}
Do you require MFA to log into computers? & \no & \textbf{Critical Gap.} Lack of endpoint MFA allows credential theft to translate directly into system access. \\
\rowcolor{criticalred!20}
Do you require MFA to access sensitive data systems? & \no & \textbf{Critical Gap.} High-value data is not adequately protected from unauthorized access. \\
Does your organization have an employee acceptable use policy? & \yes & Good Practice. \\
\rowcolor{highorange!20}
Does your organization do security awareness training for new employees? & \no & \textbf{High Risk.} New hires are a primary target for phishing and are untrained on organizational policies. \\
Does your organization do security awareness training for all employees at least once per year? & \yes & Good Practice. \\
\bottomrule
\end{tabular}
\end{center}

% --- Technical Scan Results ---
\section{Technical Scan Results}
A network scan was performed to identify open ports and exposed services on the target system.

\subsection{Scan Details}
\begin{itemize}
    \item \textbf{Target IP Address:} \texttt{10.10.10.51}
    \item \textbf{Scan Tool:} Nmap
\end{itemize}

\subsection{Open Ports and Services}
The scan identified the following open port on the target host.

\begin{center}
\begin{tabular}{c c c p{6cm}}
\toprule
\rowcolor{tablehead}
\textbf{Port} & \textbf{State} & \textbf{Service Name} & \textbf{Analyst Note} \\
\midrule
\rowcolor{criticalred!20}
3389/tcp & Open & \texttt{ms-wbt-server} & This is the standard port for Microsoft Remote Desktop Protocol (RDP). Exposed RDP is a primary vector for ransomware attacks and unauthorized access. \\
\bottomrule
\end{tabular}
\end{center}

% --- Consolidated Risk Assessment ---
\section{Consolidated Risk Assessment}
The following table synthesizes findings from the security control review, technical scan, and pre-existing risk data into a prioritized list of risks.

\begin{center}
\begin{tabular}{p{4cm} p{7.5cm} c}
\toprule
\rowcolor{tablehead}
\textbf{Risk Name} & \textbf{Description} & \textbf{Severity} \\
\midrule
\rowcolor{criticalred!20}
\textbf{Systemic RDP Exposure} & The technical scan found RDP open on \texttt{10.10.10.51}. This adds to a pre-existing finding on \texttt{10.10.10.50}, indicating a systemic issue. RDP is a high-value target for attackers. & \cellcolor{criticalred} \color{white} \textbf{Critical} \\
\midrule
\rowcolor{criticalred!20}
\textbf{Lack of Endpoint MFA} & The absence of MFA on computer logons, combined with exposed RDP, creates a direct path for an attacker with stolen credentials to gain system access and move laterally. & \cellcolor{criticalred} \color{white} \textbf{Critical} \\
\midrule
\rowcolor{criticalred!20}
\textbf{Lack of MFA for Sensitive Data} & Sensitive data systems lack a critical authentication control, making them highly vulnerable to compromise if an attacker gains a foothold in the network. & \cellcolor{criticalred} \color{white} \textbf{Critical} \\
\midrule
\rowcolor{highorange!20}
\textbf{Inadequate New-Hire Security Training} & New employees are not trained on security best practices, making them more susceptible to phishing and social engineering attacks that could lead to credential compromise. & \cellcolor{highorange} \color{white} \textbf{High} \\
\bottomrule
\end{tabular}
\end{center}

% --- Recommendations ---
\section{Recommendations}
The following prioritized recommendations are provided to mitigate the identified risks.

\subsection{Immediate Actions (To Be Completed in 0-7 Days)}
\begin{enumerate}
    \item \textbf{Contain RDP Exposure:}
    \begin{itemize}
        \item Immediately investigate the business need for RDP access on \texttt{10.10.10.51} and \texttt{10.10.10.50}.
        \item If access is not required, \textbf{disable the service and close port 3389}.
        \item If required, implement strict firewall rules to restrict RDP access to only trusted, whitelisted IP addresses.
    \end{itemize}
\end{enumerate}

\subsection{Short-Term Actions (To Be Completed in 1-3 Months)}
\begin{enumerate}
    \item \textbf{Deploy Multi-Factor Authentication (MFA):}
    \begin{itemize}
        \item Prioritize and enforce MFA for all user accounts for computer/endpoint logons. Start with administrative and privileged accounts, then expand to all users.
        \item Enforce MFA for access to all systems identified as containing sensitive data.
    \end{itemize}
    \item \textbf{Implement New-Hire Training:}
    \begin{itemize}
        \item Develop and integrate a mandatory security awareness training module into the new employee onboarding process. This training should cover phishing, acceptable use, and password security.
    \end{itemize}
\end{enumerate}

\subsection{Long-Term Strategic Actions (To Be Completed in 3-6 Months)}
\begin{enumerate}
    \item \textbf{Implement a Secure Remote Access Solution:}
    \begin{itemize}
        \item Deploy a Virtual Private Network (VPN) solution for all remote administrative access. All remote management protocols, including RDP, should only be accessible through the VPN tunnel, never directly from the internet or across untrusted network segments.
    \end{itemize}
    \item \textbf{Conduct Comprehensive Vulnerability Scanning:}
    \begin{itemize}
        \item Perform a credentialed vulnerability scan across the entire internal network to identify other misconfigurations, outdated software, and security weaknesses.
    \end{itemize}
\end{enumerate}

\end{document}
```