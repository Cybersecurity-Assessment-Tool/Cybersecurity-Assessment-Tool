```latex
\documentclass[12pt]{article}

% ----------------------------------------------------------------------
% PREAMBLE
% ----------------------------------------------------------------------
\usepackage[margin=1in]{geometry}
\usepackage{pifont} % Required for \ding
\usepackage{booktabs} % For professional-looking tables
\usepackage{hyperref} % For clickable links and references
\usepackage{url}      % For properly formatting URLs
\usepackage{seqsplit} % To split long strings in \texttt
\usepackage{graphicx}
\usepackage{xcolor}
\usepackage{array}

% Hyperlink setup
\hypersetup{
    colorlinks=true,
    linkcolor=black,
    urlcolor=blue,
}

% Define custom colors for severity levels
\definecolor{sevCritical}{HTML}{990000}
\definecolor{sevHigh}{HTML}{DD4B39}
\definecolor{sevMedium}{HTML}{F4B400}
\definecolor{sevLow}{HTML}{4285F4}

% Define a new column type for wrapped text in tables
\newcolumntype{L}[1]{>{\raggedright\let\newline\\\arraybackslash\hspace{0pt}}m{#1}}

% ----------------------------------------------------------------------
% DOCUMENT START
% ----------------------------------------------------------------------
\begin{document}

\title{Cybersecurity Posture Assessment Report \\ \large For: Blue Marble}
\author{Cybersecurity Analysis Division}
\date{\today}
\maketitle

\begin{abstract}
This report provides a comprehensive analysis of the cybersecurity posture for Blue Marble. The assessment is based on a synthesis of technical network scans, a review of organizational security controls, and an evaluation of pre-existing risk data. The analysis identified several high-impact risks, including a newly discovered system with Remote Desktop Protocol (RDP) exposed to the network, which compounds a previously identified issue. This technical vulnerability is critically exacerbated by organizational policy gaps, namely the lack of multi-factor authentication (MFA) for sensitive systems and the absence of security awareness training for new employees. Immediate remediation is strongly recommended to mitigate the significant threat of unauthorized access and potential compromise.
\end{abstract}

\newpage

% ----------------------------------------------------------------------
% TABLE OF CONTENTS
% ----------------------------------------------------------------------
\tableofcontents
\newpage

% ----------------------------------------------------------------------
% SECTION 1: ORGANIZATIONAL INFORMATION
% ----------------------------------------------------------------------
\section{Organizational Information}

This section outlines the organizational details provided for this assessment. This information is used to establish the context and scope of the review.

\begin{tabular}{@{}ll}
\toprule
\textbf{Attribute} & \textbf{Value} \\
\midrule
Organization Name & \textbf{Blue Marble} \\
Email Domain & \texttt{BlueMarble.net} \\
Website Domain & \url{www.BlueMarble.net} \\
External IP Address & \texttt{137.7.4.254} \\
\bottomrule
\end{tabular}

% ----------------------------------------------------------------------
% SECTION 2: SECURITY CONTROL REVIEW
% ----------------------------------------------------------------------
\section{Security Control Review}

The following table summarizes the organization's responses to a security controls questionnaire. "No" answers indicate significant gaps in the defensive posture and are flagged for immediate attention.

\begin{tabular}{@{} L{8cm} c L{5cm} @{}}
\toprule
\textbf{Control Question} & \textbf{Response} & \textbf{Analyst Note} \\
\midrule
Do you require MFA to access email? & \ding{51} & Good practice. Mitigates email account takeover. \\
\addlinespace
Do you require MFA to log into computers? & \ding{51} & Strong control for endpoint security. \\
\addlinespace
\textbf{Do you require MFA to access sensitive data systems?} & \textbf{\color{red}\ding{55}} & \textbf{Critical Gap.} Lack of MFA on critical systems drastically increases the risk of a breach via stolen credentials. \\
\addlinespace
Does your organization have an employee acceptable use policy? & \ding{51} & Foundational policy is in place. \\
\addlinespace
\textbf{Does your organization do security awareness training for new employees?} & \textbf{\color{red}\ding{55}} & \textbf{High Risk.} New hires are a common target for phishing and social engineering. Lack of initial training leaves the organization vulnerable. \\
\addlinespace
Does your organization do security awareness training for all employees at least once per year? & \ding{51} & Good practice for maintaining security awareness. \\
\bottomrule
\end{tabular}

% ----------------------------------------------------------------------
% SECTION 3: TECHNICAL SCAN RESULTS
% ----------------------------------------------------------------------
\section{Technical Scan Results}

A network scan was performed to identify open ports and exposed services on the target system.

\begin{itemize}
    \item \textbf{Target IP Address:} \texttt{10.10.10.51}
    \item \textbf{Scan Date:} Assumed to be current as of report generation.
\end{itemize}

The scan revealed the following open port:

\begin{tabular}{@{} l l l L{6cm} @{}}
\toprule
\textbf{Port} & \textbf{State} & \textbf{Service Name} & \textbf{Finding} \\
\midrule
3389/tcp & Open & \texttt{ms-wbt-server} & The Remote Desktop Protocol (RDP) service is exposed. This is a high-value target for attackers and is frequently exploited for initial access, often leading to ransomware deployment. \\
\bottomrule
\end{tabular}

\subsection{Correlation with Existing Risks}
This finding is particularly concerning as it indicates a systemic issue. A pre-existing risk was documented for RDP exposure on a different host (\texttt{10.10.10.50}). The discovery of a second exposed host suggests that network segmentation and firewall egress/ingress rules may not be consistently applied or enforced.

% ----------------------------------------------------------------------
% SECTION 4: CONSOLIDATED RISK ASSESSMENT
% ----------------------------------------------------------------------
\section{Consolidated Risk Assessment}

This section synthesizes findings from the security control review, technical scan, and pre-existing risk data into a consolidated list of key risks facing the organization.

\begin{tabular}{@{} l L{4cm} L{3.5cm} L{5cm} @{}}
\toprule
\textbf{Severity} & \textbf{Risk Title} & \textbf{Affected Asset(s)} & \textbf{Description} \\
\midrule
\textbf{\color{sevCritical}Critical} & Newly Discovered RDP Exposure & Host: \texttt{10.10.10.51} & Open RDP port allows direct brute-force or credential-stuffing attacks. This is a primary vector for ransomware. \\
\addlinespace
\textbf{\color{sevCritical}Critical} & Pre-Existing RDP Exposure & Host: \texttt{10.10.10.50} & A known, unmitigated critical risk. The presence of multiple exposed hosts elevates the overall organizational risk significantly. \\
\addlinespace
\textbf{\color{sevHigh}High} & Lack of MFA for Sensitive Systems & Organization-wide & The absence of MFA on critical systems means that a single compromised password could lead to a major data breach. This directly exacerbates the RDP exposure risk. \\
\addlinespace
\textbf{\color{sevHigh}High} & No Security Training for New Hires & Organization-wide & New employees are highly susceptible to phishing attacks. This gap makes it easier for attackers to steal credentials needed to exploit other vulnerabilities like exposed RDP. \\
\bottomrule
\end{tabular}

% ----------------------------------------------------------------------
% SECTION 5: RECOMMENDATIONS
% ----------------------------------------------------------------------
\section{Recommendations}

Based on the analysis, the following actions are recommended to improve the security posture of Blue Marble. Recommendations are prioritized to address the most critical risks first.

\subsection{Immediate Priority (0-7 Days)}
\begin{enumerate}
    \item \textbf{Remediate RDP Exposure:} Immediately close port 3389 on hosts \texttt{10.10.10.51} and \texttt{10.10.10.50} to all external traffic. If remote access is required, it must be placed behind a secure gateway such as a VPN.
    \item \textbf{Conduct Emergency Scan:} Perform a comprehensive scan of the entire network perimeter and internal subnets to identify any other instances of exposed RDP or other risky services.
\end{enumerate}

\subsection{Short-Term Priority (1-3 Months)}
\begin{enumerate}
    \item \textbf{Deploy Multi-Factor Authentication (MFA):} Prioritize the rollout of MFA for all access to sensitive data systems, administrative interfaces, and remote access solutions (e.g., VPN). This is the single most effective control to mitigate the risk of compromised credentials.
    \item \textbf{Implement New Hire Security Training:} Develop and mandate a security awareness training module as part of the onboarding process for all new employees. This training should cover phishing, acceptable use, and password hygiene.
\end{enumerate}

\subsection{Long-Term Priority (3-12 Months)}
\begin{enumerate}
    \item \textbf{Formalize Remote Access Strategy:} Implement a formal, secure remote access solution like a Zero Trust Network Access (ZTNA) gateway or a fully managed VPN. Prohibit direct, unauthenticated access to internal services from the internet.
    \item \textbf{Review Network Architecture and Segmentation:} Conduct a thorough review of the network architecture. Implement network segmentation to isolate critical systems, preventing lateral movement in the event of a breach.
\end{enumerate}

% ----------------------------------------------------------------------
% DOCUMENT END
% ----------------------------------------------------------------------
\end{document}
```