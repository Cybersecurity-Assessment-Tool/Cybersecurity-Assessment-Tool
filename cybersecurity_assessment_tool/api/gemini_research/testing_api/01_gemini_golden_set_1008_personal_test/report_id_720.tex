Of course. As an expert-level Cybersecurity Analyst and LaTeX Report Generator, I will analyze the provided data, handle the corrupted inputs professionally, and synthesize the findings into a complete and formal LaTeX report.

Here is the generated LaTeX code:

```latex
\documentclass[12pt]{article}

% --------------------------------------------------------------------------------
% PREAMBLE
% --------------------------------------------------------------------------------
\usepackage[margin=1in]{geometry}
\usepackage{pifont} % For checkmarks and crosses
\usepackage{booktabs} % For professional tables
\usepackage{hyperref} % For clickable links and table of contents
\usepackage{url}      % For formatting URLs
\usepackage{seqsplit} % To split long strings without breaking
\usepackage{graphicx} % For potential logos in the future
\usepackage{xcolor}   % For colors

% --- Hyperref Setup ---
\hypersetup{
    colorlinks=true,
    linkcolor=blue,
    filecolor=magenta,      
    urlcolor=cyan,
    pdftitle={Cybersecurity Posture Assessment},
    pdfpagemode=FullScreen,
}

% --- Title Information ---
\title{
    \vspace{-2cm}
    \fontsize{22pt}{26pt}\selectfont
    \textbf{Cybersecurity Posture Assessment Report}
    \vspace{1cm}
}
\author{Cybersecurity Analysis Division}
\date{\today}

% --------------------------------------------------------------------------------
% DOCUMENT START
% --------------------------------------------------------------------------------
\begin{document}

\maketitle
\thispagestyle{empty}
\newpage

\tableofcontents
\newpage

% --------------------------------------------------------------------------------
% SECTION 1: EXECUTIVE SUMMARY
% --------------------------------------------------------------------------------
\section{Executive Summary}

This report provides a cybersecurity posture assessment for \textbf{Nova Terra}. The analysis is based on a review of organizational security controls provided via a questionnaire. 

It is critical to note that the technical network scan data (\texttt{Input\_1\_Network\_Scan\_JSON}) and the list of pre-existing risks (\texttt{Input\_3\_Current\_Risks\_JSON}) were found to be corrupted or unavailable during this assessment. Consequently, this report focuses exclusively on the policy and procedure gaps identified from the organizational data. A comprehensive technical vulnerability assessment could not be performed.

The primary findings indicate significant risks related to user access and security awareness. The most critical vulnerability is the \textbf{absence of Multi-Factor Authentication (MFA) on employee email accounts}. This exposes the organization to a high risk of business email compromise (BEC), phishing, and subsequent data breaches.

Furthermore, the lack of a formal security awareness training program for both new and existing employees constitutes a high risk. This gap leaves the organization vulnerable to social engineering and other human-centric attacks.

While some positive controls are in place, such as MFA for computer and sensitive system access, the identified gaps require immediate attention to mitigate potentially severe impacts on the organization's security and operations. Recommendations for remediation are detailed in Section 6.

% --------------------------------------------------------------------------------
% SECTION 2: ORGANIZATIONAL INFORMATION
% --------------------------------------------------------------------------------
\section{Organizational Information}

The following details were provided for the assessment scope.

\begin{table}[h!]
\centering
\begin{tabular}{@{}ll@{}}
\toprule
\textbf{Attribute} & \textbf{Value} \\ \midrule
Organization Name    & Nova Terra \\
Email Domain         & \seqsplit{\texttt{NovaTerra.net}} \\
Website Domain       & \seqsplit{\texttt{www.NovaTerra.net}} \\
Primary External IP  & \seqsplit{\texttt{30.167.199.75}} \\ \bottomrule
\end{tabular}
\caption{Client Organizational Details}
\label{tab:org_info}
\end{table}

% --------------------------------------------------------------------------------
% SECTION 3: SECURITY CONTROL REVIEW
% --------------------------------------------------------------------------------
\section{Security Control Review}

The following table summarizes the organization's responses to the security controls questionnaire and provides an initial assessment of each response.

\begin{table}[h!]
\centering
\begin{tabular}{@{}p{0.5\linewidth} c p{0.25\linewidth}@{}}
\toprule
\textbf{Control Question} & \textbf{Response} & \textbf{Analyst Assessment} \\ \midrule
Do you require MFA to access email? & \ding{55} & \textcolor{red}{\textbf{Critical Gap}} \\
Do you require MFA to log into computers? & \ding{51} & Meets Best Practice \\
Do you require MFA to access sensitive data systems? & \ding{51} & Meets Best Practice \\
Does your organization have an employee acceptable use policy? & \ding{51} & Meets Best Practice \\
Does your organization do security awareness training for new employees? & \ding{55} & \textcolor{orange}{\textbf{High Risk}} \\
Does your organization do security awareness training for all employees at least once per year? & \ding{55} & \textcolor{orange}{\textbf{High Risk}} \\ \bottomrule
\end{tabular}
\caption{Security Controls Questionnaire Analysis}
\label{tab:controls}
\end{table}

% --------------------------------------------------------------------------------
% SECTION 4: TECHNICAL SCAN RESULTS
% --------------------------------------------------------------------------------
\section{Technical Scan Results}

The data file intended to contain the results of the network vulnerability scan (\texttt{Input\_1\_Network\_Scan\_JSON}) was found to be corrupted and could not be parsed. Therefore, no technical analysis of open ports, running services, or software versions could be performed for the target IP address (\seqsplit{\texttt{30.167.199.75}}).

\textbf{Assessment Impact:} Without this data, the organization's external attack surface remains unassessed. There may be unpatched services, insecure configurations, or other vulnerabilities exposed to the internet that present a direct threat. It is strongly recommended to conduct a new scan as a matter of priority.

% --------------------------------------------------------------------------------
% SECTION 5: RISK ASSESSMENT
% --------------------------------------------------------------------------------
\section{Risk Assessment}

This section summarizes the key risks identified based on the available data. Note that the pre-existing risk data (\texttt{Input\_3\_Current\_Risks\_JSON}) was unavailable and is not included.

\begin{table}[h!]
\centering
\begin{tabular}{@{}p{0.1\linewidth} p{0.25\linewidth} p{0.4\linewidth} p{0.1\linewidth}@{}}
\toprule
\textbf{Risk ID} & \textbf{Risk Name} & \textbf{Description} & \textbf{Severity} \\ \midrule
R-01 & Lack of MFA on Email & Email accounts are vulnerable to compromise via phishing or credential stuffing. This can lead to business email compromise (BEC), data exfiltration, and further internal network pivoting. & \textcolor{red}{\textbf{Critical}} \\ \addlinespace
R-02 & Inadequate Security Awareness Training & Employees are not formally trained to recognize or respond to social engineering, phishing, or other common cyber threats. This significantly increases the likelihood of an employee-triggered security incident. & \textcolor{orange}{\textbf{High}} \\ \bottomrule
\end{tabular}
\caption{Summary of Identified Risks}
\label{tab:risks}
\end{table}

% --------------------------------------------------------------------------------
% SECTION 6: RECOMMENDATIONS
% --------------------------------------------------------------------------------
\section{Recommendations}

The following prioritized recommendations are provided to address the identified risks and improve the overall security posture of \textbf{Nova Terra}.

\subsection*{Priority 1 (Critical): Implement MFA on Email}
\begin{itemize}
    \item \textbf{Action:} Immediately enforce Multi-Factor Authentication (MFA) across all user email accounts. This is the single most effective control to prevent unauthorized account access and mitigate the risk of business email compromise.
    \item \textbf{Justification:} Addresses risk \textbf{R-01}. An attacker with valid credentials cannot access an account without the second factor, drastically reducing the impact of phishing and password reuse attacks.
\end{itemize}

\subsection*{Priority 2 (High): Establish Security Awareness Training Program}
\begin{itemize}
    \item \textbf{Action:} Develop and implement a mandatory security awareness training program. This program should be required for all new hires during onboarding and conducted annually for all existing employees.
    \item \textbf{Justification:} Addresses risk \textbf{R-02}. Training should cover key topics such as phishing identification, password security, acceptable use, and incident reporting. This builds a "human firewall" and reduces susceptibility to social engineering.
\end{itemize}

\subsection*{Priority 3 (Informational): Conduct Network Vulnerability Scan}
\begin{itemize}
    \item \textbf{Action:} Commission a new, comprehensive external network vulnerability scan against the public-facing IP address \seqsplit{\texttt{30.167.199.75}}.
    \item \textbf{Justification:} The original scan data was unusable. A new scan is essential to identify and remediate any technical vulnerabilities (e.g., unpatched software, weak encryption) on the network perimeter before they can be exploited by attackers.
\end{itemize}

% --------------------------------------------------------------------------------
% DOCUMENT END
% --------------------------------------------------------------------------------
\end{document}
```