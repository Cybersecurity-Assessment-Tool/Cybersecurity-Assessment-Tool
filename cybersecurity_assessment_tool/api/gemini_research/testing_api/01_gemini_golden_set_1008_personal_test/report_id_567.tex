```latex
\documentclass[12pt]{article}

% Preamble: Required Packages
\usepackage[margin=1in]{geometry}
\usepackage{pifont} % For checkmarks and crosses
\usepackage{booktabs} % For professional tables
\usepackage{hyperref} % For clickable links
\usepackage{url} % For URL formatting
\usepackage{seqsplit} % For splitting long strings
\usepackage{xcolor} % For color
\usepackage{graphicx} % For images, if needed
\usepackage{fancyhdr} % For headers/footers

% --- Document Setup ---
\hypersetup{
    colorlinks=true,
    linkcolor=blue,
    filecolor=magenta,      
    urlcolor=cyan,
    pdftitle={Cybersecurity Posture Report},
    pdfauthor={Cybersecurity Analyst},
    pdfsubject={Security Assessment},
    pdfkeywords={Cybersecurity, Risk, Analysis},
    bookmarks=true
}

% Define severity colors
\definecolor{criticalred}{HTML}{990000}
\definecolor{highorange}{HTML}{E25822}
\definecolor{mediumyellow}{HTML}{F39C12}
\definecolor{lowblue}{HTML}{3498DB}

% Page style
\pagestyle{fancy}
\fancyhf{}
\lhead{Cybersecurity Posture Report}
\rhead{Sovereign Trust}
\cfoot{\thepage}

% --- Document Start ---
\begin{document}

% --- Title Page ---
\begin{titlepage}
    \centering
    \vspace*{1cm}
    \Huge{\textbf{Cybersecurity Posture Report}}
    \vspace{0.5cm}
    \Large{Confidential Assessment}
    \vspace{1.5cm}
    
    \includegraphics[width=0.4\textwidth]{example-image-a} % Placeholder for company logo
    
    \vfill
    
    \Large{\textbf{Prepared for:}} \\
    \vspace{0.2cm}
    Sovereign Trust
    
    \vspace{1cm}
    
    \Large{\textbf{Date of Report:}} \\
    \vspace{0.2cm}
    \today
    
    \vspace{1cm}
    
    \Large{\textbf{Generated by:}} \\
    \vspace{0.2cm}
    Expert Cybersecurity Analyst
    
\end{titlepage}

\tableofcontents
\newpage

% --- Section 1: Executive Summary ---
\section{Executive Summary}
This report provides a comprehensive analysis of the cybersecurity posture for \textbf{Sovereign Trust}. The assessment is based on a correlation of organizational data, a technical network scan, and a review of existing risks.

The overall security posture is assessed as \textbf{\textcolor{criticalred}{CRITICAL}}. Several significant, high-impact risks were identified that require immediate attention. The most critical findings include:
\begin{itemize}
    \item \textbf{Lack of Multi-Factor Authentication (MFA):} The complete absence of MFA across email, computer logins, and sensitive data systems represents a severe security gap, leaving the organization highly vulnerable to credential theft and unauthorized access.
    \item \textbf{Exposed and Outdated Database:} A MySQL database server is directly exposed to the network. The running version, MySQL 5.7.33, is End-of-Life (EOL) and no longer receives security updates, making it a prime target for exploitation.
    \item \textbf{Inadequate Security Training:} The lack of mandatory, annual security awareness training for all employees increases susceptibility to phishing, social engineering, and other human-targeted attacks.
\end{itemize}
These issues, when combined, create a high-risk environment. An attacker could potentially compromise an employee's credentials and gain direct access to a vulnerable, mission-critical database. Immediate remediation of the recommendations outlined in Section \ref{sec:recommendations} is strongly advised.

% --- Section 2: Organizational Information ---
\section{Organizational Information}
The following details were provided for the assessment.

\begin{tabular}{@{}ll}
    \toprule
    \textbf{Attribute} & \textbf{Value} \\
    \midrule
    Organization Name & Sovereign Trust \\
    Email Domain & \texttt{SovereignTrust.org} \\
    Website Domain & \url{www.SovereignTrust.org} \\
    External IP Address & \texttt{204.197.208.158} \\
    \bottomrule
\end{tabular}

% --- Section 3: Security Control Review ---
\section{Security Control Review}
A review of the organization's security controls was conducted based on a standard questionnaire. The results highlight critical gaps in identity and access management and employee security awareness.

\begin{table}[h!]
\centering
\caption{Security Controls Questionnaire Analysis}
\label{tab:controls}
\begin{tabular}{@{}p{0.6\linewidth} c p{0.3\linewidth}@{}}
    \toprule
    \textbf{Control Question} & \textbf{Response} & \textbf{Analyst Notes} \\
    \midrule
    Do you require MFA to access email? & \textcolor{criticalred}{\ding{55}} & Critical risk. A primary vector for Business Email Compromise (BEC). \\
    \addlinespace
    Do you require MFA to log into computers? & \textcolor{criticalred}{\ding{55}} & High risk. Weakens endpoint security and lateral movement defense. \\
    \addlinespace
    Do you require MFA to access sensitive data systems? & \textcolor{criticalred}{\ding{55}} & Critical risk. Allows single-factor access to critical assets like databases. \\
    \addlinespace
    Does your organization have an employee acceptable use policy? & \textcolor{green}{\ding{51}} & Good. Establishes a baseline for security expectations. \\
    \addlinespace
    Does your organization do security awareness training for new employees? & \textcolor{green}{\ding{51}} & Good. New hires are onboarded with security in mind. \\
    \addlinespace
    Does your organization do security awareness training for all employees at least once per year? & \textcolor{highorange}{\ding{55}} & High risk. Security skills decay over time; regular training is essential. \\
    \bottomrule
\end{tabular}
\end{table}

% --- Section 4: Technical Scan Results ---
\section{Technical Scan Results}
A network scan was performed on the target system to identify open ports and exposed services.

\begin{itemize}
    \item \textbf{Target IP Address:} \texttt{172.16.50.20}
\end{itemize}

\subsection{Open Ports}
The following open port was discovered on the target system.

\begin{table}[h!]
\centering
\caption{Discovered Open Ports}
\label{tab:ports}
\begin{tabular}{@{}lllll@{}}
    \toprule
    \textbf{Port} & \textbf{State} & \textbf{Service} & \textbf{Product} & \textbf{Version} \\
    \midrule
    3306/tcp & open & mysql & MySQL & 5.7.33 \\
    \bottomrule
\end{tabular}
\end{table}

\subsection{Analyst Findings}
The scan identified a MySQL database service running on port 3306. This presents two significant risks:
\begin{enumerate}
    \item \textbf{Service Exposure:} Database ports should not be exposed to general network traffic. Access should be restricted via firewall rules to only trusted application servers.
    \item \textbf{Outdated Software:} The detected version, \textbf{MySQL 5.7}, reached its official \textbf{End-of-Life (EOL) in October 2023}. This means it no longer receives security patches from the vendor. The specific version, 5.7.33, was released in 2021 and is missing years of critical security updates, making it highly vulnerable to known exploits.
\end{enumerate}

% --- Section 5: Correlated Risk Assessment ---
\section{Correlated Risk Assessment}
This section synthesizes findings from the security control review, technical scan, and pre-existing risk data into a prioritized list of organizational risks.

\begin{table}[h!]
\centering
\caption{Summary of Key Risks}
\label{tab:risks}
\begin{tabular}{@{}p{0.2\linewidth} p{0.6\linewidth} l@{}}
    \toprule
    \textbf{Risk Name} & \textbf{Description} & \textbf{Severity} \\
    \midrule
    \addlinespace
    Database Exposure and Compromise & A MySQL database running an End-of-Life version (5.7.33) is publicly exposed on port 3306. This is compounded by the lack of MFA for sensitive systems, creating a direct path for attackers to access critical data. & \textcolor{criticalred}{\textbf{Critical}} \\
    \addlinespace
    Credential Compromise \& Account Takeover & The absence of MFA on all critical systems (email, endpoints, applications) makes the organization highly susceptible to account takeover via phishing or password spraying attacks. & \textcolor{criticalred}{\textbf{Critical}} \\
    \addlinespace
    Reduced Employee Security Vigilance & Security awareness training is not conducted annually for all employees. This increases the likelihood of successful social engineering and phishing attacks, which are often the entry point for major breaches. & \textcolor{highorange}{\textbf{High}} \\
    \bottomrule
\end{tabular}
\end{table}

% --- Section 6: Recommendations ---
\section{Recommendations}
\label{sec:recommendations}
The following actions are recommended to mitigate the identified risks. They are prioritized based on impact and urgency.

\subsection{Immediate Priority ( remediate within 72 hours)}
\begin{enumerate}
    \item \textbf{Isolate the Database Server:} Immediately implement strict firewall rules to deny all access to port 3306 on server \texttt{172.16.50.20} from any source except for explicitly authorized application servers. Public access must be disabled.
    \item \textbf{Enforce Multi-Factor Authentication (MFA):} Begin immediate rollout of MFA. Prioritize as follows:
        \begin{itemize}
            \item All administrative accounts and access to sensitive systems (including databases).
            \item All user access to email (\texttt{SovereignTrust.org}).
            \item All remote access systems (e.g., VPN).
            \item All computer logins.
        \end{itemize}
\end{enumerate}

\subsection{High Priority (remediate within 30 days)}
\begin{enumerate}
    \setcounter{enumi}{2} % Continue numbering
    \item \textbf{Upgrade Outdated Database Software:} Plan and execute the migration of the MySQL 5.7 database to a currently supported version (e.g., MySQL 8.x). This is critical for receiving ongoing security patches.
    \item \textbf{Implement Annual Security Training:} Establish a mandatory annual security awareness training program for all employees. This program should cover phishing, password hygiene, and acceptable use policies.
\end{enumerate}

\subsection{Medium Priority (remediate within 90 days)}
\begin{enumerate}
    \setcounter{enumi}{4} % Continue numbering
    \item \textbf{Conduct a Comprehensive Vulnerability Scan:} Perform an authenticated vulnerability scan across all internal and external assets to identify other outdated software or misconfigurations.
    \item \textbf{Review Access Control Policies:} Conduct a full audit of user permissions on sensitive systems to ensure the principle of least privilege is strictly enforced.
\end{enumerate}

% --- Section 7: Conclusion ---
\section{Conclusion}
The current security posture of \textbf{Sovereign Trust} contains critical, high-impact vulnerabilities that must be addressed urgently. The combination of an exposed, end-of-life database and a complete lack of Multi-Factor Authentication creates a significant risk of a data breach.

By implementing the prioritized recommendations in this report, Sovereign Trust can substantially reduce its attack surface, improve its resilience against common cyber threats, and build a stronger foundation for its long-term security strategy.

\end{document}
```