```latex
\documentclass[12pt]{article}

% Preamble: Required Packages
\usepackage[margin=1in]{geometry}
\usepackage{pifont} % For checkmarks and crosses (\ding)
\usepackage{booktabs} % For professional tables
\usepackage{hyperref} % For clickable links
\usepackage{url} % For URL formatting
\usepackage{seqsplit} % For splitting long strings in \texttt
\usepackage{graphicx}
\usepackage{xcolor}

% Document Metadata
\title{Cybersecurity Assessment Report}
\author{Cybersecurity Analysis Division}
\date{November 22, 2025}

% Hyperref Setup
\hypersetup{
    colorlinks=true,
    linkcolor=blue,
    filecolor=magenta,      
    urlcolor=cyan,
    pdftitle={Cybersecurity Assessment Report},
    pdfpagemode=FullScreen,
}

\begin{document}

\maketitle
\hrule
\vspace{1cm}

\section*{1. Executive Overview}

This report details a cybersecurity assessment for \textbf{Maple Leaf Logistics}, conducted on November 22, 2025. The analysis correlates findings from a network scan, a security controls questionnaire, and a review of existing risks.

The assessment reveals several critical and high-risk vulnerabilities that require immediate attention. The most significant findings include a \textbf{complete lack of Multi-Factor Authentication (MFA)} across all critical systems, including email and sensitive data repositories. This represents a severe gap in access control.

Furthermore, technical scanning identified a public-facing web server running a \textbf{significantly outdated and vulnerable version of nginx (1.18.0)}. This outdated software exposes the organization to numerous publicly known exploits. These issues, combined with gaps in the security training process for new employees, place the organization at a high risk of unauthorized access, data breach, and ransomware attacks.

Immediate remediation of the identified risks is strongly recommended to improve the organization's security posture.

\section*{2. Organizational Information}

The following information was provided for the assessment.

\begin{itemize}
    \item \textbf{Organization Name:} Maple Leaf Logistics
    \item \textbf{Email Domain:} \texttt{MapleLeafLogistics.com}
    \item \textbf{Website Domain:} \url{www.MapleLeafLogistics.com}
    \item \textbf{External IP Address:} \texttt{4.56.70.108}
\end{itemize}

\section*{3. Security Control Review (Questionnaire)}

The following table summarizes the organization's responses to a security controls questionnaire. Answers marked with a cross (\ding{55}) indicate significant gaps in security best practices and are addressed in the Risk Assessment section.

\begin{table}[h!]
\centering
\begin{tabular}{p{0.75\textwidth} c}
\toprule
\textbf{Control Question} & \textbf{Response} \\
\midrule
Does your organization have an employee acceptable use policy? & \ding{51} \\
Does your organization do security awareness training for all employees at least once per year? & \ding{51} \\
Do you require MFA to access email? & \textcolor{red}{\ding{55}} \\
Do you require MFA to log into computers? & \textcolor{red}{\ding{55}} \\
Do you require MFA to access sensitive data systems? & \textcolor{red}{\ding{55}} \\
Does your organization do security awareness training for new employees? & \textcolor{red}{\ding{55}} \\
\bottomrule
\end{tabular}
\caption{Security Controls Questionnaire Results}
\end{table}

\section*{4. Technical Scan Results}

An Nmap scan was performed against the target IP address \texttt{192.168.10.5} on November 22, 2025. The scan identified one open port with a service running an outdated software version.

\begin{table}[h!]
\centering
\begin{tabular}{l l l l l}
\toprule
\textbf{Port} & \textbf{State} & \textbf{Service} & \textbf{Product} & \textbf{Version} \\
\midrule
443/tcp & open & https & nginx & 1.18.0 \\
\bottomrule
\end{tabular}
\caption{Open Port and Service Information}
\end{table}

\subsection*{Analysis of Technical Findings}
\begin{itemize}
    \item \textbf{Outdated Nginx Server:} The web server is running nginx version 1.18.0, which was released in April 2020. This version is no longer supported and has multiple publicly disclosed vulnerabilities (CVEs) that could be exploited by attackers to achieve remote code execution, denial of service, or information disclosure.
    \item \textbf{SSL Certificate Mismatch:} The scan revealed the server's SSL certificate was issued for the Common Name \texttt{www.acme-corp.com}, which does not match the organization's domain. This misconfiguration can cause browser trust errors and may indicate a placeholder or improperly configured certificate.
\end{itemize}

\section*{5. Risk Assessment Summary}

The following table synthesizes findings from the security questionnaire and technical scan. As no pre-existing risks were provided, this assessment is based entirely on new discoveries. Each risk has been assigned a severity level based on its potential impact and likelihood of exploitation.

\begin{table}[h!]
\centering
\begin{tabular}{p{0.1\textwidth} p{0.25\textwidth} p{0.5\textwidth} p{0.1\textwidth}}
\toprule
\textbf{ID} & \textbf{Risk Title} & \textbf{Overview} & \textbf{Severity} \\
\midrule
\textbf{RISK-01} & No MFA on Sensitive Systems & The absence of MFA on systems holding sensitive data is a critical failure. A single compromised password would be sufficient for an attacker to access the organization's most valuable information assets. & \textbf{Critical} \\
\addlinespace
\textbf{RISK-02} & Outdated \& Vulnerable Web Server & The web server at \texttt{192.168.10.5} runs nginx 1.18.0, a version with numerous known high-severity vulnerabilities. This exposes the organization to opportunistic and targeted attacks. & \textbf{Critical} \\
\addlinespace
\textbf{RISK-03} & No MFA for Email and Endpoints & Lack of MFA on email and employee computers significantly increases the risk of successful phishing attacks, business email compromise (BEC), and lateral movement within the network. & \textbf{High} \\
\addlinespace
\textbf{RISK-04} & Inadequate Onboarding Security Training & New employees do not receive security training upon being hired. This creates a window of vulnerability where new staff are more susceptible to social engineering and accidental policy violations. & \textbf{High} \\
\addlinespace
\textbf{RISK-05} & SSL Certificate Misconfiguration & The SSL certificate's name does not match the company domain, which can erode user trust and may indicate a wider server misconfiguration issue. & \textbf{Medium} \\
\bottomrule
\end{tabular}
\caption{Identified Risks and Severity}
\end{table}

\section*{6. Recommendations}

The following actionable recommendations are provided to mitigate the identified risks. They are prioritized based on severity.

\begin{description}
    \item[\textbf{R-1: Implement Multi-Factor Authentication (MFA)}]
    \textit{(Mitigates RISK-01, RISK-03)} \\
    Immediately prioritize the deployment of a robust MFA solution across the organization.
    \begin{itemize}
        \item \textbf{Phase 1 (Critical):} Enforce MFA for all access to sensitive data systems, administrator accounts, and remote access solutions (e.g., VPN).
        \item \textbf{Phase 2 (High):} Roll out MFA for all employee email accounts and endpoint logins.
    \end{itemize}

    \item[\textbf{R-2: Remediate Vulnerable Web Server}]
    \textit{(Mitigates RISK-02)} \\
    Develop and execute a patch management plan to upgrade the nginx server at \texttt{192.168.10.5} to the latest stable version. All public-facing software should be inventoried and placed on a regular patch cycle.

    \item[\textbf{R-3: Enhance Security Awareness Program}]
    \textit{(Mitigates RISK-04)} \\
    Integrate mandatory security awareness training into the new employee onboarding process. This training must cover, at a minimum, phishing identification, acceptable use policies, and password hygiene.
    
    \item[\textbf{R-4: Correct SSL Certificate Configuration}]
    \textit{(Mitigates RISK-05)} \\
    Replace the existing SSL certificate on the web server with a valid certificate that correctly corresponds to the \texttt{www.MapleLeafLogistics.com} domain.
\end{description}

\end{document}
```