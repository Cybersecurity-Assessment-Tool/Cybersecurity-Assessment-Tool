```latex
\documentclass[12pt]{article}

% Preamble: Required Packages
\usepackage[a4paper, margin=1in]{geometry}
\usepackage{pifont} % For checkmarks and crosses
\usepackage{booktabs} % For professional tables
\usepackage{hyperref} % For clickable links and ToC
\usepackage{url} % For formatting URLs
\usepackage{seqsplit} % To split long strings in \texttt
\usepackage{xcolor} % For colors in text

% Document Metadata
\title{Cybersecurity Posture Assessment Report \\ \large For Summit Peak Partners}
\author{Cybersecurity Analysis Division}
\date{\today}

% Hyperref Setup
\hypersetup{
    colorlinks=true,
    linkcolor=blue,
    filecolor=magenta,      
    urlcolor=cyan,
    pdftitle={Cybersecurity Posture Assessment Report},
    pdfpagemode=FullScreen,
}

\begin{document}

\maketitle
\thispagestyle{empty}
\newpage

\tableofcontents
\newpage

% --- Section 1: Executive Summary ---
\section{Executive Summary}
This report provides a comprehensive analysis of the cybersecurity posture for \textbf{Summit Peak Partners}, based on network scans, organizational data, and a review of existing risks. The assessment was conducted on \today.

The analysis identified several critical and high-risk findings that require immediate attention. The most significant finding is the direct exposure of a MySQL database service (Port 3306) on the host \texttt{172.16.50.20}. This exposure is compounded by the fact that the database software version (MySQL 5.7.33) is outdated and no longer receives mainstream security updates, leaving it vulnerable to known exploits.

Furthermore, a critical gap was identified in the organization's internal security controls: the absence of Multi-Factor Authentication (MFA) for employee computer logins. This significantly increases the risk of unauthorized access to internal systems should an employee's credentials be compromised.

While the organization has implemented several positive security controls, such as MFA for email and security awareness training, the identified risks present a clear and present danger to the confidentiality, integrity, and availability of sensitive data. This report outlines these findings in detail and provides actionable recommendations to mitigate the identified risks.

% --- Section 2: Organizational Information ---
\section{Organizational Information}
The following information was provided for the assessment.

\begin{tabular}{@{}ll}
\toprule
\textbf{Item} & \textbf{Details} \\
\midrule
Organization Name & \textbf{Summit Peak Partners} \\
Email Domain & \texttt{SummitPeakPartners.com} \\
Website Domain & \seqsplit{\url{www.SummitPeakPartners.com}} \\
External IP Address & \texttt{216.39.162.112} \\
\bottomrule
\end{tabular}

% --- Section 3: Security Control Review ---
\section{Security Control Review (Questionnaire)}
A review of the organization's security controls was conducted via a questionnaire. The responses are summarized below. A green checkmark (\textcolor{green}{\ding{51}}) indicates a positive control, while a red cross (\textcolor{red}{\ding{55}}) indicates a potential security gap.

\begin{tabular}{@{}p{0.75\linewidth}c}
\toprule
\textbf{Control Question} & \textbf{Response} \\
\midrule
Do you require MFA to access email? & \textcolor{green}{\ding{51}} \\
\textbf{Do you require MFA to log into computers?} & \textcolor{red}{\ding{55}} \\
Do you require MFA to access sensitive data systems? & \textcolor{green}{\ding{51}} \\
Does your organization have an employee acceptable use policy? & \textcolor{green}{\ding{51}} \\
Does your organization do security awareness training for new employees? & \textcolor{green}{\ding{51}} \\
Does your organization do security awareness training for all employees at least once per year? & \textcolor{green}{\ding{51}} \\
\bottomrule
\end{tabular}

\subsection*{Analysis of Gaps}
The primary gap identified is the \textbf{lack of MFA for computer logins}. This is a critical control for protecting against credential theft and lateral movement within the network. An attacker with valid user credentials could gain direct access to an endpoint, bypassing other security measures.

% --- Section 4: Technical Scan Results ---
\section{Technical Scan Results}
An external network scan was performed to identify open ports and exposed services.

\subsection{Scan Details}
\begin{tabular}{@{}ll}
\toprule
\textbf{Parameter} & \textbf{Value} \\
\midrule
Target IP Address & \texttt{172.16.50.20} \\
Scan Date & \today \\
\bottomrule
\end{tabular}

\subsection{Open Ports and Services}
The following open port was discovered on the target system.

\begin{tabular}{@{}lllll}
\toprule
\textbf{Port} & \textbf{State} & \textbf{Service} & \textbf{Product} & \textbf{Version} \\
\midrule
3306/tcp & open & mysql & MySQL & 5.7.33 \\
\bottomrule
\end{tabular}

\subsection*{Analysis of Findings}
The scan confirms that port \textbf{3306}, the default port for MySQL, is open to the network. Exposing a database service directly is a significant security risk, as it provides a direct vector for attackers to attempt brute-force attacks, exploit vulnerabilities, or exfiltrate data. The identified version, MySQL 5.7.33, is outdated and has known vulnerabilities that have been patched in subsequent releases.

% --- Section 5: Risk Assessment ---
\section{Risk Assessment}
The following table synthesizes findings from the security control review, technical scan, and pre-existing risk data.

\begin{tabular}{@{}p{0.1\linewidth} p{0.25\linewidth} p{0.1\linewidth} p{0.45\linewidth}}
\toprule
\textbf{Risk ID} & \textbf{Risk Name} & \textbf{Severity} & \textbf{Description} \\
\midrule
\textbf{RISK-001} & \textbf{External Database Exposure} & \textbf{Critical (7.5)} & A MySQL database on port 3306 is directly exposed to the network. This allows attackers to directly target the database for credential stuffing, exploitation, or denial-of-service attacks. \\
\addlinespace
\textbf{RISK-002} & \textbf{Lack of Endpoint MFA} & \textbf{High} & The absence of MFA on computer logins means that a single compromised password could grant an attacker full access to an employee's workstation and potentially the internal network. \\
\addlinespace
\textbf{RISK-003} & \textbf{Outdated Database Software} & \textbf{High} & The running MySQL version (5.7.33) is no longer the latest stable release in its series and is susceptible to multiple publicly known vulnerabilities that could lead to remote code execution or information disclosure. \\
\bottomrule
\end{tabular}

% --- Section 6: Recommendations ---
\section{Recommendations}
The following actions are recommended to mitigate the identified risks. Recommendations are prioritized based on severity.

\subsection{RISK-001: External Database Exposure (Critical)}
\begin{itemize}
    \item \textbf{Immediate Action:} Implement strict firewall rules to block all public access to port 3306 on host \texttt{172.16.50.20}. Access should only be permitted from trusted internal IP addresses.
    \item \textbf{Long-Term Solution:} For any required remote access to the database, implement a Virtual Private Network (VPN) with strong authentication. This ensures that all connections are encrypted and originate from authorized users.
\end{itemize}

\subsection{RISK-002: Lack of Endpoint MFA (High)}
\begin{itemize}
    \item \textbf{Immediate Action:} Prioritize the deployment of a Multi-Factor Authentication (MFA) solution for all employee computer logins. This is a fundamental control to protect against credential theft.
    \item \textbf{Policy Update:} Update the organization's access control policy to mandate the use of MFA for all system access, including workstations, servers, and critical applications.
\end{itemize}

\subsection{RISK-003: Outdated Database Software (High)}
\begin{itemize}
    \item \textbf{Immediate Action:} Develop a patch management and upgrade plan for the MySQL instance. The system should be upgraded to the latest stable version of MySQL (e.g., 8.x series) to benefit from the most recent security patches.
    \item \textbf{Long-Term Solution:} Establish a formal patch management program for all critical systems. This program should include regular vulnerability scanning and a defined process for testing and deploying security patches in a timely manner.
\end{itemize}

\end{document}
```