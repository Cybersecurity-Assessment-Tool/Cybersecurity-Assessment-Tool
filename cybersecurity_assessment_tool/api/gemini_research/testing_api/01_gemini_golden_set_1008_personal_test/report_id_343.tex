```latex
\documentclass[12pt]{article}

% -----------------------------------------------------------------------------
% PREAMBLE
% -----------------------------------------------------------------------------
\usepackage[margin=1in]{geometry}
\usepackage{pifont} % For checkmarks and crosses
\usepackage{booktabs} % For professional tables
\usepackage{hyperref} % For clickable links and references
\usepackage{url} % For formatting URLs
\usepackage{seqsplit} % To split long strings without breaking
\usepackage{graphicx}
\usepackage{xcolor}

% --- Hyperref Setup ---
\hypersetup{
    colorlinks=true,
    linkcolor=blue,
    filecolor=magenta,      
    urlcolor=cyan,
    pdftitle={Cybersecurity Assessment Report},
    pdfpagemode=FullScreen,
}

% --- Define Colors ---
\definecolor{darkred}{rgb}{0.55, 0.0, 0.0}
\definecolor{darkgreen}{rgb}{0.0, 0.39, 0.0}
\definecolor{darkblue}{rgb}{0.0, 0.0, 0.55}

% --- Custom Commands ---
\newcommand{\yes}{\textcolor{darkgreen}{\ding{51}}}
\newcommand{\no}{\textcolor{darkred}{\ding{55}}}

% -----------------------------------------------------------------------------
% DOCUMENT START
% -----------------------------------------------------------------------------
\begin{document}

% -----------------------------------------------------------------------------
% TITLE PAGE
% -----------------------------------------------------------------------------
\begin{titlepage}
    \centering
    \vspace*{2cm}
    
    \Huge
    \textbf{Cybersecurity Assessment Report}
    
    \vspace{1.5cm}
    
    \Large
    Prepared for: \\
    \vspace{0.5cm}
    \textbf{Hearth \& Home}
    
    \vfill
    
    \large
    Date of Report: \today \\
    Report ID: CYBER-2023-042
    
\end{titlepage}

\tableofcontents
\newpage

% -----------------------------------------------------------------------------
% 1. EXECUTIVE SUMMARY
% -----------------------------------------------------------------------------
\section{Executive Summary}
This report details the findings of a cybersecurity assessment conducted for Hearth \& Home. The assessment combined an analysis of organizational security controls, a technical network scan, and a review of pre-existing risks.

The organization demonstrates a solid commitment to security awareness, with established training programs for both new and existing employees. This forms a crucial part of a strong security culture.

However, several critical gaps were identified that significantly elevate the organization's risk profile. The most pressing concerns are the lack of Multi-Factor Authentication (MFA) for computer and sensitive data system access. This absence of a fundamental security control, combined with a publicly exposed Secure Shell (SSH) service on the network, creates a high-risk scenario for unauthorized access and potential system compromise. Furthermore, the lack of a formal Acceptable Use Policy (AUP) represents a significant governance gap.

Immediate remediation is required to address these vulnerabilities. Recommendations focus on implementing MFA across all critical systems, securing the exposed network service, and establishing foundational security policies to mitigate the identified risks and improve the overall security posture.

% -----------------------------------------------------------------------------
% 2. ORGANIZATIONAL INFORMATION
% -----------------------------------------------------------------------------
\section{Organizational Information}
The following details were provided for the assessment.

\begin{tabular}{@{}ll}
    \toprule
    \textbf{Attribute} & \textbf{Value} \\
    \midrule
    Organization Name & \textbf{Hearth \& Home} \\
    Email Domain & \texttt{HearthHome.com} \\
    Website Domain & \texttt{www.HearthHome.com} \\
    External IP Address & \texttt{162.98.183.53} \\
    \bottomrule
\end{tabular}

% -----------------------------------------------------------------------------
% 3. SECURITY CONTROL REVIEW (QUESTIONNAIRE)
% -----------------------------------------------------------------------------
\section{Security Control Review}
An analysis of the security questionnaire revealed several areas of concern. The following table summarizes the responses and provides an assessment based on cybersecurity best practices.

\begin{tabular}{@{}p{0.6\textwidth}cp{0.2\textwidth}@{}}
    \toprule
    \textbf{Control Question} & \textbf{Response} & \textbf{Assessment} \\
    \midrule
    Do you require MFA to access email? & \yes & Best Practice Met \\
    Do you require MFA to log into computers? & \no & \textbf{Critical Gap} \\
    Do you require MFA to access sensitive data systems? & \no & \textbf{Critical Gap} \\
    Does your organization have an employee acceptable use policy? & \no & \textbf{High Risk Gap} \\
    Does your organization do security awareness training for new employees? & \yes & Best Practice Met \\
    Does your organization do security awareness training for all employees at least once per year? & \yes & Best Practice Met \\
    \bottomrule
\end{tabular}

\subsection*{Analysis of Gaps}
\begin{itemize}
    \item \textbf{Lack of MFA:} The absence of MFA for computer and sensitive data system logins is a critical vulnerability. It means that a compromised password is the only barrier preventing an attacker from gaining access to internal systems and potentially the organization's most valuable data.
    \item \textbf{Missing Acceptable Use Policy (AUP):} An AUP is a foundational governance document. Without it, there are no clear, enforceable rules for employees regarding the use of company technology and data, increasing the risk of insider threat and misuse.
\end{itemize}

% -----------------------------------------------------------------------------
% 4. TECHNICAL SCAN RESULTS
% -----------------------------------------------------------------------------
\section{Technical Scan Results}
A network scan was performed to identify open ports and exposed services on the organization's external infrastructure.

\begin{tabular}{@{}ll}
    \toprule
    \textbf{Scan Parameter} & \textbf{Value} \\
    \midrule
    Target IP Address & \seqsplit{\texttt{2001:db8::1}} \\
    Scan Date & \today \\
    \bottomrule
\end{tabular}

\subsection*{Open Ports Discovered}
The following table details the open ports identified during the scan.

\begin{tabular}{@{}llll@{}}
    \toprule
    \textbf{Port} & \textbf{State} & \textbf{Service} & \textbf{Notes} \\
    \midrule
    22/tcp & Open & SSH (Secure Shell) & Administrative access protocol. Public exposure is a \\
           &      &                    & significant risk, especially without compensating controls. \\
    \bottomrule
\end{tabular}

\subsection*{Technical Analysis}
The scan identified an open SSH port (22) on the target IPv6 address. SSH is a common protocol for remote server administration. However, when exposed to the public internet, it becomes a primary target for brute-force attacks, where attackers attempt to guess usernames and passwords. This finding is especially concerning given the lack of MFA for computer/system access identified in the security control review.

% -----------------------------------------------------------------------------
% 5. RISK ASSESSMENT SUMMARY
% -----------------------------------------------------------------------------
\section{Risk Assessment Summary}
The following table synthesizes findings from the questionnaire and technical scan into a prioritized list of identified risks. No pre-existing vulnerabilities were reported.

\begin{tabular}{@{}lp{0.25\textwidth}p{0.45\textwidth}l@{}}
    \toprule
    \textbf{ID} & \textbf{Risk Name} & \textbf{Description} & \textbf{Severity} \\
    \midrule
    R-01 & No MFA on Sensitive Systems & The absence of MFA on systems holding sensitive data allows access with only a password, creating a high risk of data breach from credential theft. & \textbf{Critical} \\
    \addlinespace
    R-02 & Publicly Exposed SSH Service & The SSH port is open to the internet, inviting brute-force and credential stuffing attacks. This is exacerbated by the lack of MFA on endpoints (R-03). & \textbf{High} \\
    \addlinespace
    R-03 & No MFA on Employee Computers & A compromised password could grant an attacker direct access to an employee's computer, establishing a foothold within the internal network. & \textbf{High} \\
    \addlinespace
    R-04 & Missing Acceptable Use Policy & Lack of a formal AUP creates ambiguity for employees and limits the organization's ability to enforce security policies, increasing insider risk. & \textbf{Medium} \\
    \bottomrule
\end{tabular}

% -----------------------------------------------------------------------------
% 6. RECOMMENDATIONS
% -----------------------------------------------------------------------------
\section{Recommendations}
The following actions are recommended to mitigate the identified risks and strengthen the security posture of Hearth \& Home. Recommendations are prioritized based on severity.

\begin{enumerate}
    \item \textbf{[Critical] Implement MFA for All Critical Systems (R-01, R-03):} Prioritize the deployment of a robust MFA solution for all employee computer logins and for access to all systems containing sensitive or critical data. This is the single most effective control to mitigate the risk of unauthorized access via stolen credentials.
    
    \item \textbf{[High] Secure the Exposed SSH Service (R-02):}
        \begin{itemize}
            \item If remote access is not required, block port 22 at the network firewall.
            \item If access is required, restrict source IP addresses to only trusted networks (e.g., administrator office IPs, VPN gateway).
            \item Enforce public key authentication and disable password-based authentication for SSH to prevent brute-force attacks.
        \end{itemize}
        
    \item \textbf{[Medium] Develop and Implement an Acceptable Use Policy (R-04):} Create a formal AUP that clearly defines the rules and responsibilities for all employees when using company assets, data, and network resources. This policy should be reviewed by legal counsel, communicated to all staff, and formally acknowledged by each employee.
\end{enumerate}

% -----------------------------------------------------------------------------
% DOCUMENT END
% -----------------------------------------------------------------------------
\end{document}
```