```latex
\documentclass[12pt]{article}

% Preamble: Required Packages
\usepackage[margin=1in]{geometry}
\usepackage{pifont} % Required for \ding{51} (checkmark) and \ding{55} (cross)
\usepackage{booktabs} % For professional-looking tables (\toprule, \midrule, \bottomrule)
\usepackage{hyperref} % For clickable links
\usepackage{url}      % For formatting URLs
\usepackage{seqsplit} % To split long strings in \texttt environments

% Document Metadata and Hyperlink Setup
\hypersetup{
    colorlinks=true,
    linkcolor=black,
    filecolor=magenta,
    urlcolor=blue,
    pdftitle={Cybersecurity Posture Assessment Report},
    pdfauthor={Cybersecurity Analysis Division},
    pdfsubject={Security Assessment},
    pdfkeywords={Cybersecurity, Risk, Analysis}
}

\begin{document}

% --- Title Page ---
\title{Cybersecurity Posture Assessment Report}
\author{Cybersecurity Analysis Division}
\date{\today}
\maketitle
\thispagestyle{empty}
\newpage

\tableofcontents
\newpage

% --- Section 1: Executive Overview ---
\section{Executive Overview}

This report provides a comprehensive cybersecurity assessment for \textbf{Iron Bridge Legal}, synthesizing data from organizational questionnaires, technical network scans, and a review of existing risks.

The assessment reveals a mixed security posture. The organization demonstrates a strong commitment to security policy and employee awareness, with established acceptable use policies and a consistent security training program. Furthermore, the external network scan of the provided IP address (\texttt{129.52.180.86}) indicates a very secure perimeter, as no open ports were discovered.

However, two critical security gaps were identified in the organization's access control policies. The absence of mandatory Multi-Factor Authentication (MFA) for accessing email and for logging into company computers represents a significant and immediate risk. These gaps could allow an attacker with compromised credentials to gain unauthorized access to sensitive communications, client data, and the internal network.

Immediate remediation of these MFA-related vulnerabilities is strongly recommended to mitigate the risk of business email compromise, ransomware, and data breaches.

% --- Section 2: Organizational Information ---
\section{Organizational Information}

The following details were provided for the assessment. This information is used to establish the context and scope of the review.

\begin{table}[h!]
\centering
\begin{tabular}{@{}ll@{}}
\toprule
\textbf{Attribute} & \textbf{Value} \\ \midrule
Organization Name    & \textbf{Iron Bridge Legal} \\
Email Domain         & \texttt{IronBridgeLegal.com} \\
Website Domain       & \url{www.IronBridgeLegal.com} \\
External IP Address  & \texttt{129.52.180.86} \\ \bottomrule
\end{tabular}
\caption{Client Organizational Details}
\label{tab:org_info}
\end{table}

% --- Section 3: Security Control Review ---
\section{Security Control Review}

A review of the organization's self-reported security controls was conducted based on a standard security questionnaire. The findings highlight key areas of strength and weakness in the current security program. "No" answers indicate significant gaps that require attention.

\begin{table}[h!]
\centering
\begin{tabular}{@{}lcc@{}}
\toprule
\textbf{Security Control Question} & \textbf{Status} & \textbf{Assessment} \\ \midrule
Do you require MFA to access email? & \ding{55} & \textbf{Critical Gap} \\
Do you require MFA to log into computers? & \ding{55} & \textbf{Critical Gap} \\
Do you require MFA to access sensitive data systems? & \ding{51} & Best Practice Met \\
Does your organization have an employee acceptable use policy? & \ding{51} & Best Practice Met \\
Does your organization do security awareness training for new employees? & \ding{51} & Best Practice Met \\
Does your organization do security awareness training for all employees annually? & \ding{51} & Best Practice Met \\ \bottomrule
\end{tabular}
\caption{Security Controls Questionnaire Analysis (\ding{51}=Yes, \ding{55}=No)}
\label{tab:controls}
\end{table}

% --- Section 4: Technical Scan Results ---
\section{Technical Scan Results}

A network scan was performed to identify accessible services and potential vulnerabilities on the organization's perimeter.

\begin{itemize}
    \item \textbf{Scan Target:} \texttt{192.168.1.100}
    \item \textbf{Scan Date:} \today
    \item \textbf{Host Status:} Up
\end{itemize}

\subsection{Open Ports and Services}
The scan results were positive from a security standpoint. The target host was responsive, but \textbf{no open TCP or UDP ports were discovered}. All scanned ports were reported as "closed". This indicates a strong firewall configuration and a minimal attack surface, which significantly reduces the risk of external network-based attacks.

% --- Section 5: Risk Assessment ---
\section{Risk Assessment}

This section synthesizes the findings from the security control review and technical scans to identify and prioritize key risks. Although no pre-existing vulnerabilities were documented and the network scan was clean, the policy gaps represent a significant threat.

\begin{table}[h!]
\centering
\begin{tabular}{@{}p{0.25\linewidth}p{0.55\linewidth}p{0.1\linewidth}@{}}
\toprule
\textbf{Risk Name} & \textbf{Overview} & \textbf{Severity} \\ \midrule
\textbf{Email Account Compromise via Password Attack} & The lack of MFA on email accounts (\texttt{IronBridgeLegal.com}) means that a compromised password is all an attacker needs to gain full access. This can lead to business email compromise (BEC), data exfiltration, and phishing attacks against clients and partners. & \textbf{Critical} \\
\addlinespace
\textbf{Workstation Takeover via Stolen Credentials} & Without an MFA requirement for computer logins, an attacker who obtains an employee's credentials can log directly into a company workstation. This provides a foothold into the internal network, access to local files, and a platform to launch further attacks. & \textbf{Critical} \\
\bottomrule
\end{tabular}
\caption{Identified Cybersecurity Risks}
\label{tab:risks}
\end{table}

% --- Section 6: Recommendations ---
\section{Recommendations}

Based on the analysis, the following actions are recommended to mitigate the identified risks and improve the overall security posture of \textbf{Iron Bridge Legal}.

\begin{enumerate}
    \item \textbf{Implement MFA for Email Access (Priority: Immediate):}
    \begin{itemize}
        \item Enforce mandatory MFA for all users accessing the email system. This is the single most effective control to prevent email account takeovers.
        \item This can typically be configured within the administration settings of your email provider (e.g., Microsoft 365, Google Workspace).
    \end{itemize}
    
    \item \textbf{Implement MFA for Workstation Login (Priority: Immediate):}
    \begin{itemize}
        \item Require a second factor of authentication for all users logging into company-issued desktops and laptops.
        \item Solutions like Windows Hello for Business, Duo Security, or other third-party identity providers can be integrated with Active Directory or cloud identity services to enforce this control.
    \end{itemize}
    
    \item \textbf{Enhance Security Awareness Training (Priority: High):}
    \begin{itemize}
        \item While the training program is robust, incorporate specific modules that explain the importance of the new MFA controls.
        \item Continue to run phishing simulations to test employee awareness and reinforce best practices.
    \end{itemize}
    
    \item \textbf{Maintain Network Security Posture (Priority: Ongoing):}
    \begin{itemize}
        \item The current external network posture is excellent. Continue to conduct regular vulnerability scans (both external and internal) to ensure that no new services are inadvertently exposed and that the firewall configuration remains secure.
    \end{itemize}
\end{enumerate}

\end{document}
```