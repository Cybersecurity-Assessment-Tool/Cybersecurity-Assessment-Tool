```latex
\documentclass[12pt]{article}

% Preamble: Required Packages
\usepackage[margin=1in]{geometry}
\usepackage{pifont} % For checkmarks and crosses
\usepackage{booktabs} % For professional tables
\usepackage[hidelinks]{hyperref} % For clickable links without boxes
\usepackage{url} % For proper URL formatting
\usepackage{seqsplit} % To split long strings like IPs
\usepackage{graphicx} % For potential logo inclusion

% --- Document Metadata ---
\title{Cybersecurity Posture Report \\ \large For: \textbf{North Star Education}}
\author{Cybersecurity Analysis Division}
\date{\today}

\begin{document}

\maketitle
\thispagestyle{empty}
\newpage

\tableofcontents
\thispagestyle{empty}
\newpage

% --- Section 1: Executive Summary ---
\section{Executive Summary}
\setcounter{page}{1}
This report provides a comprehensive analysis of the cybersecurity posture for \textbf{North Star Education}, based on a review of organizational security controls, an external network scan, and pre-existing risk data. The assessment was conducted to identify security gaps, technical vulnerabilities, and areas for improvement.

The analysis revealed several high-risk gaps in administrative and access controls. Key findings include the absence of a formal Acceptable Use Policy, a lack of mandatory Multi-Factor Authentication (MFA) for computer logons, and an incomplete security awareness training program. These represent significant risks to the organization's data and systems, increasing susceptibility to unauthorized access and social engineering attacks.

From a technical perspective, an external scan identified an exposed Secure Shell (SSH) service on a public-facing IPv6 address. While necessary for administration, this service is a common target for attackers and requires robust hardening.

Immediate remediation efforts should focus on implementing the foundational security controls identified as missing. Addressing these administrative gaps will substantially improve the organization's overall resilience against common cyber threats. Detailed technical and policy recommendations are provided in the final section of this report.

% --- Section 2: Organizational Information ---
\section{Organizational Information}
The following details were provided for the assessment. This information establishes the scope and context for the security review.

\begin{itemize}
    \item \textbf{Organization Name:} North Star Education
    \item \textbf{Primary Email Domain:} \texttt{NorthStarEducation.com}
    \item \textbf{Primary Website Domain:} \url{www.NorthStarEducation.com}
    \item \textbf{Known External IP Address:} \texttt{7.242.93.17}
\end{itemize}

% --- Section 3: Security Control Review ---
\section{Security Control Review}
A questionnaire was completed to evaluate the implementation of key administrative and technical security controls. The results are summarized below. Answers marked with \ding{55} (No) indicate significant control gaps that require immediate attention.

\begin{table}[h!]
\centering
\caption{Security Controls Questionnaire Results}
\begin{tabular}{p{0.75\linewidth} c}
\toprule
\textbf{Control Question} & \textbf{Response} \\
\midrule
Do you require MFA to access email? & \ding{51} \\
\textbf{Do you require MFA to log into computers?} & \textbf{\ding{55}} \\
Do you require MFA to access sensitive data systems? & \ding{51} \\
\textbf{Does your organization have an employee acceptable use policy?} & \textbf{\ding{55}} \\
Does your organization do security awareness training for new employees? & \ding{51} \\
\textbf{Does your organization do security awareness training for all employees at least once per year?} & \textbf{\ding{55}} \\
\bottomrule
\end{tabular}
\end{table}

% --- Section 4: Technical Scan Results ---
\section{Technical Scan Results}
An external network scan was performed to identify open ports and exposed services on public-facing assets.

\begin{itemize}
    \item \textbf{Target IP Address:} \seqsplit{\texttt{2001:db8::1}}
    \item \textbf{Scan Date:} \today
\end{itemize}

The following table details the services discovered during the scan.

\begin{table}[h!]
\centering
\caption{Open Ports and Services}
\begin{tabular}{l l l p{0.5\linewidth}}
\toprule
\textbf{Port} & \textbf{State} & \textbf{Service} & \textbf{Analyst Notes} \\
\midrule
22/tcp & Open & SSH & Secure Shell is used for remote administration. Exposing this service to the internet increases the attack surface. No version information was available. Hardening is strongly recommended. \\
\bottomrule
\end{tabular}
\end{table}

% --- Section 5: Risk Assessment ---
\section{Risk Assessment}
This section synthesizes findings from the security control review and technical scan into a prioritized list of risks. The organization had no previously documented risks.

\begin{table}[h!]
\centering
\caption{Identified Risks}
\begin{tabular}{l p{0.55\linewidth} l}
\toprule
\textbf{Risk ID} & \textbf{Description} & \textbf{Severity} \\
\midrule
RISK-001 & \textbf{Lack of Acceptable Use Policy:} Without a formal AUP, employees lack clear guidelines on the secure and acceptable use of company assets, increasing the risk of insider threat and misuse. & \textbf{High} \\
\addlinespace
RISK-002 & \textbf{No MFA for Endpoint Logon:} The absence of MFA on computer logins creates a single point of failure. A compromised password could grant an attacker full access to an employee's workstation and network resources. & \textbf{High} \\
\addlinespace
RISK-003 & \textbf{Inadequate Security Awareness Training:} Failing to provide annual training for all employees leaves the organization vulnerable to phishing, social engineering, and other human-centric attacks. & \textbf{High} \\
\addlinespace
RISK-004 & \textbf{Exposed SSH Service:} The SSH service on \seqsplit{\texttt{2001:db8::1}} is exposed to the public internet, making it a target for brute-force attacks and exploitation of potential vulnerabilities. & Medium \\
\bottomrule
\end{tabular}
\end{table}

% --- Section 6: Recommendations ---
\section{Recommendations}
The following actions are recommended to mitigate the identified risks and improve the overall security posture of \textbf{North Star Education}.

\begin{enumerate}
    \item \textbf{Develop and Implement an Acceptable Use Policy (AUP) (RISK-001):}
    \begin{itemize}
        \item Draft a formal AUP that clearly defines rules for the use of company networks, systems, and data.
        \item Require all new and existing employees to read and formally acknowledge the policy.
        \item Review and update the policy annually or as significant organizational changes occur.
    \end{itemize}
    \vspace{1em}
    \item \textbf{Enforce MFA for All System Logons (RISK-002):}
    \begin{itemize}
        \item Immediately deploy a solution to require Multi-Factor Authentication (MFA) for all employee computer logons.
        \item This is a critical defense against credential theft and should be prioritized for all systems, not just those deemed "sensitive."
    \end{itemize}
    \vspace{1em}
    \item \textbf{Establish a Comprehensive Security Training Program (RISK-003):}
    \begin{itemize}
        \item Implement a mandatory, annual security awareness training program for all employees.
        \item Training should cover key topics such as phishing identification, password hygiene, and incident reporting.
        \item Consider periodic phishing simulations to test and reinforce employee awareness.
    \end{itemize}
    \vspace{1em}
    \item \textbf{Harden the Exposed SSH Service (RISK-004):}
    \begin{itemize}
        \item \textbf{Restrict Access:} If possible, limit SSH access to a whitelist of trusted IP addresses (e.g., office or administrator IPs).
        \item \textbf{Disable Password Authentication:} Configure the SSH server to only allow public key-based authentication, which is significantly more secure than passwords.
        \item \textbf{Implement Intrusion Prevention:} Use a tool like \texttt{fail2ban} to automatically block IP addresses that exhibit malicious behavior, such as repeated failed login attempts.
    \end{itemize}
\end{enumerate}

\end{document}
```