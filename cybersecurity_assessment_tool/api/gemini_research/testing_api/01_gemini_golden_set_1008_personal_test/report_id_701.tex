```latex
\documentclass[12pt]{article}

% --- PACKAGE IMPORTS ---
\usepackage[margin=1in]{geometry}
\usepackage{pifont} % For checkmarks and crosses
\usepackage{booktabs} % For professional tables
\usepackage{hyperref} % For clickable links
\usepackage{url}      % For proper URL formatting
\usepackage{seqsplit} % For splitting long text strings like URLs/IPs
\usepackage{xcolor}   % For colors

% --- DOCUMENT METADATA ---
\title{Cybersecurity Posture Assessment Report}
\author{Cybersecurity Analysis Division}
\date{\today}

% --- HYPERREF SETUP ---
\hypersetup{
    colorlinks=true,
    linkcolor=blue,
    filecolor=magenta,      
    urlcolor=cyan,
    pdftitle={Cybersecurity Posture Assessment Report},
    pdfpagemode=FullScreen,
}

\begin{document}

\maketitle
\hrule
\vspace{1em}
\begin{center}
    \textbf{Client:} Common Ground \\
    \textbf{Report ID:} CSR-2023-451
\end{center}
\vspace{1em}
\hrule

\newpage

\tableofcontents

\newpage

% ==============================================================================
% SECTION 1: EXECUTIVE OVERVIEW
% ==============================================================================
\section{Executive Overview}

This report provides a comprehensive analysis of the cybersecurity posture for \textbf{Common Ground}, based on a synthesis of network scan data, organizational security controls, and pre-existing risk information.

The assessment reveals a mixed security posture. The organization demonstrates maturity in identity and access management, with strong enforcement of Multi-Factor Authentication (MFA) across email, workstations, and sensitive systems. This significantly reduces the risk of credential-based attacks.

However, critical deficiencies were identified in both policy governance and technical security. The absence of a formal Acceptable Use Policy (AUP) and the lack of mandatory annual security awareness training for all staff create significant human-factor risks. These policy gaps are compounded by a critical technical finding: an exposed Remote Desktop Protocol (RDP) service on an internal host (\seqsplit{\texttt{10.10.10.51}}). This finding is particularly concerning as it indicates a systemic issue, echoing a previously identified risk of RDP exposure on another host.

Immediate action is required to remediate the exposed RDP service and to develop foundational security policies to mitigate ongoing risks and foster a security-conscious culture.

% ==============================================================================
% SECTION 2: ORGANIZATIONAL INFORMATION
% ==============================================================================
\section{Organizational Information}

The following details were provided for the assessment scope.

\begin{tabular}{@{}ll}
\toprule
\textbf{Attribute} & \textbf{Value} \\
\midrule
Organization Name & \textbf{Common Ground} \\
Email Domain & \seqsplit{\texttt{CommonGround.org}} \\
Website Domain & \seqsplit{\url{www.CommonGround.org}} \\
External IP Address & \seqsplit{\texttt{183.107.245.57}} \\
\bottomrule
\end{tabular}

% ==============================================================================
% SECTION 3: SECURITY CONTROL REVIEW
% ==============================================================================
\section{Security Control Review (Questionnaire Analysis)}

The following table summarizes the organization's responses to a security controls questionnaire. The analysis highlights gaps where responses deviate from established best practices.

\begin{table}[h!]
\centering
\begin{tabular}{@{}lc}
\toprule
\textbf{Control Question} & \textbf{Response} \\
\midrule
Do you require MFA to access email? & \textcolor{green}{\ding{51}} \\
Do you require MFA to log into computers? & \textcolor{green}{\ding{51}} \\
Do you require MFA to access sensitive data systems? & \textcolor{green}{\ding{51}} \\
Does your organization have an employee acceptable use policy? & \textcolor{red}{\ding{55}} \\
Does your organization do security awareness training for new employees? & \textcolor{green}{\ding{51}} \\
Does your organization do security awareness training for all employees at least once per year? & \textcolor{red}{\ding{55}} \\
\bottomrule
\end{tabular}
\caption{Security Controls Questionnaire Results}
\end{table}

\paragraph{Analysis:}
The organization has successfully implemented MFA across critical access points, which is a commendable strength. However, two significant policy gaps were identified:
\begin{itemize}
    \item \textbf{No Acceptable Use Policy (AUP):} The lack of an AUP means there are no formally documented rules for how employees should use company technology and data. This can lead to inconsistent security practices, misuse of assets, and legal ambiguity. This is considered a \textbf{High} risk.
    \item \textbf{No Annual Security Awareness Training:} While new hires receive training, the absence of an annual refresher for all employees is a critical oversight. The threat landscape evolves continuously, and without ongoing education, employees are more susceptible to phishing, social engineering, and other modern attack vectors. This is considered a \textbf{High} risk.
\end{itemize}

% ==============================================================================
% SECTION 4: TECHNICAL SCAN RESULTS
% ==============================================================================
\section{Technical Scan Results}

A network scan was performed to identify open ports and exposed services on the target system.

\begin{itemize}
    \item \textbf{Target IP Address:} \seqsplit{\texttt{10.10.10.51}}
    \item \textbf{Scan Status:} Host is Up
\end{itemize}

\begin{table}[h!]
\centering
\begin{tabular}{@{}llll}
\toprule
\textbf{Port} & \textbf{State} & \textbf{Service Name} & \textbf{Product / Version} \\
\midrule
3389/tcp & open & ms-wbt-server & Not Determined \\
\bottomrule
\end{tabular}
\caption{Open Ports Detected on \seqsplit{\texttt{10.10.10.51}}}
\end{table}

\paragraph{Analysis:}
The scan identified that port \textbf{3389/tcp} is open, which corresponds to the Microsoft Remote Desktop Protocol (RDP). RDP is a primary target for attackers seeking to gain unauthorized access to internal networks. Exposed RDP services are frequently exploited for ransomware deployment and lateral movement. This finding, correlated with a pre-existing risk of RDP exposure on another host (\seqsplit{\texttt{10.10.10.50}}), points to a systemic configuration weakness that requires immediate attention. This is a \textbf{Critical} risk.

% ==============================================================================
% SECTION 5: CONSOLIDATED RISK ASSESSMENT
% ==============================================================================
\section{Consolidated Risk Assessment}

This section correlates all findings from the questionnaire, technical scans, and pre-existing risk data into a unified summary.

\begin{table}[h!]
\centering
\begin{tabular}{@{}p{0.3\linewidth}p{0.5\linewidth}l}
\toprule
\textbf{Risk Name} & \textbf{Description} & \textbf{Severity} \\
\midrule
\textbf{Systemic RDP Exposure} & The Remote Desktop Protocol (port 3389) is exposed on host \seqsplit{\texttt{10.10.10.51}}. This is a recurrence of a known issue, previously identified on host \seqsplit{\texttt{10.10.10.50}}. & \textbf{Critical} \\
\addlinespace
\textbf{Lack of Acceptable Use Policy} & The organization does not have a formal policy defining the acceptable use of corporate assets, data, and networks by employees. & \textbf{High} \\
\addlinespace
\textbf{Insufficient Security Training} & While new hires are trained, there is no mandatory annual security awareness training program for all employees to address evolving threats. & \textbf{High} \\
\bottomrule
\end{tabular}
\caption{Summary of Identified Risks}
\end{table}

% ==============================================================================
% SECTION 6: RECOMMENDATIONS
% ==============================================================================
\section{Recommendations}

The following actions are recommended to mitigate the identified risks and improve the overall security posture of \textbf{Common Ground}.

\begin{enumerate}
    \item \textbf{[Immediate] Remediate Exposed RDP Service:} The RDP service on \seqsplit{\texttt{10.10.10.51}} must be secured immediately.
    \begin{itemize}
        \item \textbf{Short-Term:} If RDP access is not required, close port 3389 on the host and network firewalls.
        \item \textbf{Long-Term:} If remote access is necessary, implement a Virtual Private Network (VPN) or Zero Trust Network Access (ZTNA) solution with MFA as a prerequisite for connection.
    \end{itemize}

    \item \textbf{[High Priority] Conduct a Network-Wide RDP Audit:} Given that this is a recurring issue, perform a comprehensive internal and external scan of all network segments to identify and remediate any other instances of exposed RDP or other high-risk services.

    \item \textbf{[High Priority] Develop and Implement an Acceptable Use Policy (AUP):} Create a formal AUP that clearly outlines the rules and responsibilities for all employees when using company technology and data. This policy should be reviewed by legal counsel, communicated to all staff, and formally acknowledged via signature.

    \item \textbf{[High Priority] Establish an Annual Security Awareness Program:} Implement a mandatory security awareness training program for all employees to be completed at least once per year. The training should cover modern threats such as phishing, ransomware, social engineering, and secure data handling.
\end{enumerate}

\end{document}
```