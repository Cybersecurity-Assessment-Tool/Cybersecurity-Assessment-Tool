```latex
\documentclass[12pt]{article}

% Preamble: Required Packages
\usepackage[margin=1in]{geometry}
\usepackage{pifont} % For checkmarks and crosses
\usepackage{booktabs} % For professional tables
\usepackage{hyperref} % For clickable links
\usepackage{url} % For URL formatting
\usepackage{seqsplit} % For splitting long strings
\usepackage{graphicx}
\usepackage{xcolor}
\usepackage{datetime}

% Document Title and Author
\title{Cybersecurity Assessment Report \\ \large For: \textbf{Great Lakes}}
\author{Cybersecurity Analysis Division}
\date{\today}

\begin{document}

\maketitle
\thispagestyle{empty}
\newpage

\tableofcontents
\newpage

% --- 1. Executive Summary ---
\section{Executive Summary}

This report details the findings of a cybersecurity assessment for \textbf{Great Lakes}, conducted on \today. The assessment was based on a security controls questionnaire, a review of pre-existing risks, and an external network scan.

The analysis revealed several critical and high-risk security gaps originating from organizational policies and procedures. While the organization has implemented some foundational controls, such as an acceptable use policy and security training for new hires, there are significant deficiencies in access control and ongoing employee education.

\textbf{Key Findings:}
\begin{itemize}
    \item \textbf{Critical Risk - Lack of Multi-Factor Authentication (MFA):} The organization does not enforce MFA for accessing email or other sensitive data systems. This represents a critical vulnerability, as compromised credentials could lead directly to a significant data breach.
    \item \textbf{High Risk - Insufficient Security Training:} While new employees receive training, there is no mandatory annual security awareness training for all staff. This increases the organization's susceptibility to social engineering attacks like phishing over time.
    \item \textbf{Technical Scan:} The external network scan performed against the target IP address \texttt{[Target IP]} did not identify any open ports. This suggests a strong firewall configuration or that no services are intentionally exposed on that host.
    \item \textbf{Pre-existing Risks:} No pre-existing vulnerabilities were provided for this assessment.
\end{itemize}

Immediate remediation should focus on implementing a robust MFA policy across all critical systems. Furthermore, establishing a continuous security awareness training program is essential to bolster the human element of the organization's defense.

% --- 2. Organizational Information ---
\section{Organizational Information}

The following information was provided by the client and used as a baseline for this assessment.

\begin{tabular}{@{}ll}
    \toprule
    \textbf{Attribute} & \textbf{Value} \\
    \midrule
    Organization Name & \textbf{Great Lakes} \\
    Email Domain & \texttt{GreatLakes.com} \\
    Website Domain & \url{www.GreatLakes.com} \\
    External IP Address & \texttt{233.9.104.57} \\
    \bottomrule
\end{tabular}

% --- 3. Security Control Review ---
\section{Security Control Review}

The following table summarizes the organization's responses to the security controls questionnaire. Each response has been assessed against industry best practices. Items marked with \ding{55} indicate a significant gap in security posture.

\begin{tabular}{@{}p{0.6\linewidth}cp{0.25\linewidth}@{}}
    \toprule
    \textbf{Control Question} & \textbf{Response} & \textbf{Assessment} \\
    \midrule
    Do you require MFA to access email? & \ding{55} & \textbf{Critical Gap.} Email is a primary target for account takeover. \\
    \addlinespace
    Do you require MFA to log into computers? & \ding{51} & Meets best practice. \\
    \addlinespace
    Do you require MFA to access sensitive data systems? & \ding{55} & \textbf{Critical Gap.} Direct risk to confidential and proprietary data. \\
    \addlinespace
    Does your organization have an employee acceptable use policy? & \ding{51} & Meets best practice. \\
    \addlinespace
    Does your organization do security awareness training for new employees? & \ding{51} & Meets best practice for onboarding. \\
    \addlinespace
    Does your organization do security awareness training for all employees at least once per year? & \ding{55} & \textbf{High Risk.} Lack of continuous training increases susceptibility to phishing. \\
    \bottomrule
\end{tabular}

% --- 4. Technical Scan Results ---
\section{Technical Scan Results}

An external network vulnerability scan was conducted to identify exposed services and potential vulnerabilities on the organization's perimeter.

\begin{itemize}
    \item \textbf{Target IP Address:} \texttt{[Target IP]}
    \item \textbf{Scan Date:} \today
\end{itemize}

\subsection{Summary of Findings}
The scan against the target host \texttt{[Target IP]} completed successfully but did not identify any open TCP or UDP ports.

\textbf{Conclusion:} This result indicates one of two possibilities:
\begin{enumerate}
    \item The target system is properly secured and does not expose any services to the public internet.
    \item A firewall or other network security device is in place and configured to drop or reject all unsolicited incoming traffic, effectively hiding any running services from an external scanner.
\end{enumerate}

While this is a positive finding from a network security perspective, it provides limited insight into the overall technical posture. A comprehensive assessment would require authenticated internal scans.

% --- 5. Consolidated Risk Assessment ---
\section{Consolidated Risk Assessment}

This section consolidates all identified risks from the questionnaire analysis, technical scan, and pre-existing risk data. The risks are prioritized by severity to guide remediation efforts. No pre-existing risks were provided for this assessment.

\begin{tabular}{@{}p{0.1\linewidth}p{0.25\linewidth}p{0.45\linewidth}l@{}}
    \toprule
    \textbf{Risk ID} & \textbf{Risk Name} & \textbf{Description} & \textbf{Severity} \\
    \midrule
    RISK-001 & Lack of MFA on Email Systems & User email accounts are protected only by passwords. A single credential compromise could lead to account takeover, data exfiltration, and further internal phishing attacks. & \textcolor{red}{Critical} \\
    \addlinespace
    RISK-002 & Lack of MFA on Sensitive Data Systems & Critical systems containing sensitive data do not require MFA for access. This exposes the organization's most valuable data to high risk in the event of a credential breach. & \textcolor{red}{Critical} \\
    \addlinespace
    RISK-003 & Insufficient Security Awareness Training & The absence of a mandatory, annual security training program for all employees leads to a decay in security awareness, making staff more vulnerable to social engineering attacks. & \textcolor{orange}{High} \\
    \bottomrule
\end{tabular}

% --- 6. Recommendations ---
\section{Recommendations}

The following actions are recommended to mitigate the identified risks and improve the overall security posture of \textbf{Great Lakes}.

\begin{description}
    \item[\textcolor{red}{Critical}] \textbf{Implement Mandatory MFA (RISK-001 \& RISK-002):}
    \begin{itemize}
        \item Immediately enforce MFA for all users on the primary email platform (e.g., Microsoft 365, Google Workspace).
        \item Conduct an inventory of all systems containing sensitive data and prioritize the rollout of MFA, starting with those that are internet-accessible or hold the most critical information.
        \item Select an MFA solution that supports a range of methods (e.g., authenticator app, hardware token, biometrics) to ensure usability and adoption.
    \end{itemize}
    
    \item[\textcolor{orange}{High}] \textbf{Establish a Continuous Security Training Program (RISK-003):}
    \begin{itemize}
        \item Develop or procure a security awareness training module to be completed by all employees on an annual basis.
        \item Training content should cover modern threats, including phishing, business email compromise (BEC), ransomware, and proper data handling.
        \item Supplement annual training with regular phishing simulations to test and reinforce employee knowledge in a practical setting.
    \end{itemize}
    
    \item[\textcolor{blue}{Informational}] \textbf{Enhance Technical Visibility:}
    \begin{itemize}
        \item Schedule regular, authenticated vulnerability scans of internal and external infrastructure to gain a more complete and accurate picture of technical vulnerabilities beyond what a simple external port scan can provide.
    \end{itemize}
\end{description}

\end{document}
```