```latex
\documentclass[12pt]{article}

% --- PACKAGES ---
\usepackage[margin=1in]{geometry}
\usepackage{pifont} % For checkmarks and crosses
\usepackage{booktabs} % For professional tables
\usepackage{hyperref} % For clickable links
\usepackage{url}      % For URL formatting
\usepackage{seqsplit} % For splitting long strings like IPs
\usepackage{graphicx} % For potential logos
\usepackage{xcolor}   % For colors

% --- DOCUMENT SETUP ---
\hypersetup{
    colorlinks=true,
    linkcolor=blue,
    filecolor=magenta,      
    urlcolor=cyan,
    pdftitle={Cybersecurity Assessment Report},
    pdfpagemode=FullScreen,
}

\newcommand{\yes}{\ding{51}} % Green checkmark
\newcommand{\no}{\ding{55}}  % Red cross

\begin{document}

% --- TITLE PAGE ---
\begin{titlepage}
    \centering
    \vspace*{1cm}
    \Huge\textbf{Cybersecurity Assessment Report}
    \vspace{1.5cm}
    \Large
    \textbf{Prepared for:} \\
    \vspace{0.5cm}
    \textbf{Fable \& Lore}
    \vspace{2cm}
    \large
    \textbf{Date of Report:} \today \\
    \textbf{Assessment Date:} 2025-11-22
    \vfill
    \large
    \textit{This report contains sensitive information and should be handled with care.}
\end{titlepage}

\tableofcontents
\newpage

% --- EXECUTIVE OVERVIEW ---
\section{Executive Overview}
This report details the findings of a cybersecurity assessment conducted for \textbf{Fable \& Lore}. The assessment combined a review of organizational security controls, an external network scan, and an analysis of known risks.

The overall security posture reveals several critical and high-risk gaps that require immediate attention. While the organization has implemented foundational controls like Multi-Factor Authentication (MFA) for email and provides security awareness training, significant weaknesses exist in access control for internal systems and administrative policy.

Key findings include:
\begin{itemize}
    \item \textbf{Critical Gaps in Access Control:} MFA is not enforced for logging into computers or accessing sensitive data systems. This significantly increases the risk of unauthorized access via compromised credentials.
    \item \textbf{Outdated Web Server Software:} The external-facing web server is running an outdated version of Nginx (1.18.0), which is susceptible to numerous publicly known vulnerabilities.
    \item \textbf{Missing Administrative Controls:} The absence of a formal Acceptable Use Policy (AUP) creates ambiguity regarding secure employee conduct and increases the risk of insider threats.
\end{itemize}

Immediate remediation of these issues is recommended to reduce the organization's attack surface and protect critical assets.

% --- ORGANIZATIONAL INFORMATION ---
\section{Organizational Information}
The following details were provided for the assessment.
\begin{itemize}
    \item \textbf{Organization Name:} Fable \& Lore
    \item \textbf{Email Domain:} \texttt{FableLore.org}
    \item \textbf{Website Domain:} \url{www.FableLore.org}
    \item \textbf{External IP Address:} \seqsplit{\texttt{223.31.118.209}}
\end{itemize}

% --- SECURITY CONTROL REVIEW ---
\section{Security Control Review}
A review of administrative and organizational security controls was conducted based on a questionnaire. The responses indicate several areas for improvement, particularly concerning access control policies.

\begin{table}[h!]
\centering
\caption{Organizational Security Control Questionnaire}
\begin{tabular}{p{0.8\linewidth} c}
\toprule
\textbf{Control Question} & \textbf{Response} \\
\midrule
Do you require MFA to access email? & \yes \\
Do you require MFA to log into computers? & \no \\
Do you require MFA to access sensitive data systems? & \no \\
Does your organization have an employee acceptable use policy? & \no \\
Does your organization do security awareness training for new employees? & \yes \\
Does your organization do security awareness training for all employees at least once per year? & \yes \\
\bottomrule
\end{tabular}
\end{table}

The lack of MFA on computers and sensitive systems, combined with the absence of an Acceptable Use Policy, represents a significant risk to the organization.

% --- TECHNICAL SCAN RESULTS ---
\section{Technical Scan Results}
An Nmap scan was performed to identify open ports and services on the organization's external infrastructure.

\begin{itemize}
    \item \textbf{Scan Target:} \seqsplit{\texttt{192.168.10.5}}
    \item \textbf{Scan Date:} 2025-11-22
\end{itemize}

\begin{table}[h!]
\centering
\caption{Open Ports and Services}
\begin{tabular}{l l l l l}
\toprule
\textbf{Port} & \textbf{State} & \textbf{Service} & \textbf{Product} & \textbf{Version} \\
\midrule
443/tcp & open & https & nginx & 1.18.0 \\
\bottomrule
\end{tabular}
\end{table}

\subsection{Analysis of Technical Findings}
The scan identified one open port (443/tcp) running an Nginx web server. The detected version, \textbf{Nginx 1.18.0}, was released in April 2020. This version is significantly outdated and is known to be affected by multiple Common Vulnerabilities and Exposures (CVEs). Running outdated software on internet-facing systems presents a high risk of compromise, as attackers can exploit well-documented vulnerabilities to gain unauthorized access.

% --- RISK ASSESSMENT ---
\section{Risk Assessment}
The following table synthesizes findings from the security control review and the technical scan. No pre-existing risks were reported.

\begin{table}[h!]
\centering
\caption{Summary of Identified Risks}
\begin{tabular}{p{0.1\linewidth} p{0.3\linewidth} p{0.4\linewidth} p{0.1\linewidth}}
\toprule
\textbf{Risk ID} & \textbf{Risk Name} & \textbf{Description} & \textbf{Severity} \\
\midrule
R-01 & Lack of MFA on Endpoints & User computers are not protected by MFA, allowing an attacker with valid credentials to gain access to the endpoint and internal network. & Critical \\
\addlinespace
R-02 & Lack of MFA on Sensitive Systems & Critical data systems are not protected by MFA, exposing sensitive information to unauthorized access if an employee's account is compromised. & Critical \\
\addlinespace
R-03 & Outdated Web Server Software & The public-facing web server runs Nginx 1.18.0, which has multiple known vulnerabilities that could be exploited for remote code execution or denial of service. & High \\
\addlinespace
R-04 & Missing Acceptable Use Policy & The absence of a formal AUP creates a lack of guidance for employees on the secure use of company assets, increasing the likelihood of accidental or malicious insider threats. & High \\
\bottomrule
\end{tabular}
\end{table}

% --- RECOMMENDATIONS ---
\section{Recommendations}
Based on the identified risks, the following remediation actions are recommended to improve the security posture of \textbf{Fable \& Lore}.

\begin{enumerate}
    \item \textbf{Implement Comprehensive MFA (R-01, R-02):}
    Deploy a robust Multi-Factor Authentication solution across all employee endpoints (computer logins) and for access to all systems containing sensitive data. This is the highest priority action to mitigate the risk of account compromise.
    
    \item \textbf{Upgrade Web Server and Implement Patch Management (R-03):}
    Immediately upgrade the Nginx web server on host \seqsplit{\texttt{192.168.10.5}} from version 1.18.0 to the latest stable release. Furthermore, establish a formal patch management policy and program to ensure all systems are updated in a timely manner.
    
    \item \textbf{Develop and Enforce an Acceptable Use Policy (R-04):}
    Create a comprehensive Acceptable Use Policy (AUP) that clearly defines the rules and responsibilities for all employees when using company technology and data. This policy should be formally acknowledged by all current staff and integrated into the onboarding process for new hires.
\end{enumerate}

\end{document}
```