```latex
\documentclass[12pt]{article}

% Preamble: Required Packages
\usepackage[margin=1in]{geometry}
\usepackage{pifont} % For checkmarks and crosses
\usepackage{booktabs} % For professional tables
\usepackage{hyperref} % For clickable links and metadata
\usepackage{url} % For URL formatting
\usepackage{seqsplit} % To split long text strings in tt font
\usepackage{graphicx}
\usepackage{xcolor}

% --- Document Metadata ---
\hypersetup{
    colorlinks=true,
    linkcolor=blue,
    filecolor=magenta,      
    urlcolor=cyan,
    pdftitle={Cybersecurity Posture Report},
    pdfauthor={Cybersecurity Analysis Cell},
    pdfsubject={Security Assessment},
    pdfkeywords={Cybersecurity, Risk, Assessment},
    pdftoolbar=true,
}

% --- Custom Commands ---
\newcommand{\yes}{\ding{51}}
\newcommand{\no}{\ding{55}}

\begin{document}

% --- Title Page ---
\title{
    \vspace{2cm}
    \textbf{Cybersecurity Posture Report} \\
    \large \textit{Analysis and Recommendations} \\
    \vspace{1cm}
    \textbf{For: Orchid Isle}
}
\author{Cybersecurity Analysis Cell}
\date{\today}
\maketitle
\thispagestyle{empty}
\newpage

\tableofcontents
\newpage

% --- Section 1: Executive Summary ---
\section{Executive Summary}

This report provides a comprehensive assessment of the cybersecurity posture for \textbf{Orchid Isle}. The analysis is based on a synthesis of organizational security control data, an external network vulnerability scan, and a review of pre-existing risks.

The assessment revealed a mixed security posture. On a positive note, the external network scan of the designated target IP address (\texttt{[Target IP]}) did not identify any open ports. This suggests a strong firewall configuration and a hardened external perimeter for the scanned asset, which is a commendable security practice.

However, significant gaps were identified in the organization's administrative and access controls. The two most critical findings are:
\begin{itemize}
    \item \textbf{Critical Risk:} The absence of Multi-Factor Authentication (MFA) for logging into employee computers. This exposes the organization to substantial risk from credential theft, which could lead to unauthorized access and lateral movement within the network.
    \item \textbf{High Risk:} The lack of a formal Employee Acceptable Use Policy (AUP). This creates ambiguity regarding the secure use of company assets and data, and limits the organization's ability to enforce security standards.
\end{itemize}

While the external technical defenses appear robust for the asset scanned, the internal administrative and identity controls require immediate attention. The recommendations in this report are prioritized to address these critical and high-risk findings to strengthen the overall security posture of \textbf{Orchid Isle}.

% --- Section 2: Organizational Information ---
\section{Organizational Information}

The following details were provided for the assessment. This information is used to establish the context and scope of the review.

\begin{tabular}{@{}ll}
    \toprule
    \textbf{Attribute} & \textbf{Value} \\
    \midrule
    Organization Name & Orchid Isle \\
    Email Domain & \seqsplit{\texttt{OrchidIsle.net}} \\
    Website Domain & \url{www.OrchidIsle.net} \\
    External IP Address & \seqsplit{\texttt{105.26.55.10}} \\
    \bottomrule
\end{tabular}

% --- Section 3: Security Control Review ---
\section{Security Control Review}

A review of key organizational security controls was conducted via a questionnaire. The responses highlight areas of strength and identify significant gaps in the current security framework.

\begin{table}[h!]
\centering
\begin{tabular}{p{0.6\textwidth} c c}
    \toprule
    \textbf{Control Question} & \textbf{Response} & \textbf{Status} \\
    \midrule
    Do you require MFA to access email? & Yes & \yes \\
    Do you require MFA to log into computers? & No & \textcolor{red}{\no} \\
    Do you require MFA to access sensitive data systems? & Yes & \yes \\
    Does your organization have an employee acceptable use policy? & No & \textcolor{red}{\no} \\
    Does your organization do security awareness training for new employees? & Yes & \yes \\
    Does your organization do security awareness training for all employees at least once per year? & Yes & \yes \\
    \bottomrule
\end{tabular}
\caption{Organizational Security Control Status}
\end{table}

% --- Section 4: Technical Scan Results ---
\section{Technical Scan Results}

An external network scan was performed to identify open ports and services that could be exposed to potential attackers.

\begin{itemize}
    \item \textbf{Target IP Address:} \texttt{[Target IP]}
    \item \textbf{Scan Date:} \today
\end{itemize}

\subsection{Summary of Findings}
The scan against the target IP address did not identify any open TCP or UDP ports. All scanned ports were found to be in a `closed` or `filtered` state.

\subsection{Analysis}
This is a positive security finding. It indicates that the host is likely protected by a well-configured firewall that drops or rejects unsolicited incoming traffic. A minimal external attack surface is a fundamental principle of network security and significantly reduces the risk of remote exploitation. No vulnerabilities were discovered during this phase of the assessment.

% --- Section 5: Risk Assessment ---
\section{Risk Assessment}

This section synthesizes the findings from the security control review, technical scan, and any pre-existing risk data. The risks are prioritized by severity to guide remediation efforts. No pre-existing vulnerabilities were provided for this assessment.

\begin{table}[h!]
\centering
\begin{tabular}{p{0.1\textwidth} p{0.25\textwidth} p{0.45\textwidth} p{0.1\textwidth}}
    \toprule
    \textbf{Risk ID} & \textbf{Risk Name} & \textbf{Description} & \textbf{Severity} \\
    \midrule
    RISK-001 & Lack of MFA on Endpoints & User computers do not require MFA for login. This exposes the organization to significant risk from compromised credentials, allowing unauthorized access to internal resources and facilitating lateral movement. & \textbf{Critical} \\
    \addlinespace
    RISK-002 & Missing Acceptable Use Policy & The organization lacks a formal Acceptable Use Policy (AUP). This creates ambiguity regarding employee responsibilities for protecting company assets and data, and hinders enforcement of security standards. & \textbf{High} \\
    \bottomrule
\end{tabular}
\caption{Identified Risks and Severity}
\end{table}

% --- Section 6: Recommendations ---
\section{Recommendations}

The following actionable recommendations are provided to address the risks identified in the previous section.

\subsection{RISK-001: Implement MFA for All Endpoint Logins (Critical)}
\begin{itemize}
    \item \textbf{Priority:} \textcolor{red}{\textbf{Immediate}}
    \item \textbf{Action:} Deploy a mandatory Multi-Factor Authentication solution for all employee computer (desktop and laptop) logins. This is one of the most effective controls to prevent unauthorized access resulting from stolen or weak passwords.
    \item \textbf{Implementation Guidance:}
    \begin{itemize}
        \item Evaluate and select an MFA solution compatible with your operating systems (e.g., Windows Hello for Business, Duo, Okta).
        \item Options include authenticator apps (e.g., Google Authenticator, Microsoft Authenticator), hardware tokens (e.g., YubiKey), or biometric verification.
        \item Develop a phased rollout plan, starting with privileged users (administrators, executives) and expanding to all employees.
    \end{itemize}
\end{itemize}

\subsection{RISK-002: Develop and Implement an Acceptable Use Policy (High)}
\begin{itemize}
    \item \textbf{Priority:} \textbf{High}
    \item \textbf{Action:} Draft, approve, and disseminate a comprehensive Acceptable Use Policy (AUP) to all employees. This policy is a foundational administrative control that establishes the rules of behavior for using company technology and data.
    \item \textbf{Implementation Guidance:}
    \begin{itemize}
        \item The policy should clearly define acceptable and unacceptable uses of company networks, computers, email, and internet access.
        \item Include sections on data protection, password security, and incident reporting.
        \item Require all new and existing employees to read and formally acknowledge the policy.
        \item Schedule annual reviews of the policy to ensure it remains current with evolving threats and business needs.
    \end{itemize}
\end{itemize}

\end{document}
```