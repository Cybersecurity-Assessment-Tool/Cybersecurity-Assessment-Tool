```latex
\documentclass[12pt]{article}

% Preamble: Required Packages
\usepackage[margin=1in]{geometry}
\usepackage{pifont} % For checkmarks and crosses
\usepackage{booktabs} % For professional tables
\usepackage{hyperref} % For clickable links
\usepackage{url} % For URL formatting
\usepackage{seqsplit} % To split long text strings like URLs

% Document Metadata
\title{Cybersecurity Posture Assessment Report \\ \large For: \textbf{Open Door}}
\author{Cybersecurity Analyst}
\date{\today}

\begin{document}

\maketitle
\tableofcontents
\newpage

% ------------------------------------------------------------------
% Section 1: Executive Summary
% ------------------------------------------------------------------
\section{Executive Summary}

This report provides a cybersecurity posture assessment for \textbf{Open Door}, based on a review of organizational security controls and available technical data. The analysis synthesizes information from a security questionnaire, a network scan, and a list of pre-existing risks.

\textbf{Key Findings:}
The assessment reveals a mixed security posture. While the organization has implemented security awareness training and requires Multi-Factor Authentication (MFA) for email and sensitive systems, two critical gaps were identified that significantly increase risk:
\begin{itemize}
    \item \textbf{Lack of MFA on Endpoints:} Employee computers do not require MFA for login. This exposes the organization to significant risk from compromised credentials, as a single stolen password could grant an attacker network access.
    \item \textbf{Absence of an Acceptable Use Policy (AUP):} The organization does not have a formal AUP. This creates ambiguity regarding employee responsibilities for protecting company assets and handling data, increasing the likelihood of insider threats and policy violations.
\end{itemize}

\textbf{Data Limitations:}
It is critical to note that the provided network scan data (\texttt{Input\_1\_Network\_Scan\_JSON}) and the list of current risks (\texttt{Input\_3\_Current\_Risks\_JSON}) were corrupted and could not be parsed. Consequently, this report's findings are primarily based on the organizational questionnaire. The technical risk landscape remains unassessed.

\textbf{Overall Recommendation:}
Immediate action should be taken to address the identified high-risk gaps by mandating MFA for all computer logins and developing a comprehensive Acceptable Use Policy. A new, successful network vulnerability scan is urgently required to gain a complete picture of the organization's external attack surface.

% ------------------------------------------------------------------
% Section 2: Organizational Information
% ------------------------------------------------------------------
\section{Organizational Information}

The following details were provided by the client and used as the basis for this assessment.

\begin{tabular}{@{}ll}
    \toprule
    \textbf{Attribute} & \textbf{Value} \\
    \midrule
    Organization Name & \textbf{Open Door} \\
    Email Domain & \texttt{OpenDoor.org} \\
    Website Domain & \seqsplit{\url{www.OpenDoor.org}} \\
    External IP Address & \texttt{165.123.44.107} \\
    \bottomrule
\end{tabular}

% ------------------------------------------------------------------
% Section 3: Security Control Review
% ------------------------------------------------------------------
\section{Security Control Review}

The following table summarizes the organization's responses to a security controls questionnaire. Items marked with \ding{55} represent significant gaps in the security framework and are discussed in the Risk Assessment section.

\begin{table}[h!]
\centering
\begin{tabular}{@{}p{0.6\textwidth}cc@{}}
    \toprule
    \textbf{Control Question} & \textbf{Response} & \textbf{Assessment} \\
    \midrule
    Do you require MFA to access email? & Yes & \ding{51} \\
    Do you require MFA to log into computers? & No & \textbf{\color{red}\ding{55}} \\
    Do you require MFA to access sensitive data systems? & Yes & \ding{51} \\
    Does your organization have an employee acceptable use policy? & No & \textbf{\color{red}\ding{55}} \\
    Does your organization do security awareness training for new employees? & Yes & \ding{51} \\
    Does your organization do security awareness training for all employees at least once per year? & Yes & \ding{51} \\
    \bottomrule
\end{tabular}
\caption{Security Controls Questionnaire Results}
\end{label{tab:controls}
\end{table}

% ------------------------------------------------------------------
% Section 4: Technical Scan Results
% ------------------------------------------------------------------
\section{Technical Scan Results}

\subsection{Data Integrity Issue}

The network scan data file (\texttt{Input\_1\_Network\_Scan\_JSON}) provided for this assessment was malformed or corrupted. The system was unable to parse the contents to identify open ports, running services, or software versions.

\textbf{Impact:} A full analysis of the external attack surface for the target IP address (\texttt{165.123.44.107}) could not be performed. This represents a significant blind spot in the current assessment, as potential vulnerabilities in externally-facing services remain unidentified.

\subsection{Target Information}
\begin{itemize}
    \item \textbf{Target IP Address:} \texttt{165.123.44.107}
    \item \textbf{Scan Date:} Not Available
    \item \textbf{Open Ports \& Services:} Not Available
\end{itemize}

A new network vulnerability scan against the target IP is strongly recommended as a high-priority action.

% ------------------------------------------------------------------
% Section 5: Risk Assessment
% ------------------------------------------------------------------
\section{Risk Assessment}

This section details the risks identified from the available data. Due to the corrupted input for pre-existing risks and the network scan, this list is based solely on the Security Control Review.

\begin{table}[h!]
\centering
\begin{tabular}{@{}p{0.2\textwidth}p{0.6\textwidth}l@{}}
    \toprule
    \textbf{Risk Name} & \textbf{Overview} & \textbf{Severity} \\
    \midrule
    \textbf{Lack of MFA on Endpoints} & The absence of MFA for computer logins means that a single compromised password could allow an attacker to gain unauthorized access to an employee's machine and, potentially, the internal network. This greatly increases the risk of lateral movement and ransomware attacks. & \textbf{High} \\
    \addlinespace
    \textbf{Missing Acceptable Use Policy (AUP)} & Without a formal AUP, there are no established rules for employees regarding the use of company systems, data handling, and security responsibilities. This leads to inconsistent security practices and increases the risk of data leakage and insider threats. & \textbf{Medium} \\
    \addlinespace
    \textbf{Unknown External Vulnerabilities} & Due to the failed network scan, the organization's external attack surface is unknown. There may be unpatched services, misconfigurations, or outdated software exposed to the internet, creating undiscovered pathways for attackers. & \textbf{Unknown} \\
    \bottomrule
\end{tabular}
\caption{Identified Risks and Severity}
\label{tab:risks}
\end{table}

% ------------------------------------------------------------------
% Section 6: Recommendations
% ------------------------------------------------------------------
\section{Recommendations}

The following prioritized recommendations are provided to mitigate the identified risks and improve the overall security posture of \textbf{Open Door}.

\subsection{Priority 1: Implement MFA for Endpoint Logins (High)}
\begin{itemize}
    \item \textbf{Action:} Deploy a Multi-Factor Authentication solution for all employee computer (desktop and laptop) logins. This can be achieved through solutions integrated with your existing identity provider (e.g., Microsoft Azure AD, Okta) using methods like authenticator apps, hardware tokens, or biometrics.
    \item \textbf{Justification:} This is the single most effective control to mitigate the risk of compromised credentials. It ensures that even if an attacker steals a password, they cannot access the endpoint without the second factor.
\end{itemize}

\subsection{Priority 2: Develop and Implement an Acceptable Use Policy (Medium)}
\begin{itemize}
    \item \textbf{Action:} Draft, approve, and disseminate a formal Acceptable Use Policy (AUP) for all employees. This policy should clearly define rules for system access, data handling, internet usage, and personal device use. All employees must read and acknowledge the policy.
    \item \textbf{Justification:} An AUP establishes a baseline for secure behavior, reduces organizational liability, and empowers the organization to enforce security standards consistently.
\end{itemize}

\subsection{Priority 3: Conduct a New Network Vulnerability Scan (Informational)}
\begin{itemize}
    \item \textbf{Action:} Commission a new, comprehensive, and authenticated vulnerability scan of the external IP address \texttt{165.123.44.107} and any other public-facing assets.
    \item \textbf{Justification:} To address the gap left by the corrupted scan data, a new scan is essential to identify and remediate technical vulnerabilities before they can be exploited by attackers.
\end{itemize}

\subsection{Priority 4: Consolidate Risk Documentation (Informational)}
\begin{itemize}
    \item \textbf{Action:} Review and consolidate all existing risk documentation. The provided data on current risks was unavailable, indicating a potential gap in risk tracking.
    \item \textbf{Justification:} A centralized and up-to-date risk register is a foundational component of a mature security program, enabling informed decision-making and resource allocation.
\end{itemize}

\end{document}
```