```latex
\documentclass[12pt]{article}

% Preamble: Required Packages
\usepackage[margin=1in]{geometry}
\usepackage{pifont} % For checkmarks and crosses
\usepackage{booktabs} % For professional tables
\usepackage{hyperref} % For clickable links
\usepackage{url} % For URL formatting
\usepackage{seqsplit} % For splitting long strings
\usepackage[T1]{fontenc}

% Document Metadata
\title{Cybersecurity Posture Assessment Report}
\author{Cybersecurity Analysis Division}
\date{\today}

\begin{document}

\maketitle
\thispagestyle{empty}
\newpage

\tableofcontents
\thispagestyle{empty}
\newpage
\setcounter{page}{1}

% --- 1. Executive Overview ---
\section{Executive Overview}
This report details the findings of a cybersecurity posture assessment for \textbf{True Grit}. The evaluation combines an analysis of organizational security controls, a technical network scan, and a review of known risks.

The assessment reveals a mixed security posture. The organization demonstrates a strong commitment to identity and access management by enforcing Multi-Factor Authentication (MFA) across email, computers, and sensitive systems. This is a commendable and critical defense against unauthorized access.

However, significant gaps were identified in foundational security policies and procedures. The absence of an employee \textbf{Acceptable Use Policy (AUP)} and the lack of \textbf{security awareness training for new employees} represent high-risk vulnerabilities. These gaps expose the organization to insider threats, social engineering, and potential legal liabilities.

The external network scan of the designated target IP address did not identify any open ports. While this could indicate a strong firewall configuration, it also prevents a deeper analysis of externally-facing services.

This report provides a detailed breakdown of these findings and offers actionable recommendations to mitigate the identified risks and strengthen the overall security posture of \textbf{True Grit}.

% --- 2. Organizational Information ---
\section{Organizational Information}
The following details were provided for the assessment.

\begin{itemize}
    \item \textbf{Organization Name:} True Grit
    \item \textbf{Email Domain:} \texttt{TrueGrit.net}
    \item \textbf{Website Domain:} \url{www.TrueGrit.net}
    \item \textbf{External IP Assessed:} \texttt{182.2.130.20}
\end{itemize}

% --- 3. Security Control Review ---
\section{Security Control Review}
An assessment of internal security controls was conducted via a standardized questionnaire. The responses are summarized below. "No" answers indicate potential gaps in the security framework and are highlighted as risks.

\begin{table}[h!]
\centering
\caption{Organizational Security Control Questionnaire}
\begin{tabular}{p{0.6\textwidth} c l}
\toprule
\textbf{Control Question} & \textbf{Response} & \textbf{Assessment} \\
\midrule
Do you require MFA to access email? & \ding{51} & Strong Control \\
Do you require MFA to log into computers? & \ding{51} & Strong Control \\
Do you require MFA to access sensitive data systems? & \ding{51} & Strong Control \\
\addlinespace
Does your organization have an employee acceptable use policy? & \ding{55} & \textbf{Critical Gap} \\
Does your organization do security awareness training for new employees? & \ding{55} & \textbf{High Risk} \\
\addlinespace
Does your organization do security awareness training for all employees at least once per year? & \ding{51} & Good Practice \\
\bottomrule
\end{tabular}
\end{table}

% --- 4. Technical Scan Results ---
\section{Technical Scan Results}
A network scan was performed to identify externally accessible services and potential vulnerabilities.

\begin{itemize}
    \item \textbf{Target IP Address:} \texttt{[Target IP]}
    \item \textbf{Scan Date:} \today
\end{itemize}

\subsection{Summary of Findings}
The network scan did not detect any open ports on the target host. This result can indicate one of the following scenarios:
\begin{itemize}
    \item The host is protected by a robust firewall that is correctly configured to drop or reject unsolicited incoming traffic (stealth configuration).
    \item The host was offline or unreachable at the time of the scan.
    \item The scan was blocked by an upstream network security device.
\end{itemize}

\textbf{Conclusion:} No technical vulnerabilities could be identified from the external scan. The lack of exposed services is a positive security finding.

% --- 5. Risk Assessment Summary ---
\section{Risk Assessment Summary}
The following table synthesizes findings from the security control review, technical scan, and pre-existing risk data. The primary risks identified are procedural and policy-related.

\begin{table}[h!]
\centering
\caption{Identified Risks}
\begin{tabular}{p{0.1\textwidth} p{0.3\textwidth} p{0.4\textwidth} l}
\toprule
\textbf{Risk ID} & \textbf{Risk Name} & \textbf{Description} & \textbf{Severity} \\
\midrule
RISK-001 & Lack of Acceptable Use Policy (AUP) & Without a formal AUP, employees lack clear guidelines on the proper use of company assets. This increases the risk of data misuse, policy violations, and legal ambiguity. & \textbf{High} \\
\addlinespace
RISK-002 & No Onboarding Security Training & New employees are a prime target for social engineering. Failing to provide security training upon hiring leaves the organization vulnerable from day one. & \textbf{High} \\
\bottomrule
\end{tabular}
\end{table}

% --- 6. Recommendations ---
\section{Recommendations}
Based on the risk assessment, the following actions are recommended to enhance the organization's cybersecurity posture.

\begin{enumerate}
    \item \textbf{Develop and Implement an Acceptable Use Policy (AUP):}
    \begin{itemize}
        \item \textbf{Action:} Draft a comprehensive AUP that clearly defines the rules for using company networks, devices, software, and data. The policy should cover topics such as data handling, password security, internet usage, and incident reporting.
        \item \textbf{Impact:} Mitigates RISK-001. Establishes a clear security baseline for all employees and provides a basis for disciplinary action in case of violations.
    \end{itemize}
    \vspace{1em}

    \item \textbf{Establish a Mandatory Onboarding Security Training Program:}
    \begin{itemize}
        \item \textbf{Action:} Create a security awareness training module that is a mandatory part of the onboarding process for all new hires, to be completed within their first week. This should complement the existing annual training program.
        \item \textbf{Impact:} Mitigates RISK-002. Equips new employees with the knowledge to identify and resist common threats like phishing and social engineering, reducing the likelihood of a security incident.
    \end{itemize}
    \vspace{1em}
    
    \item \textbf{Validate Network Security Posture:}
    \begin{itemize}
        \item \textbf{Action:} While no open ports were found, it is recommended to conduct periodic, authenticated internal and external vulnerability scans to validate firewall rules and ensure no misconfigurations exist.
        \item \textbf{Impact:} Provides ongoing assurance that the external perimeter remains secure and that the initial scan results were not an anomaly.
    \end{itemize}
\end{enumerate}

\end{document}
```