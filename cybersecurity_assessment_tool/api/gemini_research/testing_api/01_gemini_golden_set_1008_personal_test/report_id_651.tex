```latex
\documentclass[12pt]{article}

% Preamble: Required Packages
\usepackage[margin=1in]{geometry}
\usepackage{pifont} % For symbols like checkmarks
\usepackage{booktabs} % For professional-looking tables
\usepackage{hyperref} % For clickable links and references
\usepackage{url}      % For formatting URLs
\usepackage{seqsplit} % For splitting long strings in texttt
\usepackage{xcolor}   % For colored text

% Document Setup
\hypersetup{
    colorlinks=true,
    linkcolor=blue,
    filecolor=magenta,
    urlcolor=cyan,
    pdftitle={Cybersecurity Posture Assessment Report},
    pdfauthor={Cybersecurity Analysis Division},
}

% Custom Commands for convenience
\newcommand{\yes}{\ding{51}} % Green checkmark
\newcommand{\no}{\ding{55}}  % Red X

% Document Title
\title{Cybersecurity Posture Assessment Report \\ \large For: \textbf{Obsidian Operatives}}
\author{Cybersecurity Analysis Division}
\date{\today}

\begin{document}

\maketitle

\section*{1. Executive Summary}

This report provides a comprehensive cybersecurity assessment for \textbf{Obsidian Operatives}, based on a synthesis of network scan data, organizational security controls, and pre-existing risk information. The analysis revealed several critical and high-risk vulnerabilities that require immediate attention.

The most significant findings include a complete lack of Multi-Factor Authentication (MFA) across all critical systems, a publicly exposed and outdated database server (MySQL 5.7.33), and an inadequate security awareness training schedule for existing employees. The exposed database directly confirms a known high-severity risk (CVSS 7.5), elevating the urgency of remediation.

These vulnerabilities, particularly the absence of MFA, create a substantial risk of unauthorized access, credential compromise, and potential data breach. We strongly recommend prioritizing the remediation steps outlined in Section 6 to mitigate these risks and improve the organization's overall security posture.

\section*{2. Organizational Information}

The following information was provided for the assessment.

\begin{tabular}{@{}ll}
\toprule
\textbf{Attribute} & \textbf{Value} \\
\midrule
Organization Name & \textbf{Obsidian Operatives} \\
Email Domain & \texttt{ObsidianOperatives.net} \\
Website Domain & \texttt{www.ObsidianOperatives.net} \\
External IP Address & \texttt{222.166.134.85} \\
\bottomrule
\end{tabular}

\section*{3. Security Control Review}

A review of the organization's security controls was conducted via a questionnaire. The results highlight significant gaps in access control and employee security training policies. The lack of MFA is a critical weakness.

\begin{table}[h!]
\centering
\begin{tabular}{p{0.7\textwidth}c}
\toprule
\textbf{Control Question} & \textbf{Status} \\
\midrule
Do you require MFA to access email? & \no \\
Do you require MFA to log into computers? & \no \\
Do you require MFA to access sensitive data systems? & \no \\
Does your organization have an employee acceptable use policy? & \yes \\
Does your organization do security awareness training for new employees? & \yes \\
Does your organization do security awareness training for all employees at least once per year? & \no \\
\bottomrule
\end{tabular}
\caption{Organizational Security Controls Questionnaire Results.}
\end{table}

\section*{4. Technical Scan Results}

An external network scan was performed on the target IP address \texttt{172.16.50.20}. The scan identified one open port exposing a critical database service to the network.

\subsection*{Open Ports and Services}

\begin{table}[h!]
\centering
\begin{tabular}{lllll}
\toprule
\textbf{Port} & \textbf{State} & \textbf{Service} & \textbf{Product} & \textbf{Version} \\
\midrule
3306/tcp & open & mysql & MySQL & 5.7.33 \\
\bottomrule
\end{tabular}
\caption{Network services identified on target \texttt{172.16.50.20}.}
\end{table}

\subsection*{Technical Analysis}

The scan confirms that a MySQL database server is directly accessible from the network. The identified version, \textbf{MySQL 5.7.33}, reached its official End of Life (EOL) in October 2023. EOL software no longer receives security updates from the vendor, making it an easy target for attackers leveraging known vulnerabilities. This finding directly correlates with and validates the pre-existing risk documented in Input 3.

\section*{5. Consolidated Risk Assessment}

The following table synthesizes findings from the security control review, technical scan, and pre-existing risk data into a prioritized list of security risks.

\begin{table}[h!]
\centering
\begin{tabular}{lp{0.55\textwidth}l}
\toprule
\textbf{Risk Name} & \textbf{Description} & \textbf{Severity} \\
\midrule
\textbf{Exposed \& Outdated Database} & A MySQL 5.7.33 database is publicly accessible on port 3306. The version is End-of-Life and unpatched. This aligns with a known risk. & \textcolor{red}{\textbf{Critical (7.5)}} \\
\addlinespace
\textbf{Critical MFA Gaps} & Multi-Factor Authentication is not enforced for email, computer logins, or access to sensitive systems, leaving them vulnerable to credential theft. & \textcolor{red}{\textbf{Critical}} \\
\addlinespace
\textbf{Inadequate Training Cadence} & While new hires receive training, the lack of an annual refresher for all employees increases susceptibility to phishing and social engineering attacks over time. & \textcolor{orange}{\textbf{High}} \\
\bottomrule
\end{tabular}
\caption{Summary of Identified Security Risks.}
\end{table}

\section*{6. Recommendations}

The following actions are recommended to mitigate the identified risks. They are prioritized based on severity and potential impact.

\begin{itemize}
    \item[\textbf{1.}] \textbf{(Immediate) Isolate the Database:}
    \begin{itemize}
        \item Immediately implement firewall rules to restrict all public access to TCP port 3306 on host \texttt{172.16.50.20}. Access should only be permitted from trusted internal IP addresses or via a secure VPN connection.
    \end{itemize}

    \item[\textbf{2.}] \textbf{(High Priority) Implement Multi-Factor Authentication (MFA):}
    \begin{itemize}
        \item Enforce MFA across all user accounts for critical services, starting with email (e.g., Office 365, Google Workspace) and any systems containing sensitive data.
        \item Develop a plan to roll out MFA for all employee computer logins.
    \end{itemize}

    \item[\textbf{3.}] \textbf{(High Priority) Remediate Outdated Software:}
    \begin{itemize}
        \item Develop a migration plan to upgrade the MySQL 5.7.33 database instance to a currently supported version (e.g., MySQL 8.x). This will ensure the system receives critical security patches.
    \end{itemize}

    \item[\textbf{4.}] \textbf{(Medium Priority) Enhance Security Training Program:}
    \begin{itemize}
        \item Establish a mandatory annual security awareness training program for all employees. This program should cover current threats such as phishing, ransomware, and proper data handling.
    \end{itemize}
\end{itemize}

\end{document}
```