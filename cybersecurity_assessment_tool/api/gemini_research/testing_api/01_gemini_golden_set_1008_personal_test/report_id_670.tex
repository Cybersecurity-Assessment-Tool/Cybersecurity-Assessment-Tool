```latex
\documentclass[12pt]{article}

% Preamble: Required Packages
\usepackage[margin=1in]{geometry}
\usepackage{pifont} % For checkmarks and crosses
\usepackage{booktabs} % For professional tables
\usepackage{hyperref} % For hyperlinks
\usepackage{url} % For URL formatting
\usepackage{seqsplit} % For splitting long text strings
\usepackage{graphicx}
\usepackage[utf8]{inputenc}

% Document Metadata
\title{Cybersecurity Posture Assessment Report}
\author{Cybersecurity Analysis Division}
\date{\today}

\hypersetup{
    colorlinks=true,
    linkcolor=blue,
    filecolor=magenta,      
    urlcolor=cyan,
    pdftitle={Cybersecurity Posture Assessment Report},
    pdfpagemode=FullScreen,
}

\begin{document}

\maketitle
\thispagestyle{empty}
\newpage
\tableofcontents
\newpage

\section{Executive Summary}

This report provides a comprehensive cybersecurity posture assessment for \textbf{Iron Oak Furniture}. The analysis is based on a review of organizational security controls, an external network scan, and pre-existing risk data.

The assessment reveals a mixed security posture. The organization has implemented foundational controls such as Multi-Factor Authentication (MFA) for email and computer access. However, critical gaps were identified that significantly increase the risk of a security incident. These include the lack of MFA for sensitive data systems, the absence of an employee Acceptable Use Policy, and no requirement for annual security awareness training for all staff.

The external network scan against the target IP address yielded no open ports, which may indicate a strong firewall configuration or a potential scanning issue. No pre-existing vulnerabilities were reported.

Immediate action is recommended to address the identified policy and access control deficiencies to mitigate risks related to data breaches and unauthorized access.

\section{Organizational Information}

The following details were provided for the assessment.

\begin{itemize}
    \item \textbf{Organization Name:} Iron Oak Furniture
    \item \textbf{Email Domain:} \texttt{IronOakFurniture.net}
    \item \textbf{Website Domain:} \url{www.IronOakFurniture.net}
    \item \textbf{External IP Address:} \texttt{190.172.203.225}
\end{itemize}

\section{Security Control Review}

A review of the organization's security controls was conducted via a questionnaire. The responses are summarized below. Items marked with a red 'X' (\ding{55}) represent significant gaps in the security framework and are discussed in the Risk Assessment section.

\begin{table}[h!]
\centering
\caption{Security Controls Questionnaire Results}
\begin{tabular}{p{0.8\linewidth}c}
\toprule
\textbf{Control Question} & \textbf{Status} \\
\midrule
Do you require MFA to access email? & \ding{51} \\
Do you require MFA to log into computers? & \ding{51} \\
Do you require MFA to access sensitive data systems? & \textbf{\color{red}\ding{55}} \\
Does your organization have an employee acceptable use policy? & \textbf{\color{red}\ding{55}} \\
Does your organization do security awareness training for new employees? & \ding{51} \\
Does your organization do security awareness training for all employees at least once per year? & \textbf{\color{red}\ding{55}} \\
\bottomrule
\end{tabular}
\end{table}

\section{Technical Scan Results}

An external network vulnerability scan was performed to identify open ports and services exposed to the internet.

\begin{itemize}
    \item \textbf{Target IP Address:} \texttt{[Target IP]}
    \item \textbf{Scan Date:} \today
\end{itemize}

\subsection{Summary of Findings}
The scan completed successfully but did not identify any open TCP or UDP ports on the target system. This outcome can indicate one of the following scenarios:
\begin{itemize}
    \item The target host has a properly configured firewall that is blocking all unsolicited incoming traffic, which is a positive security practice.
    \item The target host was offline or unreachable at the time of the scan.
    \item An Intrusion Prevention System (IPS) may have blocked the scan traffic.
\end{itemize}
As no services were discovered, no service-specific vulnerabilities could be identified from this scan.

\section{Risk Assessment}

This section correlates the findings from the security control review and technical scan. While no technical vulnerabilities were discovered, significant procedural and policy-based risks were identified.

\begin{table}[h!]
\centering
\caption{Identified Risks and Severity}
\begin{tabular}{p{0.25\linewidth}p{0.5\linewidth}p{0.15\linewidth}}
\toprule
\textbf{Risk Name} & \textbf{Overview} & \textbf{Severity} \\
\midrule
\textbf{Lack of MFA on Sensitive Systems} & The absence of MFA on systems containing sensitive data exposes this information to a high risk of unauthorized access if user credentials are compromised through phishing or other means. & \textbf{Critical} \\
\addlinespace
\textbf{No Employee Acceptable Use Policy (AUP)} & Without a formal AUP, there are no clear guidelines for employees on the acceptable use of company assets. This increases the risk of insider threat, data leakage, and unintentional security incidents. & \textbf{High} \\
\addlinespace
\textbf{Lack of Annual Security Training} & Security knowledge is perishable. Without mandatory annual training, employees are more likely to fall victim to evolving social engineering and phishing attacks, undermining the organization's security posture. & \textbf{High} \\
\bottomrule
\end{tabular}
\end{table}

\section{Recommendations}

To mitigate the identified risks and improve the overall security posture of \textbf{Iron Oak Furniture}, the following actions are recommended with priority.

\begin{enumerate}
    \item \textbf{Implement Comprehensive MFA (Priority: Critical):}
    \begin{itemize}
        \item Immediately deploy a robust MFA solution across all systems and applications that store, process, or transmit sensitive data.
        \item Prioritize systems containing customer information, financial records, and intellectual property.
        \item Ensure the MFA policy is enforced for all users, including administrators and third-party contractors.
    \end{itemize}

    \item \textbf{Develop and Enforce an Acceptable Use Policy (Priority: High):}
    \begin{itemize}
        \item Create a formal AUP that clearly defines the rules and responsibilities for all employees when using company networks, devices, and data.
        \item The policy should cover topics such as data handling, internet usage, password security, and incident reporting.
        \item Require all current and new employees to read and formally acknowledge the policy.
    \end{itemize}

    \item \textbf{Establish a Mandatory Annual Security Training Program (Priority: High):}
    \begin{itemize}
        \item Institute a security awareness training program that is mandatory for all employees on an annual basis.
        \item The training should cover current threats, including phishing, ransomware, social engineering, and best practices for data protection.
        \item Track completion to ensure 100\% participation.
    \end{itemize}
\end{enumerate}

\section{Conclusion}

\textbf{Iron Oak Furniture} has established a baseline of security with its current MFA implementation for email and workstations. However, the critical and high-risk findings identified in this report expose the organization to significant threats, particularly concerning the protection of its sensitive data.

By prioritizing the recommendations—implementing comprehensive MFA, establishing a formal AUP, and mandating annual security training—the organization can substantially reduce its risk exposure and build a more resilient and mature security program.

\end{document}
```