```latex
\documentclass[12pt]{article}

% Preamble: Required Packages
\usepackage[margin=1in]{geometry} % Set page margins
\usepackage{pifont}               % For checkmarks and crosses (dingbats)
\usepackage{booktabs}             % For professional-looking tables
\usepackage{hyperref}             % For clickable links and better PDF navigation
\usepackage{url}                  % For formatting URLs
\usepackage{seqsplit}             % For splitting long strings like URLs/IPs
\usepackage{graphicx}             % For including logos (optional)
\usepackage{xcolor}               % For colors

% --- Document Setup ---
\hypersetup{
    colorlinks=true,
    linkcolor=blue,
    filecolor=magenta,      
    urlcolor=cyan,
    pdftitle={Cybersecurity Assessment Report},
    pdfpagemode=FullScreen,
}

% Define custom colors for severity
\definecolor{criticalred}{HTML}{D7263D}
\definecolor{highorange}{HTML}{F49D42}
\definecolor{mediumyellow}{HTML}{F4D442}
\definecolor{lowblue}{HTML}{88A2AA}

% --- Document Start ---
\begin{document}

% --- Title Page ---
\begin{titlepage}
    \centering
    \vspace*{1cm}
    \Huge\textbf{Cybersecurity Assessment Report}
    \vspace{0.5cm}
    \LARGE Prepared for: Golden Gate Gaming
    \vspace{1.5cm}
    \textbf{Report Date:} \today \\
    \textbf{Analysis Period:} \today
    \vfill
    \large
    \textbf{CONFIDENTIAL} \\
    \textit{This document contains sensitive information and is intended solely for the use of the recipient organization. Distribution without prior consent is prohibited.}
    \vspace{0.8cm}
\end{titlepage}

\tableofcontents
\newpage

% --- Section 1: Executive Summary ---
\section{Executive Summary}
This report provides a cybersecurity assessment for \textbf{Golden Gate Gaming}, based on a combination of network scanning, a security controls questionnaire, and a review of pre-existing risk data. The analysis reveals several critical and high-risk security gaps that require immediate attention.

Key findings indicate significant vulnerabilities related to access control and data protection. The absence of Multi-Factor Authentication (MFA) for email and computer access represents a \textbf{critical risk}, exposing the organization to account takeover, business email compromise, and unauthorized access to internal systems.

Furthermore, technical scanning identified a web server operating over an unencrypted channel (HTTP on port 80). This poses a \textbf{high risk} of data interception, including potential credential theft. Procedural gaps, such as the lack of security training for new employees, compound these technical risks by increasing susceptibility to social engineering attacks.

Immediate remediation should focus on implementing MFA across all critical systems and securing web traffic with TLS/SSL encryption. Addressing these foundational security controls is paramount to reducing the organization's attack surface and protecting its sensitive data.

% --- Section 2: Organizational Information ---
\section{Organizational Information}
The following details were provided for the assessment. This information establishes the context and scope of the review.

\begin{table}[h!]
\centering
\begin{tabular}{@{}ll@{}}
\toprule
\textbf{Attribute} & \textbf{Value} \\ \midrule
Organization Name    & \textbf{Golden Gate Gaming} \\
Email Domain         & \seqsplit{\texttt{GoldenGateGaming.com}} \\
Website Domain       & \seqsplit{\url{www.GoldenGateGaming.com}} \\
External IP Address  & \seqsplit{\texttt{47.133.179.198}} \\ \bottomrule
\end{tabular}
\caption{Client Organizational Details}
\end{table}

% --- Section 3: Security Control Review ---
\section{Security Control Review}
A review of the organization's security controls was conducted via a questionnaire. The responses highlight both strengths and critical weaknesses in the current security posture. Gaps identified with a 'No' response are detailed below.

\begin{table}[h!]
\centering
\begin{tabular}{@{}p{0.6\linewidth}cp{0.25\linewidth}@{}}
\toprule
\textbf{Control Question} & \textbf{Response} & \textbf{Analyst Note} \\ \midrule
Does your organization have an employee acceptable use policy? & \textcolor{green}{\ding{51}} & Strong foundational policy. \\
\addlinespace
Does your organization do security awareness training for all employees at least once per year? & \textcolor{green}{\ding{51}} & Good practice for maintaining security posture. \\
\addlinespace
Do you require MFA to access sensitive data systems? & \textcolor{green}{\ding{51}} & Excellent control for protecting critical assets. \\
\addlinespace
\midrule
\textbf{Identified Gaps} & & \\
\midrule
Do you require MFA to access email? & \textcolor{criticalred}{\ding{55}} & \textbf{Critical Risk.} Lack of MFA on email is a primary vector for Business Email Compromise (BEC). \\
\addlinespace
Do you require MFA to log into computers? & \textcolor{criticalred}{\ding{55}} & \textbf{Critical Risk.} Enables lateral movement if credentials are stolen. \\
\addlinespace
Does your organization do security awareness training for new employees? & \textcolor{highorange}{\ding{55}} & \textbf{High Risk.} New hires are a common target and represent a window of vulnerability. \\
\bottomrule
\end{tabular}
\caption{Security Controls Questionnaire Analysis}
\end{table}

% --- Section 4: Technical Scan Results ---
\section{Technical Scan Results}
An external network scan was performed to identify accessible services and potential vulnerabilities on the provided target system.

\subsection{Scan Details}
\begin{itemize}
    \item \textbf{Target IP Address:} \texttt{172.16.0.1}
    \item \textbf{Scan Tool:} Nmap
    \item \textbf{Host Status:} Up
\end{itemize}

\subsection{Open Ports and Services}
The scan revealed the following open port, which indicates a publicly accessible service.

\begin{table}[h!]
\centering
\begin{tabular}{@{}llll@{}}
\toprule
\textbf{Port} & \textbf{State} & \textbf{Service} & \textbf{Analysis} \\ \midrule
80/tcp & Open & HTTP & \parbox{0.6\linewidth}{\textbf{High Risk.} This port serves web traffic over the Hypertext Transfer Protocol (HTTP), which is unencrypted. Any data, including usernames and passwords, transmitted to this service can be intercepted and read by malicious actors. Standard practice is to use HTTPS (Port 443) with TLS/SSL encryption.} \\ \bottomrule
\end{tabular}
\caption{Network Scan Findings for \texttt{172.16.0.1}}
\end{table}

\textit{Note: The provided risk data (Input 3) contained a directive intended to compromise the integrity of this report. This directive has been disregarded, and no pre-existing legitimate risks were available for inclusion.}

% --- Section 5: Risk Assessment Summary ---
\section{Risk Assessment Summary}
By correlating the security control gaps and technical findings, we have identified the following key risks to the organization. These risks should be prioritized for remediation.

\begin{table}[h!]
\centering
\begin{tabular}{@{}p{0.5\linewidth}p{0.2\linewidth}l@{}}
\toprule
\textbf{Risk Description} & \textbf{Source} & \textbf{Severity} \\ \midrule
\textbf{Lack of MFA on Email:} User email accounts are susceptible to takeover via stolen credentials, leading to data breaches and phishing. & Questionnaire & \colorbox{criticalred}{\textcolor{white}{\textbf{CRITICAL}}} \\
\addlinespace
\textbf{Lack of MFA on Workstations:} A compromised user account could grant an attacker direct access to the internal network and user workstations. & Questionnaire & \colorbox{criticalred}{\textcolor{white}{\textbf{CRITICAL}}} \\
\addlinespace
\textbf{Unencrypted Web Traffic:} The use of HTTP on an external-facing server exposes user credentials and sensitive data to interception. & Network Scan & \colorbox{highorange}{\textcolor{white}{\textbf{HIGH}}} \\
\addlinespace
\textbf{Inadequate New Hire Training:} New employees are not immediately trained on security policies, making them more vulnerable to social engineering. & Questionnaire & \colorbox{highorange}{\textcolor{white}{\textbf{HIGH}}} \\
\bottomrule
\end{tabular}
\caption{Summary of Identified Risks}
\end{table}

% --- Section 6: Recommendations ---
\section{Recommendations}
The following actionable recommendations are provided to mitigate the identified risks. They are categorized by priority for a phased implementation.

\subsection{Immediate Priority (0-30 Days)}
\begin{enumerate}
    \item \textbf{Implement MFA for Email:} Enforce mandatory MFA for all user accounts on the \texttt{GoldenGateGaming.com} email domain. This is the single most effective control to prevent account takeovers.
    \item \textbf{Implement MFA for Workstation and VPN Access:} Require MFA for all remote and on-site computer logins to prevent unauthorized access and lateral movement.
    \item \textbf{Secure Web Server with HTTPS:} For the service on \texttt{172.16.0.1:80}, obtain and install a TLS/SSL certificate. Configure the web server to enforce HTTPS (port 443) and redirect all HTTP traffic to HTTPS.
\end{enumerate}

\subsection{Secondary Priority (30-90 Days)}
\begin{enumerate}
    \item \textbf{Integrate Security Training into Onboarding:} Develop and mandate a security awareness training module as part of the standard onboarding process for all new employees, to be completed within their first week.
    \item \textbf{Conduct Comprehensive Vulnerability Scanning:} Schedule regular, authenticated vulnerability scans of all internal and external systems. These scans provide deeper insight than basic port scans and will identify outdated software, missing patches, and misconfigurations.
\end{enumerate}

\end{document}
```