Here is the complete and professional LaTeX report, synthesized from the provided data inputs.

```latex
\documentclass[12pt]{article}

% Preamble: Required Packages
\usepackage[margin=1in]{geometry}
\usepackage{pifont} % For \ding
\usepackage{booktabs} % For professional tables
\usepackage{hyperref} % For clickable links
\usepackage{url} % For URL formatting
\usepackage{seqsplit} % For splitting long strings
\usepackage{graphicx}
\usepackage{xcolor}
\usepackage{fancyhdr} % For headers and footers

% --- Document Setup ---
\hypersetup{
    colorlinks=true,
    linkcolor=blue,
    filecolor=magenta,      
    urlcolor=cyan,
}

% --- Header and Footer ---
\pagestyle{fancy}
\fancyhf{} % Clear all header and footer fields
\fancyhead[L]{Cybersecurity Posture Assessment}
\fancyhead[R]{Solid State}
\fancyfoot[C]{\thepage}
\renewcommand{\headrulewidth}{0.4pt}
\renewcommand{\footrulewidth}{0.4pt}

% --- Document Start ---
\begin{document}

% --- Title Page ---
\begin{titlepage}
    \centering
    \vspace*{1cm}
    \Huge
    \textbf{Cybersecurity Posture Assessment Report}
    \vspace{1.5cm}
    \Large
    Prepared for: \\
    \vspace{0.5cm}
    \textbf{Solid State}
    \vspace{2cm}
    \rule{0.8\textwidth}{0.4pt}
    \vspace{1cm}
    \large
    Date: \today \\
    \vspace{0.5cm}
    Report ID: CYBER-2023-001
    \vfill
    \small
    \textit{This document contains sensitive information and is intended for the exclusive use of the recipient.}
\end{titlepage}

\tableofcontents
\newpage

% --- Section 1: Executive Summary ---
\section{Executive Summary}

This report details the findings of a cybersecurity posture assessment for \textbf{Solid State}. The analysis is based on a combination of self-reported organizational data, a security controls questionnaire, and an attempted technical network scan.

The assessment revealed several critical and high-risk security gaps originating from administrative and access control policies. The most significant findings include the lack of multi-factor authentication (MFA) for accessing email and sensitive data systems. These gaps expose the organization to a high risk of business email compromise, account takeover, and data breaches. Additionally, the absence of an employee Acceptable Use Policy (AUP) creates ambiguity regarding security responsibilities and increases the potential for insider threats.

It is crucial to note that the provided technical network scan data and the list of current risks were corrupted and could not be analyzed. This represents a significant visibility gap into the organization's external attack surface and historical risk landscape.

Immediate remediation is recommended to address the identified MFA and policy deficiencies. A follow-up technical vulnerability scan is essential to gain a complete picture of the organization's security posture.

% --- Section 2: Organizational Information ---
\section{Organizational Information}

The following details were provided for the assessment. This information establishes the scope and context for the analysis.

\begin{tabular}{@{}ll}
\toprule
\textbf{Attribute} & \textbf{Value} \\
\midrule
Organization Name & Solid State \\
Email Domain & \texttt{SolidState.org} \\
Website Domain & \texttt{www.SolidState.org} \\
External IP Address & \texttt{192.164.76.205} \\
\bottomrule
\end{tabular}

% --- Section 3: Security Control Review ---
\section{Security Control Review}

A review of the security controls questionnaire was conducted to evaluate the implementation of fundamental security practices. The responses are summarized below, with "No" answers indicating significant gaps that require attention.

\begin{tabular}{@{}p{0.6\linewidth}cp{0.2\linewidth}@{}}
\toprule
\textbf{Control Question} & \textbf{Response} & \textbf{Assessment} \\
\midrule
Do you require MFA to access email? & \textcolor{red}{\ding{55}} & \textbf{Critical Gap} \\
Do you require MFA to log into computers? & \textcolor{green}{\ding{51}} & Best Practice Met \\
Do you require MFA to access sensitive data systems? & \textcolor{red}{\ding{55}} & \textbf{Critical Gap} \\
Does your organization have an employee acceptable use policy? & \textcolor{red}{\ding{55}} & \textbf{High Risk} \\
Does your organization do security awareness training for new employees? & \textcolor{green}{\ding{51}} & Best Practice Met \\
Does your organization do security awareness training for all employees at least once per year? & \textcolor{green}{\ding{51}} & Best Practice Met \\
\bottomrule
\end{tabular}

% --- Section 4: Technical Scan Results ---
\section{Technical Scan Results}

The provided network scan data file (\texttt{Input\_1\_Network\_Scan\_JSON}) was found to be corrupted and could not be parsed. 

\textbf{Status:} \textcolor{red}{Scan Failed / Data Unavailable}

\textbf{Impact:} Due to the corrupted data, no technical analysis of externally exposed services, open ports, or software versions could be performed. This creates a significant blind spot regarding potential vulnerabilities on the organization's internet-facing systems. The intended target of the scan was not specified in the recoverable data. It is strongly recommended to conduct a new scan against the known external IP address (\texttt{192.164.76.205}).

% --- Section 5: Risk Assessment ---
\section{Risk Assessment}

This section summarizes the key risks identified during the assessment. The pre-existing risk data (\texttt{Input\_3\_Current\_Risks\_JSON}) was also unavailable, so this assessment is based solely on the new findings from the security control review and the failed technical scan.

\begin{tabular}{@{}p{0.1\linewidth}p{0.25\linewidth}p{0.45\linewidth}l@{}}
\toprule
\textbf{Risk ID} & \textbf{Risk Name} & \textbf{Description} & \textbf{Severity} \\
\midrule
RISK-001 & Lack of MFA on Email & Exposes the organization to a high likelihood of business email compromise (BEC), phishing attacks, and unauthorized access to sensitive communications. & \textcolor{red}{Critical} \\
\addlinespace
RISK-002 & Lack of MFA on Sensitive Data Systems & Absence of a critical access control layer greatly increases the risk of unauthorized access, modification, or exfiltration of critical company or customer data. & \textcolor{red}{Critical} \\
\addlinespace
RISK-003 & No Employee Acceptable Use Policy (AUP) & Creates ambiguity regarding employee responsibilities for security and data handling, increasing the risk of insider threats and non-compliance. & \textcolor{orange}{High} \\
\addlinespace
RISK-004 & Incomplete Security Visibility & The failure to obtain technical scan data prevents the identification of potentially exploitable vulnerabilities on external-facing services. & \textcolor{yellow}{Medium} \\
\bottomrule
\end{tabular}

% --- Section 6: Recommendations ---
\section{Recommendations}

The following actions are recommended to mitigate the identified risks and improve the overall security posture of \textbf{Solid State}. Recommendations are prioritized based on risk severity.

\begin{tabular}{@{}p{0.1\linewidth}p{0.25\linewidth}p{0.6\linewidth}@{}}
\toprule
\textbf{Priority} & \textbf{Recommendation} & \textbf{Actionable Steps} \\
\midrule
\textbf{1. Critical} & \textbf{Enforce MFA on Email} & Immediately enable and enforce MFA for all user accounts across the \texttt{SolidState.org} email system. This is the single most effective control to prevent account takeovers. \\
\addlinespace
\textbf{1. Critical} & \textbf{Enforce MFA on Sensitive Systems} & Identify all systems containing sensitive or critical data and immediately implement MFA as a requirement for access. \\
\addlinespace
\textbf{2. High} & \textbf{Develop and Implement an AUP} & Draft a comprehensive Acceptable Use Policy that clearly defines rules for computer, network, and data usage. Ensure all current and new employees read and formally acknowledge the policy. \\
\addlinespace
\textbf{3. Medium} & \textbf{Conduct External Vulnerability Scan} & Commission a new, comprehensive vulnerability scan of the external IP address (\texttt{192.164.76.205}) and any other public-facing assets. Remediate any discovered vulnerabilities based on severity. \\
\addlinespace
\textbf{3. Medium} & \textbf{Establish a Risk Register} & Create and maintain a centralized risk register to formally track all identified vulnerabilities, their severity, and the status of remediation efforts. \\
\bottomrule
\end{tabular}

\end{document}
```