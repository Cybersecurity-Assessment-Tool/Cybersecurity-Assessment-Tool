```latex
\documentclass[12pt]{article}

% Preamble: Required Packages
\usepackage[margin=1in]{geometry}
\usepackage{pifont} % For checkmarks and crosses
\usepackage{booktabs} % For professional tables
\usepackage{hyperref} % For clickable links
\usepackage{url} % For URL formatting
\usepackage{seqsplit} % For splitting long strings in tt font
\usepackage{graphicx} % For potential logos
\usepackage{xcolor} % For colors

% Document Information
\title{Cybersecurity Posture Assessment Report}
\author{Cybersecurity Analyst}
\date{\today}

% Hyperref Setup
\hypersetup{
    colorlinks=true,
    linkcolor=blue,
    filecolor=magenta,      
    urlcolor=cyan,
    pdftitle={Cybersecurity Posture Assessment Report},
    pdfpagemode=FullScreen,
}

\begin{document}

\maketitle
\thispagestyle{empty}
\newpage

\tableofcontents
\newpage

% ==============================================================================
% 1. Executive Summary
% ==============================================================================
\section{Executive Summary}

This report provides a comprehensive cybersecurity posture assessment for \textbf{Apex Legends Group}. The analysis is based on a correlation of network scan data, a security controls questionnaire, and a review of pre-existing risk documentation.

The assessment identified several high-priority areas for improvement. The most critical findings are significant gaps in the implementation of Multi-Factor Authentication (MFA) for email and sensitive data systems. These gaps expose the organization to substantial risks, including account compromise, data breaches, and phishing attacks.

Additionally, technical scanning revealed an open port for unencrypted web traffic (HTTP), which poses a risk to data confidentiality and integrity.

This report outlines these findings in detail and provides actionable recommendations to mitigate the identified risks and strengthen the overall security posture of the organization.

% ==============================================================================
% 2. Organizational Information
% ==============================================================================
\section{Organizational Information}

The following information was provided for the assessment.

\begin{itemize}
    \item \textbf{Organization Name:} Apex Legends Group
    \item \textbf{Primary Email Domain:} \texttt{ApexLegendsGroup.com}
    \item \textbf{Primary Website Domain:} \url{www.ApexLegendsGroup.com}
    \item \textbf{External IP Address:} \texttt{217.76.251.105}
\end{itemize}

% ==============================================================================
% 3. Security Control Review
% ==============================================================================
\section{Security Control Review}

A review of the organization's security controls was conducted via a questionnaire. The responses indicate a solid foundation in policy and training but reveal critical deficiencies in access control measures. The table below summarizes the provided answers.

\begin{table}[h!]
\centering
\caption{Security Controls Questionnaire Results}
\begin{tabular}{p{0.7\linewidth} c}
\toprule
\textbf{Control Question} & \textbf{Response} \\
\midrule
Do you require MFA to access email? & \ding{55} \\
Do you require MFA to log into computers? & \ding{51} \\
Do you require MFA to access sensitive data systems? & \ding{55} \\
Does your organization have an employee acceptable use policy? & \ding{51} \\
Does your organization do security awareness training for new employees? & \ding{51} \\
Does your organization do security awareness training for all employees at least once per year? & \ding{51} \\
\bottomrule
\end{tabular}
\end{table}

\subsection*{Analysis of Gaps}
The two negative responses (\ding{55}) represent significant security gaps:
\begin{itemize}
    \item \textbf{No MFA for Email:} Email is a primary target for attackers. Without MFA, a compromised password is all that is needed for an attacker to gain access, read sensitive communications, and launch further attacks against employees and partners.
    \item \textbf{No MFA for Sensitive Data Systems:} The lack of a secondary authentication factor for systems containing sensitive data is a critical vulnerability. This significantly increases the risk of a data breach resulting from stolen credentials.
\end{itemize}

% ==============================================================================
% 4. Technical Scan Results
% ==============================================================================
\section{Technical Scan Results}

An external network scan was performed to identify accessible services and potential vulnerabilities.

\begin{itemize}
    \item \textbf{Target IP Address:} \texttt{172.16.0.1}
    \item \textbf{Scan Date:} \today
\end{itemize}

\subsection*{Open Ports}
The following table details the open ports discovered on the target system.

\begin{table}[h!]
\centering
\caption{Discovered Open Ports}
\begin{tabular}{l l l p{0.5\linewidth}}
\toprule
\textbf{Port} & \textbf{State} & \textbf{Service} & \textbf{Notes} \\
\midrule
80/tcp & open & http & The presence of an open HTTP port indicates that unencrypted web traffic is permitted. This can expose user credentials and other sensitive data transmitted to the web server. \\
\bottomrule
\end{tabular}
\end{table}

% ==============================================================================
% 5. Risk Assessment Summary
% ==============================================================================
\section{Risk Assessment Summary}

The following table synthesizes findings from the security control review, technical scan, and pre-existing risk documentation into a prioritized list.

\begin{table}[h!]
\centering
\caption{Consolidated Risk Register}
\begin{tabular}{p{0.1\linewidth} p{0.3\linewidth} p{0.15\linewidth} p{0.35\linewidth}}
\toprule
\textbf{ID} & \textbf{Risk Name} & \textbf{Severity} & \textbf{Description} \\
\midrule
RISK-001 & No MFA on Sensitive Data Systems & \textbf{Critical} & Lack of MFA on critical systems makes them highly susceptible to unauthorized access and data breach via compromised credentials. \\
\addlinespace
RISK-002 & No MFA on Email & \textbf{High} & Email accounts are vulnerable to takeover, which can lead to phishing, business email compromise (BEC), and further network intrusion. \\
\addlinespace
RISK-003 & Unencrypted Web Traffic (HTTP) & \textbf{Medium} & The web server allows unencrypted connections over port 80, exposing transmitted data to interception and manipulation (Man-in-the-Middle attacks). \\
\addlinespace
RISK-004 & System Overriden & Informational & A pre-existing risk entry with a CVSS score of 0.0 was noted. This item appears to be invalid or erroneous data and presents no actual threat. \\
\bottomrule
\end{tabular}
\end{table}

% ==============================================================================
% 6. Recommendations
% ==============================================================================
\section{Recommendations}

Based on the findings of this assessment, the following actions are recommended to mitigate the identified risks.

\subsection*{Immediate Actions (1-7 Days)}
\begin{description}
    \item[RISK-001: Enforce MFA on Sensitive Systems] \hfill \\
    Immediately prioritize and enforce MFA for all user accounts with access to systems containing sensitive or critical data.
    
    \item[RISK-002: Enforce MFA on Email] \hfill \\
    Enable and enforce MFA for all email accounts across the organization. This is one of the most effective controls against phishing and account takeover.
\end{description}

\subsection*{Short-Term Actions (1-4 Weeks)}
\begin{description}
    \item[RISK-003: Remediate Unencrypted Web Traffic] \hfill \\
    \begin{itemize}
        \item Obtain and install a valid SSL/TLS certificate on the web server at \texttt{172.16.0.1}.
        \item Configure the web server to redirect all HTTP traffic (port 80) to HTTPS (port 443).
        \item Implement HTTP Strict Transport Security (HSTS) to ensure browsers only connect over HTTPS.
    \end{itemize}
\end{description}

\subsection*{Administrative Actions}
\begin{description}
    \item[RISK-004: Review Informational Risk] \hfill \\
    Review the pre-existing risk entry "System Overriden" and close it out in the risk register, as it does not represent a valid security threat.
\end{description}

\end{document}
```