```latex
\documentclass[12pt, a4paper]{article}

% Preamble: Required Packages
\usepackage[margin=1in]{geometry}
\usepackage{pifont} % For checkmarks and crosses
\usepackage{booktabs} % For professional tables
\usepackage{hyperref} % For clickable links
\usepackage{url} % For URL formatting
\usepackage{seqsplit} % For splitting long strings without spaces
\usepackage{graphicx} % For logo (placeholder)
\usepackage{fancyhdr} % For headers/footers

% --- Document Setup ---
\hypersetup{
    colorlinks=true,
    linkcolor=blue,
    filecolor=magenta,      
    urlcolor=cyan,
    pdftitle={Cybersecurity Assessment Report},
    pdfpagemode=FullScreen,
}

\pagestyle{fancy}
\fancyhf{}
\lhead{Cybersecurity Assessment Report}
\rhead{Grizzly Peak}
\cfoot{\thepage}

% --- Document Start ---
\begin{document}

% --- Title Page ---
\begin{titlepage}
    \centering
    \vspace*{1cm}
    
    \Huge
    \textbf{Cybersecurity Assessment Report}
    
    \vspace{1.5cm}
    
    \Large
    Prepared For: \\
    \vspace{0.5cm}
    \textbf{Grizzly Peak}
    
    \vspace{2cm}
    
    \large
    Report Date: \today \\
    Analysis Period: October 2023 % Placeholder Date
    
    \vfill
    
    \large
    \textbf{CONFIDENTIAL}
    
    \vspace{0.8cm}
    \small
    This document contains sensitive information. Access is restricted to authorized personnel only. Do not distribute without explicit permission.
    
\end{titlepage}

\tableofcontents
\newpage

% --- Section 1: Executive Summary ---
\section{Executive Summary}
This report provides a comprehensive analysis of the cybersecurity posture of \textbf{Grizzly Peak}, based on a review of organizational security controls, an external network scan, and pre-existing risk data.

The assessment identified a mixed security landscape. The organization has successfully implemented multi-factor authentication (MFA) across key areas, which is a significant strength. However, critical gaps were discovered in foundational security policies and employee training. Specifically, the absence of an Acceptable Use Policy (AUP) and the lack of security training for new hires represent high-risk vulnerabilities.

Technical analysis confirmed a pre-existing critical risk related to an exposed service on a local interface (\texttt{127.0.0.1}). This, combined with the policy gaps, indicates that while certain technical controls are in place, the human and procedural elements of the security program require immediate attention to mitigate significant threats, particularly those related to insider risk and social engineering.

This report outlines the findings in detail and provides prioritized, actionable recommendations to address these vulnerabilities and strengthen the overall security posture.

% --- Section 2: Organizational Information ---
\section{Organizational Information}
The following details were provided for the assessment. This information is used to establish the context and scope of the review.

\begin{tabular}{@{}ll}
    \toprule
    \textbf{Attribute} & \textbf{Value} \\
    \midrule
    Organization Name & \textbf{Grizzly Peak} \\
    Email Domain & \texttt{GrizzlyPeak.org} \\
    Website Domain & \url{www.GrizzlyPeak.org} \\
    External IP Address & \texttt{68.57.8.143} \\
    \bottomrule
\end{tabular}

% --- Section 3: Security Control Review ---
\section{Security Control Review}
A review of administrative and procedural security controls was conducted via a questionnaire. The responses highlight critical areas for improvement in security governance and employee awareness.

\begin{tabular}{@{}p{0.75\linewidth}c}
    \toprule
    \textbf{Control Question} & \textbf{Response} \\
    \midrule
    Do you require MFA to access email? & \ding{51} \\ % Yes
    Do you require MFA to log into computers? & \ding{51} \\ % Yes
    Do you require MFA to access sensitive data systems? & \ding{51} \\ % Yes
    Does your organization have an employee acceptable use policy? & \textbf{\color{red}\ding{55}} \\ % No
    Does your organization do security awareness training for new employees? & \textbf{\color{red}\ding{55}} \\ % No
    Does your organization do security awareness training for all employees at least once per year? & \ding{51} \\ % Yes
    \bottomrule
\end{tabular}

\subsection*{Analysis of Control Gaps}
\begin{itemize}
    \item \textbf{Lack of Acceptable Use Policy (AUP):} The absence of a formal AUP is a critical policy gap. An AUP defines the rules for using company IT assets, protecting both the employee and the organization. Without it, there is no clear standard for behavior, making it difficult to enforce security policies and manage insider risk.
    \item \textbf{No Security Training for New Hires:} New employees are often prime targets for phishing and other social engineering attacks. Failing to provide security awareness training during the onboarding process leaves the organization highly vulnerable during a new employee's critical first few weeks. While annual training is conducted, the initial gap is a significant weakness.
\end{itemize}

% --- Section 4: Technical Scan Results ---
\section{Technical Scan Results}
A network scan was performed to identify open ports and exposed services on the target system.

\subsection*{Scan Details}
\begin{itemize}
    \item \textbf{Target IP Address:} \texttt{127.0.0.1}
    \item \textbf{Scan Type:} TCP Port Scan (Nmap)
\end{itemize}

\subsection*{Open Ports Discovered}
The following table details the open ports found on the target system.
\begin{table}[h!]
\centering
\begin{tabular}{@{}llll@{}}
    \toprule
    \textbf{Port} & \textbf{State} & \textbf{Service (Inferred)} & \textbf{Notes} \\
    \midrule
    22/tcp & open & SSH & The service is likely Secure Shell (SSH). No version information was available. \\
    \bottomrule
\end{tabular}
\caption{Open Ports on \texttt{127.0.0.1}}
\end{table}

\subsection*{Technical Analysis}
The scan identified an open SSH port on the localhost interface (\texttt{127.0.0.1}). While this interface is typically not accessible from external networks, an open port can still pose a risk from local privilege escalation or service misconfigurations. This finding directly corroborates a pre-existing risk identified in the organization's risk register.

% --- Section 5: Consolidated Risk Assessment ---
\section{Consolidated Risk Assessment}
This section synthesizes findings from the security control review, technical scan, and pre-existing risk data into a consolidated list of identified risks.

\begin{table}[h!]
\centering
\begin{tabular}{@{}p{0.1\linewidth}p{0.25\linewidth}p{0.4\linewidth}p{0.15\linewidth}@{}}
    \toprule
    \textbf{Risk ID} & \textbf{Risk Name} & \textbf{Description} & \textbf{Severity} \\
    \midrule
    RISK-001 & Localhost Exposed & The SSH service (port 22) is open on the local loopback interface, posing a potential risk for local exploits or misconfigurations. This was confirmed by the technical scan. & \textbf{Critical} \\
    \addlinespace
    RISK-002 & Lack of Acceptable Use Policy & The absence of a formal AUP creates ambiguity regarding proper use of IT assets, increasing the likelihood of insider threats and unsafe user behavior. & \textbf{High} \\
    \addlinespace
    RISK-003 & Inadequate New Hire Security Training & New employees are not provided with security training upon being hired, making them highly susceptible to social engineering attacks. & \textbf{High} \\
    \bottomrule
\end{tabular}
\caption{Summary of Identified Risks}
\end{table}

% --- Section 6: Recommendations ---
\section{Recommendations}
The following prioritized recommendations are provided to address the identified risks and improve the overall security posture of \textbf{Grizzly Peak}.

\subsection*{Priority 1: Critical}
\begin{itemize}
    \item \textbf{Remediate Exposed Localhost Service (RISK-001):}
    \begin{itemize}
        \item \textbf{Action:} Investigate the purpose of the SSH service running on \texttt{127.0.0.1}. If it is not required for a specific application function, the service should be disabled immediately.
        \item \textbf{Justification:} Eliminates a direct technical vulnerability and reduces the local attack surface.
    \end{itemize}
\end{itemize}

\subsection*{Priority 2: High}
\begin{itemize}
    \item \textbf{Develop and Implement an Acceptable Use Policy (RISK-002):}
    \begin{itemize}
        \item \textbf{Action:} Draft a formal AUP that clearly outlines the rules for using company networks, devices, and data. This policy should be reviewed by management, communicated to all employees, and formally acknowledged via signature.
        \item \textbf{Justification:} Establishes a clear security baseline for all employees and provides an enforceable framework for security governance.
    \end{itemize}
    \item \textbf{Institute Mandatory Onboarding Security Training (RISK-003):}
    \begin{itemize}
        \item \textbf{Action:} Integrate a mandatory security awareness training module into the new employee onboarding process. This training should cover key topics such as phishing, password security, and the new AUP.
        \item \textbf{Justification:} Arms new employees with essential security knowledge from day one, significantly reducing their susceptibility to common cyberattacks.
    \end{itemize}
\end{itemize}

\subsection*{Priority 3: Medium}
\begin{itemize}
    \item \textbf{Enhance Vulnerability Scanning:}
    \begin{itemize}
        \item \textbf{Action:} Upgrade network scanning capabilities to include service and version detection. This will allow for the proactive identification of outdated and vulnerable software.
        \item \textbf{Justification:} Moves the organization from basic port scanning to a more mature vulnerability management process.
    \end{itemize}
\end{itemize}

% --- Document End ---
\end{document}
```