```latex
\documentclass[12pt]{article}

% --- PACKAGES ---
\usepackage[margin=1in]{geometry} % Set page margins
\usepackage{pifont}                 % For checkmarks and crosses (\ding)
\usepackage{booktabs}               % For professional-looking tables
\usepackage{hyperref}               % For clickable links
\usepackage{url}                    % For URL formatting
\usepackage{seqsplit}               % For splitting long strings in \texttt
\usepackage{graphicx}               % For logos (placeholder)
\usepackage{xcolor}                 % For colors

% --- DOCUMENT SETUP ---
\hypersetup{
    colorlinks=true,
    linkcolor=blue,
    filecolor=magenta,      
    urlcolor=cyan,
    pdftitle={Cybersecurity Posture Assessment Report},
    pdfpagemode=FullScreen,
}

% --- TITLE ---
\title{
    \vspace{2cm}
    \textbf{Cybersecurity Posture Assessment Report} \\
    \large \textit{Prepared for: Phoenix Rising}
    \vspace{1cm}
}
\author{Cybersecurity Analysis Division}
\date{\today}

% --- DOCUMENT START ---
\begin{document}

\maketitle
\thispagestyle{empty}
\newpage

\tableofcontents
\newpage

% --- EXECUTIVE SUMMARY ---
\section{Executive Summary}

This report provides a comprehensive analysis of the cybersecurity posture for Phoenix Rising, based on a combination of network scanning, a security controls questionnaire, and a review of known risks. The assessment was conducted to identify vulnerabilities, security gaps, and areas for improvement in the organization's defenses.

The overall security posture is assessed as having a \textbf{Moderate Risk} level. While the organization has implemented foundational controls such as Multi-Factor Authentication (MFA) for email and computer access, critical gaps were identified that expose the organization to significant threats.

Key findings include:
\begin{itemize}
    \item \textbf{Critical Risk:} The absence of MFA on systems handling sensitive data presents a severe risk. A single compromised credential could lead to a major data breach.
    \item \textbf{High Risk:} New employees do not receive security awareness training upon being hired, creating an immediate vulnerability to social engineering and phishing attacks.
    \item \textbf{Exposed Service:} An open Secure Shell (SSH) port was detected on an external-facing IPv6 address. This service is a common target for attackers and, when combined with the lack of MFA on sensitive systems, elevates the potential impact of a successful compromise.
\end{itemize}

This report details these findings and provides actionable recommendations to mitigate the identified risks and strengthen the overall security posture of Phoenix Rising.

% --- ORGANIZATIONAL INFORMATION ---
\section{Organizational Information}

The following information was provided for the assessment.

\begin{tabular}{@{}ll}
\toprule
\textbf{Attribute} & \textbf{Value} \\
\midrule
Organization Name & Phoenix Rising \\
Email Domain & \texttt{PhoenixRising.net} \\
Website Domain & \url{www.PhoenixRising.net} \\
Primary External IP & \texttt{226.1.214.182} \\
\bottomrule
\end{tabular}

% --- SECURITY CONTROL REVIEW ---
\section{Security Control Review}

A review of the organization's security controls was conducted via a questionnaire. The responses indicate a solid foundation in some areas but also reveal significant gaps in critical security layers. The symbol \ding{51} denotes a "Yes" response (control implemented), while \ding{55} denotes a "No" response (control gap).

\begin{table}[h!]
\centering
\caption{Security Controls Questionnaire Results}
\begin{tabular}{@{}p{0.8\linewidth}c@{}}
\toprule
\textbf{Control Question} & \textbf{Response} \\
\midrule
Do you require MFA to access email? & \ding{51} \\
Do you require MFA to log into computers? & \ding{51} \\
\textbf{Do you require MFA to access sensitive data systems?} & \textbf{\color{red}\ding{55}} \\
Does your organization have an employee acceptable use policy? & \ding{51} \\
\textbf{Does your organization do security awareness training for new employees?} & \textbf{\color{red}\ding{55}} \\
Does your organization do security awareness training for all employees at least once per year? & \ding{51} \\
\bottomrule
\end{tabular}
\end{table}

\subsection*{Analysis of Control Gaps}
\begin{itemize}
    \item \textbf{MFA on Sensitive Systems:} The lack of MFA on sensitive systems is a critical vulnerability. Should an attacker compromise credentials with access to these systems (e.g., a database administrator), there is no secondary authentication factor to prevent unauthorized access and data exfiltration.
    \item \textbf{New Employee Training:} Failing to train new employees on security best practices from day one exposes the organization to unnecessary risk. New hires are often targeted by phishing campaigns and may be unaware of internal policies, making them more susceptible to social engineering attacks.
\end{itemize}

% --- TECHNICAL SCAN RESULTS ---
\section{Technical Scan Results}

An external network scan was performed to identify exposed services on the organization's network infrastructure.

\begin{itemize}
    \item \textbf{Target IP Address:} \seqsplit{\texttt{2001:db8::1}}
\end{itemize}

\begin{table}[h!]
\centering
\caption{Open Ports Detected on Target IP}
\begin{tabular}{@{}llll@{}}
\toprule
\textbf{Port} & \textbf{State} & \textbf{Service} & \textbf{Notes} \\
\midrule
22/TCP & Open & SSH & Secure Shell is used for remote administration. No version \\
       &      & (Secure Shell) & information was retrieved. Exposed SSH services are a \\
       &      &                & prime target for brute-force and credential stuffing attacks. \\
\bottomrule
\end{tabular}
\end{table}

\subsection*{Analysis of Technical Findings}
The presence of an open SSH port on an external-facing system is a significant finding. While necessary for remote administration, it must be rigorously secured. The risk is amplified by the previously identified lack of MFA on sensitive systems, as this SSH service could be a direct entry point to such a system.

% --- RISK ASSESSMENT ---
\section{Risk Assessment}

The following table summarizes the key risks identified during this assessment, combining findings from the security control review and the technical scan. The `Current Risks` input was empty, indicating no pre-existing documented risks were provided for this assessment.

\begin{table}[h!]
\centering
\caption{Summary of Identified Risks}
\begin{tabular}{@{}p{0.1\linewidth}p{0.6\linewidth}l@{}}
\toprule
\textbf{Risk ID} & \textbf{Description} & \textbf{Severity} \\
\midrule
RISK-001 & Lack of MFA on sensitive data systems allows for unauthorized access with a single compromised credential. & \textbf{Critical} \\
\\
RISK-002 & No security awareness training for new employees creates a high likelihood of compromise via social engineering. & \textbf{High} \\
\\
RISK-003 & An externally exposed SSH service provides a direct vector for attackers to attempt unauthorized remote access. This risk is compounded by RISK-001. & \textbf{Medium} \\
\bottomrule
\end{tabular}
\end{table}

% --- RECOMMENDATIONS ---
\section{Recommendations}

The following actions are recommended to mitigate the identified risks and improve the overall security posture of Phoenix Rising.

\subsection*{Immediate Actions (To Address Critical/High Risks)}

\begin{enumerate}
    \item \textbf{Implement MFA on Sensitive Systems (RISK-001):}
    \begin{itemize}
        \item \textbf{Action:} Enforce mandatory multi-factor authentication (MFA) across all systems classified as containing sensitive data, especially for administrative access.
        \item \textbf{Priority:} Immediate. This is the single most effective control to mitigate the risk of a data breach from compromised credentials.
    \end{itemize}
    \vspace{0.5cm}
    \item \textbf{Integrate Security Training into Onboarding (RISK-002):}
    \begin{itemize}
        \item \textbf{Action:} Develop and mandate a security awareness training module as part of the new employee onboarding process. This module should cover phishing, password security, and the acceptable use policy.
        \item \textbf{Priority:} Immediate. This closes a critical window of vulnerability for new hires.
    \end{itemize}
\end{enumerate}

\subsection*{Secondary Actions}

\begin{enumerate}
    \setcounter{enumi}{2}
    \item \textbf{Secure the Exposed SSH Service (RISK-003):}
    \begin{itemize}
        \item \textbf{Action:} Review the business necessity of the public-facing SSH service on \seqsplit{\texttt{2001:db8::1}}.
        \begin{itemize}
            \item If required, implement IP whitelisting to restrict access to only trusted source IP addresses.
            \item Disable password-based authentication and enforce the use of strong cryptographic keys (e.g., ED25519).
            \item Implement monitoring and alerting for failed login attempts to detect brute-force attacks.
        \end{itemize}
        \item \textbf{Priority:} High. This hardens a key potential entry point for attackers.
    \end{itemize}
\end{enumerate}

% --- CONCLUSION ---
\section{Conclusion}

Phoenix Rising has established some important security controls, but critical gaps in its defenses require immediate attention. The lack of MFA on sensitive systems and the absence of security training for new employees represent the most significant threats. By implementing the recommendations outlined in this report, the organization can substantially reduce its risk exposure and build a more resilient security posture against modern cyber threats.

\end{document}
```