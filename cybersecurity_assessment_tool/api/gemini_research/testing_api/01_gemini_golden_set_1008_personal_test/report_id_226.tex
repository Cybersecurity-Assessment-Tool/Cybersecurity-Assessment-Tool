```latex
\documentclass[12pt]{article}

% Preamble: Required Packages
\usepackage[margin=1in]{geometry}
\usepackage{pifont} % For checkmarks and crosses
\usepackage{booktabs} % For professional tables
\usepackage{hyperref} % For clickable links
\usepackage{url} % For formatting URLs
\usepackage{seqsplit} % For splitting long strings in tt font
\usepackage{graphicx} % For logo (optional)
\usepackage{xcolor} % For colors

% Document Information
\title{Cybersecurity Posture Assessment Report}
\author{Cybersecurity Analyst}
\date{\today}

% Hyperref Setup
\hypersetup{
    colorlinks=true,
    linkcolor=blue,
    filecolor=magenta,      
    urlcolor=cyan,
    pdftitle={Cybersecurity Posture Assessment Report},
    pdfpagemode=FullScreen,
}

\begin{document}

\maketitle
\hrule
\vspace{1em}

% ------------------------------------------------------------------
% 1. Executive Summary
% ------------------------------------------------------------------
\section*{1. Executive Summary}

This report provides a comprehensive analysis of the cybersecurity posture for \textbf{Tidal Wave Sports}. The assessment is based on a correlation of organizational data, a security controls questionnaire, and an external network scan.

The overall security posture reveals a mixed landscape. Positive controls are in place, such as the enforcement of Multi-Factor Authentication (MFA) for email and sensitive data systems. The external network scan of the target host \texttt{[Target IP]} did not identify any open ports, suggesting effective firewall configurations or a minimized attack surface at the time of the scan.

However, several critical and high-risk gaps were identified in procedural and policy-based controls. The most significant risks include the lack of mandatory MFA for computer logins, the absence of a formal Acceptable Use Policy (AUP), and the failure to provide security awareness training to new employees during onboarding. These gaps expose the organization to significant threats, including unauthorized access, insider threats, and social engineering attacks.

Immediate remediation is recommended to address these deficiencies, focusing on strengthening endpoint security and formalizing security policies and training programs.

% ------------------------------------------------------------------
% 2. Organizational Information
% ------------------------------------------------------------------
\section*{2. Organizational Information}

The following information was provided for the assessment.

\begin{tabular}{@{}ll}
    \toprule
    \textbf{Attribute} & \textbf{Value} \\
    \midrule
    Organization Name & \textbf{Tidal Wave Sports} \\
    Email Domain & \texttt{TidalWaveSports.org} \\
    Website Domain & \url{www.TidalWaveSports.org} \\
    External IP Address & \texttt{58.164.216.172} \\
    \bottomrule
\end{tabular}

% ------------------------------------------------------------------
% 3. Security Control Review
% ------------------------------------------------------------------
\section*{3. Security Control Review}

A review of the organization's security controls was conducted based on a questionnaire. The responses highlight key areas of strength and weakness in the current security framework. A "No" response indicates a potential control gap that increases risk.

\begin{tabular}{@{}p{0.6\textwidth}p{0.15\textwidth}p{0.15\textwidth}@{}}
    \toprule
    \textbf{Control Question} & \textbf{Response} & \textbf{Status} \\
    \midrule
    Do you require MFA to access email? & Yes & \ding{51} \\
    Do you require MFA to log into computers? & No & \textcolor{red}{\ding{55}} \\
    Do you require MFA to access sensitive data systems? & Yes & \ding{51} \\
    Does your organization have an employee acceptable use policy? & No & \textcolor{red}{\ding{55}} \\
    Does your organization do security awareness training for new employees? & No & \textcolor{red}{\ding{55}} \\
    Does your organization do security awareness training for all employees at least once per year? & Yes & \ding{51} \\
    \bottomrule
\end{tabular}

% ------------------------------------------------------------------
% 4. Technical Scan Results
% ------------------------------------------------------------------
\section*{4. Technical Scan Results}

An external network scan was performed to identify exposed services and potential vulnerabilities.

\begin{itemize}
    \item \textbf{Scan Target:} \texttt{[Target IP]}
    \item \textbf{Scan Date:} \today
    \item \textbf{Summary:} The scan completed successfully. \textbf{No open ports were detected on the target host.}
\end{itemize}

\textbf{Analysis:} This result is a positive security finding. It indicates that the target system is likely protected by a well-configured firewall that blocks unsolicited inbound traffic from the internet. This significantly reduces the external attack surface. It is also possible the host was offline or unresponsive during the scan window.

% ------------------------------------------------------------------
% 5. Risk Assessment
% ------------------------------------------------------------------
\section*{5. Risk Assessment}

This section synthesizes findings from the security control review and technical scan. While no pre-existing vulnerabilities were provided, the following new risks have been identified based on this assessment.

\begin{tabular}{@{}p{0.25\textwidth}p{0.55\textwidth}p{0.1\textwidth}@{}}
    \toprule
    \textbf{Risk Name} & \textbf{Overview} & \textbf{Severity} \\
    \midrule
    \textbf{Lack of Workstation MFA} & The absence of MFA for computer logins means that a compromised password is all an attacker needs to gain access to an employee's workstation and potentially the internal network. This is a common vector for ransomware and data breaches. & \textbf{Critical} \\
    \addlinespace
    \textbf{Missing Acceptable Use Policy (AUP)} & Without a formal AUP, there are no clear guidelines for employees on the proper use of company assets. This can lead to unintentional misuse, security incidents, and creates legal and HR challenges when enforcing security rules. & \textbf{High} \\
    \addlinespace
    \textbf{No Onboarding Security Training} & New employees are a primary target for social engineering and phishing attacks. By not providing security training during onboarding, the organization misses a critical opportunity to educate them on policies, procedures, and current threats before they gain access to sensitive systems. & \textbf{High} \\
    \bottomrule
\end{tabular}

% ------------------------------------------------------------------
% 6. Recommendations
% ------------------------------------------------------------------
\section*{6. Recommendations}

Based on the identified risks, the following actions are recommended to improve the organization's cybersecurity posture. Recommendations are prioritized by severity.

\begin{enumerate}
    \item \textbf{Implement Workstation MFA (Critical):}
    \begin{itemize}
        \item \textbf{Action:} Deploy and mandate the use of Multi-Factor Authentication for all employee computer and laptop logins.
        \item \textbf{Impact:} Drastically reduces the risk of unauthorized access resulting from stolen or weak credentials. This is one of the most effective controls against common cyberattacks.
    \end{itemize}
    \vspace{1em}
    
    \item \textbf{Develop and Implement an Acceptable Use Policy (High):}
    \begin{itemize}
        \item \textbf{Action:} Draft a formal AUP that clearly defines the rules for using company networks, devices, and data. Require all current and new employees to read and formally acknowledge the policy.
        \item \textbf{Impact:} Establishes a clear security baseline for user behavior, reduces insider risk, and provides a framework for disciplinary action in case of violations.
    \end{itemize}
    \vspace{1em}

    \item \textbf{Institute Security Onboarding for New Hires (High):}
    \begin{itemize}
        \item \textbf{Action:} Integrate a mandatory security awareness training module into the new employee onboarding process. This should cover key topics such as phishing, password security, data handling, and the new AUP.
        \item \textbf{Impact:} Equips new employees with essential security knowledge from day one, reducing their susceptibility to attacks and reinforcing a culture of security throughout the organization.
    \end{itemize}
\end{enumerate}

\end{document}
```