```latex
\documentclass[12pt]{article}

% Preamble: Required packages and document settings
\usepackage[a4paper, margin=1in]{geometry}
\usepackage{pifont} % For checkmarks and crosses (\ding{51}, \ding{55})
\usepackage{booktabs} % For professional-looking tables
\usepackage{hyperref} % For clickable links and references
\usepackage{url} % For formatting URLs
\usepackage{seqsplit} % For splitting long strings in texttt
\usepackage{graphicx}
\usepackage{xcolor}
\usepackage{fancyhdr}
\usepackage{lastpage}

% --- Document Metadata and Hyperref Setup ---
\hypersetup{
    colorlinks=true,
    linkcolor=blue,
    filecolor=magenta,      
    urlcolor=cyan,
    pdftitle={Cybersecurity Posture Assessment Report},
    pdfauthor={Cybersecurity Analyst},
    pdfsubject={Security Analysis},
    pdfkeywords={Security, Report, Analysis},
    bookmarks=true
}

% --- Header and Footer Customization ---
\pagestyle{fancy}
\fancyhf{} % Clear all header and footer fields
\fancyhead[L]{Cybersecurity Posture Assessment}
\fancyhead[R]{Crestview Analytics}
\fancyfoot[C]{\thepage\ of \pageref{LastPage}}
\renewcommand{\headrulewidth}{0.4pt}
\renewcommand{\footrulewidth}{0.4pt}

% --- Title Page Definition ---
\title{
    \vspace{2cm}
    \textbf{Cybersecurity Posture Assessment Report}\\
    \vspace{0.5cm}
    \large{Prepared for: Crestview Analytics}\\
    \vspace{2cm}
}
\author{Cybersecurity Analyst}
\date{\today}

% --- Document Start ---
\begin{document}

\maketitle
\thispagestyle{empty}
\newpage

\tableofcontents
\thispagestyle{empty}
\newpage

\pagestyle{fancy} % Apply the fancy page style from here on

% ==============================================================================
\section*{1. Executive Summary}
% ==============================================================================

This report provides a comprehensive cybersecurity posture assessment for Crestview Analytics. The analysis is based on a combination of a network vulnerability scan, a review of organizational security controls via a questionnaire, and an evaluation of pre-existing risks.

The assessment reveals a mixed security posture. From a network perimeter perspective, the organization presents a strong defensive stance; the external scan of the target IP address \texttt{192.168.1.100} found no open ports, indicating a well-configured firewall that significantly reduces the external attack surface.

However, significant gaps were identified in internal security policies and procedures. Critical areas of concern include the absence of multi-factor authentication (MFA) for sensitive data systems, the lack of a formal employee acceptable use policy, and the failure to conduct annual security awareness training for all staff. These deficiencies expose the organization to substantial risks, including unauthorized data access, insider threats, and susceptibility to social engineering attacks.

This report details these findings and provides actionable recommendations to mitigate the identified risks and strengthen the overall security posture of Crestview Analytics.

% ==============================================================================
\section*{2. Organizational Information}
% ==============================================================================

The following information was provided for the assessment:

\begin{description}
    \item[Organization Name:] Crestview Analytics
    \item[Email Domain:] \texttt{CrestviewAnalytics.com}
    \item[Website Domain:] \url{www.CrestviewAnalytics.com}
    \item[External IP Address:] \texttt{131.31.204.235}
\end{description}

% ==============================================================================
\section*{3. Security Control Review}
% ==============================================================================

A review of administrative and technical security controls was conducted based on a questionnaire. The results are summarized below. Items marked with \ding{55} represent significant gaps in the organization's security framework and are correlated with risks identified in Section 5.

\begin{table}[h!]
\centering
\caption{Security Controls Questionnaire Results}
\begin{tabular}{p{0.6\linewidth} c p{0.2\linewidth}}
\toprule
\textbf{Control Question} & \textbf{Response} & \textbf{Assessment} \\
\midrule
Do you require MFA to access email? & \ding{51} & Best Practice Met \\
\addlinespace
Do you require MFA to log into computers? & \ding{51} & Best Practice Met \\
\addlinespace
Do you require MFA to access sensitive data systems? & \textcolor{red}{\ding{55}} & \textbf{High Risk} \\
\addlinespace
Does your organization have an employee acceptable use policy? & \textcolor{red}{\ding{55}} & \textbf{High Risk} \\
\addlinespace
Does your organization do security awareness training for new employees? & \ding{51} & Best Practice Met \\
\addlinespace
Does your organization do security awareness training for all employees at least once per year? & \textcolor{red}{\ding{55}} & \textbf{High Risk} \\
\bottomrule
\end{tabular}
\end{table}

% ==============================================================================
\section*{4. Technical Scan Results}
% ==============================================================================

An external network scan was performed to identify open ports and exposed services on the organization's infrastructure.

\subsection*{4.1. Scan Details}
\begin{description}
    \item[Target IP Address:] \texttt{192.168.1.100}
    \item[Scan Type:] Nmap TCP Scan
\end{description}

\subsection*{4.2. Findings}
The scan concluded that the host was online but reported that all scanned ports were in a \textbf{`closed`} state. No open ports or active services were discovered.

\textbf{Assessment:} This is a positive security finding. A lack of open ports on an externally facing system indicates a properly configured firewall that is effectively minimizing the network attack surface by denying unsolicited incoming traffic.

% ==============================================================================
\section*{5. Risk Assessment}
% ==============================================================================

This section synthesizes findings from the security control review, technical scan, and any pre-existing known vulnerabilities. The following table outlines the newly identified risks based on this assessment. No pre-existing vulnerabilities were reported.

\begin{table}[h!]
\centering
\caption{Identified Security Risks}
\begin{tabular}{p{0.05\linewidth} p{0.25\linewidth} p{0.45\linewidth} p{0.1\linewidth}}
\toprule
\textbf{ID} & \textbf{Risk Name} & \textbf{Overview} & \textbf{Severity} \\
\midrule
RISK-001 & Lack of MFA for Sensitive Data Systems & The absence of MFA on systems storing or processing sensitive information creates a single point of failure (password compromise) for a catastrophic data breach. & \textbf{High} \\
\addlinespace
RISK-002 & No Employee Acceptable Use Policy (AUP) & Without a formal AUP, there are no clear guidelines for employees on the proper use of company assets. This increases the risk of insider threats, data leakage, and legal liability. & \textbf{High} \\
\addlinespace
RISK-003 & No Annual Security Awareness Training & Failing to provide regular, ongoing training for all staff leaves the organization highly vulnerable to phishing, social engineering, and other human-centric attacks. Initial training for new hires is insufficient to maintain security awareness. & \textbf{High} \\
\bottomrule
\end{tabular}
\end{table}

% ==============================================================================
\section*{6. Recommendations}
% ==============================================================================

The following actions are recommended to mitigate the identified risks and improve the overall security posture of Crestview Analytics.

\subsection*{6.1. Remediate RISK-001: Implement MFA for Sensitive Data}
\begin{itemize}
    \item \textbf{Action:} Immediately prioritize and implement a robust MFA solution for all systems, applications, and databases that contain sensitive, confidential, or proprietary data.
    \item \textbf{Justification:} This will add a critical layer of security, ensuring that even if user credentials are stolen, unauthorized access to the organization's most valuable data is prevented.
\end{itemize}

\subsection*{6.2. Remediate RISK-002: Develop and Enforce an AUP}
\begin{itemize}
    \item \textbf{Action:} Develop a comprehensive Acceptable Use Policy (AUP) that clearly defines the rules and responsibilities for all employees when using company technology and data. This policy should be formally acknowledged by all staff members.
    \item \textbf{Justification:} An AUP establishes a baseline for secure behavior, reduces the risk of misuse of company assets, and provides a framework for disciplinary action in case of violations.
\end{itemize}

\subsection*{6.3. Remediate RISK-003: Establish Annual Security Training}
\begin{itemize}
    \item \textbf{Action:} Institute a mandatory, annual security awareness training program for all employees, including management. This program should cover current threats such as phishing, ransomware, password security, and social engineering.
    \item \textbf{Justification:} Regular training reinforces security best practices and keeps employees vigilant against evolving cyber threats, transforming the workforce from a potential vulnerability into a strong line of defense.
\end{itemize}

% ==============================================================================
\section*{7. Conclusion}
% ==============================================================================

Crestview Analytics has established a strong network perimeter defense, as evidenced by the clean external scan. However, the organization's resilience against cyber threats is undermined by significant gaps in its internal security controls and policies.

The identified risks—lack of MFA on critical systems, no acceptable use policy, and insufficient security training—are fundamental and present a clear and immediate danger to the organization's data and operations.

By implementing the recommendations outlined in this report, Crestview Analytics can effectively mitigate these high-severity risks, mature its security program, and build a more resilient and secure operational environment.

\end{document}
```