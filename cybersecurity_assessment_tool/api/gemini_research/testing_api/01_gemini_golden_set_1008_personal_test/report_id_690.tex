```latex
\documentclass[12pt]{article}

% Preamble: Required Packages
\usepackage[margin=1in]{geometry} % For setting page margins
\usepackage{pifont}               % For using dingbats (checkmarks, crosses)
\usepackage{booktabs}             % For professional-looking tables
\usepackage{hyperref}             % For hyperlinks and document metadata
\usepackage{url}                  % For formatting URLs
\usepackage{seqsplit}             % For splitting long strings without spaces

% Document Metadata
\hypersetup{
    colorlinks=true,
    linkcolor=blue,
    filecolor=magenta,      
    urlcolor=cyan,
    pdftitle={Cybersecurity Posture Assessment Report},
    pdfauthor={Cybersecurity Analyst},
    pdfsubject={Security Analysis},
    pdfkeywords={Cybersecurity, Risk, Assessment},
}

\begin{document}

% --- Title Section ---
\title{Cybersecurity Posture Assessment Report \\ \large For: \textbf{Skyward Bound}}
\author{Cybersecurity Analyst}
\date{\today}
\maketitle

\hrule
\vspace{1em}

% --- Executive Summary ---
\section*{Executive Summary}

This report provides a comprehensive cybersecurity assessment for \textbf{Skyward Bound}, synthesizing data from organizational questionnaires, technical network scans, and a review of pre-existing risks.

The analysis reveals a mixed security posture. The organization demonstrates a commitment to security awareness training and has implemented Multi-Factor Authentication (MFA) for computer logins. A significant positive finding from the technical scan is that a previously identified risk, an unencrypted web server on port 80, appears to have been remediated, as the port was found to be closed.

However, critical gaps exist that expose the organization to significant risk. The lack of mandatory MFA for email and sensitive data systems presents a primary vector for account compromise and data breaches. Furthermore, the absence of a formal Acceptable Use Policy (AUP) creates ambiguity regarding employee responsibilities and acceptable behavior, weakening the overall security culture.

Immediate remediation efforts should focus on implementing comprehensive MFA across all critical systems and developing a formal AUP.

% --- Organizational Information ---
\section{Organizational Information}

The following details were provided for the assessment.

\begin{tabular}{@{}ll}
\toprule
\textbf{Attribute} & \textbf{Value} \\
\midrule
Organization Name & \textbf{Skyward Bound} \\
Email Domain      & \texttt{SkywardBound.org} \\
Website Domain    & \url{www.SkywardBound.org} \\
External IP Address & \texttt{109.204.72.91} \\
\bottomrule
\end{tabular}

% --- Security Control Review ---
\section{Security Control Review}

This section evaluates the organization's security posture based on a questionnaire. "No" answers indicate significant gaps in security controls that require immediate attention.

\begin{tabular}{@{}p{0.6\linewidth}ccp{0.2\linewidth}@{}}
\toprule
\textbf{Control Question} & \textbf{Response} & \textbf{Status} & \textbf{Analyst's Note} \\
\midrule
Do you require MFA to access email? & No & \ding{55} & \textbf{Critical Gap.} Email is a primary target for phishing and account takeover. \\
\addlinespace
Do you require MFA to log into computers? & Yes & \ding{51} & Good Practice. This strengthens endpoint security. \\
\addlinespace
Do you require MFA to access sensitive data systems? & No & \ding{55} & \textbf{Critical Gap.} High-value data is not adequately protected from unauthorized access. \\
\addlinespace
Does your organization have an employee acceptable use policy? & No & \ding{55} & \textbf{High Risk.} Lack of a formal policy creates legal and security risks. \\
\addlinespace
Does your organization do security awareness training for new employees? & Yes & \ding{51} & Good Practice. Establishes a security baseline for new hires. \\
\addlinespace
Does your organization do security awareness training for all employees at least once per year? & Yes & \ding{51} & Good Practice. Reinforces a culture of security. \\
\bottomrule
\end{tabular}

% --- Technical Scan Results ---
\section{Technical Scan Results}

A network scan was performed to identify open ports and services on the specified target system.

\begin{itemize}
    \item \textbf{Target IP Address:} \texttt{192.168.0.5}
    \item \textbf{Scan Date:} \today
\end{itemize}

\subsection{Scan Findings}
The scan revealed a very limited attack surface on the target host. The key finding contradicts a previously identified risk.

\begin{tabular}{@{}llll@{}}
\toprule
\textbf{Port} & \textbf{State} & \textbf{Service} & \textbf{Notes} \\
\midrule
80/tcp & \textbf{closed} & http & The port for unencrypted web traffic is closed. \\
\bottomrule
\end{tabular}

\subsection{Analysis}
The pre-existing risk data (Input 3) indicated that port 80 was open, posing a risk of unencrypted communications. The current scan shows this port is now \textbf{closed}. This is a positive security development, suggesting that the risk has been successfully remediated. It is recommended to verify that this change was intentional and is a permanent configuration.

% --- Consolidated Risk Assessment ---
\section{Consolidated Risk Assessment}

The following table summarizes the identified risks, correlating findings from the security questionnaire, technical scans, and pre-existing risk data.

\begin{tabular}{@{}p{0.3\linewidth}p{0.15\linewidth}p{0.45\linewidth}@{}}
\toprule
\textbf{Risk Name} & \textbf{Severity} & \textbf{Overview} \\
\midrule
\textbf{Lack of MFA for Email} & \textbf{Critical} & Failure to protect email accounts with MFA makes them highly susceptible to phishing and account takeover, which can lead to business email compromise and further network intrusion. \\
\addlinespace
\textbf{Lack of MFA for Sensitive Data} & \textbf{Critical} & Sensitive organizational and client data is protected only by passwords, which can be easily compromised. This exposes the organization to a high risk of a data breach. \\
\addlinespace
\textbf{No Employee Acceptable Use Policy (AUP)} & \textbf{High} & Without a formal AUP, employees may be unaware of their security responsibilities. This increases the likelihood of insider threats, misuse of assets, and policy violations. \\
\addlinespace
\textbf{Unencrypted Web Server (Remediated)} & \textbf{Informational} & A previously identified risk of an open port 80 (HTTP) appears to be remediated, as the port was found to be closed during the scan. This should be confirmed as an intentional change. \\
\bottomrule
\end{tabular}

% --- Recommendations ---
\section{Recommendations}

Based on the analysis, the following actions are recommended to mitigate the identified risks and improve the overall security posture of \textbf{Skyward Bound}.

\begin{enumerate}
    \item \textbf{Implement Comprehensive MFA (Immediate Priority):}
    \begin{itemize}
        \item Enforce MFA for all user access to the email system (e.g., Microsoft 365, Google Workspace).
        \item Enforce MFA for all systems containing sensitive or critical data, including databases, file shares, and administrative portals.
    \end{itemize}

    \item \textbf{Develop and Implement an Acceptable Use Policy (High Priority):}
    \begin{itemize}
        \item Draft a formal AUP that clearly defines the rules for using company IT assets, data handling, and internet usage.
        \item Require all employees to read and acknowledge the policy upon hiring and annually thereafter.
    \end{itemize}

    \item \textbf{Verify Port 80 Closure (Verification Task):}
    \begin{itemize}
        \item Confirm with the IT or network team that the closure of port 80 on host \texttt{192.168.0.5} was an intentional act to remediate the previously known risk.
        \item Ensure firewall rules and server configurations are permanently set to keep this port closed unless there is a specific, approved business need for it (in which case HTTPS on port 443 should be used).
    \end{itemize}
\end{enumerate}

\end{document}
```