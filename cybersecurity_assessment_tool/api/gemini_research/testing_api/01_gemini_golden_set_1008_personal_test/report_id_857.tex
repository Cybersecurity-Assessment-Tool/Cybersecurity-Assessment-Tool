```latex
\documentclass[12pt, a4paper]{article}

% Preamble: Required Packages
\usepackage[margin=1in]{geometry}
\usepackage{pifont} % For checkmarks and crosses
\usepackage{booktabs} % For professional tables
\usepackage{hyperref} % For clickable links and references
\usepackage{url} % For URL formatting
\usepackage{seqsplit} % To split long monospaced text
\usepackage{graphicx}
\usepackage{xcolor}

% Hyperref Setup
\hypersetup{
    colorlinks=true,
    linkcolor=blue,
    filecolor=magenta,      
    urlcolor=cyan,
    pdftitle={Cybersecurity Assessment Report},
    pdfpagemode=FullScreen,
}

% Define checkmark and cross symbols for clarity
\newcommand{\cmark}{\ding{51}}%
\newcommand{\xmark}{\ding{55}}%

\begin{document}

% --- Title Page ---
\begin{titlepage}
    \centering
    \vspace*{1cm}
    \Huge\textbf{Cybersecurity Assessment Report}
    \vspace{1.5cm}
    \Large
    \textbf{Prepared for:}\\
    \vspace{0.5cm}
    \textbf{Opal Sky Media}
    \vspace{2cm}
    \large
    \textbf{Date of Report:}\\
    \vspace{0.5cm}
    \today
    \vfill
    \large
    \textit{This report contains sensitive information and should be handled with care. Distribution is restricted to authorized personnel only.}
\end{titlepage}

\tableofcontents
\newpage

% --- Section 1: Executive Overview ---
\section{Executive Overview}
This report provides a comprehensive analysis of the current cybersecurity posture of \textbf{Opal Sky Media}, based on a combination of technical network scanning, a review of existing risks, and an assessment of organizational security controls.

The assessment has identified several critical and high-severity risks that require immediate attention. The most pressing findings include:
\begin{itemize}
    \item \textbf{Systemic Remote Desktop Protocol (RDP) Exposure:} A network scan identified an open RDP port on a new system (\texttt{10.10.10.51}). This finding, correlated with a pre-existing risk on another machine, indicates a recurring and dangerous misconfiguration that exposes the internal network to significant threats like ransomware and brute-force attacks.
    \item \textbf{Critical Gaps in Multi-Factor Authentication (MFA):} The organization does not enforce MFA for accessing email or for logging into employee computers. This lack of a fundamental security control dramatically increases the risk of account compromise and unauthorized access.
    \item \textbf{Deficiencies in Security Governance:} The absence of a formal Acceptable Use Policy and a mandatory annual security awareness training program for all employees weakens the organization's human firewall, making it more susceptible to social engineering and policy violations.
\end{itemize}

The combination of these vulnerabilities creates a high-risk environment. An attacker could leverage the lack of MFA to compromise an email account, obtain credentials, and then use the exposed RDP service to gain direct access to the internal network. We strongly recommend prioritizing the remediation actions outlined in Section 6 to mitigate these risks and improve the overall security posture.

% --- Section 2: Organizational Information ---
\section{Organizational Information}
The following information was provided for the assessment.

\begin{tabular}{@{}ll}
    \toprule
    \textbf{Attribute} & \textbf{Value} \\
    \midrule
    Organization Name & \textbf{Opal Sky Media} \\
    Email Domain & \texttt{OpalSkyMedia.com} \\
    Website Domain & \seqsplit{\url{www.OpalSkyMedia.com}} \\
    External IP Address & \texttt{171.126.178.253} \\
    \bottomrule
\end{tabular}

% --- Section 3: Security Control Review ---
\section{Security Control Review}
A review of organizational security controls was conducted based on a standardized questionnaire. The responses reveal significant gaps in foundational security practices.

\begin{tabular}{@{}p{0.6\linewidth} c p{0.2\linewidth}@{}}
    \toprule
    \textbf{Control Question} & \textbf{Response} & \textbf{Assessment} \\
    \midrule
    Do you require MFA to access email? & \textcolor{red}{\xmark} & \textbf{Critical Gap} \\
    Do you require MFA to log into computers? & \textcolor{red}{\xmark} & \textbf{Critical Gap} \\
    Do you require MFA to access sensitive data systems? & \textcolor{green}{\cmark} & Good Practice \\
    Does your organization have an employee acceptable use policy? & \textcolor{red}{\xmark} & \textbf{High Risk} \\
    Does your organization do security awareness training for new employees? & \textcolor{green}{\cmark} & Good Practice \\
    Does your organization do security awareness training for all employees at least once per year? & \textcolor{red}{\xmark} & \textbf{High Risk} \\
    \bottomrule
\end{tabular}

% --- Section 4: Technical Scan Results ---
\section{Technical Scan Results}
A network scan was performed to identify open ports and services on the target system.

\begin{itemize}
    \item \textbf{Target IP Address:} \texttt{10.10.10.51}
    \item \textbf{Scan Status:} Host is up and responsive.
\end{itemize}

The following open port was discovered:
\begin{table}[h!]
\centering
\begin{tabular}{@{}llll@{}}
    \toprule
    \textbf{Port} & \textbf{State} & \textbf{Service Name} & \textbf{Analysis} \\
    \midrule
    3389/tcp & Open & \texttt{ms-wbt-server} & Remote Desktop Protocol (RDP) \\
    \bottomrule
\end{tabular}
\caption{Open Ports on \texttt{10.10.10.51}}
\end{table}

\paragraph{Analysis:} The discovery of an open RDP port is a critical finding. RDP is a primary target for attackers seeking to gain unauthorized access to internal networks. It is frequently exploited through brute-force password attacks, credential stuffing, and vulnerabilities such as BlueKeep. This finding, when combined with the pre-existing risk on host \texttt{10.10.10.50}, points to a systemic issue in network configuration and security management.

% --- Section 5: Consolidated Risk Assessment ---
\section{Consolidated Risk Assessment}
This section synthesizes findings from the security control review, technical scan, and pre-existing risk data into a consolidated list of key risks.

\begin{tabular}{@{}p{0.25\linewidth} p{0.55\linewidth} l@{}}
    \toprule
    \textbf{Risk Name} & \textbf{Description} & \textbf{Severity} \\
    \midrule
    \textbf{Systemic RDP Exposure} & RDP is exposed on multiple internal systems (\texttt{10.10.10.50}, \texttt{10.10.10.51}), creating a direct path for attackers into the network. This is a common vector for ransomware. & \textbf{Critical} \\
    \addlinespace
    \textbf{Lack of Foundational MFA} & The absence of MFA on email and computer logins makes the organization highly vulnerable to credential theft and account takeover attacks. & \textbf{Critical} \\
    \addlinespace
    \textbf{Inadequate Security Policies \& Training} & The lack of an Acceptable Use Policy and annual security training for all staff increases the likelihood of human error, policy violations, and successful social engineering attacks. & \textbf{High} \\
    \bottomrule
\end{tabular}

% --- Section 6: Recommendations ---
\section{Recommendations}
Based on the analysis, the following actions are recommended to mitigate the identified risks. Recommendations are prioritized by severity.

\subsection*{Immediate Actions (Critical Priority)}
\begin{enumerate}
    \item \textbf{Remediate RDP Exposure:} Immediately close port 3389 on \texttt{10.10.10.51} and any other systems where it is not strictly required for internal operations. For all required remote access, implement a Virtual Private Network (VPN) solution that requires Multi-Factor Authentication.
    \item \textbf{Implement Comprehensive MFA:} Deploy MFA across all employee email accounts and endpoint logins (laptops/desktops) without delay. This is the single most effective control to prevent account compromise.
\end{enumerate}

\subsection*{High Priority Actions}
\begin{enumerate}
    \setcounter{enumi}{2} % Continue numbering
    \item \textbf{Develop and Enforce an Acceptable Use Policy (AUP):} Create a formal AUP that clearly defines the rules for using company technology, data, and network resources. Ensure all employees read and acknowledge the policy.
    \item \textbf{Establish Annual Security Awareness Training:} Implement a mandatory security awareness training program for all employees to be completed annually. This program should cover topics such as phishing, password security, and social engineering.
\end{enumerate}

\subsection*{Further Recommendations}
\begin{enumerate}
    \setcounter{enumi}{4} % Continue numbering
    \item \textbf{Conduct a Comprehensive Vulnerability Scan:} Perform a full internal and external vulnerability assessment to identify and remediate other potential security weaknesses across the entire network infrastructure.
\end{enumerate}

\end{document}
```