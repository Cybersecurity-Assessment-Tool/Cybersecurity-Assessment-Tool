```latex
\documentclass[12pt,a4paper]{article}

% Preamble: Required Packages
\usepackage[margin=1in]{geometry}
\usepackage{pifont} % For checkmarks and crosses
\usepackage{booktabs} % For professional tables
\usepackage{hyperref} % For clickable links
\usepackage{url} % For URL formatting
\usepackage{seqsplit} % For splitting long text strings in tt font
\usepackage{graphicx}
\usepackage{xcolor}

% Document Metadata
\hypersetup{
    colorlinks=true,
    linkcolor=blue,
    filecolor=magenta,      
    urlcolor=cyan,
    pdftitle={Cybersecurity Assessment Report},
    pdfauthor={Cybersecurity Analyst},
    pdfsubject={Security Assessment},
    pdfkeywords={Security, Nmap, Risk, Analysis},
}

% Custom Commands
\newcommand{\yes}{\ding{51}} % Green checkmark
\newcommand{\no}{\ding{55}}  % Red X

\begin{document}

% --- Title Page ---
\begin{titlepage}
    \centering
    \vspace*{1cm}
    \Huge \textbf{Cybersecurity Assessment Report}
    \vspace{1.5cm}
    \Large \textbf{Prepared for:} \\
    \vspace{0.5cm}
    \textbf{Blue Horizon Initiative}
    \vspace{2cm}
    \rule{\linewidth}{0.5mm}
    \vspace{0.5cm}
    \large \textbf{CONFIDENTIAL}
    \vspace{0.5cm}
    \rule{\linewidth}{0.5mm}
    \vfill
    \large \textbf{Date of Report:} \today \\
    \textbf{Author:} Cybersecurity Analyst
\end{titlepage}

\tableofcontents
\newpage

% --- Executive Summary ---
\section*{Executive Summary}
This report details the findings of a cybersecurity assessment conducted for Blue Horizon Initiative. The analysis synthesizes results from a network vulnerability scan, a review of organizational security controls, and pre-existing risk data.

The assessment reveals a \textbf{Critical} overall risk posture. Key findings include an externally accessible FTP server running a dangerously outdated and vulnerable version of \texttt{vsftpd} (2.3.4), which is known to contain a backdoor. This vulnerability is exacerbated by the allowance of anonymous logins, creating a direct and easily exploitable entry point for attackers.

Furthermore, a systemic lack of fundamental security controls was identified. The organization does not enforce Multi-Factor Authentication (MFA) for any system, including email, computer logins, or sensitive data access. This significantly increases the risk of account compromise. The absence of a security awareness training program leaves the organization highly susceptible to phishing and other social engineering attacks. These issues are compounded by the use of an end-of-life operating system (Windows 7) on workstations, which no longer receives security updates.

Immediate and decisive action is required to remediate these critical vulnerabilities and establish a baseline of security for the organization.

% --- Organizational Information ---
\section*{1. Organizational Information}
The following details were provided for the assessment.

\begin{itemize}
    \item \textbf{Organization Name:} Blue Horizon Initiative
    \item \textbf{Primary Email Domain:} \texttt{BlueHorizonInitiative.com}
    \item \textbf{Primary Website Domain:} \texttt{www.BlueHorizonInitiative.com}
    \item \textbf{External IP Address:} \texttt{220.39.128.23}
\end{itemize}

% --- Security Control Review ---
\section*{2. Security Control Review}
A review of organizational security policies and controls was conducted via a questionnaire. The results highlight significant gaps in foundational security practices.

\begin{table}[h!]
\centering
\caption{Security Controls Questionnaire Results}
\begin{tabular}{p{0.8\linewidth}c}
\toprule
\textbf{Control Question} & \textbf{Status} \\
\midrule
Does your organization have an employee acceptable use policy? & \yes \\
Do you require MFA to access email? & \no \\
Do you require MFA to log into computers? & \no \\
Do you require MFA to access sensitive data systems? & \no \\
Does your organization do security awareness training for new employees? & \no \\
Does your organization do security awareness training for all employees at least once per year? & \no \\
\bottomrule
\end{tabular}
\end{table}

The lack of MFA for email, computer, and data access, combined with the absence of security awareness training, represents a critical failure in security posture.

% --- Technical Scan Results ---
\section*{3. Technical Scan Results}
An external network scan was performed to identify open ports and services.

\begin{itemize}
    \item \textbf{Target IP Address:} \texttt{10.0.0.15}
\end{itemize}

\begin{table}[h!]
\centering
\caption{Open Port Analysis}
\begin{tabular}{llllll}
\toprule
\textbf{Port} & \textbf{State} & \textbf{Service} & \textbf{Product} & \textbf{Version} & \textbf{Notes} \\
\midrule
21/tcp & open & ftp & vsftpd & 2.3.4 & Anonymous login allowed \\
\bottomrule
\end{tabular}
\end{table}

\subsection*{Analysis of Findings}
The scan identified a single open port, 21/tcp, running \textbf{vsftpd version 2.3.4}. This specific version is extremely dangerous and contains a critical backdoor vulnerability (\textbf{CVE-2011-2523}). When a username containing the sequence `:)` is sent, the server triggers a backdoor that opens a command shell on port 6200. This allows a remote attacker to execute arbitrary commands on the server with root privileges.

The finding is further compounded by the fact that \textbf{Anonymous FTP login is allowed}, making the server trivial for an attacker to interact with and exploit. This configuration represents an immediate and severe threat to the integrity and confidentiality of the target system and the entire network.

% --- Consolidated Risk Assessment ---
\section*{4. Consolidated Risk Assessment}
The following table consolidates all identified risks from the technical scan, security control review, and pre-existing data. Risks are prioritized based on their potential impact and likelihood of exploitation.

\begin{table}[h!]
\centering
\caption{Risk Summary}
\begin{tabular}{p{0.15\linewidth} p{0.65\linewidth} p{0.15\linewidth}}
\toprule
\textbf{Risk Title} & \textbf{Description} & \textbf{Severity} \\
\midrule
\textbf{Vulnerable FTP Server (RCE)} & The FTP server runs \texttt{vsftpd 2.3.4}, which has a known public backdoor allowing for remote code execution (CVE-2011-2523). Anonymous login is enabled. & \textbf{Critical} \\
\addlinespace
\textbf{No Multi-Factor Authentication} & MFA is not enforced for email, computer logins, or sensitive data systems. A single compromised password could lead to a full breach. & \textbf{Critical} \\
\addlinespace
\textbf{End-of-Life OS} & Workstations are running Windows 7, which is no longer supported by Microsoft and does not receive security patches, leaving them vulnerable to known exploits. & \textbf{High} \\
\addlinespace
\textbf{No Security Awareness Training} & Employees are not trained to identify or report phishing attempts or other security threats, making them a primary target for attackers. & \textbf{High} \\
\bottomrule
\end{tabular}
\end{table}

% --- Recommendations ---
\section*{5. Remediation Plan and Recommendations}
The following actions are recommended to address the identified risks. They are prioritized to mitigate the most severe threats first.

\begin{enumerate}
    \item \textbf{Immediate Action: Remediate FTP Server}
    \begin{itemize}
        \item \textbf{Option A (Recommended):} Decommission the FTP server at \texttt{10.0.0.15} immediately. If file transfer is a business requirement, migrate to a secure protocol such as SFTP (SSH File Transfer Protocol).
        \item \textbf{Option B (Temporary):} If decommissioning is not possible, immediately upgrade \texttt{vsftpd} to the latest stable version and disable anonymous user access. The server should be firewalled from the internet until patched.
    \end{itemize}

    \item \textbf{High Priority: Implement Multi-Factor Authentication (MFA)}
    \begin{itemize}
        \item Procure and deploy an MFA solution across the organization.
        \item Enforce MFA for all users on all critical applications, starting with:
        \begin{itemize}
            \item Email (e.g., Office 365, G Suite)
            \item VPN and remote access solutions
            \item All systems containing sensitive or regulated data
            \item Workstation and server logins
        \end{itemize}
    \end{itemize}

    \item \textbf{High Priority: Upgrade End-of-Life Systems}
    \begin{itemize}
        \item Initiate a project to upgrade or replace all workstations and servers running Windows 7.
        \item Deploy a modern, supported operating system (e.g., Windows 11) to ensure security patches are received.
    \end{itemize}

    \item \textbf{Medium Priority: Establish Security Awareness Training}
    \begin{itemize}
        \item Implement a mandatory security awareness training program for all employees.
        \item Training should be part of the onboarding process for new hires and conducted annually for all staff.
        \item Topics should include phishing awareness, password hygiene, and acceptable use policies.
    \end{itemize}
\end{enumerate}

\end{document}
```