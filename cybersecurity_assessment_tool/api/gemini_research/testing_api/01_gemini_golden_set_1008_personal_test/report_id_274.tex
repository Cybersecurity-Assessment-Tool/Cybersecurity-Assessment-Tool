```latex
\documentclass[12pt]{article}

% Preamble: Required Packages
\usepackage[margin=1in]{geometry}
\usepackage{pifont} % For checkmarks and crosses
\usepackage{booktabs} % For professional tables
\usepackage{hyperref} % For clickable links
\usepackage{url}      % For URL formatting
\usepackage{seqsplit} % For splitting long text sequences
\usepackage{graphicx}
\usepackage{xcolor}

% Document Metadata
\title{Cybersecurity Posture Assessment Report}
\author{Cybersecurity Analysis Division}
\date{\today}

% Hyperref Setup
\hypersetup{
    colorlinks=true,
    linkcolor=blue,
    filecolor=magenta,      
    urlcolor=cyan,
    pdftitle={Cybersecurity Posture Assessment Report},
    pdfpagemode=FullScreen,
}

\begin{document}

\maketitle
\thispagestyle{empty}
\newpage

\tableofcontents
\thispagestyle{empty}
\newpage

\setcounter{page}{1}

% --- 1. Executive Summary ---
\section*{1. Executive Summary}

This report details the findings of a cybersecurity posture assessment for \textbf{Iron Bridge Legal}. The assessment incorporated an analysis of organizational security controls, a review of known risks, and an external network scan.

The primary findings indicate significant gaps in administrative and access controls, which present a critical risk to the organization. Specifically, the lack of Multi-Factor Authentication (MFA) on email and sensitive data systems exposes the organization to account compromise and data breaches. Furthermore, the complete absence of a security awareness training program and an acceptable use policy leaves the organization highly vulnerable to social engineering and insider threats.

From a technical perspective, the external network scan of the target system did not identify any open ports, suggesting a well-configured firewall or network perimeter. While this is a positive finding, it does not mitigate the severe internal policy and control deficiencies.

Immediate remediation efforts should focus on implementing MFA across all critical systems, establishing a comprehensive security awareness program, and developing foundational security policies.

% --- 2. Organizational Information ---
\section*{2. Organizational Information}

The following information was provided for the assessment.

\begin{tabular}{@{}ll}
\toprule
\textbf{Attribute} & \textbf{Value} \\
\midrule
Organization Name & \textbf{Iron Bridge Legal} \\
Email Domain & \texttt{IronBridgeLegal.net} \\
Website Domain & \url{www.IronBridgeLegal.net} \\
External IP Address & \texttt{92.96.242.159} \\
\bottomrule
\end{tabular}

% --- 3. Security Control Review ---
\section*{3. Security Control Review}

A review of the organization's security controls was conducted via a standardized questionnaire. The results highlight critical areas requiring immediate attention. A checkmark (\ding{51}) indicates a positive control is in place, while a cross (\ding{55}) indicates a control gap.

\begin{table}[h!]
\centering
\begin{tabular}{@{}lc}
\toprule
\textbf{Security Control Question} & \textbf{Status} \\
\midrule
Do you require MFA to log into computers? & \textcolor{green}{\ding{51}} \\
\addlinespace[0.5em]
Do you require MFA to access email? & \textcolor{red}{\ding{55}} \\
\addlinespace[0.5em]
Do you require MFA to access sensitive data systems? & \textcolor{red}{\ding{55}} \\
\addlinespace[0.5em]
Does your organization have an employee acceptable use policy? & \textcolor{red}{\ding{55}} \\
\addlinespace[0.5em]
Does your organization do security awareness training for new employees? & \textcolor{red}{\ding{55}} \\
\addlinespace[0.5em]
Does your organization do security awareness training for all employees at least once per year? & \textcolor{red}{\ding{55}} \\
\bottomrule
\end{tabular}
\caption{Organizational Security Control Status}
\end{table}

% --- 4. Technical Scan Results ---
\section*{4. Technical Scan Results}

An external network vulnerability scan was performed to identify potential exposures on the organization's perimeter.

\begin{itemize}
    \item \textbf{Scan Target:} \texttt{[Target IP]}
    \item \textbf{Scan Date:} \today
\end{itemize}

\subsection*{Findings}
The scan completed successfully, and \textbf{no open TCP or UDP ports were detected} on the target system. This indicates a strong firewall configuration that denies unsolicited inbound traffic, which is a commendable security practice for external-facing assets. This significantly reduces the external attack surface. However, this result does not provide insight into vulnerabilities that may exist on the internal network or on the systems behind the firewall.

% --- 5. Risk Assessment ---
\section*{5. Risk Assessment}

This section synthesizes the findings from the security control review, technical scan, and pre-existing risk data. The following risks have been identified and prioritized based on their potential impact on the organization.

\begin{table}[h!]
\centering
\begin{tabular}{@{}p{0.25\linewidth} p{0.5\linewidth} p{0.15\linewidth}@{}}
\toprule
\textbf{Risk Name} & \textbf{Overview} & \textbf{Severity} \\
\midrule
\textbf{Lack of MFA on Critical Systems} & Email and sensitive data systems are not protected by Multi-Factor Authentication. This allows an attacker with stolen credentials to gain direct access, leading to potential data breaches, financial fraud, and reputational damage. & \textbf{Critical} \\
\addlinespace[1em]
\textbf{No Security Awareness Program} & The absence of security awareness training for new or existing employees makes the organization highly susceptible to phishing, social engineering, and other human-targeted attacks. Employees are likely unaware of current threats and best practices. & \textbf{High} \\
\addlinespace[1em]
\textbf{Absence of Acceptable Use Policy (AUP)} & Without a formal AUP, there are no clear guidelines for employees on the acceptable use of company assets. This increases the risk of insider threats, misuse of resources, and legal liabilities. & \textbf{High} \\
\bottomrule
\end{tabular}
\caption{Identified Cybersecurity Risks}
\end{table}

% --- 6. Recommendations ---
\section*{6. Recommendations}

The following actions are recommended to mitigate the identified risks and improve the overall cybersecurity posture of \textbf{Iron Bridge Legal}.

\subsection*{Immediate Actions (0-30 Days)}
\begin{enumerate}
    \item \textbf{Implement MFA on Email and Sensitive Data Systems (Critical):}
    \begin{itemize}
        \item \textbf{Action:} Enforce MFA for all user accounts accessing the email system (\texttt{IronBridgeLegal.net}) and any systems identified as storing or processing sensitive client data.
        \item \textbf{Justification:} This is the single most effective control to prevent unauthorized access resulting from compromised credentials.
    \end{itemize}
    
    \item \textbf{Develop and Distribute an Acceptable Use Policy (High):}
    \begin{itemize}
        \item \textbf{Action:} Draft a clear and concise AUP that outlines the rules for using company networks, computers, and data. Require all employees to read and formally acknowledge the policy.
        \item \textbf{Justification:} An AUP establishes a baseline for secure employee behavior and provides a framework for enforcing security standards.
    \end{itemize}
\end{enumerate}

\subsection*{Short-Term Actions (30-90 Days)}
\begin{enumerate}
    \setcounter{enumi}{2} % Continue numbering from previous list
    \item \textbf{Establish a Security Awareness Training Program (High):}
    \begin{itemize}
        \item \textbf{Action:} Procure and implement a security awareness training solution. Enroll all current employees in foundational training, with a focus on phishing and password security. Integrate this training into the onboarding process for all new hires.
        \item \textbf{Justification:} A well-trained workforce is a critical layer of defense against common cyberattacks.
    \end{itemize}
\end{enumerate}

\end{document}
```