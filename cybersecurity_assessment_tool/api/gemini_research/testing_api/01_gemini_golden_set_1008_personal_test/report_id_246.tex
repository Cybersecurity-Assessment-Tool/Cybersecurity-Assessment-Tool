```latex
\documentclass[12pt]{article}

% --- PACKAGES ---
\usepackage[margin=1in]{geometry}
\usepackage{pifont} % For checkmarks and crosses
\usepackage{booktabs} % For professional tables
\usepackage{hyperref} % For hyperlinks
\usepackage{url} % For URL formatting
\usepackage{seqsplit} % To split long monospaced strings

% --- DOCUMENT METADATA ---
\title{Cybersecurity Posture Assessment Report}
\author{Cybersecurity Analysis Division}
\date{\today}

% --- HYPERREF SETUP ---
\hypersetup{
    colorlinks=true,
    linkcolor=blue,
    filecolor=magenta,      
    urlcolor=cyan,
    pdftitle={Cybersecurity Posture Assessment Report},
    pdfpagemode=FullScreen,
}

\begin{document}

\maketitle
\hrule
\vspace{1em}

% ==============================================================================
% SECTION 1: EXECUTIVE OVERVIEW
% ==============================================================================
\section*{Executive Overview}

This report provides a comprehensive analysis of the cybersecurity posture for \textbf{New Era}. The assessment is based on a synthesis of a network vulnerability scan, a review of organizational security controls via a questionnaire, and an evaluation of pre-existing risk documentation.

The assessment reveals a mixed security posture. On a positive note, a previously identified risk concerning an unencrypted web server on port 80 appears to have been remediated; our technical scan confirms that this port is now closed on the target system \texttt{192.168.0.5}.

However, significant and high-risk gaps were identified in the organization's administrative and access controls. The most critical findings are the lack of Multi-Factor Authentication (MFA) for computer logins and the complete absence of a security awareness training program for both new and existing employees. These deficiencies expose the organization to a heightened risk of credential compromise, unauthorized access, and successful social engineering attacks such as phishing.

Immediate action is recommended to address these control gaps to significantly improve the organization's resilience against common cyber threats.

% ==============================================================================
% SECTION 2: ORGANIZATIONAL INFORMATION
% ==============================================================================
\section{Organizational Information}

The following details were provided for the assessment scope.

\begin{itemize}
    \item \textbf{Organization Name:} New Era
    \item \textbf{Email Domain:} \texttt{NewEra.com}
    \item \textbf{Website Domain:} \seqsplit{\url{www.NewEra.com}}
    \item \textbf{External IP Address:} \texttt{225.72.5.243}
\end{itemize}

% ==============================================================================
% SECTION 3: SECURITY CONTROL REVIEW
% ==============================================================================
\section{Security Control Review}

A review of administrative and technical security controls was conducted based on the provided questionnaire. The results are summarized below. "No" answers indicate significant gaps in the security framework.

\begin{table}[h!]
\centering
\caption{Security Control Questionnaire Analysis}
\begin{tabular}{p{0.6\linewidth} c c}
\toprule
\textbf{Control Question} & \textbf{Response} & \textbf{Status} \\
\midrule
Do you require MFA to access email? & Yes & \ding{51} \\
Do you require MFA to log into computers? & No & \textbf{\color{red}\ding{55}} \\
Do you require MFA to access sensitive data systems? & Yes & \ding{51} \\
Does your organization have an employee acceptable use policy? & Yes & \ding{51} \\
Does your organization do security awareness training for new employees? & No & \textbf{\color{red}\ding{55}} \\
Does your organization do security awareness training for all employees at least once per year? & No & \textbf{\color{red}\ding{55}} \\
\bottomrule
\end{tabular}
\end{table}

% ==============================================================================
% SECTION 4: TECHNICAL SCAN RESULTS
% ==============================================================================
\section{Technical Scan Results}

A network scan was performed on the specified target to identify open ports and exposed services.

\begin{itemize}
    \item \textbf{Target IP Address:} \texttt{192.168.0.5}
    \item \textbf{Scan Date:} As per scan metadata
\end{itemize}

\begin{table}[h!]
\centering
\caption{Port Scan Findings for \texttt{192.168.0.5}}
\begin{tabular}{ccccc}
\toprule
\textbf{Port} & \textbf{State} & \textbf{Service} & \textbf{Product} & \textbf{Version} \\
\midrule
80/tcp & closed & http & N/A & N/A \\
\bottomrule
\end{tabular}
\end{table}

\paragraph{Analysis:} The scan indicates that port 80 (HTTP) is closed on the target system. This finding is significant as it contradicts a previously documented risk (\textit{Unencrypted Web Server}). This suggests the prior vulnerability has been successfully remediated. No other open ports were discovered during this scan.

% ==============================================================================
% SECTION 5: CONSOLIDATED RISK ASSESSMENT
% ==============================================================================
\section{Consolidated Risk Assessment}

The following table synthesizes findings from the security control review, technical scan, and pre-existing risk data into a prioritized list.

\begin{table}[h!]
\centering
\caption{Summary of Identified Risks}
\begin{tabular}{p{0.25\linewidth} p{0.1\linewidth} p{0.4\linewidth} p{0.2\linewidth}}
\toprule
\textbf{Risk Name} & \textbf{Severity} & \textbf{Description} & \textbf{Affected Elements} \\
\midrule
\textbf{Lack of MFA on Endpoints} & \textbf{High} & The absence of MFA for computer logins significantly increases the risk of unauthorized access resulting from stolen or weak credentials. & All user workstations and laptops. \\
\addlinespace
\textbf{No Security Awareness Training Program} & \textbf{High} & Employees are not trained to identify or respond to phishing, social engineering, or other common cyber attacks, making them a primary target for threat actors. & All employees. \\
\addlinespace
\textbf{Remediated: Unencrypted Web Server} & Informational & A previously documented risk stated Port 80 was open. The current scan confirms this port is closed, indicating the risk has been addressed. & \texttt{192.168.0.5} \\
\bottomrule
\end{tabular}
\end{table}

% ==============================================================================
% SECTION 6: RECOMMENDATIONS
% ==============================================================================
\section{Recommendations}

Based on the consolidated risk assessment, the following actions are recommended to improve the security posture of \textbf{New Era}.

\begin{enumerate}
    \item \textbf{Implement MFA for Endpoint Logins (Critical):}
    \begin{itemize}
        \item \textbf{Action:} Deploy a mandatory Multi-Factor Authentication solution for all employee computer and laptop logins. This is a critical defense against credential theft and unauthorized access.
        \item \textbf{Justification:} If an employee's password is compromised, MFA provides a second layer of security that prevents an attacker from gaining access to the endpoint and, subsequently, the internal network.
        \item \textbf{Examples:} Solutions like Windows Hello for Business, Duo Security, or other hardware/software token systems should be evaluated.
    \end{itemize}
    \vspace{1em}
    
    \item \textbf{Establish a Security Awareness Training Program (Critical):}
    \begin{itemize}
        \item \textbf{Action:} Develop and implement a formal security awareness training program. This program must be mandatory for all new hires during their onboarding process and required for all staff annually.
        \item \textbf{Justification:} The human element is often the weakest link in security. A well-trained workforce is the first line of defense against phishing, malware, and social engineering attacks.
        \item \textbf{Examples:} The program should cover topics like phishing identification, password hygiene, acceptable use of company assets, and incident reporting procedures.
    \end{itemize}
    \vspace{1em}
    
    \item \textbf{Update Risk Register (Administrative):}
    \begin{itemize}
        \item \textbf{Action:} Review the internal risk register and formally close the item related to the "Unencrypted Web Server" on port 80 for host \texttt{192.168.0.5}.
        \item \textbf{Justification:} The technical scan confirms this vulnerability has been remediated. Maintaining an accurate risk register is crucial for focusing resources on current and relevant threats.
    \end{itemize}
\end{enumerate}

\end{document}
```