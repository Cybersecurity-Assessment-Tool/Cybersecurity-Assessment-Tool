```latex
\documentclass[12pt]{article}

% -------------------
% --- PREAMBLE ---
% -------------------
\usepackage[margin=1in]{geometry}
\usepackage{pifont} % For checkmarks and crosses
\usepackage{booktabs} % For professional-looking tables
\usepackage{hyperref} % For clickable links
\usepackage{url} % For URL formatting
\usepackage{seqsplit} % To split long strings without breaking
\usepackage{xcolor} % For colors
\usepackage{graphicx}
\usepackage[T1]{fontenc}

% --- Document Metadata ---
\title{Cybersecurity Posture and Risk Assessment Report}
\author{Cybersecurity Analysis Division}
\date{\today}

% --- Hyperref Setup ---
\hypersetup{
    colorlinks=true,
    linkcolor=blue,
    filecolor=magenta,      
    urlcolor=cyan,
    pdftitle={Cybersecurity Posture and Risk Assessment Report},
    pdfpagemode=FullScreen,
}

% -------------------
% --- DOCUMENT START ---
% -------------------
\begin{document}

\maketitle
\hrule
\vspace{1cm}

% -------------------
% Section 1: Executive Summary
% -------------------
\section*{Executive Summary}
This report provides a comprehensive analysis of the cybersecurity posture for \textbf{Moxie Marketing}. The assessment is based on a correlation of data from a technical network scan, a security controls questionnaire, and a review of previously identified risks.

The primary findings indicate significant procedural and policy-based risks. Critical security gaps were identified, including the absence of Multi-Factor Authentication (MFA) for email access and computer logins. Furthermore, a lack of a formal security awareness training program for both new and existing employees presents a high risk of human error leading to security incidents.

On a technical level, the network scan of the target system \texttt{192.168.0.5} did not reveal any open or vulnerable ports. This is a positive finding for the scanned asset. However, this result contradicts a pre-existing risk concerning an unencrypted web server on Port 80. This suggests the pre-existing risk may be outdated or related to a different asset within the organization's network.

Immediate action is required to address the MFA and security training deficiencies to mitigate the most severe risks to the organization.

% -------------------
% Section 2: Organizational Information
% -------------------
\section{Organizational Information}
The following details were provided for the assessment.

\begin{itemize}
    \item \textbf{Organization Name:} Moxie Marketing
    \item \textbf{Email Domain:} \texttt{MoxieMarketing.com}
    \item \textbf{External IP Address:} \seqsplit{\texttt{99.232.192.214}}
\end{itemize}

% -------------------
% Section 3: Security Control Review
% -------------------
\section{Security Control Review}
The following table summarizes the organization's responses to the security controls questionnaire. Items marked with \ding{55} indicate significant gaps in the security framework and are discussed in the Risk Assessment section.

\begin{table}[h!]
\centering
\begin{tabular}{@{}lcc@{}}
\toprule
\textbf{Control Question} & \textbf{Response} & \textbf{Assessment} \\
\midrule
Do you require MFA to access email? & No & \textcolor{red}{\ding{55}} \\
Do you require MFA to log into computers? & No & \textcolor{red}{\ding{55}} \\
Do you require MFA to access sensitive data systems? & Yes & \textcolor{green}{\ding{51}} \\
Does your organization have an employee acceptable use policy? & Yes & \textcolor{green}{\ding{51}} \\
Does your organization do security awareness training for new employees? & No & \textcolor{red}{\ding{55}} \\
Does your organization do security awareness training for all employees at least once per year? & No & \textcolor{red}{\ding{55}} \\
\bottomrule
\end{tabular}
\caption{Security Controls Questionnaire Results}
\end{table}

% -------------------
% Section 4: Technical Scan Results
% -------------------
\section{Technical Scan Results}
A network scan was conducted to identify open ports and services on the specified target system.

\begin{itemize}
    \item \textbf{Target IP Address:} \texttt{192.168.0.5}
    \item \textbf{Scan Date:} \today
\end{itemize}

\paragraph{Summary:} The scan of the target system revealed no open ports. The host was responsive, but its network services were not exposed, indicating a secure perimeter for this specific asset.

\begin{table}[h!]
\centering
\begin{tabular}{@{}lcccc@{}}
\toprule
\textbf{Port} & \textbf{State} & \textbf{Service} & \textbf{Product} & \textbf{Version} \\
\midrule
80/tcp & closed & http & N/A & N/A \\
\bottomrule
\end{tabular}
\caption{Nmap Scan Results for \texttt{192.168.0.5}}
\end{table}

% -------------------
% Section 5: Correlated Risk Assessment
% -------------------
\section{Correlated Risk Assessment}
The following table synthesizes findings from the security questionnaire, technical scans, and pre-existing risk data.

\begin{table}[h!]
\centering
\begin{tabular}{@{}p{0.3\linewidth} p{0.5\linewidth} p{0.15\linewidth}@{}}
\toprule
\textbf{Risk Name} & \textbf{Description} & \textbf{Severity} \\
\midrule
\textbf{Lack of MFA on Email} & The absence of MFA on email accounts significantly increases the risk of account compromise via phishing or credential stuffing, potentially leading to data breaches and further internal network compromise. & \textbf{Critical} \\
\addlinespace
\textbf{Lack of MFA on Workstations} & User workstations and laptops do not require MFA for login. A compromised password could grant an attacker full access to a user's machine and any connected network resources. & \textbf{Critical} \\
\addlinespace
\textbf{Inadequate Security Training} & The organization does not provide security awareness training to new or existing employees. This creates a high susceptibility to social engineering and phishing attacks. & \textbf{High} \\
\addlinespace
\textbf{Unencrypted Web Server*} & A pre-existing risk indicates that Port 80 is open for unencrypted web traffic. This could expose sensitive data transmitted to and from the server. & Medium \\
\bottomrule
\end{tabular}
\caption{Synthesized Risk Register}
\end{table}
\textit{*Note: The current network scan of \texttt{192.168.0.5} did not validate the "Unencrypted Web Server" finding, as Port 80 was found to be closed. This risk may be outdated, apply to a different asset, or have been previously remediated.}

% -------------------
% Section 6: Recommendations
% -------------------
\section{Recommendations}
Based on the correlated risk assessment, the following actions are recommended to improve the organization's cybersecurity posture.

\subsection*{Immediate Priority (Critical Risks)}
\begin{itemize}
    \item \textbf{Implement MFA for Email:} Enforce mandatory Multi-Factor Authentication (MFA) across all email accounts immediately. This is the single most effective control to prevent unauthorized account access.
    \item \textbf{Implement MFA for Endpoints:} Deploy and enforce MFA for all workstation and laptop logins to protect against compromised credentials.
\end{itemize}

\subsection*{High Priority}
\begin{itemize}
    \item \textbf{Establish a Security Awareness Program:} Develop and implement a formal security awareness training program. This program must include:
    \begin{itemize}
        \item Mandatory training for all new employees during onboarding.
        \item Annual refresher training for all staff, covering topics like phishing, password security, and acceptable use.
    \end{itemize}
\end{itemize}

\subsection*{Verification and Review}
\begin{itemize}
    \item \textbf{Investigate Unencrypted Web Server Risk:} Conduct a thorough investigation to determine the source of the "Unencrypted Web Server" risk. Scan all public-facing assets to confirm if Port 80 is open elsewhere. If found, remediate by either closing the port or enforcing TLS/SSL encryption (HTTPS).
\end{itemize}

% -------------------
% --- DOCUMENT END ---
% -------------------
\end{document}
```