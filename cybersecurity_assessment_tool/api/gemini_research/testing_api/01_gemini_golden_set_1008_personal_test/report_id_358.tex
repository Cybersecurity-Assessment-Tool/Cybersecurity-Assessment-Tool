```latex
\documentclass[12pt, a4paper]{article}

% Preamble: Required Packages
\usepackage[margin=1in]{geometry}
\usepackage{pifont} % For checkmarks and crosses
\usepackage{booktabs} % For professional tables
\usepackage{hyperref} % For hyperlinks
\usepackage{url} % For URL formatting
\usepackage{seqsplit} % For splitting long strings
\usepackage{graphicx} % For logo (optional, but good practice)
\usepackage{xcolor} % For colors in text

% Document Information
\title{Cybersecurity Posture Assessment Report}
\author{Cybersecurity Analysis Division}
\date{\today}

% Hyperref Setup
\hypersetup{
    colorlinks=true,
    linkcolor=blue,
    filecolor=magenta,      
    urlcolor=cyan,
    pdftitle={Cybersecurity Posture Assessment Report},
    pdfpagemode=FullScreen,
}

\begin{document}

\maketitle
\hrule
\vspace{1cm}

\begin{center}
    \textbf{Client: Nebula Creative} \\
    \textbf{Report ID: CSR-2023-481}
\end{center}

\vspace{1cm}

% --- Section 1: Executive Overview ---
\section*{1.0 Executive Overview}

This report details the findings of a cybersecurity posture assessment for \textbf{Nebula Creative}. The assessment combined a review of organizational security controls via a questionnaire, an external network vulnerability scan, and an analysis of pre-existing risks.

The primary finding is a significant disparity between the organization's technical and procedural security controls. While the external network scan of the target IP address revealed no open ports—a positive indicator of a well-configured firewall—the organization has critical deficiencies in its access control policies and employee security awareness.

The absence of Multi-Factor Authentication (MFA) on critical systems such as email and sensitive data repositories, coupled with a complete lack of an acceptable use policy and security awareness training, exposes \textbf{Nebula Creative} to a high risk of compromise through credential theft, phishing, and social engineering attacks. Immediate remediation of these policy and procedural gaps is strongly recommended to mitigate the risk of a significant security incident.

% --- Section 2: Organizational Information ---
\section*{2.0 Organizational Information}

The following information was provided for the assessment.

\begin{tabular}{@{}ll}
    \toprule
    \textbf{Attribute} & \textbf{Value} \\
    \midrule
    Organization Name & Nebula Creative \\
    Email Domain & \texttt{NebulaCreative.org} \\
    Website Domain & \url{www.NebulaCreative.org} \\
    External IP Address & \texttt{131.206.221.48} \\
    \bottomrule
\end{tabular}

% --- Section 3: Security Control Review ---
\section*{3.0 Security Control Review}

A security questionnaire was completed to evaluate the organization's current procedural and administrative controls. The results indicate several critical gaps in fundamental security practices.

\begin{tabular}{@{}p{0.7\linewidth}c}
    \toprule
    \textbf{Control Question} & \textbf{Status} \\
    \midrule
    Do you require MFA to access email? & \ding{55} \\
    \textit{\textcolor{red}{Critical Gap: Lack of MFA on email exposes the organization to account takeover.}} \\
    \addlinespace
    Do you require MFA to log into computers? & \ding{51} \\
    \textit{\textcolor{green}{Good Practice: Endpoint access is properly secured.}} \\
    \addlinespace
    Do you require MFA to access sensitive data systems? & \ding{55} \\
    \textit{\textcolor{red}{Critical Gap: Sensitive data is at high risk of unauthorized access.}} \\
    \addlinespace
    Does your organization have an employee acceptable use policy? & \ding{55} \\
    \textit{\textcolor{orange}{High Risk: Lack of policy creates ambiguity and increases insider risk.}} \\
    \addlinespace
    Does your organization do security awareness training for new employees? & \ding{55} \\
    \textit{\textcolor{orange}{High Risk: New hires are not equipped to identify security threats.}} \\
    \addlinespace
    Does your organization do security awareness training for all employees at least once per year? & \ding{55} \\
    \textit{\textcolor{orange}{High Risk: Employees are highly vulnerable to phishing and social engineering.}} \\
    \bottomrule
\end{tabular}

% --- Section 4: Technical Scan Results ---
\section*{4.0 Technical Scan Results}

An external network scan was performed to identify open ports and exposed services on the organization's perimeter.

\begin{itemize}
    \item \textbf{Target IP Address:} \texttt{[Target IP]}
    \item \textbf{Scan Date:} 2023-10-27 (Date of analysis)
    \item \textbf{Findings:} The scan completed successfully and found \textbf{no open TCP or UDP ports}. This is a strong security posture from a network perspective and suggests that a properly configured firewall is in place, restricting all unsolicited inbound traffic.
\end{itemize}

% --- Section 5: Risk Assessment Summary ---
\section*{5.0 Risk Assessment Summary}

This section synthesizes findings from the security control review, technical scan, and pre-existing risk data. As no pre-existing vulnerabilities were reported, the risks below are derived solely from the control gaps identified during this assessment.

\begin{tabular}{@{}p{0.25\linewidth}p{0.5\linewidth}l}
    \toprule
    \textbf{Risk Name} & \textbf{Overview} & \textbf{Severity} \\
    \midrule
    Email Account Compromise & The absence of MFA on email accounts makes them highly susceptible to takeover via credential stuffing or phishing attacks. Compromised email can lead to data breaches and further internal compromise. & \textbf{Critical} \\
    \addlinespace
    Sensitive Data Exposure & Lack of MFA on sensitive data systems means that a single compromised password could grant an attacker direct access to the organization's most valuable information assets. & \textbf{Critical} \\
    \addlinespace
    High Susceptibility to Social Engineering & Without any security awareness training, employees are unlikely to recognize and report phishing emails, malicious links, or other social engineering tactics, making them the weakest link in the security chain. & \textbf{High} \\
    \addlinespace
    Lack of Formal IT Governance & The absence of a foundational document like an Acceptable Use Policy (AUP) leads to inconsistent security practices and a higher likelihood of unintentional insider threats or misuse of company assets. & \textbf{High} \\
    \bottomrule
\end{tabular}

% --- Section 6: Recommendations ---
\section*{6.0 Recommendations}

Based on the risks identified, the following actions are recommended to improve the cybersecurity posture of \textbf{Nebula Creative}. These are prioritized based on severity and impact.

\begin{enumerate}
    \item \textbf{Implement MFA for Email (Priority: Immediate):} Enforce MFA for all user accounts on the \texttt{NebulaCreative.org} email domain. This is the single most effective control to prevent email account takeovers.
    
    \item \textbf{Implement MFA for Sensitive Systems (Priority: Immediate):} Identify all systems containing sensitive or critical data and enforce MFA for all user access. This creates a critical defense-in-depth layer to protect core assets.
    
    \item \textbf{Establish a Security Awareness Training Program (Priority: High):} Implement a mandatory security awareness training program for all employees. Training should be conducted for new hires and repeated annually for all staff. Focus on identifying phishing, password hygiene, and reporting security incidents.
    
    \item \textbf{Develop and Implement an Acceptable Use Policy (Priority: High):} Create a formal AUP that clearly defines the rules for using company IT assets, including data handling, internet usage, and password requirements. All employees must read and acknowledge this policy.
\end{enumerate}

\end{document}
```