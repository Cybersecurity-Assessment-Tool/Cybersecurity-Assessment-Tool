Here is the complete and professional LaTeX report, generated based on the provided data inputs and your instructions.

```latex
\documentclass[12pt]{article}

% Required Packages
\usepackage[margin=1in]{geometry} % for setting page margins
\usepackage{pifont}               % for checkmarks and crosses (dingbats)
\usepackage{booktabs}             % for professional-looking tables
\usepackage{hyperref}             % for clickable links
\usepackage{url}                  % for URL formatting
\usepackage{seqsplit}             % for splitting long strings in texttt
\usepackage{graphicx}             % for images, not used here but good practice
\usepackage{xcolor}               % for colors

% Document Metadata
\title{Cybersecurity Posture Assessment Report}
\author{Cybersecurity Analysis Division}
\date{\today}

\hypersetup{
    colorlinks=true,
    linkcolor=blue,
    filecolor=magenta,      
    urlcolor=cyan,
    pdftitle={Cybersecurity Posture Assessment Report},
    pdfpagemode=FullScreen,
}

\begin{document}

\maketitle
\thispagestyle{empty} % No page number on the title page

\newpage
\tableofcontents
\thispagestyle{empty} % No page number on the ToC page

\newpage
\setcounter{page}{1} % Start page numbering at 1

% --- 1. Executive Overview ---
\section{Executive Overview}

This report provides a cybersecurity posture assessment for \textbf{Sovereign Trust}, based on a review of organizational security controls, technical scan data, and pre-existing risk information. The analysis reveals several critical and high-risk security gaps that significantly increase the organization's exposure to common cyber threats, such as phishing, credential theft, and unauthorized access.

The most critical findings stem from the organizational security questionnaire, which indicates a complete absence of Multi-Factor Authentication (MFA) for email, computer logins, and access to sensitive data systems. This lack of a fundamental security control leaves the organization highly vulnerable to account compromise. Additional high-risk gaps include the lack of a formal employee acceptable use policy and the absence of annual security awareness training for all staff.

It is important to note that the technical network scan data (\texttt{Input\_1\_Network\_Scan\_JSON}) and the list of current risks (\texttt{Input\_3\_Current\_Risks\_JSON}) were found to be corrupted and could not be processed. This data integrity issue prevented a full analysis of externally exposed services and pre-existing vulnerabilities, representing a significant blind spot in the organization's security visibility.

Recommendations have been prioritized to address the most severe risks first. Immediate implementation of MFA is paramount, followed by the development of foundational security policies and training programs.

% --- 2. Organizational Information ---
\section{Organizational Information}

The following details were provided by the client for this assessment.

\begin{itemize}
    \item \textbf{Organization Name:} Sovereign Trust
    \item \textbf{Email Domain:} \texttt{SovereignTrust.com}
    \item \textbf{Website Domain:} \url{www.SovereignTrust.com}
    \item \textbf{External IP Address:} \texttt{148.124.11.255}
\end{itemize}

% --- 3. Security Control Review ---
\section{Security Control Review (Questionnaire Analysis)}

The following table summarizes the organization's responses to the security controls questionnaire. A red cross (\ding{55}) indicates a negative response, highlighting a potential security gap that requires attention.

\begin{table}[h!]
\centering
\caption{Security Controls Questionnaire Results}
\begin{tabular}{p{0.7\textwidth}cc}
\toprule
\textbf{Control Question} & \textbf{Response} & \textbf{Status} \\
\midrule
Do you require MFA to access email? & No & \textcolor{red}{\ding{55}} \\
Do you require MFA to log into computers? & No & \textcolor{red}{\ding{55}} \\
Do you require MFA to access sensitive data systems? & No & \textcolor{red}{\ding{55}} \\
Does your organization have an employee acceptable use policy? & No & \textcolor{red}{\ding{55}} \\
Does your organization do security awareness training for new employees? & Yes & \textcolor{green}{\ding{51}} \\
Does your organization do security awareness training for all employees at least once per year? & No & \textcolor{red}{\ding{55}} \\
\bottomrule
\end{tabular}
\end{table}

The analysis of these responses indicates critical deficiencies in access control and governance. The lack of MFA across all key systems is the most severe finding.

% --- 4. Technical Scan Results ---
\section{Technical Scan Results}

The provided network scan data (\texttt{Input\_1\_Network\_Scan\_JSON}) was found to be corrupted or incomplete. As a result, a technical analysis of open ports, running services, and potential software vulnerabilities could not be performed.

This lack of visibility into the organization's external attack surface is a significant finding in itself. Without reliable and periodic scanning, the organization cannot proactively identify and remediate vulnerabilities in its internet-facing systems. It is strongly recommended to perform a new, validated external network vulnerability scan as soon as possible.

% --- 5. Risk Assessment ---
\section{Risk Assessment}

This risk assessment is based on the findings from the Security Control Review, as technical scan and pre-existing risk data were unavailable. The identified risks are foundational and expose the organization to a high likelihood of a security incident with a significant impact.

\begin{table}[h!]
\centering
\caption{Identified Risks and Severity}
\begin{tabular}{lp{3.5cm}p{6cm}l}
\toprule
\textbf{Risk ID} & \textbf{Risk Name} & \textbf{Description} & \textbf{Severity} \\
\midrule
RISK-001 & No Multi-Factor Authentication (MFA) & The absence of MFA for email, endpoints, and sensitive systems allows an attacker with stolen credentials to gain unauthorized access easily. & \textbf{Critical} \\
\addlinespace
RISK-002 & Lack of Acceptable Use Policy (AUP) & Without a formal AUP, employees are unaware of their security responsibilities, leading to inconsistent practices and increased insider risk. & \textbf{High} \\
\addlinespace
RISK-003 & Insufficient Security Awareness Training & Failing to provide annual training for all staff allows security knowledge to degrade, making employees more susceptible to phishing and social engineering. & \textbf{High} \\
\addlinespace
RISK-004 & Incomplete Security Visibility & Corrupted or missing scan data prevents the identification of technical vulnerabilities on external systems, creating a critical blind spot in the security program. & \textbf{High} \\
\bottomrule
\end{tabular}
\end{table}

% --- 6. Recommendations ---
\section{Recommendations}

The following prioritized recommendations are provided to mitigate the identified risks and improve the overall security posture of \textbf{Sovereign Trust}.

\subsection{Immediate Priority (Critical Risks)}
\begin{enumerate}
    \item \textbf{Implement Multi-Factor Authentication (MFA):}
    \begin{itemize}
        \item Immediately enable MFA for all user accounts on the primary email system (e.g., Microsoft 365, Google Workspace).
        \item Phase in MFA for all remote access solutions (VPN) and computer logins.
        \item Mandate MFA for all systems containing sensitive or regulated data.
    \end{itemize}
\end{enumerate}

\subsection{High Priority}
\begin{enumerate}
    \setcounter{enumi}{1} % Continue numbering
    \item \textbf{Develop and Implement an Acceptable Use Policy (AUP):}
    \begin{itemize}
        \item Create a formal AUP that defines the rules for using company assets, data handling, and internet usage.
        \item Require all employees to read and acknowledge the policy as a condition of employment.
    \end{itemize}
    \item \textbf{Establish a Recurring Security Awareness Program:}
    \begin{itemize}
        \item Implement a mandatory annual security awareness training program for all employees.
        \item Supplement annual training with periodic phishing simulations to measure and improve employee resilience.
    \end{itemize}
    \item \textbf{Ensure Data Integrity and Perform Technical Scans:}
    \begin{itemize}
        \item Investigate and resolve the issue causing the corruption of security data feeds.
        \item Schedule and conduct a comprehensive external network vulnerability scan immediately. Establish a recurring schedule for scans (e.g., quarterly).
    \end{itemize}
\end{enumerate}

\end{document}
```