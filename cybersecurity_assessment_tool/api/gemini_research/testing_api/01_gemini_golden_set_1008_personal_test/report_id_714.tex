```latex
\documentclass[12pt]{article}

% Preamble: Required Packages
\usepackage[margin=1in]{geometry}
\usepackage{pifont} % For \ding symbols (checkmarks/crosses)
\usepackage{booktabs} % For professional-looking tables
\usepackage{hyperref} % For clickable links and references
\usepackage{url} % For formatting URLs
\usepackage{seqsplit} % For splitting long strings in \texttt
\usepackage{xcolor} % For custom colors

% Document Information & Styling
\hypersetup{
    colorlinks=true,
    linkcolor=blue,
    filecolor=magenta,
    urlcolor=cyan,
}
\urlstyle{same} % Use the same font for URLs as the surrounding text

% Define custom colors for severity
\definecolor{criticalred}{HTML}{D7263D}
\definecolor{highorange}{HTML}{F49D40}
\definecolor{mediumyellow}{HTML}{F4D440}
\definecolor{lowblue}{HTML}{5C9EAD}
\definecolor{infogray}{HTML}{808080}

% Document Title
\title{Cybersecurity Posture Assessment Report \\ \large For: Nexus Dynamics}
\author{Cybersecurity Analysis Division}
\date{\today}

\begin{document}

\maketitle

\begin{abstract}
This report provides a comprehensive cybersecurity assessment for Nexus Dynamics. The analysis is based on a correlation of network scan data, organizational security control questionnaires, and a review of pre-existing risk registers. The assessment identifies critical deficiencies in administrative controls, particularly the lack of Multi-Factor Authentication (MFA) and a formal security awareness training program. While the technical network scan of the specified target did not reveal immediate vulnerabilities, the identified policy and procedure gaps represent a significant risk to the organization's security posture. Actionable recommendations are provided to mitigate these risks and enhance overall resilience.
\end{abstract}

\newpage

\tableofcontents

\newpage

\section{Overview and Executive Summary}

This assessment synthesizes three key data sources to provide a holistic view of the current security posture of Nexus Dynamics.

\paragraph{Key Findings:}
\begin{itemize}
    \item \textbf{Critical Control Gaps:} The organization has not implemented Multi-Factor Authentication (MFA) for email, computer logins, or access to sensitive data systems. This absence of a fundamental security control exposes the organization to a high risk of account compromise and unauthorized access.
    \item \textbf{Lack of Security Training:} There is no security awareness training program for new or existing employees. This significantly increases the organization's susceptibility to social engineering attacks, such as phishing.
    \item \textbf{Positive Technical Finding:} A network scan of the target host \texttt{192.168.0.5} found no open ports. This indicates that a previously documented risk, "Unencrypted Web Server," related to an open port 80 may have been remediated.
\end{itemize}

\paragraph{Overall Posture:} The organization's primary risks are currently rooted in administrative and procedural weaknesses rather than immediate technical vulnerabilities on the scanned asset. Addressing the lack of MFA and employee training should be the highest priority to prevent common and impactful cyber attacks.

\section{Organizational Information}

The following details were provided for the assessment.

\begin{tabular}{@{}ll}
\toprule
\textbf{Attribute} & \textbf{Value} \\
\midrule
Organization Name: & \textbf{Nexus Dynamics} \\
Email Domain: & \texttt{NexusDynamics.net} \\
Website Domain: & \url{www.NexusDynamics.net} \\
External IP Address: & \texttt{134.184.136.244} \\
\bottomrule
\end{tabular}

\section{Security Control Review}

The following table summarizes the organization's responses to a security controls questionnaire. Items marked with \ding{55} indicate a deviation from security best practices and represent a gap in the defense-in-depth strategy.

\begin{table}[h!]
\centering
\begin{tabular}{p{0.6\textwidth} c l}
\toprule
\textbf{Control Question} & \textbf{Response} & \textbf{Assessment} \\
\midrule
Do you require MFA to access email? & \ding{55} & \textcolor{criticalred}{\textbf{Critical Gap}} \\
Do you require MFA to log into computers? & \ding{55} & \textcolor{highorange}{\textbf{High Risk}} \\
Do you require MFA to access sensitive data systems? & \ding{55} & \textcolor{criticalred}{\textbf{Critical Gap}} \\
Does your organization have an employee acceptable use policy? & \ding{51} & Met \\
Does your organization do security awareness training for new employees? & \ding{55} & \textcolor{highorange}{\textbf{High Risk}} \\
Does your organization do security awareness training for all employees at least once per year? & \ding{55} & \textcolor{highorange}{\textbf{High Risk}} \\
\bottomrule
\end{tabular}
\caption{Security Controls Questionnaire Analysis}
\end{table}

\section{Technical Scan Results}

A network vulnerability scan was conducted to identify technical exposures on the specified target system.

\begin{itemize}
    \item \textbf{Target IP Address:} \texttt{192.168.0.5}
    \item \textbf{Scan Date:} Not specified in scan data.
    \item \textbf{Scanner Used:} Nmap
\end{itemize}

\subsection{Scan Summary}
The target host was responsive (status: up). The scan focused on common service ports, including TCP/80 (HTTP).

\paragraph{Findings:} The scan reported that port \textbf{80 was closed}. No open ports or active services were discovered on the target during this assessment. This is a positive security finding, as it limits the attack surface of the host.

\paragraph{Correlation Note:} This result contradicts a pre-existing risk entry titled "Unencrypted Web Server," which was predicated on Port 80 being open. The current scan data suggests this risk has been remediated or the service has been decommissioned on this host. This discrepancy should be investigated to ensure the organizational risk register is accurate.

\section{Consolidated Risk Assessment}

The following table consolidates risks identified from the security control review and pre-existing risk data, factoring in the results of the recent technical scan.

\begin{table}[h!]
\centering
\begin{tabular}{p{0.25\textwidth} l p{0.45\textwidth}}
\toprule
\textbf{Risk Name} & \textbf{Severity} & \textbf{Overview} \\
\midrule
\textbf{Absence of MFA} & \textcolor{criticalred}{\textbf{Critical}} & The lack of MFA for email, endpoints, and sensitive systems exposes the organization to credential theft, account takeover, and subsequent data breaches. \\
\addlinespace
\textbf{Inadequate Security Awareness Program} & \textcolor{highorange}{\textbf{High}} & Without regular training, employees are more likely to fall victim to phishing, malware, and other social engineering attacks, making them a vulnerable entry point for attackers. \\
\addlinespace
\textbf{Unencrypted Web Server} & \textcolor{mediumyellow}{\textbf{Medium (5.0)}} & \textit{(Potentially Remediated)} A previously identified risk of an unencrypted web service on Port 80. The recent scan indicates this port is now closed, suggesting remediation. Verification is required. \\
\bottomrule
\end{tabular}
\caption{Summary of Identified Risks}
\end{table}

\section{Recommendations}

To address the identified risks and improve the overall security posture, the following actions are recommended in order of priority.

\begin{enumerate}
    \item \textbf{Implement Multi-Factor Authentication (Critical Priority):}
    \begin{itemize}
        \item Immediately begin a phased rollout of a robust MFA solution across the organization.
        \item \textbf{Phase 1:} Protect all access to sensitive data systems and administrative accounts.
        \item \textbf{Phase 2:} Enforce MFA for remote access solutions (VPN) and all email accounts (e.g., Office 365, Google Workspace).
        \item \textbf{Phase 3:} Extend MFA to all employee computer logins.
    \end{itemize}

    \item \textbf{Establish a Security Awareness Training Program (High Priority):}
    \begin{itemize}
        \item Develop or procure a security awareness training module and make it mandatory for all new employees during their onboarding process.
        \item Implement a mandatory annual refresher training for all staff to ensure their knowledge remains current.
        \item Conduct periodic phishing simulations to test and reinforce employee awareness.
    \end{itemize}

    \item \textbf{Validate Risk Register and System Configuration (Medium Priority):}
    \begin{itemize}
        \item Internally verify the status of the service previously running on port 80 of host \texttt{192.168.0.5}.
        \item If the service was intentionally decommissioned or secured, formally update the organization's risk register to reflect that this risk has been remediated.
        \item If the service was moved, ensure a new risk assessment is performed on its new location.
    \end{itemize}
\end{enumerate}

\section{Conclusion}

Nexus Dynamics has demonstrated a positive step in securing its network perimeter, as evidenced by the technical scan results. However, the organization's resilience against modern cyber threats is severely undermined by critical gaps in its administrative security controls. The immediate implementation of Multi-Factor Authentication and the establishment of a comprehensive security awareness program are essential to building a robust, defense-in-depth security posture.

\end{document}
```