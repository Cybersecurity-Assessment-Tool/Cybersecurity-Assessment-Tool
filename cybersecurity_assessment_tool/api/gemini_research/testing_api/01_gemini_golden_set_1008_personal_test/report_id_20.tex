```latex
\documentclass[12pt]{article}

% Preamble: Required Packages
\usepackage[margin=1in]{geometry} % For setting page margins
\usepackage{pifont}               % For checkmarks and crosses (\ding)
\usepackage{booktabs}             % For professional-looking tables
\usepackage{hyperref}             % For hyperlinks, enhances PDF output
\usepackage{url}                  % For formatting URLs
\usepackage{seqsplit}             % To split long text strings without breaking
\usepackage{graphicx}             % For including logos, if needed
\usepackage{xcolor}               % For custom colors

% Document Information
\title{Cybersecurity Posture Assessment Report \\ \large For: \textbf{Phoenix Rising}}
\author{Cybersecurity Analysis Division}
\date{\today}

% Hyperref Setup for PDF metadata
\hypersetup{
    colorlinks=true,
    linkcolor=blue,
    filecolor=magenta,      
    urlcolor=cyan,
    pdftitle={Cybersecurity Posture Assessment Report for Phoenix Rising},
    pdfpagemode=FullScreen,
}

\begin{document}

\maketitle
\thispagestyle{empty}
\newpage

\tableofcontents
\newpage

% --- 1. Executive Summary ---
\section{Executive Summary}

This report provides a cybersecurity posture assessment for \textbf{Phoenix Rising}, based on a combination of network scanning, a security controls questionnaire, and a review of pre-existing risk data. The analysis was conducted on \today.

The assessment reveals that while \textbf{Phoenix Rising} has implemented several important foundational security controls, such as employee security awareness training and an acceptable use policy, there are critical gaps that expose the organization to significant risk.

Key findings include:
\begin{itemize}
    \item \textbf{Critical - Lack of Multi-Factor Authentication (MFA):} MFA is not enforced for accessing email or sensitive data systems. This represents a high-risk vulnerability, as compromised credentials could lead directly to a major data breach.
    \item \textbf{High - Unencrypted Web Traffic:} The external network scan identified an open port 80 (HTTP). This service transmits data in cleartext, making it susceptible to eavesdropping and man-in-the-middle attacks. Any credentials or sensitive information sent over this connection are not secure.
\end{itemize}

Immediate remediation of these findings is strongly recommended to reduce the organization's attack surface and protect critical assets. Detailed recommendations are provided in Section \ref{sec:recommendations}.

% --- 2. Organizational Information ---
\section{Organizational Information}

The following details were provided for the assessment. This information is used to establish the context and scope of the review.

\begin{table}[h!]
\centering
\begin{tabular}{@{}ll@{}}
\toprule
\textbf{Attribute} & \textbf{Value} \\
\midrule
Organization Name    & \textbf{Phoenix Rising} \\
Email Domain         & \texttt{PhoenixRising.org} \\
Website Domain       & \url{www.PhoenixRising.org} \\
External IP Address  & \texttt{39.35.186.55} \\
\bottomrule
\end{tabular}
\caption{Client Organizational Details}
\label{tab:org_info}
\end{table}

% --- 3. Security Control Review ---
\section{Security Control Review}

A review of the organization's security controls was conducted via a questionnaire. The results highlight both strengths and critical weaknesses in the current security posture. A "No" answer indicates a potential gap that requires attention.

\begin{table}[h!]
\centering
\begin{tabular}{@{}p{0.75\textwidth}c@{}}
\toprule
\textbf{Control Question} & \textbf{Status} \\
\midrule
Does your organization have an employee acceptable use policy? & \textcolor{green}{\ding{51}} \\
Does your organization do security awareness training for new employees? & \textcolor{green}{\ding{51}} \\
Does your organization do security awareness training for all employees at least once per year? & \textcolor{green}{\ding{51}} \\
Do you require MFA to log into computers? & \textcolor{green}{\ding{51}} \\
\midrule
\textbf{Do you require MFA to access email?} & \textbf{\textcolor{red}{\ding{55}}} \\
\textbf{Do you require MFA to access sensitive data systems?} & \textbf{\textcolor{red}{\ding{55}}} \\
\bottomrule
\end{tabular}
\caption{Security Controls Questionnaire Results}
\label{tab:controls}
\end{table}

\subsection*{Analysis of Control Gaps}
The lack of MFA on email and sensitive data systems (\textbf{\textcolor{red}{\ding{55}}}) are the most critical findings from this review. Email is a primary target for attackers for phishing and account takeover. Similarly, sensitive data systems without MFA are highly vulnerable to unauthorized access via stolen or weak credentials.

% --- 4. Technical Scan Results ---
\section{Technical Scan Results}

An external network scan was performed to identify exposed services and potential vulnerabilities visible from the public internet.

\begin{itemize}
    \item \textbf{Scan Target:} \texttt{172.16.0.1}
    \item \textbf{Scan Date:} \today
\end{itemize}

\subsection*{Open Ports Discovered}
The following table details the open ports discovered on the target system.

\begin{table}[h!]
\centering
\begin{tabular}{@{}llll@{}}
\toprule
\textbf{Port} & \textbf{Protocol} & \textbf{State} & \textbf{Service (Inferred)} \\
\midrule
80 & TCP & open & HTTP (Hypertext Transfer Protocol) \\
\bottomrule
\end{tabular}
\caption{Open Ports on Target \texttt{172.16.0.1}}
\label{tab:ports}
\end{table}

\subsection*{Technical Analysis}
The scan identified that port 80 is open, which is used for the HTTP protocol. HTTP is an unencrypted protocol. Any data, including usernames, passwords, and session cookies, transmitted over an HTTP connection can be intercepted and read by an attacker on the same network. This is a significant security risk, especially if the service handles any form of login or sensitive data exchange. Modern security standards require the use of HTTPS (HTTP over TLS/SSL on port 443) to encrypt web communications.

\textit{Note: The provided scan data did not include service version information. A more detailed scan is recommended to identify specific software versions and associated vulnerabilities.}

% --- 5. Synthesized Risk Assessment ---
\section{Synthesized Risk Assessment}

This section correlates the findings from the security control review and the technical scan to provide a consolidated view of the primary risks facing the organization. The pre-existing risk data provided in the input was determined to be invalid and has been excluded from this analysis.

\begin{table}[h!]
\centering
\begin{tabular}{@{}p{0.2\linewidth}p{0.6\linewidth}l@{}}
\toprule
\textbf{Risk ID} & \textbf{Risk Description} & \textbf{Severity} \\
\midrule
RISK-001 & \textbf{Lack of MFA on Critical Systems} \newline Email and sensitive data systems are vulnerable to account takeover due to the absence of MFA. A single compromised password could lead to a significant data breach. & \textbf{High} \\
\addlinespace
RISK-002 & \textbf{Unencrypted Web Traffic} \newline The active HTTP service on port 80 transmits data in cleartext, exposing user credentials and sensitive information to interception (e.g., Man-in-the-Middle attacks). & \textbf{High} \\
\bottomrule
\end{tabular}
\caption{Summary of Identified Risks}
\label{tab:risks}
\end{table}

% --- 6. Recommendations ---
\section{Recommendations}
\label{sec:recommendations}

Based on the synthesized risk assessment, the following actions are recommended to mitigate the identified vulnerabilities and improve the overall security posture of \textbf{Phoenix Rising}.

\subsection*{High Priority - Immediate Actions}

\begin{enumerate}
    \item \textbf{Enforce MFA on All Critical Systems (RISK-001):}
    \begin{itemize}
        \item Immediately enable and enforce MFA for all user accounts with access to email (e.g., Office 365, Google Workspace).
        \item Implement MFA for all systems identified as containing sensitive data. This includes databases, financial applications, and administrative portals.
    \end{itemize}

    \item \textbf{Implement TLS/SSL for Web Services (RISK-002):}
    \begin{itemize}
        \item Obtain and install a valid TLS/SSL certificate on the web server at \texttt{172.16.0.1}.
        \item Disable the HTTP service on port 80 or configure the server to automatically redirect all HTTP traffic to HTTPS (port 443).
        \item Ensure that HTTP Strict Transport Security (HSTS) is enabled to protect against protocol downgrade attacks.
    \end{itemize}
\end{enumerate}

\subsection*{Medium Priority - Further Actions}

\begin{enumerate}
    \item \textbf{Conduct Comprehensive Vulnerability Scanning:}
    \begin{itemize}
        \item Perform a comprehensive, credentialed vulnerability scan of the external IP address (\texttt{39.35.186.55}) and internal network segments. This scan should include service version detection to identify outdated software and specific CVEs (Common Vulnerabilities and Exposures).
    \end{itemize}
    
    \item \textbf{Review and Enhance Access Control Policies:}
    \begin{itemize}
        \item Conduct a review of user access rights to sensitive data systems, ensuring that the principle of least privilege is applied.
    \end{itemize}
\end{enumerate}

\end{document}
```