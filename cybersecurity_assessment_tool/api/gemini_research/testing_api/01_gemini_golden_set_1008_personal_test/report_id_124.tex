```latex
\documentclass[12pt, a4paper]{article}

% Preamble: Required Packages
\usepackage[margin=1in]{geometry}
\usepackage{pifont} % For checkmarks and crosses
\usepackage{booktabs} % For professional-looking tables
\usepackage{hyperref} % For clickable links
\usepackage{url} % For formatting URLs
\usepackage{seqsplit} % For splitting long strings without breaking
\usepackage{graphicx}
\usepackage{xcolor}
\usepackage{fancyhdr}

% --- Document Setup ---
\hypersetup{
    colorlinks=true,
    linkcolor=blue,
    filecolor=magenta,      
    urlcolor=cyan,
    pdftitle={Cybersecurity Posture Assessment Report},
    pdfpagemode=FullScreen,
}

% Define colors for risk levels
\definecolor{criticalred}{HTML}{D10000}
\definecolor{highorange}{HTML}{E25F00}
\definecolor{mediumyellow}{HTML}{F0C200}
\definecolor{lowblue}{HTML}{0073E6}
\definecolor{infogray}{HTML}{808080}

% Header and Footer
\pagestyle{fancy}
\fancyhf{}
\lhead{Cybersecurity Posture Assessment}
\rhead{Cloud Nine Software}
\cfoot{\thepage}

% --- Document Start ---
\begin{document}

% --- Title Page ---
\begin{titlepage}
    \centering
    \vspace*{2cm}
    
    \Huge{\textbf{Cybersecurity Posture Assessment Report}}
    
    \vspace{1.5cm}
    
    \Large{\textbf{Prepared for:}}
    
    \vspace{0.5cm}
    
    \Large{Cloud Nine Software}
    
    \vspace{2cm}
    
    \includegraphics[width=0.4\textwidth]{example-image-a} % Placeholder for company logo
    
    \vfill
    
    \large{\textbf{Date of Report:}}
    
    \vspace{0.2cm}
    
    \large{\today}
    
\end{titlepage}

\tableofcontents
\newpage

% --- 1. Executive Summary ---
\section*{1. Executive Summary}

This report details the findings of a cybersecurity posture assessment for \textbf{Cloud Nine Software}. The assessment incorporated a technical network scan, a review of existing risks, and an analysis of organizational security controls via a questionnaire.

The technical scan of the target host \texttt{192.168.1.100} revealed a strong security configuration, with no open ports detected. This indicates a well-hardened external perimeter for the assessed system.

However, the organizational security control review identified several \textbf{critical and high-risk gaps}. The complete absence of Multi-Factor Authentication (MFA) for email, computer logins, and sensitive data access represents a significant vulnerability to account compromise and unauthorized access. Furthermore, the lack of a formal Acceptable Use Policy (AUP) and security training for new employees creates an environment where security best practices are not consistently established or enforced.

While the technical posture of the scanned asset is commendable, the identified policy and access control deficiencies present a substantial risk to the organization. Immediate remediation of these gaps is strongly recommended to improve the overall security posture and mitigate the risk of a security incident.

% --- 2. Organizational Information ---
\section*{2. Organizational Information}

The following information was provided for the assessment.

\begin{tabular}{@{}ll}
\toprule
\textbf{Attribute} & \textbf{Value} \\
\midrule
Organization Name & \textbf{Cloud Nine Software} \\
Email Domain & \texttt{CloudNineSoftware.com} \\
Website Domain & \seqsplit{\texttt{www.CloudNineSoftware.com}} \\
External IP & \texttt{59.184.230.86} \\
\bottomrule
\end{tabular}

% --- 3. Security Control Review ---
\section*{3. Security Control Review}

The following table summarizes the organization's responses to the security controls questionnaire. A checkmark (\ding{51}) indicates a positive control is in place, while a cross (\ding{55}) indicates a control gap that introduces risk.

\begin{tabular}{@{}p{0.6\linewidth} c p{0.25\linewidth}@{}}
\toprule
\textbf{Control Question} & \textbf{Response} & \textbf{Assessment} \\
\midrule
Does your organization do security awareness training for all employees at least once per year? & \textcolor{green}{\ding{51}} & \textbf{Control in Place.} Annual training is a good baseline for maintaining security awareness. \\
\addlinespace
Do you require MFA to access email? & \textcolor{red}{\ding{55}} & \textbf{Critical Gap.} Email is a primary target for phishing and account takeover. \\
\addlinespace
Do you require MFA to log into computers? & \textcolor{red}{\ding{55}} & \textbf{Critical Gap.} Lack of MFA on endpoints allows for easier lateral movement after a credential compromise. \\
\addlinespace
Do you require MFA to access sensitive data systems? & \textcolor{red}{\ding{55}} & \textbf{Critical Gap.} Sensitive data is left highly vulnerable to unauthorized access. \\
\addlinespace
Does your organization have an employee acceptable use policy? & \textcolor{red}{\ding{55}} & \textbf{High Risk.} Without an AUP, employees lack clear guidelines on safe and acceptable use of company assets. \\
\addlinespace
Does your organization do security awareness training for new employees? & \textcolor{red}{\ding{55}} & \textbf{High Risk.} New employees are not equipped with essential security knowledge from day one, increasing vulnerability. \\
\bottomrule
\end{tabular}

% --- 4. Technical Scan Results ---
\section*{4. Technical Scan Results}

A network scan was performed to identify open ports and exposed services on the target system.

\begin{itemize}
    \item \textbf{Target IP Address:} \texttt{192.168.1.100}
    \item \textbf{Host Status:} UP
    \item \textbf{Findings:} The scan confirmed that the host is online, but found \textbf{no open TCP ports}. All 1000 scanned ports were reported as 'closed'. This is a positive security finding, as it indicates a minimal attack surface on the scanned host.
\end{itemize}

% --- 5. Risk Assessment Summary ---
\section*{5. Risk Assessment Summary}

This section synthesizes findings from the security control review and technical scan. While no pre-existing risks were documented and the technical scan was clean, the questionnaire revealed significant policy-based risks that require immediate attention.

\begin{tabular}{@{}p{0.25\linewidth} p{0.55\linewidth} l@{}}
\toprule
\textbf{Risk Name} & \textbf{Overview} & \textbf{Severity} \\
\midrule
\addlinespace
\textbf{Lack of Multi-Factor Authentication (MFA)} & The absence of MFA for email, endpoints, and sensitive systems makes the organization highly susceptible to credential theft, phishing attacks, and unauthorized access. A single compromised password could lead to a significant breach. & \textcolor{criticalred}{\textbf{CRITICAL}} \\
\addlinespace
\textbf{Absence of Employee Acceptable Use Policy (AUP)} & Without a formal AUP, there is no enforceable standard for employee behavior regarding IT systems and data. This can lead to unintentional data exposure, misuse of assets, and a weakened security culture. & \textcolor{highorange}{\textbf{HIGH}} \\
\addlinespace
\textbf{Inadequate Security Onboarding} & New employees are not receiving security awareness training upon joining. This gap means they may be unaware of company policies, common threats like phishing, or their security responsibilities, making them a high-value target for attackers. & \textcolor{highorange}{\textbf{HIGH}} \\
\addlinespace
\bottomrule
\end{tabular}

% --- 6. Recommendations ---
\section*{6. Recommendations}

The following actions are recommended to mitigate the identified risks and strengthen the organization's overall security posture.

\subsection*{6.1. Implement Multi-Factor Authentication (Critical)}
\begin{itemize}
    \item \textbf{Action:} Deploy a mandatory MFA policy for all employees and contractors.
    \item \textbf{Details:} Prioritize the implementation of MFA on the following systems in order:
    \begin{enumerate}
        \item Email (e.g., Office 365, Google Workspace).
        \item Remote access systems (e.g., VPN, RDP).
        \item All systems containing sensitive or critical data.
        \item All employee computer logins (endpoint MFA).
    \end{enumerate}
    \item \textbf{Impact:} Drastically reduces the risk of account takeover and unauthorized access, even if user credentials are stolen.
\end{itemize}

\subsection*{6.2. Develop and Enforce an Acceptable Use Policy (High)}
\begin{itemize}
    \item \textbf{Action:} Create a comprehensive AUP that governs the use of all company IT assets, including networks, computers, and data.
    \item \textbf{Details:} The policy should clearly define acceptable and prohibited activities. All current employees must read and formally acknowledge the policy. This acknowledgment should be integrated into the onboarding process for all new hires.
    \item \textbf{Impact:} Establishes clear security expectations for all employees and provides a basis for enforcing security standards.
\end{itemize}

\subsection*{6.3. Integrate Security into Employee Onboarding (High)}
\begin{itemize}
    \item \textbf{Action:} Develop a mandatory security awareness training module for all new employees.
    \item \textbf{Details:} This training should be a required part of the onboarding process before system access is granted. It should cover, at a minimum: phishing awareness, password security, data handling procedures, and a review of the new AUP.
    \item \textbf{Impact:} Ensures that all personnel are equipped with baseline security knowledge from their first day, reducing the organization's vulnerability to common social engineering attacks.
\end{itemize}

% --- 7. Conclusion ---
\section*{7. Conclusion}
The security assessment of \textbf{Cloud Nine Software} presents a mixed but clear picture. The organization demonstrates technical competence in hardening network assets, as evidenced by the clean network scan. However, this technical strength is severely undermined by fundamental gaps in administrative and access controls.

The lack of MFA, a formal AUP, and security onboarding for new hires are significant deficiencies that expose the organization to a high risk of compromise. We strongly urge management to prioritize the recommendations outlined in this report, focusing first on the critical rollout of Multi-Factor Authentication. By addressing these policy and procedural gaps, Cloud Nine Software can build a resilient security foundation that complements its technical capabilities.

\end{document}
```