```latex
\documentclass[12pt]{article}

% --- PACKAGE IMPORTS ---
\usepackage[margin=1in]{geometry} % Set page margins
\usepackage{pifont}               % For checkmarks and crosses (dingbats)
\usepackage{booktabs}             % For professional-looking tables
\usepackage{hyperref}             % For hyperlinks and metadata
\usepackage{url}                  % For formatting URLs
\usepackage{seqsplit}             % For splitting long strings without spaces
\usepackage{xcolor}               % For colors
\usepackage{etoolbox}             % For conditional logic in commands

% --- DOCUMENT METADATA ---
\hypersetup{
    colorlinks=true,
    linkcolor=blue,
    filecolor=magenta,      
    urlcolor=cyan,
    pdftitle={Cybersecurity Posture Assessment Report},
    pdfauthor={Cybersecurity Analyst},
    pdfsubject={Security Analysis},
    pdfkeywords={Security, Report, Analysis},
    bookmarks=true
}

% --- CUSTOM COMMANDS ---
% Command to display Yes/No with appropriate symbols and colors
\newcommand{\checkans}[1]{%
  \ifstrequal{#1}{Yes}{{\color{green}\ding{51}} Yes}{{\color{red}\ding{55}} No}%
}

% --- DOCUMENT START ---
\begin{document}

% --- TITLE PAGE ---
\begin{titlepage}
    \centering
    \vspace*{1cm}
    \Huge
    \textbf{Cybersecurity Posture Assessment Report}
    
    \vspace{1.5cm}
    \Large
    Prepared for: \textbf{Opal Sky Media}
    
    \vspace{2cm}
    \large
    \textbf{Date of Report:} \today
    
    \vfill
    
    \large
    \textbf{Generated By:} \\
    Cybersecurity Analyst
    
    \vspace{0.5cm}
    \textit{This report contains sensitive information and should be handled with care. Distribution is restricted to authorized personnel only.}
    
\end{titlepage}

\tableofcontents
\newpage

% --- SECTION 1: EXECUTIVE SUMMARY ---
\section{Executive Summary}

This report provides a cybersecurity posture assessment for \textbf{Opal Sky Media}, based on a combination of organizational data, technical network scanning, and a review of pre-existing risks. The analysis reveals a mixed security landscape with several positive controls but also identifies critical vulnerabilities that require immediate attention.

The organization has implemented foundational security practices, including endpoint MFA, employee security training, and an acceptable use policy. However, two high-risk areas were identified:

\begin{enumerate}
    \item \textbf{Lack of Multi-Factor Authentication (MFA) on Critical Systems:} The absence of MFA on email and sensitive data systems represents a critical vulnerability. This gap significantly increases the risk of account compromise through phishing or credential theft, which could lead to data breaches and unauthorized access.
    
    \item \textbf{Systemic Internal RDP Exposure:} The technical scan identified a new host (\texttt{10.10.10.51}) with an exposed Remote Desktop Protocol (RDP) service. This finding, correlated with a pre-existing risk for a different host, indicates a pattern of insecure configuration. Exposed RDP is a primary vector for lateral movement and ransomware attacks within a network.
\end{enumerate}

Immediate remediation of these issues is strongly recommended to reduce the organization's attack surface and mitigate the risk of a significant security incident. Detailed recommendations are provided in Section \ref{sec:recommendations}.

% --- SECTION 2: ORGANIZATIONAL INFORMATION ---
\section{Organizational Information}

The following information was provided by the client and used as a baseline for this assessment.

\begin{table}[h!]
\centering
\caption{Client Organizational Data}
\label{tab:orgdata}
\begin{tabular}{@{}ll@{}}
\toprule
\textbf{Attribute} & \textbf{Value} \\ \midrule
Organization Name  & \textbf{Opal Sky Media} \\
Primary Email Domain & \texttt{OpalSkyMedia.org} \\
Primary Website    & \url{www.OpalSkyMedia.org} \\
External IP Address & \texttt{121.74.118.64} \\ \bottomrule
\end{tabular}
\end{table}

% --- SECTION 3: SECURITY CONTROL REVIEW ---
\section{Security Control Review}

A review of the organization's security controls was conducted via a questionnaire. The results highlight key areas of strength and weakness in the current security posture. "No" answers indicate significant gaps that increase risk.

\begin{table}[h!]
\centering
\caption{Security Controls Questionnaire Results}
\label{tab:controls}
\begin{tabular}{@{}lc@{}}
\toprule
\textbf{Control Question} & \textbf{Status} \\ \midrule
Do you require MFA to access email? & \checkans{No} \\
Do you require MFA to log into computers? & \checkans{Yes} \\
Do you require MFA to access sensitive data systems? & \checkans{No} \\
Does your organization have an employee acceptable use policy? & \checkans{Yes} \\
Does your organization do security awareness training for new employees? & \checkans{Yes} \\
Does your organization do security awareness training for all employees once per year? & \checkans{Yes} \\ \bottomrule
\end{tabular}
\end{table}

% --- SECTION 4: TECHNICAL SCAN RESULTS ---
\section{Technical Scan Results}

A network scan was performed to identify open ports and exposed services on the target system. The scan revealed an open RDP port, which is a common target for attackers.

\begin{itemize}
    \item \textbf{Target IP Address:} \texttt{10.10.10.51}
    \item \textbf{Scan Tool:} Nmap
\end{itemize}

\begin{table}[h!]
\centering
\caption{Open Port Analysis for \texttt{10.10.10.51}}
\label{tab:scanresults}
\begin{tabular}{@{}llll@{}}
\toprule
\textbf{Port} & \textbf{State} & \textbf{Service Name} & \textbf{Description} \\ \midrule
3389/tcp & open & ms-wbt-server & Microsoft Remote Desktop Protocol (RDP) \\ \bottomrule
\end{tabular}
\end{table}

% --- SECTION 5: RISK ASSESSMENT SUMMARY ---
\section{Risk Assessment Summary}

This section synthesizes findings from the security control review, technical scan, and pre-existing risk data. Each identified risk is assigned a severity level to aid in prioritization.

\begin{table}[h!]
\centering
\caption{Consolidated Risk Register}
\label{tab:riskregister}
\begin{tabular}{@{}p{0.4\linewidth}p{0.4\linewidth}l@{}}
\toprule
\textbf{Risk Name} & \textbf{Description} & \textbf{Severity} \\ \midrule
\textbf{MFA Not Enforced on Critical Systems} & Lack of MFA on email and sensitive data systems exposes the organization to account takeover and data breaches. & \textbf{\color{red}Critical} \\
\addlinespace
\textbf{Systemic RDP Exposure} & Open RDP on internal hosts (\texttt{10.10.10.51} and a previously reported \texttt{10.10.10.50}) creates a significant vector for lateral movement and ransomware deployment. & \textbf{\color{red}Critical} \\
\bottomrule
\end{tabular}
\end{table}

% --- SECTION 6: RECOMMENDATIONS ---
\section{Recommendations}
\label{sec:recommendations}

The following actions are recommended to mitigate the identified risks. Recommendations are prioritized based on severity and potential impact.

\subsection{Immediate Priority (Implement within 7 days)}
\begin{description}
    \item[Enforce MFA on Email:] Immediately enable and enforce MFA for all user accounts on the \texttt{OpalSkyMedia.org} email platform. This is the single most effective control to prevent business email compromise.
    
    \item[Enforce MFA on Sensitive Systems:] Identify all systems classified as containing sensitive data and enforce MFA for all access, both privileged and standard.
    
    \item[Remediate RDP Exposure:]
    \begin{itemize}
        \item Disable the RDP service on \texttt{10.10.10.51} if remote access is not required.
        \item If RDP is required, restrict access at the host or network firewall level to only allow connections from specific, authorized administrative IP addresses.
        \item Investigate the previously identified host (\texttt{10.10.10.50}) to ensure it has been remediated.
    \end{itemize}
\end{description}

\subsection{Strategic Priority (Implement within 90 days)}
\begin{description}
    \item[Develop an Access Control Policy:] Create and enforce a formal policy that mandates MFA for all systems, especially those that are internet-facing or contain sensitive information.
    
    \item[Implement Network Segmentation:] Segment the internal network to limit the ability of an attacker to move laterally. For example, place servers in a separate VLAN from user workstations and restrict traffic between them.
    
    \item[Conduct Comprehensive Vulnerability Scanning:] Perform authenticated vulnerability scans across the entire internal network to proactively identify and remediate other instances of misconfiguration, missing patches, and exposed services.
\end{description}

\end{document}
```