```latex
\documentclass[12pt]{article}

% ----------------------------------------------------------------------
% PREAMBLE
% ----------------------------------------------------------------------
\usepackage[margin=1in]{geometry}
\usepackage{pifont} % For checkmarks and crosses
\usepackage{booktabs} % For professional tables
\usepackage{hyperref} % For hyperlinks
\usepackage{url}      % For URL formatting
\usepackage{seqsplit} % For splitting long strings in tt font
\usepackage{graphicx}
\usepackage{xcolor}

% --- Hyperref Setup ---
\hypersetup{
    colorlinks=true,
    linkcolor=blue,
    filecolor=magenta,      
    urlcolor=cyan,
    pdftitle={Cybersecurity Posture Report},
    pdfpagemode=FullScreen,
}

% --- Custom Commands ---
\newcommand{\yes}{\ding{51}} % Checkmark
\newcommand{\no}{\ding{55}}  % Cross

% ----------------------------------------------------------------------
% DOCUMENT START
% ----------------------------------------------------------------------
\begin{document}

% --- Title Page ---
\begin{titlepage}
    \centering
    \vspace*{1cm}
    \Huge \textbf{Cybersecurity Posture Report}
    \vspace{1.5cm}
    \Large \textbf{Prepared for:} \\
    \vspace{0.5cm}
    \huge Ironclad Logistics
    \vspace{2cm}
    \large \textbf{Date of Report:} \\
    \vspace{0.5cm}
    \Large \today
    \vfill
    \large \textbf{Generated by:} \\
    \vspace{0.5cm}
    \Large Expert Cybersecurity Analyst
\end{titlepage}

\tableofcontents
\newpage

% ----------------------------------------------------------------------
% SECTION 1: EXECUTIVE SUMMARY
% ----------------------------------------------------------------------
\section{Executive Summary}

This report provides a comprehensive analysis of the cybersecurity posture for \textbf{Ironclad Logistics}, based on a review of organizational security controls, technical network scanning, and pre-existing risk data.

The assessment reveals a mixed security posture. While the organization has implemented foundational controls such as Multi-Factor Authentication (MFA) for email and computer access, critical gaps exist that present a significant risk.

\textbf{Key Findings Include:}
\begin{itemize}
    \item \textbf{Critical Control Gap:} Sensitive data systems are not protected by Multi-Factor Authentication. This is a primary vulnerability that could lead to a severe data breach if credentials are compromised.
    \item \textbf{High-Risk Process Gap:} New employees do not receive security awareness training during onboarding, making them a prime target for social engineering attacks like phishing.
    \item \textbf{Confirmed Technical Risk:} A network scan confirmed a pre-existing critical risk: an open service (Port 22/SSH) on the localhost interface (\texttt{127.0.0.1}). This misconfiguration could be exploited by malicious software on the host system.
\end{itemize}

Immediate remediation of these identified risks is strongly recommended to reduce the organization's attack surface and protect critical assets. Actionable recommendations are detailed in Section \ref{sec:recommendations}.

% ----------------------------------------------------------------------
% SECTION 2: ORGANIZATIONAL INFORMATION
% ----------------------------------------------------------------------
\section{Organizational Information}

The following details were provided for the assessment.

\begin{tabular}{@{}ll}
    \toprule
    \textbf{Attribute} & \textbf{Value} \\
    \midrule
    Organization Name & \textbf{Ironclad Logistics} \\
    Email Domain & \texttt{IroncladLogistics.net} \\
    Website Domain & \url{www.IroncladLogistics.net} \\
    External IP Address & \texttt{110.173.229.129} \\
    \bottomrule
\end{tabular}

% ----------------------------------------------------------------------
% SECTION 3: SECURITY CONTROL REVIEW
% ----------------------------------------------------------------------
\section{Security Control Review}

A review of the organization's self-reported security controls was conducted. The results below highlight both strengths and critical weaknesses in the current security framework. Answers marked with a red \no\ represent significant gaps requiring immediate attention.

\begin{table}[h!]
\centering
\caption{Security Controls Questionnaire Analysis}
\label{tab:controls}
\begin{tabular}{@{}p{0.7\linewidth}cc@{}}
    \toprule
    \textbf{Control Question} & \textbf{Response} & \textbf{Status} \\
    \midrule
    Do you require MFA to access email? & Yes & \textcolor{green}{\yes} \\
    Do you require MFA to log into computers? & Yes & \textcolor{green}{\yes} \\
    \textbf{Do you require MFA to access sensitive data systems?} & \textbf{No} & \textbf{\textcolor{red}{\no}} \\
    Does your organization have an employee acceptable use policy? & Yes & \textcolor{green}{\yes} \\
    \textbf{Does your organization do security awareness training for new employees?} & \textbf{No} & \textbf{\textcolor{red}{\no}} \\
    Does your organization do security awareness training for all employees at least once per year? & Yes & \textcolor{green}{\yes} \\
    \bottomrule
\end{tabular}
\end{table}

\subsection*{Analysis of Control Gaps}
\begin{itemize}
    \item \textbf{MFA on Sensitive Systems:} The absence of MFA on systems containing sensitive data is a critical vulnerability. Should an attacker gain access to a user's credentials, they would have direct access to the organization's most valuable information.
    \item \textbf{New Employee Onboarding:} Failing to train new employees on security best practices from day one leaves the organization vulnerable. New hires are often targeted by phishing and social engineering attacks as they are less familiar with company policies and communication patterns.
\end{itemize}

% ----------------------------------------------------------------------
% SECTION 4: TECHNICAL SCAN RESULTS
% ----------------------------------------------------------------------
\section{Technical Scan Results}

A network scan was performed to identify open ports and exposed services on the specified target.

\begin{itemize}
    \item \textbf{Target IP:} \texttt{127.0.0.1}
    \item \textbf{Scan Date:} \today
\end{itemize}

The scan revealed the following open port, which directly correlates with the pre-existing risk documented in \texttt{Input\_3\_Current\_Risks\_JSON}.

\begin{table}[h!]
\centering
\caption{Open Port Analysis for Target \texttt{127.0.0.1}}
\label{tab:nmap}
\begin{tabular}{@{}llll@{}}
    \toprule
    \textbf{Port} & \textbf{State} & \textbf{Service (Inferred)} & \textbf{Notes} \\
    \midrule
    22 & Open & SSH & The presence of an open port on the localhost \\
       &      &       & interface is a high-risk configuration. It can be \\
       &      &       & exploited by malware or other processes on the \\
       &      &       & same machine to escalate privileges or pivot. \\
    \bottomrule
\end{tabular}
\end{table}

% ----------------------------------------------------------------------
% SECTION 5: CONSOLIDATED RISK ASSESSMENT
% ----------------------------------------------------------------------
\section{Consolidated Risk Assessment}

The following table synthesizes findings from the security control review, technical scan, and pre-existing risk data into a prioritized list of current risks.

\begin{table}[h!]
\centering
\caption{Summary of Identified Risks}
\label{tab:risks}
\begin{tabular}{@{}p{0.1\linewidth}p{0.3\linewidth}p{0.4\linewidth}l@{}}
    \toprule
    \textbf{Risk ID} & \textbf{Risk Name} & \textbf{Description} & \textbf{Severity} \\
    \midrule
    RISK-001 & Exposed Localhost Service & The technical scan confirmed a pre-existing risk of an exposed SSH service on \texttt{127.0.0.1}. This could be leveraged by local malware for persistence or privilege escalation. & \textbf{Critical} \\
    \addlinespace
    RISK-002 & Lack of MFA on Sensitive Systems & Critical data systems are accessible with only a username and password, making them highly vulnerable to credential stuffing and phishing attacks. & \textbf{Critical} \\
    \addlinespace
    RISK-003 & Inadequate Onboarding Security Training & New employees are not trained on security policies, creating a significant vulnerability to social engineering and accidental data exposure. & \textbf{High} \\
    \bottomrule
\end{tabular}
\end{table}

% ----------------------------------------------------------------------
% SECTION 6: RECOMMENDATIONS
% ----------------------------------------------------------------------
\section{Recommendations}
\label{sec:recommendations}

The following actions are recommended to mitigate the identified risks. They are prioritized based on severity and potential impact.

\begin{enumerate}
    \item \textbf{Priority 1 (RISK-001 \& RISK-002 - Critical): Implement Comprehensive Access Controls.}
    \begin{itemize}
        \item \textbf{Immediate Action:} Enforce mandatory Multi-Factor Authentication (MFA) across all systems that store, process, or access sensitive data. This is the single most effective control to mitigate the risk of unauthorized access.
        \item \textbf{Immediate Action:} Investigate the service running on port 22 on \texttt{127.0.0.1}. If this service is not essential for business operations, it must be disabled immediately. If it is required, ensure it is properly configured and firewalled to only allow access from authorized local processes.
    \end{itemize}
    \vspace{0.5cm}
    \item \textbf{Priority 2 (RISK-003 - High): Enhance Security Awareness Program.}
    \begin{itemize}
        \item \textbf{Short-Term Action:} Develop and implement a mandatory security awareness training module for all new employees as part of the standard onboarding process. This training must cover, at a minimum: acceptable use policies, phishing and social engineering identification, and proper data handling procedures.
    \end{itemize}
\end{enumerate}

\end{document}
```