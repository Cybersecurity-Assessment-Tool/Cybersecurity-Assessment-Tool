```latex
\documentclass[12pt, a4paper]{article}

% Required Packages
\usepackage[margin=1in]{geometry}
\usepackage{pifont} % For checkmarks and crosses
\usepackage{booktabs} % For professional tables
\usepackage{hyperref} % For clickable links and TOC
\usepackage{url} % For formatting URLs
\usepackage{seqsplit} % For splitting long strings in texttt
\usepackage{graphicx} % For potential logo
\usepackage{fancyhdr} % For headers and footers

% Document Metadata
\title{Cybersecurity Posture Assessment Report}
\author{Cybersecurity Analysis Division}
\date{\today}

% Hyperref Setup
\hypersetup{
    colorlinks=true,
    linkcolor=blue,
    filecolor=magenta,      
    urlcolor=cyan,
    pdftitle={Cybersecurity Posture Assessment Report},
    pdfpagemode=FullScreen,
}

% Header and Footer
\pagestyle{fancy}
\fancyhf{}
\fancyhead[L]{Oasis Wellness Security Report}
\fancyfoot[C]{\thepage}

\begin{document}

\maketitle
\thispagestyle{empty}
\newpage

\tableofcontents
\newpage

% --- 1. Executive Summary ---
\section{Executive Summary}

This report provides a comprehensive cybersecurity assessment for \textbf{Oasis Wellness}, based on network scans, a security controls questionnaire, and a review of pre-existing risks. The analysis was conducted on \today.

The assessment reveals several critical and high-risk security gaps that require immediate attention. The most severe finding is a pre-existing vulnerability, \textbf{`Localhost Exposed'}, rated with a CVSS score of 10.0 (Critical). This is corroborated by a network scan identifying an open service on the localhost interface (\texttt{127.0.0.1}).

Furthermore, significant procedural and policy-based weaknesses were identified. The organization lacks Multi-Factor Authentication (MFA) for computer and sensitive data access, and does not conduct security awareness training for new or existing employees. This combination of technical vulnerability and procedural weakness creates a high-risk environment, making the organization susceptible to credential theft, unauthorized access, and lateral movement within the network.

Key recommendations focus on the immediate remediation of the exposed service, followed by the rapid implementation of a comprehensive MFA policy and the establishment of a robust security awareness training program.

% --- 2. Organizational Information ---
\section{Organizational Information}

The following details were provided for the assessment. This information is used to establish the context and scope of the review.

\begin{table}[h!]
\centering
\begin{tabular}{@{}ll@{}}
\toprule
\textbf{Attribute} & \textbf{Value} \\ \midrule
Organization Name & Oasis Wellness \\
Primary Email Domain & \texttt{OasisWellness.org} \\
Primary Website & \url{www.OasisWellness.org} \\
External IP Address & \texttt{25.22.139.189} \\ \bottomrule
\end{tabular}
\caption{Client Organizational Details.}
\label{tab:org_info}
\end{table}

% --- 3. Security Control Review ---
\section{Security Control Review}

A security controls questionnaire was completed to evaluate the organization's current policies and procedures. The responses indicate significant gaps in foundational security practices. A "No" response highlights a deviation from security best practices and is flagged as a gap.

\begin{table}[h!]
\centering
\begin{tabular}{@{}p{0.6\linewidth}cc@{}}
\toprule
\textbf{Control Question} & \textbf{Response} & \textbf{Status} \\ \midrule
Do you require MFA to access email? & \ding{51} & Implemented \\
Do you require MFA to log into computers? & \ding{55} & \textbf{Critical Gap} \\
Do you require MFA to access sensitive data systems? & \ding{55} & \textbf{Critical Gap} \\
Does your organization have an employee acceptable use policy? & \ding{51} & Implemented \\
Does your organization do security awareness training for new employees? & \ding{55} & \textbf{High Risk} \\
Does your organization do security awareness training for all employees at least once per year? & \ding{55} & \textbf{High Risk} \\ \bottomrule
\end{tabular}
\caption{Security Controls Questionnaire Results.}
\label{tab:controls_review}
\end{table}

% --- 4. Technical Scan Results ---
\section{Technical Scan Results}

A network scan was performed to identify open ports and services on the target system. The scan provides insight into the external attack surface and potential vulnerabilities.

\begin{itemize}
    \item \textbf{Target IP:} \texttt{127.0.0.1}
    \item \textbf{Scan Tool:} Nmap
    \item \textbf{Host Status:} Up
\end{itemize}

The following table details the open ports discovered during the scan. The presence of an open SSH port on the localhost interface is a significant finding, especially when correlated with the pre-existing risk documented in Section 5.

\begin{table}[h!]
\centering
\begin{tabular}{@{}llll@{}}
\toprule
\textbf{Port} & \textbf{State} & \textbf{Service} & \textbf{Notes} \\ \midrule
22/tcp & open & ssh & No version information was available. Secure Shell access. \\ \bottomrule
\end{tabular}
\caption{Open Ports Detected on \texttt{127.0.0.1}.}
\label{tab:scan_results}
\end{table}

% --- 5. Overall Risk Assessment ---
\section{Overall Risk Assessment}

This section synthesizes the findings from the security control review, technical scans, and pre-existing risk data into a consolidated list of identified risks.

\begin{table}[h!]
\centering
\begin{tabular}{@{}p{0.25\linewidth}p{0.15\linewidth}p{0.5\linewidth}@{}}
\toprule
\textbf{Risk Title} & \textbf{Severity} & \textbf{Description} \\ \midrule
\textbf{Exposed Localhost Service} & \textbf{Critical (10.0)} & A pre-existing risk indicates a critical service intended for local access is exposed. The Nmap scan confirms an open SSH port on \texttt{127.0.0.1}, validating this high-severity finding. \\
\addlinespace
\textbf{Lack of MFA on Critical Systems} & \textbf{Critical} & MFA is not enforced for logging into computers or accessing sensitive data. This severely weakens access controls and makes the organization highly vulnerable to credential compromise. \\
\addlinespace
\textbf{No Security Awareness Training} & \textbf{High} & The absence of a security training program for new and existing employees leaves the organization vulnerable to phishing, social engineering, and other human-targeted attacks. \\ \bottomrule
\end{tabular}
\caption{Consolidated Risk Summary.}
\label{tab:risk_summary}
\end{table}

% --- 6. Recommendations ---
\section{Recommendations}

Based on the identified risks, the following prioritized recommendations are provided to improve the cybersecurity posture of \textbf{Oasis Wellness}.

\subsection{Immediate Priority (0-7 Days)}
\begin{enumerate}
    \item \textbf{Remediate Exposed Localhost Service:} Immediately investigate the `Localhost Exposed` vulnerability. Determine why the service on \texttt{127.0.0.1} (port 22/ssh) is accessible and reconfigure network rules or service bindings to restrict access to only the local machine. If this service is not required, it should be disabled.
\end{enumerate}

\subsection{High Priority (1-3 Months)}
\begin{enumerate}
    \item \textbf{Implement Comprehensive MFA:} Deploy a robust MFA solution across all endpoints (computers) and systems containing sensitive data. This is the single most effective control to prevent unauthorized access from compromised credentials.
    \item \textbf{Establish Security Awareness Training Program:} Develop and implement a mandatory security awareness training program. This must include:
    \begin{itemize}
        \item Onboarding training for all new hires before they are granted system access.
        \item Annual refresher training for all employees to keep them updated on current threats.
        \item Regular phishing simulations to test and reinforce training.
    \end{itemize}
\end{enumerate}

\subsection{Medium Priority (3-6 Months)}
\begin{enumerate}
    \item \textbf{Conduct Regular Vulnerability Scanning:} Implement a program for regular, automated vulnerability scanning of all internal and external assets. This will help proactively identify and remediate technical vulnerabilities before they can be exploited.
    \item \textbf{Review and Harden SSH Configuration:} For any required SSH services, ensure they are hardened according to best practices. This includes disabling root login, using key-based authentication instead of passwords, and implementing IP whitelisting.
\end{enumerate}

\end{document}
```