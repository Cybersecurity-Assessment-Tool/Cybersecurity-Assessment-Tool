```latex
\documentclass[12pt]{article}

% --- PACKAGE IMPORTS ---
\usepackage[margin=1in]{geometry}
\usepackage{pifont} % For checkmarks and crosses
\usepackage{booktabs} % For professional tables
\usepackage[hidelinks]{hyperref} % For clickable links without boxes
\usepackage{url} % For formatting URLs
\usepackage{seqsplit} % For splitting long strings in tt font
\usepackage{graphicx}
\usepackage{fancyhdr}
\usepackage{xcolor}
\usepackage{datetime}

% --- DOCUMENT METADATA ---
\title{Cybersecurity Assessment Report}
\author{Cybersecurity Analysis Division}
\date{\today}

% --- CUSTOM COMMANDS & SETTINGS ---
\pagestyle{fancy}
\fancyhf{} % Clear all header and footer fields
\fancyhead[L]{\textbf{Cybersecurity Assessment Report}}
\fancyhead[R]{\textbf{Maple Leaf Logistics}}
\fancyfoot[C]{\thepage}
\renewcommand{\headrulewidth}{0.4pt}
\renewcommand{\footrulewidth}{0.4pt}

% Define colors for severity
\definecolor{criticalred}{HTML}{D10000}
\definecolor{highorange}{HTML}{E25F00}
\definecolor{mediumyellow}{HTML}{F5C700}
\definecolor{lowblue}{HTML}{0073E6}
\definecolor{infogray}{HTML}{808080}

\newcommand{\severitycritical}[1]{\textcolor{criticalred}{\textbf{#1}}}
\newcommand{\severityhigh}[1]{\textcolor{highorange}{\textbf{#1}}}
\newcommand{\severitymedium}[1]{\textcolor{mediumyellow}{\textbf{#1}}}

\begin{document}

\maketitle
\thispagestyle{empty}
\newpage

\tableofcontents
\newpage

% --- SECTION 1: EXECUTIVE SUMMARY ---
\section{Executive Summary}
This report details the findings of a cybersecurity assessment for \textbf{Maple Leaf Logistics}, conducted on \today. The assessment combined a technical network scan, a review of existing risk documentation, and an analysis of organizational security controls via a questionnaire.

The overall security posture reveals a significant disparity between technical and administrative controls. While the targeted technical scan of host \texttt{192.168.0.5} showed no open ports—a positive sign of a hardened network perimeter—the organizational review uncovered critical gaps in fundamental security practices.

Key findings include:
\begin{itemize}
    \item \textbf{Critical Control Gaps:} Multi-Factor Authentication (MFA) is not enforced for accessing email or other sensitive data systems. This represents a critical vulnerability, as compromised credentials could lead directly to a significant data breach.
    \item \textbf{High-Risk Policy Deficiencies:} The organization lacks a formal Acceptable Use Policy (AUP) and does not conduct security awareness training for new or existing employees. These omissions cultivate a high-risk environment where employees are more susceptible to social engineering attacks like phishing.
    \item \textbf{Outdated Risk Information:} A previously documented risk, "Unencrypted Web Server," was found to be mitigated. The associated port (80/tcp) is now closed. This highlights the importance of maintaining an up-to-date risk register.
\end{itemize}

Immediate action should be focused on implementing MFA across all critical systems and establishing a foundational security awareness and policy program. Addressing these administrative and procedural weaknesses is paramount to safeguarding the organization's assets.

% --- SECTION 2: ORGANIZATIONAL INFORMATION ---
\section{Organizational Information}
The following details were provided for the assessment.

\begin{tabular}{@{}ll}
    \toprule
    \textbf{Attribute} & \textbf{Value} \\
    \midrule
    Organization Name & \textbf{Maple Leaf Logistics} \\
    Email Domain & \seqsplit{\texttt{MapleLeafLogistics.com}} \\
    Website Domain & \seqsplit{\url{www.MapleLeafLogistics.com}} \\
    External IP Address & \seqsplit{\texttt{221.202.232.75}} \\
    \bottomrule
\end{tabular}

% --- SECTION 3: SECURITY CONTROL REVIEW ---
\section{Security Control Review}
A review of administrative and procedural security controls was conducted based on a questionnaire. The results indicate several significant gaps in foundational security practices.

\begin{tabular}{@{}p{0.6\linewidth}cp{0.2\linewidth}@{}}
    \toprule
    \textbf{Control Question} & \textbf{Response} & \textbf{Assessment} \\
    \midrule
    Do you require MFA to access email? & \ding{55} No & \severitycritical{Critical Gap} \\
    Do you require MFA to log into computers? & \ding{51} Yes & Control in Place \\
    Do you require MFA to access sensitive data systems? & \ding{55} No & \severitycritical{Critical Gap} \\
    Does your organization have an employee acceptable use policy? & \ding{55} No & \severityhigh{High Risk} \\
    Does your organization do security awareness training for new employees? & \ding{55} No & \severityhigh{High Risk} \\
    Does your organization do security awareness training for all employees at least once per year? & \ding{55} No & \severityhigh{High Risk} \\
    \bottomrule
\end{tabular}

% --- SECTION 4: TECHNICAL SCAN RESULTS ---
\section{Technical Scan Results}
A network scan was performed on the specified target to identify accessible services and potential vulnerabilities.

\begin{itemize}
    \item \textbf{Target IP Address:} \texttt{192.168.0.5}
    \item \textbf{Scan Date:} \today
\end{itemize}

The scan revealed no open ports on the target host. This indicates a properly configured firewall or that no network services are actively listening on the ports that were scanned.

\begin{tabular}{@{}llll@{}}
    \toprule
    \textbf{Port} & \textbf{State} & \textbf{Service} & \textbf{Version} \\
    \midrule
    80/tcp & closed & http & N/A \\
    \bottomrule
\end{tabular}

% --- SECTION 5: RISK ASSESSMENT ---
\section{Risk Assessment}
This section synthesizes findings from the security control review, the technical scan, and pre-existing risk data. The most significant risks are currently administrative and procedural.

\begin{tabular}{@{}p{0.2\linewidth}p{0.1\linewidth}p{0.55\linewidth}@{}}
    \toprule
    \textbf{Risk Name} & \textbf{Severity} & \textbf{Description and Impact} \\
    \midrule
    No MFA on Email & \severitycritical{Critical} & Lack of MFA on email exposes the organization to account takeovers via phishing or credential stuffing, potentially leading to data exfiltration, financial fraud, and further system compromise. \\
    \addlinespace
    No MFA on Sensitive Data Systems & \severitycritical{Critical} & Failure to protect sensitive data systems with MFA means that a single compromised password could grant an attacker access to the organization's most valuable information assets. \\
    \addlinespace
    Lack of Security Awareness Training & \severityhigh{High} & Without regular training, employees are ill-equipped to identify and report security threats like phishing. This makes them the primary target for attackers seeking initial access. \\
    \addlinespace
    Absence of Acceptable Use Policy (AUP) & \severityhigh{High} & Without a formal AUP, there are no clear guidelines for employees on the secure use of company assets, increasing the risk of insider threat and non-compliance. \\
    \addlinespace
    Unencrypted Web Server (Port 80) & Mitigated & A pre-existing risk noted an open Port 80. Our scan on \today\ confirmed this port is \textbf{closed}, indicating the risk has been remediated or was based on outdated information. \\
    \bottomrule
\end{tabular}

% --- SECTION 6: RECOMMENDATIONS ---
\section{Recommendations}
The following actions are recommended to mitigate the identified risks and improve the overall security posture of \textbf{Maple Leaf Logistics}.

\subsection{Implement Multi-Factor Authentication (Critical)}
\begin{itemize}
    \item \textbf{Immediate Action:} Enforce MFA for all users on the primary email system (\seqsplit{\texttt{MapleLeafLogistics.com}}). This is the highest priority action.
    \item \textbf{Near-Term Action:} Identify all systems containing sensitive data and enforce MFA for all access, both internal and external.
\end{itemize}

\subsection{Establish Foundational Security Programs (High)}
\begin{itemize}
    \item \textbf{Acceptable Use Policy (AUP):} Develop and implement a formal AUP that clearly defines the rules for using company networks, devices, and data. Require all employees to read and acknowledge the policy.
    \item \textbf{Security Awareness Training:} Procure and deploy a security awareness training program. All employees must complete an initial training module upon hiring and an annual refresher course. Training should focus on identifying phishing, password hygiene, and reporting security incidents.
\end{itemize}

\subsection{Maintain Risk Register (Informational)}
\begin{itemize}
    \item \textbf{Action:} Conduct a review of the organization's risk register. The finding that the "Unencrypted Web Server" risk was outdated suggests other items may also be inaccurate. Validate and update all documented risks to ensure resources are focused on current threats.
\end{itemize}

\end{document}
```