```latex
\documentclass[12pt]{article}

% Required Packages
\usepackage[margin=1in]{geometry} % For setting page margins
\usepackage{pifont}               % For checkmarks and crosses (\ding)
\usepackage{booktabs}             % For professional-looking tables
\usepackage{graphicx}             % For including logos (optional)
\usepackage{hyperref}             % For clickable links and references
\usepackage{url}                  % For formatting URLs
\usepackage{seqsplit}             % For splitting long, unbreakable strings
\usepackage{xcolor}               % For custom colors

% Document Information
\title{Cybersecurity Risk Assessment Report \\ \large For: \textbf{Sterling Silver}}
\author{Cybersecurity Analyst}
\date{\today}

% Hyperref Setup
\hypersetup{
    colorlinks=true,
    linkcolor=blue,
    filecolor=magenta,      
    urlcolor=cyan,
    pdftitle={Cybersecurity Risk Assessment Report},
    pdfpagemode=FullScreen,
}

\begin{document}

\maketitle
\thispagestyle{empty}
\newpage

\tableofcontents
\newpage

% ------------------------------------------------------------------------------
% 1. Executive Summary
% ------------------------------------------------------------------------------
\section*{1. Executive Summary}

This report provides a comprehensive cybersecurity assessment for \textbf{Sterling Silver}, based on network scans, an organizational security questionnaire, and a review of pre-existing risks. The analysis reveals several critical and high-risk vulnerabilities that require immediate attention to mitigate the threat of unauthorized access, data breaches, and system compromise.

Key findings include a publicly accessible FTP server running a dangerously outdated and vulnerable version of \texttt{vsftpd} (2.3.4), which allows for anonymous logins. This configuration is associated with a known remote code execution vulnerability (CVE-2011-2523). Furthermore, significant gaps were identified in the organization's authentication policies, with Multi-Factor Authentication (MFA) not enforced for email or computer access. The lack of mandatory annual security awareness training for all employees exacerbates these technical risks by increasing the likelihood of human error.

The overall security posture is considered poor. Immediate remediation of the identified critical vulnerabilities is strongly recommended to protect the organization's assets and data integrity.

% ------------------------------------------------------------------------------
% 2. Organizational Information
% ------------------------------------------------------------------------------
\section*{2. Organizational Information}

The following details were provided for the assessment.

\begin{tabular}{@{}ll}
    \toprule
    \textbf{Attribute} & \textbf{Value} \\
    \midrule
    Organization Name & Sterling Silver \\
    Email Domain & \texttt{SterlingSilver.org} \\
    Website Domain & \url{www.SterlingSilver.org} \\
    External IP Address & \texttt{48.244.75.229} \\
    \bottomrule
\end{tabular}

% ------------------------------------------------------------------------------
% 3. Security Control Review (Questionnaire Analysis)
% ------------------------------------------------------------------------------
\section*{3. Security Control Review}

A review of the security questionnaire highlights significant gaps in foundational security controls. "No" answers indicate a lack of implementation and represent areas of high risk.

\begin{tabular}{@{}p{0.8\textwidth}c}
    \toprule
    \textbf{Control Question} & \textbf{Status} \\
    \midrule
    Do you require MFA to access email? & \ding{55} \\
    Do you require MFA to log into computers? & \ding{55} \\
    Do you require MFA to access sensitive data systems? & \ding{51} \\
    Does your organization have an employee acceptable use policy? & \ding{51} \\
    Does your organization do security awareness training for new employees? & \ding{51} \\
    Does your organization do security awareness training for all employees at least once per year? & \ding{55} \\
    \bottomrule
\end{tabular}
\\
\vspace{5mm}
\textbf{Key:} \ding{51} = Yes (Control in place) \quad \ding{55} = No (Control gap identified)

% ------------------------------------------------------------------------------
% 4. Technical Scan Results
% ------------------------------------------------------------------------------
\section*{4. Technical Scan Results}

A network scan was performed on the internal host \texttt{10.0.0.15}. The results indicate a critical vulnerability that must be addressed immediately.

\subsection*{Host: \texttt{10.0.0.15}}
\begin{tabular}{@{}lllll}
    \toprule
    \textbf{Port} & \textbf{State} & \textbf{Service} & \textbf{Version} & \textbf{Finding} \\
    \midrule
    21/tcp & Open & ftp & vsftpd 2.3.4 & \textbf{Critical:} Anonymous FTP login allowed. \\
    & & & & \textbf{Critical:} This version is vulnerable to a \\
    & & & & known backdoor (CVE-2011-2523), which \\
    & & & & allows for remote command execution. \\
    \bottomrule
\end{tabular}

\paragraph{Analysis:} The presence of an FTP server with anonymous login enabled poses a significant risk of unauthorized data access and exfiltration. The outdated \texttt{vsftpd} version 2.3.4 is widely known to contain a critical backdoor vulnerability. An attacker could exploit this to gain complete control over the server. The use of unencrypted FTP also exposes any transmitted credentials and data to interception.

% ------------------------------------------------------------------------------
% 5. Consolidated Risk Assessment
% ------------------------------------------------------------------------------
\section*{5. Consolidated Risk Assessment}

The following table synthesizes findings from the technical scan, security control review, and pre-existing risk data into a prioritized list of security risks.

\begin{tabular}{@{}p{0.2\textwidth}p{0.1\textwidth}p{0.6\textwidth}}
    \toprule
    \textbf{Risk Name} & \textbf{Severity} & \textbf{Overview} \\
    \midrule
    \textbf{Vulnerable FTP Server} & \textbf{Critical} & An internal server (\texttt{10.0.0.15}) is running \texttt{vsftpd 2.3.4}, which is vulnerable to remote code execution (CVE-2011-2523). Anonymous login is also enabled, allowing unauthenticated access. \\
    \addlinespace
    \textbf{Insufficient MFA Policy} & \textbf{High} & Multi-Factor Authentication is not required for accessing email or for computer logins. This significantly increases the risk of account compromise via credential theft or phishing. \\
    \addlinespace
    \textbf{Inadequate Security Training} & \textbf{High} & The lack of mandatory annual security awareness training for all staff increases susceptibility to social engineering, phishing, and other human-targeted attacks. \\
    \addlinespace
    \textbf{Outdated Windows Policy} & \textbf{Medium} & \textit{(Pre-existing risk)} Workstations are running Windows 7, which is an end-of-life operating system no longer receiving security updates. This exposes them to numerous known vulnerabilities. \\
    \bottomrule
\end{tabular}

% ------------------------------------------------------------------------------
% 6. Recommendations
% ------------------------------------------------------------------------------
\section*{6. Recommendations}

The following actions are recommended to address the identified risks, prioritized by severity.

\subsection*{Priority 1: Immediate Actions (Critical Risks)}
\begin{enumerate}
    \item \textbf{Remediate Vulnerable FTP Server:} Immediately take the server at \texttt{10.0.0.15} offline.
    \begin{itemize}
        \item If the FTP service is not required for business operations, decommission it permanently.
        \item If it is required, upgrade \texttt{vsftpd} to the latest stable version, disable anonymous access, and enforce strong authentication.
        \item Replace FTP with a secure alternative like SFTP (SSH File Transfer Protocol) or FTPS (FTP over SSL/TLS).
    \end{itemize}
\end{enumerate}

\subsection*{Priority 2: High-Priority Actions}
\begin{enumerate}
    \setcounter{enumi}{1}
    \item \textbf{Implement Comprehensive MFA:} Deploy MFA across all user accounts for email access (e.g., Office 365, G Suite) and computer logins (e.g., Windows Hello, Duo). This is one of the most effective controls for preventing account takeovers.
    \item \textbf{Establish Annual Security Training:} Institute a mandatory security awareness training program for all employees, to be completed annually. The training should cover phishing, password security, acceptable use, and incident reporting.
\end{enumerate}

\subsection*{Priority 3: Medium-Priority Actions}
\begin{enumerate}
    \setcounter{enumi}{3}
    \item \textbf{Upgrade End-of-Life Systems:} Develop and execute a plan to upgrade all Windows 7 workstations to a modern, supported operating system like Windows 10 or 11. This will ensure systems receive critical security patches.
\end{enumerate}

\end{document}
```