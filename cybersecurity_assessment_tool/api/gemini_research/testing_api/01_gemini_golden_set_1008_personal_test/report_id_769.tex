```latex
\documentclass[12pt]{article}

% Required Packages
\usepackage[margin=1in]{geometry}
\usepackage{pifont} % For checkmarks and crosses
\usepackage{booktabs} % For professional tables
\usepackage{hyperref} % For clickable links and references
\usepackage{url} % For URL formatting
\usepackage{seqsplit} % For splitting long strings
\usepackage{graphicx}
\usepackage{xcolor}
\usepackage{datetime2}

% --- Document Setup ---
% Define colors for risk levels
\definecolor{critical}{HTML}{D7263D}
\definecolor{high}{HTML}{F49D40}
\definecolor{medium}{HTML}{F4D440}
\definecolor{low}{HTML}{84B854}

% Hyperref Setup
\hypersetup{
    colorlinks=true,
    linkcolor=blue,
    filecolor=magenta,      
    urlcolor=cyan,
    pdftitle={Cybersecurity Posture Report},
    pdfpagemode=FullScreen,
}

% --- Document Body ---
\begin{document}

% --- Title Page ---
\begin{titlepage}
    \centering
    \vspace*{1cm}
    \Huge{\textbf{Cybersecurity Posture Report}}
    \vspace{0.5cm}
    \Large{Prepared for: \textbf{New Era}}
    \vspace{2cm}
    \includegraphics[width=0.4\textwidth]{example-image-a} % Placeholder for company logo
    \vfill
    \large{
        \textbf{Date of Report:} \today \\
        \textbf{Scan Date:} 2025-11-22 \\
        \textbf{Report ID:} CSR-20251122-001
    }
\end{titlepage}

\tableofcontents
\newpage

% --- Section 1: Executive Summary ---
\section{Executive Summary}
This report provides a comprehensive analysis of the cybersecurity posture for \textbf{New Era}, based on a combination of technical network scanning, a review of organizational security controls, and an assessment of known risks. The assessment was conducted on \DTMdate{2025-11-22}.

The overall security posture is assessed as having a \textbf{High} level of risk. Key findings include critical gaps in access control and the presence of outdated, vulnerable software on externally-facing systems.

The most significant findings are:
\begin{itemize}
    \item \textbf{Critical - Lack of Multi-Factor Authentication (MFA):} MFA is not enforced for accessing company email or for logging into employee computers. This exposes the organization to significant risk from credential theft and unauthorized access.
    \item \textbf{High - Outdated Web Server Software:} The external web server at \texttt{192.168.10.5} is running Nginx version \texttt{1.18.0}, which is outdated and has multiple known vulnerabilities. This could allow an attacker to compromise the server.
\end{itemize}

Immediate remediation of these issues is strongly recommended to reduce the organization's attack surface and mitigate the risk of a security breach. Detailed findings and actionable recommendations are provided in the subsequent sections of this report.

% --- Section 2: Organizational Information ---
\section{Organizational Information}
This section details the organizational data provided for the assessment.

\begin{tabular}{@{}ll}
\toprule
\textbf{Attribute} & \textbf{Value} \\
\midrule
Organization Name & \textbf{New Era} \\
Email Domain & \texttt{NewEra.net} \\
Website Domain & \url{www.NewEra.net} \\
External IP Address & \texttt{219.43.13.59} \\
\bottomrule
\end{tabular}

% --- Section 3: Security Control Review ---
\section{Security Control Review}
The following table summarizes the organization's responses to a security controls questionnaire. Items marked with \ding{55} represent significant gaps in the security framework and are addressed in the risk assessment section.

\begin{tabular}{@{}p{0.7\linewidth}c}
\toprule
\textbf{Control Question} & \textbf{Status} \\
\midrule
Does your organization have an employee acceptable use policy? & \textcolor{green}{\ding{51}} \\
Does your organization do security awareness training for new employees? & \textcolor{green}{\ding{51}} \\
Does your organization do security awareness training for all employees at least once per year? & \textcolor{green}{\ding{51}} \\
Do you require MFA to access sensitive data systems? & \textcolor{green}{\ding{51}} \\
Do you require MFA to access email? & \textcolor{red}{\ding{55}} \\
Do you require MFA to log into computers? & \textcolor{red}{\ding{55}} \\
\bottomrule
\end{tabular}

\subsection{Analysis of Control Gaps}
The primary control gaps identified are related to Multi-Factor Authentication (MFA). The absence of MFA for email and computer logins represents a critical weakness. Email is a primary target for phishing and account takeover attacks, while unprotected computer logins allow for trivial lateral movement within the network if an attacker obtains a user's credentials.

% --- Section 4: Technical Scan Results ---
\section{Technical Scan Results}
A network scan was performed to identify open ports and running services on the target system.

\begin{itemize}
    \item \textbf{Target IP Address:} \texttt{192.168.10.5}
    \item \textbf{Scan Date:} 2025-11-22T10:00:00Z
\end{itemize}

\subsection{Open Ports and Services}
The following table details the services discovered on the target host.

\begin{tabular}{@{}lllll}
\toprule
\textbf{Port} & \textbf{State} & \textbf{Service} & \textbf{Product} & \textbf{Version} \\
\midrule
443/tcp & open & https & nginx & \texttt{1.18.0} \\
\bottomrule
\end{tabular}

\subsection{Technical Findings Analysis}
The scan identified an Nginx web server running on port 443. The detected version, \textbf{\texttt{1.18.0}}, was released in April 2020. This version is significantly outdated and is no longer supported. It is known to be affected by several publicly disclosed vulnerabilities (e.g., CVE-2021-23017), which could be exploited by an attacker to compromise the server. Public-facing web servers are high-value targets, and running outdated software poses a substantial risk.

% --- Section 5: Identified Risks & Assessment ---
\section{Identified Risks \& Assessment}
This section synthesizes findings from the security control review and technical scan into a prioritized list of risks. No pre-existing vulnerabilities were reported.

\begin{tabular}{@{}p{0.1\linewidth} p{0.3\linewidth} p{0.15\linewidth} p{0.35\linewidth}}
\toprule
\textbf{Risk ID} & \textbf{Risk Name} & \textbf{Severity} & \textbf{Description} \\
\midrule
RISK-001 & Lack of MFA for Email Access & \colorbox{critical}{\color{white}\textbf{CRITICAL}} & The absence of MFA on email accounts makes them highly susceptible to account takeover via phishing or credential stuffing. A compromised email account can lead to data breaches and further attacks. \\
\addlinespace
RISK-002 & Outdated Nginx Web Server & \colorbox{high}{\color{white}\textbf{HIGH}} & The public-facing web server runs an unsupported version of Nginx with known vulnerabilities. This could allow for remote code execution, denial of service, or a full system compromise. \\
\addlinespace
RISK-003 & Lack of MFA for Computer Logins & \colorbox{high}{\color{white}\textbf{HIGH}} & Without MFA on endpoint logins, stolen or weak credentials can be used to gain direct access to the internal network, facilitating lateral movement and ransomware deployment. \\
\bottomrule
\end{tabular}

% --- Section 6: Recommendations ---
\section{Recommendations}
The following actions are recommended to mitigate the identified risks and improve the overall security posture of \textbf{New Era}.

\begin{description}
    \item[\textbf{For RISK-001 (Critical):}] \hfill \\
    \textbf{Action:} Implement and enforce MFA for all user accounts on the \texttt{NewEra.net} email system immediately. \\
    \textbf{Justification:} This is the single most effective control to prevent unauthorized access to email, which is a primary vector for business email compromise and data exfiltration.
    
    \item[\textbf{For RISK-002 (High):}] \hfill \\
    \textbf{Action:} Plan and execute an upgrade of the Nginx server on host \texttt{192.168.10.5} to the latest stable version. Implement a patch management process to ensure public-facing software is kept up-to-date. \\
    \textbf{Justification:} Patching known vulnerabilities removes low-hanging fruit for attackers and is a fundamental aspect of security hygiene for internet-facing systems.
    
    \item[\textbf{For RISK-003 (High):}] \hfill \\
    \textbf{Action:} Deploy an MFA solution for all employee computer logins (Windows, macOS, etc.). Prioritize implementation for privileged users (administrators) and executives. \\
    \textbf{Justification:} This adds a critical layer of defense against attacks that rely on stolen credentials, significantly increasing the difficulty for an attacker to move laterally within the network.
\end{description}

\end{document}
```