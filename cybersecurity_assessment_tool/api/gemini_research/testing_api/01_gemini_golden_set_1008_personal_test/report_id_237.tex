```latex
\documentclass[12pt]{article}

% Preamble: Required Packages
\usepackage[margin=1in]{geometry}
\usepackage{pifont} % For checkmarks and crosses
\usepackage{booktabs} % For professional tables
\usepackage{hyperref} % For clickable links
\usepackage{url} % For formatting URLs
\usepackage{seqsplit} % For breaking long strings
\usepackage{graphicx}
\usepackage{xcolor}

% Document Metadata
\title{Cybersecurity Posture Assessment Report}
\author{Cybersecurity Analyst}
\date{\today}

% Hyperref Setup
\hypersetup{
    colorlinks=true,
    linkcolor=blue,
    filecolor=magenta,      
    urlcolor=cyan,
    pdftitle={Cybersecurity Posture Assessment Report},
    pdfpagemode=FullScreen,
}

\begin{document}

\maketitle

\begin{abstract}
This report provides a comprehensive cybersecurity assessment for \textbf{Clear Path}. The analysis is based on the correlation of technical network scan data, an organizational security questionnaire, and a review of pre-existing risk documentation. The assessment identified a critical-risk service exposure and significant gaps in administrative security controls, including the lack of multi-factor authentication for sensitive systems and missing foundational employee policies. Immediate remediation is required to address these findings and reduce the organization's risk profile.
\end{abstract}

\tableofcontents
\newpage

% ===================================================================
% SECTION 1: EXECUTIVE SUMMARY
% ===================================================================
\section{Executive Summary}
This assessment synthesizes data from three sources: a network scan performed on \today, a security controls questionnaire, and a list of current risks. Our analysis reveals a security posture with several areas of critical concern that require immediate attention.

The most severe finding is the discovery of an open service on port 8080 of an internal system (\seqsplit{\texttt{10.5.5.5}}), which identifies itself as \textbf{``TOP SECRET DB''}. This suggests a highly sensitive database or management interface is exposed without adequate protection. This technical finding directly correlates with and elevates the risk of a key administrative gap identified in the questionnaire: the organization does not require Multi-Factor Authentication (MFA) for accessing sensitive data systems.

Furthermore, foundational security policies are absent. The lack of an Acceptable Use Policy (AUP) and security awareness training for new employees creates a permissive environment for insider threats and accidental data breaches.

While a pre-existing risk assessment noted that port 8080 was secure, our current, active scan data contradicts this and must be considered the authoritative finding. The recommendations in this report prioritize securing the exposed service, implementing mandatory MFA, and establishing essential security policies.

% ===================================================================
% SECTION 2: ORGANIZATIONAL INFORMATION
% ===================================================================
\section{Organizational Information}
The following details were provided for the assessment. This information helps establish the context and scope of the review.

\begin{itemize}
    \item \textbf{Organization Name:} Clear Path
    \item \textbf{Email Domain:} \seqsplit{\texttt{ClearPath.net}}
    \item \textbf{Website Domain:} \seqsplit{\texttt{www.ClearPath.net}}
    \item \textbf{External IP Address:} \seqsplit{\texttt{110.95.153.250}}
\end{itemize}

% ===================================================================
% SECTION 3: SECURITY CONTROL REVIEW
% ===================================================================
\section{Security Control Review}
The following table summarizes the organization's responses to a security controls questionnaire. The assessment column highlights gaps where current practices deviate from established security best practices. Answers marked with \ding{55} (No) represent significant weaknesses.

\begin{table}[h!]
\centering
\caption{Security Controls Questionnaire Analysis}
\begin{tabular}{p{0.6\textwidth} c l}
\toprule
\textbf{Control Question} & \textbf{Response} & \textbf{Assessment} \\
\midrule
Do you require MFA to access email? & \ding{51} & Good Practice \\
\addlinespace
Do you require MFA to log into computers? & \ding{51} & Good Practice \\
\addlinespace
\textbf{Do you require MFA to access sensitive data systems?} & \textbf{\color{red}\ding{55}} & \textbf{Critical Gap} \\
\addlinespace
\textbf{Does your organization have an employee acceptable use policy?} & \textbf{\color{red}\ding{55}} & \textbf{High Risk} \\
\addlinespace
\textbf{Does your organization do security awareness training for new employees?} & \textbf{\color{red}\ding{55}} & \textbf{High Risk} \\
\addlinespace
Does your organization do security awareness training for all employees at least once per year? & \ding{51} & Good Practice \\
\bottomrule
\end{tabular}
\end{table}

The review indicates a critical failure in protecting sensitive data systems through MFA. Additionally, the absence of an Acceptable Use Policy and security training for new hires points to systemic weaknesses in security governance and employee onboarding.

% ===================================================================
% SECTION 4: TECHNICAL SCAN RESULTS
% ===================================================================
\section{Technical Scan Results}
An Nmap scan was conducted to identify open ports and services on the target system. The results reveal a critical exposure.

\begin{itemize}
    \item \textbf{Scan Date:} \today
    \item \textbf{Target IP Address:} \seqsplit{\texttt{10.5.5.5}}
    \item \textbf{Target Status:} Up
\end{itemize}

\begin{table}[h!]
\centering
\caption{Open Ports and Services on \seqsplit{\texttt{10.5.5.5}}}
\begin{tabular}{l l p{0.5\textwidth}}
\toprule
\textbf{Port} & \textbf{State} & \textbf{Service Information / Banner} \\
\midrule
8080/tcp & Open & \textbf{HTTP Title: TOP SECRET DB} \\
\bottomrule
\end{tabular}
\end{table}

\textbf{Analysis:} The scan discovered an open service on port 8080. The HTTP title banner explicitly identifies the service as \textbf{``TOP SECRET DB''}. This is a critical finding, indicating that a potentially sensitive, internal database is accessible over the network. This finding directly contradicts a previous risk assessment (\textit{Input\_3\_Current\_Risks\_JSON}) which listed this port as a secure false positive. The new, active scan data supersedes the outdated information and must be treated as a confirmed, high-priority vulnerability.

% ===================================================================
% SECTION 5: CORRELATED RISK ASSESSMENT
% ===================================================================
\section{Correlated Risk Assessment}
The following table synthesizes findings from the technical scan and the security questionnaire into a prioritized list of risks.

\begin{table}[h!]
\centering
\caption{Summary of Identified Risks}
\begin{tabular}{p{0.2\textwidth} p{0.5\textwidth} l}
\toprule
\textbf{Risk ID} & \textbf{Description} & \textbf{Severity} \\
\midrule
RISK-001 & An exposed service on \seqsplit{\texttt{10.5.5.5:8080}} identifies as ``TOP SECRET DB,'' suggesting unauthorized access to highly sensitive data is possible. & \textbf{Critical} \\
\addlinespace
RISK-002 & The organization does not enforce Multi-Factor Authentication (MFA) on sensitive data systems, drastically increasing the risk of compromise via stolen credentials. & \textbf{High} \\
\addlinespace
RISK-003 & The lack of an Acceptable Use Policy (AUP) means there are no formal rules governing employee use of IT assets, increasing insider threat and misconfiguration risks. & \textbf{High} \\
\addlinespace
RISK-004 & New employees do not receive security awareness training, creating an immediate window of vulnerability as they are not informed of security best practices or policies. & \textbf{Medium} \\
\bottomrule
\end{tabular}
\end{table}

% ===================================================================
% SECTION 6: RECOMMENDATIONS
% ===================================================================
\section{Recommendations}
Based on the correlated risk assessment, the following actions are recommended to mitigate the identified vulnerabilities and improve the overall security posture of \textbf{Clear Path}.

\subsection{Immediate Actions (Priority 1: Within 24-48 Hours)}
\begin{enumerate}
    \item \textbf{Investigate and Isolate Exposed Service (RISK-001):}
    \begin{itemize}
        \item Immediately investigate the service running on \seqsplit{\texttt{10.5.5.5:8080}}.
        \item Determine the nature of the data it contains and its business purpose.
        \item Apply firewall rules to restrict all access to this port, allowing only explicitly authorized personnel or systems.
    \end{itemize}
    \item \textbf{Enforce Access Control (RISK-001 \& RISK-002):}
    \begin{itemize}
        \item If the service is a legitimate sensitive system, enforce strong authentication and mandatory MFA for all users immediately.
    \end{itemize}
\end{enumerate}

\subsection{Short-Term Actions (Priority 2: Within 30 Days)}
\begin{enumerate}
    \item \textbf{Implement Mandatory MFA Policy (RISK-002):}
    \begin{itemize}
        \item Develop and enforce a company-wide policy requiring MFA for access to all systems classified as containing sensitive data.
    \end{itemize}
    \item \textbf{Develop and Implement an Acceptable Use Policy (RISK-003):}
    \begin{itemize}
        \item Draft a formal AUP that clearly defines the rules for using company IT resources, including data handling, internet usage, and software installation.
        \item Communicate the policy to all employees and require acknowledgment of receipt.
    \end{itemize}
\end{enumerate}

\subsection{Long-Term Actions (Priority 3: Within 90 Days)}
\begin{enumerate}
    \item \textbf{Integrate Security into Employee Onboarding (RISK-004):}
    \begin{itemize}
        \item Develop a security awareness training module specifically for new hires.
        \item Make completion of this training a mandatory part of the onboarding process before system access is granted.
    \end{itemize}
    \item \textbf{Conduct Comprehensive Security Audit:}
    \begin{itemize}
        \item Commission a full internal and external vulnerability assessment or penetration test to identify any other unknown exposures.
    \end{itemize}
\end{enumerate}

\end{document}
```