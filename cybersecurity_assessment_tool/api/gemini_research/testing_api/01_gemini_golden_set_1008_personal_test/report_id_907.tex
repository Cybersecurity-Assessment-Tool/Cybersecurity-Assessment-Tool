```latex
\documentclass[12pt]{article}

% Preamble: Required Packages
\usepackage[margin=1in]{geometry}
\usepackage{pifont} % For checkmarks and crosses
\usepackage{booktabs} % For professional tables
\usepackage{xcolor} % For coloring text
\usepackage{hyperref} % For clickable links and ToC
\usepackage{url} % For formatting URLs
\usepackage{seqsplit} % For splitting long strings in texttt

% --- Document Setup ---
% Hyperref configuration
\hypersetup{
    colorlinks=true,
    linkcolor=black,
    urlcolor=blue,
    pdftitle={Cybersecurity Posture Assessment Report},
    pdfauthor={Cybersecurity Analysis Division},
    pdfsubject={Security Assessment},
    pdfkeywords={Cybersecurity, Risk, Assessment}
}

% Define colors for risk levels
\definecolor{HighRed}{RGB}{217, 83, 79}
\definecolor{MediumOrange}{RGB}{240, 173, 78}
\definecolor{LowYellow}{RGB}{92, 184, 92}

% --- Document Start ---
\begin{document}

% --- Title Page ---
\title{Cybersecurity Posture Assessment Report \\ \large For: \textbf{Foresight Strategies}}
\author{Cybersecurity Analysis Division}
\date{\today}
\maketitle

% --- Table of Contents ---
\tableofcontents
\newpage

% --- Section 1: Executive Summary ---
\section{Executive Summary}
This report provides a comprehensive assessment of the cybersecurity posture for \textbf{Foresight Strategies}. The analysis is based on a review of organizational security controls, an external network scan, and a summary of known risks.

The assessment identified several areas of concern that require immediate attention. Key findings include critical gaps in security controls, specifically the absence of Multi-Factor Authentication (MFA) for sensitive data systems and the lack of a formal security awareness training program for new employees.

Furthermore, a technical scan of the organization's external infrastructure identified an openly accessible Secure Shell (SSH) service. While this service is common, its exposure without proper hardening presents a significant attack vector.

Overall, the organization has implemented foundational security controls, such as MFA for email access. However, the identified high-risk gaps significantly increase the organization's vulnerability to common cyber threats. This report provides specific, actionable recommendations to mitigate these risks and strengthen the overall security posture.

% --- Section 2: Organizational Information ---
\section{Organizational Information}
The following information was provided for the assessment.

\begin{tabular}{@{}ll}
    \toprule
    \textbf{Attribute} & \textbf{Value} \\
    \midrule
    Organization Name & \textbf{Foresight Strategies} \\
    Email Domain & \texttt{ForesightStrategies.com} \\
    Website Domain & \url{www.ForesightStrategies.com} \\
    Primary External IP & \texttt{54.20.249.235} \\
    \bottomrule
\end{tabular}

% --- Section 3: Security Control Review ---
\section{Security Control Review}
A review of administrative and technical security controls was conducted via a standardized questionnaire. The responses indicate gaps in critical areas of identity and access management and employee security training.

\begin{table}[h!]
\centering
\caption{Security Controls Questionnaire Results}
\begin{tabular}{p{0.7\textwidth} c c}
    \toprule
    \textbf{Control Question} & \textbf{Response} & \textbf{Status} \\
    \midrule
    Do you require MFA to access email? & Yes & \ding{51} \\
    Do you require MFA to log into computers? & Yes & \ding{51} \\
    Do you require MFA to access sensitive data systems? & No & \textcolor{HighRed}{\ding{55}} \\
    Does your organization have an employee acceptable use policy? & Yes & \ding{51} \\
    Does your organization do security awareness training for new employees? & No & \textcolor{HighRed}{\ding{55}} \\
    Does your organization do security awareness training for all employees at least once per year? & Yes & \ding{51} \\
    \bottomrule
\end{tabular}
\end{table}

% --- Section 4: Technical Scan Results ---
\section{Technical Scan Results}
An external network scan was performed to identify accessible services on the organization's public-facing infrastructure.

\begin{itemize}
    \item \textbf{Target IP Address:} \seqsplit{\texttt{2001:db8::1}}
    \item \textbf{Host Status:} Up
\end{itemize}

The scan revealed the following open port:

\begin{table}[h!]
\centering
\caption{Open Port Analysis}
\begin{tabular}{c c c l p{0.4\textwidth}}
    \toprule
    \textbf{Port} & \textbf{Protocol} & \textbf{State} & \textbf{Service} & \textbf{Notes} \\
    \midrule
    22 & TCP & Open & SSH & The Secure Shell service is exposed to the internet. Version information was not obtained, but this service is a common target for brute-force attacks. \\
    \bottomrule
\end{tabular}
\end{table}

% --- Section 5: Risk Assessment ---
\section{Risk Assessment}
The following risks have been identified by correlating the security control gaps, technical findings, and pre-existing vulnerability data. The pre-existing risk list was empty at the time of this assessment.

\begin{table}[h!]
\centering
\caption{Identified Risk Summary}
\begin{tabular}{p{0.1\textwidth} p{0.25\textwidth} p{0.4\textwidth} p{0.1\textwidth}}
    \toprule
    \textbf{Risk ID} & \textbf{Risk Name} & \textbf{Description} & \textbf{Severity} \\
    \midrule
    RISK-001 & MFA Not Enforced on Sensitive Systems & The lack of MFA on systems containing sensitive data exposes the organization to significant risk of unauthorized access and data breach via compromised credentials. & \textcolor{HighRed}{\textbf{High}} \\
    \addlinespace
    RISK-002 & No Onboarding Security Training & New employees are not receiving security awareness training, making them highly susceptible to phishing and social engineering attacks from their first day. & \textcolor{HighRed}{\textbf{High}} \\
    \addlinespace
    RISK-003 & Exposed SSH Service & The SSH service on \seqsplit{\texttt{2001:db8::1}} is publicly accessible. If not configured securely (e.g., weak passwords, outdated version), it can be exploited for unauthorized server access. & \textcolor{MediumOrange}{\textbf{Medium}} \\
    \bottomrule
\end{tabular}
\end{table}

% --- Section 6: Recommendations ---
\section{Recommendations}
To mitigate the identified risks and improve the overall security posture, the following actions are recommended:

\begin{description}
    \item[\textbf{For RISK-001 (High):}] \textbf{Implement MFA on Sensitive Systems.}
    \begin{itemize}
        \item Immediately prioritize and deploy a robust MFA solution for all access to systems identified as storing or processing sensitive or critical data.
        \item Enforce this policy for all users, including administrators and third-party contractors.
    \end{itemize}

    \item[\textbf{For RISK-002 (High):}] \textbf{Establish a New Hire Security Training Program.}
    \begin{itemize}
        \item Develop and implement a mandatory security awareness training module as a standard part of the employee onboarding process.
        \item This training should cover, at a minimum, phishing awareness, acceptable use policies, and password hygiene.
    \end{itemize}

    \item[\textbf{For RISK-003 (Medium):}] \textbf{Harden and Restrict SSH Access.}
    \begin{itemize}
        \item \textbf{Review Necessity:} Determine if public access to this SSH service is business-critical. If not, it should be disabled or placed behind a VPN.
        \item \textbf{Restrict Access:} If required, configure firewall rules to restrict SSH access to only known, trusted IP addresses.
        \item \textbf{Harden Configuration:} Ensure the SSH service is fully patched. Disable password-based authentication in favor of public key cryptography and consider implementing MFA for SSH logins.
    \end{itemize}
\end{description}

\end{document}
```