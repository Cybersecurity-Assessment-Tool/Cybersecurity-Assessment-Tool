```latex
\documentclass[12pt]{article}

% Preamble: Required Packages
\usepackage[margin=1in]{geometry}
\usepackage{pifont} % For check and cross marks
\usepackage{booktabs} % For professional tables
\usepackage{hyperref} % For clickable links
\usepackage{url} % For formatting URLs
\usepackage{seqsplit} % To split long strings in texttt
\usepackage{xcolor} % For coloring text
\usepackage{graphicx} % For potential logos/images
\usepackage{array} % For table column formatting

% --- Document Setup ---
% Hyperref settings
\hypersetup{
    colorlinks=true,
    linkcolor=blue,
    filecolor=magenta,      
    urlcolor=cyan,
    pdftitle={Cybersecurity Risk Assessment Report},
    pdfpagemode=FullScreen,
}

% Define colors for risk levels
\definecolor{criticalrisk}{HTML}{D7263D}
\definecolor{highrisk}{HTML}{F46036}
\definecolor{mediumrisk}{HTML}{F29E4C}
\definecolor{lowrisk}{HTML}{2E86AB}

% Custom command for severity text
\newcommand{\severity}[2]{\colorbox{#1}{\textcolor{white}{\textbf{\phantom{|}#2\phantom{|}}}}}

\begin{document}

% --- Title Page ---
\begin{titlepage}
    \centering
    \vspace*{1cm}
    \Huge\textbf{Cybersecurity Risk Assessment Report}
    \vspace{1.5cm}
    \vfill
    \large
    \textbf{Prepared for:} \\
    Firebrand Media \\
    \vspace{2cm}
    \textbf{Prepared by:} \\
    Cybersecurity Analyst \\
    \vspace{0.5cm}
    \textbf{Date:} \\
    \today
\end{titlepage}

\tableofcontents
\newpage

% --- Executive Summary ---
\section{Executive Summary}
This report provides a comprehensive cybersecurity assessment for Firebrand Media, synthesizing information from organizational questionnaires, automated network scanning, and a review of pre-existing risks. The analysis reveals several critical and high-risk vulnerabilities that require immediate attention.

Key findings include a critically outdated and misconfigured FTP server (\texttt{vsftpd 2.3.4}) on the internal network, which is known to contain a backdoor and is currently permitting anonymous access. Furthermore, significant gaps in administrative controls were identified, including the lack of mandatory multi-factor authentication (MFA) for computer logins, the absence of an employee acceptable use policy, and no security awareness training for new hires. These issues, combined with the pre-existing risk of outdated Windows 7 workstations, create a significant attack surface.

This report outlines prioritized, actionable recommendations to mitigate these risks and strengthen the overall security posture of the organization.

% --- Organizational Information ---
\section{Organizational Information}
The following details were provided for the assessment. This information is used to establish the context and scope of the review.

\begin{tabular}{@{}ll}
    \toprule
    \textbf{Attribute} & \textbf{Value} \\
    \midrule
    Organization Name & Firebrand Media \\
    Email Domain & \texttt{FirebrandMedia.net} \\
    Website Domain & \url{www.FirebrandMedia.net} \\
    External IP Address & \texttt{171.220.93.20} \\
    \bottomrule
\end{tabular}

% --- Security Control Review ---
\section{Security Control Review}
The following table summarizes the organization's responses to a security controls questionnaire. Items marked with \ding{55} indicate a deviation from security best practices and represent a potential gap in the organization's defenses.

\begin{tabular}{p{0.7\linewidth} c c}
    \toprule
    \textbf{Control Question} & \textbf{Response} & \textbf{Status} \\
    \midrule
    Do you require MFA to access email? & \ding{51} & Aligned \\
    Do you require MFA to log into computers? & \ding{55} & \textbf{Gap Identified} \\
    Do you require MFA to access sensitive data systems? & \ding{51} & Aligned \\
    Does your organization have an employee acceptable use policy? & \ding{55} & \textbf{Gap Identified} \\
    Does your organization do security awareness training for new employees? & \ding{55} & \textbf{Gap Identified} \\
    Does your organization do security awareness training for all employees at least once per year? & \ding{51} & Aligned \\
    \bottomrule
\end{tabular}

\subsection*{Analysis of Gaps}
The identified gaps in security controls are significant. The lack of MFA on computer logins exposes the organization to unauthorized access via compromised credentials. The absence of an acceptable use policy and security training for new hires creates an environment where employees may be unaware of safe computing practices, increasing the risk of human error leading to a security incident.

% --- Technical Scan Results ---
\section{Technical Scan Results}
An internal network scan was performed to identify active services and potential vulnerabilities.

\begin{itemize}
    \item \textbf{Target IP Address:} \texttt{10.0.0.15}
    \item \textbf{Scan Summary:} One host was found to be online with one open port exposing a critical service.
\end{itemize}

\begin{tabular}{@{}llllll}
    \toprule
    \textbf{Port} & \textbf{State} & \textbf{Service} & \textbf{Product} & \textbf{Version} & \textbf{Notes} \\
    \midrule
    21/tcp & Open & ftp & vsftpd & 2.3.4 & \begin{tabular}[t]{@{}l}
        \textbf{CRITICAL:} Anonymous FTP login allowed. \\
        \textbf{CRITICAL:} This version is vulnerable to a \\
        backdoor (CVE-2011-2523).
    \end{tabular} \\
    \bottomrule
\end{tabular}

\subsection*{Analysis of Technical Findings}
The scan identified a severely outdated FTP server, \textbf{vsftpd version 2.3.4}, which was released in 2011. This specific version contains a well-documented backdoor that allows an attacker to gain a command shell on the server. Compounding this issue, the server is configured to allow \textbf{anonymous FTP login}, permitting any user on the network to access files without authentication. This configuration represents a direct and immediate threat to the confidentiality and integrity of any data stored on or accessible by this server.

% --- Risk Assessment Summary ---
\section{Risk Assessment Summary}
The following table consolidates findings from the security control review, technical scan, and pre-existing risk register. Each risk has been assigned a severity level based on its potential impact and likelihood of exploitation.

\begin{tabular}{p{0.3\linewidth} p{0.5\linewidth} >{\centering\arraybackslash}p{0.15\linewidth}}
    \toprule
    \textbf{Risk / Vulnerability} & \textbf{Overview} & \textbf{Severity} \\
    \midrule
    \textbf{Vulnerable FTP Server} & The FTP server (vsftpd 2.3.4) on \texttt{10.0.0.15} contains a known backdoor (CVE-2011-2523), allowing for remote code execution. & \severity{criticalrisk}{Critical} \\
    \addlinespace
    \textbf{Anonymous FTP Access} & The FTP server is configured to allow unauthenticated access, exposing files to any user on the network. & \severity{criticalrisk}{Critical} \\
    \addlinespace
    \textbf{Lack of MFA on Workstations} & Employee computers do not require multi-factor authentication for login, increasing the risk of unauthorized access from stolen credentials. & \severity{highrisk}{High} \\
    \addlinespace
    \textbf{Missing Acceptable Use Policy} & The absence of a formal policy creates ambiguity regarding secure employee behavior and limits the organization's ability to enforce security standards. & \severity{highrisk}{High} \\
    \addlinespace
    \textbf{No Onboarding Security Training} & New employees are not provided with security awareness training, making them more susceptible to social engineering and policy violations. & \severity{highrisk}{High} \\
    \addlinespace
    \textbf{Outdated Windows Policy} & Pre-existing risk: Computers are running Windows 7, which is an end-of-life operating system and no longer receives security updates. & \severity{mediumrisk}{Medium} \\
    \bottomrule
\end{tabular}

% --- Recommendations ---
\section{Recommendations}
Based on the risk assessment, the following prioritized actions are recommended to mitigate the identified vulnerabilities and improve the overall security posture of Firebrand Media.

\subsection*{Immediate Priority (Critical Risks)}
\begin{enumerate}
    \item \textbf{Decommission or Upgrade the Vulnerable FTP Server:} The vsftpd server on \texttt{10.0.0.15} must be taken offline immediately. 
    \begin{itemize}
        \item \textbf{Short-term:} If the service is not business-critical, disable it.
        \item \textbf{Long-term:} If FTP functionality is required, replace it with a modern, secure file transfer protocol like SFTP or FTPS, and ensure the underlying software is fully patched.
    \end{itemize}
    \item \textbf{Disable Anonymous Access:} Under no circumstances should anonymous, unauthenticated access to file shares be permitted. All access must be tied to a legitimate, authenticated user account.
\end{enumerate}

\subsection*{High Priority}
\begin{enumerate}
    \setcounter{enumi}{2}
    \item \textbf{Implement MFA for Computer Logins:} Deploy a multi-factor authentication solution for all employee workstations and laptops. This is a critical defense against credential theft.
    \item \textbf{Develop and Implement an Acceptable Use Policy (AUP):} Create a formal AUP that clearly defines the rules and expectations for employee use of company technology and data. All employees must read and acknowledge this policy.
    \item \textbf{Integrate Security Training into Onboarding:} Develop a mandatory security awareness training module for all new employees. This should be completed as part of their onboarding process before they are granted access to sensitive systems.
\end{enumerate}

\subsection*{Medium Priority}
\begin{enumerate}
    \setcounter{enumi}{5}
    \item \textbf{Plan Windows 7 Migration:} Begin planning the upgrade of all Windows 7 workstations to a supported operating system, such as Windows 10 or 11, to ensure they receive critical security patches.
\end{enumerate}

\end{document}
```