```latex
\documentclass[12pt]{article}

% Preamble: Required Packages
\usepackage[margin=1in]{geometry}
\usepackage{pifont} % For checkmarks and crosses
\usepackage{booktabs} % For professional tables
\usepackage{hyperref} % For hyperlinks
\usepackage{url}      % For URL formatting
\usepackage{seqsplit} % To split long strings in tt font

% Document Metadata and Hyperlink Setup
\hypersetup{
    colorlinks=true,
    linkcolor=black,
    urlcolor=blue,
    pdftitle={Cybersecurity Posture Report},
    pdfauthor={Cybersecurity Analyst},
    pdfsubject={Security Assessment},
    pdfkeywords={Security, Analysis, Report}
}

\begin{document}

% --- Title Section ---
\begin{titlepage}
    \centering
    \vspace*{1cm}
    \Huge{\textbf{Cybersecurity Posture Report}}
    \vspace{1.5cm}
    \Large{\textbf{For: Terraform Global}}
    \vspace{2cm}
    \normalsize
    \begin{center}
        \begin{tabular}{ll}
            \textbf{Report Date:} & \today \\
            \textbf{Analysis Period:} & \today \\
            \textbf{Author:} & Cybersecurity Analyst \\
            \textbf{Version:} & 1.0 \\
        \end{tabular}
    \end{center}
    \vfill
    \small{\textit{This report is confidential and intended solely for the use of Terraform Global. Distribution without prior written consent is prohibited.}}
\end{titlepage}

\tableofcontents
\newpage

% --- Section 1: Executive Overview ---
\section{Executive Overview}
This report provides a comprehensive assessment of the cybersecurity posture for \textbf{Terraform Global}. The analysis is based on a correlation of organizational data from a security questionnaire, an external network scan of a key asset, and a review of previously documented risks.

\paragraph{Key Findings:} The assessment reveals a mixed security posture. On one hand, the organization demonstrates a strong network perimeter defense, with the scanned external asset showing no exposed services. This is a significant strength that reduces the external attack surface. Additionally, established security awareness training programs and an acceptable use policy provide a solid foundation for a security-conscious culture.

\paragraph{Critical Risks:} Despite these strengths, two critical security gaps were identified in the organization's access control policies. The absence of Multi-Factor Authentication (MFA) for accessing email and sensitive data systems constitutes a high-impact risk. These gaps expose the organization to significant threats, including business email compromise, credential theft, and data breaches, even with a strong network firewall.

\paragraph{Overall Assessment:} While foundational and perimeter controls are in place, the identified access control deficiencies require immediate attention. The recommendations in this report are prioritized to address these critical risks first, thereby significantly improving the organization's resilience against common and impactful cyber threats.

% --- Section 2: Organizational Information ---
\section{Organizational Information}
The following details were provided for the assessment. This information is used to establish the context and scope of the analysis.

\begin{center}
    \begin{tabular}{ll}
        \toprule
        \textbf{Attribute} & \textbf{Value} \\
        \midrule
        Organization Name & \textbf{Terraform Global} \\
        Email Domain & \texttt{TerraformGlobal.net} \\
        Website Domain & \url{www.TerraformGlobal.net} \\
        External IP Assessed & \texttt{94.197.8.67} \\
        \bottomrule
    \end{tabular}
\end{center}

% --- Section 3: Security Control Review ---
\section{Security Control Review}
A security questionnaire was completed to evaluate the implementation of key administrative and technical controls. The responses are summarized below. Items marked with \ding{55} indicate a potential security gap that requires further investigation or remediation.

\begin{center}
    \begin{tabular}{p{0.6\textwidth} c c}
        \toprule
        \textbf{Security Control Question} & \textbf{Response} & \textbf{Status} \\
        \midrule
        Do you require MFA to access email? & No & \ding{55} \\
        Do you require MFA to log into computers? & Yes & \ding{51} \\
        Do you require MFA to access sensitive data systems? & No & \ding{55} \\
        Does your organization have an employee acceptable use policy? & Yes & \ding{51} \\
        Does your organization do security awareness training for new employees? & Yes & \ding{51} \\
        Does your organization do security awareness training for all employees at least once per year? & Yes & \ding{51} \\
        \bottomrule
    \end{tabular}
\end{center}

% --- Section 4: Technical Scan Results ---
\section{Technical Scan Results}
An external network scan was performed to identify exposed services and potential vulnerabilities on a public-facing asset.

\begin{itemize}
    \item \textbf{Scan Target:} \texttt{192.168.1.100}
    \item \textbf{Scan Date:} \today
\end{itemize}

\subsection{Summary of Findings}
The scan results were positive, indicating a strong perimeter security posture for the targeted host.
\begin{itemize}
    \item \textbf{Host Status:} Up
    \item \textbf{Open Ports:} None Detected
    \item \textbf{Port State:} All 1000 scanned ports were reported as `closed`.
\end{itemize}

\subsection{Analysis}
A host with no open ports is not externally accessible and presents a minimal attack surface. This is an excellent security configuration, likely enforced by a well-configured firewall. This finding significantly mitigates the risk of external network-based attacks against this specific asset.

% --- Section 5: Risk Assessment ---
\section{Risk Assessment}
This section synthesizes findings from the security control review, technical scans, and pre-existing risk documentation. No pre-existing vulnerabilities were documented. The following new risks have been identified based on this assessment.

\begin{center}
    \begin{tabular}{p{0.1\textwidth} p{0.25\textwidth} p{0.45\textwidth} p{0.1\textwidth}}
        \toprule
        \textbf{Risk ID} & \textbf{Risk Name} & \textbf{Overview} & \textbf{Severity} \\
        \midrule
        RISK-001 & Lack of MFA on Email Accounts & Email accounts are not protected by a second factor of authentication. A compromised password could lead to account takeover, business email compromise, and phishing of internal staff. & \textbf{Critical} \\
        \addlinespace
        RISK-002 & Lack of MFA on Sensitive Data Systems & Critical systems containing sensitive data are accessible with only a password. A single compromised credential could result in a major data breach, leading to regulatory fines and reputational damage. & \textbf{Critical} \\
        \bottomrule
    \end{tabular}
\end{center}

% --- Section 6: Recommendations ---
\section{Recommendations}
The following actionable recommendations are provided to mitigate the identified risks. They are prioritized based on severity and potential impact.

\subsection{Priority 1: Remediate Critical Access Control Gaps}
\begin{description}
    \item[REC-001: Enforce MFA on All Email Accounts] \\
    \textbf{Action:} Immediately enable and enforce Multi-Factor Authentication (MFA) for all user accounts accessing the \texttt{TerraformGlobal.net} email system. This is the single most effective control to prevent business email compromise.
    
    \item[REC-002: Enforce MFA on Sensitive Systems] \\
    \textbf{Action:} Conduct an internal review to identify all systems containing sensitive data. Subsequently, enable and enforce MFA for all user and administrator access to these systems. This control is critical for protecting the organization's most valuable data assets.
\end{description}

\subsection{Priority 2: Maintain and Verify Security Posture}
\begin{description}
    \item[REC-003: Implement Continuous Vulnerability Scanning] \\
    \textbf{Action:} While the one-time external scan was clean, the security landscape is dynamic. Implement a program for regular, automated vulnerability scanning of all external and internal assets to ensure new weaknesses are identified and remediated promptly.
\end{description}

\end{document}
```