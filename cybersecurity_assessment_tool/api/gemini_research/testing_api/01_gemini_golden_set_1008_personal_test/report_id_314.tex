```latex
\documentclass[12pt]{article}

% Preamble: Required Packages
\usepackage[margin=1in]{geometry}
\usepackage{pifont} % For checkmarks and crosses
\usepackage{booktabs} % For professional tables
\usepackage{hyperref} % For clickable links and ToC
\usepackage{url} % For formatting URLs
\usepackage{seqsplit} % To split long monospaced strings
\usepackage{xcolor} % For colors

% Hyperref Setup
\hypersetup{
    colorlinks=true,
    linkcolor=blue,
    filecolor=magenta,      
    urlcolor=cyan,
    pdftitle={Cybersecurity Assessment Report},
    pdfpagemode=FullScreen,
}

% Document Title Block
\title{Cybersecurity Assessment Report \\ \large For Swift Current Labs}
\author{Cybersecurity Analyst}
\date{\today}

\begin{document}

\maketitle
\tableofcontents
\newpage

% --- 1. Executive Summary ---
\section{Executive Summary}
This report provides a comprehensive cybersecurity assessment for Swift Current Labs, based on network scans, a security controls questionnaire, and a review of pre-existing risks. The analysis reveals a mixed security posture with several areas of immediate and high-priority concern.

While the organization has implemented strong multi-factor authentication (MFA) controls, critical vulnerabilities were discovered that expose the network to significant risk. A server at \texttt{10.0.0.15} is running a dangerously outdated and misconfigured FTP service (\texttt{vsftpd 2.3.4}) with anonymous access enabled. This specific version is known to be vulnerable to remote code execution (CVE-2011-2523), representing a critical threat.

Furthermore, foundational security policies are lacking. The absence of an Acceptable Use Policy and mandatory annual security awareness training for all employees creates significant governance and human-element risks. These gaps, combined with the pre-existing issue of outdated Windows 7 workstations, indicate that immediate and decisive action is required to mitigate the identified threats and strengthen the overall security posture.

% --- 2. Organizational Information ---
\section{Organizational Information}
The following details were provided for the assessment.

\begin{tabular}{@{}ll}
\toprule
\textbf{Attribute} & \textbf{Value} \\
\midrule
Organization Name & Swift Current Labs \\
Email Domain & \texttt{SwiftCurrentLabs.net} \\
Website Domain & \url{www.SwiftCurrentLabs.net} \\
External IP Address & \texttt{205.96.171.213} \\
\bottomrule
\end{tabular}

% --- 3. Security Control Review ---
\section{Security Control Review}
A review of the organization's security controls was conducted via a questionnaire. The results are summarized below. Answers marked with a red 'X' (\ding{55}) indicate a deviation from security best practices and represent a gap in the organization's defenses.

\begin{table}[h!]
\centering
\begin{tabular}{@{}p{0.8\linewidth}c@{}}
\toprule
\textbf{Control Question} & \textbf{Status} \\
\midrule
Do you require MFA to access email? & \textcolor{green}{\ding{51}} \\
Do you require MFA to log into computers? & \textcolor{green}{\ding{51}} \\
Do you require MFA to access sensitive data systems? & \textcolor{green}{\ding{51}} \\
Does your organization have an employee acceptable use policy? & \textcolor{red}{\ding{55}} \\
Does your organization do security awareness training for new employees? & \textcolor{green}{\ding{51}} \\
Does your organization do security awareness training for all employees at least once per year? & \textcolor{red}{\ding{55}} \\
\bottomrule
\end{tabular}
\caption{Security Controls Questionnaire Results}
\end{table}

\subsection*{Analysis of Control Gaps}
\begin{itemize}
    \item \textbf{Missing Acceptable Use Policy (AUP):} This is a critical governance gap. An AUP is a foundational document that sets clear expectations for employees on how to use company resources securely. Without it, there is no formal basis for enforcing security rules or correcting unsafe behavior.
    \item \textbf{Lack of Annual Security Training:} While new employees receive training, security is an ongoing effort. Threats evolve, and employee knowledge degrades over time. The absence of annual refresher training for all staff significantly increases the risk of successful phishing attacks and other social engineering tactics.
\end{itemize}

% --- 4. Technical Scan Results ---
\section{Technical Scan Results}
An external network scan was performed to identify open ports and exposed services. The following findings were identified on the target system.

\subsection*{Host: \texttt{10.0.0.15}}
A single host was found to be active and responsive to the scan.

\begin{table}[h!]
\centering
\begin{tabular}{@{}lllll@{}}
\toprule
\textbf{Port} & \textbf{State} & \textbf{Service} & \textbf{Product / Version} & \textbf{Notes} \\
\midrule
21/tcp & Open & ftp & vsftpd 2.3.4 & \begin{tabular}[t]{@{}l@{}} \textbf{CRITICAL:} Anonymous FTP login allowed. \\ This version is vulnerable to a backdoor \\ command execution exploit (CVE-2011-2523). \end{tabular} \\
\bottomrule
\end{tabular}
\caption{Open Ports and Services on \texttt{10.0.0.15}}
\end{table}

\subsection*{Analysis of Technical Findings}
The finding on port 21 is of the highest severity. The combination of two major issues creates an easily exploitable attack vector:
\begin{enumerate}
    \item \textbf{Anonymous FTP Access:} This allows any attacker on the network to connect to the FTP server without credentials. They can potentially list, download, or upload files, leading to data exfiltration or the introduction of malware.
    \item \textbf{Vulnerable Software (vsftpd 2.3.4):} This specific version contains a well-known, critical backdoor vulnerability. An attacker can gain a command shell on the server by sending a specific string in the username field, resulting in a complete system compromise.
\end{enumerate}

% --- 5. Consolidated Risk Assessment ---
\section{Consolidated Risk Assessment}
The following table synthesizes findings from the technical scan, controls review, and pre-existing risk register into a prioritized list.

\begin{table}[h!]
\centering
\begin{tabular}{@{}p{0.1\linewidth}p{0.3\linewidth}p{0.15\linewidth}p{0.35\linewidth}@{}}
\toprule
\textbf{Risk ID} & \textbf{Description} & \textbf{Severity} & \textbf{Affected Systems / Impact} \\
\midrule
RISK-001 & Vulnerable FTP service (\texttt{vsftpd 2.3.4}) with anonymous login enabled. & \textbf{Critical} & Server at \texttt{10.0.0.15}. Potential for remote code execution and full system compromise. \\
\addlinespace
RISK-002 & Lack of annual security awareness training for all employees. & \textbf{High} & All employees. Increased susceptibility to phishing, social engineering, and malware infection. \\
\addlinespace
RISK-003 & No formal Acceptable Use Policy (AUP) for employees. & \textbf{High} & Entire organization. Lack of enforceable security standards for employee behavior. \\
\addlinespace
RISK-004 & Workstations are running outdated Windows 7 operating systems. & \textbf{Medium} & End-user workstations. Systems are no longer receiving security updates, making them vulnerable to known exploits. \\
\bottomrule
\end{tabular}
\caption{Summary of Identified Risks}
\end{table}

% --- 6. Recommendations ---
\section{Recommendations}
The following actions are recommended to mitigate the identified risks. They are prioritized based on severity and potential impact.

\subsection*{Immediate Priority}
\begin{itemize}
    \item \textbf{Remediate RISK-001:} Immediately take the server at \texttt{10.0.0.15} offline or firewall port 21. If FTP is business-critical, upgrade \texttt{vsftpd} to the latest stable version and disable anonymous access. If FTP is not required, the service should be disabled and uninstalled permanently. Secure alternatives like SFTP (SSH File Transfer Protocol) should be used for all file transfers.
\end{itemize}

\subsection*{High Priority}
\begin{itemize}
    \item \textbf{Remediate RISK-003:} Develop and implement a formal Acceptable Use Policy (AUP). This policy should be distributed to all employees, who must read and acknowledge it as a condition of using company resources.
    \item \textbf{Remediate RISK-002:} Establish a mandatory annual security awareness training program for all employees. This program should cover current threats such as phishing, ransomware, and proper data handling.
\end{itemize}

\subsection*{Medium Priority}
\begin{itemize}
    \item \textbf{Remediate RISK-004:} Continue with the planned project to upgrade or replace all workstations running Windows 7. This operating system is End-of-Life and poses a persistent risk to the organization.
\end{itemize}

\end{document}
```