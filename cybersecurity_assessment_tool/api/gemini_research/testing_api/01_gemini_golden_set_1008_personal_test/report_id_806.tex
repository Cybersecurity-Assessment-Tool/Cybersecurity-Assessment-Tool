```latex
\documentclass[12pt]{article}

% Preamble: Required Packages
\usepackage[margin=1in]{geometry}
\usepackage{pifont} % For checkmarks and crosses
\usepackage{booktabs} % For professional tables
\usepackage{hyperref} % For clickable links
\usepackage{url}      % For URL formatting
\usepackage{seqsplit} % For splitting long strings in tt font

% Document Metadata
\hypersetup{
    colorlinks=true,
    linkcolor=black,
    urlcolor=blue,
    pdftitle={Cybersecurity Assessment Report},
    pdfauthor={Cybersecurity Analyst},
    pdfsubject={Security Assessment},
    pdfkeywords={Security, Assessment, Report}
}

\title{Cybersecurity Assessment Report \\ \large For: Iron River Finance}
\author{Cybersecurity Analyst}
\date{\today}

\begin{document}

\maketitle
\thispagestyle{empty}
\newpage

\tableofcontents
\newpage

% --- 1. Executive Overview ---
\section{Executive Overview}
This report details the findings of a cybersecurity assessment conducted for Iron River Finance. The analysis is based on a network scan, a review of organizational security controls via a questionnaire, and an evaluation of pre-existing risks.

The assessment reveals several \textbf{critical and high-risk security deficiencies} that expose the organization to significant threats, including data breaches, unauthorized access, and system compromise. Key findings include:
\begin{itemize}
    \item \textbf{Critical Network Vulnerability:} An internal server (\texttt{10.0.0.15}) is running a dangerously outdated and misconfigured FTP service (vsftpd 2.3.4), which is vulnerable to remote code execution and allows anonymous, unauthenticated access.
    \item \textbf{Critical Lack of Access Controls:} Multi-Factor Authentication (MFA) is completely absent across all critical systems, including email, computer logins, and access to sensitive data. This represents a fundamental failure in identity and access management.
    \item \textbf{Systemic Policy Gaps:} The organization lacks a formal employee acceptable use policy and does not conduct any security awareness training. This indicates a low level of security maturity and increases the risk of human error leading to a security incident.
    \item \textbf{Pre-existing Infrastructure Risk:} An existing identified risk regarding outdated Windows 7 workstations remains a concern and is compounded by the newly discovered vulnerabilities.
\end{itemize}

Immediate and decisive action is required to remediate these findings. Recommendations provided in this report are prioritized to address the most critical risks first.

% --- 2. Organizational Information ---
\section{Organizational Information}
The following information was provided for the assessment.

\begin{tabular}{@{}ll}
\toprule
\textbf{Attribute} & \textbf{Value} \\
\midrule
Organization Name & Iron River Finance \\
Email Domain & \texttt{IronRiverFinance.com} \\
Website Domain & \url{www.IronRiverFinance.com} \\
External IP Address & \texttt{138.176.178.113} \\
\bottomrule
\end{tabular}

% --- 3. Security Control Review ---
\section{Security Control Review (Questionnaire)}
A review of administrative and procedural security controls was conducted based on a standardized questionnaire. The responses indicate significant gaps in foundational security practices. A summary of the findings is presented below.

\begin{tabular}{@{}p{0.75\linewidth}c}
\toprule
\textbf{Control Question} & \textbf{Response} \\
\midrule
Do you require MFA to access email? & \ding{55} \\
Do you require MFA to log into computers? & \ding{55} \\
Do you require MFA to access sensitive data systems? & \ding{55} \\
Does your organization have an employee acceptable use policy? & \ding{55} \\
Does your organization do security awareness training for new employees? & \ding{55} \\
Does your organization do security awareness training for all employees at least once per year? & \ding{55} \\
\bottomrule
\end{tabular}

\vspace{1em}
\noindent \textbf{Note:} A checkmark (\ding{51}) indicates the control is in place; a cross (\ding{55}) indicates a control gap. All responses indicate critical control gaps.

% --- 4. Technical Scan Results ---
\section{Technical Scan Results}
A network scan was performed to identify active services and potential vulnerabilities on the specified target system.

\subsection{Target: \texttt{10.0.0.15}}
The scan identified one host as active and revealed the following open port and service.

\begin{tabular}{@{}p{0.1\linewidth} p{0.1\linewidth} p{0.3\linewidth} p{0.4\linewidth}@{}}
\toprule
\textbf{Port} & \textbf{State} & \textbf{Service / Version} & \textbf{Finding Details} \\
\midrule
21/tcp & Open & FTP / vsftpd 2.3.4 & \textbf{Critical Finding.} This version is from 2011 and is associated with a known backdoor vulnerability (CVE-2011-2523), allowing for remote code execution.
\\
\addlinespace
& & & \textbf{Critical Misconfiguration.} The scan confirmed that \textbf{Anonymous FTP login is allowed}. This permits any unauthenticated user on the network to access files on the server, posing a severe data exfiltration risk. \\
\bottomrule
\end{tabular}

% --- 5. Risk Assessment ---
\section{Risk Assessment}
The following table synthesizes findings from the technical scan, security control review, and pre-existing risk data. Risks are prioritized based on their potential impact and likelihood of exploitation.

\begin{tabular}{@{}p{0.1\linewidth} p{0.4\linewidth} p{0.15\linewidth} p{0.25\linewidth}@{}}
\toprule
\textbf{Risk ID} & \textbf{Description} & \textbf{Severity} & \textbf{Affected Systems} \\
\midrule
RISK-001 & A publicly known backdoor vulnerability (CVE-2011-2523) and anonymous access are present on the FTP server. & \textbf{Critical} & Server at \texttt{10.0.0.15} \\
\addlinespace
RISK-002 & Lack of Multi-Factor Authentication (MFA) for accessing sensitive data systems. & \textbf{Critical} & All sensitive data systems, Core business applications \\
\addlinespace
RISK-003 & Lack of MFA for company email and computer logins. & High & All employee workstations, Email infrastructure \\
\addlinespace
RISK-004 & No security awareness training program for employees. & High & All employees \\
\addlinespace
RISK-005 & Workstations are running the unsupported Windows 7 operating system. & Medium & All workstations \\
\addlinespace
RISK-006 & No formal Acceptable Use Policy (AUP) for employees. & Medium & Organizational Governance \\
\bottomrule
\end{tabular}

% --- 6. Recommendations ---
\section{Recommendations}
The following actions are recommended to mitigate the identified risks. They are prioritized to address the most severe threats first.

\subsection{Immediate Actions (Critical Priority)}
\begin{enumerate}
    \item \textbf{Remediate Vulnerable FTP Server (RISK-001):}
    \begin{itemize}
        \item Immediately take the FTP service on \texttt{10.0.0.15} offline.
        \item If file transfer is a business requirement, decommission the FTP service permanently and replace it with a secure alternative such as SFTP (SSH File Transfer Protocol).
        \item If the service cannot be taken offline, immediately disable anonymous access and apply all available security patches to the vsftpd service. A full upgrade is strongly preferred.
    \end{itemize}
    \item \textbf{Implement MFA for Sensitive Systems (RISK-002):}
    \begin{itemize}
        \item Immediately begin a project to deploy MFA on all systems that house or provide access to sensitive company and client data. This is the single most effective control to prevent unauthorized access.
    \end{itemize}
\end{enumerate}

\subsection{High Priority Actions}
\begin{enumerate}
    \item \textbf{Deploy MFA for Email and Endpoints (RISK-003):}
    \begin{itemize}
        \item Enforce MFA for all user accounts on the company email platform (e.g., Office 365, Google Workspace).
        \item Implement MFA for logging into all company-managed workstations and laptops.
    \end{itemize}
    \item \textbf{Establish Security Awareness Training (RISK-004):}
    \begin{itemize}
        \item Procure and implement a security awareness training program for all employees.
        \item Ensure all new hires complete the training as part of their onboarding process.
        \item Conduct annual refresher training for all staff to ensure continued awareness of threats like phishing and social engineering.
    \end{itemize}
\end{enumerate}

\subsection{Medium Priority Actions}
\begin{enumerate}
    \item \textbf{Upgrade Outdated Operating Systems (RISK-005):}
    \begin{itemize}
        \item Continue and prioritize the plan to upgrade all workstations from Windows 7, which is no longer supported with security updates, to a modern, supported operating system like Windows 10 or 11.
    \end{itemize}
    \item \textbf{Develop an Acceptable Use Policy (RISK-006):}
    \begin{itemize}
        \item Draft, approve, and distribute an Acceptable Use Policy (AUP) that clearly defines the rules for using company IT assets. Ensure all employees read and acknowledge the policy.
    \end{itemize}
\end{enumerate}

\end{document}
```