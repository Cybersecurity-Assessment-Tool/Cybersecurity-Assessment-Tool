```latex
\documentclass[12pt]{article}

% ----------------------------------------------------------------------
% PREAMBLE
% ----------------------------------------------------------------------
\usepackage[margin=1in]{geometry}
\usepackage{pifont} % For checkmarks and crosses
\usepackage{booktabs} % For professional tables
\usepackage{hyperref} % For clickable links
\usepackage{url} % For URL formatting
\usepackage{seqsplit} % To split long strings in tt font
\usepackage{graphicx}
\usepackage{fancyhdr}
\usepackage{xcolor}
\usepackage{titlesec}

% --- Customization ---
\definecolor{darkblue}{rgb}{0.0, 0.0, 0.55}
\hypersetup{
    colorlinks=true,
    linkcolor=darkblue,
    filecolor=darkblue,      
    urlcolor=darkblue,
    citecolor=darkblue,
}

\pagestyle{fancy}
\fancyhf{} % Clear all header and footer fields
\fancyhead[L]{Cybersecurity Assessment Report}
\fancyhead[R]{\textbf{Verve \& Vigor}}
\fancyfoot[C]{\thepage}
\renewcommand{\headrulewidth}{0.4pt}
\renewcommand{\footrulewidth}{0.4pt}

\titleformat{\section}
  {\normalfont\Large\bfseries\color{darkblue}}
  {\thesection}{1em}{}

% ----------------------------------------------------------------------
% DOCUMENT START
% ----------------------------------------------------------------------
\begin{document}

\title{
    \vspace{1cm}
    \textbf{Cybersecurity Posture and Risk Assessment Report} \\
    \large For: \textbf{Verve \& Vigor}
    \vspace{0.5cm}
}
\author{Cybersecurity Analysis Division}
\date{\today}

\maketitle
\thispagestyle{empty}
\newpage

\tableofcontents
\newpage

% ----------------------------------------------------------------------
% SECTION 1: EXECUTIVE SUMMARY
% ----------------------------------------------------------------------
\section{Executive Summary}

This report presents a cybersecurity assessment for \textbf{Verve \& Vigor}, synthesizing data from a network scan, a security controls questionnaire, and a review of pre-existing risk documentation. The analysis reveals a mixed security posture with several critical and high-risk vulnerabilities that require immediate attention.

While the organization has implemented some foundational controls, such as Multi-Factor Authentication (MFA) for computer logins and security training for new hires, significant gaps exist. The most pressing issues identified are:

\begin{itemize}
    \item \textbf{Critical Lack of MFA:} The absence of mandatory MFA for accessing email and sensitive data systems exposes the organization to significant risks of account compromise and data breach.
    \item \textbf{Inadequate Security Training:} The lack of a mandatory annual security awareness training program for all employees increases susceptibility to phishing, social engineering, and other human-centric attacks.
    \item \textbf{Validated Critical Vulnerability:} A network scan confirmed a pre-identified critical risk, "Localhost Exposed," associated with an open SSH port on \texttt{127.0.0.1}. This finding carries a CVSS score of 10.0 and must be remediated immediately.
\end{itemize}

This report provides a detailed breakdown of these findings and offers actionable recommendations to mitigate the identified risks and strengthen the overall security posture of \textbf{Verve \& Vigor}.

% ----------------------------------------------------------------------
% SECTION 2: ORGANIZATIONAL INFORMATION
% ----------------------------------------------------------------------
\section{Organizational Information}

The following details were provided for the assessment. This information is used to establish the context and scope of the review.

\begin{tabular}{@{}ll}
\toprule
\textbf{Attribute} & \textbf{Value} \\
\midrule
Organization Name & \textbf{Verve \& Vigor} \\
Email Domain & \seqsplit{\texttt{VerveVigor.net}} \\
Website Domain & \seqsplit{\url{www.VerveVigor.net}} \\
External IP Address & \seqsplit{\texttt{133.129.109.3}} \\
\bottomrule
\end{tabular}

% ----------------------------------------------------------------------
% SECTION 3: SECURITY CONTROL REVIEW
% ----------------------------------------------------------------------
\section{Security Control Review}

A review of the organization's security controls was conducted via a questionnaire. The responses indicate foundational policies are in place, but critical gaps exist in access control and ongoing employee education. A "No" response (\ding{55}) highlights a significant weakness.

\begin{table}[h!]
\centering
\caption{Security Controls Questionnaire Results}
\begin{tabular}{@{}lc}
\toprule
\textbf{Control Question} & \textbf{Response} \\
\midrule
Do you require MFA to access email? & \textcolor{red}{\ding{55}} \\
Do you require MFA to log into computers? & \textcolor{green}{\ding{51}} \\
Do you require MFA to access sensitive data systems? & \textcolor{red}{\ding{55}} \\
Does your organization have an employee acceptable use policy? & \textcolor{green}{\ding{51}} \\
Does your organization do security awareness training for new employees? & \textcolor{green}{\ding{51}} \\
Does your organization do security awareness training for all employees at least once per year? & \textcolor{red}{\ding{55}} \\
\bottomrule
\end{tabular}
\end{table}

\subsection*{Analysis of Gaps}
The questionnaire reveals three primary areas of concern:
\begin{itemize}
    \item \textbf{MFA for Email:} Email is a primary vector for phishing attacks and a gateway to other systems. The lack of MFA is a high-risk finding.
    \item \textbf{MFA for Sensitive Data:} Failure to protect sensitive data systems with MFA significantly increases the risk of a data breach resulting from compromised credentials.
    \item \textbf{Annual Security Training:} Threats evolve constantly. Without annual refresher training, employees are more likely to fall victim to new attack techniques, negating the impact of initial onboarding training.
\end{itemize}

% ----------------------------------------------------------------------
% SECTION 4: TECHNICAL SCAN RESULTS
% ----------------------------------------------------------------------
\section{Technical Scan Results}
A network scan was performed to identify open ports and services on the specified target. The results validate a known critical risk.

\begin{itemize}
    \item \textbf{Target IP Address:} \texttt{127.0.0.1}
    \item \textbf{Scan Tool:} Nmap
    \item \textbf{Host Status:} Up
\end{itemize}

\begin{table}[h!]
\centering
\caption{Open Ports Detected on \texttt{127.0.0.1}}
\begin{tabular}{@{}llll}
\toprule
\textbf{Port} & \textbf{State} & \textbf{Service (Inferred)} & \textbf{Notes} \\
\midrule
22/tcp & open & SSH & Secure Shell. Used for remote administration. \\
\bottomrule
\end{tabular}
\end{table}

\subsection*{Analysis of Findings}
The scan confirms that port 22 (SSH) is open on the localhost interface. While localhost is typically not exposed externally, this finding directly correlates with the pre-existing risk "Localhost Exposed" (CVSS 10.0). This indicates that a misconfiguration may be exposing this internal service, or that the service itself is inherently vulnerable. An open SSH port, if improperly configured or running an outdated version, can be a vector for unauthorized access.

% ----------------------------------------------------------------------
% SECTION 5: CONSOLIDATED RISK ASSESSMENT
% ----------------------------------------------------------------------
\section{Consolidated Risk Assessment}
The following table synthesizes findings from the security questionnaire, the technical scan, and pre-existing risk data into a consolidated list of identified risks.

\begin{table}[h!]
\centering
\caption{Summary of Identified Risks}
\begin{tabular}{@{}p{0.15\linewidth} p{0.55\linewidth} p{0.2\linewidth}}
\toprule
\textbf{Risk Name} & \textbf{Description} & \textbf{Severity} \\
\midrule
\textbf{Localhost Exposed} & The technical scan confirmed an open SSH port on \texttt{127.0.0.1}, validating a known vulnerability. This represents a severe misconfiguration or software flaw. & \textbf{Critical (10.0)} \\
\addlinespace
\textbf{Lack of MFA on Email} & The absence of MFA on email accounts greatly increases the risk of business email compromise, phishing success, and subsequent lateral movement. & \textbf{High} \\
\addlinespace
\textbf{Lack of MFA on Sensitive Systems} & Sensitive data systems are not protected by MFA, making them prime targets for credential-based attacks and data exfiltration. & \textbf{High} \\
\addlinespace
\textbf{Inadequate Security Training} & Without mandatory annual training, employees' ability to recognize and respond to evolving cyber threats diminishes over time. & \textbf{Medium} \\
\bottomrule
\end{tabular}
\end{table}

% ----------------------------------------------------------------------
% SECTION 6: RECOMMENDATIONS
% ----------------------------------------------------------------------
\section{Recommendations}
The following actions are recommended to mitigate the identified risks. They are prioritized based on severity.

\subsection*{Immediate Priority (Critical)}
\begin{enumerate}
    \item \textbf{Remediate Localhost Exposure (Risk: Localhost Exposed):}
    \begin{itemize}
        \item Immediately investigate the service running on \texttt{127.0.0.1:22}.
        \item If the service is not essential for business operations, disable it.
        \item If the service is required, ensure it is fully patched, configured according to security best practices, and firewalled to prevent any unintended external access.
    \end{itemize}
\end{enumerate}

\subsection*{High Priority}
\begin{enumerate}
    \setcounter{enumi}{1}
    \item \textbf{Implement Comprehensive MFA (Risks: Lack of MFA on Email \& Sensitive Systems):}
    \begin{itemize}
        \item Develop a project plan to deploy and enforce MFA for all users on email platforms (e.g., O365, Google Workspace) and all systems identified as containing sensitive data.
        \item Set a mandatory enforcement deadline, after which non-MFA logins are disabled.
    \end{itemize}
\end{enumerate}

\subsection*{Medium Priority}
\begin{enumerate}
    \setcounter{enumi}{2}
    \item \textbf{Establish Annual Security Training (Risk: Inadequate Security Training):}
    \begin{itemize}
        \item Procure or develop a security awareness training program that is mandatory for all employees to complete annually.
        \item The training should cover current threats such as phishing, ransomware, and social engineering.
        \item Track completion rates to ensure 100\% compliance.
    \end{itemize}
\end{enumerate}

% ----------------------------------------------------------------------
% DOCUMENT END
% ----------------------------------------------------------------------
\end{document}
```