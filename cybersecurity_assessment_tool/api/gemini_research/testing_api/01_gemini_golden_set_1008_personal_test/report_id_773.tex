```latex
\documentclass[12pt]{article}

% Preamble: Required Packages
\usepackage[margin=1in]{geometry}
\usepackage{pifont} % For \ding
\usepackage{booktabs} % For professional tables (\toprule, \midrule, \bottomrule)
\usepackage[hidelinks]{hyperref}
\usepackage{url}
\usepackage{seqsplit} % For breaking long strings
\usepackage[T1]{fontenc}
\usepackage{graphicx}
\usepackage{array}

% Define custom colors for severity
\usepackage[table]{xcolor}
\definecolor{sevCritical}{HTML}{990000}
\definecolor{sevHigh}{HTML}{DD4B39}
\definecolor{sevMedium}{HTML}{F4B400}
\definecolor{sevLow}{HTML}{4285F4}

% Document Information
\title{
    \vspace{-1.5cm}
    \includegraphics[width=0.3\textwidth]{example-image-a} \\ % Placeholder for a logo
    \vspace{0.5cm}
    \textbf{Cybersecurity Posture Assessment Report} \\
    \large \vspace{0.2cm}
    Prepared for: Golden Gate Gaming
}
\author{Cybersecurity Analysis Division}
\date{\today}

\begin{document}

\maketitle
\thispagestyle{empty}
\newpage

\tableofcontents
\newpage

% ==============================================================================
% 1. Executive Summary
% ==============================================================================
\section*{1. Executive Summary}

This report details the findings of a cybersecurity posture assessment conducted for \textbf{Golden Gate Gaming}. The assessment synthesized data from a network vulnerability scan, a review of existing risks, and an organizational security controls questionnaire.

The analysis identified several critical and high-risk vulnerabilities that require immediate attention. Key findings include the discovery of an additional system with an exposed Remote Desktop Protocol (RDP) port, compounding a previously known risk. This technical vulnerability is significantly amplified by critical gaps in foundational security controls, most notably the lack of Multi-Factor Authentication (MFA) for email and computer access.

Furthermore, the absence of an employee acceptable use policy and a mandatory annual security awareness training program indicates systemic weaknesses in the organization's security culture and governance. These gaps create an environment where both technical and human-related security incidents are more likely to occur and have a greater impact.

Immediate remediation of the exposed services and the implementation of MFA are paramount to reducing the organization's attack surface and mitigating the risk of a security breach.

% ==============================================================================
% 2. Organizational Information
% ==============================================================================
\section*{2. Organizational Information}

The following information was provided by the client and used as a baseline for this assessment.

\begin{table}[h!]
\centering
\begin{tabular}{@{}ll@{}}
\toprule
\textbf{Attribute} & \textbf{Value} \\ \midrule
Organization Name & Golden Gate Gaming \\
Email Domain & \texttt{GoldenGateGaming.org} \\
Website Domain & \url{www.GoldenGateGaming.org} \\
External IP Address & \texttt{28.186.59.131} \\ \bottomrule
\end{tabular}
\caption{Client Organizational Details}
\end{table}

% ==============================================================================
% 3. Security Control Review
% ==============================================================================
\section*{3. Security Control Review}

A review of internal security controls was conducted via a questionnaire. The responses reveal significant gaps in user access controls and security governance, which are detailed below.

\begin{table}[h!]
\centering
\renewcommand{\arraystretch}{1.3}
\begin{tabular}{>{\raggedright\arraybackslash}p{8cm}cc}
\toprule
\textbf{Control Question} & \textbf{Response} & \textbf{Assessment} \\ \midrule
Do you require MFA to access email? & No & \ding{55} \\
Do you require MFA to log into computers? & No & \ding{55} \\
Do you require MFA to access sensitive data systems? & Yes & \ding{51} \\
Does your organization have an employee acceptable use policy? & No & \ding{55} \\
Does your organization do security awareness training for new employees? & Yes & \ding{51} \\
Does your organization do security awareness training for all employees at least once per year? & No & \ding{55} \\ \bottomrule
\end{tabular}
\caption{Security Controls Questionnaire Results}
\end{table}

\subsection*{Analysis of Control Gaps}
\begin{itemize}
    \item \textbf{Lack of MFA:} The absence of MFA for email and computer logins is a critical vulnerability. This significantly increases the risk of unauthorized access via credential theft, phishing, or brute-force attacks.
    \item \textbf{Policy Gaps:} Lacking an acceptable use policy means there are no formal guidelines for employees on how to use company assets securely.
    \item \textbf{Training Deficiencies:} While new employees receive training, the lack of an annual refresher for all staff means that security knowledge degrades over time, making employees more susceptible to social engineering attacks.
\end{itemize}

% ==============================================================================
% 4. Technical Scan Results
% ==============================================================================
\section*{4. Technical Scan Results}

A network scan was performed to identify open ports and exposed services on the target system.

\subsection*{Scan Details}
\begin{itemize}
    \item \textbf{Target IP:} \texttt{10.10.10.51}
    \item \textbf{Target Status:} Up
\end{itemize}

\subsection*{Open Ports Discovered}
The following table details the open ports and services discovered on the target host.

\begin{table}[h!]
\centering
\begin{tabular}{@{}llll@{}}
\toprule
\textbf{Port} & \textbf{State} & \textbf{Service Name} & \textbf{Notes} \\ \midrule
3389/tcp & open & \texttt{ms-wbt-server} & Microsoft Remote Desktop Protocol (RDP) \\ \bottomrule
\end{tabular}
\caption{Open Ports on \texttt{10.10.10.51}}
\end{table}

\subsection*{Technical Analysis}
The identification of an open RDP port on host \texttt{10.10.10.51} is a significant finding. RDP is a primary target for attackers seeking to gain initial access to a network. This finding is especially concerning as the list of pre-existing risks already identified RDP exposure on a different host (\texttt{10.10.10.50}), indicating a systemic issue with network segmentation and service exposure. When combined with the lack of MFA for computer logins, this creates a direct path for an attacker to compromise internal systems.

% ==============================================================================
% 5. Consolidated Risk Assessment
% ==============================================================================
\section*{5. Consolidated Risk Assessment}

The following table synthesizes findings from the security control review, technical scan, and pre-existing risk register into a prioritized list of security risks.

\begin{table}[h!]
\centering
\renewcommand{\arraystretch}{1.4}
\begin{tabular}{@{}lp{8cm}l@{}}
\toprule
\textbf{Risk Name} & \textbf{Description} & \textbf{Severity} \\ \midrule
Lack of MFA for Critical Systems & The absence of MFA on email and computer logins allows for account takeover if credentials are compromised. This is a foundational security control failure. & \colorbox{sevCritical}{\color{white}\textbf{Critical}} \\
\addlinespace
Widespread RDP Exposure & RDP is exposed on at least two internal hosts (\texttt{10.10.10.50}, \texttt{10.10.10.51}). This service is a frequent target for ransomware and other attacks. & \colorbox{sevCritical}{\color{white}\textbf{Critical}} \\
\addlinespace
Inadequate Security Training Program & Lack of annual security training for all employees increases susceptibility to phishing and social engineering, which are the root causes of most breaches. & \colorbox{sevHigh}{\color{white}\textbf{High}} \\
\addlinespace
No Acceptable Use Policy (AUP) & Without a formal AUP, there is no enforceable standard for employee behavior regarding company IT assets, leading to inconsistent and risky practices. & \colorbox{sevHigh}{\color{white}\textbf{High}} \\ \bottomrule
\end{tabular}
\caption{Summary of Identified Risks}
\end{table}

% ==============================================================================
% 6. Recommendations
% ==============================================================================
\section*{6. Recommendations}

Based on the risk assessment, the following actions are recommended to improve the security posture of \textbf{Golden Gate Gaming}. Recommendations are prioritized by severity.

\subsection*{Immediate Priority (Critical Risks)}
\begin{enumerate}
    \item \textbf{Remediate RDP Exposure:} Immediately disable or firewall access to RDP (port 3389) on hosts \texttt{10.10.10.50} and \texttt{10.10.10.51}. If remote access is required, it must be placed behind a secure gateway, such as a VPN with MFA.
    \item \textbf{Implement MFA Everywhere:} Enforce MFA for all users on all critical systems, starting immediately with email (e.g., Office 365, G Suite) and computer logins (e.g., Windows Hello, Duo).
\end{enumerate}

\subsection*{High Priority}
\begin{enumerate}
    \setcounter{enumi}{2}
    \item \textbf{Establish an Annual Security Training Program:} Develop and mandate an annual security awareness training program for all employees. The training should cover phishing, password hygiene, social engineering, and the organization's security policies.
    \item \textbf{Develop and Implement an Acceptable Use Policy (AUP):} Create a formal AUP that clearly defines the rules for using company technology and data. All employees must read and acknowledge this policy as a condition of their employment.
\end{enumerate}

\end{document}
```