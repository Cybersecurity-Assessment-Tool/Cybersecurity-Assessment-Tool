```latex
\documentclass[12pt]{article}

% Preamble: Required Packages
\usepackage[margin=1in]{geometry}
\usepackage{pifont} % For \ding symbols (checkmarks/crosses)
\usepackage{booktabs} % For professional-looking tables
\usepackage{hyperref} % For clickable links and references
\usepackage{url}      % For properly formatting URLs
\usepackage{seqsplit} % To prevent long strings from overflowing
\usepackage{graphicx} % For potential logos
\usepackage{xcolor}   % For custom colors

% Document Setup
\hypersetup{
    colorlinks=true,
    linkcolor=blue,
    filecolor=magenta,
    urlcolor=cyan,
    pdftitle={Cybersecurity Posture Assessment Report},
    pdfauthor={Cybersecurity Analysis Division},
}

% Define custom commands for status indicators
\newcommand{\yes}{\ding{51}}
\newcommand{\no}{\ding{55}}

% --- Document Start ---
\begin{document}

% --- Title Page ---
\begin{titlepage}
    \centering
    \vspace*{1cm}
    \Huge\textbf{Cybersecurity Posture Assessment Report}
    \vspace{1.5cm}
    \Large
    \textbf{Prepared for:}\\
    Evergreen Alliance
    \vspace{2cm}
    \large
    \textbf{Date of Report:}\\
    \today
    \vfill
    \large
    \textbf{Analysis Division}\\
    Cybersecurity Threat Intelligence
\end{titlepage}

\tableofcontents
\newpage

% --- Section 1: Executive Summary ---
\section*{Executive Summary}

This report provides a comprehensive assessment of the cybersecurity posture for \textbf{Evergreen Alliance}, based on technical network scans, a review of organizational security controls, and an analysis of pre-existing risks. The assessment was conducted to identify vulnerabilities, evaluate current security practices, and provide actionable recommendations to mitigate identified threats.

The analysis revealed several \textbf{critical and high-risk vulnerabilities} that require immediate attention. Key findings include:
\begin{itemize}
    \item \textbf{Exposed Vulnerable Service:} A public-facing FTP server running \texttt{vsftpd 2.3.4} was identified. This specific version is known to contain a critical backdoor vulnerability (CVE-2011-2523), which could allow an attacker to gain remote command execution. The server is further misconfigured to allow anonymous logins, significantly increasing its risk profile.
    \item \textbf{Systemic Lack of Multi-Factor Authentication (MFA):} The organization does not enforce MFA for accessing email, logging into computers, or accessing sensitive data systems. This represents a critical gap in access control, leaving accounts vulnerable to compromise via credential theft or brute-force attacks.
    \item \textbf{Deficient Administrative Controls:} The absence of a formal employee acceptable use policy and a recurring annual security awareness training program indicates significant gaps in administrative security controls. This increases the likelihood of human error leading to a security incident.
    \item \textbf{Pre-existing Infrastructure Risks:} The organization is aware of an existing risk related to outdated Windows 7 workstations, which are no longer supported and do not receive security updates.
\end{itemize}

Immediate remediation of the vulnerable FTP server and implementation of MFA are the highest priorities. Recommendations are detailed in Section 5 of this report.

% --- Section 2: Organizational Information ---
\section*{Organizational Information}
This section details the organizational data provided for this assessment.

\begin{tabular}{@{}ll}
\toprule
\textbf{Attribute} & \textbf{Value} \\
\midrule
Organization Name & Evergreen Alliance \\
Email Domain & \texttt{EvergreenAlliance.org} \\
Website Domain & \url{www.EvergreenAlliance.org} \\
External IP Address & \texttt{64.142.77.126} \\
\bottomrule
\end{tabular}

% --- Section 3: Security Control Review ---
\section*{Security Control Review}
The following table summarizes the organization's responses to a security controls questionnaire. Answers marked with \no\ indicate a deviation from security best practices and represent a potential risk.

\begin{tabular}{@{}p{0.8\linewidth}c}
\toprule
\textbf{Control Question} & \textbf{Status} \\
\midrule
Do you require MFA to access email? & \no \\
Do you require MFA to log into computers? & \no \\
Do you require MFA to access sensitive data systems? & \no \\
Does your organization have an employee acceptable use policy? & \no \\
Does your organization do security awareness training for new employees? & \yes \\
Does your organization do security awareness training for all employees at least once per year? & \no \\
\bottomrule
\end{tabular}

\subsection*{Analysis of Control Gaps}
The lack of MFA across all critical systems (email, endpoints, sensitive data) is a \textbf{critical control gap}. This single weakness dramatically increases the risk of unauthorized access. Furthermore, the absence of an acceptable use policy and annual security training creates an environment where employees are more likely to engage in risky behavior, unaware of their security responsibilities.

% --- Section 4: Technical Scan Results ---
\section*{Technical Scan Results}
An external network scan was performed on the target system to identify open ports and exposed services.

\begin{itemize}
    \item \textbf{Target IP Address:} \texttt{10.0.0.15}
    \item \textbf{Scan Date:} Data provided on \today
\end{itemize}

\begin{tabular}{@{}lllll}
\toprule
\textbf{Port} & \textbf{State} & \textbf{Service} & \textbf{Version} & \textbf{Notes} \\
\midrule
21/tcp & Open & FTP & vsftpd 2.3.4 & \textbf{Critical:} Anonymous login allowed. \\
 & & & & This version is known to be \\
 & & & & vulnerable to a backdoor \\
 & & & & (CVE-2011-2523). \\
\bottomrule
\end{tabular}

\subsection*{Technical Findings Analysis}
The scan identified a critically vulnerable File Transfer Protocol (FTP) server. The service version, \textbf{\texttt{vsftpd 2.3.4}}, contains a well-documented backdoor that was inserted into the source code. If exploited, this vulnerability allows an attacker to execute arbitrary commands on the server with root-level privileges. The misconfiguration allowing anonymous FTP login makes the server an easy target for attackers to discover and exploit.

% --- Section 5: Consolidated Risk Assessment ---
\section*{Consolidated Risk Assessment}
This section synthesizes findings from the security control review, technical scan, and pre-existing risk data into a prioritized list of organizational risks.

\begin{tabular}{@{}p{0.25\linewidth}p{0.5\linewidth}p{0.15\linewidth}}
\toprule
\textbf{Risk Name} & \textbf{Overview} & \textbf{Severity} \\
\midrule
\textbf{Vulnerable FTP Service} & An internet-facing FTP server is running \texttt{vsftpd 2.3.4}, which contains a critical backdoor vulnerability (CVE-2011-2523). The service also permits anonymous login. & \textbf{Critical} \\
\addlinespace
\textbf{No Multi-Factor Authentication} & The lack of MFA for email, computer logins, and sensitive systems exposes the organization to account takeover, data breaches, and ransomware attacks. & \textbf{Critical} \\
\addlinespace
\textbf{Insufficient Policies \& Training} & The absence of an Acceptable Use Policy and annual security training leads to a weak security culture and increases the risk of insider threats and human error. & \textbf{High} \\
\addlinespace
\textbf{Outdated Windows Policy} & Workstations are running Windows 7, an end-of-life operating system that no longer receives security updates, leaving them vulnerable to known exploits. & \textbf{Medium} \\
\bottomrule
\end{tabular}

% --- Section 6: Recommendations ---
\section*{Recommendations}
The following prioritized recommendations are provided to address the identified risks and improve the overall security posture of Evergreen Alliance.

\subsection*{Immediate Priority (0-7 Days)}
\begin{enumerate}
    \item \textbf{Remediate Vulnerable FTP Server:} Immediately take the FTP server offline.
        \begin{itemize}
            \item If the service is not required for business operations, decommission it permanently.
            \item If it is required, upgrade the \texttt{vsftpd} software to the latest stable version and reconfigure it to disable anonymous access. Enforce strong authentication and use SFTP (SSH File Transfer Protocol) instead of FTP.
        \end{itemize}
\end{enumerate}

\subsection*{High Priority (1-3 Months)}
\begin{enumerate}
    \setcounter{enumi}{1}
    \item \textbf{Implement Multi-Factor Authentication (MFA):}
        \begin{itemize}
            \item Prioritize enabling MFA for all email accounts (e.g., via Microsoft 365 or Google Workspace settings).
            \item Deploy an MFA solution for all computer and remote access logins.
            \item Enforce MFA for all applications and systems that store or process sensitive data.
        \end{itemize}
    \item \textbf{Develop and Implement Security Policies:}
        \begin{itemize}
            \item Create and enforce an \textbf{Acceptable Use Policy (AUP)} that defines rules for employee use of company assets.
            \item Establish a \textbf{Password Policy} that requires strong, unique passwords for all accounts.
        \end{itemize}
\end{enumerate}

\subsection*{Medium Priority (3-6 Months)}
\begin{enumerate}
    \setcounter{enumi}{3}
    \item \textbf{Establish a Security Awareness Program:}
        \begin{itemize}
            \item Implement a mandatory annual security awareness training program for all employees covering topics like phishing, password security, and the new AUP.
        \end{itemize}
    \item \textbf{Upgrade Outdated Systems:}
        \begin{itemize}
            \item Begin the process of upgrading or replacing all workstations running Windows 7 with a modern, supported operating system like Windows 11.
        \end{itemize}
\end{enumerate}

% --- Document End ---
\end{document}
```