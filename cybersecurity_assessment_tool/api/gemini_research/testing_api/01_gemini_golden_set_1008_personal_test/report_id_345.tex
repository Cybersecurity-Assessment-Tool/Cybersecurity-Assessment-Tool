```latex
\documentclass[12pt]{article}

% Preamble: Required Packages and Document Setup
\usepackage[margin=1in]{geometry}
\usepackage{pifont} % For checkmarks and crosses
\usepackage{booktabs} % For professional tables
\usepackage[hidelinks]{hyperref} % For clickable links
\usepackage{url} % For URL formatting
\usepackage{seqsplit} % To split long strings in tt font
\usepackage{graphicx}
\usepackage{fancyhdr}
\usepackage{lastpage}
\usepackage{xcolor}

% --- Document Metadata and Custom Commands ---
\newcommand{\organizationName}{Phoenix Rising}
\newcommand{\reportDate}{\today}

% Define severity colors
\definecolor{sevCritical}{HTML}{940000}
\definecolor{sevHigh}{HTML}{D14000}
\definecolor{sevMedium}{HTML}{E8A000}
\definecolor{sevLow}{HTML}{3488A4}
\definecolor{sevInfo}{HTML}{666666}

% Header and Footer Configuration
\pagestyle{fancy}
\fancyhf{} % Clear all header and footer fields
\fancyhead[L]{\organizationName}
\fancyhead[R]{Cybersecurity Posture Assessment}
\fancyfoot[C]{\thepage\ of \pageref{LastPage}}
\renewcommand{\headrulewidth}{0.4pt}
\renewcommand{\footrulewidth}{0.4pt}

% --- Document Start ---
\begin{document}

% --- Title Page ---
\begin{titlepage}
    \centering
    \vspace*{2cm}
    \Huge
    \textbf{Cybersecurity Posture Assessment Report}
    \vfill
    \large
    \textbf{Prepared for:} \\
    \vspace{0.5cm}
    \LARGE{\organizationName}
    \vfill
    \large
    \textbf{Date of Report:} \\
    \vspace{0.5cm}
    \Large{\reportDate}
    \vfill
    \textit{This report contains sensitive information and should be handled with care.}
\end{titlepage}

\tableofcontents
\newpage

% --- Section 1: Executive Summary ---
\section{Executive Summary}

This report provides a comprehensive analysis of the cybersecurity posture for \organizationName, based on a combination of technical network scanning, a review of organizational security controls, and an evaluation of pre-existing risk data.

The assessment reveals a mixed security posture. The organization demonstrates strong identity and access management practices, with Multi-Factor Authentication (MFA) widely implemented across key systems. However, two significant areas of concern were identified that elevate the overall risk profile:

\begin{itemize}
    \item \textbf{Critical Finding:} An internal network scan of target \texttt{10.5.5.5} discovered an openly accessible web service on port 8080 with the title \textbf{"TOP SECRET DB"}. This suggests a potential high-value database is exposed internally without adequate access controls, posing a severe risk of data exfiltration or unauthorized access. This finding directly contradicts previous risk assessments which marked this port as secure.
    
    \item \textbf{High-Risk Gap:} A procedural gap was identified in the employee onboarding process. The organization does not provide mandatory security awareness training for new employees, creating a window of vulnerability where new hires are more susceptible to phishing, social engineering, and policy violations.
\end{itemize}

Immediate remediation is required for the critical technical finding. Strategic improvements are recommended to address the identified policy gap to bolster the organization's human firewall.

% --- Section 2: Organizational Information ---
\section{Organizational Information}
The following information was provided for the assessment.

\begin{table}[h!]
\centering
\begin{tabular}{@{}ll@{}}
\toprule
\textbf{Attribute} & \textbf{Value} \\ \midrule
Organization Name & \organizationName \\
Email Domain & \texttt{PhoenixRising.org} \\
Website Domain & \url{www.PhoenixRising.org} \\
External IP Address & \texttt{67.64.160.186} \\ \bottomrule
\end{tabular}
\caption{Client Organizational Details}
\label{tab:org_info}
\end{table}

% --- Section 3: Security Control Review ---
\section{Security Control Review}
A review of administrative and procedural security controls was conducted via a questionnaire. The results are summarized below. A checkmark (\ding{51}) indicates a positive control is in place, while a cross (\ding{55}) indicates a potential security gap.

\begin{table}[h!]
\centering
\begin{tabular}{@{}lc@{}}
\toprule
\textbf{Security Control Question} & \textbf{Status} \\ \midrule
Do you require MFA to access email? & \ding{51} \\
Do you require MFA to log into computers? & \ding{51} \\
Do you require MFA to access sensitive data systems? & \ding{51} \\
Does your organization have an employee acceptable use policy? & \ding{51} \\
Does your organization do security awareness training for new employees? & \textcolor{red}{\ding{55}} \\
Does your organization do security awareness training for all employees annually? & \ding{51} \\ \bottomrule
\end{tabular}
\caption{Security Controls Questionnaire Results}
\label{tab:controls}
\end{table}

\paragraph{Analysis:} The organization has effectively implemented MFA across critical access points. However, the lack of security awareness training for new employees represents a significant gap. New hires are often prime targets for malicious actors, and this gap leaves the organization vulnerable during the critical initial employment period.

% --- Section 4: Technical Scan Results ---
\section{Technical Scan Results}
A network scan was performed on the specified internal target to identify open ports and exposed services.

\begin{itemize}
    \item \textbf{Target IP Address:} \texttt{10.5.5.5}
    \item \textbf{Scanner Used:} Nmap
\end{itemize}

The scan identified the following open port:
\begin{table}[h!]
\centering
\begin{tabular}{@{}llll@{}}
\toprule
\textbf{Port} & \textbf{State} & \textbf{Service/Script Output} \\ \midrule
8080/tcp & Open & \texttt{http-title: TOP SECRET DB} \\ \bottomrule
\end{tabular}
\caption{Open Ports Detected on \texttt{10.5.5.5}}
\label{tab:scan_results}
\end{table}

\paragraph{Analysis:} The finding on port 8080 is of \textbf{critical concern}. The service title explicitly suggests it is a "TOP SECRET" database. Its availability on an open port, even internally, presents a severe risk. This directly contradicts the information from the existing risk register (\textit{Input\_3\_Current\_Risks\_JSON}), which incorrectly classified this port as a secure false positive. This indicates a failure in the risk validation process.

% --- Section 5: Consolidated Risk Assessment ---
\section{Consolidated Risk Assessment}
The following table synthesizes findings from the security control review, technical scan, and pre-existing data into a prioritized list of risks.

\begin{table}[h!]
\centering
\begin{tabular}{@{}lp{2.5cm}p{7.5cm}@{}}
\toprule
\textbf{ID} & \textbf{Risk Name} & \textbf{Description} \\ \midrule
\textbf{RISK-001} & \textcolor{sevCritical}{\textbf{Critical}} \newline Unsecured Internal Database Exposure & A service running on \texttt{10.5.5.5:8080} is titled "TOP SECRET DB" and is openly accessible on the internal network. This poses an immediate threat of unauthorized access and data leakage of potentially highly sensitive information. \\
\addlinespace
\textbf{RISK-002} & \textcolor{sevHigh}{\textbf{High}} \newline Lack of Onboarding Security Training & New employees do not receive security awareness training upon being hired. This creates a significant vulnerability, as untrained users are more likely to fall victim to phishing, mishandle data, or violate security policies. \\
\addlinespace
\textbf{RISK-003} & \textcolor{sevInfo}{\textbf{Informational}} \newline Outdated Risk Assessment Data & The existing risk register incorrectly states that port 8080 is secure. The current scan proves this assessment is outdated and inaccurate, highlighting a potential weakness in the risk management lifecycle and validation process. \\ \bottomrule
\end{tabular}
\caption{Summary of Identified Risks}
\label{tab:risks}
\end{table}

% --- Section 6: Recommendations ---
\section{Recommendations}
Based on the consolidated risk assessment, the following actions are recommended to mitigate the identified risks and improve the overall security posture of \organizationName.

\subsection{Immediate Actions (To Be Completed in < 72 Hours)}
\begin{itemize}
    \item \textbf{For RISK-001:}
    \begin{enumerate}
        \item \textbf{Investigate \& Contain:} Immediately investigate the service on \texttt{10.5.5.5:8080}. Identify the system owner, the type of data it contains, and why it is exposed.
        \item \textbf{Apply Access Controls:} If the service is business-critical, implement strict firewall rules to ensure it is only accessible from authorized systems and users. If it is non-essential or a development system, it should be shut down immediately.
    \end{enumerate}
\end{itemize}

\subsection{Strategic Actions (To Be Completed in < 90 Days)}
\begin{itemize}
    \item \textbf{For RISK-002:}
    \begin{enumerate}
        \item \textbf{Develop Onboarding Training:} Create or procure a mandatory security awareness training module specifically for new employees. This module should be a required part of the formal onboarding process before system access is granted.
        \item \textbf{Content Focus:} Training should cover, at a minimum: phishing identification, acceptable use policies, password hygiene, and data handling procedures.
    \end{enumerate}
    \item \textbf{For RISK-003:}
    \begin{enumerate}
        \item \textbf{Review Risk Management Process:} Implement a formal process to periodically review and validate all documented risks, especially those marked as low-severity or false positives.
        \item \textbf{Integrate Scanning:} Ensure that vulnerability scan results are used to automatically trigger a review of related items in the risk register to prevent outdated information from persisting.
    \end{enumerate}
\end{itemize}

\end{document}
```