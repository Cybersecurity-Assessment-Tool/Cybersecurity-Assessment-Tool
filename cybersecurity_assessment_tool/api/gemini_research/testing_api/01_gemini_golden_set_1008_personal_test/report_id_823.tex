```latex
\documentclass[12pt]{article}

% Preamble: Required Packages
\usepackage[margin=1in]{geometry}
\usepackage{pifont} % For checkmarks and crosses
\usepackage{booktabs} % For professional tables
\usepackage{hyperref} % For clickable links
\usepackage{url} % For URL formatting
\usepackage{seqsplit} % To split long strings in texttt
\usepackage{graphicx}
\usepackage{xcolor}
\usepackage{fancyhdr}

% Document Information
\title{Cybersecurity Posture Assessment Report}
\author{Cybersecurity Analysis Division}
\date{\today}

% Hyperref Setup
\hypersetup{
    colorlinks=true,
    linkcolor=blue,
    filecolor=magenta,      
    urlcolor=cyan,
    pdftitle={Cybersecurity Posture Assessment Report},
    pdfpagemode=FullScreen,
}

% Header and Footer
\pagestyle{fancy}
\fancyhf{}
\fancyhead[L]{Cybersecurity Assessment Report}
\fancyhead[R]{Quantum Reach}
\fancyfoot[C]{\thepage}

\begin{document}

\maketitle
\thispagestyle{empty}
\newpage

\tableofcontents
\newpage

% --- 1. Executive Summary ---
\section{Executive Summary}

This report details the findings of a cybersecurity posture assessment for \textbf{Quantum Reach}. The assessment combined a review of organizational security controls, an external network scan, and an analysis of known risks.

The overall security posture is considered \textbf{weak} and requires immediate remediation. Several critical-risk gaps were identified in fundamental security controls. The most pressing issues include a lack of Multi-Factor Authentication (MFA) for computer and sensitive data access, a complete absence of a security awareness training program, and an exposed administrative service (SSH) on an external-facing IPv6 address.

These deficiencies create a significant risk of unauthorized access, data breach, and successful social engineering attacks. We strongly recommend prioritizing the implementation of MFA and restricting access to the exposed SSH service. A detailed list of risks and actionable recommendations is provided in Sections 5 and 6.

% --- 2. Organizational Information ---
\section{Organizational Information}

The following information was provided for the assessment.

\begin{table}[h!]
\centering
\begin{tabular}{@{}ll@{}}
\toprule
\textbf{Attribute} & \textbf{Value} \\ \midrule
Organization Name & \textbf{Quantum Reach} \\
Email Domain & \texttt{QuantumReach.com} \\
Website Domain & \url{www.QuantumReach.com} \\
Primary External IP & \texttt{59.57.174.193} \\
Scanned Target IP & \seqsplit{\texttt{2001:db8::1}} \\ \bottomrule
\end{tabular}
\caption{Client Organizational Data}
\label{tab:org_info}
\end{table}

% --- 3. Security Control Review ---
\section{Security Control Review}

A review of administrative and policy-based security controls was conducted via a questionnaire. The results highlight critical gaps in user access controls and employee security education. The symbol \textcolor{green}{\ding{51}} indicates a positive control is in place, while \textcolor{red}{\ding{55}} indicates a control gap.

\begin{table}[h!]
\centering
\begin{tabular}{@{}p{0.6\textwidth}cp{0.25\textwidth}@{}}
\toprule
\textbf{Control Question} & \textbf{Status} & \textbf{Analyst Note} \\ \midrule
Do you require MFA to access email? & \textcolor{green}{\ding{51}} & Good Practice. Protects primary communication channel. \\
\addlinespace
Do you require MFA to log into computers? & \textcolor{red}{\ding{55}} & \textbf{Critical Gap.} Lack of endpoint MFA allows for easier lateral movement if credentials are stolen. \\
\addlinespace
Do you require MFA to access sensitive data systems? & \textcolor{red}{\ding{55}} & \textbf{Critical Gap.} Increases the risk of a data breach from compromised accounts. \\
\addlinespace
Does your organization have an employee acceptable use policy? & \textcolor{red}{\ding{55}} & \textbf{High Risk.} Creates ambiguity and lacks a formal standard for employee system usage. \\
\addlinespace
Does your organization do security awareness training for new employees? & \textcolor{red}{\ding{55}} & \textbf{Critical Gap.} New staff are not equipped to identify or report security threats. \\
\addlinespace
Does your organization do security awareness training for all employees at least once per year? & \textcolor{red}{\ding{55}} & \textbf{Critical Gap.} The organization is highly vulnerable to phishing and social engineering. \\ \bottomrule
\end{tabular}
\caption{Security Controls Questionnaire Results}
\label{tab:controls}
\end{table}

% --- 4. Technical Scan Results ---
\section{Technical Scan Results}

An external network scan was performed on the target IP address \seqsplit{\texttt{2001:db8::1}} to identify open ports and exposed services.

\subsection{Summary of Findings}
The scan identified one open port, which corresponds to the Secure Shell (SSH) service. This service is typically used for remote system administration. Exposing SSH directly to the public internet is a high-risk configuration, as it is a common target for brute-force and credential stuffing attacks.

\begin{table}[h!]
\centering
\begin{tabular}{@{}llll@{}}
\toprule
\textbf{Port} & \textbf{State} & \textbf{Service} & \textbf{Notes} \\ \midrule
22/tcp & open & ssh & No version information was obtained. Secure configuration is critical. \\ \bottomrule
\end{tabular}
\caption{Open Ports on \seqsplit{\texttt{2001:db8::1}}}
\label{tab:scan_results}
\end{table}

% --- 5. Consolidated Risk Assessment ---
\section{Consolidated Risk Assessment}

The following table synthesizes findings from the security control review and the technical scan. No pre-existing vulnerabilities were reported. The risks are prioritized by severity.

\begin{table}[h!]
\centering
\begin{tabular}{@{}p{0.1\textwidth}p{0.5\textwidth}p{0.15\textwidth}@{}}
\toprule
\textbf{Risk ID} & \textbf{Description} & \textbf{Severity} \\ \midrule
\textbf{RISK-001} & \textbf{Lack of MFA on Endpoints and Sensitive Systems:} User accounts are protected only by passwords, making them vulnerable to compromise and subsequent unauthorized access to computers and critical data. & \textbf{Critical} \\
\addlinespace
\textbf{RISK-002} & \textbf{Absence of Security Awareness Training:} Employees are not trained to recognize or respond to common cyber threats like phishing, significantly increasing the likelihood of a successful attack. & \textbf{Critical} \\
\addlinespace
\textbf{RISK-003} & \textbf{Exposed SSH Service:} The administrative SSH port (22) is open to the internet, creating a direct attack vector for external threats to attempt to gain privileged access to the network. & \textbf{High} \\
\addlinespace
\textbf{RISK-004} & \textbf{Missing Acceptable Use Policy (AUP):} Without a formal AUP, there is no enforceable policy governing the secure and appropriate use of company assets, leading to potential misuse and insider threats. & \textbf{High} \\ \bottomrule
\end{tabular}
\caption{Summary of Identified Risks}
\label{tab:risks}
\end{table}

% --- 6. Recommendations ---
\section{Recommendations}

Based on the identified risks, we recommend the following actions, prioritized into phases.

\subsection{Immediate Actions (0-30 Days)}
\begin{itemize}
    \item \textbf{Implement MFA (RISK-001):} Immediately enforce MFA for all employees for logging into their computers and accessing any systems containing sensitive data.
    \item \textbf{Restrict SSH Access (RISK-003):} Configure firewall rules to restrict access to TCP port 22 on \seqsplit{\texttt{2001:db8::1}}. Access should be limited to specific, trusted IP addresses or be placed behind a Virtual Private Network (VPN).
\end{itemize}

\subsection{Short-Term Actions (30-90 Days)}
\begin{itemize}
    \item \textbf{Deploy Security Awareness Training (RISK-002):} Procure and deploy a security awareness training program. All current employees must complete the initial training, and it must be integrated into the onboarding process for all new hires.
    \item \textbf{Develop and Enforce an AUP (RISK-004):} Create a formal Acceptable Use Policy that all employees must read and acknowledge. This policy should define the rules for using company IT resources.
\end{itemize}

\subsection{Long-Term Actions (90+ Days)}
\begin{itemize}
    \item \textbf{Conduct Comprehensive Vulnerability Assessment:} Perform a more in-depth, authenticated vulnerability scan across the internal and external network to identify software vulnerabilities and misconfigurations.
    \item \textbf{Establish Continuous Monitoring:} Implement security monitoring solutions to detect and alert on suspicious activity within the network.
\end{itemize}

% --- 7. Conclusion ---
\section{Conclusion}

The assessment for \textbf{Quantum Reach} has revealed several fundamental weaknesses in its cybersecurity defenses. While the organization has taken a positive first step by securing email with MFA, this protection must be extended to all critical access points. The combination of an exposed administrative service and an untrained workforce presents a clear and present danger to the organization's data and operations.

By implementing the recommendations outlined in this report, Quantum Reach can significantly reduce its risk exposure and build a more resilient security foundation. We advise treating these findings with urgency to mitigate the identified threats effectively.

\end{document}
```