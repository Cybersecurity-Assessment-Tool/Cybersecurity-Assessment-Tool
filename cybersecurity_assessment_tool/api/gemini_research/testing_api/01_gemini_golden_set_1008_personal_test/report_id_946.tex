```latex
\documentclass[12pt]{article}

% Preamble: Required Packages
\usepackage[margin=1in]{geometry}
\usepackage{pifont} % For checkmarks and crosses (\ding)
\usepackage{booktabs} % For professional tables
\usepackage{hyperref} % For clickable links
\usepackage{url} % For formatting URLs
\usepackage{seqsplit} % For splitting long strings in texttt
\usepackage{graphicx}
\usepackage{xcolor}
\usepackage{titling}

% Document Styling
\hypersetup{
    colorlinks=true,
    linkcolor=blue,
    filecolor=magenta,      
    urlcolor=cyan,
    pdftitle={Cybersecurity Posture Assessment},
    pdfpagemode=FullScreen,
}

% --- Document Title ---
\title{
    \vspace{-2cm}
    \includegraphics[width=0.3\textwidth]{example-image-a}\\[1cm] % Placeholder for company logo
    \textbf{Cybersecurity Posture Assessment Report}
    \hrule
}
\author{Cybersecurity Analysis Division}
\date{\today}

% --- Document Body ---
\begin{document}

\maketitle

\begin{abstract}
\noindent This report provides a comprehensive cybersecurity posture assessment for \textbf{Silver Leaf Collective}. The analysis is based on a synthesis of self-reported organizational security controls, an external network vulnerability scan, and a review of pre-existing risk documentation. The assessment identifies critical security gaps related to access control and employee training, alongside positive findings such as a strong network perimeter. Immediate remediation of the identified high-impact risks is strongly recommended to protect sensitive data and mitigate the threat of unauthorized access.
\end{abstract}

\newpage

\tableofcontents

\newpage

\section*{1.0 Overview and Executive Summary}

This assessment was conducted to evaluate the current cybersecurity posture of \textbf{Silver Leaf Collective}. The evaluation methodology combined three key data sources:
\begin{itemize}
    \item \textbf{Organizational Data:} A questionnaire detailing existing security policies and controls.
    \item \textbf{Technical Scan Data:} An external network port scan to identify exposed services.
    \item \textbf{Pre-existing Risks:} A review of previously documented vulnerabilities.
\end{itemize}

\subsection*{Key Findings}
The organization demonstrates a foundational level of security awareness with an established acceptable use policy and annual security training. Furthermore, the external network scan revealed no open ports, indicating a well-configured firewall protecting the network perimeter.

However, several critical vulnerabilities were identified that significantly increase the organization's risk profile:
\begin{itemize}
    \item \textbf{Critical - Lack of Multi-Factor Authentication (MFA):} MFA is not enforced for accessing email or other sensitive data systems. This represents a severe weakness, as a single compromised password could lead to a major data breach.
    \item \textbf{High - Inadequate Onboarding Security:} New employees do not receive security awareness training upon being hired. This oversight leaves the organization vulnerable to social engineering attacks, as new staff are often prime targets.
\end{itemize}

This report details these findings and provides actionable recommendations to mitigate the identified risks and strengthen the overall security posture.

\section*{2.0 Organizational Information}

The following information was provided for the assessment.

\begin{table}[h!]
\centering
\begin{tabular}{@{}ll@{}}
\toprule
\textbf{Attribute} & \textbf{Value} \\ \midrule
Organization Name & \textbf{Silver Leaf Collective} \\
Email Domain & \texttt{SilverLeafCollective.com} \\
Website Domain & \seqsplit{\url{www.SilverLeafCollective.com}} \\
External IP Address & \texttt{56.112.178.135} \\ \bottomrule
\end{tabular}
\caption{Client Organizational Details.}
\end{table}

\section*{3.0 Security Control Review}

A review of the organization's self-reported security controls was conducted. The table below summarizes the responses to the security questionnaire. Gaps in security best practices are indicated with a red cross (\ding{55}).

\begin{table}[h!]
\centering
\begin{tabular}{@{}p{0.75\linewidth}c@{}}
\toprule
\textbf{Control Question} & \textbf{Response} \\ \midrule
Do you require MFA to access email? & \textcolor{red}{\ding{55}} \\
Do you require MFA to log into computers? & \textcolor{green}{\ding{51}} \\
Do you require MFA to access sensitive data systems? & \textcolor{red}{\ding{55}} \\
Does your organization have an employee acceptable use policy? & \textcolor{green}{\ding{51}} \\
Does your organization do security awareness training for new employees? & \textcolor{red}{\ding{55}} \\
Does your organization do security awareness training for all employees at least once per year? & \textcolor{green}{\ding{51}} \\ \bottomrule
\end{tabular}
\caption{Security Controls Questionnaire Results.}
\end{table}

\subsection*{Analysis of Control Gaps}
The questionnaire reveals critical deficiencies in access control and employee lifecycle management. The absence of MFA for email and sensitive systems are high-impact vulnerabilities. Business Email Compromise (BEC) and data exfiltration are significantly more likely without this fundamental control. Additionally, failing to train new employees on security best practices from day one creates an immediate and unnecessary risk.

\section*{4.0 Technical Scan Results}

An external network scan was performed against the target IP address to identify any exposed services or potential vulnerabilities.

\begin{itemize}
    \item \textbf{Target IP Address:} \texttt{[Target IP]}
    \item \textbf{Scan Date:} Not available in scan data.
\end{itemize}

\subsection*{Findings}
The scan completed successfully and found \textbf{no open ports}.

\subsection*{Interpretation}
This is a positive security finding. It indicates that a firewall is in place and properly configured to block unsolicited inbound traffic from the internet (a "default deny" posture). This significantly reduces the external attack surface. However, this does not preclude the existence of other vulnerabilities, such as those in web applications or those exploitable via phishing attacks.

\section*{5.0 Risk Assessment Summary}

The following table synthesizes risks identified from the security control review, technical scan, and pre-existing risk data. Since no pre-existing risks were provided, this table focuses on newly identified vulnerabilities.

\begin{table}[h!]
\centering
\begin{tabular}{@{}p{0.1\linewidth}p{0.2\linewidth}p{0.5\linewidth}l@{}}
\toprule
\textbf{Risk ID} & \textbf{Risk Name} & \textbf{Description} & \textbf{Severity} \\ \midrule
R-01 & Lack of MFA for Email Access & Email accounts are secured only by passwords. A single compromised password could lead to Business Email Compromise (BEC), data theft, and further network infiltration. & \textbf{\textcolor{red}{Critical}} \\
\addlinespace
R-02 & Lack of MFA for Sensitive Systems & Systems containing sensitive organizational or customer data are not protected by a second authentication factor, greatly increasing the risk of a data breach from stolen credentials. & \textbf{\textcolor{red}{Critical}} \\
\addlinespace
R-03 & No Security Training for New Hires & New employees are not trained on security policies and threats during onboarding, making them highly susceptible to phishing and other social engineering attacks. & \textbf{\textcolor{orange}{High}} \\ \bottomrule
\end{tabular}
\caption{Identified Security Risks.}
\end{table}

\section*{6.0 Recommendations}

Based on the analysis, the following actions are recommended to mitigate the identified risks and improve the overall security posture of \textbf{Silver Leaf Collective}. Recommendations are prioritized by severity.

\subsection*{Critical Priority}
\begin{description}
    \item[Rec-01 (Ref: R-01)] \textbf{Enforce MFA on All Email Accounts.} Immediately enable and enforce MFA for all user mailboxes. This is the single most effective control to prevent unauthorized account access and BEC attacks.
    \item[Rec-02 (Ref: R-02)] \textbf{Deploy MFA for Sensitive Systems.} Conduct an inventory of all systems that store or process sensitive data. Deploy and mandate the use of MFA for access to these systems without exception.
\end{description}

\subsection*{High Priority}
\begin{description}
    \item[Rec-03 (Ref: R-03)] \textbf{Integrate Security Training into Onboarding.} Develop a mandatory security awareness training module for all new employees. This training must be completed as part of the onboarding process before system access is granted. It should cover, at a minimum, phishing identification, password security, and the acceptable use policy.
\end{description}

\subsection*{Informational}
\begin{description}
    \item[Rec-04] \textbf{Maintain Strong Perimeter Security.} Continue to maintain the current firewall configuration that limits the external attack surface. Regularly review firewall rules to ensure they remain aligned with business needs and security best practices.
\end{description}

\end{document}
```