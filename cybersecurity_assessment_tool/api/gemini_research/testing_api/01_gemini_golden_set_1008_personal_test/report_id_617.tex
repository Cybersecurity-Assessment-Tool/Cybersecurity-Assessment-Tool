```latex
\documentclass[12pt]{article}

% ----------------------------------------------------------------------
% PREAMBLE
% ----------------------------------------------------------------------
\usepackage[margin=1in]{geometry}
\usepackage{pifont} % Required for \ding
\usepackage{booktabs} % For professional-looking tables
\usepackage{hyperref}
\usepackage{url}
\usepackage{seqsplit} % For breaking long, unbreakable strings
\usepackage{xcolor}
\usepackage{graphicx}
\usepackage{titling}

% --- Customization ---
\hypersetup{
    colorlinks=true,
    linkcolor=black,
    urlcolor=blue,
}

% Define colors for severity
\definecolor{criticalred}{HTML}{D10000}
\definecolor{highorange}{HTML}{E25F00}

% --- Document Metadata ---
\title{\textbf{Cybersecurity Posture Assessment Report}}
\author{Cybersecurity Analyst}
\date{\today}

% ----------------------------------------------------------------------
% DOCUMENT START
% ----------------------------------------------------------------------
\begin{document}

\maketitle
\thispagestyle{empty}
\newpage
\tableofcontents
\newpage

% ----------------------------------------------------------------------
% 1. EXECUTIVE SUMMARY
% ----------------------------------------------------------------------
\section{Executive Summary}

This report provides a cybersecurity assessment for \textbf{Orchid Isle}, synthesizing data from technical network scans, a security controls questionnaire, and a review of pre-existing risks. The analysis reveals several critical and high-severity risks that require immediate attention to mitigate potential threats to the organization's data and operations.

Key findings indicate a pattern of insecure Remote Desktop Protocol (RDP) exposure across multiple systems, a critical gap in enforcing Multi-Factor Authentication (MFA) for email access, and a lack of foundational security awareness training for new employees. These vulnerabilities significantly increase the risk of unauthorized access, business email compromise (BEC), and ransomware attacks.

This document details each finding and provides actionable, prioritized recommendations to strengthen the organization's overall security posture.

% ----------------------------------------------------------------------
% 2. ORGANIZATIONAL INFORMATION
% ----------------------------------------------------------------------
\section{Organizational Information}
The following information was provided for the assessment.

\begin{center}
\begin{tabular}{@{}ll}
\toprule
\textbf{Attribute} & \textbf{Value} \\
\midrule
Organization Name & \textbf{Orchid Isle} \\
Email Domain & \texttt{OrchidIsle.net} \\
External IP Address & \texttt{138.84.195.205} \\
\bottomrule
\end{tabular}
\end{center}

% ----------------------------------------------------------------------
% 3. SECURITY CONTROL REVIEW
% ----------------------------------------------------------------------
\section{Security Control Review}
This section evaluates the organization's implemented security controls based on a self-assessment questionnaire. Gaps in these controls often represent significant organizational risk. A green checkmark (\ding{51}) indicates a positive control is in place, while a red cross (\ding{55}) indicates a control gap.

\begin{center}
\begin{tabular}{p{0.8\linewidth}c}
\toprule
\textbf{Control Question} & \textbf{Status} \\
\midrule
Do you require MFA to access email? & \textcolor{criticalred}{\ding{55}} \\
Do you require MFA to log into computers? & \textcolor{green}{\ding{51}} \\
Do you require MFA to access sensitive data systems? & \textcolor{green}{\ding{51}} \\
Does your organization have an employee acceptable use policy? & \textcolor{green}{\ding{51}} \\
Does your organization do security awareness training for new employees? & \textcolor{criticalred}{\ding{55}} \\
Does your organization do security awareness training for all employees at least once per year? & \textcolor{green}{\ding{51}} \\
\bottomrule
\end{tabular}
\end{center}

\subsection*{Analysis of Control Gaps}
\begin{itemize}
    \item \textbf{No MFA for Email:} This is a critical vulnerability. Email accounts are primary targets for phishing and account takeover attacks, which can lead to data breaches and financial fraud.
    \item \textbf{No Security Training for New Employees:} New hires are often more susceptible to social engineering attacks. Failing to provide immediate security training leaves a recurring window of vulnerability as the organization grows.
\end{itemize}


% ----------------------------------------------------------------------
% 4. TECHNICAL SCAN RESULTS
% ----------------------------------------------------------------------
\section{Technical Scan Results}
A network scan was performed to identify externally accessible services and potential vulnerabilities on the target host.

\subsection*{Host: \texttt{10.10.10.51}}
The scan identified the following open port, which indicates a service directly exposed to the network.

\begin{center}
\begin{tabular}{llll}
\toprule
\textbf{Port} & \textbf{State} & \textbf{Service Name} & \textbf{Analysis} \\
\midrule
3389/tcp & open & \texttt{ms-wbt-server} & Remote Desktop Protocol (RDP) \\
\bottomrule
\end{tabular}
\end{center}

\subsection*{Analysis of Technical Findings}
The presence of an open RDP port is a significant security risk. RDP is a common vector for ransomware and other malware. Exposing this service without mitigating controls like a VPN or IP address whitelisting is strongly discouraged. This finding, correlated with pre-existing risk data, suggests a systemic issue with RDP exposure within the organization.

% ----------------------------------------------------------------------
% 5. CORRELATED RISK ASSESSMENT
% ----------------------------------------------------------------------
\section{Correlated Risk Assessment}
This section synthesizes findings from all data sources to provide a holistic view of the most pressing security risks.

\begin{center}
\begin{tabular}{p{0.25\linewidth}p{0.55\linewidth}l}
\toprule
\textbf{Risk Title} & \textbf{Overview} & \textbf{Severity} \\
\midrule
\textbf{Widespread RDP Exposure} & The technical scan identified RDP exposed on host \texttt{10.10.10.51}. This correlates with a pre-existing risk identifying the same issue on host \texttt{10.10.10.50}. This pattern indicates a lack of network hardening standards. & \textcolor{criticalred}{\textbf{Critical}} \\
\addlinespace
\textbf{Lack of MFA for Email Access} & The security questionnaire confirmed that MFA is not enforced for email access. This exposes the organization to severe risks of business email compromise (BEC), phishing, and subsequent data breaches. & \textcolor{criticalred}{\textbf{Critical}} \\
\addlinespace
\textbf{Inadequate New-Hire Security Training} & The questionnaire revealed that new employees do not receive security awareness training upon joining. This creates a persistent vulnerability, as untrained staff are prime targets for social engineering attacks. & \textcolor{highorange}{\textbf{High}} \\
\bottomrule
\end{tabular}
\end{center}

% ----------------------------------------------------------------------
% 6. RECOMMENDATIONS
% ----------------------------------------------------------------------
\section{Recommendations}
The following actions are recommended to mitigate the identified risks. Recommendations are prioritized by severity.

\subsection*{6.1 Critical Risk: Widespread RDP Exposure}
\begin{itemize}
    \item \textbf{Immediate Action:} For all systems exposing RDP (\texttt{10.10.10.51}, \texttt{10.10.10.50}, and any others), immediately implement firewall rules to restrict access to only trusted IP addresses. If no immediate business need exists, the port should be closed entirely.
    \item \textbf{Long-Term Strategy:} Implement a Virtual Private Network (VPN) or a remote desktop gateway solution for all remote administrative access. This provides a secure, encrypted tunnel and removes the need to expose RDP directly to the network. Enforce strong, unique passwords and MFA for all remote access accounts.
\end{itemize}

\subsection*{6.2 Critical Risk: Lack of MFA for Email Access}
\begin{itemize}
    \item \textbf{Immediate Action:} Enable and enforce MFA for all user accounts on the \texttt{OrchidIsle.net} email domain. Prioritize enforcement for administrative and executive accounts.
    \item \textbf{Long-Term Strategy:} Develop a policy that mandates MFA for all cloud services and critical systems. Conduct a user awareness campaign to educate employees on the importance of MFA and how to use it.
\end{itemize}

\subsection*{6.3 High Risk: Inadequate New-Hire Security Training}
\begin{itemize}
    \item \textbf{Immediate Action:} Develop a baseline security awareness training module and integrate it into the mandatory onboarding process for all new employees, to be completed within their first week of employment.
    \item \textbf{Long-Term Strategy:} Formalize the security training program. Ensure the new-hire training covers key topics such as phishing, password hygiene, and the acceptable use policy. Supplement this with the existing annual refresher training for all staff.
\end{itemize}

\end{document}
```