```latex
\documentclass[12pt]{article}

% Preamble: Required Packages
\usepackage[margin=1in]{geometry}
\usepackage{pifont} % For checkmarks and crosses
\usepackage{booktabs} % For professional tables
\usepackage[hidelinks]{hyperref} % For clickable links without boxes
\usepackage{url} % For URL formatting
\usepackage{seqsplit} % For splitting long strings
\usepackage{fancyhdr} % For headers/footers
\usepackage{xcolor} % For colors
\usepackage{graphicx}
\usepackage{lipsum} % For placeholder text if needed

% --- Document Setup ---
\definecolor{darkblue}{rgb}{0.0, 0.0, 0.55}
\definecolor{darkred}{rgb}{0.55, 0.0, 0.0}

\hypersetup{
    colorlinks=true,
    linkcolor=darkblue,
    filecolor=darkblue,      
    urlcolor=darkblue,
    citecolor=darkblue,
}

\pagestyle{fancy}
\fancyhf{} % Clear all header and footer fields
\fancyhead[L]{\textbf{Cybersecurity Posture Assessment}}
\fancyhead[R]{Confidential}
\fancyfoot[C]{\thepage}
\renewcommand{\headrulewidth}{0.4pt}
\renewcommand{\footrulewidth}{0.4pt}

% --- Document Start ---
\begin{document}

% --- Title Page ---
\begin{titlepage}
    \centering
    \vspace*{1cm}
    
    \Huge
    \textbf{Cybersecurity Posture Assessment Report}
    
    \vspace{1.5cm}
    
    \Large
    Prepared for: \\
    \vspace{0.5cm}
    \textbf{Granite Shield}
    
    \vspace{2cm}
    
    \large
    Report Date: \today
    
    \vfill
    
    \large
    \textbf{Cybersecurity Analysis Division} \\
    \texttt{confidential@security-analyst.local}
    
\end{titlepage}

\tableofcontents
\newpage

% --- Section 1: Executive Summary ---
\section{Executive Summary}
This report details the findings of a cybersecurity posture assessment conducted for \textbf{Granite Shield}. The assessment combined a review of organizational security controls, an automated network scan of internal assets, and a correlation with existing risk documentation.

The analysis revealed several high-impact risks requiring immediate attention. The most critical finding is the exposure of an internal service on port 8080, explicitly titled \textbf{"TOP SECRET DB"}. This finding directly contradicts the current risk register, which incorrectly lists this port as a secured false positive. This indicates not only a severe technical vulnerability but also a significant failure in the risk management process.

Furthermore, critical security control gaps were identified, including the absence of Multi-Factor Authentication (MFA) for sensitive data systems and the lack of a formal employee Acceptable Use Policy. The combination of these vulnerabilities creates a high-risk environment where a single compromised credential could lead to a significant data breach.

This report outlines these findings in detail and provides a prioritized list of actionable recommendations to mitigate the identified risks and strengthen the organization's overall security posture.

% --- Section 2: Organizational Information ---
\section{Organizational Information}
The following details were provided for the assessment scope.
\begin{itemize}
    \item \textbf{Organization Name:} Granite Shield
    \item \textbf{Email Domain:} \texttt{GraniteShield.org}
    \item \textbf{Website Domain:} \url{www.GraniteShield.org}
    \item \textbf{External IP Address:} \texttt{97.143.141.217}
\end{itemize}

% --- Section 3: Security Control Review ---
\section{Security Control Review}
A questionnaire was completed to evaluate the implementation of fundamental security controls. The results are summarized below. "No" answers represent significant gaps in the security framework.

\begin{table}[h!]
\centering
\caption{Organizational Security Control Questionnaire}
\begin{tabular}{p{0.8\linewidth} c}
\toprule
\textbf{Control Question} & \textbf{Response} \\
\midrule
Do you require MFA to access email? & \ding{51} \\
Do you require MFA to log into computers? & \ding{51} \\
\color{darkred}Do you require MFA to access sensitive data systems? & \color{darkred}\ding{55} \\
\color{darkred}Does your organization have an employee acceptable use policy? & \color{darkred}\ding{55} \\
Does your organization do security awareness training for new employees? & \ding{51} \\
Does your organization do security awareness training for all employees at least once per year? & \ding{51} \\
\bottomrule
\end{tabular}
\end{table}

\subsection*{Analysis of Control Gaps}
Two critical gaps were identified:
\begin{enumerate}
    \item \textbf{No MFA for Sensitive Data Systems:} The lack of MFA on systems housing sensitive data is a critical vulnerability. This control is an industry standard for protecting against credential theft and unauthorized access. Its absence significantly increases the risk of a data breach.
    \item \textbf{No Acceptable Use Policy (AUP):} An AUP is a foundational governance document that defines responsibilities for employees when using company assets. Without an AUP, the organization lacks a formal mechanism to enforce security standards and address policy violations, increasing insider risk.
\end{enumerate}

% --- Section 4: Technical Scan Results ---
\section{Technical Scan Results}
An Nmap scan was performed on the specified internal target to identify open ports and exposed services.

\subsection*{Target: \texttt{10.5.5.5}}
The scan revealed the following open port and service information.

\begin{table}[h!]
\centering
\caption{Open Ports on \texttt{10.5.5.5}}
\begin{tabular}{l l p{0.6\linewidth}}
\toprule
\textbf{Port} & \textbf{State} & \textbf{Service / Banner Information} \\
\midrule
8080/tcp & Open & \textbf{HTTP Title:} TOP SECRET DB \\
\bottomrule
\end{tabular}
\end{table}

\subsection*{Analysis of Technical Findings}
The primary finding is an open HTTP service on port 8080 with a title suggesting it is a highly sensitive database. Exposing such a service, even on an internal network, presents a severe risk of unauthorized data access.

\textbf{Crucially, this finding directly contradicts the provided risk register}, which states: \textit{"Port 8080 is confirmed secure and false positive."} This discrepancy indicates that the existing risk management process is flawed, as a critical vulnerability was previously misclassified and ignored. The current scan confirms the risk is active and severe.

% --- Section 5: Consolidated Risk Assessment ---
\section{Consolidated Risk Assessment}
The following table synthesizes findings from the security control review, technical scan, and existing risk data into a prioritized list of risks.

\begin{table}[h!]
\centering
\caption{Summary of Identified Risks}
\begin{tabular}{p{0.1\linewidth} p{0.45\linewidth} p{0.15\linewidth} p{0.2\linewidth}}
\toprule
\textbf{Risk ID} & \textbf{Description} & \textbf{Severity} & \textbf{Primary Impact} \\
\midrule
\textbf{RISK-001} & An internal service titled "TOP SECRET DB" is exposed on port 8080, indicating potential unprotected access to sensitive data. & \textbf{Critical} & Data Breach, Confidentiality Loss \\
\addlinespace
\textbf{RISK-002} & Lack of Multi-Factor Authentication (MFA) on sensitive data systems allows access with only a password. & \textbf{Critical} & Unauthorized Access, Account Takeover \\
\addlinespace
\textbf{RISK-003} & The organization lacks an Acceptable Use Policy (AUP), leading to unclear security responsibilities for employees. & \textbf{High} & Insider Threat, Non-Compliance \\
\addlinespace
\textbf{RISK-004} & The current risk register is inaccurate, having incorrectly dismissed a critical vulnerability (RISK-001) as a false positive. & \textbf{High} & Governance Failure, Ineffective Remediation \\
\bottomrule
\end{tabular}
\end{table}

% --- Section 6: Recommendations ---
\section{Recommendations}
The following actionable recommendations are provided to mitigate the identified risks. They are prioritized based on severity and potential impact.

\subsection*{Immediate Actions (To Be Completed within 72 Hours)}
\begin{enumerate}
    \item \textbf{Contain RISK-001:} Immediately apply a firewall rule to block all access to \texttt{10.5.5.5:8080} from non-essential personnel.
    \item \textbf{Investigate RISK-001:} Identify the owner and purpose of the service on port 8080. Determine the nature of the data it contains and review access logs for any signs of unauthorized activity.
\end{enumerate}

\subsection*{High-Priority Actions (To Be Completed within 30 Days)}
\begin{enumerate}
    \item \textbf{Remediate RISK-002:} Enforce mandatory MFA across all systems identified as containing sensitive data. Provide training and support to users during the rollout.
    \item \textbf{Remediate RISK-003:} Develop and implement a formal employee Acceptable Use Policy. Require all current employees and new hires to read and acknowledge the policy as a condition of network access.
\end{enumerate}

\subsection*{Medium-Priority Actions (To Be Completed within 90 Days)}
\begin{enumerate}
    \item \textbf{Remediate RISK-004:} Conduct a comprehensive review of the entire risk management and vulnerability assessment process. Ensure that all findings are properly validated, tracked, and remediated. Update the risk register to accurately reflect the status of RISK-001.
    \item \textbf{Conduct Internal Scan:} Perform a comprehensive, authenticated vulnerability scan across the entire internal network to identify any other misconfigurations or exposed services.
\end{enumerate}

\end{document}
```