```latex
\documentclass[12pt]{article}

% Preamble: Required Packages
\usepackage[margin=1in]{geometry}
\usepackage{pifont} % For checkmarks and crosses
\usepackage{booktabs} % For professional tables
\usepackage{hyperref} % For clickable links and ToC
\usepackage{url} % For formatting URLs
\usepackage{seqsplit} % To split long strings without breaking
\usepackage{xcolor} % For colors

% Hyperref Setup
\hypersetup{
    colorlinks=true,
    linkcolor=blue,
    filecolor=magenta,      
    urlcolor=cyan,
    pdftitle={Cybersecurity Assessment Report},
    pdfpagemode=FullScreen,
}

% Document Start
\begin{document}

% Title Page
\title{
    Cybersecurity Assessment Report \\
    \large For: Structure \& Form
}
\author{Cybersecurity Analyst Group}
\date{\today}
\maketitle

\hrule
\vspace{1em}
\begin{abstract}
This report provides a detailed analysis of the cybersecurity posture of Structure \& Form. The assessment is based on a synthesis of network scan data, an organizational security questionnaire, and a review of known pre-existing risks. The findings indicate a robust network perimeter but highlight critical gaps in internal security controls and policies. This document outlines the identified risks and provides actionable recommendations to enhance the organization's overall security resilience.
\end{abstract}
\vspace{1em}
\hrule

\newpage

\tableofcontents

\newpage

% Section 1: Executive Summary
\section{Overview and Executive Summary}
This assessment was conducted to evaluate the security posture of Structure \& Form. The evaluation combined a technical network scan, a review of administrative security controls via a questionnaire, and an analysis of previously identified risks.

\paragraph{Key Findings:} The primary finding of this assessment is a significant disparity between the organization's technical perimeter security and its internal administrative controls.
\begin{itemize}
    \item \textbf{Technical Posture (Strong):} The external network scan of the target host \texttt{192.168.1.100} revealed no open ports. This indicates a well-configured firewall and a strong network perimeter, effectively minimizing the external attack surface.
    \item \textbf{Administrative Posture (Weak):} The security control review identified critical deficiencies in fundamental security practices. The lack of Multi-Factor Authentication (MFA) for email and computer access represents a critical risk, leaving the organization highly vulnerable to credential theft and account takeover attacks. Furthermore, the absence of an Acceptable Use Policy and mandatory security training for new hires creates significant unaddressed risk from insider threats and social engineering.
\end{itemize}

\paragraph{Overall Assessment:} While the organization has successfully secured its network perimeter, it remains highly exposed to common cyber threats due to weak identity and access management and a lack of foundational security policies. The recommendations in this report focus on addressing these critical gaps to build a more comprehensive and resilient security program.

% Section 2: Organizational Information
\section{Organizational Information}
The following information was provided for the assessment. This data forms the baseline context for the analysis.

\begin{tabular}{@{}ll}
    \toprule
    \textbf{Attribute} & \textbf{Value} \\
    \midrule
    Organization Name & Structure \& Form \\
    Email Domain & \texttt{StructureForm.org} \\
    Website Domain & \url{www.StructureForm.org} \\
    External IP Address & \texttt{140.163.33.138} \\
    \bottomrule
\end{tabular}

% Section 3: Security Control Review
\section{Security Control Review (Questionnaire Analysis)}
The following table details the responses to the security controls questionnaire. "No" answers indicate significant gaps in the security framework and are flagged as risks.

\begin{table}[h!]
\centering
\begin{tabular}{@{}p{8cm}ccc}
    \toprule
    \textbf{Control Question} & \textbf{Response} & \textbf{Assessment} \\
    \midrule
    Do you require MFA to access email? & \textcolor{red}{\ding{55}} No & \textcolor{red}{\textbf{Critical Gap}} \\
    Do you require MFA to log into computers? & \textcolor{red}{\ding{55}} No & \textcolor{red}{\textbf{High Risk}} \\
    Do you require MFA to access sensitive data systems? & \textcolor{green}{\ding{51}} Yes & Compliant \\
    Does your organization have an employee acceptable use policy? & \textcolor{red}{\ding{55}} No & \textcolor{red}{\textbf{High Risk}} \\
    Does your organization do security awareness training for new employees? & \textcolor{red}{\ding{55}} No & \textcolor{red}{\textbf{High Risk}} \\
    Does your organization do security awareness training for all employees at least once per year? & \textcolor{green}{\ding{51}} Yes & Compliant \\
    \bottomrule
\end{tabular}
\caption{Security Controls Questionnaire Results.}
\label{tab:controls}
\end{table}

% Section 4: Technical Scan Results
\section{Technical Scan Results}
An external network scan was performed to identify open ports and exposed services.

\begin{itemize}
    \item \textbf{Scan Target:} \texttt{192.168.1.100}
    \item \textbf{Scan Date:} \today
    \item \textbf{Scanner Used:} Nmap
\end{itemize}

\subsection{Summary of Findings}
The scan confirmed that the target host is online and responsive. However, \textbf{no open ports were detected}. All 1000 scanned ports were found to be in a 'closed' state.

\paragraph{Analysis:} This result is positive and indicates a strong network security posture for the scanned host. A lack of open ports significantly reduces the attack surface available to external adversaries, suggesting that a firewall is properly configured to deny unsolicited inbound traffic. No further analysis of services, products, or versions was possible as no services were exposed.

% Section 5: Risk Assessment
\section{Risk Assessment Summary}
This section synthesizes findings from the security control review and technical scan. As no pre-existing vulnerabilities were reported and the network scan found no technical flaws, the risks listed below are derived directly from the identified policy and procedure gaps.

\begin{table}[h!]
\centering
\begin{tabular}{@{}lp{5cm}p{5.5cm}l@{}}
    \toprule
    \textbf{ID} & \textbf{Risk Name} & \textbf{Description} & \textbf{Severity} \\
    \midrule
    RISK-001 & Inadequate Email Security & No MFA on email accounts makes them highly susceptible to phishing and credential compromise, leading to potential Business Email Compromise (BEC) and data breaches. & \textbf{Critical} \\
    \addlinespace
    RISK-002 & Weak Endpoint Access Controls & No MFA on computer logins allows an attacker with stolen credentials to easily gain access to workstations, facilitating lateral movement and data theft. & \textbf{High} \\
    \addlinespace
    RISK-003 & Lack of Acceptable Use Policy (AUP) & The absence of a formal AUP creates ambiguity for employees regarding safe technology use and exposes the organization to legal and compliance risks. & \textbf{High} \\
    \addlinespace
    RISK-004 & Inadequate New Hire Onboarding & Failing to provide security training to new employees leaves a critical window of vulnerability, as new staff are often prime targets for social engineering attacks. & \textbf{High} \\
    \bottomrule
\end{tabular}
\caption{Summary of Identified Risks.}
\label{tab:risks}
\end{table}

% Section 6: Recommendations
\section{Recommendations}
The following actions are recommended to mitigate the identified risks and strengthen the overall security posture of Structure \& Form.

\subsection{RISK-001: Implement MFA for Email (Critical)}
Immediately prioritize the deployment of Multi-Factor Authentication (MFA) across all email accounts. This is the single most effective control to prevent unauthorized account access resulting from stolen credentials.
\begin{itemize}
    \item \textbf{Action:} Enable MFA within your email provider's security settings (e.g., Microsoft 365, Google Workspace).
    \item \textbf{Options:} Utilize authenticator apps (e.g., Microsoft Authenticator, Google Authenticator) or hardware tokens for the second factor.
\end{itemize}

\subsection{RISK-002: Enforce MFA for Computer Logins (High)}
To protect against unauthorized use of company computers, MFA should be required for all workstation and laptop logins. This adds a crucial layer of security to protect data at rest and prevent lateral movement.
\begin{itemize}
    \item \textbf{Action:} Implement an MFA solution for your operating system environment (e.g., Windows Hello for Business, Duo Security, or YubiKey).
\end{itemize}

\subsection{RISK-003: Develop and Implement an Acceptable Use Policy (High)}
A formal AUP is a foundational document for cybersecurity governance. It establishes clear rules for all employees regarding the use of company technology and data.
\begin{itemize}
    \item \textbf{Action:} Draft an AUP that covers topics such as data handling, internet usage, personal device usage (BYOD), and password security.
    \item \textbf{Implementation:} Require all employees to read and formally acknowledge the policy.
\end{itemize}

\subsection{RISK-004: Integrate Security Training into Onboarding (High)}
New employees are uniquely vulnerable and must be equipped with security knowledge from day one.
\begin{itemize}
    \item \textbf{Action:} Develop a security awareness training module specifically for new hires. This should be a mandatory part of the onboarding process.
    \item \textbf{Content:} The training should cover phishing identification, password hygiene, and the new Acceptable Use Policy.
\end{itemize}

\end{document}
```