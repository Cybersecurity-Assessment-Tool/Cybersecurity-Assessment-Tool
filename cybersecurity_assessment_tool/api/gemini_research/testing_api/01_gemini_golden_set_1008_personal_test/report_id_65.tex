```latex
\documentclass[12pt]{article}

% ----------------------------------------------------------------------
% PREAMBLE
% ----------------------------------------------------------------------
\usepackage[margin=1in]{geometry}
\usepackage{pifont} % For checkmarks and crosses
\usepackage{booktabs} % For professional tables
\usepackage{hyperref} % For clickable links and TOC
\usepackage{url} % For formatting URLs
\usepackage{seqsplit} % For splitting long strings in tt font
\usepackage[utf8]{inputenc}
\usepackage{graphicx}
\usepackage{xcolor}

% Hyperref setup
\hypersetup{
    colorlinks=true,
    linkcolor=black,
    filecolor=magenta,      
    urlcolor=blue,
    pdftitle={Cybersecurity Assessment Report},
    pdfpagemode=FullScreen,
}

% Define checkmark and cross symbols for clarity
\newcommand{\cmark}{\ding{51}}%
\newcommand{\xmark}{\ding{55}}%

% ----------------------------------------------------------------------
% DOCUMENT START
% ----------------------------------------------------------------------
\begin{document}

% --- TITLE PAGE ---
\title{
    \vspace{2cm}
    \textbf{Cybersecurity Assessment Report} \\
    \large{For: \textbf{Green Sprout Organic}}
    \vspace{1cm}
}
\author{Cybersecurity Analysis Division}
\date{\today}
\maketitle
\thispagestyle{empty}
\newpage

% --- TABLE OF CONTENTS ---
\tableofcontents
\newpage

% ----------------------------------------------------------------------
% SECTION 1: EXECUTIVE OVERVIEW
% ----------------------------------------------------------------------
\section{Executive Overview}

This report details the findings of a cybersecurity assessment conducted for \textbf{Green Sprout Organic}. The assessment combined a technical network scan, a review of existing risk documentation, and an analysis of organizational security controls.

The overall security posture is determined to be at a \textbf{Critical Risk} level. This is based on the convergence of several high-impact findings:

\begin{itemize}
    \item \textbf{Critical Service Exposure:} The technical scan identified a new instance of an exposed Remote Desktop Protocol (RDP) service on host \texttt{10.10.10.51}. This finding is compounded by a pre-existing, documented risk of RDP exposure on another host (\texttt{10.10.10.50}), indicating a systemic issue with network security configuration. RDP is a primary vector for ransomware attacks.
    
    \item \textbf{Lack of Foundational Controls:} The organization has not implemented Multi-Factor Authentication (MFA) for any systems, including email, computer logins, or access to sensitive data. This significantly increases the risk of a successful breach via compromised credentials.
    
    \item \textbf{Significant Policy Gaps:} There is a complete absence of security awareness training and a formal acceptable use policy. This leaves the organization highly susceptible to social engineering and phishing attacks, which are the most common methods for initial credential theft.
\end{itemize}

The combination of an easily exploitable external service (RDP) with a lack of mandatory MFA and a workforce untrained in security best practices creates a direct and immediate path for a malicious actor to compromise the network. Immediate remediation is strongly advised.

% ----------------------------------------------------------------------
% SECTION 2: ORGANIZATIONAL INFORMATION
% ----------------------------------------------------------------------
\section{Organizational Information}

The following information was provided for the assessment.

\begin{table}[h!]
\centering
\begin{tabular}{@{}ll@{}}
\toprule
\textbf{Attribute} & \textbf{Value} \\ \midrule
Organization Name & \textbf{Green Sprout Organic} \\
Email Domain & \texttt{GreenSproutOrganic.net} \\
Website Domain & \seqsplit{\url{www.GreenSproutOrganic.net}} \\
External IP Address & \texttt{214.164.25.250} \\ \bottomrule
\end{tabular}
\caption{Client Organizational Details.}
\end{table}

% ----------------------------------------------------------------------
% SECTION 3: SECURITY CONTROL REVIEW
% ----------------------------------------------------------------------
\section{Security Control Review}

A review of administrative and policy-based security controls was conducted via a questionnaire. The results reveal critical gaps in foundational security practices. A response of \textcolor{red}{\xmark} indicates a missing control and a significant area of risk.

\begin{table}[h!]
\centering
\begin{tabular}{@{}p{0.8\linewidth}c@{}}
\toprule
\textbf{Control Question} & \textbf{Response} \\ \midrule
Do you require MFA to access email? & \textcolor{red}{\xmark} \\
Do you require MFA to log into computers? & \textcolor{red}{\xmark} \\
Do you require MFA to access sensitive data systems? & \textcolor{red}{\xmark} \\
Does your organization have an employee acceptable use policy? & \textcolor{red}{\xmark} \\
Does your organization do security awareness training for new employees? & \textcolor{red}{\xmark} \\
Does your organization do security awareness training for all employees at least once per year? & \textcolor{red}{\xmark} \\ \bottomrule
\end{tabular}
\caption{Security Controls Questionnaire Results.}
\end{table}

\subsection*{Analysis}
The complete absence of these fundamental controls is a major security concern. Without MFA, a single compromised password can grant an attacker full access to an employee's email, computer, and sensitive data. The lack of security training and policies makes the initial password compromise through phishing or other social engineering tactics highly probable.

% ----------------------------------------------------------------------
% SECTION 4: TECHNICAL SCAN RESULTS
% ----------------------------------------------------------------------
\section{Technical Scan Results}

An external network scan was performed to identify open ports and exposed services on the target system.

\subsection*{Host: \texttt{10.10.10.51}}
The scan revealed the following open port on the target host:

\begin{table}[h!]
\centering
\begin{tabular}{@{}llll@{}}
\toprule
\textbf{Port} & \textbf{State} & \textbf{Service Name} & \textbf{Description} \\ \midrule
3389/tcp & open & \texttt{ms-wbt-server} & Microsoft Remote Desktop Protocol (RDP) \\ \bottomrule
\end{tabular}
\caption{Open Ports Detected on \texttt{10.10.10.51}.}
\end{table}

\subsection*{Analysis}
The discovery of an open RDP port is a critical finding. RDP is a frequent target for brute-force password attacks and exploitation of known vulnerabilities (e.g., BlueKeep). When exposed directly to the internet, it provides a gateway for attackers to gain remote control over the system. This finding, combined with the lack of MFA, elevates the risk of a full network compromise.

% ----------------------------------------------------------------------
% SECTION 5: CONSOLIDATED RISK ASSESSMENT
% ----------------------------------------------------------------------
\section{Consolidated Risk Assessment}

The following table synthesizes findings from the technical scan, controls review, and pre-existing risk data into a consolidated list of key risks.

\begin{table}[h!]
\centering
\begin{tabular}{@{}p{0.25\linewidth}p{0.45\linewidth}p{0.15\linewidth}@{}}
\toprule
\textbf{Risk Name} & \textbf{Description} & \textbf{Severity} \\ \midrule
\textbf{Systemic RDP Exposure} & Port 3389 (RDP) is open on \texttt{10.10.10.51} (new finding) and \texttt{10.10.10.50} (existing risk). This indicates a pattern of insecure configuration. & \textbf{Critical} \\
\addlinespace
\textbf{No Multi-Factor Authentication (MFA)} & The lack of MFA across all systems means that compromised credentials provide direct, unfettered access to corporate resources. & \textbf{Critical} \\
\addlinespace
\textbf{No Security Awareness Program} & Employees are not trained to recognize or report phishing attempts, making the organization an easy target for credential harvesting and social engineering. & \textbf{High} \\
\addlinespace
\textbf{No Acceptable Use Policy (AUP)} & Without a formal AUP, there are no established rules for employees regarding data handling, password security, or appropriate use of company assets. & \textbf{High} \\
\bottomrule
\end{tabular}
\caption{Summary of Identified Risks.}
\end{table}

% ----------------------------------------------------------------------
% SECTION 6: RECOMMENDATIONS
% ----------------------------------------------------------------------
\section{Recommendations}

The following recommendations are prioritized to address the most critical risks first.

\subsection*{Immediate Actions (Within 72 Hours)}
\begin{enumerate}
    \item \textbf{Remediate RDP Exposure:} Immediately close port 3389 to the public internet on hosts \texttt{10.10.10.51} and \texttt{10.10.10.50}. If remote access is essential, it \textbf{must} be placed behind a Virtual Private Network (VPN) that requires MFA for access.
\end{enumerate}

\subsection*{High-Priority Actions (Within 30 Days)}
\begin{enumerate}
    \item \textbf{Implement MFA:} Deploy MFA for all employees. Prioritize the following services:
    \begin{itemize}
        \item Email (e.g., Office 365, Google Workspace).
        \item Any VPN or remote access solution.
        \item All administrative accounts.
    \end{itemize}
    \item \textbf{Develop Foundational Policies:} Create and enforce a basic Acceptable Use Policy (AUP) that outlines employee responsibilities for protecting company data and resources.
\end{enumerate}

\subsection*{Medium-Priority Actions (Within 90 Days)}
\begin{enumerate}
    \item \textbf{Establish Security Awareness Training:} Implement a mandatory security awareness training program for all employees. This should be a condition of employment for new hires and conducted at least annually for all staff. The training should cover phishing, password security, and social engineering.
    \item \textbf{Conduct a Comprehensive Vulnerability Scan:} Perform an authenticated vulnerability scan across the internal network to identify other potential misconfigurations, missing patches, or weak passwords.
\end{enumerate}

\end{document}
```