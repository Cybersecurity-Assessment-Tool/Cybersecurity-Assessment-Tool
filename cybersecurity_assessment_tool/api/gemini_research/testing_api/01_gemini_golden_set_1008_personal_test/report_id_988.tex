```latex
\documentclass[12pt, a4paper]{article}

% Preamble: Required Packages
\usepackage[margin=1in]{geometry}
\usepackage{pifont} % For checkmarks and crosses
\usepackage{booktabs} % For professional tables
\usepackage{hyperref} % For clickable links and references
\usepackage{url} % For formatting URLs
\usepackage{seqsplit} % To split long strings in tt font
\usepackage{graphicx}
\usepackage{xcolor}
\usepackage{fancyhdr}

% Document Metadata
\hypersetup{
    colorlinks=true,
    linkcolor=blue,
    filecolor=magenta,      
    urlcolor=cyan,
    pdftitle={Cybersecurity Posture Report},
    pdfauthor={Cybersecurity Analysis Division},
    pdfsubject={Security Assessment},
    pdfkeywords={Cybersecurity, Risk, Assessment},
}

% Header and Footer
\pagestyle{fancy}
\fancyhf{} % clear all header and footer fields
\fancyhead[L]{Cybersecurity Posture Report}
\fancyhead[R]{\textbf{Verve \& Vigor}}
\fancyfoot[C]{\thepage}

% --- DOCUMENT START ---
\begin{document}

\begin{titlepage}
    \centering
    \vspace*{1cm}
    \Huge{\textbf{Cybersecurity Posture Report}}
    \vspace{0.5cm}
    \Large{Prepared for: \textbf{Verve \& Vigor}}
    \vspace{1.5cm}
    \includegraphics[width=0.4\textwidth]{example-image-a} % Placeholder for a logo
    \vfill
    \Large{\textbf{Date of Report:}} \\
    \large{\today}
    \vspace{1cm}
    \large{\textbf{Report ID:}} \\
    \large{CYBER-SEC-REP-2023-001}
    \vspace{2cm}
    \normalsize{\textit{This document contains sensitive information and is intended for the exclusive use of the recipient.}}
\end{titlepage}

\tableofcontents
\newpage

% --- EXECUTIVE SUMMARY ---
\section{Executive Summary}
This report provides a comprehensive analysis of the cybersecurity posture for \textbf{Verve \& Vigor}. The assessment combines a review of organizational security controls, an external network scan, and an evaluation of known risks.

The technical network scan of the target host \texttt{192.168.1.100} revealed a strong security configuration, with no open ports detected. This indicates effective firewalling and a minimal attack surface for this specific asset, which is a significant positive finding.

However, the organizational security control review identified several critical and high-risk gaps. The most pressing concerns are the absence of Multi-Factor Authentication (MFA) for computer logins and access to sensitive data systems. These gaps expose the organization to significant risk from credential theft and unauthorized access. Additionally, the lack of a formal Acceptable Use Policy (AUP) represents a foundational governance weakness.

This report outlines these findings in detail and provides prioritized, actionable recommendations to mitigate the identified risks and strengthen the overall security posture of the organization.

% --- ORGANIZATIONAL INFORMATION ---
\section{Organizational Information}
The following details were provided for the assessment.
\begin{center}
\begin{tabular}{ll}
\toprule
\textbf{Attribute} & \textbf{Value} \\
\midrule
Organization Name & \textbf{Verve \& Vigor} \\
Email Domain      & \texttt{VerveVigor.com} \\
Website Domain    & \url{www.VerveVigor.com} \\
External IP Address & \texttt{97.29.27.72} \\
\bottomrule
\end{tabular}
\end{center}

% --- SECURITY CONTROL REVIEW ---
\section{Security Control Review}
The following table summarizes the organization's responses to a security controls questionnaire. Each response is assessed against industry best practices. Items marked with \ding{55} represent significant gaps in the current security posture.

\begin{center}
\begin{tabular}{p{0.5\linewidth}ccc}
\toprule
\textbf{Control Question} & \textbf{Response} & \textbf{Assessment} \\
\midrule
Do you require MFA to access email? & \ding{51} & Aligned \\
\addlinespace
Do you require MFA to log into computers? & \ding{55} & \textbf{High Risk Gap} \\
\addlinespace
Do you require MFA to access sensitive data systems? & \ding{55} & \textbf{Critical Risk Gap} \\
\addlinespace
Does your organization have an employee acceptable use policy? & \ding{55} & \textbf{Policy Gap} \\
\addlinespace
Does your organization do security awareness training for new employees? & \ding{51} & Aligned \\
\addlinespace
Does your organization do security awareness training for all employees at least once per year? & \ding{51} & Aligned \\
\bottomrule
\end{tabular}
\end{center}
\vspace{1em}
\textbf{Analysis:} The organization has implemented crucial controls such as MFA for email and a consistent security awareness training program. However, the absence of MFA on workstations and sensitive systems, coupled with the lack of an Acceptable Use Policy, creates exploitable vulnerabilities that must be addressed urgently.

% --- TECHNICAL SCAN RESULTS ---
\section{Technical Scan Results}
A network scan was performed to identify open ports and services exposed on the target system.

\begin{itemize}
    \item \textbf{Target IP Address:} \texttt{192.168.1.100}
    \item \textbf{Scan Date:} \today
\end{itemize}

\subsection{Summary of Findings}
The scan results were positive, indicating a strong defensive posture for the targeted host.
\begin{itemize}
    \item \textbf{Host Status:} Up
    \item \textbf{Open Ports:} 0
    \item \textbf{Port State:} All 1000 scanned ports were in a 'closed' state.
\end{itemize}

\textbf{Interpretation:} No open ports were detected on the target system. This suggests that the host is either not running any network-facing services or is protected by a well-configured firewall that blocks all incoming connection attempts. This significantly reduces the external attack surface of this asset. No further technical vulnerabilities were identified for this host.

% --- RISK ASSESSMENT ---
\section{Risk Assessment}
This section synthesizes findings from the security control review, technical scan, and pre-existing risk data. As no pre-existing vulnerabilities were reported, the following risks are derived directly from this assessment.

\begin{center}
\begin{tabular}{p{0.25\linewidth}p{0.15\linewidth}p{0.5\linewidth}}
\toprule
\textbf{Risk Title} & \textbf{Severity} & \textbf{Overview} \\
\midrule
\addlinespace
\textbf{No MFA on Sensitive Data Systems} & \textcolor{red}{\textbf{Critical}} & The absence of MFA on critical systems means that a single compromised password could lead to a catastrophic data breach. This is the most severe risk identified. \\
\addlinespace
\textbf{No MFA on Computer Logins} & \textcolor{orange}{\textbf{High}} & Stolen or weak user credentials can be used to gain direct access to company workstations, providing a foothold for lateral movement, data theft, or ransomware deployment. \\
\addlinespace
\textbf{Absence of Acceptable Use Policy (AUP)} & \textcolor{yellow!80!black}{\textbf{Medium}} & Without a formal AUP, there is no clear guidance for employees on the proper use of company assets. This creates ambiguity and legal/HR challenges in the event of an insider threat or policy violation. \\
\bottomrule
\end{tabular}
\end{center}

% --- RECOMMENDATIONS ---
\section{Recommendations}
The following prioritized actions are recommended to mitigate the identified risks and improve the organization's overall security posture.

\begin{enumerate}
    \item \textbf{[Critical] Implement MFA for Sensitive Systems:}
    \begin{itemize}
        \item \textbf{Action:} Immediately prioritize and deploy a robust Multi-Factor Authentication solution for all systems identified as containing sensitive or critical data. This includes databases, financial applications, and administrative portals.
        \item \textbf{Impact:} Drastically reduces the risk of unauthorized access to the organization's most valuable data assets.
    \end{itemize}
    \vspace{1em}
    \item \textbf{[High] Enforce MFA for All Computer Logins:}
    \begin{itemize}
        \item \textbf{Action:} Roll out MFA for all employee and privileged user logins to company-managed workstations and servers (both on-premise and remote).
        \item \textbf{Impact:} Protects against endpoint compromise resulting from credential theft, a common vector for ransomware and other advanced attacks.
    \end{itemize}
    \vspace{1em}
    \item \textbf{[Medium] Develop and Implement an Acceptable Use Policy:}
    \begin{itemize}
        \item \textbf{Action:} Draft a formal AUP that clearly defines the rules and expectations for using company technology, data, and network resources. This policy should be reviewed by legal/HR and communicated to all employees, who must formally acknowledge it.
        \item \textbf{Impact:} Establishes a clear security baseline for employee behavior, reduces insider risk, and provides a framework for enforcing security standards.
    \end{itemize}
\end{enumerate}

\end{document}
```