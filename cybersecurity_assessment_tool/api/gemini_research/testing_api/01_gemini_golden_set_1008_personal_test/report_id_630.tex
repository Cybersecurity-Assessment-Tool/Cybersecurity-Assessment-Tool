```latex
\documentclass[12pt]{article}

% Preamble: Required Packages
\usepackage[margin=1in]{geometry}
\usepackage{pifont} % For checkmarks and crosses
\usepackage{booktabs} % For professional tables
\usepackage{hyperref} % For clickable links
\usepackage{url} % For formatting URLs
\usepackage{seqsplit} % For splitting long strings in texttt
\usepackage{graphicx}
\usepackage{fancyhdr}
\usepackage{lastpage}
\usepackage{xcolor}

% --- Document Setup ---
\hypersetup{
    colorlinks=true,
    linkcolor=blue,
    filecolor=magenta,      
    urlcolor=cyan,
    pdftitle={Cybersecurity Posture Report},
    pdfpagemode=FullScreen,
}

\pagestyle{fancy}
\fancyhf{}
\lhead{\textbf{Cybersecurity Posture Report}}
\rhead{\textbf{Ember Glow Hospitality}}
\cfoot{Page \thepage\ of \pageref{LastPage}}
\renewcommand{\headrulewidth}{0.4pt}
\renewcommand{\footrulewidth}{0.4pt}

% --- Custom Commands ---
\newcommand{\yes}{\ding{51}}
\newcommand{\no}{\ding{55}}

\begin{document}

% --- Title Page ---
\begin{titlepage}
    \centering
    \vspace*{2cm}
    
    \Huge
    \textbf{Cybersecurity Posture Report}
    
    \vspace{1.5cm}
    
    \Large
    Prepared for: \\
    \vspace{0.5cm}
    \textbf{Ember Glow Hospitality}
    
    \vfill
    
    \large
    Report Date: \today
    
    \vspace{1cm}
    
    \normalsize
    This report contains a comprehensive analysis of the organization's security posture based on provided organizational data, technical network scans, and a review of existing risks. The findings and recommendations herein are intended to help strengthen the organization's defenses against cyber threats.
    
\end{titlepage}

\tableofcontents
\newpage

% --- Section 1: Executive Summary ---
\section{Executive Summary}

This report provides a cybersecurity assessment for \textbf{Ember Glow Hospitality}. The analysis is based on a security controls questionnaire, a limited technical network scan of an internal host, and a review of previously identified risks.

The primary findings indicate significant gaps in fundamental administrative and identity management controls. The lack of Multi-Factor Authentication (MFA) for email and computer access represents a \textbf{critical risk}, exposing the organization to account takeover and unauthorized access. Furthermore, the absence of an employee acceptable use policy and security training for new hires creates a weak security culture and increases the likelihood of human error leading to a security incident.

On a positive note, the technical scan of the target host \texttt{192.168.0.5} showed a minimal attack surface. A previously documented risk concerning an open unencrypted web port (Port 80) appears to have been remediated on this specific host, as the port was found to be closed.

Immediate action should be focused on implementing MFA across all critical systems and establishing foundational security policies and training programs.

% --- Section 2: Organizational Information ---
\section{Organizational Information}

The following details were provided for the assessment.

\begin{tabular}{@{}ll}
    \toprule
    \textbf{Attribute} & \textbf{Value} \\
    \midrule
    Organization Name & \textbf{Ember Glow Hospitality} \\
    Email Domain & \seqsplit{\texttt{EmberGlowHospitality.com}} \\
    Website Domain & \seqsplit{\url{www.EmberGlowHospitality.com}} \\
    External IP Address & \texttt{80.135.253.231} \\
    \bottomrule
\end{tabular}

% --- Section 3: Security Control Review ---
\section{Security Control Review}

A review of the organization's security controls was conducted via a questionnaire. The responses highlight key areas of strength and weakness in the current security posture. "No" answers indicate significant gaps that require immediate attention.

\begin{table}[h!]
\centering
\begin{tabular}{@{}p{0.7\textwidth}cc@{}}
    \toprule
    \textbf{Control Question} & \textbf{Response} & \textbf{Status} \\
    \midrule
    Do you require MFA to access email? & No & \no \\
    Do you require MFA to log into computers? & No & \no \\
    Do you require MFA to access sensitive data systems? & Yes & \yes \\
    Does your organization have an employee acceptable use policy? & No & \no \\
    Does your organization do security awareness training for new employees? & No & \no \\
    Does your organization do security awareness training for all employees at least once per year? & Yes & \yes \\
    \bottomrule
\end{tabular}
\caption{Security Controls Questionnaire Results}
\end{table}

% --- Section 4: Technical Scan Results ---
\section{Technical Scan Results}

A network scan was performed to identify open ports and services on the specified target.

\begin{itemize}
    \item \textbf{Scan Target:} \texttt{192.168.0.5}
    \item \textbf{Scan Date:} Not provided in scan data; report generated on \today.
\end{itemize}

The scan of the target host revealed a very limited attack surface with no open ports detected. This indicates a positive security configuration for this specific device.

\begin{table}[h!]
\centering
\begin{tabular}{@{}llll@{}}
    \toprule
    \textbf{Port} & \textbf{State} & \textbf{Service} & \textbf{Product / Version} \\
    \midrule
    80/tcp & closed & http & Not Applicable \\
    \bottomrule
\end{tabular}
\caption{Nmap Scan Results for \texttt{192.168.0.5}}
\end{table}

\subsection*{Analysis}
The scan results are encouraging for the specific host tested. Notably, the pre-existing risk "Unencrypted Web Server" (Port 80 open) was not found on this target. This suggests that the risk may have been remediated or does not apply to this particular system. Verification across all relevant systems is recommended.

% --- Section 5: Consolidated Risk Assessment ---
\section{Consolidated Risk Assessment}

The following table synthesizes findings from the security questionnaire, technical scan, and pre-existing risk data into a consolidated list of current risks and observations.

\begin{table}[h!]
\centering
\begin{tabular}{@{}p{0.2\textwidth}p{0.6\textwidth}l@{}}
    \toprule
    \textbf{Risk Name} & \textbf{Description} & \textbf{Severity} \\
    \midrule
    \textbf{No MFA on Email} & Lack of Multi-Factor Authentication on email accounts makes them highly susceptible to phishing and credential stuffing attacks, which can lead to data breaches and further network compromise. & \textcolor{red}{\textbf{Critical}} \\
    \addlinespace
    \textbf{No MFA on Endpoints} & Lack of MFA for computer logins allows an attacker with stolen credentials to gain direct access to an endpoint and the corporate network. & \textcolor{red}{\textbf{High}} \\
    \addlinespace
    \textbf{No Acceptable Use Policy} & Without a formal policy, employees are unaware of their responsibilities for protecting company data, leading to inconsistent security practices and increased insider risk. & \textcolor{orange}{Medium} \\
    \addlinespace
    \textbf{No Onboarding Training} & New employees are not trained on security best practices, making them significantly more vulnerable to social engineering and policy violations from their first day. & \textcolor{orange}{Medium} \\
    \addlinespace
    \textbf{Unencrypted Web Server} & \textit{(Observation)} A pre-existing risk noted an open Port 80. Our scan of \texttt{192.168.0.5} found this port to be closed, indicating potential remediation. This should be verified across all production web servers. & Informational \\
    \bottomrule
\end{tabular}
\caption{Summary of Identified Risks}
\end{table}

% --- Section 6: Recommendations ---
\section{Recommendations}

Based on the consolidated risk assessment, the following actions are recommended to improve the security posture of \textbf{Ember Glow Hospitality}. Recommendations are prioritized by severity.

\subsection*{Critical / High Priority}
\begin{enumerate}
    \item \textbf{Implement MFA for Email Access:} Immediately enforce MFA for all user accounts accessing the email system. This is the single most effective control to prevent unauthorized account access.
    \item \textbf{Implement MFA for Endpoint Login:} Deploy MFA for all employees logging into their company-issued computers. This adds a critical layer of defense against credential theft.
\end{enumerate}

\subsection*{Medium Priority}
\begin{enumerate}
    \item \textbf{Develop an Acceptable Use Policy (AUP):} Create and distribute a formal AUP that clearly defines the rules for using company technology and data. All employees must read and acknowledge this policy.
    \item \textbf{Integrate Security Training into Onboarding:} Develop a mandatory security awareness training module for all new hires. This should be a required step in the onboarding process before system access is granted.
\end{enumerate}

\subsection*{Verification and Review}
\begin{enumerate}
    \item \textbf{Confirm Remediation of Port 80:} Conduct a thorough scan of all public-facing assets to confirm that the "Unencrypted Web Server" risk has been fully remediated and that Port 80 is closed or redirects to an encrypted HTTPS connection (Port 443) where required.
\end{enumerate}

\end{document}
```