```latex
\documentclass[12pt]{article}

% Preamble: Required Packages
\usepackage[margin=1in]{geometry}
\usepackage{pifont} % For checkmarks and crosses
\usepackage{booktabs} % For professional tables
\usepackage[hidelinks]{hyperref} % For clickable links without boxes
\usepackage{url} % For URL formatting
\usepackage{seqsplit} % For splitting long strings to prevent overflow
\usepackage[T1]{fontenc}
\usepackage{xcolor}
\usepackage{graphicx}

% Document Information
\title{Cybersecurity Posture Assessment Report \\ \large For: Ironclad Logistics}
\author{Cybersecurity Analyst Group}
\date{\today}

\begin{document}

\maketitle
\thispagestyle{empty}
\newpage

\tableofcontents
\thispagestyle{empty}
\newpage

\setcounter{page}{1}

% ==============================================================================
% Section 1: Executive Overview
% ==============================================================================
\section{Executive Overview}

This report details the findings of a cybersecurity posture assessment for Ironclad Logistics. The assessment combined an analysis of organizational security controls, a technical network scan of the external perimeter, and a review of pre-existing risks.

\paragraph{Key Findings:} The organization demonstrates a strong perimeter defense, as the technical scan of the designated target IP address (\texttt{[Target IP]}) revealed no open ports. This indicates a well-configured firewall or network access control list (ACL). However, significant gaps were identified in internal security controls.

The most critical findings stem from the security questionnaire analysis:
\begin{itemize}
    \item \textbf{Critical Risk:} Multi-Factor Authentication (MFA) is not enforced for accessing sensitive data systems. This exposes critical assets to a high risk of unauthorized access should credentials be compromised.
    \item \textbf{High Risk:} The organization lacks a formal security awareness training program for both new and existing employees. This increases susceptibility to social engineering, phishing attacks, and unintentional insider threats.
\end{itemize}

\paragraph{Overall Posture:} The overall security posture is assessed as \textbf{Moderate}. While the external network perimeter appears secure, the identified gaps in access control and employee security training present significant risks that could be exploited by threat actors. Immediate remediation of these issues is strongly recommended to protect organizational data and systems.

% ==============================================================================
% Section 2: Organizational Information
% ==============================================================================
\section{Organizational Information}

The following details were provided for the assessment.

\begin{tabular}{@{}ll}
    \toprule
    \textbf{Attribute} & \textbf{Value} \\
    \midrule
    Organization Name & Ironclad Logistics \\
    Email Domain & \texttt{IroncladLogistics.com} \\
    Website Domain & \url{www.IroncladLogistics.com} \\
    External IP Address & \texttt{96.148.36.177} \\
    \bottomrule
\end{tabular}

% ==============================================================================
% Section 3: Security Control Review (Questionnaire Analysis)
% ==============================================================================
\section{Security Control Review (Questionnaire Analysis)}

The following table summarizes the organization's responses to the security controls questionnaire. Each response is assessed against industry best practices. A green checkmark (\ding{51}) indicates alignment with best practices, while a red cross (\ding{55}) indicates a significant gap.

\begin{table}[h!]
\centering
\begin{tabular}{@{}p{0.6\textwidth}ccp{0.2\textwidth}@{}}
    \toprule
    \textbf{Control Question} & \textbf{Response} & \textbf{Assessment} \\
    \midrule
    Do you require MFA to access email? & Yes & \ding{51} \\
    Do you require MFA to log into computers? & Yes & \ding{51} \\
    \addlinespace
    \color{red!80!black}Do you require MFA to access sensitive data systems? & \color{red!80!black}No & \color{red!80!black}\ding{55} \\
    \addlinespace
    Does your organization have an employee acceptable use policy? & Yes & \ding{51} \\
    \addlinespace
    \color{red!80!black}Does your organization do security awareness training for new employees? & \color{red!80!black}No & \color{red!80!black}\ding{55} \\
    \addlinespace
    \color{red!80!black}Does your organization do security awareness training for all employees at least once per year? & \color{red!80!black}No & \color{red!80!black}\ding{55} \\
    \bottomrule
\end{tabular}
\caption{Security Control Questionnaire Results.}
\end{table}

% ==============================================================================
% Section 4: Technical Scan Results
% ==============================================================================
\section{Technical Scan Results}

A network scan was performed to identify open ports and exposed services on the organization's external perimeter.

\begin{itemize}
    \item \textbf{Target IP Address:} \texttt{[Target IP]}
    \item \textbf{Scan Date:} Not provided in input data.
\end{itemize}

\subsection{Scan Summary}
\textbf{No open ports were detected on the target host.}

\paragraph{Interpretation:} This result is positive and suggests a strong firewall configuration that denies all unsolicited inbound traffic. This significantly reduces the external attack surface. It may also indicate that the host was offline or that the scan was blocked by an Intrusion Prevention System (IPS). Periodic rescans are recommended to validate this finding.

% ==============================================================================
% Section 5: Consolidated Risk Assessment
% ==============================================================================
\section{Consolidated Risk Assessment}

This section correlates findings from the security control review and technical scan to present a consolidated list of identified risks. No pre-existing vulnerabilities were provided for this assessment.

\begin{table}[h!]
\centering
\begin{tabular}{@{}p{0.2\textwidth}p{0.5\textwidth}l@{}}
    \toprule
    \textbf{Risk Name} & \textbf{Description} & \textbf{Severity} \\
    \midrule
    \addlinespace
    Lack of MFA on Sensitive Systems & The absence of MFA on systems containing sensitive data creates a single point of failure. If an employee's credentials are stolen, an attacker can gain direct access to critical information. & \textbf{\textcolor{red}{Critical}} \\
    \addlinespace
    No Security Awareness Training Program & Without regular training, employees are more likely to fall victim to phishing, malware, and other social engineering attacks, potentially leading to credential theft, data breaches, or ransomware incidents. & \textbf{\textcolor{orange}{High}} \\
    \addlinespace
    \bottomrule
\end{tabular}
\caption{Summary of Identified Risks.}
\end{table}

% ==============================================================================
% Section 6: Recommendations
% ==============================================================================
\section{Recommendations}

The following actionable recommendations are prioritized based on the severity of the associated risks.

\subsection{Priority 1: Critical}
\begin{description}
    \item[Implement MFA for Sensitive Systems:] Immediately deploy a robust Multi-Factor Authentication solution for all systems classified as containing sensitive or critical data. This includes databases, financial applications, and administrative portals. This is the single most effective control to mitigate the risk of unauthorized access.
\end{description}

\subsection{Priority 2: High}
\begin{description}
    \item[Establish a Security Awareness Program:] Develop and implement a comprehensive security awareness training program. This program should include:
    \begin{itemize}
        \item Mandatory onboarding training for all new employees covering topics like phishing, password hygiene, and acceptable use.
        \item Annual refresher training for all staff to reinforce key concepts and address emerging threats.
        \item Periodic simulated phishing campaigns to test and improve employee vigilance.
    \end{itemize}
\end{description}

\subsection{Priority 3: Informational}
\begin{description}
    \item[Conduct Periodic Network Scans:] Continue to perform regular, authenticated and unauthenticated scans of the external and internal networks. This will help ensure that the firewall configuration remains secure and identify any new vulnerabilities that may arise from system changes.
\end{description}

\end{document}
```