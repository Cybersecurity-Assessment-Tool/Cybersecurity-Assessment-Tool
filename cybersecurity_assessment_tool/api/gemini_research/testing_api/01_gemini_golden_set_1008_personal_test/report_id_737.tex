```latex
\documentclass[12pt]{article}

% Preamble: Required Packages
\usepackage[margin=1in]{geometry}
\usepackage{pifont} % For checkmarks and crosses
\usepackage{booktabs} % For professional tables
\usepackage[hidelinks]{hyperref} % For clickable links without boxes
\usepackage{url} % For formatting URLs
\usepackage{seqsplit} % For splitting long strings in tt font
\usepackage{graphicx}
\usepackage{fancyhdr}

% Document Information
\title{Cybersecurity Posture Assessment Report}
\author{Cybersecurity Analyst}
\date{\today}

% Header and Footer
\pagestyle{fancy}
\fancyhf{}
\lhead{\textbf{Silver Leaf Collective} -- Confidential}
\rhead{Page \thepage}
\rfoot{\today}

\begin{document}

\maketitle
\thispagestyle{empty}
\newpage

\tableofcontents
\newpage

% --- 1. Executive Overview ---
\section{Executive Overview}

This report details the findings of a cybersecurity posture assessment conducted for \textbf{Silver Leaf Collective}. The assessment combined a technical network scan, a review of existing risks, and an analysis of organizational security controls based on a questionnaire.

The assessment identified several critical and high-risk vulnerabilities that expose the organization to significant threats, including ransomware, business email compromise (BEC), and unauthorized data access.

Key findings include:
\begin{itemize}
    \item \textbf{Systemic Remote Access Exposure:} The technical scan discovered an open Remote Desktop Protocol (RDP) port on an internal server (\texttt{10.10.10.51}). This finding is highly concerning as it correlates with a pre-existing critical risk of RDP exposure on another server (\texttt{10.10.10.50}), indicating a pattern of insecure configuration.
    \item \textbf{Critical Identity and Access Gaps:} The organization does not enforce Multi-Factor Authentication (MFA) for email access. This is a critical security gap that significantly increases the risk of account takeovers and BEC attacks.
    \item \textbf{Foundational Policy Deficiencies:} The absence of a formal Employee Acceptable Use Policy and a lack of security awareness training for new hires create a weak security culture. This makes the organization more susceptible to social engineering and insider threats.
\end{itemize}

Immediate remediation of these issues is strongly recommended to reduce the organization's attack surface and improve its overall security posture. Actionable recommendations are provided in Section \ref{sec:recommendations}.

% --- 2. Organizational Information ---
\section{Organizational Information}

The following information was provided for the assessment.

\begin{tabular}{@{}ll}
    \toprule
    \textbf{Attribute} & \textbf{Value} \\
    \midrule
    Organization Name & \textbf{Silver Leaf Collective} \\
    Email Domain & \texttt{SilverLeafCollective.com} \\
    Website Domain & \url{www.SilverLeafCollective.com} \\
    Known External IP & \texttt{116.247.9.85} \\
    \bottomrule
\end{tabular}

% --- 3. Security Control Review ---
\section{Security Control Review}

A review of administrative and policy-based security controls was conducted via a questionnaire. The responses reveal significant gaps in foundational security practices. A "No" response indicates a missing control and a potential area of high risk.

\begin{table}[h!]
\centering
\caption{Security Control Questionnaire Analysis}
\begin{tabular}{@{}p{0.8\linewidth}c@{}}
    \toprule
    \textbf{Control Question} & \textbf{Response} \\
    \midrule
    Do you require MFA to access email? & \ding{55} \\
    Do you require MFA to log into computers? & \ding{51} \\
    Do you require MFA to access sensitive data systems? & \ding{51} \\
    Does your organization have an employee acceptable use policy? & \ding{55} \\
    Does your organization do security awareness training for new employees? & \ding{55} \\
    Does your organization do security awareness training for all employees at least once per year? & \ding{51} \\
    \bottomrule
\end{tabular}
\end{table}

\subsection*{Analysis of Control Gaps}
\begin{itemize}
    \item \textbf{No MFA for Email:} This is a critical vulnerability. Email is a primary target for attackers seeking to launch phishing campaigns, commit financial fraud, or gain a foothold in the network.
    \item \textbf{No Acceptable Use Policy (AUP):} The lack of an AUP means there are no formally documented rules for employee use of IT assets, which can lead to unintentional security incidents and policy violations.
    \item \textbf{No Security Training for New Hires:} New employees are not receiving security training upon joining the organization. This is a missed opportunity to establish a strong security culture from day one, leaving them more vulnerable to social engineering attacks.
\end{itemize}

% --- 4. Technical Scan Results ---
\section{Technical Scan Results}

A network scan was performed to identify open ports and services on the target system.

\begin{itemize}
    \item \textbf{Target IP Address:} \texttt{10.10.10.51}
\end{itemize}

The following table summarizes the findings for the host that was found to be online.

\begin{table}[h!]
\centering
\caption{Open Port Scan Results for \texttt{10.10.10.51}}
\begin{tabular}{@{}llll@{}}
    \toprule
    \textbf{Port} & \textbf{State} & \textbf{Service Name} & \textbf{Notes} \\
    \midrule
    3389/tcp & open & \texttt{ms-wbt-server} & Remote Desktop Protocol (RDP). \\
    \bottomrule
\end{tabular}
\end{table}

\subsection*{Analysis of Technical Findings}
The scan identified that port \textbf{3389 (RDP)} is open on the server at \texttt{10.10.10.51}. RDP is a common protocol used for remote administration, but when exposed insecurely, it is one of the most frequent targets for ransomware gangs and other malicious actors. This finding, when correlated with pre-existing risks, points to a systemic control failure.

% --- 5. Correlated Risk Assessment ---
\section{Correlated Risk Assessment}

This section synthesizes findings from the security control review, the technical scan, and pre-existing risk data to provide a holistic view of the organization's risk posture.

\begin{table}[h!]
\centering
\caption{Summary of Identified Risks}
\begin{tabular}{@{}p{0.2\linewidth}p{0.45\linewidth}p{0.15\linewidth}p{0.1\linewidth}@{}}
    \toprule
    \textbf{Risk Title} & \textbf{Description} & \textbf{Affected Assets} & \textbf{Severity} \\
    \midrule
    \textbf{Widespread RDP Exposure} & The scan found open RDP on \texttt{10.10.10.51}. This compounds a known risk on \texttt{10.10.10.50}, indicating a systemic failure to secure remote access protocols. This is a primary vector for ransomware. & Internal Servers & \textbf{Critical} \\
    \addlinespace
    \textbf{Email Account Compromise} & Lack of MFA on email exposes all user accounts to takeover via phishing or credential stuffing. This can lead to data breaches and financial fraud (BEC). & All Email Accounts & \textbf{Critical} \\
    \addlinespace
    \textbf{Weak Security Culture \& Policy} & The absence of an AUP and security training for new hires creates an environment where employees are more likely to fall victim to social engineering or cause accidental breaches. & All Employees, Company Data & \textbf{High} \\
    \bottomrule
\end{tabular}
\end{table}

% --- 6. Recommendations ---
\section{Recommendations}
\label{sec:recommendations}

The following actions are recommended to mitigate the identified risks and strengthen the overall security posture of \textbf{Silver Leaf Collective}.

\subsection{Immediate Actions (To Be Completed within 72 Hours)}
\begin{itemize}
    \item \textbf{Remediate RDP Exposure:}
        \begin{itemize}
            \item Immediately close port 3389 on \texttt{10.10.10.51} and any other internal systems where it is not strictly required and firewall-protected.
            \item If remote access is necessary, implement a Virtual Private Network (VPN) with MFA as the sole method for accessing internal resources like RDP.
            \item Conduct a comprehensive internal network scan to identify and remediate all other instances of exposed RDP.
        \end{itemize}
    \item \textbf{Enable MFA for Email:}
        \begin{itemize}
            \item Immediately enforce MFA for all user accounts accessing the email system (\texttt{SilverLeafCollective.com}). This is the single most effective control against account takeovers.
        \end{itemize}
\end{itemize}

\subsection{Short-Term Actions (To Be Completed within 30 Days)}
\begin{itemize}
    \item \textbf{Develop and Implement an Acceptable Use Policy (AUP):}
        \begin{itemize}
            \item Draft a formal AUP that clearly outlines the rules for using company technology and data.
            \item Require all current and new employees to read and formally acknowledge the policy.
        \end{itemize}
    \item \textbf{Establish New-Hire Security Training:}
        \begin{itemize}
            \item Create a mandatory security awareness training module to be included in the onboarding process for all new employees. This should cover topics such as phishing, password security, and the new AUP.
        \end{itemize}
\end{itemize}

\subsection{Long-Term Strategic Actions}
\begin{itemize}
    \item \textbf{Conduct Network Architecture Review:} Perform a full review of network segmentation and firewall configurations to ensure a defense-in-depth strategy is in place and that the principle of least privilege is enforced.
    \item \textbf{Formalize Vulnerability Management:} Implement a formal vulnerability management program that includes regular scanning, risk-based prioritization, and timely remediation of identified vulnerabilities.
\end{itemize}

\end{document}
```