```latex
\documentclass[12pt, a4paper]{article}

% Preamble: Required Packages
\usepackage[margin=1in]{geometry}
\usepackage{pifont} % For checkmarks and crosses
\usepackage{booktabs} % For professional tables
\usepackage{hyperref} % For clickable links
\usepackage{url} % For URL formatting
\usepackage{seqsplit} % To split long strings in tt font
\usepackage{graphicx}
\usepackage{xcolor}
\usepackage{fancyhdr}

% --- Document Setup ---
\hypersetup{
    colorlinks=true,
    linkcolor=blue,
    filecolor=magenta,      
    urlcolor=cyan,
    pdftitle={Cybersecurity Assessment Report},
    pdfpagemode=FullScreen,
}

% Define colors for severity levels
\definecolor{critical}{HTML}{990000}
\definecolor{high}{HTML}{D13F04}
\definecolor{medium}{HTML}{E89805}
\definecolor{low}{HTML}{3A7D44}

% Header and Footer
\pagestyle{fancy}
\fancyhf{}
\fancyhead[L]{Cybersecurity Assessment Report}
\fancyhead[R]{Pacific Rim Exports}
\fancyfoot[C]{\thepage}

% --- Document Body ---
\begin{document}

% --- Title Page ---
\begin{titlepage}
    \centering
    \vspace*{1cm}
    \includegraphics[width=0.4\textwidth]{example-image-a} % Placeholder for company logo
    \vfill
    \huge\bfseries
    Cybersecurity Assessment Report
    \vspace{1cm}
    \Large\bfseries
    Prepared for: Pacific Rim Exports
    \vspace{2cm}
    \normalsize
    Report Date: \today \\
    \vspace{0.5cm}
    Generated by: Expert Cybersecurity Analyst
\end{titlepage}

\tableofcontents
\newpage

% --- Section 1: Executive Summary ---
\section{Executive Summary}
This report provides a comprehensive cybersecurity assessment for Pacific Rim Exports, based on a combination of technical network scanning, a review of organizational security controls, and an analysis of known risks.

The assessment revealed a mixed security posture. On a positive note, the technical network scan of the target host (\texttt{192.168.1.100}) found no open ports, indicating a strong network hardening and firewall configuration for that specific system.

However, significant and high-risk gaps were identified in the organization's procedural and policy-based security controls. The lack of mandatory Multi-Factor Authentication (MFA) for email and computer access represents a critical vulnerability that could be exploited for account takeover and unauthorized access. Furthermore, the absence of a formal Acceptable Use Policy and the failure to conduct annual security awareness training for all staff create substantial operational and human-centric risks.

Immediate remediation should focus on implementing MFA across all critical systems, developing foundational security policies, and establishing a recurring security training program to mitigate these identified threats.

% --- Section 2: Organizational Information ---
\section{Organizational Information}
The following information was provided for the assessment.

\begin{table}[h!]
\centering
\begin{tabular}{@{}ll@{}}
\toprule
\textbf{Attribute} & \textbf{Value} \\ \midrule
Organization Name & Pacific Rim Exports \\
Email Domain & \texttt{PacificRimExports.com} \\
Website Domain & \url{www.PacificRimExports.com} \\
External IP Address & \seqsplit{\texttt{183.104.75.190}} \\ \bottomrule
\end{tabular}
\caption{Client Organizational Details}
\end{table}

% --- Section 3: Security Control Review ---
\section{Security Control Review}
The following table summarizes the organization's responses to a security controls questionnaire. Items marked with \ding{55} indicate a deviation from security best practices and represent a potential risk.

\begin{table}[h!]
\centering
\begin{tabular}{@{}p{0.6\linewidth} c p{0.25\linewidth}@{}}
\toprule
\textbf{Control Question} & \textbf{Status} & \textbf{Analyst Comment} \\ \midrule
Do you require MFA to access email? & \ding{55} & \textcolor{high}{\textbf{High Risk.}} Email is a primary target for phishing and account takeover. \\
\addlinespace
Do you require MFA to log into computers? & \ding{55} & \textcolor{high}{\textbf{High Risk.}} Lack of MFA on endpoints allows for easier lateral movement after a compromise. \\
\addlinespace
Do you require MFA to access sensitive data systems? & \ding{51} & Positive control in place for sensitive data. \\
\addlinespace
Does your organization have an employee acceptable use policy? & \ding{55} & \textcolor{high}{\textbf{High Risk.}} A foundational governance document is missing. \\
\addlinespace
Does your organization do security awareness training for new employees? & \ding{51} & Good practice for onboarding. \\
\addlinespace
Does your organization do security awareness training for all employees at least once per year? & \ding{55} & \textcolor{high}{\textbf{High Risk.}} Security knowledge degrades over time; continuous training is essential. \\ \bottomrule
\end{tabular}
\caption{Security Controls Questionnaire Analysis}
\end{table}

% --- Section 4: Technical Scan Results ---
\section{Technical Scan Results}
An external network scan was performed to identify exposed services and potential vulnerabilities on the perimeter.

\begin{itemize}
    \item \textbf{Scan Date:} \today
    \item \textbf{Target IP:} \texttt{192.168.1.100}
\end{itemize}

\subsection{Summary of Findings}
The scan of the target host confirmed that the host is online and responsive. However, \textbf{no open TCP or UDP ports were discovered}. All 1000 scanned TCP ports and common UDP ports were found to be in a 'closed' state.

\subsubsection{Conclusion}
This is a positive security finding. It indicates that the target system is either not running any network-facing services or is protected by a well-configured firewall that blocks all unsolicited incoming traffic. This significantly reduces the attack surface of the scanned host.

% --- Section 5: Risk Assessment ---
\section{Risk Assessment}
This section synthesizes findings from the security control review and technical scans. No pre-existing vulnerabilities were reported. The following new risks have been identified based on this assessment.

\begin{table}[h!]
\centering
\begin{tabular}{@{}p{0.25\linewidth} p{0.5\linewidth} p{0.15\linewidth}@{}}
\toprule
\textbf{Risk Name} & \textbf{Overview} & \textbf{Severity} \\ \midrule
\addlinespace
Lack of Multi-Factor Authentication (MFA) & The absence of MFA for email and computer logins exposes the organization to a high risk of credential theft, phishing attacks, and unauthorized access. A compromised password alone could lead to a full system breach. & \textcolor{high}{\textbf{High}} \\
\addlinespace
Absence of Employee Acceptable Use Policy (AUP) & Without a formal AUP, there are no clear guidelines for employees regarding the secure use of company assets, data handling, and internet access. This creates legal and operational risks and can lead to insider threats. & \textcolor{high}{\textbf{High}} \\
\addlinespace
Inadequate Security Awareness Training & While new hires receive training, the lack of an annual refresher for all staff means the workforce is not kept up-to-date on evolving threats like new phishing techniques. This increases the likelihood of a successful social engineering attack. & \textcolor{medium}{\textbf{Medium}} \\
\addlinespace
\bottomrule
\end{tabular}
\caption{Identified Risks and Severity}
\end{table}

% --- Section 6: Recommendations ---
\section{Recommendations}
Based on the findings of this assessment, the following actions are recommended to improve the cybersecurity posture of Pacific Rim Exports.

\subsection{High Priority Recommendations}
\begin{enumerate}
    \item \textbf{Implement Mandatory Multi-Factor Authentication (MFA):}
    \begin{itemize}
        \item \textbf{Action:} Deploy MFA across all user accounts for email access (e.g., via Microsoft 365 or Google Workspace controls) and for all computer/endpoint logins (e.g., via Windows Hello for Business, Duo, or similar solutions).
        \item \textbf{Impact:} Drastically reduces the risk of account compromise from stolen credentials and is the single most effective control to mitigate unauthorized access.
    \end{itemize}
    \item \textbf{Develop and Implement an Acceptable Use Policy (AUP):}
    \begin{itemize}
        \item \textbf{Action:} Create a formal AUP document that all employees must read and sign. This policy should clearly define the rules for using company networks, devices, email, and internet access, as well as guidelines for data handling.
        \item \textbf{Impact:} Establishes a clear security baseline for employee behavior, reduces insider risk, and provides a framework for enforcing security standards.
    \end{itemize}
\end{enumerate}

\subsection{Medium Priority Recommendations}
\begin{enumerate}
    \setcounter{enumi}{2} % Continue numbering from previous list
    \item \textbf{Establish an Annual Security Awareness Training Program:}
    \begin{itemize}
        \item \textbf{Action:} Institute a mandatory security awareness training program that all employees must complete at least once per year. The training should cover current threats such as phishing, ransomware, and social engineering.
        \item \textbf{Impact:} Reinforces security best practices, keeps employees vigilant against new attack vectors, and strengthens the "human firewall."
    \end{itemize}
\end{enumerate}

\subsection{Commendations}
\begin{itemize}
    \item \textbf{Strong Network Hardening:} The scanned host at \texttt{192.168.1.100} demonstrated excellent security hardening, with no exposed services. This configuration should be maintained and used as a standard for other systems where possible.
\end{itemize}

\end{document}
```