```latex
\documentclass[12pt]{article}

% --- PACKAGES ---
\usepackage[margin=1in]{geometry}
\usepackage{pifont}                 % For checkmarks and crosses
\usepackage{booktabs}               % For professional tables
\usepackage{hyperref}               % For clickable links
\usepackage{url}                    % For URL formatting
\usepackage{seqsplit}               % To split long strings in texttt
\usepackage{graphicx}
\usepackage{xcolor}
\usepackage{fancyhdr}               % For headers and footers
\usepackage{lastpage}               % To get the total number of pages

% --- DOCUMENT SETUP ---
\hypersetup{
    colorlinks=true,
    linkcolor=blue,
    filecolor=magenta,      
    urlcolor=cyan,
    pdftitle={Cybersecurity Posture Assessment Report},
    pdfpagemode=FullScreen,
}

% --- HEADER & FOOTER ---
\pagestyle{fancy}
\fancyhf{} % Clear all header and footer fields
\lhead{Cybersecurity Assessment Report}
\rhead{For: Structure \& Form}
\cfoot{Page \thepage\ of \pageref{LastPage}}
\renewcommand{\headrulewidth}{0.4pt}
\renewcommand{\footrulewidth}{0.4pt}

% --- DOCUMENT START ---
\begin{document}

% --- TITLE PAGE ---
\begin{titlepage}
    \centering
    \vspace*{1cm}
    \textbf{\Huge Cybersecurity Posture Assessment Report}
    \vspace{1.5cm}
    
    \textbf{\Large Prepared for:} \\
    \vspace{0.5cm}
    \textbf{\Large Structure \& Form}
    
    \vfill
    
    \textbf{\large Report Date:} \\
    \vspace{0.5cm}
    \large \today
    
    \vspace{1.5cm}
    \textbf{\large Generated by:} \\
    \vspace{0.5cm}
    \large Cybersecurity Analysis Division
\end{titlepage}

\tableofcontents
\newpage

% --- EXECUTIVE SUMMARY ---
\section{Executive Summary}
This report provides a comprehensive cybersecurity posture assessment for \textbf{Structure \& Form}, based on an analysis of organizational data, technical network scans, and a review of pre-existing risk documentation. The assessment synthesizes these data points to provide a holistic view of the organization's security landscape.

The key findings indicate that while some security controls are in place, significant gaps exist in fundamental areas of endpoint security and corporate policy. The two most critical risks identified are the lack of Multi-Factor Authentication (MFA) for computer logins and the absence of a formal Employee Acceptable Use Policy. These represent substantial vulnerabilities that could be exploited by threat actors.

A technical scan of the internal host \texttt{192.168.0.5} did not reveal any open ports, which is a positive finding for that specific asset. However, this result conflicts with a pre-existing documented risk concerning an unencrypted web server on Port 80. This discrepancy requires further investigation to determine if the risk has been remediated or applies to a different asset within the organization's network.

This report concludes with prioritized, actionable recommendations to mitigate the identified risks and strengthen the overall security posture of \textbf{Structure \& Form}.

% --- ORGANIZATIONAL INFORMATION ---
\section{Organizational Information}
The following details were provided for the assessment. This information helps establish the context and scope of the review.

\begin{tabular}{@{}ll}
    \toprule
    \textbf{Attribute} & \textbf{Value} \\
    \midrule
    Organization Name & \textbf{Structure \& Form} \\
    Email Domain & \texttt{StructureForm.net} \\
    Website Domain & \url{www.StructureForm.net} \\
    Known External IP & \texttt{67.117.36.97} \\
    \bottomrule
\end{tabular}

% --- SECURITY CONTROL REVIEW ---
\section{Security Control Review}
A review of the organization's security controls was conducted based on a standardized questionnaire. The responses highlight both strengths and critical weaknesses in the current security framework.

\begin{tabular}{@{}p{0.6\linewidth}cp{0.2\linewidth}@{}}
    \toprule
    \textbf{Control Question} & \textbf{Response} & \textbf{Assessment} \\
    \midrule
    Do you require MFA to access email? & \ding{51} Yes & Control in Place \\
    \addlinespace
    Do you require MFA to log into computers? & \textbf{\color{red}\ding{55} No} & \textbf{Critical Gap} \\
    \addlinespace
    Do you require MFA to access sensitive data systems? & \ding{51} Yes & Control in Place \\
    \addlinespace
    Does your organization have an employee acceptable use policy? & \textbf{\color{red}\ding{55} No} & \textbf{High Risk} \\
    \addlinespace
    Does your organization do security awareness training for new employees? & \ding{51} Yes & Control in Place \\
    \addlinespace
    Does your organization do security awareness training for all employees at least once per year? & \ding{51} Yes & Control in Place \\
    \bottomrule
\end{tabular}

\subsection*{Analysis of Control Gaps}
\begin{itemize}
    \item \textbf{No MFA on Computers:} The absence of MFA on workstations is a critical vulnerability. Compromised credentials could grant an attacker direct access to an endpoint, bypassing perimeter defenses. This significantly increases the risk of ransomware and data breaches.
    \item \textbf{No Acceptable Use Policy (AUP):} Lacking an AUP creates ambiguity regarding the proper use of company assets and data. It weakens the organization's ability to enforce security standards and manage insider risk, both accidental and malicious.
\end{itemize}

% --- TECHNICAL SCAN RESULTS ---
\section{Technical Scan Results}
A network scan was performed to identify open ports and exposed services on the specified target system.

\begin{itemize}
    \item \textbf{Target IP Address:} \texttt{192.168.0.5}
    \item \textbf{Scan Date:} \today
\end{itemize}

The scan results for the target host are summarized below.

\begin{tabular}{@{}lllll@{}}
    \toprule
    \textbf{Port} & \textbf{State} & \textbf{Service} & \textbf{Product} & \textbf{Version} \\
    \midrule
    80/tcp & closed & http & N/A & N/A \\
    \bottomrule
\end{tabular}

\subsection*{Analysis of Technical Findings}
The scan of host \texttt{192.168.0.5} indicates that it is not exposing any services on the ports checked. The finding that port 80 is closed is a positive security posture for this specific device.

However, this result is in direct conflict with a pre-existing risk documented in Input 3, which states that "Port 80 is open." This suggests one of the following possibilities:
\begin{enumerate}
    \item The pre-existing risk has been remediated since it was last documented.
    \item The pre-existing risk applies to a different asset, such as the external IP \texttt{67.117.36.97}, and not the scanned internal host.
    \item The original risk assessment was inaccurate.
\end{enumerate}
Further investigation is required to validate and locate the source of the "Unencrypted Web Server" risk.

% --- CONSOLIDATED RISK ASSESSMENT ---
\section{Consolidated Risk Assessment}
The following table summarizes the key risks identified through the correlation of the security questionnaire, technical scans, and existing risk documentation.

\begin{tabular}{@{}lp{0.4\linewidth}p{0.3\linewidth}l@{}}
    \toprule
    \textbf{ID} & \textbf{Risk Name} & \textbf{Description} & \textbf{Severity} \\
    \midrule
    ORG-001 & Lack of Endpoint MFA & User computers are accessible with only a password, exposing them to credential stuffing and phishing attacks. & \textbf{Critical} \\
    \addlinespace
    ORG-002 & No Acceptable Use Policy & Absence of a formal policy governing the use of company IT assets, leading to increased insider risk. & \textbf{High} \\
    \addlinespace
    NET-001 & Unencrypted Web Server & A web server is allegedly accessible via unencrypted HTTP (Port 80). \textit{Note: Not confirmed by the recent scan of \texttt{192.168.0.5}.} & Medium \\
    \bottomrule
\end{tabular}

% --- RECOMMENDATIONS ---
\section{Recommendations}
The following actions are recommended to mitigate the identified risks and improve the overall security posture of \textbf{Structure \& Form}.

\subsection*{Immediate Priorities (Critical \& High Risks)}
\begin{description}
    \item[Remediate ORG-001 (Lack of Endpoint MFA):]
        \begin{itemize}
            \item \textbf{Action:} Implement and enforce mandatory Multi-Factor Authentication (MFA) for all employee computer and laptop logins.
            \item \textbf{Guidance:} Leverage built-in solutions like Windows Hello for Business or third-party identity providers (e.g., Duo, Okta) to add a second factor of authentication (e.g., authenticator app, hardware token, or biometric scan).
        \end{itemize}

    \item[Remediate ORG-002 (No Acceptable Use Policy):]
        \begin{itemize}
            \item \textbf{Action:} Develop, approve, and implement a formal Acceptable Use Policy (AUP) for all employees and contractors.
            \item \textbf{Guidance:} The policy should clearly define rules for handling data, using company equipment, accessing the internet, and reporting security incidents. Ensure all personnel read and acknowledge the policy upon onboarding and annually thereafter. Templates from organizations like the SANS Institute can serve as a valuable starting point.
        \end{itemize}
\end{description}

\subsection*{Further Investigation}
\begin{description}
    \item[Investigate NET-001 (Unencrypted Web Server):]
        \begin{itemize}
            \item \textbf{Action:} Conduct a thorough investigation to validate the "Unencrypted Web Server" risk.
            \item \textbf{Guidance:} Perform external port scans against the public IP address (\texttt{67.117.36.97}) and internal scans against other critical servers. If an open Port 80 is discovered on any production system, immediately implement one of the following:
            \begin{enumerate}
                \item Close the port if it is not required for business operations.
                \item Implement a permanent (301) redirect from HTTP to HTTPS (Port 443) and ensure a valid, up-to-date TLS/SSL certificate is in place.
            \end{enumerate}
        \end{itemize}
\end{description}

\end{document}
```