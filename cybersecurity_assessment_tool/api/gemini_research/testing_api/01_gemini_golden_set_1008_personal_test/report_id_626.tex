```latex
\documentclass[12pt, a4paper]{article}

% Preamble: Required Packages
\usepackage[margin=1in]{geometry}
\usepackage{pifont} % For checkmarks and crosses
\usepackage{booktabs} % For professional-looking tables
\usepackage{hyperref} % For clickable links
\usepackage{url} % For formatting URLs
\usepackage{seqsplit} % For splitting long strings in tt font
\usepackage{graphicx}
\usepackage{fancyhdr}
\usepackage{lastpage}

% Document Metadata
\title{Cybersecurity Assessment Report}
\author{Cybersecurity Analysis Division}
\date{\today}

% Hyperref Setup
\hypersetup{
    colorlinks=true,
    linkcolor=black,
    urlcolor=blue,
    pdftitle={Cybersecurity Assessment Report},
    pdfauthor={Cybersecurity Analysis Division},
    pdfsubject={Security Assessment},
    pdfkeywords={Security, Assessment, Report, Vulnerability}
}

% Header and Footer
\pagestyle{fancy}
\fancyhf{}
\lhead{Confidential Security Report}
\rhead{Apex Legends Group}
\cfoot{Page \thepage\ of \pageref{LastPage}}
\renewcommand{\headrulewidth}{0.4pt}
\renewcommand{\footrulewidth}{0.4pt}

\begin{document}

\maketitle
\thispagestyle{empty}
\newpage

\tableofcontents
\newpage

\section{Executive Summary}

This report details the findings of a cybersecurity assessment conducted for \textbf{Apex Legends Group}. The assessment combined a review of organizational security controls, an external network scan, and an analysis of pre-existing risks.

Several high-impact vulnerabilities and security gaps were identified that expose the organization to significant threats, including unauthorized access, data breach, and potential system compromise. The most critical findings are:

\begin{itemize}
    \item \textbf{Critically Vulnerable FTP Server:} An externally accessible FTP server was found running \texttt{vsftpd 2.3.4}, a version with a known remote code execution backdoor (CVE-2011-2523). The server also permits anonymous login, allowing unauthenticated access to files.
    \item \textbf{Lack of Multi-Factor Authentication (MFA):} MFA is not enforced for accessing sensitive data systems. This represents a critical failure in access control, making these systems highly susceptible to compromise via stolen credentials.
    \item \textbf{Inadequate Security Training:} The organization does not provide annual security awareness training for all employees, increasing susceptibility to phishing and social engineering attacks.
    \item \textbf{Outdated Operating Systems:} A known risk of workstations running the unsupported Windows 7 operating system persists, leaving them vulnerable to exploitation.
\end{itemize}

Immediate remediation of the vulnerable FTP server and implementation of MFA on sensitive systems are strongly recommended to mitigate the most severe risks.

\section{Organizational Information}

The following information was provided by the client and used as a baseline for this assessment.

\begin{table}[h!]
\centering
\caption{Client Organizational Data}
\begin{tabular}{@{}ll@{}}
\toprule
\textbf{Attribute} & \textbf{Value} \\ \midrule
Organization Name & \textbf{Apex Legends Group} \\
Email Domain & \texttt{ApexLegendsGroup.org} \\
Website Domain & \url{http://www.ApexLegendsGroup.org} \\
External IP Address & \texttt{103.69.24.105} \\ \bottomrule
\end{tabular}
\end{table}

\section{Security Control Review}

A review of internal security controls was conducted based on a standardized questionnaire. The responses highlight significant gaps in the organization's security posture. A summary of findings is presented in Table 2.

\begin{table}[h!]
\centering
\caption{Security Control Questionnaire Analysis}
\begin{tabular}{@{}p{0.6\textwidth}cc@{}}
\toprule
\textbf{Control Question} & \textbf{Response} & \textbf{Assessment} \\ \midrule
Do you require MFA to access email? & Yes & \ding{51} \\
Do you require MFA to log into computers? & Yes & \ding{51} \\
\textbf{Do you require MFA to access sensitive data systems?} & \textbf{No} & \textbf{\ding{55} Critical Gap} \\
Does your organization have an employee acceptable use policy? & Yes & \ding{51} \\
Does your organization do security awareness training for new employees? & Yes & \ding{51} \\
\textbf{Does your organization do security awareness training for all employees at least once per year?} & \textbf{No} & \textbf{\ding{55} High Risk} \\ \bottomrule
\end{tabular}
\end{table}

The lack of MFA on sensitive systems and the absence of annual security training are critical deficiencies that must be addressed to protect organizational assets and data.

\section{Technical Scan Results}

A network scan was performed against the target IP address \texttt{10.0.0.15}. The scan identified one open port with a critically vulnerable service.

\begin{table}[h!]
\centering
\caption{Open Port Analysis for Target: \texttt{10.0.0.15}}
\begin{tabular}{@{}lllll@{}}
\toprule
\textbf{Port} & \textbf{State} & \textbf{Service} & \textbf{Product / Version} & \textbf{Notes} \\ \midrule
21/tcp & Open & ftp & vsftpd 2.3.4 & \begin{tabular}[c]{@{}l@{}}Vulnerable version (CVE-2011-2523).\\ Anonymous FTP login allowed.\end{tabular} \\ \bottomrule
\end{tabular}
\end{table}

\subsection*{Finding Details: Vulnerable FTP Service}
The FTP service is running \texttt{vsftpd version 2.3.4}. This specific version is widely known to contain an intentional backdoor, assigned \textbf{CVE-2011-2523}. An attacker can gain a command shell on the underlying server by sending a specific sequence of characters as the username. Compounding this issue, the server is configured to allow anonymous logins, making it trivial for an attacker to connect and attempt this exploit. This vulnerability presents a direct path to system compromise.

\section{Consolidated Risk Assessment}

The following table synthesizes findings from the security control review, technical scan, and pre-existing risk data into a prioritized list.

\begin{table}[h!]
\centering
\caption{Summary of Identified Risks}
\begin{tabular}{@{}p{0.2\textwidth}p{0.55\textwidth}l@{}}
\toprule
\textbf{Risk Title} & \textbf{Description} & \textbf{Severity} \\ \midrule
\textbf{Vulnerable FTP Server (CVE-2011-2523)} & A public-facing FTP server is running a version with a known backdoor, allowing for remote code execution. Anonymous login is enabled. & \textbf{Critical} \\
\addlinespace
\textbf{No MFA on Sensitive Systems} & Lack of MFA on critical systems dramatically increases the risk of unauthorized access via compromised credentials. & \textbf{Critical} \\
\addlinespace
\textbf{Inadequate Security Training} & The absence of annual training leaves employees ill-equipped to recognize and report modern phishing and social engineering attacks. & \textbf{High} \\
\addlinespace
\textbf{Outdated Windows Policy} & Pre-existing risk of workstations running Windows 7, which is end-of-life and no longer receives security updates from Microsoft. & \textbf{Medium} \\ \bottomrule
\end{tabular}
\end{table}

\section{Recommendations}

Based on the consolidated risk assessment, the following remediation actions are recommended, prioritized by severity.

\subsection*{Immediate Actions (Critical Risks)}

\begin{enumerate}
    \item \textbf{Remediate Vulnerable FTP Server:}
    \begin{itemize}
        \item \textbf{Option A (Preferred):} Decommission the FTP server at \texttt{10.0.0.15} immediately if it is not business-critical.
        \item \textbf{Option B:} If the service is required, take it offline immediately, upgrade \texttt{vsftpd} to the latest stable version, disable anonymous access, and implement a strong password policy. The server should be placed behind a firewall with strict access control lists (ACLs) allowing traffic only from trusted IP addresses.
    \end{itemize}
    \item \textbf{Implement MFA on Sensitive Systems:}
    \begin{itemize}
        \item Immediately begin the process of deploying a robust MFA solution for all access to systems identified as containing sensitive or critical data. This is the single most effective control to prevent unauthorized access from compromised credentials.
    \end{itemize}
\end{enumerate}

\subsection*{Near-Term Actions (High \& Medium Risks)}

\begin{enumerate}
    \setcounter{enumi}{2}
    \item \textbf{Establish Annual Security Awareness Training:}
    \begin{itemize}
        \item Procure and implement a mandatory annual security awareness training program for all employees. The training should cover phishing, password hygiene, acceptable use, and incident reporting.
    \end{itemize}
    \item \textbf{Accelerate OS Upgrade Plan:}
    \begin{itemize}
        \item Continue and prioritize the existing plan to upgrade all workstations from Windows 7 to a currently supported operating system (e.g., Windows 10/11) to ensure they receive critical security patches.
    \end{itemize}
\end{enumerate}

\end{document}
```