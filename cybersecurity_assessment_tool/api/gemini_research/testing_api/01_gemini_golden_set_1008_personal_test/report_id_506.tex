```latex
\documentclass[12pt]{article}

% Preamble: Required Packages
\usepackage[margin=1in]{geometry}
\usepackage{pifont}                 % For symbols like checkmarks (\ding{51}) and crosses (\ding{55})
\usepackage{booktabs}               % For professional-looking tables (\toprule, \midrule, \bottomrule)
\usepackage{hyperref}               % For creating hyperlinks in the document
\usepackage{url}                    % For typesetting URLs
\usepackage{seqsplit}               % To split long strings without spaces
\usepackage{graphicx}
\usepackage{xcolor}
\usepackage{fancyhdr}

% Document Styling
\hypersetup{
    colorlinks=true,
    linkcolor=blue,
    filecolor=magenta,      
    urlcolor=cyan,
}
\pagestyle{fancy}
\fancyhf{}
\fancyhead[L]{\textbf{Cybersecurity Posture Assessment}}
\fancyfoot[C]{\thepage}
\renewcommand{\headrulewidth}{0.4pt}
\renewcommand{\footrulewidth}{0.4pt}

% Define colors for risk levels
\definecolor{criticalrisk}{HTML}{D7263D}
\definecolor{highrisk}{HTML}{F46036}
\definecolor{mediumrisk}{HTML}{F9A620}
\definecolor{bestpractice}{HTML}{2E7D32}

\begin{document}

% --- Title Page ---
\begin{titlepage}
    \centering
    \vspace*{1cm}
    \Huge \textbf{Cybersecurity Posture Assessment Report}
    \vspace{1.5cm}
    \Large \textbf{Prepared for:} \\
    \vspace{0.5cm}
    \huge True Grit
    \vfill
    \large \textbf{Date of Report:} \\
    \vspace{0.5cm}
    \Large \today
    \vspace{1.5cm}
    \large \textbf{Generated By:} \\
    \vspace{0.5cm}
    \Large Cybersecurity Analysis Division
\end{titlepage}

\tableofcontents
\newpage

% --- Section 1: Executive Overview ---
\section{Executive Overview}

This report provides a comprehensive cybersecurity posture assessment for \textbf{True Grit}, based on a correlation of network scan data, organizational security control responses, and a review of pre-existing risks.

The analysis reveals several \textbf{critical vulnerabilities} that expose the organization to significant threats, including data breaches, ransomware attacks, and unauthorized system access. The most severe findings include:

\begin{itemize}
    \item \textbf{Exposed Vulnerable Service:} A public-facing FTP server (\texttt{10.0.0.15}) is running a critically outdated version of \texttt{vsftpd} (2.3.4), which is known to contain a backdoor vulnerability (\textbf{CVE-2011-2523}). This service also permits anonymous, unauthenticated access.
    \item \textbf{Systemic Lack of Multi-Factor Authentication (MFA):} MFA is not enforced for email, computer logins, or access to sensitive data systems. This drastically increases the risk of account compromise and lateral movement within the network.
    \item \textbf{Inadequate Security Training:} While new employees receive security training, there is no annual refresher program, leading to a degradation of security awareness over time.
    \item \textbf{Outdated Operating Systems:} Pre-existing risks confirm the use of Windows 7, an end-of-life operating system that no longer receives security updates.
\end{itemize}

The combination of these issues places the organization in a high-risk category. Immediate and decisive action is required to remediate these findings and strengthen the overall security posture. This report outlines specific, actionable recommendations to address each identified risk.

% --- Section 2: Organizational Information ---
\section{Organizational Information}

The following details were provided for the assessment.

\begin{tabular}{@{}ll}
    \toprule
    \textbf{Attribute} & \textbf{Value} \\
    \midrule
    Organization Name & True Grit \\
    Email Domain & \texttt{TrueGrit.com} \\
    Website Domain & \url{www.TrueGrit.com} \\
    External IP & \texttt{159.92.202.228} \\
    \bottomrule
\end{tabular}

% --- Section 3: Security Control Review ---
\section{Security Control Review}

A review of the organization's security controls via questionnaire highlighted significant gaps in access control and employee security awareness. A "No" response indicates a deviation from established cybersecurity best practices.

\begin{table}[h!]
\centering
\caption{Security Control Questionnaire Analysis}
\begin{tabular}{p{8cm} c l}
    \toprule
    \textbf{Security Control Question} & \textbf{Response} & \textbf{Status} \\
    \midrule
    Do you require MFA to access email? & \ding{55} & \textcolor{criticalrisk}{\textbf{Critical Gap}} \\
    Do you require MFA to log into computers? & \ding{55} & \textcolor{highrisk}{\textbf{High Risk}} \\
    Do you require MFA to access sensitive data systems? & \ding{55} & \textcolor{criticalrisk}{\textbf{Critical Gap}} \\
    Does your organization have an employee acceptable use policy? & \ding{51} & \textcolor{bestpractice}{Best Practice Met} \\
    Does your organization do security awareness training for new employees? & \ding{51} & \textcolor{bestpractice}{Best Practice Met} \\
    Does your organization do security awareness training for all employees at least once per year? & \ding{55} & \textcolor{highrisk}{\textbf{High Risk}} \\
    \bottomrule
\end{tabular}
\end{table}

% --- Section 4: Technical Scan Results ---
\section{Technical Scan Results}

An Nmap scan was conducted to identify open ports and exposed services on the target system. The scan revealed a critically vulnerable service.

\begin{itemize}
    \item \textbf{Target IP Address:} \texttt{10.0.0.15}
\end{itemize}

\begin{table}[h!]
\centering
\caption{Open Port Analysis}
\begin{tabular}{lllll}
    \toprule
    \textbf{Port} & \textbf{State} & \textbf{Service} & \textbf{Version} & \textbf{Notes} \\
    \midrule
    21/tcp & Open & ftp & vsftpd 2.3.4 & \begin{tabular}{@{}l@{}}Anonymous FTP login allowed.\\ \textbf{\textcolor{criticalrisk}{Critical Vulnerability (CVE-2011-2523)}}.\end{tabular} \\
    \bottomrule
\end{tabular}
\end{table}

\subsection*{Finding Detail: vsftpd 2.3.4 Backdoor (CVE-2011-2523)}
The version of \texttt{vsftpd} detected on port 21 is known to contain a malicious backdoor. An attacker can gain a command shell on the system by sending a specific sequence of characters as the username. This provides direct, unauthorized access to the server. Combined with anonymous FTP access, this presents an immediate and severe threat.

% --- Section 5: Consolidated Risk Assessment ---
\section{Consolidated Risk Assessment}

The following table synthesizes findings from the network scan, security control review, and pre-existing risk data into a prioritized list.

\begin{table}[h!]
\centering
\caption{Summary of Identified Risks}
\begin{tabular}{p{4cm} p{1.5cm} p{7cm}}
    \toprule
    \textbf{Risk Name} & \textbf{Severity} & \textbf{Overview} \\
    \midrule
    FTP Server Backdoor (CVE-2011-2523) & \textcolor{criticalrisk}{\textbf{Critical}} & The FTP server at \texttt{10.0.0.15} is running a version of vsftpd with a known remote command execution backdoor. \\
    \addlinespace
    Anonymous FTP Access & \textcolor{criticalrisk}{\textbf{Critical}} & The FTP server allows unauthenticated access, which could lead to data exfiltration or the upload of malicious files. \\
    \addlinespace
    Widespread Lack of MFA & \textcolor{criticalrisk}{\textbf{Critical}} & Absence of MFA on email, endpoints, and sensitive systems makes credential-based attacks trivial to execute. \\
    \addlinespace
    Lack of Annual Security Awareness Training & \textcolor{highrisk}{\textbf{High}} & Without regular refreshers, employees are more susceptible to phishing and social engineering, which are primary vectors for credential theft. \\
    \addlinespace
    Outdated Windows 7 Policy & \textcolor{mediumrisk}{\textbf{Medium}} & The use of an end-of-life OS (Windows 7) means workstations are not receiving security patches for newly discovered vulnerabilities. \\
    \bottomrule
\end{tabular}
\end{table}

% --- Section 6: Recommendations ---
\section{Recommendations}

The following actions are recommended to mitigate the identified risks. They are prioritized based on severity and potential impact.

\subsection{Immediate Actions (Critical Priority)}
\begin{enumerate}
    \item \textbf{Remediate Vulnerable FTP Server:} Immediately take the server at \texttt{10.0.0.15} offline.
    \begin{itemize}
        \item If the FTP service is not essential, it should be decommissioned permanently.
        \item If it is required, the server must be rebuilt with a patched version of \texttt{vsftpd} (version 3.0.0 or later) and reconfigured to \textbf{disable anonymous access}. Secure alternatives like SFTP (SSH File Transfer Protocol) should be implemented instead.
    \end{itemize}
    \item \textbf{Deploy Multi-Factor Authentication (MFA):} Begin an emergency rollout of MFA across the organization.
    \begin{itemize}
        \item \textbf{Phase 1:} Enforce MFA for all email accounts, VPN/remote access, and administrator accounts.
        \item \textbf{Phase 2:} Expand MFA to all sensitive data systems and applications.
        \item \textbf{Phase 3:} Implement MFA for all standard user computer logins.
    \end{itemize}
\end{enumerate}

\subsection{High-Priority Actions}
\begin{enumerate}
    \item \textbf{Establish Annual Security Training Program:} Procure and implement a mandatory annual security awareness training program for all employees. This program should cover modern threats such as phishing, ransomware, and business email compromise.
\end{enumerate}

\subsection{Medium-Priority Actions}
\begin{enumerate}
    \item \textbf{Accelerate Operating System Upgrades:} Expedite the project to upgrade all workstations running Windows 7 to a modern, supported operating system such as Windows 10 or Windows 11. Create a firm deadline for decommissioning all end-of-life systems.
\end{enumerate}

\end{document}
```