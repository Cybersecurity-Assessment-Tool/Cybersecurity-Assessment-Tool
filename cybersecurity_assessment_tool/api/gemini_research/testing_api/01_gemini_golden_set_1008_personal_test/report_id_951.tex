```latex
\documentclass[12pt]{article}

% Preamble: Required Packages
\usepackage[margin=1in]{geometry}
\usepackage{pifont} % For check and cross marks
\usepackage{booktabs} % For professional tables
\usepackage{hyperref} % For clickable links
\usepackage{url} % For formatting URLs
\usepackage{seqsplit} % For splitting long strings to prevent overflow
\usepackage{graphicx} % For potential logos (not used here, but good practice)
\usepackage[utf8]{inputenc}

% Document Metadata
\title{Cybersecurity Posture Assessment Report}
\author{Cybersecurity Analysis Division}
\date{\today}

\begin{document}

\maketitle
\thispagestyle{empty}
\newpage
\tableofcontents
\newpage

% --- 1. Executive Summary ---
\section{Executive Summary}

This report provides a comprehensive cybersecurity posture assessment for \textbf{Vivid Vision}. The analysis is based on the correlation of three data sources: an external network scan, a security controls questionnaire, and a list of pre-existing known risks.

The assessment reveals several critical and high-risk findings that require immediate attention. The primary areas of concern are:

\begin{itemize}
    \item \textbf{Critical Infrastructure Exposure:} A MySQL database server (\texttt{172.16.50.20}) is directly exposed to the network on port 3306. The running software, MySQL version 5.7.33, is End-of-Life (EOL) and no longer receives security updates, significantly elevating the risk of compromise. This finding was confirmed by both the technical scan and pre-existing risk data.
    \item \textbf{Critical Gaps in Identity and Access Management:} Multi-Factor Authentication (MFA) is not enforced for employee email or computer logins. This represents a fundamental weakness in the organization's defense against common attacks like phishing and credential theft.
    \item \textbf{High-Risk Gaps in Security Awareness:} The organization lacks a consistent security awareness training program for both new and existing employees. This increases susceptibility to social engineering and other human-targeted attacks.
\end{itemize}

Immediate remediation of the exposed database and the implementation of MFA are strongly recommended to mitigate the most severe risks.

% --- 2. Organizational Information ---
\section{Organizational Information}

The following details were provided for the assessment.

\begin{itemize}
    \item \textbf{Organization Name:} \textbf{Vivid Vision}
    \item \textbf{Primary Email Domain:} \texttt{VividVision.net}
    \item \textbf{Primary Website Domain:} \url{www.VividVision.net}
    \item \textbf{Known External IP Address:} \texttt{9.239.175.213}
\end{itemize}

% --- 3. Security Control Review ---
\section{Security Control Review}

A review of the organization's security controls was conducted via a questionnaire. The responses highlight significant gaps in foundational security practices. A "No" response indicates a missing control and a potential area of high risk.

\begin{table}[h!]
\centering
\caption{Security Controls Questionnaire Results}
\begin{tabular}{p{0.7\linewidth} c c}
\toprule
\textbf{Control Question} & \textbf{Response} & \textbf{Status} \\
\midrule
Do you require MFA to access email? & No & \ding{55} \\
Do you require MFA to log into computers? & No & \ding{55} \\
Do you require MFA to access sensitive data systems? & Yes & \ding{51} \\
Does your organization have an employee acceptable use policy? & Yes & \ding{51} \\
Does your organization do security awareness training for new employees? & No & \ding{55} \\
Does your organization do security awareness training for all employees at least once per year? & No & \ding{55} \\
\bottomrule
\end{tabular}
\end{table}

\subsection*{Analysis}
The lack of MFA for email and computer access is a critical vulnerability. Email is a primary target for phishing attacks, and compromising an account can lead to widespread system access. Similarly, the absence of a formal security awareness training program leaves the organization highly vulnerable to social engineering attacks, which are often the initial vector for major security breaches.

% --- 4. Technical Scan Results ---
\section{Technical Scan Results}

A network scan was performed to identify open ports and exposed services.

\begin{itemize}
    \item \textbf{Target IP Address:} \texttt{172.16.50.20}
    \item \textbf{Scan Date:} Not specified in scan data.
\end{itemize}

The scan identified the following open port:

\begin{table}[h!]
\centering
\caption{Open Ports and Services on \texttt{172.16.50.20}}
\begin{tabular}{l l l l l}
\toprule
\textbf{Port} & \textbf{State} & \textbf{Service} & \textbf{Product} & \textbf{Version} \\
\midrule
3306/tcp & open & mysql & MySQL & 5.7.33 \\
\bottomrule
\end{tabular}
\end{table}

\subsection*{Analysis}
The technical scan confirms the pre-existing risk of database exposure. Port 3306 is the default port for MySQL, and its exposure to the network makes it a target for brute-force attacks, credential stuffing, and exploitation of known vulnerabilities.

Crucially, \textbf{MySQL version 5.7 reached its official End-of-Life (EOL) in October 2023}. This means it no longer receives security patches from the vendor. Any new vulnerabilities discovered in this version will remain unpatched, making this exposed service an extremely high-risk target for attackers.

% --- 5. Consolidated Risk Assessment ---
\section{Consolidated Risk Assessment}

The following table synthesizes findings from all data sources into a prioritized list of organizational risks.

\begin{table}[h!]
\centering
\caption{Summary of Identified Risks}
\begin{tabular}{p{0.15\linewidth} p{0.6\linewidth} l}
\toprule
\textbf{Risk Title} & \textbf{Description} & \textbf{Severity} \\
\midrule
\textbf{Exposed \& EOL Database Service} & Port 3306 (MySQL) is open to the network, and the service version (5.7.33) is End-of-Life, meaning it is unsupported and unpatched against new vulnerabilities. & \textbf{Critical} \\
\addlinespace
\textbf{Lack of Essential MFA} & Multi-Factor Authentication is not enforced on core systems like email and workstations, drastically increasing the risk of account compromise. & \textbf{Critical} \\
\addlinespace
\textbf{Insufficient Security Awareness} & The absence of a formal and recurring security training program for employees increases the likelihood of successful phishing and social engineering attacks. & \textbf{High} \\
\bottomrule
\end{tabular}
\end{table}

% --- 6. Recommendations ---
\section{Recommendations}

The following actionable recommendations are provided to address the identified risks. They are prioritized based on severity and potential impact.

\subsection{Immediate Priority (Remediate within 72 hours)}
\begin{enumerate}
    \item \textbf{Restrict Database Access:} Implement strict firewall rules to block all public access to TCP port 3306 on host \texttt{172.16.50.20}. Access should only be permitted from specific, trusted internal IP addresses (e.g., application servers).
    \item \textbf{Enforce MFA on Email:} Immediately enable and enforce MFA for all user accounts on the \texttt{VividVision.net} email domain. This is the single most effective control to prevent email account takeovers.
\end{enumerate}

\subsection{Short-Term Priority (Remediate within 30-60 days)}
\begin{enumerate}
    \item \textbf{Upgrade End-of-Life Database:} Plan and execute the migration of the MySQL 5.7 database to a currently supported version (e.g., MySQL 8.x). This is essential for long-term security and stability.
    \item \textbf{Deploy Security Awareness Training:} Procure and deploy a security awareness training module for all employees. This should be a mandatory part of the onboarding process for new hires and conducted at least annually for all staff.
    \item \textbf{Enforce MFA on Endpoints:} Roll out MFA for all computer and laptop logins to secure physical and remote access to company devices.
\end{enumerate}

\subsection{Long-Term Strategic Recommendations}
\begin{enumerate}
    \item \textbf{Implement a Vulnerability Management Program:} Establish a formal process for regularly scanning internal and external assets, identifying vulnerabilities, and tracking them to remediation.
    \item \textbf{Consider Network Segmentation:} As a long-term goal, investigate network segmentation to isolate critical systems (like databases) from general user networks, further reducing the attack surface. A VPN should be required for any remote administrative access.
\end{enumerate}

\end{document}
```