```latex
\documentclass[12pt]{article}

% ----------------------------------------------------------------------
% PREAMBLE
% ----------------------------------------------------------------------
\usepackage[margin=1in]{geometry}
\usepackage{pifont} % For checkmarks and crosses
\usepackage{booktabs} % For professional tables
\usepackage{hyperref} % For clickable links and ToC
\usepackage{url} % For formatting URLs
\usepackage{seqsplit} % For splitting long text strings
\usepackage{xcolor} % For colors
\usepackage{graphicx} % For logo (placeholder)

% Define colors for severity levels
\definecolor{criticalred}{HTML}{D7263D}
\definecolor{highorange}{HTML}{F49D40}
\definecolor{mediumyellow}{HTML}{F4D440}
\definecolor{lowblue}{HTML}{5486F3}

% Hyperref setup
\hypersetup{
    colorlinks=true,
    linkcolor=blue,
    filecolor=magenta,      
    urlcolor=cyan,
    pdftitle={Cybersecurity Assessment Report},
    pdfpagemode=FullScreen,
}

% Define commands for Yes/No symbols
\newcommand{\yes}{\ding{51}}
\newcommand{\no}{\ding{55}}

% ----------------------------------------------------------------------
% DOCUMENT START
% ----------------------------------------------------------------------
\begin{document}

% ----------------------------------------------------------------------
% TITLE PAGE
% ----------------------------------------------------------------------
\begin{titlepage}
    \centering
    \vspace*{1cm}
    
    \Huge
    \textbf{Cybersecurity Risk Assessment Report}
    
    \vspace{1.5cm}
    
    \Large
    Prepared for: \\
    \vspace{0.5cm}
    \textbf{Verve \& Vigor}
    
    \vspace{2cm}
    
    {\large \today}
    
    \vfill
    
    \large
    \textbf{Confidential} \\
    This document contains sensitive information. Distribution is restricted.
    
\end{titlepage}

% ----------------------------------------------------------------------
% TABLE OF CONTENTS
% ----------------------------------------------------------------------
\tableofcontents
\newpage

% ----------------------------------------------------------------------
% SECTION 1: EXECUTIVE SUMMARY
% ----------------------------------------------------------------------
\section{Executive Summary}

This report details the findings of a cybersecurity assessment conducted for \textbf{Verve \& Vigor}. The evaluation combined a review of organizational security controls, an external network scan, and an analysis of pre-existing risk data.

The assessment identified several significant security weaknesses that expose the organization to a high risk of unauthorized access, data breach, and operational disruption. Key findings include:

\begin{itemize}
    \item \textbf{Critical Database Exposure:} A MySQL database server on an internal network (\texttt{172.16.50.20}) was found with port 3306 open. This finding validates a known risk and presents a direct path for an attacker to access or compromise sensitive data. The detected MySQL version (5.7.33) is also outdated and likely missing critical security patches.
    
    \item \textbf{Critical Control Gap - No MFA for Email:} The organization does not require Multi-Factor Authentication (MFA) for email access. As email is a primary target for phishing and account takeover attacks, this gap severely increases the risk of a successful breach.
    
    \item \textbf{High-Risk Policy Gap - No Acceptable Use Policy (AUP):} The absence of a formal AUP creates ambiguity regarding the secure and appropriate use of company assets, increasing the likelihood of insider threats and non-compliant behavior.
\end{itemize}

Immediate remediation of these vulnerabilities is strongly recommended to reduce the organization's attack surface and improve its overall security posture. Actionable recommendations are provided in Section \ref{sec:recommendations} of this report.

\newpage

% ----------------------------------------------------------------------
% SECTION 2: ORGANIZATIONAL INFORMATION
% ----------------------------------------------------------------------
\section{Organizational Information}

The following details were provided for the assessment scope.

\begin{tabular}{@{}ll}
    \toprule
    \textbf{Attribute} & \textbf{Value} \\
    \midrule
    Organization Name & Verve \& Vigor \\
    Email Domain & \texttt{VerveVigor.com} \\
    External IP Address & \texttt{224.8.242.29} \\
    \bottomrule
\end{tabular}

% ----------------------------------------------------------------------
% SECTION 3: SECURITY CONTROL REVIEW
% ----------------------------------------------------------------------
\section{Security Control Review}

A review of administrative and technical security controls was conducted based on a standardized questionnaire. The responses indicate a mixed level of maturity. While foundational controls like security awareness training are in place, critical gaps exist in access control and policy enforcement.

\begin{table}[h!]
\centering
\caption{Security Controls Questionnaire Analysis}
\begin{tabular}{@{}p{0.6\linewidth} c p{0.2\linewidth}@{}}
    \toprule
    \textbf{Control Question} & \textbf{Response} & \textbf{Assessment} \\
    \midrule
    Do you require MFA to access email? & \no & \textcolor{criticalred}{\textbf{Critical Gap}} \\
    Do you require MFA to log into computers? & \yes & Meets Best Practice \\
    Do you require MFA to access sensitive data systems? & \yes & Meets Best Practice \\
    Does your organization have an employee acceptable use policy? & \no & \textcolor{highorange}{\textbf{High Risk}} \\
    Does your organization do security awareness training for new employees? & \yes & Meets Best practice \\
    Does your organization do security awareness training for all employees at least once per year? & \yes & Meets Best Practice \\
    \bottomrule
\end{tabular}
\end{table}

\newpage

% ----------------------------------------------------------------------
% SECTION 4: TECHNICAL SCAN RESULTS
% ----------------------------------------------------------------------
\section{Technical Scan Results}

An Nmap scan was performed on the target system to identify open ports and running services.

\begin{itemize}
    \item \textbf{Target IP Address:} \texttt{172.16.50.20}
\end{itemize}

\subsection{Open Ports and Services}
The scan revealed one open port, indicating a database service is directly accessible from the network.

\begin{table}[h!]
\centering
\caption{Discovered Open Ports}
\begin{tabular}{@{}lllll@{}}
    \toprule
    \textbf{Port} & \textbf{State} & \textbf{Service} & \textbf{Product} & \textbf{Version} \\
    \midrule
    3306/tcp & open & mysql & MySQL & 5.7.33 \\
    \bottomrule
\end{tabular}
\end{table}

\subsection{Analysis}
The exposure of a MySQL database (port 3306) is a significant security risk. It allows attackers to directly interact with the database, potentially leading to data exfiltration, modification, or denial of service. Furthermore, MySQL version 5.7.33 is outdated; the 5.7 branch reached its end-of-life in October 2023 and no longer receives security updates. This exposes the service to numerous publicly known vulnerabilities.

% ----------------------------------------------------------------------
% SECTION 5: CONSOLIDATED RISK ASSESSMENT
% ----------------------------------------------------------------------
\section{Consolidated Risk Assessment}
This section synthesizes findings from the security control review, technical scan, and pre-existing risk data into a consolidated list of key risks facing the organization.

\begin{table}[h!]
\centering
\caption{Summary of Identified Risks}
\begin{tabular}{@{}p{0.25\linewidth} p{0.15\linewidth} p{0.5\linewidth}@{}}
    \toprule
    \textbf{Risk Name} & \textbf{Severity} & \textbf{Overview} \\
    \midrule
    \textbf{Database Exposure} & \textcolor{highorange}{\textbf{High (7.5)}} & The MySQL database service (port 3306) is open to the network, allowing direct connection attempts. The running version is outdated and no longer supported with security patches. \\
    \addlinespace
    \textbf{Lack of MFA for Email} & \textcolor{criticalred}{\textbf{Critical}} & Email accounts, a primary gateway to other systems, are protected only by passwords. This makes them highly susceptible to phishing, credential stuffing, and account takeover. \\
    \addlinespace
    \textbf{Missing Acceptable Use Policy} & \textcolor{highorange}{\textbf{High}} & Without a formal policy, there is no enforceable standard for employee use of IT assets, increasing the risk of misuse, data leakage, and insider threat. \\
    \bottomrule
\end{tabular}
\end{table}

\newpage

% ----------------------------------------------------------------------
% SECTION 6: RECOMMENDATIONS
% ----------------------------------------------------------------------
\section{Recommendations}
\label{sec:recommendations}

The following actions are recommended to mitigate the identified risks. Recommendations are prioritized based on severity and potential impact.

\subsection{Remediate Database Exposure (High Priority)}
\begin{itemize}
    \item \textbf{Immediate Action:} Implement strict firewall rules to restrict all access to TCP port 3306 on host \texttt{172.16.50.20}. Access should only be permitted from specific, trusted application servers or administrative hosts.
    \item \textbf{Short-Term Action:} Plan and execute an upgrade of the MySQL 5.7.33 instance to a currently supported version (e.g., MySQL 8.x) to ensure security patches are applied.
    \item \textbf{Long-Term Strategy:} Re-architect network segmentation to ensure database servers are located in a secure, non-public zone. Access should be brokered through a VPN or bastion host for all administrative purposes.
\end{itemize}

\subsection{Implement MFA for Email (Critical Priority)}
\begin{itemize}
    \item \textbf{Immediate Action:} Enable and enforce Multi-Factor Authentication (MFA) for all user accounts on the \texttt{VerveVigor.com} email platform. This is the single most effective control to prevent unauthorized account access.
\end{itemize}

\subsection{Develop and Implement an Acceptable Use Policy (High Priority)}
\begin{itemize}
    \item \textbf{Short-Term Action:} Draft an Acceptable Use Policy (AUP) that clearly defines the rules and responsibilities for all employees when using company technology and data.
    \item \textbf{Long-Term Strategy:} Integrate the AUP into the new employee onboarding process. Require all current and future employees to read and formally acknowledge the policy on an annual basis.
\end{itemize}

% ----------------------------------------------------------------------
% DOCUMENT END
% ----------------------------------------------------------------------
\end{document}
```