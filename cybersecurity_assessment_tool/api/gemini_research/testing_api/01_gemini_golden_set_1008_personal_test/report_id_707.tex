```latex
\documentclass[12pt]{article}

% Preamble: Required Packages
\usepackage[margin=1in]{geometry}
\usepackage{pifont} % For check and cross marks
\usepackage{booktabs} % For professional tables
\usepackage{hyperref} % For clickable links and better PDF navigation
\usepackage{url} % For formatting URLs
\usepackage{seqsplit} % For splitting long strings in texttt
\usepackage{graphicx}
\usepackage{fancyhdr}
\usepackage{lastpage}

% Document Metadata
\title{Cybersecurity Posture Assessment Report}
\author{Cybersecurity Analysis Division}
\date{\today}

% Hyperref Setup
\hypersetup{
    colorlinks=true,
    linkcolor=black,
    urlcolor=blue,
    pdftitle={Cybersecurity Posture Assessment Report},
    pdfauthor={Cybersecurity Analysis Division},
    pdfsubject={Security Assessment},
    pdfkeywords={Cybersecurity, Risk, Assessment, Nmap, Policy}
}

% Header and Footer
\pagestyle{fancy}
\fancyhf{}
\lhead{\textbf{Hearth \& Home} - Confidential}
\rhead{Page \thepage\ of \pageref{LastPage}}

\begin{document}

\maketitle
\thispagestyle{empty}
\newpage

\tableofcontents
\newpage

% --- 1. Executive Overview ---
\section{Executive Overview}

This report provides a comprehensive cybersecurity posture assessment for \textbf{Hearth \& Home}. The analysis is based on a synthesis of technical network scan data, a review of organizational security controls via a questionnaire, and an evaluation of pre-existing risk documentation.

The assessment reveals a mixed security posture. While foundational controls like Multi-Factor Authentication (MFA) for email and computer access are in place, several critical and high-risk gaps were identified that require immediate attention.

\paragraph{Key Findings:}
\begin{itemize}
    \item \textbf{Critical Technical Risk:} A network service was found exposed on the localhost interface (\texttt{127.0.0.1}), confirming a pre-identified critical vulnerability. This indicates a significant system misconfiguration.
    \item \textbf{High-Risk Policy Gaps:} The organization lacks a formal Acceptable Use Policy (AUP) and does not conduct annual security awareness training for all employees. These administrative gaps significantly increase the risk of insider threats and susceptibility to social engineering attacks.
    \item \textbf{High-Risk Access Control Gap:} Sensitive data systems are not protected by MFA, creating a substantial risk of unauthorized access and potential data breach.
\end{itemize}

Immediate remediation efforts should focus on addressing the critical localhost exposure and implementing MFA on all sensitive systems. Concurrently, developing and enforcing key security policies and training programs is essential for improving the organization's long-term security resilience.

% --- 2. Organizational Information ---
\section{Organizational Information}

The following information was provided for the assessment.

\begin{table}[h!]
\centering
\begin{tabular}{@{}ll@{}}
\toprule
\textbf{Attribute} & \textbf{Value} \\ \midrule
Organization Name & \textbf{Hearth \& Home} \\
Email Domain & \seqsplit{\texttt{HearthHome.com}} \\
Website Domain & \seqsplit{\url{www.HearthHome.com}} \\
External IP Address & \seqsplit{\texttt{125.145.206.231}} \\ \bottomrule
\end{tabular}
\caption{Client Organizational Details.}
\label{tab:org_info}
\end{table}

% --- 3. Security Control Review ---
\section{Security Control Review}

A review of administrative and technical security controls was conducted based on a questionnaire. The responses highlight key areas of strength and weakness in the current security program. Gaps identified with a \ding{55} represent significant risks.

\begin{table}[h!]
\centering
\begin{tabular}{@{}p{0.6\textwidth}cc@{}}
\toprule
\textbf{Control Question} & \textbf{Response} & \textbf{Status} \\ \midrule
Do you require MFA to access email? & Yes & \ding{51} \\
Do you require MFA to log into computers? & Yes & \ding{51} \\
\textbf{Do you require MFA to access sensitive data systems?} & \textbf{No} & \textbf{\ding{55}} \\
\textbf{Does your organization have an employee acceptable use policy?} & \textbf{No} & \textbf{\ding{55}} \\
Does your organization do security awareness training for new employees? & Yes & \ding{51} \\
\textbf{Does your organization do security awareness training for all employees at least once per year?} & \textbf{No} & \textbf{\ding{55}} \\ \bottomrule
\end{tabular}
\caption{Security Control Questionnaire Analysis.}
\label{tab:controls}
\end{table}

% --- 4. Technical Scan Results ---
\section{Technical Scan Results}

A network scan was performed to identify open ports and services on the target system. The results are detailed below.

\subsection{Nmap Scan Findings for \texttt{127.0.0.1}}
The scan identified one open port on the localhost interface. Exposing services, even on localhost, can be a security risk if the system is compromised or if other services can be forced to interact with it (e.g., Server-Side Request Forgery). This finding directly correlates with and validates the pre-existing risk documented in Input 3.

\begin{table}[h!]
\centering
\begin{tabular}{@{}llll@{}}
\toprule
\textbf{Port/Protocol} & \textbf{State} & \textbf{Service (Inferred)} & \textbf{Version} \\ \midrule
22/tcp & open & ssh & Not Detected \\ \bottomrule
\end{tabular}
\caption{Open Ports Detected on \texttt{127.0.0.1}.}
\label{tab:nmap_results}
\end{table}

% --- 5. Consolidated Risk Assessment ---
\section{Consolidated Risk Assessment}

The following table synthesizes findings from the technical scan, control review, and pre-existing risk data into a prioritized list.

\begin{table}[h!]
\centering
\begin{tabular}{@{}p{0.1\textwidth}p{0.3\textwidth}p{0.4\textwidth}l@{}}
\toprule
\textbf{ID} & \textbf{Risk Title} & \textbf{Description} & \textbf{Severity} \\ \midrule
RISK-001 & \textbf{Localhost Exposed} & The Nmap scan confirmed that port 22 (SSH) is open on the localhost interface, validating a known critical misconfiguration. & \textbf{Critical} \\
\addlinespace
RISK-002 & \textbf{No MFA for Sensitive Systems} & Access to critical systems containing sensitive data is not protected by MFA, elevating the risk of unauthorized access via compromised credentials. & \textbf{High} \\
\addlinespace
RISK-003 & \textbf{Lack of Acceptable Use Policy (AUP)} & The absence of a formal AUP creates ambiguity regarding employee responsibilities and acceptable behavior when using company IT assets. & \textbf{High} \\
\addlinespace
RISK-004 & \textbf{Insufficient Security Awareness Training} & Training is not conducted annually for all staff, increasing the organization's vulnerability to phishing and other social engineering attacks. & \textbf{High} \\ \bottomrule
\end{tabular}
\caption{Summary of Identified Risks.}
\label{tab:risk_summary}
\end{table}

% --- 6. Recommendations ---
\section{Recommendations}

Based on the consolidated risk assessment, the following actions are recommended to mitigate the identified vulnerabilities and improve the overall security posture of \textbf{Hearth \& Home}.

\begin{itemize}
    \item[\textbf{1.}] \textbf{(Critical) Remediate Localhost Exposure (RISK-001):}
    \begin{itemize}
        \item Immediately investigate the service running on port 22 on \texttt{127.0.0.1}.
        \item Determine its business purpose. If it is not required, disable the service entirely.
        \item If the service is required, ensure it is properly configured and firewalled to prevent unauthorized access or interaction.
    \end{itemize}
    \vspace{0.5em}
    \item[\textbf{2.}] \textbf{(High) Implement MFA on Sensitive Systems (RISK-002):}
    \begin{itemize}
        \item Prioritize and inventory all systems that store, process, or transmit sensitive data.
        \item Deploy and enforce a strong MFA solution for all user access to these identified systems.
    \end{itemize}
    \vspace{0.5em}
    \item[\textbf{3.}] \textbf{(High) Develop and Implement an AUP (RISK-003):}
    \begin{itemize}
        \item Draft a comprehensive Acceptable Use Policy that clearly defines rules for the use of company networks, systems, and data.
        \item Require all current and new employees to read and formally acknowledge the policy.
    \end{itemize}
    \vspace{0.5em}
    \item[\textbf{4.}] \textbf{(High) Establish Annual Security Training (RISK-004):}
    \begin{itemize}
        \item Institute a mandatory security awareness training program for all employees to be completed on an annual basis.
        \item The training curriculum should cover modern threats, including phishing, malware, password security, and social engineering.
    \end{itemize}
\end{itemize}

\end{document}
```