```latex
\documentclass[12pt]{article}

% --- PACKAGES ---
\usepackage[margin=1in]{geometry}
\usepackage{pifont} % For checkmarks and crosses
\usepackage{booktabs} % For professional tables
\usepackage{hyperref} % For clickable links
\usepackage{url} % For URL formatting
\usepackage{seqsplit} % To split long strings without breaking
\usepackage[utf8]{inputenc}

% --- DOCUMENT METADATA ---
\title{Cybersecurity Posture Assessment Report}
\author{Cybersecurity Analyst}
\date{\today}

% --- HYPERREF SETUP ---
\hypersetup{
    colorlinks=true,
    linkcolor=black,
    urlcolor=blue,
    pdftitle={Cybersecurity Posture Assessment Report},
    pdfauthor={Cybersecurity Analyst},
}

\begin{document}

\maketitle
\hrule

\section*{Executive Overview}
This report provides a comprehensive cybersecurity posture assessment for \textbf{Apex Legends Group}. The analysis is based on a synthesis of organizational security control data, a technical network scan, and a review of pre-existing risks.

The assessment reveals a mixed security posture. While the organization demonstrates a strong network perimeter on the scanned asset, with no open ports detected, critical gaps exist in its identity and access management and employee onboarding processes. The absence of Multi-Factor Authentication (MFA) for email access represents a \textbf{Critical} risk, exposing the organization to significant threats like business email compromise and phishing attacks. Furthermore, the lack of security awareness training for new employees constitutes a \textbf{High} risk, as new hires are often prime targets for social engineering.

This report details these findings and provides prioritized, actionable recommendations to mitigate the identified risks and strengthen the overall security framework.

\section{Organizational Information}
The following details were provided for the assessment:
\begin{itemize}
    \item \textbf{Organization Name:} Apex Legends Group
    \item \textbf{Email Domain:} \texttt{ApexLegendsGroup.com}
    \item \textbf{Website Domain:} \url{www.ApexLegendsGroup.com}
    \item \textbf{External IP Address:} \texttt{199.204.186.150}
\end{itemize}

\section{Security Control Review}
The following table summarizes the organization's responses to a security controls questionnaire. A green checkmark (\ding{51}) indicates a positive control is in place, while a red cross (\ding{55}) indicates a potential security gap.

\begin{table}[h!]
\centering
\begin{tabular}{p{0.7\textwidth}c}
\toprule
\textbf{Security Control Question} & \textbf{Response} \\
\midrule
Do you require MFA to access email? & \ding{55} \\
Do you require MFA to log into computers? & \ding{51} \\
Do you require MFA to access sensitive data systems? & \ding{51} \\
Does your organization have an employee acceptable use policy? & \ding{51} \\
Does your organization do security awareness training for new employees? & \ding{55} \\
Does your organization do security awareness training for all employees at least once per year? & \ding{51} \\
\bottomrule
\end{tabular}
\caption{Organizational Security Controls Questionnaire}
\label{tab:controls}
\end{table}

\section{Technical Scan Results}
A network scan was performed to identify open ports and services on the specified target system.
\begin{itemize}
    \item \textbf{Scan Target:} \texttt{192.168.1.100}
    \item \textbf{Scan Date:} \today
\end{itemize}

\subsection{Summary of Findings}
The scan confirmed that the host is online and responsive. However, \textbf{no open ports were detected}. All scanned ports were found to be in a "closed" state. This is a positive security finding, as it indicates a properly configured firewall and a minimal attack surface for this specific host.

\begin{table}[h!]
\centering
\begin{tabular}{ll}
\toprule
\textbf{Metric} & \textbf{Result} \\
\midrule
Host Status & Up \\
Open Ports & 0 \\
Filtered Ports & 0 \\
Closed Ports & All Scanned Ports \\
\bottomrule
\end{tabular}
\caption{Nmap Scan Summary for \texttt{192.168.1.100}}
\label{tab:nmap}
\end{table}

\section{Risk Assessment}
The following table correlates the findings from the security control review, technical scan, and pre-existing vulnerability data. The primary risks identified stem from organizational policy and process gaps rather than technical vulnerabilities on the scanned asset.

\begin{table}[h!]
\centering
\begin{tabular}{p{0.25\textwidth}p{0.5\textwidth}l}
\toprule
\textbf{Risk Name} & \textbf{Overview} & \textbf{Severity} \\
\midrule
\textbf{Lack of MFA on Email Accounts} & Email accounts do not require Multi-Factor Authentication (MFA). This makes them highly susceptible to compromise via phishing, credential stuffing, and password spray attacks, which can lead to data breaches and financial fraud. & \textbf{Critical} \\
\addlinespace
\textbf{Inadequate New Employee Onboarding} & New employees do not receive security awareness training upon joining. This leaves them unprepared to identify and report social engineering attacks, making them a significant weak point in the organization's human firewall. & \textbf{High} \\
\bottomrule
\end{tabular}
\caption{Synthesized Risk Summary}
\label{tab:risks}
\end{table}

\section{Recommendations}
Based on the analysis, the following actions are recommended to mitigate the identified risks and improve the security posture of \textbf{Apex Legends Group}.

\subsection{Recommendation 1: Implement MFA for Email Access (Critical)}
\textbf{Action:} Enforce mandatory Multi-Factor Authentication for all user accounts on the \texttt{ApexLegendsGroup.com} email domain immediately. This is the single most effective control to prevent unauthorized account access and business email compromise.
\begin{itemize}
    \item \textbf{Priority:} Immediate
    \item \textbf{Impact:} High
    \item \textbf{Effort:} Medium
\end{itemize}

\subsection{Recommendation 2: Integrate Security Training into Onboarding (High)}
\textbf{Action:} Develop and mandate a security awareness training module for all new hires as part of the standard onboarding process. This training should cover, at a minimum, phishing identification, password security, and the company's acceptable use policy.
\begin{itemize}
    \item \textbf{Priority:} High
    \item \textbf{Impact:} High
    \item \textbf{Effort:} Medium
\end{itemize}

\subsection{Recommendation 3: Maintain Strong Network Perimeter (Informational)}
\textbf{Action:} The network scan of \texttt{192.168.1.100} showed no open ports, which is an excellent security practice. Continue to apply the principle of least privilege to all external-facing systems and conduct regular vulnerability scans to ensure this hardened posture is maintained.
\begin{itemize}
    \item \textbf{Priority:} Informational
    \item \textbf{Impact:} N/A
    \item \textbf{Effort:} Low (Maintenance)
\end{itemize}

\end{document}
```