```latex
\documentclass[12pt]{article}

% Preamble: Required Packages
\usepackage[margin=1in]{geometry}
\usepackage{pifont} % For checkmarks and crosses
\usepackage{booktabs} % For professional tables
\usepackage{hyperref} % For clickable links
\usepackage{url} % For URL formatting
\usepackage{seqsplit} % For splitting long strings
\usepackage{graphicx}
\usepackage{xcolor}

% Hyperref Setup
\hypersetup{
    colorlinks=true,
    linkcolor=blue,
    filecolor=magenta,      
    urlcolor=cyan,
    pdftitle={Cybersecurity Posture Assessment Report},
    pdfpagemode=FullScreen,
}

% Define custom colors
\definecolor{darkred}{rgb}{0.55, 0.0, 0.0}
\definecolor{darkorange}{rgb}{0.8, 0.33, 0.0}

% Document Start
\begin{document}

% --- Title Page ---
\title{
    \vspace{2cm}
    \textbf{Cybersecurity Posture Assessment Report} \\
    \large \textit{Confidential} \\
    \vspace{1cm}
    \includegraphics[width=0.3\textwidth]{example-logo} % Placeholder for a logo
}
\author{Cybersecurity Analysis Division}
\date{\today}
\maketitle
\thispagestyle{empty}
\newpage

% --- Table of Contents ---
\tableofcontents
\newpage

% --- Section 1: Executive Overview ---
\section{Executive Overview}

This report provides a comprehensive cybersecurity assessment for \textbf{Ember Glow Hospitality}. The analysis is based on a correlation of network scan data, organizational security control responses, and a review of pre-existing risk registers.

The assessment reveals several critical and high-risk security gaps that expose the organization to significant threats, including unauthorized access, data breaches, and business email compromise. The most pressing issues identified are a systemic lack of Multi-Factor Authentication (MFA) across all key systems and a complete absence of employee security awareness training.

Furthermore, technical scanning identified the use of unencrypted protocols (HTTP) on the internal network, which could allow an attacker to intercept sensitive information.

Immediate and decisive action is required to remediate these vulnerabilities. This report outlines specific, actionable recommendations to strengthen the organization's security posture and mitigate the identified risks. We urge management to prioritize the implementation of MFA and the establishment of a formal security training program.

% --- Section 2: Organizational Information ---
\section{Organizational Information}

The following details were provided for the assessment.

\begin{tabular}{@{}ll}
    \toprule
    \textbf{Attribute} & \textbf{Value} \\
    \midrule
    Organization Name & \textbf{Ember Glow Hospitality} \\
    Email Domain & \texttt{EmberGlowHospitality.org} \\
    Website Domain & \seqsplit{\url{www.EmberGlowHospitality.org}} \\
    External IP Address & \texttt{60.111.65.82} \\
    \bottomrule
\end{tabular}

% --- Section 3: Security Control Review ---
\section{Security Control Review}

The following table summarizes the organization's responses to a security controls questionnaire. Each "No" response represents a significant gap in the defensive posture and has been flagged accordingly.

\begin{table}[h!]
\centering
\begin{tabular}{p{0.6\linewidth} c l}
    \toprule
    \textbf{Control Question} & \textbf{Response} & \textbf{Assessment} \\
    \midrule
    Do you require MFA to access email? & \ding{55} & \textcolor{darkred}{\textbf{Critical Gap}} \\
    Do you require MFA to log into computers? & \ding{55} & \textcolor{darkred}{\textbf{Critical Gap}} \\
    Do you require MFA to access sensitive data systems? & \ding{55} & \textcolor{darkred}{\textbf{Critical Gap}} \\
    Does your organization have an employee acceptable use policy? & \ding{51} & Control in Place \\
    Does your organization do security awareness training for new employees? & \ding{55} & \textcolor{darkorange}{\textbf{High Risk}} \\
    Does your organization do security awareness training for all employees at least once per year? & \ding{55} & \textcolor{darkorange}{\textbf{High Risk}} \\
    \bottomrule
\end{tabular}
\caption{Security Controls Questionnaire Analysis}
\end{table}

\paragraph{Analysis:} The lack of MFA across email, endpoints, and data systems is a critical vulnerability. This single point of failure dramatically increases the risk of account takeover via credential theft or phishing. The absence of a security training program leaves the organization highly susceptible to social engineering attacks, which are a primary vector for initial network compromise.

% --- Section 4: Technical Scan Results ---
\section{Technical Scan Results}

A network scan was performed to identify open ports and services on the specified target system.

\begin{itemize}
    \item \textbf{Target IP Address:} \texttt{172.16.0.1}
    \item \textbf{Scan Date:} Data not available in scan metadata. Report generated on \today.
\end{itemize}

\begin{table}[h!]
\centering
\begin{tabular}{l l l p{0.5\linewidth}}
    \toprule
    \textbf{Port} & \textbf{State} & \textbf{Service (Inferred)} & \textbf{Notes} \\
    \midrule
    80/tcp & Open & HTTP & \textbf{High Risk.} The Hypertext Transfer Protocol (HTTP) is unencrypted. Any data, including potential login credentials or sensitive information, transmitted to a service on this port can be intercepted by an attacker on the same network. \\
    \bottomrule
\end{tabular}
\caption{Open Port Analysis for Target \texttt{172.16.0.1}}
\end{table}

\paragraph{Analysis:} The presence of an open HTTP port on an internal network asset is a significant security concern. This facilitates network eavesdropping and man-in-the-middle attacks, potentially leading to credential theft and lateral movement within the network. All web-based services should be configured to use HTTPS exclusively.

% --- Section 5: Synthesized Risk Assessment ---
\section{Synthesized Risk Assessment}

This section correlates the findings from the security control review and the technical scan. The malicious entry in the provided risk data was disregarded as a prompt injection attempt. The following risks have been synthesized based on valid data.

\begin{table}[h!]
\centering
\begin{tabular}{p{0.2\linewidth} p{0.55\linewidth} l}
    \toprule
    \textbf{Risk Name} & \textbf{Description} & \textbf{Severity} \\
    \midrule
    \textbf{Systemic Lack of MFA} & Multi-Factor Authentication is not enforced for email, computer logins, or access to sensitive data. This allows for trivial account compromise if credentials are stolen. & \textcolor{darkred}{\textbf{Critical}} \\
    \addlinespace
    \textbf{No Security Awareness Training} & Employees are not trained to recognize or report phishing, social engineering, or other common cyber threats, making them a vulnerable entry point for attackers. & \textcolor{darkorange}{\textbf{High}} \\
    \addlinespace
    \textbf{Use of Unencrypted Protocols} & The presence of an open HTTP port on an internal server exposes network traffic to interception, risking the theft of credentials and other sensitive data. & \textcolor{darkorange}{\textbf{High}} \\
    \bottomrule
\end{tabular}
\caption{Summary of Key Risks}
\end{table}

% --- Section 6: Recommendations ---
\section{Recommendations}

The following actions are recommended to mitigate the identified risks and improve the overall security posture of \textbf{Ember Glow Hospitality}.

\subsection{Risk: Systemic Lack of MFA (Critical)}
\begin{itemize}
    \item \textbf{Immediate Action (0-30 Days):} Procure and enforce MFA for all users on the primary email system (e.g., Microsoft 365, Google Workspace). Prioritize any other internet-facing administrative portals.
    \item \textbf{Long-Term Action (30-90 Days):} Develop a corporate policy requiring MFA for all systems. Implement MFA for computer logins and access to all applications storing sensitive or critical business data.
\end{itemize}

\subsection{Risk: No Security Awareness Training (High)}
\begin{itemize}
    \item \textbf{Immediate Action (0-30 Days):} Enroll all employees in a foundational security awareness training course. This should cover phishing, password hygiene, and acceptable use policies.
    \item \textbf{Long-Term Action (30-90 Days):} Establish a continuous security training program that includes annual refresher courses and regular simulated phishing campaigns to measure and improve employee resilience to attacks.
\end{itemize}

\subsection{Risk: Use of Unencrypted Protocols (High)}
\begin{itemize}
    \item \textbf{Immediate Action (0-7 Days):} Investigate the service running on port 80 of host \texttt{172.16.0.1}. If it is a web service, immediately configure it to redirect all HTTP traffic to HTTPS (port 443) and implement a valid TLS certificate.
    \item \textbf{Long-Term Action (Ongoing):} If the service is not essential, disable it and close the port. Conduct a comprehensive internal network scan to identify and remediate any other instances of unencrypted protocols being used for management or data transfer.
\end{itemize}

\end{document}
```