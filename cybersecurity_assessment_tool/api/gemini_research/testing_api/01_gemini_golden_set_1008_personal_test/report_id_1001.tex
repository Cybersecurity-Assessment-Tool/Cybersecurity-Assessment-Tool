```latex
\documentclass[12pt]{article}

% --- PACKAGES ---
\usepackage[margin=1in]{geometry}
\usepackage{pifont} % For checkmarks and crosses
\usepackage{booktabs} % For professional tables
\usepackage{hyperref} % For clickable links
\usepackage{url} % For URL formatting
\usepackage{seqsplit} % For splitting long strings in tt font
\usepackage[utf8]{inputenc}
\usepackage{graphicx}
\usepackage{xcolor}
\usepackage{fancyhdr}

% --- DOCUMENT SETUP ---
\hypersetup{
    colorlinks=true,
    linkcolor=black,
    urlcolor=blue,
}
\pagestyle{fancy}
\fancyhf{}
\lhead{Cybersecurity Assessment Report}
\rhead{Willow Creek Health}
\cfoot{\thepage}
\renewcommand{\headrulewidth}{0.4pt}
\renewcommand{\footrulewidth}{0.4pt}

% --- TITLE ---
\title{
    \vspace{1cm}
    \textbf{Cybersecurity Posture Assessment Report} \\
    \large \textit{Prepared for Willow Creek Health}
}
\author{Cybersecurity Analysis Division}
\date{\today}

\begin{document}

\maketitle
\thispagestyle{empty}
\newpage

\tableofcontents
\newpage

% --- EXECUTIVE SUMMARY ---
\section{Executive Summary}
This report provides a comprehensive analysis of the cybersecurity posture of Willow Creek Health, based on a combination of technical network scanning, a review of existing risks, and an organizational security controls questionnaire.

The assessment reveals several \textbf{critical-risk} findings that require immediate attention. The primary concern is the combination of externally exposed services, specifically Remote Desktop Protocol (RDP), and significant gaps in foundational security controls. The lack of Multi-Factor Authentication (MFA) for computer and sensitive system access, coupled with the absence of a formal security awareness training program, creates a high-risk environment. This configuration is a common precursor to severe security incidents, including ransomware attacks and data breaches.

Immediate remediation should focus on securing the network perimeter by closing unnecessary ports and implementing a Virtual Private Network (VPN) for remote access. Concurrently, deploying MFA across all critical systems is paramount to mitigating the risk of unauthorized access via compromised credentials.

% --- ORGANIZATIONAL INFORMATION ---
\section{Organizational Information}
The following details were provided for the assessment scope.
\begin{itemize}
    \item \textbf{Organization Name:} Willow Creek Health
    \item \textbf{Email Domain:} \texttt{WillowCreekHealth.org}
    \item \textbf{Website Domain:} \seqsplit{\texttt{www.WillowCreekHealth.org}}
    \item \textbf{Known External IP:} \texttt{121.17.121.210}
\end{itemize}

% --- SECURITY CONTROL REVIEW ---
\section{Security Control Review}
An analysis of the security questionnaire highlights significant gaps in administrative and technical controls. "No" responses indicate a deviation from security best practices and are detailed below.

\begin{table}[h!]
\centering
\caption{Security Controls Questionnaire Analysis}
\begin{tabular}{p{8cm} c p{4cm}}
\toprule
\textbf{Control Question} & \textbf{Response} & \textbf{Assessment} \\
\midrule
Do you require MFA to access email? & \ding{51} & Compliant. \\
Do you require MFA to log into computers? & \textbf{\color{red}\ding{55}} & \textbf{Critical Gap.} Increases risk of unauthorized access. \\
Do you require MFA to access sensitive data systems? & \textbf{\color{red}\ding{55}} & \textbf{Critical Gap.} Exposes critical data to compromise. \\
Does your organization have an employee acceptable use policy? & \textbf{\color{red}\ding{55}} & \textbf{High Risk.} Lack of foundational governance. \\
Does your organization do security awareness training for new employees? & \textbf{\color{red}\ding{55}} & \textbf{High Risk.} Employees are not equipped to identify threats. \\
Does your organization do security awareness training for all employees at least once per year? & \textbf{\color{red}\ding{55}} & \textbf{High Risk.} Security skills degrade without reinforcement. \\
\bottomrule
\end{tabular}
\end{table}

% --- TECHNICAL SCAN RESULTS ---
\section{Technical Scan Results}
A network scan was performed on the specified target to identify open ports and exposed services.

\begin{itemize}
    \item \textbf{Target IP Address:} \texttt{10.10.10.51}
    \item \textbf{Scan Status:} Host is UP.
\end{itemize}

\begin{table}[h!]
\centering
\caption{Open Port Analysis for \texttt{10.10.10.51}}
\begin{tabular}{c c l p{6cm}}
\toprule
\textbf{Port} & \textbf{State} & \textbf{Service Name} & \textbf{Analysis} \\
\midrule
3389/tcp & OPEN & \texttt{ms-wbt-server} & This port is used for Microsoft Remote Desktop Protocol (RDP). Exposing RDP directly to the network is a critical security risk and a primary vector for ransomware attacks. This finding, combined with the pre-existing risk on host \texttt{10.10.10.50}, indicates a systemic issue. \\
\bottomrule
\end{tabular}
\end{table}

% --- CONSOLIDATED RISK ASSESSMENT ---
\section{Consolidated Risk Assessment}
The following table synthesizes findings from the technical scan, the control review, and pre-existing risk data into a prioritized list.

\begin{table}[h!]
\centering
\caption{Summary of Identified Risks}
\begin{tabular}{p{4cm} p{8cm} l}
\toprule
\textbf{Risk Name} & \textbf{Description} & \textbf{Severity} \\
\midrule
\textbf{Systemic RDP Exposure} & Remote Desktop Protocol is exposed on multiple internal systems (\texttt{10.10.10.50}, \texttt{10.10.10.51}), creating direct entry points for attackers. This is a well-known target for brute-force and credential stuffing attacks. & \textbf{Critical} \\
\addlinespace
\textbf{Lack of Multi-Factor Authentication} & MFA is not enforced for computer or sensitive system logins. This allows an attacker with valid credentials (e.g., from a phishing attack) to gain full access without needing a second authentication factor. & \textbf{Critical} \\
\addlinespace
\textbf{No Security Awareness Program} & The complete absence of security awareness training means employees are likely unable to recognize or report phishing attempts, malware, or other social engineering tactics. & \textbf{High} \\
\addlinespace
\textbf{Missing Acceptable Use Policy} & The lack of a foundational policy defining the acceptable use of company assets creates ambiguity for employees and exposes the organization to legal and compliance risks. & \textbf{Medium} \\
\bottomrule
\end{tabular}
\end{table}

% --- RECOMMENDATIONS ---
\section{Recommendations}
Based on the analysis, the following actions are recommended to mitigate the identified risks and improve the overall security posture of Willow Creek Health.

\subsection{Immediate Priority (Remediate within 72 hours)}
\begin{enumerate}
    \item \textbf{Remediate RDP Exposure:} Immediately block access to TCP port 3389 on hosts \texttt{10.10.10.50} and \texttt{10.10.10.51} from any untrusted network.
    \item \textbf{Implement a Secure Remote Access Solution:} If remote access to these systems is required, deploy a Virtual Private Network (VPN) solution. Access to the VPN must be protected with Multi-Factor Authentication.
\end{enumerate}

\subsection{High Priority (Remediate within 30 days)}
\begin{enumerate}
    \item \textbf{Deploy Multi-Factor Authentication (MFA):} Enforce MFA for all users for logging into computers and accessing any systems containing sensitive data. This is the single most effective control to prevent unauthorized access.
    \item \textbf{Establish a Security Awareness Program:} Implement a mandatory security awareness training program for all new and existing employees. The training should be conducted annually at a minimum and cover topics such as phishing, password security, and incident reporting.
\end{enumerate}

\subsection{Medium Priority (Remediate within 90 days)}
\begin{enumerate}
    \item \textbf{Develop and Implement an Acceptable Use Policy (AUP):} Create a formal AUP document that all employees must read and acknowledge. This policy should outline the rules for using corporate networks, devices, and data.
    \item \textbf{Conduct a Comprehensive Vulnerability Scan:} Perform a credentialed vulnerability scan across the entire internal network to identify other potential exposures, outdated software, and misconfigurations.
\end{enumerate}

\end{document}
```