```latex
\documentclass[12pt]{article}

% Preamble: Required Packages
\usepackage[margin=1in]{geometry}
\usepackage{pifont} % For checkmarks and crosses
\usepackage{booktabs} % For professional tables
\usepackage{hyperref} % For clickable links
\usepackage{url} % For URL formatting
\usepackage{seqsplit} % To split long strings in tt font
\usepackage{graphicx}
\usepackage{xcolor}

% Hyperref Setup
\hypersetup{
    colorlinks=true,
    linkcolor=blue,
    filecolor=magenta,      
    urlcolor=cyan,
    pdftitle={Cybersecurity Posture Report},
    pdfpagemode=FullScreen,
}

% Document Title
\title{Cybersecurity Posture Report \\ \large For: Tidal Wave Sports}
\author{Cybersecurity Analysis Division}
\date{\today}

\begin{document}

\maketitle
\tableofcontents
\newpage

% --- 1. Executive Summary ---
\section{Executive Summary}

This report provides a comprehensive analysis of the cybersecurity posture for \textbf{Tidal Wave Sports}, based on a synthesis of technical network scans, a review of organizational security controls, and an assessment of existing risks. The evaluation was conducted to identify vulnerabilities, security gaps, and areas for improvement.

The key findings indicate a mixed security posture with several critical areas requiring immediate attention. While the organization has implemented Multi-Factor Authentication (MFA) for email access, significant gaps exist in other domains. The absence of MFA for computer logins and access to sensitive data systems presents a high risk of unauthorized access. Furthermore, critical administrative controls, such as an employee acceptable use policy and security training for new hires, are not in place.

From a technical standpoint, an external scan identified an exposed Secure Shell (SSH) service on a public-facing IPv6 address. This, combined with the identified administrative control gaps, elevates the risk of a successful brute-force or credential-stuffing attack.

This report concludes with a prioritized list of actionable recommendations designed to mitigate the identified risks and strengthen the overall security framework of the organization.

% --- 2. Organizational Information ---
\section{Organizational Information}

The following information was provided and used as the basis for this assessment.

\begin{tabular}{@{}ll}
\toprule
\textbf{Attribute} & \textbf{Value} \\
\midrule
Organization Name & \textbf{Tidal Wave Sports} \\
Email Domain & \texttt{TidalWaveSports.net} \\
Website Domain & \url{www.TidalWaveSports.net} \\
External IP Address & \texttt{152.249.71.69} \\
\bottomrule
\end{tabular}

% --- 3. Security Control Review ---
\section{Security Control Review}

A review of administrative and procedural security controls was conducted via a questionnaire. The responses reveal critical gaps in the organization's security policies and enforcement mechanisms. A summary of the findings is presented in Table \ref{tab:controls}.

\begin{table}[h!]
\centering
\caption{Organizational Security Control Status}
\label{tab:controls}
\begin{tabular}{@{}lc}
\toprule
\textbf{Control Question} & \textbf{Response} \\
\midrule
Do you require MFA to access email? & \ding{51} \\ % Yes
Do you require MFA to log into computers? & \textcolor{red}{\ding{55}} \\ % No
Do you require MFA to access sensitive data systems? & \textcolor{red}{\ding{55}} \\ % No
Does your organization have an employee acceptable use policy? & \textcolor{red}{\ding{55}} \\ % No
Does your organization do security awareness training for new employees? & \textcolor{red}{\ding{55}} \\ % No
Does your organization do security awareness training for all employees at least once per year? & \ding{51} \\ % Yes
\bottomrule
\end{tabular}
\end{table}

\subsection*{Analysis of Control Gaps}
The responses marked with \textcolor{red}{\ding{55}} highlight significant vulnerabilities:
\begin{itemize}
    \item \textbf{Lack of Endpoint and System MFA:} The absence of MFA on computer logins and sensitive data systems means that a single compromised password could grant an attacker broad access to internal resources.
    \item \textbf{Missing Acceptable Use Policy (AUP):} Without a formal AUP, employees may be unaware of security expectations, acceptable online behavior, and data handling procedures, increasing the likelihood of unintentional security breaches.
    \item \textbf{No Onboarding Security Training:} New employees are not receiving security awareness training upon joining the organization. This is a critical missed opportunity to establish a strong security culture from day one and prevent common errors.
\end{itemize}

% --- 4. Technical Scan Results ---
\section{Technical Scan Results}

An external network scan was performed on the infrastructure associated with the organization. The scan identified the following open ports on a public-facing host.

\begin{table}[h!]
\centering
\caption{Open Ports Detected on Target: \seqsplit{\texttt{2001:db8::1}}}
\label{tab:scan}
\begin{tabular}{@{}llll}
\toprule
\textbf{Port} & \textbf{State} & \textbf{Service (Inferred)} & \textbf{Notes} \\
\midrule
22/tcp & Open & SSH & Secure Shell remote administration protocol. \\
\bottomrule
\end{tabular}
\end{table}

\subsection*{Analysis of Technical Findings}
The scan revealed that port 22 (SSH) is open on the IPv6 address \seqsplit{\texttt{2001:db8::1}}. While SSH is a standard tool for remote server management, its exposure to the public internet makes it a primary target for automated brute-force attacks. Without proper security configurations—such as IP address whitelisting, key-based authentication, and intrusion detection systems (e.g., fail2ban)—this service poses a significant entry risk. The lack of detailed version information from this scan prevents an assessment for specific software vulnerabilities, but the exposure itself is a notable risk.

% --- 5. Correlated Risk Assessment ---
\section{Correlated Risk Assessment}

This section synthesizes the findings from the security control review and the technical scan to provide a consolidated view of the most critical risks facing the organization. No pre-existing vulnerabilities were reported.

\begin{table}[h!]
\centering
\caption{Summary of Identified Risks}
\label{tab:risks}
\resizebox{\textwidth}{!}{%
\begin{tabular}{@{}llll}
\toprule
\textbf{ID} & \textbf{Risk Name} & \textbf{Description} & \textbf{Severity} \\
\midrule
RISK-001 & Inadequate Access Control & The absence of MFA on workstations and sensitive systems & \textbf{High} \\
         & (MFA Gaps) & drastically increases the risk of unauthorized access via & \\
         &            & compromised credentials. & \\
\addlinespace
RISK-002 & Exposed Management Service & Port 22 (SSH) is publicly accessible, exposing the system & \textbf{Medium} \\
         & (SSH) & to brute-force attacks and potential exploitation. This risk & \\
         &       & is amplified by the internal MFA gaps (RISK-001). & \\
\addlinespace
RISK-003 & Weak Policy and Training & The lack of an Acceptable Use Policy and security training & \textbf{High} \\
         & Framework & for new hires creates an environment susceptible to human & \\
         &           & error and policy violations, which are leading causes of breaches. & \\
\bottomrule
\end{tabular}%
}
\end{table}

% --- 6. Recommendations ---
\section{Recommendations}

Based on the correlated risk assessment, the following prioritized recommendations are provided to mitigate the identified vulnerabilities and improve the overall security posture of \textbf{Tidal Wave Sports}.

\begin{enumerate}
    \item \textbf{[High Priority] Implement Comprehensive MFA (RISK-001):}
    \begin{itemize}
        \item Immediately deploy and enforce MFA for all user logins to company computers and laptops.
        \item Enforce MFA for access to all internal and cloud-based systems that store or process sensitive data.
    \end{itemize}

    \item \textbf{[High Priority] Develop and Implement Security Policies (RISK-003):}
    \begin{itemize}
        \item Create a formal Acceptable Use Policy (AUP) that clearly defines rules for using company assets, data handling, and internet usage. Require all employees to read and acknowledge the policy.
        \item Integrate mandatory security awareness training into the new employee onboarding process to establish a baseline of security knowledge.
    \end{itemize}

    \item \textbf{[Medium Priority] Secure Exposed SSH Service (RISK-002):}
    \begin{itemize}
        \item If remote access is required, restrict access to the SSH service on \seqsplit{\texttt{2001:db8::1}} to a whitelist of trusted IP addresses.
        \item Disable password-based authentication for SSH and enforce the use of public/private key pairs.
        \item Implement an intrusion prevention tool like \texttt{fail2ban} to automatically block IPs that exhibit malicious behavior (e.g., repeated failed login attempts).
    \end{itemize}
\end{enumerate}

\end{document}
```