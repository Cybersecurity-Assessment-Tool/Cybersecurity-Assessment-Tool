```latex
\documentclass[12pt]{article}

% --- PACKAGE IMPORTS ---
\usepackage[margin=1in]{geometry}
\usepackage{pifont} % For checkmarks and crosses
\usepackage{booktabs} % For professional tables
\usepackage{hyperref} % For clickable links
\usepackage{url} % For formatting URLs
\usepackage{seqsplit} % For splitting long strings to prevent overflow

% --- DOCUMENT METADATA ---
\hypersetup{
    colorlinks=true,
    linkcolor=black,
    urlcolor=blue,
    pdftitle={Cybersecurity Posture Report},
    pdfauthor={Cybersecurity Analyst},
    pdfsubject={Security Assessment},
    pdfkeywords={Security, Analysis, Report}
}

\begin{document}

% --- TITLE PAGE ---
\begin{titlepage}
    \centering
    \vspace*{\stretch{1.0}}
    \Huge{\textbf{Cybersecurity Posture Report}}
    \vspace{0.5cm}
    \LARGE{Prepared for: Blue Marble}
    \vspace{1.5cm}
    \large{Generated: \today}
    \vspace{\stretch{2.0}}
    \normalsize{
        \textbf{Author:} Cybersecurity Analyst \\
        \textbf{Classification:} Confidential
    }
    \vfill
\end{titlepage}

% --- TABLE OF CONTENTS ---
\tableofcontents
\newpage

% --- EXECUTIVE SUMMARY ---
\section*{Executive Summary}
This report provides a comprehensive analysis of the cybersecurity posture of Blue Marble, based on a combination of network scanning, a security controls questionnaire, and a review of pre-existing risks. The assessment identified several critical and high-risk areas requiring immediate attention.

Key findings indicate significant gaps in identity and access management, specifically the lack of multi-factor authentication (MFA) for computer and sensitive data system access. Furthermore, the absence of annual security awareness training for all employees elevates the risk of human error, such as falling victim to phishing attacks.

Technical analysis revealed the use of an unencrypted web service (HTTP) on a scanned internal host, which exposes data to interception. While the organization has foundational controls in place, such as an acceptable use policy and MFA for email, the identified vulnerabilities create a significant risk of unauthorized access and potential data compromise. Recommendations in this report are prioritized to address these critical gaps and strengthen the overall security framework.

% --- ORGANIZATIONAL INFORMATION ---
\section*{Organizational Information}
The following details were provided for the assessment scope.
\begin{itemize}
    \item \textbf{Organization Name:} Blue Marble
    \item \textbf{Email Domain:} \texttt{BlueMarble.net}
    \item \textbf{Primary Website:} \url{www.BlueMarble.net}
    \item \textbf{External IP Address:} \texttt{76.128.24.192}
\end{itemize}

% --- SECURITY CONTROL REVIEW ---
\section*{Security Control Review}
A review of the organization's security controls was conducted via a questionnaire. The responses highlight critical gaps in access control and employee training policies. A summary of the findings is presented in Table 1.

\begin{table}[h!]
\centering
\caption{Security Controls Questionnaire Results}
\label{tab:controls}
\begin{tabular}{@{}lc@{}}
\toprule
\textbf{Control Question} & \textbf{Response} \\ \midrule
Do you require MFA to access email? & \ding{51} \\
Do you require MFA to log into computers? & \textbf{\color{red}\ding{55}} \\
Do you require MFA to access sensitive data systems? & \textbf{\color{red}\ding{55}} \\
Does your organization have an employee acceptable use policy? & \ding{51} \\
Does your organization do security awareness training for new employees? & \ding{51} \\
Does your organization do security awareness training for all employees at least once per year? & \textbf{\color{red}\ding{55}} \\ \bottomrule
\end{tabular}
\end{table}

\paragraph{Analysis:} The lack of MFA on computer logins and sensitive data systems (\textbf{critical risk}) significantly increases the likelihood of a successful breach via stolen credentials. The absence of mandatory annual security training for all staff (\textbf{high risk}) leaves the organization vulnerable to social engineering and phishing attacks, which are primary vectors for credential theft.

% --- TECHNICAL SCAN RESULTS ---
\section*{Technical Scan Results}
A network scan was performed to identify open ports and services on the specified target system. The results indicate the presence of an unencrypted web service.

\begin{itemize}
    \item \textbf{Target IP Address:} \texttt{172.16.0.1}
    \item \textbf{Scan Date:} \today
\end{itemize}

\begin{table}[h!]
\centering
\caption{Open Port Analysis}
\label{tab:ports}
\begin{tabular}{@{}llll@{}}
\toprule
\textbf{Port} & \textbf{State} & \textbf{Service} & \textbf{Notes} \\ \midrule
80/tcp & Open & HTTP & Unencrypted web traffic. Exposes data to interception. \\ \bottomrule
\end{tabular}
\end{table}

\paragraph{Analysis:} The presence of an open HTTP port is a \textbf{high-risk} finding. The HTTP protocol does not encrypt data in transit, meaning that any information, including potential login credentials or sensitive content, can be captured and read by an attacker on the same network. All web services should be configured to use HTTPS (HTTP over TLS/SSL) to ensure confidentiality and integrity.

% --- CONSOLIDATED RISK ASSESSMENT ---
\section*{Consolidated Risk Assessment}
The following table synthesizes findings from the security control review and technical scan into a prioritized list of identified risks. Note: The provided "Current Risks" data contained a non-actionable, invalid entry and was excluded from this analysis.

\begin{table}[h!]
\centering
\caption{Summary of Identified Risks}
\label{tab:risks}
\begin{tabular}{@{}p{0.1\textwidth}p{0.6\textwidth}l@{}}
\toprule
\textbf{Risk ID} & \textbf{Description} & \textbf{Severity} \\ \midrule
RISK-001 & \textbf{Lack of MFA on Endpoints and Sensitive Systems:} User accounts for computers and sensitive data systems are protected only by passwords, making them highly vulnerable to takeover via credential theft. & Critical \\
\addlinespace
RISK-002 & \textbf{Unencrypted Web Traffic (HTTP):} The active HTTP service on \texttt{172.16.0.1} transmits data in cleartext, allowing for potential eavesdropping and session hijacking. & High \\
\addlinespace
RISK-003 & \textbf{Inadequate Security Awareness Training:} Without mandatory annual training, employees are more likely to be unaware of current threats, increasing the risk of successful phishing and social engineering attacks. & High \\ \bottomrule
\end{tabular}
\end{table}

% --- RECOMMENDATIONS ---
\section*{Recommendations}
To mitigate the identified risks and improve the overall security posture, the following actions are recommended:

\begin{enumerate}
    \item \textbf{Implement Comprehensive MFA (RISK-001):}
    \begin{itemize}
        \item \textbf{Immediate Action:} Prioritize and enforce MFA for all access to systems containing sensitive or critical data.
        \item \textbf{Long-Term Action:} Roll out MFA for all employee computer logins to protect endpoints from unauthorized access.
    \end{itemize}

    \item \textbf{Enforce Secure Web Communication (RISK-002):}
    \begin{itemize}
        \item \textbf{Immediate Action:} Reconfigure the web service on \texttt{172.16.0.1} to use HTTPS with a valid TLS certificate. Disable the HTTP port 80 or implement a permanent redirect to HTTPS port 443.
    \end{itemize}

    \item \textbf{Establish a Continuous Security Training Program (RISK-003):}
    \begin{itemize}
        \item \textbf{Immediate Action:} Develop and schedule mandatory annual security awareness training for all current employees.
        \item \textbf{Long-Term Action:} Incorporate periodic phishing simulations and ongoing security updates into the training program to ensure its continued effectiveness.
    \end{itemize}
\end{enumerate}

% --- CONCLUSION ---
\section*{Conclusion}
The assessment reveals that while Blue Marble has implemented some foundational security controls, there are critical deficiencies that expose the organization to significant cyber threats. The most pressing issues are the gaps in multi-factor authentication and the lack of a recurring security training program. By implementing the prioritized recommendations outlined in this report, Blue Marble can substantially reduce its risk exposure and build a more resilient security posture.

\end{document}
```