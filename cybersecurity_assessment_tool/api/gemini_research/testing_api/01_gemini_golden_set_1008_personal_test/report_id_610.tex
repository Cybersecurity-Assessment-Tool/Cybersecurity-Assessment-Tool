```latex
\documentclass[12pt]{article}

% --- PACKAGES ---
\usepackage[margin=1in]{geometry}
\usepackage{pifont} % For checkmarks and crosses
\usepackage{booktabs} % For professional tables
\usepackage{hyperref} % For clickable links
\usepackage{url} % For URL formatting
\usepackage{seqsplit} % For splitting long strings to prevent overflow

% --- DOCUMENT METADATA ---
\title{Cybersecurity Posture Assessment Report \\ \large For: \textbf{Modern Myth}}
\author{Cybersecurity Analysis Division}
\date{\today}

\hypersetup{
    colorlinks=true,
    linkcolor=black,
    urlcolor=blue,
    pdftitle={Cybersecurity Posture Assessment Report for Modern Myth},
    pdfauthor={Cybersecurity Analysis Division},
}

% --- DOCUMENT START ---
\begin{document}

\maketitle
\thispagestyle{empty}
\newpage

\tableofcontents
\newpage

% ==============================================================================
\section*{1.0 Executive Summary}
% ==============================================================================

This report details the findings of a cybersecurity posture assessment conducted for \textbf{Modern Myth}. The assessment incorporated an analysis of organizational security controls via a questionnaire, a technical network scan of the designated external asset, and a review of pre-existing risk documentation.

The external network scan of the target IP address \texttt{[Target IP]} did not reveal any open ports or services, suggesting a hardened external perimeter or that the asset was not responsive at the time of the scan. This is a positive finding from a network security perspective.

However, the security control review identified several critical gaps in the organization's policies and procedures. The most significant risks stem from the lack of Multi-Factor Authentication (MFA) for sensitive data systems and the absence of a formal security awareness training program for employees. These deficiencies expose \textbf{Modern Myth} to a heightened risk of unauthorized access, data breaches, and social engineering attacks.

This report provides a detailed breakdown of these findings and offers actionable recommendations to mitigate the identified risks and strengthen the overall security posture.

% ==============================================================================
\section*{2.0 Organizational Information}
% ==============================================================================

The following information was provided for the assessment.

\begin{tabular}{@{}ll}
\toprule
\textbf{Attribute} & \textbf{Value} \\
\midrule
Organization Name & \textbf{Modern Myth} \\
Email Domain & \texttt{ModernMyth.org} \\
Website Domain & \url{www.ModernMyth.org} \\
External IP Scanned & \texttt{147.118.13.6} \\
\bottomrule
\end{tabular}

% ==============================================================================
\section*{3.0 Security Control Review}
% ==============================================================================

An assessment of internal security controls was conducted based on a standardized questionnaire. The responses indicate areas of both strength and weakness. "No" answers represent significant gaps in the security framework and are correlated to risks identified in Section 5.0.

\begin{tabular}{@{}p{0.8\linewidth}c@{}}
\toprule
\textbf{Control Question} & \textbf{Response} \\
\midrule
Do you require MFA to access email? & \ding{51} \\ % Yes
Do you require MFA to log into computers? & \ding{51} \\ % Yes
\textbf{Do you require MFA to access sensitive data systems?} & \textbf{\color{red}\ding{55}} \\ % No
Does your organization have an employee acceptable use policy? & \ding{51} \\ % Yes
\textbf{Does your organization do security awareness training for new employees?} & \textbf{\color{red}\ding{55}} \\ % No
\textbf{Does your organization do security awareness training for all employees at least once per year?} & \textbf{\color{red}\ding{55}} \\ % No
\bottomrule
\end{tabular}

% ==============================================================================
\section*{4.0 Technical Scan Results}
% ==============================================================================

A network vulnerability scan was performed on the target system to identify potential weaknesses visible from the public internet.

\begin{itemize}
    \item \textbf{Target IP:} \texttt{[Target IP]}
    \item \textbf{Scan Date:} Current Assessment Period
\end{itemize}

\subsection*{Findings}
\textbf{No open ports or services were detected on the target system during the scan.}

This result indicates a strong network perimeter configuration for the scanned asset. No vulnerabilities related to exposed services could be identified at this time.

% ==============================================================================
\section*{5.0 Risk Assessment Summary}
% ==============================================================================

The following table summarizes the key risks identified during this assessment. These risks are derived from the security control gaps (Section 3.0), technical findings (Section 4.0), and any pre-existing risks provided. Given the inputs, the primary risks are procedural and policy-based.

\begin{tabular}{@{}p{0.1\linewidth}p{0.25\linewidth}p{0.45\linewidth}p{0.1\linewidth}@{}}
\toprule
\textbf{Risk ID} & \textbf{Risk Name} & \textbf{Overview} & \textbf{Severity} \\
\midrule
RISK-001 & Lack of MFA for Sensitive Systems & The absence of MFA on systems containing sensitive data drastically increases the risk of unauthorized access from compromised credentials. A threat actor with a valid username and password can gain direct access to critical assets. & \textbf{Critical} \\
\addlinespace
RISK-002 & Inadequate Security Awareness Training & The lack of a formal security awareness training program for new and existing employees leaves the organization highly susceptible to phishing, social engineering, and malware-based attacks that rely on human error. & \textbf{High} \\
\bottomrule
\end{tabular}

% ==============================================================================
\section*{6.0 Recommendations}
% ==============================================================================

The following actionable recommendations are provided to address the identified risks and improve the overall security posture of \textbf{Modern Myth}.

\subsection*{Recommendation for RISK-001: Implement MFA for Sensitive Systems}
\begin{itemize}
    \item \textbf{Action:} Enforce Multi-Factor Authentication (MFA) across all applications, databases, and administrative interfaces that store, process, or transmit sensitive organizational or customer data.
    \item \textbf{Priority:} \textbf{Critical}. This should be addressed immediately to prevent potential data breaches.
    \item \textbf{Impact:} Significantly reduces the risk of unauthorized access via stolen credentials, which is a primary vector in cyberattacks.
\end{itemize}

\subsection*{Recommendation for RISK-002: Establish a Security Training Program}
\begin{itemize}
    \item \textbf{Action:} Develop and implement a mandatory security awareness training program. The program should include initial training for all new hires and annual refresher training for all staff.
    \item \textbf{Priority:} \textbf{High}. A well-informed workforce is a critical layer of defense.
    \item \textbf{Impact:} Reduces the likelihood of successful phishing and social engineering attacks by empowering employees to recognize and report threats. This builds a stronger security culture within the organization.
\end{itemize}

% --- DOCUMENT END ---
\end{document}
```