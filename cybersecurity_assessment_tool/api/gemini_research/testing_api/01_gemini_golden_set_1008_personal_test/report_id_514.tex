```latex
\documentclass[12pt]{article}

% Preamble: Required Packages
\usepackage[margin=1in]{geometry}
\usepackage{pifont}
\usepackage{booktabs}
\usepackage{hyperref}
\usepackage{url}
\usepackage{seqsplit}
\usepackage{graphicx}

% Document Metadata
\title{Cybersecurity Posture Assessment Report}
\author{Cybersecurity Analysis Team}
\date{\today}

% Hyperref Setup
\hypersetup{
    colorlinks=true,
    linkcolor=black,
    urlcolor=blue,
    pdftitle={Cybersecurity Posture Assessment Report},
    pdfauthor={Cybersecurity Analysis Team},
}

\begin{document}

\maketitle
\thispagestyle{empty}
\newpage

\tableofcontents
\newpage

% --- 1. Executive Summary ---
\section{Executive Summary}
This report provides a comprehensive cybersecurity assessment for \textbf{Green Sprout Organic}, based on a review of organizational security controls, an external network scan, and an analysis of pre-existing risks. The assessment was conducted to identify vulnerabilities, evaluate the current security posture, and provide actionable recommendations for risk mitigation.

The analysis reveals a mixed security posture. The organization has implemented several important controls, such as requiring Multi-Factor Authentication (MFA) for email and sensitive data access. However, two significant gaps were identified that present a high level of risk:
\begin{itemize}
    \item \textbf{Critical Risk:} The absence of mandatory MFA for logging into company computers exposes the organization to significant risk from compromised credentials.
    \item \textbf{High Risk:} The lack of security awareness training for new employees leaves a critical window of vulnerability, as new hires are often targeted by social engineering attacks.
\end{itemize}

The external network scan of the target IP address \texttt{[Target IP]} did not identify any open ports. This is a positive finding, suggesting a well-configured firewall or that no services are intentionally exposed.

Recommendations in this report focus on immediately addressing the identified critical and high-risk gaps to strengthen the organization's overall defense against common cyber threats.

% --- 2. Organizational Information ---
\section{Organizational Information}
The following details were provided for the assessment.

\begin{tabular}{@{}ll}
    \toprule
    \textbf{Attribute} & \textbf{Value} \\
    \midrule
    Organization Name & \textbf{Green Sprout Organic} \\
    Primary Email Domain & \texttt{GreenSproutOrganic.net} \\
    Primary Website & \url{http://www.GreenSproutOrganic.net} \\
    External IP Address & \texttt{182.79.53.84} \\
    \bottomrule
\end{tabular}

% --- 3. Security Control Review ---
\section{Security Control Review}
A review of the organization's security controls was conducted based on a standardized questionnaire. The findings are summarized below. "No" answers indicate potential gaps in the security framework.

\begin{tabular}{@{}p{0.6\linewidth} c p{0.2\linewidth}@{}}
    \toprule
    \textbf{Control Question} & \textbf{Response} & \textbf{Assessment} \\
    \midrule
    Do you require MFA to access email? & \ding{51} Yes & Best Practice Met \\
    \addlinespace
    Do you require MFA to log into computers? & \color{red}\ding{55} No & \textbf{Critical Gap} \\
    \addlinespace
    Do you require MFA to access sensitive data systems? & \ding{51} Yes & Best Practice Met \\
    \addlinespace
    Does your organization have an employee acceptable use policy? & \ding{51} Yes & Good \\
    \addlinespace
    Does your organization do security awareness training for new employees? & \color{red}\ding{55} No & \textbf{High Risk} \\
    \addlinespace
    Does your organization do security awareness training for all employees at least once per year? & \ding{51} Yes & Good \\
    \bottomrule
\end{tabular}

% --- 4. Technical Scan Results ---
\section{Technical Scan Results}
An external network vulnerability scan was performed to identify open ports and exposed services.

\begin{itemize}
    \item \textbf{Target IP Address:} \texttt{[Target IP]}
    \item \textbf{Scan Date:} 2023-10-27 (Date of data processing)
\end{itemize}

\subsection{Summary of Findings}
The scan completed successfully but did not detect any open TCP or UDP ports on the target host. This indicates that the host is likely protected by a firewall that is correctly configured to block unsolicited inbound traffic. This is a positive security finding.

% --- 5. Risk Assessment ---
\section{Risk Assessment}
This section synthesizes findings from the security control review and technical scan. No pre-existing vulnerabilities were reported. The primary risks identified are related to internal security policies and procedures.

\begin{tabular}{@{}p{0.25\linewidth} p{0.5\linewidth} p{0.15\linewidth}@{}}
    \toprule
    \textbf{Risk Name} & \textbf{Overview} & \textbf{Severity} \\
    \midrule
    Lack of Endpoint MFA & The absence of MFA on computer logins means a compromised password is all an attacker needs to gain access to an employee's workstation and potentially the internal network. & \textbf{Critical} \\
    \addlinespace
    No Security Training During Onboarding & New employees are not receiving security awareness training upon joining. This makes them highly susceptible to phishing, social engineering, and other common initial access attacks before they are integrated into the annual training cycle. & \textbf{High} \\
    \bottomrule
\end{tabular}

% --- 6. Recommendations ---
\section{Recommendations}
The following actions are recommended to mitigate the identified risks and improve the overall security posture of \textbf{Green Sprout Organic}.

\subsection{Immediate Actions (Critical \& High Risks)}
\begin{enumerate}
    \item \textbf{Implement MFA for All Endpoint Logins (Critical):}
    \begin{itemize}
        \item \textbf{Action:} Deploy and enforce a Multi-Factor Authentication solution (e.g., authenticator app, hardware token, or biometrics) for all user logins to desktops and laptops.
        \item \textbf{Justification:} This single control dramatically reduces the risk of unauthorized access from stolen or weak credentials, protecting endpoints from being the initial entry point for an attack.
    \end{itemize}
    \vspace{1em}
    \item \textbf{Integrate Security Training into New Employee Onboarding (High):}
    \begin{itemize}
        \item \textbf{Action:} Develop a mandatory security awareness training module that is a required part of the onboarding process for all new hires, to be completed within their first week of employment.
        \item \textbf{Justification:} This ensures that new staff are immediately aware of company security policies, how to identify and report phishing attempts, and their responsibilities in protecting company data. This closes a significant window of vulnerability.
    \end{itemize}
\end{enumerate}

\subsection{General Recommendations}
\begin{itemize}
    \item \textbf{Continuous Monitoring:} Continue to perform regular external and internal vulnerability scans to ensure new services are not inadvertently exposed.
    \item \textbf{Policy Review:} Periodically review and update the Acceptable Use Policy and other security documentation to reflect changes in technology and the threat landscape.
\end{itemize}

\end{document}
```