```latex
\documentclass[12pt, a4paper]{article}

% Preamble: Required Packages
\usepackage[margin=1in]{geometry}
\usepackage{pifont} % For \ding
\usepackage{booktabs} % For professional tables
\usepackage{hyperref} % For clickable links
\usepackage{url} % For URL formatting
\usepackage{seqsplit} % To split long strings in texttt
\usepackage{xcolor} % For colors
\usepackage{graphicx} % For potential logos/images
\usepackage{fancyhdr} % For headers and footers

% --- Document Setup ---
\hypersetup{
    colorlinks=true,
    linkcolor=blue,
    filecolor=magenta,      
    urlcolor=cyan,
    pdftitle={Cybersecurity Posture Assessment Report},
    pdfauthor={Cybersecurity Analyst},
    pdfsubject={Security Analysis},
    pdfkeywords={Security, Report, Analysis},
}

% Define colors for severity
\definecolor{criticalred}{HTML}{D10000}
\definecolor{highorange}{HTML}{E57300}
\definecolor{mediumyellow}{HTML}{FFBF00}
\definecolor{lowblue}{HTML}{0073E6}

% --- Header and Footer ---
\pagestyle{fancy}
\fancyhf{}
\fancyhead[L]{Cybersecurity Posture Assessment}
\fancyhead[R]{\textbf{Hearth \& Home}}
\fancyfoot[C]{\thepage}
\renewcommand{\headrulewidth}{0.4pt}
\renewcommand{\footrulewidth}{0.4pt}

% --- Document Start ---
\begin{document}

% --- Title Page ---
\begin{titlepage}
    \centering
    \vspace*{1cm}
    
    \Huge
    \textbf{Cybersecurity Posture Assessment Report}
    
    \vspace{1.5cm}
    
    \Large
    Prepared for: \\
    \vspace{0.5cm}
    \textbf{Hearth \& Home}
    
    \vspace{2cm}
    
    \large
    \textbf{Date of Report:} \today
    
    \vfill
    
    \large
    \textbf{CONFIDENTIAL}
    
    \vspace{0.8cm}
    
    \normalsize
    This document contains sensitive information. Access is restricted to authorized personnel only. Unauthorized distribution is strictly prohibited.
    
\end{titlepage}

\tableofcontents
\newpage

% --- Section 1: Executive Overview ---
\section{Executive Overview}
This report provides a comprehensive cybersecurity assessment for \textbf{Hearth \& Home}, based on an analysis of network scan data, organizational security controls, and known existing risks. The assessment reveals several critical and high-severity vulnerabilities that expose the organization to significant threats, including data breaches, ransomware, and unauthorized access.

The overall security posture is determined to be \textbf{critically weak}. Key findings include:
\begin{itemize}
    \item \textbf{Critically Vulnerable External Service:} A public-facing FTP server is running a dangerously outdated version of vsftpd (2.3.4) from 2011, which is susceptible to remote code execution (CVE-2011-2523).
    \item \textbf{Lack of Multi-Factor Authentication (MFA):} MFA is not enforced for email or computer access, drastically increasing the risk of account compromise from phishing or credential theft.
    \item \textbf{Deficient Security Policies and Training:} The organization lacks a formal Acceptable Use Policy and does not conduct security awareness training for employees. This creates a high-risk environment where employees are more likely to fall victim to social engineering attacks.
    \item \textbf{Insecure Network Configuration:} The vulnerable FTP server also permits anonymous login, allowing unauthenticated users to access files, which could lead to data leakage or the introduction of malware.
\end{itemize}

Immediate remediation of these issues is strongly recommended to mitigate the imminent risk of a significant security incident. Detailed findings and actionable recommendations are provided in the subsequent sections.

% --- Section 2: Organizational Information ---
\section{Organizational Information}
The following details were provided for the assessment. This information helps to establish the context and scope of the review.

\begin{table}[h!]
\centering
\begin{tabular}{@{}ll@{}}
\toprule
\textbf{Attribute} & \textbf{Value} \\ \midrule
Organization Name & \textbf{Hearth \& Home} \\
Email Domain & \texttt{HearthHome.net} \\
Website Domain & \seqsplit{\url{www.HearthHome.net}} \\
External IP Address & \texttt{195.129.219.134} \\ \bottomrule
\end{tabular}
\caption{Client Organizational Details}
\end{table}

% --- Section 3: Security Control Review ---
\section{Security Control Review}
A review of organizational security controls was conducted via a questionnaire. The responses indicate significant gaps in foundational security practices. The results are summarized below.

\begin{table}[h!]
\centering
\begin{tabular}{@{}lc@{}}
\toprule
\textbf{Security Control Question} & \textbf{Response} \\ \midrule
Do you require MFA to access email? & \textcolor{criticalred}{\ding{55}} \\
Do you require MFA to log into computers? & \textcolor{criticalred}{\ding{55}} \\
Do you require MFA to access sensitive data systems? & \textcolor{lowblue}{\ding{51}} \\
Does your organization have an employee acceptable use policy? & \textcolor{highorange}{\ding{55}} \\
Does your organization do security awareness training for new employees? & \textcolor{criticalred}{\ding{55}} \\
Does your organization do security awareness training for all employees annually? & \textcolor{criticalred}{\ding{55}} \\ \bottomrule
\end{tabular}
\caption{Security Controls Questionnaire Results}
\end{table}

\subsection{Analysis of Control Gaps}
The responses highlighted with a \textcolor{criticalred}{\ding{55}} (No) represent significant weaknesses:
\begin{itemize}
    \item \textbf{Lack of MFA:} The absence of MFA for email and computer logins is a critical vulnerability. Email is a primary target for attackers, and a compromised account can lead to widespread system access, data theft, and financial fraud.
    \item \textbf{Lack of Security Training:} Without security awareness training, employees are ill-equipped to identify and report phishing attempts, malware, or other social engineering tactics. This makes the organization highly susceptible to initial access attempts by attackers.
    \item \textbf{Missing Acceptable Use Policy:} An Acceptable Use Policy (AUP) is a foundational document that sets clear expectations for employee behavior when using company resources. Its absence can lead to unintentional security risks and complicates incident response.
\end{itemize}

% --- Section 4: Technical Scan Results ---
\section{Technical Scan Results}
A network scan was performed to identify open ports and services on the target system. The findings reveal a critical misconfiguration on a network host.

\subsection{Host: \texttt{10.0.0.15}}
\begin{itemize}
    \item \textbf{Status:} Up
    \item \textbf{Summary:} The host is responsive and exposes a highly vulnerable FTP service.
\end{itemize}

\begin{table}[h!]
\centering
\begin{tabular}{@{}lllll@{}}
\toprule
\textbf{Port} & \textbf{Service} & \textbf{Product} & \textbf{Version} & \textbf{Finding} \\ \midrule
21/tcp & ftp & vsftpd & 2.3.4 & \begin{tabular}[c]{@{}l@{}}\textbf{Critical Vulnerability (CVE-2011-2523)}\\ Anonymous FTP Login Allowed\end{tabular} \\ \bottomrule
\end{tabular}
\caption{Open Ports and Services on \texttt{10.0.0.15}}
\end{table}

\subsection{Analysis of Technical Findings}
\begin{itemize}
    \item \textbf{Vulnerable FTP Service (CVE-2011-2523):} The version of vsftpd (2.3.4) running on port 21 is over a decade old and contains a well-known critical vulnerability. An attacker can exploit this flaw to gain a remote shell and execute arbitrary code on the server, leading to a full system compromise.
    \item \textbf{Anonymous FTP Access:} The server is configured to allow anonymous logins. This allows any user on the internet to connect and potentially access, download, or upload files. This configuration poses a severe risk of data leakage and could allow an attacker to plant malware on the system.
\end{itemize}

% --- Section 5: Consolidated Risk Assessment ---
\section{Consolidated Risk Assessment}
This section synthesizes findings from the security control review, technical scan, and pre-existing risk data into a prioritized list of risks facing the organization.

\begin{table}[h!]
\centering
\resizebox{\textwidth}{!}{%
\begin{tabular}{@{}llll@{}}
\toprule
\textbf{Risk Description} & \textbf{Severity} & \textbf{Affected Systems} & \textbf{Finding Source} \\ \midrule
\begin{tabular}[c]{@{}l@{}}Vulnerable FTP Server (vsftpd 2.3.4)\\ allows remote code execution.\end{tabular} & \textcolor{criticalred}{\textbf{Critical}} & Server at \texttt{10.0.0.15} & Network Scan \\
\hline
\begin{tabular}[c]{@{}l@{}}Lack of MFA for email and computer\\ access exposes accounts to takeover.\end{tabular} & \textcolor{criticalred}{\textbf{Critical}} & All Employees, Email System & Questionnaire \\
\hline
\begin{tabular}[c]{@{}l@{}}No security awareness training program\\ increases susceptibility to phishing.\end{tabular} & \textcolor{criticalred}{\textbf{Critical}} & All Employees & Questionnaire \\
\hline
\begin{tabular}[c]{@{}l@{}}Outdated Windows 7 operating systems\\ are unsupported and vulnerable.\end{tabular} & \textcolor{highorange}{\textbf{High}} & Workstations & Pre-existing Risk \\
\hline
\begin{tabular}[c]{@{}l@{}}Anonymous FTP login enabled, risking\\ data leakage or malware upload.\end{tabular} & \textcolor{highorange}{\textbf{High}} & Server at \texttt{10.0.0.15} & Network Scan \\
\hline
\begin{tabular}[c]{@{}l@{}}Absence of an Acceptable Use Policy\\ leads to inconsistent security behavior.\end{tabular} & \textcolor{mediumyellow}{\textbf{Medium}} & Organization-wide Policy & Questionnaire \\ \bottomrule
\end{tabular}%
}
\caption{Summary of Identified Risks}
\end{table}

% --- Section 6: Recommendations ---
\section{Recommendations}
The following actions are recommended to address the identified risks. They are prioritized to ensure the most critical vulnerabilities are remediated first.

\subsection{Immediate Actions (Next 72 Hours)}
\begin{enumerate}
    \item \textbf{Remediate Vulnerable FTP Server:} Immediately take the server at \texttt{10.0.0.15} offline. If the FTP service is business-critical, it must be upgraded to a modern, patched version. If it is not, the service should be permanently disabled.
    \item \textbf{Disable Anonymous FTP:} Regardless of the upgrade path, anonymous FTP access must be disabled immediately to prevent unauthorized access.
    \item \textbf{Enforce MFA on Critical Accounts:} Begin the emergency rollout of MFA for all administrative accounts and key personnel, focusing on email and remote access systems.
\end{enumerate}

\subsection{Short-Term Actions (Next 30-90 Days)}
\begin{enumerate}
    \item \textbf{Full MFA Deployment:} Complete the rollout of MFA for all employees across email, VPN, and computer login systems.
    \item \textbf{Develop and Implement Security Training:} Contract with a security awareness training provider or develop an in-house program. Conduct an initial training session for all employees covering phishing, password hygiene, and acceptable use.
    \item \textbf{Plan Windows 7 Upgrades:} Create a formal project plan to upgrade or replace all workstations running Windows 7. Prioritize systems used by employees with access to sensitive data.
    \item \textbf{Draft an Acceptable Use Policy (AUP):} Develop and ratify a formal AUP. Require all employees to read and acknowledge the policy.
\end{enumerate}

\subsection{Long-Term Strategic Actions (6-12 Months)}
\begin{enumerate}
    \item \textbf{Establish a Vulnerability Management Program:} Implement a formal process for regularly scanning internal and external systems for vulnerabilities, tracking findings, and ensuring timely remediation.
    \item \textbf{Conduct Regular Security Assessments:} Schedule annual penetration tests and vulnerability assessments with a third-party security firm to validate security controls and identify new weaknesses.
\end{enumerate}

\end{document}
```