```latex
\documentclass[12pt]{article}

% Preamble: Required Packages
\usepackage[a4paper, margin=1in]{geometry} % Page layout
\usepackage{pifont}                      % For checkmarks and crosses (\ding)
\usepackage{booktabs}                    % For professional tables
\usepackage{graphicx}                    % For logos, etc. (optional)
\usepackage[table]{xcolor}               % For colors in tables
\usepackage{hyperref}                    % For clickable links and references
\usepackage{url}                         % For formatting URLs
\usepackage{seqsplit}                    % For splitting long strings without spaces

% Hyperref Setup
\hypersetup{
    colorlinks=true,
    linkcolor=black,
    filecolor=magenta,      
    urlcolor=blue,
    pdftitle={Cybersecurity Assessment Report},
    pdfauthor={Cybersecurity Analyst},
    pdfsubject={Security Assessment},
    pdfkeywords={Security, Report, Analysis},
    bookmarks=true,
    bookmarksopen=true
}

% Define custom colors
\definecolor{lightgray}{gray}{0.9}
\definecolor{severitycritical}{HTML}{990000}
\definecolor{severityhigh}{HTML}{D14302}
\definecolor{severitymedium}{HTML}{E5A50A}

% Document Start
\begin{document}

% --- Title Page ---
\begin{titlepage}
    \centering
    \vspace*{2cm}
    \Huge\textbf{Cybersecurity Assessment Report}
    \vspace{1.5cm}
    \Large\textbf{Prepared for:} \\
    \vspace{0.5cm}
    \huge Echo Chamber Arts
    \vfill
    \large
    \textbf{Date of Report:} \today \\
    \vspace{0.5cm}
    \textbf{Analysis Period:} October 2023 \\
    \vspace{0.5cm}
    \textbf{Author:} Cybersecurity Analyst
\end{titlepage}

% --- Table of Contents ---
\tableofcontents
\newpage

% --- Section 1: Executive Summary ---
\section{Executive Summary}

This report details the findings of a cybersecurity assessment conducted for Echo Chamber Arts. The analysis synthesized data from an external network scan, a security controls questionnaire, and a review of pre-existing documented risks.

The assessment reveals a \textbf{critical risk posture}. Several significant vulnerabilities and procedural gaps were identified that expose the organization to a high likelihood of a security breach. 

Key findings include:
\begin{itemize}
    \item \textbf{Critical Service Exposure:} A network service (SSH on port 22) intended for local machine administration (\texttt{127.0.0.1}) is exposed to the public internet. This represents a direct and severe threat.
    \item \textbf{Lack of Multi-Factor Authentication (MFA):} MFA is not enforced for email or computer access. This is a critical gap, as compromised credentials could lead to an immediate and widespread system compromise.
    \item \textbf{Policy and Training Deficiencies:} The absence of a formal Acceptable Use Policy and mandatory security training for new employees creates a weak human security layer, making the organization highly susceptible to phishing and social engineering attacks.
\end{itemize}

Immediate remediation of the exposed service and the implementation of MFA are paramount to reducing the organization's risk profile. Further recommendations are detailed in Section \ref{sec:recommendations} to address all identified gaps and establish a more resilient security posture.

% --- Section 2: Organizational Information ---
\section{Organizational Information}

The following details were provided for the assessment scope. This information is used to contextualize the findings and tailor recommendations.

\begin{tabular}{@{}ll}
    \toprule
    \textbf{Attribute} & \textbf{Value} \\
    \midrule
    Organization Name & Echo Chamber Arts \\
    Primary Email Domain & \texttt{EchoChamberArts.org} \\
    Primary Website & \url{www.EchoChamberArts.org} \\
    External IP Address & \texttt{214.82.49.43} \\
    \bottomrule
\end{tabular}

% --- Section 3: Security Control Review ---
\section{Security Control Review}

A questionnaire was completed to evaluate the current state of administrative and procedural security controls. The responses highlight significant gaps in foundational security practices.

\begin{table}[h!]
\centering
\caption{Security Controls Questionnaire Analysis}
\label{tab:controls}
\begin{tabular}{p{9cm}cc}
    \toprule
    \textbf{Control Question} & \textbf{Response} & \textbf{Assessment} \\
    \midrule
    \rowcolor{lightgray}
    Do you require MFA to access email? & \textbf{No} & \textcolor{severitycritical}{\ding{55} Critical Gap} \\
    Do you require MFA to log into computers? & \textbf{No} & \textcolor{severityhigh}{\ding{55} High Risk} \\
    \rowcolor{lightgray}
    Do you require MFA to access sensitive data systems? & Yes & \textcolor{green}{\ding{51} Best Practice Met} \\
    Does your organization have an employee acceptable use policy? & \textbf{No} & \textcolor{severitymedium}{\ding{55} Policy Gap} \\
    \rowcolor{lightgray}
    Does your organization do security awareness training for new employees? & \textbf{No} & \textcolor{severityhigh}{\ding{55} High Risk} \\
    Does your organization do security awareness training for all employees at least once per year? & Yes & \textcolor{green}{\ding{51} Best Practice Met} \\
    \bottomrule
\end{tabular}
\end{table}

% --- Section 4: Technical Scan Results ---
\section{Technical Scan Results}

An external network scan was performed on the provided IP address to identify open ports and exposed services.

\begin{itemize}
    \item \textbf{Scan Target:} \texttt{127.0.0.1} (Correlated with pre-existing risk data)
    \item \textbf{Scan Tool:} Nmap
\end{itemize}

The scan identified the following open port:

\begin{table}[h!]
\centering
\caption{Open Port Analysis}
\label{tab:ports}
\begin{tabular}{lllll}
    \toprule
    \textbf{Port} & \textbf{State} & \textbf{Service} & \textbf{Product/Version} & \textbf{Notes} \\
    \midrule
    22/tcp & Open & SSH & Not Identified & \begin{tabular}[t]{@{}l@{}}Exposing SSH to the internet is extremely risky. \\ The service is running on the localhost interface, \\ which should not be publicly accessible.\end{tabular} \\
    \bottomrule
\end{tabular}
\end{table}

\textbf{Analysis:} The presence of an open SSH port on the localhost address (\texttt{127.0.0.1}) is a critical misconfiguration. This suggests a firewall or network routing issue is exposing an internal management interface directly to the public internet. This finding corroborates the pre-existing risk titled "Localhost Exposed" and elevates its urgency.

% --- Section 5: Consolidated Risk Assessment ---
\section{Consolidated Risk Assessment}

This section synthesizes findings from the questionnaire, technical scan, and pre-existing risk data into a unified list of security risks facing the organization.

\begin{table}[h!]
\centering
\caption{Summary of Identified Risks}
\label{tab:risks}
\begin{tabular}{p{4.5cm}p{6.5cm}l}
    \toprule
    \textbf{Risk Name} & \textbf{Description} & \textbf{Severity} \\
    \midrule
    \rowcolor{lightgray}
    Localhost Exposed via SSH & The SSH management service on port 22 is accessible from the public internet, providing a direct vector for attackers to attempt brute-force or credential-based attacks. & \textcolor{severitycritical}{Critical} \\
    
    Lack of MFA for Email & Without MFA, a single compromised password is all an attacker needs to gain full access to an employee's mailbox, leading to data theft, phishing, and further compromise. & \textcolor{severitycritical}{Critical} \\
    
    \rowcolor{lightgray}
    No Security Training for New Hires & New employees are a primary target for attackers. Without initial training, they are highly likely to fall victim to phishing or other social engineering tactics. & \textcolor{severityhigh}{High} \\
    
    Lack of MFA for Computer Login & Stolen credentials could allow an attacker to log directly into a company computer, gaining access to the internal network and sensitive files. & \textcolor{severityhigh}{High} \\
    
    \rowcolor{lightgray}
    Missing Acceptable Use Policy (AUP) & The absence of a formal AUP means there are no clear guidelines for employees on safe technology use, creating legal and security ambiguities. & \textcolor{severitymedium}{Medium} \\
    \bottomrule
\end{tabular}
\end{table}

% --- Section 6: Recommendations ---
\section{Recommendations}
\label{sec:recommendations}

The following actions are recommended to mitigate the identified risks. They are prioritized based on severity and potential impact.

\subsection*{Critical Priority}
\begin{itemize}
    \item \textbf{Remediate Exposed SSH Service:} Immediately reconfigure the firewall and network routing rules to ensure that port 22 on the localhost interface (\texttt{127.0.0.1}) is not accessible from the public internet. Access to management services should be restricted to trusted internal IP addresses or a VPN.
    \item \textbf{Enforce MFA on All Email Accounts:} Immediately enable and enforce Multi-Factor Authentication for all email accounts. This is the single most effective control to prevent account takeovers resulting from compromised credentials.
\end{itemize}

\subsection*{High Priority}
\begin{itemize}
    \item \textbf{Implement MFA for Computer Logins:} Deploy MFA for all employee computer and laptop logins to protect against unauthorized access to endpoint devices and the internal network.
    \item \textbf{Develop New Hire Security Training:} Create a mandatory security awareness training module that all new employees must complete as part of their onboarding process. This should cover phishing, password security, and acceptable use.
\end{itemize}

\subsection*{Medium Priority}
\begin{itemize}
    \item \textbf{Establish an Acceptable Use Policy (AUP):} Develop and implement a formal AUP that clearly defines the rules and expectations for using company technology and data. All employees should be required to read and acknowledge the policy.
\end{itemize}

\end{document}
```