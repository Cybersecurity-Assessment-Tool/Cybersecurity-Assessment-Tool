```latex
\documentclass[12pt]{article}

% Preamble: Required Packages and Document Setup
\usepackage[margin=1in]{geometry}
\usepackage{pifont} % For checkmarks and crosses
\usepackage{booktabs} % For professional tables
\usepackage{hyperref} % For clickable links
\usepackage{url} % For URL formatting
\usepackage{seqsplit} % For splitting long strings in tt font
\usepackage{graphicx}
\usepackage{xcolor}
\usepackage{fancyhdr}
\usepackage{lastpage}

% --- Document Metadata ---
\title{Cybersecurity Posture Assessment Report}
\author{Cybersecurity Analysis Division}
\date{November 22, 2025}

% --- Custom Commands & Settings ---
\hypersetup{
    colorlinks=true,
    linkcolor=blue,
    filecolor=magenta,      
    urlcolor=cyan,
    pdftitle={Cybersecurity Posture Assessment Report},
    pdfpagemode=FullScreen,
}

% Define colors for risk levels
\definecolor{severitycritical}{HTML}{990000}
\definecolor{severityhigh}{HTML}{D13200}
\definecolor{severitymedium}{HTML}{FFC300}

% --- Header and Footer ---
\pagestyle{fancy}
\fancyhf{} % Clear all header and footer fields
\fancyhead[L]{\textbf{Cybersecurity Assessment Report}}
\fancyhead[R]{\textbf{[Organization Name]}}
\fancyfoot[C]{Page \thepage\ of \pageref{LastPage}}
\fancyfoot[R]{\small Confidential}
\renewcommand{\headrulewidth}{0.4pt}
\renewcommand{\footrulewidth}{0.4pt}

% --- Document Body ---
\begin{document}

\maketitle
\thispagestyle{empty}
\newpage

\tableofcontents
\newpage

% ==============================================================================
\section{Executive Summary}
% ==============================================================================

This report details the findings of a cybersecurity posture assessment conducted for \textbf{[Organization Name]} on November 22, 2025. The assessment combined a review of organizational security controls, a technical network scan of external-facing assets, and an analysis of pre-existing risks.

\paragraph{Key Findings:} The organization demonstrates a strong foundation in administrative and procedural security. The security controls questionnaire revealed a mature approach to essential safeguards, including the mandatory use of Multi-Factor Authentication (MFA) and a robust security awareness training program. These are commendable practices that significantly reduce risks related to human error and account compromise.

However, a critical technical vulnerability was identified. The external network scan of the target IP address \texttt{[Target IP]} revealed a public-facing web server running an outdated version of Nginx (1.18.0). This software version, released in 2020, is no longer supported and is known to be vulnerable to numerous publicly disclosed exploits. This finding represents a high-priority risk that could be leveraged by attackers to compromise the server, disrupt services, or access sensitive data.

\paragraph{Overall Posture:} The organization's overall security posture is assessed as \textbf{Guarded}. While procedural controls are strong, the presence of a significant, unmitigated technical vulnerability on a critical public asset requires immediate attention. Recommendations in this report are prioritized to address this finding and enhance the organization's proactive security management.

% ==============================================================================
\section{Organizational Information}
% ==============================================================================

The following information was used as the basis for this assessment. Due to the anonymized nature of the provided data, placeholders have been used where necessary.

\begin{itemize}
    \item \textbf{Organization Name:} \textbf{[Organization Name]}
    \item \textbf{Primary Email Domain:} \texttt{[Domain]}
    \item \textbf{Known External IP:} \texttt{[Client IP]}
    \item \textbf{Assessed Target IP:} \texttt{[Target IP]}
\end{itemize}

% ==============================================================================
\section{Security Control Review}
% ==============================================================================

A review of the organization's security controls was conducted via a standardized questionnaire. The responses indicate a strong commitment to fundamental security practices. A summary of the responses is provided in Table \ref{tab:controls}. The green checkmark (\textcolor{green}{\ding{51}}) indicates an affirmative response, which aligns with security best practices.

\begin{table}[h!]
\centering
\caption{Organizational Security Controls Questionnaire}
\label{tab:controls}
\begin{tabular}{p{0.75\textwidth} c}
\toprule
\textbf{Control Question} & \textbf{Response} \\
\midrule
Do you require MFA to access email? & \textcolor{green}{\ding{51}} \\
Do you require MFA to log into computers? & \textcolor{green}{\ding{51}} \\
Do you require MFA to access sensitive data systems? & \textcolor{green}{\ding{51}} \\
Does your organization have an employee acceptable use policy? & \textcolor{green}{\ding{51}} \\
Does your organization do security awareness training for new employees? & \textcolor{green}{\ding{51}} \\
Does your organization do security awareness training for all employees at least once per year? & \textcolor{green}{\ding{51}} \\
\bottomrule
\end{tabular}
\end{table}

\paragraph{Analysis:} All responses were affirmative, indicating that the organization has implemented critical administrative controls. These measures are effective at mitigating a wide range of common cyber threats, particularly phishing, credential theft, and insider threats.

% ==============================================================================
\section{Technical Scan Results}
% ==============================================================================

An external network scan was performed to identify open ports and services visible on the public internet.

\begin{itemize}
    \item \textbf{Scan Date:} November 22, 2025
    \item \textbf{Target IP Address:} \texttt{[Target IP]}
\end{itemize}

\begin{table}[h!]
\centering
\caption{Open Ports and Services Detected}
\label{tab:nmap}
\begin{tabular}{l l l l l}
\toprule
\textbf{Port} & \textbf{State} & \textbf{Service} & \textbf{Product} & \textbf{Version} \\
\midrule
443/TCP & Open & HTTPS & Nginx & 1.18.0 \\
\bottomrule
\end{tabular}
\end{table}

\paragraph{Analysis:} The scan identified a single open port, 443/TCP, which is standard for hosting a secure website via HTTPS. The service was identified as Nginx version 1.18.0. This version was released in April 2020 and is considered outdated and end-of-life. It is missing several years of critical security patches, making it an attractive target for automated exploitation tools. This represents the most significant technical risk discovered during the assessment.

% ==============================================================================
\section{Risk Assessment}
% ==============================================================================

This section synthesizes findings from the security control review and technical scan. The pre-existing risk register was empty. A new high-severity risk has been identified based on the technical scan results.

\begin{table}[h!]
\centering
\caption{Identified Risks}
\label{tab:risks}
\begin{tabular}{p{0.1\textwidth} p{0.3\textwidth} p{0.15\textwidth} p{0.35\textwidth}}
\toprule
\textbf{ID} & \textbf{Risk Name} & \textbf{Severity} & \textbf{Description} \\
\midrule
RISK-001 & Outdated Web Server Software (Nginx 1.18.0) & \textcolor{severityhigh}{\textbf{High}} & The public-facing web server at \texttt{[Target IP]} is running a vulnerable, end-of-life version of Nginx. This exposes the server to numerous known vulnerabilities, which could lead to remote code execution, denial of service, or a full system compromise. \\
\bottomrule
\end{tabular}
\end{table}

% ==============================================================================
\section{Recommendations}
% ==============================================================================

The following prioritized recommendations are provided to address the identified risks and improve the overall security posture of \textbf{[Organization Name]}.

\subsection*{Priority 1: Remediate Critical Vulnerability}

\begin{description}
    \item[Recommendation:] \textbf{Upgrade Nginx Server Immediately.}
    \item[Action:] Plan and execute an upgrade of the Nginx server at \texttt{[Target IP]} from version 1.18.0 to the latest stable version recommended by the vendor. This process should include pre-upgrade testing in a staging environment to ensure application compatibility.
    \item[Justification:] This is the most critical action required. Mitigating this vulnerability closes a direct and easily exploitable pathway for attackers, protecting the organization's data, reputation, and service availability.
\end{description}

\subsection*{Priority 2: Enhance Proactive Security}

\begin{description}
    \item[Recommendation:] \textbf{Implement a Formal Vulnerability Management Program.}
    \item[Action:] Establish a documented process to regularly scan, identify, prioritize, and remediate vulnerabilities across all external and internal systems. This program should define timelines for patching based on vulnerability severity.
    \item[Justification:] The presence of severely outdated software suggests a potential gap in proactive vulnerability management. A formal program ensures that such risks are identified and addressed systematically, rather than reactively, preventing future occurrences.
\end{description}

\subsection*{Priority 3: Maintain and Review Controls}

\begin{description}
    \item[Recommendation:] \textbf{Continue and Formalize Review of Administrative Controls.}
    \item[Action:] Maintain the excellent security controls currently in place (MFA, security training, etc.). Schedule an annual review of these policies to ensure they remain effective and aligned with the evolving threat landscape and business requirements.
    \item[Justification:] Strong controls provide a robust defense-in-depth strategy. Regular reviews ensure their continued effectiveness and demonstrate due diligence.
\end{description}

\end{document}
```