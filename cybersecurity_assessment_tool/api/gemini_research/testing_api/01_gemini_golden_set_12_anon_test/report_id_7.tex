```latex
\documentclass[12pt]{article}

% --- PACKAGES ---
\usepackage[margin=1in]{geometry}
\usepackage{pifont} % For checkmarks and crosses
\usepackage{booktabs} % For professional tables
\usepackage{hyperref} % For hyperlinks
\usepackage{url}      % For URL formatting
\usepackage{seqsplit} % For splitting long strings in texttt
\usepackage[utf8]{inputenc}

% --- DOCUMENT METADATA ---
\title{Cybersecurity Assessment Report \\ for \\ \textbf{[Organization Name]}}
\author{Cybersecurity Analyst}
\date{\today}

% --- HYPERREF SETUP ---
\hypersetup{
    colorlinks=true,
    linkcolor=black,
    filecolor=magenta,      
    urlcolor=blue,
    pdftitle={Cybersecurity Assessment Report},
    pdfpagemode=FullScreen,
}

\begin{document}

\maketitle
\newpage

\tableofcontents
\newpage

% --- EXECUTIVE SUMMARY ---
\section{Executive Summary}

This report details the findings of a cybersecurity assessment conducted for \textbf{[Organization Name]}. The analysis is based on a combination of network scanning, a review of organizational security controls, and an evaluation of pre-existing risks.

The assessment reveals a critically deficient security posture. Foundational security controls, such as Multi-Factor Authentication (MFA) and employee security awareness training, are entirely absent. This leaves the organization highly vulnerable to common cyberattacks, including phishing, credential theft, and unauthorized access.

Furthermore, a technical scan identified an exposed management service (SSH) on the external network perimeter. When combined with the lack of MFA and user training, this exposed service presents a high-probability attack vector for malicious actors.

Immediate and decisive action is required to remediate these critical-risk findings. Recommendations are prioritized to address the most severe threats first, focusing on implementing MFA, establishing a security training program, and securing the network perimeter.

% --- ORGANIZATIONAL INFORMATION ---
\section{Organizational Information}
This section provides the key identifying information for the organization under review. As the provided data was anonymized, placeholders have been used.

\begin{itemize}
    \item \textbf{Organization Name:} \textbf{[Organization Name]}
    \item \textbf{Primary Email Domain:} \texttt{[Domain]}
    \item \textbf{External IP Address Scanned:} \texttt{[Client IP]}
\end{itemize}

% --- SECURITY CONTROL REVIEW ---
\section{Security Control Review (Questionnaire Analysis)}
The following table summarizes the organization's responses to a security controls questionnaire. The responses indicate significant gaps in fundamental security policies and procedures. A "No" response (\ding{55}) signifies a deviation from security best practices and introduces substantial risk.

\begin{table}[h!]
\centering
\caption{Security Controls Questionnaire Results}
\begin{tabular}{p{0.6\linewidth} c p{0.2\linewidth}}
\toprule
\textbf{Control Question} & \textbf{Response} & \textbf{Assessment} \\
\midrule
Do you require MFA to access email? & \ding{55} & Critical Gap \\
Do you require MFA to log into computers? & \ding{55} & Critical Gap \\
Do you require MFA to access sensitive data systems? & \ding{55} & Critical Gap \\
Does your organization have an employee acceptable use policy? & \ding{55} & High Risk \\
Does your organization do security awareness training for new employees? & \ding{55} & Critical Gap \\
Does your organization do security awareness training for all employees at least once per year? & \ding{55} & Critical Gap \\
\bottomrule
\end{tabular}
\end{table}

The complete absence of MFA is the most severe finding. It dramatically increases the risk of account compromise from phishing or password spraying attacks. The lack of security training and a formal acceptable use policy exacerbates this risk, as employees are not equipped with the knowledge to recognize or respond to threats.

% --- TECHNICAL SCAN RESULTS ---
\section{Technical Scan Results}
An external network scan was performed to identify exposed services. The target IP address was anonymized in the source data.

\begin{itemize}
    \item \textbf{Target IP Address:} \texttt{[Target IP]}
    \item \textbf{Scan Date:} \today
\end{itemize}

The scan identified the following open port(s):

\begin{table}[h!]
\centering
\caption{Open Port Analysis}
\begin{tabular}{l l l p{0.5\linewidth}}
\toprule
\textbf{Port/Protocol} & \textbf{State} & \textbf{Service (Inferred)} & \textbf{Notes} \\
\midrule
22/tcp & Open & SSH (Secure Shell) & Exposing SSH directly to the public internet is a significant risk. It is a primary target for automated brute-force attacks. Access should be restricted via a firewall or VPN and hardened with key-based authentication. \\
\bottomrule
\end{tabular}
\end{table}

% --- CONSOLIDATED RISK ASSESSMENT ---
\section{Consolidated Risk Assessment}
This section synthesizes findings from the security control review and technical scan into a consolidated list of identified risks. The pre-existing risk list provided was empty; therefore, all risks below are new findings from this assessment.

\begin{table}[h!]
\centering
\caption{Summary of Identified Risks}
\begin{tabular}{p{0.15\linewidth} p{0.25\linewidth} p{0.4\linewidth} l}
\toprule
\textbf{Risk ID} & \textbf{Risk Name} & \textbf{Description} & \textbf{Severity} \\
\midrule
RISK-001 & Absence of Multi-Factor Authentication (MFA) & The lack of MFA for email, workstations, and sensitive systems allows an attacker with valid credentials to gain unauthorized access without a second authentication factor. & \textbf{Critical} \\
\addlinespace
RISK-002 & Lack of Security Awareness Program & Without training, employees are unable to identify and appropriately respond to threats like phishing and social engineering, making them a vulnerable entry point. & \textbf{Critical} \\
\addlinespace
RISK-003 & Exposed SSH Management Service & The SSH service on \texttt{[Target IP]} is open to the public internet, making it a target for brute-force and credential stuffing attacks. & High \\
\addlinespace
RISK-004 & Missing Acceptable Use Policy (AUP) & The absence of a formal AUP means there are no documented rules for employees regarding the secure use of company assets, leading to inconsistent and insecure practices. & High \\
\bottomrule
\end{tabular}
\end{table}

% --- RECOMMENDATIONS ---
\section{Recommendations}
The following actions are recommended to mitigate the identified risks. They are prioritized based on severity and potential impact.

\subsection*{Priority 1: Critical Risks}
\begin{enumerate}
    \item \textbf{Implement Multi-Factor Authentication (MFA):}
    \begin{itemize}
        \item \textbf{Action:} Immediately enable MFA for all users across all critical systems.
        \item \textbf{Priority Systems:} Start with email (e.g., Office 365, Google Workspace), followed by VPN access, and access to any systems containing sensitive data.
        \item \textbf{Justification:} This is the single most effective control to prevent unauthorized access resulting from compromised credentials.
    \end{itemize}
    \item \textbf{Establish a Security Awareness Training Program:}
    \begin{itemize}
        \item \textbf{Action:} Procure and deploy a security awareness training module for all employees. Conduct a baseline phishing simulation to measure initial susceptibility.
        \item \textbf{Details:} Training should be mandatory for all new hires and conducted at least annually for all staff. Topics must include phishing, password hygiene, and acceptable use.
        \item \textbf{Justification:} An educated workforce is a critical layer of defense against social engineering and phishing attacks.
    \end{itemize}
\end{enumerate}

\subsection*{Priority 2: High Risks}
\begin{enumerate}
    \setcounter{enumi}{2} % Continue numbering
    \item \textbf{Secure the Exposed SSH Service:}
    \begin{itemize}
        \item \textbf{Action:} Restrict access to the SSH service on port 22.
        \item \textbf{Options:} The best practice is to place the service behind a VPN. If direct access is required, restrict source IP addresses to only trusted networks in the firewall.
        \item \textbf{Hardening:} Disable password-based authentication and enforce the use of strong cryptographic keys (e.g., ED25519) for all SSH access.
    \end{itemize}
    \item \textbf{Develop and Implement an Acceptable Use Policy (AUP):}
    \begin{itemize}
        \item \textbf{Action:} Draft a formal AUP that clearly defines the rules for using company networks, devices, and data.
        \item \textbf{Distribution:} The policy must be distributed to all employees, and acknowledgment of receipt and understanding should be tracked.
        \item \textbf{Justification:} An AUP establishes a baseline for secure behavior and provides a framework for enforcing security standards.
    \end{itemize}
\end{enumerate}

\end{document}
```