```latex
\documentclass[12pt]{article}

% Required Packages
\usepackage[margin=1in]{geometry}
\usepackage{pifont} % For \ding
\usepackage{booktabs} % For professional tables (\toprule, \midrule, \bottomrule)
\usepackage{hyperref} % For clickable links and ToC
\usepackage{url}
\usepackage{seqsplit} % For breaking long strings in \texttt
\usepackage{xcolor}   % For coloring text

% Hyperref Setup
\hypersetup{
    colorlinks=true,
    linkcolor=black,
    filecolor=magenta,
    urlcolor=blue,
    pdftitle={Cybersecurity Posture Assessment Report},
    pdfauthor={Cybersecurity Analysis Division},
}

% Define a custom color for severity
\definecolor{criticalred}{HTML}{990000}
\definecolor{highorange}{HTML}{E69138}

% Document Metadata
\title{Cybersecurity Posture Assessment Report}
\author{Cybersecurity Analysis Division}
\date{\today}

\begin{document}

\maketitle
\thispagestyle{empty}

\newpage
\tableofcontents
\newpage

%======================================================================
\section{Executive Summary}
%======================================================================

This report provides a comprehensive analysis of the cybersecurity posture for \textbf{[Organization Name]}. The assessment is based on a synthesis of network scan data, a security controls questionnaire, and a review of pre-existing risks.

The analysis revealed two critical-risk findings that require immediate attention:
\begin{enumerate}
    \item \textbf{Publicly Exposed End-of-Life Database:} A network scan identified a MySQL database (version 5.7.33) on port 3306 open to the public network. This version is past its End-of-Life (EOL) and no longer receives security updates, exposing it to numerous known vulnerabilities. This finding confirms and elevates the pre-existing risk "Database Exposure".
    \item \textbf{Critical Gap in Access Controls:} The security questionnaire revealed that Multi-Factor Authentication (MFA) is not required for accessing sensitive data systems. This policy gap, combined with the exposed database, creates a significant risk of unauthorized access and data breach.
\end{enumerate}

The overall security posture is assessed as \textbf{High-Risk}. Immediate remediation of the identified vulnerabilities is strongly recommended to reduce the likelihood of a security incident.

%======================================================================
\section{Organizational Information}
%======================================================================

The following information was used as the basis for this assessment. Due to the anonymized nature of the provided data, placeholders have been used where necessary.

\begin{itemize}
    \item \textbf{Organization Name:} \textbf{[Organization Name]}
    \item \textbf{Primary Email Domain:} \texttt{[Domain]}
    \item \textbf{Assessed External IP:} \texttt{[Client IP]}
\end{itemize}

%======================================================================
\section{Security Control Review}
%======================================================================

A review of the organization's security controls was conducted via a questionnaire. The responses are summarized below. A green checkmark (\ding{51}) indicates a positive control is in place, while a red cross (\ding{55}) indicates a potential control gap.

\begin{table}[h!]
\centering
\caption{Security Controls Questionnaire Results}
\label{tab:controls}
\begin{tabular}{p{0.8\textwidth}c}
\toprule
\textbf{Control Question} & \textbf{Status} \\
\midrule
Do you require MFA to access email? & \ding{51} \\
Do you require MFA to log into computers? & \ding{51} \\
\textbf{Do you require MFA to access sensitive data systems?} & \textcolor{criticalred}{\ding{55}} \\
Does your organization have an employee acceptable use policy? & \ding{51} \\
Does your organization do security awareness training for new employees? & \ding{51} \\
Does your organization do security awareness training for all employees at least once per year? & \ding{51} \\
\bottomrule
\end{tabular}
\end{table}

\subsection*{Analysis of Control Gaps}
The primary finding from the questionnaire is the \textbf{lack of mandatory MFA for sensitive data systems}. While MFA is commendably enforced for email and computer logins, its absence on systems housing sensitive data represents a critical security oversight. This allows an attacker with stolen credentials to gain direct access to the organization's most valuable assets. This gap is particularly concerning given the technical findings in the next section.

%======================================================================
\section{Technical Scan Results}
%======================================================================

An external network scan was performed to identify exposed services and potential vulnerabilities.

\subsection*{Network Port Scan Findings}
The scan was performed against the target IP address \texttt{[Target IP]}. The results are detailed in Table \ref{tab:scanresults}.

\begin{table}[h!]
\centering
\caption{Open Ports and Services Detected}
\label{tab:scanresults}
\begin{tabular}{l l l l l}
\toprule
\textbf{Port} & \textbf{State} & \textbf{Service} & \textbf{Product} & \textbf{Version} \\
\midrule
3306/tcp & open & mysql & MySQL & 5.7.33 \\
\bottomrule
\end{tabular}
\end{table}

\subsection*{Analysis of Technical Findings}
The scan identified a publicly accessible MySQL database service. This configuration is highly discouraged as it exposes the database directly to threats from the public internet, such as brute-force attacks, credential stuffing, and exploitation of known vulnerabilities.

Furthermore, the detected version, \textbf{MySQL 5.7.33}, is a significant concern. The MySQL 5.7 branch reached its official End-of-Life (EOL) in October 2023. This means it no longer receives security patches from the vendor, and any vulnerabilities discovered since that date will remain unpatched. An attacker could leverage a known public exploit to compromise the database, potentially leading to a full system compromise or a severe data breach.

%======================================================================
\section{Consolidated Risk Assessment}
%======================================================================

This section correlates the findings from the security control review, the technical scan, and pre-existing risk data to provide a consolidated view of the current risk landscape.

\begin{table}[h!]
\centering
\caption{Summary of Identified Risks}
\label{tab:risks}
\begin{tabular}{p{0.1\textwidth} p{0.5\textwidth} p{0.15\textwidth} p{0.15\textwidth}}
\toprule
\textbf{ID} & \textbf{Risk Description} & \textbf{Severity} & \textbf{Affected Systems} \\
\midrule
\textbf{RISK-001} & A publicly exposed MySQL database is running an End-of-Life (EOL) version (5.7.33), making it highly susceptible to known, unpatched vulnerabilities. & \textcolor{criticalred}{\textbf{Critical (7.5)}} & \texttt{[Target IP]}:3306 \\
\midrule
\textbf{RISK-002} & Lack of enforced Multi-Factor Authentication (MFA) on sensitive data systems allows for single-factor (password-only) authentication, greatly increasing the risk of compromise from stolen credentials. & \textcolor{criticalred}{\textbf{Critical}} & All sensitive data systems \\
\bottomrule
\end{tabular}
\end{table}

%======================================================================
\section{Recommendations}
%======================================================================

The following actions are recommended to mitigate the identified risks. Recommendations are prioritized based on severity and potential impact.

\subsection*{RISK-001: Remediate Exposed End-of-Life Database}
\begin{itemize}
    \item \textbf{Immediate (Containment):} Implement strict firewall rules to block all public access to TCP port 3306. Access should be restricted to a minimal set of trusted IP addresses required for administration or application functionality.
    \item \textbf{Short-Term (Remediation):} Initiate a project to upgrade the MySQL 5.7 instance to a currently supported version (e.g., MySQL 8.x). This will ensure the system receives critical security patches.
    \item \textbf{Long-Term (Strategic):} Review the application architecture. Databases should not be directly exposed to the internet. If remote access is required, it should be facilitated through a secure channel such as a Virtual Private Network (VPN) or a bastion host.
\end{itemize}

\subsection*{RISK-002: Enforce MFA on Sensitive Systems}
\begin{itemize}
    \item \textbf{Immediate (Policy):} Update the organization's access control policy to mandate the use of MFA for all accounts (user and administrative) on systems classified as sensitive.
    \item \textbf{Short-Term (Implementation):} Identify all sensitive data systems, including database management interfaces, and deploy an MFA solution. Prioritize the system identified in RISK-001.
    \item \textbf{Long-Term (Compliance):} Establish a quarterly audit process to verify that MFA is correctly implemented and enforced on all in-scope systems, including any new systems deployed.
\end{itemize}

\end{document}
```