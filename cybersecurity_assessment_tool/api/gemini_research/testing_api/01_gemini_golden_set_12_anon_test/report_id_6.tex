```latex
\documentclass[12pt]{article}

% Preamble: Required Packages
\usepackage[margin=1in]{geometry}
\usepackage{pifont} % For checkmarks (\ding{51}) and crosses (\ding{55})
\usepackage{booktabs} % For professional-looking tables
\usepackage{hyperref} % For clickable links
\usepackage{url}      % For URL formatting
\usepackage{seqsplit} % To split long strings in \texttt
\usepackage{xcolor}   % For colored text

% Define custom colors for status indicators
\definecolor{darkgreen}{rgb}{0.0, 0.5, 0.0}
\definecolor{darkred}{rgb}{0.6, 0.0, 0.0}

\begin{document}

% --- Title Page ---
\title{Cybersecurity Posture Assessment Report}
\author{Cybersecurity Analyst}
\date{\today}
\maketitle

\newpage

% --- Table of Contents ---
\tableofcontents
\newpage

% --- Section 1: Executive Overview ---
\section{Executive Overview}

This report provides a comprehensive cybersecurity assessment for \textbf{[Organization Name]}, based on an analysis of organizational security controls, a technical network scan, and a review of pre-existing risk data.

The assessment reveals a strong and improving security posture. The organization demonstrates a mature approach to procedural security, with all reviewed administrative controls, such as Multi-Factor Authentication (MFA) and security awareness training, fully implemented. This foundational strength is commendable.

A key finding from our technical analysis is the successful remediation of a previously identified vulnerability. A pre-existing risk noted the presence of an unencrypted web server on port 80. However, our recent network scan confirms that this port is now \textbf{closed}, effectively mitigating the associated risk. This proactive remediation demonstrates a positive security trajectory.

While the current posture is robust, this report concludes with recommendations focused on maintaining this high standard through continuous monitoring and formal validation of security changes.

% --- Section 2: Organizational Information ---
\section{Organizational Information}

This assessment pertains to the following entity and its associated network assets. The information below is based on data provided for this analysis.

\begin{itemize}
    \item \textbf{Organization Name:} \textbf{[Organization Name]}
    \item \textbf{Primary Email Domain:} \texttt{[Domain]}
    \item \textbf{Scanned External IP:} \texttt{[Client IP]}
\end{itemize}

% --- Section 3: Security Control Review ---
\section{Security Control Review}

The following table summarizes the organization's responses to a security controls questionnaire. The responses indicate a strong commitment to implementing fundamental security best practices. A green checkmark (\textcolor{darkgreen}{\ding{51}}) indicates that the control is in place.

\begin{table}[h!]
\centering
\caption{Organizational Security Controls Questionnaire}
\begin{tabular}{p{0.7\linewidth} c c}
\toprule
\textbf{Control Question} & \textbf{Response} & \textbf{Status} \\
\midrule
Do you require MFA to access email? & Yes & \textcolor{darkgreen}{\ding{51}} \\
Do you require MFA to log into computers? & Yes & \textcolor{darkgreen}{\ding{51}} \\
Do you require MFA to access sensitive data systems? & Yes & \textcolor{darkgreen}{\ding{51}} \\
Does your organization have an employee acceptable use policy? & Yes & \textcolor{darkgreen}{\ding{51}} \\
Does your organization do security awareness training for new employees? & Yes & \textcolor{darkgreen}{\ding{51}} \\
Does your organization do security awareness training for all employees at least once per year? & Yes & \textcolor{darkgreen}{\ding{51}} \\
\bottomrule
\end{tabular}
\end{table}

\textbf{Analysis:} The organization has successfully implemented all reviewed administrative controls. This significantly reduces the risk of account compromise and unauthorized access, and fosters a security-conscious culture. No gaps were identified in this review.

% --- Section 4: Technical Scan Results ---
\section{Technical Scan Results}

A network scan was conducted to identify open ports and exposed services on the organization's external-facing infrastructure.

\begin{itemize}
    \item \textbf{Target IP Address:} \texttt{[Target IP]}
    \item \textbf{Host Status:} Up
\end{itemize}

The scan results are detailed in the table below.

\begin{table}[h!]
\centering
\caption{Port Scan Findings}
\begin{tabular}{l l l p{0.5\linewidth}}
\toprule
\textbf{Port} & \textbf{State} & \textbf{Service} & \textbf{Analyst Notes} \\
\midrule
80/tcp & \textbf{closed} & http & The port for unencrypted web traffic was found to be closed. This is a positive security finding and mitigates risks associated with unencrypted communications. \\
\bottomrule
\end{tabular}
\end{table}

\textbf{Analysis:} The technical scan did not identify any open ports on the target host. The finding that port 80/tcp is closed is particularly significant as it directly contradicts a previously documented risk, indicating successful remediation.

% --- Section 5: Correlated Risk Assessment ---
\section{Correlated Risk Assessment}

This section synthesizes the findings from the technical scan with the list of previously known vulnerabilities. The status of each risk has been updated based on the most recent data.

\begin{table}[h!]
\centering
\caption{Current Risk Register Status}
\begin{tabular}{p{0.3\linewidth} c p{0.3\linewidth} l}
\toprule
\textbf{Risk Name} & \textbf{Severity} & \textbf{Overview} & \textbf{Status} \\
\midrule
Unencrypted Web Server & 5.0 (Medium) & The original risk stated that Port 80 was open, exposing an unencrypted service. & \textbf{\textcolor{darkgreen}{Remediated}} \\
\bottomrule
\end{tabular}
\end{table}

\textbf{Analysis:} The primary known risk, "Unencrypted Web Server," is now considered \textbf{Remediated}. The technical scan from Section 4 provides direct evidence that port 80 is no longer accessible, thereby neutralizing the threat. This is a significant improvement to the organization's security posture.

% --- Section 6: Recommendations ---
\section{Recommendations}

Based on the analysis, the organization has a strong security posture. The following recommendations are focused on maintaining and formalizing these excellent security practices.

\begin{enumerate}
    \item \textbf{Validate and Document Remediation:} Formally update the internal risk register to reflect that the "Unencrypted Web Server" vulnerability has been closed. Ensure that the change is documented in the change management system to prevent accidental reversion in the future.

    \item \textbf{Implement Continuous Monitoring:} While the current state is secure, configurations can change. Implement an automated scanning or monitoring solution to provide alerts if critical ports (e.g., 80/tcp, 22/tcp, 3389/tcp) are ever opened on external-facing assets without authorization.

    \item \textbf{Maintain Strong Procedural Controls:} Continue the rigorous enforcement of MFA and the security awareness training program. These controls are the bedrock of the organization's defense-in-depth strategy and should be periodically reviewed and updated.
\end{enumerate}

\end{document}
```