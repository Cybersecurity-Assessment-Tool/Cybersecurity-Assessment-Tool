```latex
\documentclass[12pt, a4paper]{article}

% Preamble: Required Packages
\usepackage[margin=1in]{geometry}
\usepackage{pifont} % For checkmarks and crosses (\ding)
\usepackage{booktabs} % For professional tables
\usepackage{hyperref} % For clickable links
\usepackage{url} % For URL formatting
\usepackage{seqsplit} % To split long strings in \texttt
\usepackage{graphicx}
\usepackage{xcolor}
\usepackage{fancyhdr}
\usepackage{lastpage}

% --- Document Setup ---
\hypersetup{
    colorlinks=true,
    linkcolor=blue,
    filecolor=magenta,      
    urlcolor=cyan,
    pdftitle={Cybersecurity Posture Assessment Report},
    pdfpagemode=FullScreen,
}

% --- Header & Footer ---
\pagestyle{fancy}
\fancyhf{} % Clear all header and footer fields
\fancyhead[L]{Cybersecurity Posture Assessment}
\fancyhead[R]{\textbf{[Organization Name]}}
\fancyfoot[C]{\thepage\ of \pageref{LastPage}}
\renewcommand{\headrulewidth}{0.4pt}
\renewcommand{\footrulewidth}{0.4pt}

% --- Document Start ---
\begin{document}

% --- Title Page ---
\begin{titlepage}
    \centering
    \vfill
    {\Huge\bfseries Cybersecurity Posture Assessment Report\par}
    \vspace{1.5cm}
    {\Large For: \textbf{[Organization Name]}}\par
    \vspace{1cm}
    {\large Report Date: \today}\par
    \vspace{0.5cm}
    {\large Confidential}\par
    \vfill
    \vfill
\end{titlepage}

\tableofcontents
\newpage

% --- Section 1: Executive Summary ---
\section{Executive Summary}
This report details the findings of a cybersecurity posture assessment conducted for \textbf{[Organization Name]}. The assessment combined an analysis of organizational security controls, a technical network scan of external infrastructure, and a review of pre-existing risks.

The overall security posture is determined to be at a \textbf{CRITICAL} risk level. This conclusion is based on the identification of several high-impact vulnerabilities and control gaps that expose the organization to significant threats, including unauthorized access, data breach, and service disruption.

Key findings include:
\begin{itemize}
    \item \textbf{Pre-existing Critical Risk:} A critical vulnerability, ``Localhost Exposed'' (CVSS 10.0), was identified as an existing issue, requiring immediate investigation and remediation.
    \item \textbf{Lack of Multi-Factor Authentication (MFA):} The security questionnaire indicates that MFA is not implemented for email, computer logins, or access to sensitive data systems. This absence of a fundamental security control represents a critical gap.
    \item \textbf{Exposed Secure Shell (SSH) Service:} The network scan identified an open SSH port (22/TCP) on the external IP address \texttt{[Target IP]}. When combined with the lack of MFA, this creates a significant risk of brute-force attacks and unauthorized server access.
\end{itemize}

Immediate and decisive action is required to address these findings. Recommendations are provided in Section \ref{sec:recommendations} to guide remediation efforts and mitigate the identified risks.

% --- Section 2: Organizational Information ---
\section{Organizational Information}
This section provides the organizational details used as the basis for this assessment. Due to the anonymized nature of the input data, placeholders have been used where necessary.

\begin{table}[h!]
\centering
\caption{Client Details}
\begin{tabular}{@{}ll@{}}
\toprule
\textbf{Attribute} & \textbf{Value} \\ \midrule
Organization Name & \textbf{[Organization Name]} \\
Primary Domain & \texttt{[Domain]} \\
External IP Address Scanned & \texttt{[Client IP]} \\ \bottomrule
\end{tabular}
\end{table}

% --- Section 3: Security Control Review ---
\section{Security Control Review (Questionnaire)}
An analysis of the organization's security controls was performed based on a questionnaire. The responses indicate significant gaps in foundational security practices. An answer of ``N/A'' (Not Applicable) has been interpreted as the control not being implemented, thus representing a gap.

\begin{table}[h!]
\centering
\caption{Security Questionnaire Analysis}
\label{tab:questionnaire}
\begin{tabular}{@{}p{0.7\linewidth}c@{}}
\toprule
\textbf{Control Question} & \textbf{Implemented?} \\ \midrule
Do you require MFA to access email? & \ding{55} \\
Do you require MFA to log into computers? & \ding{55} \\
Do you require MFA to access sensitive data systems? & \ding{55} \\ \bottomrule
\end{tabular}
\end{table}

The complete absence of MFA across all queried systems is a critical weakness. MFA is an industry-standard control that provides a vital layer of defense against credential theft and unauthorized access.

% --- Section 4: Technical Scan Results ---
\section{Technical Scan Results}
A network scan was performed on the target IP address to identify open ports and exposed services.

\begin{itemize}
    \item \textbf{Target IP Address:} \texttt{[Target IP]}
    \item \textbf{Scan Status:} Host is UP.
\end{itemize}

\subsection{Open Ports}
The following ports were found to be open and accessible from the public internet.

\begin{table}[h!]
\centering
\caption{Open Port Findings}
\label{tab:ports}
\begin{tabular}{@{}lllll@{}}
\toprule
\textbf{Port} & \textbf{State} & \textbf{Service} & \textbf{Version} & \textbf{Notes} \\ \midrule
22/tcp & open & ssh & Unknown & Secure Shell access is exposed. \\ \bottomrule
\end{tabular}
\end{table}

\textbf{Analysis:} The exposure of the SSH service (port 22) is a high-risk finding. This service is a common target for automated brute-force attacks by malicious actors attempting to gain unauthorized access to the server. Without proper security configurations, such as IP whitelisting and key-based authentication, this service presents a significant entry point into the organization's network.

% --- Section 5: Risk Assessment & Findings ---
\section{Risk Assessment \& Findings}
This section synthesizes the results from the control review, technical scan, and pre-existing risk data into a consolidated list of findings. Each finding is assigned a severity level to aid in prioritization.

\begin{table}[h!]
\centering
\caption{Consolidated Risk Register}
\label{tab:risks}
\begin{tabular}{@{}p{0.1\linewidth}p{0.4\linewidth}p{0.15\linewidth}p{0.2\linewidth}@{}}
\toprule
\textbf{ID} & \textbf{Finding} & \textbf{Severity} & \textbf{Affected Asset(s)} \\ \midrule
\textbf{RISK-001} & \textbf{Localhost Exposed} \newline A pre-existing critical risk was identified, suggesting a severe misconfiguration. & \textbf{Critical} & \texttt{[Target IP]} \\
\addlinespace
\textbf{RISK-002} & \textbf{Lack of Multi-Factor Authentication} \newline MFA is not enforced for email, endpoints, or sensitive systems, leaving them vulnerable to credential compromise. & \textbf{Critical} & All Users \& Systems \\
\addlinespace
\textbf{RISK-003} & \textbf{SSH Port Exposed to the Internet} \newline The SSH management port is open to the public, increasing the risk of brute-force attacks and unauthorized access. & \textbf{High} & \texttt{[Target IP]} \\ \bottomrule
\end{tabular}
\end{table}

% --- Section 6: Recommendations ---
\section{Recommendations}
\label{sec:recommendations}
The following actions are recommended to mitigate the identified risks. They are prioritized based on severity.

\subsection{RISK-001: Localhost Exposed (Critical)}
\begin{itemize}
    \item \textbf{Immediate Action:} Conduct an urgent investigation into the "Localhost Exposed" vulnerability. This may involve reviewing network firewall rules, service configurations (e.g., web servers, databases), and application bindings to ensure that services intended for internal use only are not bound to public-facing network interfaces.
    \item \textbf{Long-Term Fix:} Implement a change management and security review process for all new deployments and configuration changes to prevent similar misconfigurations in the future.
\end{itemize}

\subsection{RISK-002: Lack of Multi-Factor Authentication (Critical)}
\begin{itemize}
    \item \textbf{Immediate Action:} Prioritize the deployment of an MFA solution for all remote access systems (e.g., VPN, SSH) and privileged user accounts.
    \item \textbf{Phased Rollout:}
        \begin{enumerate}
            \item Phase 1: Enforce MFA on all email accounts (e.g., Office 365, Google Workspace).
            \item Phase 2: Enforce MFA for access to all systems containing sensitive data.
            \item Phase 3: Implement MFA for all standard user computer logins.
        \end{enumerate}
\end{itemize}

\subsection{RISK-003: SSH Port Exposed to the Internet (High)}
\begin{itemize}
    \item \textbf{Immediate Action:} If remote SSH access is not required, block port 22 at the network firewall immediately.
    \item \textbf{If Access is Required:}
        \begin{enumerate}
            \item \textbf{Restrict Access:} Configure firewall rules to whitelist only trusted source IP addresses that require SSH access. Deny all other connections.
            \item \textbf{Disable Password Authentication:} Modify the SSH daemon configuration (`sshd_config`) to disable password-based logins and permit only public key authentication. This mitigates the risk of brute-force attacks.
            \item \textbf{Regular Patching:} Ensure the SSH service is kept up-to-date with the latest security patches.
        \end{enumerate}
\end{itemize}

\end{document}
```