```latex
\documentclass[12pt]{article}

% --- PACKAGES ---
\usepackage[margin=1in]{geometry}
\usepackage{pifont} % For checkmarks and crosses
\usepackage{booktabs} % For professional tables
\usepackage{hyperref} % For clickable links
\usepackage{url} % For URL formatting
\usepackage{seqsplit} % To split long text strings without breaking
\usepackage{graphicx}
\usepackage{xcolor}
\usepackage{fancyhdr}
\usepackage{lastpage}

% --- DOCUMENT SETUP ---
\hypersetup{
    colorlinks=true,
    linkcolor=blue,
    filecolor=magenta,      
    urlcolor=cyan,
    pdftitle={Cybersecurity Posture Assessment Report},
    pdfauthor={Cybersecurity Analyst},
    pdfsubject={Security Report},
    pdfkeywords={Security, Assessment, Report},
}

% --- HEADER & FOOTER ---
\pagestyle{fancy}
\fancyhf{}
\lhead{Cybersecurity Posture Assessment}
\rhead{\textbf{[Organization Name]}}
\cfoot{Page \thepage\ of \pageref{LastPage}}
\renewcommand{\headrulewidth}{0.4pt}
\renewcommand{\footrulewidth}{0.4pt}

% --- DOCUMENT START ---
\begin{document}

% --- TITLE PAGE ---
\begin{titlepage}
    \centering
    \vspace*{2cm}
    
    {\Huge \textbf{Cybersecurity Posture Assessment Report}\par}
    \vspace{1.5cm}
    
    {\Large Prepared for:\par}
    \vspace{0.5cm}
    {\huge \textbf{[Organization Name]}}\par
    
    \vfill
    
    {\large \today\par}
    
    \vspace{1cm}
    
    \textit{This report contains sensitive information and should be handled with care.}
    
\end{titlepage}

\newpage
\tableofcontents
\newpage

% --- EXECUTIVE SUMMARY ---
\section{Executive Summary}

This report provides a comprehensive analysis of the current cybersecurity posture for \textbf{[Organization Name]}. The assessment is based on a correlation of external network scan data, a review of existing risks, and an internal security controls questionnaire.

The overall security posture is assessed as \textbf{CRITICAL}. Several severe vulnerabilities and control gaps were identified that expose the organization to a high likelihood of a significant security incident, such as a ransomware attack or data breach.

Key findings include:
\begin{itemize}
    \item \textbf{Direct Internet Exposure of Remote Desktop Protocol (RDP):} The external network scan confirmed that RDP (port 3389) is open to the public internet on a key asset. This is a well-known and highly targeted attack vector.
    \item \textbf{Systemic Lack of Multi-Factor Authentication (MFA):} The organization does not enforce MFA for email, computer logins, or access to sensitive data systems. This represents a critical failure in identity and access management, making user accounts highly susceptible to compromise.
    \item \textbf{Foundational Policy and Training Gaps:} The status of an employee acceptable use policy and security awareness training programs is unknown. This suggests a potential lack of fundamental governance, risk, and compliance (GRC) controls, which are essential for establishing a security-conscious culture.
\end{itemize}

Immediate remediation of the identified critical risks is strongly recommended to reduce the organization's attack surface and prevent a potential compromise.

% --- ORGANIZATIONAL INFORMATION ---
\section{Organizational Information}

The following information was used as the basis for this assessment. Where data was not provided, placeholders have been used.

\begin{tabular}{@{}ll}
    \toprule
    \textbf{Attribute} & \textbf{Value} \\
    \midrule
    Organization Name & \textbf{[Organization Name]} \\
    Email Domain & \seqsplit{\texttt{[Domain]}} \\
    External IP Address (Scan Source) & \seqsplit{\texttt{[Client IP]}} \\
    \bottomrule
\end{tabular}

% --- SECURITY CONTROL REVIEW ---
\section{Security Control Review}

The following table summarizes the responses from the security controls questionnaire. "No" answers indicate significant gaps that increase organizational risk. "Unknown" answers require immediate verification to determine the status of the control.

\begin{table}[h!]
    \centering
    \caption{Security Controls Questionnaire Analysis}
    \begin{tabular}{p{0.6\linewidth} c l}
        \toprule
        \textbf{Control Question} & \textbf{Response} & \textbf{Assessment} \\
        \midrule
        Do you require MFA to access email? & \ding{55} & \textcolor{red}{\textbf{Critical Gap}} \\
        Do you require MFA to log into computers? & \ding{55} & \textcolor{red}{\textbf{Critical Gap}} \\
        Do you require MFA to access sensitive data systems? & \ding{55} & \textcolor{red}{\textbf{Critical Gap}} \\
        \addlinespace
        Does your organization have an employee acceptable use policy? & Unknown & \textcolor{orange}{High Risk / Needs Verification} \\
        Does your organization do security awareness training for new employees? & Unknown & \textcolor{orange}{High Risk / Needs Verification} \\
        Does your organization do security awareness training for all employees at least once per year? & Unknown & \textcolor{orange}{High Risk / Needs Verification} \\
        \bottomrule
    \end{tabular}
\end{table}

% --- TECHNICAL SCAN RESULTS ---
\section{Technical Scan Results}

An external network scan was performed to identify exposed services. The scan confirmed the presence of an open port associated with a high-risk service.

\begin{itemize}
    \item \textbf{Target IP Address:} \seqsplit{\texttt{[Target IP]}}
\end{itemize}

\begin{table}[h!]
    \centering
    \caption{Open Ports Detected on \seqsplit{\texttt{[Target IP]}}}
    \begin{tabular}{c c c l l}
        \toprule
        \textbf{Port} & \textbf{Protocol} & \textbf{State} & \textbf{Service} & \textbf{Details} \\
        \midrule
        3389 & TCP & Open & ms-wbt-server & Remote Desktop Protocol (RDP) \\
        \bottomrule
    \end{tabular}
\end{table}

\paragraph{Analysis:} The exposure of RDP (port 3389) to the public internet is a critical security risk. This service is a primary target for attackers who use brute-force credential attacks and exploit known vulnerabilities to gain initial access to internal networks. This finding, combined with the lack of MFA, creates a direct and easily exploitable pathway for a compromise.

% --- CONSOLIDATED RISK ASSESSMENT ---
\section{Consolidated Risk Assessment}

The following table synthesizes findings from the technical scan, control review, and pre-existing risk data into a consolidated list of prioritized risks.

\begin{table}[h!]
    \centering
    \caption{Summary of Identified Risks}
    \begin{tabular}{p{0.15\linewidth} p{0.55\linewidth} p{0.15\linewidth}}
        \toprule
        \textbf{Risk Name} & \textbf{Description} & \textbf{Severity} \\
        \midrule
        \textbf{Public RDP Exposure} & Port 3389 (RDP) is open to the internet on \seqsplit{\texttt{[Target IP]}}, allowing attackers to attempt brute-force logins or exploit vulnerabilities. This risk is confirmed by both the scan and existing risk data. & \textbf{\textcolor{red}{Critical (9.0)}} \\
        \addlinespace
        \textbf{Lack of MFA} & Multi-Factor Authentication is not enforced for email, computer logins, or sensitive data systems. This critically weakens account security against phishing and credential theft. & \textbf{\textcolor{red}{Critical}} \\
        \addlinespace
        \textbf{Missing Security Policies \& Training} & Foundational controls such as an Acceptable Use Policy and security awareness training are not confirmed to be in place, increasing the risk of insider threats and human error. & \textbf{\textcolor{orange}{High}} \\
        \bottomrule
    \end{tabular}
\end{table}

% --- RECOMMENDATIONS ---
\section{Recommendations}

The following actions are recommended to mitigate the identified risks. They are prioritized based on severity and exploitability.

\subsection{Immediate Priority (Remediate within 72 hours)}

\paragraph{Risk: Public RDP Exposure}
\begin{itemize}
    \item \textbf{Short-Term Fix:} Immediately implement a firewall rule to block all inbound traffic to TCP port 3389 on \seqsplit{\texttt{[Target IP]}} from the public internet. Access should be restricted to trusted IP addresses only, if absolutely necessary.
    \item \textbf{Long-Term Fix:} For any required remote access, implement a Virtual Private Network (VPN) solution. The VPN must be configured to require Multi-Factor Authentication for all users.
    \item \textbf{Justification:} This is the single most critical and exploitable vulnerability identified. Closing this port will immediately eliminate a primary vector for ransomware attacks.
\end{itemize}

\subsection{High Priority (Remediate within 30 days)}

\paragraph{Risk: Lack of Multi-Factor Authentication (MFA)}
\begin{itemize}
    \item \textbf{Action:} Procure and deploy an MFA solution. Enforce its use across all critical applications and systems, prioritizing the following:
    \begin{enumerate}
        \item Email (e.g., Office 365, Google Workspace)
        \item VPN and other remote access solutions
        \item Administrative access to servers, network devices, and cloud platforms
        \item Access to systems containing sensitive data
    \end{enumerate}
    \item \textbf{Justification:} According to industry data, MFA prevents over 99.9\% of account compromise attacks. It is the most effective security control for protecting user identities.
\end{itemize}

\paragraph{Risk: Missing Security Policies \& Training}
\begin{itemize}
    \item \textbf{Action:}
    \begin{enumerate}
        \item \textbf{Verify Controls:} Immediately determine if an Acceptable Use Policy (AUP) and security awareness training programs exist.
        \item \textbf{Develop \& Implement:} If they do not exist, develop a formal AUP that all employees must sign. Procure and enroll all employees in a security awareness training program, with mandatory modules for new hires and annual refreshers for all staff.
    \end{enumerate}
    \item \textbf{Justification:} These foundational controls establish a baseline for secure employee behavior, reduce the risk of human error (e.g., falling for phishing attacks), and create a defensible security culture.
\end{itemize}

\end{document}
```