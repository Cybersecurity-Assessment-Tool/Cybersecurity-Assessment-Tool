```latex
\documentclass[12pt]{article}

% Preamble: Required Packages
\usepackage[margin=1in]{geometry}
\usepackage{pifont}         % For using symbols like checkmarks (\ding{51}) and crosses (\ding{55})
\usepackage{booktabs}       % For professional-looking tables (\toprule, \midrule, \bottomrule)
\usepackage{hyperref}       % For creating hyperlinks within the document
\usepackage{url}            % For formatting URLs
\usepackage{seqsplit}       % For splitting long, unbreakable strings like hashes or tokens
\usepackage{graphicx}
\usepackage{xcolor}

% Hyperlink Setup
\hypersetup{
    colorlinks=true,
    linkcolor=black,
    filecolor=magenta,
    urlcolor=blue,
    pdftitle={Cybersecurity Risk Assessment Report},
    pdfpagemode=FullScreen,
}

% Custom Commands for Status Indicators
\newcommand{\yes}{\ding{51}}
\newcommand{\no}{\ding{55}}

% Document Start
\begin{document}

% --- Title Page ---
\begin{titlepage}
    \centering
    \vspace*{1cm}
    \Huge\textbf{Cybersecurity Risk Assessment Report}
    \vspace{1.5cm}
    \Large
    Prepared for: \textbf{[Organization Name]}
    \vspace{2cm}
    \rule{\linewidth}{0.5mm}
    \vspace{0.4cm}
    \text{CONFIDENTIAL}
    \rule{\linewidth}{0.5mm}
    \vfill
    \large
    Report Date: \today \\
    Analysis Period: October 2023 % Placeholder date as none was provided
    \vspace{0.8cm}
    \textit{This report contains sensitive information regarding the security posture of the organization. Distribution should be limited to authorized personnel.}
\end{titlepage}

\tableofcontents
\newpage

% --- 1. Executive Summary ---
\section{Executive Summary}
This report provides a comprehensive analysis of the current cybersecurity posture of \textbf{[Organization Name]}. The assessment is based on a correlation of external network scans, a review of internal security controls via a questionnaire, and an analysis of pre-existing risk data.

The assessment reveals a \textbf{critical risk posture}. This is driven by the combination of a severe, publicly-exposed vulnerability on an external-facing server and a complete absence of fundamental internal security controls, such as Multi-Factor Authentication (MFA), security policies, and employee awareness training.

An externally accessible FTP server was identified running a dangerously outdated and vulnerable version of \texttt{vsftpd}, which is known to contain a backdoor. This vulnerability could allow an attacker to gain complete control of the server. This technical flaw is compounded by organizational gaps, including the lack of MFA, which leaves user accounts highly susceptible to compromise via phishing or password spraying attacks.

\textbf{Immediate action is required} to remediate the critical external vulnerability. Following this, a strategic initiative must be undertaken to implement foundational security controls to mitigate the high-impact organizational risks identified.

% --- 2. Organizational Information ---
\section{Organizational Information}
This section details the information provided for the assessment. Placeholders are used where data was not supplied.

\begin{itemize}
    \item \textbf{Organization Name:} \textbf{[Organization Name]}
    \item \textbf{Primary Email Domain:} \texttt{[Domain]}
    \item \textbf{Assumed Client External IP:} \texttt{[Client IP]}
    \item \textbf{Target IP in Network Scan:} \texttt{[Target IP]}
\end{itemize}

% --- 3. Security Control Review ---
\section{Security Control Review}
The following table summarizes the organization's responses to a security controls questionnaire. A red cross (\no) indicates a significant gap in the organization's defensive posture.

\begin{table}[h!]
\centering
\caption{Security Controls Questionnaire Analysis}
\begin{tabular}{p{0.75\linewidth} c}
\toprule
\textbf{Control Question} & \textbf{Status} \\
\midrule
Do you require MFA to access email? & \no \\
Do you require MFA to log into computers? & \no \\
Do you require MFA to access sensitive data systems? & \no \\
Does your organization have an employee acceptable use policy? & \no \\
Does your organization do security awareness training for new employees? & \no \\
Does your organization do security awareness training for all employees at least once per year? & \no \\
\bottomrule
\end{tabular}
\end{table}

\textbf{Analysis:} The complete absence of these foundational controls represents a critical organizational risk. Without MFA, the organization is highly vulnerable to account takeover attacks. The lack of an acceptable use policy and security training means that employees are likely unaware of security best practices, making them susceptible to social engineering and phishing attacks.

% --- 4. Technical Scan Results ---
\section{Technical Scan Results}
An external network scan was performed on the target IP address \texttt{[Target IP]}. The results are detailed below.

\subsection{Open Ports and Services}
\begin{table}[h!]
\centering
\caption{Open Ports on Target: \texttt{[Target IP]}}
\begin{tabular}{l l l l}
\toprule
\textbf{Port} & \textbf{State} & \textbf{Service} & \textbf{Product / Version} \\
\midrule
21/tcp & open & ftp & vsftpd 2.3.4 \\
\bottomrule
\end{tabular}
\end{table}

\subsection{Key Findings}
\begin{itemize}
    \item \textbf{Critical Vulnerability - vsftpd 2.3.4 Backdoor (CVE-2011-2523):} The version of the FTP server software, \texttt{vsftpd 2.3.4}, is extremely old and contains a well-known, critical backdoor vulnerability. An attacker can exploit this flaw by sending a specific sequence of characters in a username field, which triggers a malicious payload and opens a command shell on port 6200. This effectively gives the attacker complete control over the server.
    \item \textbf{Insecure Configuration - Anonymous FTP Login:} The scan confirmed that anonymous FTP login is allowed. This permits any user on the internet to connect to the server and access files without authentication. This configuration is highly insecure and could lead to data leakage or be used by attackers to stage malicious files.
\end{itemize}

% --- 5. Consolidated Risk Assessment ---
\section{Consolidated Risk Assessment}
The following table synthesizes findings from the technical scan, control review, and pre-existing risk data into a prioritized list.

\begin{table}[h!]
\centering
\caption{Summary of Identified Risks}
\begin{tabular}{p{0.25\linewidth} p{0.5\linewidth} l}
\toprule
\textbf{Risk Name} & \textbf{Description} & \textbf{Severity} \\
\midrule
\textbf{Vulnerable FTP Server} & An external FTP server is running \texttt{vsftpd 2.3.4}, which has a public backdoor exploit (CVE-2011-2523). Anonymous login is also enabled. & \textbf{Critical} \\
\addlinespace
\textbf{No Multi-Factor Authentication (MFA)} & MFA is not enforced for email, computer logins, or access to sensitive systems, leaving accounts vulnerable to simple password compromise. & \textbf{Critical} \\
\addlinespace
\textbf{No Security Policies or Training} & The organization lacks an Acceptable Use Policy and does not conduct security awareness training, leading to a high likelihood of human error. & \textbf{High} \\
\addlinespace
\textbf{Outdated Windows Policy} & Workstations are running the unsupported Windows 7 operating system, which no longer receives security updates. (CVSS 5.0) & \textbf{Medium} \\
\bottomrule
\end{tabular}
\end{table}

% --- 6. Recommendations ---
\section{Recommendations}
Based on the consolidated risk assessment, the following actions are recommended. They are prioritized to address the most severe threats first.

\begin{enumerate}
    \item \textbf{Immediate Priority: Remediate Vulnerable FTP Server} \\
    The FTP server on \texttt{[Target IP]} presents a clear and present danger to the organization.
    \begin{itemize}
        \item \textbf{Action:} Take the server offline \textbf{immediately}.
        \item \textbf{Justification:} The known backdoor provides a direct path for an attacker to compromise the system and potentially pivot into the internal network.
        \item \textbf{Long-Term:} If FTP is a business requirement, it must be replaced with a modern, secure file transfer solution (e.g., SFTP) on a fully patched server. Anonymous access must be disabled unless explicitly and safely required.
    \end{itemize}
    \vspace{0.5cm}
    \item \textbf{High Priority: Implement Multi-Factor Authentication (MFA)} \\
    The lack of MFA is a critical gap that significantly increases the risk of a breach.
    \begin{itemize}
        \item \textbf{Action:} Procure and implement an MFA solution across all critical systems, starting with email, VPN access, and administrative accounts.
        \item \textbf{Justification:} MFA is one of the most effective controls for preventing unauthorized access resulting from stolen credentials.
    \end{itemize}
    \vspace{0.5cm}
    \item \textbf{High Priority: Establish Security Policies and Training} \\
    A strong security culture is essential for defense-in-depth.
    \begin{itemize}
        \item \textbf{Action:} Develop and enforce a formal Acceptable Use Policy (AUP). Implement a mandatory security awareness training program for all employees upon hiring and annually thereafter.
        \item \textbf{Justification:} This reduces the risk of security incidents caused by human error, such as falling for phishing scams or mishandling sensitive data.
    \end{itemize}
    \vspace{0.5cm}
    \item \textbf{Medium Priority: Address Outdated Windows Policy} \\
    The risk of running unsupported operating systems should be addressed systematically.
    \begin{itemize}
        \item \textbf{Action:} Continue with the existing recommendation to upgrade all Windows 7 workstations to a supported operating system (e.g., Windows 10/11).
        \item \textbf{Justification:} Unsupported systems do not receive security patches, leaving them vulnerable to newly discovered exploits.
    \end{itemize}
\end{enumerate}

\end{document}
```