```latex
\documentclass[12pt]{article}

% Required Packages
\usepackage[margin=1in]{geometry} % Sets page margins
\usepackage{pifont}               % For checkmark and X symbols (\ding)
\usepackage{booktabs}             % For professional-looking tables
\usepackage[hidelinks]{hyperref}  % For clickable links without boxes
\usepackage{url}                  % For formatting URLs
\usepackage{seqsplit}             % For splitting long strings in texttt

% Document Metadata
\title{Cybersecurity Posture Assessment Report}
\author{Cybersecurity Analyst}
\date{\today}

\begin{document}

\maketitle

\begin{abstract}
    This report provides a comprehensive cybersecurity assessment for \textbf{[Organization Name]}. The analysis is based on a synthesis of network scan data, a review of organizational security controls, and an evaluation of pre-existing risks. The assessment reveals several critical and high-risk vulnerabilities that require immediate attention. Key findings include the external exposure of an end-of-life database system, a systemic lack of Multi-Factor Authentication (MFA) across all critical services, and significant gaps in foundational security policies. These issues, when combined, create a high likelihood of a potential security breach. This report outlines the identified risks and provides actionable recommendations to mitigate them and improve the organization's overall security posture.
\end{abstract}

\tableofcontents
\newpage

% ===================================================================
\section{Overview and Scope}
% ===================================================================

This assessment was conducted to evaluate the current cybersecurity posture of \textbf{[Organization Name]}. The scope of this analysis includes:
\begin{itemize}
    \item A review of the organization's self-reported security controls.
    \item A technical analysis of an external network scan performed against the client's infrastructure.
    \item A correlation of technical findings and policy gaps with known, pre-existing risks.
\end{itemize}
The primary objective is to identify, analyze, and prioritize security risks and to provide clear, actionable recommendations for remediation.

% ===================================================================
\section{Organizational Information}
% ===================================================================

The following information was used as the basis for this assessment.
\begin{itemize}
    \item \textbf{Organization Name:} \textbf{[Organization Name]}
    \item \textbf{Email Domain:} \texttt{[Domain]}
    \item \textbf{External IP Scanned:} \texttt{[Client IP]}
\end{itemize}

% ===================================================================
\section{Security Control Review}
% ===================================================================

A review of the organization's security questionnaire revealed significant gaps in fundamental security controls. The following table summarizes the responses provided. A checkmark (\ding{51}) indicates a positive control is in place, while an X (\ding{55}) indicates a control gap.

\begin{table}[h!]
\centering
\caption{Organizational Security Controls Questionnaire}
\begin{tabular}{@{}lc@{}}
\toprule
\textbf{Control Question} & \textbf{Status} \\ \midrule
Do you require MFA to access email? & \ding{55} \\
Do you require MFA to log into computers? & \ding{55} \\
Do you require MFA to access sensitive data systems? & \ding{55} \\
Does your organization have an employee acceptable use policy? & \ding{55} \\
Does your organization do security awareness training for new employees? & \ding{51} \\
Does your organization do security awareness training for all employees annually? & \ding{51} \\ \bottomrule
\end{tabular}
\end{table}

\subsection{Analysis of Control Gaps}
The absence of Multi-Factor Authentication (MFA) across email, computer logins, and sensitive data systems represents a \textbf{critical risk}. This deficiency significantly lowers the barrier for an attacker, as a single compromised password could lead to a full-scale breach. Additionally, the lack of an Acceptable Use Policy (AUP) is a high-risk governance gap, leaving the organization without a formal framework to enforce secure employee behavior.

% ===================================================================
\section{Technical Scan Results}
% ===================================================================

An external network scan was performed against the target IP address \texttt{[Target IP]}. The scan identified one open port, which presents a significant security risk.

\begin{table}[h!]
\centering
\caption{Open Ports Detected on \texttt{[Target IP]}}
\begin{tabular}{@{}lllll@{}}
\toprule
\textbf{Port} & \textbf{State} & \textbf{Service} & \textbf{Product} & \textbf{Version} \\ \midrule
3306/tcp      & open           & mysql            & MySQL            & 5.7.33           \\ \bottomrule
\end{tabular}
\end{table}

\subsection{Technical Analysis}
\begin{itemize}
    \item \textbf{Exposed Database Service:} The scan confirms that a MySQL database server on port 3306 is directly accessible from the network. Exposing a database directly is a highly dangerous practice, as it allows attackers to directly target the database with brute-force attacks, credential stuffing, or exploits.
    \item \textbf{End-of-Life Software:} The detected version, \textbf{MySQL 5.7.33}, reached its official End of Life (EOL) in October 2023. This means it no longer receives security patches from the vendor. Any new vulnerabilities discovered in this version will remain unpatched, leaving the system perpetually vulnerable to exploitation. This is a \textbf{critical finding}.
\end{itemize}

% ===================================================================
\section{Consolidated Risk Assessment}
% ===================================================================

By correlating the security control gaps, technical findings, and pre-existing risk data, we have compiled a prioritized list of security risks facing the organization.

\begin{table}[h!]
\centering
\caption{Summary of Identified Risks}
\begin{tabular}{@{}p{0.3\linewidth}p{0.5\linewidth}l@{}}
\toprule
\textbf{Risk Name} & \textbf{Description} & \textbf{Severity} \\ \midrule
\textbf{Lack of Multi-Factor Authentication} &
The absence of MFA on email, endpoints, and data systems makes unauthorized access trivial if credentials are compromised. &
\textbf{Critical} \\

\textbf{Use of End-of-Life Software} &
The exposed MySQL 5.7 database is no longer supported and does not receive security updates, making it an easy target for exploitation. &
\textbf{Critical} \\

\textbf{Database Exposure} &
The MySQL database port (3306) is open to the network, inviting direct attacks. This confirms the pre-existing risk identified in Input 3. &
\textbf{High} \\

\textbf{Missing Acceptable Use Policy} &
The lack of a formal AUP creates ambiguity regarding secure practices for employees and limits the organization's ability to enforce security standards. &
\textbf{High} \\
\bottomrule
\end{tabular}
\end{table}

% ===================================================================
\section{Recommendations}
% ===================================================================

The following actions are recommended to mitigate the identified risks. They are prioritized based on severity and ease of implementation.

\subsection{Immediate Actions (To Be Completed Within 72 Hours)}
\begin{enumerate}
    \item \textbf{Restrict Access to Database Port:} Immediately implement firewall rules to block all public access to TCP port 3306. Access should be restricted to only trusted internal IP addresses.
    \item \textbf{Develop EOL Migration Plan:} Begin immediate planning to migrate the MySQL 5.7 database to a currently supported version (e.g., MySQL 8.0 or a managed cloud equivalent).
\end{enumerate}

\subsection{High-Priority Actions (To Be Completed Within 30 Days)}
\begin{enumerate}
    \item \textbf{Implement Multi-Factor Authentication (MFA):}
        \begin{itemize}
            \item Prioritize enabling MFA for all email accounts.
            \item Enable MFA for all systems containing sensitive data, especially those related to the exposed database.
            \item Plan the rollout of MFA for all employee computer logins.
        \end{itemize}
    \item \textbf{Develop and Implement an Acceptable Use Policy (AUP):} Draft, approve, and distribute an AUP to all employees. Require acknowledgment of the policy and integrate it into the new-hire onboarding process.
\end{enumerate}

\subsection{Long-Term Strategic Actions (To Be Completed Within 6 Months)}
\begin{enumerate}
    \item \textbf{Complete Database Migration:} Execute the migration plan to move off the end-of-life MySQL 5.7 instance to a fully patched and supported platform.
    \item \textbf{Implement a VPN for Remote Access:} For any required remote database administration, implement a Virtual Private Network (VPN) solution. This will ensure that sensitive services are never directly exposed to the internet.
\end{enumerate}

\end{document}
```