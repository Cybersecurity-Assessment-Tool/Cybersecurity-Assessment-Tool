```latex
\documentclass[12pt]{article}

% --- PACKAGE IMPORTS ---
\usepackage[margin=1in]{geometry}
\usepackage{pifont} % For checkmarks and crosses
\usepackage{booktabs} % For professional tables
\usepackage{hyperref} % For hyperlinks
\usepackage{url} % For URL formatting
\usepackage{seqsplit} % To split long strings in texttt
\usepackage[T1]{fontenc}

% --- DOCUMENT METADATA ---
\title{Cybersecurity Posture Assessment Report}
\author{Cybersecurity Analysis Division}
\date{\today}

\hypersetup{
    colorlinks=true,
    linkcolor=black,
    urlcolor=blue,
    pdftitle={Cybersecurity Posture Assessment Report},
    pdfauthor={Cybersecurity Analysis Division},
}

% --- BEGIN DOCUMENT ---
\begin{document}

\maketitle
\hrule
\vspace{1em}

% ==============================================================================
% 1. EXECUTIVE SUMMARY
% ==============================================================================
\section*{Executive Summary}

This report provides a cybersecurity assessment for \textbf{[Organization Name]}, synthesizing findings from a technical network scan, a review of existing risks, and an organizational security controls questionnaire.

The analysis indicates a notable contrast in the organization's security posture. On one hand, administrative controls, as reported in the questionnaire, appear robust. The organization has implemented key policies such as Multi-Factor Authentication (MFA), acceptable use policies, and security awareness training.

However, the technical scan revealed critical-level vulnerabilities that significantly elevate the organization's risk profile. A MySQL database service is publicly exposed to the internet on a server at \texttt{[Target IP]}. Furthermore, this database is running on MySQL version 5.7.33, which is an End-of-Life (EOL) product and no longer receives security updates. This combination of direct exposure and unpatchable software creates a high-impact target for malicious actors.

Immediate remediation is required to restrict access to the exposed database, followed by a plan to upgrade the underlying software to a supported version.

% ==============================================================================
% 2. ORGANIZATIONAL INFORMATION
% ==============================================================================
\section{Organizational Information}

The following details were used as the basis for this assessment. Anonymized placeholders are used where data was not provided.

\begin{itemize}
    \item \textbf{Organization Name:} \textbf{[Organization Name]}
    \item \textbf{Primary Domain:} \texttt{[Domain]}
    \item \textbf{External IP Scanned:} \texttt{[Target IP]}
    \item \textbf{Scan Date:} \today
\end{itemize}

% ==============================================================================
% 3. SECURITY CONTROL REVIEW (QUESTIONNAIRE)
% ==============================================================================
\section{Security Control Review}

The following table summarizes the organization's self-reported security controls. A green checkmark (\ding{51}) indicates a positive response ("Yes"), while a red cross (\ding{55}) would indicate a negative response ("No"). The organization reports full compliance across all surveyed controls, which is a strong indicator of a mature security governance program.

\begin{table}[h!]
\centering
\caption{Organizational Security Controls Questionnaire}
\begin{tabular}{lc}
\toprule
\textbf{Control Question} & \textbf{Status} \\
\midrule
Do you require MFA to access email? & \ding{51} \\
Do you require MFA to log into computers? & \ding{51} \\
Do you require MFA to access sensitive data systems? & \ding{51} \\
Does your organization have an employee acceptable use policy? & \ding{51} \\
Does your organization do security awareness training for new employees? & \ding{51} \\
Does your organization do security awareness training for all employees? & \ding{51} \\
\bottomrule
\end{tabular}
\end{table}

% ==============================================================================
% 4. TECHNICAL SCAN RESULTS
% ==============================================================================
\section{Technical Scan Results}

An external network scan was performed on the target IP address \texttt{[Target IP]}. The scan identified the following open port and running service.

\begin{table}[h!]
\centering
\caption{Open Ports and Services Detected}
\begin{tabular}{lllll}
\toprule
\textbf{Port} & \textbf{State} & \textbf{Service} & \textbf{Version} & \textbf{Notes} \\
\midrule
3306/tcp & Open & mysql & MySQL 5.7.33 & \textbf{Critical Finding} \\
\bottomrule
\end{tabular}
\end{table}

\subsection*{Analysis of Technical Findings}
The key finding is the exposure of a MySQL database service (port 3306) to the public internet. Database servers are high-value targets for attackers seeking to exfiltrate or ransom data and should not be directly accessible.

Compounding this issue is the software version in use. \textbf{MySQL 5.7 reached its official End-of-Life (EOL) in October 2023.} This means it no longer receives security patches from the vendor, and any newly discovered vulnerabilities will remain unpatched. Running EOL software, especially on a publicly exposed critical service, constitutes a severe and unacceptable risk.

% ==============================================================================
% 5. RISK ASSESSMENT SUMMARY
% ==============================================================================
\section{Risk Assessment Summary}

The following table correlates the pre-existing risk data with new findings from the technical scan. The severity is based on the CVSS v3.1 standard, where 7.0-8.9 is High and 9.0-10.0 is Critical.

\begin{table}[h!]
\centering
\caption{Identified and Correlated Risks}
\begin{tabular}{p{0.25\linewidth} p{0.1\linewidth} p{0.55\linewidth}}
\toprule
\textbf{Risk Name} & \textbf{Severity} & \textbf{Overview} \\
\midrule
\textbf{Database Exposure} & High (7.5) & The MySQL database port (3306) is open to the public internet, allowing attackers to directly attempt to connect, brute-force credentials, or exploit vulnerabilities. This finding confirms the pre-existing risk. \\
\addlinespace
\textbf{End-of-Life Software} & Critical (9.8) & The exposed MySQL service is running version 5.7.33, which is past its end-of-life. The system is unpatchable against any current or future vulnerabilities, making a compromise highly likely over time. \\
\bottomrule
\end{tabular}
\end{table}

% ==============================================================================
% 6. RECOMMENDATIONS
% ==============================================================================
\section{Recommendations}

Based on the correlated findings, the following remediation actions are recommended, prioritized by urgency.

\begin{enumerate}
    \item \textbf{Immediate: Restrict Access via Firewall.}
    \begin{itemize}
        \item \textbf{Action:} Configure the network firewall to deny all inbound traffic to TCP port 3306 on \texttt{[Target IP]} from the internet. Access should only be permitted from trusted, internal IP addresses.
        \item \textbf{Impact:} High. This action will immediately mitigate the exposure risk and is the most critical first step.
    \end{itemize}
    \vspace{1em}
    
    \item \textbf{Short-Term: Plan and Execute Software Upgrade.}
    \begin{itemize}
        \item \textbf{Action:} Develop a migration plan to upgrade the MySQL 5.7.33 database to a currently supported version (e.g., MySQL 8.x). This involves testing application compatibility, backing up data, and performing the upgrade in a controlled manner.
        \item \textbf{Impact:} High. This action mitigates the risk of exploitation via unpatchable vulnerabilities inherent in the EOL software.
    \end{itemize}
    \vspace{1em}
    
    \item \textbf{Long-Term: Implement Secure Access Architecture.}
    \begin{itemize}
        \item \textbf{Action:} For any required external administrative access, implement a secure solution such as a Virtual Private Network (VPN) or a bastion host (jump box). This ensures that sensitive services are never directly exposed to the internet, providing a defense-in-depth posture.
        \item \textbf{Impact:} Medium. This is a strategic improvement that reduces the likelihood of similar configuration errors in the future.
    \end{itemize}
\end{enumerate}

\end{document}
```