```latex
\documentclass[12pt]{article}

% --- PACKAGE IMPORTS ---
\usepackage[margin=1in]{geometry}
\usepackage{pifont} % For checkmarks and crosses
\usepackage{booktabs} % For professional tables
\usepackage{hyperref} % For clickable links
\usepackage{url} % For URL formatting
\usepackage{seqsplit} % For splitting long strings
\usepackage[T1]{fontenc}

% --- DOCUMENT METADATA ---
\title{Cybersecurity Posture Assessment Report \\ \large For: \textbf{[Organization Name]}}
\author{Cybersecurity Analysis Division}
\date{\today}

% --- HYPERREF SETUP ---
\hypersetup{
    colorlinks=true,
    linkcolor=black,
    urlcolor=blue,
    pdftitle={Cybersecurity Posture Assessment Report},
    pdfauthor={Cybersecurity Analysis Division},
}

\begin{document}

\maketitle
\hrule
\vspace{1cm}

% ==============================================================================
% EXECUTIVE SUMMARY
% ==============================================================================
\section*{Executive Summary}

This report provides a comprehensive assessment of the cybersecurity posture for \textbf{[Organization Name]}, based on an analysis of network scan data, organizational security controls, and pre-existing risk information. The assessment reveals several critical and high-risk security deficiencies that require immediate attention.

Key findings indicate a significant risk of unauthorized access and potential data breach. A technical scan identified a publicly accessible MySQL database running an outdated and unsupported version. This technical vulnerability is compounded by critical gaps in organizational security controls, most notably the widespread lack of Multi-Factor Authentication (MFA) across email, computer logins, and sensitive data systems. Furthermore, the absence of an employee acceptable use policy and security training for new hires points to a systemic weakness in the organization's security culture.

This report consolidates these findings into a prioritized list of actionable recommendations designed to mitigate the identified risks and strengthen the overall security posture of the organization.

% ==============================================================================
% ORGANIZATIONAL INFORMATION
% ==============================================================================
\section*{Organizational Information}

The following details were used as the basis for this assessment. Due to missing data in the provided inputs, placeholders have been used.

\begin{table}[h!]
\centering
\begin{tabular}{@{}ll@{}}
\toprule
\textbf{Attribute} & \textbf{Value} \\ \midrule
Organization Name & \textbf{[Organization Name]} \\
Primary Domain & \texttt{[Domain]} \\
External IP Address Scanned & \texttt{[Client IP]} \\ \bottomrule
\end{tabular}
\caption{Client Organizational Details}
\end{table}

% ==============================================================================
% SECURITY CONTROL REVIEW (FROM QUESTIONNAIRE)
% ==============================================================================
\section*{Security Control Review}

An analysis of the organization's security questionnaire reveals significant gaps in fundamental security controls. A "No" response, indicated by \ding{55}, highlights a missing control that increases organizational risk.

\begin{table}[h!]
\centering
\begin{tabular}{@{}lc@{}}
\toprule
\textbf{Control Question} & \textbf{Response} \\ \midrule
Do you require MFA to access email? & \ding{55} \\
Do you require MFA to log into computers? & \ding{55} \\
Do you require MFA to access sensitive data systems? & \ding{55} \\
Does your organization have an employee acceptable use policy? & \ding{55} \\
Does your organization do security awareness training for new employees? & \ding{55} \\
Does your organization do security awareness training for all employees annually? & \ding{51} \\ \bottomrule
\end{tabular}
\caption{Security Control Questionnaire Analysis}
\end{table}

\subsection*{Analysis}
The lack of MFA for email, computer, and sensitive system access is a \textbf{critical vulnerability}. This single point of failure makes the organization highly susceptible to credential theft and account takeover attacks. The absence of an acceptable use policy and security training for new hires further exacerbates risk by failing to establish a baseline of secure employee behavior.

% ==============================================================================
% TECHNICAL SCAN RESULTS
% ==============================================================================
\section*{Technical Scan Results}

A network scan was performed to identify open ports and exposed services. The target IP address was not specified in the input data and is represented by a placeholder.

\begin{table}[h!]
\centering
\begin{tabular}{@{}lllll@{}}
\toprule
\textbf{Target IP} & \textbf{Port} & \textbf{State} & \textbf{Service} & \textbf{Version} \\ \midrule
\texttt{[Target IP]} & 3306/tcp & open & mysql & MySQL 5.7.33 \\ \bottomrule
\end{tabular}
\caption{Nmap Scan Findings}
\end{table}

\subsection*{Analysis}
The scan confirms that a MySQL database service is directly exposed to the network on port 3306. This configuration is highly discouraged as it exposes the database to brute-force attacks, credential stuffing, and direct exploitation of vulnerabilities.

Furthermore, the identified version, \textbf{MySQL 5.7.33}, is outdated. The MySQL 5.7 series reached its end of life in October 2023 and no longer receives security patches. Running unsupported software with known vulnerabilities presents a high risk of compromise.

% ==============================================================================
% CONSOLIDATED RISK ASSESSMENT
% ==============================================================================
\section*{Consolidated Risk Assessment}

The following table synthesizes findings from the security control review, technical scan, and pre-existing risk data into a consolidated list of identified risks.

\begin{table}[h!]
\centering
\begin{tabular}{@{}lp{7cm}l@{}}
\toprule
\textbf{Risk ID} & \textbf{Risk Title \& Description} & \textbf{Severity} \\ \midrule
\textbf{R-01} & \textbf{Public Database Exposure} \newline A MySQL database is directly accessible from the network, exposing it to external threats. This finding correlates with a known risk. & \textbf{Critical} \\
\addlinespace
\textbf{R-02} & \textbf{Lack of Multi-Factor Authentication} \newline The absence of MFA for critical systems (email, logins) creates a high risk of account compromise through simple password guessing or phishing. & \textbf{Critical} \\
\addlinespace
\textbf{R-03} & \textbf{Outdated Database Software} \newline The exposed MySQL server is running version 5.7.33, which is past its end-of-life and contains known, unpatched vulnerabilities. & \textbf{High} \\
\addlinespace
\textbf{R-04} & \textbf{Absence of Security Policies \& Training} \newline Lack of an acceptable use policy and new-hire training results in an inconsistent security baseline and increases the likelihood of human error. & \textbf{High} \\ \bottomrule
\end{tabular}
\caption{Summary of Identified Risks}
\end{table}

% ==============================================================================
% RECOMMENDATIONS
% ==============================================================================
\section*{Recommendations}

The following prioritized recommendations are provided to mitigate the identified risks and improve the organization's security posture.

\subsection*{Immediate Priority (P0)}
\begin{itemize}
    \item \textbf{Restrict Database Access:} Immediately implement firewall rules to restrict access to port 3306. Access should only be permitted from trusted internal IP addresses or via a secure VPN connection. Public access must be disabled.
\end{itemize}

\subsection*{High Priority (P1)}
\begin{itemize}
    \item \textbf{Implement MFA:} Deploy mandatory Multi-Factor Authentication (MFA) across all critical platforms, starting with email, remote access solutions (VPNs), and administrative accounts. Extend this to all employee computer and sensitive data system logins.
    \item \textbf{Develop Security Policies:} Create and enforce a formal Employee Acceptable Use Policy that clearly defines rules for using company assets, data handling, and internet usage.
\end{itemize}

\subsection*{Medium Priority (P2)}
\begin{itemize}
    \item \textbf{Upgrade Database Server:} Plan and execute the upgrade of the MySQL 5.7.33 server to a currently supported version (e.g., MySQL 8.x) to ensure it receives ongoing security patches.
    \item \textbf{Establish Security Training Program:} Implement a mandatory security awareness training program for all new employees upon hiring. Ensure the existing annual training program is comprehensive and addresses current threats like phishing and social engineering.
\end{itemize}

\end{document}
```