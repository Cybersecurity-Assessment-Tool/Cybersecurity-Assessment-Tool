```latex
\documentclass[12pt, a4paper]{article}

% Preamble: Required Packages
\usepackage[margin=1in]{geometry}
\usepackage{pifont} % For checkmarks and crosses
\usepackage{booktabs} % For professional tables
\usepackage{hyperref} % For hyperlinks
\usepackage{url} % For URL formatting
\usepackage{seqsplit} % To split long strings in tt font
\usepackage{graphicx}
\usepackage[table]{xcolor}
\usepackage{fancyhdr}
\usepackage{lastpage}

% --- Document Setup ---
\hypersetup{
    colorlinks=true,
    linkcolor=blue,
    filecolor=magenta,
    urlcolor=cyan,
    pdftitle={Cybersecurity Posture Assessment Report},
    pdfpagemode=FullScreen,
}

\pagestyle{fancy}
\fancyhf{} % Clear all header and footer fields
\fancyhead[L]{Cybersecurity Posture Assessment}
\fancyhead[R]{\textbf{[Organization Name]}}
\fancyfoot[C]{\thepage\ of \pageref{LastPage}}
\renewcommand{\headrulewidth}{0.4pt}
\renewcommand{\footrulewidth}{0.4pt}

% --- Custom Commands ---
\newcommand{\yes}{\ding{51}}
\newcommand{\no}{\ding{55}}

% --- Document Title ---
\title{
    \vspace{2cm}
    \textbf{Cybersecurity Posture Assessment Report} \\
    \large \textit{Confidential}
    \vspace{1.5cm}
}
\author{Cybersecurity Analysis Division}
\date{\today}

% --- BEGIN DOCUMENT ---
\begin{document}

\maketitle
\thispagestyle{empty}
\newpage

\tableofcontents
\newpage

% --- Section 1: Executive Summary ---
\section{Executive Summary}
This report provides a comprehensive cybersecurity posture assessment for \textbf{[Organization Name]}, based on an analysis of network scan data, organizational security controls, and a review of pre-existing risks. The assessment was conducted on \today.

The analysis reveals a mixed security posture. The organization has implemented foundational controls such as Multi-Factor Authentication (MFA) for email and computer access. The external network scan of the target IP address did not identify any open ports, which is a positive security finding.

However, two critical gaps in security controls were identified through the organizational questionnaire:
\begin{itemize}
    \item \textbf{Lack of MFA for Sensitive Data Systems:} The absence of MFA on systems housing sensitive data presents a critical risk, significantly increasing the potential impact of a compromised credential.
    \item \textbf{No Security Training for New Employees:} New hires are not provided with security awareness training during onboarding, leaving a high-risk window where they are more susceptible to social engineering and policy violations.
\end{itemize}

Furthermore, a notable discrepancy was found between the technical scan results, which show Port 80 as closed, and the current risk register, which lists an active risk for an "Unencrypted Web Server" on an open Port 80. This suggests that the risk may have been remediated, but the documentation has not been updated.

Immediate action is recommended to address the identified control gaps. The highest priorities are the enforcement of MFA on all sensitive systems and the integration of security training into the employee onboarding process.

% --- Section 2: Organizational Information ---
\section{Organizational Information}
This section details the organizational information used as a basis for this assessment. Due to the anonymized nature of the provided data, placeholders have been used where necessary.

\begin{table}[h!]
\centering
\caption{Client Organizational Details}
\label{tab:org_info}
\begin{tabular}{@{}ll@{}}
\toprule
\textbf{Attribute} & \textbf{Value} \\ \midrule
Organization Name & \textbf{[Organization Name]} \\
Primary Email Domain & \texttt{[Domain]} \\
External IP Address Scanned & \texttt{[Client IP]} \\
Target IP Address for Scan & \texttt{[Target IP]} \\ \bottomrule
\end{tabular}
\end{table}

% --- Section 3: Security Control Review ---
\section{Security Control Review}
The following table summarizes the organization's responses to a security controls questionnaire. This review provides insight into the maturity and implementation of key administrative and technical security policies. Gaps are highlighted for immediate attention.

\begin{table}[h!]
\centering
\caption{Security Controls Questionnaire Analysis}
\label{tab:controls}
\begin{tabular}{@{}p{8.5cm}ccp{3cm}@{}}
\toprule
\textbf{Control Question} & \multicolumn{2}{c}{\textbf{Response}} & \textbf{Assessment} \\ \midrule
Do you require MFA to access email? & Yes & \yes & Implemented \\
Do you require MFA to log into computers? & Yes & \yes & Implemented \\
\rowcolor{red!15} Do you require MFA to access sensitive data systems? & No & \no & \textbf{Critical Gap} \\
Does your organization have an employee acceptable use policy? & Yes & \yes & Implemented \\
\rowcolor{orange!20} Does your organization do security awareness training for new employees? & No & \no & \textbf{High-Risk Gap} \\
Does your organization do security awareness training for all employees at least once per year? & Yes & \yes & Implemented \\ \bottomrule
\end{tabular}
\end{table}

% --- Section 4: Technical Scan Results ---
\section{Technical Scan Results}
An external network scan was performed to identify open ports and exposed services on the designated target system.

\subsection{Scan Summary}
\begin{itemize}
    \item \textbf{Scanner Used:} Nmap
    \item \textbf{Target IP Address:} \texttt{[Target IP]}
    \item \textbf{Scan Date:} Scan data provided on \today
\end{itemize}

\subsection{Findings}
The scan revealed no open ports on the target host. All tested ports, including common web service ports, were found to be in a `closed` state. This is a positive security finding, as it indicates a properly configured firewall that minimizes the external attack surface.

\begin{table}[h!]
\centering
\caption{Port Scan Results for Target: \texttt{[Target IP]}}
\label{tab:scan_results}
\begin{tabular}{@{}llll@{}}
\toprule
\textbf{Port} & \textbf{State} & \textbf{Service} & \textbf{Product/Version} \\ \midrule
80/tcp & closed & http & N/A \\
\multicolumn{4}{l}{\textit{Note: Only significant port states are listed. No open ports were detected.}} \\
\bottomrule
\end{tabular}
\end{table}

% --- Section 5: Risk Assessment & Correlation ---
\section{Risk Assessment \& Correlation}
This section synthesizes findings from the security control review, technical scan, and pre-existing risk data. Each identified risk is assigned a severity level to guide prioritization.

\begin{table}[h!]
\centering
\caption{Consolidated Risk Summary}
\label{tab:risk_summary}
\begin{tabular}{@{}p{1cm}p{3.5cm}p{6cm}p{2cm}@{}}
\toprule
\textbf{ID} & \textbf{Risk Title} & \textbf{Description} & \textbf{Severity} \\ \midrule
\rowcolor{red!15}
R-01 & \textbf{Lack of MFA on Sensitive Systems} & The absence of mandatory MFA for accessing sensitive data systems exposes critical assets to unauthorized access via credential theft or compromise. & \textbf{Critical} \\
\rowcolor{orange!20}
R-02 & \textbf{No Security Training for New Employees} & New hires are not trained on security policies and threats, making them a high-value target for social engineering attacks during their initial, vulnerable period. & \textbf{High} \\
\rowcolor{yellow!25}
R-03 & \textbf{Risk Register Discrepancy} & The current risk register lists "Unencrypted Web Server" (Port 80 open) as an active risk. However, our scan confirms Port 80 is closed. This indicates a potential gap in risk management tracking and validation. & \textbf{Informational} \\
\bottomrule
\end{tabular}
\end{table}

% --- Section 6: Recommendations ---
\section{Recommendations}
Based on the analysis, the following actions are recommended to mitigate the identified risks and improve the overall security posture of \textbf{[Organization Name]}. Recommendations are prioritized by severity.

\begin{enumerate}
    \item \textbf{Priority: Critical - Remediate R-01} \\
    \textbf{Action:} Enforce Multi-Factor Authentication (MFA) across all systems and applications that store, process, or transmit sensitive organizational or customer data.
    \begin{itemize}
        \item \textit{Justification:} This is the single most effective control to prevent unauthorized access resulting from compromised credentials. It directly protects the organization's most valuable data assets.
    \end{itemize}
    \vspace{0.5cm}

    \item \textbf{Priority: High - Remediate R-02} \\
    \textbf{Action:} Develop and implement a mandatory security awareness training module as part of the standard new employee onboarding process.
    \begin{itemize}
        \item \textit{Justification:} Educating employees from day one establishes a security-conscious culture and reduces the likelihood of successful phishing and other social engineering attacks.
    \end{itemize}
    \vspace{0.5cm}

    \item \textbf{Priority: Medium - Address R-03} \\
    \textbf{Action:} Conduct a verification of the "Unencrypted Web Server" risk. Confirm that Port 80 is indeed closed across all relevant external infrastructure. If confirmed, formally close the risk item and update the risk register to reflect its remediation.
    \begin{itemize}
        \item \textit{Justification:} An accurate and up-to-date risk register is essential for effective security governance, resource allocation, and compliance.
    \end{itemize}
\end{enumerate}

\end{document}
```