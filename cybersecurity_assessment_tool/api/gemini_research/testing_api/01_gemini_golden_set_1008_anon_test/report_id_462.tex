```latex
\documentclass[12pt]{article}

% ----------------------------------------------------------------------
% PREAMBLE
% ----------------------------------------------------------------------
\usepackage[margin=1in]{geometry}
\usepackage{pifont} % For checkmarks and crosses
\usepackage{booktabs} % For professional tables
\usepackage{hyperref} % For hyperlinks and metadata
\usepackage{url}      % For formatting URLs
\usepackage{seqsplit} % To split long strings in tt font
\usepackage{xcolor}   % For colors

% Hyperref Setup
\hypersetup{
    colorlinks=true,
    linkcolor=black,
    urlcolor=blue,
    pdftitle={Cybersecurity Assessment Report},
    pdfauthor={Cybersecurity Analyst},
    pdfsubject={Security Analysis},
    pdfkeywords={Cybersecurity, Nmap, Risk Assessment}
}

% Define custom colors
\definecolor{darkred}{rgb}{0.55, 0.0, 0.0}
\definecolor{darkgreen}{rgb}{0.0, 0.39, 0.0}

% Define custom commands for Yes/No
\newcommand{\yes}{\textcolor{darkgreen}{\ding{51}}}
\newcommand{\no}{\textcolor{darkred}{\ding{55}}}

% ----------------------------------------------------------------------
% DOCUMENT START
% ----------------------------------------------------------------------
\begin{document}

% ----------------------------------------------------------------------
% TITLE PAGE
% ----------------------------------------------------------------------
\begin{titlepage}
    \centering
    \vspace*{\stretch{1.0}}
    \Huge{\textbf{Cybersecurity Assessment Report}}
    \vspace{1.5cm}
    \Large{\textbf{For: \textbf{[Organization Name]}}}
    \vspace{2.5cm}
    \large{
        \begin{tabular}{ll}
            \textbf{Date of Report:} & \today \\
            \textbf{Author:} & Cybersecurity Analyst \\
            \textbf{Classification:} & Confidential \\
        \end{tabular}
    }
    \vspace*{\stretch{2.0}}
    \small{\textit{This report contains sensitive information regarding the security posture of the organization. Distribution should be limited to authorized personnel only.}}
\end{titlepage}

\tableofcontents
\newpage

% ----------------------------------------------------------------------
% SECTION 1: EXECUTIVE SUMMARY
% ----------------------------------------------------------------------
\section{Executive Summary}

This report provides a comprehensive cybersecurity assessment for \textbf{[Organization Name]}, synthesizing data from a network vulnerability scan, a security controls questionnaire, and a review of pre-existing risks.

The assessment reveals a mixed security posture. On a positive note, a previously identified high-risk vulnerability, an unencrypted web server on Port 80, has been successfully remediated. Our technical scan confirms that this port is now closed, which significantly improves the organization's external network security.

However, the analysis of organizational security controls has identified critical gaps in the employee security awareness program. The lack of mandatory security training for both new and existing employees represents a high-risk exposure. This deficiency makes the organization highly susceptible to human-centric attacks such as phishing, social engineering, and malware infection via user error.

Our primary recommendation is the immediate implementation of a comprehensive security awareness training program. While the technical security posture has shown improvement, addressing the human element is now the most critical step to building a resilient and holistic security defense.

% ----------------------------------------------------------------------
% SECTION 2: ORGANIZATIONAL INFORMATION
% ----------------------------------------------------------------------
\section{Organizational Information}

The following details were used as the basis for this assessment. Due to the anonymized nature of the provided data, placeholders are used where specific information was not available.

\begin{itemize}
    \item \textbf{Organization Name:} \textbf{[Organization Name]}
    \item \textbf{Primary Domain:} \texttt{[Domain]}
    \item \textbf{Scanned External IP:} \texttt{[Client IP]}
\end{itemize}

% ----------------------------------------------------------------------
% SECTION 3: SECURITY CONTROL REVIEW
% ----------------------------------------------------------------------
\section{Security Control Review}

A security questionnaire was completed to evaluate the current state of administrative and organizational security controls. The responses are summarized in the table below. Items marked with a \no{} indicate significant gaps in the security framework.

\begin{table}[h!]
\centering
\caption{Security Controls Questionnaire Results}
\begin{tabular}{p{0.8\textwidth} c}
\toprule
\textbf{Control Question} & \textbf{Response} \\
\midrule
Do you require MFA to access email? & \yes \\
Do you require MFA to log into computers? & \yes \\
Do you require MFA to access sensitive data systems? & \yes \\
Does your organization have an employee acceptable use policy? & \yes \\
\midrule
\rowcolor{red!10}
Does your organization do security awareness training for new employees? & \no \\
\rowcolor{red!10}
Does your organization do security awareness training for all employees at least once per year? & \no \\
\bottomrule
\end{tabular}
\end{table}

\paragraph{Analysis:} The organization has implemented strong Multi-Factor Authentication (MFA) controls across key systems, which is commendable. However, the complete absence of a security awareness training program is a critical vulnerability. Without proper training, employees are the weakest link in the security chain and are more likely to fall victim to common cyberattacks, potentially bypassing other technical controls.

% ----------------------------------------------------------------------
% SECTION 4: TECHNICAL SCAN RESULTS
% ----------------------------------------------------------------------
\section{Technical Scan Results}

An external network scan was performed to identify open ports and exposed services on the organization's public-facing infrastructure.

\begin{itemize}
    \item \textbf{Target IP Address:} \texttt{[Target IP]}
    \item \textbf{Scan Date:} [Scan Date]
\end{itemize}

The scan revealed a very limited external attack surface, with no open ports detected. The table below details the status of a specific port of interest.

\begin{table}[h!]
\centering
\caption{Nmap Scan Port Summary}
\begin{tabular}{cccc}
\toprule
\textbf{Port} & \textbf{State} & \textbf{Service} & \textbf{Product / Version} \\
\midrule
80/tcp & closed & http & N/A \\
\bottomrule
\end{tabular}
\end{table}

\paragraph{Analysis:} The technical scan results are highly positive. The finding that Port 80 is \textbf{closed} directly contradicts a previously identified risk ("Unencrypted Web Server"). This indicates that the organization has taken successful remedial action to close an unnecessary port, thereby reducing its attack surface and mitigating the risk of unencrypted communications. No other vulnerabilities were identified during this external scan.

% ----------------------------------------------------------------------
% SECTION 5: CONSOLIDATED RISK ASSESSMENT
% ----------------------------------------------------------------------
\section{Consolidated Risk Assessment}

This section correlates findings from the security control review, the technical scan, and pre-existing risk data to provide a holistic view of the current risk landscape.

\begin{table}[h!]
\centering
\caption{Summary of Identified Risks}
\begin{tabular}{p{0.3\textwidth} p{0.5\textwidth} c}
\toprule
\textbf{Risk Name} & \textbf{Description} & \textbf{Severity} \\
\midrule
\rowcolor{red!10}
Lack of Security Awareness Training & The absence of a formal training program for new and existing employees exposes the organization to a high likelihood of successful phishing and social engineering attacks. & \textbf{High} \\
\midrule
\rowcolor{green!10}
Unencrypted Web Server \textit{(Remediated)} & A previously identified risk of an open Port 80 has been successfully addressed. The port is now confirmed closed, and the risk is considered resolved. & Informational \\
\bottomrule
\end{tabular}
\end{table}

% ----------------------------------------------------------------------
% SECTION 6: RECOMMENDATIONS
% ----------------------------------------------------------------------
\section{Recommendations}

Based on the consolidated risk assessment, the following actions are recommended to enhance the security posture of \textbf{[Organization Name]}.

\subsection{High Priority: Implement a Security Awareness Program}
The most critical action is to address the gap in employee security training. A robust program is the most effective defense against human-targeted attacks.

\begin{itemize}
    \item \textbf{Onboarding Training:} Develop and mandate a security awareness training module for all new hires as a standard part of the onboarding process. This should cover topics such as phishing identification, password hygiene, and the acceptable use policy.
    \item \textbf{Annual Refresher Training:} Implement mandatory annual security training for all staff. This ensures that knowledge remains current and reinforces best practices.
    \item \textbf{Phishing Simulations:} Conduct periodic, simulated phishing campaigns to test employee vigilance and provide targeted, in-the-moment training for those who click on malicious links.
\end{itemize}

\subsection{Informational: Maintain Proactive Vulnerability Management}
The organization should be commended for remediating the Port 80 vulnerability. This proactive stance should be maintained.

\begin{itemize}
    \item \textbf{Continue Regular Scanning:} Continue to perform regular, automated vulnerability scans of all external and internal infrastructure to identify and address new risks as they emerge.
    \item \textbf{Formalize Patch Management:} Ensure a formal patch management policy is in place to apply security updates to all systems in a timely manner.
\end{itemize}

% ----------------------------------------------------------------------
% DOCUMENT END
% ----------------------------------------------------------------------
\end{document}
```