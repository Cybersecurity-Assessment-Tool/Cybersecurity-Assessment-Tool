```latex
\documentclass[12pt]{article}

% ----------------------------------------------------------------------
% PACKAGES
% ----------------------------------------------------------------------
\usepackage[margin=1in]{geometry} % For setting page margins
\usepackage{pifont}               % For checkmark and X symbols (\ding)
\usepackage{booktabs}             % For professional-looking tables
\usepackage{hyperref}             % For clickable links and table of contents
\usepackage{url}                  % For formatting URLs
\usepackage{seqsplit}             % To split long strings without breaking words
\usepackage{graphicx}             % For including logos/images
\usepackage{xcolor}               % For custom colors

% ----------------------------------------------------------------------
% DOCUMENT SETUP & CUSTOM COMMANDS
% ----------------------------------------------------------------------
\hypersetup{
    colorlinks=true,
    linkcolor=blue,
    filecolor=magenta,      
    urlcolor=cyan,
    pdftitle={Cybersecurity Posture Assessment Report},
    pdfauthor={Cybersecurity Analysis Division},
}

\newcommand{\yes}{\ding{51}} % Green checkmark
\newcommand{\no}{\ding{55}}  % Red X

% ----------------------------------------------------------------------
% BEGIN DOCUMENT
% ----------------------------------------------------------------------
\begin{document}

\title{Cybersecurity Posture Assessment Report}
\author{Cybersecurity Analysis Division}
\date{\today}
\maketitle

\begin{abstract}
This report provides a comprehensive analysis of the cybersecurity posture for \textbf{[Organization Name]}. The assessment is based on a synthesis of external network scan data, a review of self-reported security controls, and an evaluation of known existing risks. The findings indicate several critical and high-severity risks that require immediate attention, including a publicly exposed, vulnerable FTP server and significant gaps in access control policies, specifically the lack of Multi-Factor Authentication (MFA). Actionable recommendations are provided to mitigate these risks and improve the overall security posture.
\end{abstract}

\tableofcontents
\newpage

\section{Organizational Information}
This report pertains to the following organization and assets. The information provided has been anonymized as per the engagement parameters.

\begin{center}
\begin{tabular}{@{}ll}
\toprule
\textbf{Attribute} & \textbf{Value} \\
\midrule
\textbf{Organization Name:} & \textbf{[Organization Name]} \\
\textbf{Primary Domain:}    & \texttt{[Domain]} \\
\textbf{External IP Scanned:} & \texttt{[Client IP]} \\
\bottomrule
\end{tabular}
\end{center}

\section{Security Control Review}
This section reviews the organization's self-reported security controls based on the provided questionnaire. Identified gaps often represent significant policy or procedural risks that can be exploited by threat actors.

\begin{center}
\begin{tabular}{p{0.7\textwidth}c}
\toprule
\textbf{Control Question} & \textbf{Status} \\
\midrule
Do you require MFA to access email? & \no \\
Do you require MFA to log into computers? & \no \\
Do you require MFA to access sensitive data systems? & \yes \\
Does your organization have an employee acceptable use policy? & \yes \\
Does your organization do security awareness training for new employees? & \no \\
Does your organization do security awareness training for all employees at least once per year? & \yes \\
\bottomrule
\end{tabular}
\end{center}

\subsection*{Analysis}
The review highlights critical gaps in identity and access management. The absence of Multi-Factor Authentication (MFA) for email and general computer logins drastically increases the risk of account compromise via credential theft or phishing. Furthermore, the lack of security awareness training for new employees creates an immediate vulnerability, as new staff are often prime targets for social engineering attacks before they are familiar with corporate policies.

\section{Technical Scan Results}
An external network scan was conducted against the client's perimeter to identify open ports and services exposed to the internet.

\subsection*{Target Information}
\begin{itemize}
    \item \textbf{Target IP:} \texttt{[Target IP]}
    \item \textbf{Scan Date:} Date Not Provided (Report Generation: \today)
    \item \textbf{Host Status:} Up
\end{itemize}

\subsection*{Open Ports and Services}
The following table details the services discovered on the target system.

\begin{center}
\begin{tabular}{lllll}
\toprule
\textbf{Port} & \textbf{State} & \textbf{Service} & \textbf{Version} & \textbf{Notes} \\
\midrule
21/tcp & Open & ftp & vsftpd 2.3.4 & Anonymous FTP login allowed \\
\bottomrule
\end{tabular}
\end{center}

\subsection*{Analysis}
The scan identified a critically vulnerable service. \textbf{vsftpd version 2.3.4} is known to contain a backdoor vulnerability (\textbf{CVE-2011-2523}). An attacker can gain a remote command shell on the server by sending a specific string as the username. Compounding this issue, the FTP server is configured to allow \textbf{anonymous logins}, making it trivial for an unauthorized user to access the system and exploit this vulnerability. This represents an immediate and severe threat to the organization's network integrity and confidentiality.

\section{Consolidated Risk Assessment}
The following table summarizes the key identified risks, correlating findings from the security control review, technical scan, and pre-existing risk data.

\begin{center}
\begin{tabular}{p{0.3\textwidth}p{0.5\textwidth}l}
\toprule
\textbf{Risk Name} & \textbf{Overview} & \textbf{Severity} \\
\midrule
\textbf{Vulnerable External FTP Server} & The external server is running vsftpd 2.3.4, which has a known remote code execution backdoor (CVE-2011-2523). Anonymous login is enabled, allowing unauthenticated access. & \textbf{Critical} \\
\addlinespace
\textbf{No MFA for Critical Systems} & Multi-Factor Authentication is not enforced for email access or computer logins, making credential theft or phishing a high-impact event that could lead to widespread compromise. & \textbf{Critical} \\
\addlinespace
\textbf{Inadequate Employee Onboarding} & New employees do not receive security awareness training, making them highly susceptible to social engineering attacks from their first day. & \textbf{High} \\
\addlinespace
\textbf{Outdated Windows Policy} & Computers are running Windows 7, an unsupported operating system that no longer receives security updates, leaving them vulnerable to known exploits. (CVSS 5.0) & Medium \\
\bottomrule
\end{tabular}
\end{center}

\newpage
\section{Recommendations}
Based on the assessment, the following actions are recommended to mitigate the identified risks. Recommendations are prioritized by severity to address the most urgent threats first.

\subsection*{Immediate Actions (Critical Risks)}
\begin{enumerate}
    \item \textbf{Remediate Vulnerable FTP Server:}
        \begin{itemize}
            \item \textbf{Action:} Immediately take the FTP server at \texttt{[Target IP]} offline.
            \item \textbf{Investigation:} Conduct a forensic analysis of the server to determine if it has already been compromised.
            \item \textbf{Remediation:} If the FTP service is a business requirement, rebuild the server using a modern, patched operating system and the latest stable version of the FTP software. Disable anonymous access and configure strong authentication. If the service is not required, decommission it permanently.
        \end{itemize}
    \item \textbf{Implement Multi-Factor Authentication (MFA):}
        \begin{itemize}
            \item \textbf{Action:} Begin the phased rollout of MFA across the organization immediately.
            \item \textbf{Prioritization:} Enforce MFA for all email access (cloud and on-premise) and for all administrative/privileged accounts within the next 30 days.
            \item \textbf{Roadmap:} Develop a project plan to enforce MFA for all general computer and system logins within the next quarter.
        \end{itemize}
\end{enumerate}

\subsection*{Near-Term Actions (High \& Medium Risks)}
\begin{enumerate}
    \setcounter{enumi}{2} % Continue numbering from the previous list
    \item \textbf{Enhance Security Awareness Program:}
        \begin{itemize}
            \item \textbf{Action:} Integrate mandatory security awareness training into the new employee onboarding process. This must be completed before a new hire is granted system access.
            \item \textbf{Content:} The training should cover, at a minimum: phishing identification, password hygiene, and the organization's acceptable use policy.
        \end{itemize}
    \item \textbf{Upgrade End-of-Life Systems:}
        \begin{itemize}
            \item \textbf{Action:} Accelerate the project to upgrade all remaining Windows 7 workstations to a supported operating system (e.g., Windows 10/11).
            \item \textbf{Mitigation:} For any systems that cannot be immediately upgraded, isolate them in a segmented network with restricted access to and from the main corporate network.
        \end{itemize}
\end{enumerate}

\end{document}
```