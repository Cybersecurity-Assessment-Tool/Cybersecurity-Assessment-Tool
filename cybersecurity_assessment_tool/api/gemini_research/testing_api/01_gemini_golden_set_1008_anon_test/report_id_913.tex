```latex
\documentclass[12pt]{article}

% Preamble: Required Packages
\usepackage[margin=1in]{geometry}
\usepackage{pifont} % For checkmarks and crosses (\ding{51} and \ding{55})
\usepackage{booktabs} % For professional-looking tables
\usepackage{hyperref}
\usepackage{url}
\usepackage{seqsplit} % To break long strings in \texttt
\usepackage{xcolor}
\usepackage{graphicx}
\usepackage{titling}

% --- Document Setup ---
\hypersetup{
    colorlinks=true,
    linkcolor=blue,
    filecolor=magenta,
    urlcolor=cyan,
    pdftitle={Cybersecurity Posture Assessment Report},
    pdfauthor={Cybersecurity Analysis Division},
}

% Custom commands for consistency
\newcommand{\yes}{\textcolor{green}{\ding{51}}}
\newcommand{\no}{\textcolor{red}{\ding{55}}}
\newcommand{\orgname}{\textbf{[Organization Name]}}
\newcommand{\clientip}{\texttt{[Client IP]}}
\newcommand{\clientdomain}{\texttt{[Domain]}}
\newcommand{\targetip}{\texttt{[Target IP]}}

% --- Title Information ---
\title{Cybersecurity Posture Assessment Report}
\author{Cybersecurity Analysis Division}
\date{November 22, 2025}

\begin{document}

\maketitle
\thispagestyle{empty}
\newpage

\tableofcontents
\newpage

% --- Section 1: Executive Summary ---
\section{Executive Summary}

This report provides a comprehensive assessment of the cybersecurity posture for \orgname. The analysis is based on a correlation of network scan data, a review of internal security controls via a questionnaire, and an evaluation of pre-existing risks. The assessment was conducted on November 22, 2025.

The overall security posture requires immediate attention. Key findings indicate critical gaps in both technical and administrative controls. An externally facing web server was found to be running a significantly outdated and vulnerable version of Nginx (1.18.0). This technical vulnerability is compounded by critical administrative control deficiencies, including the lack of mandatory Multi-Factor Authentication (MFA) for computer logins, the absence of an employee Acceptable Use Policy (AUP), and an incomplete security awareness training program.

These combined findings present a significant and exploitable attack surface. We have categorized the identified risks and provided actionable recommendations to mitigate them and improve the organization's overall resilience against cyber threats.

% --- Section 2: Organizational Information ---
\section{Organizational Information}

This section details the information provided by the client organization for the scope of this assessment.
\begin{itemize}
    \item \textbf{Organization Name:} \orgname
    \item \textbf{Primary Email Domain:} \clientdomain
    \item \textbf{External IP Address Scanned:} \clientip
\end{itemize}

% --- Section 3: Security Control Review ---
\section{Security Control Review}

The following table summarizes the organization's responses to a security controls questionnaire. A red \no\ indicates a significant gap in security controls that increases organizational risk.

\begin{table}[h!]
\centering
\caption{Security Controls Questionnaire Results}
\label{tab:controls}
\begin{tabular}{p{0.7\textwidth} c c}
\toprule
\textbf{Control Question} & \textbf{Response} & \textbf{Status} \\
\midrule
Do you require MFA to access email? & Yes & \yes \\
Do you require MFA to log into computers? & No & \no \\
Do you require MFA to access sensitive data systems? & Yes & \yes \\
Does your organization have an employee acceptable use policy? & No & \no \\
Does your organization do security awareness training for new employees? & Yes & \yes \\
Does your organization do security awareness training for all employees at least once per year? & No & \no \\
\bottomrule
\end{tabular}
\end{table}

\subsection*{Analysis of Control Gaps}
The review identified three critical control gaps:
\begin{enumerate}
    \item \textbf{Lack of Endpoint MFA:} The absence of MFA for computer logins significantly increases the risk of unauthorized access from compromised credentials.
    \item \textbf{Missing Acceptable Use Policy (AUP):} Without a formal AUP, there is no enforceable standard for employee behavior regarding company assets, which can lead to unintentional data exposure or misuse.
    \item \textbf{Inadequate Security Awareness Training:} While new employees receive training, the lack of an annual refresher for all staff means that the workforce's ability to recognize and respond to evolving threats like phishing diminishes over time.
\end{enumerate}

% --- Section 4: Technical Scan Results ---
\section{Technical Scan Results}

An external network scan was performed to identify exposed services and potential vulnerabilities.

\begin{itemize}
    \item \textbf{Scan Date:} 2025-11-22T10:00:00Z
    \item \textbf{Target IP Address:} \targetip
\end{itemize}

\begin{table}[h!]
\centering
\caption{Open Ports and Services Detected on \targetip}
\label{tab:nmap}
\begin{tabular}{l l l l l}
\toprule
\textbf{Port} & \textbf{State} & \textbf{Service} & \textbf{Product} & \textbf{Version} \\
\midrule
443/tcp & open & https & nginx & 1.18.0 \\
\bottomrule
\end{tabular}
\end{table}

\subsection*{Analysis of Technical Findings}
The scan identified an Nginx web server, version \textbf{1.18.0}, accessible to the public internet. This version was released in April 2020 and is now considered obsolete. It contains multiple known vulnerabilities, including but not limited to those related to request smuggling and improper input validation. An attacker could exploit these vulnerabilities to bypass security controls, access sensitive information, or achieve remote code execution.

% --- Section 5: Consolidated Risk Assessment ---
\section{Consolidated Risk Assessment}

This section synthesizes findings from the security control review and the technical scan. No pre-existing vulnerabilities were reported in the provided data. The following table outlines the newly identified risks.

\begin{table}[h!]
\centering
\caption{Summary of Identified Risks}
\label{tab:risks}
\begin{tabular}{p{0.1\textwidth} p{0.25\textwidth} p{0.45\textwidth} l}
\toprule
\textbf{Risk ID} & \textbf{Risk Name} & \textbf{Description} & \textbf{Severity} \\
\midrule
RISK-001 & Outdated Web Server Software & The public-facing web server runs Nginx 1.18.0, an outdated version with multiple known critical vulnerabilities. & \textbf{Critical} \\
\addlinespace
RISK-002 & Lack of Endpoint MFA & User computers can be accessed with only a password, exposing the internal network to takeover if credentials are stolen. & \textbf{Critical} \\
\addlinespace
RISK-003 & Missing Acceptable Use Policy & The absence of a formal AUP creates ambiguity regarding secure practices and limits the organization's ability to enforce security standards. & High \\
\addlinespace
RISK-004 & Insufficient Security Training & Security awareness is not consistently reinforced across the organization, making employees more susceptible to social engineering attacks. & High \\
\bottomrule
\end{tabular}
\end{table}

% --- Section 6: Recommendations ---
\section{Recommendations}

Based on the consolidated risk assessment, we recommend the following actions, prioritized by severity, to mitigate the identified risks and strengthen the security posture of \orgname.

\begin{enumerate}
    \item \textbf{Remediate Outdated Nginx Server (RISK-001):}
    \begin{itemize}
        \item \textbf{Action:} Immediately plan and execute an upgrade of the Nginx server on \targetip from version 1.18.0 to the latest stable version recommended by the vendor.
        \item \textbf{Justification:} Patching this server will close known security vulnerabilities, preventing potential exploitation that could lead to a system compromise or data breach.
    \end{itemize}
    \vspace{1em}
    \item \textbf{Implement Mandatory Endpoint MFA (RISK-002):}
    \begin{itemize}
        \item \textbf{Action:} Procure and deploy a Multi-Factor Authentication solution for all employee computer and remote access logins. Enforce its use as a mandatory policy.
        \item \textbf{Justification:} This adds a critical layer of security that protects against unauthorized access even if user passwords are compromised.
    \end{itemize}
    \vspace{1em}
    \item \textbf{Develop and Enforce an Acceptable Use Policy (RISK-003):}
    \begin{itemize}
        \item \textbf{Action:} Draft, approve, and implement a comprehensive AUP that clearly defines the rules for using company IT assets, data, and internet access. All employees must read and acknowledge the policy.
        \item \textbf{Justification:} An AUP establishes a baseline for secure behavior, reduces insider risk, and provides a framework for disciplinary action in case of policy violations.
    \end{itemize}
    \vspace{1em}
    \item \textbf{Establish Annual Security Awareness Training (RISK-004):}
    \begin{itemize}
        \item \textbf{Action:} Institute a mandatory, annual security awareness training program for all employees. The training should cover current threats such as phishing, malware, and social engineering.
        \item \textbf{Justification:} Regular training reinforces security concepts, keeps employees informed about new threats, and reduces the likelihood of human error leading to a security incident.
    \end{itemize}
\end{enumerate}

\end{document}
```