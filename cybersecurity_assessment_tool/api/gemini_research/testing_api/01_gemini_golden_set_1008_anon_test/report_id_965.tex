```latex
\documentclass[12pt]{article}

% Preamble: Required Packages
\usepackage[margin=1in]{geometry}
\usepackage{pifont} % For checkmarks and crosses
\usepackage{booktabs} % For professional-looking tables
\usepackage{hyperref} % For clickable links
\usepackage{url} % For URL formatting
\usepackage{seqsplit} % For splitting long strings to prevent overflow
\usepackage{xcolor} % For colors in text
\usepackage{graphicx} % For logo if needed
\usepackage{fancyhdr} % For headers and footers

% Document Metadata
\title{Cybersecurity Posture and Risk Assessment Report}
\author{Cybersecurity Analysis Division}
\date{\today}

% Header and Footer Configuration
\pagestyle{fancy}
\fancyhf{} % Clear all header and footer fields
\rhead{\textbf{[Organization Name]}}
\lhead{Confidential Cybersecurity Report}
\cfoot{\thepage}

\begin{document}

\maketitle
\thispagestyle{empty}
\newpage

\tableofcontents
\newpage

% --- 1. Executive Summary ---
\section{Executive Summary}

This report provides a comprehensive cybersecurity assessment for \textbf{[Organization Name]}, synthesizing data from a network vulnerability scan, a security controls questionnaire, and a review of pre-existing risks.

The analysis revealed a \textbf{CRITICAL} risk posture primarily driven by the direct exposure of a Remote Desktop Protocol (RDP) service on port 3389 to the public internet. This technical vulnerability is significantly amplified by critical gaps in administrative controls. Specifically, the absence of mandatory Multi-Factor Authentication (MFA) for computer logins creates a direct path for attackers with compromised credentials to gain internal network access.

Furthermore, foundational security practices are missing, including a formal Acceptable Use Policy and mandatory security awareness training for new employees. These deficiencies cultivate a high-risk environment where employees are more susceptible to social engineering attacks, such as phishing, which could lead to the initial credential compromise needed to exploit the exposed RDP service.

Immediate remediation is required to address the exposed RDP service. Following this, a strategic effort must be undertaken to implement endpoint MFA and establish foundational security policies and training programs to reduce the overall attack surface and improve the organization's security maturity.

% --- 2. Organizational Information ---
\section{Organizational Information}

This section details the organizational context for this assessment. The information provided is based on the data supplied for the analysis.

\begin{itemize}
    \item \textbf{Organization Name:} \textbf{[Organization Name]}
    \item \textbf{Primary Domain:} \texttt{[Domain]}
    \item \textbf{External IP Scanned:} \texttt{[Client IP]}
\end{itemize}

% --- 3. Security Control Review ---
\section{Security Control Review}

The following table summarizes the organization's responses to a security controls questionnaire. Items marked with \textcolor{red}{\ding{55}} represent significant gaps in the current security posture and require immediate attention.

\begin{table}[h!]
\centering
\caption{Security Controls Questionnaire Analysis}
\begin{tabular}{p{0.7\linewidth} c}
\toprule
\textbf{Control Question} & \textbf{Status} \\
\midrule
Do you require MFA to access email? & \textcolor{green!80!black}{\ding{51}} \\
\rowcolor{red!10}
Do you require MFA to log into computers? & \textcolor{red}{\ding{55}} \\
Do you require MFA to access sensitive data systems? & \textcolor{green!80!black}{\ding{51}} \\
\rowcolor{red!10}
Does your organization have an employee acceptable use policy? & \textcolor{red}{\ding{55}} \\
\rowcolor{red!10}
Does your organization do security awareness training for new employees? & \textcolor{red}{\ding{55}} \\
Does your organization do security awareness training for all employees at least once per year? & \textcolor{green!80!black}{\ding{51}} \\
\bottomrule
\end{tabular}
\end{table}

\subsection*{Analysis of Gaps}
\begin{itemize}
    \item \textbf{No MFA for Computer Logins:} This is a critical weakness. It allows an attacker with a single valid password to gain access to an endpoint, bypassing a crucial security layer.
    \item \textbf{No Acceptable Use Policy (AUP):} An AUP is a foundational document that sets expectations for employee behavior regarding company assets and data. Its absence can lead to inconsistent security practices and a lack of recourse for policy violations.
    \item \textbf{No New Employee Security Training:} Failing to train new hires on security best practices from day one leaves a window of high vulnerability. New employees are often prime targets for social engineering attacks.
\end{itemize}

% --- 4. Technical Scan Results ---
\section{Technical Scan Results}

An external network scan was performed on the target IP address to identify open ports and exposed services. The findings are detailed below.

\begin{table}[h!]
\centering
\caption{Open Ports Detected on Target: \texttt{[Target IP]}}
\begin{tabular}{c c l l}
\toprule
\textbf{Port} & \textbf{State} & \textbf{Service Name} & \textbf{Analysis} \\
\midrule
3389/tcp & Open & \texttt{ms-wbt-server} & Critical Risk. This is the port for Microsoft \\
& & & Remote Desktop Protocol (RDP). Public exposure \\
& & & is a common vector for ransomware attacks. \\
\bottomrule
\end{tabular}
\end{table}

\subsection*{Analysis of Findings}
The scan confirms that the Remote Desktop Protocol (RDP) service is directly accessible from the public internet on the host \texttt{[Target IP]}. This configuration is extremely dangerous and is actively targeted by threat actors for initial access into corporate networks. This finding validates the pre-existing risk documented in the organization's risk register.

% --- 5. Consolidated Risk Assessment ---
\section{Consolidated Risk Assessment}

This section correlates the findings from the security control review, the technical scan, and pre-existing risk data to provide a holistic view of the primary risks facing the organization.

\begin{table}[h!]
\centering
\caption{Summary of Key Identified Risks}
\begin{tabular}{p{0.25\linewidth} p{0.1\linewidth} p{0.55\linewidth}}
\toprule
\textbf{Risk Title} & \textbf{Severity} & \textbf{Description} \\
\midrule
\textbf{Public RDP Exposure} & \textbf{Critical} & The RDP service on \texttt{[Target IP]} is exposed to the internet, allowing attackers to attempt brute-force or credential stuffing attacks. This is a direct gateway into the internal network. \\
\addlinespace
\textbf{Lack of Endpoint MFA} & \textbf{High} & The absence of MFA on computer logins means that a single compromised password is all an attacker needs to gain access. This directly exacerbates the risk of the exposed RDP service. \\
\addlinespace
\textbf{Foundational Policy \& Training Gaps} & \textbf{High} & The lack of an AUP and security training for new hires creates a weak security culture. This increases the likelihood of human error, such as falling for phishing attacks, which is a primary source of credential theft. \\
\bottomrule
\end{tabular}
\end{table}

% --- 6. Recommendations ---
\section{Recommendations}

Based on the consolidated risk assessment, the following actions are recommended to mitigate the identified vulnerabilities and improve the overall security posture of \textbf{[Organization Name]}.

\subsection{Immediate Actions (0-7 Days)}
\begin{enumerate}
    \item \textbf{Remediate RDP Exposure:} Immediately close port 3389 to the public internet on the perimeter firewall for the host at \texttt{[Target IP]}. All remote access should be facilitated through a secure, MFA-enabled Virtual Private Network (VPN).
\end{enumerate}

\subsection{High-Priority Actions (1-3 Months)}
\begin{enumerate}
    \setcounter{enumi}{1} % Continue numbering
    \item \textbf{Implement Endpoint MFA:} Procure and deploy an MFA solution for all computer logins (both local and remote). This is the single most effective control to prevent unauthorized access from compromised credentials.
    \item \textbf{Develop and Implement an Acceptable Use Policy (AUP):} Draft a formal AUP that clearly defines the rules and expectations for all employees when using company technology and data. Require all employees to read and acknowledge the policy.
\end{enumerate}

\subsection{Medium-Priority Actions (3-6 Months)}
\begin{enumerate}
    \setcounter{enumi}{3} % Continue numbering
    \item \textbf{Establish New Hire Security Training:} Integrate a mandatory security awareness training module into the new employee onboarding process. This training should cover key topics such as phishing, password hygiene, and the new AUP.
\end{enumerate}

\end{document}
```