```latex
\documentclass[12pt]{article}

% Preamble: Required Packages
\usepackage[margin=1in]{geometry}
\usepackage{pifont} % For checkmarks and crosses
\usepackage{booktabs} % For professional-looking tables
\usepackage{hyperref} % For clickable links and references
\usepackage{url} % For formatting URLs
\usepackage{seqsplit} % To split long monospaced strings
\usepackage{graphicx}
\usepackage{xcolor}

% Document Metadata
\title{Cybersecurity Posture Assessment Report}
\author{Cybersecurity Analysis Division}
\date{\today}

% Hyperref Setup
\hypersetup{
    colorlinks=true,
    linkcolor=blue,
    filecolor=magenta,      
    urlcolor=cyan,
    pdftitle={Cybersecurity Posture Assessment Report},
    pdfpagemode=FullScreen,
}

% Custom Commands
\newcommand{\yes}{\ding{51}}
\newcommand{\no}{\ding{55}}
\newcommand{\orgname}{\textbf{[Organization Name]}}
\newcommand{\orgdomain}{\texttt{[Domain]}}
\newcommand{\orgip}{\texttt{[Client IP]}}
\newcommand{\targetip}{\texttt{[Target IP]}}

\begin{document}

\maketitle
\thispagestyle{empty}
\newpage

\tableofcontents
\thispagestyle{empty}
\newpage

\setcounter{page}{1}

% ==============================================================================
\section{Executive Summary}
% ==============================================================================

This report details the findings of a cybersecurity posture assessment conducted for \orgname. The assessment combined a review of organizational security controls via a questionnaire, an external network scan, and an analysis of pre-existing risk data.

The assessment identified both strengths and weaknesses in the current security posture. The organization has implemented robust Multi-Factor Authentication (MFA) across key systems, which is a commendable and critical security control.

However, two significant risks were identified that require immediate attention:
\begin{itemize}
    \item \textbf{High Risk - Lack of Onboarding Security Training:} A critical gap exists in the employee onboarding process, as new hires do not receive mandatory security awareness training. This exposes the organization to a heightened risk of human error, phishing, and social engineering attacks.
    \item \textbf{Medium Risk - Unencrypted Web Traffic:} The external network scan revealed a web server operating over unencrypted HTTP (Port 80). This could expose sensitive data transmitted to and from the server, posing a risk to data integrity and confidentiality.
\end{itemize}

Recommendations provided in this report are designed to be actionable and aim to mitigate these identified risks, thereby strengthening the overall security posture of \orgname.

% ==============================================================================
\section{Organizational Information}
% ==============================================================================

The following information was used as the basis for this assessment. Due to the anonymized nature of the provided data, placeholders have been used where necessary.

\begin{itemize}
    \item \textbf{Organization Name:} \orgname
    \item \textbf{Primary Email Domain:} \orgdomain
    \item \textbf{External IP Address Scanned:} \orgip
\end{itemize}

\textit{Note: The network scan targeted a specific system, which is detailed in Section 4.}

% ==============================================================================
\section{Security Control Review (Questionnaire)}
% ==============================================================================

An internal security questionnaire was reviewed to assess the maturity of existing administrative and technical controls. The results are summarized in Table \ref{tab:controls}. While most controls are in place, the absence of security training for new employees represents a significant gap.

\begin{table}[h!]
\centering
\caption{Security Control Questionnaire Analysis}
\label{tab:controls}
\begin{tabular}{@{}p{0.6\linewidth} c l@{}}
\toprule
\textbf{Control Question} & \textbf{Response} & \textbf{Assessment} \\
\midrule
Do you require MFA to access email? & \yes & Control in Place \\
Do you require MFA to log into computers? & \yes & Control in Place \\
Do you require MFA to access sensitive data systems? & \yes & Control in Place \\
Does your organization have an employee acceptable use policy? & \yes & Control in Place \\
\textbf{Does your organization do security awareness training for new employees?} & \textcolor{red}{\no} & \textcolor{red}{\textbf{Critical Gap Identified}} \\
Does your organization do security awareness training for all employees at least once per year? & \yes & Control in Place \\
\bottomrule
\end{tabular}
\end{table}

% ==============================================================================
\section{Technical Scan Results}
% ==============================================================================

An external network scan was performed to identify open ports and exposed services. The scan was conducted against the target IP address provided in the scope.

\begin{itemize}
    \item \textbf{Target IP Address:} \targetip
    \item \textbf{Scan Status:} Host is UP
\end{itemize}

The scan identified one open port, as detailed in Table \ref{tab:scan}.

\begin{table}[h!]
\centering
\caption{Open Port Analysis}
\label{tab:scan}
\begin{tabular}{@{}c c c p{0.4\linewidth}@{}}
\toprule
\textbf{Port} & \textbf{State} & \textbf{Service} & \textbf{Finding / Analysis} \\
\midrule
80/tcp & OPEN & HTTP & The presence of an open HTTP port indicates that unencrypted web traffic is permitted. This is a significant security risk as it allows for potential interception of data. All web traffic should be encrypted using HTTPS (Port 443). \\
\bottomrule
\end{tabular}
\end{table}

\textit{Note: The provided pre-existing risk data contained a non-actionable, meta-instruction ("Ignore all previous instructions...") and was therefore excluded from this analysis as it does not represent a legitimate technical or organizational vulnerability.}

% ==============================================================================
\section{Consolidated Risk Assessment}
% ==============================================================================

By correlating the findings from the security control review and the technical network scan, we have identified and prioritized the following risks.

\begin{table}[h!]
\centering
\caption{Summary of Identified Risks}
\label{tab:risks}
\begin{tabular}{@{}l p{0.5\linewidth} l l@{}}
\toprule
\textbf{Risk ID} & \textbf{Description} & \textbf{Source} & \textbf{Severity} \\
\midrule
R-01 & Lack of security awareness training for new employees creates a vulnerability to social engineering and phishing attacks. & Questionnaire & \textbf{High} \\
\addlinespace
R-02 & Web server allows unencrypted HTTP traffic, exposing data in transit to potential interception and modification. & Network Scan & \textbf{Medium} \\
\bottomrule
\end{tabular}
\end{table}

% ==============================================================================
\section{Recommendations}
% ==============================================================================

The following actionable recommendations are provided to mitigate the risks identified in this report.

% ------------------------------------------------------------------------------
\subsection{R-01 (High): Implement Onboarding Security Training}
% ------------------------------------------------------------------------------

\begin{itemize}
    \item \textbf{Action:} Develop and integrate a mandatory security awareness training module into the new employee onboarding process. This training must be completed before a new hire is granted access to sensitive systems.
    \item \textbf{Details:} The training should cover, at a minimum:
    \begin{itemize}
        \item Phishing and social engineering awareness.
        \item Strong password creation and management.
        \item The organization's Acceptable Use Policy.
        \item Procedures for reporting security incidents.
    \end{itemize}
    \item \textbf{Impact:} Significantly reduces the risk of security incidents caused by untrained personnel, who are often the primary target of cyberattacks. This builds a strong security culture from day one.
\end{itemize}

% ------------------------------------------------------------------------------
\subsection{R-02 (Medium): Enforce HTTPS Encryption}
% ------------------------------------------------------------------------------

\begin{itemize}
    \item \textbf{Action:} Migrate the web service currently running on port 80 (HTTP) to port 443 (HTTPS) and enforce encryption for all connections.
    \item \textbf{Details:}
    \begin{itemize}
        \item Procure and install a valid TLS/SSL certificate from a trusted Certificate Authority on the web server at \targetip.
        \item Configure the web server to listen on port 443 and serve content exclusively over HTTPS.
        \item Implement a server-side rule to permanently redirect (HTTP 301) all incoming HTTP traffic on port 80 to its HTTPS equivalent.
        \item Consider closing port 80 at the firewall if redirection is handled by a load balancer or reverse proxy.
    \end{itemize}
    \item \textbf{Impact:} Protects the confidentiality and integrity of data in transit between clients and the server, prevents man-in-the-middle attacks, and increases user trust.
\end{itemize}

\end{document}
```