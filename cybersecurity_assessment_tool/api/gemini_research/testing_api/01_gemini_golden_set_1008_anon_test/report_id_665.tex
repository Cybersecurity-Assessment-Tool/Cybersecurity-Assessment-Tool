```latex
\documentclass[12pt]{article}

% --- PACKAGES ---
\usepackage[utf8]{inputenc}
\usepackage[a4paper, margin=1in]{geometry}
\usepackage{pifont} % Required for \ding
\usepackage{booktabs} % For professional-looking tables
\usepackage{hyperref} % For clickable links and references
\usepackage{url}      % For properly formatting URLs
\usepackage{seqsplit} % For splitting long text strings like hashes or IPs

% --- DOCUMENT METADATA & SETUP ---
\hypersetup{
    colorlinks=true,
    linkcolor=black,
    urlcolor=blue,
}

\title{Cybersecurity Posture Assessment Report}
\author{Cybersecurity Analysis Division}
\date{\today}

% --- DOCUMENT START ---
\begin{document}

\maketitle

\section*{1. Executive Summary}

This report details the findings of a cybersecurity assessment conducted for \textbf{[Organization Name]}. The assessment incorporated an analysis of organizational security controls, a technical network scan of the external perimeter, and a review of pre-existing risk documentation.

The assessment identified critical vulnerabilities that place the organization at a high risk of compromise. The primary findings include:
\begin{itemize}
    \item \textbf{Critical Network Exposure:} A publicly exposed Remote Desktop Protocol (RDP) service was confirmed on the external IP address \seqsplit{\texttt{[Client IP]}}. This is a common and high-impact vector for ransomware attacks.
    \item \textbf{Significant Authentication Gaps:} Multi-Factor Authentication (MFA) is not enforced for accessing email or for computer logins. This lack of a fundamental security control makes credential-based attacks significantly more likely to succeed.
    \item \textbf{Insufficient Employee Onboarding:} New employees do not receive security awareness training, creating a persistent vulnerability to social engineering and phishing attacks.
\end{itemize}

These findings, when correlated, paint a picture of an organization highly susceptible to external attack. Immediate remediation is required to lower the risk profile to an acceptable level. Detailed recommendations are provided in Section 6.

\section*{2. Organizational Information}

The following information was used as the basis for this assessment.
\begin{itemize}
    \item \textbf{Organization Name:} \textbf{[Organization Name]}
    \item \textbf{Primary Domain:} \texttt{[Domain]}
    \item \textbf{External IP Scanned:} \seqsplit{\texttt{[Client IP]}}
\end{itemize}

\section*{3. Security Control Review}

A review of the organization's security controls was conducted via a standardized questionnaire. The results highlight significant gaps in identity and access management and employee security training. A checkmark (\ding{51}) indicates a positive control is in place, while a cross (\ding{55}) indicates a control gap.

\begin{table}[h!]
\centering
\begin{tabular}{p{0.75\textwidth} c}
\toprule
\textbf{Control Question} & \textbf{Status} \\
\midrule
Do you require MFA to access email? & \ding{55} \\
Do you require MFA to log into computers? & \ding{55} \\
Do you require MFA to access sensitive data systems? & \ding{51} \\
Does your organization have an employee acceptable use policy? & \ding{51} \\
Does your organization do security awareness training for new employees? & \ding{55} \\
Does your organization do security awareness training for all employees at least once per year? & \ding{51} \\
\bottomrule
\end{tabular}
\caption{Organizational Security Control Status}
\end{table}

\paragraph{Analysis:} The review reveals critical deficiencies. The absence of MFA for email and computer logins drastically increases the risk of account takeover and lateral movement following a credential compromise. Furthermore, the failure to train new employees on security best practices leaves the organization vulnerable from day one of their employment, undermining the effectiveness of the annual training program.

\section*{4. Technical Scan Results}

An external network scan was performed to identify exposed services on the organization's perimeter. The scan was targeted at the IP address provided.

\subsection*{Scan Details}
\begin{itemize}
    \item \textbf{Target IP:} \texttt{[Target IP]}
    \item \textbf{Scan Type:} TCP Port Scan
\end{itemize}

The scan confirmed the following service is open and accessible from the public internet:

\begin{table}[h!]
\centering
\begin{tabular}{llll}
\toprule
\textbf{Port} & \textbf{State} & \textbf{Service Name} & \textbf{Description} \\
\midrule
3389/tcp & open & ms-wbt-server & Microsoft Remote Desktop Protocol (RDP) \\
\bottomrule
\end{tabular}
\caption{Open Ports Detected on Target IP}
\end{table}

\paragraph{Analysis:} The presence of an open RDP port (3389) is a \textbf{critical finding}. This service allows for direct administrative access to a Windows system. It is a primary target for threat actors, who use brute-force password attacks, credential stuffing, and exploitation of RDP vulnerabilities to gain initial access to a network. This finding directly confirms the pre-existing risk documented in Input 3 and elevates its urgency.

\section*{5. Consolidated Risk Assessment}

By correlating the security control gaps, technical findings, and existing risk data, we have compiled a summary of the top risks facing the organization.

\begin{table}[h!]
\centering
\begin{tabular}{p{0.25\textwidth} p{0.15\textwidth} p{0.5\textwidth}}
\toprule
\textbf{Risk Name} & \textbf{Severity} & \textbf{Description} \\
\midrule
\textbf{Public RDP Exposure} & \textbf{Critical (9.0)} & The technical scan confirmed that Port 3389 is open to the internet, allowing direct access to the Remote Desktop Protocol. This creates a direct pathway for attackers into the internal network. \\
\addlinespace
\textbf{Lack of MFA Enforcement} & \textbf{Critical} & MFA is not required for email or computer access. A single compromised password could lead to a full breach of email data or allow an attacker who gains RDP access to move laterally. \\
\addlinespace
\textbf{Insufficient Security Training} & \textbf{High} & New employees do not receive security awareness training, making them highly susceptible to phishing attacks designed to steal credentials, which could then be used against the exposed RDP service. \\
\bottomrule
\end{tabular}
\caption{Summary of Key Risks}
\end{table}

\section*{6. Recommendations}

The following actions are recommended to mitigate the identified risks. They are prioritized based on urgency and potential impact.

\subsection*{Immediate Actions (Critical Priority)}
\begin{enumerate}
    \item \textbf{Mitigate RDP Exposure:} Immediately block all inbound traffic to TCP port 3389 on the external firewall for the IP address \texttt{[Target IP]}. If remote access is business-critical, restrict access exclusively to known, trusted source IP addresses as a temporary measure until a long-term solution is implemented.
\end{enumerate}

\subsection*{High Priority Actions}
\begin{enumerate}
    \item \textbf{Implement Comprehensive MFA:} Deploy and enforce MFA across all user accounts for both email access (e.g., Office 365, Google Workspace) and computer logins (e.g., Windows Hello, Duo). This is the single most effective control to prevent credential-based attacks.
    \item \textbf{Establish New Hire Security Training:} Develop and implement a mandatory security awareness training module as part of the new employee onboarding process. This training should cover phishing, acceptable use, and password security at a minimum.
\end{enumerate}

\subsection*{Strategic Recommendations}
\begin{enumerate}
    \item \textbf{Adopt a Secure Remote Access Solution:} Decommission direct RDP access permanently. Replace it with a modern, secure remote access solution such as a Virtual Private Network (VPN) with MFA or a Zero Trust Network Access (ZTNA) platform.
    \item \textbf{Implement Continuous Vulnerability Management:} Establish a program for regular, automated vulnerability scanning of both external and internal assets to proactively identify and remediate weaknesses before they can be exploited.
\end{enumerate}

\end{document}
```