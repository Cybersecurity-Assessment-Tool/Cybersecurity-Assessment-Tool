```latex
\documentclass[12pt]{article}

% Preamble: Required Packages
\usepackage[margin=1in]{geometry} % For setting page margins
\usepackage{pifont}                 % For checkmarks and crosses (\ding)
\usepackage{booktabs}               % For professional-looking tables
\usepackage[hidelinks]{hyperref}    % For clickable links without boxes
\usepackage{url}                    % For formatting URLs
\usepackage{seqsplit}               % For splitting long strings without spaces
\usepackage[table]{xcolor}          % For coloring table rows

% Document Information
\title{Cybersecurity Posture Assessment Report}
\author{Cybersecurity Analysis Division}
\date{\today}

\begin{document}

\maketitle

\begin{abstract}
This report provides a comprehensive analysis of the cybersecurity posture for \textbf{[Organization Name]}. The assessment is based on a synthesis of external network scan data, a review of internal security controls via a questionnaire, and an evaluation of previously identified risks. The analysis reveals a minimal external attack surface from the provided scan, but identifies critical gaps in internal security policies, particularly concerning Multi-Factor Authentication (MFA) and employee security training. Actionable recommendations are provided to mitigate the identified risks.
\end{abstract}

\section*{1. Overview and Executive Summary}

The objective of this assessment was to evaluate the security posture of \textbf{[Organization Name]} by correlating technical scan data with organizational security practices.

\paragraph{Key Findings:}
\begin{itemize}
    \item \textbf{Positive Finding:} The external network scan of the target IP address, \texttt{[Target IP]}, did not identify any open ports. This indicates a strong network perimeter configuration for the scanned asset, significantly reducing the external attack surface.
    \item \textbf{Critical Risk:} The organization has not implemented Multi-Factor Authentication (MFA) for critical access points, including employee email and computer logins. This represents a significant vulnerability to credential theft and unauthorized access.
    \item \textbf{High Risk:} Core security policies are missing. The absence of an employee Acceptable Use Policy (AUP) and a mandatory security awareness training program for new hires creates a high-risk environment susceptible to insider threats and social engineering.
    \item \textbf{Risk Status Update:} A previously documented risk, "Unencrypted Web Server," related to an open Port 80, appears to be remediated, as the recent scan shows this port is now closed. This should be internally verified.
\end{itemize}

\paragraph{Conclusion:} While the external network posture appears robust for the scanned target, the primary risks to the organization are internal and procedural. Immediate focus should be placed on implementing MFA and establishing foundational security policies to mitigate the most severe threats.

\section*{2. Organizational Information}

This section details the information provided about the organization.
\begin{itemize}
    \item \textbf{Organization Name:} \textbf{[Organization Name]}
    \item \textbf{Email Domain:} \texttt{[Domain]}
    \item \textbf{External IP Address:} \texttt{[Client IP]}
\end{itemize}

\section*{3. Security Control Review}

The following table summarizes the organization's responses to the security controls questionnaire. A checkmark (\ding{51}) indicates a positive control is in place, while a cross (\ding{55}) indicates a control gap that introduces risk.

\begin{table}[h!]
\centering
\caption{Security Controls Questionnaire Analysis}
\label{tab:controls}
\begin{tabular}{p{0.7\linewidth} c c}
\toprule
\textbf{Control Question} & \textbf{Response} & \textbf{Status} \\
\midrule
Do you require MFA to access email? & No & \ding{55} \\
Do you require MFA to log into computers? & No & \ding{55} \\
Do you require MFA to access sensitive data systems? & Yes & \ding{51} \\
Does your organization have an employee acceptable use policy? & No & \ding{55} \\
Does your organization do security awareness training for new employees? & No & \ding{55} \\
Does your organization do security awareness training for all employees at least once per year? & Yes & \ding{51} \\
\bottomrule
\end{tabular}
\end{table}

The identified gaps in MFA and employee policies are significant and are addressed in the Risk Assessment section of this report.

\section*{4. Technical Scan Results}

An external network scan was performed to identify accessible services and potential vulnerabilities.

\begin{itemize}
    \item \textbf{Scan Target:} \texttt{[Target IP]}
    \item \textbf{Scanner Used:} Nmap
\end{itemize}

\begin{table}[h!]
\centering
\caption{Network Scan Port Summary}
\label{tab:scan}
\begin{tabular}{l l l}
\toprule
\textbf{Port} & \textbf{State} & \textbf{Service/Product/Version} \\
\midrule
80/tcp & closed & Not Applicable \\
\bottomrule
\end{tabular}
\end{table}

\paragraph{Analysis:} The scan results are positive, indicating a well-configured firewall or host. No open ports were discovered, which severely limits the avenues for an external attacker to compromise this system directly. The scan specifically confirmed that port 80 (HTTP) is closed, which contradicts a pre-existing risk finding and suggests recent remediation efforts have been successful.

\section*{5. Consolidated Risk Assessment}

This section synthesizes findings from the security control review, technical scan, and pre-existing risk data into a prioritized list.

\definecolor{critical}{HTML}{990000}
\definecolor{high}{HTML}{D14302}
\definecolor{info}{HTML}{1E90FF}

\begin{table}[h!]
\centering
\caption{Prioritized Risk Register}
\label{tab:risks}
\begin{tabular}{p{0.25\linewidth} p{0.5\linewidth} p{0.15\linewidth}}
\toprule
\textbf{Risk Name} & \textbf{Description} & \textbf{Severity} \\
\midrule
\rowcolor{critical!25}
\textbf{Lack of Foundational MFA} & The absence of MFA for email and computer logins exposes the organization to a high likelihood of account compromise via phishing or credential stuffing attacks. & \textbf{Critical} \\
\addlinespace
\rowcolor{high!25}
\textbf{Inadequate Employee Security Policies} & The lack of an Acceptable Use Policy and security training for new hires means employees are not equipped with the knowledge to handle company assets securely, increasing the risk of data leakage and policy violations. & \textbf{High} \\
\addlinespace
\rowcolor{info!25}
\textbf{Potentially Remediated Web Server Risk} & A previously identified risk of an "Unencrypted Web Server" (CVSS 5.0) on Port 80 appears to be resolved, as the port is now closed. This requires internal confirmation. & \textbf{Informational} \\
\bottomrule
\end{tabular}
\end{table}

\section*{6. Recommendations}

Based on the analysis, we recommend the following actions, prioritized by severity:

\begin{enumerate}
    \item \textbf{[Immediate Priority] Implement Multi-Factor Authentication (MFA):}
    \begin{itemize}
        \item Immediately enable MFA for all user accounts on the primary email system (e.g., Microsoft 365, Google Workspace).
        \item Develop a plan to roll out MFA for all employee computer logins. This is a critical defense against unauthorized access to corporate devices and data.
    \end{itemize}
    
    \item \textbf{[High Priority] Establish Foundational Security Policies:}
    \begin{itemize}
        \item Develop and implement a formal Acceptable Use Policy (AUP) that all employees must read and sign. This policy should clearly define the rules for using company IT assets.
        \item Create a mandatory security awareness training module for all new hires as part of the onboarding process. This training should cover key topics such as phishing, password security, and data handling.
    \end{itemize}
    
    \item \textbf{[Verification] Confirm Remediation of Port 80:}
    \begin{itemize}
        \item Internally verify that the closure of port 80 on \texttt{[Target IP]} was intentional and is in line with business requirements.
        \item If a web server is required at this address in the future, ensure it is deployed securely using HTTPS on port 443 with a valid SSL/TLS certificate.
        \item Formally update the internal risk register to mark the "Unencrypted Web Server" risk as "Remediated."
    \end{itemize}
\end{enumerate}

\end{document}
```