```latex
\documentclass[12pt]{article}

% Preamble: Required Packages
\usepackage[margin=1in]{geometry}
\usepackage{pifont} % For \ding{51} (checkmark) and \ding{55} (cross)
\usepackage{booktabs} % For professional-looking tables (\toprule, \midrule, \bottomrule)
\usepackage{hyperref} % For creating hyperlinks
\usepackage{url}      % For formatting URLs
\usepackage{seqsplit} % To split long, unbreakable strings like hashes or tokens
\usepackage{graphicx} % For logo if needed
\usepackage{xcolor}   % For color definitions

% Hyperlink Setup
\hypersetup{
    colorlinks=true,
    linkcolor=black,
    filecolor=magenta,
    urlcolor=blue,
    pdftitle={Cybersecurity Posture Assessment Report},
    pdfpagemode=FullScreen,
}

% Define colors for severity
\definecolor{criticalred}{HTML}{D7263D}
\definecolor{highorange}{HTML}{F49D40}
\definecolor{mediumyellow}{HTML}{F4D440}

% --- Document Start ---
\begin{document}

% --- Title Page ---
\begin{titlepage}
    \centering
    \vspace*{1cm}
    \Huge\textbf{Cybersecurity Posture Assessment Report}
    \vspace{1.5cm}
    \Large
    \textbf{Prepared for:} \\
    \vspace{0.5cm}
    \textbf{[Organization Name]}
    \vspace{2cm}
    \large
    \textbf{Date of Report:} \today \\
    \textbf{Date of Scan:} November 22, 2025
    \vfill
    \large
    \textbf{Generated By:} \\
    Cybersecurity Analyst
\end{titlepage}

\tableofcontents
\newpage

% --- Section 1: Executive Summary ---
\section{Executive Summary}
This report details the findings of a cybersecurity posture assessment conducted for \textbf{[Organization Name]}. The assessment combined an external network scan, a review of existing risks, and an analysis of organizational security controls via a questionnaire.

The overall security posture is considered to be at a \textbf{high risk level}. Several critical deficiencies were identified that significantly increase the organization's exposure to common cyber threats such as ransomware, data breaches, and unauthorized access.

Key findings include:
\begin{itemize}
    \item \textbf{Critical Lack of Multi-Factor Authentication (MFA):} MFA is not enforced for accessing email, logging into computers, or accessing sensitive data systems. This represents a critical vulnerability, as a single compromised password could lead to a widespread system breach.
    \item \textbf{Outdated Public-Facing Software:} The external-facing web server is running Nginx version 1.18.0, which is outdated and no longer receives security patches. This exposes the organization to numerous known vulnerabilities that could be exploited by attackers.
    \item \textbf{Deficient Security Policies and Training:} The organization lacks a formal employee acceptable use policy and does not conduct annual security awareness training for all staff. These gaps in administrative controls weaken the human element of security, making the organization more susceptible to social engineering and phishing attacks.
\end{itemize}

Immediate and decisive action is required to remediate these issues. Prioritized recommendations are provided in Section \ref{sec:recommendations} to guide mitigation efforts.

% --- Section 2: Organizational Information ---
\section{Organizational Information}
This section provides a summary of the organizational details relevant to this assessment.
\begin{itemize}
    \item \textbf{Organization Name:} \textbf{[Organization Name]}
    \item \textbf{Primary Email Domain:} \texttt{[Domain]}
    \item \textbf{Scanned External IP Address:} \texttt{[Client IP]}
\end{itemize}

% --- Section 3: Security Control Review ---
\section{Security Control Review (Questionnaire Analysis)}
The following table summarizes the organization's responses to a security controls questionnaire. A red cross (\ding{55}) indicates a negative response, highlighting a potential gap in security posture.

\begin{table}[h!]
\centering
\caption{Security Controls Questionnaire Results}
\begin{tabular}{p{0.75\linewidth} c}
\toprule
\textbf{Control Question} & \textbf{Response} \\
\midrule
Do you require MFA to access email? & \textcolor{criticalred}{\ding{55}} \\
Do you require MFA to log into computers? & \textcolor{criticalred}{\ding{55}} \\
Do you require MFA to access sensitive data systems? & \textcolor{criticalred}{\ding{55}} \\
Does your organization have an employee acceptable use policy? & \textcolor{criticalred}{\ding{55}} \\
Does your organization do security awareness training for new employees? & \textcolor{green}{\ding{51}} \\
Does your organization do security awareness training for all employees at least once per year? & \textcolor{highorange}{\ding{55}} \\
\bottomrule
\end{tabular}
\end{table}

\paragraph{Analysis:} The complete absence of MFA for critical access points (email, computers, sensitive data) is a severe security weakness. The lack of an acceptable use policy and annual security training further elevates the risk, as employees may be unaware of their security responsibilities.

% --- Section 4: Technical Scan Results ---
\section{Technical Scan Results}
An external network scan was performed on the organization's public-facing infrastructure.

\subsection{Scan Metadata}
\begin{itemize}
    \item \textbf{Target IP:} \texttt{[Target IP]}
    \item \textbf{Scan Date:} 2025-11-22T10:00:00Z
\end{itemize}

\subsection{Open Ports and Services}
The following table details the open ports and services discovered during the scan.

\begin{table}[h!]
\centering
\caption{Discovered Open Ports}
\begin{tabular}{l l l l l}
\toprule
\textbf{Port} & \textbf{State} & \textbf{Service} & \textbf{Product} & \textbf{Version} \\
\midrule
443/tcp & open & https & nginx & 1.18.0 \\
\bottomrule
\end{tabular}
\end{table}

\paragraph{Analysis:} The scan identified a web server running \textbf{Nginx version 1.18.0}. This version was released in April 2020 and reached its end-of-life for security patches in May 2022. Running outdated software, especially on a public-facing service like a web server, presents a high risk. Numerous vulnerabilities have been discovered in Nginx since this version was released, which could be exploited by attackers to achieve remote code execution, denial of service, or information disclosure.

% --- Section 5: Risk Assessment ---
\section{Risk Assessment}
This section synthesizes the findings from the questionnaire and technical scan into a prioritized list of identified risks. No pre-existing risks were provided for this assessment.

\begin{table}[h!]
\centering
\caption{Summary of Identified Risks}
\begin{tabular}{p{0.1\linewidth} p{0.25\linewidth} p{0.45\linewidth} p{0.1\linewidth}}
\toprule
\textbf{ID} & \textbf{Risk Name} & \textbf{Description} & \textbf{Severity} \\
\midrule
RISK-001 & Lack of Multi-Factor Authentication (MFA) & The absence of MFA on all critical systems allows an attacker with stolen credentials to gain unauthorized access to email, workstations, and sensitive data. & \textcolor{criticalred}{Critical} \\
\addlinespace
RISK-002 & Outdated Web Server Software & The public-facing web server runs Nginx 1.18.0, an end-of-life version with known, unpatched vulnerabilities. This could serve as an initial entry point for an attack. & \textcolor{highorange}{High} \\
\addlinespace
RISK-003 & Inadequate Security Policies and Training & The lack of an Acceptable Use Policy and mandatory annual security training increases the likelihood of human error, such as falling for phishing attacks. & \textcolor{highorange}{High} \\
\bottomrule
\end{tabular}
\end{table}

% --- Section 6: Recommendations ---
\section{Recommendations}
\label{sec:recommendations}
The following actionable recommendations are provided to mitigate the identified risks. They are prioritized based on severity.

\begin{enumerate}
    \item \textbf{Implement Multi-Factor Authentication (Critical - RISK-001):}
    \begin{itemize}
        \item Immediately enable MFA for all user accounts across all critical platforms, including email (e.g., Office 365, Google Workspace), VPN access, and any systems containing sensitive data.
        \item Develop a policy that mandates MFA for all new employees and systems.
    \end{itemize}
    
    \item \textbf{Upgrade Web Server Software (High - RISK-002):}
    \begin{itemize}
        \item Plan and execute an upgrade of the Nginx server from version 1.18.0 to a current, stable, and supported version.
        \item Implement a patch management process to ensure all public-facing systems are regularly updated to protect against newly discovered vulnerabilities.
    \end{itemize}
    
    \item \textbf{Develop and Implement Security Policies and Training (High - RISK-003):}
    \begin{itemize}
        \item Create a formal Acceptable Use Policy (AUP) that clearly defines the rules for using company IT assets. Require all employees to read and acknowledge the policy.
        \item Institute a mandatory annual security awareness training program for all employees. The training should cover topics such as phishing, password security, and social engineering.
    \end{itemize}
\end{enumerate}

% --- Document End ---
\end{document}
```