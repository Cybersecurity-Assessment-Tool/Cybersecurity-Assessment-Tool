```latex
\documentclass[12pt]{article}

% 1. Document Setup & Packages
\usepackage[margin=1in]{geometry}
\usepackage{pifont} % For checkmarks and crosses
\usepackage{booktabs} % For professional tables
\usepackage{hyperref} % For clickable links
\usepackage{url} % For URL formatting
\usepackage{seqsplit} % For splitting long strings without breaking
\usepackage{graphicx}
\usepackage[table]{xcolor} % For coloring table cells and text
\usepackage{lastpage} % To get the total number of pages
\usepackage{fancyhdr} % For headers and footers

% --- Color Definitions ---
\definecolor{darkblue}{rgb}{0.0, 0.0, 0.55}
\definecolor{darkred}{rgb}{0.55, 0.0, 0.0}
\definecolor{headergray}{gray}{0.95}

% --- Hyperref Setup ---
\hypersetup{
    colorlinks=true,
    linkcolor=darkblue,
    filecolor=darkblue,      
    urlcolor=darkblue,
    citecolor=darkblue,
    pdftitle={Cybersecurity Posture Assessment Report},
    pdfauthor={Cybersecurity Analyst},
    pdfsubject={Security Report},
    pdfkeywords={Cybersecurity, Analysis, Report},
}

% --- Header and Footer Setup ---
\pagestyle{fancy}
\fancyhf{} % Clear all header and footer fields
\fancyhead[L]{Cybersecurity Posture Assessment}
\fancyhead[R]{\textbf{[Organization Name]}}
\fancyfoot[C]{\thepage\ of \pageref{LastPage}}
\renewcommand{\headrulewidth}{0.4pt}
\renewcommand{\footrulewidth}{0.4pt}

% --- Custom Commands ---
\newcommand{\yes}{\ding{51}}
\newcommand{\no}{{\color{darkred}\ding{55}}}

% 2. Document Start
\begin{document}

% --- Title Page ---
\begin{titlepage}
    \centering
    \vspace*{2cm}
    
    {\Huge\bfseries Cybersecurity Posture Assessment Report\par}
    
    \vspace{1.5cm}
    
    {\Large Prepared for:\par}
    \vspace{0.5cm}
    {\Huge\bfseries \textbf{[Organization Name]}}\par
    
    \vfill
    
    {\large \today\par}
    
    \vspace{1cm}
    
    {\large Confidential\par}
    
\end{titlepage}

\tableofcontents
\newpage

% 3. Executive Summary
\section{Executive Summary}
This report provides a comprehensive assessment of the cybersecurity posture for \textbf{[Organization Name]}. The analysis is based on a correlation of organizational security control data, technical network scan results, and a review of pre-existing risks.

The assessment identified two significant areas of concern requiring immediate attention. A critical gap exists in the enforcement of Multi-Factor Authentication (MFA) for accessing sensitive data systems. Additionally, a high-risk process gap was noted, as new employees do not receive mandatory security awareness training during their onboarding. These deficiencies expose the organization to significant risks, including unauthorized data access, credential compromise, and social engineering attacks.

On a positive note, the external network scan of the target system \texttt{[Target IP]} did not identify any open ports or exposed services. This suggests a strong perimeter firewall configuration. However, the internal policy and process gaps remain the primary focus for remediation.

This report outlines these findings in detail and provides actionable recommendations to mitigate the identified risks and strengthen the overall security posture.

% 4. Organizational Information
\section{Organizational Information}
The following details were used as the basis for this assessment. Where information was not provided, placeholders have been used.

\begin{table}[h!]
\centering
\begin{tabular}{@{}ll@{}}
\toprule
\textbf{Attribute} & \textbf{Value} \\ \midrule
Organization Name & \textbf{[Organization Name]} \\
Primary Email Domain & \texttt{[Domain]} \\
External IP Address (Client) & \texttt{[Client IP]} \\ \bottomrule
\end{tabular}
\caption{Client Organizational Details.}
\label{tab:org_info}
\end{table}

% 5. Security Control Review (from Questionnaire)
\section{Security Control Review}
An analysis of the organization's security questionnaire reveals its current state of administrative and technical controls. The following table summarizes the responses. Items marked with a \no\ represent significant gaps in the security framework and are discussed in the Risk Assessment section.

\begin{table}[h!]
\centering
\rowcolors{2}{gray!10}{white}
\begin{tabular}{p{0.8\linewidth}c}
\toprule
\rowcolor{headergray}
\textbf{Control Question} & \textbf{Status} \\ \midrule
Do you require MFA to access email? & \yes \\
Do you require MFA to log into computers? & \yes \\
Do you require MFA to access sensitive data systems? & \no \\
Does your organization have an employee acceptable use policy? & \yes \\
Does your organization do security awareness training for new employees? & \no \\
Does your organization do security awareness training for all employees at least once per year? & \yes \\ \bottomrule
\end{tabular}
\caption{Security Control Questionnaire Analysis.}
\label{tab:control_review}
\end{table}

% 6. Technical Scan Results
\section{Technical Scan Results}
An external network vulnerability scan was conducted to identify exposed services and potential vulnerabilities on the organization's perimeter.

\begin{itemize}
    \item \textbf{Scan Target:} \texttt{[Target IP]}
    \item \textbf{Scan Date:} Not specified in scan data.
\end{itemize}

\subsection{Findings}
The scan results indicate that \textbf{no open ports or services were detected} on the target system. This is a positive security finding, suggesting that the external firewall is properly configured to deny unsolicited inbound traffic. While this reduces the external attack surface, it does not mitigate risks from internal threats or application-layer vulnerabilities on approved services (e.g., a web server on port 443).

% 7. Risk Assessment
\section{Risk Assessment}
This section synthesizes findings from the security control review, technical scans, and pre-existing risk data. The primary risks identified are related to internal security policies and procedures.

\begin{table}[h!]
\centering
\rowcolors{2}{gray!10}{white}
\begin{tabular}{p{0.25\linewidth}p{0.55\linewidth}p{0.1\linewidth}}
\toprule
\rowcolor{headergray}
\textbf{Risk Name} & \textbf{Overview} & \textbf{Severity} \\ \midrule
\textbf{Lack of MFA for Sensitive Data Systems} & The absence of MFA on systems containing sensitive or critical data significantly increases the risk of unauthorized access. A single compromised password could lead to a major data breach. & \textbf{Critical} \\
\textbf{No Security Awareness Training for New Employees} & New hires are not provided with security awareness training upon joining. This makes them highly susceptible to phishing, social engineering, and other common attacks, as they are unfamiliar with organizational security policies. & \textbf{High} \\
\multicolumn{3}{l}{\textit{No pre-existing vulnerabilities were reported in the provided data.}} \\
\bottomrule
\end{tabular}
\caption{Summary of Identified Risks.}
\label{tab:risk_assessment}
\end{table}

% 8. Recommendations
\section{Recommendations}
Based on the analysis, the following actions are recommended to mitigate the identified risks and improve the overall security posture of \textbf{[Organization Name]}.

\subsection{Critical Priority}
\begin{itemize}
    \item \textbf{Implement MFA for All Sensitive Systems:}
    \begin{itemize}
        \item \textbf{Action:} Immediately prioritize the deployment and enforcement of a robust Multi-Factor Authentication (MFA) solution for all user accounts (including administrative and service accounts) that have access to sensitive data systems, databases, and critical infrastructure.
        \item \textbf{Impact:} Drastically reduces the risk of unauthorized access via compromised credentials, protecting the organization's most valuable data assets.
    \end{itemize}
\end{itemize}

\subsection{High Priority}
\begin{itemize}
    \item \textbf{Establish a New Hire Security Training Program:}
    \begin{itemize}
        \item \textbf{Action:} Develop and integrate a mandatory security awareness training module into the new employee onboarding process. This training should cover, at a minimum: acceptable use policies, phishing and social engineering recognition, password hygiene, and incident reporting procedures.
        \item \textbf{Impact:} Equips new employees with the fundamental knowledge to act as a human firewall, reducing the likelihood of successful social engineering attacks and policy violations from day one.
    \end{itemize}
\end{itemize}

\subsection{General Recommendations}
\begin{itemize}
    \item \textbf{Continuous Monitoring:} Although the external scan was clean, it is crucial to perform regular, authenticated vulnerability scans on both external and internal assets to maintain a comprehensive view of the security landscape.
    \item \textbf{Policy Review:} Periodically review and update all security policies, including the acceptable use policy, to ensure they align with current threats and business objectives.
\end{itemize}

% 9. Document End
\end{document}
```