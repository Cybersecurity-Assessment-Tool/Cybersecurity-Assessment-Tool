```latex
\documentclass[12pt]{article}

% --- PACKAGES ---
\usepackage[margin=1in]{geometry}
\usepackage{pifont} % Required for \ding
\usepackage{booktabs} % For professional-looking tables
\usepackage{hyperref} % For hyperlinks, references, and metadata
\usepackage{url}      % For formatting URLs
\usepackage{seqsplit} % To prevent long strings from overflowing
\usepackage{xcolor}   % For custom colors

% --- DOCUMENT METADATA ---
\hypersetup{
    colorlinks=true,
    linkcolor=blue,
    filecolor=magenta,      
    urlcolor=cyan,
    pdftitle={Cybersecurity Posture Assessment Report},
    pdfauthor={Cybersecurity Analysis Division},
    pdfsubject={Security Assessment},
    pdfkeywords={Cybersecurity, Nmap, Risk, Assessment},
}

% --- CUSTOM COMMANDS ---
\newcommand{\yes}{\ding{51}} % Green checkmark
\newcommand{\no}{\ding{55}}  % Red X mark

% --- DOCUMENT START ---
\begin{document}

% --- TITLE PAGE ---
\title{Cybersecurity Posture Assessment Report \\ \large For \textbf{[Organization Name]}}
\author{Cybersecurity Analysis Division}
\date{\today}
\maketitle

\newpage

% --- TABLE OF CONTENTS ---
\tableofcontents
\newpage

% --- EXECUTIVE SUMMARY ---
\section{Executive Summary}
This report details the findings of a cybersecurity posture assessment for \textbf{[Organization Name]}. The assessment combined a technical network scan, a review of organizational security controls via questionnaire, and an analysis of pre-existing risk data. The overall security posture is determined to be at a \textbf{High Risk} level.

Key findings include a publicly exposed and outdated MySQL database service, which presents a critical and immediate threat. This technical vulnerability, previously identified, is confirmed to still be active. This risk is significantly compounded by crucial gaps in organizational governance, specifically the absence of an employee acceptable use policy and the lack of security awareness training for new hires. These deficiencies create a permissive environment for both external attacks and internal threats. 

Immediate remediation is required to address the exposed database, followed by strategic improvements to security policies and training programs to build a more resilient security foundation.

% --- ORGANIZATIONAL INFORMATION ---
\section{Organizational Information}
The following information was used as the basis for this assessment. Anonymized placeholders are used where data was not provided.

\begin{tabular}{@{}ll}
\toprule
\textbf{Attribute} & \textbf{Value} \\
\midrule
Organization Name & \textbf{[Organization Name]} \\
Primary Domain & \texttt{[Domain]} \\
External IP Scanned & \texttt{[Client IP]} \\
\bottomrule
\end{tabular}

% --- SECURITY CONTROL REVIEW ---
\section{Security Control Review}
The following table summarizes the organization's responses to a security controls questionnaire. Items marked with a red \no\ represent significant gaps in the current security framework and are addressed in the Risk Assessment section.

\vspace{1em}
\begin{tabular}{@{}p{0.7\textwidth}c@{}}
\toprule
\textbf{Control Question} & \textbf{Response} \\
\midrule
Do you require MFA to access email? & \yes \\
Do you require MFA to log into computers? & \yes \\
Do you require MFA to access sensitive data systems? & \yes \\
Does your organization have an employee acceptable use policy? & \no \\
Does your organization do security awareness training for new employees? & \no \\
Does your organization do security awareness training for all employees at least once per year? & \yes \\
\bottomrule
\end{tabular}

% --- TECHNICAL SCAN RESULTS ---
\section{Technical Scan Results}
An external network scan was performed against the target IP address \texttt{[Target IP]}. The scan identified the following open ports and services accessible from the public internet.

\subsection{Open Ports and Services}
\begin{tabular}{@{}llll@{}}
\toprule
\textbf{Port} & \textbf{State} & \textbf{Service} & \textbf{Version} \\
\midrule
3306/tcp & open & mysql & MySQL 5.7.33 \\
\bottomrule
\end{tabular}

\subsection{Analysis of Technical Findings}
The scan confirms that a MySQL database service is directly exposed to the internet on port 3306. This is a critical security risk, as it allows attackers worldwide to directly target the database for brute-force attacks, exploitation of vulnerabilities, or denial-of-service attacks.

Furthermore, the identified version, \textbf{MySQL 5.7.33}, reached its official End of Life (EOL) in October 2023. End-of-life software no longer receives security updates from the vendor. This means any newly discovered vulnerabilities will remain unpatched, significantly increasing the risk of a successful compromise. This finding directly correlates with the pre-existing risk documented in "Database Exposure".

% --- RISK ASSESSMENT ---
\section{Synthesized Risk Assessment}
The following table correlates findings from the technical scan, control review, and existing risk data to provide a unified view of the organization's top security risks.

\vspace{1em}
\begin{tabular}{@{}p{0.25\textwidth}p{0.45\textwidth}p{0.1\textwidth}p{0.15\textwidth}@{}}
\toprule
\textbf{Risk Name} & \textbf{Overview} & \textbf{Severity} & \textbf{Affected Elements} \\
\midrule
\textbf{Exposed \& Outdated Database} & A MySQL database (v5.7.33) is publicly accessible on port 3306. The version is past its End of Life and no longer receives security patches. This is a known, high-impact risk. & \textbf{Critical} & \texttt{[Target IP]}:3306 \\
\addlinespace
\textbf{Lack of Acceptable Use Policy} & The organization does not have a formal policy governing the acceptable use of company assets. This leads to inconsistent security practices and a lack of employee accountability. & High & All Employees, IT Governance \\
\addlinespace
\textbf{Inadequate New Hire Onboarding} & New employees do not receive security awareness training, making them highly susceptible to phishing and social engineering attacks from their first day of employment. & High & New Employees, Human Resources \\
\bottomrule
\end{tabular}

% --- RECOMMENDATIONS ---
\section{Recommendations}
Based on the assessment, the following actions are recommended to mitigate the identified risks. Recommendations are prioritized by urgency.

\subsection{Immediate Priority (Critical)}
\begin{itemize}
    \item \textbf{Restrict Database Access:} Immediately implement firewall rules to \textbf{block all public access} to port 3306 on \texttt{[Target IP]}. Access should only be permitted from trusted, whitelisted IP addresses or via a secure VPN.
    \item \textbf{Plan Database Upgrade:} Begin planning the migration from MySQL 5.7.33 to a currently supported version (e.g., MySQL 8.x). This is essential to ensure the database receives critical security patches against future vulnerabilities.
\end{itemize}

\subsection{High Priority}
\begin{itemize}
    \item \textbf{Develop Acceptable Use Policy (AUP):} Draft and implement a comprehensive AUP that all employees must read and acknowledge. This policy should define clear rules for data handling, internet usage, password security, and incident reporting.
    \item \textbf{Implement New Hire Security Training:} Create a mandatory security awareness training module as part of the new employee onboarding process. This should cover key topics such as phishing identification, social engineering, and the new AUP.
\end{itemize}

\subsection{Long-Term Recommendations}
\begin{itemize}
    \item \textbf{Establish Secure Remote Access:} For long-term needs, implement a Virtual Private Network (VPN) solution to provide encrypted and authenticated access to internal resources like the database, which is more secure than IP whitelisting.
    \item \textbf{Formalize Vulnerability Management:} Implement a regular schedule of internal and external vulnerability scanning and a formal patch management program to proactively identify and remediate security weaknesses across all systems.
\end{itemize}

\end{document}
```