```latex
\documentclass[12pt]{article}

% Preamble: Required Packages
\usepackage[margin=1in]{geometry}
\usepackage{pifont} % For checkmarks and crosses
\usepackage{booktabs} % For professional tables
\usepackage[hidelinks]{hyperref} % For clickable links without boxes
\usepackage{url} % For URL formatting
\usepackage{seqsplit} % To split long monospaced strings
\usepackage{graphicx}
\usepackage{fancyhdr}
\usepackage{lastpage}
\usepackage{xcolor}

% Define colors for severity
\definecolor{criticalred}{HTML}{D73B3E}
\definecolor{highorange}{HTML}{F58426}
\definecolor{mediumyellow}{HTML}{FFC842}
\definecolor{lowblue}{HTML}{3498DB}
\definecolor{infogray}{HTML}{7F8C8D}

% Header and Footer Configuration
\pagestyle{fancy}
\fancyhf{} % Clear all header and footer fields
\fancyhead[L]{Cybersecurity Posture Assessment}
\fancyhead[R]{\textbf{[Organization Name]}}
\fancyfoot[C]{\thepage\ of \pageref{LastPage}}
\renewcommand{\headrulewidth}{0.4pt}
\renewcommand{\footrulewidth}{0.4pt}

% Document Start
\begin{document}

% --- Title Page ---
\begin{titlepage}
    \centering
    \vspace*{2cm}
    
    {\Huge \textbf{Cybersecurity Posture Assessment Report}\par}
    \vspace{1.5cm}
    
    {\Large \textbf{Prepared For:}}
    \vspace{0.5cm}
    
    {\Huge \textbf{[Organization Name]}}
    \vspace{2cm}
    
    \includegraphics[width=0.3\textwidth]{example-image-a} % Placeholder for a logo
    
    \vfill
    
    {\large \today\par}
    
    \vspace{1cm}
    
    {\small This document is confidential and intended solely for the use of the individual or entity to whom it is addressed.}
\end{titlepage}

\tableofcontents
\newpage

% --- Section 1: Executive Summary ---
\section{Executive Summary}

This report provides a comprehensive analysis of the cybersecurity posture of \textbf{[Organization Name]}, based on a synthesis of network scan data, a review of organizational security controls, and an assessment of current risks. The evaluation was conducted to identify vulnerabilities, security gaps, and areas for improvement.

The assessment revealed several critical and high-risk findings that require immediate attention. Key among these are significant gaps in the implementation of Multi-Factor Authentication (MFA) for core services like email and computer access. Furthermore, a technical scan identified an externally exposed, unencrypted web service (HTTP on port 80), which poses a direct threat to data confidentiality and integrity. The absence of a formal Employee Acceptable Use Policy exacerbates these risks by lacking a foundational governance framework for user behavior.

While the organization has implemented security awareness training, the identified technical and procedural weaknesses create a high-risk environment susceptible to account compromise, data breaches, and unauthorized access. This report outlines actionable recommendations prioritized by severity to mitigate these risks and strengthen the overall security framework.

% --- Section 2: Organizational Information ---
\section{Organizational Information}

This section details the organizational data used as a basis for this assessment. The information was derived from the provided datasets.

\begin{itemize}
    \item \textbf{Organization Name:} \textbf{[Organization Name]}
    \item \textbf{Primary Email Domain:} \texttt{[Domain]}
    \item \textbf{Assumed Client IP Block:} \texttt{[Client IP]}
    \item \textbf{Specific Target Scanned:} \seqsplit{\texttt{[Target IP]}}
\end{itemize}

% --- Section 3: Security Control Review ---
\section{Security Control Review}

The following table summarizes the organization's responses to a security controls questionnaire. Each response is evaluated against industry best practices. "No" answers indicate significant gaps in the security framework.

\begin{table}[h!]
\centering
\caption{Security Controls Questionnaire Analysis}
\begin{tabular}{p{0.6\linewidth} c p{0.2\linewidth}}
\toprule
\textbf{Control Question} & \textbf{Response} & \textbf{Assessment} \\
\midrule
Do you require MFA to access email? & \ding{55} & \textcolor{criticalred}{\textbf{Critical Gap}} \\
Do you require MFA to log into computers? & \ding{55} & \textcolor{criticalred}{\textbf{Critical Gap}} \\
Do you require MFA to access sensitive data systems? & \ding{51} & Compliant \\
Does your organization have an employee acceptable use policy? & \ding{55} & \textcolor{highorange}{\textbf{High Risk}} \\
Does your organization do security awareness training for new employees? & \ding{51} & Compliant \\
Does your organization do security awareness training for all employees at least once per year? & \ding{51} & Compliant \\
\bottomrule
\end{tabular}
\end{table}

\paragraph{Analysis:} The lack of MFA for email and computer logins represents a critical vulnerability. Email is a primary target for attackers seeking to gain an initial foothold, and its compromise can lead to widespread system access. The absence of an Acceptable Use Policy creates ambiguity regarding the secure use of company assets, increasing the risk of insider threats and non-compliance.

% --- Section 4: Technical Scan Results ---
\section{Technical Scan Results}

An external network scan was performed to identify open ports and exposed services. The following findings were observed on the target system.

\begin{itemize}
    \item \textbf{Target IP Address:} \seqsplit{\texttt{[Target IP]}}
    \item \textbf{Scan Date:} Not Specified
\end{itemize}

\begin{table}[h!]
\centering
\caption{Open Port Analysis}
\begin{tabular}{l l l p{0.5\linewidth}}
\toprule
\textbf{Port} & \textbf{State} & \textbf{Service} & \textbf{Notes} \\
\midrule
80/tcp & Open & http & The service is running over an unencrypted channel. This exposes all transmitted data, including potential credentials or sensitive information, to interception (Man-in-the-Middle attacks). \\
\bottomrule
\end{tabular}
\end{table}

\paragraph{Analysis:} The presence of an open port 80 (HTTP) is a significant security risk. Modern security standards mandate the use of encrypted protocols like HTTPS (port 443) for all web traffic to protect data in transit. This finding indicates that either a legacy system is in use or a modern system is misconfigured.

% --- Section 5: Risk Assessment Summary ---
\section{Risk Assessment Summary}

This section correlates the findings from the security control review and the technical scan to provide a summarized view of the most pressing risks.

\begin{table}[h!]
\centering
\caption{Synthesized Risk Register}
\begin{tabular}{p{0.1\linewidth} p{0.25\linewidth} p{0.15\linewidth} p{0.4\linewidth}}
\toprule
\textbf{Risk ID} & \textbf{Finding} & \textbf{Severity} & \textbf{Description} \\
\midrule
RISK-001 & Lack of Multi-Factor Authentication (MFA) & \textcolor{criticalred}{\textbf{Critical}} & The absence of MFA on email and computer logins makes user accounts highly susceptible to takeover via credential stuffing, phishing, or password spraying attacks. \\
\addlinespace
RISK-002 & Exposed Unencrypted Web Service & \textcolor{highorange}{\textbf{High}} & An open HTTP port on an external-facing system allows for the interception of all data transmitted to and from the service. This could lead to the theft of login credentials, session cookies, or other sensitive data. \\
\addlinespace
RISK-003 & Missing Acceptable Use Policy (AUP) & \textcolor{mediumyellow}{\textbf{Medium}} & Without a formal AUP, there is no enforceable standard for employee behavior regarding IT assets. This increases the likelihood of accidental data exposure, malware infections, and misuse of resources. \\
\bottomrule
\end{tabular}
\end{table}

% --- Section 6: Recommendations ---
\section{Recommendations}

The following actionable recommendations are provided to mitigate the identified risks and improve the overall security posture of \textbf{[Organization Name]}.

\subsection*{RISK-001: Implement Comprehensive MFA (Critical)}
\begin{itemize}
    \item \textbf{Immediate Action (0-30 days):} Enforce MFA for all user accounts on the primary email system. This is the single most effective control to prevent unauthorized account access.
    \item \textbf{Long-Term Action (30-90 days):} Develop and execute a plan to deploy MFA for all remote access solutions (e.g., VPN) and local computer logins.
\end{itemize}

\subsection*{RISK-002: Remediate Unencrypted Web Service (High)}
\begin{itemize}
    \item \textbf{Immediate Action (0-7 days):} Investigate the service running on port 80 of \seqsplit{\texttt{[Target IP]}}. If the service is not business-critical, disable it and block the port at the firewall.
    \item \textbf{Long-Term Action (7-45 days):} If the service is required, migrate it to use HTTPS exclusively. This involves obtaining and installing a valid TLS certificate and configuring the web server to redirect all HTTP traffic to HTTPS.
\end{itemize}

\subsection*{RISK-003: Develop and Implement an AUP (Medium)}
\begin{itemize}
    \item \textbf{Immediate Action (0-30 days):} Draft a formal Employee Acceptable Use Policy. This policy should clearly define the rules for handling company data, using the internet and email, and managing passwords.
    \item \textbf{Long-Term Action (30-60 days):} Integrate the AUP into the new employee onboarding process. Require all current employees to read and formally acknowledge the policy. Conduct brief training to ensure comprehension.
\end{itemize}

\end{document}
```