```latex
\documentclass[12pt]{article}

% Preamble: Required Packages
\usepackage[margin=1in]{geometry}
\usepackage{pifont} % For checkmarks and crosses
\usepackage{booktabs} % For professional tables
\usepackage{hyperref} % For hyperlinks
\usepackage{url} % For URL formatting
\usepackage{seqsplit} % For splitting long strings in tt font
\usepackage{graphicx}
\usepackage{fancyhdr}
\usepackage{xcolor}

% Document Metadata and Hyperref Setup
\hypersetup{
    colorlinks=true,
    linkcolor=blue,
    filecolor=magenta,      
    urlcolor=cyan,
    pdftitle={Cybersecurity Posture Report},
    pdfpagemode=FullScreen,
}

% Header and Footer Configuration
\pagestyle{fancy}
\fancyhf{}
\lhead{Cybersecurity Posture Report}
\rhead{\textbf{[Organization Name]}}
\cfoot{\thepage}
\renewcommand{\headrulewidth}{0.4pt}
\renewcommand{\footrulewidth}{0.4pt}

% Define custom colors
\definecolor{darkred}{rgb}{0.55, 0.0, 0.0}
\definecolor{darkorange}{rgb}{0.8, 0.33, 0.0}

\begin{document}

% --- TITLE PAGE ---
\begin{titlepage}
    \centering
    \vspace*{1cm}
    \includegraphics[width=0.3\textwidth]{example-image-a} % Placeholder logo
    \vfill
    \huge\bfseries
    Cybersecurity Posture Report
    \vspace{1.5cm}
    \Large
    Prepared for: \textbf{[Organization Name]}
    \vspace{1cm}
    \normalsize
    Report Date: \today \\
    Scan Date: 2025-11-22
    \vfill
    \small
    This report contains sensitive information and is intended solely for the use of the designated recipient.
\end{titlepage}

\tableofcontents
\newpage

% --- EXECUTIVE SUMMARY ---
\section{Executive Summary}

This report details the findings of a cybersecurity assessment conducted for \textbf{[Organization Name]}. The evaluation combined a review of organizational security controls, an external network scan, and an analysis of pre-existing risks.

The assessment identified several areas of concern that expose the organization to significant cyber threats. While foundational controls like Multi-Factor Authentication (MFA) for email and computers are in place, critical gaps were discovered.

\textbf{Key Findings:}
\begin{itemize}
    \item \textbf{Critical Risk - Lack of MFA on Sensitive Systems:} The absence of mandatory MFA for accessing sensitive data systems represents a critical vulnerability. A single compromised credential could lead to a major data breach.
    \item \textbf{High Risk - Inadequate Employee Onboarding:} New employees do not receive security awareness training, creating a high-risk window of vulnerability from the moment they join the organization.
    \item \textbf{High Risk - Outdated Web Server Software:} The external-facing web server at \texttt{[Target IP]} is running an outdated version of Nginx (1.18.0). This version has known vulnerabilities that could be exploited by attackers to compromise the server.
\end{itemize}

Immediate remediation of these issues is strongly recommended to reduce the organization's risk profile and enhance its overall security posture. Detailed recommendations are provided in Section \ref{sec:recommendations}.

% --- ORGANIZATIONAL INFORMATION ---
\section{Organizational Information}

The following details were used as the basis for this assessment. As per the provided data, placeholder values are used where specific information was not available.

\begin{tabular}{@{}ll}
    \toprule
    \textbf{Attribute} & \textbf{Value} \\
    \midrule
    Organization Name & \textbf{[Organization Name]} \\
    Primary Email Domain & \texttt{[Domain]} \\
    External IP Address Scanned & \texttt{[Client IP]} \\
    \bottomrule
\end{tabular}

% --- SECURITY CONTROL REVIEW ---
\section{Security Control Review}

A review of the organization's security policies and procedures was conducted via a questionnaire. The responses highlight critical gaps in the current security framework. A "No" answer indicates a deviation from security best practices.

\begin{table}[h!]
\centering
\caption{Security Controls Questionnaire Analysis}
\begin{tabular}{p{0.6\linewidth} c l}
    \toprule
    \textbf{Control Question} & \textbf{Response} & \textbf{Assessment} \\
    \midrule
    Do you require MFA to access email? & \ding{51} & Meets best practice. \\
    \addlinespace
    Do you require MFA to log into computers? & \ding{51} & Meets best practice. \\
    \addlinespace
    Do you require MFA to access sensitive data systems? & \textcolor{darkred}{\ding{55}} & \textbf{Critical Gap} \\
    \addlinespace
    Does your organization have an employee acceptable use policy? & \ding{51} & Meets best practice. \\
    \addlinespace
    Does your organization do security awareness training for new employees? & \textcolor{darkred}{\ding{55}} & \textbf{High Risk} \\
    \addlinespace
    Does your organization do security awareness training for all employees at least once per year? & \ding{51} & Meets best practice. \\
    \bottomrule
\end{tabular}
\end{table}

% --- TECHNICAL SCAN RESULTS ---
\section{Technical Scan Results}

An external network scan was performed on \textbf{2025-11-22} against the target IP address \texttt{[Target IP]}. The scan identified the following open ports and services.

\subsection{Open Ports}

\begin{table}[h!]
\centering
\caption{Open Ports on \texttt{[Target IP]}}
\begin{tabular}{l l l l l}
    \toprule
    \textbf{Port} & \textbf{State} & \textbf{Service} & \textbf{Product} & \textbf{Version} \\
    \midrule
    443/tcp & open & https & nginx & \seqsplit{\texttt{1.18.0}} \\
    \bottomrule
\end{tabular}
\end{table}

\subsection{Technical Findings}

The scan revealed that the web server is running \textbf{Nginx version 1.18.0}. This version was released in April 2020 and is now considered outdated. Legacy software versions often contain publicly disclosed vulnerabilities that are actively exploited by malicious actors. Running this version exposes the server to potential compromise, which could lead to website defacement, data theft, or the server being used as a pivot point for further attacks into the internal network.

% --- RISK ASSESSMENT ---
\section{Risk Assessment}

This section synthesizes the findings from the security control review and the technical scan. The pre-existing risk register (Input 3) was empty, so all risks listed below are new findings from this assessment.

\begin{table}[h!]
\centering
\caption{Summary of Identified Risks}
\begin{tabular}{p{0.1\linewidth} p{0.6\linewidth} l}
    \toprule
    \textbf{Risk ID} & \textbf{Description} & \textbf{Severity} \\
    \midrule
    RISK-001 & Lack of MFA on sensitive data systems allows for unauthorized access via compromised credentials. & \textcolor{darkred}{\textbf{Critical}} \\
    \addlinespace
    RISK-002 & Outdated Nginx web server (v1.18.0) is exposed to the internet, presenting a target with known vulnerabilities. & \textcolor{darkorange}{\textbf{High}} \\
    \addlinespace
    RISK-003 & New employees are not provided with security awareness training, making them susceptible to social engineering attacks. & \textcolor{darkorange}{\textbf{High}} \\
    \bottomrule
\end{tabular}
\end{table}

% --- RECOMMENDATIONS ---
\section{Recommendations}
\label{sec:recommendations}

Based on the identified risks, the following actions are recommended to improve the security posture of \textbf{[Organization Name]}. Recommendations are prioritized by severity.

\begin{enumerate}
    \item \textbf{Implement MFA for Sensitive Systems (RISK-001):}
    \begin{itemize}
        \item \textbf{Action:} Enforce mandatory Multi-Factor Authentication (MFA) for all user accounts (including administrative and service accounts) that have access to systems classified as containing sensitive or critical data.
        \item \textbf{Priority:} Immediate.
    \end{itemize}
    \vspace{0.5cm}
    
    \item \textbf{Upgrade Web Server Software (RISK-002):}
    \begin{itemize}
        \item \textbf{Action:} Plan and execute an upgrade of the Nginx server on \texttt{[Target IP]} from version 1.18.0 to the latest stable version. This should be done in a test environment first to ensure compatibility.
        \item \textbf{Priority:} High.
    \end{itemize}
    \vspace{0.5cm}
    
    \item \textbf{Establish Onboarding Security Training (RISK-003):}
    \begin{itemize}
        \item \textbf{Action:} Develop a mandatory security awareness training module and integrate it into the new employee onboarding process. This training should cover key topics such as phishing, password security, and the acceptable use policy.
        \item \textbf{Priority:} High.
    \end{itemize}
\end{enumerate}

\end{document}
```