```latex
\documentclass[12pt]{article}

% Preamble: Required Packages
\usepackage[margin=1in]{geometry}
\usepackage{pifont} % For checkmarks and crosses
\usepackage{booktabs} % For professional tables
\usepackage{hyperref} % For clickable links
\usepackage{url}      % For URL formatting
\usepackage{seqsplit} % For splitting long strings
\usepackage{graphicx} % For potential logos
\usepackage{xcolor}   % For colors in text

% Document Metadata
\title{Cybersecurity Posture Assessment Report}
\author{Cybersecurity Analysis Division}
\date{\today}

% Hyperref Setup
\hypersetup{
    colorlinks=true,
    linkcolor=blue,
    filecolor=magenta,      
    urlcolor=cyan,
    pdftitle={Cybersecurity Posture Assessment Report},
    pdfpagemode=FullScreen,
}

\begin{document}

\maketitle
\thispagestyle{empty}
\newpage

\tableofcontents
\newpage

% --- 1. Executive Summary ---
\section{Executive Summary}

This report provides a comprehensive cybersecurity assessment for \textbf{[Organization Name]}, conducted on \textbf{[Scan Date]}. The analysis synthesizes data from an external network scan, a security controls questionnaire, and a review of pre-existing risks.

The assessment reveals a mixed security posture. The organization demonstrates strong foundational controls in identity and access management, with consistent enforcement of Multi-Factor Authentication (MFA) across email, computers, and sensitive data systems. This significantly reduces the risk of unauthorized access through compromised credentials.

However, critical deficiencies were identified in the organization's governance and human-layer security. The complete absence of an employee Acceptable Use Policy (AUP) and any form of security awareness training presents a high risk. These gaps leave the organization vulnerable to insider threats, social engineering attacks like phishing, and inconsistent security practices.

From a technical perspective, the external network scan identified an open Secure Shell (SSH) port (22/TCP). While necessary for remote administration, its exposure to the public internet increases the attack surface and requires immediate mitigation through access control restrictions.

Key recommendations focus on urgently establishing a formal security awareness training program, developing and implementing a comprehensive AUP, and hardening the exposed SSH service by restricting access to authorized IP addresses only. Addressing these policy and training gaps is paramount to maturing the organization's overall security posture.

% --- 2. Organizational Information ---
\section{Organizational Information}

The following details were used as the basis for this assessment. Due to missing data in the provided inputs, placeholders have been used where necessary.

\begin{table}[h!]
\centering
\begin{tabular}{@{}ll@{}}
\toprule
\textbf{Attribute} & \textbf{Value} \\ \midrule
Organization Name    & \textbf{[Organization Name]} \\
Primary Domain       & \texttt{[Domain]} \\
External IP Address  & \texttt{[Client IP]} \\
Target IP Scanned    & \texttt{[Target IP]} \\ \bottomrule
\end{tabular}
\caption{Client Organizational Details}
\label{tab:org_info}
\end{table}

% --- 3. Security Control Review ---
\section{Security Control Review}

A review of the organization's security controls was conducted via a questionnaire. The results highlight significant gaps in administrative and policy-based controls. While technical access controls are strong, the lack of employee guidance and training represents a critical vulnerability.

\begin{table}[h!]
\centering
\begin{tabular}{@{}lc@{}}
\toprule
\textbf{Control Question} & \textbf{Status} \\ \midrule
Do you require MFA to access email? & \textcolor{green}{\ding{51}} \\
Do you require MFA to log into computers? & \textcolor{green}{\ding{51}} \\
Do you require MFA to access sensitive data systems? & \textcolor{green}{\ding{51}} \\
Does your organization have an employee acceptable use policy? & \textcolor{red}{\ding{55}} \\
Does your organization do security awareness training for new employees? & \textcolor{red}{\ding{55}} \\
Does your organization do security awareness training for all employees annually? & \textcolor{red}{\ding{55}} \\ \bottomrule
\end{tabular}
\caption{Security Controls Questionnaire Results (\ding{51} = Yes, \ding{55} = No)}
\label{tab:controls}
\end{table}

\subsection*{Analysis of Control Gaps}
\begin{itemize}
    \item \textbf{Acceptable Use Policy (AUP):} The absence of an AUP means there are no formal, documented rules for employees regarding the use of company networks, systems, and data. This can lead to misuse of assets, data leakage, and legal ambiguity.
    \item \textbf{Security Awareness Training:} The lack of any security training for new or existing employees is a critical failure. Employees are the first line of defense against threats like phishing and social engineering. Without training, they are significantly more likely to fall victim to attacks, potentially leading to credential theft, malware infection, or data breaches.
\end{itemize}

% --- 4. Technical Scan Results ---
\section{Technical Scan Results}

An external network scan was performed on the target IP address \texttt{[Target IP]}. The scan identified one open port, which indicates a publicly accessible service.

\begin{table}[h!]
\centering
\begin{tabular}{@{}lllll@{}}
\toprule
\textbf{Port} & \textbf{Protocol} & \textbf{State} & \textbf{Service} & \textbf{Notes} \\ \midrule
22 & TCP & open & ssh & Secure Shell is used for remote administration. \\
& & & & No version information was available. \\ \bottomrule
\end{tabular}
\caption{Open Ports Detected on \texttt{[Target IP]}}
\label{tab:scan_results}
\end{table}

\subsection*{Analysis of Technical Findings}
The presence of an open SSH port (22) on the external perimeter is a common but notable finding. If not securely configured, this service can be a target for:
\begin{itemize}
    \item \textbf{Brute-force attacks:} Automated attempts to guess usernames and passwords.
    \item \textbf{Exploitation of vulnerabilities:} If the SSH server software is outdated, it may be susceptible to known exploits.
\end{itemize}
While essential for remote management, access to this port should be strictly controlled.

% --- 5. Consolidated Risk Assessment ---
\section{Consolidated Risk Assessment}

This section correlates the findings from the security control review and the technical scan. The `vulnerabilities` list from the input data was empty, so all risks listed below are new findings from this assessment.

\begin{table}[h!]
\centering
\begin{tabular}{@{}p{0.2\linewidth}p{0.15\linewidth}p{0.55\linewidth}@{}}
\toprule
\textbf{Risk Name} & \textbf{Severity} & \textbf{Overview} \\ \midrule
Lack of Security Awareness Training & \textbf{High} & Employees are not equipped to identify or respond to common cyber threats like phishing, malware, or social engineering. This makes the organization highly susceptible to human-targeted attacks. \\
\addlinespace
Lack of Acceptable Use Policy & \textbf{High} & Without a formal AUP, there is no enforceable standard for employee behavior on corporate assets. This increases the risk of insider threat, data misuse, and installation of unauthorized software. \\
\addlinespace
Exposed SSH Service & \textbf{Medium} & The SSH management port is open to the public internet, increasing the attack surface. It is a potential target for brute-force login attempts and exploitation if not properly maintained and configured. \\ \bottomrule
\end{tabular}
\caption{Summary of Identified Risks}
\label{tab:risks}
\end{table}

% --- 6. Recommendations ---
\section{Recommendations}

The following actionable recommendations are provided to mitigate the identified risks and improve the overall security posture of \textbf{[Organization Name]}.

\begin{description}
    \item[\textbf{Risk: Lack of Security Awareness Training (High)}]
        \begin{itemize}
            \item \textbf{Immediate Action:} Implement a mandatory security awareness training program for all employees. This should be part of the onboarding process for new hires and conducted at least annually for all staff.
            \item \textbf{Content:} Training should cover phishing recognition, password hygiene, safe internet use, and reporting security incidents.
            \item \textbf{Testing:} Conduct periodic, simulated phishing campaigns to test and reinforce employee learning.
        \end{itemize}

    \item[\textbf{Risk: Lack of Acceptable Use Policy (High)}]
        \begin{itemize}
            \item \textbf{Immediate Action:} Develop a comprehensive Acceptable Use Policy (AUP) that clearly defines the rules for using company technology and data.
            \item \textbf{Distribution:} Require all employees to read and formally acknowledge the policy. This acknowledgment should be documented.
            \item \textbf{Enforcement:} Establish clear consequences for policy violations.
        \end{itemize}
        
    \item[\textbf{Risk: Exposed SSH Service (Medium)}]
        \begin{itemize}
            \item \textbf{Immediate Action:} Restrict access to port 22 to a whitelist of trusted IP addresses (e.g., administrator offices, VPN gateway). Deny all other connections by default.
            \item \textbf{Hardening:} Ensure the SSH service is configured securely. This includes disabling root login, using key-based authentication instead of passwords, and keeping the SSH server software fully patched.
        \end{itemize}
\end{description}

% --- 7. Conclusion ---
\section{Conclusion}

The security posture of \textbf{[Organization Name]} has a solid foundation in MFA-based access control but is critically undermined by a lack of security governance and employee training. The human element remains the most significant vulnerability. The technical finding of an exposed SSH port, while less severe, still presents an unnecessary risk that is simple to mitigate.

By implementing the recommendations outlined in this report—specifically by establishing a robust security training program, enforcing a clear AUP, and securing the network perimeter—the organization can significantly reduce its risk exposure and build a more resilient and mature cybersecurity program.

\end{document}
```