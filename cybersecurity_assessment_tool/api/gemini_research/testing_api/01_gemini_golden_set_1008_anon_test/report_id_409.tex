```latex
\documentclass[12pt]{article}

% Preamble: Required Packages
\usepackage[margin=1in]{geometry}
\usepackage{pifont} % For checkmarks and crosses
\usepackage{booktabs} % For professional tables
\usepackage{hyperref} % For hyperlinks
\usepackage{url} % For URL formatting
\usepackage{seqsplit} % For splitting long strings
\usepackage{graphicx}
\usepackage{fancyhdr}

% Document Metadata and Styling
\hypersetup{
    colorlinks=true,
    linkcolor=blue,
    filecolor=magenta,      
    urlcolor=cyan,
    pdftitle={Cybersecurity Assessment Report},
    pdfauthor={Cybersecurity Analyst},
    pdfsubject={Security Analysis},
    pdfkeywords={Security, Report, Analysis},
}

\pagestyle{fancy}
\fancyhf{}
\lhead{Cybersecurity Assessment Report}
\rhead{\textbf{[Organization Name]}}
\cfoot{\thepage}

\begin{document}

% --- Title Page ---
\begin{titlepage}
    \centering
    \vspace*{1cm}
    \includegraphics[width=0.3\textwidth]{example-image-a} % Placeholder for a logo
    
    \vspace{1.5cm}
    
    \Huge
    \textbf{Cybersecurity Assessment Report}
    
    \vspace{1.5cm}
    
    \Large
    Prepared for: \textbf{[Organization Name]}
    
    \vspace{2cm}
    
    \normalsize
    \begin{tabular}{ll}
        \textbf{Assessment Date:} & November 22, 2025 \\
        \textbf{Report Version:} & 1.0 \\
        \textbf{Author:} & Cybersecurity Analyst \\
    \end{tabular}
    
    \vfill
    
    \small
    \textit{This document contains sensitive information. Distribution is restricted to authorized personnel only.}
\end{titlepage}

\tableofcontents
\newpage

% --- Section 1: Executive Summary ---
\section{Executive Summary}
This report details the findings of a cybersecurity assessment conducted for \textbf{[Organization Name]} on November 22, 2025. The assessment combined a review of organizational security controls, an external network scan, and an analysis of pre-existing risks to provide a holistic view of the organization's current security posture.

The assessment identified several areas of concern that require immediate attention. Key findings include:
\begin{itemize}
    \item \textbf{Critical Control Gap:} The absence of Multi-Factor Authentication (MFA) for computer logins presents a critical risk. A single compromised password could grant an attacker direct access to an endpoint, facilitating lateral movement and data exfiltration.
    \item \textbf{High-Risk External Service:} The external-facing web server is running an outdated version of Nginx (1.18.0). This version is several years old and has multiple publicly disclosed vulnerabilities, making it a prime target for exploitation.
    \item \textbf{High-Risk Human Factor Gap:} The lack of mandatory, annual security awareness training for all employees increases the organization's susceptibility to phishing, social engineering, and other human-centric attacks.
\end{itemize}

While the organization has implemented some positive security controls, such as MFA for email and sensitive systems, the identified gaps significantly weaken its overall defensive posture. This report provides a detailed breakdown of these risks and offers actionable recommendations to mitigate them effectively.

% --- Section 2: Organizational Information ---
\section{Organizational Information}
This section provides general information about the organization and the scope of this assessment.
\begin{itemize}
    \item \textbf{Organization Name:} \textbf{[Organization Name]}
    \item \textbf{Primary Email Domain:} \texttt{[Domain]}
    \item \textbf{External IP Address Scanned:} \texttt{[Client IP]}
    \item \textbf{Assessment Date:} November 22, 2025
\end{itemize}

% --- Section 3: Security Control Review ---
\section{Security Control Review (Questionnaire Analysis)}
An analysis of the organization's security questionnaire revealed its current stance on key security controls. The following table summarizes the responses. A green checkmark (\ding{51}) indicates a positive control is in place, while a red cross (\ding{55}) highlights a control gap.

\begin{table}[h!]
\centering
\caption{Security Controls Questionnaire Summary}
\begin{tabular}{p{0.7\linewidth} c c}
\toprule
\textbf{Control Question} & \textbf{Response} & \textbf{Status} \\
\midrule
Do you require MFA to access email? & Yes & \ding{51} \\
Do you require MFA to log into computers? & No & \ding{55} \\
Do you require MFA to access sensitive data systems? & Yes & \ding{51} \\
Does your organization have an employee acceptable use policy? & Yes & \ding{51} \\
Does your organization do security awareness training for new employees? & Yes & \ding{51} \\
Does your organization do security awareness training for all employees at least once per year? & No & \ding{55} \\
\bottomrule
\end{tabular}
\end{table}

\subsection*{Analysis of Control Gaps}
Two significant control gaps were identified:
\begin{enumerate}
    \item \textbf{No MFA for Computer Logins:} This is a critical deficiency. Endpoint devices are a primary target for attackers. Without MFA, a compromised password (obtained via phishing or credential stuffing) is all that is needed to gain access to an employee's computer and the network resources it can reach.
    \item \textbf{No Annual Security Awareness Training:} While training for new hires is a good first step, the threat landscape evolves continuously. Without annual refresher training, employees are more likely to fall victim to modern phishing and social engineering tactics, undermining other security controls.
\end{enumerate}

% --- Section 4: Technical Scan Results ---
\section{Technical Scan Results}
An external network scan was performed using Nmap to identify open ports and services visible on the public internet.

\begin{itemize}
    \item \textbf{Target IP Address:} \texttt{[Target IP]}
    \item \textbf{Scan Date:} 2025-11-22T10:00:00Z
\end{itemize}

The scan revealed the following open port:

\begin{table}[h!]
\centering
\caption{Open Ports and Services}
\begin{tabular}{l l l l l}
\toprule
\textbf{Port} & \textbf{State} & \textbf{Service} & \textbf{Product} & \textbf{Version} \\
\midrule
443/TCP & Open & https & nginx & 1.18.0 \\
\bottomrule
\end{tabular}
\end{table}

\subsection*{Technical Analysis}
The scan identified a single open port, 443, which is standard for HTTPS traffic. However, the service running is \textbf{Nginx version 1.18.0}. This version was released in April 2020 and is now considered outdated. It is known to be vulnerable to several Common Vulnerabilities and Exposures (CVEs), including but not limited to issues related to request smuggling and DNS resolver vulnerabilities (e.g., CVE-2021-23017). Running outdated software on an internet-facing server presents a high risk of compromise.

% --- Section 5: Consolidated Risk Assessment ---
\section{Consolidated Risk Assessment}
This section synthesizes findings from the organizational review and technical scan into a consolidated list of identified risks. No pre-existing vulnerabilities were reported.

\begin{table}[h!]
\centering
\caption{Summary of Identified Risks}
\begin{tabular}{p{0.1\linewidth} p{0.4\linewidth} p{0.15\linewidth} p{0.25\linewidth}}
\toprule
\textbf{Risk ID} & \textbf{Description} & \textbf{Severity} & \textbf{Impact} \\
\midrule
ORG-001 & Lack of Multi-Factor Authentication (MFA) for computer logins. & \textbf{Critical} & A compromised password could lead to direct endpoint and network access. \\
\addlinespace
TEC-001 & Outdated Nginx web server (v1.18.0) exposed to the internet. & \textbf{High} & The server is vulnerable to known exploits, potentially leading to a system compromise or data breach. \\
\addlinespace
ORG-002 & Security awareness training is not conducted annually for all employees. & \textbf{High} & Increased susceptibility to phishing and social engineering attacks, leading to credential theft or malware infection. \\
\bottomrule
\end{tabular}
\end{table}

% --- Section 6: Recommendations ---
\section{Recommendations}
Based on the findings, the following prioritized actions are recommended to improve the security posture of \textbf{[Organization Name]}.

\subsection*{Priority 1: Critical}
\begin{itemize}
    \item \textbf{Implement MFA for Endpoint Logins (Risk ID: ORG-001):} Immediately deploy a robust MFA solution (e.g., authenticator app, hardware token, or biometrics) for all employee computer and remote access logins. This is the single most effective control to mitigate the risk of compromised credentials.
\end{itemize}

\subsection*{Priority 2: High}
\begin{itemize}
    \item \textbf{Upgrade Web Server Software (Risk ID: TEC-001):} Plan and execute an upgrade of the Nginx server from version 1.18.0 to the latest stable, patched version. Before deployment, test the new version in a staging environment to ensure application compatibility.
    \item \textbf{Establish Annual Security Training (Risk ID: ORG-002):} Institute a mandatory annual security awareness training program for all employees. The training should cover current threats such as phishing, ransomware, and proper data handling. Track completion to ensure compliance.
\end{itemize}

\end{document}
```