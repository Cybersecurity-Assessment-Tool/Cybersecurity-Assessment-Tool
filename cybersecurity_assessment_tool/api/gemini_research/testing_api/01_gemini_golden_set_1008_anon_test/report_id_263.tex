```latex
\documentclass[12pt]{article}

% ----------------------------------------------------------------------
% PREAMBLE
% ----------------------------------------------------------------------
\usepackage[margin=1in]{geometry}
\usepackage{pifont} % Required for \ding
\usepackage{booktabs} % For professional-looking tables
\usepackage{hyperref} % For clickable links and table of contents
\usepackage{url}      % For properly formatting URLs
\usepackage{seqsplit} % For breaking long strings in \texttt
\usepackage{graphicx}
\usepackage{xcolor}

\hypersetup{
    colorlinks=true,
    linkcolor=black,
    filecolor=magenta,      
    urlcolor=blue,
    pdftitle={Cybersecurity Posture Assessment Report},
    pdfpagemode=FullScreen,
}

\newcommand{\yes}{\ding{51}}
\newcommand{\no}{\ding{55}}

% ----------------------------------------------------------------------
% DOCUMENT START
% ----------------------------------------------------------------------
\begin{document}

\title{
    \vspace{2cm}
    \textbf{Cybersecurity Posture Assessment Report} \\
    \large For: \textbf{[Organization Name]}
    \vspace{1cm}
}
\author{Cybersecurity Analysis Division}
\date{\today}

\maketitle
\thispagestyle{empty}

\newpage

\tableofcontents

\newpage

% ----------------------------------------------------------------------
% 1. EXECUTIVE SUMMARY
% ----------------------------------------------------------------------
\section{Executive Summary}

This report provides a comprehensive assessment of the cybersecurity posture for \textbf{[Organization Name]}. The analysis is based on a correlation of network scan data, a review of organizational security controls, and an evaluation of pre-existing documented risks.

The assessment reveals a mixed security posture. The organization has implemented critical controls such as Multi-Factor Authentication (MFA) for email and sensitive data systems. However, several significant gaps were identified that present a high level of risk. 

Key findings include:
\begin{itemize}
    \item \textbf{Critical Pre-existing Vulnerability:} A documented risk, "Localhost Exposed," is rated with the highest possible severity (CVSS 10.0) and requires immediate investigation and remediation.
    \item \textbf{Endpoint Security Gaps:} The absence of MFA for computer logins represents a major vulnerability. If an attacker compromises user credentials, they could gain direct access to an endpoint without this critical secondary control.
    \item \textbf{Exposed Network Services:} An external scan identified an open Secure Shell (SSH) port. When combined with the lack of endpoint MFA, this significantly increases the risk of a successful brute-force or credential stuffing attack.
    \item \textbf{Inadequate Employee Onboarding:} New employees do not receive security awareness training, creating a window of vulnerability where they are more susceptible to social engineering attacks like phishing.
\end{itemize}

This report details these findings and provides actionable recommendations to mitigate the identified risks and strengthen the organization's overall defensive capabilities.

% ----------------------------------------------------------------------
% 2. ORGANIZATIONAL INFORMATION
% ----------------------------------------------------------------------
\section{Organizational Information}

This section outlines the basic information for the organization under review. The data provided was anonymized for this assessment.

\begin{table}[h!]
\centering
\begin{tabular}{@{}ll@{}}
\toprule
\textbf{Attribute} & \textbf{Value} \\ \midrule
Organization Name & \textbf{[Organization Name]} \\
Primary Email Domain & \texttt{[Domain]} \\
External IP Address Scanned & \texttt{[Client IP]} \\ \bottomrule
\end{tabular}
\caption{Client Information.}
\end{table}

% ----------------------------------------------------------------------
% 3. SECURITY CONTROL REVIEW
% ----------------------------------------------------------------------
\section{Security Control Review}

A review of the organization's security controls was conducted via a standardized questionnaire. The results below highlight implemented controls and identify significant gaps. A green checkmark (\yes) indicates a positive response, while a red 'X' (\no) indicates a negative response that represents a potential security gap.

\begin{table}[h!]
\centering
\begin{tabular}{@{}p{0.8\textwidth}c@{}}
\toprule
\textbf{Security Control Question} & \textbf{Status} \\ \midrule
Do you require MFA to access email? & \yes \\
Do you require MFA to log into computers? & \textcolor{red}{\no} \\
Do you require MFA to access sensitive data systems? & \yes \\
Does your organization have an employee acceptable use policy? & \yes \\
Does your organization do security awareness training for new employees? & \textcolor{red}{\no} \\
Does your organization do security awareness training for all employees at least once per year? & \yes \\ \bottomrule
\end{tabular}
\caption{Organizational Security Control Status.}
\end{table}

\subsection{Analysis of Control Gaps}
Two critical control gaps were identified from the questionnaire:
\begin{enumerate}
    \item \textbf{No MFA for Computer Logins:} This is a high-risk gap. Credential theft is a primary vector for attackers. Without MFA at the endpoint login, a compromised password could grant an attacker full access to a user's machine, network resources, and data.
    \item \textbf{No Security Training for New Employees:} New hires are often prime targets for phishing and other social engineering attacks. Failing to provide immediate security training leaves the organization vulnerable during the critical onboarding period.
\end{enumerate}

% ----------------------------------------------------------------------
% 4. TECHNICAL SCAN RESULTS
% ----------------------------------------------------------------------
\section{Technical Scan Results}

An external network scan was performed on the target IP address to identify open ports and exposed services.

\begin{itemize}
    \item \textbf{Target IP Address:} \texttt{[Target IP]}
    \item \textbf{Scan Status:} Host is up.
\end{itemize}

\begin{table}[h!]
\centering
\begin{tabular}{@{}llll@{}}
\toprule
\textbf{Port} & \textbf{State} & \textbf{Service} & \textbf{Notes} \\ \midrule
22/tcp & open & SSH & Secure Shell for remote administration. \\ \bottomrule
\end{tabular}
\caption{Open Ports Detected on \texttt{[Target IP]}.}
\end{table}

\subsection{Analysis of Technical Findings}
The scan revealed that port 22 (SSH) is open to the internet. While SSH is a legitimate tool for remote administration, its public exposure is a significant security risk. It provides a direct attack surface for:
\begin{itemize}
    \item \textbf{Brute-force attacks:} Automated tools can attempt to guess usernames and passwords.
    \item \textbf{Credential stuffing:} Attackers can use credentials stolen from other data breaches.
    \item \textbf{Exploitation of vulnerabilities:} If the SSH server software is outdated, it may be vulnerable to known exploits.
\end{itemize}
This finding is especially concerning given the lack of MFA on computer logins, as a single compromised password could lead to a system breach via this exposed service.

% ----------------------------------------------------------------------
% 5. CONSOLIDATED RISK ASSESSMENT
% ----------------------------------------------------------------------
\section{Consolidated Risk Assessment}
This section synthesizes the findings from the security control review, technical scan, and pre-existing risk documentation into a consolidated list of identified risks.

\begin{table}[h!]
\centering
\begin{tabular}{@{}lp{0.6\textwidth}l@{}}
\toprule
\textbf{Risk ID} & \textbf{Risk Description} & \textbf{Severity} \\ \midrule
RISK-001 & A pre-existing vulnerability "Localhost Exposed" exists with a CVSS score of 10.0 on asset \texttt{[Target IP]}. & \textbf{Critical} \\
RISK-002 & The SSH service is exposed on the external network, creating a vector for unauthorized access. & High \\
RISK-003 & Endpoint systems (computers) lack MFA, allowing a single compromised password to grant an attacker system access. & High \\
RISK-004 & New employees are not provided with security awareness training, increasing susceptibility to social engineering. & Medium \\
\bottomrule
\end{tabular}
\caption{Summary of Identified Risks.}
\end{table}

% ----------------------------------------------------------------------
% 6. RECOMMENDATIONS
% ----------------------------------------------------------------------
\section{Recommendations}
The following prioritized recommendations are provided to mitigate the identified risks and improve the overall security posture of \textbf{[Organization Name]}.

\begin{enumerate}
    \item \textbf{Remediate Critical Vulnerability (RISK-001):}
    \begin{itemize}
        \item \textbf{Action:} Immediately investigate the "Localhost Exposed" vulnerability. The CVSS score of 10.0 indicates a severe and easily exploitable flaw that could lead to a complete system compromise.
        \item \textbf{Priority:} \textbf{CRITICAL}
    \end{itemize}
    \vspace{0.5cm}
    \item \textbf{Secure Exposed SSH Service (RISK-002):}
    \begin{itemize}
        \item \textbf{Action:} If remote access is required, restrict SSH access to authorized IP addresses only using a firewall. If it is not required, disable the service. For authorized access, enforce the use of public key authentication and disable password-based logins entirely.
        \item \textbf{Priority:} High
    \end{itemize}
    \vspace{0.5cm}
    \item \textbf{Implement Endpoint MFA (RISK-003):}
    \begin{itemize}
        \item \textbf{Action:} Deploy a mandatory MFA solution for all computer and server logins. This acts as a critical compensating control that protects against credential theft.
        \item \textbf{Priority:} High
    \end{itemize}
    \vspace{0.5cm}
    \item \textbf{Enhance Security Awareness Program (RISK-004):}
    \begin{itemize}
        \item \textbf{Action:} Integrate mandatory security awareness training into the new employee onboarding process. This training should occur before a new hire is granted access to sensitive systems or data.
        \item \textbf{Priority:} Medium
    \end{itemize}
\end{enumerate}

\end{document}
```