```latex
\documentclass[12pt]{article}

% Preamble: Required Packages
\usepackage[margin=1in]{geometry} % For setting page margins
\usepackage{pifont}               % For checkmark and X symbols (\ding)
\usepackage{booktabs}             % For professional-looking tables
\usepackage{hyperref}             % For hyperlinks
\usepackage{url}                  % For formatting URLs
\usepackage{seqsplit}             % To split long strings in texttt
\usepackage{graphicx}             % For including images (e.g., logo)
\usepackage{xcolor}               % For custom colors

% Document Information
\title{Cybersecurity Posture Assessment Report}
\author{Cybersecurity Analysis Division}
\date{\today}

% Hyperref Setup
\hypersetup{
    colorlinks=true,
    linkcolor=blue,
    filecolor=magenta,      
    urlcolor=cyan,
    pdftitle={Cybersecurity Posture Assessment Report},
    pdfpagemode=FullScreen,
}

\begin{document}

\maketitle
\hrule
\vspace{1em}

%----------------------------------------------------------------------------------------
%   1. EXECUTIVE SUMMARY
%----------------------------------------------------------------------------------------
\section*{Executive Summary}

This report provides a comprehensive cybersecurity assessment for \textbf{[Organization Name]}, based on an analysis of organizational security controls, an external network scan, and a review of known risks. The assessment was conducted on \today.

The key findings indicate significant procedural and policy-based risks within the organization. While the external network perimeter appears hardened, with no open services detected, critical gaps exist in internal security controls. Specifically, the absence of Multi-Factor Authentication (MFA) for computer and sensitive data system access, coupled with a lack of foundational security policies and employee training, presents a high likelihood of compromise through social engineering or credential theft.

Our recommendations prioritize the immediate implementation of MFA across all critical systems, the development of an Acceptable Use Policy, and the establishment of a mandatory security awareness training program to mitigate these identified risks.

%----------------------------------------------------------------------------------------
%   2. ORGANIZATIONAL INFORMATION
%----------------------------------------------------------------------------------------
\section{Organizational Information}

The following details were used as the basis for this assessment. Due to missing data in the provided inputs, placeholders have been used where necessary.

\begin{itemize}
    \item \textbf{Organization Name:} \textbf{[Organization Name]}
    \item \textbf{Primary Email Domain:} \texttt{[Domain]}
    \item \textbf{External IP Scanned:} \texttt{[Client IP]}
    \item \textbf{Target of Network Scan:} \texttt{[Target IP]}
\end{itemize}

%----------------------------------------------------------------------------------------
%   3. SECURITY CONTROL REVIEW (QUESTIONNAIRE)
%----------------------------------------------------------------------------------------
\section{Security Control Review}

A review of the organization's security controls was conducted via a standardized questionnaire. The responses reveal critical gaps in identity management and security governance. A summary of the findings is presented in Table 1.

\begin{table}[h!]
\centering
\caption{Security Controls Questionnaire Analysis}
\begin{tabular}{p{0.6\linewidth} c p{0.2\linewidth}}
\toprule
\textbf{Control Question} & \textbf{Response} & \textbf{Assessment} \\
\midrule
Do you require MFA to access email? & \ding{51} Yes & Good Practice \\
\addlinespace
Do you require MFA to log into computers? & \textbf{\color{red}\ding{55} No} & \textbf{Critical Gap} \\
\addlinespace
Do you require MFA to access sensitive data systems? & \textbf{\color{red}\ding{55} No} & \textbf{Critical Gap} \\
\addlinespace
Does your organization have an employee acceptable use policy? & \textbf{\color{red}\ding{55} No} & High Risk \\
\addlinespace
Does your organization do security awareness training for new employees? & \textbf{\color{red}\ding{55} No} & High Risk \\
\addlinespace
Does your organization do security awareness training for all employees at least once per year? & \textbf{\color{red}\ding{55} No} & High Risk \\
\bottomrule
\end{tabular}
\end{table}

%----------------------------------------------------------------------------------------
%   4. TECHNICAL SCAN RESULTS
%----------------------------------------------------------------------------------------
\section{Technical Scan Results}

An external network vulnerability scan was performed against the designated target IP address to identify exposed services and potential vulnerabilities.

\begin{itemize}
    \item \textbf{Scan Target:} \texttt{[Target IP]}
    \item \textbf{Scan Date:} \today
    \item \textbf{Findings:} The scan completed successfully and did not detect any open TCP or UDP ports on the target system.
\end{itemize}

\paragraph{Analysis:} The absence of open ports suggests a well-configured perimeter firewall that enforces a default-deny policy for unsolicited inbound traffic. While this is a positive security posture for the network edge, it provides no visibility into the internal network's security state. An external scan alone is insufficient to assess the overall technical security of the organization.

%----------------------------------------------------------------------------------------
%   5. CONSOLIDATED RISK ASSESSMENT
%----------------------------------------------------------------------------------------
\section{Consolidated Risk Assessment}

This section synthesizes findings from the security control review and technical scan. As the pre-existing risk register was empty, the following risks are derived directly from this assessment.

\begin{table}[h!]
\centering
\caption{Summary of Identified Risks}
\begin{tabular}{p{0.25\linewidth} p{0.5\linewidth} l}
\toprule
\textbf{Risk / Vulnerability} & \textbf{Description} & \textbf{Severity} \\
\midrule
\addlinespace
Lack of MFA on Endpoints and Systems & The absence of MFA on computer logins and sensitive systems allows an attacker with valid credentials (e.g., from a phishing attack) to gain unauthorized access. & \textbf{Critical} \\
\addlinespace
Absence of Security Policies & Without a formal Acceptable Use Policy (AUP), employees lack clear guidelines on the secure use of company assets, increasing the risk of unintentional insider threats and policy violations. & \textbf{High} \\
\addlinespace
No Security Awareness Training Program & Employees are not trained to recognize or respond to common cyber threats like phishing, social engineering, or malware, making them the weakest link in the organization's defense. & \textbf{High} \\
\bottomrule
\end{tabular}
\end{table}

%----------------------------------------------------------------------------------------
%   6. RECOMMENDATIONS
%----------------------------------------------------------------------------------------
\section{Recommendations}

Based on the consolidated risk assessment, the following actions are recommended to improve the cybersecurity posture of \textbf{[Organization Name]}. Recommendations are prioritized by severity.

\subsection*{Immediate Priority (Critical)}
\begin{enumerate}
    \item \textbf{Implement Comprehensive MFA:} Deploy Multi-Factor Authentication (MFA) for all employee computer logins and for access to all systems containing sensitive or critical data. This is the single most effective control to mitigate the risk of credential compromise.
\end{enumerate}

\subsection*{High Priority}
\begin{enumerate}
    \setcounter{enumi}{1}
    \item \textbf{Develop and Enforce an Acceptable Use Policy (AUP):} Create a formal AUP that clearly defines the rules and responsibilities for all users of company IT assets. This policy should be reviewed and signed by all employees.
    \item \textbf{Establish a Security Awareness Training Program:} Implement a mandatory security awareness training program for all new hires and conduct annual refresher training for all staff. The program should cover key topics such as phishing, password security, and data handling.
\end{enumerate}

\subsection*{Informational}
\begin{enumerate}
    \setcounter{enumi}{3}
    \item \textbf{Conduct Internal Vulnerability Scanning:} To complement the strong external posture, schedule regular authenticated vulnerability scans of the internal network. This will provide crucial visibility into potential weaknesses behind the firewall, such as unpatched systems or misconfigurations.
\end{enumerate}

\end{document}
```