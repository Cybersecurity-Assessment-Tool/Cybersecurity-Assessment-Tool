```latex
\documentclass[12pt]{article}

% --- PACKAGES ---
\usepackage[margin=1in]{geometry}
\usepackage{pifont} % For checkmarks and crosses
\usepackage{booktabs} % For professional tables
\usepackage{hyperref} % For hyperlinks
\usepackage{url} % For URL formatting
\usepackage{seqsplit} % To split long monospaced strings
\usepackage{graphicx}
\usepackage{xcolor}

% --- DOCUMENT SETUP ---
\hypersetup{
    colorlinks=true,
    linkcolor=blue,
    filecolor=magenta,      
    urlcolor=cyan,
    pdftitle={Cybersecurity Posture Assessment Report},
    pdfpagemode=FullScreen,
}

\newcommand{\yes}{\ding{51}}
\newcommand{\no}{\ding{55}}

% --- DOCUMENT START ---
\begin{document}

% --- TITLE PAGE ---
\begin{titlepage}
    \centering
    \vfill
    {\Huge\bfseries Cybersecurity Posture Assessment Report\par}
    \vspace{1.5cm}
    {\Large For: \textbf{[Organization Name]}}\par
    \vspace{2cm}
    {\large \today\par}
    \vfill
    {\large Prepared by: Cybersecurity Analyst Group\par}
\end{titlepage}

\tableofcontents
\newpage

% --- EXECUTIVE SUMMARY ---
\section{Executive Summary}
This report provides a comprehensive analysis of the cybersecurity posture for \textbf{[Organization Name]}, based on a combination of network scanning, a security controls questionnaire, and a review of pre-existing risk data. The assessment was conducted to identify vulnerabilities, security gaps, and areas for improvement.

The analysis reveals a mixed security posture. The organization has implemented foundational controls such as Multi-Factor Authentication (MFA) for email and computer access, and conducts annual security training. However, several critical and high-risk gaps were identified that expose the organization to significant threats:

\begin{itemize}
    \item \textbf{Critical Risk - Lack of MFA on Sensitive Systems:} The absence of MFA for accessing sensitive data systems is a critical vulnerability, significantly increasing the risk of unauthorized access and data breach.
    \item \textbf{High Risk - Inadequate Onboarding Training:} New employees do not receive security awareness training, leaving a critical window of vulnerability where they are more susceptible to social engineering and phishing attacks.
    \item \textbf{High Risk - Unencrypted Web Traffic:} The external network scan identified an open port 80 (HTTP), indicating that web traffic is not encrypted. This exposes data, including potential credentials, to eavesdropping and man-in-the-middle attacks.
\end{itemize}

Immediate action is required to address these findings. Recommendations focus on enforcing MFA across all critical systems, integrating security training into the employee onboarding process, and migrating all web services to use encrypted HTTPS.

\section{Organizational Information}
The following details were used as the basis for this assessment. Note that placeholders are used where specific data was not provided.

\begin{itemize}
    \item \textbf{Organization Name:} \textbf{[Organization Name]}
    \item \textbf{Primary Domain:} \texttt{[Domain]}
    \item \textbf{External IP Scanned:} \texttt{[Client IP]}
\end{itemize}

\section{Security Control Review}
The following table summarizes the organization's responses to the security controls questionnaire. Each response has been evaluated against industry best practices. Items marked with \no{} represent significant security gaps that require immediate attention.

\begin{table}[h!]
\centering
\caption{Security Controls Questionnaire Analysis}
\begin{tabular}{p{0.5\textwidth} c p{0.3\textwidth}}
\toprule
\textbf{Control Question} & \textbf{Response} & \textbf{Analyst Notes} \\
\midrule
Do you require MFA to access email? & \yes & Strong control. Protects primary communication channel. \\
\addlinespace
Do you require MFA to log into computers? & \yes & Good practice for endpoint security. \\
\addlinespace
Do you require MFA to access sensitive data systems? & \no & \textbf{Critical Gap.} Leaves high-value assets vulnerable to credential compromise. \\
\addlinespace
Does your organization have an employee acceptable use policy? & \yes & Foundational policy is in place. \\
\addlinespace
Does your organization do security awareness training for new employees? & \no & \textbf{High Risk.} New hires are a primary target for attackers and are untrained. \\
\addlinespace
Does your organization do security awareness training for all employees at least once per year? & \yes & Good practice for maintaining security awareness. \\
\bottomrule
\end{tabular}
\end{table}

\section{Technical Scan Results}
An external network scan was performed on the provided target IP address to identify open ports and exposed services.

\begin{itemize}
    \item \textbf{Target IP:} \texttt{[Target IP]}
    \item \textbf{Scan Date:} Data not provided in scan results.
\end{itemize}

The scan revealed the following open ports:

\begin{table}[h!]
\centering
\caption{Open Port Analysis}
\begin{tabular}{l l l p{0.5\textwidth}}
\toprule
\textbf{Port} & \textbf{State} & \textbf{Service} & \textbf{Notes} \\
\midrule
80/tcp & open & http (assumed) & The presence of an open port 80 indicates unencrypted HTTP traffic. This is a significant security risk, as it allows for the interception of data in transit. All web traffic should be encrypted using HTTPS (port 443). \\
\bottomrule
\end{tabular}
\end{table}

\textit{Note: The provided scan data was minimal. A comprehensive vulnerability scan would be required to identify specific software versions and associated CVEs.}

\section{Risk Assessment}
This section synthesizes findings from the security control review and the technical scan. No valid pre-existing vulnerabilities were provided for analysis; the risks below are derived directly from this assessment.

\begin{table}[h!]
\centering
\caption{Summary of Identified Risks}
\begin{tabular}{p{0.1\textwidth} p{0.4\textwidth} l p{0.3\textwidth}}
\toprule
\textbf{Risk ID} & \textbf{Description} & \textbf{Severity} & \textbf{Affected Elements} \\
\midrule
RISK-001 & Lack of MFA on sensitive systems allows an attacker with stolen credentials to gain direct access to critical data. & \textbf{Critical} & Sensitive data repositories, financial systems, customer databases. \\
\addlinespace
RISK-002 & Unencrypted HTTP traffic allows for eavesdropping and man-in-the-middle attacks, potentially exposing user credentials and sensitive data. & \textbf{High} & Public-facing web server at \texttt{[Target IP]}, all users of the web service. \\
\addlinespace
RISK-003 & New employees are not given security awareness training, making them highly susceptible to phishing and social engineering attacks. & \textbf{High} & All new employees, organizational data, network integrity. \\
\bottomrule
\end{tabular}
\end{table}

\section{Recommendations}
The following actions are recommended to mitigate the identified risks and improve the overall security posture of \textbf{[Organization Name]}.

\subsection{Immediate Actions (0-30 Days)}
\begin{description}
    \item[Remediate RISK-001:] \textbf{Enforce MFA on Sensitive Systems.}
    \begin{itemize}
        \item \textbf{Action:} Immediately begin deployment of MFA across all systems identified as containing sensitive data.
        \item \textbf{Justification:} This is the single most effective control to prevent unauthorized access to critical assets and should be the top priority.
    \end{itemize}
    \item[Remediate RISK-002:] \textbf{Migrate to HTTPS.}
    \begin{itemize}
        \item \textbf{Action:} Install a valid TLS/SSL certificate on the web server at \texttt{[Target IP]}. Configure the server to redirect all HTTP traffic to HTTPS and disable port 80 if possible.
        \item \textbf{Justification:} Encrypting all web traffic protects data integrity and confidentiality, preventing credential theft and session hijacking.
    \end{itemize}
\end{description}

\subsection{Strategic Actions (30-90 Days)}
\begin{description}
    \item[Remediate RISK-003:] \textbf{Implement Onboarding Security Training.}
    \begin{itemize}
        \item \textbf{Action:} Develop or procure a security awareness training module and integrate it into the mandatory onboarding process for all new employees and contractors.
        \item \textbf{Justification:} This closes a major gap in the human firewall and reduces the likelihood of success for phishing and social engineering attacks from day one of a new employee's tenure.
    \end{itemize}
\end{description}

\end{document}
```