```latex
\documentclass[12pt]{article}

% Preamble: Required Packages
\usepackage[margin=1in]{geometry}
\usepackage{pifont} % For checkmarks and crosses
\usepackage{booktabs} % For professional tables
\usepackage{hyperref} % For clickable links
\usepackage{url} % For URL formatting
\usepackage{seqsplit} % For splitting long strings
\usepackage{graphicx} % For logo
\usepackage{fancyhdr} % For header/footer

% Document Metadata
\title{Cybersecurity Posture Assessment Report}
\author{Cybersecurity Analysis Division}
\date{\today}

% Header and Footer Configuration
\pagestyle{fancy}
\fancyhf{} % Clear all header and footer fields
\fancyhead[L]{Cybersecurity Assessment Report}
\fancyhead[R]{\textbf{[Organization Name]}}
\fancyfoot[C]{\thepage}
\renewcommand{\headrulewidth}{0.4pt}
\renewcommand{\footrulewidth}{0.4pt}

\begin{document}

\maketitle
\thispagestyle{empty}
\newpage

\tableofcontents
\newpage

% --- Section 1: Executive Summary ---
\section{Executive Summary}

This report provides a comprehensive analysis of the cybersecurity posture for \textbf{[Organization Name]}, conducted on \today. The assessment synthesizes data from an external network scan, a review of organizational security controls, and an analysis of pre-existing risks.

The external network scan of the perimeter IP address \texttt{[Client IP]} revealed no open ports. This is a positive finding, suggesting a well-hardened network perimeter and a correctly configured firewall.

However, the review of internal security controls identified several critical gaps that significantly elevate the organization's risk profile. Key findings include:
\begin{itemize}
    \item \textbf{Critical Risk:} Lack of Multi-Factor Authentication (MFA) for computer logins, which exposes the organization to severe risks from credential compromise and unauthorized access.
    \item \textbf{High Risk:} Absence of a formal Employee Acceptable Use Policy (AUP), leading to ambiguity in security responsibilities and increasing the likelihood of insider threats.
    \item \textbf{High Risk:} Failure to provide security awareness training to new employees during onboarding, leaving a critical window of vulnerability.
\end{itemize}

While the external defenses appear robust, the identified internal control weaknesses require immediate attention. This report outlines these risks in detail and provides actionable recommendations to mitigate them and strengthen the overall security posture of \textbf{[Organization Name]}.

% --- Section 2: Organizational Information ---
\section{Organizational Information}

This section details the organizational data used as the basis for this assessment. Due to the anonymized nature of the input data, placeholders have been used where necessary.

\begin{itemize}
    \item \textbf{Organization Name:} \textbf{[Organization Name]}
    \item \textbf{Primary Domain:} \texttt{[Domain]}
    \item \textbf{Assessed External IP:} \texttt{[Client IP]}
\end{itemize}

% --- Section 3: Security Control Review ---
\section{Security Control Review}

The following table summarizes the organization's responses to a security controls questionnaire. A checkmark (\ding{51}) indicates a positive control is in place, while a cross (\ding{55}) indicates a control gap that represents a potential risk.

\begin{table}[h!]
\centering
\caption{Organizational Security Controls Questionnaire}
\begin{tabular}{p{0.75\linewidth} c}
\toprule
\textbf{Control Question} & \textbf{Response} \\
\midrule
Do you require MFA to access email? & \ding{51} \\
Do you require MFA to log into computers? & \textbf{\color{red}\ding{55}} \\
Do you require MFA to access sensitive data systems? & \ding{51} \\
Does your organization have an employee acceptable use policy? & \textbf{\color{red}\ding{55}} \\
Does your organization do security awareness training for new employees? & \textbf{\color{red}\ding{55}} \\
Does your organization do security awareness training for all employees at least once per year? & \ding{51} \\
\bottomrule
\end{tabular}
\end{table}

The identified gaps are analyzed further in the Risk Assessment section of this report.

% --- Section 4: Technical Scan Results ---
\section{Technical Scan Results}

An external network vulnerability scan was performed to identify open ports, running services, and potential vulnerabilities on the public-facing infrastructure.

\begin{itemize}
    \item \textbf{Scan Target:} \texttt{[Target IP]}
    \item \textbf{Scan Date:} \today
\end{itemize}

\subsection{Findings}
The scan completed successfully and found \textbf{no open TCP or UDP ports} on the target system.

\subsection{Analysis}
This result is a strong positive indicator. It suggests that a firewall is properly configured to deny all unsolicited inbound traffic, adhering to the security principle of least privilege. A hardened external perimeter significantly reduces the attack surface available to external threat actors.

% --- Section 5: Risk Assessment ---
\section{Risk Assessment}

This section correlates the findings from the security control review and the technical scan. While no technical vulnerabilities were discovered externally, the policy and procedural gaps identified in the questionnaire present a significant risk to the organization.

\begin{table}[h!]
\centering
\caption{Identified Risks and Severity}
\begin{tabular}{p{0.25\linewidth} p{0.5\linewidth} p{0.15\linewidth}}
\toprule
\textbf{Risk Name} & \textbf{Overview} & \textbf{Severity} \\
\midrule
\textbf{Lack of Endpoint MFA} & User computers do not require MFA for login. A single compromised password could grant an attacker full access to an employee's machine and potentially the internal network. & \textbf{Critical} \\
\addlinespace
\textbf{Missing Acceptable Use Policy (AUP)} & The absence of a formal AUP creates ambiguity regarding the secure and acceptable use of company assets. This increases the risk of unintentional data exposure and malicious insider activity. & \textbf{High} \\
\addlinespace
\textbf{Inadequate Onboarding Security Training} & New employees are not provided with security awareness training upon joining. This makes them prime targets for phishing and social engineering attacks before they are familiar with company policies. & \textbf{High} \\
\bottomrule
\end{tabular}
\end{table}

% --- Section 6: Recommendations ---
\section{Recommendations}

Based on the analysis, the following actions are recommended to mitigate the identified risks and improve the overall security posture of \textbf{[Organization Name]}. Recommendations are prioritized by severity.

\begin{enumerate}
    \item \textbf{Implement MFA for All Computer Logins (Critical):}
    \begin{itemize}
        \item \textbf{Action:} Deploy a mandatory Multi-Factor Authentication solution for all employee computer and laptop logins. This is the single most effective control to prevent unauthorized access resulting from stolen credentials.
        \item \textbf{Examples:} Windows Hello for Business, Duo Security, Okta, or similar identity and access management (IAM) solutions.
    \end{itemize}
    \vspace{1em}

    \item \textbf{Develop and Enforce an Acceptable Use Policy (High):}
    \begin{itemize}
        \item \textbf{Action:} Draft a comprehensive AUP that clearly defines the rules for using company networks, devices, and data. This policy must be distributed to all employees, and a formal acknowledgment of receipt and understanding should be required.
    \end{itemize}
    \vspace{1em}

    \item \textbf{Establish a Mandatory New Hire Security Training Program (High):}
    \begin{itemize}
        \item \textbf{Action:} Integrate a security awareness training module into the standard onboarding process for all new employees. This training should be completed before full access to sensitive data or systems is granted. The training should cover phishing, password security, and the new AUP.
    \end{itemize}
\end{enumerate}

\end{document}
```