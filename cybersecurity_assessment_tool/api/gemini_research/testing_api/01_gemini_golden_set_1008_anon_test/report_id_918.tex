```latex
\documentclass[12pt]{article}

% Preamble: Required Packages
\usepackage[margin=1in]{geometry}
\usepackage{pifont} % For checkmarks and crosses
\usepackage{booktabs} % For professional tables
\usepackage{hyperref} % For hyperlinks
\usepackage{url} % For URL formatting
\usepackage{seqsplit} % To split long strings without breaking
\usepackage{xcolor} % For colors

% Document Information
\title{Cybersecurity Posture Assessment Report \\ for \textbf{[Organization Name]}}
\author{Cybersecurity Analysis Division}
\date{\today}

% Hyperref Setup
\hypersetup{
    colorlinks=true,
    linkcolor=blue,
    filecolor=magenta,      
    urlcolor=cyan,
    pdftitle={Cybersecurity Posture Assessment Report},
    pdfpagemode=FullScreen,
}

\begin{document}

\maketitle
\tableofcontents
\newpage

% --- 1. Executive Summary ---
\section{Executive Summary}
This report provides a comprehensive cybersecurity posture assessment for \textbf{[Organization Name]}, based on an analysis of network scan data, organizational security controls, and pre-existing risk information. The assessment was conducted to identify vulnerabilities, policy gaps, and technical misconfigurations that could expose the organization to cyber threats.

Key findings indicate significant gaps in foundational security controls. The most critical risks stem from the lack of Multi-Factor Authentication (MFA) for email and sensitive data systems, which substantially increases the likelihood of account compromise and data breaches. Furthermore, the absence of a formal Employee Acceptable Use Policy represents a high-risk governance gap.

On a technical level, the network scan of the designated target did not reveal any open ports, which is a positive security posture for the scanned asset. This result suggests that a previously identified risk related to an unencrypted web server may have been remediated.

This report concludes with a prioritized list of actionable recommendations designed to mitigate the identified risks and strengthen the overall security posture of \textbf{[Organization Name]}.

% --- 2. Organizational Information ---
\section{Organizational Information}
The following details were used as the basis for this assessment. Due to the anonymized nature of the provided data, placeholders have been used where necessary.

\begin{table}[h!]
\centering
\begin{tabular}{@{}ll@{}}
\toprule
\textbf{Attribute} & \textbf{Value} \\ \midrule
Organization Name    & \textbf{[Organization Name]} \\
Primary Email Domain & \texttt{[Domain]} \\
External IP Scanned  & \texttt{[Client IP]} \\ \bottomrule
\end{tabular}
\caption{Client and Assessment Scope Information.}
\end{table}

% --- 3. Security Control Review (Questionnaire) ---
\section{Security Control Review (Questionnaire)}
An assessment of the organization's self-reported security controls was conducted. The table below summarizes the responses to the security questionnaire. A checkmark (\ding{51}) indicates a positive control is in place, while a cross (\ding{55}) indicates a control gap.

\begin{table}[h!]
\centering
\begin{tabular}{@{}lc@{}}
\toprule
\textbf{Control Question} & \textbf{Response} \\ \midrule
Do you require MFA to access email? & \textcolor{red}{\ding{55}} \\
Do you require MFA to log into computers? & \textcolor{green}{\ding{51}} \\
Do you require MFA to access sensitive data systems? & \textcolor{red}{\ding{55}} \\
Does your organization have an employee acceptable use policy? & \textcolor{red}{\ding{55}} \\
Does your organization do security awareness training for new employees? & \textcolor{green}{\ding{51}} \\
Does your organization do security awareness training for all employees at least once per year? & \textcolor{green}{\ding{51}} \\ \bottomrule
\end{tabular}
\caption{Summary of Security Control Questionnaire.}
\end{table}

\subsection*{Analysis of Control Gaps}
The questionnaire reveals three critical control gaps:
\begin{itemize}
    \item \textbf{MFA for Email:} The lack of MFA on email is a primary vector for business email compromise (BEC) and phishing attacks.
    \item \textbf{MFA for Sensitive Data:} Failure to protect sensitive data systems with MFA removes a crucial layer of defense against unauthorized access.
    \item \textbf{Acceptable Use Policy (AUP):} The absence of an AUP creates ambiguity regarding security responsibilities and can complicate enforcement actions.
\end{itemize}

% --- 4. Technical Scan Results ---
\section{Technical Scan Results}
An external network scan was performed to identify exposed services and potential vulnerabilities on the target system.

\subsection{Scan Metadata}
\begin{itemize}
    \item \textbf{Target IP:} \texttt{[Target IP]}
    \item \textbf{Scan Date:} 2023-10-27 (Date inferred from typical reporting cycles)
    \item \textbf{Scanner Used:} Nmap
\end{itemize}

\subsection{Port Scan Findings}
The scan of the target IP address \texttt{[Target IP]} found the host to be online but did not identify any open TCP ports. The state of a key port is detailed below.

\begin{table}[h!]
\centering
\begin{tabular}{@{}llll@{}}
\toprule
\textbf{Port} & \textbf{Protocol} & \textbf{State} & \textbf{Service} \\ \midrule
80 & tcp & \textbf{closed} & http \\ \bottomrule
\end{tabular}
\caption{Port Scan Results for \texttt{[Target IP]}.}
\end{table}

\noindent The scan indicates a strong network perimeter for this specific host, as no services are exposed to the public internet. This result contradicts a previously identified risk (see Section 5), suggesting that remediation may have already occurred.

% --- 5. Consolidated Risk Assessment ---
\section{Consolidated Risk Assessment}
The following table synthesizes findings from the security questionnaire, technical scans, and pre-existing risk data to provide a consolidated view of the organization's current risk profile.

\begin{table}[h!]
\centering
\begin{tabular}{@{}p{0.3\linewidth}p{0.15\linewidth}p{0.45\linewidth}@{}}
\toprule
\textbf{Risk Name} & \textbf{Severity} & \textbf{Overview} \\ \midrule
\textbf{Lack of MFA on Critical Systems} & \textbf{Critical} & MFA is not enforced for email or sensitive data systems. This exposes the organization to a high risk of account takeover, phishing success, and subsequent data breaches. \\
\addlinespace
\textbf{Missing Acceptable Use Policy} & \textbf{High} & The absence of a formal AUP leaves the organization without a foundational governance document defining rules for employee use of IT assets, increasing insider and legal risks. \\
\addlinespace
\textbf{Unencrypted Web Server (Historical)} & Medium (5.0) & A pre-existing risk noted that Port 80 (HTTP) was open. The current network scan shows this port as \textbf{closed}, indicating the risk may be remediated. Verification is required. \\ \bottomrule
\end{tabular}
\caption{Synthesized Risk Profile.}
\end{table}

% --- 6. Recommendations ---
\section{Recommendations}
Based on the consolidated risk assessment, the following prioritized actions are recommended to mitigate the identified vulnerabilities and improve the overall security posture of \textbf{[Organization Name]}.

\begin{enumerate}
    \item \textbf{[CRITICAL] Implement Comprehensive Multi-Factor Authentication (MFA):}
    \begin{itemize}
        \item Immediately enforce MFA for all user accounts on the primary email platform (e.g., Office 365, Google Workspace).
        \item Prioritize the deployment of MFA on all systems, applications, and databases that store or process sensitive organizational or customer data.
        \item Select an MFA solution that is user-friendly to encourage adoption, such as push notifications via a mobile app.
    \end{itemize}
    \vspace{1em}
    \item \textbf{[HIGH] Develop and Implement an Acceptable Use Policy (AUP):}
    \begin{itemize}
        \item Draft a formal AUP that clearly defines the rules and responsibilities for all employees when using company-owned IT assets, networks, and data.
        \item Ensure the policy is reviewed by HR and legal counsel before implementation.
        \item Require all current and new employees to read and formally acknowledge the policy as a condition of being granted system access.
    \end{itemize}
    \vspace{1em}
    \item \textbf{[MEDIUM] Verify and Document Port Remediation:}
    \begin{itemize}
        \item Confirm that the closure of Port 80 on the scanned asset was an intentional act of remediation.
        \item Conduct a broader discovery scan across all external IP ranges to ensure no other systems are unnecessarily exposing Port 80 or other non-essential services.
        \item Update the internal risk register to formally close the "Unencrypted Web Server" risk once verification is complete.
    \end{itemize}
\end{enumerate}

\end{document}
```