```latex
\documentclass[12pt, a4paper]{article}

% Preamble: Required Packages
\usepackage[margin=1in]{geometry}
\usepackage{pifont} % For checkmarks and crosses
\usepackage{booktabs} % For professional tables
\usepackage{hyperref} % For hyperlinks and metadata
\usepackage{url} % For formatting URLs
\usepackage{seqsplit} % To split long strings in tt font
\usepackage{graphicx} % For logo (placeholder)
\usepackage{xcolor} % For colors

% Document Metadata and Hyperlink Setup
\hypersetup{
    colorlinks=true,
    linkcolor=blue,
    filecolor=magenta,      
    urlcolor=cyan,
    pdftitle={Cybersecurity Posture Assessment Report},
    pdfauthor={Cybersecurity Analyst},
    pdfsubject={Security Analysis},
    pdfkeywords={Cybersecurity, Risk Assessment, Nmap, LaTeX},
    bookmarks=true
}

% Define custom colors for severity
\definecolor{criticalred}{HTML}{D7263D}
\definecolor{highorange}{HTML}{F49D40}
\definecolor{mediumyellow}{HTML}{F4D440}
\definecolor{lowblue}{HTML}{56A3A6}
\definecolor{infogray}{HTML}{808080}

% Custom commands for severity
\newcommand{\sevCRITICAL}[1]{\textcolor{criticalred}{\textbf{#1}}}
\newcommand{\sevHIGH}[1]{\textcolor{highorange}{\textbf{#1}}}
\newcommand{\sevMEDIUM}[1]{\textcolor{mediumyellow}{\textbf{#1}}}
\newcommand{\sevLOW}[1]{\textcolor{lowblue}{\textbf{#1}}}

% Checkmark and Crossmark shortcuts
\newcommand{\cmark}{\ding{51}}
\newcommand{\xmark}{\ding{55}}

\begin{document}

% --- Title Page ---
\begin{titlepage}
    \centering
    \vspace*{1cm}
    
    \Huge{\textbf{Cybersecurity Posture Assessment Report}}
    
    \vspace{1.5cm}
    
    \Large{\textbf{Prepared for:}}
    
    \vspace{0.5cm}
    
    \Huge{\textbf{[Organization Name]}}
    
    \vfill
    
    \Large{\textbf{Date of Report:}}
    
    \vspace{0.5cm}
    
    \Large{\today}
    
    \vspace{1.5cm}
    
    \large{\textit{This report contains sensitive information and should be handled with care.}}
    
\end{titlepage}

\tableofcontents
\newpage

% --- Section 1: Executive Summary ---
\section{Executive Summary}
This report provides a comprehensive cybersecurity assessment for \textbf{[Organization Name]}. The analysis is based on a combination of technical network scanning, a review of organizational security controls via a questionnaire, and an evaluation of pre-existing risk documentation.

The assessment reveals several \sevCRITICAL{critical} and \sevHIGH{high-risk} vulnerabilities that require immediate attention. Key findings include:
\begin{itemize}
    \item \textbf{Critical Access Control Gaps:} Multi-Factor Authentication (MFA) is not enforced for computer logins or access to sensitive data systems. This significantly increases the risk of unauthorized access from compromised credentials.
    \item \textbf{Insufficient Employee Training:} The organization lacks a formal security awareness training program for new or existing employees, making it highly susceptible to phishing and social engineering attacks.
    * \textbf{Exposed Management Services:} An external scan identified an open SSH port (22), a common target for attackers. When combined with weak authentication policies, this poses a direct threat to the network perimeter.
    \item \textbf{Pre-existing Critical Risk:} A documented vulnerability, "Localhost Exposed," with a CVSS score of 10.0, indicates a severe, unmitigated threat that must be prioritized for immediate remediation.
\end{itemize}
The overall security posture is considered poor due to these fundamental gaps in security controls. This report provides specific, actionable recommendations to mitigate the identified risks and strengthen the organization's defenses.

% --- Section 2: Organizational Information ---
\section{Organizational Information}
The following details were used as the basis for this assessment.
\begin{itemize}
    \item \textbf{Organization Name:} \textbf{[Organization Name]}
    \item \textbf{Primary Domain:} \seqsplit{\texttt{[Domain]}}
    \item \textbf{Monitored External IP:} \seqsplit{\texttt{[Client IP]}}
\end{itemize}

% --- Section 3: Security Control Review ---
\section{Security Control Review (Questionnaire Analysis)}
An analysis of the security questionnaire reveals significant gaps in foundational security controls. "No" answers indicate a failure to meet baseline security practices and are flagged as high-risk or critical deficiencies.

\begin{table}[h!]
\centering
\caption{Security Controls Questionnaire Results}
\label{tab:controls}
\begin{tabular}{p{0.6\linewidth} c l}
\toprule
\textbf{Control Question} & \textbf{Status} & \textbf{Assessment} \\
\midrule
Do you require MFA to access email? & \cmark & Implemented \\
Do you require MFA to log into computers? & \xmark & \sevHIGH{High Risk Gap} \\
Do you require MFA to access sensitive data systems? & \xmark & \sevCRITICAL{Critical Gap} \\
Does your organization have an employee acceptable use policy? & \cmark & Implemented \\
Does your organization do security awareness training for new employees? & \xmark & \sevHIGH{High Risk Gap} \\
Does your organization do security awareness training for all employees at least once per year? & \xmark & \sevHIGH{High Risk Gap} \\
\bottomrule
\end{tabular}
\end{table}

% --- Section 4: Technical Scan Results ---
\section{Technical Scan Results}
An external network scan was performed on the target IP address to identify accessible services.
\begin{itemize}
    \item \textbf{Target IP Address:} \seqsplit{\texttt{[Target IP]}}
    \item \textbf{Scan Date:} Data provided on \today
\end{itemize}

The scan revealed the following open port:

\begin{table}[h!]
\centering
\caption{Open Ports Detected on Target IP}
\label{tab:ports}
\begin{tabular}{l l l l}
\toprule
\textbf{Port} & \textbf{State} & \textbf{Service (Inferred)} & \textbf{Product/Version} \\
\midrule
22/tcp & open & SSH & Not Detected \\
\bottomrule
\end{tabular}
\end{table}

\paragraph{Analysis:} The presence of an open Secure Shell (SSH) port is a significant finding. SSH is a powerful administrative protocol, and its exposure to the public internet makes it a primary target for brute-force attacks and exploitation of vulnerabilities. The risk is severely amplified by the lack of MFA for computer logins, as a single compromised password could lead to a full system compromise.

% --- Section 5: Consolidated Risk Assessment ---
\section{Consolidated Risk Assessment}
The following table synthesizes findings from the questionnaire, technical scan, and pre-existing risk data into a consolidated list of security risks.

\begin{table}[h!]
\centering
\caption{Summary of Identified Risks}
\label{tab:risks}
\begin{tabular}{p{0.3\linewidth} p{0.5\linewidth} l}
\toprule
\textbf{Risk Title} & \textbf{Description} & \textbf{Severity} \\
\midrule
\textbf{Localhost Exposed} & Pre-existing documented vulnerability with the highest possible CVSS score. Details suggest a critical service is improperly exposed. & \sevCRITICAL{Critical} \\
\addlinespace
\textbf{No MFA on Sensitive Systems} & Lack of MFA for sensitive data access allows an attacker with stolen credentials to directly access and exfiltrate core business data. & \sevCRITICAL{Critical} \\
\addlinespace
\textbf{No MFA on Endpoints} & Lack of MFA for computer logins allows for trivial lateral movement and system takeover if an employee's password is compromised. & \sevHIGH{High} \\
\addlinespace
\textbf{Exposed SSH Service} & The administrative SSH port is open to the public internet, inviting automated brute-force attacks and exploitation attempts. & \sevHIGH{High} \\
\addlinespace
\textbf{No Security Awareness Training} & Employees are not trained to recognize or report security threats, making the organization highly vulnerable to phishing and social engineering. & \sevHIGH{High} \\
\bottomrule
\end{tabular}
\end{table}

% --- Section 6: Recommendations ---
\section{Recommendations}
To address the identified risks, the following actions are recommended, prioritized by severity.

\subsection*{Priority 1: Critical Risks (Immediate Action Required)}
\begin{enumerate}
    \item \textbf{Remediate "Localhost Exposed" Vulnerability:} Immediately investigate the pre-existing risk documented as "Localhost Exposed" (CVSS 10.0). This likely involves reconfiguring a service or firewall to ensure it is not accessible from the internet. This is the highest priority action item.
    \item \textbf{Implement MFA for Sensitive Data Access:} Deploy a mandatory MFA policy for all applications, databases, and systems that contain sensitive or critical business data. This is the most effective control to prevent data breaches resulting from credential theft.
\end{enumerate}

\subsection*{Priority 2: High Risks (Action within 30-60 Days)}
\begin{enumerate}
    \setcounter{enumi}{2} % Continue numbering
    \item \textbf{Enforce MFA for All Computer Logins:} Deploy MFA for all employee and administrative logins to workstations and servers. This hardens endpoints against unauthorized use.
    \item \textbf{Restrict SSH Access:} Reconfigure the network firewall to restrict access to port 22. Access should only be permitted from a whitelist of trusted IP addresses (e.g., corporate offices or a bastion host). Furthermore, disable password-based authentication for SSH in favor of public key cryptography.
    \item \textbf{Establish a Security Awareness Program:} Implement a security awareness training program immediately. All new hires must complete the training as part of their onboarding, and all existing employees must complete it annually. The training must cover phishing, password hygiene, and acceptable use policies.
\end{enumerate}

\end{document}
```