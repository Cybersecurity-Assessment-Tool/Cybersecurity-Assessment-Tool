```latex
\documentclass[12pt]{article}

% --- PACKAGES ---
\usepackage[margin=1in]{geometry}
\usepackage{pifont} % For checkmarks and crosses
\usepackage{booktabs} % For professional tables
\usepackage{hyperref} % For clickable links
\usepackage{url} % For URL formatting
\usepackage{seqsplit} % For splitting long strings
\usepackage{graphicx}
\usepackage{xcolor}
\usepackage{fancyhdr}

% --- DOCUMENT METADATA ---
\hypersetup{
    colorlinks=true,
    linkcolor=blue,
    filecolor=magenta,      
    urlcolor=cyan,
    pdftitle={Cybersecurity Posture Assessment Report},
    pdfauthor={Automated Security Analysis System},
    pdfsubject={Security Report},
    pdfkeywords={Cybersecurity, Analysis, LaTeX},
}

% --- CUSTOM COMMANDS & SETTINGS ---
\pagestyle{fancy}
\fancyhf{}
\fancyhead[L]{Cybersecurity Posture Assessment}
\fancyhead[R]{\textbf{[Organization Name]}}
\fancyfoot[C]{\thepage}
\renewcommand{\headrulewidth}{0.4pt}
\renewcommand{\footrulewidth}{0.4pt}

% Define colors for risk levels
\definecolor{criticalred}{HTML}{D7263D}
\definecolor{highorange}{HTML}{F49D40}
\definecolor{mediumyellow}{HTML}{F3E96B}
\definecolor{lowgreen}{HTML}{A7C957}
\definecolor{infoblue}{HTML}{68A2B9}

% --- DOCUMENT START ---
\begin{document}

% --- TITLE PAGE ---
\begin{titlepage}
    \centering
    \vspace*{1cm}
    \includegraphics[width=0.3\textwidth]{example-image-a} % Placeholder for company logo
    
    \vspace{1.5cm}
    
    \Huge
    \textbf{Cybersecurity Posture Assessment Report}
    
    \vspace{1.5cm}
    
    \Large
    Prepared for: \textbf{[Organization Name]}
    
    \vspace{2cm}
    
    \large
    Date of Report: \today
    
    \vfill
    
    \large
    \textit{This report contains sensitive information and should be handled with care.}
    
\end{titlepage}

\tableofcontents
\newpage

% --- EXECUTIVE SUMMARY ---
\section{Executive Summary}
This report provides a comprehensive assessment of the cybersecurity posture for \textbf{[Organization Name]}, synthesizing data from a network vulnerability scan, a security controls questionnaire, and a review of pre-existing risks.

The analysis reveals a significant disparity between the organization's external network security and its internal security policies. While the external scan of the target IP address (\texttt{[Target IP]}) did not identify any open, vulnerable ports, the security controls review uncovered critical gaps. 

Key findings indicate a high risk of compromise stemming from inadequate identity and access management and a lack of foundational security policies. Specifically, the absence of Multi-Factor Authentication (MFA) for email and computer access, the lack of an Acceptable Use Policy (AUP), and no security training for new hires represent urgent vulnerabilities. These policy-based weaknesses create a substantial risk of social engineering, phishing, and insider threats, which could bypass an otherwise secure network perimeter.

Immediate action is recommended to address these critical policy and procedural gaps to reduce the likelihood of a security incident.

% --- ORGANIZATIONAL INFORMATION ---
\section{Organizational Information}
The following details were used as the basis for this assessment. Where information was not provided, placeholders have been used.

\begin{table}[h!]
\centering
\caption{Client Information}
\begin{tabular}{@{}ll@{}}
\toprule
\textbf{Attribute} & \textbf{Value} \\ \midrule
Organization Name & \textbf{[Organization Name]} \\
Primary Domain & \texttt{[Domain]} \\
External IP Scanned & \texttt{[Client IP]} \\ \bottomrule
\end{tabular}
\end{table}

% --- SECURITY CONTROL REVIEW ---
\section{Security Control Review}
The following table summarizes the organization's responses to a security controls questionnaire. "No" answers indicate significant gaps in the security framework and are highlighted for immediate attention.

\begin{table}[h!]
\centering
\caption{Security Controls Questionnaire Analysis}
\begin{tabular}{@{}p{0.5\linewidth}ccc@{}}
\toprule
\textbf{Control Question} & \textbf{Response} & \textbf{Analyst Note} \\ \midrule
Do you require MFA to access email? & \ding{55} & \textcolor{criticalred}{\textbf{Critical Risk}} \\
Do you require MFA to log into computers? & \ding{55} & \textcolor{highorange}{\textbf{High Risk}} \\
Do you require MFA to access sensitive data systems? & \ding{51} & Best Practice Met \\
Does your organization have an employee acceptable use policy? & \ding{55} & \textcolor{highorange}{\textbf{High Risk}} \\
Does your organization do security awareness training for new employees? & \ding{55} & \textcolor{highorange}{\textbf{High Risk}} \\
Does your organization do security awareness training for all employees at least once per year? & \ding{51} & Best Practice Met \\ \bottomrule
\end{tabular}
\end{table}

\subsection*{Analysis of Control Gaps}
\begin{itemize}
    \item \textbf{Lack of MFA for Email:} This is a critical vulnerability. Email accounts are the primary target for phishing attacks and are often the key to resetting passwords for other critical systems. A compromise here could lead to a widespread breach.
    \item \textbf{Lack of MFA for Computers:} The absence of MFA on endpoints (desktops/laptops) increases the risk of unauthorized access if user credentials are stolen.
    \item \textbf{No Acceptable Use Policy (AUP):} An AUP is a foundational policy that sets clear expectations for employees regarding the use of company assets. Its absence can lead to inconsistent security practices and ambiguity during a security incident.
    \item \textbf{No New Hire Security Training:} New employees are often targeted by attackers. Failing to provide security training during onboarding leaves a critical window of vulnerability open.
\end{itemize}

% --- TECHNICAL SCAN RESULTS ---
\section{Technical Scan Results}
An external network scan was performed on the target IP address to identify open ports and exposed services.

\subsection{Scan Details}
\begin{itemize}
    \item \textbf{Target IP:} \texttt{[Target IP]}
    \item \textbf{Scanner Used:} Nmap
    \item \textbf{Scan Date:} Not specified in scan data
\end{itemize}

\subsection{Findings}
The scan revealed that the target host is online, but no open ports were discovered. The status of a key port is detailed below.

\begin{table}[h!]
\centering
\caption{Port Scan Summary for \texttt{[Target IP]}}
\begin{tabular}{@{}llll@{}}
\toprule
\textbf{Port} & \textbf{Protocol} & \textbf{State} & \textbf{Service/Note} \\ \midrule
80 & TCP & \textbf{closed} & HTTP \\ \bottomrule
\end{tabular}
\end{table}

\subsection*{Analysis of Technical Findings}
The external perimeter of the scanned IP address appears to be hardened, with no services exposed to the public internet. The finding that port 80 (HTTP) is closed is a positive security control. This result directly contradicts a pre-existing risk documented in the next section, suggesting that the risk may have been remediated.

% --- RISK ASSESSMENT SUMMARY ---
\section{Risk Assessment Summary}
This section correlates findings from the security control review, the technical scan, and pre-existing risk data.

\begin{table}[h!]
\centering
\caption{Consolidated Risk Register}
\begin{tabular}{@{}p{0.3\linewidth}p{0.5\linewidth}l@{}}
\toprule
\textbf{Risk Name} & \textbf{Overview} & \textbf{Severity} \\ \midrule
\textbf{No MFA on Email} & Lack of MFA on email exposes the organization to account takeover and phishing. & \textcolor{criticalred}{\textbf{Critical}} \\
\addlinespace
\textbf{No MFA on Endpoints} & Lack of MFA on computer logins weakens endpoint security against stolen credentials. & \textcolor{highorange}{\textbf{High}} \\
\addlinespace
\textbf{No Acceptable Use Policy} & Absence of a formal AUP creates inconsistent security behavior and legal ambiguity. & \textcolor{highorange}{\textbf{High}} \\
\addlinespace
\textbf{No Onboarding Security Training} & New employees are not trained on security best practices, making them susceptible to attacks. & \textcolor{highorange}{\textbf{High}} \\
\addlinespace
\textbf{Unencrypted Web Server} & \textit{(From Input 3)} Port 80 is open, exposing unencrypted web traffic. \textbf{Note:} Our scan found this port to be \textbf{closed}, indicating this risk may be remediated. & \textcolor{mediumyellow}{\textbf{Medium}} \\
\bottomrule
\end{tabular}
\end{table}

% --- RECOMMENDATIONS ---
\section{Recommendations}
Based on the analysis, we recommend the following actions, prioritized by severity.

\subsection{Priority 1: Critical}
\begin{enumerate}
    \item \textbf{Implement MFA for Email Immediately:}
    \begin{itemize}
        \item \textbf{Action:} Enforce MFA for all user accounts on the primary email system (\texttt{[Domain]}).
        \item \textbf{Justification:} This is the single most effective control to prevent email account takeovers, which are a primary vector for major security breaches.
    \end{itemize}
\end{enumerate}

\subsection{Priority 2: High}
\begin{enumerate}
    \setcounter{enumi}{1}
    \item \textbf{Deploy MFA for Endpoint Logins:}
    \begin{itemize}
        \item \textbf{Action:} Roll out MFA for all company-managed computer logins (desktops and laptops).
        \item \textbf{Justification:} Protects against unauthorized physical or remote access resulting from compromised credentials.
    \end{itemize}
    \item \textbf{Develop and Implement an Acceptable Use Policy (AUP):}
    \begin{itemize}
        \item \textbf{Action:} Draft, approve, and distribute an AUP that all employees must read and acknowledge. The policy should cover data handling, internet usage, and software installation.
        \item \textbf{Justification:} Establishes a clear baseline for secure employee behavior and provides a framework for enforcement.
    \end{itemize}
    \item \textbf{Integrate Security Training into New Hire Onboarding:}
    \begin{itemize}
        \item \textbf{Action:} Create a mandatory security awareness training module for all new employees to complete within their first week.
        \item \textbf{Justification:} Reduces the risk posed by new hires who are unfamiliar with company policies and are common targets for social engineering.
    \end{itemize}
\end{enumerate}

\subsection{Priority 3: Verification}
\begin{enumerate}
    \setcounter{enumi}{4}
    \item \textbf{Verify and Close "Unencrypted Web Server" Risk:}
    \begin{itemize}
        \item \textbf{Action:} Internally confirm that the service on port 80 is no longer active. Formally update the organization's risk register to reflect that this vulnerability is closed.
        \item \textbf{Justification:} Ensures that risk documentation is accurate and reflects the current security posture.
    \end{itemize}
\end{enumerate}

\end{document}
```