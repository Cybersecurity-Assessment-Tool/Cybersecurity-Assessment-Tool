```latex
\documentclass[12pt]{article}

% --- PACKAGES ---
\usepackage[margin=1in]{geometry}
\usepackage{pifont} % For checkmarks and crosses
\usepackage{booktabs} % For professional tables
\usepackage{hyperref} % For hyperlinks
\usepackage{url} % For URL formatting
\usepackage{seqsplit} % For splitting long strings in tt font
\usepackage{graphicx}
\usepackage{xcolor}

% --- DOCUMENT INFORMATION ---
\title{Cybersecurity Posture Assessment Report}
\author{Cybersecurity Analyst}
\date{\today}

% --- HYPERREF SETUP ---
\hypersetup{
    colorlinks=true,
    linkcolor=blue,
    filecolor=magenta,      
    urlcolor=cyan,
    pdftitle={Cybersecurity Posture Assessment Report},
    pdfpagemode=FullScreen,
}

% --- COMMANDS ---
\newcommand{\yes}{\ding{51}}
\newcommand{\no}{\ding{55}}
\newcommand{\orgname}{\textbf{[Organization Name]}}
\newcommand{\clientdomain}{\texttt{[Domain]}}
\newcommand{\clientip}{\texttt{[Client IP]}}
\newcommand{\targetip}{\texttt{[Target IP]}}

\begin{document}

\maketitle
\thispagestyle{empty}
\newpage

\tableofcontents
\newpage

% ===================================================================
% 1. EXECUTIVE SUMMARY
% ===================================================================
\section{Executive Summary}

This report provides a comprehensive analysis of the cybersecurity posture for \orgname. The assessment is based on a combination of organizational data, a network vulnerability scan, and a review of pre-existing risks.

The analysis reveals several critical and high-risk security gaps that require immediate attention. Key findings include:
\begin{itemize}
    \item \textbf{Lack of Multi-Factor Authentication (MFA) on Email:} The absence of MFA for email access presents a critical risk, making user accounts highly susceptible to compromise through phishing and credential stuffing attacks.
    \item \textbf{Deficient Security Awareness Training:} The organization does not provide security awareness training for new or existing employees. This significantly increases the likelihood of human error leading to security incidents.
    \item \textbf{Exposed Network Services:} An external network scan identified an open SSH port (22), which could serve as an entry point for attackers if not properly secured.
    \item \textbf{Pre-existing Critical Vulnerability:} A critical vulnerability, "Localhost Exposed," with a CVSS score of 10.0, was noted in existing risk data. This indicates a severe misconfiguration that must be remediated immediately.
\end{itemize}

The combination of these factors places the organization at a high risk of a significant security breach. This report outlines detailed findings and provides actionable recommendations to mitigate these risks and improve the overall security posture.

% ===================================================================
% 2. ORGANIZATIONAL INFORMATION
% ===================================================================
\section{Organizational Information}

This section details the organizational information used for this assessment. The data has been anonymized as per the engagement requirements.

\begin{itemize}
    \item \textbf{Organization Name:} \orgname
    \item \textbf{Email Domain:} \clientdomain
    \item \textbf{External IP Address:} \clientip
\end{itemize}

% ===================================================================
% 3. SECURITY CONTROL REVIEW (QUESTIONNAIRE)
% ===================================================================
\section{Security Control Review}

The following table summarizes the organization's responses to a security controls questionnaire. Items marked with a red cross (\no) indicate significant gaps in the security framework and are discussed in the Risk Assessment section.

\begin{table}[h!]
\centering
\caption{Security Controls Questionnaire Results}
\begin{tabular}{p{0.7\linewidth} c c}
\toprule
\textbf{Control Question} & \textbf{Status} & \textbf{Risk Level} \\
\midrule
Do you require MFA to access email? & \textcolor{red}{\no} & \textcolor{red}{\textbf{Critical}} \\
Do you require MFA to log into computers? & \textcolor{green}{\yes} & Low \\
Do you require MFA to access sensitive data systems? & \textcolor{green}{\yes} & Low \\
Does your organization have an employee acceptable use policy? & \textcolor{green}{\yes} & Low \\
Does your organization do security awareness training for new employees? & \textcolor{red}{\no} & \textcolor{red}{\textbf{High}} \\
Does your organization do security awareness training for all employees at least once per year? & \textcolor{red}{\no} & \textcolor{red}{\textbf{High}} \\
\bottomrule
\end{tabular}
\end{table}

% ===================================================================
% 4. TECHNICAL SCAN RESULTS
% ===================================================================
\section{Technical Scan Results}

An external network scan was performed to identify exposed services and potential vulnerabilities. The scan was conducted using Nmap.

\begin{itemize}
    \item \textbf{Target IP Address:} \targetip
    \item \textbf{Scan Date:} Not specified in scan data. Report generated on \today.
    \item \textbf{Host Status:} Up
\end{itemize}

\subsection{Open Ports}
The following table details the open ports discovered on the target system.

\begin{table}[h!]
\centering
\caption{Discovered Open Ports}
\begin{tabular}{c c l l}
\toprule
\textbf{Port} & \textbf{State} & \textbf{Service (Inferred)} & \textbf{Notes} \\
\midrule
22/tcp & Open & SSH & No version information was available. Exposed SSH \\
& & & can be a target for brute-force attacks. \\
\bottomrule
\end{tabular}
\end{table}

% ===================================================================
% 5. RISK ASSESSMENT
% ===================================================================
\section{Risk Assessment}

This section synthesizes findings from the security control review, technical scan, and pre-existing risk data into a consolidated list of identified risks.

\begin{table}[h!]
\centering
\caption{Summary of Identified Risks}
\begin{tabular}{p{0.3\linewidth} p{0.15\linewidth} p{0.45\linewidth}}
\toprule
\textbf{Risk Name} & \textbf{Severity} & \textbf{Description} \\
\midrule
\textbf{Localhost Exposed} & \textbf{Critical (10.0)} & A pre-existing and unmitigated vulnerability indicating a service intended for internal use is likely exposed to the internet. This represents a severe and immediate threat. \\
\addlinespace
\textbf{No MFA on Email} & \textbf{Critical} & The lack of MFA on the primary communication platform makes the organization highly vulnerable to account takeovers via phishing, leading to data breaches and further internal compromise. \\
\addlinespace
\textbf{No Security Awareness Training} & \textbf{High} & Without training, employees are unprepared to identify and respond to social engineering and phishing attacks, making them the weakest link in the organization's defense. \\
\addlinespace
\textbf{Exposed SSH Service} & \textbf{High} & The SSH port is open to the internet. Without strong controls (e.g., key-based authentication, IP whitelisting), it is a prime target for brute-force and credential stuffing attacks. \\
\bottomrule
\end{tabular}
\end{table}

% ===================================================================
% 6. RECOMMENDATIONS
% ===================================================================
\section{Recommendations}

The following actions are recommended to mitigate the identified risks. Recommendations are prioritized based on severity.

\subsection{Critical Priority}
\begin{enumerate}
    \item \textbf{Remediate "Localhost Exposed" Vulnerability:}
    \begin{itemize}
        \item \textbf{Action:} Immediately investigate the "Localhost Exposed" finding. Identify the misconfigured service and restrict its access to the internal network only. This is the highest priority action.
    \end{itemize}

    \item \textbf{Implement MFA on Email Accounts:}
    \begin{itemize}
        \item \textbf{Action:} Enforce mandatory MFA for all user accounts on the email system (\clientdomain).
        \item \textbf{Impact:} Drastically reduces the risk of email account compromise.
    \end{itemize}
\end{enumerate}

\subsection{High Priority}
\begin{enumerate}
    \setcounter{enumi}{2} % Continue numbering
    \item \textbf{Establish a Security Awareness Training Program:}
    \begin{itemize}
        \item \textbf{Action:} Procure and implement a security awareness training solution. Mandate training for all new hires upon onboarding and for all staff on an annual basis. Conduct regular phishing simulations to reinforce learning.
        \item \textbf{Impact:} Builds a human firewall and reduces susceptibility to social engineering.
    \end{itemize}

    \item \textbf{Secure the Exposed SSH Service:}
    \begin{itemize}
        \item \textbf{Action:} Review the business need for external SSH access to \targetip.
        \begin{itemize}
            \item If not required, block port 22 at the firewall.
            \item If required, implement strong security controls: disable password-based authentication in favor of public key cryptography, restrict access to specific source IP addresses (IP whitelisting), and implement an intrusion detection tool like Fail2Ban.
        \end{itemize}
        \item \textbf{Impact:} Hardens a potential entry point against unauthorized access.
    \end{itemize}
\end{enumerate}

\end{document}
```