```latex
\documentclass[12pt]{article}

% ----------------------------------------------------------------------
% PREAMBLE
% ----------------------------------------------------------------------

% Package for page geometry
\usepackage[a4paper, margin=1in]{geometry}

% Package for checkmarks and crosses
\usepackage{pifont}

% Package for professional tables
\usepackage{booktabs}

% Package for hyperlinks and URLs
\usepackage{hyperref}
\usepackage{url}

% Package for splitting long sequences of text (like URLs or hashes)
\usepackage{seqsplit}

% Document metadata
\title{Cybersecurity Assessment Report}
\author{Cybersecurity Analysis Division}
\date{\today}

% Hyperref setup for PDF metadata
\hypersetup{
    pdftitle={Cybersecurity Assessment Report},
    pdfauthor={Cybersecurity Analysis Division},
    pdfsubject={Security Analysis},
    pdfkeywords={Cybersecurity, Risk Assessment, Network Scan},
    colorlinks=true,
    linkcolor=black,
    urlcolor=blue,
}

% ----------------------------------------------------------------------
% DOCUMENT START
% ----------------------------------------------------------------------

\begin{document}

\maketitle
\thispagestyle{empty}
\newpage

\tableofcontents
\thispagestyle{empty}
\newpage

% ----------------------------------------------------------------------
% SECTION 1: EXECUTIVE SUMMARY
% ----------------------------------------------------------------------
\section{Executive Summary}
\setcounter{page}{1}

This report presents a cybersecurity assessment for \textbf{[Organization Name]}, combining an analysis of organizational security controls, a technical network scan, and a review of pre-existing risks.

The assessment identified two critical security gaps related to identity and access management. The lack of mandatory Multi-Factor Authentication (MFA) for both email access and computer logins exposes the organization to a high risk of account compromise, data breaches, and ransomware attacks. These policy-based vulnerabilities are currently the most significant threats to the organization's security posture.

On a positive note, the technical network scan confirmed that a previously identified medium-severity risk, an "Unencrypted Web Server" on Port 80, has been successfully remediated. The scan found no open ports on the target system, indicating a strong network perimeter at the time of the assessment.

Our primary recommendations focus on the immediate implementation of MFA across all critical systems to mitigate the identified access control weaknesses.

% ----------------------------------------------------------------------
% SECTION 2: ORGANIZATIONAL INFORMATION
% ----------------------------------------------------------------------
\section{Organizational Information}

This assessment was conducted for the following entity. The information provided was anonymized.

\begin{itemize}
    \item \textbf{Organization Name:} \textbf{[Organization Name]}
    \item \textbf{Primary Email Domain:} \texttt{[Domain]}
    \item \textbf{External IP Address Scanned:} \texttt{[Client IP]}
\end{itemize}

% ----------------------------------------------------------------------
% SECTION 3: SECURITY CONTROL REVIEW
% ----------------------------------------------------------------------
\section{Security Control Review}

The following table summarizes the organization's responses to a security controls questionnaire. The analysis highlights gaps in current policies and procedures.

\begin{table}[h!]
\centering
\caption{Security Controls Questionnaire Results}
\begin{tabular}{p{0.75\linewidth} c}
\toprule
\textbf{Control Question} & \textbf{Response} \\
\midrule
Do you require MFA to access email? & \ding{55} \\
Do you require MFA to log into computers? & \ding{55} \\
Do you require MFA to access sensitive data systems? & \ding{51} \\
Does your organization have an employee acceptable use policy? & \ding{51} \\
Does your organization do security awareness training for new employees? & \ding{51} \\
Does your organization do security awareness training for all employees at least once per year? & \ding{51} \\
\bottomrule
\end{tabular}
\end{table}

\subsection*{Analysis of Controls}
The organization demonstrates a strong foundation in security policy and awareness training. However, the absence of MFA for email and computer logins represents a critical vulnerability. Email accounts are a primary target for phishing and business email compromise (BEC) attacks, while unprotected computer logins can lead to complete system takeover. These two gaps significantly increase the organization's risk profile.

% ----------------------------------------------------------------------
% SECTION 4: TECHNICAL SCAN RESULTS
% ----------------------------------------------------------------------
\section{Technical Scan Results}

An external network scan was performed on the client's provided IP address to identify accessible services and potential vulnerabilities.

\begin{itemize}
    \item \textbf{Target IP Address:} \texttt{[Target IP]}
    \item \textbf{Scan Date:} \today
\end{itemize}

\begin{table}[h!]
\centering
\caption{Nmap Scan Results - Open Ports}
\begin{tabular}{l l l l}
\toprule
\textbf{Port} & \textbf{State} & \textbf{Service} & \textbf{Product / Version} \\
\midrule
80 & closed & http & N/A \\
\multicolumn{4}{l}{\textit{Note: The scan found no other open ports.}} \\
\bottomrule
\end{tabular}
\end{table}

\subsection*{Analysis of Scan Results}
The technical scan of the target system revealed a secure external perimeter at the time of testing. No open ports were discovered.

Crucially, this scan confirms that Port 80 (HTTP) is now closed. This directly addresses a previously identified risk, "Unencrypted Web Server," which is now considered remediated. This is a significant security improvement.

% ----------------------------------------------------------------------
% SECTION 5: RISK ASSESSMENT
% ----------------------------------------------------------------------
\section{Risk Assessment}

This section synthesizes findings from the security control review, the technical scan, and pre-existing risk data into a consolidated list.

\begin{table}[h!]
\centering
\caption{Consolidated Risk Register}
\begin{tabular}{p{0.25\linewidth} p{0.4\linewidth} l l}
\toprule
\textbf{Risk Name} & \textbf{Description} & \textbf{Severity} & \textbf{Status} \\
\midrule
\textbf{Lack of MFA for Email Access} & Failure to enforce MFA on email accounts allows for account takeover with only a compromised password. & \textbf{High} & \textbf{Newly Identified} \\
\addlinespace
\textbf{Lack of MFA for Endpoint Login} & Failure to enforce MFA on computer logins allows unauthorized access via stolen credentials, undermining network security. & \textbf{High} & \textbf{Newly Identified} \\
\addlinespace
Unencrypted Web Server & Port 80 was previously open, exposing the potential for unencrypted web traffic and information disclosure. & Medium & \textbf{Remediated} \\
\bottomrule
\end{tabular}
\end{table}

% ----------------------------------------------------------------------
% SECTION 6: RECOMMENDATIONS
% ----------------------------------------------------------------------
\section{Recommendations}

Based on the analysis, the following actions are recommended to improve the organization's security posture. Recommendations are prioritized by criticality.

\subsection*{Priority 1: Implement MFA for Email (Critical)}
\begin{itemize}
    \item \textbf{Action:} Enforce mandatory Multi-Factor Authentication for all user mailboxes. This is the single most effective control to prevent business email compromise and phishing-related account takeovers.
    \item \textbf{Timeline:} Immediate (within 30 days).
\end{itemize}

\subsection*{Priority 2: Implement MFA for Endpoint Logins (Critical)}
\begin{itemize}
    \item \textbf{Action:} Deploy and enforce MFA for all company-managed workstations and laptops (Windows, macOS, etc.). This prevents unauthorized users with stolen credentials from accessing the internal network.
    \item \textbf{Timeline:} Immediate (within 30 days).
\end{itemize}

\subsection*{Priority 3: Update Risk Register (Administrative)}
\begin{itemize}
    \item \textbf{Action:} Formally document the closure of the "Unencrypted Web Server" risk in the organization's internal risk register. This ensures that documentation accurately reflects the current, improved security posture.
    \item \textbf{Timeline:} Next review cycle (within 90 days).
\end{itemize}

% ----------------------------------------------------------------------
% DOCUMENT END
% ----------------------------------------------------------------------
\end{document}
```