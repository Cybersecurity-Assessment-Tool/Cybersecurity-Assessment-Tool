```latex
\documentclass[12pt]{article}

% Preamble: Required Packages
\usepackage[margin=1in]{geometry}
\usepackage{pifont} % For checkmarks and crosses
\usepackage{booktabs} % For professional tables
\usepackage{hyperref} % For clickable links and ToC
\usepackage{url} % For formatting URLs
\usepackage{seqsplit} % For splitting long text sequences without breaking
\usepackage{xcolor} % For custom colors
\usepackage{graphicx} % For potential logos or images
\usepackage{datetime} % For \today

% --- Document Setup ---
\hypersetup{
    colorlinks=true,
    linkcolor=blue,
    filecolor=magenta,      
    urlcolor=cyan,
    pdftitle={Cybersecurity Posture Assessment},
    pdfpagemode=FullScreen,
}

% Define colors for risk levels
\definecolor{severitycritical}{HTML}{990000}
\definecolor{severityhigh}{HTML}{D2691E}
\definecolor{severitymedium}{HTML}{FFA500}
\definecolor{severitylow}{HTML}{32CD32}

% --- Document Start ---
\begin{document}

% --- Title Page ---
\begin{titlepage}
    \centering
    \vspace*{\fill}
    \Huge{\textbf{Cybersecurity Posture Assessment Report}}
    \vspace{1.5cm}
    \Large{\textbf{Prepared for:}} \\
    \vspace{0.5cm}
    \huge{\textbf{[Organization Name]}}
    \vspace{2cm}
    \large{\textbf{Date of Report:}} \\
    \vspace{0.2cm}
    \large{\today}
    \vspace*{\fill}
    \textit{This report is confidential and intended solely for the use of the recipient.}
\end{titlepage}

% --- Table of Contents ---
\tableofcontents
\newpage

% --- Section 1: Executive Summary ---
\section{Executive Summary}
This report provides a comprehensive analysis of the cybersecurity posture for \textbf{[Organization Name]}, based on a review of organizational security controls, an external network scan, and an assessment of pre-existing risks.

The assessment reveals a notable contrast in the organization's security maturity. On one hand, the security questionnaire indicates a strong foundation of administrative controls. Policies such as mandatory Multi-Factor Authentication (MFA) for critical systems and consistent security awareness training are in place, demonstrating a commendable commitment to user-level security.

However, the technical scan identified a critical-risk vulnerability that overshadows these positive aspects. A MySQL database service (port 3306) was found to be publicly exposed on the target system \texttt{[Target IP]}. This direct exposure creates a significant attack surface for threat actors. Compounding this issue, the detected MySQL version (5.7.33) is now End-of-Life (EOL) as of October 2023 and no longer receives security updates from the vendor.

The primary recommendation is to take immediate action to restrict all public access to the database service. Following this, a plan must be developed to migrate the database to a currently supported version to mitigate risks from unpatched vulnerabilities.

% --- Section 2: Organizational Information ---
\section{Organizational Information}
The following details were used as the basis for this assessment. Where information was not provided, placeholders have been used.

\begin{itemize}
    \item \textbf{Organization Name:} \textbf{[Organization Name]}
    \item \textbf{Primary Email Domain:} \texttt{[Domain]}
    \item \textbf{Client External IP:} \texttt{[Client IP]}
    \item \textbf{Scan Target:} \texttt{[Target IP]}
\end{itemize}

% --- Section 3: Security Control Review ---
\section{Security Control Review}
A security questionnaire was completed to evaluate the administrative and policy-based controls within the organization. The responses, summarized in Table \ref{tab:controls}, indicate a strong adherence to foundational security best practices.

\begin{table}[h!]
\centering
\caption{Security Controls Questionnaire Results}
\label{tab:controls}
\begin{tabular}{p{0.8\linewidth} c}
\toprule
\textbf{Control Question} & \textbf{Response} \\
\midrule
\seqsplit{Do you require MFA to access email?} & \textcolor{green}{\ding{51}} \\
\seqsplit{Do you require MFA to log into computers?} & \textcolor{green}{\ding{51}} \\
\seqsplit{Do you require MFA to access sensitive data systems?} & \textcolor{green}{\ding{51}} \\
\seqsplit{Does your organization have an employee acceptable use policy?} & \textcolor{green}{\ding{51}} \\
\seqsplit{Does your organization do security awareness training for new employees?} & \textcolor{green}{\ding{51}} \\
\seqsplit{Does your organization do security awareness training for all employees at least once per year?} & \textcolor{green}{\ding{51}} \\
\bottomrule
\end{tabular}
\end{table}

\noindent All provided answers were affirmative (\textcolor{green}{\ding{51}}), indicating that the organization has successfully implemented key security controls related to access management and employee awareness. No policy-based gaps were identified from this review.

% --- Section 4: Technical Scan Results ---
\section{Technical Scan Results}
An external network scan was performed against the target IP address to identify open ports and exposed services.

\subsection{Scan Results for \texttt{[Target IP]}}
The scan revealed one open port, detailed in Table \ref{tab:scanresults}.

\begin{table}[h!]
\centering
\caption{Open Ports and Services Detected}
\label{tab:scanresults}
\begin{tabular}{llll}
\toprule
\textbf{Port} & \textbf{State} & \textbf{Service} & \textbf{Product \& Version} \\
\midrule
3306/tcp & open & mysql & MySQL 5.7.33 \\
\bottomrule
\end{tabular}
\end{table}

\subsubsection{Analysis of Findings}
The discovery of port 3306 open to the public internet is a significant security concern. This port is the default for the MySQL database protocol. Exposing a database server directly to the internet is highly discouraged as it allows attackers to:
\begin{itemize}
    \item Perform brute-force attacks to guess credentials.
    \item Exploit known vulnerabilities in the database software.
    \item Potentially exfiltrate, modify, or destroy sensitive data.
\end{itemize}
Furthermore, the identified version, MySQL 5.7.33, belongs to the 5.7 series which reached its official End-of-Life (EOL) in October 2023. EOL software no longer receives security patches, meaning any vulnerabilities discovered after this date will remain unpatched and exploitable.

% --- Section 5: Risk Assessment ---
\section{Risk Assessment}
The following table synthesizes findings from the technical scan and pre-existing risk data. The correlation of these items provides a clear view of the most pressing threats.

\begin{table}[h!]
\centering
\caption{Consolidated Risk Summary}
\label{tab:risks}
\begin{tabular}{p{0.25\linewidth} p{0.1\linewidth} p{0.35\linewidth} p{0.2\linewidth}}
\toprule
\textbf{Risk Name} & \textbf{Severity} & \textbf{Overview} & \textbf{Affected Elements} \\
\midrule
\textbf{End-of-Life (EOL) Database Software} & \textcolor{severitycritical}{\textbf{Critical}} & The MySQL 5.7.33 software is EOL and no longer receives security updates, leaving it perpetually vulnerable to new exploits. & MySQL 5.7.33 on \texttt{[Target IP]} \\
\addlinespace
\textbf{Publicly Exposed Database Service} & \textcolor{severityhigh}{\textbf{High}} & The MySQL database port (3306) is open to the internet, creating a direct attack vector for unauthorized access and exploitation. This finding validates a known risk. & Port 3306 on \texttt{[Target IP]} \\
\bottomrule
\end{tabular}
\end{table}

% --- Section 6: Recommendations ---
\section{Recommendations}
The following actionable recommendations are prioritized to address the identified risks effectively.

\subsection{Immediate Priority (Urgent)}
\begin{enumerate}
    \item \textbf{Restrict Access to Port 3306:} Immediately implement firewall rules to block all public internet traffic to TCP port 3306 on \texttt{[Target IP]}. Access should be restricted to a whitelist of trusted IP addresses only. If no external access is required, the port should be closed to the internet entirely. This directly mitigates the \textit{Publicly Exposed Database Service} risk.
\end{enumerate}

\subsection{High Priority (Within 3 Months)}
\begin{enumerate}
    \item \textbf{Plan and Execute Database Migration:} Develop a migration plan to upgrade the MySQL 5.7 database to a fully supported version (e.g., MySQL 8.x or a managed cloud database service). This is the only effective long-term solution to the \textit{End-of-Life Software} risk and is critical for maintaining a secure environment.
\end{enumerate}

\subsection{Strategic Recommendations}
\begin{enumerate}
    \item \textbf{Implement a Secure Remote Access Solution:} For any legitimate need for remote database administration, establish a Virtual Private Network (VPN) or a zero-trust network access (ZTNA) solution. This provides an encrypted and authenticated channel for administrators without exposing the database service directly to the internet.
    \item \textbf{Establish a Vulnerability Management Program:} Proactively identify risks by implementing a regular schedule of internal and external vulnerability scanning. This practice helps detect outdated software, misconfigurations, and other security weaknesses before they can be exploited.
\end{enumerate}

% --- Document End ---
\end{document}
```