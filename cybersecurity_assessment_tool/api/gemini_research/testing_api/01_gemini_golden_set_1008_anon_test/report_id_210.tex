```latex
\documentclass[12pt, a4paper]{article}

% Required Packages
\usepackage[margin=1in]{geometry}
\usepackage{pifont} % For checkmarks and crosses
\usepackage{booktabs} % For professional tables
\usepackage{hyperref} % For hyperlinks and metadata
\usepackage{url} % For URL formatting
\usepackage{seqsplit} % To split long strings in tt font
\usepackage{xcolor} % For custom colors
\usepackage{graphicx} % For logo (placeholder)
\usepackage{lastpage} % For page numbers
\usepackage{fancyhdr} % For header/footer

% --- Document Metadata and Hyperlink Setup ---
\hypersetup{
    colorlinks=true,
    linkcolor=blue,
    filecolor=magenta,      
    urlcolor=cyan,
    pdftitle={Cybersecurity Posture Report},
    pdfauthor={Cybersecurity Analyst},
    pdfsubject={Security Assessment},
    pdfkeywords={Security, Risk, Assessment},
    bookmarks=true
}

% --- Color Definitions for Risk Levels ---
\definecolor{criticalrisk}{HTML}{D73B3E}
\definecolor{highrisk}{HTML}{F58158}
\definecolor{mediumrisk}{HTML}{F9C642}
\definecolor{lowrisk}{HTML}{78A843}
\definecolor{inforisk}{HTML}{4372C4}
\definecolor{tableheader}{gray}{0.9}

% --- Header and Footer Configuration ---
\pagestyle{fancy}
\fancyhf{} % Clear all header and footer fields
\fancyhead[L]{Cybersecurity Posture Report}
\fancyhead[R]{\textbf{[Organization Name]}}
\fancyfoot[C]{\thepage\ of \pageref{LastPage}}
\renewcommand{\headrulewidth}{0.4pt}
\renewcommand{\footrulewidth}{0.4pt}

\begin{document}

% --- Title Page ---
\begin{titlepage}
    \centering
    \vfill
    \begin{center}
        \Huge\bfseries Cybersecurity Posture Report
    \end{center}
    \vspace{1.5cm}
    \begin{center}
        \Large Prepared for: \\
        \vspace{0.5cm}
        \Huge\bfseries \textbf{[Organization Name]}
    \end{center}
    \vfill
    \begin{center}
        \large \today
    \end{center}
\end{titlepage}

\newpage
\tableofcontents
\newpage

% --- Executive Summary ---
\section*{Executive Summary}

This report provides a comprehensive analysis of the cybersecurity posture for \textbf{[Organization Name]}. The assessment is based on a correlation of technical network scan data, a review of existing security controls via a questionnaire, and an evaluation of previously identified risks.

The analysis reveals a mixed security posture. The organization has implemented several positive controls, including Multi-Factor Authentication (MFA) for computer and sensitive system access. A significant positive finding is that a previously identified medium-risk vulnerability, an unencrypted web server on Port 80, appears to have been remediated, as the port was found to be closed during our scan.

However, two critical gaps were identified that significantly increase the organization's risk profile:
\begin{enumerate}
    \item \textbf{Lack of MFA on Email:} The absence of MFA for email access is a critical vulnerability. Email accounts are a primary target for attackers seeking to launch Business Email Compromise (BEC) attacks, pivot to other systems, or exfiltrate sensitive data.
    \item \textbf{Lack of Annual Security Training:} Without regular, recurring security awareness training for all employees, the organization is highly susceptible to phishing, social engineering, and other human-centric attacks.
\end{enumerate}

Immediate remediation of these two issues is strongly recommended to substantially improve the organization's resilience against common cyber threats. Detailed findings and actionable recommendations are provided in the subsequent sections of this report.

% --- Organizational Information ---
\section*{Organizational Information}

This section details the organizational information used for this assessment. As per our template mode for anonymized data, placeholders are used where specific information was not provided.

\begin{tabular}{@{}ll}
    \toprule
    \textbf{Attribute} & \textbf{Value} \\
    \midrule
    Organization Name & \textbf{[Organization Name]} \\
    Primary Email Domain & \texttt{[Domain]} \\
    External IP Address Scanned & \texttt{[Client IP]} \\
    \bottomrule
\end{tabular}

% --- Security Control Review ---
\section*{Security Control Review}

The following table summarizes the organization's security controls based on the provided questionnaire. "No" answers indicate significant gaps in the security framework and are highlighted for immediate attention.

\begin{table}[h!]
\centering
\caption{Security Controls Questionnaire Analysis}
\begin{tabular}{@{}p{0.55\linewidth}ccc}
    \toprule
    \rowcolor{tableheader}
    \textbf{Control Question} & \textbf{Response} & \textbf{Status} \\
    \midrule
    Do you require MFA to access email? & \ding{55} & \textcolor{criticalrisk}{\textbf{Critical Gap}} \\
    Do you require MFA to log into computers? & \ding{51} & \textcolor{lowrisk}{In Place} \\
    Do you require MFA to access sensitive data systems? & \ding{51} & \textcolor{lowrisk}{In Place} \\
    Does your organization have an employee acceptable use policy? & \ding{51} & \textcolor{lowrisk}{In Place} \\
    Does your organization do security awareness training for new employees? & \ding{51} & \textcolor{lowrisk}{In Place} \\
    Does your organization do security awareness training for all employees at least once per year? & \ding{55} & \textcolor{highrisk}{\textbf{High Risk}} \\
    \bottomrule
\end{tabular}
\end{table}

% --- Technical Scan Results ---
\section*{Technical Scan Results}

A network scan was performed to identify open ports and exposed services on the organization's external infrastructure.

\begin{itemize}
    \item \textbf{Target IP Address:} \texttt{[Target IP]}
    \item \textbf{Scan Date:} Data provided on \today
\end{itemize}

The scan revealed a very limited external attack surface, which is a positive security practice. The key finding is detailed below.

\begin{table}[h!]
\centering
\caption{Network Scan Port Summary}
\begin{tabular}{@{}cccc@{}}
    \toprule
    \rowcolor{tableheader}
    \textbf{Port} & \textbf{Protocol} & \textbf{State} & \textbf{Notes} \\
    \midrule
    80 & TCP & \textbf{closed} & Port 80 (HTTP) was actively scanned and confirmed to be closed. \\
    \bottomrule
\end{tabular}
\end{table}

\textbf{Analysis:} The scan indicates that the previously identified risk associated with an "Unencrypted Web Server" on Port 80 has been remediated. No other open ports or vulnerable services were discovered during this assessment.

% --- Risk Assessment Summary ---
\section*{Risk Assessment Summary}

This section synthesizes findings from the security control review, technical scan, and pre-existing risk data into a prioritized list of current risks.

\begin{table}[h!]
\centering
\caption{Prioritized Risk Register}
\begin{tabular}{@{}p{0.3\linewidth}p{0.15\linewidth}p{0.45\linewidth}@{}}
    \toprule
    \rowcolor{tableheader}
    \textbf{Risk Name} & \textbf{Severity} & \textbf{Overview} \\
    \midrule
    \textbf{Lack of MFA on Email} & \textcolor{criticalrisk}{\textbf{Critical}} & The absence of MFA on email exposes the organization to a high likelihood of account compromise, leading to data breaches, financial fraud (BEC), and further network intrusion. \\
    \addlinespace
    \textbf{Insufficient Security Awareness Training} & \textcolor{highrisk}{\textbf{High}} & Without mandatory annual training, employees are more likely to fall victim to phishing and social engineering attacks, undermining other technical security controls. \\
    \addlinespace
    \textbf{Unencrypted Web Server} & \textcolor{lowrisk}{\textbf{Remediated}} & \textit{(From previous risk data)}. This risk is now considered remediated. Our technical scan confirmed that Port 80 is closed, mitigating the threat of unencrypted communication. \\
    \bottomrule
\end{tabular}
\end{table}

% --- Recommendations ---
\section*{Recommendations}

The following actionable recommendations are prioritized to address the most significant risks identified during the assessment.

\subsection*{Priority 1: Implement MFA on Email (Critical)}
\begin{itemize}
    \item \textbf{Action:} Immediately enforce MFA for all user access to the email system (\texttt{[Domain]}). This should apply to webmail, desktop clients, and mobile devices.
    \item \textbf{Justification:} This is the single most effective control to prevent unauthorized access to email accounts, which are a gateway for attackers. It directly mitigates the risk of Business Email Compromise and credential theft.
    \item \textbf{Resources:} Consult documentation for your email provider (e.g., Microsoft 365, Google Workspace) on enabling and enforcing MFA.
\end{itemize}

\subsection*{Priority 2: Establish Annual Security Training (High)}
\begin{itemize}
    \item \textbf{Action:} Develop and implement a mandatory security awareness training program for all employees, to be completed annually. The training should cover, at a minimum, phishing identification, password hygiene, and acceptable use policies.
    \item \textbf{Justification:} A well-trained workforce is a critical layer of defense. Regular training reduces the likelihood of human error leading to a security incident and fosters a culture of security within the organization.
    \item \textbf{Resources:} Numerous third-party vendors (e.g., KnowBe4, Proofpoint) offer comprehensive training and phishing simulation platforms.
\end{itemize}

\subsection*{Informational: Acknowledge Risk Remediation}
\begin{itemize}
    \item \textbf{Action:} Formally update internal risk documentation to mark the "Unencrypted Web Server" vulnerability as closed.
    \item \textbf{Justification:} Acknowledging and documenting successful remediation efforts is a key component of a mature risk management program. The team responsible should be commended for closing this security gap.
\end{itemize}

\end{document}
```