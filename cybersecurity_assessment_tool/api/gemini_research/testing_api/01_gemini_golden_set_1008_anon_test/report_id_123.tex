```latex
\documentclass[12pt]{article}

% Preamble: Required Packages
\usepackage[margin=1in]{geometry}
\usepackage{pifont} % For check and cross marks
\usepackage{booktabs} % For professional tables
\usepackage{hyperref} % For clickable links and references
\usepackage{url} % For formatting URLs
\usepackage{seqsplit} % To split long strings without breaking
\usepackage{xcolor} % For coloring text
\usepackage{graphicx} % For logo (placeholder)

% --- Document Setup ---
% Define colors for risk levels
\definecolor{criticalred}{HTML}{D73027}
\definecolor{highorange}{HTML}{F46D43}
\definecolor{mediumyellow}{HTML}{FEE090}
\definecolor{lowblue}{HTML}{4575B4}

% Hyperref setup for PDF metadata
\hypersetup{
    colorlinks=true,
    linkcolor=blue,
    filecolor=magenta,      
    urlcolor=cyan,
    pdftitle={Cybersecurity Posture Assessment Report},
    pdfauthor={Cybersecurity Analyst},
    pdfsubject={Security Analysis},
    pdfkeywords={Security, Report, Analysis},
    bookmarks=true
}

% --- Document Start ---
\begin{document}

% --- Title Page ---
\begin{titlepage}
    \centering
    \vspace*{1cm}
    
    \Huge \textbf{Cybersecurity Posture Assessment Report}
    
    \vspace{1.5cm}
    
    \Large \textbf{Prepared for:} \\
    \vspace{0.5cm}
    \textbf{[Organization Name]}
    
    \vspace{2cm}
    
    \Large \textbf{Date of Report:} \\
    \vspace{0.5cm}
    \today
    
    \vfill
    
    \large \textit{This report contains sensitive information and should be handled with care. Distribution is restricted to authorized personnel only.}
    
\end{titlepage}

\tableofcontents
\newpage

% --- Section 1: Executive Summary ---
\section{Executive Summary}
This report provides a comprehensive analysis of the security posture of \textbf{[Organization Name]}, based on a network scan, a review of security controls, and an assessment of known risks. The assessment was conducted to identify vulnerabilities, policy gaps, and misconfigurations that could expose the organization to cyber threats.

The key findings indicate a significant level of risk. Critical deficiencies were identified in foundational security controls, including a complete lack of Multi-Factor Authentication (MFA) for email and computer access, and an absence of security awareness training and acceptable use policies.

Technically, the assessment discovered a publicly exposed MySQL database service (Port 3306). This service is running an End-of-Life (EOL) version of MySQL (5.7.33), which no longer receives security updates and is highly likely to contain unpatched, exploitable vulnerabilities.

The combination of these policy gaps and technical vulnerabilities places the organization at a high risk of data breach, ransomware, and unauthorized access. Immediate and decisive action is required to remediate the identified issues. This report outlines prioritized, actionable recommendations to mitigate these risks and improve the overall security posture.

% --- Section 2: Organizational Information ---
\section{Organizational Information}
The following details were used as the basis for this assessment. Due to the anonymized nature of the provided data, placeholders have been used.

\begin{itemize}
    \item \textbf{Organization Name:} \textbf{[Organization Name]}
    \item \textbf{Primary Domain:} \texttt{[Domain]}
    \item \textbf{External IP Scanned:} \texttt{[Client IP]}
\end{itemize}

% --- Section 3: Security Control Review ---
\section{Security Control Review}
A review of organizational security controls was conducted via a questionnaire. The results highlight critical gaps in administrative and access control policies. The absence of MFA, user policies, and security training represents a fundamental weakness that significantly increases the organization's attack surface.

\begin{table}[h!]
\centering
\caption{Security Controls Questionnaire Results}
\begin{tabular}{@{}lc@{}}
\toprule
\textbf{Control Question} & \textbf{Status} \\ \midrule
Do you require MFA to access email? & \ding{55} \\
Do you require MFA to log into computers? & \ding{55} \\
Do you require MFA to access sensitive data systems? & \ding{51} \\
Does your organization have an employee acceptable use policy? & \ding{55} \\
Does your organization do security awareness training for new employees? & \ding{55} \\
Does your organization do security training for all employees annually? & \ding{55} \\ \bottomrule
\end{tabular}
\end{table}

\begin{itemize}
    \item[\ding{51}] = Control Implemented
    \item[\ding{55}] = Control Gap Identified
\end{itemize}

% --- Section 4: Technical Scan Results ---
\section{Technical Scan Results}
An external network scan was performed on the target IP address to identify open ports and exposed services.

\begin{itemize}
    \item \textbf{Target IP Address:} \texttt{[Target IP]}
\end{itemize}

The scan revealed one open port, which exposes a critical database service directly to the network.

\begin{table}[h!]
\centering
\caption{Open Ports and Services Detected}
\begin{tabular}{@{}lllll@{}}
\toprule
\textbf{Port} & \textbf{State} & \textbf{Service} & \textbf{Product} & \textbf{Version} \\ \midrule
3306/tcp & open & mysql & MySQL & 5.7.33 \\ \bottomrule
\end{tabular}
\end{table}

\paragraph{Analysis of Findings:}
The most critical finding is the exposed MySQL database on port 3306. Direct exposure of a database to the internet is a severe security risk, making it a prime target for brute-force attacks, credential stuffing, and exploitation of known vulnerabilities.

Furthermore, the detected version, \textbf{MySQL 5.7.33}, reached its official End-of-Life (EOL) in October 2023. EOL software no longer receives security patches from the vendor, meaning any vulnerabilities discovered after this date will remain unpatched, leaving the system perpetually vulnerable.

% --- Section 5: Consolidated Risk Assessment ---
\section{Consolidated Risk Assessment}
The following table synthesizes findings from the security control review, technical scan, and pre-existing risk data. Risks are rated based on their potential impact and likelihood of exploitation.

\begin{table}[h!]
\centering
\caption{Risk Summary}
\resizebox{\textwidth}{!}{%
\begin{tabular}{@{}llll@{}}
\toprule
\textbf{Risk Name} & \textbf{Severity} & \textbf{Overview} & \textbf{Affected Elements} \\ \midrule
\textcolor{criticalred}{\textbf{End-of-Life Database Software}} & \textcolor{criticalred}{\textbf{Critical}} & An exposed database is running MySQL 5.7, which is EOL and unpatched. & \texttt{[Target IP]}:3306 \\
\textcolor{criticalred}{\textbf{Lack of MFA on Critical Systems}} & \textcolor{criticalred}{\textbf{Critical}} & No MFA on email or workstations allows a single password compromise to grant wide access. & User Accounts, Workstations, Email \\
\textcolor{highorange}{\textbf{Database Exposure}} & \textcolor{highorange}{\textbf{High}} & The MySQL database port (3306) is open to the network, inviting direct attacks. & \texttt{[Target IP]}:3306 \\
\textcolor{highorange}{\textbf{Lack of Security Training}} & \textcolor{highorange}{\textbf{High}} & The absence of a security awareness program makes staff highly susceptible to phishing. & All Employees \\ \bottomrule
\end{tabular}
}
\end{table}

% --- Section 6: Recommendations ---
\section{Recommendations}
The following prioritized recommendations are provided to address the identified risks. Immediate action should be taken on all Critical and High priority items.

\subsection{Immediate Priority (Critical)}
\begin{enumerate}
    \item \textbf{Restrict Access to Database Port:} Immediately implement firewall rules to block all public access to TCP port 3306. Access should only be permitted from trusted, internal IP addresses.
    \item \textbf{Upgrade End-of-Life MySQL Server:} Plan and execute an urgent migration from MySQL 5.7 to a currently supported version (e.g., MySQL 8.0 or later). This is essential to receive security patches for new vulnerabilities.
    \item \textbf{Enforce Multi-Factor Authentication (MFA):} Immediately enable and enforce MFA for all users on all critical systems, starting with email (e.g., Office 365, Google Workspace) and computer logins (e.g., Windows Hello, Duo).
\end{enumerate}

\subsection{High Priority}
\begin{enumerate}
    \item \textbf{Develop and Implement Security Policies:} Draft and enforce an \textit{Acceptable Use Policy} that clearly defines rules for company assets and data handling.
    \item \textbf{Establish Security Awareness Training:} Implement a mandatory security awareness training program for all employees. This should be required for new hires during onboarding and conducted for all staff at least annually.
\end{enumerate}

\subsection{Long-Term Recommendations}
\begin{enumerate}
    \item \textbf{Implement Network Segmentation:} Isolate critical servers, such as databases, in a separate, restricted network zone to limit lateral movement in the event of a breach.
    \item \textbf{Utilize a VPN for Remote Access:} For any required remote administration of the database or other internal systems, enforce the use of a Virtual Private Network (VPN) with MFA.
\end{enumerate}

\end{document}
```