```latex
\documentclass[12pt]{article}

% Preamble: Required Packages
\usepackage[margin=1in]{geometry}
\usepackage{pifont} % For checkmarks and crosses
\usepackage{booktabs} % For professional tables
\usepackage{hyperref} % For hyperlinks
\usepackage{url}      % For URL formatting
\usepackage{seqsplit} % For splitting long strings in tt font
\usepackage{graphicx} % For potential logos
\usepackage{xcolor}   % For colors

% Document Information
\title{Cybersecurity Posture Assessment Report}
\author{Cybersecurity Analysis Division}
\date{\today}

% Hyperref Setup
\hypersetup{
    colorlinks=true,
    linkcolor=blue,
    filecolor=magenta,      
    urlcolor=cyan,
    pdftitle={Cybersecurity Posture Assessment Report},
    pdfpagemode=FullScreen,
}

\begin{document}

\maketitle
\hrule
\vspace{1cm}
\begin{center}
    \textbf{Client:} \textbf{[Organization Name]} \\
    \textbf{Report Date:} \today
\end{center}
\vspace{1cm}
\hrule
\newpage

\tableofcontents
\newpage

% ------------------------------------------------------------------
% Section 1: Executive Summary
% ------------------------------------------------------------------
\section{Executive Summary}

This report provides a comprehensive analysis of the cybersecurity posture for \textbf{[Organization Name]}, based on a review of organizational security controls, a technical network scan, and an evaluation of pre-existing risks. The assessment was conducted on \today.

The analysis revealed several critical and high-risk gaps in the organization's security controls. Most notably, the absence of Multi-Factor Authentication (MFA) for email and computer access represents a critical vulnerability. These gaps significantly increase the risk of unauthorized access, account compromise, and data breaches. Additionally, the lack of a formal Acceptable Use Policy and mandatory security training for new employees indicates foundational weaknesses in security governance.

The external network scan of the target IP address \texttt{[Target IP]} did not identify any open ports. While this may suggest a properly configured firewall, it does not mitigate the severe internal and policy-based risks identified.

Immediate remediation of the identified control gaps is strongly recommended to reduce the organization's attack surface and improve its overall defensive capabilities.

% ------------------------------------------------------------------
% Section 2: Organizational Information
% ------------------------------------------------------------------
\section{Organizational Information}

This section details the information provided for the assessment. The data has been anonymized as per the engagement requirements.

\begin{itemize}
    \item \textbf{Organization Name:} \textbf{[Organization Name]}
    \item \textbf{Primary Email Domain:} \texttt{[Domain]}
    \item \textbf{Assessed External IP:} \texttt{[Client IP]}
\end{itemize}

% ------------------------------------------------------------------
% Section 3: Security Control Review
% ------------------------------------------------------------------
\section{Security Control Review}

A review of the organization's security controls was conducted via a standardized questionnaire. The responses indicate significant areas for improvement, particularly concerning access control and security governance. A summary of the findings is presented in Table \ref{tab:controls}.

\begin{table}[h!]
\centering
\caption{Organizational Security Control Questionnaire}
\label{tab:controls}
\begin{tabular}{p{0.6\linewidth} c l}
\toprule
\textbf{Control Question} & \textbf{Response} & \textbf{Assessment} \\
\midrule
Do you require MFA to access email? & \ding{55} & \textcolor{red}{\textbf{Critical Gap}} \\
Do you require MFA to log into computers? & \ding{55} & \textcolor{red}{\textbf{Critical Gap}} \\
Do you require MFA to access sensitive data systems? & \ding{51} & Good Practice \\
Does your organization have an employee acceptable use policy? & \ding{55} & \textcolor{orange}{High Risk} \\
Does your organization do security awareness training for new employees? & \ding{55} & \textcolor{orange}{High Risk} \\
Does your organization do security awareness training for all employees at least once per year? & \ding{51} & Good Practice \\
\bottomrule
\end{tabular}
\end{table}

\paragraph{Analysis:} The lack of MFA for email and computer access are critical findings. Email is the primary vector for phishing attacks, and compromised credentials could lead to a full account takeover. Similarly, unprotected computer logons remove a vital layer of defense against unauthorized endpoint access. The absence of a formal Acceptable Use Policy and security training for new hires creates an environment where employees may be unaware of security best practices, increasing the likelihood of human error leading to a security incident.

% ------------------------------------------------------------------
% Section 4: Technical Scan Results
% ------------------------------------------------------------------
\section{Technical Scan Results}

An external network vulnerability scan was performed to identify exposed services and potential vulnerabilities on the organization's perimeter.

\begin{itemize}
    \item \textbf{Scan Target:} \texttt{[Target IP]}
    \item \textbf{Scan Date:} \today
\end{itemize}

\subsection{Scan Summary}
The scan completed without identifying any open TCP or UDP ports on the target system. 

\paragraph{Interpretation:} No exposed services were detected. This is a positive finding and may indicate that a well-configured firewall is in place, properly blocking unsolicited inbound traffic. However, this result does not provide insight into the security of internal systems or the risks associated with the policy gaps identified in Section 3.

% ------------------------------------------------------------------
% Section 5: Consolidated Risk Assessment
% ------------------------------------------------------------------
\section{Consolidated Risk Assessment}

This section synthesizes findings from the security control review, technical scan, and pre-existing risk data. The primary risks identified stem from organizational policy and procedure gaps. No pre-existing vulnerabilities were provided for this assessment.

\begin{table}[h!]
\centering
\caption{Identified Risks and Severity}
\label{tab:risks}
\begin{tabular}{p{0.1\linewidth} p{0.3\linewidth} p{0.15\linewidth} p{0.35\linewidth}}
\toprule
\textbf{Risk ID} & \textbf{Finding} & \textbf{Severity} & \textbf{Description} \\
\midrule
RISK-001 & Lack of MFA for Email Access & \textcolor{red}{\textbf{Critical}} & Absence of a second authentication factor for email exposes the organization to account takeover via phishing or credential stuffing. \\
\addlinespace
RISK-002 & Lack of MFA for Computer Logons & \textcolor{red}{\textbf{Critical}} & Stolen or weak user credentials can be used to gain direct access to company endpoints, data, and internal network resources. \\
\addlinespace
RISK-003 & Missing Acceptable Use Policy (AUP) & \textcolor{orange}{\textbf{High}} & Without a formal AUP, employees lack clear guidelines on the secure and acceptable use of company assets, increasing insider risk. \\
\addlinespace
RISK-004 & No Security Training for New Hires & \textcolor{orange}{\textbf{High}} & New employees are not equipped with the necessary security knowledge during their critical onboarding phase, making them more susceptible to social engineering attacks. \\
\bottomrule
\end{tabular}
\end{table}

% ------------------------------------------------------------------
% Section 6: Recommendations
% ------------------------------------------------------------------
\section{Recommendations}

Based on the findings in this report, the following actions are recommended to mitigate the identified risks and strengthen the overall security posture of \textbf{[Organization Name]}.

\subsection{Immediate (Critical) Actions}
\begin{itemize}
    \item \textbf{Remediate RISK-001:} Procure and enforce an MFA solution for all user access to the corporate email system (e.g., Microsoft 365, Google Workspace). This should be the highest priority.
    \item \textbf{Remediate RISK-002:} Implement and enforce MFA for all employee computer and remote access (VPN) logons. Solutions like Windows Hello for Business, Duo, or other identity providers can fulfill this requirement.
\end{itemize}

\subsection{Near-Term (High Priority) Actions}
\begin{itemize}
    \item \textbf{Remediate RISK-003:} Develop, approve, and disseminate a formal Employee Acceptable Use Policy. This policy should be signed by all employees and reviewed annually.
    \item \textbf{Remediate RISK-004:} Integrate a mandatory security awareness training module into the new employee onboarding process. This training should cover key topics such as phishing, password security, and the new Acceptable Use Policy.
\end{itemize}

\end{document}
```