```latex
\documentclass[12pt]{article}

% Preamble: Required Packages
\usepackage[margin=1in]{geometry}
\usepackage{pifont} % For checkmarks and crosses
\usepackage{booktabs} % For professional tables
\usepackage{hyperref} % For hyperlinks and metadata
\usepackage{url} % For formatting URLs
\usepackage{seqsplit} % For splitting long strings in texttt

% Document Metadata
\hypersetup{
    colorlinks=true,
    linkcolor=blue,
    filecolor=magenta,      
    urlcolor=cyan,
    pdftitle={Cybersecurity Posture Assessment Report},
    pdfauthor={Cybersecurity Analyst},
    pdfsubject={Security Analysis},
    pdfkeywords={Cybersecurity, Risk Assessment, Network Scan},
}

% Define checkmark and cross symbols for convenience
\newcommand{\cmark}{\ding{51}}%
\newcommand{\xmark}{\ding{55}}%

\begin{document}

% --- Title Page ---
\begin{titlepage}
    \centering
    \vspace*{1cm}
    \Huge
    \textbf{Cybersecurity Posture Assessment Report}
    
    \vspace{1.5cm}
    \Large
    Prepared for: \\
    \vspace{0.5cm}
    \textbf{[Organization Name]}
    
    \vfill
    
    \Large
    \textbf{Date of Assessment:} November 22, 2025 \\
    \textbf{Report Generated By:} Expert Cybersecurity Analyst
    
\end{titlepage}

\tableofcontents
\newpage

% --- Section 1: Executive Summary ---
\section{Executive Summary}
This report provides a comprehensive analysis of the cybersecurity posture for \textbf{[Organization Name]}, based on data collected on November 22, 2025. The assessment combines a review of organizational security controls, an external network vulnerability scan, and an evaluation of pre-existing risks.

The analysis revealed several critical and high-risk findings that require immediate attention. Key areas of concern include:
\begin{itemize}
    \item \textbf{Lack of Foundational Security Controls:} The organization has not implemented Multi-Factor Authentication (MFA) for critical access points such as email and computer logins. This represents a critical vulnerability.
    \item \textbf{Outdated Public-Facing Software:} The external network scan identified a web server running an outdated version of nginx (1.18.0), which is known to have multiple security vulnerabilities. This exposes the organization to potential compromise.
    \item \textbf{Absence of Security Governance:} There is a significant gap in security governance, evidenced by the lack of an employee acceptable use policy and a formal security awareness training program.
\end{itemize}

Overall, the organization's current security posture is considered high-risk. The recommendations outlined in this report are designed to address these deficiencies and significantly improve the defensive capabilities against common cyber threats.

% --- Section 2: Organizational Information ---
\section{Organizational Information}
This section details the organizational data used for this assessment. Due to the anonymized nature of the input, placeholders have been used where specific data was not provided.

\begin{tabular}{@{}ll}
    \toprule
    \textbf{Attribute} & \textbf{Value} \\
    \midrule
    Organization Name & \textbf{[Organization Name]} \\
    Primary Domain & \texttt{[Domain]} \\
    External IP Address & \texttt{[Client IP]} \\
    \bottomrule
\end{tabular}

% --- Section 3: Security Control Review ---
\section{Security Control Review}
The following table summarizes the organization's responses to a security controls questionnaire. The assessment column highlights alignment with cybersecurity best practices. "No" answers indicate significant gaps in the security framework.

\begin{table}[h!]
\centering
\caption{Security Controls Questionnaire Analysis}
\begin{tabular}{p{8cm} c l}
    \toprule
    \textbf{Control Question} & \textbf{Response} & \textbf{Assessment} \\
    \midrule
    Do you require MFA to access email? & \xmark & \textbf{Critical Gap} \\
    Do you require MFA to log into computers? & \xmark & \textbf{Critical Gap} \\
    Do you require MFA to access sensitive data systems? & \cmark & Best Practice \\
    Does your organization have an employee acceptable use policy? & \xmark & High Risk \\
    Does your organization do security awareness training for new employees? & \xmark & High Risk \\
    Does your organization do security awareness training for all employees at least once per year? & \xmark & High Risk \\
    \bottomrule
\end{tabular}
\end{table}

The lack of MFA for email and computer access, combined with the absence of security policies and training, creates a permissive environment for unauthorized access and insider threats.

% --- Section 4: Technical Scan Results ---
\section{Technical Scan Results}
An external network scan was performed against the organization's public-facing infrastructure. The target for this scan was \texttt{[Target IP]}.

\subsection{Open Ports and Services}
The scan identified the following open port and service, which is accessible from the public internet.

\begin{table}[h!]
\centering
\caption{Identified Open Ports and Services}
\begin{tabular}{l l l l}
    \toprule
    \textbf{Port} & \textbf{Service} & \textbf{Product} & \textbf{Version} \\
    \midrule
    443/TCP & HTTPS & nginx & 1.18.0 \\
    \bottomrule
\end{tabular}
\end{table}

\subsection{Technical Finding: Outdated Web Server}
The web server is running \textbf{nginx version 1.18.0}, which was released in April 2020. This version is outdated and no longer receives security updates. It is known to be vulnerable to multiple Common Vulnerabilities and Exposures (CVEs), which could allow an attacker to cause a denial of service, bypass security restrictions, or potentially execute arbitrary code. This finding is classified as a \textbf{High} severity risk.

% --- Section 5: Consolidated Risk Assessment ---
\section{Consolidated Risk Assessment}
This section synthesizes findings from the security control review, technical scan, and pre-existing risk data. No pre-existing vulnerabilities were reported. The following new risks have been identified.

\begin{table}[h!]
\centering
\caption{Summary of Identified Risks}
\begin{tabular}{p{1.5cm} p{6.5cm} l l}
    \toprule
    \textbf{Risk ID} & \textbf{Description} & \textbf{Source} & \textbf{Severity} \\
    \midrule
    RISK-001 & Lack of MFA for email and endpoint access significantly increases the risk of account compromise via stolen credentials. & Questionnaire & \textbf{Critical} \\
    \addlinespace
    RISK-002 & The public-facing web server runs outdated and vulnerable nginx software, exposing the organization to external attack. & Network Scan & \textbf{High} \\
    \addlinespace
    RISK-003 & Absence of an acceptable use policy and security awareness training leads to a high likelihood of human error causing a security incident. & Questionnaire & \textbf{High} \\
    \bottomrule
\end{tabular}
\end{table}

% --- Section 6: Recommendations ---
\section{Recommendations}
The following actionable recommendations are provided to mitigate the identified risks and improve the overall security posture of \textbf{[Organization Name]}. They are prioritized based on severity and potential impact.

\subsection{Immediate Priority (0-30 Days)}
\begin{enumerate}
    \item \textbf{Implement Multi-Factor Authentication (MFA):}
    \begin{itemize}
        \item \textbf{Action:} Enforce MFA for all users on all critical systems, prioritizing email (e.g., Office 365, Google Workspace) and computer logins (e.g., via Windows Hello for Business, Duo).
        \item \textbf{Justification:} Mitigates RISK-001 by preventing unauthorized access even if passwords are compromised. This is the single most effective control to implement.
    \end{itemize}
    
    \item \textbf{Upgrade Web Server Software:}
    \begin{itemize}
        \item \textbf{Action:} Upgrade the nginx server on host \texttt{[Target IP]} from version 1.18.0 to a current, stable, and supported version (e.g., 1.26.x or newer).
        \item \textbf{Justification:} Mitigates RISK-002 by patching numerous known vulnerabilities, reducing the external attack surface.
    \end{itemize}
\end{enumerate}

\subsection{High Priority (30-90 Days)}
\begin{enumerate}
    \setcounter{enumi}{2} % Continue numbering
    \item \textbf{Develop and Implement Security Policies:}
    \begin{itemize}
        \item \textbf{Action:} Create and enforce a formal Employee Acceptable Use Policy (AUP). This policy should clearly define the rules for using company assets, data handling, and internet usage.
        \item \textbf{Justification:} Mitigates RISK-003 by establishing a baseline for expected user behavior and creating accountability.
    \end{itemize}

    \item \textbf{Establish Security Awareness Training Program:}
    \begin{itemize}
        \item \textbf{Action:} Implement a mandatory security awareness training program for all new hires and conduct annual refresher training for all employees. Topics should include phishing, password security, and social engineering.
        \item \textbf{Justification:} Mitigates RISK-003 by empowering employees to become the first line of defense, reducing the likelihood of successful phishing and social engineering attacks.
    \end{itemize}
\end{enumerate}

\end{document}
```