```latex
\documentclass[12pt]{article}

% Preamble: Required Packages
\usepackage[margin=1in]{geometry}
\usepackage{pifont} % For checkmarks and crosses (\ding)
\usepackage{booktabs} % For professional tables
\usepackage{hyperref} % For clickable links
\usepackage{url} % For URL formatting
\usepackage{seqsplit} % For splitting long strings
\usepackage{graphicx}
\usepackage{xcolor}
\usepackage{fancyhdr}

% Document Metadata & Header/Footer
\hypersetup{
    colorlinks=true,
    linkcolor=blue,
    filecolor=magenta,      
    urlcolor=cyan,
    pdftitle={Cybersecurity Assessment Report},
    pdfauthor={Cybersecurity Analyst},
    pdfsubject={Security Posture Analysis},
    pdfkeywords={Cybersecurity, Nmap, Risk, Assessment},
}

\pagestyle{fancy}
\fancyhf{}
\lhead{Cybersecurity Assessment Report}
\rhead{\textbf{[Organization Name]}}
\cfoot{\thepage}

\begin{document}

% --- Title Page ---
\begin{titlepage}
    \centering
    \vspace*{\stretch{1.0}}
    \Huge\textbf{Cybersecurity Assessment Report}
    \vspace{0.5cm}
    \LARGE For
    \vspace{0.5cm}
    \Huge\textbf{[Organization Name]}
    \vspace{\stretch{2.0}}
    \large
    \textbf{Date of Report:} \today \\
    \textbf{Author:} Cybersecurity Analyst
    \vspace*{\stretch{1.0}}
\end{titlepage}

\tableofcontents
\newpage

% --- Section 1: Executive Overview ---
\section{Executive Overview}
This report details the findings of a cybersecurity assessment conducted for \textbf{[Organization Name]}. The analysis synthesized data from an external network scan, a security controls questionnaire, and a review of pre-existing risks.

The assessment revealed several \textbf{critical} and \textbf{high-risk} vulnerabilities that expose the organization to a significant threat of unauthorized access, data breach, and system compromise.

Key findings include:
\begin{itemize}
    \item \textbf{Critical Perimeter Vulnerability:} A publicly accessible FTP server was identified running a dangerously outdated version of vsftpd (2.3.4). This version is known to contain a critical backdoor vulnerability (CVE-2011-2523). Furthermore, the server is misconfigured to allow anonymous login, permitting unauthenticated access to files.
    \item \textbf{Critical Identity and Access Management Gaps:} Multi-Factor Authentication (MFA) is not enforced for accessing email or other sensitive data systems. This dramatically increases the risk of account compromise and lateral movement by an attacker.
    \item \textbf{Significant Policy and Governance Deficiencies:} The organization lacks a formal employee acceptable use policy and does not conduct annual security awareness training for all staff. These gaps weaken the human element of security, often the first line of defense.
    \item \textbf{Unsupported Internal Systems:} The organization is aware of and continues to operate computers running Windows 7, an end-of-life operating system that no longer receives security updates from Microsoft.
\end{itemize}

Immediate remediation of the external FTP server and the implementation of MFA are strongly recommended to mitigate the most severe risks.

% --- Section 2: Organizational Information ---
\section{Organizational Information}
This report is based on the following information provided for the assessment.
\begin{itemize}
    \item \textbf{Organization Name:} \textbf{[Organization Name]}
    \item \textbf{Primary Domain:} \texttt{[Domain]}
    \item \textbf{External IP Scanned:} \texttt{[Client IP]}
\end{itemize}

% --- Section 3: Security Control Review ---
\section{Security Control Review}
The following table summarizes the organization's responses to a security controls questionnaire. The assessment column highlights gaps against established cybersecurity best practices.

\begin{table}[h!]
\centering
\caption{Security Controls Questionnaire Analysis}
\begin{tabular}{p{0.6\textwidth} c l}
\toprule
\textbf{Control Question} & \textbf{Response} & \textbf{Assessment} \\
\midrule
Do you require MFA to access email? & \ding{55} & \textcolor{red}{\textbf{Critical Gap}} \\
Do you require MFA to log into computers? & \ding{51} & Best Practice Met \\
Do you require MFA to access sensitive data systems? & \ding{55} & \textcolor{red}{\textbf{Critical Gap}} \\
Does your organization have an employee acceptable use policy? & \ding{55} & \textcolor{orange}{High-Risk Finding} \\
Does your organization do security awareness training for new employees? & \ding{51} & Best Practice Met \\
Does your organization do security awareness training for all employees at least once per year? & \ding{55} & \textcolor{orange}{High-Risk Finding} \\
\bottomrule
\end{tabular}
\end{table}

% --- Section 4: Technical Scan Results ---
\section{Technical Scan Results}
An external network scan was performed against the target IP address \texttt{[Target IP]}. The scan identified one host with a critical vulnerability.

\subsection{Host: \texttt{[Target IP]}}
The host was found to be responsive and exposed one service to the public internet.

\begin{table}[h!]
\centering
\caption{Open Ports and Services on \texttt{[Target IP]}}
\begin{tabular}{l l l l p{0.3\textwidth}}
\toprule
\textbf{Port} & \textbf{State} & \textbf{Service} & \textbf{Version} & \textbf{Details} \\
\midrule
21/tcp & Open & ftp & vsftpd 2.3.4 & \textcolor{red}{\textbf{Critical Finding:}} Anonymous FTP login is allowed. The software version is vulnerable to a known backdoor (CVE-2011-2523). \\
\bottomrule
\end{tabular}
\end{table}

% --- Section 5: Consolidated Risk Assessment ---
\section{Consolidated Risk Assessment}
This section correlates all findings from the questionnaire, technical scan, and pre-existing risk data into a consolidated list of security risks.

\begin{table}[h!]
\centering
\caption{Summary of Identified Risks}
\begin{tabular}{p{0.25\textwidth} p{0.5\textwidth} l}
\toprule
\textbf{Risk Name} & \textbf{Description} & \textbf{Severity} \\
\midrule
\textbf{Vulnerable Public FTP Server} & An outdated and misconfigured FTP server is exposed to the internet, allowing anonymous access and potential for remote code execution via a known backdoor. & \textcolor{red}{\textbf{Critical}} \\
\textbf{Lack of MFA on Critical Systems} & Email and sensitive data systems are protected only by passwords, making them highly susceptible to compromise through phishing or credential stuffing attacks. & \textcolor{red}{\textbf{Critical}} \\
\textbf{Inadequate Security Policies \& Training} & The absence of an Acceptable Use Policy and annual security training increases the likelihood of insider threats and successful social engineering attacks. & \textcolor{orange}{High} \\
\textbf{Unsupported Operating System} & Workstations are running Windows 7, which is End-of-Life and no longer receives security patches, leaving them vulnerable to known exploits. & \textcolor{orange}{Medium} \\
\bottomrule
\end{tabular}
\end{table}

% --- Section 6: Recommendations ---
\section{Recommendations}
The following actions are recommended to mitigate the identified risks and improve the overall security posture of \textbf{[Organization Name]}.

\subsection{Immediate Priority (Remediate within 72 hours)}
\subsubsection{Vulnerable Public FTP Server}
\begin{itemize}
    \item \textbf{Action 1:} Immediately disable or firewall port 21 on the public-facing server at \texttt{[Target IP]}.
    \item \textbf{Action 2:} Decommission the vsftpd 2.3.4 service permanently.
    \item \textbf{Action 3:} If external file transfer is a business requirement, implement a secure alternative such as SFTP (SSH File Transfer Protocol) or a modern, managed file transfer solution.
    \item \textbf{Action 4:} Ensure any replacement solution requires strong authentication and does not permit anonymous access.
\end{itemize}

\subsection{High Priority (Remediate within 30 days)}
\subsubsection{Lack of MFA on Critical Systems}
\begin{itemize}
    \item \textbf{Action 1:} Procure and implement an MFA solution for the organization's email system (e.g., \texttt{[Domain]}).
    \item \textbf{Action 2:} Enforce MFA for all user accounts, especially those with administrative privileges.
    \item \textbf{Action 3:} Identify all systems classified as containing sensitive data and enforce MFA for access.
\end{itemize}

\subsubsection{Unsupported Operating System}
\begin{itemize}
    \item \textbf{Action 1:} Accelerate the plan to upgrade all remaining Windows 7 workstations to a supported operating system, such as Windows 10 or Windows 11.
    \item \textbf{Action 2:} Isolate any Windows 7 machines that cannot be immediately upgraded from the main network to limit their exposure.
\end{itemize}

\subsection{Medium Priority (Remediate within 90 days)}
\subsubsection{Inadequate Security Policies \& Training}
\begin{itemize}
    \item \textbf{Action 1:} Develop and ratify a formal Employee Acceptable Use Policy (AUP) that all employees must read and acknowledge.
    \item \textbf{Action 2:} Implement a mandatory, annual security awareness training program for all employees. This program should cover topics such as phishing, password hygiene, and data handling.
\end{itemize}

\end{document}
```