```latex
\documentclass[12pt, a4paper]{article}

% Preamble: Required Packages
\usepackage[margin=1in]{geometry}
\usepackage{pifont} % For checkmarks and crosses
\usepackage{booktabs} % For professional tables
\usepackage{hyperref} % For clickable links and table of contents
\usepackage{url} % For formatting URLs
\usepackage{xcolor} % For colors
\usepackage{graphicx} % For logo
\usepackage{seqsplit} % For splitting long strings like IPs

% --- Document Setup ---
% Define colors for risk levels
\definecolor{criticalred}{HTML}{D73B3E}
\definecolor{highorange}{HTML}{F5A623}
\definecolor{mediumyellow}{HTML}{F8E71C}
\definecolor{lowblue}{HTML}{4A90E2}
\definecolor{infogray}{HTML}{9B9B9B}

% Hyperref setup
\hypersetup{
    colorlinks=true,
    linkcolor=blue,
    filecolor=magenta,      
    urlcolor=cyan,
    pdftitle={Cybersecurity Assessment Report},
    pdfpagemode=FullScreen,
}

% --- Document Start ---
\begin{document}

% --- Title Page ---
\begin{titlepage}
    \centering
    \vspace*{1cm}
    
    \Huge
    \textbf{Cybersecurity Assessment Report}
    
    \vspace{1.5cm}
    
    \Large
    Prepared for: \\
    \vspace{0.5cm}
    \textbf{[Organization Name]}
    
    \vspace{2cm}
    
    \large
    \textbf{Report Date:} \today \\
    \textbf{Author:} Cybersecurity Analyst
    
    \vfill
    
    \large
    \textit{This report contains sensitive information and should be handled with the utmost confidentiality.}
    
\end{titlepage}

% --- Table of Contents ---
\tableofcontents
\newpage

% --- Section 1: Executive Summary ---
\section{Executive Summary}
This report provides a comprehensive cybersecurity assessment for \textbf{[Organization Name]}, based on an analysis of network scan data, organizational security controls, and known risks. The assessment identified critical deficiencies across both administrative and technical security domains, resulting in a high-risk security posture.

Key findings indicate a complete absence of fundamental security controls. All reviewed administrative controls, including Multi-Factor Authentication (MFA), employee security training, and an acceptable use policy, are not implemented. This exposes the organization to a wide range of threats, including phishing, credential theft, and insider threats.

Technically, the external network scan revealed a web server operating over an unencrypted channel (HTTP on port 80). This presents a significant risk of data interception and credential compromise for any user or system interacting with it.

Immediate and decisive action is required to remediate these critical vulnerabilities. Recommendations prioritize the implementation of MFA, the development of foundational security policies and training programs, and the securing of external-facing services with industry-standard encryption.

% --- Section 2: Organizational Information ---
\section{Organizational Information}
The following details were used as the basis for this assessment. Due to the anonymized nature of the provided data, placeholders have been used where necessary.

\begin{table}[h!]
\centering
\begin{tabular}{@{}ll@{}}
\toprule
\textbf{Attribute} & \textbf{Value} \\ \midrule
Organization Name & \textbf{[Organization Name]} \\
Primary Email Domain & \texttt{[Domain]} \\
External IP Address (Scanned) & \texttt{[Client IP]} \\ \bottomrule
\end{tabular}
\caption{Client Organizational Details.}
\label{tab:org_info}
\end{table}

% --- Section 3: Security Control Review ---
\section{Security Control Review (Questionnaire Analysis)}
A review of the organization's administrative security controls was conducted via a questionnaire. The responses indicate critical gaps in foundational security practices. A summary of the findings is presented in Table \ref{tab:controls}. The symbol \ding{55} denotes a "No" response, highlighting a control gap.

\begin{table}[h!]
\centering
\begin{tabular}{@{}p{0.5\textwidth}cp{0.3\textwidth}@{}}
\toprule
\textbf{Control Question} & \textbf{Response} & \textbf{Analyst Assessment} \\ \midrule
Do you require MFA to access email? & \ding{55} & \textcolor{criticalred}{\textbf{Critical Gap.}} Lack of MFA on email exposes the primary communication channel to account takeover. \\
\addlinespace
Do you require MFA to log into computers? & \ding{55} & \textcolor{criticalred}{\textbf{Critical Gap.}} Compromised credentials can lead directly to endpoint and internal network access. \\
\addlinespace
Do you require MFA to access sensitive data systems? & \ding{55} & \textcolor{criticalred}{\textbf{Critical Gap.}} The organization's most valuable data is protected only by single-factor authentication. \\
\addlinespace
Does your organization have an employee acceptable use policy? & \ding{55} & \textcolor{highorange}{\textbf{High Risk.}} Without a policy, there is no enforceable standard for employee behavior or device usage. \\
\addlinespace
Does your organization do security awareness training for new employees? & \ding{55} & \textcolor{highorange}{\textbf{High Risk.}} New hires are not equipped with the knowledge to identify and avoid common threats like phishing. \\
\addlinespace
Does your organization do security awareness training for all employees at least once per year? & \ding{55} & \textcolor{criticalred}{\textbf{Critical Gap.}} The lack of ongoing training leaves the entire workforce vulnerable to evolving social engineering tactics. \\ \bottomrule
\end{tabular}
\caption{Analysis of Security Control Questionnaire.}
\label{tab:controls}
\end{table}

% --- Section 4: Technical Scan Results ---
\section{Technical Scan Results}
An external network scan was performed to identify open ports and services exposed to the internet. The scan targeted the IP address provided for the assessment.

\begin{itemize}
    \item \textbf{Target IP Address:} \texttt{[Target IP]}
    \item \textbf{Scan Tool:} Nmap
    \item \textbf{Scan Date:} \today
\end{itemize}

The scan identified one open port, as detailed in Table \ref{tab:scan_results}.

\begin{table}[h!]
\centering
\begin{tabular}{@{}llll@{}}
\toprule
\textbf{Port} & \textbf{State} & \textbf{Service (Inferred)} & \textbf{Analysis} \\ \midrule
80/tcp & Open & HTTP & \parbox{0.5\textwidth}{\textcolor{highorange}{\textbf{High Risk.}} This port serves web content over the Hypertext Transfer Protocol (HTTP), which is unencrypted. All data, including potential login credentials or sensitive information, is transmitted in cleartext. This makes the service highly susceptible to eavesdropping and Man-in-the-Middle (MitM) attacks.} \\ \bottomrule
\end{tabular}
\caption{External Network Scan Findings.}
\label{tab:scan_results}
\end{table}

% --- Section 5: Consolidated Risk Assessment ---
\section{Consolidated Risk Assessment}
This section synthesizes findings from the security control review and technical scan into a consolidated list of identified risks. Each risk is assigned a severity level based on its potential impact and likelihood.

\begin{table}[h!]
\centering
\begin{tabular}{@{}p{0.1\textwidth}p{0.4\textwidth}p{0.25\textwidth}l@{}}
\toprule
\textbf{Risk ID} & \textbf{Description} & \textbf{Affected Elements} & \textbf{Severity} \\ \midrule
RISK-001 & \textbf{Lack of Multi-Factor Authentication (MFA):} User accounts are protected only by passwords, making them highly vulnerable to compromise via phishing, credential stuffing, or password spraying. & All user accounts, email system, endpoints, sensitive data systems. & \textcolor{criticalred}{\textbf{Critical}} \\
\addlinespace
RISK-002 & \textbf{Inadequate Security Policies and Training:} The absence of an Acceptable Use Policy and security awareness training creates a culture where employees are unaware of security best practices and cyber threats. & All employees, organizational data, network integrity. & \textcolor{criticalred}{\textbf{Critical}} \\
\addlinespace
RISK-003 & \textbf{Unencrypted Web Traffic:} The public-facing web server uses HTTP instead of HTTPS. This exposes all transmitted data to interception, including user credentials and session cookies. & External Server (\texttt{[Target IP]}), website visitors, any user authenticating via the site. & \textcolor{highorange}{\textbf{High}} \\ \bottomrule
\end{tabular}
\caption{Summary of Identified Risks.}
\label{tab:risks}
\end{table}

% --- Section 6: Recommendations ---
\section{Recommendations}
The following actions are recommended to mitigate the identified risks and improve the overall security posture of \textbf{[Organization Name]}. Recommendations are prioritized based on severity.

\begin{enumerate}
    \item \textbf{Implement Multi-Factor Authentication (Critical)} \\
    Immediately enforce MFA across all critical systems. The implementation should be prioritized as follows:
    \begin{itemize}
        \item Email system (e.g., Office 365, Google Workspace).
        \item Remote access solutions (VPNs).
        \item Access to all systems containing sensitive or regulated data.
        \item Endpoint logins for all employees.
    \end{itemize}

    \item \textbf{Develop and Implement Foundational Security Policies (Critical)} \\
    Establish a baseline for secure operations by creating, approving, and communicating an \textit{Acceptable Use Policy (AUP)}. This policy should clearly define the rules for using company assets, data, and networks.

    \item \textbf{Establish a Security Awareness Training Program (Critical)} \\
    Implement a mandatory security awareness training program for all employees. This program must be conducted for all new hires upon onboarding and for all staff at least annually. Training should cover phishing, password hygiene, social engineering, and the new AUP.

    \item \textbf{Secure Web Services with HTTPS (High)} \\
    Immediately decommission the unencrypted HTTP service on port 80. Obtain and install a TLS/SSL certificate on the web server at \texttt{[Target IP]} and configure it to serve all traffic exclusively over HTTPS (port 443). Implement HTTP Strict Transport Security (HSTS) to enforce encrypted connections.
\end{enumerate}

\end{document}
```