```latex
\documentclass[12pt]{article}

% Preamble: Required Packages
\usepackage[a4paper, margin=1in]{geometry}
\usepackage{pifont} % For checkmarks and crosses
\usepackage{booktabs} % For professional tables
\usepackage{hyperref} % For clickable links
\usepackage{url} % For formatting URLs
\usepackage{seqsplit} % For splitting long strings in texttt
\usepackage[T1]{fontenc}

% Document Metadata
\title{Cybersecurity Posture Assessment Report}
\author{Cybersecurity Analysis Division}
\date{\today}

\hypersetup{
    colorlinks=true,
    linkcolor=black,
    urlcolor=blue,
    pdftitle={Cybersecurity Posture Assessment Report},
    pdfauthor={Cybersecurity Analysis Division},
}

\begin{document}

\maketitle
\thispagestyle{empty}
\newpage

\tableofcontents
\newpage

% --- 1. Executive Summary ---
\section{Executive Summary}

This report provides a cybersecurity assessment for \textbf{[Organization Name]}, based on an analysis of network scan data, organizational security controls, and pre-existing risk information.

The assessment has identified a \textbf{critical risk} related to the direct exposure of Remote Desktop Protocol (RDP) on port 3389 to the public internet. This configuration is a primary target for ransomware attacks and unauthorized access attempts. This technical vulnerability is critically compounded by significant gaps in organizational security controls, most notably the lack of Multi-Factor Authentication (MFA) for computer and sensitive data system access.

Furthermore, foundational governance controls, such as a formal Acceptable Use Policy and mandatory annual security training for all employees, are not in place. These deficiencies create an environment where both technical and human-related security incidents are more likely to occur and have a greater impact.

Immediate remediation is required to address the exposed RDP service. Strategic initiatives must be undertaken to implement comprehensive MFA and establish foundational security policies and training programs.

% --- 2. Organizational Information ---
\section{Organizational Information}

The following information was used as a baseline for this assessment. Due to the anonymized nature of the provided data, placeholders have been used where necessary.

\begin{itemize}
    \item \textbf{Organization Name:} \textbf{[Organization Name]}
    \item \textbf{Email Domain:} \texttt{[Domain]}
    \item \textbf{External IP Address Scanned:} \texttt{[Client IP]}
\end{itemize}

% --- 3. Security Control Review ---
\section{Security Control Review}

A review of the organization's security controls was conducted via a questionnaire. The responses highlight significant gaps in access control and security governance. A "No" response indicates a missing control and a potential area of high risk.

\begin{table}[h!]
\centering
\caption{Organizational Security Controls Questionnaire}
\label{tab:controls}
\begin{tabular}{p{0.7\linewidth} c}
\toprule
\textbf{Control Question} & \textbf{Response} \\
\midrule
Do you require MFA to access email? & \ding{51} \\ % Yes
Do you require MFA to log into computers? & \textbf{\color{red}\ding{55}} \\ % No
Do you require MFA to access sensitive data systems? & \textbf{\color{red}\ding{55}} \\ % No
Does your organization have an employee acceptable use policy? & \textbf{\color{red}\ding{55}} \\ % No
Does your organization do security awareness training for new employees? & \ding{51} \\ % Yes
Does your organization do security awareness training for all employees at least once per year? & \textbf{\color{red}\ding{55}} \\ % No
\bottomrule
\end{tabular}
\end{table}

\subsection*{Analysis of Control Gaps}
\begin{itemize}
    \item \textbf{Lack of MFA:} The absence of MFA for computer and sensitive data system logins is a critical weakness. This allows an attacker with valid credentials (e.g., obtained via phishing or brute force) to gain direct access to internal systems without a secondary authentication challenge.
    \item \textbf{Policy and Training Gaps:} The lack of an Acceptable Use Policy and annual security training for all staff indicates a low level of security maturity. This increases the risk of insider threats (both malicious and accidental) and makes employees more vulnerable to social engineering attacks.
\end{itemize}

% --- 4. Technical Scan Results ---
\section{Technical Scan Results}

An external network scan was performed to identify open ports and exposed services.

\begin{itemize}
    \item \textbf{Target IP Address:} \texttt{[Target IP]}
    \item \textbf{Scan Utility:} Nmap
\end{itemize}

\begin{table}[h!]
\centering
\caption{Open Ports Detected on \texttt{[Target IP]}}
\label{tab:nmap}
\begin{tabular}{l l l l}
\toprule
\textbf{Port} & \textbf{State} & \textbf{Service} & \textbf{Analysis} \\
\midrule
3389/tcp & open & ms-wbt-server & Critical Risk \\
\bottomrule
\end{tabular}
\end{table}

\subsection*{Analysis of Technical Findings}
The scan identified that port \textbf{3389/tcp} is open to the internet. This port is used for Microsoft's Remote Desktop Protocol (RDP). Exposing RDP directly to the public internet is an extremely dangerous practice and is a common attack vector for threat actors.
\begin{itemize}
    \item \textbf{Brute-Force Attacks:} Automated tools constantly scan the internet for open RDP ports and attempt to guess or use stolen credentials to gain access.
    \item \textbf{Ransomware Vector:} Many ransomware gangs, such as Conti and REvil, specifically target exposed RDP to gain initial access to a network before deploying their malware.
    \item \textbf{Exploitation:} RDP has had several critical remote code execution vulnerabilities in the past (e.g., BlueKeep - CVE-2019-0708). Even if patched, it remains a high-value target.
\end{itemize}
This finding directly confirms the pre-existing risk documented in the organization's risk register ("RDP Exposure") and elevates its urgency.

% --- 5. Correlated Risk Assessment ---
\section{Correlated Risk Assessment}

This section synthesizes findings from the security control review, technical scan, and existing risk data into a prioritized list of risks facing the organization.

\begin{table}[h!]
\centering
\caption{Summary of Identified Risks}
\label{tab:risks}
\begin{tabular}{p{0.1\linewidth} p{0.5\linewidth} p{0.15\linewidth} p{0.15\linewidth}}
\toprule
\textbf{ID} & \textbf{Risk Description} & \textbf{Severity} & \textbf{Source} \\
\midrule
\textbf{R-01} & \textbf{Direct RDP Exposure on the Internet Perimeter.} This allows attackers to attempt brute-force logins or exploit RDP vulnerabilities to gain unauthorized remote access to the internal network. & \textbf{Critical} & Technical Scan, Existing Risks \\
\addlinespace
\textbf{R-02} & \textbf{Lack of Multi-Factor Authentication (MFA) for System Access.} The absence of MFA on computers and sensitive systems means that a single compromised password provides an attacker with direct access. This risk critically compounds R-01. & \textbf{Critical} & Questionnaire \\
\addlinespace
\textbf{R-03} & \textbf{Inadequate Security Governance and Training.} The lack of an Acceptable Use Policy and mandatory annual training increases the likelihood of human error leading to security incidents, such as credential compromise. & \textbf{High} & Questionnaire \\
\bottomrule
\end{tabular}
\end{table}

% --- 6. Recommendations ---
\section{Recommendations}

The following actions are recommended to mitigate the identified risks. They are prioritized based on severity and potential impact.

\subsection{Immediate Actions (To Be Completed within 24 Hours)}
\begin{enumerate}
    \item \textbf{Mitigate RDP Exposure (R-01):} Immediately implement a firewall rule to block all inbound traffic to TCP port 3389 on the external IP address \texttt{[Target IP]}.
    \item \textbf{Implement Secure Remote Access:} If remote access is required, it must be placed behind a Virtual Private Network (VPN) that requires MFA for all connections. Do not re-enable RDP access from the internet.
\end{enumerate}

\subsection{High-Priority Actions (To Be Completed within 30 Days)}
\begin{enumerate}
    \item \textbf{Deploy MFA (R-02):} Procure and deploy an MFA solution for all remote access (VPN) and all user logins to computers and sensitive data systems.
    \item \textbf{Develop Acceptable Use Policy (R-03):} Create and ratify a formal Acceptable Use Policy (AUP) that all employees must read and acknowledge. This policy should govern the use of company assets, data handling, and security responsibilities.
    \item \textbf{Implement Annual Security Training (R-03):} Enroll all employees in a mandatory annual security awareness training program. The training should cover topics such as phishing, password security, and the new AUP.
\end{enumerate}

\subsection{General Recommendations}
\begin{enumerate}
    \item \textbf{Conduct Comprehensive Vulnerability Scanning:} Perform authenticated vulnerability scans on both external and internal networks to identify other potential weaknesses, such as unpatched software or misconfigurations.
    \item \textbf{Review Firewall Rules:} Conduct a full audit of all firewall rules to ensure that only necessary ports are open to the internet and that the principle of least privilege is applied.
\end{enumerate}

\end{document}
```