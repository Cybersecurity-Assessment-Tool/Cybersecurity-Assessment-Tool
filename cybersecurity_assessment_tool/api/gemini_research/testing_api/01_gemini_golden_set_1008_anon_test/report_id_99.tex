```latex
\documentclass[12pt]{article}

% === PACKAGES ===
\usepackage[margin=1in]{geometry}
\usepackage{pifont} % For \ding symbols
\usepackage{booktabs} % For professional tables
\usepackage{hyperref} % For clickable links
\usepackage{url} % For URL formatting
\usepackage{seqsplit} % To split long strings in tt font
\usepackage{xcolor} % For colors
\usepackage{graphicx} % For logo (placeholder)

% === DOCUMENT SETUP ===
\hypersetup{
    colorlinks=true,
    linkcolor=blue,
    filecolor=magenta,      
    urlcolor=cyan,
    pdftitle={Cybersecurity Assessment Report},
    pdfauthor={Cybersecurity Analyst},
    pdfsubject={Security Assessment},
    pdfkeywords={Security, Risk, Assessment},
}

% === TITLE SECTION ===
\title{
    \vspace{-1.5cm}
    \begin{flushleft}
        \rule{\linewidth}{1pt} \\
        \LARGE \textbf{Cybersecurity Assessment Report} \\
        \rule{\linewidth}{1pt}
    \end{flushleft}
}
\author{Cybersecurity Analyst}
\date{\today}

% ==============================================================================
% === BEGIN DOCUMENT ===========================================================
% ==============================================================================
\begin{document}
\maketitle
\thispagestyle{empty}
\newpage

\tableofcontents
\newpage

% ==============================================================================
% 1. EXECUTIVE SUMMARY
% ==============================================================================
\section{Executive Summary}

This report details the findings of a cybersecurity assessment conducted for \textbf{[Organization Name]}. The analysis correlates results from an external network scan, a review of existing risks, and a security controls questionnaire.

The assessment identified several high-risk and critical vulnerabilities that require immediate attention. The most critical finding is a \textbf{publicly exposed MySQL database service} on port 3306. This service is running an outdated version (5.7.33) with known vulnerabilities, presenting a direct and significant threat of data breach.

This technical vulnerability is compounded by critical gaps in administrative security controls. Specifically, the organization \textbf{lacks Multi-Factor Authentication (MFA)} for accessing email and sensitive data systems. Furthermore, there is a complete absence of a formal \textbf{security awareness training program} for employees.

This combination of an exposed, vulnerable system and weak access/human controls creates a high-likelihood path for threat actors to compromise sensitive data. Immediate remediation of the exposed database and implementation of MFA are paramount to reducing the organization's risk profile. This report provides detailed findings and actionable recommendations to address these issues.

% ==============================================================================
% 2. ORGANIZATIONAL INFORMATION
% ==============================================================================
\section{Organizational Information}

The following details were used as the basis for this assessment. Placeholder values are used where specific data was not provided.

\begin{itemize}
    \item \textbf{Organization Name:} \textbf{[Organization Name]}
    \item \textbf{Primary Email Domain:} \texttt{[Domain]}
    \item \textbf{External IP Address Scanned:} \texttt{[Client IP]}
\end{itemize}

% ==============================================================================
% 3. SECURITY CONTROL REVIEW (QUESTIONNAIRE)
% ==============================================================================
\section{Security Control Review}

A review of the organization's administrative security controls was conducted via a questionnaire. The results highlight significant gaps in user access controls and employee security training. "No" answers indicate a deviation from security best practices and are flagged as risks.

\begin{table}[h!]
\centering
\caption{Security Controls Questionnaire Analysis}
\begin{tabular}{p{0.6\linewidth} c p{0.2\linewidth}}
\toprule
\textbf{Control Question} & \textbf{Status} & \textbf{Assessment} \\
\midrule
Do you require MFA to access email? & \ding{55} & \textcolor{red}{\textbf{Critical Gap}} \\
Do you require MFA to log into computers? & \ding{51} & Implemented \\
Do you require MFA to access sensitive data systems? & \ding{55} & \textcolor{red}{\textbf{Critical Gap}} \\
Does your organization have an employee acceptable use policy? & \ding{51} & Implemented \\
Does your organization do security awareness training for new employees? & \ding{55} & \textcolor{orange}{High Risk} \\
Does your organization do security awareness training for all employees at least once per year? & \ding{55} & \textcolor{orange}{High Risk} \\
\bottomrule
\end{tabular}
\end{table}

% ==============================================================================
% 4. TECHNICAL SCAN RESULTS
% ==============================================================================
\section{Technical Scan Results}

An external network scan was performed to identify open ports and exposed services. The scan was conducted on \today.

\begin{itemize}
    \item \textbf{Scan Target:} \texttt{[Target IP]}
\end{itemize}

The scan revealed one open port, which corresponds to a publicly accessible database service. Details are provided in Table 2.

\begin{table}[h!]
\centering
\caption{Open Port Analysis}
\begin{tabular}{l l l l l}
\toprule
\textbf{Port} & \textbf{State} & \textbf{Service} & \textbf{Version} & \textbf{Notes} \\
\midrule
3306/tcp & Open & MySQL & 5.7.33 & \parbox[t]{5cm}{\textcolor{red}{\textbf{Critical Risk.}} Publicly exposed database. This version is outdated and has known public vulnerabilities (e.g., CVE-2021-2154).} \\
\bottomrule
\end{tabular}
\end{table}

% ==============================================================================
% 5. CONSOLIDATED RISK ASSESSMENT
% ==============================================================================
\section{Consolidated Risk Assessment}

This section synthesizes findings from all data sources into a consolidated list of identified risks. Each risk is assigned a severity level based on its potential impact and likelihood of exploitation.

\begin{table}[h!]
\centering
\caption{Risk Summary}
\begin{tabular}{p{0.25\linewidth} p{0.12\linewidth} p{0.53\linewidth}}
\toprule
\textbf{Risk Name} & \textbf{Severity} & \textbf{Description} \\
\midrule
\textbf{Public Database Exposure} & \textcolor{red}{Critical (7.5)} & The MySQL database service on port 3306 is exposed to the public internet. The running version (5.7.33) is known to be vulnerable, allowing potential attackers to compromise, exfiltrate, or destroy data. This finding was confirmed by both the network scan and pre-existing risk data. \\
\addlinespace
\textbf{Lack of Multi-Factor Authentication (MFA)} & \textcolor{red}{Critical} & The absence of MFA for email and sensitive data systems means that a compromised password is all an attacker needs to gain access. This drastically increases the risk of business email compromise and data breaches. \\
\addlinespace
\textbf{Insufficient Security Awareness Training} & \textcolor{orange}{High} & Without a formal training program, employees are significantly more susceptible to phishing and other social engineering attacks. This weakness serves as a primary vector for initial credential compromise, which can then be used to exploit other weaknesses like the lack of MFA. \\
\bottomrule
\end{tabular}
\end{table}

% ==============================================================================
% 6. RECOMMENDATIONS
% ==============================================================================
\section{Recommendations}

The following actionable recommendations are provided to mitigate the identified risks. They are prioritized based on severity and urgency.

\subsection{Immediate Actions (Critical Priority)}
\begin{enumerate}
    \item \textbf{Restrict Access to Database Port:} Immediately apply firewall rules to block all public access to TCP port 3306 on target \texttt{[Target IP]}. Access should be restricted to a whitelist of trusted internal IP addresses only.
    \item \textbf{Implement MFA on Critical Systems:} Prioritize the deployment of MFA for all users on the primary email system (\texttt{[Domain]}) and all systems identified as containing sensitive data. This is the single most effective control to prevent account takeovers.
\end{enumerate}

\subsection{Short-Term Actions (High Priority)}
\begin{enumerate}
    \item \textbf{Patch Vulnerable MySQL Server:} Plan and execute an upgrade of the MySQL 5.7.33 server to a currently supported and patched version. This will mitigate the risk posed by known, exploitable vulnerabilities.
    \item \textbf{Deploy Foundational Security Training:} Enroll all employees in a security awareness training program. The training must cover, at a minimum, phishing identification, password security, and the organization's acceptable use policy. This program should be mandatory for all new hires.
\end{enumerate}

\subsection{Long-Term Strategic Improvements}
\begin{enumerate}
    \item \textbf{Implement Network Segmentation:} Develop a strategy to move critical infrastructure, such as databases, into a separate, non-public network zone. Access to this zone should only be permitted via a secure gateway, such as a Virtual Private Network (VPN) or bastion host.
    \item \textbf{Establish an Annual Training Cadence:} Formalize the security awareness program by making it a mandatory annual requirement for all employees to ensure that security knowledge remains current.
\end{enumerate}

\end{document}
```