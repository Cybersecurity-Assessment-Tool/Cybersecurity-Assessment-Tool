```latex
\documentclass[12pt]{article}

% ----------------------------------------------------------------------
%  PACKAGES
% ----------------------------------------------------------------------
\usepackage[margin=1in]{geometry}
\usepackage{pifont} % For checkmarks and crosses
\usepackage{booktabs} % For professional tables
\usepackage{hyperref} % For hyperlinks
\usepackage{url}      % For URL formatting
\usepackage{seqsplit} % For splitting long strings in texttt
\usepackage{graphicx} % For logo (optional)
\usepackage{xcolor}   % For colors

% ----------------------------------------------------------------------
%  DOCUMENT CONFIGURATION
% ----------------------------------------------------------------------
\hypersetup{
    colorlinks=true,
    linkcolor=blue,
    filecolor=magenta,      
    urlcolor=cyan,
    pdftitle={Cybersecurity Posture Report},
    pdfpagemode=FullScreen,
}

\newcommand{\yes}{\ding{51}}
\newcommand{\no}{\ding{55}}

% ----------------------------------------------------------------------
%  TITLE SECTION
% ----------------------------------------------------------------------
\title{
    \vspace{-2cm}
    \rule{\textwidth}{2pt}\\[0.5cm]
    \textbf{Cybersecurity Posture Report}\\[0.2cm]
    \large For: \textbf{[Organization Name]}\\[0.5cm]
    \rule{\textwidth}{1pt}
}
\author{Cybersecurity Analyst}
\date{\today}

% ----------------------------------------------------------------------
%  BEGIN DOCUMENT
% ----------------------------------------------------------------------
\begin{document}

\maketitle
\thispagestyle{empty}
\newpage

\tableofcontents
\newpage

% ----------------------------------------------------------------------
%  1. EXECUTIVE SUMMARY
% ----------------------------------------------------------------------
\section{Executive Summary}

This report provides a comprehensive analysis of the cybersecurity posture for \textbf{[Organization Name]}, based on data collected from a network vulnerability scan, a security controls questionnaire, and a review of pre-existing risks.

The assessment reveals a mixed security posture. On the one hand, the organization demonstrates strong foundational controls in identity and access management, with Multi-Factor Authentication (MFA) consistently enforced across email, computers, and sensitive data systems. Furthermore, the external network scan of the target IP address revealed no open ports, indicating a robust firewall configuration and a minimal external attack surface.

However, significant administrative and procedural gaps were identified that present a high level of risk. The absence of a formal Employee Acceptable Use Policy (AUP) and the lack of mandatory annual security awareness training for all staff are critical deficiencies. These gaps leave the organization vulnerable to insider threats, social engineering, and phishing attacks, which often bypass technical controls.

Immediate action is recommended to address these policy and training-related risks to build a more resilient and comprehensive security program.

% ----------------------------------------------------------------------
%  2. ORGANIZATIONAL INFORMATION
% ----------------------------------------------------------------------
\section{Organizational Information}

This section details the information provided by the client organization. The data is used to establish the context for the security assessment.

\begin{itemize}
    \item \textbf{Organization Name:} \textbf{[Organization Name]}
    \item \textbf{Primary Domain:} \texttt{[Domain]}
    \item \textbf{External IP Scanned:} \texttt{[Client IP]}
\end{itemize}

% ----------------------------------------------------------------------
%  3. SECURITY CONTROL REVIEW
% ----------------------------------------------------------------------
\section{Security Control Review}

The following table summarizes the organization's responses to a security controls questionnaire. A green checkmark (\yes) indicates a control is in place, while a red cross (\no) highlights a potential gap. Gaps identified here are directly correlated with the risks outlined in Section 5.

\begin{table}[h!]
\centering
\begin{tabular}{p{0.75\linewidth} c}
\toprule
\textbf{Control Question} & \textbf{Response} \\
\midrule
Do you require MFA to access email? & \yes \\
Do you require MFA to log into computers? & \yes \\
Do you require MFA to access sensitive data systems? & \yes \\
Does your organization have an employee acceptable use policy? & \textcolor{red}{\no} \\
Does your organization do security awareness training for new employees? & \yes \\
Does your organization do security awareness training for all employees at least once per year? & \textcolor{red}{\no} \\
\bottomrule
\end{tabular}
\caption{Security Controls Questionnaire Results}
\end{table}

\textbf{Analysis:} The organization has successfully implemented critical MFA controls, significantly reducing the risk of unauthorized access via compromised credentials. However, the lack of an Acceptable Use Policy and annual security awareness training for all employees are critical administrative failings that expose the organization to significant human-centric risks.

% ----------------------------------------------------------------------
%  4. TECHNICAL SCAN RESULTS
% ----------------------------------------------------------------------
\section{Technical Scan Results}

An external network scan was performed to identify open ports and exposed services on the organization's public-facing infrastructure.

\begin{itemize}
    \item \textbf{Target IP Address:} \texttt{[Target IP]}
    \item \textbf{Scan Date:} \today
    \item \textbf{Scanner Used:} Nmap
\end{itemize}

\subsection{Summary of Findings}
The scan results were positive, indicating a strong network perimeter defense.
\begin{itemize}
    \item \textbf{Host Status:} Up
    \item \textbf{Open Ports Found:} 0
    \item \textbf{Filtered/Closed Ports:} All 1000 scanned ports were in a 'closed' state.
\end{itemize}

\textbf{Conclusion:} No listening services were detected on the target system. This suggests that a firewall is properly configured to block unsolicited incoming traffic, which is a security best practice. No technical vulnerabilities were identified from this external scan.

% ----------------------------------------------------------------------
%  5. RISK ASSESSMENT
% ----------------------------------------------------------------------
\section{Risk Assessment}

This section synthesizes findings from the security control review, technical scan, and pre-existing risk data. The following risks have been identified and prioritized based on their potential impact on the organization.

\begin{table}[h!]
\centering
\begin{tabular}{p{0.2\linewidth} p{0.55\linewidth} p{0.15\linewidth}}
\toprule
\textbf{Risk Name} & \textbf{Overview} & \textbf{Severity} \\
\midrule
\textbf{Lack of Employee Acceptable Use Policy} & Without a formal AUP, employees lack clear guidelines on the acceptable use of company assets, data, and systems. This increases the risk of intentional or unintentional misuse, data leakage, and potential legal or compliance violations. & \textbf{Critical} \\
\addlinespace[0.5em]
\textbf{Lack of Annual Security Awareness Training} & Without regular, recurring training, employee awareness of evolving threats like phishing, malware, and social engineering degrades. This makes the organization highly susceptible to human-centric attacks that target employees as the weakest link. & \textbf{High} \\
\bottomrule
\end{tabular}
\caption{Identified Cybersecurity Risks}
\end{table}

\textit{Note: No pre-existing vulnerabilities were provided for this assessment.}

% ----------------------------------------------------------------------
%  6. RECOMMENDATIONS
% ----------------------------------------------------------------------
\section{Recommendations}

Based on the identified risks, the following prioritized actions are recommended to improve the organization's cybersecurity posture.

\subsection{Priority 1: Critical Risk Mitigation}
\begin{itemize}
    \item \textbf{Action:} Develop and implement a comprehensive Employee Acceptable Use Policy (AUP).
    \item \textbf{Details:} This policy should clearly define the rules and expectations for using company networks, devices, software, and data. It must be formally reviewed, approved by management, and acknowledged in writing by all new and existing employees. The AUP serves as a foundational document for security governance.
\end{itemize}

\subsection{Priority 2: High Risk Mitigation}
\begin{itemize}
    \item \textbf{Action:} Establish a mandatory annual security awareness training program for all employees.
    \item \textbf{Details:} The training program should be conducted at least once per year and cover current cyber threats, such as phishing, ransomware, and social engineering. It should also reinforce company policies (including the new AUP) and outline procedures for reporting security incidents.
\end{itemize}

% ----------------------------------------------------------------------
%  7. CONCLUSION
% ----------------------------------------------------------------------
\section{Conclusion}

\textbf{[Organization Name]} has established a commendable technical security foundation with strong MFA enforcement and a secure network perimeter. However, the organization's current posture is undermined by critical gaps in administrative controls. The human element remains the most significant area of risk.

By implementing the recommendations in this report—specifically by creating an Acceptable Use Policy and instituting annual security awareness training—the organization can significantly reduce its susceptibility to common cyberattacks and build a more robust, defense-in-depth security culture.

\end{document}
```