```latex
\documentclass[12pt, a4paper]{article}

% Preamble: Required Packages
\usepackage[margin=1in]{geometry}
\usepackage{pifont} % For checkmarks and crosses
\usepackage{booktabs} % For professional-looking tables
\usepackage{hyperref} % For clickable links
\usepackage{url} % For formatting URLs
\usepackage{seqsplit} % For splitting long strings in texttt
\usepackage[utf8]{inputenc}

% Document Metadata
\title{Cybersecurity Posture Assessment Report}
\author{Cybersecurity Analysis Division}
\date{\today}

% Hyperref Setup
\hypersetup{
    colorlinks=true,
    linkcolor=black,
    filecolor=magenta,      
    urlcolor=blue,
    pdftitle={Cybersecurity Posture Assessment Report},
    pdfpagemode=FullScreen,
}

\begin{document}

\maketitle
\thispagestyle{empty}
\newpage

\tableofcontents
\newpage

% --- 1. Executive Summary ---
\section{Executive Summary}

This report provides a comprehensive analysis of the cybersecurity posture for \textbf{[Organization Name]}. The assessment is based on a correlation of network scan data, a security controls questionnaire, and a review of pre-existing documented risks.

The analysis reveals a mixed security posture. The organization has implemented several important controls, including mandatory Multi-Factor Authentication (MFA) for computer and sensitive system access, and maintains a security awareness training program.

However, several critical and high-risk vulnerabilities were identified that require immediate attention. A critical pre-existing risk, ``Localhost Exposed,'' with a CVSS score of 10.0, remains a primary concern. Furthermore, a critical administrative gap was found: the lack of mandatory MFA for email access, which exposes the organization to significant risk from phishing and account takeover attacks. The external network scan also identified an open SSH port (\texttt{22/tcp}) on a key asset, which could serve as a direct entry point for attackers if not properly secured.

This report details these findings and provides actionable recommendations to mitigate the identified risks and strengthen the overall security posture of \textbf{[Organization Name]}.

% --- 2. Organizational Information ---
\section{Organizational Information}

This section outlines the basic information for the organization under review. The data provided was anonymized for this report.

\begin{itemize}
    \item \textbf{Organization Name:} \textbf{[Organization Name]}
    \item \textbf{Primary Email Domain:} \texttt{[Domain]}
    \item \textbf{Scanned External IP:} \texttt{[Client IP]}
\end{itemize}

% --- 3. Security Control Review ---
\section{Security Control Review}

A security controls questionnaire was completed to assess the administrative and policy-based safeguards in place. The results are summarized below. A checkmark (\ding{51}) indicates a positive control, while a cross (\ding{55}) indicates a potential gap.

\begin{table}[h!]
\centering
\caption{Security Controls Questionnaire Results}
\begin{tabular}{p{0.8\linewidth} c}
\toprule
\textbf{Control Question} & \textbf{Status} \\
\midrule
Do you require MFA to access email? & \ding{55} \\
Do you require MFA to log into computers? & \ding{51} \\
Do you require MFA to access sensitive data systems? & \ding{51} \\
Does your organization have an employee acceptable use policy? & \ding{51} \\
Does your organization do security awareness training for new employees? & \ding{51} \\
Does your organization do security awareness training for all employees at least once per year? & \ding{51} \\
\bottomrule
\end{tabular}
\end{table}

\subsection*{Analysis of Gaps}
The most significant finding from the questionnaire is the \textbf{lack of mandatory MFA for email access}. Email is the primary vector for corporate phishing attacks, business email compromise (BEC), and initial access attempts. Without MFA, a single compromised password can lead to a full email account takeover, potentially exposing sensitive communications, enabling further internal phishing, and facilitating password resets for other integrated services. This is classified as a \textbf{Critical Risk}.

% --- 4. Technical Scan Results ---
\section{Technical Scan Results}

An external network scan was performed to identify open ports and exposed services on the organization's perimeter.

\begin{itemize}
    \item \textbf{Target IP Address:} \texttt{[Target IP]}
    \item \textbf{Scan Date:} \textbf{[Scan Date]}
    \item \textbf{Scanner Used:} Nmap
\end{itemize}

\subsection*{Open Ports Discovered}
The scan revealed the following open port on the target host:

\begin{table}[h!]
\centering
\caption{Open Ports on \texttt{[Target IP]}}
\begin{tabular}{c c c c}
\toprule
\textbf{Port} & \textbf{State} & \textbf{Service} & \textbf{Version} \\
\midrule
22/tcp & open & ssh & \textit{Not Determined} \\
\bottomrule
\end{tabular}
\end{table}

\subsection*{Technical Analysis}
The scan identified that port \texttt{22/tcp}, commonly used for the Secure Shell (SSH) protocol, is open to the public internet. SSH is a powerful administrative tool, and its exposure presents a significant risk. Potential threats include:
\begin{itemize}
    \item \textbf{Brute-force attacks:} Automated tools can be used to guess usernames and passwords.
    \item \textbf{Credential stuffing:} Attackers may use credentials stolen from other data breaches to attempt logins.
    \item \textbf{Exploitation of vulnerabilities:} If the SSH server software is outdated, it may be vulnerable to known exploits that could lead to remote code execution. The scan was unable to determine the software version, which prevents a full vulnerability assessment.
\end{itemize}
Exposing SSH directly to the internet is strongly discouraged. This finding is classified as a \textbf{High Risk}.

% --- 5. Correlated Risk Assessment ---
\section{Correlated Risk Assessment}

This section synthesizes findings from the security questionnaire, technical scan, and pre-existing risk documentation into a unified summary.

\begin{table}[h!]
\centering
\caption{Summary of Identified Risks}
\begin{tabular}{p{0.25\linewidth} p{0.45\linewidth} p{0.1\linewidth} p{0.1\linewidth}}
\toprule
\textbf{Risk Name} & \textbf{Description} & \textbf{Severity} & \textbf{Source} \\
\midrule
\textbf{Localhost Exposed} & A pre-existing critical vulnerability was documented with a CVSS score of 10.0. Details of this risk should be prioritized for immediate remediation. & Critical & Input 3 \\
\addlinespace
\textbf{No MFA on Email} & The lack of Multi-Factor Authentication on email accounts exposes the organization to account takeovers, phishing, and data breaches. & Critical & Input 2 \\
\addlinespace
\textbf{Exposed SSH Service} & Port 22 (SSH) is open to the public internet, creating a vector for brute-force attacks and potential exploitation of the service. & High & Input 1 \\
\bottomrule
\end{tabular}
\end{table}

% --- 6. Recommendations ---
\section{Recommendations}

Based on the correlated risk assessment, the following actions are recommended to mitigate the identified vulnerabilities and improve the overall security posture.

\subsection*{Immediate Priority (0-7 Days)}
\begin{enumerate}
    \item \textbf{Address Critical ``Localhost Exposed'' Vulnerability:} Immediately investigate and remediate the pre-existing risk documented as "Localhost Exposed". Given its CVSS 10.0 score, this represents the most severe threat.
    \item \textbf{Enforce MFA on All Email Accounts:} Immediately enable and enforce MFA for all user email accounts. This is the single most effective control to prevent email account takeovers.
\end{enumerate}

\subsection*{High Priority (7-30 Days)}
\begin{enumerate}
    \item \textbf{Secure the Exposed SSH Service:}
        \begin{itemize}
            \item \textbf{If not required for business:} Close port 22 on the external firewall immediately.
            \item \textbf{If required:} Restrict access to trusted IP addresses only (IP whitelisting) via firewall rules. If dynamic access is needed, place the service behind a Virtual Private Network (VPN).
            \item \textbf{Harden Configuration:} Disable password-based authentication and enforce the use of strong cryptographic keys (e.g., ED25519) for all SSH access.
        \end{itemize}
\end{enumerate}

\subsection*{Medium Priority (30-90 Days)}
\begin{enumerate}
    \item \textbf{Conduct Authenticated Vulnerability Scanning:} Perform a comprehensive, credentialed vulnerability scan on the asset at \texttt{[Target IP]} and other critical systems. This will provide detailed software version information and identify a wider range of vulnerabilities that are not visible from an external, unauthenticated perspective.
\end{enumerate}

\end{document}
```