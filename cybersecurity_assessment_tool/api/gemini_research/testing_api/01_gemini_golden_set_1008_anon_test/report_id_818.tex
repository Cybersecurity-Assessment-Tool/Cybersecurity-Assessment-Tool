```latex
\documentclass[12pt]{article}

% Preamble: Required Packages
\usepackage[margin=1in]{geometry}
\usepackage{pifont} % For checkmarks and crosses
\usepackage{booktabs} % For professional tables
\usepackage{hyperref} % For clickable links
\usepackage{url} % For formatting URLs
\usepackage{seqsplit} % For splitting long strings in texttt
\usepackage{xcolor} % For colors

% Document Information
\title{Cybersecurity Posture Assessment Report}
\author{Cybersecurity Analyst}
\date{\today}

% Define severity colors
\definecolor{sev_critical}{HTML}{990000}
\definecolor{sev_high}{HTML}{D14200}
\definecolor{sev_medium}{HTML}{E79400}

\begin{document}

\maketitle
\thispagestyle{empty}
\clearpage

\tableofcontents
\clearpage

% --- 1. Executive Summary ---
\section{Executive Summary}

This report provides a comprehensive cybersecurity assessment for \textbf{[Organization Name]}, conducted on \today. The analysis is based on a synthesis of external network scan data, a review of internal security controls via a questionnaire, and an evaluation of pre-existing risk documentation.

The assessment reveals a mixed security posture. The organization has implemented several important security controls, including mandatory Multi-Factor Authentication (MFA) for computer and sensitive system access, and maintains a security awareness training program.

However, several critical vulnerabilities were identified that require immediate attention. An external-facing FTP server was found to be dangerously misconfigured, allowing anonymous access and running a known vulnerable version of \texttt{vsftpd}. Furthermore, a critical policy gap exists where MFA is not enforced for email access, exposing the organization to significant risk of account compromise and business email compromise (BEC) attacks. These issues, combined with the pre-existing risk of outdated Windows 7 workstations, create attack vectors that could lead to data breaches, unauthorized access, or service disruption.

This report details these findings and provides actionable recommendations to mitigate the identified risks and strengthen the overall security posture of \textbf{[Organization Name]}.

% --- 2. Organizational Information ---
\section{Organizational Information}

This section contains the high-level information used as the basis for this assessment.

\begin{itemize}
    \item \textbf{Organization Name:} \textbf{[Organization Name]}
    \item \textbf{Primary Email Domain:} \texttt{[Domain]}
    \item \textbf{External IP Address Scanned:} \texttt{[Client IP]}
\end{itemize}

% --- 3. Security Control Review ---
\section{Security Control Review}

The following table summarizes the organization's responses to a security controls questionnaire. These answers provide insight into the current policies and procedures governing information security.

\begin{table}[h!]
\centering
\caption{Security Controls Questionnaire Results}
\begin{tabular}{p{0.7\linewidth} c}
\toprule
\textbf{Control Question} & \textbf{Status} \\
\midrule
Do you require MFA to access email? & \ding{55} \\
Do you require MFA to log into computers? & \ding{51} \\
Do you require MFA to access sensitive data systems? & \ding{51} \\
Does your organization have an employee acceptable use policy? & \ding{51} \\
Does your organization do security awareness training for new employees? & \ding{51} \\
Does your organization do security awareness training for all employees at least once per year? & \ding{51} \\
\bottomrule
\end{tabular}
\end{table}

\subsection*{Analysis}
The review of security controls is largely positive, indicating a mature approach to endpoint security (MFA on computers), data protection (MFA on sensitive systems), and employee education. However, the lack of MFA for email access (\ding{55}) is a \textbf{critical security gap}. Email is a primary vector for phishing and account takeover attacks. Without MFA, a single compromised password could grant an attacker access to sensitive communications and a trusted platform from which to launch further attacks within the organization.

% --- 4. Technical Scan Results ---
\section{Technical Scan Results}

An external network scan was performed on the target IP address to identify open ports and exposed services.

\begin{itemize}
    \item \textbf{Target IP Address:} \texttt{[Target IP]}
    \item \textbf{Scan Tool:} Nmap
\end{itemize}

\begin{table}[h!]
\centering
\caption{Open Ports and Services Detected}
\begin{tabular}{l l l l}
\toprule
\textbf{Port} & \textbf{Service} & \textbf{Product / Version} & \textbf{Notes} \\
\midrule
21/tcp & ftp & vsftpd 2.3.4 & Anonymous FTP login allowed \\
\bottomrule
\end{tabular}
\end{table}

\subsection*{Analysis}
The technical scan identified one open port, which presents two distinct and critical risks:
\begin{enumerate}
    \item \textbf{Anonymous FTP Access:} The server is configured to allow "Anonymous FTP login". This is a severe misconfiguration that permits any unauthenticated user on the internet to connect to the server and potentially access, download, or upload files. This could lead to a data breach or the distribution of malicious content.
    \item \textbf{Vulnerable Software Version:} The FTP service is running \texttt{vsftpd version 2.3.4}. This specific version is widely known to be vulnerable to a critical backdoor vulnerability (CVE-2011-2523), which can allow an attacker to execute arbitrary commands on the server with root-level privileges.
\end{enumerate}

% --- 5. Consolidated Risk Assessment ---
\section{Consolidated Risk Assessment}

This section synthesizes findings from the security control review, technical scan, and pre-existing risk data into a consolidated list of identified risks.

\begin{table}[h!]
\centering
\caption{Summary of Identified Risks}
\begin{tabular}{p{0.25\linewidth} p{0.5\linewidth} p{0.15\linewidth}}
\toprule
\textbf{Risk Name} & \textbf{Description} & \textbf{Severity} \\
\midrule
\textbf{Vulnerable FTP Service} & The external FTP server runs \texttt{vsftpd 2.3.4}, which is vulnerable to remote code execution (CVE-2011-2523). & \textcolor{sev_critical}{\textbf{Critical}} \\
\addlinespace
\textbf{Anonymous FTP Access} & The FTP server is misconfigured to allow unauthenticated anonymous access, risking data exfiltration or unauthorized uploads. & \textcolor{sev_critical}{\textbf{Critical}} \\
\addlinespace
\textbf{Lack of MFA on Email} & Email accounts are secured only by passwords, making them highly susceptible to phishing and account takeover attacks. & \textcolor{sev_critical}{\textbf{Critical}} \\
\addlinespace
\textbf{Outdated Windows Policy} & Workstations are running Windows 7, which is End-of-Life and no longer receives security updates from Microsoft. & \textcolor{sev_medium}{\textbf{Medium}} \\
\bottomrule
\end{tabular}
\end{table}

% --- 6. Recommendations ---
\section{Recommendations}

The following actions are recommended to mitigate the identified risks. They are prioritized based on severity and potential impact.

\begin{enumerate}
    \item \textbf{Remediate FTP Server Vulnerabilities (Immediate Priority):}
        \begin{itemize}
            \item \textbf{Action:} If the FTP service is not essential for business operations, disable and block port 21 at the firewall immediately.
            \item \textbf{Action:} If the service is required, immediately upgrade the \texttt{vsftpd} software to the latest stable version to patch the known backdoor vulnerability.
            \item \textbf{Action:} Immediately disable anonymous FTP login in the server's configuration file (\texttt{vsftpd.conf}).
        \end{itemize}

    \item \textbf{Enforce MFA for Email Access (Immediate Priority):}
        \begin{itemize}
            \item \textbf{Action:} Procure and implement an MFA solution for the organization's email platform.
            \item \textbf{Action:} Develop a phased rollout plan to enforce MFA for all users, starting with privileged accounts (e.g., administrators, finance, executives). The goal should be 100\% adoption within 30 days.
        \end{itemize}

    \item \textbf{Upgrade End-of-Life Workstations (High Priority):}
        \begin{itemize}
            \item \textbf{Action:} Continue with the existing plan to upgrade all Windows 7 workstations to a modern, supported operating system (e.g., Windows 10/11).
            \item \textbf{Action:} Isolate any remaining Windows 7 machines from the primary network if they cannot be immediately upgraded.
        \end{itemize}
\end{enumerate}

% --- 7. Conclusion ---
\section{Conclusion}

While \textbf{[Organization Name]} has established a solid foundation for its security program, this assessment has identified critical, high-impact vulnerabilities that require immediate action. The externally exposed FTP server and the lack of MFA on email represent the most significant threats to the organization's security and data integrity.

By implementing the recommendations outlined in this report, the organization can significantly reduce its attack surface, mitigate the risk of a security breach, and improve its overall cybersecurity resilience.

\end{document}
```