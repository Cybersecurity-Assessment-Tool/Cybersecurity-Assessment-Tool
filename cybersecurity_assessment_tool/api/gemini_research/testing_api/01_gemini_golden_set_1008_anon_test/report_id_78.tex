```latex
\documentclass[12pt]{article}

% ----------------------------------------------------------------------
% PREAMBLE
% ----------------------------------------------------------------------
\usepackage[margin=1in]{geometry}
\usepackage{pifont} % For checkmarks and crosses
\usepackage{booktabs} % For professional tables
\usepackage{hyperref} % For clickable links
\usepackage{url}      % For URL formatting
\usepackage{seqsplit} % To split long strings in tt font
\usepackage{xcolor}   % For colors
\usepackage{graphicx} % For logo (placeholder)

% Define colors for severity
\definecolor{sevhigh}{HTML}{D9534F}
\definecolor{sevmedium}{HTML}{F0AD4E}
\definecolor{sevlow}{HTML}{5CB85C}
\definecolor{sevinfo}{HTML}{5BC0DE}

% Hyperref setup
\hypersetup{
    colorlinks=true,
    linkcolor=blue,
    filecolor=magenta,      
    urlcolor=cyan,
    pdftitle={Cybersecurity Posture Report},
    pdfpagemode=FullScreen,
}

% ----------------------------------------------------------------------
% DOCUMENT START
% ----------------------------------------------------------------------
\begin{document}

% --- TITLE PAGE ---
\begin{titlepage}
    \centering
    \vspace*{2cm}
    
    \Huge \textbf{Cybersecurity Posture Report}
    
    \vspace{1.5cm}
    
    \Large Prepared for:
    
    \vspace{0.5cm}
    
    \textbf{\huge [Organization Name]}
    
    \vfill
    
    \large
    \textbf{Date of Report:} \today \\
    \textbf{Report ID:} CSR-2024-001
    
    \vspace{1cm}
    
    \textit{This report contains sensitive information and is intended solely for the recipient.}
    
\end{titlepage}

\tableofcontents
\newpage

% ----------------------------------------------------------------------
% 1. EXECUTIVE OVERVIEW
% ----------------------------------------------------------------------
\section{Executive Overview}

This report provides a comprehensive analysis of the cybersecurity posture for \textbf{[Organization Name]}, synthesizing findings from a security controls questionnaire, an external network scan, and a review of pre-existing risks.

The assessment reveals a mixed security posture. The organization demonstrates strong identity and access management controls, with Multi-Factor Authentication (MFA) widely implemented across key systems. Security awareness training programs are also well-established.

However, a critical administrative gap was identified: the absence of a formal Employee Acceptable Use Policy (AUP). This exposes the organization to significant insider risk and lacks a foundational governance document for user behavior.

On a positive note, the technical network scan of the external IP address \texttt{[Client IP]} shows a minimal attack surface. A previously identified risk concerning an unencrypted web server (Port 80) has been successfully remediated, as the port is now confirmed to be closed. This proactive remediation is a commendable security improvement.

Key recommendations focus on immediately developing and implementing an AUP to address the policy gap and formally updating the internal risk register to reflect the remediated vulnerability.

% ----------------------------------------------------------------------
% 2. ORGANIZATIONAL INFORMATION
% ----------------------------------------------------------------------
\section{Organizational Information}

The following details were used as the basis for this assessment. Due to the anonymized nature of the data provided, placeholders have been used where necessary.

\begin{itemize}
    \item \textbf{Organization Name:} \textbf{[Organization Name]}
    \item \textbf{Primary Domain:} \texttt{[Domain]}
    \item \textbf{Assessed External IP:} \texttt{[Client IP]}
    \item \textbf{Target of Network Scan:} \texttt{[Target IP]}
\end{itemize}

% ----------------------------------------------------------------------
% 3. SECURITY CONTROL REVIEW
% ----------------------------------------------------------------------
\section{Security Control Review}

A security questionnaire was conducted to evaluate the implementation of key administrative and technical controls. The results are summarized below.

\subsection{Questionnaire Results}

\begin{table}[h!]
\centering
\caption{Security Controls Questionnaire Analysis}
\label{tab:controls}
\begin{tabular}{p{0.8\linewidth} c}
\toprule
\textbf{Control Question} & \textbf{Response} \\
\midrule
Do you require MFA to access email? & \textcolor{green}{\ding{51}} \\
Do you require MFA to log into computers? & \textcolor{green}{\ding{51}} \\
Do you require MFA to access sensitive data systems? & \textcolor{green}{\ding{51}} \\
Does your organization have an employee acceptable use policy? & \textcolor{red}{\ding{55}} \\
Does your organization do security awareness training for new employees? & \textcolor{green}{\ding{51}} \\
Does your organization do security awareness training for all employees at least once per year? & \textcolor{green}{\ding{51}} \\
\bottomrule
\end{tabular}
\end{table}

\subsection{Analysis of Controls}
The organization has effectively implemented MFA and security awareness training, which are crucial for defending against common cyber threats like phishing and credential theft.

However, the response \textbf{"No"} to having an employee acceptable use policy represents a \textbf{critical governance gap}. An AUP is a foundational document that sets expectations for employee behavior, defines acceptable use of company assets, and provides a basis for enforcement actions. Its absence can lead to inconsistent security practices and increased insider risk.

% ----------------------------------------------------------------------
% 4. TECHNICAL SCAN RESULTS
% ----------------------------------------------------------------------
\section{Technical Scan Results}

An external network scan was performed using Nmap on \today\ to identify open ports and exposed services on the target system.

\begin{itemize}
    \item \textbf{Target IP Address:} \texttt{[Target IP]}
    \item \textbf{Host Status:} Up
\end{itemize}

\begin{table}[h!]
\centering
\caption{Nmap Port Scan Findings}
\label{tab:nmap}
\begin{tabular}{l l l l}
\toprule
\textbf{Port} & \textbf{State} & \textbf{Service} & \textbf{Notes} \\
\midrule
80/tcp & closed & http & Port is not accessible from the internet. \\
\bottomrule
\end{tabular}
\end{table}

\subsection{Technical Analysis}
The scan results are highly favorable. The target system exposes a minimal attack surface, with no open ports detected. The confirmation that port 80 (HTTP) is \textbf{closed} is a significant finding. This directly contradicts a previously documented risk, indicating that successful remediation has occurred. A hardened external perimeter is a primary defense against opportunistic attackers.

% ----------------------------------------------------------------------
% 5. CONSOLIDATED RISK ASSESSMENT
% ----------------------------------------------------------------------
\section{Consolidated Risk Assessment}

This section correlates findings from the security control review, the technical scan, and pre-existing risk data into a unified summary.

\begin{table}[h!]
\centering
\caption{Risk Summary}
\label{tab:risks}
\begin{tabular}{p{0.25\linewidth} p{0.45\linewidth} l l}
\toprule
\textbf{Risk Name} & \textbf{Description} & \textbf{Severity} & \textbf{Status} \\
\midrule
\textbf{Lack of Acceptable Use Policy} & The organization lacks a formal policy defining acceptable use of IT assets, data handling, and security responsibilities for employees. & \colorbox{sevhigh}{\color{white} High} & \textbf{Active} \\
\midrule
\textbf{Unencrypted Web Server} & The pre-existing risk register noted that Port 80 was open, potentially exposing unencrypted web traffic. & \colorbox{sevmedium}{\color{white} Medium} & \textbf{Remediated} \\
\bottomrule
\end{tabular}
\end{table}

\subsection{Risk Analysis}
The primary active risk is administrative in nature. The lack of an AUP undermines other security controls by failing to establish a clear baseline for user conduct.

The "Unencrypted Web Server" risk, previously rated as Medium, is now considered \textbf{Remediated}. The technical scan provides definitive evidence that the associated port is closed, mitigating the risk of data interception over unencrypted channels.

% ----------------------------------------------------------------------
% 6. RECOMMENDATIONS
% ----------------------------------------------------------------------
\section{Recommendations}

The following actions are recommended to address the identified risks and improve the overall security posture of \textbf{[Organization Name]}.

\subsection{Priority 1: Develop and Implement an Acceptable Use Policy}
\begin{itemize}
    \item \textbf{Risk Addressed:} Lack of Acceptable Use Policy (High)
    \item \textbf{Action:} Draft, approve, and disseminate a formal Acceptable Use Policy for all employees and contractors. This policy should cover, at a minimum:
    \begin{itemize}
        \item Proper handling of sensitive data.
        \item Rules for internet, email, and software usage.
        \item Password and account security requirements.
        \item Consequences for policy violations.
    \end{itemize}
    \item \textbf{Impact:} Establishes clear security expectations, reduces insider threats, and provides a legal and administrative framework for enforcing security standards.
\end{itemize}

\subsection{Priority 2: Update Internal Risk Register}
\begin{itemize}
    \item \textbf{Risk Addressed:} Unencrypted Web Server (Medium)
    \item \textbf{Action:} Formally update the organization's internal risk register to mark the vulnerability associated with Port 80 as "Remediated" or "Closed". Include the date of this report's technical scan as evidence.
    \item \textbf{Impact:} Ensures the risk register is an accurate, up-to-date reflection of the current security posture, allowing for better prioritization of future security efforts.
\end{itemize}

% ----------------------------------------------------------------------
% DOCUMENT END
% ----------------------------------------------------------------------
\end{document}
```