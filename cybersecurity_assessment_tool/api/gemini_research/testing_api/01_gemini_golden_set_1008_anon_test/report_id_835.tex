```latex
\documentclass[12pt]{article}

% Preamble: Required Packages
\usepackage[margin=1in]{geometry}
\usepackage{pifont} % For checkmarks and crosses (\ding)
\usepackage{booktabs} % For professional tables
\usepackage{hyperref} % For clickable links and metadata
\usepackage{url} % For formatting URLs
\usepackage{seqsplit} % For splitting long strings in \texttt
\usepackage{xcolor} % For colors in text

% Document Metadata
\hypersetup{
    colorlinks=true,
    linkcolor=blue,
    filecolor=magenta,      
    urlcolor=cyan,
    pdftitle={Cybersecurity Posture Assessment Report},
    pdfauthor={Cybersecurity Analyst},
    pdfsubject={Security Assessment},
    pdfkeywords={Cybersecurity, Risk, Assessment},
}

% Title and Author Information
\title{Cybersecurity Posture Assessment Report \\ \large For \textbf{[Organization Name]}}
\author{Cybersecurity Analyst}
\date{\today}

\begin{document}

\maketitle
\tableofcontents
\newpage

% --- 1. Executive Summary ---
\section{Executive Summary}
This report details the findings of a cybersecurity posture assessment conducted for \textbf{[Organization Name]}. The assessment combined a review of organizational security controls, an external network scan, and an analysis of known risks.

The analysis revealed several critical and high-risk security gaps that require immediate attention. Key findings include:
\begin{itemize}
    \item \textbf{Lack of Multi-Factor Authentication (MFA):} MFA is not enforced for accessing email or logging into computers. This represents a critical vulnerability, as compromised credentials could directly lead to unauthorized access to sensitive communications and internal systems.
    \item \textbf{Incomplete Security Awareness Training:} While annual training is in place, new employees do not receive security awareness training as part of their onboarding process. This leaves the organization vulnerable to phishing and social engineering attacks, as new staff are often prime targets.
    \item \textbf{Exposed Management Service:} An external network scan identified an open SSH port (22/TCP) on the public-facing IP address \texttt{[Client IP]}. If not properly secured, this service could be a target for brute-force attacks or exploitation, potentially leading to a full system compromise.
\end{itemize}

The combination of these findings indicates a significant risk of security incidents, including data breaches and ransomware attacks. This report provides a detailed breakdown of these risks and offers actionable recommendations to mitigate them effectively.

% --- 2. Organizational Information ---
\section{Organizational Information}
This section provides the high-level details of the organization under review. The data was collected from the provided organizational data sheet.

\begin{tabular}{@{}ll}
    \toprule
    \textbf{Attribute} & \textbf{Value} \\
    \midrule
    Organization Name & \textbf{[Organization Name]} \\
    Primary Email Domain & \texttt{[Domain]} \\
    External IP Address Scanned & \texttt{[Client IP]} \\
    \bottomrule
\end{tabular}

% --- 3. Security Control Review ---
\section{Security Control Review}
A review of administrative and technical security controls was conducted based on a standardized questionnaire. The results below highlight the current state of implemented policies and procedures. Gaps identified with a red cross (\textcolor{red}{\ding{55}}) are discussed in the Risk Assessment section.

\begin{table}[h!]
\centering
\caption{Organizational Security Controls Questionnaire}
\begin{tabular}{@{}p{0.8\linewidth}c@{}}
    \toprule
    \textbf{Control Question} & \textbf{Status} \\
    \midrule
    Do you require MFA to access email? & \textcolor{red}{\ding{55}} \\
    Do you require MFA to log into computers? & \textcolor{red}{\ding{55}} \\
    Do you require MFA to access sensitive data systems? & \textcolor{green}{\ding{51}} \\
    Does your organization have an employee acceptable use policy? & \textcolor{green}{\ding{51}} \\
    Does your organization do security awareness training for new employees? & \textcolor{red}{\ding{55}} \\
    Does your organization do security awareness training for all employees at least once per year? & \textcolor{green}{\ding{51}} \\
    \bottomrule
\end{tabular}
\end{table}

% --- 4. Technical Scan Results ---
\section{Technical Scan Results}
An external network vulnerability scan was performed against the organization's public-facing infrastructure to identify open ports and exposed services.

\begin{itemize}
    \item \textbf{Target IP Address:} \texttt{[Target IP]}
    \item \textbf{Scan Tool:} Nmap
    \item \textbf{Scan Date:} Not specified in scan data.
\end{itemize}

\subsection{Open Ports Discovered}
The scan revealed the following open port, which is accessible from the public internet.

\begin{table}[h!]
\centering
\caption{Open Ports on \texttt{[Target IP]}}
\begin{tabular}{@{}llll@{}}
    \toprule
    \textbf{Port} & \textbf{State} & \textbf{Service (Common)} & \textbf{Notes} \\
    \midrule
    22/TCP & open & SSH (Secure Shell) & No version information was available. Exposed SSH \\
    & & & can be a target for brute-force password attacks. \\
    \bottomrule
\end{tabular}
\end{table}

\subsection{Technical Analysis}
The presence of an open SSH port is a significant finding. Secure Shell is a powerful administrative protocol that provides direct command-line access to a server. While essential for remote management, its exposure to the entire internet without proper hardening (e.g., IP whitelisting, key-based authentication) creates a substantial security risk.

% --- 5. Risk Assessment ---
\section{Risk Assessment}
This section correlates the findings from the security control review and the technical scan to define specific, measurable risks to the organization. Pre-existing risks were also reviewed; none were identified in the provided data.

\begin{table}[h!]
\centering
\caption{Identified Risk Summary}
\begin{tabular}{@{}p{0.1\linewidth}p{0.4\linewidth}p{0.2\linewidth}p{0.2\linewidth}@{}}
    \toprule
    \textbf{ID} & \textbf{Risk Description} & \textbf{Severity} & \textbf{Likely Impact} \\
    \midrule
    \textbf{R-01} & \textbf{Lack of MFA on Email and Endpoints:} User accounts are protected only by passwords, making them highly susceptible to compromise via phishing or credential stuffing. & \textbf{Critical} & Data Breach, Ransomware, Business Email Compromise \\
    \addlinespace
    \textbf{R-02} & \textbf{Exposed SSH Management Port:} The SSH service on \texttt{[Target IP]} is accessible from the internet, creating a vector for brute-force or credential-based attacks. & \textbf{High} & Server Compromise, Unauthorized Access \\
    \addlinespace
    \textbf{R-03} & \textbf{Inadequate Security Awareness Training:} New employees are not trained on security policies and threats, increasing the likelihood of successful social engineering attacks. & \textbf{High} & Malware Infection, Credential Theft \\
    \bottomrule
\end{tabular}
\end{table}

% --- 6. Recommendations ---
\section{Recommendations}
The following actions are recommended to mitigate the identified risks and improve the overall security posture of \textbf{[Organization Name]}.

\subsection{Immediate Actions (1-30 Days)}
\begin{itemize}
    \item \textbf{Remediation for R-01 (MFA):}
        \begin{itemize}
            \item Immediately enforce MFA for all users on the primary email system.
            \item Develop and execute a plan to roll out MFA for all computer logins, prioritizing administrative and executive accounts.
        \end{itemize}
    \item \textbf{Remediation for R-02 (SSH):}
        \begin{itemize}
            \item If SSH access from the public internet is not required, block port 22 at the network firewall immediately.
            \item If external access is required, restrict access to a whitelist of trusted IP addresses.
            \item Enforce public key authentication and disable password-based authentication for SSH.
        \end{itemize}
\end{itemize}

\subsection{Strategic Actions (30-90 Days)}
\begin{itemize}
    \item \textbf{Remediation for R-03 (Training):}
        \begin{itemize}
            \item Integrate mandatory security awareness training into the new employee onboarding process.
            \item Conduct a company-wide phishing simulation to establish a baseline for employee awareness and measure improvement over time.
        \end{itemize}
    \item \textbf{Further Technical Hardening:}
        \begin{itemize}
            \item Commission a comprehensive vulnerability scan or penetration test to identify outdated software versions and further misconfigurations on external-facing systems.
        \end{itemize}
\end{itemize}

\end{document}
```