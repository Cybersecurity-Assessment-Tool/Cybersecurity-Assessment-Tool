```latex
\documentclass[12pt]{article}

% Required Packages
\usepackage[margin=1in]{geometry}
\usepackage{pifont} % For \ding symbols (checkmarks and X's)
\usepackage{booktabs} % For professional-looking tables
\usepackage{hyperref} % For URLs and internal links
\usepackage{url}      % For typesetting URLs
\usepackage{seqsplit} % For breaking long strings in tt font
\usepackage[utf8]{inputenc}

% Document Metadata
\title{Cybersecurity Posture Assessment Report}
\author{Cybersecurity Analysis Division}
\date{November 22, 2025}

\hypersetup{
    colorlinks=true,
    linkcolor=black,
    urlcolor=blue,
}

\begin{document}

\maketitle
\tableofcontents
\newpage

% ------------------------------------------------------------------
% Section 1: Executive Summary
% ------------------------------------------------------------------
\section{Executive Summary}

This report provides a comprehensive analysis of the cybersecurity posture for \textbf{[Organization Name]}, based on data collected on November 22, 2025. The assessment combines a review of organizational security controls, an external network scan, and an evaluation of pre-existing risks.

The analysis revealed several areas of significant concern that elevate the organization's risk profile. The most critical findings include:
\begin{itemize}
    \item \textbf{Lack of Multi-Factor Authentication (MFA) for Email:} This is a critical security gap that exposes the organization to a high risk of business email compromise (BEC), phishing attacks, and unauthorized account access.
    \item \textbf{Absence of Security Awareness Training:} The organization does not provide security awareness training to new or existing employees. This deficiency makes personnel highly susceptible to social engineering and phishing attacks, which are primary vectors for initial compromise.
    \item \textbf{Outdated Web Server Software:} The external network scan identified an outdated version of Nginx (1.18.0) running on a public-facing server. This software version is no longer supported and likely contains known, unpatched vulnerabilities that could be exploited by attackers.
\end{itemize}

Immediate remediation of these issues is strongly recommended to reduce the likelihood of a significant security incident. The overall security posture requires immediate and focused improvement.

% ------------------------------------------------------------------
% Section 2: Organizational Information
% ------------------------------------------------------------------
\section{Organizational Information}

The following details were used as the basis for this assessment. Due to the anonymized nature of the provided data, placeholders have been used where necessary.

\begin{itemize}
    \item \textbf{Organization Name:} \textbf{[Organization Name]}
    \item \textbf{Primary Domain:} \texttt{[Domain]}
    \item \textbf{External IP Address Scanned:} \texttt{[Client IP]}
\end{itemize}

% ------------------------------------------------------------------
% Section 3: Security Control Review
% ------------------------------------------------------------------
\section{Security Control Review}

A review of the organization's security controls was conducted via a standardized questionnaire. The responses indicate critical gaps in personnel and access control policies. A "No" response (\ding{55}) highlights a deviation from security best practices.

\begin{table}[h!]
\centering
\caption{Security Controls Questionnaire Results}
\begin{tabular}{p{0.75\linewidth} c}
\toprule
\textbf{Control Question} & \textbf{Response} \\
\midrule
Do you require MFA to access email? & \ding{55} \\
Do you require MFA to log into computers? & \ding{51} \\
Do you require MFA to access sensitive data systems? & \ding{51} \\
Does your organization have an employee acceptable use policy? & \ding{51} \\
Does your organization do security awareness training for new employees? & \ding{55} \\
Does your organization do security awareness training for all employees at least once per year? & \ding{55} \\
\bottomrule
\end{tabular}
\label{tab:controls}
\end{table}

% ------------------------------------------------------------------
% Section 4: Technical Scan Results
% ------------------------------------------------------------------
\section{Technical Scan Results}

An external network scan was performed against the target IP address to identify open ports and exposed services.

\begin{itemize}
    \item \textbf{Scan Target:} \texttt{[Target IP]}
    \item \textbf{Scan Date:} 2025-11-22
\end{itemize}

The following open ports and services were discovered:

\begin{table}[h!]
\centering
\caption{Open Ports and Services}
\begin{tabular}{l l l l}
\toprule
\textbf{Port} & \textbf{State} & \textbf{Service} & \textbf{Product \& Version} \\
\midrule
443/TCP & open & https & nginx 1.18.0 \\
\bottomrule
\end{tabular}
\label{tab:scanresults}
\end{table}

The scan identified an Nginx web server, version 1.18.0. This version's mainline support ended in April 2021, making it outdated and susceptible to numerous publicly disclosed vulnerabilities.

% ------------------------------------------------------------------
% Section 5: Risk Assessment
% ------------------------------------------------------------------
\section{Risk Assessment}

The following table synthesizes findings from the security control review and technical scan into a prioritized list of risks. No pre-existing vulnerabilities were reported.

\begin{table}[h!]
\centering
\caption{Identified Risks and Severity}
\begin{tabular}{p{0.1\linewidth} p{0.25\linewidth} p{0.45\linewidth} p{0.1\linewidth}}
\toprule
\textbf{ID} & \textbf{Risk Name} & \textbf{Overview} & \textbf{Severity} \\
\midrule
RISK-001 & No MFA on Email & The absence of MFA on email accounts greatly increases the risk of account takeover via credential stuffing or phishing. This can lead to data breaches and business email compromise. & \textbf{Critical} \\
\addlinespace
RISK-002 & No Security Awareness Training & Employees are not trained to recognize or report security threats. This makes the organization highly vulnerable to phishing, malware, and social engineering attacks. & \textbf{High} \\
\addlinespace
RISK-003 & Outdated Web Server (Nginx 1.18.0) & The public-facing web server is running an unsupported version of Nginx with known vulnerabilities. An attacker could exploit these flaws to compromise the server and gain access to the internal network. & \textbf{High} \\
\bottomrule
\end{tabular}
\label{tab:risks}
\end{table}

% ------------------------------------------------------------------
% Section 6: Recommendations
% ------------------------------------------------------------------
\section{Recommendations}

The following actions are recommended to mitigate the identified risks and improve the overall security posture of \textbf{[Organization Name]}.

\subsection*{RISK-001: No MFA on Email (Critical)}
\begin{itemize}
    \item \textbf{Immediate Action:} Procure and implement a robust MFA solution for the organization's email system.
    \item \textbf{Policy:} Mandate the use of MFA for all employees and service accounts with access to email.
    \item \textbf{Timeline:} This should be treated as the highest priority and completed within 30 days.
\end{itemize}

\subsection*{RISK-002: No Security Awareness Training (High)}
\begin{itemize}
    \item \textbf{Immediate Action:} Establish a formal security awareness training program. This program should be mandatory for all new hires during their onboarding process.
    \item \textbf{Ongoing Action:} Conduct annual refresher training for all employees. Supplement this with periodic simulated phishing campaigns to measure effectiveness and provide targeted training.
    \item \textbf{Timeline:} A program should be selected and initial training for all staff should be completed within 90 days.
\end{itemize}

\subsection*{RISK-003: Outdated Web Server (High)}
\begin{itemize}
    \item \textbf{Immediate Action:} Develop a patch management plan to upgrade the Nginx 1.18.0 instance to a current, stable, and fully supported version.
    \item \textbf{Testing:} The upgrade should be tested in a non-production environment first to ensure compatibility with existing applications.
    \item \textbf{Timeline:} The upgrade should be completed within 60 days. A formal patch management policy should be implemented to prevent future occurrences.
\end{itemize}

\end{document}
```