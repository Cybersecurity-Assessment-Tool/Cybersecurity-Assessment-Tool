```latex
\documentclass[12pt]{article}

% Preamble: Required Packages
\usepackage[margin=1in]{geometry}
\usepackage{pifont} % For checkmarks and crosses
\usepackage{booktabs} % For professional tables
\usepackage{hyperref} % For clickable links and better PDF metadata
\usepackage{url} % For formatting URLs
\usepackage{seqsplit} % For splitting long strings without breaking
\usepackage[utf8]{inputenc}

% Document Metadata
\hypersetup{
    colorlinks=true,
    linkcolor=black,
    filecolor=magenta,      
    urlcolor=blue,
    pdftitle={Cybersecurity Posture Assessment Report},
    pdfauthor={Cybersecurity Analysis Division},
    pdfsubject={Security Assessment},
    pdfkeywords={Security, Risk, Analysis},
    bookmarks=true
}

\title{Cybersecurity Posture Assessment Report}
\author{Cybersecurity Analysis Division}
\date{\today}

\begin{document}

\maketitle
\thispagestyle{empty}
\newpage

\tableofcontents
\newpage

% --- 1. Executive Summary ---
\section{Executive Summary}

This report provides a comprehensive cybersecurity assessment for \textbf{[Organization Name]}, synthesizing findings from a network vulnerability scan, a review of organizational security controls, and an analysis of pre-existing risk data.

The assessment reveals a mixed security posture. On a positive note, a previously identified vulnerability related to an unencrypted web server on Port 80 has been successfully remediated; our technical scan confirms this port is now closed. This demonstrates a positive step towards improving the external security perimeter.

However, the review of internal security controls has identified two significant, high-impact gaps:
\begin{enumerate}
    \item \textbf{Lack of Multi-Factor Authentication (MFA) for Sensitive Systems:} The absence of MFA on critical data systems constitutes a critical risk, leaving the organization's most valuable assets vulnerable to unauthorized access via compromised credentials.
    \item \textbf{Insufficient Security Onboarding for New Employees:} New hires do not receive mandatory security awareness training, making them, and by extension the organization, highly susceptible to social engineering and phishing attacks.
\end{enumerate}

While the external technical posture has improved, these internal control deficiencies present a clear and present danger to the organization's security and data integrity. The recommendations provided in this report are prioritized to address these critical gaps first.

% --- 2. Organizational Information ---
\section{Organizational Information}

This section details the information provided for the assessment. As this report was generated in a template mode, placeholders have been used where specific data was not available.

\begin{itemize}
    \item \textbf{Organization Name:} \textbf{[Organization Name]}
    \item \textbf{Primary Email Domain:} \texttt{[Domain]}
    \item \textbf{Known External IP:} \texttt{[Client IP]}
    \item \textbf{Target IP Scanned:} \texttt{[Target IP]}
\end{itemize}

% --- 3. Security Control Review ---
\section{Security Control Review}

The following table summarizes the organization's responses to a security controls questionnaire. "No" answers indicate significant gaps in the security framework and are highlighted as areas of high concern.

\begin{table}[h!]
\centering
\caption{Organizational Security Controls Questionnaire}
\begin{tabular}{p{0.5\textwidth} c p{0.3\textwidth}}
\toprule
\textbf{Control Question} & \textbf{Response} & \textbf{Assessment} \\
\midrule
Do you require MFA to access email? & \ding{51} & Satisfactory \\
\addlinespace
Do you require MFA to log into computers? & \ding{51} & Satisfactory \\
\addlinespace
Do you require MFA to access sensitive data systems? & \textbf{\color{red}\ding{55}} & \textbf{Critical Gap} \\
\addlinespace
Does your organization have an employee acceptable use policy? & \ding{51} & Satisfactory \\
\addlinespace
Does your organization do security awareness training for new employees? & \textbf{\color{red}\ding{55}} & \textbf{High Risk} \\
\addlinespace
Does your organization do security awareness training for all employees at least once per year? & \ding{51} & Satisfactory \\
\bottomrule
\end{tabular}
\end{table}

% --- 4. Technical Scan Results ---
\section{Technical Scan Results}

An external network scan was performed on the target IP address to identify open ports and exposed services.

\begin{itemize}
    \item \textbf{Target IP:} \texttt{[Target IP]}
    \item \textbf{Scan Date:} \today
\end{itemize}

\paragraph{Analysis:} The scan revealed that the target host is online, but no open ports were detected. Notably, Port 80 (HTTP), which was listed as an active risk in previous documentation, was found to be \textbf{closed}. This is a significant positive finding, indicating that the risk of unencrypted web traffic from this host has been successfully remediated.

\begin{table}[h!]
\centering
\caption{Nmap Scan Results for \texttt{[Target IP]}}
\begin{tabular}{c c c c c}
\toprule
\textbf{Port} & \textbf{State} & \textbf{Service} & \textbf{Product} & \textbf{Version} \\
\midrule
80 & closed & http & N/A & N/A \\
\bottomrule
\end{tabular}
\end{table}

% --- 5. Consolidated Risk Assessment ---
\section{Consolidated Risk Assessment}

This section correlates findings from the security control review, the technical scan, and pre-existing risk data into a unified risk register.

\begin{table}[h!]
\centering
\caption{Summary of Identified Risks}
\begin{tabular}{p{0.3\textwidth} p{0.5\textwidth} c}
\toprule
\textbf{Risk Name} & \textbf{Description} & \textbf{Severity} \\
\midrule
\addlinespace
\textbf{Lack of MFA for Sensitive Systems} & The absence of mandatory MFA on systems containing sensitive data exposes the organization to a high likelihood of data breach from credential theft. & \textbf{Critical} \\
\addlinespace
\textbf{Inadequate Employee Onboarding Security} & New employees are not provided with security awareness training, making them a primary target for social engineering and phishing attacks. & \textbf{High} \\
\addlinespace
Unencrypted Web Server & Port 80 was previously identified as open, allowing for unencrypted HTTP traffic. Our scan confirms this port is now closed. & \textbf{Remediated} \\
\bottomrule
\end{tabular}
\end{table}

% --- 6. Recommendations ---
\section{Recommendations}

The following actionable recommendations are prioritized based on the risk assessment. Addressing these items will significantly improve the organization's overall security posture.

\subsection{Priority 1: Critical}

\subsubsection{Implement MFA for All Sensitive Data Systems}
\begin{itemize}
    \item \textbf{Action:} Enforce Multi-Factor Authentication (MFA) across all applications, databases, and administrative interfaces that process or store sensitive or regulated data. This policy should be implemented with immediate effect.
    \item \textbf{Impact:} Drastically reduces the risk of unauthorized access and data breaches. Even if user credentials are stolen, MFA provides a critical second layer of defense.
\end{itemize}

\subsection{Priority 2: High}

\subsubsection{Integrate Security Training into Employee Onboarding}
\begin{itemize}
    \item \textbf{Action:} Develop and mandate a security awareness training module as a required step in the new employee onboarding process. The training must cover, at a minimum: phishing identification, acceptable use of company assets, password hygiene, and data handling policies.
    \item \textbf{Impact:} Reduces the human-factor risk by equipping new staff with the knowledge to recognize and report security threats, thereby strengthening the organization's first line of defense.
\end{itemize}

\subsection{Informational}

\subsubsection{Update Internal Risk Register}
\begin{itemize}
    \item \textbf{Action:} Formally document the remediation of the "Unencrypted Web Server" vulnerability in the internal risk register. The entry should be marked as "Closed" or "Remediated" with a reference to the date this scan confirmed the fix.
    \item \textbf{Impact:} Ensures the risk register remains an accurate and reliable source for tracking the organization's security posture.
\end{itemize}

\end{document}
```