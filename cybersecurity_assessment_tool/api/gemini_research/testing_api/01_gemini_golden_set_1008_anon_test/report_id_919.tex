```latex
\documentclass[12pt]{article}

% Preamble: Required Packages
\usepackage[margin=1in]{geometry} % Set page margins
\usepackage{pifont}               % For checkmark and X symbols (\ding)
\usepackage{booktabs}             % For professional-looking tables
\usepackage{hyperref}             % For hyperlinks, metadata
\usepackage{url}                  % For URL formatting
\usepackage{seqsplit}             % To split long strings without breaking
\usepackage{graphicx}             % For including logos, etc.
\usepackage{fancyhdr}             % For headers and footers
\usepackage{xcolor}               % For custom colors

% Document Metadata
\hypersetup{
    colorlinks=true,
    linkcolor=blue,
    filecolor=magenta,      
    urlcolor=cyan,
    pdftitle={Cybersecurity Posture Assessment Report},
    pdfauthor={Cybersecurity Analysis Division},
    pdfsubject={Security Report},
    pdfkeywords={Security, Assessment, Network Scan, Risk},
}

% Define custom colors
\definecolor{darkblue}{rgb}{0.0, 0.0, 0.5}
\definecolor{darkred}{rgb}{0.5, 0.0, 0.0}
\definecolor{darkgreen}{rgb}{0.0, 0.5, 0.0}

% Header and Footer Configuration
\pagestyle{fancy}
\fancyhf{} % Clear all header and footer fields
\fancyhead[L]{\textbf{Cybersecurity Posture Assessment}}
\fancyhead[R]{\textbf{[Organization Name]}}
\fancyfoot[C]{\thepage}
\renewcommand{\headrulewidth}{0.4pt}
\renewcommand{\footrulewidth}{0.4pt}

\begin{document}

% --- Title Page ---
\begin{titlepage}
    \centering
    \vspace*{2cm}
    
    {\Huge\bfseries Cybersecurity Posture Assessment Report\par}
    
    \vspace{1.5cm}
    
    {\Large \textbf{Prepared for:}} \\
    \vspace{0.5cm}
    {\Huge \textbf{[Organization Name]}}\par
    
    \vfill % Pushes content to the bottom
    
    {\large \textbf{Date of Report:} \today}\par
    \vspace{0.5cm}
    {\large \textbf{Analysis by:} Cybersecurity Analysis Division}\par
    
\end{titlepage}

\tableofcontents
\newpage

% --- Section 1: Executive Summary ---
\section{Executive Summary}
This report provides a comprehensive analysis of the cybersecurity posture for \textbf{[Organization Name]}, based on a synthesis of network scan data, a security controls questionnaire, and a review of pre-existing risks. The assessment was conducted to identify vulnerabilities, security gaps, and areas for improvement in the organization's defenses.

The analysis revealed several high-risk areas requiring immediate attention. Key findings include critical gaps in foundational security controls, specifically the absence of Multi-Factor Authentication (MFA) for computer logins and the lack of a formal security awareness training program for employees. These gaps significantly increase the risk of unauthorized access and successful social engineering attacks, such as phishing.

Furthermore, technical scanning identified a publicly exposed Secure Shell (SSH) service. While necessary for remote administration, an improperly configured or unmonitored SSH service is a prime target for automated brute-force attacks.

The combination of these findings indicates an elevated risk profile. This report provides detailed analysis and actionable recommendations to mitigate these risks and strengthen the overall security posture.

% --- Section 2: Organizational Information ---
\section{Organizational Information}
This section details the organizational data used as a basis for this assessment. Due to the anonymized nature of the input data, placeholders have been used where specific information was not provided.

\begin{itemize}
    \item \textbf{Organization Name:} \textbf{[Organization Name]}
    \item \textbf{Primary Email Domain:} \seqsplit{\texttt{[Domain]}}
    \item \textbf{External IP Address Scanned:} \seqsplit{\texttt{[Client IP]}}
\end{itemize}

% --- Section 3: Security Control Review ---
\section{Security Control Review}
The following table summarizes the organization's responses to a security controls questionnaire. The assessment column highlights whether the current practice aligns with security best practices or represents a significant gap.

\begin{table}[h!]
\centering
\caption{Security Controls Questionnaire Analysis}
\label{tab:controls}
\begin{tabular}{p{8cm} c p{3cm}}
\toprule
\textbf{Control Question} & \textbf{Response} & \textbf{Assessment} \\
\midrule
Do you require MFA to access email? & \ding{51} & \textcolor{darkgreen}{Control in Place} \\
\addlinespace
Do you require MFA to log into computers? & \ding{55} & \textcolor{darkred}{Critical Gap} \\
\addlinespace
Do you require MFA to access sensitive data systems? & \ding{51} & \textcolor{darkgreen}{Control in Place} \\
\addlinespace
Does your organization have an employee acceptable use policy? & \ding{51} & \textcolor{darkgreen}{Control in Place} \\
\addlinespace
Does your organization do security awareness training for new employees? & \ding{55} & \textcolor{darkred}{High Risk} \\
\addlinespace
Does your organization do security awareness training for all employees at least once per year? & \ding{55} & \textcolor{darkred}{High Risk} \\
\bottomrule
\end{tabular}
\end{table}

The review identifies two primary areas of concern:
\begin{enumerate}
    \item \textbf{Endpoint Access Control:} The lack of MFA on computer logins is a critical vulnerability. If an employee's password is compromised, an attacker could gain direct access to their workstation and, potentially, the entire internal network.
    \item \textbf{Human Firewall:} The absence of a security awareness training program for both new and existing employees leaves the organization highly vulnerable to phishing, social engineering, and other human-centric attacks.
\end{enumerate}

% --- Section 4: Technical Scan Results ---
\section{Technical Scan Results}
An external network scan was performed to identify open ports and exposed services. The target for this scan was \seqsplit{\texttt{[Target IP]}}.

\begin{table}[h!]
\centering
\caption{Open Port Analysis}
\label{tab:ports}
\begin{tabular}{c c l p{7cm}}
\toprule
\textbf{Port} & \textbf{State} & \textbf{Service} & \textbf{Details} \\
\midrule
22/tcp & Open & SSH (Presumed) & The Secure Shell service is exposed to the public internet. This port is a common target for automated brute-force attacks. No version information was available from the scan, preventing an assessment for known vulnerabilities. \\
\bottomrule
\end{tabular}
\end{table}

\textbf{Analysis:} The exposure of SSH (Port 22) is a notable finding. This service is essential for remote system administration but must be carefully secured. Without proper controls, such as IP whitelisting, strong password policies, public key authentication, and intrusion detection, it presents a significant attack vector.

% --- Section 5: Consolidated Risk Assessment ---
\section{Consolidated Risk Assessment}
This section synthesizes findings from the security control review, technical scan, and pre-existing risk data. As no pre-existing vulnerabilities were reported, the following risks are derived directly from this assessment.

\begin{table}[h!]
\centering
\caption{Summary of Identified Risks}
\label{tab:risks}
\begin{tabular}{p{1.5cm} p{4cm} p{6.5cm} c}
\toprule
\textbf{Risk ID} & \textbf{Risk Name} & \textbf{Description} & \textbf{Severity} \\
\midrule
RISK-001 & Lack of MFA on Endpoints & The absence of MFA for computer logins allows an attacker with stolen credentials to gain direct access to an endpoint and the internal network. & \textbf{High} \\
\addlinespace
RISK-002 & Inadequate Security Awareness Training & Without training, employees are more likely to fall victim to phishing and social engineering, leading to credential theft, malware infection, or data breaches. & \textbf{High} \\
\addlinespace
RISK-003 & Exposed SSH Administrative Service & The SSH service on \seqsplit{\texttt{[Target IP]}} is open to the internet, making it a target for brute-force login attempts and exploitation of potential vulnerabilities. & \textbf{Medium} \\
\bottomrule
\end{tabular}
\end{table}

% --- Section 6: Recommendations ---
\section{Recommendations}
The following actionable recommendations are provided to mitigate the identified risks and improve the organization's overall security posture.

\subsection{Risk-001: Lack of MFA on Endpoints (High)}
\begin{itemize}
    \item \textbf{Immediate Action:} Implement a mandatory Multi-Factor Authentication (MFA) policy for all employee computer logins (e.g., Windows, macOS). Solutions like Windows Hello for Business, Duo, or other third-party integrations should be deployed.
    \item \textbf{Strategic Goal:} Ensure MFA is a standard component of the identity and access management (IAM) lifecycle for all systems, not just sensitive ones.
\end{itemize}

\subsection{Risk-002: Inadequate Security Awareness Training (High)}
\begin{itemize}
    \item \textbf{Immediate Action:} Enroll all current employees in a foundational security awareness training program. This program should be mandatory for all new hires as part of their onboarding process.
    \item \textbf{Ongoing Action:} Establish an annual, mandatory security awareness refresher course for all staff. Supplement this with regular phishing simulation campaigns to test and reinforce learned behaviors.
\end{itemize}

\subsection{Risk-003: Exposed SSH Administrative Service (Medium)}
\begin{itemize}
    \item \textbf{Immediate Action:} If remote access is required, restrict access to the SSH port (22) to a whitelist of trusted IP addresses using a firewall. This immediately reduces the attack surface.
    \item \textbf{Best Practices:}
        \begin{itemize}
            \item Disable password-based authentication and enforce the use of public key cryptography.
            \item Ensure the SSH service is running an up-to-date version with no known critical vulnerabilities.
            \item Implement a tool like \texttt{fail2ban} to automatically block IPs that exhibit malicious behavior, such as repeated failed login attempts.
        \end{itemize}
\end{itemize}

\end{document}
```