```latex
\documentclass[12pt]{article}

% Preamble: Required Packages
\usepackage[margin=1in]{geometry} % Set page margins
\usepackage{pifont}                 % For symbols like checkmarks (\ding)
\usepackage{booktabs}               % For professional-looking tables
\usepackage{hyperref}               % For clickable links and references
\usepackage{url}                    % For formatting URLs
\usepackage{seqsplit}               % For splitting long strings in \texttt
\usepackage{xcolor}                 % For text coloring

% --- Document Setup ---
% Define colors for severity and status
\definecolor{darkgreen}{rgb}{0.0, 0.5, 0.0}
\definecolor{darkred}{rgb}{0.8, 0.0, 0.0}
\definecolor{darkorange}{rgb}{1.0, 0.55, 0.0}

% Hyperlink setup
\hypersetup{
    colorlinks=true,
    linkcolor=blue,
    filecolor=magenta,
    urlcolor=cyan,
}

% Custom commands for Yes/No symbols
\newcommand{\yes}{\textcolor{darkgreen}{\ding{51}}}
\newcommand{\no}{\textcolor{darkred}{\ding{55}}}

% --- Document Start ---
\begin{document}

% --- Title Page ---
\title{Cybersecurity Posture Assessment Report}
\author{Cybersecurity Analysis Division}
\date{\today}
\maketitle
\thispagestyle{empty}
\newpage

\tableofcontents
\newpage

% --- Executive Summary ---
\section{Executive Summary}

This report provides a comprehensive analysis of the cybersecurity posture for \textbf{[Organization Name]}. The assessment combines a review of self-reported security controls, an external network vulnerability scan, and an evaluation of pre-existing risk data.

The analysis revealed several high-priority risks that require immediate attention. Key findings include:
\begin{itemize}
    \item \textbf{Critical Network Exposure:} A MySQL database server was found to be publicly accessible from the internet. The running version, MySQL 5.7.33, is past its official End of Life (EOL) and no longer receives security updates, making it a prime target for exploitation.
    \item \textbf{Critical Authentication Gaps:} Multi-Factor Authentication (MFA) is not enforced for accessing critical systems, including employee email and computer logins. This significantly weakens defenses against common attacks like phishing and credential theft.
\end{itemize}

These findings indicate a high likelihood of potential compromise. This report details these risks and provides prioritized, actionable recommendations to mitigate them and strengthen the overall security posture.

% --- Organizational Information ---
\section{Organizational Information}
This section details the scope and context of the assessment.
\begin{center}
\begin{tabular}{@{}ll}
\toprule
\textbf{Attribute} & \textbf{Value} \\
\midrule
Organization Name & \textbf{[Organization Name]} \\
Primary Domain & \texttt{[Domain]} \\
External IP Scanned & \texttt{[Client IP]} \\
Target IP Scanned & \texttt{[Target IP]} \\
\bottomrule
\end{tabular}
\end{center}

% --- Security Control Review ---
\section{Security Control Review}
The following table summarizes the organization's responses to a security controls questionnaire. Gaps identified with a \no{} symbol represent significant areas for improvement and are discussed in the analysis below.

\begin{center}
\begin{tabular}{p{0.7\textwidth}c}
\toprule
\textbf{Control Question} & \textbf{Status} \\
\midrule
Do you require MFA to access email? & \no \\
Do you require MFA to log into computers? & \no \\
Do you require MFA to access sensitive data systems? & \yes \\
Does your organization have an employee acceptable use policy? & \yes \\
Does your organization do security awareness training for new employees? & \yes \\
Does your organization do security awareness training for all employees at least once per year? & \yes \\
\bottomrule
\end{tabular}
\end{center}

\subsection*{Analysis of Control Gaps}
The lack of MFA for email and computer logins represents a \textbf{critical security gap}. Email is a primary vector for phishing attacks, and a compromised account can lead to business email compromise, data exfiltration, and further network intrusion. Similarly, the absence of MFA on workstations removes a crucial layer of defense against unauthorized access should an employee's credentials be stolen.

% --- Technical Scan Results ---
\section{Technical Scan Results}
An external network scan was performed to identify open ports and services exposed to the internet.

\subsection*{Host: \texttt{[Target IP]}}
The scan revealed the following open port:
\begin{center}
\begin{tabular}{ccccc}
\toprule
\textbf{Port} & \textbf{State} & \textbf{Service} & \textbf{Product} & \textbf{Version} \\
\midrule
3306/tcp & open & mysql & MySQL & 5.7.33 \\
\bottomrule
\end{tabular}
\end{center}

\subsection*{Analysis of Technical Findings}
The scan identified an open MySQL database port (3306). Exposing a database directly to the internet is a highly dangerous practice. The detected version, \textbf{MySQL 5.7.33}, is a major concern as the MySQL 5.7 series reached its \textbf{End of Life (EOL) in October 2023}. EOL software does not receive security updates, making it highly susceptible to exploitation from numerous publicly known vulnerabilities. This finding directly confirms and elevates the pre-existing "Database Exposure" risk.

% --- Consolidated Risk Assessment ---
\section{Consolidated Risk Assessment}
This section synthesizes findings from all data sources into a prioritized list of risks.

\begin{center}
\begin{tabular}{p{0.25\textwidth}p{0.5\textwidth}l}
\toprule
\textbf{Risk Name} & \textbf{Overview} & \textbf{Severity} \\
\midrule
\textbf{Exposed End-of-Life Database} & The MySQL database (v5.7.33) is publicly accessible on port 3306. This version is past its End of Life and is no longer patched, exposing it to numerous known vulnerabilities. & \textbf{\textcolor{darkred}{Critical}} \\
\addlinespace
\textbf{Lack of MFA for Email} & The absence of Multi-Factor Authentication on email accounts significantly increases the risk of account compromise through phishing or credential stuffing attacks. & \textbf{\textcolor{darkred}{Critical}} \\
\addlinespace
\textbf{Lack of MFA for Workstations} & User computers do not require MFA for login. A compromised password would be sufficient for an attacker to gain access to a workstation and the internal network. & \textbf{\textcolor{darkorange}{High}} \\
\bottomrule
\end{tabular}
\end{center}

% --- Recommendations ---
\section{Recommendations}
Based on the identified risks, we provide the following prioritized recommendations to improve the organization's security posture.

\subsection*{Priority 1: Remediate Critical Risks (Immediate Action Required)}
\begin{enumerate}
    \item \textbf{Restrict Database Access:} Immediately implement firewall rules to \textbf{block all public access} to TCP port 3306 on host \texttt{[Target IP]}. Access should be restricted to trusted internal IP addresses or through a secure VPN connection only.
    \item \textbf{Implement MFA for Email:} Immediately begin the rollout of MFA for all email accounts. This is the single most effective control to prevent account takeovers and business email compromise.
    \item \textbf{Plan Database Upgrade:} Develop an urgent plan to migrate the MySQL 5.7.33 database to a currently supported version (e.g., MySQL 8.0 or later). The existing version is a critical liability as it no longer receives security patches.
\end{enumerate}

\subsection*{Priority 2: Remediate High Risks (Action within 30-60 days)}
\begin{enumerate}
    \item \textbf{Implement MFA for Workstations:} Roll out MFA for all computer and laptop logins. This adds a critical layer of security to protect endpoints and the internal network from unauthorized access.
\end{enumerate}

\subsection*{Priority 3: Continuous Improvement}
\begin{enumerate}
    \item \textbf{Regular Vulnerability Scanning:} Implement a regular, automated vulnerability scanning schedule for all external-facing assets to proactively identify and remediate new risks.
    \item \textbf{Review Security Policies:} While an acceptable use policy exists, review and update it annually to reflect the current threat landscape and organizational changes.
\end{enumerate}

\end{document}
```