```latex
\documentclass[12pt]{article}

% --- PACKAGES ---
\usepackage[a4paper, margin=1in]{geometry}
\usepackage{pifont} % For checkmarks and crosses
\usepackage{booktabs} % For professional tables
\usepackage{hyperref} % For hyperlinks
\usepackage{url} % For URL formatting
\usepackage{seqsplit} % For splitting long strings
\usepackage{graphicx}
\usepackage{xcolor}

% --- DOCUMENT SETUP ---
\hypersetup{
    colorlinks=true,
    linkcolor=blue,
    filecolor=magenta,      
    urlcolor=cyan,
    pdftitle={Cybersecurity Posture Report},
    pdfpagemode=FullScreen,
}

\newcommand{\yes}{\ding{51}}
\newcommand{\no}{\ding{55}}

% --- DOCUMENT START ---
\begin{document}

% --- TITLE PAGE ---
\begin{titlepage}
    \centering
    \vspace*{1cm}
    \Huge\textbf{Cybersecurity Posture Report}
    \vspace{1.5cm}
    \Large
    Prepared for: \textbf{[Organization Name]}
    \vspace{2cm}
    \includegraphics[width=0.4\textwidth]{example-image-a} % Placeholder for a logo
    \vfill
    \large
    \textbf{Date of Assessment:} November 22, 2025 \\
    \textbf{Report Generated By:} Cybersecurity Analyst
\end{titlepage}

\tableofcontents
\newpage

% --- EXECUTIVE SUMMARY ---
\section*{Executive Summary}

This report provides a comprehensive analysis of the cybersecurity posture for \textbf{[Organization Name]}, based on a combination of network scanning, a security controls questionnaire, and a review of pre-existing risks. The assessment, conducted on November 22, 2025, identified several critical areas requiring immediate attention.

While the organization has implemented multi-factor authentication (MFA) for email and sensitive data systems, significant gaps exist in endpoint security, employee security awareness, and formal policy. Specifically, the lack of MFA for computer logins represents a critical vulnerability that could facilitate unauthorized access and lateral movement within the network. Furthermore, the absence of an acceptable use policy and any security awareness training program leaves the organization highly susceptible to social engineering and insider threats.

The external network scan revealed a public-facing web server running an outdated version of Nginx (1.18.0). This software is several years old and is known to have multiple published vulnerabilities, presenting a direct and exploitable vector for external attackers.

Immediate remediation efforts should focus on upgrading the vulnerable web server, implementing mandatory MFA for all endpoint devices, and establishing a foundational security awareness and policy program.

% --- ORGANIZATIONAL INFORMATION ---
\section*{Organizational Information}
\begin{itemize}
    \item \textbf{Organization Name:} \textbf{[Organization Name]}
    \item \textbf{Primary Domain:} \texttt{[Domain]}
    \item \textbf{Scanned External IP:} \texttt{[Client IP]}
\end{itemize}

% --- SECURITY CONTROL REVIEW ---
\section*{Security Control Review}

The following table summarizes the organization's responses to a security controls questionnaire. Items marked with a red cross (\no) indicate significant gaps in the current security framework.

\begin{table}[h!]
\centering
\caption{Security Controls Questionnaire Results}
\begin{tabular}{p{0.8\linewidth} c}
\toprule
\textbf{Control Question} & \textbf{Status} \\
\midrule
Do you require MFA to access email? & \yes \\
Do you require MFA to log into computers? & \textcolor{red}{\no} \\
Do you require MFA to access sensitive data systems? & \yes \\
Does your organization have an employee acceptable use policy? & \textcolor{red}{\no} \\
Does your organization do security awareness training for new employees? & \textcolor{red}{\no} \\
Does your organization do security awareness training for all employees at least once per year? & \textcolor{red}{\no} \\
\bottomrule
\end{tabular}
\end{table}

\subsection*{Analysis of Control Gaps}
The questionnaire reveals critical deficiencies in foundational security practices:
\begin{itemize}
    \item \textbf{No MFA on Computers:} This is a critical oversight. If an attacker compromises a user's credentials (e.g., through phishing), they can gain direct access to the user's computer and potentially pivot to other systems on the internal network without facing a second authentication challenge.
    \item \textbf{No Acceptable Use Policy (AUP):} The absence of an AUP means there are no formal, documented rules for how employees should use company assets. This creates ambiguity and makes it difficult to enforce security standards.
    \item \textbf{No Security Awareness Training:} Employees are the first line of defense. Without training, they are significantly more likely to fall victim to phishing, malware, and other social engineering attacks, which are the leading causes of security breaches.
\end{itemize}

% --- TECHNICAL SCAN RESULTS ---
\section*{Technical Scan Results}

An external network scan was performed to identify open ports and exposed services.

\subsection*{Scan Details}
\begin{itemize}
    \item \textbf{Target IP Address:} \texttt{[Target IP]}
    \item \textbf{Scan Date:} November 22, 2025
\end{itemize}

\subsection*{Open Ports and Services}
The following table details the services discovered on the target system.

\begin{table}[h!]
\centering
\caption{Discovered Open Ports}
\begin{tabular}{lllll}
\toprule
\textbf{Port} & \textbf{State} & \textbf{Service} & \textbf{Product} & \textbf{Version} \\
\midrule
443/TCP & open & https & nginx & 1.18.0 \\
\bottomrule
\end{tabular}
\end{table}

\subsection*{Technical Analysis}
The scan identified a single open port, 443 (HTTPS), running an \textbf{Nginx 1.18.0} web server. This version was released in April 2020 and is now considered severely outdated. It is known to be vulnerable to multiple Common Vulnerabilities and Exposures (CVEs), such as CVE-2021-23017. Running outdated software on a public-facing server presents a high-impact risk, as it provides a readily available entry point for attackers to compromise the system.

% --- RISK ASSESSMENT ---
\section*{Risk Assessment}

The following risks have been identified by correlating the findings from the security control review and the technical scan.

\begin{table}[h!]
\centering
\caption{Identified Security Risks}
\begin{tabular}{p{0.1\linewidth} p{0.25\linewidth} p{0.45\linewidth} p{0.1\linewidth}}
\toprule
\textbf{Risk ID} & \textbf{Risk Name} & \textbf{Overview} & \textbf{Severity} \\
\midrule
RISK-001 & Outdated Public-Facing Web Server & The external web server is running Nginx 1.18.0, a version with known, exploitable vulnerabilities. This could lead to server compromise, data breach, or service disruption. & \textbf{High} \\
\addlinespace
RISK-002 & Lack of Endpoint MFA & The absence of MFA on computer logins allows an attacker with stolen credentials to gain unauthorized access to the internal network, facilitating data theft and lateral movement. & \textbf{Critical} \\
\addlinespace
RISK-003 & Deficient Security Awareness Program & The lack of an AUP and security training makes employees vulnerable to phishing and other social engineering attacks, increasing the likelihood of an initial breach. & \textbf{High} \\
\bottomrule
\end{tabular}
\end{table}

% --- RECOMMENDATIONS ---
\section*{Recommendations}

To mitigate the identified risks and improve the overall security posture, the following actions are recommended with high priority.

\begin{enumerate}
    \item \textbf{Upgrade Nginx Server (RISK-001):}
    \begin{itemize}
        \item Immediately plan and execute an upgrade of the Nginx server from version 1.18.0 to the latest stable version.
        \item Implement a formal patch management policy to ensure all public-facing systems are updated in a timely manner.
    \end{itemize}

    \item \textbf{Implement Endpoint MFA (RISK-002):}
    \begin{itemize}
        \item Deploy and enforce a mandatory Multi-Factor Authentication solution for all employee computer and laptop logins.
        \item Prioritize implementation for privileged users (e.g., administrators) and users with access to sensitive data.
    \end{itemize}

    \item \textbf{Establish Security Policy and Training (RISK-003):}
    \begin{itemize}
        \item Develop and ratify a formal Employee Acceptable Use Policy (AUP) that clearly defines rules for using company technology and data.
        \item Implement a mandatory security awareness training program for all employees upon hiring and at least annually thereafter. Training should cover key topics like phishing recognition, password hygiene, and data handling.
    \end{itemize}
\end{enumerate}

% --- CONCLUSION ---
\section*{Conclusion}
This assessment highlights critical vulnerabilities in the security posture of \textbf{[Organization Name]}. While some controls are in place, the combination of an outdated external service, lack of endpoint MFA, and a non-existent security awareness program creates a high-risk environment. We strongly urge the organization to prioritize the recommendations outlined in this report to significantly reduce its attack surface and strengthen its defenses against common cyber threats.

\end{document}
```