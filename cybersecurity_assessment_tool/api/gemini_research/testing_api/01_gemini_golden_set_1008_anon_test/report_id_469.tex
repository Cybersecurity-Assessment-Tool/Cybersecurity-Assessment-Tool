```latex
\documentclass[12pt]{article}

% Preamble: Required Packages
\usepackage[margin=1in]{geometry}
\usepackage{pifont} % For checkmarks and crosses
\usepackage{booktabs} % For professional tables
\usepackage{hyperref} % For clickable links and ToC
\usepackage{url} % For formatting URLs
\usepackage{seqsplit} % For splitting long strings in tt font
\usepackage[T1]{fontenc}

% Document Metadata
\title{Cybersecurity Posture Assessment Report}
\author{Cybersecurity Analysis Division}
\date{November 22, 2025}

\hypersetup{
    colorlinks=true,
    linkcolor=black,
    urlcolor=blue,
    pdftitle={Cybersecurity Posture Assessment Report},
    pdfauthor={Cybersecurity Analysis Division},
}

\begin{document}

\maketitle
\thispagestyle{empty}
\newpage
\tableofcontents
\newpage

% --- Section 1: Executive Overview ---
\section{Executive Overview}

This report provides a comprehensive cybersecurity posture assessment for \textbf{[Organization Name]}, conducted on November 22, 2025. The analysis is based on a combination of self-reported organizational data, an external network scan, and a review of pre-existing risks.

The assessment reveals a mixed security posture. While the organization has implemented some essential controls, such as requiring Multi-Factor Authentication (MFA) for email and sensitive systems, several critical and high-risk gaps were identified. These gaps expose the organization to significant threats, including unauthorized access, data breaches, and system compromise.

Key findings include:
\begin{itemize}
    \item \textbf{Critical Endpoint Security Gap:} The absence of mandatory MFA for computer logins represents a significant vulnerability, potentially allowing unauthorized access to local systems and network resources.
    \item \textbf{Outdated Public-Facing Software:} An external scan identified an outdated and unsupported version of the Nginx web server (1.18.0), which is known to have security vulnerabilities.
    \item \textbf{Insufficient Security Training:} The lack of a mandatory annual security awareness training program for all employees increases the risk of human error leading to security incidents, such as phishing attacks.
\end{itemize}

Immediate remediation of these findings is strongly recommended to reduce the organization's attack surface and improve its overall defensive capabilities. Detailed recommendations are provided in Section \ref{sec:recommendations}.

% --- Section 2: Organizational Information ---
\section{Organizational Information}

This section details the organizational context and scope of the assessment. The information provided is based on data supplied by the client and metadata from the technical scan.

\begin{table}[h!]
\centering
\begin{tabular}{@{}ll@{}}
\toprule
\textbf{Attribute} & \textbf{Value} \\ \midrule
Organization Name & \textbf{[Organization Name]} \\
Primary Email Domain & \texttt{[Domain]} \\
External IP Scanned & \texttt{[Client IP]} \\
Assessment Date & November 22, 2025 \\ \bottomrule
\end{tabular}
\caption{Assessment Scope and Details}
\label{tab:org_info}
\end{table}

% --- Section 3: Security Control Review ---
\section{Security Control Review}

The following table summarizes the organization's responses to a security controls questionnaire. A checkmark (\ding{51}) indicates a positive control is in place, while a cross (\ding{55}) indicates a control gap that may introduce risk.

\begin{table}[h!]
\centering
\begin{tabular}{@{}lc@{}}
\toprule
\textbf{Control Question} & \textbf{Response} \\ \midrule
Do you require MFA to access email? & \ding{51} \\
Do you require MFA to log into computers? & \textbf{\color{red}\ding{55}} \\
Do you require MFA to access sensitive data systems? & \ding{51} \\
Does your organization have an employee acceptable use policy? & \ding{51} \\
Does your organization do security awareness training for new employees? & \ding{51} \\
Does your organization do security awareness training for all employees at least once per year? & \textbf{\color{red}\ding{55}} \\ \bottomrule
\end{tabular}
\caption{Security Controls Questionnaire Results}
\label{tab:controls}
\end{table}

\subsection*{Analysis of Control Gaps}
Two significant control gaps were identified from the questionnaire:
\begin{itemize}
    \item \textbf{No MFA for Computer Logins:} This is a critical weakness. If an employee's credentials are stolen (e.g., via phishing), an attacker could gain direct access to their workstation and, potentially, the internal network.
    \item \textbf{No Annual Security Awareness Training:} Security threats evolve rapidly. Failing to provide regular, updated training for all staff members leaves the organization highly susceptible to social engineering and phishing attacks.
\end{itemize}

% --- Section 4: Technical Scan Results ---
\section{Technical Scan Results}

An external network scan was performed against the organization's public-facing infrastructure to identify open ports and exposed services.

\begin{itemize}
    \item \textbf{Target IP Address:} \texttt{[Target IP]}
    \item \textbf{Scan Date:} 2025-11-22T10:00:00Z
\end{itemize}

\begin{table}[h!]
\centering
\begin{tabular}{@{}llll@{}}
\toprule
\textbf{Port} & \textbf{State} & \textbf{Service} & \textbf{Version} \\ \midrule
443/tcp & open & https (nginx) & 1.18.0 \\ \bottomrule
\end{tabular}
\caption{Open Ports and Services Detected}
\label{tab:scan_results}
\end{table}

\subsection*{Analysis of Technical Findings}
The scan identified a single open port, 443 (HTTPS), running an Nginx web server. The detected version, \textbf{1.18.0}, was released in April 2020 and is now considered outdated and is no longer supported. This version is associated with known security vulnerabilities (e.g., CVE-2021-23017) that could be exploited by attackers to compromise the server. Exposing end-of-life software to the internet presents a high risk to the organization.

% --- Section 5: Consolidated Risk Assessment ---
\section{Consolidated Risk Assessment}

This section synthesizes findings from the security control review and the technical scan. No pre-existing vulnerabilities were reported. The following table prioritizes the newly identified risks based on their potential impact.

\begin{table}[h!]
\centering
\begin{tabular}{@{}lp{4cm}lp{5.5cm}@{}}
\toprule
\textbf{ID} & \textbf{Risk Name} & \textbf{Severity} & \textbf{Description} \\ \midrule
RISK-001 & Lack of MFA on Workstations & \textbf{Critical} & The absence of MFA on computer logins allows an attacker with stolen credentials to gain direct access to endpoints and the internal network. \\
\addlinespace
RISK-002 & Outdated Nginx Web Server & \textbf{High} & The public-facing web server is running an unsupported version of Nginx (1.18.0) with known vulnerabilities, exposing it to remote compromise. \\
\addlinespace
RISK-003 & Inadequate Security Awareness Training & \textbf{High} & Without mandatory annual training, employees are more likely to fall victim to phishing and other social engineering attacks, leading to credential theft or malware infection. \\ \bottomrule
\end{tabular}
\caption{Identified Risks and Severity}
\label{tab:risks}
\end{table}

% --- Section 6: Recommendations ---
\section{Recommendations}
\label{sec:recommendations}

The following actionable recommendations are provided to mitigate the identified risks and strengthen the organization's overall security posture.

\subsection*{RISK-001: Lack of MFA on Workstations (Critical)}
\begin{itemize}
    \item \textbf{Action:} Implement a mandatory Multi-Factor Authentication (MFA) policy for all computer and workstation logins (both local and remote).
    \item \textbf{Details:} Utilize solutions compatible with your existing operating systems (e.g., Windows Hello for Business, Duo, or other third-party tools). Prioritize implementation for privileged users (administrators) and executives.
\end{itemize}

\subsection*{RISK-002: Outdated Nginx Web Server (High)}
\begin{itemize}
    \item \textbf{Action:} Immediately upgrade the Nginx server on host \texttt{[Target IP]} from version 1.18.0 to a current, stable, and fully patched version.
    \item \textbf{Details:} Follow a robust patch management process that includes testing the upgrade in a non-production environment before deploying to production. Subscribe to vendor security advisories to ensure timely patching in the future.
\end{itemize}

\subsection*{RISK-003: Inadequate Security Awareness Training (High)}
\begin{itemize}
    \item \textbf{Action:} Establish a mandatory annual security awareness training program for all employees.
    \item \textbf{Details:} The program should cover current threats such as phishing, ransomware, proper data handling, and password security. Supplement annual training with regular phishing simulation campaigns to measure effectiveness and provide targeted follow-up training to susceptible users.
\end{itemize}

\end{document}
```