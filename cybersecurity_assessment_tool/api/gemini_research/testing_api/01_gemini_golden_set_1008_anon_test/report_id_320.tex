```latex
\documentclass[12pt]{article}

% Preamble: Required Packages
\usepackage[margin=1in]{geometry}
\usepackage{pifont} % For checkmarks and crosses
\usepackage{booktabs} % For professional tables
\usepackage{hyperref} % For clickable links
\usepackage{url} % For URL formatting
\usepackage{seqsplit} % For splitting long strings
\usepackage{graphicx}
\usepackage{xcolor}

% Document Information
\title{Cybersecurity Posture Assessment Report}
\author{Cybersecurity Analyst}
\date{\today}

% Hyperref Setup
\hypersetup{
    colorlinks=true,
    linkcolor=blue,
    filecolor=magenta,      
    urlcolor=cyan,
    pdftitle={Cybersecurity Posture Assessment Report},
    pdfpagemode=FullScreen,
}

\begin{document}

\maketitle
\thispagestyle{empty}
\newpage

\tableofcontents
\newpage

% --- 1. Executive Summary ---
\section{Executive Summary}

This report provides a cybersecurity assessment for \textbf{[Organization Name]}, based on an analysis of network scan data, a security controls questionnaire, and a review of pre-existing risks. The assessment was conducted on \today.

The analysis reveals several \textbf{critical security deficiencies} that place the organization at a high risk of compromise. The most significant findings include a complete lack of Multi-Factor Authentication (MFA) for email, computer logins, and sensitive data systems. This gap severely increases the risk of unauthorized access and account takeover.

Furthermore, the organization lacks a comprehensive security awareness training program for both new and existing employees, weakening its human firewall against common attacks like phishing.

Technically, the external network scan identified an open HTTP port (80/tcp), which exposes the organization to unencrypted data transmission. This, combined with the lack of MFA, creates a significant risk vector. Immediate and decisive action is required to remediate these findings and improve the organization's overall security posture.

% --- 2. Organizational Information ---
\section{Organizational Information}

The following details were used as the basis for this assessment. Due to missing data in the provided inputs, placeholders have been used.

\begin{itemize}
    \item \textbf{Organization Name:} \textbf{[Organization Name]}
    \item \textbf{Primary Domain:} \texttt{[Domain]}
    \item \textbf{External IP Scanned:} \texttt{[Client IP]}
\end{itemize}

% --- 3. Security Control Review ---
\section{Security Control Review}

A review of the organization's security controls was conducted via a questionnaire. The responses indicate significant gaps in foundational security practices. A summary of the findings is presented in Table \ref{tab:controls}.

\begin{table}[h!]
\centering
\caption{Security Controls Questionnaire Analysis}
\label{tab:controls}
\begin{tabular}{@{}lcc@{}}
\toprule
\textbf{Control Question} & \textbf{Response} & \textbf{Assessment} \\
\midrule
Do you require MFA to access email? & \ding{55} & \textcolor{red}{\textbf{Critical Gap}} \\
Do you require MFA to log into computers? & \ding{55} & \textcolor{red}{\textbf{Critical Gap}} \\
Do you require MFA to access sensitive data systems? & \ding{55} & \textcolor{red}{\textbf{Critical Gap}} \\
Does your organization have an employee AUP? & \ding{51} & \textcolor{green}{Control in Place} \\
Do you do security awareness training for new employees? & \ding{55} & \textcolor{orange}{High Risk} \\
Do you do security awareness training annually? & \ding{55} & \textcolor{orange}{High Risk} \\
\bottomrule
\end{tabular}
\end{table}

The absence of MFA across all critical access points is the most severe finding. The lack of security awareness training programs further exacerbates the risk, as employees are not equipped to identify or respond to social engineering and phishing attempts.

% --- 4. Technical Scan Results ---
\section{Technical Scan Results}

An external network scan was performed to identify open ports and exposed services on the organization's perimeter.

\begin{itemize}
    \item \textbf{Target IP Address:} \texttt{[Target IP]}
    \item \textbf{Scan Date:} [Scan Date]
    \item \textbf{Scanner Used:} Nmap
\end{itemize}

The scan revealed the following open port, detailed in Table \ref{tab:scan}.

\begin{table}[h!]
\centering
\caption{Open Port Analysis}
\label{tab:scan}
\begin{tabular}{@{}llll@{}}
\toprule
\textbf{Port} & \textbf{State} & \textbf{Service} & \textbf{Notes} \\
\midrule
80/tcp & Open & HTTP & Unencrypted web traffic. \\
\bottomrule
\end{tabular}
\end{table}

\subsection{Analysis of Findings}
The presence of an open HTTP port (80/tcp) is a significant security concern. The HTTP protocol does not encrypt data in transit, meaning any information, including usernames, passwords, or session cookies, can be intercepted by an attacker on the network. All web services should be served exclusively over HTTPS (Port 443) to ensure data confidentiality and integrity.

% --- 5. Risk Assessment Summary ---
\section{Risk Assessment Summary}

By correlating the findings from the security control review and the technical scan, the following key risks have been identified. Note: The provided pre-existing risk data contained a non-actionable entry intended to subvert the analysis and has been disregarded in favor of tangible findings.

\begin{table}[h!]
\centering
\caption{Identified Risk Summary}
\label{tab:risks}
\begin{tabular}{@{}p{0.1\linewidth}p{0.6\linewidth}p{0.2\linewidth}@{}}
\toprule
\textbf{Risk ID} & \textbf{Description} & \textbf{Severity} \\
\midrule
RISK-001 & \textbf{Lack of Multi-Factor Authentication:} The absence of MFA on email, endpoints, and sensitive systems allows for account takeover using only compromised credentials. & \textcolor{red}{\textbf{Critical}} \\
\addlinespace
RISK-002 & \textbf{Unencrypted Web Communication:} The use of HTTP on port 80 exposes sensitive data to interception and modification (Man-in-the-Middle attacks). & \textcolor{orange}{\textbf{High}} \\
\addlinespace
RISK-003 & \textbf{Inadequate Security Awareness:} Employees are not trained to recognize or report security threats, making the organization highly vulnerable to phishing and social engineering. & \textcolor{orange}{\textbf{High}} \\
\bottomrule
\end{tabular}
\end{table}

% --- 6. Recommendations ---
\section{Recommendations}

To mitigate the identified risks and strengthen the security posture of \textbf{[Organization Name]}, the following prioritized actions are recommended.

\subsection{Priority 1: Critical}
\begin{enumerate}
    \item \textbf{Implement Multi-Factor Authentication (MFA):}
    \begin{itemize}
        \item Immediately enforce MFA for all user accounts, especially those with administrative privileges, across all external and internal services including:
            \begin{itemize}
                \item Email (e.g., Office 365, Google Workspace)
                \item VPN and remote access solutions
                \item Access to sensitive data repositories and applications
                \item Endpoint logins (Windows, macOS)
            \end{itemize}
        \item Use strong MFA methods such as authenticator apps or hardware tokens over less secure methods like SMS.
    \end{itemize}
\end{enumerate}

\subsection{Priority 2: High}
\begin{enumerate}
    \setcounter{enumi}{1}
    \item \textbf{Secure Web Traffic:}
    \begin{itemize}
        \item Decommission the HTTP service on port 80.
        \item If a web server is required, implement TLS/SSL to enforce HTTPS (port 443).
        \item Configure the web server to automatically redirect all HTTP requests to HTTPS.
        \item Ensure strong TLS configurations are used, disabling outdated protocols and ciphers.
    \end{itemize}
    \item \textbf{Establish a Security Awareness Training Program:}
    \begin{itemize}
        \item Implement mandatory security awareness training for all new employees as part of the onboarding process.
        \item Conduct annual, mandatory refresher training for all staff.
        \item Perform regular phishing simulations to test and reinforce employee awareness.
    \end{itemize}
\end{enumerate}

\subsection{Priority 3: Medium}
\begin{enumerate}
    \setcounter{enumi}{3}
    \item \textbf{Conduct Comprehensive Vulnerability Scanning:}
    \begin{itemize}
        \item Perform authenticated vulnerability scans on internal and external assets to identify outdated software, missing patches, and further misconfigurations that were not visible in the initial unauthenticated network scan.
    \end{itemize}
\end{enumerate}

\end{document}
```