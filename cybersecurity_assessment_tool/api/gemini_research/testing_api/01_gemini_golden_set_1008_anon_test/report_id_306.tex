```latex
\documentclass[12pt]{article}

% --- PACKAGES ---
\usepackage[margin=1in]{geometry}
\usepackage{pifont} % Required for \ding{51} (checkmark) and \ding{55} (cross)
\usepackage{booktabs} % For professional-looking tables
\usepackage[hidelinks]{hyperref}
\usepackage{url}
\usepackage{seqsplit} % To split long strings in \texttt
\usepackage{xcolor} % For colors
\usepackage{fancyhdr} % For headers and footers
\usepackage{lastpage} % To get the total number of pages

% --- DOCUMENT METADATA ---
\title{Cybersecurity Posture Assessment Report}
\author{Cybersecurity Analysis Division}
\date{\today}

% --- HEADER & FOOTER DEFINITION ---
\pagestyle{fancy}
\fancyhf{} % Clear all header and footer fields
\fancyhead[L]{\textbf{Cybersecurity Assessment}}
\fancyhead[R]{\textbf{[Organization Name]}}
\fancyfoot[C]{\thepage\ of \pageref{LastPage}}
\renewcommand{\headrulewidth}{0.4pt}
\renewcommand{\footrulewidth}{0.4pt}

\begin{document}

\maketitle
\thispagestyle{empty}
\newpage

\tableofcontents
\newpage

% ===================================================================
% SECTION 1: EXECUTIVE SUMMARY
% ===================================================================
\section*{Executive Summary}

This report provides a comprehensive cybersecurity assessment for \textbf{[Organization Name]}, based on an analysis of network scan data, organizational security controls, and pre-existing risk information. The assessment was conducted on \today.

The analysis reveals several critical and high-risk gaps in the organization's security posture, primarily related to identity and access management and foundational security policies. The most pressing concerns are the lack of Multi-Factor Authentication (MFA) for email and computer access, the absence of an employee acceptable use policy, and no mandatory security training for new hires. These deficiencies significantly increase the risk of unauthorized access, data breaches, and successful phishing attacks.

On the technical front, a network scan of the target IP address \texttt{[Target IP]} did not identify any open ports. This result contradicts a pre-existing risk entry that indicates an unencrypted web server is active on Port 80. This discrepancy requires immediate investigation to confirm whether the risk has been remediated or if the scan scope was incomplete.

This report outlines the detailed findings and provides a prioritized list of actionable recommendations to mitigate the identified risks and strengthen the overall security posture of \textbf{[Organization Name]}.

\vspace{1cm}

% ===================================================================
% SECTION 2: ORGANIZATIONAL INFORMATION
% ===================================================================
\section*{Organizational Information}

The following information was used as the basis for this assessment. Due to the anonymized nature of the provided data, placeholders have been used where necessary.

\begin{tabular}{@{}ll}
    \toprule
    \textbf{Attribute} & \textbf{Value} \\
    \midrule
    Organization Name & \textbf{[Organization Name]} \\
    Email Domain & \texttt{[Domain]} \\
    External IP Address (Target) & \texttt{[Client IP]} \\
    Scan Target & \texttt{[Target IP]} \\
    Scan Date & \today \\
    \bottomrule
\end{tabular}

\newpage

% ===================================================================
% SECTION 3: SECURITY CONTROL REVIEW
% ===================================================================
\section*{Security Control Review (Questionnaire Analysis)}

The following table summarizes the organization's responses to a security controls questionnaire. Each "No" response represents a significant gap in the security framework and has been flagged with an associated risk level.

\begin{table}[h!]
\centering
\begin{tabular}{@{}p{8.5cm}ccp{3cm}@{}}
    \toprule
    \textbf{Control Question} & \textbf{Response} & \textbf{Status} & \textbf{Assessment} \\
    \midrule
    Do you require MFA to access email? & No & \ding{55} & \textcolor{red}{\textbf{Critical Gap}} \\
    Do you require MFA to log into computers? & No & \ding{55} & \textcolor{red}{\textbf{Critical Gap}} \\
    Do you require MFA to access sensitive data systems? & Yes & \ding{51} & Control Implemented \\
    Does your organization have an employee acceptable use policy? & No & \ding{55} & High Risk \\
    Does your organization do security awareness training for new employees? & No & \ding{55} & High Risk \\
    Does your organization do security awareness training for all employees at least once per year? & Yes & \ding{51} & Control Implemented \\
    \bottomrule
\end{tabular}
\caption{Security Controls Questionnaire Results}
\end{table}

\subsection*{Analysis of Control Gaps}
\begin{itemize}
    \item \textbf{Lack of MFA:} The absence of MFA for email and computer logins is a critical vulnerability. Compromised credentials could directly lead to unauthorized access to sensitive communications and internal network resources.
    \item \textbf{Missing Acceptable Use Policy (AUP):} Without a formal AUP, there is no clear standard for employee behavior regarding company assets, increasing the risk of insider threats and unintentional data exposure.
    \item \textbf{No New Hire Training:} Failing to train new employees on security best practices from day one leaves the organization vulnerable, as new staff may be unaware of policies, threat identification, or proper data handling procedures.
\end{itemize}

\newpage

% ===================================================================
% SECTION 4: TECHNICAL SCAN RESULTS
% ===================================================================
\section*{Technical Scan Results}

A network port scan was performed on the designated target IP address to identify accessible services.

\subsection*{Nmap Scan Details}
\begin{itemize}
    \item \textbf{Target IP:} \texttt{[Target IP]}
    \item \textbf{Scan Type:} TCP Port Scan
    \item \textbf{Target Status:} Up
\end{itemize}

\begin{table}[h!]
\centering
\begin{tabular}{@{}llll@{}}
    \toprule
    \textbf{Port} & \textbf{State} & \textbf{Service} & \textbf{Version} \\
    \midrule
    80/tcp & closed & http & N/A \\
    \bottomrule
\end{tabular}
\caption{Port Scan Results for \texttt{[Target IP]}}
\end{table}

\subsection*{Analysis of Technical Findings}
The scan of \texttt{[Target IP]} revealed no open TCP ports. Port 80, commonly used for unencrypted web traffic, was explicitly identified as \textbf{closed}.

\textbf{Crucially, this technical finding contradicts the information provided in the "Current Risks" data}, which lists an active vulnerability related to an open Port 80. This discrepancy suggests one of the following possibilities:
\begin{enumerate}
    \item The previously identified risk has been successfully remediated.
    \item The current scan targeted a different asset than the one associated with the pre-existing risk.
    \item The service on Port 80 is only intermittently available or protected by a firewall that blocked this specific scan.
\end{enumerate}
Further investigation is required to resolve this conflict and validate the organization's external attack surface.

\newpage

% ===================================================================
% SECTION 5: CONSOLIDATED RISK ASSESSMENT
% ===================================================================
\section*{Consolidated Risk Assessment}

The following table synthesizes findings from the security questionnaire, technical scans, and pre-existing risk data into a unified list of current risks.

\begin{table}[h!]
\centering
\begin{tabular}{@{}p{4cm}p{7cm}l@{}}
    \toprule
    \textbf{Risk Name} & \textbf{Overview} & \textbf{Severity} \\
    \midrule
    No MFA for Email Access & Lack of MFA on email accounts allows for account takeover with only a password, exposing all communications. & \textbf{Critical} \\
    \addlinespace
    No MFA for Endpoint Login & Lack of MFA on computer logins allows an attacker with stolen credentials to gain direct access to the internal network. & \textbf{Critical} \\
    \addlinespace
    Lack of Acceptable Use Policy & Without a formal policy, employees may misuse company assets or mishandle data without consequence, increasing insider risk. & High \\
    \addlinespace
    No Security Training for New Hires & New employees are not trained on security policies and threat awareness, making them prime targets for phishing and social engineering. & High \\
    \addlinespace
    Unencrypted Web Server (Unverified) & A pre-existing risk states Port 80 is open. \textbf{Note:} The current scan on \texttt{[Target IP]} found this port closed. The status of this risk is unconfirmed and requires investigation. & Medium (5.0) \\
    \bottomrule
\end{tabular}
\caption{Summary of Identified Risks}
\end{table}

% ===================================================================
% SECTION 6: RECOMMENDATIONS
% ===================================================================
\section*{Recommendations}

Based on the analysis, the following actions are recommended to mitigate the identified risks. Recommendations are prioritized by severity.

\subsection*{Priority 1: Critical Risks}
\begin{enumerate}
    \item \textbf{Implement MFA Across All Critical Systems:}
    \begin{itemize}
        \item \textbf{Action:} Immediately enforce MFA for all users on the primary email system (e.g., Microsoft 365, Google Workspace).
        \item \textbf{Action:} Deploy and enforce MFA for all remote and on-site computer/endpoint logins.
    \end{itemize}
\end{enumerate}

\subsection*{Priority 2: High Risks}
\begin{enumerate}
    \setcounter{enumi}{1}
    \item \textbf{Develop and Implement Foundational Security Policies:}
    \begin{itemize}
        \item \textbf{Action:} Draft, approve, and distribute an official Employee Acceptable Use Policy (AUP). Require all employees to read and acknowledge it.
        \item \textbf{Action:} Integrate mandatory security awareness training into the new employee onboarding process. This should occur before a new hire is granted access to sensitive systems.
    \end{itemize}
\end{enumerate}

\subsection*{Priority 3: Verification and Investigation}
\begin{enumerate}
    \setcounter{enumi}{2}
    \item \textbf{Investigate Port 80 Discrepancy:}
    \begin{itemize}
        \item \textbf{Action:} Conduct a comprehensive vulnerability scan across all known external IP addresses for the organization to confirm the status of Port 80 and other services.
        \item \textbf{Action:} If Port 80 is found open on any asset, immediately implement the long-term recommendation of migrating the service to HTTPS (Port 443) and closing Port 80.
        \item \textbf{Action:} Update the central risk register to accurately reflect the findings of the investigation.
    \end{itemize}
\end{enumerate}

\end{document}
```