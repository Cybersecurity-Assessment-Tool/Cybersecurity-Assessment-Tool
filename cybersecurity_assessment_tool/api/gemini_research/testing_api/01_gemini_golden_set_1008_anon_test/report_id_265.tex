```latex
\documentclass[12pt]{article}

% Preamble: Required Packages
\usepackage[margin=1in]{geometry}
\usepackage{pifont} % For checkmarks and crosses
\usepackage{booktabs} % For professional tables
\usepackage{hyperref} % For hyperlinks, not used here but good practice
\usepackage{url} % For formatting URLs
\usepackage{seqsplit} % For splitting long text strings in tt font
\usepackage{graphicx} % For potential logos/images
\usepackage{xcolor} % For colors

% Hyperref Setup
\hypersetup{
    colorlinks=true,
    linkcolor=blue,
    filecolor=magenta,      
    urlcolor=cyan,
    pdftitle={Cybersecurity Assessment Report},
    pdfpagemode=FullScreen,
}

% Define custom colors for severity
\definecolor{criticalred}{HTML}{D7263D}
\definecolor{highorange}{HTML}{F49D37}
\definecolor{mediumyellow}{HTML}{F2DF3A}
\definecolor{lowblue}{HTML}{4A90E2}

% Checkmark and Cross definitions
\newcommand{\cmark}{\ding{51}}
\newcommand{\xmark}{\ding{55}}

\begin{document}

% --- Title Page ---
\begin{titlepage}
    \centering
    \vspace*{1cm}
    \Huge\textbf{Cybersecurity Assessment Report}
    \vspace{1.5cm}
    \Large
    \textbf{Prepared for:}\\
    \vspace{0.5cm}
    \textbf{[Organization Name]}
    \vspace{2cm}
    \large
    \textbf{Date of Report:}\\
    \today
    \vspace{1cm}
    \large
    \textbf{Date of Assessment:}\\
    November 22, 2025
    \vfill
    \large
    \textbf{Generated By:}\\
    Expert Cybersecurity Analyst
\end{titlepage}

\tableofcontents
\newpage

% --- Section 1: Executive Summary ---
\section{Executive Summary}
This report details the findings of a cybersecurity assessment conducted for \textbf{[Organization Name]}. The assessment combined a review of organizational security controls, an external network vulnerability scan, and an analysis of pre-existing risks.

The overall security posture requires immediate attention. Key findings include critical-risk vulnerabilities related to outdated public-facing software and significant gaps in foundational security controls, such as Multi-Factor Authentication (MFA) for endpoint devices. Furthermore, the lack of security awareness training for new employees presents a high risk, making the organization more susceptible to social engineering and phishing attacks.

This report provides a consolidated view of these risks and offers actionable recommendations to mitigate them, thereby strengthening the organization's defenses against common cyber threats.

% --- Section 2: Organizational Information ---
\section{Organizational Information}
The following details were used as the basis for this assessment. Due to the anonymized nature of the provided data, placeholders have been used where necessary.

\begin{itemize}
    \item \textbf{Organization Name:} \textbf{[Organization Name]}
    \item \textbf{Primary Domain:} \texttt{[Domain]}
    \item \textbf{External IP Address Scanned:} \texttt{[Client IP]}
\end{itemize}

% --- Section 3: Security Control Review ---
\section{Security Control Review (Questionnaire Analysis)}
A review of the organization's security policies and controls was conducted via a standardized questionnaire. The responses reveal critical gaps in the current security framework. "No" answers indicate a failure to meet baseline security best practices and are highlighted below.

\begin{table}[h!]
\centering
\caption{Security Controls Questionnaire Responses}
\begin{tabular}{p{0.75\linewidth} c}
\toprule
\textbf{Control Question} & \textbf{Response} \\
\midrule
Do you require MFA to access email? & \cmark \\
Do you require MFA to log into computers? & \textcolor{criticalred}{\xmark} \\
Do you require MFA to access sensitive data systems? & \cmark \\
Does your organization have an employee acceptable use policy? & \cmark \\
Does your organization do security awareness training for new employees? & \textcolor{highorange}{\xmark} \\
Does your organization do security awareness training for all employees at least once per year? & \cmark \\
\bottomrule
\end{tabular}
\end{table}

\subsection*{Analysis of Control Gaps}
Two significant control gaps were identified:
\begin{itemize}
    \item \textbf{Lack of Endpoint MFA:} The absence of MFA for computer logins is a critical weakness. If an employee's password is stolen or cracked, an attacker could gain direct access to their workstation, potentially accessing sensitive local data and using it as a pivot point to move laterally across the network.
    \item \textbf{No Onboarding Security Training:} New employees are a primary target for phishing and social engineering attacks. Failing to provide immediate security awareness training during onboarding leaves a significant window of vulnerability. This oversight undermines the effectiveness of the annual training program.
\end{itemize}

% --- Section 4: Technical Scan Results ---
\section{Technical Scan Results}
An external network scan was performed on \textbf{November 22, 2025}, targeting the primary external IP address. The scan identified one open port with a service running outdated software.

\begin{itemize}
    \item \textbf{Target IP Address:} \texttt{[Target IP]}
    \item \textbf{Scan Date:} 2025-11-22T10:00:00Z
\end{itemize}

\begin{table}[h!]
\centering
\caption{Open Ports and Services Detected}
\begin{tabular}{l l l l l}
\toprule
\textbf{Port} & \textbf{State} & \textbf{Service} & \textbf{Product} & \textbf{Version} \\
\midrule
443/tcp & Open & https & nginx & \textcolor{criticalred}{1.18.0} \\
\bottomrule
\end{tabular}
\end{table}

\subsection*{Analysis of Technical Findings}
The scan revealed that the public-facing web server is running \textbf{Nginx version 1.18.0}. This version was released in April 2020 and is now considered end-of-life. It is known to be vulnerable to several publicly disclosed security flaws, including but not limited to CVE-2021-23017 (1-byte memory overwrite). Running outdated software on an internet-facing system presents a critical risk, as it can be exploited by automated tools to gain unauthorized access, exfiltrate data, or disrupt service.

% --- Section 5: Consolidated Risk Assessment ---
\section{Consolidated Risk Assessment}
This section synthesizes findings from the security control review and the technical scan. No pre-existing risks were provided for this assessment. The following new risks have been identified and prioritized.

\begin{table}[h!]
\centering
\caption{Summary of Identified Risks}
\begin{tabular}{p{0.1\linewidth} p{0.25\linewidth} p{0.45\linewidth} p{0.1\linewidth}}
\toprule
\textbf{ID} & \textbf{Risk Name} & \textbf{Description} & \textbf{Severity} \\
\midrule
\textbf{R-01} & Outdated Web Server Software & The public-facing web server runs Nginx 1.18.0, an end-of-life version with multiple known vulnerabilities. & \textcolor{criticalred}{Critical} \\
\addlinespace
\textbf{R-02} & Lack of Endpoint MFA & The absence of MFA on computer logins exposes the organization to unauthorized access via compromised credentials. & \textcolor{criticalred}{Critical} \\
\addlinespace
\textbf{R-03} & Inadequate New Employee Security Training & New hires do not receive security awareness training upon joining, making them highly susceptible to phishing and social engineering. & \textcolor{highorange}{High} \\
\bottomrule
\end{tabular}
\end{table}

% --- Section 6: Recommendations ---
\section{Recommendations}
The following actions are recommended to mitigate the identified risks and improve the overall security posture of \textbf{[Organization Name]}.

\subsection*{R-01: Outdated Web Server Software (Critical)}
\begin{itemize}
    \item \textbf{Immediate Action:} Upgrade the Nginx server from version 1.18.0 to the latest stable mainline or stable version. This will patch known vulnerabilities and introduce security enhancements.
    \item \textbf{Strategic Action:} Implement a formal patch and vulnerability management program for all internet-facing systems. This program should include regular scanning, risk assessment, and a defined timeline for applying critical security patches.
\end{itemize}

\subsection*{R-02: Lack of Endpoint MFA (Critical)}
\begin{itemize}
    \item \textbf{Immediate Action:} Enable MFA for all user accounts, especially those with administrative privileges, for logging into company workstations and laptops. Solutions can include authenticator apps, hardware tokens, or biometric verification.
    \item \textbf{Strategic Action:} Review and update the identity and access management (IAM) policy to mandate MFA for all critical systems, not just endpoints and email.
\end{itemize}

\subsection*{R-03: Inadequate New Employee Security Training (High)}
\begin{itemize}
    \item \textbf{Immediate Action:} Develop and integrate a mandatory security awareness training module into the new employee onboarding process. This training should be completed within the first week of employment.
    \item \textbf{Strategic Action:} The onboarding module should cover key topics such as phishing identification, password hygiene, acceptable use of company assets, and how to report a security incident. This will build a strong security culture from day one.
\end{itemize}

\end{document}
```