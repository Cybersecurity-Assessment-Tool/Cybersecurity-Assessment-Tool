```latex
\documentclass[12pt, a4paper]{article}

% Preamble: Required Packages
\usepackage[margin=1in]{geometry}
\usepackage{pifont} % For checkmarks and crosses
\usepackage{booktabs} % For professional tables
\usepackage{hyperref} % For clickable links
\usepackage{url} % For URL formatting
\usepackage{seqsplit} % To split long strings in \texttt
\usepackage{xcolor} % For colors
\usepackage{graphicx} % For logo
\usepackage{fancyhdr} % For headers/footers
\usepackage{lastpage} % To get total page count

% --- Document Setup ---

% Define colors for severity levels
\definecolor{sev_critical}{HTML}{940000}
\definecolor{sev_high}{HTML}{D13F00}
\definecolor{sev_medium}{HTML}{E8A600}
\definecolor{sev_low}{HTML}{3E8E41}

% Hyperlink setup
\hypersetup{
    colorlinks=true,
    linkcolor=blue,
    filecolor=magenta,      
    urlcolor=cyan,
    pdftitle={Cybersecurity Risk Assessment Report},
    pdfpagemode=FullScreen,
}

% Header and Footer setup
\pagestyle{fancy}
\fancyhf{} % Clear all header and footer fields
\fancyhead[L]{\textbf{Cybersecurity Risk Assessment Report}}
\fancyhead[R]{\textbf{[Organization Name]}}
\fancyfoot[C]{Confidential}
\fancyfoot[R]{Page \thepage\ of \pageref{LastPage}}
\renewcommand{\headrulewidth}{0.4pt}
\renewcommand{\footrulewidth}{0.4pt}

% --- Document Start ---
\begin{document}

% --- Title Page ---
\begin{titlepage}
    \centering
    \vfill
    
    {\Huge\bfseries Cybersecurity Risk Assessment Report\par}
    \vspace{1.5cm}
    
    {\Large Prepared for:\par}
    \vspace{0.5cm}
    {\Huge\bfseries \textbf{[Organization Name]}}\par
    
    \vfill
    
    {\large \today\par}
    
    \vspace{1cm}
    \textit{This document is confidential and intended solely for the use of the recipient.}
    
\end{titlepage}

\tableofcontents
\newpage

% --- Section 1: Executive Summary ---
\section{Executive Summary}
This report provides a comprehensive analysis of the cybersecurity posture of \textbf{[Organization Name]}, based on technical network scans, a review of existing security controls, and an assessment of known risks. The assessment was conducted to identify vulnerabilities and provide actionable recommendations to enhance the organization's security.

The most critical finding is the direct exposure of a Remote Desktop Protocol (RDP) service on port 3389 to the public internet at IP address \texttt{[Client IP]}. This vulnerability, rated with a CVSS score of 9.0 (Critical), is a primary target for ransomware attacks and unauthorized access.

This critical technical vulnerability is dangerously compounded by significant gaps in administrative and access controls identified through the security questionnaire. Key concerns include:
\begin{itemize}
    \item \textbf{Lack of Multi-Factor Authentication (MFA):} MFA is not required for computer logins or access to sensitive data systems. This drastically increases the risk of a successful breach via compromised credentials.
    \item \textbf{Missing Governance Policies:} The absence of an employee Acceptable Use Policy (AUP) indicates a lack of foundational security governance.
\end{itemize}

The combination of an exposed, high-value service (RDP) with weak internal access controls creates a high-probability, high-impact risk scenario. Immediate remediation of the exposed RDP service is paramount. Following this, the organization must prioritize the implementation of MFA across all critical systems to build a more resilient defense.

% --- Section 2: Organizational Information ---
\section{Organizational Information}
This section details the information provided by the client organization. The data is used as a baseline for the assessment. Note that identity information was not provided and placeholders are used.

\begin{table}[h!]
\centering
\begin{tabular}{@{}ll@{}}
\toprule
\textbf{Attribute} & \textbf{Value} \\ \midrule
Organization Name & \textbf{[Organization Name]} \\
Primary Email Domain & \texttt{[Domain]} \\
External IP Address Scanned & \texttt{[Client IP]} \\ \bottomrule
\end{tabular}
\caption{Client Organizational Details.}
\label{tab:org_info}
\end{table}

% --- Section 3: Security Control Review ---
\section{Security Control Review}
The following table summarizes the organization's responses to a security controls questionnaire. A \textcolor{green}{\ding{51}} indicates a positive control is in place, while a \textcolor{red}{\ding{55}} indicates a control gap that introduces risk.

\begin{table}[h!]
\centering
\begin{tabular}{@{}p{0.5\textwidth}cp{0.3\textwidth}@{}}
\toprule
\textbf{Control Question} & \textbf{Response} & \textbf{Analyst Notes} \\ \midrule
Do you require MFA to access email? & \textcolor{green}{\ding{51}} & Good. Protects a primary communication channel. \\
\addlinespace
Do you require MFA to log into computers? & \textcolor{red}{\ding{55}} & \textbf{High Risk.} Lack of MFA on endpoints allows an attacker with valid credentials to gain full access. \\
\addlinespace
Do you require MFA to access sensitive data systems? & \textcolor{red}{\ding{55}} & \textbf{High Risk.} Critical data is vulnerable to credential theft or brute-force attacks. \\
\addlinespace
Does your organization have an employee acceptable use policy? & \textcolor{red}{\ding{55}} & \textbf{Critical Gap.} Lack of a formal policy creates ambiguity and increases the risk of insider threat. \\
\addlinespace
Does your organization do security awareness training for new employees? & \textcolor{green}{\ding{51}} & Good. Establishes a security baseline for new hires. \\
\addlinespace
Does your organization do security awareness training for all employees at least once per year? & \textcolor{green}{\ding{51}} & Good. Reinforces security concepts annually. \\ \bottomrule
\end{tabular}
\caption{Security Controls Questionnaire Analysis.}
\label{tab:controls}
\end{table}

% --- Section 4: Technical Scan Results ---
\section{Technical Scan Results}
An external network scan was performed on the target IP address to identify open ports and exposed services.
\begin{itemize}
    \item \textbf{Target IP Address:} \texttt{[Target IP]}
\end{itemize}

\begin{table}[h!]
\centering
\begin{tabular}{@{}llll@{}}
\toprule
\textbf{Port} & \textbf{State} & \textbf{Service Name} & \textbf{Analyst Notes} \\ \midrule
3389/tcp & open & ms-wbt-server & \textbf{Critical Risk.} Microsoft Remote Desktop Protocol (RDP). \\
 & & & This service should never be exposed directly to the \\
 & & & internet. It is a common vector for ransomware and \\
 & & & brute-force attacks. \\
\bottomrule
\end{tabular}
\caption{Open Ports Detected on \texttt{[Client IP]}.}
\label{tab:scan_results}
\end{table}

% --- Section 5: Consolidated Risk Assessment ---
\section{Consolidated Risk Assessment}
This section synthesizes findings from the security control review, technical scans, and pre-existing risk data into a consolidated list of identified risks.

\begin{table}[h!]
\centering
\begin{tabular}{@{}p{0.25\textwidth}p{0.15\textwidth}p{0.5\textwidth}@{}}
\toprule
\textbf{Risk Title} & \textbf{Severity} & \textbf{Description} \\ \midrule
\textbf{Publicly Exposed RDP Service} & \textcolor{sev_critical}{\textbf{Critical (9.0)}} & The Nmap scan confirms that RDP is open on \texttt{[Client IP]}. This aligns with the pre-existing risk data. This exposure provides a direct path for attackers into the internal network. \\
\addlinespace
\textbf{Lack of MFA on Endpoints and Systems} & \textcolor{sev_high}{\textbf{High}} & The absence of MFA on computer logins and sensitive systems means that a single compromised password could lead to a full-scale breach. This risk is severely amplified by the exposed RDP service. \\
\addlinespace
\textbf{Missing Acceptable Use Policy (AUP)} & \textcolor{sev_medium}{\textbf{Medium}} & This is a foundational governance failure. Without an AUP, employees lack clear guidelines on protecting company assets, handling data, and using network resources, increasing the likelihood of unintentional security incidents. \\
\bottomrule
\end{tabular}
\caption{Summary of Identified Risks.}
\label{tab:risks}
\end{table}

% --- Section 6: Recommendations ---
\section{Recommendations}
The following prioritized recommendations are provided to address the identified risks and improve the overall security posture of \textbf{[Organization Name]}.

\subsection{Priority 1: Remediate Exposed RDP Service (Immediate)}
This is the most critical vulnerability and must be addressed immediately to prevent a breach.
\begin{enumerate}
    \item \textbf{Short-Term Fix (Emergency Change):} Implement a firewall rule to \textbf{block all inbound traffic to TCP port 3389} on \texttt{[Client IP]} from the internet. Access should only be permitted from trusted, internal IP addresses if absolutely necessary.
    \item \textbf{Long-Term Fix:} Deploy a Virtual Private Network (VPN) solution for all remote access needs. A VPN provides an encrypted tunnel and a single, managed point of entry, which is significantly more secure than exposing services like RDP directly.
\end{enumerate}

\subsection{Priority 2: Implement Multi-Factor Authentication (High)}
MFA is one of the most effective controls for preventing unauthorized access.
\begin{enumerate}
    \item \textbf{Endpoint MFA:} Deploy MFA for all user logins to Windows, macOS, and Linux workstations and servers. Solutions like Duo, Microsoft Authenticator, or other identity providers can facilitate this.
    \item \textbf{Application MFA:} Enforce MFA for access to all systems containing sensitive data, including databases, file shares, and critical business applications.
\end{enumerate}

\subsection{Priority 3: Establish Security Governance (Medium)}
Strengthening administrative controls is essential for long-term security.
\begin{enumerate}
    \item \textbf{Develop an Acceptable Use Policy (AUP):} Create and implement a formal AUP that clearly defines the rules and responsibilities for all employees when using company technology and data. Ensure all employees read and acknowledge the policy.
    \item \textbf{Enhance Security Awareness Training:} While a training program exists, it should be updated to include specific modules on the dangers of exposed services, the importance of strong passwords, and the role of MFA in protecting the organization.
\end{enumerate}

\end{document}
```