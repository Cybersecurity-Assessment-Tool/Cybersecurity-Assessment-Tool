```latex
\documentclass[12pt]{article}

% Preamble: Required Packages
\usepackage[margin=1in]{geometry}
\usepackage{pifont} % For checkmarks and crosses
\usepackage{booktabs} % For professional tables
\usepackage{hyperref} % For hyperlinks
\usepackage{url} % For URL formatting
\usepackage{seqsplit} % To split long strings in tt font
\usepackage{graphicx}
\usepackage{xcolor}

% Hyperref Setup
\hypersetup{
    colorlinks=true,
    linkcolor=blue,
    filecolor=magenta,      
    urlcolor=cyan,
    pdftitle={Cybersecurity Assessment Report},
    pdfpagemode=FullScreen,
}

% Document Start
\begin{document}

% --- Title Page ---
\begin{titlepage}
    \centering
    \vfill
    {\Huge\bfseries Cybersecurity Assessment Report\par}
    \vspace{1.5cm}
    {\Large For: \textbf{[Organization Name]}\par}
    \vspace{2cm}
    {\large Report Date: \today\par}
    \vfill
    {\large Generated by: Cybersecurity Analyst\par}
\end{titlepage}

\tableofcontents
\newpage

% --- 1. Executive Summary ---
\section{Executive Summary}
This report provides a comprehensive analysis of the cybersecurity posture of \textbf{[Organization Name]}, based on a network scan, a security controls questionnaire, and a review of pre-existing risks. The assessment was conducted to identify vulnerabilities, policy gaps, and misconfigurations that could expose the organization to significant cyber threats.

The analysis revealed several critical and high-risk findings. A key vulnerability is the direct network exposure of an outdated and End-of-Life (EOL) MySQL database (version 5.7.33), which presents an immediate and severe risk of data breach. This technical finding is compounded by significant gaps in organizational security controls, most notably the lack of Multi-Factor Authentication (MFA) for computer and sensitive system access. Furthermore, the absence of an employee acceptable use policy and a mandatory annual security training program indicates a need for foundational improvements to the organization's security culture.

Immediate remediation is required to address the exposed database. Strategic initiatives must be launched to implement comprehensive MFA and develop core security policies to mitigate the identified risks effectively.

% --- 2. Organizational Information ---
\section{Organizational Information}
This section details the information provided by the client for this assessment.
\begin{itemize}
    \item \textbf{Organization Name:} \textbf{[Organization Name]}
    \item \textbf{Primary Domain:} \texttt{[Domain]}
    \item \textbf{External IP Scanned:} \texttt{[Client IP]}
\end{itemize}

% --- 3. Security Control Review ---
\section{Security Control Review}
A security questionnaire was completed to evaluate the organization's current administrative and policy-based controls. The responses are summarized below. Items marked with \ding{55} represent significant gaps in the security framework and require immediate attention.

\begin{table}[h!]
\centering
\caption{Security Controls Questionnaire Analysis}
\begin{tabular}{@{}lcc@{}}
\toprule
\textbf{Control Question} & \textbf{Response} & \textbf{Assessment} \\
\midrule
Do you require MFA to access email? & \ding{51} Yes & Good Practice \\
Do you require MFA to log into computers? & \ding{55} No & \textbf{Critical Gap} \\
Do you require MFA to access sensitive data systems? & \ding{55} No & \textbf{Critical Gap} \\
Does your organization have an employee acceptable use policy? & \ding{55} No & \textbf{High Risk} \\
Does your organization do security awareness training for new employees? & \ding{51} Yes & Good Practice \\
Does your organization do security training for all employees annually? & \ding{55} No & \textbf{High Risk} \\
\bottomrule
\end{tabular}
\end{table}

\subsection*{Analysis of Gaps}
\begin{itemize}
    \item \textbf{MFA Gaps:} The absence of MFA on computer logins and sensitive systems drastically increases the risk of unauthorized access from compromised credentials. A single stolen password could lead to a significant breach.
    \item \textbf{Policy and Training Gaps:} Lacking an Acceptable Use Policy (AUP) and annual security training means employees may be unaware of their security responsibilities, making the organization more susceptible to phishing and social engineering attacks.
\end{itemize}

% --- 4. Technical Scan Results ---
\section{Technical Scan Results}
A network scan was performed on the target system to identify open ports and exposed services.

\begin{itemize}
    \item \textbf{Target IP Address:} \texttt{[Target IP]}
    \item \textbf{Scan Utility:} Nmap
\end{itemize}

\begin{table}[h!]
\centering
\caption{Open Ports Detected on Target Host}
\begin{tabular}{@{}lllll@{}}
\toprule
\textbf{Port} & \textbf{State} & \textbf{Service} & \textbf{Product} & \textbf{Version} \\
\midrule
3306/tcp & open & mysql & MySQL & 5.7.33 \\
\bottomrule
\end{tabular}
\end{table}

\subsection*{Technical Analysis}
The scan identified that port \textbf{3306 (MySQL)} is open to the network. This is a critical finding for two primary reasons:
\begin{enumerate}
    \item \textbf{Direct Exposure:} Exposing a database port directly is highly discouraged. It allows attackers to directly attempt brute-force password attacks, exploit vulnerabilities, or perform denial-of-service attacks against the database.
    \item \textbf{Outdated Software:} The detected version, \textbf{MySQL 5.7.33}, is significant. The MySQL 5.7 series reached its official End-of-Life (EOL) in October 2023. This means it no longer receives security updates from the vendor, and any newly discovered vulnerabilities will remain unpatched, leaving the system perpetually at risk.
\end{enumerate}

This technical finding directly correlates with and validates the pre-existing risk documented in "Database Exposure."

% --- 5. Consolidated Risk Assessment ---
\section{Consolidated Risk Assessment}
The following table synthesizes findings from the security control review, technical scan, and pre-existing risk data into a prioritized list.

\begin{table}[h!]
\centering
\caption{Summary of Identified Risks}
\begin{tabular}{@{}p{0.3\linewidth}p{0.5\linewidth}l@{}}
\toprule
\textbf{Risk Name} & \textbf{Description} & \textbf{Severity} \\
\midrule
\textbf{Exposed \& Outdated Database} & The MySQL database (v5.7.33) is exposed on port 3306. The software is End-of-Life and no longer receives security patches. & \textbf{Critical} \\
\addlinespace
\textbf{Insufficient MFA Coverage} & MFA is not enforced for computer logins or access to sensitive data systems, leaving them vulnerable to credential theft. & \textbf{High} \\
\addlinespace
\textbf{Lack of Foundational Policies \& Training} & The absence of an Acceptable Use Policy and mandatory annual security training weakens the human element of security, increasing susceptibility to phishing and misuse. & \textbf{High} \\
\bottomrule
\end{tabular}
\end{table}

% --- 6. Recommendations ---
\section{Recommendations}
The following actions are recommended to mitigate the identified risks. They are prioritized based on severity and potential impact.

\subsection*{Immediate Priority (Critical)}
\begin{enumerate}
    \item \textbf{Restrict Database Access:} Immediately implement strict firewall rules to block all public access to port 3306. Access should be restricted to a whitelist of specific, trusted internal IP addresses only.
    \item \textbf{Upgrade End-of-Life Database:} Plan and execute an urgent migration from MySQL 5.7.33 to a currently supported version (e.g., MySQL 8.x). This is essential to receive security patches for future vulnerabilities.
\end{enumerate}

\subsection*{High Priority}
\begin{enumerate}
    \item \textbf{Implement Comprehensive MFA:} Deploy Multi-Factor Authentication for all employee computer logins and for all systems classified as containing sensitive data. This is a critical defense against credential compromise.
    \item \textbf{Develop and Enforce an Acceptable Use Policy (AUP):} Create a formal AUP that clearly defines the rules and responsibilities for employees when using company technology and data.
    \item \textbf{Establish Annual Security Training:} Institute a mandatory security awareness training program for all employees, to be completed annually. This program should cover topics such as phishing, password security, and data handling.
\end{enumerate}

\subsection*{Long-Term Strategy}
\begin{itemize}
    \item For secure, remote database access, consider implementing a Virtual Private Network (VPN) that requires authorized users to connect to the internal network before accessing resources like the database.
\end{itemize}

% --- 7. Conclusion ---
\section{Conclusion}
The assessment has identified critical vulnerabilities and systemic weaknesses in the current cybersecurity posture of \textbf{[Organization Name]}. The exposed, outdated database presents a clear and present danger that must be addressed immediately. Concurrently, strengthening foundational controls through the enforcement of MFA, policy development, and regular training is crucial for building a resilient and defensible security environment. We strongly advise that the recommendations in this report be actioned without delay to reduce the risk of a significant security incident.

\end{document}
```