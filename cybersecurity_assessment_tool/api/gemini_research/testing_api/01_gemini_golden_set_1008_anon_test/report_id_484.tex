```latex
\documentclass[12pt]{article}

% Preamble: Required Packages
\usepackage[margin=1in]{geometry}
\usepackage{pifont} % For checkmarks and crosses
\usepackage{booktabs} % For professional tables
\usepackage{hyperref} % For clickable links
\usepackage{url} % For formatting URLs
\usepackage{seqsplit} % To split long monospaced text
\usepackage{xcolor} % For colors

% Document Metadata
\title{Cybersecurity Assessment Report}
\author{Cybersecurity Analysis Division}
\date{\today}

% Hyperref Setup
\hypersetup{
    colorlinks=true,
    linkcolor=black,
    urlcolor=blue,
    pdftitle={Cybersecurity Assessment Report},
    pdfauthor={Cybersecurity Analysis Division},
}

\begin{document}

\maketitle
\thispagestyle{empty}
\newpage

\tableofcontents
\thispagestyle{empty}
\newpage

\section{Executive Summary}

This report details the findings of a cybersecurity assessment conducted for \textbf{[Organization Name]}. The evaluation combined a review of organizational security controls, an external network vulnerability scan, and an analysis of pre-existing risks.

The assessment revealed a mixed security posture. On a positive note, the external network scan of the target system, \texttt{[Client IP]}, indicated a strong perimeter defense, with no open ports detected. This suggests a well-configured firewall that effectively minimizes the external attack surface.

However, significant and high-risk gaps were identified in internal administrative and access controls. The two most critical findings are:
\begin{itemize}
    \item \textbf{Critical Risk:} The absence of Multi-Factor Authentication (MFA) for accessing sensitive data systems. This exposes the organization's most valuable data to significant risk of unauthorized access through credential compromise.
    \item \textbf{High Risk:} The lack of a formal Employee Acceptable Use Policy (AUP). This foundational governance document is essential for setting security expectations, managing insider risk, and ensuring regulatory compliance.
\end{itemize}

Immediate action is recommended to address these control gaps to reduce the likelihood of a security incident. Recommendations are detailed in Section \ref{sec:recommendations}.

\section{Organizational Information}

The following information was used as the basis for this assessment. Due to the anonymized nature of the provided data, placeholders are used where necessary.

\begin{itemize}
    \item \textbf{Organization Name:} \textbf{[Organization Name]}
    \item \textbf{Primary Domain:} \texttt{[Domain]}
    \item \textbf{External IP Scanned:} \texttt{[Client IP]}
\end{itemize}

\section{Security Control Review}

A review of administrative and technical controls was conducted via a security questionnaire. The responses indicate a solid foundation in some areas, such as security awareness training and MFA for email and workstations. However, critical gaps were identified, as detailed in the table below.

\begin{table}[h!]
\centering
\caption{Security Controls Questionnaire Analysis}
\label{tab:controls}
\begin{tabular}{p{0.6\linewidth} c p{0.25\linewidth}}
\toprule
\textbf{Control Question} & \textbf{Response} & \textbf{Assessment} \\
\midrule
Do you require MFA to access email? & \ding{51} Yes & Control Implemented \\
Do you require MFA to log into computers? & \ding{51} Yes & Control Implemented \\
Do you require MFA to access sensitive data systems? & \textcolor{red}{\ding{55} No} & \textbf{Critical Control Gap} \\
Does your organization have an employee acceptable use policy? & \textcolor{red}{\ding{55} No} & \textbf{High-Risk Governance Gap} \\
Does your organization do security awareness training for new employees? & \ding{51} Yes & Control Implemented \\
Does your organization do security awareness training for all employees at least once per year? & \ding{51} Yes & Control Implemented \\
\bottomrule
\end{tabular}
\end{table}

\section{Technical Scan Results}

An external network scan was performed to identify open ports and services exposed to the internet.

\begin{itemize}
    \item \textbf{Target IP:} \texttt{[Target IP]}
    \item \textbf{Scan Date:} \today
    \item \textbf{Scanner Used:} Nmap
\end{itemize}

\subsection{Summary of Findings}
The scan confirmed that the target host was online and responsive (\texttt{status: up}). However, the scan found \textbf{no open TCP or UDP ports}. All 65,535 ports were in a \texttt{closed} state.

\subsection{Analysis}
This is a positive security finding. A host with no externally accessible ports presents a minimal attack surface to external adversaries. This result strongly suggests the presence of a properly configured stateful firewall that is correctly implementing a default-deny policy for all inbound traffic. This is a significant strength in the organization's perimeter defense strategy.

\section{Overall Risk Assessment}

This section correlates the findings from the security control review, technical scan, and pre-existing risk data. While the technical posture is strong, the administrative control gaps present a clear and immediate danger to the organization.

\begin{table}[h!]
\centering
\caption{Identified Risks and Severity}
\label{tab:risks}
\begin{tabular}{p{0.3\linewidth} p{0.5\linewidth} l}
\toprule
\textbf{Risk Name} & \textbf{Description} & \textbf{Severity} \\
\midrule
\textbf{Lack of MFA on Sensitive Systems} & Failure to implement MFA on systems containing sensitive data leaves them highly vulnerable to unauthorized access via compromised credentials (e.g., phishing, password spraying). & \textbf{Critical} \\
\addlinespace
\textbf{No Employee Acceptable Use Policy (AUP)} & The absence of a formal AUP creates ambiguity regarding technology usage, increases the risk of unintentional data exposure or malicious insider activity, and poses compliance challenges. & \textbf{High} \\
\addlinespace
\textbf{No Known Pre-Existing Vulnerabilities} & The provided data indicated no known or tracked vulnerabilities at the time of this assessment. & Informational \\
\bottomrule
\end{tabular}
\end{table}

\section{Recommendations}
\label{sec:recommendations}

The following actions are recommended to mitigate the identified risks and improve the overall security posture of \textbf{[Organization Name]}.

\subsection{Recommendation 1: Implement MFA for Sensitive Systems (Critical)}
\begin{itemize}
    \item \textbf{Action:} Immediately prioritize the evaluation and deployment of a robust MFA solution for all user and administrative access to systems identified as containing or processing sensitive data. This includes databases, financial applications, and critical infrastructure management interfaces.
    \item \textbf{Justification:} MFA is one of the most effective controls for preventing unauthorized access. Even if an attacker steals valid user credentials, they will be unable to access the protected system without the second factor. This single change drastically reduces the risk of a data breach.
\end{itemize}

\subsection{Recommendation 2: Develop and Implement an Acceptable Use Policy (High)}
\begin{itemize}
    \item \textbf{Action:} Draft, approve, and disseminate a formal AUP that clearly defines the rules and expectations for employee use of company technology, data, and network resources. This policy should be acknowledged in writing by all employees during onboarding and annually thereafter.
    \item \textbf{Justification:} An AUP is a foundational cybersecurity governance document. It establishes a baseline for secure behavior, provides a legal framework for enforcing security rules, reduces organizational liability, and is often a requirement for regulatory compliance.
\end{itemize}

\end{document}
```