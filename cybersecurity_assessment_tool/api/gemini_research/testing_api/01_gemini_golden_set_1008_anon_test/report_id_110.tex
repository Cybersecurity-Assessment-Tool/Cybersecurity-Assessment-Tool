```latex
\documentclass[12pt]{article}

% --- PACKAGE IMPORTS ---
\usepackage[margin=1in]{geometry}
\usepackage{pifont} % For checkmarks and crosses
\usepackage{booktabs} % For professional tables
\usepackage{hyperref} % For clickable links
\usepackage{url} % For URL formatting
\usepackage{seqsplit} % For splitting long strings
\usepackage{graphicx}
\usepackage{xcolor}

% --- DOCUMENT METADATA ---
\title{Cybersecurity Posture Assessment Report}
\author{Cybersecurity Analysis Division}
\date{\today}

% --- HYPERREF SETUP ---
\hypersetup{
    colorlinks=true,
    linkcolor=blue,
    filecolor=magenta,      
    urlcolor=cyan,
    pdftitle={Cybersecurity Posture Assessment Report},
    pdfpagemode=FullScreen,
}

% --- BEGIN DOCUMENT ---
\begin{document}

\maketitle

\begin{abstract}
    This report provides a comprehensive analysis of the cybersecurity posture for \textbf{[Organization Name]}. The assessment is based on a synthesis of external network scan data, a review of internal security controls via a questionnaire, and an evaluation of previously identified risks. The analysis reveals several critical- and high-severity vulnerabilities that require immediate attention. Key findings include an exposed and dangerously outdated FTP server, a critical lack of Multi-Factor Authentication (MFA) on essential systems, and significant gaps in foundational security policies and training.
\end{abstract}

\newpage

\section*{1. Overview and Scope}
This assessment was conducted to evaluate the external security posture and internal security controls of \textbf{[Organization Name]}. The scope of this report includes:
\begin{itemize}
    \item \textbf{Organizational Information Review:} Analysis of client-provided data.
    \item \textbf{Security Control Analysis:} Evaluation of responses to a security best-practices questionnaire.
    \item \textbf{Technical Network Scan:} An external vulnerability scan of the public-facing IP address.
    \item \textbf{Risk Correlation:} Synthesis of all data points to provide a holistic risk profile and actionable recommendations.
\end{itemize}

\section*{2. Organizational Information}
The following information was used as the basis for this assessment. Due to the anonymized nature of the input data, placeholders have been used where necessary.

\begin{tabular}{@{}ll}
    \toprule
    \textbf{Attribute} & \textbf{Value} \\
    \midrule
    Organization Name & \textbf{[Organization Name]} \\
    Primary Domain & \texttt{[Domain]} \\
    External IP Address Scanned & \texttt{[Client IP]} \\
    Target IP from Scan Data & \texttt{[Target IP]} \\
    \bottomrule
\end{tabular}

\section*{3. Security Control Review}
The following table summarizes the organization's self-reported security controls. "No" answers indicate significant gaps in the security framework and are flagged as high or critical risks.

\begin{table}[h!]
\centering
\caption{Security Controls Questionnaire Analysis}
\begin{tabular}{@{}p{0.6\linewidth} c p{0.2\linewidth}@{}}
    \toprule
    \textbf{Control Question} & \textbf{Response} & \textbf{Analyst Note} \\
    \midrule
    Do you require MFA to access email? & \ding{51} & Best practice met. \\
    \addlinespace
    Do you require MFA to log into computers? & \textbf{\color{red}\ding{55}} & \textbf{Critical Gap.} Lack of MFA on endpoints allows for easier lateral movement after a credential compromise. \\
    \addlinespace
    Do you require MFA to access sensitive data systems? & \textbf{\color{red}\ding{55}} & \textbf{Critical Gap.} The organization's most valuable data is not protected by a fundamental security layer. \\
    \addlinespace
    Does your organization have an employee acceptable use policy? & \textbf{\color{red}\ding{55}} & \textbf{High Risk.} Lack of a formal policy creates ambiguity and legal/operational risk regarding employee system usage. \\
    \addlinespace
    Does your organization do security awareness training for new employees? & \ding{51} & Good baseline practice. \\
    \addlinespace
    Does your organization do security awareness training for all employees at least once per year? & \textbf{\color{red}\ding{55}} & \textbf{High Risk.} Security is a moving target; without recurring training, employee knowledge becomes outdated, increasing susceptibility to phishing and social engineering. \\
    \bottomrule
\end{tabular}
\end{table}

\section*{4. Technical Scan Results}
An external network scan was performed against the target IP address \texttt{[Target IP]}. The scan identified one open port with a critical vulnerability.

\begin{table}[h!]
\centering
\caption{Open Port Analysis}
\begin{tabular}{@{}lllll@{}}
    \toprule
    \textbf{Port} & \textbf{State} & \textbf{Service} & \textbf{Version} & \textbf{Finding} \\
    \midrule
    21/tcp & Open & FTP & vsftpd 2.3.4 & \textbf{Anonymous Login Allowed} \\
    \bottomrule
\end{tabular}
\end{table}

\subsection*{Analysis of Findings}
The presence of an open FTP port is a concern, but the specific version and configuration are \textbf{critical risks}:
\begin{itemize}
    \item \textbf{Vulnerable Version:} \texttt{vsftpd 2.3.4} is a notoriously vulnerable version released in 2011. It is susceptible to a critical backdoor vulnerability (CVE-2011-2523) that allows an attacker to execute arbitrary commands on the server with root privileges.
    \item \textbf{Anonymous Login:} The server is configured to allow anonymous FTP login. This permits any user on the internet to connect, view, upload, or download files without authentication. This configuration is extremely dangerous and is often exploited to store malicious files or exfiltrate data.
\end{itemize}

\section*{5. Consolidated Risk Assessment}
The following table consolidates risks identified from the security control review, technical scan, and pre-existing risk data.

\begin{table}[h!]
\centering
\caption{Summary of Identified Risks}
\begin{tabular}{@{}p{0.15\linewidth} p{0.65\linewidth} p{0.1\linewidth}@{}}
    \toprule
    \textbf{Risk Title} & \textbf{Description} & \textbf{Severity} \\
    \midrule
    \textbf{Exposed Vulnerable FTP Server} & An internet-facing server is running vsftpd 2.3.4, which contains a known remote code execution backdoor. & \textbf{Critical} \\
    \addlinespace
    \textbf{Lack of MFA on Endpoints and Systems} & Workstations and sensitive data systems are accessible with only a password, exposing them to credential stuffing and phishing attacks. & \textbf{Critical} \\
    \addlinespace
    \textbf{Anonymous FTP Access} & The FTP server allows unauthenticated access, creating a high risk of data leakage or the hosting of malicious content. & \textbf{High} \\
    \addlinespace
    \textbf{Inadequate Security Policies and Training} & The absence of an Acceptable Use Policy and annual security training increases the likelihood of human error leading to a security incident. & \textbf{High} \\
    \addlinespace
    \textbf{Outdated Windows Policy} & (Pre-existing risk) Workstations are running an unsupported OS (Windows 7), which no longer receives security updates. & \textbf{Medium} \\
    \bottomrule
\end{tabular}
\end{table}

\section*{6. Recommendations}
Based on the analysis, we recommend the following actions, prioritized by severity.

\subsection*{Priority 1: Immediate Actions (Within 24-48 Hours)}
\begin{enumerate}
    \item \textbf{Remediate the FTP Server:}
        \begin{itemize}
            \item \textbf{Option A (Preferred):} Decommission the FTP server immediately. If FTP is not essential for business operations, the service should be disabled and the port closed at the firewall.
            \item \textbf{Option B (If Required):} If FTP is business-critical, immediately upgrade \texttt{vsftpd} to the latest stable version and disable anonymous login. Access should be restricted to known IP addresses at the firewall level.
        \end{itemize}
\end{enumerate}

\subsection*{Priority 2: High-Impact Actions (Within 30 Days)}
\begin{enumerate}
    \setcounter{enumi}{1}
    \item \textbf{Implement Multi-Factor Authentication (MFA):}
        \begin{itemize}
            \item Deploy MFA for all user logins to company workstations.
            \item Enforce MFA for access to all systems containing sensitive or critical data.
        \end{itemize}
    \item \textbf{Develop Foundational Policies:}
        \begin{itemize}
            \item Draft and implement a formal Acceptable Use Policy (AUP) that all employees must read and sign.
        \end{itemize}
\end{enumerate}

\subsection*{Priority 3: Foundational Improvements (Within 90 Days)}
\begin{enumerate}
    \setcounter{enumi}{3}
    \item \textbf{Establish Recurring Security Training:}
        \begin{itemize}
            \item Implement a mandatory annual security awareness training program for all employees. This should cover topics like phishing, password hygiene, and social engineering.
        \end{itemize}
    \item \textbf{Address OS Lifecycle:}
        \begin{itemize}
            \item Act on the pre-existing risk by accelerating the plan to upgrade or replace all workstations running Windows 7.
        \end{itemize}
\end{enumerate}

\end{document}
```