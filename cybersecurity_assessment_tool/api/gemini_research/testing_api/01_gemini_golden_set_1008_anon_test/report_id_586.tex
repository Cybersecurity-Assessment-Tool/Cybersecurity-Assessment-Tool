```latex
\documentclass[12pt]{article}

% --- PACKAGES ---
\usepackage[margin=1in]{geometry}
\usepackage{pifont} % For checkmarks and crosses
\usepackage{booktabs} % For professional tables
\usepackage{hyperref} % For clickable links and ToC
\usepackage{url} % For URL formatting
\usepackage{seqsplit} % To split long strings in texttt
\usepackage{xcolor} % For colors
\usepackage{graphicx} % For logo (placeholder)

% --- DOCUMENT SETUP ---
\hypersetup{
    colorlinks=true,
    linkcolor=blue,
    filecolor=magenta,      
    urlcolor=cyan,
    pdftitle={Cybersecurity Posture Report},
    pdfpagemode=FullScreen,
}

% --- CUSTOM COMMANDS & COLORS ---
\newcommand{\yes}{\ding{51}}
\newcommand{\no}{\ding{55}}
\definecolor{critical}{HTML}{990000}
\definecolor{high}{HTML}{D14302}
\definecolor{medium}{HTML}{E5A50A}
\definecolor{low}{HTML}{3E8E41}

\begin{document}

% --- TITLE PAGE ---
\begin{titlepage}
    \centering
    \vspace*{1cm}
    
    \Huge
    \textbf{Cybersecurity Posture Report}
    
    \vspace{1.5cm}
    
    \Large
    Prepared for: \\
    \vspace{0.5cm}
    \textbf{[Organization Name]}
    
    \vspace{2cm}
    
    \normalsize
    Report Date: \today \\
    Analysis Period: \today
    
    \vfill
    
    \large
    \textbf{Generated by} \\
    Cybersecurity Analyst
    
\end{titlepage}

\tableofcontents
\newpage

% --- EXECUTIVE SUMMARY ---
\section{Executive Summary}
This report provides a comprehensive analysis of the cybersecurity posture for \textbf{[Organization Name]}. The assessment is based on a correlation of organizational data, a review of security controls via a questionnaire, and an external network scan.

The analysis revealed several critical and high-risk gaps in the organization's security controls, primarily related to policy and identity management. Specifically, Multi-Factor Authentication (MFA) is not enforced for email or access to sensitive data systems, leaving critical assets vulnerable to compromise. Furthermore, the absence of a formal employee Acceptable Use Policy and a lack of security awareness training for new hires create significant organizational risk.

On a technical level, the external network scan of the target IP address \texttt{[Target IP]} did not identify any open ports or exposed services. While this indicates a strong network perimeter defense or that the host was not responsive at the time of the scan, it does not mitigate the severe internal and policy-related risks identified.

Immediate action is required to address the identified control gaps to reduce the risk of unauthorized access, data breaches, and other security incidents.

% --- ORGANIZATIONAL INFORMATION ---
\section{Organizational Information}
This section details the information provided by the client organization. Based on the data supplied, several key identifiers were not available and are represented by placeholders.

\begin{tabular}{@{}ll}
    \toprule
    \textbf{Attribute} & \textbf{Value} \\
    \midrule
    Organization Name & \textbf{[Organization Name]} \\
    Email Domain & \texttt{[Domain]} \\
    External IP Address & \texttt{[Client IP]} \\
    \bottomrule
\end{tabular}

% --- SECURITY CONTROL REVIEW ---
\section{Security Control Review}
The following table summarizes the organization's responses to a security controls questionnaire. Answers marked with \no{} indicate a potential gap in security posture and are discussed in the Risk Assessment section.

\begin{table}[h!]
\centering
\begin{tabular}{@{}p{0.6\textwidth}ccp{0.2\textwidth}@{}}
    \toprule
    \textbf{Control Question} & \textbf{Response} & \textbf{Assessment} \\
    \midrule
    Do you require MFA to access email? & \no & \textcolor{critical}{\textbf{Critical Gap}} \\
    Do you require MFA to log into computers? & \yes & Met \\
    Do you require MFA to access sensitive data systems? & \no & \textcolor{critical}{\textbf{Critical Gap}} \\
    Does your organization have an employee acceptable use policy? & \no & \textcolor{high}{\textbf{High Risk}} \\
    Does your organization do security awareness training for new employees? & \no & \textcolor{high}{\textbf{High Risk}} \\
    Does your organization do security awareness training for all employees at least once per year? & \yes & Met \\
    \bottomrule
\end{tabular}
\caption{Security Controls Questionnaire Analysis}
\end{table}

% --- TECHNICAL SCAN RESULTS ---
\section{Technical Scan Results}
An external network vulnerability scan was conducted to identify exposed services and potential vulnerabilities on the organization's perimeter.

\begin{itemize}
    \item \textbf{Target IP Address:} \texttt{[Target IP]}
    \item \textbf{Scan Date:} \today
    \item \textbf{Summary:} The scan completed successfully but did not discover any open TCP or UDP ports on the target host.
\end{itemize}

\textbf{Analysis:} No externally facing services were identified. This suggests that the network perimeter is well-secured by a firewall that drops or rejects unsolicited traffic. While this is a positive finding from an external threat perspective, it is crucial to remember that many security incidents originate from internal threats or phishing attacks that bypass perimeter defenses.

% --- RISK ASSESSMENT ---
\section{Risk Assessment}
This section synthesizes findings from the security control review and technical scan. The primary risks identified are procedural and policy-based, stemming from the questionnaire responses. No pre-existing vulnerabilities were provided for this assessment.

\begin{table}[h!]
\centering
\begin{tabular}{@{}p{0.25\textwidth}p{0.5\textwidth}p{0.15\textwidth}@{}}
    \toprule
    \textbf{Risk Name} & \textbf{Overview} & \textbf{Severity} \\
    \midrule
    \textbf{No MFA on Email} & The lack of MFA on email accounts significantly increases the risk of business email compromise (BEC), phishing success, and unauthorized access to sensitive communications. & \textcolor{critical}{\textbf{Critical}} \\
    \addlinespace
    \textbf{No MFA on Sensitive Data Systems} & Critical data systems are protected only by username and password. A single credential leak could lead to a major data breach. & \textcolor{critical}{\textbf{Critical}} \\
    \addlinespace
    \textbf{Missing Acceptable Use Policy (AUP)} & Without a formal AUP, employees may be unaware of security responsibilities and acceptable behavior, increasing the likelihood of insider threats and unintentional policy violations. & \textcolor{high}{\textbf{High}} \\
    \addlinespace
    \textbf{No New Hire Security Training} & New employees are not provided with security awareness training during onboarding. This makes them highly susceptible to social engineering and phishing attacks from their first day. & \textcolor{high}{\textbf{High}} \\
    \bottomrule
\end{tabular}
\caption{Identified Risks and Severity}
\end{table}

% --- RECOMMENDATIONS ---
\section{Recommendations}
The following actions are recommended to mitigate the identified risks and improve the overall security posture of \textbf{[Organization Name]}.

\begin{description}
    \item[\textcolor{critical}{Critical Priority:}]
    \begin{enumerate}
        \item \textbf{Implement MFA for Email:} Immediately enforce MFA for all user access to the email system (e.g., Microsoft 365, Google Workspace). This is the single most effective control to prevent account takeovers.
        \item \textbf{Implement MFA for Sensitive Systems:} Identify all systems containing sensitive, confidential, or regulated data and enforce MFA for all user and administrative access.
    \end{enumerate}
    
    \item[\textcolor{high}{High Priority:}]
    \begin{enumerate}
        \item \textbf{Develop and Implement an AUP:} Create a formal Acceptable Use Policy that clearly defines rules for using company assets, data handling, and internet usage. Ensure all employees read and acknowledge the policy.
        \item \textbf{Integrate Security Training into Onboarding:} Develop a mandatory security awareness training module for all new hires. This training should cover key topics such as phishing, password security, and the new AUP.
    \end{enumerate}
\end{description}

% --- CONCLUSION ---
\section{Conclusion}
The current security posture of \textbf{[Organization Name]} presents several significant, addressable risks. While the external network perimeter appears secure, critical deficiencies in identity and access management and security governance create vulnerabilities that must be remediated promptly. 

By implementing the recommendations outlined in this report, particularly the enforcement of Multi-Factor Authentication and the establishment of foundational security policies, the organization can substantially reduce its risk exposure and build a more resilient security foundation.

\end{document}
```