```latex
\documentclass[12pt]{article}

% Preamble: Required Packages
\usepackage[margin=1in]{geometry}
\usepackage{pifont} % For checkmarks and crosses
\usepackage{booktabs} % For professional tables
\usepackage{graphicx} % For potential logos
\usepackage[hidelinks]{hyperref} % For clickable links without boxes
\usepackage{url} % For URL formatting
\usepackage{seqsplit} % For splitting long strings in texttt
\usepackage{xcolor} % For color definitions

% Define colors for severity
\definecolor{sevhigh}{HTML}{D9534F}
\definecolor{sevmedium}{HTML}{F0AD4E}

% Document Information
\title{Cybersecurity Posture Assessment Report \\ \large For \textbf{[Organization Name]}}
\author{Cybersecurity Analysis Division}
\date{\today}

\begin{document}

\maketitle
\thispagestyle{empty}
\newpage

\tableofcontents
\newpage

% --- 1. Executive Summary ---
\section*{1. Executive Summary}

This report provides a cybersecurity posture assessment for \textbf{[Organization Name]}, based on an analysis of organizational security controls, an external network scan, and a review of known risks. The assessment was conducted on \today.

Overall, the organization demonstrates a strong commitment to identity and access management through the consistent enforcement of Multi-Factor Authentication (MFA) across key systems. This significantly reduces the risk of unauthorized access via compromised credentials.

However, two critical areas of concern were identified. Firstly, there is a complete absence of a formal security awareness training program for both new and existing employees. This represents a \textbf{High} risk, as it leaves the organization vulnerable to phishing, social engineering, and other human-centric attacks. Secondly, an externally facing Secure Shell (SSH) service was detected on the network perimeter. While necessary for remote administration, an improperly configured SSH service is a common target for attackers.

This report details these findings and provides actionable recommendations to mitigate the identified risks and strengthen the organization's overall security posture.

% --- 2. Organizational Information ---
\section*{2. Organizational Information}

The following information was used as the basis for this assessment. Due to the anonymized nature of the provided data, placeholders have been used where necessary.

\begin{table}[h!]
\centering
\begin{tabular}{@{}ll@{}}
\toprule
\textbf{Attribute} & \textbf{Value} \\ \midrule
Organization Name & \textbf{[Organization Name]} \\
Email Domain & \texttt{[Domain]} \\
External IP Address & \texttt{[Client IP]} \\ \bottomrule
\end{tabular}
\caption{Client Organizational Details}
\end{table}

% --- 3. Security Control Review ---
\section*{3. Security Control Review}

A review of foundational security controls was conducted via a questionnaire. The responses indicate a mature approach to access control but highlight significant deficiencies in security awareness. The table below summarizes the findings. "No" answers are considered security gaps that require immediate attention.

\begin{table}[h!]
\centering
\begin{tabular}{@{}p{0.7\linewidth}c@{}}
\toprule
\textbf{Control Question} & \textbf{Response} \\ \midrule
Do you require MFA to access email? & \ding{51} \\
Do you require MFA to log into computers? & \ding{51} \\
Do you require MFA to access sensitive data systems? & \ding{51} \\
Does your organization have an employee acceptable use policy? & \ding{51} \\
Does your organization do security awareness training for new employees? & {\color{red}\ding{55}} \\
Does your organization do security awareness training for all employees at least once per year? & {\color{red}\ding{55}} \\ \bottomrule
\end{tabular}
\caption{Security Controls Questionnaire Results}
\end{table}

\subsection*{Analysis of Control Gaps}
The lack of a security awareness training program is a critical vulnerability. Without formal training, employees are significantly more likely to fall victim to phishing attacks, mishandle sensitive data, or unknowingly violate security policies. This gap undermines the effectiveness of technical controls and exposes the organization to a wide range of threats.

% --- 4. Technical Scan Results ---
\section*{4. Technical Scan Results}

An external network scan was performed against the public-facing IP address \texttt{[Client IP]} to identify exposed services. The target for this scan was specified as \texttt{[Target IP]}.

\subsection*{Open Ports}
The scan identified the following open port accessible from the public internet:

\begin{table}[h!]
\centering
\begin{tabular}{@{}llll@{}}
\toprule
\textbf{Port} & \textbf{State} & \textbf{Service (Inferred)} & \textbf{Notes} \\ \midrule
22/tcp & open & SSH & Secure Shell for remote administration. \\ \bottomrule
\end{tabular}
\caption{Externally Exposed Services}
\end{table}

\subsection*{Technical Analysis}
The Secure Shell (SSH) service on port 22 is commonly used for legitimate remote system administration. However, its exposure to the internet makes it a prime target for automated brute-force attacks and exploitation of potential software vulnerabilities. The scan did not retrieve version information, so it is not possible to determine if the running SSH server is outdated. Secure configuration is paramount for any internet-facing administrative service.

% --- 5. Risk Assessment Summary ---
\section*{5. Risk Assessment Summary}

The following table synthesizes findings from the security control review and the technical scan into a prioritized list of risks. No pre-existing vulnerabilities were reported.

\begin{table}[h!]
\centering
\begin{tabular}{@{}p{0.1\linewidth}p{0.6\linewidth}p{0.15\linewidth}@{}}
\toprule
\textbf{ID} & \textbf{Risk Description} & \textbf{Severity} \\ \midrule
RISK-001 & \textbf{Lack of Security Awareness Training:} Employees are not trained to identify or respond to common cyber threats like phishing and social engineering. This significantly increases the likelihood of a security breach originating from human error. & \colorbox{sevhigh}{\color{white}\textbf{High}} \\
\addlinespace
RISK-002 & \textbf{Externally Exposed SSH Service:} The SSH service on \texttt{[Client IP]} is accessible from the public internet. If misconfigured (e.g., allows password authentication, uses weak ciphers, or is an outdated version), it could be compromised by attackers. & \colorbox{sevmedium}{\color{white}\textbf{Medium}} \\ \bottomrule
\end{tabular}
\caption{Identified Risks and Severity}
\end{table}

% --- 6. Recommendations ---
\section*{6. Recommendations}

The following actionable recommendations are provided to mitigate the identified risks and improve the overall security posture of \textbf{[Organization Name]}.

\subsection*{RISK-001: Lack of Security Awareness Training (High)}
\begin{itemize}
    \item \textbf{Immediate Action:} Implement a mandatory security awareness training program for all new employees as part of the onboarding process.
    \item \textbf{Ongoing Action:} Establish an annual, mandatory security awareness refresher course for all staff. This program should cover, at a minimum: phishing identification, password security, safe browsing habits, and data handling policies.
    \item \textbf{Enhancement:} Conduct periodic, simulated phishing campaigns to test employee awareness and reinforce training concepts.
\end{itemize}

\subsection*{RISK-002: Externally Exposed SSH Service (Medium)}
\begin{itemize}
    \item \textbf{Audit:} Immediately review the configuration of the SSH service running on \texttt{[Client IP]}.
    \item \textbf{Harden Configuration:}
        \begin{itemize}
            \item Disable password-based authentication and enforce the use of public key cryptography.
            \item Disable root user login over SSH.
            \item Ensure the SSH server software is patched and up-to-date.
            \item Use strong, modern cryptographic ciphers and algorithms.
        \end{itemize}
    \item \textbf{Restrict Access:} If feasible, restrict SSH access to a whitelist of trusted IP addresses (e.g., office or administrator home IPs) using a firewall.
\end{itemize}

\end{document}
```