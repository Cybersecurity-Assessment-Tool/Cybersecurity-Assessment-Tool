```latex
\documentclass[12pt]{article}

% Preamble: Required Packages
\usepackage[margin=1in]{geometry}
\usepackage{pifont} % For checkmarks and crosses (\ding)
\usepackage{booktabs} % For professional-looking tables
\usepackage{hyperref} % For clickable links
\usepackage{url}      % For URL formatting
\usepackage{seqsplit} % To split long strings in tt font
\usepackage{xcolor}   % For custom colors
\usepackage{graphicx} % For potential logos (placeholder)

% --- Document Setup ---
\hypersetup{
    colorlinks=true,
    linkcolor=blue,
    filecolor=magenta,
    urlcolor=cyan,
    pdftitle={Cybersecurity Posture Assessment Report},
    pdfauthor={Cybersecurity Analysis Division},
}

% --- Custom Commands ---
\newcommand{\yes}{\textcolor{green!60!black}{\ding{51}}} % Green checkmark
\newcommand{\no}{\textcolor{red!80!black}{\ding{55}}}   % Red X

% --- Document Title ---
\title{
    \vspace{-1cm}
    \rule{\textwidth}{1pt}\par
    \textbf{Cybersecurity Posture Assessment Report} \\
    \large for \\
    \textbf{[Organization Name]}
    \rule{\textwidth}{1pt}\par
    \vspace{0.5cm}
}
\author{Cybersecurity Analysis Division}
\date{November 22, 2025}

% --- Begin Document ---
\begin{document}

\maketitle
\thispagestyle{empty}
\newpage

\tableofcontents
\newpage

% ==============================================================================
\section*{1. Executive Overview}
% ==============================================================================

This report details the findings of a cybersecurity posture assessment conducted for \textbf{[Organization Name]} on November 22, 2025. The assessment combined an external network scan, a review of organizational security controls via a questionnaire, and an analysis of pre-existing risks.

The organization demonstrates a strong foundation in identity and access management, with commendable implementation of Multi-Factor Authentication (MFA) across email, computer logins, and sensitive data systems. An acceptable use policy and security training for new hires are also in place, which are positive indicators of a security-conscious culture.

However, two significant risks were identified that require immediate attention:

\begin{enumerate}
    \item \textbf{Critical Risk - Outdated Web Server Software:} An external scan identified an Nginx web server running version 1.18.0. This version is several years old, no longer receives security updates, and is known to be vulnerable to multiple high-severity exploits. This exposes the organization to potential data breaches, denial-of-service attacks, and system compromise.

    \item \textbf{High Risk - Security Training Gap:} The organization does not provide mandatory annual security awareness training for all employees. This creates a significant gap in the human firewall, as evolving threats like sophisticated phishing and social engineering tactics may go unrecognized, increasing the likelihood of a security incident.
\end{enumerate}

No pre-existing vulnerabilities were reported for this assessment. The overall security posture is considered \textbf{Moderate Risk}. While foundational controls are strong, the critical-risk technical vulnerability and the high-risk administrative gap present clear and present dangers to the organization's security. This report provides actionable recommendations to mitigate these risks effectively.

% ==============================================================================
\section*{2. Organizational Information}
% ==============================================================================

The following information was used as the basis for this assessment. As identity data was not provided, placeholders have been used.

\begin{itemize}
    \item \textbf{Organization Name:} \textbf{[Organization Name]}
    \item \textbf{Primary Email Domain:} \texttt{[Domain]}
    \item \textbf{External IP Address Scanned:} \texttt{[Client IP]}
\end{itemize}

% ==============================================================================
\section*{3. Security Control Review (Questionnaire)}
% ==============================================================================

The following table summarizes the organization's responses to the security controls questionnaire. A \yes\ indicates a positive control is in place, while a \no\ indicates a potential security gap.

\begin{table}[h!]
\centering
\caption{Security Controls Questionnaire Results}
\label{tab:controls}
\begin{tabular}{p{0.75\textwidth} c}
\toprule
\textbf{Control Question} & \textbf{Response} \\
\midrule
Do you require MFA to access email? & \yes \\
Do you require MFA to log into computers? & \yes \\
Do you require MFA to access sensitive data systems? & \yes \\
Does your organization have an employee acceptable use policy? & \yes \\
Does your organization do security awareness training for new employees? & \yes \\
Does your organization do security awareness training for all employees at least once per year? & \no \\
\bottomrule
\end{tabular}
\end{table}

\subsection*{Analysis}
The questionnaire reveals a significant gap in the organization's security training program. While new employees receive initial training, the absence of a mandatory annual refresher course for \textit{all} staff is a high-risk oversight. The threat landscape evolves continuously, and without ongoing education, employees become more susceptible to modern phishing, ransomware, and social engineering attacks. This gap undermines other technical controls and increases the overall risk profile.

% ==============================================================================
\section*{4. Technical Scan Results}
% ==============================================================================

An external network scan was performed against the target IP address to identify open ports and exposed services.

\begin{itemize}
    \item \textbf{Scan Target:} \texttt{[Target IP]}
    \item \textbf{Scan Date:} 2025-11-22
\end{itemize}

\begin{table}[h!]
\centering
\caption{Open Ports and Services Detected}
\label{tab:nmap}
\begin{tabular}{l l l l l}
\toprule
\textbf{Port} & \textbf{State} & \textbf{Service} & \textbf{Product} & \textbf{Version} \\
\midrule
443/tcp & open & https & nginx & 1.18.0 \\
\bottomrule
\end{tabular}
\end{table}

\subsection*{Analysis}
The scan identified a single open port, 443 (HTTPS), running an Nginx web server. The detected version, \textbf{1.18.0}, is critically outdated. This version was released in April 2020 and has since been superseded by numerous stable and mainline releases.

Running outdated software, especially a public-facing web server, is a severe security risk. Nginx 1.18.0 is known to be vulnerable to multiple Common Vulnerabilities and Exposures (CVEs), including those that could lead to request smuggling, denial of service, or information disclosure. Attackers actively scan the internet for such unpatched systems, making this a high-priority target for remediation.

% ==============================================================================
\section*{5. Consolidated Risk Assessment}
% ==============================================================================

The following table consolidates the findings from the security control review and the technical scan into a prioritized list of risks.

\begin{table}[h!]
\centering
\caption{Summary of Identified Risks}
\label{tab:risks}
\begin{tabular}{l p{0.5\textwidth} l l}
\toprule
\textbf{ID} & \textbf{Risk Description} & \textbf{Source} & \textbf{Severity} \\
\midrule
RISK-001 & Outdated and vulnerable Nginx web server (v1.18.0) exposed to the internet. & Network Scan & \textbf{\textcolor{red!80!black}{Critical}} \\
\addlinespace
RISK-002 & Lack of mandatory annual security awareness training for all employees. & Questionnaire & \textbf{\textcolor{orange!90!black}{High}} \\
\bottomrule
\end{tabular}
\end{table}

% ==============================================================================
\section*{6. Recommendations}
% ==============================================================================

To address the identified risks and improve the overall security posture of \textbf{[Organization Name]}, the following actions are recommended.

\begin{enumerate}
    \item \textbf{Upgrade Nginx Web Server (RISK-001 - Critical):}
    \begin{itemize}
        \item \textbf{Action:} Immediately plan and execute an upgrade of the Nginx server on host \texttt{[Target IP]} from version 1.18.0 to the latest stable version.
        \item \textbf{Justification:} Upgrading will patch numerous known vulnerabilities, protecting the server and its data from compromise. This is the most urgent technical remediation required.
        \item \textbf{Follow-up:} Implement a patch management policy to ensure all public-facing software is reviewed and updated on a regular basis (e.g., quarterly).
    \end{itemize}
    \vspace{0.5cm}
    \item \textbf{Implement Annual Security Training (RISK-002 - High):}
    \begin{itemize}
        \item \textbf{Action:} Develop and implement a mandatory annual security awareness training program for all employees, contractors, and executives.
        \item \textbf{Justification:} Regular training reinforces security best practices and educates staff on new and evolving threats, strengthening the organization's "human firewall" and reducing the risk of security incidents caused by human error.
        \item \textbf{Follow-up:} Track completion rates to ensure 100\% compliance and consider periodic phishing simulations to test and validate the effectiveness of the training.
    \end{itemize}
\end{enumerate}

\end{document}
```