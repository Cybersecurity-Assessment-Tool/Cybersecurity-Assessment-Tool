```latex
\documentclass[12pt, a4paper]{article}

% --- PACKAGE IMPORTS ---
\usepackage[margin=1in]{geometry} % Set page margins
\usepackage{pifont}                 % For checkmarks and crosses (\ding)
\usepackage{booktabs}               % For professional-looking tables
\usepackage{hyperref}               % For hyperlinks and metadata
\usepackage{url}                    % For typesetting URLs
\usepackage{seqsplit}               % To split long strings without spaces
\usepackage{graphicx}               % For including logos, etc.
\usepackage[table]{xcolor}          % For coloring table cells

% --- DOCUMENT METADATA ---
\hypersetup{
    colorlinks=true,
    linkcolor=blue,
    filecolor=magenta,      
    urlcolor=cyan,
    pdftitle={Cybersecurity Assessment Report},
    pdfauthor={Cybersecurity Analyst},
    pdfsubject={Security Assessment},
    pdfkeywords={Cybersecurity, Risk, Assessment, Nmap, RDP},
}

% --- CUSTOM COMMANDS ---
\newcommand{\riskcritical}[1]{\colorbox{red!80}{\color{white}\textbf{#1}}}
\newcommand{\riskhigh}[1]{\colorbox{orange!80}{\color{white}\textbf{#1}}}
\newcommand{\riskmedium}[1]{\colorbox{yellow!80}{\color{black}\textbf{#1}}}
\newcommand{\yes}{\ding{51}} % Green checkmark
\newcommand{\no}{\ding{55}}  % Red X

% --- TITLE ---
\title{
    \vspace{-1cm}
    \includegraphics[width=0.3\textwidth]{https://i.imgur.com/2Yv4qNA.png} \\ % Placeholder Logo
    \vspace{1cm}
    \textbf{Cybersecurity Risk Assessment Report} \\
    \large For: \textbf{[Organization Name]}
}
\author{Cybersecurity Analysis Division}
\date{\today}

% --- DOCUMENT START ---
\begin{document}

\maketitle
\thispagestyle{empty}
\newpage

\tableofcontents
\thispagestyle{empty}
\newpage

% =========================================================================
\section{Executive Summary}
% =========================================================================
This report details the findings of a cybersecurity assessment for \textbf{[Organization Name]}, conducted on \today. The analysis correlates results from an external network scan, a review of existing risks, and a security controls questionnaire.

The assessment identified two critical-priority risks that require immediate attention. The primary finding is the direct public exposure of the Remote Desktop Protocol (RDP) service on port 3389 at the client's external IP address. This configuration represents a significant and immediate threat, as it is a primary vector for ransomware attacks and unauthorized access.

This technical vulnerability is compounded by an organizational policy gap: the lack of Multi-Factor Authentication (MFA) on systems containing sensitive data. The combination of an exposed remote access service and potentially weak authentication controls creates a direct pathway for an adversary to compromise the internal network.

Immediate remediation of the RDP exposure and implementation of comprehensive MFA are strongly recommended to mitigate the high likelihood of a security breach.

% =========================================================================
\section{Organizational Information}
% =========================================================================
The following information was used as the basis for this assessment. Due to the anonymized nature of the provided data, placeholders have been used where necessary.

\begin{description}
    \item[Organization Name:] \textbf{[Organization Name]}
    \item[Primary Email Domain:] \texttt{[Domain]}
    \item[External IP Scanned:] \texttt{[Client IP]}
\end{description}

% =========================================================================
\section{Security Control Review}
% =========================================================================
The following table summarizes the organization's responses to a security controls questionnaire. While most responses align with best practices, a critical gap was identified regarding the protection of sensitive data systems.

\begin{table}[h!]
\centering
\caption{Security Controls Questionnaire Analysis}
\label{tab:controls}
\begin{tabular}{p{0.6\linewidth} c p{0.2\linewidth}}
\toprule
\textbf{Control Question} & \textbf{Response} & \textbf{Assessment} \\
\midrule
Do you require MFA to access email? & \yes & Best Practice Met \\
Do you require MFA to log into computers? & \yes & Best Practice Met \\
\rowcolor{red!15}
Do you require MFA to access sensitive data systems? & \no & \textbf{Critical Gap} \\
Does your organization have an employee acceptable use policy? & \yes & Best practice Met \\
Does your organization do security awareness training for new employees? & \yes & Best practice Met \\
Does your organization do security awareness training for all employees at least once per year? & \yes & Best practice Met \\
\bottomrule
\end{tabular}
\end{table}

% =========================================================================
\section{Technical Scan Results}
% =========================================================================
An external network port scan was conducted to identify exposed services on the organization's perimeter. The scan confirmed the presence of an open RDP port, which corroborates the pre-existing risk data.

\begin{description}
    \item[Target IP:] \texttt{[Target IP]}
    \item[Scan Date:] \today
\end{description}

\begin{table}[h!]
\centering
\caption{Open Port Findings}
\label{tab:ports}
\begin{tabular}{l l l p{0.4\linewidth}}
\toprule
\textbf{Port} & \textbf{State} & \textbf{Service} & \textbf{Analyst Notes} \\
\midrule
\rowcolor{red!15}
3389/tcp & open & ms-wbt-server & High Risk. This is the default port for Microsoft Remote Desktop Protocol (RDP). Public exposure is a common entry point for ransomware and targeted attacks. \\
\bottomrule
\end{tabular}
\end{table}

% =========================================================================
\section{Correlated Risk Assessment}
% =========================================================================
This section synthesizes all data points into a consolidated view of the top risks facing the organization. The technical findings from the network scan directly validate the pre-existing known risks and are exacerbated by the identified policy gaps.

\begin{table}[h!]
\centering
\caption{Summary of Key Risks}
\label{tab:risks}
\begin{tabular}{p{0.2\linewidth} p{0.5\linewidth} l}
\toprule
\textbf{Risk Name} & \textbf{Description} & \textbf{Severity} \\
\midrule
\rowcolor{red!15}
Public RDP Exposure & The Remote Desktop Protocol service on port 3389 is publicly accessible on \texttt{[Target IP]}. This allows attackers worldwide to attempt brute-force password attacks or exploit RDP vulnerabilities to gain unauthorized access to the internal network. & \riskcritical{Critical} \\
\rowcolor{orange!15}
Lack of MFA on Sensitive Systems & The organization does not enforce MFA for accessing sensitive data systems. If the exposed RDP server provides access to such a system, this gap removes a critical layer of defense, making a breach from compromised credentials trivial. & \riskhigh{High} \\
\bottomrule
\end{tabular}
\end{table}

% =========================================================================
\section{Recommendations}
% =========================================================================
Based on the correlated findings, the following actions are recommended to reduce the organization's risk exposure. Recommendations are prioritized by urgency.

\begin{enumerate}
    \item \textbf{[Immediate] Remediate RDP Exposure:}
    \begin{itemize}
        \item \textbf{Primary Fix:} Immediately close port 3389 on the external firewall for IP \texttt{[Target IP]}. Access to internal resources should be provided through a secure Virtual Private Network (VPN) that requires MFA.
        \item \textbf{Alternative Fix:} If immediate VPN implementation is not possible, restrict access to port 3389 to a whitelist of known, trusted source IP addresses. This is a temporary measure and is less secure than a VPN.
    \end{itemize}

    \item \textbf{[High Priority] Implement Comprehensive MFA:}
    \begin{itemize}
        \item Deploy a mandatory MFA policy for all remote access solutions (VPN, RDP Gateways) and all internal systems classified as containing sensitive data. This action directly mitigates the risk of credential theft leading to a data breach.
    \end{itemize}

    \item \textbf{[Ongoing] Enhance Security Posture:}
    \begin{itemize}
        \item Implement a regular vulnerability scanning program for all external-facing assets to proactively identify and remediate exposures like the one found in this assessment.
        \item Ensure robust logging and monitoring are in place for all remote access, especially for authentication successes and failures.
    \end{itemize}
\end{enumerate}

\end{document}
```