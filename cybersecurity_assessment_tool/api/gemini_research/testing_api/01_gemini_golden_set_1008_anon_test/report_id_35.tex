```latex
\documentclass[12pt]{article}

% --- PACKAGE IMPORTS ---
\usepackage[margin=1in]{geometry}
\usepackage{pifont} % For checkmarks and crosses
\usepackage{booktabs} % For professional tables
\usepackage{hyperref} % For hyperlinks
\usepackage{url} % For URL formatting
\usepackage{seqsplit} % To split long strings without breaking
\usepackage{graphicx}
\usepackage{xcolor}

% --- DOCUMENT METADATA ---
\hypersetup{
    colorlinks=true,
    linkcolor=blue,
    filecolor=magenta,      
    urlcolor=cyan,
    pdftitle={Cybersecurity Posture Assessment Report},
    pdfauthor={Cybersecurity Analyst},
    pdfkeywords={security, assessment, report},
}

% --- COMMAND DEFINITIONS ---
\newcommand{\yes}{\ding{51}}
\newcommand{\no}{\ding{55}}

% --- DOCUMENT START ---
\begin{document}

% --- TITLE PAGE ---
\begin{titlepage}
    \centering
    \vspace*{1cm}
    \Huge\textbf{Cybersecurity Posture Assessment Report}
    \vspace{1.5cm}
    \Large
    \textbf{Prepared for:} \textbf{[Organization Name]}\\
    \vspace{1cm}
    \textbf{Date:} \today\\
    \vfill
    \large
    \textbf{Generated By:}\\
    Expert Cybersecurity Analyst\\
    \vspace{0.5cm}
    \textit{This report contains sensitive information and should be handled with care.}
\end{titlepage}

\tableofcontents
\newpage

% --- EXECUTIVE SUMMARY ---
\section{Executive Summary}
This report provides a comprehensive analysis of the cybersecurity posture for \textbf{[Organization Name]}, based on network scans, a security controls questionnaire, and a review of pre-existing risks. The assessment was conducted on \today.

The analysis revealed several areas of concern requiring immediate attention. A critical gap was identified in the access control policy: Multi-Factor Authentication (MFA) is not enforced for sensitive data systems. This oversight significantly increases the risk of unauthorized access and data breach.

Furthermore, technical scans identified an exposed Secure Shell (SSH) service (port 22) on the external network perimeter at \texttt{[Target IP]}. While necessary for remote administration, an improperly configured or unmonitored SSH service is a primary target for attackers.

Finally, a pre-existing critical risk, "Localhost Exposed," with a CVSS score of 10.0, remains an outstanding issue that must be prioritized for remediation.

Overall, while foundational security controls like employee training and basic MFA are in place, critical vulnerabilities exist that elevate the organization's risk profile. The recommendations in this report are designed to directly address these findings and strengthen the overall security posture.

% --- ORGANIZATIONAL INFORMATION ---
\section{Organizational Information}
The following details were used as the basis for this assessment. Due to the anonymized nature of the provided data, placeholders have been used where necessary.

\begin{table}[h!]
\centering
\caption{Client Details}
\begin{tabular}{@{}ll@{}}
\toprule
\textbf{Attribute} & \textbf{Value} \\ \midrule
Organization Name & \textbf{[Organization Name]} \\
Primary Domain & \texttt{[Domain]} \\
External IP Address Scanned & \texttt{[Client IP]} \\ \bottomrule
\end{tabular}
\end{table}

% --- SECURITY CONTROL REVIEW ---
\section{Security Control Review (Questionnaire Analysis)}
A review of the organization's security controls was conducted via a questionnaire. The responses indicate a good foundation in security awareness and standard access controls, but highlight a critical weakness in protecting high-value assets.

\begin{table}[h!]
\centering
\caption{Security Controls Questionnaire Results}
\begin{tabular}{@{}p{0.7\linewidth}c@{}}
\toprule
\textbf{Question} & \textbf{Status} \\ \midrule
Do you require MFA to access email? & \yes \\
Do you require MFA to log into computers? & \yes \\
\textbf{Do you require MFA to access sensitive data systems?} & \textcolor{red}{\no} \\
Does your organization have an employee acceptable use policy? & \yes \\
Does your organization do security awareness training for new employees? & \yes \\
Does your organization do security awareness training for all employees at least once per year? & \yes \\ \bottomrule
\end{tabular}
\end{table}

\subsection*{Analysis}
The "No" response to requiring MFA for sensitive data systems is a \textbf{Critical Finding}. Standard user credentials (username and password) are highly susceptible to compromise through phishing, credential stuffing, or malware. Without a second factor of authentication, sensitive systems are vulnerable to unauthorized access, which could lead to a significant data breach.

% --- TECHNICAL SCAN RESULTS ---
\section{Technical Scan Results}
An external network scan was performed to identify exposed services and potential vulnerabilities on the client's perimeter.

\subsection{Scan Metadata}
\begin{itemize}
    \item \textbf{Target IP Address:} \texttt{[Target IP]}
    \item \textbf{Scan Date:} Information not available in scan data.
    \item \textbf{Scanner Used:} Nmap
\end{itemize}

\subsection{Open Ports and Services}
The scan revealed the following open port on the target host.

\begin{table}[h!]
\centering
\caption{Discovered Open Ports}
\begin{tabular}{@{}lllll@{}}
\toprule
\textbf{Port} & \textbf{State} & \textbf{Service (Inferred)} & \textbf{Product} & \textbf{Version} \\ \midrule
22/tcp & open & SSH & \textit{Not Available} & \textit{Not Available} \\ \bottomrule
\end{tabular}
\end{table}

\subsection*{Analysis}
The presence of an open SSH port (22) indicates that remote administrative access is enabled from the public internet. This is a common practice but carries significant risk if not properly secured. Without detailed version information, it is not possible to check for specific known vulnerabilities. However, any internet-facing SSH service is a constant target for brute-force login attempts and automated attacks.

% --- CONSOLIDATED RISK ASSESSMENT ---
\section{Consolidated Risk Assessment}
The following table synthesizes findings from the security questionnaire, technical scans, and pre-existing risk logs to provide a unified view of the organization's current top risks.

\begin{table}[h!]
\centering
\caption{Summary of Identified Risks}
\begin{tabular}{@{}p{0.1\linewidth}p{0.45\linewidth}p{0.15\linewidth}p{0.2\linewidth}@{}}
\toprule
\textbf{Risk ID} & \textbf{Description} & \textbf{Severity} & \textbf{Source} \\ \midrule
RISK-001 & Lack of MFA on sensitive data systems allows for single-factor authentication compromise, potentially leading to a major data breach. & \textbf{Critical} & Questionnaire \\
\addlinespace
RISK-002 & Publicly exposed SSH service without sufficient hardening is a prime target for brute-force attacks and unauthorized access attempts. & \textbf{High} & Network Scan \\
\addlinespace
RISK-003 & A pre-existing risk, "Localhost Exposed," indicates a critical service is improperly bound to a public interface. (CVSS 10.0) & \textbf{Critical} & Pre-existing Risk Log \\ \bottomrule
\end{tabular}
\end{table}

% --- RECOMMENDATIONS ---
\section{Recommendations}
The following actionable recommendations are provided to mitigate the identified risks and improve the overall security posture of \textbf{[Organization Name]}.

\subsection{RISK-001: Remediate MFA Gap (Priority: Immediate)}
\begin{itemize}
    \item \textbf{Action:} Enforce phishing-resistant Multi-Factor Authentication (MFA) across all systems classified as containing sensitive data. This includes databases, file servers, and critical applications.
    \item \textbf{Justification:} This is the most effective single control to prevent unauthorized access resulting from compromised credentials. It directly addresses the critical finding from the security controls review.
\end{itemize}

\subsection{RISK-002: Harden Exposed SSH Service (Priority: High)}
\begin{itemize}
    \item \textbf{Action 1 (Best):} If possible, restrict SSH access to a list of trusted IP addresses (allow-listing) via firewall rules. Access from outside this list should be denied.
    \item \textbf{Action 2 (Required):} Disable password-based authentication and enforce the use of public key cryptography for all SSH logins.
    \item \textbf{Action 3 (Required):} Disable direct root login over SSH. Administrators should log in with a standard user account and elevate privileges as needed.
    \item \textbf{Action 4 (Recommended):} Implement an intrusion prevention tool like \texttt{Fail2Ban} to automatically block IP addresses that exhibit malicious behavior, such as repeated failed login attempts.
\end{itemize}

\subsection{RISK-003: Investigate and Remediate "Localhost Exposed" (Priority: Immediate)}
\begin{itemize}
    \item \textbf{Action:} An urgent investigation must be conducted to identify the service associated with the "Localhost Exposed" finding. The service must be reconfigured to bind only to the local loopback interface (127.0.0.1 or ::1) and not to the public network interface.
    \item \textbf{Justification:} A CVSS score of 10.0 represents the highest possible risk. This vulnerability likely exposes a highly privileged service to the public internet, which could lead to a complete system compromise. This issue must be treated as an active incident.
\end{itemize}

% --- DOCUMENT END ---
\end{document}
```