```latex
\documentclass[12pt]{article}

% Preamble: Required Packages
\usepackage[utf8]{inputenc}
\usepackage[margin=1in]{geometry}
\usepackage{pifont} % For checkmarks and crosses
\usepackage{booktabs} % For professional tables
\usepackage{hyperref} % For hyperlinks
\usepackage{url} % For URL formatting
\usepackage{seqsplit} % For splitting long strings
\usepackage{xcolor} % For colors
\usepackage{graphicx} % For images, if needed

% Document Metadata
\title{Cybersecurity Posture Assessment Report}
\author{Cybersecurity Analysis Division}
\date{November 22, 2025}

\hypersetup{
    colorlinks=true,
    linkcolor=blue,
    filecolor=magenta,      
    urlcolor=cyan,
    pdftitle={Cybersecurity Posture Assessment Report},
    pdfpagemode=FullScreen,
}

\begin{document}

\maketitle
\thispagestyle{empty}
\newpage

\tableofcontents
\newpage

% --- Section 1: Executive Summary ---
\section{Executive Summary}

This report details the findings of a cybersecurity posture assessment conducted for \textbf{[Organization Name]} on November 22, 2025. The assessment combined a review of organizational security controls, an external network scan, and an analysis of pre-existing risks to provide a holistic view of the organization's security posture.

The analysis revealed several areas of concern requiring immediate attention. While the organization has implemented some positive security controls, such as Multi-Factor Authentication (MFA) for computer and sensitive system access, critical gaps exist.

Key findings include:
\begin{itemize}
    \item \textbf{Critical Risk:} The lack of mandatory MFA for email access presents a significant vulnerability. Email is a primary target for attackers, and a compromised account can lead to widespread system compromise and data breaches.
    \item \textbf{High Risk:} The external-facing web server at \texttt{[Target IP]} is running an outdated version of Nginx (1.18.0), which has publicly documented vulnerabilities. This exposes the server to potential exploitation.
    \item \textbf{High Risk:} The absence of a formal Employee Acceptable Use Policy (AUP) creates ambiguity regarding security responsibilities and acceptable behavior, increasing the likelihood of insider threats and policy violations.
\end{itemize}

This report provides a detailed breakdown of these findings and offers actionable recommendations to mitigate the identified risks and strengthen the overall security posture of \textbf{[Organization Name]}.

% --- Section 2: Organizational Information ---
\section{Organizational Information}

This section provides high-level details about the organization and the scope of this assessment. As identity data was not provided, placeholders have been used.

\begin{table}[h!]
\centering
\begin{tabular}{@{}ll@{}}
\toprule
\textbf{Item} & \textbf{Detail} \\ \midrule
Organization Name & \textbf{[Organization Name]} \\
Primary Email Domain & \texttt{[Domain]} \\
External IP Address Scanned & \texttt{[Client IP]} \\
Assessment Date & November 22, 2025 \\ \bottomrule
\end{tabular}
\caption{Organizational and Assessment Details}
\end{table}

% --- Section 3: Security Control Review ---
\section{Security Control Review}

A review of organizational security controls was conducted via a questionnaire. The responses indicate the current state of implemented policies and procedures. "No" answers represent significant gaps in the security framework.

\begin{table}[h!]
\centering
\begin{tabular}{@{}p{0.8\linewidth}c@{}}
\toprule
\textbf{Control Question} & \textbf{Response} \\ \midrule
Do you require MFA to access email? & \textcolor{red}{\ding{55}} \\
Do you require MFA to log into computers? & \textcolor{green}{\ding{51}} \\
Do you require MFA to access sensitive data systems? & \textcolor{green}{\ding{51}} \\
Does your organization have an employee acceptable use policy? & \textcolor{red}{\ding{55}} \\
Does your organization do security awareness training for new employees? & \textcolor{green}{\ding{51}} \\
Does your organization do security awareness training for all employees at least once per year? & \textcolor{green}{\ding{51}} \\ \bottomrule
\end{tabular}
\caption{Security Controls Questionnaire Results}
\end{table}

\subsection*{Analysis of Control Gaps}
\begin{itemize}
    \item \textbf{No MFA on Email:} This is a critical deficiency. Email accounts are high-value targets for phishing and account takeover attacks. Without MFA, a single compromised password could grant an attacker access to sensitive communications and a trusted identity from which to launch further attacks.
    \item \textbf{No Acceptable Use Policy (AUP):} An AUP is a foundational policy that defines the rules and responsibilities for employees using company IT assets. Its absence can lead to inconsistent security practices, misuse of resources, and a lack of legal recourse in the event of an insider incident.
\end{itemize}

% --- Section 4: Technical Scan Results ---
\section{Technical Scan Results}

An external network scan was performed to identify open ports and exposed services on the organization's perimeter.

\begin{itemize}
    \item \textbf{Target IP:} \texttt{[Target IP]}
    \item \textbf{Scan Date:} 2025-11-22
\end{itemize}

\begin{table}[h!]
\centering
\begin{tabular}{@{}llllll@{}}
\toprule
\textbf{Port} & \textbf{State} & \textbf{Service} & \textbf{Product} & \textbf{Version} & \textbf{Notes} \\ \midrule
443/tcp & open & https & nginx & 1.18.0 & \textbf{Outdated Version} \\ \bottomrule
\end{tabular}
\caption{Open Ports and Services Detected}
\end{table}

\subsection*{Analysis of Technical Findings}
The scan identified an Nginx web server, version 1.18.0, listening on port 443 (HTTPS). This version was released in April 2020 and is now significantly outdated. It is associated with several publicly known vulnerabilities (CVEs) that could be exploited by an attacker to achieve remote code execution, denial of service, or information disclosure. Running unsupported and unpatched software on an internet-facing server presents a high level of risk.

% --- Section 5: Consolidated Risk Assessment ---
\section{Consolidated Risk Assessment}

This section synthesizes findings from the security control review and the technical scan into a consolidated list of identified risks. No pre-existing risks were reported.

\begin{table}[h!]
\centering
\begin{tabular}{@{}p{0.1\linewidth}p{0.25\linewidth}p{0.45\linewidth}p{0.1\linewidth}@{}}
\toprule
\textbf{Risk ID} & \textbf{Risk Name} & \textbf{Description} & \textbf{Severity} \\ \midrule
RISK-001 & Lack of MFA on Email & The absence of MFA on email accounts exposes the organization to a high risk of account takeover, phishing, and subsequent data breaches. & \textbf{Critical} \\
\addlinespace
RISK-002 & Outdated Web Server Software & The external web server runs Nginx 1.18.0, a version with known vulnerabilities that could allow an attacker to compromise the server. & \textbf{High} \\
\addlinespace
RISK-003 & Missing Acceptable Use Policy & The lack of a formal AUP creates ambiguity for employees, increasing the risk of insider threats and non-compliance. & \textbf{High} \\ \bottomrule
\end{tabular}
\caption{Summary of Identified Risks}
\end{table}

% --- Section 6: Recommendations ---
\section{Recommendations}

The following actions are recommended to mitigate the identified risks and improve the overall security posture of \textbf{[Organization Name]}. Recommendations are prioritized by severity.

\subsection*{RISK-001: Lack of MFA on Email (Critical)}
\begin{itemize}
    \item \textbf{Immediate Action:} Procure and implement a robust Multi-Factor Authentication (MFA) solution for the organization's email system.
    \item \textbf{Policy:} Mandate the use of MFA for all users, including administrative and service accounts, without exception.
    \item \textbf{Timeline:} This should be treated as the highest priority remediation project, with a target completion of 30 days.
\end{itemize}

\subsection*{RISK-002: Outdated Web Server Software (High)}
\begin{itemize}
    \item \textbf{Immediate Action:} Develop a patch management plan to upgrade the Nginx server on host \texttt{[Target IP]} to the latest stable version. Test the upgrade in a non-production environment first to ensure compatibility.
    \item \textbf{Ongoing Action:} Implement a formal vulnerability management program that includes regular, automated scanning of external assets to identify and remediate outdated software in a timely manner.
\end{itemize}

\subsection*{RISK-003: Missing Acceptable Use Policy (High)}
\begin{itemize}
    \item \textbf{Immediate Action:} Draft and ratify a comprehensive Acceptable Use Policy (AUP) that clearly outlines the rules for using company technology and data.
    \item \textbf{Implementation:} Distribute the policy to all current employees for review and acknowledgment. Incorporate the AUP into the onboarding process for all new hires.
    \item \textbf{Training:} Conduct training to ensure all employees understand the policy's contents and their responsibilities.
\end{itemize}

\end{document}
```