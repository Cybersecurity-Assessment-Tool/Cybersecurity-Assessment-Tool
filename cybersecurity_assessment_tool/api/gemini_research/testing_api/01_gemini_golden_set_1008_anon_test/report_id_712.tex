```latex
\documentclass[12pt]{article}

% Preamble: Required Packages
\usepackage[margin=1in]{geometry}
\usepackage{pifont} % For checkmarks and crosses
\usepackage{booktabs} % For professional tables
\usepackage{hyperref} % For hyperlinks
\usepackage{url} % For URL formatting
\usepackage{seqsplit} % For splitting long strings in tt font
\usepackage{graphicx}
\usepackage{xcolor}

% Document Metadata
\title{Cybersecurity Posture Assessment Report}
\author{Cybersecurity Analysis Division}
\date{\today}

% Hyperref Setup
\hypersetup{
    colorlinks=true,
    linkcolor=blue,
    filecolor=magenta,      
    urlcolor=cyan,
    pdftitle={Cybersecurity Posture Assessment Report},
    pdfpagemode=FullScreen,
}

\begin{document}

\maketitle
\thispagestyle{empty}
\newpage

\tableofcontents
\newpage

% --- Executive Summary ---
\section{Executive Summary}

This report provides a comprehensive analysis of the cybersecurity posture for \textbf{[Organization Name]}. The assessment is based on a synthesis of an external network scan, a review of internal security controls via a questionnaire, and an evaluation of pre-existing risks.

The external network scan of the target IP address \texttt{[Client IP]} revealed a strong perimeter security posture. No open ports were discovered, indicating that the firewall is effectively configured to deny unsolicited inbound traffic. This is a significant security strength.

However, the internal security control review identified several critical and high-risk administrative gaps. The most severe finding is the lack of Multi-Factor Authentication (MFA) for computer logins, which exposes the organization to significant risk from compromised credentials. Additionally, the absence of a formal Acceptable Use Policy (AUP) and a mandatory security awareness training program for new hires represents a substantial weakness in the organization's security governance.

This report outlines these findings in detail and provides prioritized, actionable recommendations to mitigate the identified risks and enhance the overall security posture.

% --- Organizational Information ---
\section{Organizational Information}

The following information was used as the basis for this assessment. As identity data was not provided, placeholders have been used.

\begin{table}[h!]
\centering
\begin{tabular}{@{}ll@{}}
\toprule
\textbf{Attribute} & \textbf{Value} \\ \midrule
Organization Name & \textbf{[Organization Name]} \\
Primary Domain & \seqsplit{\texttt{[Domain]}} \\
External IP Scanned & \seqsplit{\texttt{[Client IP]}} \\ \bottomrule
\end{tabular}
\caption{Client Organizational Details}
\label{tab:org_info}
\end{table}

% --- Security Control Review ---
\section{Security Control Review}

An internal security questionnaire was completed to evaluate the status of key administrative and technical controls. The responses indicate several areas requiring immediate attention. "No" answers represent significant gaps in the security framework.

\begin{table}[h!]
\centering
\begin{tabular}{@{}p{0.7\textwidth}cc@{}}
\toprule
\textbf{Control Question} & \textbf{Response} & \textbf{Status} \\ \midrule
Do you require MFA to access email? & Yes & \ding{51} \\
\rowcolor{red!15} Do you require MFA to log into computers? & No & \ding{55} \\
Do you require MFA to access sensitive data systems? & Yes & \ding{51} \\
\rowcolor{red!15} Does your organization have an employee acceptable use policy? & No & \ding{55} \\
\rowcolor{red!15} Does your organization do security awareness training for new employees? & No & \ding{55} \\
Does your organization do security awareness training for all employees at least once per year? & Yes & \ding{51} \\ \bottomrule
\end{tabular}
\caption{Security Control Questionnaire Analysis}
\label{tab:controls}
\end{table}

% --- Technical Scan Results ---
\section{Technical Scan Results}

An external network vulnerability scan was performed to identify potential weaknesses visible from the public internet.

\subsection{Nmap Scan Findings}
\begin{itemize}
    \item \textbf{Target IP:} \seqsplit{\texttt{[Target IP]}}
    \item \textbf{Host Status:} Up
    \item \textbf{Scan Summary:} The scan revealed an excellent security posture for the target host. All scanned ports were found to be in a \texttt{closed} state, meaning no services were exposed to the internet on the ports tested. This indicates a properly configured firewall and a minimal attack surface.
    \item \textbf{Open Ports Detected:} None.
\end{itemize}

% --- Risk Assessment ---
\section{Risk Assessment}

This section synthesizes findings from the security control review and technical scan. While the external technical posture is strong, significant risks stem from internal policy and procedure gaps. No pre-existing vulnerabilities were provided for this assessment.

\begin{table}[h!]
\centering
\begin{tabular}{@{}p{0.1\textwidth}p{0.25\textwidth}p{0.15\textwidth}p{0.4\textwidth}@{}}
\toprule
\textbf{Risk ID} & \textbf{Risk Name} & \textbf{Severity} & \textbf{Description} \\ \midrule
RISK-001 & Lack of MFA on Endpoints & \textbf{Critical} & Workstations and laptops lack Multi-Factor Authentication. If an employee's password is stolen, an attacker can gain direct access to the internal network and company resources. \\
\addlinespace
RISK-002 & Missing Acceptable Use Policy & \textbf{High} & The absence of a formal Acceptable Use Policy (AUP) creates ambiguity for employees regarding the secure and appropriate use of company assets, data, and network resources. \\
\addlinespace
RISK-003 & No Security Training for New Hires & \textbf{High} & New employees do not receive security awareness training during onboarding. This makes them highly susceptible to phishing, social engineering, and other common cyberattacks. \\ \bottomrule
\end{tabular}
\caption{Summary of Identified Risks}
\label{tab:risks}
\end{table}

% --- Recommendations ---
\section{Recommendations}

The following prioritized recommendations are provided to address the risks identified in this report and improve the overall security posture of \textbf{[Organization Name]}.

\subsection{Priority 1: Critical Risk Mitigation}
\begin{itemize}
    \item \textbf{Implement MFA for Endpoint Logins (RISK-001):}
    \begin{itemize}
        \item Deploy a Multi-Factor Authentication solution for all employee computer and laptop logins (e.g., Windows Hello for Business, Duo, Okta).
        \item This action is the single most effective control to prevent unauthorized access resulting from compromised credentials.
    \end{itemize}
\end{itemize}

\subsection{Priority 2: High Risk Mitigation}
\begin{itemize}
    \item \textbf{Develop an Acceptable Use Policy (RISK-002):}
    \begin{itemize}
        \item Create a formal AUP document that clearly defines the rules and expectations for employees when using company technology and data.
        \item Ensure the policy is reviewed by leadership, communicated to all staff, and formally acknowledged by every employee.
    \end{itemize}
    \item \textbf{Establish Onboarding Security Training (RISK-003):}
    \begin{itemize}
        \item Develop a mandatory security awareness training module that is a required part of the new employee onboarding process.
        \item This training should cover key topics such as phishing identification, password security, data handling, and the new Acceptable Use Policy.
    \end{itemize}
\end{itemize}

\subsection{Strengths to Maintain}
\begin{itemize}
    \item \textbf{Perimeter Security:} Continue to maintain the strong external network security posture. Regularly review firewall rules to ensure that no unnecessary ports are opened and that the principle of least privilege is enforced.
\end{itemize}

\end{document}
```