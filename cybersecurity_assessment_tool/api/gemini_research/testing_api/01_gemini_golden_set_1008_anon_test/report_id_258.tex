```latex
\documentclass[12pt]{article}

% --- PACKAGES ---
\usepackage[margin=1in]{geometry} % Set page margins
\usepackage{pifont}               % For checkmarks and crosses (\ding)
\usepackage{booktabs}             % For professional-looking tables
\usepackage{graphicx}             % For including logos (optional)
\usepackage{hyperref}             % For hyperlinks
\usepackage{url}                  % For formatting URLs
\usepackage{seqsplit}             % For splitting long strings in texttt
\usepackage[utf8]{inputenc}       % For UTF-8 input
\usepackage{xcolor}               % For colors

% --- DOCUMENT METADATA ---
\title{Cybersecurity Posture Assessment Report}
\author{Cybersecurity Analysis Division}
\date{\today}

% --- HYPERREF SETUP ---
\hypersetup{
    colorlinks=true,
    linkcolor=blue,
    filecolor=magenta,      
    urlcolor=cyan,
    pdftitle={Cybersecurity Posture Assessment Report},
    pdfpagemode=FullScreen,
}

\begin{document}

\maketitle
\thispagestyle{empty}
\newpage

\tableofcontents
\newpage

% --- SECTION 1: EXECUTIVE SUMMARY ---
\section{Executive Summary}

This report provides a comprehensive analysis of the cybersecurity posture for \textbf{[Organization Name]}, conducted on \today. The assessment is based on a synthesis of a network vulnerability scan, a security controls questionnaire, and a review of previously identified risks.

\paragraph{Key Findings:} The overall security posture presents a mixed landscape. The external network scan of the target host \texttt{[Target IP]} revealed a strong perimeter, with no open ports discovered. This is a positive finding and indicates that a previously identified risk, "Unencrypted Web Server," has likely been remediated as port 80 was found to be closed.

However, a critical gap was identified in the organization's internal security controls. The questionnaire revealed a lack of Multi-Factor Authentication (MFA) for logging into employee computers. This represents a high-risk exposure, as compromised credentials could grant an attacker direct access to endpoint systems, bypassing other security layers.

\paragraph{Overall Assessment:} While the organization demonstrates good network hygiene on the scanned asset, the identified gap in endpoint access control significantly elevates the risk of a security breach. Immediate action is recommended to address the MFA deficiency to protect against common attack vectors like phishing and credential theft.

% --- SECTION 2: ORGANIZATIONAL INFORMATION ---
\section{Organizational Information}

The following information was used as the basis for this assessment. Due to the anonymized nature of the provided data, placeholders have been used where necessary.

\begin{table}[h!]
\centering
\begin{tabular}{@{}ll@{}}
\toprule
\textbf{Attribute} & \textbf{Value} \\ \midrule
Organization Name & \textbf{[Organization Name]} \\
Primary Domain & \texttt{[Domain]} \\
External IP Address (Source) & \texttt{[Client IP]} \\
Target IP Address (Scanned) & \texttt{[Target IP]} \\
\bottomrule
\end{tabular}
\caption{Client and Target Information.}
\label{tab:org_info}
\end{table}

% --- SECTION 3: SECURITY CONTROL REVIEW ---
\section{Security Control Review (Questionnaire Analysis)}

A review of the organization's security controls was conducted via a questionnaire. The responses indicate a solid foundation in policy and awareness training. However, a critical gap in endpoint authentication was identified.

\begin{table}[h!]
\centering
\begin{tabular}{@{}p{0.6\linewidth}cp{0.25\linewidth}@{}}
\toprule
\textbf{Control Question} & \textbf{Response} & \textbf{Analyst Notes} \\ \midrule
Do you require MFA to access email? & \ding{51} & Commendable. Protects a primary communication vector. \\
\addlinespace
Do you require MFA to log into computers? & \textbf{\color{red}\ding{55}} & \textbf{Critical Gap.} Lack of endpoint MFA is a high-risk finding. \\
\addlinespace
Do you require MFA to access sensitive data systems? & \ding{51} & Good practice. Protects critical information assets. \\
\addlinespace
Does your organization have an employee acceptable use policy? & \ding{51} & Foundational policy is in place. \\
\addlinespace
Does your organization do security awareness training for new employees? & \ding{51} & Positive. Onboarding is a key time for training. \\
\addlinespace
Does your organization do security awareness training for all employees at least once per year? & \ding{51} & Strong. Reinforces a culture of security. \\
\bottomrule
\end{tabular}
\caption{Analysis of Security Control Questionnaire.}
\label{tab:controls}
\end{table}

% --- SECTION 4: TECHNICAL SCAN RESULTS ---
\section{Technical Scan Results}
An external network scan was performed on the target host \texttt{[Target IP]} to identify open ports and exposed services.

\subsection{Scan Summary}
The scan results were positive, indicating a well-configured network perimeter for the target host. No open ports were discovered. This significantly reduces the external attack surface.

\begin{table}[h!]
\centering
\begin{tabular}{@{}lllll@{}}
\toprule
\textbf{Port} & \textbf{Protocol} & \textbf{State} & \textbf{Service} & \textbf{Notes} \\ \midrule
80 & TCP & closed & http & Port is not listening. \\
\bottomrule
\end{tabular}
\caption{Nmap Scan Results for Target: \texttt{[Target IP]}.}
\label{tab:scan_results}
\end{table}

\subsection{Correlation with Existing Risks}
The provided list of existing risks included an item named "Unencrypted Web Server," which was based on the assumption that port 80 was open. Our scan confirms that port 80 is \textbf{closed}. This new data suggests that the previously identified risk has been successfully remediated or was a false positive. We recommend updating the internal risk register to reflect this corrected status.

% --- SECTION 5: CONSOLIDATED RISK ASSESSMENT ---
\section{Consolidated Risk Assessment}
This section synthesizes findings from the security questionnaire, technical scan, and pre-existing risk data into a consolidated list of current risks.

\begin{table}[h!]
\centering
\begin{tabular}{@{}p{0.1\linewidth}p{0.25\linewidth}p{0.4\linewidth}l@{}}
\toprule
\textbf{Risk ID} & \textbf{Risk Name} & \textbf{Description} & \textbf{Severity} \\ \midrule
RISK-001 & Lack of MFA for Endpoint Logins & The absence of MFA on computer logins allows an attacker with stolen credentials to gain direct access to an endpoint, potentially leading to data theft or ransomware. & \textbf{High} \\
\addlinespace
RISK-002 & Unencrypted Web Server & \textit{(Status Update)} A pre-existing risk noted an open port 80. Our scan confirms this port is now closed. This risk appears to be remediated. & Informational \\
\bottomrule
\end{tabular}
\caption{Summary of Identified and Updated Risks.}
\label{tab:risk_summary}
\end{table}

% --- SECTION 6: RECOMMENDATIONS ---
\section{Recommendations}
Based on the analysis, the following actions are recommended to improve the organization's security posture. Recommendations are prioritized based on risk severity.

\subsection{Priority 1 (High): Implement MFA for Endpoint Logins}
\begin{itemize}
    \item \textbf{Action:} Mandate the use of Multi-Factor Authentication for all user logins to company-managed computers (desktops and laptops).
    \item \textbf{Justification:} This is the most critical action to mitigate the risk of unauthorized access resulting from credential compromise (e.g., phishing, password spraying). It provides a crucial layer of defense at the endpoint level.
    \item \textbf{Suggested Solutions:}
    \begin{itemize}
        \item Native platform solutions like Windows Hello for Business.
        \item Third-party identity providers such as Duo Security, Okta, or Microsoft Authenticator.
    \end{itemize}
\end{itemize}

\subsection{Priority 2 (Informational): Update Risk Register}
\begin{itemize}
    \item \textbf{Action:} Formally close the risk item "Unencrypted Web Server" (RISK-002) in the organization's internal risk register.
    \item \textbf{Justification:} The technical scan data confirms that the mitigating control (closing port 80) is in place. Maintaining an accurate risk register is essential for effective security management and prioritization of resources.
\end{itemize}

\end{document}
```