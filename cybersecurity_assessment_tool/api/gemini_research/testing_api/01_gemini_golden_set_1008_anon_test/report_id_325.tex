```latex
\documentclass[12pt]{article}

% Preamble: Required Packages
\usepackage[margin=1in]{geometry}
\usepackage{pifont} % For checkmarks and crosses (\ding{51}, \ding{55})
\usepackage{booktabs} % For professional-looking tables
\usepackage[hidelinks]{hyperref} % For clickable links
\usepackage{url} % For typesetting URLs
\usepackage{seqsplit} % For splitting long strings without spaces
\usepackage{graphicx} % For logo (placeholder)
\usepackage{fancyhdr} % For header/footer

% --- Document Metadata ---
\title{Cybersecurity Posture Assessment Report}
\author{Cybersecurity Analysis Division}
\date{November 22, 2025}

% --- Header and Footer ---
\pagestyle{fancy}
\fancyhf{} % Clear all header and footer fields
\fancyhead[L]{\textbf{Cybersecurity Assessment Report}}
\fancyhead[R]{\textbf{[Organization Name]}}
\fancyfoot[C]{\thepage}
\renewcommand{\headrulewidth}{0.4pt}
\renewcommand{\footrulewidth}{0.4pt}

\begin{document}

\maketitle
\thispagestyle{empty}
\newpage

\tableofcontents
\newpage

% --- 1. Executive Summary ---
\section{Executive Summary}

This report provides a comprehensive analysis of the cybersecurity posture for \textbf{[Organization Name]}, conducted on November 22, 2025. The assessment combines a review of organizational security controls, an external network vulnerability scan, and an evaluation of pre-existing risks.

The organization demonstrates a solid foundation in identity and access management, with commendable implementation of Multi-Factor Authentication (MFA) across email, workstations, and sensitive data systems. An acceptable use policy and security training for new hires are also in place, indicating a baseline commitment to security.

However, two significant risks were identified that require immediate attention:

\begin{enumerate}
    \item \textbf{Outdated Public-Facing Software:} The external web server at \texttt{[Target IP]} is running an outdated version of Nginx (1.18.0). This version is no longer supported and has multiple known vulnerabilities, exposing the organization to potential compromise.
    \item \textbf{Security Training Gap:} The organization does not conduct mandatory annual security awareness training for all employees. This gap significantly increases the risk of human-error incidents, such as successful phishing attacks, which could lead to data breaches or ransomware events.
\end{enumerate}

These findings are classified as \textbf{High Risk}. This report details these risks and provides actionable recommendations to mitigate them and enhance the overall security posture of \textbf{[Organization Name]}.

% --- 2. Organizational Information ---
\section{Organizational Information}

This section contains the high-level information used for this assessment.

\begin{tabular}{@{}ll}
    \toprule
    \textbf{Attribute} & \textbf{Value} \\
    \midrule
    Organization Name & \textbf{[Organization Name]} \\
    Primary Domain & \texttt{[Domain]} \\
    External IP Address & \texttt{[Client IP]} \\
    Assessment Date & 2025-11-22 \\
    \bottomrule
\end{tabular}

% --- 3. Security Control Review ---
\section{Security Control Review}

The following table summarizes the organization's responses to the security controls questionnaire. While most controls are implemented, a critical gap was identified in the security awareness training program.

\begin{table}[h!]
\centering
\begin{tabular}{p{0.6\textwidth}cc}
    \toprule
    \textbf{Control Question} & \textbf{Response} & \textbf{Status} \\
    \midrule
    Do you require MFA to access email? & Yes & \ding{51} \\
    Do you require MFA to log into computers? & Yes & \ding{51} \\
    Do you require MFA to access sensitive data systems? & Yes & \ding{51} \\
    Does your organization have an employee acceptable use policy? & Yes & \ding{51} \\
    Does your organization do security awareness training for new employees? & Yes & \ding{51} \\
    \addlinespace
    \textbf{Does your organization do security awareness training for all} & \textbf{No} & \textbf{\ding{55} Gap Identified} \\
    \textbf{employees at least once per year?} & & \\
    \bottomrule
\end{tabular}
\caption{Organizational Security Controls Questionnaire Results.}
\label{tab:controls}
\end{table}

% --- 4. Technical Scan Results ---
\section{Technical Scan Results}

An external network scan was performed to identify open ports and exposed services.

\begin{itemize}
    \item \textbf{Scan Target:} \texttt{[Target IP]}
    \item \textbf{Scan Date:} 2025-11-22
\end{itemize}

The scan revealed one open port, which is detailed below. The Nginx web server version detected is significantly outdated and poses a security risk.

\begin{table}[h!]
\centering
\begin{tabular}{lllll}
    \toprule
    \textbf{Port} & \textbf{State} & \textbf{Service} & \textbf{Product} & \textbf{Version} \\
    \midrule
    443/tcp & open & https & nginx & 1.18.0 \\
    \bottomrule
\end{tabular}
\caption{Open Ports and Services Detected on \texttt{[Target IP]}.}
\label{tab:nmap}
\end{table}

% --- 5. Risk Assessment ---
\section{Risk Assessment}

This section synthesizes findings from the security control review and the technical scan into a prioritized list of risks. No pre-existing vulnerabilities were reported in the provided data.

\begin{table}[h!]
\centering
\begin{tabular}{p{0.1\textwidth} p{0.25\textwidth} p{0.45\textwidth} p{0.1\textwidth}}
    \toprule
    \textbf{Risk ID} & \textbf{Risk Name} & \textbf{Description} & \textbf{Severity} \\
    \midrule
    \textbf{RISK-001} & Outdated Nginx Web Server & The web server at \texttt{[Target IP]} is running Nginx version 1.18.0, which is end-of-life. This exposes the server to numerous publicly known vulnerabilities (e.g., CVE-2021-23017) that could be exploited by attackers to compromise the system. & \textbf{High} \\
    \addlinespace
    \textbf{RISK-002} & Inadequate Security Awareness Training & The lack of mandatory, annual security training for all employees creates a significant vulnerability. A well-trained workforce is the first line of defense against phishing and social engineering attacks. This gap increases the likelihood of a security breach originating from human error. & \textbf{High} \\
    \bottomrule
\end{tabular}
\caption{Summary of Identified Risks.}
\label{tab:risks}
\end{table}

% --- 6. Recommendations ---
\section{Recommendations}

The following actions are recommended to mitigate the identified risks and strengthen the organization's security posture.

\subsection{Remediation for RISK-001: Outdated Nginx Web Server}
\begin{enumerate}
    \item \textbf{Immediate Action:} Upgrade the Nginx instance on server \texttt{[Target IP]} from version 1.18.0 to the latest stable version recommended by the vendor. This will patch known security vulnerabilities.
    \item \textbf{Ongoing Strategy:} Implement a formal patch management policy and schedule for all public-facing systems. This should include quarterly reviews and timely application of security patches to prevent similar issues in the future.
\end{enumerate}

\subsection{Remediation for RISK-002: Inadequate Security Awareness Training}
\begin{enumerate}
    \item \textbf{Immediate Action:} Procure and deploy a security awareness training program for all current employees. This training should be mandatory and completed within the next quarter.
    \item \textbf{Ongoing Strategy:} Establish a recurring, annual security awareness training program. The content should be updated yearly to reflect the current threat landscape, covering topics such as phishing, ransomware, password security, and incident reporting.
\end{enumerate}

\end{document}
```