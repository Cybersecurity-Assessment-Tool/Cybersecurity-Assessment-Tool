```latex
\documentclass[12pt]{article}

% Preamble: Required Packages
\usepackage[a4paper, margin=1in]{geometry}
\usepackage{pifont} % For checkmarks and crosses
\usepackage{booktabs} % For professional-looking tables
\usepackage{hyperref} % For clickable links
\usepackage{url} % For formatting URLs
\usepackage{seqsplit} % For splitting long strings to prevent overflow
\usepackage{graphicx} % For potential logos
\usepackage{xcolor} % For colors in text

% Document Information
\title{Cybersecurity Posture Assessment Report}
\author{Cybersecurity Analysis Division}
\date{\today}

\begin{document}

\maketitle

\begin{center}
    \textbf{Prepared for:} \textbf{[Organization Name]} \\
    \textbf{Report Date:} \today
\end{center}

\hrule\vspace{1em}

% --- 1. Executive Summary ---
\section*{1. Executive Summary}

This report provides a cybersecurity posture assessment for \textbf{[Organization Name]}, based on an analysis of network scan data, organizational security controls, and pre-existing risk information. The assessment was conducted on \today.

Overall, the organization demonstrates a strong commitment to identity and access management, with mandatory Multi-Factor Authentication (MFA) and a robust security awareness training program. These are commendable foundational controls.

However, two critical areas of concern were identified that require immediate attention:
\begin{itemize}
    \item \textbf{Critical External Exposure:} A network scan identified an externally accessible MySQL database server on port 3306. The running version, MySQL 5.7.33, is past its End-of-Life (EOL) and no longer receives security updates, posing a severe risk of data breach.
    \item \textbf{High-Risk Policy Gap:} The organization lacks a formal Employee Acceptable Use Policy (AUP). This absence creates significant risk related to insider threats, data handling, and legal liability.
\end{itemize}

This report details these findings and provides actionable recommendations to mitigate the identified risks and strengthen the organization's overall security posture.

% --- 2. Organizational Information ---
\section*{2. Organizational Information}

The following information was used as the basis for this assessment. Due to the anonymized nature of the input data, placeholders have been used where necessary.

\begin{table}[h!]
\centering
\begin{tabular}{@{}ll@{}}
\toprule
\textbf{Attribute} & \textbf{Value} \\ \midrule
Organization Name & \textbf{[Organization Name]} \\
Primary Domain & \texttt{[Domain]} \\
External IP Scanned & \texttt{[Client IP]} \\
Target of Technical Scan & \texttt{[Target IP]} \\ \bottomrule
\end{tabular}
\caption{Client and Assessment Scope}
\end{table}

% --- 3. Security Control Review ---
\section*{3. Security Control Review}

An assessment of administrative and technical security controls was conducted via a questionnaire. The results indicate a strong foundation in user authentication and training but highlight a critical gap in governance policy.

\begin{table}[h!]
\centering
\begin{tabular}{@{}lc@{}}
\toprule
\textbf{Security Control Question} & \textbf{Status} \\ \midrule
Do you require MFA to access email? & \ding{51} \\
Do you require MFA to log into computers? & \ding{51} \\
Do you require MFA to access sensitive data systems? & \ding{51} \\
Does your organization have an employee acceptable use policy? & \textcolor{red}{\ding{55}} \\
Does your organization do security awareness training for new employees? & \ding{51} \\
Does your organization do training for all employees at least once per year? & \ding{51} \\ \bottomrule
\end{tabular}
\caption{Organizational Security Controls Questionnaire (\ding{51}=Yes, \ding{55}=No)}
\end{table}

\paragraph{Analysis:} The consistent implementation of MFA across key systems is an excellent security practice that significantly reduces the risk of account compromise. Similarly, the comprehensive security awareness training program is a vital defense against social engineering attacks. However, the absence of an Acceptable Use Policy is a major administrative control failure. An AUP is essential for setting clear expectations for employees, defining acceptable behavior, and establishing a legal framework for enforcing security policies.

% --- 4. Technical Scan Results ---
\section*{4. Technical Scan Results}

A network scan was performed to identify open ports and services exposed on the target system. The scan revealed a critical vulnerability.

\begin{table}[h!]
\centering
\begin{tabular}{@{}lllll@{}}
\toprule
\textbf{Port} & \textbf{State} & \textbf{Service} & \textbf{Product} & \textbf{Version} \\ \midrule
3306/tcp & open & mysql & MySQL & 5.7.33 \\ \bottomrule
\end{tabular}
\caption{Open Ports and Services Detected on \texttt{[Target IP]}}
\end{table}

\paragraph{Analysis:} The scan confirms the pre-existing risk of "Database Exposure." Port 3306 is open to the network, exposing a MySQL database server. The identified version, \textbf{MySQL 5.7.33}, is particularly alarming as its official support and security updates ended in October 2023. Running an End-of-Life service, especially one that is publicly accessible, exposes the organization to a wide range of known, unpatchable vulnerabilities that can be easily exploited by attackers to achieve remote code execution or a full data breach.

% --- 5. Consolidated Risk Assessment ---
\section*{5. Consolidated Risk Assessment}

The following table synthesizes findings from the security control review, technical scan, and pre-existing risk data into a prioritized list.

\begin{table}[h!]
\centering
\begin{tabular}{@{}p{0.3\linewidth}p{0.15\linewidth}p{0.45\linewidth}@{}}
\toprule
\textbf{Risk Name} & \textbf{Severity} & \textbf{Overview} \\ \midrule
\textbf{Unsupported Database Exposed to Internet} & \textbf{Critical (9.8)} & An unsupported, End-of-Life version of MySQL (5.7.33) is publicly accessible on port 3306. This service no longer receives security patches, making it highly susceptible to compromise. This finding confirms and elevates the pre-identified risk. \\
\addlinespace
\textbf{Missing Acceptable Use Policy (AUP)} & \textbf{High (7.2)} & The lack of a formal AUP creates ambiguity regarding the proper use of company assets and data. This increases the risk of insider threat (both malicious and accidental) and weakens the organization's legal standing in case of a policy violation. \\
\bottomrule
\end{tabular}
\caption{Summary of Identified Risks}
\end{table}

% --- 6. Recommendations ---
\section*{6. Recommendations}

The following actionable recommendations are provided to address the identified risks. They are prioritized based on severity and potential impact.

\subsection*{Priority 1: Immediate Actions (To be completed within 24-48 hours)}
\begin{enumerate}
    \item \textbf{Contain Database Exposure:} Immediately implement firewall rules to restrict all public access to TCP port 3306 on the affected server (\texttt{[Target IP]}). Access should only be permitted from trusted internal IP addresses or via a secure VPN connection. This is the most critical first step to mitigate the exposure risk.
\end{enumerate}

\subsection*{Priority 2: Short-Term Actions (To be completed within 30 days)}
\begin{enumerate}
    \item \textbf{Upgrade Unsupported Database:} Plan and execute an urgent upgrade of the MySQL 5.7.33 server to a currently supported version (e.g., MySQL 8.0 or higher). This will ensure the service receives critical security patches.
    \item \textbf{Develop and Implement an AUP:} Draft a comprehensive Acceptable Use Policy that clearly defines rules for employee use of company networks, systems, and data. The policy should be reviewed by legal counsel, approved by management, and communicated to all employees.
\end{enumerate}

\subsection*{Priority 3: Long-Term Actions (To be completed within 90 days)}
\begin{enumerate}
    \item \textbf{Implement Secure Remote Access:} If remote database administration is required, deploy a Virtual Private Network (VPN) solution. This will provide an encrypted and authenticated tunnel for administrative access, eliminating the need for any direct public exposure of database ports.
    \item \textbf{Enforce AUP Acknowledgment:} Require all current and new employees to formally read and acknowledge the new Acceptable Use Policy as a condition of being granted access to corporate resources.
    \item \textbf{Conduct Regular Vulnerability Scanning:} Implement a recurring, authenticated vulnerability scanning program for all external and internal assets to proactively identify and remediate similar issues in the future.
\end{enumerate}

\end{document}
```