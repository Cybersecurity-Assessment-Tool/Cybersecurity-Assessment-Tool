```latex
\documentclass[12pt]{article}

% Preamble: Required Packages
\usepackage[margin=1in]{geometry}
\usepackage{pifont} % For \ding
\usepackage{booktabs} % For professional tables
\usepackage{hyperref} % For clickable links and ToC
\usepackage{url} % For URL formatting
\usepackage{seqsplit} % To split long strings in tt font
\usepackage{xcolor} % For colors in tables
\usepackage{graphicx} % For potential logos or images

% --- Document Metadata ---
\title{Cybersecurity Posture Assessment Report}
\author{Cybersecurity Analysis Division}
\date{\today}

% --- Hyperref Setup ---
\hypersetup{
    colorlinks=true,
    linkcolor=blue,
    filecolor=magenta,      
    urlcolor=cyan,
    pdftitle={Cybersecurity Posture Assessment Report},
    pdfpagemode=FullScreen,
}

% --- Custom Commands ---
\newcommand{\yes}{\textcolor{green!70!black}{\ding{51}}} % Green checkmark
\newcommand{\no}{\textcolor{red}{\ding{55}}} % Red X

\begin{document}

\maketitle
\thispagestyle{empty}
\newpage

\tableofcontents
\newpage

% ==============================================================================
% SECTION 1: EXECUTIVE SUMMARY
% ==============================================================================
\section{Executive Summary}

This report provides a comprehensive analysis of the cybersecurity posture for \textbf{[Organization Name]}. The assessment is based on a correlation of network scan data, a security controls questionnaire, and a review of pre-existing documented risks.

The overall security posture requires immediate attention. The analysis revealed several critical and high-risk findings that expose the organization to significant threats, including account compromise, data breaches, and unauthorized system access.

Key findings include:
\begin{itemize}
    \item \textbf{Critical Pre-existing Vulnerability:} A documented risk, ``Localhost Exposed,'' with a CVSS score of 10.0, indicates a severe misconfiguration that must be remediated immediately.
    \item \textbf{Critical Gaps in Access Control:} Multi-Factor Authentication (MFA) is not enforced for accessing email or other sensitive data systems. This dramatically increases the risk of unauthorized access through credential theft or phishing attacks.
    \item \textbf{Exposed Network Services:} An external scan identified an open Secure Shell (SSH) port (22/TCP) on \texttt{[Target IP]}. If not properly configured, this service is a primary target for brute-force and credential-stuffing attacks.
    \item \textbf{Inadequate Security Training:} New employees do not receive security awareness training, creating a significant vulnerability to social engineering and phishing from their first day.
\end{itemize}

This report details these findings and provides actionable recommendations prioritized by severity to help \textbf{[Organization Name]} strengthen its defenses and mitigate the identified risks.

% ==============================================================================
% SECTION 2: ORGANIZATIONAL INFORMATION
% ==============================================================================
\section{Organizational Information}

This assessment was conducted for the following entity. The information provided was anonymized for the purpose of this report generation.

\begin{itemize}
    \item \textbf{Organization Name:} \textbf{[Organization Name]}
    \item \textbf{Primary Domain:} \texttt{[Domain]}
    \item \textbf{External IP Scanned:} \texttt{[Client IP]}
\end{itemize}

% ==============================================================================
% SECTION 3: SECURITY CONTROL REVIEW
% ==============================================================================
\section{Security Control Review}

The following table summarizes the organization's responses to a security controls questionnaire. Gaps in these controls often represent significant organizational risks that can be exploited by threat actors. Items marked with a \no{} indicate a deviation from security best practices and are addressed in the Risk Assessment section.

\begin{table}[h!]
\centering
\caption{Security Controls Questionnaire Analysis}
\label{tab:controls}
\begin{tabular}{p{0.6\textwidth} c p{0.2\textwidth}}
\toprule
\textbf{Control Question} & \textbf{Response} & \textbf{Assessment} \\
\midrule
Do you require MFA to access email? & \no & Critical Gap \\
Do you require MFA to log into computers? & \yes & Good Practice \\
Do you require MFA to access sensitive data systems? & \no & Critical Gap \\
Does your organization have an employee acceptable use policy? & \yes & Good Practice \\
Does your organization do security awareness training for new employees? & \no & High Risk \\
Does your organization do security awareness training for all employees at least once per year? & \yes & Good Practice \\
\bottomrule
\end{tabular}
\end{table}

% ==============================================================================
% SECTION 4: TECHNICAL SCAN RESULTS
% ==============================================================================
\section{Technical Scan Results}

An external network scan was performed to identify open ports and exposed services on the organization's public-facing infrastructure.

\subsection{Nmap Scan Findings}
\begin{itemize}
    \item \textbf{Target IP Address:} \texttt{[Target IP]}
    \item \textbf{Host Status:} Up
    \item \textbf{Scan Date:} \today
\end{itemize}

The scan revealed the following open port(s):

\begin{table}[h!]
\centering
\caption{Open Ports on \texttt{[Target IP]}}
\label{tab:ports}
\begin{tabular}{c c c l}
\toprule
\textbf{Port} & \textbf{State} & \textbf{Service} & \textbf{Notes} \\
\midrule
22/TCP & open & SSH & Secure Shell is used for remote administration. \\
& & & Exposing SSH to the internet is a common \\
& & & attack vector for brute-force attacks. \\
& & & No version information was available. \\
\bottomrule
\end{tabular}
\end{table}

% ==============================================================================
% SECTION 5: CONSOLIDATED RISK ASSESSMENT
% ==============================================================================
\section{Consolidated Risk Assessment}

The following table consolidates findings from the security control review, technical scan, and pre-existing risk documentation. Risks are prioritized based on their potential impact on the organization.

\begin{table}[h!]
\centering
\caption{Summary of Identified Risks}
\label{tab:risks}
\begin{tabular}{p{0.1\textwidth} p{0.25\textwidth} p{0.45\textwidth} c}
\toprule
\textbf{Risk ID} & \textbf{Risk Title} & \textbf{Description} & \textbf{Severity} \\
\midrule
RISK-001 & Localhost Exposed & A pre-existing documented vulnerability with a CVSS score of 10.0. Indicates a service intended for internal use only is exposed to the network. & \textbf{Critical} \\
\addlinespace
RISK-002 & No MFA on Email and Sensitive Systems & Lack of MFA on critical systems makes them highly vulnerable to compromise via phishing or credential theft, leading to potential data breaches. & \textbf{Critical} \\
\addlinespace
RISK-003 & Exposed SSH Service & The SSH port is open to the internet, creating a target for automated brute-force attacks and exploitation if any vulnerabilities exist in the service. & \textbf{High} \\
\addlinespace
RISK-004 & No Security Training for New Hires & New employees are not trained on security best practices, making them highly susceptible to social engineering and phishing attacks. & \textbf{High} \\
\bottomrule
\end{tabular}
\end{table}

% ==============================================================================
% SECTION 6: RECOMMENDATIONS
% ==============================================================================
\section{Recommendations}

The following actions are recommended to mitigate the identified risks. They are prioritized to address the most critical vulnerabilities first.

\subsection{Priority 1: Remediate "Localhost Exposed" (RISK-001)}
\begin{itemize}
    \item \textbf{Immediate Action:} Investigate the system at \texttt{[Affected System]} to identify the service that is improperly exposed.
    \item \textbf{Remediation:} Reconfigure the service's network binding to listen only on the local loopback interface (127.0.0.1 or ::1) or apply strict firewall rules to ensure it is not accessible from the external network.
\end{itemize}

\subsection{Priority 2: Enforce Multi-Factor Authentication (RISK-002)}
\begin{itemize}
    \item \textbf{Immediate Action:} Enable and enforce MFA for all users on the primary email system (e.g., Microsoft 365, Google Workspace).
    \item \textbf{Short-Term Action:} Conduct an inventory of all systems classified as containing sensitive data. Develop a plan to enforce MFA on all of these systems within the next 30-60 days.
\end{itemize}

\subsection{Priority 3: Secure Exposed SSH Service (RISK-003)}
\begin{itemize}
    \item \textbf{Immediate Action:} If remote access via SSH is not required, block port 22 at the network firewall.
    \item \textbf{Best Practices (If SSH is required):}
    \begin{enumerate}
        \item \textbf{Use Key-Based Authentication:} Disable password-based authentication and require the use of strong SSH keys.
        \item \textbf{Restrict Access:} Use firewall rules to limit SSH access to trusted IP addresses only.
        \item \textbf{Implement Intrusion Prevention:} Deploy a tool like Fail2Ban to automatically block IPs that exhibit malicious behavior (e.g., multiple failed login attempts).
        \item \textbf{Disable Root Login:} Prohibit direct login as the root user over SSH.
    \end{enumerate}
\end{itemize}

\subsection{Priority 4: Implement Onboarding Security Training (RISK-004)}
\begin{itemize}
    \item \textbf{Action:} Develop or procure a mandatory security awareness training module that is integrated into the new employee onboarding process.
    \item \textbf{Content:} The training should cover, at a minimum, phishing identification, password hygiene, acceptable use of company assets, and how to report a security incident.
\end{itemize}

\end{document}
```