```latex
\documentclass[12pt]{article}

% --- PACKAGE IMPORTS ---
\usepackage[margin=1in]{geometry}
\usepackage{pifont}          % For checkmarks and crosses (\ding)
\usepackage{booktabs}        % For professional-looking tables
\usepackage{hyperref}        % For clickable links and references
\usepackage{url}             % For formatting URLs
\usepackage{seqsplit}        % For splitting long text strings in \texttt
\usepackage{xcolor}          % For custom colors

% --- HYPERREF SETUP ---
\hypersetup{
    colorlinks=true,
    linkcolor=blue,
    filecolor=magenta,
    urlcolor=cyan,
    pdftitle={Cybersecurity Posture Assessment Report},
    pdfauthor={Cybersecurity Analyst},
}

% --- CUSTOM COMMANDS ---
\newcommand{\yes}{\ding{51}} % Green checkmark
\newcommand{\no}{\ding{55}}  % Red X

% --- DOCUMENT METADATA ---
\title{Cybersecurity Posture Assessment Report}
\author{\textbf{[Organization Name]}}
\date{\today}

% --- DOCUMENT START ---
\begin{document}

\maketitle

\section*{1. Executive Summary}

This report provides a comprehensive assessment of the cybersecurity posture for \textbf{[Organization Name]}, based on an analysis of network scan data, a security controls questionnaire, and a review of pre-existing risks.

The assessment reveals a mixed security posture. The organization has implemented several positive security controls, including mandatory Multi-Factor Authentication (MFA) for computer and sensitive data system access, and maintains a consistent security awareness training program for all employees. Furthermore, a technical scan indicates that a previously identified risk of an unencrypted web server on port 80 has likely been remediated, as the port was found to be closed.

However, two critical gaps were identified that significantly increase the organization's risk profile:
\begin{itemize}
    \item \textbf{Lack of MFA for Email:} The absence of MFA on email accounts (\texttt{[Domain]}) exposes the organization to a high risk of business email compromise, phishing, and account takeover.
    \item \textbf{Absence of an Acceptable Use Policy (AUP):} Without a formal AUP, there is no defined standard for employee behavior regarding company assets, which can lead to unintentional data exposure and insider threats.
\end{itemize}

This report details these findings and provides prioritized, actionable recommendations to mitigate the identified risks and strengthen the overall security posture.

\section*{2. Organizational Information}

The following information was used as the basis for this assessment.
\begin{center}
\begin{tabular}{ll}
\toprule
\textbf{Attribute} & \textbf{Value} \\
\midrule
Organization Name & \textbf{[Organization Name]} \\
Primary Email Domain & \texttt{[Domain]} \\
External IP Address Scanned & \texttt{[Client IP]} \\
\bottomrule
\end{tabular}
\end{center}

\section*{3. Security Control Review}

The following table summarizes the organization's self-reported security controls. "No" answers represent significant gaps that require immediate attention.

\begin{center}
\begin{tabular}{p{0.6\linewidth} c p{0.25\linewidth}}
\toprule
\textbf{Control Question} & \textbf{Response} & \textbf{Analyst Note} \\
\midrule
Do you require MFA to access email? & \no & \textbf{Critical Gap.} Increases risk of account compromise and phishing success. \\
\addlinespace
Do you require MFA to log into computers? & \yes & Best practice is being followed. \\
\addlinespace
Do you require MFA to access sensitive data systems? & \yes & Best practice is being followed. \\
\addlinespace
Does your organization have an employee acceptable use policy? & \no & \textbf{High Risk.} Creates ambiguity and increases risk of insider threat. \\
\addlinespace
Does your organization do security awareness training for new employees? & \yes & Positive control in place. \\
\addlinespace
Does your organization do security awareness training for all employees at least once per year? & \yes & Positive control in place. \\
\bottomrule
\end{tabular}
\end{center}

\section*{4. Technical Scan Results}

An external network scan was performed on the target IP address to identify accessible services. The scan results are summarized below.

\begin{itemize}
    \item \textbf{Target IP Address:} \texttt{[Target IP]}
    \item \textbf{Host Status:} UP
\end{itemize}

\begin{center}
\begin{tabular}{llll}
\toprule
\textbf{Port} & \textbf{State} & \textbf{Service} & \textbf{Finding} \\
\midrule
80/tcp & closed & http & The port is not accessible from the internet. \\
\bottomrule
\end{tabular}
\end{center}

\subsection*{Analysis of Technical Findings}
The scan indicates that port 80 (HTTP) is closed to external traffic. This is a positive finding, as it mitigates the risk associated with unencrypted web communication. This result contradicts a pre-existing risk entry (see Section 5), suggesting that remediation action has already been taken. No other open ports were discovered on the target during this scan.

\section*{5. Risk Assessment Summary}

The following table synthesizes findings from the security control review, technical scan, and pre-existing risk data into a prioritized list.

\begin{center}
\begin{tabular}{p{0.2\linewidth} p{0.55\linewidth} l}
\toprule
\textbf{Risk Name} & \textbf{Description} & \textbf{Severity} \\
\midrule
\textbf{Lack of MFA for Email} & Email accounts are secured only by a password, making them highly vulnerable to phishing, credential stuffing, and subsequent account takeover. This is the primary vector for business email compromise. & \textbf{Critical} \\
\addlinespace
\textbf{Missing Acceptable Use Policy (AUP)} & The absence of a formal policy defining acceptable use of company systems and data creates legal and security risks. It can lead to mishandling of sensitive information and inappropriate use of network resources. & \textbf{High} \\
\addlinespace
\textbf{Unencrypted Web Server (Port 80)} & A pre-existing risk noted that port 80 was open. Our technical scan found this port to be \textbf{closed}, indicating this risk has likely been remediated. & \textbf{Informational / Remediated} \\
\bottomrule
\end{tabular}
\end{center}

\section*{6. Recommendations}

To address the identified risks, we recommend the following actions, prioritized by severity.

\begin{enumerate}
    \item \textbf{[Critical] Enforce Multi-Factor Authentication (MFA) for Email:}
    \begin{itemize}
        \item \textbf{Action:} Immediately enable and enforce MFA for all user accounts on the \texttt{[Domain]} email system.
        \item \textbf{Justification:} This is the single most effective control to prevent unauthorized account access and mitigate the impact of compromised credentials. It directly protects against business email compromise.
    \end{itemize}
    \vspace{1em}
    \item \textbf{[High] Develop and Implement an Acceptable Use Policy (AUP):}
    \begin{itemize}
        \item \textbf{Action:} Draft a formal AUP that clearly outlines the rules for using company computers, networks, and data. Require all employees to read and formally acknowledge the policy.
        \item \textbf{Justification:} An AUP establishes a baseline for secure behavior, reduces organizational liability, and empowers the organization to enforce security standards.
    \end{itemize}
    \vspace{1em}
    \item \textbf{[Informational] Verify and Close Remediated Risk:}
    \begin{itemize}
        \item \textbf{Action:} Confirm with the network or system administration team that the closure of port 80 was an intentional act. Formally close the internal risk ticket associated with the "Unencrypted Web Server" vulnerability.
        \item \textbf{Justification:} This ensures that risk management documentation is accurate and reflects the current security posture.
    \end{itemize}
\end{enumerate}

\end{document}
```