```latex
\documentclass[12pt]{article}

% Preamble: Required Packages
\usepackage[margin=1in]{geometry}
\usepackage{pifont} % For checkmarks and crosses
\usepackage{booktabs} % For professional tables
\usepackage{hyperref} % For clickable links
\usepackage{url} % For formatting URLs
\usepackage{seqsplit} % For splitting long strings
\usepackage{graphicx}
\usepackage{xcolor}
\usepackage{fancyhdr}

% Define custom colors for severity
\definecolor{sev_critical}{HTML}{990000}
\definecolor{sev_high}{HTML}{DD0000}
\definecolor{sev_medium}{HTML}{FF8C00}
\definecolor{sev_low}{HTML}{F0E68C}

% Hyperref setup
\hypersetup{
    colorlinks=true,
    linkcolor=blue,
    filecolor=magenta,      
    urlcolor=cyan,
    pdftitle={Cybersecurity Posture Report},
    pdfpagemode=FullScreen,
}

% Header and Footer
\pagestyle{fancy}
\fancyhf{}
\lhead{Confidential Security Report}
\rhead{\textbf{[Organization Name]}}
\cfoot{\thepage}

\begin{document}

% --- Title Page ---
\begin{titlepage}
    \centering
    \vspace*{1cm}
    \Huge\textbf{Cybersecurity Posture Report}
    \vspace{1.5cm}
    \Large
    Prepared for:\\
    \vspace{0.5cm}
    \textbf{[Organization Name]}
    \vspace{2cm}
    \rule{0.8\textwidth}{0.4pt}
    \vspace{0.4cm}
    \large
    \textbf{Date of Report:} \today \\
    \textbf{Analysis Period:} \today
    \vspace{0.4cm}
    \rule{0.8\textwidth}{0.4pt}
    \vfill
    \normalsize
    This document contains sensitive and confidential information. Distribution is restricted to authorized personnel only.
\end{titlepage}

\tableofcontents
\newpage

% --- Section 1: Executive Summary ---
\section{Executive Summary}
This report provides a comprehensive analysis of the current cybersecurity posture for \textbf{[Organization Name]}, based on network scans, a security controls questionnaire, and a review of pre-existing risks.

The assessment reveals a \textbf{critical risk exposure} from externally facing services, combined with significant gaps in internal security controls. An externally accessible FTP server was discovered running a dangerously outdated version with a known remote code execution vulnerability (CVE-2011-2523) and configured to allow anonymous access. This presents a direct and immediate path for an attacker to compromise the network.

Furthermore, critical internal security controls are lacking. The absence of Multi-Factor Authentication (MFA) on employee computers and the failure to provide security awareness training to new hires create a high-risk internal environment. These gaps, when combined with the known issue of outdated Windows 7 workstations, significantly increase the likelihood of a successful cyberattack, such as a ransomware incident or data breach.

Urgent remediation is required. Recommendations are prioritized to address the most critical threats first, focusing on securing the network perimeter and strengthening internal access controls and employee security awareness.

% --- Section 2: Organizational Information ---
\section{Organizational Information}
This section outlines the basic information for the organization under review. The data provided was anonymized for this assessment.

\begin{table}[h!]
\centering
\begin{tabular}{@{}ll@{}}
\toprule
\textbf{Attribute} & \textbf{Value} \\ \midrule
Organization Name & \textbf{[Organization Name]} \\
Email Domain & \texttt{[Domain]} \\
External IP Address & \texttt{[Client IP]} \\
Target of Scan & \texttt{[Target IP]} \\ \bottomrule
\end{tabular}
\caption{Client and Assessment Scope Information.}
\end{table}

% --- Section 3: Security Control Review ---
\section{Security Control Review}
The following table summarizes the responses from the organizational security questionnaire. "No" answers indicate significant gaps in the security framework and are flagged as risks.

\begin{table}[h!]
\centering
\begin{tabular}{@{}p{0.5\linewidth}ccc@{}}
\toprule
\textbf{Control Question} & \textbf{Response} & \textbf{Assessment} \\ \midrule
Do you require MFA to access email? & Yes & \ding{51} \\
\addlinespace
Do you require MFA to log into computers? & No & \textcolor{red}{\ding{55}} \\
\addlinespace
Do you require MFA to access sensitive data systems? & Yes & \ding{51} \\
\addlinespace
Does your organization have an employee acceptable use policy? & Yes & \ding{51} \\
\addlinespace
Does your organization do security awareness training for new employees? & No & \textcolor{red}{\ding{55}} \\
\addlinespace
Does your organization do security awareness training for all employees at least once per year? & Yes & \ding{51} \\ \bottomrule
\end{tabular}
\caption{Security Controls Questionnaire Analysis.}
\end{table}

\subsection*{Analysis of Control Gaps}
\begin{itemize}
    \item \textbf{No MFA on Computer Logins:} This is a high-risk gap. If an employee's credentials are stolen (e.g., via phishing), an attacker can log into their computer without a second factor of authentication, gaining a significant foothold on the internal network.
    \item \textbf{No Security Training for New Employees:} New hires are a primary target for social engineering and phishing attacks. Failing to train them upon entry leaves the organization vulnerable, as they are not equipped to recognize and report threats.
\end{itemize}

% --- Section 4: Technical Scan Results ---
\section{Technical Scan Results}
An Nmap scan was conducted against the target IP address \texttt{[Target IP]}. The scan identified one open port with a critically vulnerable service.

\begin{table}[h!]
\centering
\begin{tabular}{@{}l l l p{0.4\linewidth}@{}}
\toprule
\textbf{Port/Proto} & \textbf{State} & \textbf{Service} & \textbf{Details \& Findings} \\ \midrule
21/tcp & OPEN & ftp & \textbf{Product:} vsftpd 2.3.4 \\
& & & \textbf{Finding 1 (Critical):} This version contains a known backdoor (CVE-2011-2523) which allows for remote command execution. \\
& & & \textbf{Finding 2 (Critical):} The server is configured to allow anonymous FTP login, permitting unauthorized access to files. \\
\bottomrule
\end{tabular}
\caption{Open Ports and Vulnerability Details.}
\end{table}

\subsection*{Analysis of Technical Findings}
The presence of an externally accessible FTP server running \textbf{vsftpd version 2.3.4} is a critical vulnerability. This specific version was trojanized, containing a backdoor that opens a command shell on port 6200 when a username containing a ":)" sequence is sent. This vulnerability allows an unauthenticated attacker to gain complete control over the server. The additional finding of \textbf{anonymous FTP login} exacerbates this risk, as it provides a simple method for attackers to potentially upload malicious files or exfiltrate sensitive data.

% --- Section 5: Consolidated Risk Assessment ---
\section{Consolidated Risk Assessment}
This section synthesizes findings from the technical scan, control review, and pre-existing risk data into a prioritized list.

\begin{table}[h!]
\centering
\begin{tabular}{@{}p{0.1\linewidth} p{0.25\linewidth} p{0.4\linewidth} l@{}}
\toprule
\textbf{Risk ID} & \textbf{Risk Title} & \textbf{Description} & \textbf{Severity} \\ \midrule
R-01 & \textbf{Vulnerable External FTP Server} & An internet-facing FTP server is running vsftpd 2.3.4, which has a critical backdoor vulnerability (CVE-2011-2523). Anonymous login is also enabled. & \colorbox{sev_critical}{\color{white} CRITICAL} \\
\addlinespace
R-02 & \textbf{No MFA on Workstation Logins} & Lack of MFA on computer logins allows an attacker with stolen credentials to easily access the internal network and move laterally. & \colorbox{sev_high}{\color{white} HIGH} \\
\addlinespace
R-03 & \textbf{No Onboarding Security Training} & New employees are not trained on security best practices, making them highly susceptible to phishing and social engineering attacks. & \colorbox{sev_high}{\color{white} HIGH} \\
\addlinespace
R-04 & \textbf{Outdated Windows 7 Workstations} & Pre-existing risk: Workstations are running Windows 7, an unsupported OS that no longer receives security updates, making them vulnerable to exploitation. & \colorbox{sev_medium}{\color{black} MEDIUM} \\
\bottomrule
\end{tabular}
\caption{Prioritized Risk Register.}
\end{table}

% --- Section 6: Recommendations ---
\section{Recommendations}
The following actions are recommended to mitigate the identified risks. They are prioritized based on severity.

\subsection*{Priority 1: Critical Risk (Immediate Action Required)}
\begin{description}
    \item[R-01: Vulnerable External FTP Server]
    \begin{itemize}
        \item \textbf{Immediate:} Take the FTP server offline immediately to prevent exploitation.
        \item \textbf{Short-Term:} If the FTP service is required, migrate to a secure file transfer protocol like SFTP (SSH File Transfer Protocol) or FTPS. Ensure the new service is fully patched and properly configured.
        \item \textbf{Configuration:} Disable anonymous access. Access should be restricted by IP address and require strong, unique credentials.
    \end{itemize}
\end{description}

\subsection*{Priority 2: High Risks (Action within 30 days)}
\begin{description}
    \item[R-02: No MFA on Workstation Logins]
    \begin{itemize}
        \item \textbf{Action:} Procure and deploy a Multi-Factor Authentication solution for all employee workstation and laptop logins. Solutions like Duo, Okta, or Windows Hello for Business should be considered.
        \item \textbf{Policy:} Update the access control policy to mandate MFA for all system logins.
    \end{itemize}
    \item[R-03: No Onboarding Security Training]
    \begin{itemize}
        \item \textbf{Action:} Develop or purchase a security awareness training module and integrate it into the mandatory onboarding process for all new employees and contractors.
        \item \textbf{Content:} The training must cover phishing, password security, acceptable use, and incident reporting procedures.
    \end{itemize}
\end{description}

\subsection*{Priority 3: Medium Risk (Action within 90 days)}
\begin{description}
    \item[R-04: Outdated Windows 7 Workstations]
    \begin{itemize}
        \item \textbf{Action:} Accelerate the existing project to upgrade or replace all remaining Windows 7 workstations with a modern, supported operating system (e.g., Windows 11).
        \item \textbf{Interim Control:} Until upgraded, ensure these machines are isolated on a separate network segment with restricted access to critical systems.
    \end{itemize}
\end{description}

\end{document}
```