```latex
\documentclass[12pt]{article}

% Preamble: Required Packages
\usepackage[margin=1in]{geometry}
\usepackage{pifont} % Required for checkmark and X symbols (\ding)
\usepackage{booktabs} % For professional-looking tables
\usepackage{hyperref} % For clickable links and document metadata
\usepackage{url} % For formatting URLs
\usepackage{seqsplit} % To split long strings without breaking words

% Document Metadata
\hypersetup{
    colorlinks=true,
    linkcolor=blue,
    filecolor=magenta,      
    urlcolor=cyan,
    pdftitle={Cybersecurity Assessment Report},
    pdfauthor={Cybersecurity Analyst Group},
    pdfsubject={Security Analysis},
    pdfkeywords={Cybersecurity, Nmap, Risk Assessment},
}

\begin{document}

% Title Page
\title{Cybersecurity Assessment Report \\ \large For \textbf{[Organization Name]}}
\author{Cybersecurity Analyst Group}
\date{November 22, 2025}
\maketitle

\newpage

% Table of Contents
\tableofcontents

\newpage

% --- Section 1: Executive Summary ---
\section{Executive Summary}
This report details the findings of a cybersecurity assessment conducted on November 22, 2025. The assessment combined a review of organizational security controls, an external network scan, and an analysis of known risks to evaluate the overall security posture of \textbf{[Organization Name]}.

The organization has implemented several important security controls, including Multi-Factor Authentication (MFA) for email and sensitive systems, as well as a robust security awareness training program. These measures provide a solid foundation for its security posture.

However, two high-risk vulnerabilities were identified that require immediate attention. 
\begin{enumerate}
    \item \textbf{Lack of Endpoint MFA:} The absence of mandatory MFA for computer logins represents a critical gap. A single compromised password could grant an attacker direct access to an employee's workstation and, potentially, the internal network.
    \item \textbf{Outdated Web Server Software:} The external-facing web server at \texttt{[Target IP]} is running Nginx version 1.18.0, a version that is significantly outdated. This exposes the organization to numerous publicly known vulnerabilities that could be exploited to compromise the server.
\end{enumerate}

While the organization's proactive stance on some controls is commendable, the identified risks are significant. We strongly recommend prioritizing the remediation steps outlined in Section \ref{sec:recommendations} to mitigate these threats and improve the overall security posture.

% --- Section 2: Organizational Information ---
\section{Organizational Information}
This section provides a summary of the organizational details used for this assessment. The information was provided by the client or discovered during the reconnaissance phase.

\begin{description}
    \item[Organization Name:] \textbf{[Organization Name]}
    \item[Primary Email Domain:] \texttt{[Domain]}
    \item[External IP Scanned:] \texttt{[Client IP]}
    \item[Assessment Date:] November 22, 2025
\end{description}

% --- Section 3: Security Control Review ---
\section{Security Control Review}
The following table summarizes the organization's responses to a security controls questionnaire. A green checkmark (\ding{51}) indicates a positive control is in place, while a red 'X' (\ding{55}) indicates a potential security gap.

\begin{table}[h!]
\centering
\caption{Security Controls Questionnaire Results}
\begin{tabular}{p{0.75\linewidth} c}
\toprule
\textbf{Control Question} & \textbf{Response} \\
\midrule
Do you require MFA to access email? & \ding{51} \\
Do you require MFA to log into computers? & \textbf{\color{red}\ding{55}} \\
Do you require MFA to access sensitive data systems? & \ding{51} \\
Does your organization have an employee acceptable use policy? & \ding{51} \\
Does your organization do security awareness training for new employees? & \ding{51} \\
Does your organization do security awareness training for all employees at least once per year? & \ding{51} \\
\bottomrule
\end{tabular}
\end{table}

\subsection*{Analysis}
The questionnaire reveals a critical gap in endpoint security. The lack of MFA on computer logins significantly increases the risk of unauthorized access resulting from stolen or weak credentials. An attacker who obtains a user's password could log in to their machine without any additional verification, gaining a foothold within the corporate network.

% --- Section 4: Technical Scan Results ---
\section{Technical Scan Results}
An external network scan was performed on the target IP address to identify open ports and exposed services.

\begin{description}
    \item[Target IP:] \texttt{[Target IP]}
    \item[Scan Date:] 2025-11-22T10:00:00Z
\end{description}

The following table details the services discovered during the scan.

\begin{table}[h!]
\centering
\caption{Open Ports and Services on \texttt{[Target IP]}}
\begin{tabular}{lllll}
\toprule
\textbf{Port} & \textbf{State} & \textbf{Service} & \textbf{Product} & \textbf{Version} \\
\midrule
443/tcp & open & https & nginx & 1.18.0 \\
\bottomrule
\end{tabular}
\end{table}

\subsection*{Analysis}
The scan identified an Nginx web server, version 1.18.0, accessible on port 443 (HTTPS). This version was released in April 2020 and is now considered outdated. The current stable version of Nginx has received numerous security patches and updates since 2020. Running this old version exposes the server to a wide range of publicly disclosed vulnerabilities (CVEs), which could lead to information disclosure, denial of service, or remote code execution.

% --- Section 5: Risk Assessment ---
\section{Risk Assessment}
This section synthesizes the findings from the security control review and technical scan. No pre-existing risks were provided for this assessment.

\begin{table}[h!]
\centering
\caption{Summary of Identified Risks}
\begin{tabular}{p{0.15\linewidth} p{0.6\linewidth} l}
\toprule
\textbf{Risk ID} & \textbf{Description} & \textbf{Severity} \\
\midrule
RISK-001 & \textbf{Lack of MFA for Endpoint Logon:} A compromised password allows direct, unauthorized access to employee computers and the internal network. & \textbf{High} \\
\addlinespace
RISK-002 & \textbf{Outdated Web Server Software (Nginx 1.18.0):} The public-facing web server is running an old version with known vulnerabilities, making it a target for automated exploits. & \textbf{High} \\
\bottomrule
\end{tabular}
\end{table}

% --- Section 6: Recommendations ---
\section{Recommendations}
\label{sec:recommendations}
Based on the identified risks, we provide the following actionable recommendations to enhance the security posture of \textbf{[Organization Name]}.

\subsection*{RISK-001: Lack of MFA for Endpoint Logon (High)}
\begin{description}
    \item[Action:] Implement a mandatory Multi-Factor Authentication (MFA) solution for all employee computer and laptop logins (Windows, macOS, Linux).
    \item[Details:] This control acts as a critical line of defense against credential theft. Solutions can include push notifications via an authenticator app (e.g., Microsoft Authenticator, Duo), hardware security keys (e.g., YubiKey), or biometrics. This significantly reduces the risk of unauthorized access.
\end{description}

\subsection*{RISK-002: Outdated Web Server Software (High)}
\begin{description}
    \item[Action:] Upgrade the Nginx web server on host \texttt{[Target IP]} from version 1.18.0 to the latest stable version.
    \item[Details:] A robust patch management program is essential for mitigating known vulnerabilities. Before deploying to production, the upgrade should be thoroughly tested in a staging environment to ensure full compatibility with existing applications. We also recommend subscribing to security advisories for all critical, public-facing software to ensure timely patching in the future.
\end{description}

\end{document}
```