```latex
\documentclass[12pt]{article}

% 1. DOCUMENT SETUP & PACKAGES
\usepackage[a4paper, margin=1in]{geometry}
\usepackage{pifont} % Required for checkmark and cross symbols (\ding)
\usepackage{booktabs} % For professional-looking tables
\usepackage{hyperref} % For hyperlinks, TOC, etc.
\usepackage{url} % For formatting URLs
\usepackage{seqsplit} % To split long strings in tt font, preventing overflow
\usepackage{fancyhdr} % For headers and footers
\usepackage{xcolor} % For colors

% --- Hyperref Setup ---
\hypersetup{
    colorlinks=true,
    linkcolor=blue,
    filecolor=magenta,      
    urlcolor=cyan,
    pdftitle={Cybersecurity Posture Assessment Report},
    pdfpagemode=FullScreen,
}

% --- Header and Footer Setup ---
\pagestyle{fancy}
\fancyhf{} % Clear all header and footer fields
\fancyhead[L]{\textbf{Cybersecurity Posture Assessment}}
\fancyhead[R]{\textbf{[Organization Name]}}
\fancyfoot[C]{\thepage}
\renewcommand{\headrulewidth}{0.4pt}
\renewcommand{\footrulewidth}{0.4pt}

% 2. DOCUMENT METADATA
\title{Cybersecurity Posture Assessment Report}
\author{Cybersecurity Analysis Division}
\date{\today}

% 3. DOCUMENT BODY
\begin{document}

\maketitle
\thispagestyle{empty} % No header/footer on the title page

\newpage
\tableofcontents
\newpage

% ==============================================================================
\section{Executive Summary}
% ==============================================================================
This report provides a comprehensive analysis of the cybersecurity posture for \textbf{[Organization Name]}. The assessment is based on a synthesis of data from an external network scan, a security controls questionnaire, and a review of pre-existing risks.

The analysis reveals a mixed security posture. While foundational controls like Multi-Factor Authentication (MFA) are implemented for email and computer access, critical gaps exist in protecting sensitive data systems. Furthermore, the external network scan identified a severely outdated and misconfigured FTP server, presenting an immediate and critical threat of unauthorized data access and system compromise. The lack of mandatory security awareness training for new employees exacerbates these risks by increasing the likelihood of human error.

Immediate remediation is required for the public-facing FTP server and the implementation of MFA for sensitive systems. Strategic improvements are also necessary for employee onboarding and endpoint management to build a more resilient security foundation.

% ==============================================================================
\section{Organizational Information}
% ==============================================================================
The following information was used as the basis for this assessment. As per the template mode, placeholders are used where data was not provided.

\begin{itemize}
    \item \textbf{Organization Name:} \textbf{[Organization Name]}
    \item \textbf{Primary Email Domain:} \texttt{[Domain]}
    \item \textbf{Assessed External IP:} \texttt{[Client IP]}
\end{itemize}

% ==============================================================================
\section{Security Control Review}
% ==============================================================================
The following table summarizes the organization's responses to the security controls questionnaire. Items marked with \ding{55} represent significant gaps in the security framework and are discussed in the Risk Assessment section.

\begin{table}[h!]
\centering
\caption{Security Controls Questionnaire Analysis}
\begin{tabular}{p{0.6\textwidth} c p{0.2\textwidth}}
\toprule
\textbf{Control Question} & \textbf{Response} & \textbf{Assessment} \\
\midrule
Do you require MFA to access email? & \ding{51} & Good Practice \\
Do you require MFA to log into computers? & \ding{51} & Good Practice \\
Do you require MFA to access sensitive data systems? & \textcolor{red}{\ding{55}} & \textbf{Critical Gap} \\
Does your organization have an employee acceptable use policy? & \ding{51} & Good Practice \\
Does your organization do security awareness training for new employees? & \textcolor{red}{\ding{55}} & \textbf{High Risk} \\
Does your organization do security awareness training for all employees at least once per year? & \ding{51} & Good Practice \\
\bottomrule
\end{tabular}
\end{table}

% ==============================================================================
\section{Technical Scan Results}
% ==============================================================================
An external network scan was performed against the target IP address \texttt{[Target IP]}. The scan identified one open port with a critically vulnerable service.

\begin{table}[h!]
\centering
\caption{Open Port Analysis}
\begin{tabular}{l l l l}
\toprule
\textbf{Port/Proto} & \textbf{State} & \textbf{Service/Version} & \textbf{Finding} \\
\midrule
21/tcp & OPEN & FTP & \textbf{Anonymous FTP Login Allowed} \\
       &      & \seqsplit{\texttt{vsftpd 2.3.4}} & Critically outdated version \\
       &      &                                 & (Known RCE vulnerability CVE-2011-2523) \\
\bottomrule
\end{tabular}
\end{table}

\subsection{Analysis of Findings}
The presence of an open FTP port is concerning, but two factors elevate this to a critical risk:
\begin{enumerate}
    \item \textbf{Anonymous Login:} This configuration allows any user on the internet to connect to the server without authentication, potentially accessing, downloading, or uploading files. This could lead to a severe data breach.
    \item \textbf{Outdated Version:} The running version, \texttt{vsftpd 2.3.4}, was released in 2011 and contains a well-known, critical backdoor vulnerability (CVE-2011-2523). An attacker can exploit this flaw to gain a command shell on the underlying server, leading to a complete system compromise.
\end{enumerate}

% ==============================================================================
\section{Consolidated Risk Assessment}
% ==============================================================================
The following table consolidates findings from all data sources into a prioritized list of risks.

\begin{table}[h!]
\centering
\caption{Summary of Identified Risks}
\begin{tabular}{p{0.15\textwidth} p{0.25\textwidth} p{0.5\textwidth}}
\toprule
\textbf{Severity} & \textbf{Risk Name} & \textbf{Description} \\
\midrule
\textbf{CRITICAL} & \textbf{Outdated \& Misconfigured FTP Server} & An internet-facing FTP server (\texttt{vsftpd 2.3.4}) allows anonymous login and is vulnerable to remote code execution (CVE-2011-2523). \\
\addlinespace
\textbf{CRITICAL} & \textbf{No MFA for Sensitive Systems} & Lack of MFA on systems storing sensitive data exposes the organization to significant risk of data breach from compromised credentials. \\
\addlinespace
\textbf{HIGH} & \textbf{Inadequate New Employee Onboarding} & New employees do not receive security awareness training, making them more susceptible to phishing and social engineering attacks. \\
\addlinespace
\textbf{MEDIUM} & \textbf{Outdated Windows Policy} & (Pre-existing risk) Workstations are running Windows 7, which is end-of-life and no longer receives security updates from Microsoft. \\
\bottomrule
\end{tabular}
\end{table}

% ==============================================================================
\section{Recommendations}
% ==============================================================================
The following actions are recommended to mitigate the identified risks. Recommendations are prioritized based on severity and impact.

\subsection{Immediate Actions (To Be Completed within 72 Hours)}
\begin{enumerate}
    \item \textbf{Remediate FTP Server Vulnerability:}
    \begin{itemize}
        \item If the FTP service is not essential, \textbf{disable and block port 21} at the firewall immediately.
        \item If the service is essential, immediately \textbf{disable anonymous login}.
        \item Plan to \textbf{upgrade or replace the vsftpd service} with a modern, secure file transfer solution (e.g., SFTP) within the next 30 days.
    \end{itemize}
    \item \textbf{Implement MFA for Sensitive Systems:}
    \begin{itemize}
        \item Identify all systems classified as containing sensitive data.
        \item Enforce a mandatory MFA policy for all user and administrative access to these systems.
    \end{itemize}
\end{enumerate}

\subsection{Short-Term Actions (To Be Completed within 90 Days)}
\begin{enumerate}
    \item \textbf{Enhance Employee Onboarding Process:}
    \begin{itemize}
        \item Develop a mandatory security awareness training module for all new hires.
        \item This training must be completed before access to sensitive systems is granted.
        \item Topics should include phishing identification, acceptable use policies, and data handling procedures.
    \end{itemize}
    \item \textbf{Accelerate Endpoint Modernization:}
    \begin{itemize}
        \item Continue with the existing recommendation to upgrade all Windows 7 workstations to a supported operating system (e.g., Windows 10/11).
        \item Prioritize the upgrade for users who handle sensitive information.
    \end{itemize}
\end{enumerate}

\end{document}
```