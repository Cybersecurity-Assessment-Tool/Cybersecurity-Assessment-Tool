```latex
\documentclass[12pt]{article}

% --- PACKAGES ---
\usepackage[margin=1in]{geometry}
\usepackage{pifont} % For checkmarks and crosses
\usepackage{booktabs} % For professional tables
\usepackage{hyperref} % For hyperlinks
\usepackage{url} % For URL formatting
\usepackage{seqsplit} % For splitting long text strings
\usepackage{graphicx}
\usepackage{xcolor}

% --- DOCUMENT METADATA ---
\title{Cybersecurity Posture Assessment Report}
\author{Cybersecurity Analysis Division}
\date{\today}

% --- HYPERREF SETUP ---
\hypersetup{
    colorlinks=true,
    linkcolor=blue,
    filecolor=magenta,      
    urlcolor=cyan,
    pdftitle={Cybersecurity Posture Assessment Report},
    pdfpagemode=FullScreen,
}

\begin{document}

\maketitle
\thispagestyle{empty}
\newpage

\tableofcontents
\thispagestyle{empty}
\newpage

\setcounter{page}{1}

% ==============================================================================
% SECTION 1: EXECUTIVE SUMMARY
% ==============================================================================
\section{Executive Summary}

This report details the findings of a cybersecurity posture assessment for \textbf{[Organization Name]}. The evaluation is based on the synthesis of three data sources: an external network scan, a security controls questionnaire, and a list of pre-existing documented risks.

The organization demonstrates a foundational security posture by implementing Multi-Factor Authentication (MFA) for computer and sensitive system access, and by maintaining an acceptable use policy. However, significant and high-impact gaps were identified that expose the organization to substantial risk.

\textbf{Key Findings:}
\begin{itemize}
    \item \textbf{Critical Risk:} The lack of mandatory MFA for email access (\texttt{[Domain]}) presents a critical vulnerability. This significantly increases the risk of Business Email Compromise (BEC), phishing success, and subsequent unauthorized access to sensitive data.
    \item \textbf{High Risk:} Security awareness training is not conducted annually for all employees. This gap perpetuates a higher risk of human error-related security incidents, as security knowledge degrades over time without reinforcement.
    \item \textbf{Positive Finding:} The external network scan of the target IP address (\texttt{[Target IP]}) did not identify any open ports. This indicates a strong network perimeter configuration for the scanned asset.
    \item \textbf{Risk Discrepancy:} A pre-existing risk noted an open port 80 (Unencrypted Web Server). This was not confirmed by our recent scan, suggesting the risk may have been remediated or applies to an asset outside the current scan's scope. Further investigation is recommended to formally close this risk.
\end{itemize}

Immediate remediation should focus on enforcing MFA for email and establishing a mandatory, recurring security awareness training program.

% ==============================================================================
% SECTION 2: ORGANIZATIONAL INFORMATION
% ==============================================================================
\section{Organizational Information}

The following details were used as the basis for this assessment. Due to the anonymized nature of the provided data, placeholders have been used.

\begin{tabular}{@{}ll}
    \toprule
    \textbf{Attribute} & \textbf{Value} \\
    \midrule
    Organization Name & \textbf{[Organization Name]} \\
    Primary Email Domain & \texttt{[Domain]} \\
    Scanned External IP & \texttt{[Client IP]} \\
    \bottomrule
\end{tabular}

% ==============================================================================
% SECTION 3: SECURITY CONTROL REVIEW
% ==============================================================================
\section{Security Control Review}

The following table summarizes the organization's responses to a security controls questionnaire. A green checkmark (\ding{51}) indicates a positive control is in place, while a red cross (\ding{55}) indicates a control gap that introduces risk.

\begin{table}[h!]
\centering
\begin{tabular}{@{}lc@{}}
    \toprule
    \textbf{Control Question} & \textbf{Response} \\
    \midrule
    Do you require MFA to access email? & \textcolor{red}{\ding{55}} \\
    Do you require MFA to log into computers? & \textcolor{green}{\ding{51}} \\
    Do you require MFA to access sensitive data systems? & \textcolor{green}{\ding{51}} \\
    Does your organization have an employee acceptable use policy? & \textcolor{green}{\ding{51}} \\
    Does your organization do security awareness training for new employees? & \textcolor{green}{\ding{51}} \\
    Does your organization do security awareness training for all employees at least once per year? & \textcolor{red}{\ding{55}} \\
    \bottomrule
\end{tabular}
\caption{Security Controls Questionnaire Results}
\end{table}

\subsection*{Analysis of Control Gaps}
The two identified gaps are significant. The absence of MFA on email is a critical oversight, as email accounts are a primary target for attackers. The lack of annual, recurring security training for all staff members means that the "human firewall" is not being maintained, increasing susceptibility to social engineering and phishing attacks.

% ==============================================================================
% SECTION 4: TECHNICAL SCAN RESULTS
% ==============================================================================
\section{Technical Scan Results}

An external network scan was performed using Nmap to identify open ports and services on the designated client IP address.

\begin{itemize}
    \item \textbf{Target IP:} \texttt{[Target IP]}
    \item \textbf{Host Status:} Up
    \item \textbf{Scan Date:} \today
\end{itemize}

The scan results are detailed in the table below.

\begin{table}[h!]
\centering
\begin{tabular}{@{}llll@{}}
    \toprule
    \textbf{Port} & \textbf{State} & \textbf{Service} & \textbf{Version} \\
    \midrule
    80 & closed & http & N/A \\
    \bottomrule
\end{tabular}
\caption{Nmap Port Scan Results for \texttt{[Target IP]}}
\end{table}

\subsection*{Analysis of Technical Findings}
The scan revealed no open ports on the target host. A "closed" state indicates that the port is accessible (it receives and responds to Nmap probe packets), but there is no application listening on it. This is a secure configuration and presents no immediate vulnerabilities from this scan.

% ==============================================================================
% SECTION 5: RISK ASSESSMENT
% ==============================================================================
\section{Risk Assessment}

This section synthesizes findings from the security control review, technical scan, and pre-existing risk data into a consolidated list of current risks.

\begin{table}[h!]
\centering
\begin{tabular}{@{}lp{6cm}l@{}}
    \toprule
    \textbf{Risk ID} & \textbf{Risk Name \& Description} & \textbf{Severity} \\
    \midrule
    RISK-001 & \textbf{Lack of MFA for Email Access} \newline User email accounts are not protected by MFA, making them highly susceptible to compromise via phishing or credential stuffing attacks. & \textbf{Critical} \\
    \addlinespace
    RISK-002 & \textbf{Inadequate Security Awareness Training} \newline Security training is not conducted annually for all employees, increasing the risk of human error leading to security incidents. & \textbf{High} \\
    \addlinespace
    RISK-003 & \textbf{Unencrypted Web Server (Pre-existing)} \newline A pre-existing risk indicates Port 80 was open. \textit{Note: The current scan found this port closed. This risk may be remediated or may apply to a different asset.} & Medium (5.0) \\
    \bottomrule
\end{tabular}
\caption{Consolidated Risk Summary}
\end{table}

% ==============================================================================
% SECTION 6: RECOMMENDATIONS
% ==============================================================================
\section{Recommendations}

The following actionable recommendations are provided to mitigate the identified risks and improve the overall security posture of \textbf{[Organization Name]}.

\subsection{RISK-001: Lack of MFA for Email Access (Critical)}
\begin{itemize}
    \item \textbf{Immediate Action:} Procure and enforce a mandatory MFA solution for all user access to the email system (\texttt{[Domain]}). This should be the highest priority remediation task.
    \item \textbf{Justification:} Email is the gateway to an organization. A compromised email account can lead to data breaches, financial fraud (BEC), and further network intrusion. MFA is the single most effective control to prevent account takeovers.
\end{itemize}

\subsection{RISK-002: Inadequate Security Awareness Training (High)}
\begin{itemize}
    \item \textbf{Strategic Action:} Establish a formal, mandatory security awareness training program for all employees to be completed annually. Supplement this with periodic phishing simulation campaigns to measure effectiveness.
    \item \textbf{Justification:} A well-trained workforce is a critical layer of defense. Regular training ensures that security remains top-of-mind, reduces the likelihood of successful social engineering attacks, and fosters a strong security culture.
\end{itemize}

\subsection{RISK-003: Unencrypted Web Server (Medium)}
\begin{itemize}
    \item \textbf{Verification Action:} Conduct an internal review to confirm if the service on Port 80 has been decommissioned. If so, update the risk register to formally close this finding. If the risk pertains to another asset, a new scan should be scheduled for that target.
    \item \textbf{Justification:} While our scan showed no immediate threat, it is crucial to maintain an accurate risk register. Validating and formally closing remediated risks ensures that security resources are focused on current, active threats.
\end{itemize}

\end{document}
```