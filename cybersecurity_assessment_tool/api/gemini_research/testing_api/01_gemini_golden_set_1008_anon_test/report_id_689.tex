```latex
\documentclass[12pt]{article}

% ----------------------------------------------------------------------
% METADATA & PACKAGES
% ----------------------------------------------------------------------
\usepackage[margin=1in]{geometry}
\usepackage{pifont} % For checkmarks and crosses
\usepackage{booktabs} % For professional tables
\usepackage[hidelinks]{hyperref} % For clickable links without boxes
\usepackage{url} % For URL formatting
\usepackage{seqsplit} % For splitting long strings in tt font
\usepackage{graphicx}
\usepackage{xcolor}

\definecolor{darkblue}{rgb}{0.0, 0.0, 0.55}
\definecolor{darkred}{rgb}{0.55, 0.0, 0.0}

\newcommand{\yes}{\ding{51}}
\newcommand{\no}{\ding{55}}

\hypersetup{
    colorlinks=true,
    linkcolor=darkblue,
    filecolor=magenta,      
    urlcolor=darkblue,
    pdftitle={Cybersecurity Posture Report},
    pdfpagemode=FullScreen,
}

% ----------------------------------------------------------------------
% DOCUMENT START
% ----------------------------------------------------------------------
\begin{document}

% ----------------------------------------------------------------------
% TITLE PAGE
% ----------------------------------------------------------------------
\begin{titlepage}
    \centering
    \vspace*{1cm}
    \Huge\textbf{Cybersecurity Posture Report}
    \vspace{1.5cm}
    \Large
    \textbf{Prepared for:}\\
    \vspace{0.5cm}
    \textbf{[Organization Name]}
    \vspace{2cm}
    \rule{\linewidth}{0.5mm}
    \vspace{0.5cm}
    \large
    \textbf{Date of Report:} \today \\
    \textbf{Analysis Scope:} External Network Scan \& Internal Policy Review
    \vspace{0.5cm}
    \rule{\linewidth}{0.5mm}
    \vfill
    \large
    \textit{This report contains sensitive information and should be handled with the utmost confidentiality.}
\end{titlepage}

\tableofcontents
\newpage

% ----------------------------------------------------------------------
% 1. EXECUTIVE SUMMARY
% ----------------------------------------------------------------------
\section*{1. Executive Summary}

This report provides a comprehensive analysis of the cybersecurity posture for \textbf{[Organization Name]}, based on an external network scan, a review of existing risks, and an assessment of internal security controls.

The analysis identified a \textbf{critical risk}: the direct exposure of a Remote Desktop Protocol (RDP) service on port 3389 to the public internet. This configuration is a primary target for ransomware attacks and unauthorized access attempts.

Furthermore, significant gaps were identified in foundational security controls. The lack of mandatory Multi-Factor Authentication (MFA) for email and sensitive data systems drastically increases the risk of account compromise. These policy-level weaknesses, combined with the technical vulnerability, create a high-risk environment that requires immediate attention.

Key recommendations focus on immediately isolating the exposed RDP service, implementing comprehensive MFA, and formalizing security policies and training. Prioritizing these actions will substantially improve the organization's resilience against common and impactful cyber threats.

% ----------------------------------------------------------------------
% 2. ORGANIZATIONAL INFORMATION
% ----------------------------------------------------------------------
\section*{2. Organizational Information}

The following details were used as the basis for this assessment. The data has been anonymized as per the reporting template.

\begin{itemize}
    \item \textbf{Organization Name:} \textbf{[Organization Name]}
    \item \textbf{Primary Email Domain:} \texttt{[Domain]}
    \item \textbf{External IP Address Scanned:} \texttt{[Client IP]}
\end{itemize}

% ----------------------------------------------------------------------
% 3. SECURITY CONTROL REVIEW (QUESTIONNAIRE)
% ----------------------------------------------------------------------
\section*{3. Security Control Review}

An assessment of internal security policies and procedures was conducted via a standardized questionnaire. The results below highlight critical gaps in the organization's security framework. "No" answers indicate a deviation from security best practices.

\begin{table}[h!]
\centering
\caption{Security Controls Questionnaire Results}
\begin{tabular}{p{0.6\linewidth} c c}
\toprule
\textbf{Control Question} & \textbf{Status} & \textbf{Assessment} \\
\midrule
Do you require MFA to access email? & \no & \textcolor{darkred}{\textbf{Critical Gap}} \\
Do you require MFA to log into computers? & \yes & Met \\
Do you require MFA to access sensitive data systems? & \no & \textcolor{darkred}{\textbf{Critical Gap}} \\
Does your organization have an employee acceptable use policy? & \no & \textcolor{darkred}{High Risk} \\
Does your organization do security awareness training for new employees? & \no & \textcolor{darkred}{High Risk} \\
Does your organization do security awareness training for all employees at least once per year? & \yes & Met \\
\bottomrule
\end{tabular}
\end{table}

The absence of MFA on email and sensitive systems is a primary concern. Email is a common entry point for attackers, and its compromise can lead to widespread system access. The lack of formal policies and new-hire training indicates a reactive, rather than proactive, approach to security culture.

% ----------------------------------------------------------------------
% 4. TECHNICAL SCAN RESULTS
% ----------------------------------------------------------------------
\section*{4. Technical Scan Results}

An external network scan was performed against the target IP address to identify exposed services.

\begin{itemize}
    \item \textbf{Target IP Address:} \texttt{[Target IP]}
    \item \textbf{Scan Status:} Host is Up
\end{itemize}

\begin{table}[h!]
\centering
\caption{Open Ports Detected on \texttt{[Target IP]}}
\begin{tabular}{c c p{0.5\linewidth}}
\toprule
\textbf{Port} & \textbf{State} & \textbf{Service Name / Description} \\
\midrule
3389/tcp & Open & \texttt{ms-wbt-server} (Microsoft Remote Desktop Protocol) \\
\bottomrule
\end{tabular}
\end{table}

\subsection*{Analysis of Findings}
The scan confirms that port \textbf{3389/tcp} is open to the internet. This port is used for Microsoft's Remote Desktop Protocol (RDP), which allows for remote administrative control of a Windows system. Exposing RDP directly to the public internet is extremely dangerous and is a leading cause of ransomware infections and security breaches. Attackers constantly scan the internet for open RDP ports to exploit via brute-force password attacks or known vulnerabilities.

% ----------------------------------------------------------------------
% 5. CORRELATED RISK ASSESSMENT
% ----------------------------------------------------------------------
\section*{5. Correlated Risk Assessment}

This section synthesizes the findings from the security control review, the technical scan, and pre-existing risk data into a prioritized list of security risks.

\begin{table}[h!]
\centering
\caption{Summary of Identified Risks}
\begin{tabular}{p{0.25\linewidth} p{0.5\linewidth} c}
\toprule
\textbf{Risk Title} & \textbf{Description} & \textbf{Severity} \\
\midrule
\textbf{RDP Exposure} & Port 3389 is open to the internet, allowing attackers to attempt unauthorized access. This finding confirms a known high-severity risk. & \textbf{Critical} \\
\addlinespace
\textbf{Lack of MFA} & No MFA is enforced on email or sensitive systems. A compromised password could grant an attacker direct access to critical assets. & \textbf{Critical} \\
\addlinespace
\textbf{Weak Policy Framework} & The absence of an Acceptable Use Policy and security training for new hires creates an environment where employees are more likely to engage in risky behavior. & High \\
\bottomrule
\end{tabular}
\end{table}

% ----------------------------------------------------------------------
% 6. RECOMMENDATIONS
% ----------------------------------------------------------------------
\section*{6. Recommendations}

The following actions are recommended to mitigate the identified risks. They are prioritized based on severity and potential impact.

\subsection*{Priority 1: Immediate Actions (Within 24 Hours)}
\begin{enumerate}
    \item \textbf{Isolate Exposed RDP Service:} Immediately block all inbound traffic to port 3389 on the external firewall for IP address \texttt{[Target IP]}.
    \item \textbf{Implement Secure Remote Access:} If remote access is required, deploy a Virtual Private Network (VPN) solution with mandatory Multi-Factor Authentication. Access to RDP should only be permitted through the VPN tunnel.
\end{enumerate}

\subsection*{Priority 2: High-Impact Actions (Within 30 Days)}
\begin{enumerate}
    \item \textbf{Enforce MFA on All External Services:} Enable and enforce MFA for all users on email (e.g., Office 365, Google Workspace) and any other cloud or sensitive data systems.
    \item \textbf{Conduct a Password Audit:} Force a password reset for all user accounts, especially for any accounts with administrative privileges. Ensure strong password policies are in place.
\end{enumerate}

\subsection*{Priority 3: Foundational Improvements (Within 90 Days)}
\begin{enumerate}
    \item \textbf{Develop an Acceptable Use Policy (AUP):} Create and distribute a formal AUP that outlines the rules for using company IT assets, including data handling, internet usage, and password management.
    \item \textbf{Implement New-Hire Security Training:} Develop a mandatory security awareness training module that all new employees must complete as part of their onboarding process.
\end{enumerate}

% ----------------------------------------------------------------------
% DOCUMENT END
% ----------------------------------------------------------------------
\end{document}
```