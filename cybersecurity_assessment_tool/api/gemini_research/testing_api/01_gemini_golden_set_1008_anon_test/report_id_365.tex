```latex
\documentclass[12pt]{article}

% === PACKAGES ===
\usepackage[margin=1in]{geometry}
\usepackage{pifont}         % For symbols like checkmarks and crosses
\usepackage{booktabs}       % For professional-looking tables
\usepackage{xcolor}         % For color definitions
\usepackage{hyperref}       % For hyperlinks and document metadata
\usepackage{url}            % For formatting URLs
\usepackage{seqsplit}       % For splitting long, unbreakable strings
\usepackage{graphicx}
\usepackage{fancyhdr}

% === COMMAND DEFINITIONS ===
\newcommand{\yes}{\ding{51}}
\newcommand{\no}{\textcolor{red}{\ding{55}}}

% === HYPERREF SETUP ===
\definecolor{darkblue}{rgb}{0.0, 0.0, 0.55}
\hypersetup{
    colorlinks=true,
    linkcolor=darkblue,
    filecolor=darkblue,
    urlcolor=darkblue,
    citecolor=darkblue,
    pdftitle={Cybersecurity Posture Assessment Report},
    pdfauthor={Cybersecurity Analysis Division},
    pdfsubject={Security Assessment}
}

% === DOCUMENT START ===
\begin{document}

% === TITLE PAGE ===
\title{Cybersecurity Posture Assessment Report}
\author{Cybersecurity Analysis Division}
\date{\today}
\maketitle
\thispagestyle{empty}
\newpage

% === TABLE OF CONTENTS ===
\tableofcontents
\newpage

% === EXECUTIVE SUMMARY ===
\section*{Executive Summary}
This report provides a cybersecurity assessment for \textbf{[Organization Name]}. The analysis combines a review of organizational security controls, an external network scan, and pre-existing risk data.

The assessment reveals a \textbf{Critical} overall risk posture. A primary vulnerability, the direct exposure of Remote Desktop Protocol (RDP) to the public internet, was confirmed on the external IP address \texttt{[Client IP]}. This vulnerability is significantly exacerbated by critical gaps in security controls, including the absence of Multi-Factor Authentication (MFA) on sensitive systems and a lack of a formal security awareness training program. These combined factors create a high probability of a security breach, such as a ransomware attack or data exfiltration. Immediate remediation is strongly advised.

% === ORGANIZATIONAL INFORMATION ===
\section{Organizational Information}
This section details the scope and target of the assessment based on the provided data.

\begin{tabular}{@{}ll}
\toprule
\textbf{Attribute} & \textbf{Value} \\
\midrule
Organization Name & \textbf{[Organization Name]} \\
Primary Domain & \texttt{[Domain]} \\
External IP Scanned & \texttt{[Client IP]} \\
\bottomrule
\end{tabular}

% === SECURITY CONTROL REVIEW ===
\section{Security Control Review}
The following table summarizes the organization's current security controls based on the provided questionnaire. Gaps identified with a red cross (\no) represent significant areas of risk that require attention.

\begin{center}
\begin{tabular}{p{0.7\linewidth} c}
\toprule
\textbf{Control Question} & \textbf{Status} \\
\midrule
Do you require MFA to access email? & \yes \\
Do you require MFA to log into computers? & \yes \\
Do you require MFA to access sensitive data systems? & \no \\
Does your organization have an employee acceptable use policy? & \yes \\
Does your organization do security awareness training for new employees? & \no \\
Does your organization do security awareness training for all employees at least once per year? & \no \\
\bottomrule
\end{tabular}
\end{center}

\subsection*{Analysis of Control Gaps}
The review highlights two critical areas of concern:
\begin{itemize}
    \item \textbf{Lack of MFA on Sensitive Systems:} While MFA is enforced for email and computer logins, its absence on systems containing sensitive data represents a critical control failure. A single compromised credential could lead to a significant data breach.
    \item \textbf{No Security Awareness Training:} The complete absence of a security awareness training program for both new and existing employees leaves the organization highly vulnerable to social engineering attacks, such as phishing, which are a primary vector for initial compromise.
\end{itemize}

% === TECHNICAL SCAN RESULTS ===
\section{Technical Scan Results}
An external network scan was performed on the target IP address to identify open ports and exposed services.
\subsection*{Target}
The scan was performed against the IP address: \texttt{[Target IP]}

\subsection*{Open Ports}
The following table details the services found to be accessible from the public internet.

\begin{center}
\begin{tabular}{l l l l}
\toprule
\textbf{Port} & \textbf{State} & \textbf{Service} & \textbf{Notes} \\
\midrule
3389/tcp & open & ms-wbt-server & Remote Desktop Protocol (RDP) \\
\bottomrule
\end{tabular}
\end{center}

\subsection*{Analysis of Technical Findings}
The scan confirmed that port 3389, used for Microsoft's Remote Desktop Protocol (RDP), is open to the public internet. RDP is a common target for brute-force attacks and exploitation of vulnerabilities (e.g., BlueKeep). Exposing RDP directly is a high-risk practice and is strongly discouraged by cybersecurity authorities worldwide. This finding directly corroborates the pre-existing risk identified as "RDP Exposure."

% === SYNTHESIZED RISK ASSESSMENT ===
\section{Synthesized Risk Assessment}
The following table correlates findings from the security control review, technical scan, and pre-existing risk data to provide a unified view of the organization's top security risks.

\begin{center}
\begin{tabular}{p{0.25\linewidth} p{0.5\linewidth} l}
\toprule
\textbf{Risk Name} & \textbf{Description} & \textbf{Severity} \\
\midrule
\textbf{Exposed Remote Desktop Service} & Port 3389 (RDP) is open to the internet, allowing attackers to attempt brute-force logins or exploit RDP vulnerabilities. This is confirmed by both the network scan and existing risk data. & \textbf{Critical} \\
\addlinespace
\textbf{Lack of MFA on Sensitive Systems} & The absence of MFA on critical data systems means a compromised password is all an attacker needs to access and exfiltrate sensitive information. This risk is magnified by the exposed RDP. & \textbf{Critical} \\
\addlinespace
\textbf{Inadequate Security Awareness Training} & Employees are not trained to recognize or report phishing and other social engineering attacks. This increases the likelihood of credential compromise, which could be used to access the exposed RDP or other services. & \textbf{High} \\
\bottomrule
\end{tabular}
\end{center}

% === RECOMMENDATIONS ===
\section{Recommendations}
Based on the identified risks, the following actions are recommended to improve the organization's security posture. Recommendations are prioritized by urgency.

\subsection*{Immediate Actions (Urgency: Critical)}
\begin{enumerate}
    \item \textbf{Remediate RDP Exposure:} Immediately close port 3389 on the external firewall for \texttt{[Client IP]}. If remote access is required, implement a Virtual Private Network (VPN) with Multi-Factor Authentication. Do not expose RDP directly to the internet.
\end{enumerate}

\subsection*{High-Priority Actions (Urgency: High)}
\begin{enumerate}
    \setcounter{enumi}{1} % Continue numbering
    \item \textbf{Implement MFA on All Sensitive Systems:} Deploy MFA on all applications and systems that store, process, or transmit sensitive data. This is a critical compensating control that protects against credential theft.
    \item \textbf{Establish a Security Awareness Program:} Implement a mandatory security awareness training program for all employees. This should be part of the new-hire onboarding process and conducted annually for all staff. The training should cover phishing, password hygiene, and acceptable use policies.
\end{enumerate}

\subsection*{Long-Term Recommendations}
\begin{enumerate}
    \setcounter{enumi}{3} % Continue numbering
    \item \textbf{Implement a Vulnerability Management Program:} Regularly scan internal and external networks for vulnerabilities to proactively identify and remediate risks before they can be exploited.
    \item \textbf{Review and Update Policies:} Ensure the Acceptable Use Policy is reviewed and updated annually and that employees attest to it.
\end{enumerate}

\end{document}
```