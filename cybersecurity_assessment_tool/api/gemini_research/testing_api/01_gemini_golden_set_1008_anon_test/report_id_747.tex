```latex
\documentclass[12pt, a4paper]{article}

% --- PACKAGE IMPORTS ---
\usepackage[margin=1in]{geometry} % Set page margins
\usepackage{pifont}                 % For checkmarks and crosses (\ding)
\usepackage{booktabs}               % For professional tables (\toprule, \midrule, \bottomrule)
\usepackage{hyperref}               % For hyperlinks and document metadata
\usepackage{url}                    % For typesetting URLs
\usepackage{seqsplit}               % For splitting long strings without spaces
\usepackage{graphicx}               % For including logos (optional)
\usepackage{xcolor}                 % For custom colors

% --- DOCUMENT METADATA ---
\hypersetup{
    colorlinks=true,
    linkcolor=blue,
    filecolor=magenta,      
    urlcolor=cyan,
    pdftitle={Cybersecurity Posture Assessment Report},
    pdfauthor={Cybersecurity Analyst},
    pdfsubject={Security Analysis},
    pdfkeywords={Security, Nmap, Risk Assessment},
}

% --- TITLE SECTION ---
\title{
    \vspace{-2cm} % Adjust vertical space
    \rule{\textwidth}{2pt} \\ [0.5cm]
    \textbf{Cybersecurity Posture Assessment Report} \\ [0.2cm]
    \rule{\textwidth}{1pt}
}
\author{Cybersecurity Analyst}
\date{\today}

% --- BEGIN DOCUMENT ---
\begin{document}

\maketitle
\thispagestyle{empty}
\newpage

\tableofcontents
\newpage

% =========================================================================
\section{Executive Summary}
% =========================================================================

This report provides a comprehensive cybersecurity assessment for \textbf{[Organization Name]}, based on an analysis of network scan data, a security controls questionnaire, and a review of pre-existing risks. The assessment was conducted on \today.

The analysis reveals a \textbf{critical risk posture}. Key findings include the widespread lack of Multi-Factor Authentication (MFA) for critical services like email and computer logins, which significantly increases the risk of account compromise and unauthorized access.

Furthermore, a technical scan identified a publicly exposed MySQL database service (port 3306). This service is running an outdated and End-of-Life (EOL) version of MySQL (5.7.33), which contains known vulnerabilities. This exposure, combined with weak access controls, creates a direct and severe threat to data confidentiality and integrity.

Finally, foundational security gaps were identified, including the absence of an employee Acceptable Use Policy and a lack of recurring security awareness training. These policy and training deficiencies weaken the organization's overall security culture and resilience against common threats like phishing.

Immediate and decisive action is required to mitigate these risks. Recommendations are prioritized in Section \ref{sec:recommendations} to address the most critical vulnerabilities first.

% =========================================================================
\section{Organizational Information}
% =========================================================================

The following information was used as the basis for this assessment. Due to the anonymized nature of the provided data, placeholders have been used where necessary.

\begin{itemize}
    \item \textbf{Organization Name:} \textbf{[Organization Name]}
    \item \textbf{Primary Domain:} \texttt{[Domain]}
    \item \textbf{External IP Scanned:} \texttt{[Client IP]}
\end{itemize}

% =========================================================================
\section{Security Control Review}
% =========================================================================

A review of the organization's security controls was conducted via a questionnaire. The results are summarized in Table \ref{tab:controls}. Answers marked with a red 'X' (\ding{55}) indicate significant gaps in the security framework.

\begin{table}[h!]
\centering
\caption{Security Controls Questionnaire Results}
\label{tab:controls}
\begin{tabular}{p{0.75\textwidth} c}
\toprule
\textbf{Control Question} & \textbf{Response} \\
\midrule
Do you require MFA to access email? & \textcolor{red}{\ding{55}} \\
Do you require MFA to log into computers? & \textcolor{red}{\ding{55}} \\
Do you require MFA to access sensitive data systems? & \textcolor{green}{\ding{51}} \\
Does your organization have an employee acceptable use policy? & \textcolor{red}{\ding{55}} \\
Does your organization do security awareness training for new employees? & \textcolor{green}{\ding{51}} \\
Does your organization do security awareness training for all employees at least once per year? & \textcolor{red}{\ding{55}} \\
\bottomrule
\end{tabular}
\end{table}

\subsection*{Analysis of Control Gaps}
The questionnaire reveals critical deficiencies in access control and security governance:
\begin{itemize}
    \item \textbf{Lack of MFA:} The absence of MFA on email and computer logins is a critical vulnerability. Email is a primary target for attackers seeking to gain an initial foothold, and compromised credentials can be used to move laterally within the network.
    \item \textbf{Policy and Training Gaps:} The lack of an Acceptable Use Policy means there are no formal rules governing how employees should use company assets, increasing the risk of misuse. While new employees receive training, the absence of an annual refresher for all staff leaves the organization vulnerable to evolving threats like sophisticated phishing attacks.
\end{itemize}

% =========================================================================
\section{Technical Scan Results}
% =========================================================================

An external network scan was performed against the target IP address \texttt{[Target IP]}. The scan identified one open port, detailed in Table \ref{tab:scan}.

\begin{table}[h!]
\centering
\caption{Open Port Analysis}
\label{tab:scan}
\begin{tabular}{l l l l l}
\toprule
\textbf{Port} & \textbf{State} & \textbf{Service} & \textbf{Version} & \textbf{Notes} \\
\midrule
3306/tcp & open & mysql & MySQL 5.7.33 & \textbf{Critical Risk} \\
\bottomrule
\end{tabular}
\end{table}

\subsection*{Analysis of Technical Findings}
The scan confirms the pre-existing risk of "Database Exposure". The following points are of critical concern:
\begin{itemize}
    \item \textbf{Public Exposure:} The MySQL database port (3306) is open to the public internet. This allows any attacker to attempt to connect, brute-force credentials, or exploit vulnerabilities in the service. Database servers should never be directly exposed to the internet.
    \item \textbf{End-of-Life Software:} The identified version, \textbf{MySQL 5.7.33}, is an End-of-Life (EOL) product. Official support and security patches for MySQL 5.7 ended in October 2023. This version is known to have multiple publicly disclosed vulnerabilities that will not be fixed by the vendor, making it an easy target for exploitation.
\end{itemize}

% =========================================================================
\section{Consolidated Risk Assessment}
% =========================================================================

By correlating the security control gaps, technical findings, and pre-existing risk data, we have compiled a consolidated list of the most significant risks facing the organization.

\begin{table}[h!]
\centering
\caption{Summary of Identified Risks}
\label{tab:risks}
\begin{tabular}{p{0.2\textwidth} p{0.55\textwidth} p{0.15\textwidth}}
\toprule
\textbf{Risk Name} & \textbf{Description} & \textbf{Severity} \\
\midrule
\textbf{Exposed EOL Database Service} & A MySQL 5.7 database is publicly accessible on port 3306. The software is End-of-Life and contains unpatched vulnerabilities. & \textbf{Critical} \\
\addlinespace
\textbf{Insufficient Access Controls} & MFA is not enforced for email or computer logins, making user accounts highly susceptible to takeover via credential theft or phishing. & \textbf{Critical} \\
\addlinespace
\textbf{Lack of Security Governance} & The absence of an Acceptable Use Policy and recurring security training for all staff creates a weak security culture and increases human-related risks. & \textbf{High} \\
\bottomrule
\end{tabular}
\end{table}

% =========================================================================
\section{Recommendations}
\label{sec:recommendations}
% =========================================================================

The following recommendations are prioritized to provide a clear, actionable path to improving the organization's security posture.

\subsection*{Immediate Priority (Mitigate within 72 hours)}
\begin{enumerate}
    \item \textbf{Restrict Access to MySQL Database:} Immediately implement firewall rules to block all public access to TCP port 3306 on \texttt{[Target IP]}. Access should be restricted to trusted internal IP addresses only. For remote administration, a Virtual Private Network (VPN) must be used.
    \item \textbf{Enforce MFA on Email:} Immediately enable and enforce MFA for all user accounts on the email system. This is the single most effective control to prevent business email compromise.
\end{enumerate}

\subsection*{High Priority (Mitigate within 30-60 days)}
\begin{enumerate}
    \setcounter{enumi}{2} % Continue numbering
    \item \textbf{Upgrade MySQL Database:} Plan and execute the migration of the MySQL 5.7 database to a currently supported version (e.g., MySQL 8.x). This will ensure the service receives critical security patches.
    \item \textbf{Implement Comprehensive MFA:} Extend MFA enforcement to all computer logins and other sensitive systems to create a consistent, strong authentication standard.
    \item \textbf{Develop and Implement an Acceptable Use Policy (AUP):} Create a formal AUP that all employees must read and acknowledge. This policy should define the rules for using company technology and data.
    \item \textbf{Establish Annual Security Awareness Training:} Implement a mandatory security awareness training program for all employees, to be completed annually. This program should cover topics such as phishing, password security, and social engineering.
\end{enumerate}

\end{document}
```