```latex
\documentclass[12pt]{article}

% Preamble: Required Packages
\usepackage[margin=1in]{geometry}
\usepackage{pifont} % For checkmarks and crosses
\usepackage{booktabs} % For professional tables
\usepackage{hyperref} % For hyperlinks and report metadata
\usepackage{url} % For formatting URLs
\usepackage{seqsplit} % To split long monospaced strings
\usepackage{graphicx}
\usepackage{xcolor}
\usepackage{fancyhdr}
\usepackage{lastpage}

% --- Document Metadata ---
\hypersetup{
    colorlinks=true,
    linkcolor=blue,
    filecolor=magenta,      
    urlcolor=cyan,
    pdftitle={Cybersecurity Posture Assessment Report},
    pdfauthor={Cybersecurity Analyst},
    pdfsubject={Security Assessment},
    pdfkeywords={Security, Analysis, Report},
    bookmarks=true
}

% --- Header and Footer ---
\pagestyle{fancy}
\fancyhf{} % Clear all header and footer fields
\fancyhead[L]{Cybersecurity Posture Assessment Report}
\fancyhead[R]{\textbf{[Organization Name]}}
\fancyfoot[C]{Page \thepage\ of \pageref{LastPage}}
\renewcommand{\headrulewidth}{0.4pt}
\renewcommand{\footrulewidth}{0.4pt}

% --- Checkmark and Cross Definitions ---
\newcommand{\yes}{\ding{51}}
\newcommand{\no}{\ding{55}}

\begin{document}

% --- Title Page ---
\begin{titlepage}
    \centering
    \vspace*{1cm}
    \includegraphics[width=0.4\textwidth]{example-image-a} % Placeholder logo
    \vfill
    \Huge\bfseries
    Cybersecurity Posture Assessment Report
    \vspace{1cm}
    \LARGE
    Prepared for: \textbf{[Organization Name]}
    \vspace{2cm}
    \large
    Report Date: \today
    \vfill
    \normalsize
    This report contains sensitive information and is intended solely for the use of \textbf{[Organization Name]}. Distribution is prohibited without prior written consent.
\end{titlepage}

\tableofcontents
\newpage

% --- Section 1: Executive Summary ---
\section{Executive Summary}

This report provides a comprehensive assessment of the cybersecurity posture for \textbf{[Organization Name]}. The analysis is based on a synthesis of a technical network scan, a review of administrative security controls via a questionnaire, and an evaluation of previously identified risks.

The assessment revealed critical gaps in foundational security controls. The lack of mandatory Multi-Factor Authentication (MFA) for email and computer access represents a significant and immediate threat, exposing the organization to high risks of account compromise and unauthorized access. Furthermore, the absence of annual security awareness training for all employees weakens the organization's primary defense against social engineering attacks like phishing.

On a positive note, the external network scan of the target system did not identify any open ports, indicating a strong network perimeter at that specific point. This finding contradicts a pre-existing risk concerning an unencrypted web server on Port 80, suggesting that the vulnerability may have been successfully remediated.

Immediate action is required to address the identified MFA and training deficiencies to mitigate substantial risks to the organization's data and operations.

\vspace{1cm}

\begin{tabular}{@{}ll}
    \toprule
    \textbf{Overall Posture Assessment:} & \textcolor{red}{\textbf{High Risk}} \\
    \bottomrule
\end{tabular}

% --- Section 2: Organizational Information ---
\section{Organizational Information}

This section details the information provided for the scope of this assessment. Due to the anonymized nature of the input data, placeholders have been used where necessary.

\begin{tabular}{@{}ll}
    \toprule
    \textbf{Attribute} & \textbf{Value} \\
    \midrule
    Organization Name & \textbf{[Organization Name]} \\
    Primary Email Domain & \texttt{[Domain]} \\
    Client External IP & \texttt{[Client IP]} \\
    \bottomrule
\end{tabular}

% --- Section 3: Security Control Review ---
\section{Security Control Review (Questionnaire Analysis)}

The following table summarizes the organization's responses to the security controls questionnaire. "No" answers indicate significant gaps in the security framework and are flagged for immediate attention.

\begin{tabular}{p{0.5\linewidth} c p{0.35\linewidth}}
    \toprule
    \textbf{Control Question} & \textbf{Response} & \textbf{Analyst Notes} \\
    \midrule
    Do you require MFA to access email? & \textcolor{red}{\no} & \textbf{Critical Gap.} Lack of MFA on email is a primary vector for business email compromise (BEC) and phishing attacks. \\
    \addlinespace
    Do you require MFA to log into computers? & \textcolor{red}{\no} & \textbf{Critical Gap.} Compromised credentials could lead to direct endpoint and network access, bypassing perimeter defenses. \\
    \addlinespace
    Do you require MFA to access sensitive data systems? & \textcolor{green}{\yes} & Positive control. This reduces the risk of unauthorized access to critical assets. \\
    \addlinespace
    Does your organization have an employee acceptable use policy? & \textcolor{green}{\yes} & Foundational policy is in place, setting clear expectations for employees. \\
    \addlinespace
    Does your organization do security awareness training for new employees? & \textcolor{green}{\yes} & Good practice for onboarding new staff and establishing a security-first mindset. \\
    \addlinespace
    Does your organization do security awareness training for all employees at least once per year? & \textcolor{red}{\no} & \textbf{High Risk.} Security knowledge degrades over time. Lack of recurring training leaves the organization vulnerable to evolving threats. \\
    \bottomrule
\end{tabular}

% --- Section 4: Technical Scan Results ---
\section{Technical Scan Results}

A network scan was performed to identify accessible services and potential vulnerabilities on the organization's external-facing infrastructure.

\begin{itemize}
    \item \textbf{Scan Target:} \texttt{[Target IP]}
    \item \textbf{Scan Date:} The scan was performed prior to \today.
    \item \textbf{Summary of Findings:} The scan confirmed that the target host is online. However, no open ports were discovered in the default scan profile. Port 80 (HTTP) was explicitly checked and found to be in a \textbf{closed} state. This is a positive security finding, as it indicates a properly configured firewall and a minimal attack surface for this host.
\end{itemize}

\begin{table}[h!]
\centering
\caption{Port Scan Details for \texttt{[Target IP]}}
\begin{tabular}{@{}llll@{}}
    \toprule
    \textbf{Port} & \textbf{State} & \textbf{Service} & \textbf{Notes} \\
    \midrule
    80 & closed & http & The port is not listening for connections. \\
    \bottomrule
\end{tabular}
\end{table}

% --- Section 5: Consolidated Risk Assessment ---
\section{Consolidated Risk Assessment}

This table synthesizes findings from the security control review, technical scan, and pre-existing risk data into a prioritized list.

\begin{tabular}{p{0.1\linewidth} p{0.25\linewidth} p{0.4\linewidth} p{0.15\linewidth}}
    \toprule
    \textbf{Risk ID} & \textbf{Risk Name} & \textbf{Description} & \textbf{Severity} \\
    \midrule
    RISK-001 & \textbf{Lack of MFA for Email and Endpoints} & The absence of MFA on critical access points (email, computers) drastically increases the likelihood and impact of a credential-based attack. & \textcolor{red}{\textbf{Critical}} \\
    \addlinespace
    RISK-002 & \textbf{Inadequate Security Awareness Training} & Without mandatory, recurring annual training, employees are more susceptible to social engineering, phishing, and malware attacks. & \textcolor{orange}{\textbf{High}} \\
    \addlinespace
    RISK-003 & \textbf{Unencrypted Web Server (Historical)} & A pre-existing risk noted an open Port 80. Our technical scan found this port to be closed, suggesting remediation. This risk is included for tracking but may be resolved. & \textcolor{gray}{Medium (Potentially Remediated)} \\
    \bottomrule
\end{tabular}

% --- Section 6: Recommendations ---
\section{Recommendations}

The following actions are recommended to mitigate the identified risks and improve the overall security posture of \textbf{[Organization Name]}.

\subsection{Immediate Priority (Critical Risks)}
\begin{itemize}
    \item \textbf{RISK-001: Enforce MFA Everywhere.}
        \begin{itemize}
            \item Immediately enable and enforce MFA for all user accounts across all email platforms (e.g., Microsoft 365, Google Workspace).
            \item Deploy and require MFA for all computer logins, both local and remote. Solutions like Windows Hello for Business, Duo, or other third-party tools should be implemented.
        \end{itemize}
\end{itemize}

\subsection{High Priority}
\begin{itemize}
    \item \textbf{RISK-002: Implement Annual Security Training.}
        \begin{itemize}
            \item Procure and deploy a security awareness training platform.
            \item Develop a formal policy requiring all employees, including executives and contractors, to complete security awareness training at least once per year.
            \item Track completion rates to ensure 100\% compliance.
        \end{itemize}
\end{itemize}

\subsection{Verification and Housekeeping}
\begin{itemize}
    \item \textbf{RISK-003: Confirm Remediation.}
        \begin{itemize}
            \item Internally verify that Port 80 on host \texttt{[Target IP]} is intentionally closed and that the service is no longer required.
            \item If the remediation is confirmed, formally update the internal risk register to close the "Unencrypted Web Server" risk.
        \end{itemize}
\end{itemize}

% --- Section 7: Conclusion ---
\section{Conclusion}

The assessment for \textbf{[Organization Name]} has identified critical, high-impact vulnerabilities in its administrative and procedural security controls. While the organization's network perimeter appears well-configured at the scanned point, its resilience against common cyberattacks is severely undermined by the lack of MFA and consistent employee security training.

We strongly urge management to prioritize the implementation of the recommendations outlined in this report. Addressing these foundational security gaps will significantly reduce the organization's risk exposure and build a more robust defense against modern cyber threats.

\end{document}
```