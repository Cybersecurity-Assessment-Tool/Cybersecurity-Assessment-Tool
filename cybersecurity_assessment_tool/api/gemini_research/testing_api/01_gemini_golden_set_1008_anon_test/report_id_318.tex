```latex
\documentclass[12pt]{article}

% Preamble: Required Packages
\usepackage[margin=1in]{geometry} % For setting margins
\usepackage{pifont}                 % For dingbats symbols like checkmarks
\usepackage{booktabs}               % For professional-looking tables
\usepackage[hidelinks]{hyperref}    % For hyperlinks, hidelinks removes the boxes
\usepackage{url}                    % For typesetting URLs
\usepackage{seqsplit}               % To split long strings in texttt
\usepackage{xcolor}                 % For using colors
\usepackage{graphicx}               % For scaling content

% --- Custom Commands & Definitions ---
\newcommand{\cmark}{\textcolor{green}{\ding{51}}}
\newcommand{\xmark}{\textcolor{red}{\ding{55}}}

% Define colors for risk levels
\definecolor{critical}{RGB}{217, 83, 79}
\definecolor{high}{RGB}{240, 173, 78}
\definecolor{medium}{RGB}{91, 192, 222}

% Command for a colored severity box
\newcommand{\severitybox}[2]{%
  \colorbox{#1}{\textcolor{white}{\textbf{\strut\ #2\ }}}%
}

% --- Document Metadata ---
\title{Cybersecurity Posture Assessment Report}
\author{Cybersecurity Analyst}
\date{\today}

\begin{document}

\maketitle
\thispagestyle{empty}
\newpage
\tableofcontents
\newpage

% ==============================================================================
\section{Executive Summary}
% ==============================================================================

This report details the findings of a cybersecurity assessment conducted for \textbf{[Organization Name]}. The assessment incorporated a review of organizational security controls via a questionnaire, an external network scan, and a correlation with pre-existing risk data.

The analysis revealed two significant, high-impact security gaps in organizational policy and procedure. The lack of mandatory Multi-Factor Authentication (MFA) for email access represents a \textbf{critical risk}, exposing the organization to Business Email Compromise (BEC) and account takeover attacks. Additionally, the absence of annual security awareness training for all employees constitutes a \textbf{high risk}, as it allows security knowledge to degrade over time, making staff more susceptible to social engineering attacks.

On the technical front, an external network scan of the target IP address \texttt{[Target IP]} showed no open ports. This is a positive security finding. Notably, this result contradicts a pre-existing documented risk concerning an "Unencrypted Web Server" on Port 80. The scan indicates that this port is currently closed, suggesting the risk may have been remediated or was a false positive.

Recommendations are prioritized to address the most critical findings first: immediate implementation of MFA for email, followed by the establishment of a recurring security training program.

% ==============================================================================
\section{Organizational Information}
% ==============================================================================

The following information was used as the basis for this assessment. As identity data was not provided, placeholders have been used.

\begin{table}[h!]
\centering
\begin{tabular}{@{}ll@{}}
\toprule
\textbf{Attribute} & \textbf{Value} \\ \midrule
Organization Name  & \textbf{[Organization Name]} \\
Email Domain       & \texttt{[Domain]} \\
External IP Range  & \texttt{[Client IP]} \\ \bottomrule
\end{tabular}
\caption{Client Organizational Details}
\label{tab:org_info}
\end{table}

% ==============================================================================
\section{Security Control Review}
% ==============================================================================

A security questionnaire was completed to evaluate the current state of administrative and policy-based controls. The results are summarized below.

\begin{table}[h!]
\centering
\resizebox{\textwidth}{!}{%
\begin{tabular}{@{}lc@{}}
\toprule
\textbf{Control Question} & \textbf{Response} \\ \midrule
Do you require MFA to access email? & \xmark \\
Do you require MFA to log into computers? & \cmark \\
Do you require MFA to access sensitive data systems? & \cmark \\
Does your organization have an employee acceptable use policy? & \cmark \\
Does your organization do security awareness training for new employees? & \cmark \\
Does your organization do security awareness training for all employees at least once per year? & \xmark \\ \bottomrule
\end{tabular}%
}
\caption{Security Controls Questionnaire Results}
\label{tab:controls}
\end{table}

\subsection*{Analysis of Control Gaps}
The questionnaire identified two significant control deficiencies:
\begin{itemize}
    \item \textbf{No MFA for Email:} Email is a primary target for attackers. Without MFA, a compromised password is all that is needed for an attacker to gain access to an employee's mailbox, which can lead to data breaches, financial fraud, and further infiltration of the network. This is considered a critical gap.
    \item \textbf{No Annual Security Awareness Training:} While training for new hires is in place, the lack of a recurring, annual training program for all staff is a high-risk oversight. The threat landscape evolves continuously, and employee knowledge must be refreshed to remain effective against modern phishing and social engineering tactics.
\end{itemize}

% ==============================================================================
\section{Technical Scan Results}
% ==============================================================================

An external network scan was performed using Nmap to identify open ports and exposed services on the designated target system.

\subsection*{Scan Details}
\begin{itemize}
    \item \textbf{Target IP:} \texttt{[Target IP]}
    \item \textbf{Scan Date:} \today
    \item \textbf{Scan Type:} TCP Connect Scan
\end{itemize}

\subsection*{Findings}
The scan reported the target host as "up," but found no open TCP ports. The status of a key port is detailed below.

\begin{table}[h!]
\centering
\begin{tabular}{@{}lll@{}}
\toprule
\textbf{Port} & \textbf{State} & \textbf{Service} \\ \midrule
80/tcp        & closed         & http             \\ \bottomrule
\end{tabular}
\caption{Nmap Scan Results for \texttt{[Target IP]}}
\label{tab:nmap_results}
\end{table}

\subsection*{Analysis of Technical Findings}
The scan results are positive from a security perspective, as no services were found to be exposed to the public internet on the scanned host. 

\textbf{Important Correlation Note:} A pre-existing risk on file, "Unencrypted Web Server," states that Port 80 is open. This scan's finding that Port 80 is \textbf{closed} directly contradicts the existing risk data. This suggests one of the following possibilities:
\begin{enumerate}
    \item The vulnerability has been successfully remediated since it was last assessed.
    \item The original finding was a false positive.
    \item The scan was performed against a different IP than the one associated with the risk.
\end{enumerate}
Further investigation is required to formally close or re-validate the pre-existing risk.

% ==============================================================================
\section{Consolidated Risk Assessment}
% ==============================================================================

The following table synthesizes findings from the security control review, technical scan, and pre-existing risk documentation into a consolidated list.

\begin{table}[h!]
\centering
\resizebox{\textwidth}{!}{%
\begin{tabular}{@{}llll@{}}
\toprule
\textbf{ID} & \textbf{Risk / Finding} & \textbf{Severity} & \textbf{Comments} \\ \midrule
\textbf{R-01} & Lack of MFA on Email Accounts & \severitybox{critical}{Critical} & Active policy gap. Exposes organization to BEC and account takeover. \\
\addlinespace
\textbf{R-02} & No Annual Security Awareness Training & \severitybox{high}{High} & Active policy gap. Increases susceptibility to social engineering. \\
\addlinespace
\textbf{R-03} & Unencrypted Web Server (Port 80) & \severitybox{medium}{Medium} & \begin{tabular}[c]{@{}l@{}}Status: \textbf{Not Validated}. Pre-existing risk contradicted by\\ current scan, which shows the port as closed.\end{tabular} \\ \bottomrule
\end{tabular}%
}
\caption{Consolidated Risk Register}
\label{tab:risk_register}
\end{table}

% ==============================================================================
\section{Recommendations}
% ==============================================================================

Based on the consolidated risk assessment, the following actions are recommended, ordered by priority.

\begin{description}
    \item[\textbf{Priority 1 (Critical):}] \textbf{Implement MFA for Email.}
    \begin{itemize}
        \item \textbf{Action:} Enforce mandatory Multi-Factor Authentication (MFA) across all email accounts immediately.
        \item \textbf{Justification:} This is the single most effective control to prevent unauthorized access to email and mitigate the risk of Business Email Compromise (BEC).
    \end{itemize}
    \vspace{0.5cm}
    \item[\textbf{Priority 2 (High):}] \textbf{Establish Annual Security Awareness Training.}
    \begin{itemize}
        \item \textbf{Action:} Develop or procure a security awareness training program and mandate its completion by all employees on an annual basis. The training should cover phishing, password security, and acceptable use policies.
        \item \textbf{Justification:} A well-trained workforce is a critical layer of defense. Regular training ensures that security remains a top-of-mind concern for all staff.
    \end{itemize}
    \vspace{0.5cm}
    \item[\textbf{Priority 3 (Medium):}] \textbf{Investigate and Resolve Port 80 Risk Discrepancy.}
    \begin{itemize}
        \item \textbf{Action:} Review asset management records to confirm the IP address of the web server associated with risk R-03. Re-scan the confirmed IP if different. If the port is confirmed to be closed on the correct asset, update the risk register to formally close the finding.
        \item \textbf{Justification:} Maintaining an accurate risk register is essential for focusing remediation efforts where they are most needed. This action will clarify the true state of the external perimeter.
    \end{itemize}
\end{description}

% ==============================================================================
\section{Conclusion}
% ==============================================================================

The overall security posture of \textbf{[Organization Name]} can be significantly improved by addressing the identified procedural and policy-based gaps. While the external network perimeter appears secure based on the recent scan, the lack of foundational controls like MFA on email and recurring security training presents a substantial and immediate risk to the organization.

We strongly advise prioritizing the implementation of the recommendations outlined in this report to build a more resilient and secure operational environment.

\end{document}
```