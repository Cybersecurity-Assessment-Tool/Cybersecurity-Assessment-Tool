```latex
\documentclass[12pt]{article}

% Preamble: Required Packages
\usepackage[margin=1in]{geometry}
\usepackage{pifont} % For checkmarks and crosses
\usepackage{booktabs} % For professional tables
\usepackage{hyperref} % For clickable links
\usepackage{url} % For URL formatting
\usepackage{seqsplit} % For splitting long strings in tt font
\usepackage{graphicx}
\usepackage{xcolor}

% Hyperref Setup
\hypersetup{
    colorlinks=true,
    linkcolor=blue,
    filecolor=magenta,      
    urlcolor=cyan,
    pdftitle={Cybersecurity Posture Report},
    pdfpagemode=FullScreen,
}

% Define check and cross marks for convenience
\newcommand{\cmark}{\ding{51}}%
\newcommand{\xmark}{\ding{55}}%

% Document Start
\begin{document}

% --- Title Page ---
\begin{titlepage}
    \centering
    \vspace*{1cm}
    \includegraphics[width=0.4\textwidth]{example-image-a} % Placeholder logo
    \vspace{1.5cm}
    
    \huge\textbf{Cybersecurity Posture Report}
    \vspace{1.5cm}
    
    \Large For: \textbf{[Organization Name]}
    \vspace{2cm}
    
    \normalsize
    \begin{tabular}{ll}
        \textbf{Date of Report:} & \today \\
        \textbf{Scan Date:} & 2025-11-22 \\
        \textbf{Prepared by:} & Expert Cybersecurity Analyst \\
    \end{tabular}
    
    \vfill
    
    \small\textit{This report is confidential and intended solely for the use of \textbf{[Organization Name]}. Unauthorized distribution is prohibited.}
\end{titlepage}

\tableofcontents
\newpage

% --- Section 1: Executive Summary ---
\section{Executive Summary}
This report provides a comprehensive analysis of the cybersecurity posture for \textbf{[Organization Name]}, based on a combination of technical network scanning, a security controls questionnaire, and a review of pre-existing risks. The assessment was conducted to identify vulnerabilities, policy gaps, and areas for security enhancement.

\paragraph{Key Findings:} The organization has established several positive security controls, including the enforcement of Multi-Factor Authentication (MFA) for email and computer access, and a consistent security awareness training program. These measures form a solid foundation for a defensive security strategy.

However, two high-risk issues were identified that require immediate attention:
\begin{enumerate}
    \item \textbf{Critical Control Gap:} Multi-Factor Authentication is not required for accessing sensitive data systems. This gap significantly increases the risk of unauthorized access to critical information assets in the event of credential compromise.
    \item \textbf{Vulnerable External Service:} The external-facing web server, scanned at \texttt{[Target IP]}, is running an outdated version of Nginx (1.18.0). This version is known to have multiple publicly disclosed vulnerabilities, exposing the organization to potential web-based attacks.
\end{enumerate}

\paragraph{Overall Posture:} While foundational policies are in place, the identified critical gaps in technical controls currently place the organization at a high risk of a security breach. The recommendations provided in this report are designed to mitigate these specific risks and improve the overall security posture.

% --- Section 2: Organizational Information ---
\section{Organizational Information}
The following details were used as the basis for this assessment. As per the provided data, placeholders have been used where specific information was not available.

\begin{table}[h!]
\centering
\begin{tabular}{@{}ll@{}}
\toprule
\textbf{Attribute} & \textbf{Value} \\ \midrule
Organization Name & \textbf{[Organization Name]} \\
Primary Domain & \texttt{[Domain]} \\
External IP Address (Provided) & \texttt{[Client IP]} \\
Target IP Address (Scanned) & \texttt{[Target IP]} \\
\bottomrule
\end{tabular}
\caption{Client Organizational Details.}
\end{table}

% --- Section 3: Security Control Review ---
\section{Security Control Review (Questionnaire)}
An assessment of internal security controls was conducted based on a standardized questionnaire. The responses indicate the current state of policy enforcement and security practices. A "No" response highlights a significant gap that increases organizational risk.

\begin{table}[h!]
\centering
\begin{tabular}{@{}p{0.6\linewidth} c p{0.2\linewidth}@{}}
\toprule
\textbf{Control Question} & \textbf{Response} & \textbf{Assessment} \\ \midrule
Do you require MFA to access email? & \textcolor{green}{\cmark} & Best Practice Met \\
Do you require MFA to log into computers? & \textcolor{green}{\cmark} & Best Practice Met \\
Do you require MFA to access sensitive data systems? & \textcolor{red}{\xmark} & \textbf{Critical Gap} \\
Does your organization have an employee acceptable use policy? & \textcolor{green}{\cmark} & Best Practice Met \\
Does your organization do security awareness training for new employees? & \textcolor{green}{\cmark} & Best Practice Met \\
Does your organization do security awareness training for all employees at least once per year? & \textcolor{green}{\cmark} & Best Practice Met \\
\bottomrule
\end{tabular}
\caption{Security Controls Questionnaire Analysis.}
\end{table}

% --- Section 4: Technical Scan Results ---
\section{Technical Scan Results}
An external network scan was performed against the target IP address \texttt{[Target IP]} on \textbf{2025-11-22}. The scan identified the following open ports and services accessible from the public internet.

\begin{table}[h!]
\centering
\begin{tabular}{@{}lllll@{}}
\toprule
\textbf{Port} & \textbf{State} & \textbf{Service} & \textbf{Product} & \textbf{Version} \\ \midrule
443/tcp & open & https & nginx & 1.18.0 \\
\bottomrule
\end{tabular}
\caption{Open Ports Detected on \texttt{[Target IP]}.}
\end{table}

\subsection{Analysis of Findings}
The scan identified a single service, HTTPS, running on an Nginx web server. 

\paragraph{Outdated Nginx Version:} The detected version, \textbf{Nginx 1.18.0}, was released in April 2020. This is a legacy version that is no longer receiving security patches. It is known to be vulnerable to several Common Vulnerabilities and Exposures (CVEs), including but not limited to CVE-2021-23017 (DNS resolver vulnerability). Running outdated software on internet-facing systems is a high-risk practice, as it provides a direct vector for attackers to compromise the server and potentially gain access to the internal network.

% --- Section 5: Consolidated Risk Assessment ---
\section{Consolidated Risk Assessment}
This section synthesizes findings from the security control review and the technical scan. No pre-existing vulnerabilities were reported in the input data.

\begin{table}[h!]
\centering
\begin{tabular}{@{}p{0.05\linewidth} p{0.5\linewidth} p{0.2\linewidth} p{0.1\linewidth}@{}}
\toprule
\textbf{ID} & \textbf{Risk Description} & \textbf{Source} & \textbf{Severity} \\ \midrule
\textbf{R-01} & Lack of Multi-Factor Authentication (MFA) on sensitive data systems allows for single-factor (password-only) compromise of critical assets. & Questionnaire & \textbf{High} \\
\addlinespace
\textbf{R-02} & The public-facing web server is running an outdated and vulnerable version of Nginx (1.18.0), exposing it to remote exploitation. & Network Scan & \textbf{High} \\
\bottomrule
\end{tabular}
\caption{Summary of Identified Risks.}
\end{table}

% --- Section 6: Recommendations ---
\section{Recommendations}
The following actions are recommended to mitigate the identified risks and strengthen the overall security posture of \textbf{[Organization Name]}.

\subsection{R-01: Implement MFA on Sensitive Systems (High)}
\paragraph{Action:} Immediately develop and execute a plan to deploy Multi-Factor Authentication across all systems classified as containing sensitive data. This includes databases, file shares, financial applications, and administrative portals.
\paragraph{Justification:} A compromised password is one of the most common initial attack vectors. Enforcing MFA adds a critical layer of defense, ensuring that even if an attacker obtains valid credentials, they cannot access the protected system without the second factor. This single control dramatically reduces the risk of a data breach.

\subsection{R-02: Upgrade Outdated Nginx Server (High)}
\paragraph{Action:} The Nginx server at \texttt{[Target IP]} must be upgraded to the latest stable version as soon as possible. Before upgrading, create a snapshot or backup of the server. After the upgrade, verify that all web applications function correctly.
\paragraph{Justification:} Patching and updating software is a fundamental security requirement. Upgrading Nginx will mitigate known vulnerabilities, protecting the server from automated attacks and targeted exploitation. Regular patch management cycles should be implemented for all internet-facing systems.

\end{document}
```