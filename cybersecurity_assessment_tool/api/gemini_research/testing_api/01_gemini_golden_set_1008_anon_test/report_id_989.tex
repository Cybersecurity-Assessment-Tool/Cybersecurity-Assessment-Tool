```latex
\documentclass[12pt, a4paper]{article}

% --- PACKAGE IMPORTS ---
\usepackage[margin=1in]{geometry} % Set page margins
\usepackage{pifont}                 % For checkmarks and crosses (\ding)
\usepackage{booktabs}               % For professional-looking tables
\usepackage{xcolor}                 % For custom colors
\usepackage{hyperref}               % For hyperlinks
\usepackage{url}                    % For URL formatting
\usepackage{seqsplit}               % For splitting long strings without spaces

% --- DOCUMENT SETUP ---
% Define colors for the report
\definecolor{darkblue}{rgb}{0.0, 0.0, 0.55}
\definecolor{darkred}{rgb}{0.55, 0.0, 0.0}
\definecolor{darkgreen}{rgb}{0.0, 0.39, 0.0}

% Hyperlink setup
\hypersetup{
    colorlinks=true,
    linkcolor=darkblue,
    filecolor=magenta,      
    urlcolor=darkblue,
    citecolor=darkblue,
}

% --- DOCUMENT START ---
\begin{document}

% --- TITLE PAGE ---
\title{
    \vspace{2cm}
    \textbf{Cybersecurity Posture Assessment Report} \\
    \large For \\
    \vspace{0.5cm}
    \textbf{[Organization Name]}
}
\author{Cybersecurity Analysis Division}
\date{\today}
\maketitle
\thispagestyle{empty}
\newpage

% --- TABLE OF CONTENTS ---
\tableofcontents
\newpage

% --- EXECUTIVE SUMMARY ---
\section{Executive Summary}
This report provides a cybersecurity posture assessment for \textbf{[Organization Name]}, based on an analysis of network scan data, a security controls questionnaire, and a review of pre-existing risks. The assessment was conducted on \today.

The analysis revealed several critical and high-risk findings that require immediate attention. A network scan confirmed that a Remote Desktop Protocol (RDP) service on port 3389 is publicly exposed. This finding correlates directly with a known critical risk and presents a significant and immediate threat to the organization's network integrity.

Furthermore, a review of internal security controls identified serious procedural and policy gaps. Key among these are the lack of multi-factor authentication (MFA) for sensitive data systems, the absence of an employee acceptable use policy, and a non-existent security awareness training program.

These combined technical and procedural vulnerabilities place the organization at a high risk of a security breach, data exfiltration, and ransomware attacks. This report outlines the detailed findings and provides prioritized, actionable recommendations to mitigate these risks and strengthen the overall security posture.

% --- ORGANIZATIONAL INFORMATION ---
\section{Organizational Information}
The following details were used as the basis for this assessment. Due to the anonymized nature of the provided data, placeholders have been used where necessary.

\begin{itemize}
    \item \textbf{Organization Name:} \textbf{[Organization Name]}
    \item \textbf{Primary Domain:} \texttt{[Domain]}
    \item \textbf{External IP Address Scanned:} \texttt{[Client IP]}
\end{itemize}

% --- SECURITY CONTROL REVIEW ---
\section{Security Control Review}
An assessment of the organization's security controls was performed based on a questionnaire. The responses indicate significant gaps in foundational security practices. "No" answers represent a deviation from best practices and are marked as high-risk gaps.

\begin{table}[h!]
\centering
\caption{Security Controls Questionnaire Analysis}
\label{tab:controls}
\begin{tabular}{p{0.6\linewidth} c c}
\toprule
\textbf{Security Control Question} & \textbf{Response} & \textbf{Status} \\
\midrule
Do you require MFA to access email? & Yes & \textcolor{darkgreen}{\ding{51}} \\
Do you require MFA to log into computers? & Yes & \textcolor{darkgreen}{\ding{51}} \\
\midrule
\rowcolor{red!15}
Do you require MFA to access sensitive data systems? & No & \textcolor{darkred}{\ding{55}} \\
\rowcolor{red!15}
Does your organization have an employee acceptable use policy? & No & \textcolor{darkred}{\ding{55}} \\
\rowcolor{red!15}
Does your organization do security awareness training for new employees? & No & \textcolor{darkred}{\ding{55}} \\
\rowcolor{red!15}
Does your organization do security awareness training for all employees at least once per year? & No & \textcolor{darkred}{\ding{55}} \\
\bottomrule
\end{tabular}
\end{table}

\subsection*{Analysis of Gaps}
\begin{itemize}
    \item \textbf{Lack of MFA on Sensitive Systems:} The absence of MFA on systems containing sensitive data is a critical vulnerability. It significantly increases the risk of unauthorized access should an attacker compromise a user's credentials.
    \item \textbf{Missing Acceptable Use Policy (AUP):} An AUP is a foundational document that governs the use of company IT assets. Without it, there is no formal guidance for employees, leading to inconsistent and potentially insecure practices.
    \item \textbf{No Security Awareness Training:} The complete lack of a security awareness program leaves employees uninformed about common threats like phishing, social engineering, and malware. This makes them a primary target for attackers.
\end{itemize}

% --- TECHNICAL SCAN RESULTS ---
\section{Technical Scan Results}
A network scan was performed to identify open ports and exposed services on the organization's external-facing infrastructure.

\subsection*{Scan Metadata}
\begin{itemize}
    \item \textbf{Scan Date:} \today
    \item \textbf{Target IP Address:} \texttt{[Target IP]}
\end{itemize}

\subsection*{Open Port Findings}
The scan identified the following open port, which presents a critical risk.

\begin{table}[h!]
\centering
\caption{Open Ports Identified on \texttt{[Target IP]}}
\label{tab:ports}
\begin{tabular}{l l l}
\toprule
\textbf{Port / Protocol} & \textbf{State} & \textbf{Service Name} \\
\midrule
\rowcolor{red!15}
3389/tcp & open & \texttt{ms-wbt-server} (Microsoft Remote Desktop) \\
\bottomrule
\end{tabular}
\end{table}

\subsection*{Technical Analysis}
The scan confirms that the Remote Desktop Protocol (RDP) service is directly exposed to the public internet. RDP is a frequent target for attackers who use brute-force attacks, credential stuffing, and exploitation of vulnerabilities (such as BlueKeep) to gain unauthorized access to internal networks. This finding is considered a critical risk.

% --- CORRELATED RISK ASSESSMENT ---
\section{Correlated Risk Assessment}
This section synthesizes the findings from the security control review, the technical scan, and pre-existing risk data into a consolidated list of prioritized risks.

\begin{table}[h!]
\centering
\caption{Summary of Identified Risks}
\label{tab:risks}
\begin{tabular}{p{0.1\linewidth} p{0.25\linewidth} p{0.45\linewidth} l}
\toprule
\textbf{Risk ID} & \textbf{Risk Name} & \textbf{Description} & \textbf{Severity} \\
\midrule
\rowcolor{red!25}
RISK-001 & Public RDP Exposure & The RDP service on port 3389 is exposed to the internet, confirmed by the network scan. This allows attackers to attempt brute-force or exploit-based attacks directly against a critical entry point. & \textbf{Critical (9.0)} \\
\midrule
\rowcolor{red!15}
RISK-002 & Inadequate Access Controls & Sensitive data systems lack MFA protection. This risk is severely amplified by RISK-001, as a successful RDP compromise could lead directly to sensitive data access without further authentication challenges. & \textbf{High} \\
\midrule
\rowcolor{red!15}
RISK-003 & Lack of Security Awareness & The absence of any security training program makes employees highly susceptible to phishing and social engineering attacks, which are common methods for stealing credentials used in RDP attacks. & \textbf{High} \\
\midrule
\rowcolor{orange!20}
RISK-004 & Missing Foundational Policies & The organization lacks a formal Acceptable Use Policy, leading to an undefined security baseline for employee behavior and IT resource usage. & \textbf{Medium} \\
\bottomrule
\end{tabular}
\end{table}

% --- RECOMMENDATIONS ---
\section{Recommendations}
The following actionable recommendations are provided to mitigate the identified risks. They are prioritized based on severity and potential impact.

\subsection*{Immediate Priority (Critical)}
\begin{enumerate}
    \item \textbf{Remediate RDP Exposure (RISK-001):}
    \begin{itemize}
        \item \textbf{Short-Term:} Immediately close port 3389 on the external firewall. Access should be blocked for all inbound traffic from the internet.
        \item \textbf{Long-Term:} If remote access is required, implement a secure Virtual Private Network (VPN) solution with Multi-Factor Authentication (MFA). All RDP access should occur exclusively through the secure VPN tunnel.
    \end{itemize}
\end{enumerate}

\subsection*{High Priority}
\begin{enumerate}
    \setcounter{enumi}{1}
    \item \textbf{Implement MFA on All Sensitive Systems (RISK-002):}
    \begin{itemize}
        \item Deploy a robust MFA solution and enforce its use for all access to applications and systems containing sensitive or critical data. This provides a crucial layer of defense against credential compromise.
    \end{itemize}
    \item \textbf{Establish a Security Awareness Program (RISK-003):}
    \begin{itemize}
        \item Implement mandatory security awareness training for all new hires as part of their onboarding process.
        \item Conduct annual security awareness and phishing simulation training for all employees to ensure they remain vigilant against current threats.
    \end{itemize}
\end{enumerate}

\subsection*{Medium Priority}
\begin{enumerate}
    \setcounter{enumi}{3}
    \item \textbf{Develop and Implement an Acceptable Use Policy (RISK-004):}
    \begin{itemize}
        \item Draft, approve, and disseminate an AUP that clearly defines the rules and expectations for using the organization's IT resources. Ensure all employees read and acknowledge the policy.
    \end{itemize}
\end{enumerate}

% --- DOCUMENT END ---
\end{document}
```