```latex
\documentclass[12pt, a4paper]{article}

% Preamble: Required Packages and Document Setup
\usepackage[margin=1in]{geometry}
\usepackage{pifont} % For checkmarks and crosses
\usepackage{booktabs} % For professional tables
\usepackage{hyperref} % For clickable links
\usepackage{url} % For URL formatting
\usepackage{seqsplit} % For splitting long strings in tt font
\usepackage{graphicx}
\usepackage{xcolor}
\usepackage{fancyhdr}

% Define colors for risk levels
\definecolor{criticalred}{HTML}{D10000}
\definecolor{highorange}{HTML}{E25F00}
\definecolor{mediumyellow}{HTML}{F2C000}
\definecolor{lowblue}{HTML}{0073E6}
\definecolor{infogray}{HTML}{808080}

% Setup Hyperref
\hypersetup{
    colorlinks=true,
    linkcolor=blue,
    filecolor=magenta,      
    urlcolor=cyan,
    pdftitle={Cybersecurity Posture Report},
    pdfpagemode=FullScreen,
}

% Header and Footer
\pagestyle{fancy}
\fancyhf{}
\fancyhead[L]{Cybersecurity Posture Report}
\fancyhead[R]{\textbf{[Organization Name]}}
\fancyfoot[C]{\thepage}

% --- Document Start ---
\begin{document}

% --- Title Page ---
\begin{titlepage}
    \centering
    \vspace*{1cm}
    \includegraphics[width=0.4\textwidth]{example-image-a} % Placeholder for a logo
    
    \vspace{1.5cm}
    
    \Huge
    \textbf{Cybersecurity Posture Report}
    
    \vspace{1.5cm}
    
    \Large
    Prepared for: \textbf{[Organization Name]}
    
    \vspace{2cm}
    
    \large
    Report Date: \today
    
    \vfill
    
    \large
    \textit{This report contains sensitive information and should be handled with care.}
    
\end{titlepage}

\tableofcontents
\newpage

% --- Section 1: Executive Summary ---
\section{Executive Summary}

This report provides a comprehensive analysis of the cybersecurity posture of \textbf{[Organization Name]}, based on a review of organizational security controls, an external network scan, and pre-existing risk data. The assessment was conducted on \today.

The overall security posture is assessed as \textbf{HIGH RISK}. Several critical and high-risk vulnerabilities were identified that expose the organization to significant threats, including unauthorized access, data breaches, and phishing attacks.

Key findings include:
\begin{itemize}
    \item \textbf{Critical - Widespread Lack of Multi-Factor Authentication (MFA):} The organization does not enforce MFA for email, computer logins, or access to sensitive data systems. This represents a critical control gap, significantly increasing the risk of account compromise.
    \item \textbf{High - Inadequate Security Awareness Training:} There is no security awareness training program for new or existing employees. This deficiency makes the organization highly susceptible to social engineering and phishing attacks.
    \item \textbf{Medium - Insecure Web Server Configuration:} The external network scan identified a web server operating over unencrypted HTTP (Port 80). This exposes any transmitted data to interception and manipulation.
\end{itemize}

Immediate and decisive action is required to remediate these findings. Recommendations focus on implementing foundational security controls, such as MFA and security training, and securing public-facing services. A detailed breakdown of findings and actionable recommendations is provided in the subsequent sections.

% --- Section 2: Organizational Information ---
\section{Organizational Information}

This section outlines the organizational data provided for this assessment. Due to the anonymized nature of the input data, placeholders have been used where necessary.

\begin{table}[h!]
\centering
\caption{Client Organizational Details}
\label{tab:org_info}
\begin{tabular}{@{}ll@{}}
\toprule
\textbf{Attribute} & \textbf{Value} \\ \midrule
Organization Name & \textbf{[Organization Name]} \\
Primary Email Domain & \texttt{[Domain]} \\
External IP Address & \texttt{[Client IP]} \\ \bottomrule
\end{tabular}
\end{table}

% --- Section 3: Security Control Review ---
\section{Security Control Review}
The following table summarizes the organization's responses to a security controls questionnaire. Each "No" answer represents a potential gap in the security framework and has been assessed for its impact.

\begin{table}[h!]
\centering
\caption{Security Controls Questionnaire Analysis}
\label{tab:controls}
\begin{tabular}{@{}p{0.6\linewidth}ccp{0.2\linewidth}@{}}
\toprule
\textbf{Control Question} & \textbf{Response} & \textbf{Status} & \textbf{Assessment} \\ \midrule
Do you require MFA to access email? & No & \ding{55} & \textcolor{criticalred}{\textbf{Critical Gap}} \\
Do you require MFA to log into computers? & No & \ding{55} & \textcolor{criticalred}{\textbf{Critical Gap}} \\
Do you require MFA to access sensitive data systems? & No & \ding{55} & \textcolor{criticalred}{\textbf{Critical Gap}} \\
Does your organization have an employee acceptable use policy? & Yes & \ding{51} & Good Practice \\
Does your organization do security awareness training for new employees? & No & \ding{55} & \textcolor{highorange}{\textbf{High Risk}} \\
Does your organization do security awareness training for all employees at least once per year? & No & \ding{55} & \textcolor{highorange}{\textbf{High Risk}} \\ \bottomrule
\end{tabular}
\end{table}

% --- Section 4: Technical Scan Results ---
\section{Technical Scan Results}
An external network scan was performed to identify open ports and services visible on the public internet.

\begin{itemize}
    \item \textbf{Target IP Address:} \texttt{[Target IP]}
    \item \textbf{Scan Date:} \today
    \item \textbf{Host Status:} Up
\end{itemize}

The scan revealed the following open port:

\begin{table}[h!]
\centering
\caption{Open Port Analysis}
\label{tab:ports}
\begin{tabular}{@{}lllll@{}}
\toprule
\textbf{Port} & \textbf{State} & \textbf{Service} & \textbf{Product/Version} & \textbf{Assessment} \\ \midrule
80/tcp & open & http & Not identified & \textcolor{mediumyellow}{\textbf{Medium Risk}} \\ \bottomrule
\end{tabular}
\end{table}

\paragraph{Finding: Unencrypted Web Traffic (HTTP)}
Port 80 (HTTP) is open to the public internet. The HTTP protocol transmits data in cleartext, meaning that any information, including potential login credentials or sensitive data exchanged with the server, can be easily intercepted by an attacker. Standard practice is to use HTTPS (Port 443), which encrypts the data in transit.

% --- Section 5: Risk Assessment ---
\section{Risk Assessment}
This section synthesizes findings from the security control review, technical scan, and pre-existing risk data into a consolidated list of identified risks.

\begin{table}[h!]
\centering
\caption{Consolidated Risk Register}
\label{tab:risks}
\begin{tabular}{@{}p{0.1\linewidth}p{0.25\linewidth}p{0.15\linewidth}p{0.4\linewidth}@{}}
\toprule
\textbf{Risk ID} & \textbf{Risk Name} & \textbf{Severity} & \textbf{Description} \\ \midrule
RISK-001 & Widespread Lack of MFA & \textcolor{criticalred}{\textbf{Critical}} & The absence of MFA for email, endpoints, and sensitive systems dramatically increases the likelihood of successful account takeovers via credential theft or phishing. \\
\addlinespace
RISK-002 & Inadequate Security Awareness Program & \textcolor{highorange}{\textbf{High}} & Employees are not trained to recognize or respond to security threats like phishing, making them a primary target for attackers seeking initial access to the network. \\
\addlinespace
RISK-003 & Unencrypted Web Traffic (HTTP) & \textcolor{mediumyellow}{\textbf{Medium}} & The web server on \texttt{[Target IP]} uses HTTP, exposing all communications to potential eavesdropping and man-in-the-middle attacks. \\
\addlinespace
RISK-004 & System Overriden (Pre-existing) & \textcolor{infogray}{Informational} & A pre-existing risk was noted with a CVSS score of 0.0. The overview states "System Overriden". This item is logged for tracking but poses no immediate technical threat based on the provided data. \\ \bottomrule
\end{tabular}
\end{table}

% --- Section 6: Recommendations ---
\section{Recommendations}
The following actions are recommended to mitigate the identified risks and improve the overall security posture of \textbf{[Organization Name]}.

\subsection{RISK-001: Implement Multi-Factor Authentication (Critical)}
\begin{itemize}
    \item \textbf{Immediate Action:} Enable MFA for all administrator accounts and external-facing services, especially email (e.g., Office 365, Google Workspace).
    \item \textbf{Short-Term Action:} Develop a phased rollout plan to enforce MFA for all employees for computer logins and access to all systems containing sensitive data.
    \item \textbf{Tools:} Utilize solutions like authenticator apps (Google Authenticator, Microsoft Authenticator), hardware tokens, or biometrics.
\end{itemize}

\subsection{RISK-002: Establish a Security Awareness Program (High)}
\begin{itemize}
    \item \textbf{Immediate Action:} Enroll all employees in a foundational security awareness training course covering phishing, password hygiene, and acceptable use.
    \item \textbf{Long-Term Action:} Implement a continuous training program that includes mandatory annual training for all staff, targeted training for new hires during onboarding, and regular phishing simulation campaigns to test and reinforce learning.
\end{itemize}

\subsection{RISK-003: Secure Web Server Communication (Medium)}
\begin{itemize}
    \item \textbf{Immediate Action:} Obtain and install a valid TLS/SSL certificate on the web server at \texttt{[Target IP]}.
    \item \textbf{Short-Term Action:} Reconfigure the web server to redirect all HTTP traffic from port 80 to HTTPS on port 443. Consider closing port 80 entirely if redirection is not required for legacy compatibility.
\end{itemize}

\end{document}
```