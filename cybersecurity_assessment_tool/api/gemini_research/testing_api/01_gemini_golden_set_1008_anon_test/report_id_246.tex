```latex
\documentclass[12pt]{article}

% Preamble: Required Packages
\usepackage[margin=1in]{geometry}
\usepackage{pifont} % For checkmarks and crosses
\usepackage{booktabs} % For professional tables
\usepackage{hyperref} % For hyperlinks
\usepackage{url}      % For URL formatting
\usepackage{seqsplit} % For splitting long strings
\usepackage{xcolor}   % For colors

% Document Information
\title{Cybersecurity Posture Assessment Report}
\author{Cybersecurity Analyst}
\date{\today}

% Hyperref Setup
\hypersetup{
    colorlinks=true,
    linkcolor=blue,
    filecolor=magenta,      
    urlcolor=cyan,
    pdftitle={Cybersecurity Posture Assessment Report},
    pdfpagemode=FullScreen,
}

\begin{document}

\maketitle
\thispagestyle{empty}
\newpage

\tableofcontents
\newpage

% --- 1. Executive Summary ---
\section{Executive Summary}
This report provides a comprehensive cybersecurity assessment for \textbf{[Organization Name]}, based on an analysis of network scan data, organizational security controls, and pre-existing risk information. The assessment was conducted to identify vulnerabilities, evaluate the effectiveness of current security measures, and provide actionable recommendations to enhance the organization's security posture.

Key findings indicate significant gaps in administrative and procedural controls. Specifically, the absence of Multi-Factor Authentication (MFA) on sensitive systems, the lack of an employee Acceptable Use Policy (AUP), and a non-existent security awareness training program present critical risks.

On a technical level, the external network scan of \texttt{[Client IP]} showed that port 80 (HTTP) is currently closed. This finding contradicts a pre-existing risk record that stated the port was open. This suggests a potential remediation has occurred, which should be formally verified and documented.

The primary recommendations focus on addressing the identified policy and training deficiencies to mitigate human-factor risks and implementing stronger authentication controls to protect critical assets.

% --- 2. Organizational Information ---
\section{Organizational Information}
This section details the information provided by the client organization. Due to the anonymized nature of the input data, placeholders are used where necessary.

\begin{itemize}
    \item \textbf{Organization Name:} \textbf{[Organization Name]}
    \item \textbf{Primary Domain:} \texttt{[Domain]}
    \item \textbf{External IP Scanned:} \texttt{[Client IP]}
\end{itemize}

% --- 3. Security Control Review ---
\section{Security Control Review}
The following table summarizes the organization's responses to a security controls questionnaire. A green checkmark (\textcolor{green}{\ding{51}}) indicates a positive control is in place, while a red cross (\textcolor{red}{\ding{55}}) highlights a control gap that introduces risk.

\begin{table}[h!]
\centering
\caption{Security Controls Questionnaire Analysis}
\begin{tabular}{p{0.8\linewidth} c}
\toprule
\textbf{Control Question} & \textbf{Status} \\
\midrule
Do you require MFA to access email? & \textcolor{green}{\ding{51}} \\
Do you require MFA to log into computers? & \textcolor{green}{\ding{51}} \\
Do you require MFA to access sensitive data systems? & \textcolor{red}{\ding{55}} \\
Does your organization have an employee acceptable use policy? & \textcolor{red}{\ding{55}} \\
Does your organization do security awareness training for new employees? & \textcolor{red}{\ding{55}} \\
Does your organization do security awareness training for all employees at least once per year? & \textcolor{red}{\ding{55}} \\
\bottomrule
\end{tabular}
\end{table}

The review reveals critical weaknesses in identity and access management for sensitive systems and a complete absence of foundational cybersecurity policies and employee training.

% --- 4. Technical Scan Results ---
\section{Technical Scan Results}
An external network scan was performed on the target host to identify open ports and exposed services.

\begin{itemize}
    \item \textbf{Target IP Address:} \texttt{[Target IP]}
    \item \textbf{Scan Date:} Assumed to be current
    \item \textbf{Host Status:} Up
\end{itemize}

\begin{table}[h!]
\centering
\caption{Open Port Analysis}
\begin{tabular}{c c c c}
\toprule
\textbf{Port} & \textbf{State} & \textbf{Service} & \textbf{Notes} \\
\midrule
80 & closed & http & The port for unencrypted web traffic is not accessible. \\
\bottomrule
\end{tabular}
\end{table}

\subsection{Analysis of Technical Findings}
The scan indicates that the target host is responsive but does not have any common ports open, including port 80 (HTTP). This is a positive security finding. However, this result directly contradicts a pre-existing risk entry (\textit{Unencrypted Web Server}) which assumes this port is open. This discrepancy suggests the risk may be outdated or has been remediated.

% --- 5. Risk Assessment Summary ---
\section{Risk Assessment Summary}
This section correlates findings from the security control review, technical scan, and pre-existing risk data into a consolidated list of identified risks.

\begin{table}[h!]
\centering
\caption{Consolidated Risk Register}
\begin{tabular}{p{0.3\linewidth} p{0.5\linewidth} l}
\toprule
\textbf{Risk Name} & \textbf{Overview} & \textbf{Severity} \\
\midrule
\textbf{Lack of MFA on Sensitive Systems} & The absence of MFA on systems containing sensitive data significantly increases the risk of unauthorized access via compromised credentials. & \textbf{Critical} \\
\textbf{No Security Awareness Training} & Employees are not trained on security best practices, making them highly susceptible to phishing, social engineering, and other common attacks. & High \\
\textbf{Absence of Acceptable Use Policy} & Without a formal AUP, there are no clear guidelines for employees on the acceptable use of company assets, increasing the risk of misuse and insider threats. & High \\
\textbf{Unencrypted Web Server} & A pre-existing risk noted that Port 80 was open, allowing unencrypted traffic. The current scan shows this port is closed, indicating the risk is likely remediated. & Medium (Potentially Remediated) \\
\bottomrule
\end{tabular}
\end{table}

% --- 6. Recommendations ---
\section{Recommendations}
Based on the risk assessment, the following actions are recommended to improve the organization's security posture.

\subsection{Critical Priority}
\begin{itemize}
    \item \textbf{Implement MFA on Sensitive Systems:} Immediately enforce MFA for all user accounts with access to sensitive data systems. This is the single most effective control to prevent unauthorized access.
\end{itemize}

\subsection{High Priority}
\begin{itemize}
    \item \textbf{Establish a Security Awareness Training Program:}
        \begin{itemize}
            \item Develop and implement a mandatory security awareness training module for all new employees during onboarding.
            \item Conduct annual refresher training for all staff to keep them informed about evolving threats.
            \item Consider periodic phishing simulations to test and reinforce training.
        \end{itemize}
    \item \textbf{Develop and Enforce an Acceptable Use Policy (AUP):}
        \begin{itemize}
            \item Create a comprehensive AUP that defines rules for the use of company networks, systems, and data.
            \item Ensure all employees read and formally acknowledge the policy.
        \end{itemize}
\end{itemize}

\subsection{Verification and Housekeeping}
\begin{itemize}
    \item \textbf{Verify Port 80 Closure:} Confirm that the closure of port 80 was an intentional act of remediation. If so, formally update the internal risk register to close the "Unencrypted Web Server" finding.
\end{itemize}

\end{document}
```