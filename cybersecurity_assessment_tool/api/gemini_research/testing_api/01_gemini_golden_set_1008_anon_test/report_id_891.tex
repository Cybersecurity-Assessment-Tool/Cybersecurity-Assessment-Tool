```latex
\documentclass[12pt]{article}

% Preamble: Required Packages
\usepackage[margin=1in]{geometry}
\usepackage{pifont} % For check and x marks (\ding{51}, \ding{55})
\usepackage{booktabs} % For professional tables
\usepackage{hyperref} % For clickable links
\usepackage{url}      % For URL formatting
\usepackage{seqsplit} % For splitting long strings without spaces
\usepackage{graphicx}
\usepackage[table]{xcolor}
\usepackage{fancyhdr}

% --- Document Metadata ---
\hypersetup{
    colorlinks=true,
    linkcolor=blue,
    filecolor=magenta,      
    urlcolor=cyan,
    pdftitle={Cybersecurity Posture Assessment Report},
    pdfauthor={Cybersecurity Analysis Cell},
    pdfsubject={Security Assessment},
    pdfkeywords={Cybersecurity, Nmap, Risk, Assessment},
}

% --- Header and Footer ---
\pagestyle{fancy}
\fancyhf{} % clear all header and footer fields
\fancyhead[L]{Cybersecurity Posture Assessment}
\fancyhead[R]{\textbf{[Organization Name]}}
\fancyfoot[C]{\thepage}
\renewcommand{\headrulewidth}{0.4pt}
\renewcommand{\footrulewidth}{0.4pt}

% --- Custom Commands ---
\newcommand{\yes}{\ding{51}}
\newcommand{\no}{\ding{55}}
\newcommand{\cellgray}{\cellcolor{gray!20}}

\begin{document}

% --- Title Page ---
\begin{titlepage}
    \centering
    \vspace*{1cm}
    \includegraphics[width=0.3\textwidth]{example-image-a} % Placeholder logo
    \vfill
    \huge\textbf{Cybersecurity Posture Assessment Report}
    \vspace{1cm}
    \Large
    \textbf{Prepared for:}\\
    \vspace{0.5cm}
    \textbf{[Organization Name]}
    \vspace{2cm}
    \large
    \textbf{Date of Report:}\\
    \today
    \vfill
    \textit{This document contains sensitive and confidential information. Distribution is restricted.}
\end{titlepage}

\tableofcontents
\newpage

% --- Section 1: Executive Summary ---
\section{Executive Summary}
This report details the findings of a cybersecurity posture assessment conducted for \textbf{[Organization Name]}. The assessment combined an automated network scan, a review of existing risk documentation, and an analysis of organizational security controls via a questionnaire.

The overall security posture is assessed as \textbf{High-Risk}. Several critical vulnerabilities and policy gaps were identified that require immediate attention.

\textbf{Key Critical Findings:}
\begin{itemize}
    \item \textbf{Publicly Exposed Database:} A MySQL database server (port 3306) was found to be directly accessible from the network. This exposes sensitive data to unauthorized access and attack.
    \item \textbf{End-of-Life Software:} The exposed MySQL server is running version 5.7.33, which reached its official End-of-Life (EOL) in October 2023. This version no longer receives security updates, leaving it vulnerable to known exploits.
    \item \textbf{Insufficient Access Controls:} Multi-Factor Authentication (MFA) is not enforced for critical access points, including employee email and computer logins. This significantly increases the risk of account compromise and unauthorized access.
    \item \textbf{Policy Gaps:} The absence of a formal Employee Acceptable Use Policy creates ambiguity and increases the likelihood of insider threats, whether malicious or accidental.
\end{itemize}

Immediate remediation should focus on isolating the database server from public access and implementing MFA across all critical systems. A detailed breakdown of all findings and actionable recommendations is provided in the subsequent sections of this report.

% --- Section 2: Organizational Information ---
\section{Organizational Information}
This section provides the high-level details of the organization under review. The information is based on data provided prior to the assessment.

\begin{tabular}{@{}ll}
\toprule
\textbf{Attribute} & \textbf{Value} \\
\midrule
Organization Name & \textbf{[Organization Name]} \\
Primary Email Domain & \texttt{[Domain]} \\
External IP Address Scanned & \texttt{[Client IP]} \\
Target IP Address Scanned & \texttt{[Target IP]} \\
\bottomrule
\end{tabular}

% --- Section 3: Security Control Review ---
\section{Security Control Review (Questionnaire Analysis)}
The following table summarizes the organization's responses to a security controls questionnaire. The assessment column indicates alignment with cybersecurity best practices. Answers marked with a \no{} represent significant gaps in the current security posture.

\begin{table}[h!]
\centering
\caption{Security Controls Questionnaire Results}
\begin{tabular}{@{}p{0.6\linewidth} c l}
\toprule
\textbf{Control Question} & \textbf{Response} & \textbf{Assessment} \\
\midrule
Do you require MFA to access email? & \no & \cellgray \textbf{Critical Gap} \\
Do you require MFA to log into computers? & \no & \cellgray \textbf{High Risk} \\
Do you require MFA to access sensitive data systems? & \yes & Best Practice Met \\
Does your organization have an employee acceptable use policy? & \no & \cellgray \textbf{High Risk} \\
Does your organization do security awareness training for new employees? & \yes & Best Practice Met \\
Does your organization do security awareness training for all employees at least once per year? & \yes & Best Practice Met \\
\bottomrule
\end{tabular}
\end{table}

The lack of MFA for email and computer access, combined with the absence of an Acceptable Use Policy, are foundational weaknesses that undermine the organization's overall security.

% --- Section 4: Technical Scan Results ---
\section{Technical Scan Results}
An external network scan was performed to identify open ports and exposed services on the target system.

\begin{itemize}
    \item \textbf{Scan Target:} \texttt{[Target IP]}
    \item \textbf{Scan Date:} Scan data processed on \today
\end{itemize}

The scan identified the following open port:

\begin{table}[h!]
\centering
\caption{Open Ports Detected on \texttt{[Target IP]}}
\begin{tabular}{@{}llllll}
\toprule
\textbf{Port} & \textbf{State} & \textbf{Service} & \textbf{Product} & \textbf{Version} & \textbf{Notes} \\
\midrule
3306/tcp & open & mysql & MySQL & 5.7.33 & \cellgray \textbf{End-of-Life (EOL)} \\
\bottomrule
\end{tabular}
\end{table}

\textbf{Analysis:} The scan confirms the pre-existing risk of database exposure. The MySQL service on port 3306 is open to the network. More critically, the detected version, \textbf{MySQL 5.7.33}, is no longer supported by its developer as of October 2023. It does not receive security patches for newly discovered vulnerabilities, making it a prime target for attackers.

% --- Section 5: Correlated Risk Assessment ---
\section{Correlated Risk Assessment}
This section synthesizes findings from the questionnaire, technical scan, and pre-existing risk data into a consolidated list of key risks.

\begin{table}[h!]
\centering
\caption{Summary of Identified Risks}
\resizebox{\textwidth}{!}{%
\begin{tabular}{@{}lp{0.25\linewidth}p{0.1\linewidth}p{0.4\linewidth}@{}}
\toprule
\textbf{Risk ID} & \textbf{Risk Title} & \textbf{Severity} & \textbf{Description} \\
\midrule
\rowcolor{red!25}
RISK-001 & Unsupported Database Software & \textbf{Critical} & The MySQL server is running version 5.7.33, which is End-of-Life (EOL) and no longer receives security updates. \\
\rowcolor{orange!25}
RISK-002 & Publicly Exposed Database Service & \textbf{High} & The MySQL database port (3306) is open to the network, allowing direct connection attempts from potential attackers. (CVSS: 7.5) \\
\rowcolor{red!25}
RISK-003 & Lack of MFA for Email Access & \textbf{Critical} & The absence of MFA on email accounts makes them highly susceptible to phishing and account takeover attacks. \\
\rowcolor{orange!25}
RISK-004 & Lack of MFA for Endpoints & \textbf{High} & The absence of MFA on computer logins weakens endpoint security and facilitates lateral movement for an attacker who has compromised user credentials. \\
\rowcolor{orange!25}
RISK-005 & Missing Acceptable Use Policy & \textbf{High} & The lack of a formal AUP creates operational and legal risks, as employees are not provided with clear rules for handling company data and systems. \\
\bottomrule
\end{tabular}
}
\end{table}

% --- Section 6: Recommendations ---
\section{Recommendations}
The following actionable recommendations are provided to mitigate the identified risks. They are prioritized based on severity and potential impact.

\subsection{Immediate Actions (Priority: Critical)}
\begin{enumerate}
    \item \textbf{Isolate the Database Server (RISK-002):} Immediately implement firewall rules to restrict all access to TCP port 3306. Access should be denied from all public IP addresses and only permitted from trusted internal application servers.
    \item \textbf{Deploy MFA for Email (RISK-003):} Immediately enable and enforce Multi-Factor Authentication for all user email accounts. This is the single most effective control to prevent account takeovers.
\end{enumerate}

\subsection{Short-Term Actions (Priority: High)}
\begin{enumerate}
    \item \textbf{Plan and Execute Database Upgrade (RISK-001):} Develop a plan to migrate the MySQL 5.7.33 database to a fully supported version (e.g., MySQL 8.x). This project should be initiated within the next 30 days.
    \item \textbf{Deploy MFA for Endpoints (RISK-004):} Roll out MFA for all employee computer and laptop logins to enhance endpoint security.
    \item \textbf{Develop and Implement an Acceptable Use Policy (RISK-005):} Draft, approve, and disseminate an AUP that clearly defines the rules for using company IT assets. Require all employees to read and acknowledge the policy.
\end{enumerate}

\subsection{Long-Term Strategic Improvements}
\begin{enumerate}
    \item \textbf{Implement Network Segmentation:} As a long-term solution, migrate the database server to a private, segregated network segment. Access should only be possible from a trusted application network or via a secure Virtual Private Network (VPN) for administrative purposes.
    \item \textbf{Establish a Vulnerability Management Program:} Implement a program for regular, automated vulnerability scanning of all external and internal assets to proactively identify and remediate risks like EOL software and misconfigurations.
\end{enumerate}

\end{document}
```