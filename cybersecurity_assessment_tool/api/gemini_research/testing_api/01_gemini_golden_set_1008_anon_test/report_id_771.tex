```latex
\documentclass[12pt]{article}

% --- PACKAGES ---
\usepackage[margin=1in]{geometry}
\usepackage{pifont} % For checkmarks and crosses (\ding)
\usepackage{booktabs} % For professional tables
\usepackage[hidelinks]{hyperref} % For clickable links without boxes
\usepackage{url} % For URL formatting
\usepackage{seqsplit} % For splitting long strings in texttt

% --- DOCUMENT INFORMATION ---
\title{
    Cybersecurity Posture Assessment Report \\
    \large For: \textbf{[Organization Name]}
}
\author{Cybersecurity Analysis Division}
\date{\today}

% --- DOCUMENT START ---
\begin{document}

\maketitle
\thispagestyle{empty}
\newpage

\tableofcontents
\newpage

% ===================================================================
% SECTION 1: EXECUTIVE SUMMARY
% ===================================================================
\section{Executive Summary}

This report provides a comprehensive assessment of the cybersecurity posture for \textbf{[Organization Name]}, based on an analysis of technical network scans, a review of organizational security controls, and pre-existing risk data. The assessment was conducted on \today.

The overall security posture is determined to be at a \textbf{High-Risk} level. This is primarily due to the discovery of a critical vulnerability: an internet-facing database running an End-of-Life (EOL) version of MySQL (5.7.33), which no longer receives security updates. This technical finding is compounded by significant gaps in foundational security policies and employee training.

Key findings include:
\begin{itemize}
    \item \textbf{Critical Service Exposure:} A MySQL database (port 3306) is directly exposed to the network, presenting a large attack surface.
    \item \textbf{End-of-Life Software:} The exposed database is running MySQL 5.7.33, which reached its end of life in October 2023 and is vulnerable to numerous unpatched exploits.
    \item \textbf{Policy and Training Deficiencies:} The organization lacks a formal Acceptable Use Policy and does not conduct security awareness training for new or existing employees. This elevates the risk of human error, such as weak password usage or falling victim to phishing attacks, which could lead to a compromise of the exposed database.
\end{itemize}

Immediate action is required to remediate the exposed database and EOL software. Furthermore, establishing a robust security awareness program and formalizing security policies are crucial steps to improving the organization's long-term defensive capabilities.

% ===================================================================
% SECTION 2: ORGANIZATIONAL INFORMATION
% ===================================================================
\section{Organizational Information}

The following details were used as the basis for this assessment. The placeholders indicate that this information was not provided and should be updated in the organization's records.

\begin{itemize}
    \item \textbf{Organization Name:} \textbf{[Organization Name]}
    \item \textbf{Primary Domain:} \texttt{[Domain]}
    \item \textbf{External IP Scanned:} \texttt{[Client IP]}
\end{itemize}

% ===================================================================
% SECTION 3: SECURITY CONTROL REVIEW
% ===================================================================
\section{Security Control Review}

A review of the organization's security controls was conducted via a questionnaire. The responses highlight critical gaps in administrative and procedural controls, particularly concerning employee awareness and policy enforcement.

\subsection{Questionnaire Responses}

\begin{table}[h!]
\centering
\caption{Security Control Questionnaire Results}
\begin{tabular}{p{0.75\linewidth} c}
\toprule
\textbf{Control Question} & \textbf{Response} \\
\midrule
Do you require MFA to access email? & \ding{51} \\
Do you require MFA to log into computers? & \ding{51} \\
Do you require MFA to access sensitive data systems? & \ding{51} \\
Does your organization have an employee acceptable use policy? & \textbf{\color{red}\ding{55}} \\
Does your organization do security awareness training for new employees? & \textbf{\color{red}\ding{55}} \\
Does your organization do security awareness training for all employees at least once per year? & \textbf{\color{red}\ding{55}} \\
\bottomrule
\end{tabular}
\end{table}

\subsection{Analysis of Gaps}
The organization has successfully implemented Multi-Factor Authentication (MFA) across key systems, which is a commendable strength. However, the negative responses (\ding{55}) represent significant risks:
\begin{itemize}
    \item \textbf{Lack of Acceptable Use Policy:} Without a formal policy, there are no established rules for how employees should use company assets, handle data, or what constitutes prohibited activity. This creates legal and security ambiguities.
    \item \textbf{No Security Awareness Training:} Employees are the first line of defense. Without training, they are significantly more vulnerable to social engineering, phishing, and other common attack vectors. This deficiency directly increases the likelihood of a security breach.
\end{itemize}

% ===================================================================
% SECTION 4: TECHNICAL SCAN RESULTS
% ===================================================================
\section{Technical Scan Results}

A network scan was performed on the target IP address to identify open ports and exposed services.

\begin{itemize}
    \item \textbf{Target IP:} \texttt{[Target IP]}
    \item \textbf{Scan Status:} Host is Up
\end{itemize}

\begin{table}[h!]
\centering
\caption{Open Port Analysis}
\begin{tabular}{l l l l l}
\toprule
\textbf{Port} & \textbf{State} & \textbf{Service} & \textbf{Product} & \textbf{Version} \\
\midrule
3306/tcp & open & mysql & MySQL & 5.7.33 \\
\bottomrule
\end{tabular}
\end{table}

\subsection{Technical Analysis}
The scan identified one open port, 3306, which is the default port for the MySQL database service. This finding is critical for two reasons:
\begin{enumerate}
    \item \textbf{Direct Database Exposure:} Exposing a database directly to the public internet is extremely dangerous. It allows attackers to perform reconnaissance, attempt brute-force password attacks, and exploit any vulnerabilities present in the database software.
    \item \textbf{End-of-Life (EOL) Software:} The identified version, \textbf{MySQL 5.7.33}, reached its official End of Life in \textbf{October 2023}. This means it no longer receives security patches from the vendor, and any vulnerabilities discovered since that date will remain unpatched. This elevates the risk of compromise from "High" to "Critical."
\end{enumerate}

% ===================================================================
% SECTION 5: CONSOLIDATED RISK ASSESSMENT
% ===================================================================
\section{Consolidated Risk Assessment}

The following table synthesizes findings from the organizational review, technical scan, and pre-existing risk data into a consolidated list of security risks.

\begin{table}[h!]
\centering
\caption{Summary of Identified Risks}
\begin{tabular}{p{0.2\linewidth} p{0.45\linewidth} p{0.1\linewidth} p{0.15\linewidth}}
\toprule
\textbf{Risk Name} & \textbf{Description} & \textbf{Severity} & \textbf{Affected Assets} \\
\midrule
\textbf{EOL Software Exposure} & A MySQL 5.7.33 database, which is no longer supported with security patches, is exposed to the network. & \textbf{Critical} (9.8) & Server at \texttt{[Target IP]} \\
\addlinespace
\textbf{Database Exposure} & The MySQL database port (3306) is open, allowing direct connection attempts from any location. & \textbf{High} (7.5) & Port 3306 (MySQL) \\
\addlinespace
\textbf{Lack of Security Policies \& Training} & The absence of an Acceptable Use Policy and security awareness training increases the likelihood of human-related security incidents. & \textbf{High} & All Employees, All Systems \\
\bottomrule
\end{tabular}
\end{table}

% ===================================================================
% SECTION 6: RECOMMENDATIONS
% ===================================================================
\section{Recommendations}

The following actions are recommended to mitigate the identified risks. They are prioritized based on severity and potential impact.

\subsection{Critical Priority}
\begin{enumerate}
    \item \textbf{Immediately Restrict Access to Port 3306:} Configure firewall rules to deny all public access to TCP port 3306. Access should only be permitted from specific, trusted IP addresses or through a secure VPN connection.
    \item \textbf{Plan and Execute Database Upgrade:} Develop an urgent plan to migrate the MySQL 5.7.33 database to a currently supported version (e.g., MySQL 8.x). This is essential to ensure the system receives critical security patches.
\end{enumerate}

\subsection{High Priority}
\begin{enumerate}
    \item \textbf{Develop and Implement an Acceptable Use Policy (AUP):} Draft a formal AUP that clearly defines the rules for using company technology and data. This policy should be communicated to all employees and acknowledged in writing.
    \item \textbf{Establish a Security Awareness Training Program:}
    \begin{itemize}
        \item Implement a mandatory training module for all new hires as part of their onboarding process.
        \item Conduct annual security awareness training for all staff to keep them informed about current threats, such as phishing and social engineering.
    \end{itemize}
\end{enumerate}

\subsection{Medium Priority}
\begin{enumerate}
    \item \textbf{Implement Continuous Vulnerability Scanning:} Deploy an automated scanning solution to proactively identify and report on technical vulnerabilities across the organization's infrastructure on an ongoing basis.
\end{enumerate}

\end{document}
```