```latex
\documentclass[12pt, a4paper]{article}

% --- PACKAGES ---
\usepackage[margin=1in]{geometry}
\usepackage{pifont} % For checkmarks and crosses
\usepackage{booktabs} % For professional tables
\usepackage{hyperref} % For hyperlinks
\usepackage{url} % For URL formatting
\usepackage{seqsplit} % To split long monospaced text
\usepackage{graphicx} % For logo
\usepackage{fancyhdr} % For headers/footers
\usepackage{xcolor} % For colors

% --- DOCUMENT METADATA & STYLING ---
\hypersetup{
    colorlinks=true,
    linkcolor=blue,
    filecolor=magenta,      
    urlcolor=cyan,
    pdftitle={Cybersecurity Posture Assessment Report},
    pdfpagemode=FullScreen,
}

\definecolor{darkblue}{rgb}{0.0, 0.0, 0.55}
\definecolor{darkred}{rgb}{0.55, 0.0, 0.0}

% --- HEADER & FOOTER ---
\pagestyle{fancy}
\fancyhf{} % Clear all header and footer fields
\fancyhead[L]{Cybersecurity Posture Assessment}
\fancyhead[R]{\textbf{[Organization Name]}}
\fancyfoot[C]{\thepage}
\renewcommand{\headrulewidth}{0.4pt}
\renewcommand{\footrulewidth}{0.4pt}

% --- TITLE ---
\title{
    \vspace{2cm}
    \textbf{Cybersecurity Posture Assessment Report}\\
    \large \today
}
\author{Cybersecurity Analysis Division}
\date{}

% --- DOCUMENT START ---
\begin{document}

\maketitle
\thispagestyle{empty}
\newpage

\tableofcontents
\newpage

% --- EXECUTIVE SUMMARY ---
\section{Executive Summary}

This report details the findings of a cybersecurity posture assessment conducted for \textbf{[Organization Name]}. The evaluation combined a review of organizational security controls via a questionnaire, an external network vulnerability scan, and an analysis of pre-existing risks.

The assessment identified a significant administrative gap in the employee onboarding process. Specifically, the lack of mandatory security awareness training for new hires presents a \textbf{High} risk. New employees are often prime targets for social engineering attacks, and this gap leaves the organization vulnerable during a critical transitional period. While an annual training program is in place for all staff, the initial lack of training for new personnel must be addressed immediately.

On a positive note, the external network scan of the designated target IP address (\texttt{[Target IP]}) did not reveal any open ports or services. This suggests a strong perimeter defense and a well-configured firewall, which is a commendable security practice. No technical vulnerabilities were discovered from this external perspective.

The primary recommendation is to implement a mandatory security awareness training module into the standard employee onboarding process. This single action will significantly mitigate the identified risk and improve the organization's overall security resilience.

% --- ORGANIZATIONAL INFORMATION ---
\section{Organizational Information}

The following details were used as the basis for this assessment. Where information was not provided, placeholders have been used.

\begin{itemize}
    \item \textbf{Organization Name:} \textbf{[Organization Name]}
    \item \textbf{Primary Domain:} \texttt{[Domain]}
    \item \textbf{Scanned External IP:} \texttt{[Client IP]}
\end{itemize}

% --- SECURITY CONTROL REVIEW ---
\section{Security Control Review (Questionnaire)}

A review of administrative and technical controls was conducted based on a standardized questionnaire. The responses indicate a generally strong adoption of multi-factor authentication (MFA) and policy frameworks. However, a critical gap was identified in the security training lifecycle.

\begin{table}[h!]
\centering
\caption{Security Controls Questionnaire Results}
\begin{tabular}{p{0.8\linewidth} c}
\toprule
\textbf{Control Question} & \textbf{Response} \\
\midrule
Do you require MFA to access email? & \textcolor{green}{\ding{51}} \\
Do you require MFA to log into computers? & \textcolor{green}{\ding{51}} \\
Do you require MFA to access sensitive data systems? & \textcolor{green}{\ding{51}} \\
Does your organization have an employee acceptable use policy? & \textcolor{green}{\ding{51}} \\
\textbf{Does your organization do security awareness training for new employees?} & \textcolor{darkred}{\ding{55}} \\
Does your organization do security awareness training for all employees at least once per year? & \textcolor{green}{\ding{51}} \\
\bottomrule
\end{tabular}
\end{table}

\paragraph{Analysis:} The failure to provide security awareness training to new employees represents a significant oversight. This period is when employees are most vulnerable to social engineering and are still learning organizational norms and policies. This gap is the primary source of risk identified in this report.

% --- TECHNICAL SCAN RESULTS ---
\section{Technical Scan Results}

An external network scan was performed to identify potential vulnerabilities in internet-facing systems.

\begin{itemize}
    \item \textbf{Target IP:} \texttt{[Target IP]}
    \item \textbf{Scan Date:} \today
\end{itemize}

\subsection{Summary of Findings}
The network scan did not identify any open TCP or UDP ports on the target host. This is a positive security finding, indicating that the host is likely protected by a well-configured firewall that properly implements a default-deny policy for unsolicited inbound traffic. Alternatively, the host may have been offline or configured not to respond to scan probes.

Based on this scan, no externally-facing vulnerabilities were discovered.

% --- RISK ASSESSMENT ---
\section{Risk Assessment}

This section synthesizes findings from the security control review, technical scan, and any pre-existing risk data. The following table details the identified risks to the organization.

\begin{table}[h!]
\centering
\caption{Identified Risk Summary}
\begin{tabular}{p{0.1\linewidth} p{0.25\linewidth} p{0.45\linewidth} p{0.1\linewidth}}
\toprule
\textbf{Risk ID} & \textbf{Risk Name} & \textbf{Description} & \textbf{Severity} \\
\midrule
RISK-001 & Lack of Onboarding Security Training & New employees do not receive security awareness training upon being hired. This creates a window of vulnerability where new staff are unaware of security policies, phishing threats, and data handling procedures, making them prime targets for social engineering attacks. & \textbf{High} \\
\bottomrule
\end{tabular}
\end{table}

\paragraph{Note:} No pre-existing vulnerabilities were provided for this assessment. The risk identified above is based solely on the data gathered during this engagement.

% --- RECOMMENDATIONS ---
\section{Recommendations}

The following actionable recommendations are provided to mitigate the identified risks and improve the overall security posture of \textbf{[Organization Name]}.

\begin{table}[h!]
\centering
\caption{Remediation Plan}
\begin{tabular}{p{0.1\linewidth} p{0.7\linewidth} p{0.1\linewidth}}
\toprule
\textbf{Risk ID} & \textbf{Recommendation} & \textbf{Priority} \\
\midrule
RISK-001 & Implement a mandatory security awareness training module as part of the new employee onboarding process. This training should cover, at a minimum: acceptable use policies, phishing and social engineering identification, password security, and procedures for reporting security incidents. This will close the initial vulnerability gap and ensure a consistent security baseline for all employees from day one. & \textbf{High} \\
\bottomrule
\end{tabular}
\end{table}

\end{document}
```