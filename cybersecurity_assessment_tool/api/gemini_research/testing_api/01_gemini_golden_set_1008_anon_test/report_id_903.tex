```latex
\documentclass[12pt]{article}

% Preamble: Required Packages
\usepackage[margin=1in]{geometry}
\usepackage{pifont} % For checkmarks and crosses
\usepackage{booktabs} % For professional tables
\usepackage{hyperref} % For clickable links
\usepackage{url} % For formatting URLs
\usepackage{seqsplit} % For splitting long strings
\usepackage{graphicx}
\usepackage{xcolor}

% --- Document Metadata ---
\title{Cybersecurity Posture Assessment Report}
\author{Cybersecurity Analysis Division}
\date{\today}

% --- Hyperref Setup ---
\hypersetup{
    colorlinks=true,
    linkcolor=blue,
    filecolor=magenta,      
    urlcolor=cyan,
    pdftitle={Cybersecurity Posture Assessment Report},
    pdfpagemode=FullScreen,
}

\begin{document}

\maketitle
\thispagestyle{empty}
\newpage
\tableofcontents
\newpage

% ==============================================================================
% SECTION 1: EXECUTIVE SUMMARY
% ==============================================================================
\section{Executive Summary}

This report provides a comprehensive cybersecurity assessment for \textbf{[Organization Name]}. The analysis is based on a correlation of external network scan data, a review of internal security controls via a questionnaire, and an evaluation of pre-existing risk documentation.

The assessment reveals a critical risk posture. A key external-facing asset, a MySQL database server at \texttt{[Client IP]}, is publicly exposed. This server is running MySQL version 5.7.33, which is an End-of-Life (EOL) product and no longer receives security updates. This technical vulnerability is severely compounded by significant gaps in administrative controls, most notably the lack of Multi-Factor Authentication (MFA) for computer and sensitive system access.

Immediate action is required to restrict access to the exposed database. Subsequently, the organization must prioritize the implementation of foundational security controls, including comprehensive MFA and the development of core security policies, to mitigate the high likelihood of a security breach.

% ==============================================================================
% SECTION 2: ORGANIZATIONAL INFORMATION
% ==============================================================================
\section{Organizational Information}

The following details were used as the basis for this assessment. Placeholders are used where data was not provided.

\begin{tabular}{@{}ll}
\toprule
\textbf{Attribute} & \textbf{Value} \\
\midrule
Organization Name & \textbf{[Organization Name]} \\
Primary Domain & \texttt{[Domain]} \\
External IP Scanned & \texttt{[Client IP]} \\
Target of Scan & \texttt{[Target IP]} \\
\bottomrule
\end{tabular}

% ==============================================================================
% SECTION 3: SECURITY CONTROL REVIEW (QUESTIONNAIRE)
% ==============================================================================
\section{Security Control Review}

An internal security questionnaire was reviewed to assess the maturity of administrative and procedural controls. The results indicate significant gaps in foundational security practices. A "No" answer (\ding{55}) represents a control failure that increases organizational risk.

\begin{table}[h!]
\centering
\caption{Security Controls Questionnaire Analysis}
\begin{tabular}{@{}p{0.7\textwidth}c}
\toprule
\textbf{Control Question} & \textbf{Status} \\
\midrule
Do you require MFA to access email? & \textcolor{green}{\ding{51}} \\
Do you require MFA to log into computers? & \textcolor{red}{\ding{55}} \\
Do you require MFA to access sensitive data systems? & \textcolor{red}{\ding{55}} \\
Does your organization have an employee acceptable use policy? & \textcolor{red}{\ding{55}} \\
Does your organization do security awareness training for new employees? & \textcolor{red}{\ding{55}} \\
Does your organization do security awareness training for all employees at least once per year? & \textcolor{green}{\ding{51}} \\
\bottomrule
\end{tabular}
\end{table}

\subsection*{Analysis of Control Gaps}
\begin{itemize}
    \item \textbf{Lack of MFA:} The absence of MFA on computer logins and sensitive data systems is a critical weakness. Should an employee's credentials be compromised, an attacker would have direct access to internal resources without needing a second authentication factor.
    \item \textbf{Missing Acceptable Use Policy (AUP):} Without a formal AUP, there is no documented standard for how employees should use company assets, handle data, or behave online, creating legal and security ambiguities.
    \item \textbf{No Onboarding Security Training:} Failing to train new employees on security best practices from day one leaves a critical window of vulnerability and fosters an insecure culture.
\end{itemize}

% ==============================================================================
% SECTION 4: TECHNICAL SCAN RESULTS
% ==============================================================================
\section{Technical Scan Results}

An external network scan was performed on the target IP address to identify open ports and exposed services.

\subsection*{Scan Details}
\begin{itemize}
    \item \textbf{Target IP:} \texttt{[Target IP]} (Derived from \texttt{[Client IP]})
    \item \textbf{Scan Status:} Host is UP.
\end{itemize}

\subsection*{Open Ports Discovered}
The following table details the services found to be publicly accessible.

\begin{table}[h!]
\centering
\caption{Exposed Network Services}
\begin{tabular}{@{}lllll}
\toprule
\textbf{Port} & \textbf{State} & \textbf{Service} & \textbf{Product} & \textbf{Version} \\
\midrule
3306/tcp & open & mysql & MySQL & 5.7.33 \\
\bottomrule
\end{tabular}
\end{table}

\subsection*{Technical Analysis}
The scan identified a publicly accessible MySQL database on port 3306. This is a significant security risk.
\begin{itemize}
    \item \textbf{Direct Database Exposure:} Exposing a database directly to the internet is highly discouraged. It makes the database a direct target for brute-force attacks, credential stuffing, and exploitation of known vulnerabilities.
    \item \textbf{End-of-Life (EOL) Software:} The detected version, \textbf{MySQL 5.7.33}, reached its official End-of-Life in \textbf{October 2023}. EOL software no longer receives security patches from the vendor, meaning any newly discovered vulnerabilities will remain unpatched, leaving the system permanently vulnerable to exploitation.
\end{itemize}

% ==============================================================================
% SECTION 5: CORRELATED RISK ASSESSMENT
% ==============================================================================
\section{Correlated Risk Assessment}

This section synthesizes findings from the security control review, technical scan, and pre-existing risk data to provide a holistic view of the primary risks facing the organization.

\begin{table}[h!]
\centering
\caption{Summary of Key Risks}
\begin{tabular}{@{}p{0.25\textwidth}p{0.5\textwidth}p{0.15\textwidth}}
\toprule
\textbf{Risk Name} & \textbf{Description} & \textbf{Severity} \\
\midrule
\textbf{Exposed End-of-Life Database} & A MySQL 5.7 (EOL) database is publicly accessible on port 3306. This exposes the organization to data breach via unpatchable vulnerabilities and brute-force attacks. This risk is confirmed by both the scan and existing risk documentation. & \textbf{Critical} \\
\addlinespace
\textbf{Insufficient MFA Implementation} & Lack of MFA on computers and sensitive systems allows an attacker with stolen credentials to gain deep access to the internal network and data, including the exposed database. & \textbf{High} \\
\addlinespace
\textbf{Foundational Policy Gaps} & The absence of an Acceptable Use Policy and security training for new hires creates a weak security culture, increasing the likelihood of human error leading to incidents like credential compromise. & \textbf{Medium} \\
\bottomrule
\end{tabular}
\end{table}

% ==============================================================================
% SECTION 6: RECOMMENDATIONS
% ==============================================================================
\section{Recommendations}

The following actionable recommendations are provided to address the identified risks. They are prioritized based on severity and ease of implementation.

\subsection*{Immediate Actions (To be completed within 7 days)}
\begin{enumerate}
    \item \textbf{Restrict Database Access:} Immediately implement a firewall rule to \textbf{block all public inbound traffic} to TCP port 3306 on \texttt{[Client IP]}. Access should be restricted to a whitelist of trusted IP addresses or require a Virtual Private Network (VPN) connection.
\end{enumerate}

\subsection*{Short-Term Actions (To be completed within 1-3 months)}
\begin{enumerate}
    \item \textbf{Deploy Comprehensive MFA:} Enforce MFA for all users across all critical systems, prioritizing:
    \begin{itemize}
        \item All computer/endpoint logins (Windows, macOS).
        \item All access to sensitive data systems, including the MySQL database.
    \end{itemize}
    \item \textbf{Develop and Implement Core Policies:}
    \begin{itemize}
        \item Draft and ratify an Employee Acceptable Use Policy (AUP).
        \item Create a mandatory security awareness module as part of the new employee onboarding process.
    \end{itemize}
\end{enumerate}

\subsection*{Long-Term Actions (To be completed within 3-6 months)}
\begin{enumerate}
    \item \textbf{Upgrade End-of-Life Database:} Plan and execute the migration of the MySQL 5.7.33 database to a fully supported version (e.g., MySQL 8.x). This is critical for long-term security and stability, as it ensures the system receives ongoing security patches.
\end{enumerate}

\end{document}
```