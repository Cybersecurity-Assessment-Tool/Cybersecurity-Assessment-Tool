```latex
\documentclass[12pt, a4paper]{article}

% Preamble: Required Packages
\usepackage[margin=1in]{geometry}
\usepackage{pifont} % For checkmarks and crosses
\usepackage{booktabs} % For professional tables
\usepackage{hyperref} % For clickable links
\usepackage{url} % For URL formatting
\usepackage{seqsplit} % To split long text sequences
\usepackage{xcolor} % For colors
\usepackage{graphicx} % For logo (placeholder)
\usepackage{fancyhdr} % For header/footer

% --- Document Setup ---
\hypersetup{
    colorlinks=true,
    linkcolor=blue,
    filecolor=magenta,      
    urlcolor=cyan,
    pdftitle={Cybersecurity Posture Assessment Report},
    pdfpagemode=FullScreen,
}

% Define colors for risk levels
\definecolor{criticalrisk}{HTML}{D32F2F}
\definecolor{highrisk}{HTML}{F57C00}
\definecolor{mediumrisk}{HTML}{FBC02D}
\definecolor{lowrisk}{HTML}{388E3C}

% --- Header and Footer ---
\pagestyle{fancy}
\fancyhf{} % clear all header and footer fields
\fancyhead[L]{\textbf{Cybersecurity Posture Assessment}}
\fancyhead[R]{\textbf{[Organization Name]}}
\fancyfoot[C]{\thepage}
\renewcommand{\headrulewidth}{0.4pt}
\renewcommand{\footrulewidth}{0.4pt}

% --- Document Start ---
\begin{document}

% --- Title Page ---
\begin{titlepage}
    \centering
    \vspace*{1cm}
    
    \Huge{\textbf{Cybersecurity Posture Assessment Report}}
    
    \vspace{1.5cm}
    
    \Large{\textbf{Prepared for:}} \\
    \vspace{0.5cm}
    \huge{\textbf{[Organization Name]}}
    
    \vfill
    
    \Large{\textbf{Analysis Date:}} \\
    \large{\today}
    
    \vspace{0.5cm}
    
    \Large{\textbf{Author:}} \\
    \large{Cybersecurity Analyst}

\end{titlepage}

\tableofcontents
\newpage

% --- Section 1: Executive Summary ---
\section{Executive Summary}
This report provides a comprehensive analysis of the cybersecurity posture for \textbf{[Organization Name]}, based on a review of organizational security controls, an external network scan, and pre-existing risk data. The assessment was conducted to identify vulnerabilities, evaluate current security practices, and provide actionable recommendations to enhance the organization's security resilience.

\paragraph{Key Findings:} The organization has implemented several positive security controls, including mandatory Multi-Factor Authentication (MFA) for email and computer access, and maintains an acceptable use policy. However, two critical gaps were identified through the security questionnaire:
\begin{itemize}
    \item \textbf{Critical Risk:} Multi-Factor Authentication is not required for accessing sensitive data systems. This significantly increases the risk of unauthorized access to critical information assets.
    \item \textbf{High Risk:} Security awareness training is not conducted annually for all employees, which can lead to a workforce that is more susceptible to social engineering attacks like phishing.
\end{itemize}

\paragraph{Technical Assessment:} The external network scan performed on the target IP address \texttt{[Target IP]} revealed no open ports. Notably, Port 80 (HTTP), which was listed as a concern in a previous risk assessment, was found to be closed. This indicates that the previously identified risk of an "Unencrypted Web Server" has likely been remediated.

\paragraph{Overall Posture:} While foundational controls are in place, the identified gaps in MFA for sensitive systems and recurring employee training represent significant weaknesses. Addressing these issues should be the top priority to mitigate the risk of a data breach.

\newpage

% --- Section 2: Organizational Information ---
\section{Organizational Information}
This section details the organizational data used for this assessment. As the data provided was anonymized, placeholders have been used.

\begin{table}[h!]
\centering
\caption{Client Organizational Details}
\begin{tabular}{@{}ll@{}}
\toprule
\textbf{Attribute} & \textbf{Value} \\ \midrule
Organization Name    & \textbf{[Organization Name]} \\
Primary Domain       & \texttt{[Domain]} \\
External IP Scanned  & \texttt{[Target IP]} \\ \bottomrule
\end{tabular}
\end{table}

% --- Section 3: Security Control Review ---
\section{Security Control Review}
The following table summarizes the organization's responses to a security controls questionnaire. The status indicates whether the control is in place ("Yes") or not ("No"). "No" answers are considered significant findings that require remediation.

\begin{table}[h!]
\centering
\caption{Security Controls Questionnaire Results}
\label{tab:controls}
\begin{tabular}{@{}p{0.7\textwidth}c@{}}
\toprule
\textbf{Control Question} & \textbf{Status} \\ \midrule
Do you require MFA to access email? & \textcolor{green}{\ding{51}} \\
Do you require MFA to log into computers? & \textcolor{green}{\ding{51}} \\
\textbf{Do you require MFA to access sensitive data systems?} & \textbf{\textcolor{red}{\ding{55}}} \\
Does your organization have an employee acceptable use policy? & \textcolor{green}{\ding{51}} \\
Does your organization do security awareness training for new employees? & \textcolor{green}{\ding{51}} \\
\textbf{Does your organization do security awareness training for all employees at least once per year?} & \textbf{\textcolor{red}{\ding{55}}} \\ \bottomrule
\end{tabular}
\end{table}

\subsection{Analysis of Control Gaps}
The review identified two primary control gaps:
\begin{itemize}
    \item \textbf{Lack of MFA on Sensitive Systems:} The absence of MFA on systems containing sensitive data is a critical vulnerability. Should an attacker compromise a user's credentials, they would have direct access to the organization's most valuable information.
    \item \textbf{Lack of Annual Security Training:} The threat landscape is constantly evolving. Without regular, recurring training, employees may not be able to recognize and appropriately respond to modern threats like sophisticated phishing emails, leading to a higher likelihood of a security incident.
\end{itemize}

% --- Section 4: Technical Scan Results ---
\section{Technical Scan Results}
An external network scan was conducted using Nmap to identify open ports and services on the public-facing infrastructure.

\begin{itemize}
    \item \textbf{Target IP Address:} \texttt{[Target IP]}
    \item \textbf{Scan Date:} Not specified in scan data.
    \item \textbf{Host Status:} Up
\end{itemize}

\begin{table}[h!]
\centering
\caption{Port Scan Results for \texttt{[Target IP]}}
\label{tab:scanresults}
\begin{tabular}{@{}llll@{}}
\toprule
\textbf{Port} & \textbf{State} & \textbf{Service} & \textbf{Notes} \\ \midrule
80/tcp & closed & http & Port is not accessible from the internet. \\ \bottomrule
\end{tabular}
\end{table}

\subsection{Analysis of Scan Findings}
The scan results are positive, indicating a minimal external attack surface. No open ports were discovered. This finding is particularly important as it contradicts a pre-existing risk concerning an "Unencrypted Web Server" on Port 80. The current scan indicates that this port is closed, suggesting the risk has been successfully remediated.

\newpage

% --- Section 5: Consolidated Risk Assessment ---
\section{Consolidated Risk Assessment}
This section synthesizes findings from the security control review, technical scan, and pre-existing risk data into a consolidated list of risks.

\begin{table}[h!]
\centering
\caption{Summary of Identified Risks}
\label{tab:risks}
\begin{tabular}{@{}p{0.3\textwidth}p{0.4\textwidth}p{0.1\textwidth}p{0.15\textwidth}@{}}
\toprule
\textbf{Risk Name} & \textbf{Overview} & \textbf{Severity} & \textbf{Status} \\ \midrule
\textbf{Lack of MFA on Sensitive Systems} & User accounts for sensitive systems are protected only by passwords, making them vulnerable to credential theft and unauthorized access. & \textcolor{criticalrisk}{\textbf{Critical}} & \textbf{Active} \\
\addlinespace
\textbf{Inadequate Security Awareness Training} & Employees do not receive annual security training, increasing susceptibility to social engineering and phishing attacks. & \textcolor{highrisk}{\textbf{High}} & \textbf{Active} \\
\addlinespace
Unencrypted Web Server & Previous finding indicated that Port 80 was open, exposing unencrypted HTTP traffic. & \textcolor{mediumrisk}{Medium} & \textbf{Likely Remediated} \\ \bottomrule
\end{tabular}
\end{table}

% --- Section 6: Recommendations ---
\section{Recommendations}
The following recommendations are prioritized based on the risk assessment to provide a clear roadmap for improving the organization's security posture.

\subsection{Priority 1: Critical Risk}
\begin{description}
    \item[Implement MFA on Sensitive Systems:]
    \begin{itemize}
        \item \textbf{Action:} Enforce mandatory Multi-Factor Authentication (MFA) for all user accounts (including administrative and service accounts) that have access to systems classified as containing sensitive or critical data.
        \item \textbf{Impact:} Drastically reduces the risk of unauthorized access due to compromised credentials. This is the single most effective control to prevent unauthorized access.
        \item \textbf{Timeline:} Immediate (within 30 days).
    \end{itemize}
\end{description}

\subsection{Priority 2: High Risk}
\begin{description}
    \item[Establish Annual Security Awareness Training:]
    \begin{itemize}
        \item \textbf{Action:} Develop and deploy a mandatory security awareness training program for all employees, to be completed annually. The program should cover key topics such as phishing, password security, and acceptable use.
        \item \textbf{Impact:} Creates a more security-conscious culture and reduces the likelihood of human error leading to a security incident.
        \item \textbf{Timeline:} Within 90 days.
    \end{itemize}
\end{description}

\subsection{Priority 3: Verification}
\begin{description}
    \item[Validate Risk Remediation:]
    \begin{itemize}
        \item \textbf{Action:} Formally confirm that the closure of Port 80 was an intentional and permanent change. Update the internal risk register to mark the "Unencrypted Web Server" risk as "Remediated".
        \item \textbf{Impact:} Ensures the risk register is accurate and reflects the current security posture.
        \item \textbf{Timeline:} Within 30 days.
    \end{itemize}
\end{description}

\end{document}
```