```latex
\documentclass[12pt, a4paper]{article}

% Preamble: Required Packages
\usepackage[margin=1in]{geometry}
\usepackage{pifont} % For checkmarks and crosses
\usepackage{booktabs} % For professional tables
\usepackage{hyperref} % For hyperlinks
\usepackage{url} % For URL formatting
\usepackage{seqsplit} % To split long strings without breaking
\usepackage{graphicx}
\usepackage{xcolor}

% Document Metadata
\title{Cybersecurity Posture Assessment Report}
\author{Cybersecurity Analysis Division}
\date{November 22, 2025}

% Hyperref Setup
\hypersetup{
    colorlinks=true,
    linkcolor=black,
    urlcolor=blue,
    pdftitle={Cybersecurity Posture Assessment Report},
    pdfauthor={Cybersecurity Analysis Division},
    pdfsubject={Security Assessment},
    pdfkeywords={Security, Nmap, Risk, Analysis}
}

% --- Document Start ---
\begin{document}

\begin{titlepage}
    \centering
    \vfill
    {\Huge\bfseries Cybersecurity Posture Assessment Report\par}
    \vspace{1.5cm}
    {\Large For\par}
    \vspace{0.5cm}
    {\Huge\bfseries \textbf{[Organization Name]}\par}
    \vfill
    {\large \today\par}
    \vspace{1cm}
    {\large Report ID: CSA-2025-1122-01\par}
\end{titlepage}

\tableofcontents
\newpage

% --- Section 1: Executive Summary ---
\section{Executive Summary}
This report details the findings of a cybersecurity posture assessment conducted on November 22, 2025. The assessment combined a review of organizational security controls, an external network scan, and an analysis of pre-existing risks.

The overall security posture of \textbf{[Organization Name]} requires immediate attention. Several critical and high-risk vulnerabilities were identified that expose the organization to significant threats, including unauthorized access, data breaches, and service disruption.

Key findings include:
\begin{itemize}
    \item \textbf{Critical Control Gap:} Multi-Factor Authentication (MFA) is not enforced for email access, creating a substantial risk of Business Email Compromise (BEC) and account takeovers.
    \item \textbf{High-Risk Technical Vulnerability:} The external-facing web server is running an outdated and vulnerable version of Nginx (1.18.0), which is susceptible to multiple known exploits.
    \item \textbf{Significant Policy and Training Deficiencies:} The organization lacks a formal Acceptable Use Policy and does not provide annual security awareness training for all employees. These gaps weaken the human element of the security program and increase susceptibility to social engineering attacks.
\end{itemize}

This report provides a detailed breakdown of these findings and offers actionable recommendations to mitigate the identified risks and strengthen the organization's defensive capabilities.

% --- Section 2: Organizational Information ---
\section{Organizational Information}
This section contains the high-level information for the organization under review. The data provided for this assessment was anonymized.

\begin{table}[h!]
\centering
\begin{tabular}{@{}ll@{}}
\toprule
\textbf{Attribute} & \textbf{Value} \\ \midrule
Organization Name & \textbf{[Organization Name]} \\
Primary Email Domain & \texttt{[Domain]} \\
Assessed External IP & \texttt{[Client IP]} \\
Assessment Date & November 22, 2025 \\ \bottomrule
\end{tabular}
\caption{Client Organizational Details}
\end{table}

% --- Section 3: Security Control Review ---
\section{Security Control Review}
A questionnaire was used to evaluate the implementation of fundamental security controls. The responses indicate several significant gaps in the current security framework. A "No" response highlights a missing control that should be addressed.

\begin{table}[h!]
\centering
\begin{tabular}{@{}p{0.75\linewidth}c@{}}
\toprule
\textbf{Control Question} & \textbf{Response} \\ \midrule
Do you require MFA to access email? & \textcolor{red}{\ding{55}} \\
Do you require MFA to log into computers? & \textcolor{green}{\ding{51}} \\
Do you require MFA to access sensitive data systems? & \textcolor{green}{\ding{51}} \\
Does your organization have an employee acceptable use policy? & \textcolor{red}{\ding{55}} \\
Does your organization do security awareness training for new employees? & \textcolor{green}{\ding{51}} \\
Does your organization do security awareness training for all employees at least once per year? & \textcolor{red}{\ding{55}} \\ \bottomrule
\end{tabular}
\caption{Security Controls Questionnaire Results}
\end{table}

\subsection*{Analysis of Control Gaps}
The lack of MFA for email is the most critical finding from this review. Email accounts are a primary target for attackers, and a compromised account can lead to widespread data breaches and financial fraud. Furthermore, the absence of an Acceptable Use Policy and annual security training for all staff creates an environment where employees are more likely to engage in risky behavior or fall victim to phishing and other social engineering attacks.

% --- Section 4: Technical Scan Results ---
\section{Technical Scan Results}
An external network scan was performed against the target IP address to identify open ports and exposed services.

\begin{itemize}
    \item \textbf{Scan Target:} \texttt{[Target IP]}
    \item \textbf{Scan Date:} 2025-11-22T10:00:00Z
\end{itemize}

\begin{table}[h!]
\centering
\begin{tabular}{@{}lllll@{}}
\toprule
\textbf{Port} & \textbf{State} & \textbf{Service} & \textbf{Product} & \textbf{Version} \\ \midrule
443/tcp & open & https & nginx & 1.18.0 \\ \bottomrule
\end{tabular}
\caption{Open Ports and Services Detected}
\end{table}

\subsection*{Analysis of Technical Findings}
The scan identified a web server running \textbf{Nginx version 1.18.0}. This version was released in April 2020 and is now considered obsolete. It is affected by several publicly disclosed vulnerabilities, including but not limited to CVE-2021-23017, which can lead to request smuggling and security policy bypass. Running outdated software on internet-facing systems presents a high risk of compromise, as automated attack tools constantly scan for and exploit such vulnerabilities.

% --- Section 5: Consolidated Risk Assessment ---
\section{Consolidated Risk Assessment}
The following table synthesizes the findings from the security control review, technical scan, and pre-existing risk data. Since no pre-existing vulnerabilities were reported, all risks listed below were identified during this assessment.

\begin{table}[h!]
\centering
\begin{tabular}{@{}p{0.1\linewidth}p{0.3\linewidth}p{0.4\linewidth}l@{}}
\toprule
\textbf{ID} & \textbf{Risk Name} & \textbf{Description} & \textbf{Severity} \\ \midrule
\textbf{R-01} & Lack of MFA for Email & Absence of MFA on email accounts allows for account takeover with only compromised credentials, leading to potential BEC and data exfiltration. & \textbf{Critical} \\
\addlinespace
\textbf{R-02} & Outdated Web Server Software & The public-facing Nginx server (v1.18.0) is outdated and has known vulnerabilities that can be exploited by remote attackers. & \textbf{High} \\
\addlinespace
\textbf{R-03} & Inadequate Security Training Program & Lack of mandatory annual training for all employees increases the likelihood of human error, such as falling for phishing attacks. & \textbf{High} \\
\addlinespace
\textbf{R-04} & Missing Acceptable Use Policy (AUP) & Without a formal AUP, there are no clear guidelines for employees on the proper use of company assets, increasing insider threat and compliance risks. & \textbf{Medium} \\ \bottomrule
\end{tabular}
\caption{Summary of Identified Risks}
\end{table}

% --- Section 6: Recommendations ---
\section{Recommendations}
Based on the consolidated risk assessment, we recommend the following actions, prioritized by severity, to improve the security posture of \textbf{[Organization Name]}.

\begin{enumerate}
    \item \textbf{Risk R-01 (Critical): Enforce MFA for Email Access}
    \begin{itemize}
        \item \textbf{Action:} Immediately enable and enforce MFA for all user accounts across the email platform (e.g., Microsoft 365, Google Workspace).
        \item \textbf{Impact:} Drastically reduces the risk of account compromise and Business Email Compromise (BEC). This is the single most effective control to implement.
    \end{itemize}
    \vspace{0.5cm}
    \item \textbf{Risk R-02 (High): Upgrade Web Server Software}
    \begin{itemize}
        \item \textbf{Action:} Plan and execute an upgrade of the Nginx server from version 1.18.0 to a current, stable, and patched version. Establish a patch management process to ensure the server stays up-to-date.
        \item \textbf{Impact:} Eliminates known vulnerabilities, protecting the server and its data from automated and targeted attacks.
    \end{itemize}
    \vspace{0.5cm}
    \item \textbf{Risk R-03 (High): Implement Annual Security Training}
    \begin{itemize}
        \item \textbf{Action:} Procure or develop a security awareness training program and make it mandatory for all employees to complete it on an annual basis. The training should cover phishing, password security, and data handling.
        \item \textbf{Impact:} Strengthens the "human firewall" and reduces the organization's susceptibility to social engineering.
    \end{itemize}
    \vspace{0.5cm}
    \item \textbf{Risk R-04 (Medium): Develop and Implement an AUP}
    \begin{itemize}
        \item \textbf{Action:} Draft a formal Acceptable Use Policy (AUP) that clearly defines the rules for using company IT assets. Ensure all employees read and acknowledge the policy.
        \item \textbf{Impact:} Establishes a clear security baseline for employee behavior and provides a foundation for disciplinary action in case of policy violations.
    \end{itemize}
\end{enumerate}

\end{document}
```