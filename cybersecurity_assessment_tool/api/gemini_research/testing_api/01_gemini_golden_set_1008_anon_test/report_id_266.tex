```latex
\documentclass[12pt]{article}

% Preamble: Required Packages
\usepackage[utf8]{inputenc}
\usepackage[margin=1in]{geometry}
\usepackage{pifont} % For checkmarks and crosses
\usepackage{booktabs} % For professional tables
\usepackage{hyperref} % For clickable links
\usepackage{url} % For URL formatting
\usepackage{seqsplit} % For splitting long strings
\usepackage{xcolor} % For colors
\usepackage{graphicx}

% Define colors for severity
\definecolor{critical}{RGB}{255, 0, 0}
\definecolor{high}{RGB}{255, 128, 0}
\definecolor{medium}{RGB}{255, 255, 0}
\definecolor{low}{RGB}{0, 176, 80}

% Document Metadata
\title{Cybersecurity Posture Assessment Report}
\author{Cybersecurity Analyst}
\date{\today}

\begin{document}

\maketitle
\thispagestyle{empty}
\newpage

\tableofcontents
\newpage

% --- 1. Executive Summary ---
\section{Executive Summary}
This report provides a comprehensive cybersecurity assessment for \textbf{[Organization Name]}, based on an analysis of network scan data, organizational security controls, and pre-existing risk information. The assessment reveals several critical and high-severity vulnerabilities that require immediate attention.

Key findings indicate significant gaps in fundamental security controls, including the absence of Multi-Factor Authentication (MFA) for email and sensitive data systems, a lack of employee security policies, and no formal security awareness training program.

Furthermore, a technical network scan identified a publicly exposed FTP server on \texttt{[Client IP]} running a critically outdated and vulnerable version of \texttt{vsftpd}. This service permits anonymous access, presenting an immediate and severe risk of unauthorized access and potential system compromise. The combination of weak administrative controls and a direct, exploitable technical vulnerability creates a high-risk environment that could lead to a significant data breach.

This report outlines the identified risks and provides prioritized, actionable recommendations to mitigate these threats and improve the overall security posture of the organization.

% --- 2. Organizational Information ---
\section{Organizational Information}
The following information was used as the basis for this assessment. Due to missing data in the provided inputs, placeholders have been used.

\begin{table}[h!]
\centering
\begin{tabular}{@{}ll@{}}
\toprule
\textbf{Attribute} & \textbf{Value} \\ \midrule
Organization Name & \textbf{[Organization Name]} \\
Primary Domain & \texttt{[Domain]} \\
External IP Address Assessed & \texttt{[Client IP]} \\ \bottomrule
\end{tabular}
\caption{Client Organizational Details}
\end{table}

% --- 3. Security Control Review ---
\section{Security Control Review}
A review of the organization's security controls was conducted via a questionnaire. The results highlight critical deficiencies in access control and security governance. Answers marked with \ding{55} (No) represent significant security gaps that increase organizational risk.

\begin{table}[h!]
\centering
\begin{tabular}{@{}lc@{}}
\toprule
\textbf{Control Question} & \textbf{Status} \\ \midrule
Do you require MFA to access email? & \ding{55} \\
Do you require MFA to log into computers? & \ding{51} \\
Do you require MFA to access sensitive data systems? & \ding{55} \\
Does your organization have an employee acceptable use policy? & \ding{55} \\
Does your organization do security awareness training for new employees? & \ding{55} \\
Does your organization do security awareness training annually? & \ding{55} \\ \bottomrule
\end{tabular}
\caption{Security Controls Questionnaire Results (\ding{51}=Yes, \ding{55}=No)}
\end{table}

\paragraph{Analysis:} The lack of MFA on email and sensitive data systems is a critical vulnerability, leaving key assets protected only by passwords, which are susceptible to phishing, brute-force attacks, and credential stuffing. The absence of an Acceptable Use Policy and a security awareness training program indicates a low level of security maturity and leaves the organization highly vulnerable to human error and social engineering attacks.

% --- 4. Technical Scan Results ---
\section{Technical Scan Results}
An external network scan was performed on the target IP address to identify exposed services.

\begin{itemize}
    \item \textbf{Target IP Address:} \texttt{[Target IP]}
    \item \textbf{Scan Date:} \today
\end{itemize}

The scan revealed the following open port:

\begin{table}[h!]
\centering
\begin{tabular}{@{}llllll@{}}
\toprule
\textbf{Port} & \textbf{State} & \textbf{Service} & \textbf{Product} & \textbf{Version} & \textbf{Notes} \\ \midrule
21/tcp & open & ftp & vsftpd & 2.3.4 & \parbox{5cm}{\textbf{Critical Finding:} Anonymous FTP login is allowed. This version is vulnerable to a backdoor command execution flaw (CVE-2011-2523).} \\ \bottomrule
\end{tabular}
\caption{Open Port Analysis}
\end{table}

\paragraph{Analysis:} The presence of an open FTP port with anonymous login enabled is a severe security risk. It allows any attacker to connect to the server and potentially upload malicious files or download sensitive data. The specific version of \texttt{vsftpd (2.3.4)} contains a well-known, critical backdoor vulnerability (\href{https://nvd.nist.gov/vuln/detail/CVE-2011-2523}{CVE-2011-2523}) which, if exploited, could grant an attacker complete control over the server. This finding requires immediate remediation.

% --- 5. Correlated Risk Assessment ---
\section{Correlated Risk Assessment}
The following table synthesizes findings from the security control review, technical scan, and pre-existing risk data to provide a holistic view of the organization's risk profile.

\begin{table}[h!]
\centering
\begin{tabular}{@{}lp{7cm}l@{}}
\toprule
\textbf{Risk ID} & \textbf{Risk Title \& Description} & \textbf{Severity} \\ \midrule
RISK-001 & \textbf{Exposed Vulnerable FTP Server:} An outdated FTP server (\texttt{vsftpd 2.3.4}) is exposed to the internet, allows anonymous login, and is vulnerable to remote code execution (CVE-2011-2523). & \colorbox{critical}{\color{white}\textbf{CRITICAL}} \\
\addlinespace
RISK-002 & \textbf{Insufficient Access Control:} Lack of MFA for critical systems like email and sensitive data repositories exposes the organization to account takeover and data breach. & \colorbox{high}{\color{white}\textbf{HIGH}} \\
\addlinespace
RISK-003 & \textbf{Missing Security Governance:} The absence of an Acceptable Use Policy and security awareness training creates a weak security culture, making the organization susceptible to phishing and insider threats. & \colorbox{high}{\color{white}\textbf{HIGH}} \\
\addlinespace
RISK-004 & \textbf{Outdated Workstation OS:} Workstations are running Windows 7, an unsupported operating system that no longer receives security updates, leaving them vulnerable to known exploits. & \colorbox{medium}{\color{black}\textbf{MEDIUM}} \\ \bottomrule
\end{tabular}
\caption{Summary of Identified Risks}
\end{table}

% --- 6. Recommendations ---
\section{Recommendations}
The following actions are recommended to mitigate the identified risks. They are prioritized based on severity and potential impact.

\subsection{Immediate Priority (Remediate within 24-48 hours)}
\begin{itemize}
    \item \textbf{RISK-001 - Remediate FTP Server:}
    \begin{enumerate}
        \item Immediately take the FTP server offline or firewall it from public access.
        \item If the service is business-critical, upgrade the \texttt{vsftpd} software to the latest stable version and disable anonymous access.
        \item If FTP is not required, decommission the service entirely.
        \item Consider replacing FTP with a secure alternative such as SFTP (SSH File Transfer Protocol).
    \end{enumerate}
\end{itemize}

\subsection{High Priority (Remediate within 30 days)}
\begin{itemize}
    \item \textbf{RISK-002 - Implement MFA:}
    \begin{enumerate}
        \item Enable and enforce MFA for all user accounts on the email system.
        \item Identify all systems classified as containing sensitive data and enforce MFA for access.
    \end{enumerate}
    \item \textbf{RISK-003 - Establish Security Governance:}
    \begin{enumerate}
        \item Develop and ratify a formal Acceptable Use Policy (AUP) for all employees.
        \item Implement a mandatory security awareness training program for all new and existing employees, to be conducted at least annually.
    \end{enumerate}
\end{itemize}

\subsection{Medium Priority (Remediate within 90 days)}
\begin{itemize}
    \item \textbf{RISK-004 - Upgrade Operating Systems:}
    \begin{enumerate}
        \item Develop a plan to upgrade all workstations from Windows 7 to a modern, supported operating system such as Windows 10 or Windows 11.
        \item Prioritize the upgrade for systems used by employees with access to sensitive data.
    \end{enumerate}
\end{itemize}

\end{document}
```