```latex
\documentclass[12pt, a4paper]{article}

% Preamble: Required Packages
\usepackage[margin=1in]{geometry}
\usepackage{pifont} % For \ding symbols
\usepackage{booktabs} % For professional tables
\usepackage{hyperref} % For hyperlinks
\usepackage{url} % For URL formatting
\usepackage{seqsplit} % To split long strings in texttt
\usepackage[utf8]{inputenc}
\usepackage{graphicx}
\usepackage{xcolor}

% Document Metadata
\title{Cybersecurity Assessment Report}
\author{Cybersecurity Analysis Division}
\date{\today}

% Hyperref Setup
\hypersetup{
    colorlinks=true,
    linkcolor=blue,
    filecolor=magenta,      
    urlcolor=cyan,
    pdftitle={Cybersecurity Assessment Report},
    pdfpagemode=FullScreen,
}

\begin{document}

\maketitle
\thispagestyle{empty}
\newpage

\tableofcontents
\newpage

% --- Executive Summary ---
\section{Executive Summary}
This report provides a comprehensive cybersecurity assessment for \textbf{[Organization Name]}. The analysis is based on a correlation of network scan data, a security controls questionnaire, and a review of pre-existing documented risks.

The overall security posture is determined to be at a \textbf{CRITICAL} risk level. This assessment is driven by several significant findings:
\begin{itemize}
    \item \textbf{Critical Pre-existing Vulnerability:} A known risk, ``Localhost Exposed,'' with a CVSS score of 10.0, indicates a critical service is improperly exposed to the internet on \texttt{[Target IP]}. This requires immediate remediation.
    \item \textbf{Lack of Multi-Factor Authentication (MFA):} MFA is not enforced for accessing email or other sensitive data systems. This exposes the organization to a high risk of account compromise and subsequent data breaches.
    \item \textbf{Administrative Control Gaps:} The absence of a formal Acceptable Use Policy (AUP) and a mandatory security awareness training program for new employees creates significant organizational risk.
    \item \textbf{Exposed Management Service:} The network scan identified an open SSH port (22/TCP) on the external IP address \texttt{[Target IP]}, increasing the attack surface and risk of brute-force attacks.
\end{itemize}
Immediate action is required to address the critical vulnerabilities outlined in this report. Detailed recommendations are provided in Section \ref{sec:recommendations} to mitigate these risks and improve the organization's overall security posture.

% --- Organizational Information ---
\section{Organizational Information}
This section details the information provided by the client organization.
\begin{table}[h!]
\centering
\begin{tabular}{@{}ll@{}}
\toprule
\textbf{Attribute} & \textbf{Value} \\ \midrule
Organization Name & \textbf{[Organization Name]} \\
Primary Domain & \texttt{[Domain]} \\
External IP Scanned & \texttt{[Client IP]} \\ \bottomrule
\end{tabular}
\caption{Client Profile}
\label{tab:org_info}
\end{table}

% --- Security Control Review ---
\section{Security Control Review}
The following table summarizes the organization's responses to a security controls questionnaire. A checkmark (\ding{51}) indicates a positive control is in place, while an X (\ding{55}) indicates a control gap that presents a risk.

\begin{table}[h!]
\centering
\begin{tabular}{@{}lc@{}}
\toprule
\textbf{Control Question} & \textbf{Status} \\ \midrule
Do you require MFA to access email? & \ding{55} \\
Do you require MFA to log into computers? & \ding{51} \\
Do you require MFA to access sensitive data systems? & \ding{55} \\
Does your organization have an employee acceptable use policy? & \ding{55} \\
Does your organization do security awareness training for new employees? & \ding{55} \\
Does your organization do security awareness training for all employees annually? & \ding{51} \\ \bottomrule
\end{tabular}
\caption{Security Controls Questionnaire Results}
\label{tab:controls}
\end{table}

\subsection*{Analysis of Control Gaps}
The questionnaire reveals critical gaps in identity and access management and administrative controls. The lack of MFA on email and sensitive data systems are high-impact vulnerabilities that significantly increase the likelihood of a successful phishing attack or credential compromise leading to a major breach. Furthermore, the absence of an Acceptable Use Policy and security training for new hires indicates a weakness in the foundational security culture of the organization.

% --- Technical Scan Results ---
\section{Technical Scan Results}
An external network scan was performed to identify open ports and services exposed to the internet.

\subsection*{Scan Details}
\begin{itemize}
    \item \textbf{Target IP Address:} \texttt{[Target IP]}
    \item \textbf{Scan Date:} Not specified in scan data.
\end{itemize}

\subsection*{Open Ports and Services}
The scan identified the following open port on the target host:
\begin{table}[h!]
\centering
\begin{tabular}{@{}lllll@{}}
\toprule
\textbf{Port} & \textbf{Protocol} & \textbf{State} & \textbf{Service} & \textbf{Version} \\ \midrule
22 & TCP & open & ssh (inferred) & Not Available \\ \bottomrule
\end{tabular}
\caption{Open Port Findings for \texttt{[Target IP]}}
\label{tab:nmap_results}
\end{table}

\subsection*{Technical Analysis}
The presence of an open SSH port (22) on an external-facing IP address presents a significant risk. This service is a common target for automated brute-force attacks that attempt to guess user credentials. While SSH is a secure protocol, its exposure should be limited to trusted sources whenever possible. This finding, combined with the lack of MFA on sensitive systems, creates a potential pathway for unauthorized access.

% --- Consolidated Risk Assessment ---
\section{Consolidated Risk Assessment}
This section synthesizes findings from all data sources into a prioritized list of identified risks.

\begin{table}[h!]
\centering
\resizebox{\textwidth}{!}{%
\begin{tabular}{@{}llll@{}}
\toprule
\textbf{Risk Name} & \textbf{Severity} & \textbf{Description} & \textbf{Affected Asset(s)} \\ \midrule
Localhost Exposed & \textbf{Critical} & A service intended for local access is exposed to the internet. & \texttt{[Target IP]} \\
Lack of MFA on Email & \textbf{Critical} & No MFA requirement for email access, enabling account takeovers. & Email System \\
Lack of MFA on Sensitive Data & \textbf{Critical} & No MFA requirement for sensitive systems, risking data breach. & Sensitive Data Systems \\
SSH Port Exposed to Internet & High & SSH service is open to the public internet, inviting brute-force attacks. & \texttt{[Target IP]} \\
Missing New Hire Training & High & New employees are not trained on security, making them easy targets. & New Employees \\
Missing Acceptable Use Policy & High & Lack of a formal AUP creates ambiguity in security responsibilities. & All Employees \\ \bottomrule
\end{tabular}%
}
\caption{Summary of Identified Risks}
\label{tab:risk_summary}
\end{table}

% --- Recommendations ---
\section{Recommendations}
\label{sec:recommendations}
The following actions are recommended to mitigate the identified risks. Recommendations are prioritized based on severity and potential impact.

\subsection*{Priority 1: Immediate Actions (Remediate within 72 hours)}
\begin{enumerate}
    \item \textbf{Remediate "Localhost Exposed" Vulnerability:}
    \begin{itemize}
        \item \textbf{Action:} Immediately reconfigure the firewall and the affected service on \texttt{[Target IP]} to ensure it is not bound to the public IP address. Access should be restricted to the local machine (\texttt{127.0.0.1}) or a private network.
        \item \textbf{Justification:} A CVSS 10.0 vulnerability represents the highest possible risk and could lead to a complete system compromise.
    \end{itemize}
\end{enumerate}

\subsection*{Priority 2: High-Priority Actions (Remediate within 30 days)}
\begin{enumerate}
    \item \textbf{Implement Multi-Factor Authentication (MFA):}
    \begin{itemize}
        \item \textbf{Action:} Enforce MFA for all users on the primary email system and all systems identified as containing sensitive data.
        \item \textbf{Justification:} This is the single most effective control to prevent account compromise resulting from stolen or weak credentials.
    \end{itemize}
    \item \textbf{Restrict SSH Access:}
    \begin{itemize}
        \item \textbf{Action:} Configure firewall rules to restrict access to port 22 on \texttt{[Target IP]} to only known, trusted IP addresses. If public access is a business requirement, implement an intrusion prevention tool like \texttt{fail2ban}.
        \item \textbf{Justification:} This action significantly reduces the attack surface and mitigates the risk of brute-force attacks.
    \end{itemize}
\end{enumerate}

\subsection*{Priority 3: Foundational Improvements (Implement within 90 days)}
\begin{enumerate}
    \item \textbf{Develop and Implement an Acceptable Use Policy (AUP):}
    \begin{itemize}
        \item \textbf{Action:} Draft a formal AUP that clearly defines the rules for using company IT assets. Require all employees to read and acknowledge the policy.
        \item \textbf{Justification:} An AUP establishes a baseline for secure behavior and provides a framework for enforcing security rules.
    \end{itemize}
    \item \textbf{Establish a New Hire Security Training Program:}
    \begin{itemize}
        \item \textbf{Action:} Create a mandatory security awareness training module as part of the onboarding process for all new employees.
        \item \textbf{Justification:} New hires are often targeted by attackers. Early training reduces their susceptibility to phishing and social engineering attacks.
    \end{itemize}
\end{enumerate}

\end{document}
```