```latex
\documentclass[12pt]{article}

% Preamble: Required Packages
\usepackage[a4paper, margin=1in]{geometry} % Page layout
\usepackage{pifont}                       % For checkmarks and crosses (\ding)
\usepackage{booktabs}                     % For professional tables
\usepackage{graphicx}                     % For logos, etc.
\usepackage[table]{xcolor}                % For coloring table rows
\usepackage{hyperref}                     % For clickable links
\usepackage{url}                          % For formatting URLs
\usepackage{seqsplit}                     % For splitting long strings in texttt

% --- Document Metadata ---
\hypersetup{
    colorlinks=true,
    linkcolor=blue,
    filecolor=magenta,      
    urlcolor=cyan,
    pdftitle={Cybersecurity Assessment Report},
    pdfauthor={Cybersecurity Analyst},
    pdfsubject={Security Posture Analysis},
    pdfkeywords={Cybersecurity, Nmap, Risk, Assessment},
}

% --- Custom Commands ---
\newcommand{\yes}{\ding{51}} % Green checkmark
\newcommand{\no}{\ding{55}}  % Red cross

% --- Title Page ---
\title{
    \vspace{2cm}
    \textbf{Cybersecurity Assessment Report} \\
    \large For \\
    \vspace{0.5cm}
    \textbf{[Organization Name]}
    \vspace{3cm}
}
\author{Cybersecurity Analyst}
\date{\today}

% ==============================================================================
% --- BEGIN DOCUMENT ---
% ==============================================================================
\begin{document}

\maketitle
\newpage

% --- Table of Contents ---
\tableofcontents
\newpage

% ==============================================================================
% Section 1: Executive Summary
% ==============================================================================
\section{Executive Summary}

This report details the findings of a cybersecurity assessment conducted for \textbf{[Organization Name]}. The assessment combined a review of organizational security controls, an external network vulnerability scan, and an analysis of pre-existing risks.

The overall security posture requires immediate attention. A critical vulnerability was identified on an externally-facing server, which could allow an attacker to gain unauthorized access to the network. Specifically, an outdated and misconfigured FTP service (\texttt{vsftpd 2.3.4}) is exposed, which is vulnerable to a known remote code execution backdoor and permits anonymous logins.

Furthermore, significant gaps were identified in the organization's foundational security policies. The absence of an Acceptable Use Policy and a lack of security awareness training for new employees create a high-risk environment susceptible to insider threats and social engineering attacks.

This report provides a detailed breakdown of these risks and outlines prioritized, actionable recommendations to mitigate them effectively. We urge management to review the recommendations in Section \ref{sec:recommendations} and allocate resources to address the critical findings as a matter of priority.

% ==============================================================================
% Section 2: Organizational Information
% ==============================================================================
\section{Organizational Information}

The following details were used as the basis for this assessment. Due to the anonymized nature of the provided data, placeholders have been used where necessary.

\begin{table}[h!]
\centering
\begin{tabular}{ll}
\toprule
\textbf{Attribute} & \textbf{Value} \\
\midrule
Organization Name & \textbf{[Organization Name]} \\
Primary Domain & \texttt{[Domain]} \\
External IP Scanned & \texttt{[Client IP]} \\
Assessment Date & \today \\
\bottomrule
\end{tabular}
\caption{Client and Assessment Details}
\end{table}

% ==============================================================================
% Section 3: Security Control Review
% ==============================================================================
\section{Security Control Review}

A review of the organization's security controls was conducted via a questionnaire. The responses indicate a strong implementation of Multi-Factor Authentication (MFA) but reveal critical deficiencies in policy and training.

\begin{table}[h!]
\centering
\rowcolors{2}{gray!10}{white}
\begin{tabular}{p{0.7\textwidth}c}
\toprule
\textbf{Control Question} & \textbf{Response} \\
\midrule
Do you require MFA to access email? & \yes \\
Do you require MFA to log into computers? & \yes \\
Do you require MFA to access sensitive data systems? & \yes \\
Does your organization have an employee acceptable use policy? & \no \\
Does your organization do security awareness training for new employees? & \no \\
Does your organization do security awareness training for all employees at least once per year? & \yes \\
\bottomrule
\end{tabular}
\caption{Security Controls Questionnaire Results}
\end{table}

\subsection*{Analysis}
The "No" responses represent significant security gaps:
\begin{itemize}
    \item \textbf{No Acceptable Use Policy (AUP):} Without a formal AUP, employees lack clear guidelines on the safe and appropriate use of company assets. This increases the risk of unintentional data exposure, malware infections, and insider threats.
    \item \textbf{No Security Training for New Employees:} New hires are often prime targets for phishing and social engineering attacks. Failing to provide immediate security training during onboarding leaves the organization vulnerable from day one of a new employee's tenure.
\end{itemize}

% ==============================================================================
% Section 4: Technical Scan Results
% ==============================================================================
\section{Technical Scan Results}

An external network scan was performed against the target IP address \texttt{[Target IP]}. The scan identified one open port with a critically vulnerable service.

\begin{table}[h!]
\centering
\rowcolors{2}{gray!10}{white}
\begin{tabular}{lllll}
\toprule
\textbf{Port} & \textbf{State} & \textbf{Service} & \textbf{Product / Version} & \textbf{Notes} \\
\midrule
21/tcp & Open & ftp & vsftpd 2.3.4 & \begin{tabular}[t]{@{}l@{}}Anonymous FTP login allowed. \\ Version is vulnerable to a \\ known backdoor (CVE-2011-2523).\end{tabular} \\
\bottomrule
\end{tabular}
\caption{Open Ports and Services Detected on \texttt{[Target IP]}}
\end{table}

\subsection*{Analysis}
The FTP service running on port 21 presents a severe and immediate threat:
\begin{itemize}
    \item \textbf{Vulnerable Version:} \texttt{vsftpd 2.3.4}, released in 2011, contains a critical backdoor vulnerability (\href{https://nvd.nist.gov/vuln/detail/CVE-2011-2523}{CVE-2011-2523}). A remote attacker can exploit this to gain a command shell on the server, effectively taking full control of the system.
    \item \textbf{Anonymous Login:} The server is configured to allow anonymous logins. This allows any user on the internet to connect to the FTP server to upload or download files. This could be abused to exfiltrate sensitive data or to host malicious content.
\end{itemize}

% ==============================================================================
% Section 5: Risk Assessment
% ==============================================================================
\section{Risk Assessment}

The following table synthesizes findings from the security control review, technical scan, and pre-existing risk data into a prioritized list.

\begin{table}[h!]
\centering
\begin{tabular}{p{0.1\textwidth}p{0.45\textwidth}p{0.15\textwidth}p{0.2\textwidth}}
\toprule
\textbf{Risk ID} & \textbf{Description} & \textbf{Severity} & \textbf{Affected Systems} \\
\midrule
\rowcolor{red!20}
RISK-001 & Exposed FTP server with a known remote code execution vulnerability (vsftpd 2.3.4) and anonymous login enabled. & \textbf{Critical} & External Server (\texttt{[Target IP]}) \\
\addlinespace
\rowcolor{orange!20}
RISK-002 & Lack of an Acceptable Use Policy (AUP) for employees, increasing the risk of insider threat and misuse of assets. & High & All Employees, All Systems \\
\addlinespace
\rowcolor{orange!20}
RISK-003 & No mandatory security awareness training for new employees during their onboarding process. & High & New Employees, Organization-wide \\
\addlinespace
\rowcolor{yellow!20}
RISK-004 & Workstations are running the outdated and unsupported Windows 7 operating system. & Medium & End-user Workstations \\
\bottomrule
\end{tabular}
\caption{Summary of Identified Risks}
\end{table}

% ==============================================================================
% Section 6: Recommendations
% ==============================================================================
\section{Recommendations}
\label{sec:recommendations}

The following actions are recommended to mitigate the identified risks. They are prioritized based on severity.

\subsection{RISK-001: Exposed Vulnerable FTP Server (Critical)}
\begin{itemize}
    \item \textbf{Immediate Action (Containment):} Immediately disable the FTP service on the public-facing server at \texttt{[Target IP]} or use a firewall to block all external access to TCP port 21. This is the most critical step to prevent exploitation.
    \item \textbf{Short-Term Fix:} If FTP is an absolute business necessity, upgrade the \texttt{vsftpd} service to the latest stable version and immediately disable anonymous login. Access should be restricted via firewall rules to only known, trusted IP addresses.
    \item \textbf{Long-Term Strategy:} Decommission the FTP service entirely. Migrate all file transfer workflows to a secure, encrypted protocol such as SFTP (SSH File Transfer Protocol) or a modern, managed file transfer (MFT) solution.
\end{itemize}

\subsection{RISK-002: No Acceptable Use Policy (High)}
\begin{itemize}
    \item \textbf{Immediate Action:} Begin drafting an Acceptable Use Policy (AUP). Templates are available from sources like SANS. The policy should clearly define rules for using company networks, devices, email, and internet access.
    \item \textbf{Short-Term Fix:} Finalize and formally adopt the AUP. Distribute the policy to all current employees and require a signed acknowledgment of receipt and understanding. Incorporate this step into the new employee onboarding process.
\end{itemize}

\subsection{RISK-003: No New Employee Security Training (High)}
\begin{itemize}
    \item \textbf{Immediate Action:} Develop or procure a foundational security awareness training module. This module should cover, at a minimum, phishing identification, password hygiene, and the new Acceptable Use Policy.
    \item \textbf{Short-Term Fix:} Make this training a mandatory part of the onboarding process for all new hires, to be completed within their first week of employment. Enroll all employees hired in the last 6-12 months who did not receive training.
\end{itemize}

\subsection{RISK-004: Outdated Windows 7 Workstations (Medium)}
\begin{itemize}
    \item \textbf{Short-Term Fix:} Re-confirm the existing plan to upgrade all Windows 7 machines. Prioritize the upgrade of machines used by employees with access to sensitive data.
    \item \textbf{Long-Term Strategy:} Implement a hardware and software lifecycle management policy to ensure that all systems are retired and replaced before they reach their end-of-life (EOL) date.
\end{itemize}

\end{document}
% ==============================================================================
% --- END DOCUMENT ---
% ==============================================================================
```