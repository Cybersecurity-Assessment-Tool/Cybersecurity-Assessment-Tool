```latex
\documentclass[12pt]{article}

% Preamble: Required Packages
\usepackage[margin=1in]{geometry}
\usepackage{pifont} % For checkmarks and crosses
\usepackage{booktabs} % For professional tables
\usepackage{hyperref} % For hyperlinks
\usepackage{url} % For URL formatting
\usepackage{seqsplit} % For splitting long strings to prevent overflow

% Hyperref Setup
\hypersetup{
    colorlinks=true,
    linkcolor=blue,
    filecolor=magenta,      
    urlcolor=cyan,
    pdftitle={Cybersecurity Assessment Report},
    pdfauthor={Cybersecurity Analyst},
    pdfsubject={Security Assessment},
    pdfkeywords={Security, Report, Analysis},
    bookmarks=true
}

% Document Start
\begin{document}

% --- Title Page ---
\begin{titlepage}
    \centering
    \vspace*{1cm}
    \Huge \textbf{Cybersecurity Assessment Report}
    \vspace{1.5cm}
    \large
    \begin{tabular}{ll}
        \textbf{Client:} & \textbf{[Organization Name]} \\
        \textbf{Date of Report:} & \today \\
        \textbf{Author:} & Cybersecurity Analyst \\
    \end{tabular}
    \vfill
    \large
    \textbf{CONFIDENTIAL}
    \vspace{0.5cm}
    
    \small This document contains sensitive information intended only for the designated recipient. Unauthorized distribution is strictly prohibited.
\end{titlepage}

\tableofcontents
\newpage

% --- Executive Overview ---
\section*{Executive Overview}
This report details the findings of a cybersecurity assessment conducted for \textbf{[Organization Name]}. The assessment combined an automated network scan, a review of existing risks, and an analysis of organizational security controls based on a questionnaire.

The overall security posture reveals several critical and high-risk gaps that require immediate attention. Key findings include a complete lack of Multi-Factor Authentication (MFA) for email and computer access, which exposes the organization to significant risks of account compromise and unauthorized access. Furthermore, foundational security policies, such as an Acceptable Use Policy and security training for new hires, are absent.

A technical scan identified an open Secure Shell (SSH) port on an external-facing system. When correlated with the lack of MFA, this finding presents a heightened risk of brute-force attacks and potential system compromise.

This report outlines these risks in detail and provides prioritized, actionable recommendations to mitigate them and strengthen the organization's overall defensive capabilities.

% --- Organizational Information ---
\section*{Organizational Information}
The following information was used as the basis for this assessment. Placeholders are used where data was not provided.

\begin{table}[h!]
\centering
\begin{tabular}{@{}ll@{}}
\toprule
\textbf{Attribute} & \textbf{Value} \\ \midrule
Organization Name & \textbf{[Organization Name]} \\
Primary Domain & \texttt{[Domain]} \\
External IP Scanned & \texttt{[Client IP]} \\ \bottomrule
\end{tabular}
\caption{Client Profile}
\end{table}

% --- Security Control Review ---
\section*{Security Control Review}
The following table summarizes the organization's responses to a security controls questionnaire. Items marked with a cross (\ding{55}) represent significant gaps in the security framework and are discussed in the Risk Assessment section.

\begin{table}[h!]
\centering
\begin{tabular}{@{}p{0.7\textwidth}c@{}}
\toprule
\textbf{Control Question} & \textbf{Response} \\ \midrule
Do you require MFA to access email? & \ding{55} \\
Do you require MFA to log into computers? & \ding{55} \\
Do you require MFA to access sensitive data systems? & \ding{51} \\
Does your organization have an employee acceptable use policy? & \ding{55} \\
Does your organization do security awareness training for new employees? & \ding{55} \\
Does your organization do security awareness training for all employees at least once per year? & \ding{51} \\ \bottomrule
\end{tabular}
\caption{Security Questionnaire Analysis (\ding{51}=Yes, \ding{55}=No)}
\end{table}

% --- Technical Scan Results ---
\section*{Technical Scan Results}
An external network scan was performed to identify open ports and exposed services. The scan provided a snapshot of the organization's external attack surface.

\begin{itemize}
    \item \textbf{Target IP Address:} \texttt{[Target IP]}
    \item \textbf{Scan Status:} Host is up.
\end{itemize}

\subsection*{Open Ports}
The following table details the ports found to be open and accessible from the internet.

\begin{table}[h!]
\centering
\begin{tabular}{@{}llll@{}}
\toprule
\textbf{Port} & \textbf{State} & \textbf{Service} & \textbf{Notes} \\ \midrule
22/tcp & open & ssh & Secure Shell (SSH) for remote administration. \\ \bottomrule
\end{tabular}
\caption{Discovered Open Ports}
\end{table}

\textbf{Analysis:} The presence of an open SSH port (22) indicates that a system is configured for remote command-line administration. While necessary for management, public exposure increases the risk of brute-force password attacks and exploitation of potential vulnerabilities in the SSH service itself. This risk is significantly amplified by the lack of enforced MFA for computer logins.

% --- Risk Assessment ---
\section*{Risk Assessment}
This section synthesizes the findings from the security control review and the technical scan. No pre-existing vulnerabilities were reported. All identified risks are new findings from this assessment.

\begin{table}[h!]
\centering
\begin{tabular}{@{}p{0.05\textwidth}p{0.3\textwidth}p{0.15\textwidth}p{0.4\textwidth}@{}}
\toprule
\textbf{ID} & \textbf{Risk / Vulnerability} & \textbf{Severity} & \textbf{Description} \\ \midrule
\textbf{R-01} & No MFA for Email Access & \textbf{Critical} & Lack of MFA on email accounts makes them highly susceptible to phishing and credential theft, which can lead to business email compromise (BEC), data breaches, and further internal attacks. \\
\addlinespace
\textbf{R-02} & No MFA for Computer Logins & \textbf{Critical} & Without MFA, compromised user credentials (e.g., from a phishing attack) can be used directly to gain unauthorized access to workstations and potentially the internal network. \\
\addlinespace
\textbf{R-03} & Inadequate Security Policies and Training & \textbf{High} & The absence of an Acceptable Use Policy and security training for new hires creates an environment where employees are unaware of security best practices, making them more likely to fall victim to social engineering attacks. \\
\addlinespace
\textbf{R-04} & Exposed SSH Service with Weak Authentication Controls & \textbf{High} & The open SSH port, combined with the lack of enforced MFA (R-02), creates a direct path for an attacker with stolen credentials to gain shell access to a critical system. \\ \bottomrule
\end{tabular}
\caption{Summary of Identified Risks}
\end{table}

% --- Recommendations ---
\section*{Recommendations}
Based on the identified risks, the following prioritized actions are recommended to improve the organization's security posture.

\begin{description}
    \item[\textbf{1. (Critical) Implement Comprehensive MFA:}]
    \begin{itemize}
        \item Immediately enforce MFA for all user accounts, especially for email access (R-01) and workstation/server logins (R-02).
        \item Prioritize phishing-resistant MFA methods like FIDO2/WebAuthn where possible.
    \end{itemize}

    \item[\textbf{2. (High) Develop Foundational Security Programs:}]
    \begin{itemize}
        \item Create and enforce a formal Employee Acceptable Use Policy that clearly defines rules for using company assets and data (R-03).
        \item Establish a mandatory security awareness training program for all new employees upon hiring and continue with annual refreshers for all staff (R-03).
    \end{itemize}

    \item[\textbf{3. (High) Secure the SSH Service:}]
    \begin{itemize}
        \item If remote access via SSH is not required from the public internet, block port 22 at the firewall (R-04).
        \item If public access is required, restrict access to a whitelist of trusted IP addresses.
        \item Enforce the use of public key authentication and disable password-based logins for SSH.
        \item Implement an intrusion detection tool like \texttt{fail2ban} to automatically block IPs that exhibit brute-force behavior.
    \end{itemize}
\end{description}

% --- Conclusion ---
\section*{Conclusion}
The assessment for \textbf{[Organization Name]} has identified critical vulnerabilities in its authentication and policy frameworks. While some positive controls are in place, such as MFA for sensitive systems and annual security training, the gaps in email and endpoint security represent an unacceptable level of risk.

By implementing the recommendations outlined in this report—prioritizing the enforcement of MFA and the development of core security policies—the organization can significantly reduce its attack surface and build a more resilient security posture.

\end{document}
```