```latex
\documentclass[12pt]{article}

% Preamble: Required Packages
\usepackage[margin=1in]{geometry}
\usepackage{pifont} % Required for \ding
\usepackage{booktabs} % For professional-looking tables
\usepackage{hyperref} % For clickable links and references
\usepackage{url}      % For properly formatting URLs
\usepackage{seqsplit} % For splitting long strings to prevent overflow

% Document Metadata
\title{Cybersecurity Risk Assessment Report}
\author{Cybersecurity Analysis Division}
\date{\today}

% Hyperref Setup
\hypersetup{
    colorlinks=true,
    linkcolor=black,
    urlcolor=blue,
    pdftitle={Cybersecurity Risk Assessment Report},
    pdfauthor={Cybersecurity Analysis Division},
}

\begin{document}

\maketitle

\section*{1. Executive Overview}

This report details the findings of a cybersecurity assessment conducted for \textbf{[Organization Name]}. The assessment incorporated an external network scan, a review of organizational security controls, and an analysis of pre-existing risk data.

The overall security posture is determined to be at a \textbf{High Risk} level. Critical deficiencies were identified that expose the organization to significant threats. Key findings include:
\begin{itemize}
    \item \textbf{Exposed End-of-Life Database:} An external scan revealed a MySQL database (version 5.7.33) directly accessible from the internet. This version reached its official End-of-Life (EOL) in October 2023 and no longer receives security updates, making it a prime target for exploitation.
    \item \textbf{Critical Gaps in Multi-Factor Authentication (MFA):} MFA is not enforced for computer logins or access to sensitive data systems. This significantly increases the risk of unauthorized access via compromised credentials.
    \item \textbf{Inadequate Security Training:} The lack of annual security awareness training for all employees heightens the organization's susceptibility to phishing and social engineering attacks.
\end{itemize}

Immediate remediation of these issues is strongly recommended to reduce the likelihood of a security breach. Actionable recommendations are provided in Section 6 of this report.

\section*{2. Organizational Information}

The following information was used as the basis for this assessment. As per the provided data, placeholders have been used where specific details were not available.

\begin{tabular}{@{}ll@{}}
\toprule
\textbf{Attribute} & \textbf{Value} \\
\midrule
Organization Name & \textbf{[Organization Name]} \\
Primary Domain & \texttt{[Domain]} \\
Assessed External IP & \texttt{[Client IP]} \\
Scan Target IP & \texttt{[Target IP]} \\
Scan Date & 2024-05-21 \\ % Assuming a recent date as none was in the input
\bottomrule
\end{tabular}

\section*{3. Security Control Review}

The following table summarizes the organization's responses to a security controls questionnaire. A checkmark (\ding{51}) indicates a positive control is in place, while a cross (\ding{55}) indicates a control gap that presents a risk.

\begin{center}
\begin{tabular}{@{}p{0.75\textwidth}c@{}}
\toprule
\textbf{Control Question} & \textbf{Status} \\
\midrule
Do you require MFA to access email? & \ding{51} \\
Do you require MFA to log into computers? & \textbf{\ding{55}} \\
Do you require MFA to access sensitive data systems? & \textbf{\ding{55}} \\
Does your organization have an employee acceptable use policy? & \ding{51} \\
Does your organization do security awareness training for new employees? & \ding{51} \\
Does your organization do security awareness training for all employees at least once per year? & \textbf{\ding{55}} \\
\bottomrule
\end{tabular}
\end{center}

\section*{4. Technical Scan Results}

An external network scan was performed on the target IP address. The results identified the following open ports and services.

\begin{center}
\begin{tabular}{@{}llll@{}}
\toprule
\textbf{Port} & \textbf{Service} & \textbf{Product \& Version} & \textbf{Analyst Notes} \\
\midrule
3306/tcp & mysql & MySQL 5.7.33 & \textbf{Critical:} Publicly exposed. This version is End-of-Life. \\
\bottomrule
\end{tabular}
\end{center}

\subsection*{Analysis of Technical Findings}
The most significant finding is the exposed MySQL database on port 3306. The running version, \textbf{MySQL 5.7.33}, is no longer supported by its developer as of October 2023. This means it does not receive security patches for newly discovered vulnerabilities. An exposed, unpatched, EOL database is a critical risk and a common target for automated attacks.

\section*{5. Risk Assessment Summary}

The following table synthesizes findings from the security control review, technical scan, and pre-existing risk data into a prioritized list of organizational risks.

\begin{center}
\begin{tabular}{@{}p{0.25\textwidth}p{0.55\textwidth}l@{}}
\toprule
\textbf{Risk Title} & \textbf{Description} & \textbf{Severity} \\
\midrule
\textbf{Exposed End-of-Life Database} & The MySQL database on port 3306 is accessible from the internet and is running an unsupported, unpatched version. This creates a high risk of data breach. & \textbf{Critical} \\
\addlinespace
\textbf{Lack of MFA on Critical Systems} & MFA is not required for computer logins or access to sensitive data. A single compromised password could lead to widespread system access and data exfiltration. & \textbf{Critical} \\
\addlinespace
\textbf{Inadequate Security Awareness Training} & The absence of annual security training for all staff increases the likelihood of successful phishing attacks, which are a primary vector for credential theft. & \textbf{High} \\
\bottomrule
\end{tabular}
\end{center}

\section*{6. Recommendations}

The following prioritized actions are recommended to mitigate the identified risks and improve the organization's overall security posture.

\subsection*{Priority 1: Immediate Actions (Within 72 Hours)}
\begin{enumerate}
    \item \textbf{Restrict Database Access:} Immediately implement firewall rules to block all public internet access to TCP port 3306 on \texttt{[Target IP]}. Access should only be permitted from trusted internal IP addresses or via a secure VPN connection.
    \item \textbf{Enforce MFA on Sensitive Systems:} Begin the rollout of MFA for all employees to access sensitive data systems. This is a critical compensating control while other weaknesses are being addressed.
\end{enumerate}

\subsection*{Priority 2: Short-Term Actions (Within 30-60 Days)}
\begin{enumerate}
    \item \textbf{Upgrade the MySQL Database:} Plan and execute the migration from MySQL 5.7.33 to a fully supported version (e.g., MySQL 8.x). This will ensure the system receives ongoing security patches.
    \item \textbf{Enforce MFA for All Logins:} Expand the MFA rollout to require it for all computer and remote access logins (e.g., VPN, RDP).
    \item \textbf{Implement Annual Security Training:} Procure and schedule a mandatory security awareness training program for all employees. This program should be repeated annually to reinforce best practices.
\end{enumerate}

\subsection*{Priority 3: Long-Term Strategic Actions (Within 6-12 Months)}
\begin{enumerate}
    \item \textbf{Establish a Vulnerability Management Program:} Develop a formal process for regularly scanning internal and external systems for vulnerabilities, tracking findings, and ensuring timely remediation.
    \item \textbf{Implement a Secure Access Architecture:} Formalize the use of a Virtual Private Network (VPN) or Zero Trust Network Access (ZTNA) solution for all remote access to internal resources, eliminating the need for any direct public exposure of services like databases.
\end{enumerate}

\end{document}
```