```latex
\documentclass[12pt]{article}

% Preamble: Required Packages
\usepackage[margin=1in]{geometry}
\usepackage{pifont} % For using dingbats like checkmarks and crosses
\usepackage{booktabs} % For professional-looking tables
\usepackage{hyperref} % For clickable links and references
\usepackage{url}      % For formatting URLs
\usepackage{seqsplit} % To split long strings in \texttt
\usepackage{xcolor}   % For custom colors

% --- Document Metadata ---
\title{Cybersecurity Posture Assessment Report \\ \large For \textbf{[Organization Name]}}
\author{Cybersecurity Analysis Division}
\date{\today}

% --- Custom Commands & Settings ---
\hypersetup{
    colorlinks=true,
    linkcolor=blue,
    filecolor=magenta,      
    urlcolor=cyan,
}
\newcommand{\yes}{\ding{51}} % Green checkmark
\newcommand{\no}{\ding{55}}  % Red X

\begin{document}

\maketitle
\thispagestyle{empty}
\newpage

\tableofcontents
\newpage

% ==============================================================================
\section*{1. Executive Summary}
% ==============================================================================

This report provides a cybersecurity posture assessment for \textbf{[Organization Name]}, based on an analysis of organizational security controls, a network vulnerability scan, and a review of pre-existing risks. The assessment was conducted to identify key security gaps and provide actionable recommendations to enhance the organization's defensive capabilities.

The analysis revealed several areas of concern requiring immediate attention. The most critical finding is the absence of Multi-Factor Authentication (MFA) for email access, which exposes the organization to a high risk of business email compromise and phishing attacks. Additionally, the lack of a formal Employee Acceptable Use Policy represents a significant governance gap.

On the technical front, a network scan identified an exposed Secure Shell (SSH) service on the external network perimeter. While necessary for remote administration, this service must be properly hardened to prevent unauthorized access.

The organization demonstrates strengths in requiring MFA for computer and sensitive data system access, as well as maintaining a security awareness training program. By addressing the critical vulnerabilities identified in this report, \textbf{[Organization Name]} can significantly improve its overall security posture.

% ==============================================================================
\section*{2. Organizational Information}
% ==============================================================================

The following information was used as the basis for this assessment. Placeholders are used where data was not provided.

\begin{tabular}{@{}ll}
\toprule
\textbf{Attribute} & \textbf{Value} \\
\midrule
Organization Name & \textbf{[Organization Name]} \\
Primary Email Domain & \seqsplit{\texttt{[Domain]}} \\
External IP Address Scanned & \seqsplit{\texttt{[Client IP]}} \\
\bottomrule
\end{tabular}

% ==============================================================================
\section*{3. Security Control Review}
% ==============================================================================

A review of the organization's security controls was conducted via a standardized questionnaire. The responses indicate key strengths and weaknesses in the current security program. "No" answers highlight significant gaps that increase organizational risk.

\begin{tabular}{@{}p{0.75\linewidth}c@{}}
\toprule
\textbf{Control Question} & \textbf{Response} \\
\midrule
Do you require MFA to access email? & \no \\
Do you require MFA to log into computers? & \yes \\
Do you require MFA to access sensitive data systems? & \yes \\
Does your organization have an employee acceptable use policy? & \no \\
Does your organization do security awareness training for new employees? & \yes \\
Does your organization do security awareness training for all employees at least once per year? & \yes \\
\bottomrule
\end{tabular}

% ==============================================================================
\section*{4. Technical Scan Results}
% ==============================================================================

An external network scan was performed on the target IP address to identify open ports and exposed services.

\subsection*{Scan Details}
\begin{itemize}
    \item \textbf{Target IP:} \seqsplit{\texttt{[Target IP]}}
    \item \textbf{Scan Type:} TCP Port Scan
\end{itemize}

\subsection*{Open Ports Discovered}
The following table details the ports found to be open and accessible from the public internet. No detailed service version information was available from this scan.

\begin{tabular}{@{}llll@{}}
\toprule
\textbf{Port} & \textbf{State} & \textbf{Service (Inferred)} & \textbf{Notes} \\
\midrule
22/tcp & open & SSH (Secure Shell) & Common for remote server administration. \\
\bottomrule
\end{tabular}

\subsection*{Technical Analysis}
The presence of an open SSH port (22) indicates that a system is configured for remote command-line administration. While this is a standard operational practice, exposing SSH directly to the internet without proper hardening creates a significant attack vector. Attackers frequently scan for open SSH ports to perform brute-force password attacks or exploit known vulnerabilities in outdated SSH server versions.

% ==============================================================================
\section*{5. Consolidated Risk Assessment}
% ==============================================================================

The following table synthesizes findings from the security control review and the technical scan into a prioritized list of risks. No pre-existing vulnerabilities were provided for inclusion.

\begin{tabular}{@{}p{0.1\linewidth}p{0.2\linewidth}p{0.5\linewidth}p{0.1\linewidth}@{}}
\toprule
\textbf{ID} & \textbf{Risk Name} & \textbf{Description} & \textbf{Severity} \\
\midrule
RISK-001 & \textbf{Lack of MFA on Email} & Email accounts are protected only by passwords, making them highly vulnerable to phishing, credential stuffing, and account takeover attacks. & \textbf{Critical} \\
\addlinespace
RISK-002 & \textbf{No Acceptable Use Policy} & The absence of a formal policy defining acceptable use of company assets leads to inconsistent user behavior and a lack of enforceable security standards. & \textbf{High} \\
\addlinespace
RISK-003 & \textbf{Exposed SSH Service} & The SSH management port is open to the public internet, creating a target for automated brute-force attacks and potential exploitation if not securely configured. & \textbf{Medium} \\
\bottomrule
\end{tabular}

% ==============================================================================
\section*{6. Recommendations}
% ==============================================================================

The following actions are recommended to mitigate the identified risks and strengthen the overall security posture of \textbf{[Organization Name]}.

\subsection*{RISK-001: Lack of MFA on Email (Critical)}
\begin{enumerate}
    \item \textbf{Immediate Action:} Enforce MFA for all user and administrative email accounts without exception.
    \item \textbf{Configuration:} Prioritize phishing-resistant MFA methods such as FIDO2 security keys or authenticator apps over less secure methods like SMS.
    \item \textbf{User Training:} Communicate the change to all employees and provide clear instructions on how to enroll in and use MFA.
\end{enumerate}

\subsection*{RISK-002: No Acceptable Use Policy (High)}
\begin{enumerate}
    \item \textbf{Policy Development:} Draft a comprehensive Employee Acceptable Use Policy (AUP) that covers topics such as data handling, internet usage, personal device usage (BYOD), and security incident reporting.
    \item \textbf{Implementation:} Require all current employees to read and formally acknowledge the new policy. Incorporate this step into the onboarding process for all new hires.
    \item \textbf{Enforcement:} Establish clear consequences for policy violations and ensure they are applied consistently.
\end{enumerate}

\subsection*{RISK-003: Exposed SSH Service (Medium)}
\begin{enumerate}
    \item \textbf{Review Necessity:} Confirm the business requirement for exposing SSH to the public internet. If possible, restrict access to trusted IP addresses via firewall rules or place the service behind a VPN.
    \item \textbf{Harden Configuration:} If exposure is necessary, implement the following security best practices:
    \begin{itemize}
        \item Disable password-based authentication and enforce the use of public key cryptography.
        \item Disable root user login over SSH.
        \item Implement an intrusion detection system like Fail2Ban to automatically block IPs that exhibit malicious behavior.
        \item Ensure the SSH server software is kept up-to-date with the latest security patches.
    \end{itemize}
\end{enumerate}

\end{document}
```