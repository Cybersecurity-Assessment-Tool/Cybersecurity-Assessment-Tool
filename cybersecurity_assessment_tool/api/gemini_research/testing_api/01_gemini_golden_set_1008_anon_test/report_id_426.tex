```latex
\documentclass[12pt]{article}

% Preamble: Required Packages
\usepackage[margin=1in]{geometry}
\usepackage{pifont} % For checkmarks and crosses (dingbats)
\usepackage{booktabs} % For professional-looking tables
\usepackage{hyperref} % For clickable links
\usepackage{url} % For URL formatting
\usepackage{seqsplit} % To split long strings without breaking
\usepackage{graphicx}
\usepackage{xcolor}
\usepackage{fancyhdr}
\usepackage{lastpage}

% --- Document Setup ---
\hypersetup{
    colorlinks=true,
    linkcolor=blue,
    filecolor=magenta,      
    urlcolor=cyan,
    pdftitle={Cybersecurity Posture Assessment Report},
    pdfpagemode=FullScreen,
}

% Define colors for table headers and severity
\definecolor{tablehead}{HTML}{2E4053}
\definecolor{critical}{HTML}{C70039}
\definecolor{high}{HTML}{FF5733}
\definecolor{medium}{HTML}{FFC300}
\definecolor{low}{HTML}{DAF7A6}
\definecolor{info}{HTML}{5DADE2}

% --- Header and Footer ---
\pagestyle{fancy}
\fancyhf{}
\lhead{Cybersecurity Posture Assessment}
\rhead{\textbf{[Organization Name]}}
\cfoot{Page \thepage\ of \pageref{LastPage}}
\renewcommand{\headrulewidth}{0.4pt}
\renewcommand{\footrulewidth}{0.4pt}

\begin{document}

% --- Title Page ---
\begin{titlepage}
    \centering
    \vspace*{2cm}
    
    \Huge \textbf{Cybersecurity Posture Assessment Report}
    
    \vspace{1.5cm}
    
    \Large Prepared for:
    
    \vspace{0.5cm}
    
    \textbf{\Huge [Organization Name]}
    
    \vspace{2cm}
    
    {\large \today}
    
    \vfill
    
    \large \textit{This report contains sensitive information and should be handled with care.}
    
\end{titlepage}

\tableofcontents
\newpage

% --- Section 1: Executive Summary ---
\section{Executive Summary}

This report provides a comprehensive cybersecurity posture assessment for \textbf{[Organization Name]}, based on an analysis of organizational security controls, an external network scan, and a review of pre-existing risks.

The assessment identified several critical and high-risk gaps in the organization's security policies and access controls. The most pressing concerns are the lack of multi-factor authentication (MFA) for email access, the absence of an employee acceptable use policy, and incomplete security awareness training for new hires. These deficiencies significantly increase the risk of business email compromise, insider threats, and successful phishing attacks.

On a positive note, the external network scan of the target IP address, \texttt{[Target IP]}, did not reveal any open ports. This suggests that a previously identified risk concerning an unencrypted web server on Port 80 has likely been remediated.

Immediate action should be focused on implementing MFA for email, developing foundational security policies, and ensuring all employees receive security training upon hiring. Addressing these core issues will substantially improve the organization's resilience against common cyber threats.

% --- Section 2: Organizational Information ---
\section{Organizational Information}

The following details were used as the basis for this assessment. Due to the anonymized nature of the provided data, placeholders have been used where specific information was not available.

\begin{table}[h!]
\centering
\begin{tabular}{@{}ll@{}}
\toprule
\textbf{Attribute} & \textbf{Value} \\ \midrule
Organization Name & \textbf{[Organization Name]} \\
Primary Email Domain & \texttt{[Domain]} \\
External IP Address (Provided) & \texttt{[Client IP]} \\ \bottomrule
\end{tabular}
\caption{Client Organizational Details}
\end{table}

% --- Section 3: Security Control Review ---
\section{Security Control Review}

A review of the organization's security controls was conducted via a questionnaire. The responses highlight significant gaps in fundamental security practices. A "No" answer indicates a deviation from best practices and represents a potential risk.

\begin{table}[h!]
\centering
\begin{tabular}{@{}p{0.5\textwidth} c p{0.3\textwidth}@{}}
\toprule
\rowcolor{tablehead}
\textcolor{white}{\textbf{Control Question}} & \textcolor{white}{\textbf{Response}} & \textcolor{white}{\textbf{Analyst Note}} \\ \midrule

Do you require MFA to access email? & \textcolor{red}{\ding{55}} & \textbf{Critical Risk.} Lack of MFA on email is a primary vector for account takeovers. \\ \addlinespace

Do you require MFA to log into computers? & \textcolor{green}{\ding{51}} & Good Practice. This strengthens endpoint security. \\ \addlinespace

Do you require MFA to access sensitive data systems? & \textcolor{green}{\ding{51}} & Good Practice. Protects critical organizational data. \\ \addlinespace

Does your organization have an employee acceptable use policy? & \textcolor{red}{\ding{55}} & \textbf{High Risk.} Absence of a formal policy creates ambiguity and legal exposure. \\ \addlinespace

Does your organization do security awareness training for new employees? & \textcolor{red}{\ding{55}} & \textbf{High Risk.} New hires are a prime target for social engineering attacks. \\ \addlinespace

Does your organization do security awareness training for all employees at least once per year? & \textcolor{green}{\ding{51}} & Good Practice. Regular training reinforces security-conscious behavior. \\ \bottomrule
\end{tabular}
\caption{Organizational Security Controls Questionnaire}
\end{table}

% --- Section 4: Technical Scan Results ---
\section{Technical Scan Results}

An external network scan was performed using Nmap to identify accessible services and potential vulnerabilities on the perimeter.

\begin{itemize}
    \item \textbf{Target IP Address:} \texttt{[Target IP]}
    \item \textbf{Scan Date:} Not specified in scan data
    \item \textbf{Host Status:} Up
\end{itemize}

The scan results indicate that the host is online, but no open ports were discovered. Specifically, Port 80 (HTTP), which was listed as a concern in pre-existing risk data, was found to be \textbf{closed}. This is a positive security finding, suggesting the service is either not running, firewalled, or has been intentionally disabled.

\begin{table}[h!]
\centering
\begin{tabular}{@{}lll@{}}
\toprule
\rowcolor{tablehead}
\textcolor{white}{\textbf{Port}} & \textcolor{white}{\textbf{State}} & \textcolor{white}{\textbf{Service}} \\ \midrule
80/tcp & Closed & http \\ \bottomrule
\end{tabular}
\caption{Nmap Scan Results for \texttt{[Target IP]}}
\end{table}

% --- Section 5: Risk Assessment ---
\section{Risk Assessment}

This section synthesizes findings from the security control review, technical scan, and pre-existing risk data into a consolidated list of current risks.

\begin{table}[h!]
\centering
\begin{tabular}{@{}p{0.3\textwidth} p{0.5\textwidth} l@{}}
\toprule
\rowcolor{tablehead}
\textcolor{white}{\textbf{Risk Name}} & \textcolor{white}{\textbf{Overview}} & \textcolor{white}{\textbf{Severity}} \\ \midrule

\textbf{Lack of MFA on Email} & The absence of MFA on email accounts allows an attacker with valid credentials (e.g., from a phishing attack or password reuse) to gain full access to an employee's mailbox, leading to data breaches and further attacks. & \colorbox{critical}{\textcolor{white}{\textbf{Critical}}} \\ \addlinespace

\textbf{Missing Acceptable Use Policy (AUP)} & Without a formal AUP, employees lack clear guidelines on the acceptable use of company assets. This increases the risk of insider threats, data leakage, and non-compliance. & \colorbox{high}{\textcolor{white}{\textbf{High}}} \\ \addlinespace

\textbf{Inadequate Security Awareness Training} & Failing to train new hires on security best practices leaves them vulnerable to social engineering and phishing attacks from their first day, creating a significant entry point for attackers. & \colorbox{high}{\textcolor{white}{\textbf{High}}} \\ \addlinespace

\textbf{Unencrypted Web Server (Potentially Remediated)} & A pre-existing risk noted that Port 80 was open. Our current scan shows this port is closed, indicating the risk may be remediated. Verification across all external assets is recommended. & \colorbox{info}{\textcolor{white}{\textbf{Info}}} \\ \bottomrule
\end{tabular}
\caption{Consolidated Risk Summary}
\end{table}

% --- Section 6: Recommendations ---
\section{Recommendations}

The following prioritized recommendations are provided to mitigate the identified risks and improve the overall security posture of \textbf{[Organization Name]}.

\subsection{Priority 1: Critical}
\begin{enumerate}
    \item \textbf{Implement MFA for Email Access:} Immediately enforce MFA for all user accounts on the email platform (e.g., Microsoft 365, Google Workspace). This is the single most effective control to prevent business email compromise.
\end{enumerate}

\subsection{Priority 2: High}
\begin{enumerate}
    \setcounter{enumi}{1}
    \item \textbf{Develop and Implement an Acceptable Use Policy (AUP):} Create a formal AUP that clearly defines the rules for using company networks, devices, and data. This policy should be communicated to all employees and signed as part of the onboarding process.
    \item \textbf{Establish a New Hire Security Training Program:} Integrate mandatory security awareness training into the employee onboarding process. This training should cover key topics such as phishing, password security, and the new AUP.
\end{enumerate}

\subsection{Priority 3: Informational}
\begin{enumerate}
    \setcounter{enumi}{3}
    \item \textbf{Verify Remediation of Port 80:} Conduct a comprehensive scan of all external-facing IP addresses to confirm that no services are running on Port 80 (HTTP). All web traffic should be redirected to Port 443 (HTTPS) with strong TLS encryption.
\end{enumerate}

% --- Section 7: Conclusion ---
\section{Conclusion}

The assessment reveals that while the external network perimeter of the scanned target appears secure, significant internal policy and access control weaknesses exist within \textbf{[Organization Name]}. The identified gaps in MFA, employee policies, and training represent a substantial risk to the organization.

By implementing the prioritized recommendations in this report, the organization can take immediate and effective steps to strengthen its defenses, reduce its attack surface, and build a more resilient security foundation.

\end{document}
```