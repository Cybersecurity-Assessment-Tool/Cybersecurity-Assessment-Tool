```latex
\documentclass[12pt]{article}

% Required Packages
\usepackage[margin=1in]{geometry}
\usepackage{pifont} % For \ding
\usepackage{booktabs} % For professional tables
\usepackage{hyperref} % For hyperlinks
\usepackage{url} % For URL formatting
\usepackage{seqsplit} % For splitting long strings in tt font
\usepackage{xcolor} % For colors

% Document Metadata
\title{Cybersecurity Posture Assessment Report}
\author{Cybersecurity Analysis Division}
\date{\today}

% Hyperref Setup
\hypersetup{
    colorlinks=true,
    linkcolor=blue,
    filecolor=magenta,      
    urlcolor=cyan,
    pdftitle={Cybersecurity Posture Assessment Report},
    pdfpagemode=FullScreen,
}

\begin{document}

\maketitle
\thispagestyle{empty}
\newpage
\tableofcontents
\thispagestyle{empty}
\newpage
\setcounter{page}{1}

% --- 1. Executive Summary ---
\section{Executive Summary}

This report provides a comprehensive cybersecurity assessment for \textbf{[Organization Name]}, based on an analysis of network scan data, organizational security controls, and pre-existing risk documentation. The assessment synthesizes these inputs to provide a holistic view of the organization's current security posture.

The analysis identified several critical and high-severity risks that require immediate attention. Key findings include:
\begin{itemize}
    \item \textbf{Critical External Exposure:} An externally facing FTP server was found running a dangerously outdated version of \texttt{vsftpd} (2.3.4), which is known to contain a critical backdoor vulnerability (CVE-2011-2523). The service is also misconfigured to allow anonymous logins, posing a severe and immediate threat.
    \item \textbf{Insufficient Access Controls:} The organization does not enforce Multi-Factor Authentication (MFA) for accessing email or other sensitive data systems. This represents a critical gap in identity and access management, significantly increasing the risk of account compromise and data breaches.
    \item \textbf{Inadequate Employee Training:} Security awareness training is not conducted annually for all employees, leading to a heightened risk of successful social engineering and phishing attacks.
    \item \textbf{Known End-of-Life Systems:} The organization is aware of its use of Windows 7, an End-of-Life operating system that no longer receives security updates, leaving workstations vulnerable to exploitation.
\end{itemize}

The combination of these findings indicates a reactive security posture with significant vulnerabilities across technical, administrative, and physical controls. This report outlines actionable recommendations to mitigate these risks and strengthen the overall security framework.

% --- 2. Organizational Information ---
\section{Organizational Information}

This section details the information provided by the client. As this assessment was conducted in a template mode, placeholders are used for sensitive data.

\begin{tabular}{@{}ll}
    \toprule
    \textbf{Attribute} & \textbf{Value} \\
    \midrule
    Organization Name & \textbf{[Organization Name]} \\
    Primary Email Domain & \texttt{[Domain]} \\
    External IP Address Scanned & \texttt{[Client IP]} \\
    \bottomrule
\end{tabular}

% --- 3. Security Control Review (Questionnaire) ---
\section{Security Control Review (Questionnaire)}

An internal security questionnaire was reviewed to assess the maturity of existing administrative controls. The responses are summarized below. Items marked with \ding{55} indicate significant gaps in the security program.

\begin{table}[h!]
\centering
\begin{tabular}{@{}p{0.7\linewidth}c@{}}
    \toprule
    \textbf{Control Question} & \textbf{Response} \\
    \midrule
    Do you require MFA to access email? & \textcolor{red}{\ding{55}} \\
    Do you require MFA to log into computers? & \textcolor{green}{\ding{51}} \\
    Do you require MFA to access sensitive data systems? & \textcolor{red}{\ding{55}} \\
    Does your organization have an employee acceptable use policy? & \textcolor{green}{\ding{51}} \\
    Does your organization do security awareness training for new employees? & \textcolor{green}{\ding{51}} \\
    Does your organization do security awareness training for all employees at least once per year? & \textcolor{red}{\ding{55}} \\
    \bottomrule
\end{tabular}
\caption{Organizational Security Control Responses}
\end{table}

\subsection*{Analysis}
The questionnaire reveals critical deficiencies in access control and security training. The absence of MFA on email and sensitive data systems exposes the organization to significant risk from credential theft and phishing attacks. While a baseline for training new hires exists, the lack of mandatory annual training for all staff leaves the organization vulnerable to evolving social engineering tactics.

% --- 4. Technical Scan Results ---
\section{Technical Scan Results}

An external network scan was performed on the target IP address to identify open ports and exposed services.

\subsection*{Scan Target}
\textbf{Target IP:} \texttt{[Target IP]}

\subsection*{Open Ports and Services}
The following table details the services discovered during the scan.

\begin{table}[h!]
\centering
\begin{tabular}{@{}lllll@{}}
    \toprule
    \textbf{Port} & \textbf{State} & \textbf{Service} & \textbf{Version} & \textbf{Notes} \\
    \midrule
    21/tcp & Open & ftp & \seqsplit{\texttt{vsftpd 2.3.4}} & \textbf{Critical.} Anonymous login allowed. \\
    & & & & Version is vulnerable to CVE-2011-2523. \\
    \bottomrule
\end{tabular}
\caption{Discovered Network Services}
\end{table}

\subsection*{Analysis}
The technical scan identified a critical vulnerability. Port 21 (FTP) is open and running \texttt{vsftpd} version 2.3.4. This specific version, released in 2011, contains a well-documented backdoor vulnerability (CVE-2011-2523) that allows for remote command execution. Furthermore, the service is configured to allow anonymous FTP logins, permitting unauthenticated users to access the file system. This configuration represents a direct and immediate threat to the integrity and confidentiality of the network.

% --- 5. Consolidated Risk Assessment ---
\section{Consolidated Risk Assessment}

The following table synthesizes findings from the questionnaire, technical scan, and pre-existing risk documentation into a consolidated list of security risks.

\begin{table}[h!]
\centering
\begin{tabular}{@{}p{0.25\linewidth}p{0.45\linewidth}l@{}}
    \toprule
    \textbf{Risk Name} & \textbf{Description} & \textbf{Severity} \\
    \midrule
    \textbf{Insecure FTP Server} & An outdated and misconfigured FTP server (\texttt{vsftpd 2.3.4}) is exposed to the internet, allowing anonymous access and vulnerable to remote code execution. & \textbf{Critical} \\
    \addlinespace
    \textbf{No MFA on Email} & Lack of multi-factor authentication for email access significantly increases the risk of business email compromise and phishing. & \textbf{Critical} \\
    \addlinespace
    \textbf{No MFA on Sensitive Systems} & Sensitive data systems are not protected by MFA, creating a high risk of unauthorized access and data exfiltration. & \textbf{Critical} \\
    \addlinespace
    \textbf{Insufficient Security Training} & Security awareness training is not conducted annually for all employees, weakening the human firewall against social engineering. & \textbf{High} \\
    \addlinespace
    \textbf{Outdated Operating Systems} & Workstations are running Windows 7, an End-of-Life OS that no longer receives security updates from the vendor. & \textbf{Medium} \\
    \bottomrule
\end{tabular}
\caption{Summary of Identified Risks}
\end{table}

% --- 6. Recommendations ---
\section{Recommendations}

Based on the consolidated risk assessment, the following actions are recommended to mitigate the identified vulnerabilities and improve the organization's security posture.

\subsection*{Immediate Priority (Critical Risks)}
\begin{enumerate}
    \item \textbf{Remediate Insecure FTP Server:}
    \begin{itemize}
        \item Take the FTP server offline immediately.
        \item Conduct a forensic analysis to determine if the server has already been compromised.
        \item If FTP is a business requirement, replace it with a secure protocol such as SFTP (SSH File Transfer Protocol) and enforce strong, unique credentials for all users. Anonymous access must be disabled.
    \end{itemize}
    \item \textbf{Implement Multi-Factor Authentication (MFA):}
    \begin{itemize}
        \item Enforce MFA across all user accounts for email access (e.g., Office 365, Google Workspace) within the next 7 days.
        \item Develop and execute a plan to roll out MFA for all systems containing sensitive data within the next 30 days.
    \end{itemize}
\end{enumerate}

\subsection*{High Priority}
\begin{enumerate}
    \setcounter{enumi}{2}
    \item \textbf{Establish Annual Security Training:}
    \begin{itemize}
        \item Procure and implement a security awareness training program for all employees.
        \item Make completion of this training mandatory on an annual basis. Track completion rates to ensure compliance.
    \end{itemize}
    \item \textbf{Accelerate OS Upgrade Plan:}
    \begin{itemize}
        \item Prioritize the existing project to upgrade all Windows 7 workstations to a currently supported operating system (e.g., Windows 10/11).
        \item Isolate any machines that cannot be immediately upgraded from the main corporate network to limit their exposure.
    \end{itemize}
\end{enumerate}

\end{document}
```