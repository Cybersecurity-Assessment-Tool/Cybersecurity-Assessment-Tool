```latex
\documentclass[12pt]{article}

% Preamble: Required Packages and Document Setup
\usepackage[margin=1in]{geometry}
\usepackage{pifont} % For checkmarks and crosses
\usepackage{booktabs} % For professional tables
\usepackage{hyperref} % For clickable links and references
\usepackage{url} % For formatting URLs
\usepackage{seqsplit} % For splitting long strings to prevent overflow
\usepackage{xcolor} % For custom colors
\usepackage{graphicx} % For logos or images
\usepackage{lastpage} % To get the total number of pages
\usepackage{fancyhdr} % For custom headers and footers

% --- Document Metadata and Hyperref Setup ---
\hypersetup{
    colorlinks=true,
    linkcolor=blue,
    filecolor=magenta,      
    urlcolor=cyan,
    pdftitle={Cybersecurity Posture Assessment Report},
    pdfauthor={Cybersecurity Analysis Division},
    pdfsubject={Security Assessment},
    pdfkeywords={Cybersecurity, Risk, Assessment, Nmap, Policy},
    bookmarks=true
}

% --- Custom Colors ---
\definecolor{darkblue}{rgb}{0.0, 0.0, 0.55}
\definecolor{darkred}{rgb}{0.55, 0.0, 0.0}

% --- Header and Footer Configuration ---
\pagestyle{fancy}
\fancyhf{} % Clear all header and footer fields
\fancyhead[L]{\textbf{Cybersecurity Posture Assessment}}
\fancyhead[R]{\textbf{[Organization Name]}}
\fancyfoot[C]{\thepage\ of \pageref{LastPage}}
\renewcommand{\headrulewidth}{0.4pt}
\renewcommand{\footrulewidth}{0.4pt}

% --- Document Start ---
\begin{document}

% --- Title Page ---
\begin{titlepage}
    \centering
    \vspace*{1cm}
    
    \includegraphics[width=0.4\textwidth]{example-image-a} % Placeholder for a logo
    
    \vspace{1.5cm}
    
    \Huge\textbf{Cybersecurity Posture Assessment Report}
    
    \vspace{1.5cm}
    
    \Large Prepared for: \\
    \vspace{0.5cm}
    \huge\textbf{[Organization Name]}
    
    \vfill
    
    \large
    \textbf{Date of Report:} \today \\
    \textbf{Report ID:} CSA-2023-10-27-001
    
\end{titlepage}

\tableofcontents
\newpage

% --- Section 1: Executive Summary ---
\section{Executive Summary}

This report details the findings of a cybersecurity posture assessment conducted for \textbf{[Organization Name]}. The assessment combined a review of organizational security controls, an external network scan, and an analysis of pre-existing risks to provide a holistic view of the organization's security posture.

\paragraph{Key Findings:} The external network scan of the provided IP address revealed a positive security posture from a network perimeter perspective. No open ports or exposed services were detected, suggesting a well-configured firewall that effectively limits the external attack surface.

However, the review of organizational security controls identified two significant gaps that present a high level of risk to the organization:
\begin{itemize}
    \item \textbf{Critical Risk:} The absence of Multi-Factor Authentication (MFA) on email accounts. Email is a primary target for attackers, and this gap leaves the organization highly vulnerable to phishing, business email compromise, and account takeovers.
    \item \textbf{High Risk:} The lack of mandatory, annual security awareness training for all employees. A well-trained workforce is a critical layer of defense. Without ongoing training, employees are more susceptible to evolving social engineering and phishing tactics.
\end{itemize}

\paragraph{Conclusion:} While the technical network perimeter is secure, the identified policy and procedure-based weaknesses must be addressed urgently. The recommendations in this report focus on mitigating these critical human-centric risks to significantly improve the overall security posture of \textbf{[Organization Name]}.

% --- Section 2: Organizational Information ---
\section{Organizational Information}
This section provides the context for the assessment based on the information provided.

\begin{tabular}{@{}ll}
    \toprule
    \textbf{Detail} & \textbf{Information} \\
    \midrule
    Organization Name & \textbf{[Organization Name]} \\
    Primary Domain & \texttt{[Domain]} \\
    External IP Scanned & \texttt{[Client IP]} \\
    \bottomrule
\end{tabular}

% --- Section 3: Security Control Review ---
\section{Security Control Review}
The following table summarizes the responses to the security questionnaire. "No" answers indicate potential gaps in security controls that often translate to significant organizational risk.

\begin{table}[h!]
\centering
\caption{Security Controls Questionnaire Analysis}
\begin{tabular}{p{0.55\textwidth} c p{0.2\textwidth}}
    \toprule
    \textbf{Control Question} & \textbf{Response} & \textbf{Assessment} \\
    \midrule
    Do you require MFA to access email? & \textcolor{darkred}{\ding{55}} & \textbf{Critical Gap} \\
    \addlinespace
    Do you require MFA to log into computers? & \textcolor{darkblue}{\ding{51}} & Best Practice Met \\
    \addlinespace
    Do you require MFA to access sensitive data systems? & \textcolor{darkblue}{\ding{51}} & Best Practice Met \\
    \addlinespace
    Does your organization have an employee acceptable use policy? & \textcolor{darkblue}{\ding{51}} & Best Practice Met \\
    \addlinespace
    Does your organization do security awareness training for new employees? & \textcolor{darkblue}{\ding{51}} & Best Practice Met \\
    \addlinespace
    Does your organization do security awareness training for all employees at least once per year? & \textcolor{darkred}{\ding{55}} & \textbf{High Risk} \\
    \bottomrule
\end{tabular}
\end{table}

% --- Section 4: Technical Scan Results ---
\section{Technical Scan Results}
An external network scan was performed to identify exposed services and potential vulnerabilities on the organization's network perimeter.

\subsection{Scan Details}
\begin{itemize}
    \item \textbf{Target IP:} \texttt{[Target IP]}
    \item \textbf{Scan Date:} \today
    \item \textbf{Scanner Used:} Nmap
\end{itemize}

\subsection{Port Scan Findings}
The scan confirmed that the target host was online and responsive. However, the results were positive from a security standpoint:
\begin{itemize}
    \item \textbf{Open Ports:} 0
    \item \textbf{Filtered Ports:} 0
    \item \textbf{Closed Ports:} All 1000 scanned ports were in a 'closed' state.
\end{itemize}

\paragraph{Analysis:} The absence of any open or filtered ports indicates a strong firewall configuration. This "default deny" posture is a security best practice, as it minimizes the external attack surface and prevents unauthorized access to internal systems. No vulnerabilities were identified through this scan.

% --- Section 5: Risk Assessment Summary ---
\section{Risk Assessment Summary}
This section synthesizes findings from the security control review and technical scan. While no pre-existing vulnerabilities were reported and the technical scan was clean, the policy gaps represent significant, actionable risks.

\begin{table}[h!]
\centering
\caption{Identified Risks}
\begin{tabular}{p{0.1\textwidth} p{0.25\textwidth} p{0.45\textwidth} p{0.1\textwidth}}
    \toprule
    \textbf{Risk ID} & \textbf{Risk Name} & \textbf{Description} & \textbf{Severity} \\
    \midrule
    R-01 & Lack of MFA for Email Access & Email accounts are not protected by Multi-Factor Authentication. This exposes the organization to a high likelihood of account compromise via phishing or credential theft, leading to data breaches and further network intrusion. & \textbf{Critical} \\
    \addlinespace
    R-02 & Inadequate Security Awareness Training & Security training is not conducted annually for all staff. This leaves the organization vulnerable to evolving social engineering tactics, as employees' awareness of current threats diminishes over time. & \textbf{High} \\
    \bottomrule
\end{tabular}
\end{table}

% --- Section 6: Recommendations ---
\section{Recommendations}
Based on the identified risks, the following prioritized actions are recommended to improve the organization's cybersecurity posture.

\begin{enumerate}
    \item \textbf{Implement MFA for Email (Risk R-01 - Critical):}
    \begin{itemize}
        \item \textbf{Action:} Immediately enable and enforce Multi-Factor Authentication (MFA) for all user accounts with access to the organization's email system.
        \item \textbf{Justification:} This is the single most effective control to prevent unauthorized account access and mitigate the risk of business email compromise (BEC) and phishing-related data breaches.
    \end{itemize}
    \vspace{0.5cm}

    \item \textbf{Establish Annual Security Training (Risk R-02 - High):}
    \begin{itemize}
        \item \textbf{Action:} Develop and implement a mandatory annual security awareness training program for all employees. This program should be updated yearly to include information on current threats.
        \item \textbf{Justification:} Regular training reinforces security best practices and equips employees to recognize and report modern threats like sophisticated phishing and social engineering attempts, turning a potential weakness into a strong line of defense.
    \end{itemize}
    \vspace{0.5cm}
    
    \item \textbf{Maintain Strong Network Perimeter:}
    \begin{itemize}
        \item \textbf{Action:} Continue the current practice of maintaining a restrictive firewall policy. Conduct periodic external network scans (at least quarterly) to validate that no new services have been inadvertently exposed.
        \item \textbf{Justification:} While the current configuration is secure, regular validation ensures that it remains so through system changes and updates.
    \end{itemize}
\end{enumerate}

\end{document}
```