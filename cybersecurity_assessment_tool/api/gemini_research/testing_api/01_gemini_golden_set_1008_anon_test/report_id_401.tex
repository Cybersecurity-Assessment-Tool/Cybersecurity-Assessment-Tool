```latex
\documentclass[12pt]{article}

% Preamble: Required Packages
\usepackage[margin=1in]{geometry}
\usepackage{pifont} % For checkmarks and crosses (\ding{51}, \ding{55})
\usepackage{booktabs} % For professional tables
\usepackage{hyperref} % For clickable links and table of contents
\usepackage{url} % For formatting URLs
\usepackage{seqsplit} % To prevent long strings from overflowing
\usepackage{graphicx} % For logo (optional, placeholder included)
\usepackage{xcolor} % For colors

% Hyperref and Color Setup
\hypersetup{
    colorlinks=true,
    linkcolor=black,
    filecolor=magenta,      
    urlcolor=blue,
    pdftitle={Cybersecurity Posture Assessment Report},
    pdfpagemode=FullScreen,
}

% Define custom colors for severity
\definecolor{criticalred}{HTML}{D7263D}
\definecolor{highorange}{HTML}{F49D40}
\definecolor{mediumyellow}{HTML}{F4D440}

% Document Start
\begin{document}

% --- TITLE PAGE ---
\begin{titlepage}
    \centering
    \vspace*{1cm}
    
    \Huge\textbf{Cybersecurity Posture Assessment Report}
    
    \vspace{1.5cm}
    
    \Large Prepared for: \\
    \vspace{0.5cm}
    \textbf{[Organization Name]}
    
    \vspace{2cm}
    
    \Large Prepared by: \\
    \vspace{0.5cm}
    Cybersecurity Analyst
    
    \vfill
    
    {\large \today}
\end{titlepage}

\tableofcontents
\newpage

% --- EXECUTIVE SUMMARY ---
\section{Executive Summary}
This report provides a comprehensive assessment of the cybersecurity posture for \textbf{[Organization Name]}, based on an analysis of network scans, organizational security controls, and pre-existing risk data.

The assessment has identified several \textbf{critical-risk findings} that require immediate attention. The primary concern is the direct exposure of Remote Desktop Protocol (RDP) on port 3389 to the public internet. This vulnerability, with a CVSS score of 9.0, creates a direct pathway for attackers into the internal network.

This technical vulnerability is severely compounded by critical gaps in organizational security controls. Specifically, the organization does not enforce Multi-Factor Authentication (MFA) for logging into computers, accessing email, or connecting to sensitive data systems. The combination of exposed RDP and the lack of MFA presents an extreme risk of a successful ransomware attack or data breach through brute-force or credential-stuffing techniques.

Immediate remediation should focus on securing the exposed RDP service by placing it behind a Virtual Private Network (VPN) with MFA enabled. Subsequently, a comprehensive rollout of MFA across all critical systems is strongly advised.

% --- ORGANIZATIONAL INFORMATION ---
\section{Organizational Information}
This section details the information provided by the client organization. The data is used to establish the context for the technical and procedural assessment.

\begin{tabular}{@{}ll}
\toprule
\textbf{Attribute} & \textbf{Value} \\
\midrule
Organization Name & \textbf{[Organization Name]} \\
Primary Domain & \texttt{[Domain]} \\
External IP Address Scanned & \texttt{[Client IP]} \\
\bottomrule
\end{tabular}

% --- SECURITY CONTROL REVIEW ---
\section{Security Control Review}
A review of the organization's security controls was conducted via a questionnaire. The responses highlight significant gaps in foundational security practices, particularly concerning access control. A "No" response indicates a missing control and a potential area of high risk.

\begin{table}[h!]
\centering
\caption{Organizational Security Control Questionnaire Results}
\begin{tabular}{@{}p{0.75\textwidth}c@{}}
\toprule
\textbf{Control Question} & \textbf{Response} \\
\midrule
Do you require MFA to access email? & \textcolor{criticalred}{\ding{55}} \\
Do you require MFA to log into computers? & \textcolor{criticalred}{\ding{55}} \\
Do you require MFA to access sensitive data systems? & \textcolor{criticalred}{\ding{55}} \\
Does your organization have an employee acceptable use policy? & \textcolor{highorange}{\ding{55}} \\
Does your organization do security awareness training for new employees? & \textcolor{green}{\ding{51}} \\
Does your organization do security awareness training for all employees at least once per year? & \textcolor{green}{\ding{51}} \\
\bottomrule
\end{tabular}
\end{table}

\subsection*{Analysis}
The lack of MFA across all critical access points (email, computers, sensitive data) is a \textbf{critical deficiency}. This single point of failure means that a compromised password can lead to a full system compromise. The absence of an Acceptable Use Policy represents a significant governance gap, failing to set clear expectations for employees on the secure use of company assets. The existing security awareness training program is a positive control that should be maintained and enhanced.

% --- TECHNICAL SCAN RESULTS ---
\section{Technical Scan Results}
An external network scan was performed on the provided target IP address to identify open ports and exposed services.

\begin{itemize}
    \item \textbf{Target IP Address:} \texttt{[Target IP]}
    \item \textbf{Scan Date:} Assumed to be recent, as per input data.
\end{itemize}

\begin{table}[h!]
\centering
\caption{Open Ports Detected on \texttt{[Target IP]}}
\begin{tabular}{@{}llll@{}}
\toprule
\textbf{Port} & \textbf{State} & \textbf{Service Name} & \textbf{Common Use} \\
\midrule
3389/tcp & Open & \texttt{ms-wbt-server} & Microsoft Remote Desktop Protocol (RDP) \\
\bottomrule
\end{tabular}
\end{table}

\subsection*{Analysis}
The scan confirms that port 3389 is open, exposing the Microsoft Remote Desktop Protocol (RDP) service directly to the internet. RDP is a primary target for attackers who use automated tools to scan for exposed servers. Once found, they attempt to gain access via brute-force password guessing, exploiting known vulnerabilities, or using stolen credentials. This finding directly corroborates the high-severity risk identified in \texttt{Input\_3\_Current\_Risks\_JSON}.

% --- RISK ASSESSMENT SUMMARY ---
\section{Risk Assessment Summary}
The following table synthesizes findings from the security control review, technical scan, and pre-existing risk data into a prioritized list of identified risks.

\begin{table}[h!]
\centering
\caption{Synthesized Risk Register}
\begin{tabular}{@{}p{0.1\textwidth}p{0.2\textwidth}p{0.5\textwidth}l@{}}
\toprule
\textbf{Risk ID} & \textbf{Risk Name} & \textbf{Description} & \textbf{Severity} \\
\midrule
RISK-001 & \textbf{Exposed RDP Service} & The RDP service on port 3389 is open to the internet, allowing attackers to attempt unauthorized access. This is a common vector for ransomware. & \textcolor{criticalred}{\textbf{Critical}} \\
\addlinespace
RISK-002 & \textbf{Lack of MFA} & The absence of MFA for computer, email, and sensitive data access means that a single compromised password can grant an attacker full access. & \textcolor{criticalred}{\textbf{Critical}} \\
\addlinespace
RISK-003 & \textbf{Missing Acceptable Use Policy} & The lack of a formal policy creates ambiguity for employees regarding security responsibilities and exposes the organization to insider threats and compliance issues. & \textcolor{highorange}{\textbf{High}} \\
\bottomrule
\end{tabular}
\end{table}

% --- RECOMMENDATIONS ---
\section{Recommendations}
Based on the identified risks, the following prioritized recommendations are provided to improve the organization's cybersecurity posture.

\subsection{Immediate Priority (Critical Risks)}
\begin{enumerate}
    \item \textbf{Remediate RDP Exposure (RISK-001):}
    \begin{itemize}
        \item \textbf{Immediate Fix:} Close port 3389 on the external firewall for IP address \texttt{[Client IP]}. All external access to this port should be blocked immediately.
        \item \textbf{Long-Term Solution:} If remote access is required, implement a Virtual Private Network (VPN) solution. Access to the internal network should only be granted after a user successfully authenticates to the VPN, which should be configured to require Multi-Factor Authentication.
    \end{itemize}
    
    \item \textbf{Implement Multi-Factor Authentication (RISK-002):}
    \begin{itemize}
        \item Immediately begin a phased rollout of MFA for all employees.
        \item Prioritize enabling MFA on:
            \begin{enumerate}
                \item All remote access systems (VPN).
                \item Email accounts (e.g., Office 365, Google Workspace).
                \item Access to sensitive data systems and critical infrastructure.
                \item Local computer logins where supported.
            \end{enumerate}
    \end{itemize}
\end{enumerate}

\subsection{High Priority (High Risks)}
\begin{enumerate}
    \setcounter{enumi}{2} % Continue numbering from previous list
    \item \textbf{Develop and Implement an Acceptable Use Policy (AUP) (RISK-003):}
    \begin{itemize}
        \item Draft a formal AUP that clearly defines the rules and responsibilities for all employees when using company IT assets.
        \item The policy should be reviewed by management and legal counsel, distributed to all staff, and formally acknowledged via signature.
        \item Incorporate AUP training into the new employee onboarding process.
    \end{itemize}
\end{enumerate}

\end{document}
```