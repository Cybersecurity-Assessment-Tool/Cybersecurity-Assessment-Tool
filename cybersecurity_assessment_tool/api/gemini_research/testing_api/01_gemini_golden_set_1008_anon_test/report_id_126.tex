```latex
\documentclass[12pt]{article}

% Preamble: Required Packages and Document Setup
\usepackage[margin=1in]{geometry}
\usepackage{pifont} % Required for checkmarks (\ding{51}) and crosses (\ding{55})
\usepackage{booktabs} % For professional-looking tables
\usepackage{xcolor} % For custom colors
\usepackage{graphicx}
\usepackage{url}
\usepackage{seqsplit} % To handle long, unbreakable strings like hashes or URLs
\usepackage[hidelinks]{hyperref} % For clickable links without distracting boxes

% Define custom colors for the report
\definecolor{darkblue}{rgb}{0.0, 0.0, 0.55}
\definecolor{darkred}{rgb}{0.55, 0.0, 0.0}

% Setup hyperref for clean links
\hypersetup{
    colorlinks=true,
    linkcolor=darkblue,
    filecolor=darkblue,
    urlcolor=darkblue,
    citecolor=darkblue,
}

% --- Document Start ---
\begin{document}

\title{Cybersecurity Posture Assessment Report}
\author{Cybersecurity Analysis Division}
\date{\today}
\maketitle

\begin{abstract}
    This report provides a comprehensive analysis of the cybersecurity posture for \textbf{[Organization Name]}. The assessment is based on the synthesis of a technical network scan, a review of organizational security controls via a questionnaire, and an evaluation of previously identified risks. The analysis reveals critical gaps in identity and access management, specifically the lack of multi-factor authentication (MFA) for email and computer access. Furthermore, significant procedural deficiencies were noted, including the absence of a formal employee acceptable use policy and a mandatory annual security awareness training program. On a positive note, a technical scan confirmed that a previously identified risk involving an unencrypted web server (Port 80) appears to have been remediated. This report outlines these findings in detail and provides prioritized, actionable recommendations to mitigate the identified risks and strengthen the overall security posture.
\end{abstract}

\newpage

% Section 1: Organizational Information
\section{Organizational Information}
This assessment was conducted for the following entity. The information provided has been anonymized as per the engagement parameters.

\begin{itemize}
    \item \textbf{Organization Name:} \textbf{[Organization Name]}
    \item \textbf{Primary Email Domain:} \texttt{[Domain]}
    \item \textbf{External IP Scanned:} \texttt{[Client IP]}
\end{itemize}

% Section 2: Security Control Review (from Questionnaire)
\section{Security Control Review}
The following table summarizes the organization's responses to a security controls questionnaire. The status indicates alignment with common cybersecurity best practices. A checkmark (\ding{51}) indicates an affirmative response, while a cross (\ding{55}) indicates a negative response that represents a potential security gap.

\begin{table}[h!]
\centering
\caption{Organizational Security Controls Questionnaire Results}
\label{tab:controls}
\begin{tabular}{@{}lcc@{}}
\toprule
\textbf{Control Question} & \textbf{Response} & \textbf{Status} \\
\midrule
Do you require MFA to access email? & No & \textcolor{darkred}{\ding{55}} \\
Do you require MFA to log into computers? & No & \textcolor{darkred}{\ding{55}} \\
Do you require MFA to access sensitive data systems? & Yes & \textcolor{darkgreen}{\ding{51}} \\
Does your organization have an employee acceptable use policy? & No & \textcolor{darkred}{\ding{55}} \\
Does your organization do security awareness training for new employees? & Yes & \textcolor{darkgreen}{\ding{51}} \\
Does your organization do security awareness training for all employees at least once per year? & No & \textcolor{darkred}{\ding{55}} \\
\bottomrule
\end{tabular}
\end{table}

\subsection*{Analysis}
The questionnaire results highlight several critical and high-risk gaps in the organization's security program. The absence of MFA for primary communication (email) and endpoint access (computers) presents a significant risk of account compromise and unauthorized access. Additionally, the lack of a formal acceptable use policy and mandatory annual security training for all staff weakens the organization's human firewall, making it more susceptible to social engineering and insider threats.

% Section 3: Technical Scan Results
\section{Technical Scan Results}
An external network scan was performed to identify open ports and exposed services on the organization's perimeter.

\begin{itemize}
    \item \textbf{Target IP Address:} \texttt{[Target IP]}
    \item \textbf{Scan Utility:} Nmap
\end{itemize}

The scan revealed the following port status:

\begin{table}[h!]
\centering
\caption{Network Port Scan Findings}
\label{tab:scan}
\begin{tabular}{@{}ccccc@{}}
\toprule
\textbf{Port} & \textbf{Protocol} & \textbf{State} & \textbf{Service} & \textbf{Notes} \\
\midrule
80 & TCP & closed & http & The port for unencrypted web traffic is not open. \\
\bottomrule
\end{tabular}
\end{table}

\subsection*{Analysis}
The technical scan indicates a positive security posture for the scanned target. The finding that port 80 (HTTP) is \textbf{closed} is significant. It prevents unencrypted communication with a web server, which is a common vector for man-in-the-middle attacks. This technical finding directly contradicts a previously documented risk, suggesting that remediation action has been taken.

% Section 4: Correlated Risk Assessment
\section{Correlated Risk Assessment}
This section synthesizes findings from the security control review, the technical scan, and pre-existing risk documentation to provide a holistic view of the current risk landscape.

\begin{table}[h!]
\centering
\caption{Summary of Identified Risks}
\label{tab:risks}
\begin{tabular}{@{}p{0.3\linewidth}p{0.4\linewidth}ll@{}}
\toprule
\textbf{Risk Title} & \textbf{Description} & \textbf{Severity} & \textbf{Status} \\
\midrule
\textbf{Lack of MFA on Critical Systems} & Email and computer logins are protected only by passwords, making them highly vulnerable to phishing and brute-force attacks. & \textbf{Critical} & Active \\
\addlinespace
\textbf{No Annual Security Training} & Without regular training, employees are more likely to fall victim to evolving cyber threats, such as sophisticated phishing attacks. & High & Active \\
\addlinespace
\textbf{No Acceptable Use Policy (AUP)} & The absence of a formal AUP creates ambiguity regarding safe technology use and lacks a basis for enforcing security standards. & High & Active \\
\addlinespace
\textbf{Unencrypted Web Server} & Port 80 was previously documented as open, exposing the organization to unencrypted data transmission. & Medium & \textbf{Mitigated} \\
\bottomrule
\end{tabular}
\end{table}

\subsection*{Risk Correlation Note}
The risk titled "Unencrypted Web Server," sourced from Input 3, is now marked as \textbf{Mitigated}. The technical scan from Input 1 provided definitive evidence that port 80 is closed, resolving the previously identified vulnerability. All other risks, derived from the questionnaire in Input 2, are considered active and require immediate attention.

% Section 5: Recommendations
\section{Recommendations}
Based on the correlated risk assessment, the following prioritized actions are recommended to enhance the organization's cybersecurity posture.

\subsection*{Priority 1: Implement Comprehensive MFA (Critical)}
\begin{itemize}
    \item \textbf{Action:} Immediately enable and enforce Multi-Factor Authentication (MFA) for all user accounts across all critical platforms.
    \item \textbf{Details:} Prioritize the email system (e.g., Office 365, Google Workspace) and all computer/endpoint logins (e.g., Windows Hello for Business, Duo Security). This is the single most effective control to prevent unauthorized account access.
\end{itemize}

\subsection*{Priority 2: Establish Foundational Security Policies (High)}
\begin{itemize}
    \item \textbf{Action:} Develop, approve, and disseminate a formal Employee Acceptable Use Policy (AUP).
    \item \textbf{Details:} The AUP should clearly define the rules for using company technology, data handling responsibilities, and the consequences of non-compliance. All employees must read and acknowledge the policy.
\end{itemize}

\subsection*{Priority 3: Mandate Annual Security Awareness Training (High)}
\begin{itemize}
    \item \textbf{Action:} Institute a mandatory security awareness training program for all employees, to be completed annually.
    \item \textbf{Details:} The training should cover current threats such as phishing, ransomware, and social engineering. This complements the existing new-hire training and ensures that security remains a top-of-mind concern for all staff.
\end{itemize}

\subsection*{Priority 4: Update Risk Register (Informational)}
\begin{itemize}
    \item \textbf{Action:} Formally update the internal risk register to reflect that the vulnerability associated with the open Port 80 has been mitigated.
    \item \textbf{Details:} Verify that the service previously running on port 80 has been decommissioned or properly secured and is no longer required. This ensures accurate risk tracking.
\end{itemize}

\end{document}
```