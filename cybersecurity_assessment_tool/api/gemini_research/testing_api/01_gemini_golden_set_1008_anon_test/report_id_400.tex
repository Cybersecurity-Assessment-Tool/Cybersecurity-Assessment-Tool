```latex
\documentclass[12pt]{article}

% Preamble: Required Packages and Document Setup
\usepackage[margin=1in]{geometry}
\usepackage{pifont} % For checkmarks and crosses
\usepackage{booktabs} % For professional-looking tables
\usepackage{hyperref} % For clickable links and better PDF navigation
\usepackage{url} % For formatting URLs
\usepackage{seqsplit} % For splitting long strings in texttt
\usepackage{graphicx} % For potential logos
\usepackage{xcolor} % For colors in text

% Hyperref setup
\hypersetup{
    colorlinks=true,
    linkcolor=blue,
    filecolor=magenta,      
    urlcolor=cyan,
    pdftitle={Cybersecurity Posture Assessment Report},
    pdfauthor={Cybersecurity Analyst},
    pdfsubject={Security Assessment},
    pdfkeywords={Security, Report, Analysis},
    bookmarks=true
}

% Custom commands for consistency
\newcommand{\yes}{\ding{51}}
\newcommand{\no}{\ding{55}}
\newcommand{\orgname}{\textbf{[Organization Name]}}
\newcommand{\orgdomain}{\texttt{[Domain]}}
\newcommand{\orgip}{\texttt{[Client IP]}}
\newcommand{\targetip}{\texttt{[Target IP]}}

\begin{document}

% --- Title Page ---
\begin{titlepage}
    \centering
    \vspace*{1cm}
    
    \Huge
    \textbf{Cybersecurity Posture Assessment Report}
    
    \vspace{1.5cm}
    
    \Large
    Prepared for: \orgname
    
    \vspace{2cm}
    
    \includegraphics[width=0.4\textwidth]{example-image-a} % Placeholder for a logo
    
    \vfill
    
    \Large
    \textbf{Date of Report:} \today \\
    \textbf{Author:} Cybersecurity Analyst
    
\end{titlepage}

\tableofcontents
\newpage

% --- 1. Executive Summary ---
\section{Executive Summary}

This report details the findings of a cybersecurity posture assessment conducted for \orgname. The assessment combined an external network scan, a review of existing risks, and an analysis of organizational security controls based on a questionnaire.

The key finding is a significant disparity between the organization's technical and procedural security postures. The external network scan of the target system, \targetip, did not reveal any open ports, suggesting a hardened external perimeter at the time of the scan. This is a positive technical finding.

However, the analysis of the security questionnaire reveals \textbf{critical gaps} in fundamental security controls. The absence of Multi-Factor Authentication (MFA) for computer and sensitive data system access, coupled with a complete lack of a formal security awareness training program and an acceptable use policy, exposes the organization to a high risk of compromise through social engineering, phishing, and insider threats.

Immediate remediation should focus on implementing MFA across all critical systems and establishing a baseline security awareness and policy framework. While the external posture appears strong from this limited scan, the internal weaknesses represent a far more probable and impactful vector for a security breach.

% --- 2. Organizational Information ---
\section{Organizational Information}

This section provides the high-level details for the organization under review. The information has been anonymized as per the engagement requirements.

\begin{tabular}{@{}ll}
    \toprule
    \textbf{Attribute} & \textbf{Value} \\
    \midrule
    Organization Name & \orgname \\
    Primary Email Domain & \orgdomain \\
    External IP Address & \orgip \\
    \bottomrule
\end{tabular}

% --- 3. Security Control Review ---
\section{Security Control Review}

The following table summarizes the organization's responses to a security controls questionnaire. Each response is assessed against industry best practices. "No" answers indicate significant control gaps that increase organizational risk.

\begin{table}[h!]
\centering
\caption{Security Controls Questionnaire Analysis}
\begin{tabular}{@{}p{0.6\linewidth} c p{0.25\linewidth}@{}}
    \toprule
    \textbf{Control Question} & \textbf{Response} & \textbf{Assessment} \\
    \midrule
    Do you require MFA to access email? & \yes & Best Practice Met \\
    \addlinespace
    Do you require MFA to log into computers? & \no & \textcolor{red}{\textbf{Critical Gap}} \\
    \addlinespace
    Do you require MFA to access sensitive data systems? & \no & \textcolor{red}{\textbf{Critical Gap}} \\
    \addlinespace
    Does your organization have an employee acceptable use policy? & \no & \textcolor{orange}{High Risk} \\
    \addlinespace
    Does your organization do security awareness training for new employees? & \no & \textcolor{orange}{High Risk} \\
    \addlinespace
    Does your organization do security awareness training for all employees at least once per year? & \no & \textcolor{orange}{High Risk} \\
    \bottomrule
\end{tabular}
\end{table}

% --- 4. Technical Scan Results ---
\section{Technical Scan Results}

An external network port scan was performed to identify accessible services and potential vulnerabilities on the organization's public-facing infrastructure.

\begin{itemize}
    \item \textbf{Target IP Address:} \targetip
    \item \textbf{Scan Date:} \textbf{[Scan Date]}
    \item \textbf{Scanner Used:} Nmap
\end{itemize}

\subsection{Scan Summary}
The scan confirmed that the target host was online and responsive. However, no open TCP or UDP ports were discovered within the scanned range. All tested ports were reported as being in a \textbf{`closed`} state.

\textbf{Conclusion:} From the perspective of this external scan, the target system presents a hardened perimeter with no exposed services. This significantly reduces the external attack surface. No immediate technical vulnerabilities were identified.

% --- 5. Risk Assessment ---
\section{Risk Assessment}

This section synthesizes findings from all data sources into a consolidated list of identified risks. The risks are primarily derived from the procedural and policy gaps identified in the Security Control Review, as no pre-existing risks were reported and no technical vulnerabilities were found.

\begin{table}[h!]
\centering
\caption{Consolidated Risk Register}
\begin{tabular}{@{}p{0.25\linewidth} p{0.5\linewidth} l@{}}
    \toprule
    \textbf{Risk / Vulnerability} & \textbf{Description} & \textbf{Severity} \\
    \midrule
    Lack of MFA on Endpoints and Systems & The absence of MFA on computer logins and sensitive systems allows an attacker with stolen credentials to gain immediate, unauthorized access. & \textcolor{red}{\textbf{Critical}} \\
    \addlinespace
    Absence of Security Awareness Training & Employees are not trained to recognize or respond to phishing, social engineering, or other common cyber threats, making them highly susceptible to attacks. & \textcolor{orange}{\textbf{High}} \\
    \addlinespace
    Missing Acceptable Use Policy (AUP) & Without a formal AUP, there are no clear guidelines for employees on the proper use of company assets, data handling, or security responsibilities, leading to inconsistent and insecure practices. & \textcolor{orange}{\textbf{High}} \\
    \bottomrule
\end{tabular}
\end{table}

% --- 6. Recommendations ---
\section{Recommendations}

Based on the risk assessment, the following prioritized actions are recommended to mitigate the identified vulnerabilities and improve the overall security posture of \orgname.

\subsection{Priority 1: Implement Comprehensive MFA (Critical)}
\begin{itemize}
    \item \textbf{Action:} Immediately deploy and enforce Multi-Factor Authentication (MFA) for all user accounts for logging into company computers (desktops and laptops).
    \item \textbf{Action:} Enforce MFA for access to all systems containing sensitive or critical business data, including internal applications, databases, and cloud services.
    \item \textbf{Justification:} This is the single most effective control to prevent unauthorized access resulting from compromised credentials.
\end{itemize}

\subsection{Priority 2: Establish a Security Awareness Program (High)}
\begin{itemize}
    \item \textbf{Action:} Develop a mandatory security awareness training module for all new employees as part of their onboarding process.
    \item \textbf{Action:} Implement an annual, mandatory security awareness refresher course for all existing employees. Topics should include phishing identification, password security, and incident reporting.
    \item \textbf{Justification:} A well-trained workforce is the first line of defense against the most common cyber attacks.
\end{itemize}

\subsection{Priority 3: Develop and Enforce an Acceptable Use Policy (High)}
\begin{itemize}
    \item \textbf{Action:} Draft a comprehensive Acceptable Use Policy (AUP) that clearly defines the rules and responsibilities for using company technology, networks, and data.
    \item \textbf{Action:} Require all employees to read and formally acknowledge the AUP. Integrate this policy into the employee handbook and onboarding process.
    \item \textbf{Justification:} An AUP establishes a clear security baseline for employee behavior and provides a framework for enforcing security standards.
\end{itemize}

\subsection{Priority 4: Enhance Technical Scanning (Medium)}
\begin{itemize}
    \item \textbf{Action:} Schedule regular, authenticated internal and external vulnerability scans to gain a more comprehensive view of the technical security posture.
    \item \textbf{Justification:} While the initial external scan was positive, it represents only a snapshot. Continuous and more in-depth scanning is required to identify vulnerabilities that may arise from system changes or new threats.
\end{itemize}

\end{document}
```