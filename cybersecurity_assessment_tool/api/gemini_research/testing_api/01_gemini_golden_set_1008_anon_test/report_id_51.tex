```latex
\documentclass[12pt]{article}

% --- PACKAGES ---
\usepackage[margin=1in]{geometry}
\usepackage{pifont}          % For symbols like checkmarks
\usepackage{booktabs}        % For professional-looking tables
\usepackage{hyperref}        % For clickable links and references
\usepackage{url}             % For formatting URLs
\usepackage{seqsplit}        % For splitting long strings in tt font
\usepackage{xcolor}          % For coloring text

% --- DOCUMENT SETUP ---
\hypersetup{
    colorlinks=true,
    linkcolor=blue,
    urlcolor=cyan,
}

% --- CUSTOM COMMANDS ---
\newcommand{\yes}{\textcolor{green}{\ding{51}}} % Green checkmark
\newcommand{\no}{\textcolor{red}{\ding{55}}}    % Red X

% --- DOCUMENT START ---
\begin{document}

% --- TITLE PAGE ---
\title{Cybersecurity Posture Assessment Report}
\author{Cybersecurity Analysis Division}
\date{\today}
\maketitle
\thispagestyle{empty}
\newpage

\tableofcontents
\newpage

% --- EXECUTIVE SUMMARY ---
\section{Executive Summary}
This report provides a cybersecurity posture assessment for \textbf{[Organization Name]}, based on an analysis of technical network scans, a security controls questionnaire, and pre-existing risk data.

The assessment reveals a \textbf{critical risk posture} requiring immediate attention. The primary finding is a publicly accessible MySQL database server (\texttt{[Target IP]}:3306) running an \textbf{End-of-Life (EOL) version (MySQL 5.7.33)}. EOL software no longer receives security updates, leaving it perpetually exposed to known and future vulnerabilities.

This critical technical vulnerability is compounded by significant deficiencies in fundamental administrative controls. The organization lacks a formal Acceptable Use Policy (AUP) and does not conduct any security awareness training for its employees. This combination of an exposed, unpatchable system and an untrained workforce creates a high-probability, high-impact scenario for a potential data breach.

While the implementation of Multi-Factor Authentication (MFA) is a commendable strength, it is not sufficient to protect against these foundational weaknesses. Immediate action is required to restrict access to the exposed database and address the policy and training gaps.

% --- ORGANIZATIONAL INFORMATION ---
\section{Organizational Information}
This section details the information provided about the organization. Placeholders are used where data was not available.

\begin{center}
\begin{tabular}{@{}ll}
\toprule
\textbf{Attribute} & \textbf{Value} \\
\midrule
Organization Name & \textbf{[Organization Name]} \\
Primary Domain & \texttt{[Domain]} \\
Client External IP & \texttt{[Client IP]} \\
Scanned Target IP & \texttt{[Target IP]} \\
\bottomrule
\end{tabular}
\end{center}

% --- SECURITY CONTROL REVIEW ---
\section{Security Control Review}
This section evaluates the organization's administrative and operational security controls based on a self-reported questionnaire. Gaps identified here often indicate systemic risks that can undermine technical defenses.

\begin{center}
\begin{tabular}{p{0.7\textwidth} c}
\toprule
\textbf{Control Question} & \textbf{Status} \\
\midrule
Do you require MFA to access email? & \yes \\
Do you require MFA to log into computers? & \yes \\
Do you require MFA to access sensitive data systems? & \yes \\
\addlinespace
Does your organization have an employee acceptable use policy? & \no \\
Does your organization do security awareness training for new employees? & \no \\
Does your organization do security awareness training for all employees at least once per year? & \no \\
\bottomrule
\end{tabular}
\end{center}

\subsection*{Analysis}
The organization has successfully implemented strong Multi-Factor Authentication (MFA) controls across key systems, which significantly strengthens identity and access management. However, there are critical deficiencies in foundational administrative controls. The complete absence of an Acceptable Use Policy (AUP) and a security awareness training program creates a significant risk. Employees are likely unaware of their security responsibilities, making them more susceptible to social engineering, phishing, and accidental data breaches.

% --- TECHNICAL SCAN RESULTS ---
\section{Technical Scan Results}
A network scan was performed on the target system to identify open ports and exposed services.

\subsection*{Scan Details}
\begin{itemize}
    \item \textbf{Target IP:} \texttt{[Target IP]}
    \item \textbf{Scan Date:} Not specified in scan data.
\end{itemize}

\subsection*{Open Ports Discovered}
The following table details the services found to be accessible from the network.

\begin{center}
\begin{tabular}{lllll}
\toprule
\textbf{Port} & \textbf{State} & \textbf{Service} & \textbf{Product} & \textbf{Version} \\
\midrule
3306/tcp & open & mysql & MySQL & 5.7.33 \\
\bottomrule
\end{tabular}
\end{center}

\subsection*{Analysis}
The scan identified a critical exposure: port 3306 is open, indicating a publicly accessible MySQL database. The running version, \textbf{MySQL 5.7.33}, is particularly alarming as the entire MySQL 5.7 branch reached its \textbf{End of Life (EOL) in October 2023}. EOL software no longer receives security updates, patches, or support from the vendor, leaving it perpetually vulnerable to newly discovered exploits. This exposure represents a direct and severe threat to the confidentiality, integrity, and availability of the data stored within the database. This finding directly corroborates the pre-existing risk titled "Database Exposure".

% --- RISK ASSESSMENT ---
\section{Consolidated Risk Assessment}
The following table summarizes the key risks identified by correlating organizational data, technical scans, and pre-existing risk information.

\begin{center}
\begin{tabular}{p{0.25\textwidth} p{0.5\textwidth} p{0.15\textwidth}}
\toprule
\textbf{Risk Name} & \textbf{Description} & \textbf{Severity} \\
\midrule
\textbf{Exposed End-of-Life Database} & The MySQL database (v5.7.33) is publicly accessible on port 3306 and is past its End-of-Life. It is unpatched against any vulnerabilities discovered since October 2023. This confirms the pre-existing risk "Database Exposure" with a CVSS score of 7.5 (High). & \textbf{Critical} \\
\addlinespace
\textbf{Lack of Security Awareness Program} & The organization does not provide security training to new or existing employees. This increases the likelihood of human error, such as falling for phishing attacks, which could lead to credential compromise and a breach of the exposed database. & \textbf{High} \\
\addlinespace
\textbf{Absence of Acceptable Use Policy} & Without a formal AUP, there are no established rules for employee behavior regarding company assets and data. This can lead to insecure practices and complicates enforcement of security standards. & \textbf{High} \\
\bottomrule
\end{tabular}
\end{center}

% --- RECOMMENDATIONS ---
\section{Recommendations}
Based on the findings, the following actions are recommended to mitigate the identified risks. Recommendations are prioritized by urgency.

\subsection*{Immediate (0-7 Days)}
\begin{enumerate}
    \item \textbf{Restrict Database Access:} Immediately implement firewall rules to restrict access to TCP port 3306 on host \texttt{[Target IP]}. Access should only be permitted from trusted, internal IP addresses or via a secure VPN. Public access must be blocked entirely.
\end{enumerate}

\subsection*{Short-Term (1-3 Months)}
\begin{enumerate}
    \setcounter{enumi}{1} % Continue numbering
    \item \textbf{Upgrade End-of-Life Software:} Plan and execute the migration of the MySQL 5.7 database to a currently supported version (e.g., MySQL 8.x). This is critical to ensure the system can receive security patches against future vulnerabilities.
\end{enumerate}

\subsection*{Mid-Term (3-6 Months)}
\begin{enumerate}
    \setcounter{enumi}{2} % Continue numbering
    \item \textbf{Implement Security Awareness Training:} Develop and deploy a mandatory security awareness training program. This must include an initial module for all new hires and an annual refresher course for all staff. Topics should cover phishing, password hygiene, and data handling responsibilities.
    \item \textbf{Establish an Acceptable Use Policy (AUP):} Draft, approve, and disseminate an AUP that clearly defines the rules for using company IT assets, networks, and data. All employees should be required to read and acknowledge the policy as a condition of use.
\end{enumerate}

\end{document}
```