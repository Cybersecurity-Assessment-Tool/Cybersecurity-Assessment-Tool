```latex
\documentclass[12pt]{article}

% Preamble: Required Packages
\usepackage[margin=1in]{geometry}
\usepackage{pifont} % For checkmarks (\ding{51}) and crosses (\ding{55})
\usepackage{booktabs} % For professional-looking tables
\usepackage{hyperref} % For clickable links and references
\usepackage{url}
\usepackage{seqsplit} % To split long strings in \texttt
\usepackage{graphicx}
\usepackage{xcolor}
\usepackage{array}

% Define colors for risk levels
\definecolor{critical}{HTML}{990000}
\definecolor{high}{HTML}{D14343}
\definecolor{medium}{HTML}{EFAF43}
\definecolor{low}{HTML}{8FBC8F}

% Hyperref setup
\hypersetup{
    colorlinks=true,
    linkcolor=blue,
    filecolor=magenta,      
    urlcolor=cyan,
    pdftitle={Cybersecurity Posture Assessment Report},
    pdfpagemode=FullScreen,
}

% Document Title
\title{Cybersecurity Posture Assessment Report}
\author{Cybersecurity Analysis Division}
\date{\today}

\begin{document}

\maketitle
\tableofcontents
\newpage

% --- 1. Executive Summary ---
\section{Executive Summary}

This report provides a comprehensive cybersecurity assessment for \textbf{[Organization Name]}, based on an analysis of network scan data, organizational security controls, and pre-existing risk documentation. The assessment synthesizes technical findings with procedural and policy-based controls to provide a holistic view of the organization's security posture.

The analysis revealed several areas of significant concern requiring immediate attention. Key findings include:
\begin{itemize}
    \item \textbf{Critical External Exposure:} A MySQL database (Port 3306) is directly exposed to the internet. The running version, MySQL 5.7.33, is an End-of-Life (EOL) product and no longer receives security updates, posing a critical risk of compromise.
    \item \textbf{Critical Access Control Gaps:} Multi-Factor Authentication (MFA) is not enforced for workstation logins, leaving a primary vector for lateral movement and privilege escalation vulnerable to credential-based attacks.
    \item \textbf{High-Risk Gaps in Security Culture:} While new employees receive security training, there is no mandatory annual training for all staff. This gap allows for security knowledge to decay, increasing susceptibility to social engineering attacks like phishing.
\end{itemize}

The overall security posture is rated as \textbf{High Risk}. The combination of an exposed, outdated database and weak internal access controls creates a significant likelihood of a security incident with a high potential impact. This report outlines specific, actionable recommendations to mitigate these identified risks.

% --- 2. Organizational Information ---
\section{Organizational Information}
This section details the organizational context for this assessment. The information provided is based on the data supplied for the analysis.

\begin{tabular}{@{}ll}
\toprule
\textbf{Attribute} & \textbf{Value} \\
\midrule
Organization Name & \textbf{[Organization Name]} \\
Primary Email Domain & \texttt{[Domain]} \\
External IP Scanned & \texttt{[Client IP]} \\
\bottomrule
\end{tabular}

% --- 3. Security Control Review (Questionnaire) ---
\section{Security Control Review}
The following table summarizes the organization's self-reported security controls based on the provided questionnaire. A checkmark (\ding{51}) indicates a positive control is in place, while a cross (\ding{55}) indicates a control gap.

\begin{table}[h!]
\centering
\begin{tabular}{>{\raggedright\arraybackslash}p{10cm} c}
\toprule
\textbf{Control Question} & \textbf{Response} \\
\midrule
Do you require MFA to access email? & \textcolor{green}{\ding{51}} \\
Do you require MFA to log into computers? & \textcolor{red}{\ding{55}} \\
Do you require MFA to access sensitive data systems? & \textcolor{green}{\ding{51}} \\
Does your organization have an employee acceptable use policy? & \textcolor{green}{\ding{51}} \\
Does your organization do security awareness training for new employees? & \textcolor{green}{\ding{51}} \\
Does your organization do security awareness training for all employees at least once per year? & \textcolor{red}{\ding{55}} \\
\bottomrule
\end{tabular}
\caption{Organizational Security Control Status}
\end{table}

\subsection*{Analysis of Control Gaps}
Two critical control gaps were identified from the questionnaire:
\begin{itemize}
    \item \textbf{No MFA for Computer Logins:} This is a significant weakness. If an employee's credentials are stolen, an attacker can gain direct access to the internal network, bypassing other perimeter controls.
    \item \textbf{No Annual Security Awareness Training:} The threat landscape evolves continuously. Without regular, recurring training, employees are more likely to fall victim to modern phishing and social engineering tactics.
\end{itemize}

% --- 4. Technical Scan Results ---
\section{Technical Scan Results}
An external network scan was performed to identify open ports and exposed services. The findings below are correlated with the pre-existing risk data and confirm the public exposure of a critical asset.

\begin{itemize}
    \item \textbf{Target IP Address:} \texttt{[Target IP]}
    \item \textbf{Scan Status:} Host is UP
\end{itemize}

\begin{table}[h!]
\centering
\begin{tabular}{llllll}
\toprule
\textbf{Port} & \textbf{State} & \textbf{Service} & \textbf{Product} & \textbf{Version} & \textbf{Analyst Notes} \\
\midrule
3306/tcp & open & mysql & MySQL & 5.7.33 & \textcolor{critical}{\textbf{CRITICAL: EOL Software}} \\
\bottomrule
\end{tabular}
\caption{Open Ports and Services Detected}
\end{table}

\subsection*{Analysis of Technical Findings}
The scan confirms that a MySQL database on port 3306 is open to the public internet. This configuration is highly discouraged and directly validates the "Database Exposure" risk identified in Input 3.

Furthermore, the detected version, \textbf{MySQL 5.7.33}, reached its official End-of-Life (EOL) in \textbf{October 2023}. EOL software no longer receives security patches from the vendor, meaning any newly discovered vulnerabilities will remain unpatched. This elevates the risk of this exposure from High to Critical.

% --- 5. Consolidated Risk Assessment ---
\section{Consolidated Risk Assessment}
This section synthesizes findings from the security questionnaire, technical scan, and pre-existing risk data into a consolidated list of key risks facing the organization.

\begin{table}[h!]
\centering
\begin{tabular}{p{2.5cm} p{3.5cm} p{7cm}}
\toprule
\textbf{Risk ID} & \textbf{Risk Title} & \textbf{Description} \\
\midrule
\textbf{RISK-001} & \textcolor{critical}{Exposed End-of-Life Database} & A MySQL 5.7.33 database (port 3306) is open to the internet. The software is End-of-Life and unpatched, making it a prime target for automated attacks. This confirms and elevates the pre-existing "Database Exposure" risk. \\
\addlinespace
\textbf{RISK-002} & \textcolor{high}{Lack of Workstation MFA} & The absence of MFA for computer logins exposes the organization to significant risk from compromised credentials, enabling unauthorized access and lateral movement within the network. \\
\addlinespace
\textbf{RISK-003} & \textcolor{high}{Inadequate Security Awareness Program} & The lack of mandatory annual security training for all employees weakens the human firewall, increasing the likelihood of successful phishing, malware infection, and other social engineering attacks. \\
\bottomrule
\end{tabular}
\caption{Summary of Identified Risks}
\end{table}

% --- 6. Recommendations ---
\section{Recommendations}
The following actionable recommendations are provided to mitigate the identified risks. They are prioritized based on severity and potential impact.

\subsection{RISK-001: Remediate Exposed Database (Priority: Immediate)}
\begin{itemize}
    \item \textbf{Short-Term Containment:} Immediately implement strict firewall rules to restrict all access to port 3306 from the public internet. Access should only be permitted from trusted, internal IP addresses.
    \item \textbf{Long-Term Remediation:}
        \begin{enumerate}
            \item Plan and execute an urgent upgrade of the MySQL 5.7.33 instance to a currently supported version (e.g., MySQL 8.x).
            \item Implement a Virtual Private Network (VPN) or other secure access solution (e.g., Zero Trust Network Access) for all remote database administration. Direct public exposure should never be the default configuration.
        \end{enumerate}
\end{itemize}

\subsection{RISK-002: Implement Workstation MFA (Priority: High)}
\begin{itemize}
    \item \textbf{Action:} Procure and deploy an MFA solution for all employee workstations (Windows, macOS, etc.).
    \item \textbf{Implementation:} Phase the rollout, starting with privileged users (administrators, executives) and expanding to the entire organization within the next quarter.
    \item \textbf{Justification:} This single control dramatically reduces the risk of account takeover and unauthorized internal access.
\end{itemize}

\subsection{RISK-003: Enhance Security Awareness Program (Priority: High)}
\begin{itemize}
    \item \textbf{Action:} Establish a formal, mandatory security awareness training program for all employees to be completed annually.
    \item \textbf{Content:} The training should cover modern threats, including phishing, ransomware, password hygiene, and acceptable use policies.
    \item \textbf{Reinforcement:} Supplement annual training with regular phishing simulations to measure effectiveness and provide targeted coaching to employees who need it.
\end{itemize}

\end{document}
```