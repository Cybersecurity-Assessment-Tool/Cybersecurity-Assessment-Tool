```latex
\documentclass[12pt]{article}

% Required Packages
\usepackage[margin=1in]{geometry}
\usepackage{pifont} % For checkmarks and crosses
\usepackage{booktabs} % For professional tables
\usepackage{hyperref} % For hyperlinks
\usepackage{url} % For URL formatting
\usepackage{seqsplit} % To split long strings without breaking
\usepackage{graphicx}
\usepackage{xcolor}

% Hyperref Setup
\hypersetup{
    colorlinks=true,
    linkcolor=blue,
    filecolor=magenta,      
    urlcolor=cyan,
    pdftitle={Cybersecurity Assessment Report},
    pdfpagemode=FullScreen,
}

% Define check and cross marks for convenience
\newcommand{\cmark}{\ding{51}}%
\newcommand{\xmark}{\ding{55}}%

\begin{document}

% --- Title Page ---
\title{
    \vspace{2cm}
    \textbf{Cybersecurity Assessment Report} \\
    \large For \textbf{[Organization Name]}
    \vspace{1.5cm}
}
\author{Cybersecurity Analyst}
\date{\today}
\maketitle
\newpage

% --- Table of Contents ---
\tableofcontents
\newpage

% --- Section 1: Executive Summary ---
\section{Executive Summary}

This report presents a cybersecurity assessment for \textbf{[Organization Name]}, conducted on \today. The analysis is based on a review of organizational security controls, an external network scan, and a list of pre-existing risks.

The assessment identified two significant areas of concern requiring immediate attention. The most critical finding is the absence of Multi-Factor Authentication (MFA) for email access, which exposes the organization to a high risk of Business Email Compromise (BEC) and unauthorized account access. Additionally, the lack of mandatory annual security awareness training for all employees constitutes a high risk, as it diminishes the organization's resilience against phishing and social engineering attacks.

On a positive note, the organization has implemented foundational controls, including MFA for computer and sensitive system access, an acceptable use policy, and security training for new hires. The external network scan of the target host \texttt{[Target IP]} did not reveal any open ports, suggesting a properly configured firewall or that the host was unresponsive at the time of the scan. No pre-existing vulnerabilities were provided for this assessment.

Recommendations are detailed in Section \ref{sec:recommendations} and are prioritized to address the most critical risks first.

% --- Section 2: Organizational Information ---
\section{Organizational Information}

The following information was used as the basis for this assessment. Due to the anonymized nature of the provided data, placeholders have been used where necessary.

\begin{table}[h!]
\centering
\begin{tabular}{@{}ll@{}}
\toprule
\textbf{Attribute} & \textbf{Value} \\ \midrule
Organization Name & \textbf{[Organization Name]} \\
Primary Domain & \texttt{[Domain]} \\
External IP Address Assessed & \texttt{[Client IP]} \\ \bottomrule
\end{tabular}
\caption{Client Organizational Details}
\label{tab:org_info}
\end{table}

% --- Section 3: Security Control Review ---
\section{Security Control Review}

A review of organizational security controls was conducted based on a standard questionnaire. The responses indicate key strengths and weaknesses in the current security posture. Gaps identified with a `No` (\xmark) response are correlated with findings in the Risk Assessment section.

\begin{table}[h!]
\centering
\begin{tabular}{@{}p{0.8\textwidth}c@{}}
\toprule
\textbf{Control Question} & \textbf{Response} \\ \midrule
Do you require MFA to access email? & \textcolor{red}{\xmark} \\
Do you require MFA to log into computers? & \textcolor{green}{\cmark} \\
Do you require MFA to access sensitive data systems? & \textcolor{green}{\cmark} \\
Does your organization have an employee acceptable use policy? & \textcolor{green}{\cmark} \\
Does your organization do security awareness training for new employees? & \textcolor{green}{\cmark} \\
Does your organization do security awareness training for all employees at least once per year? & \textcolor{red}{\xmark} \\ \bottomrule
\end{tabular}
\caption{Security Controls Questionnaire Results}
\label{tab:controls}
\end{table}

% --- Section 4: Technical Scan Results ---
\section{Technical Scan Results}

An external, unauthenticated network scan was performed against the designated target IP address.

\begin{itemize}
    \item \textbf{Target IP:} \texttt{[Target IP]}
    \item \textbf{Scan Date:} [Scan Date Not Provided]
\end{itemize}

\subsection{Scan Summary}
The scan did not identify any open TCP or UDP ports on the target host. This result typically indicates one of the following:
\begin{itemize}
    \item The presence of a well-configured firewall that is effectively blocking all unsolicited inbound traffic.
    - The host was offline or not responsive to network probes at the time of the scan.
    - An Intrusion Prevention System (IPS) may have blocked the scan traffic.
\end{itemize}

No vulnerabilities could be identified from this external perspective. Further internal, authenticated scanning is recommended for a more comprehensive technical assessment.

% --- Section 5: Consolidated Risk Assessment ---
\section{Consolidated Risk Assessment}

This section synthesizes findings from the security control review, technical scan, and any pre-existing risks. The risks below were identified during this assessment, as no pre-existing vulnerabilities were provided for review.

\begin{table}[h!]
\centering
\begin{tabular}{@{}p{0.1\textwidth}p{0.25\textwidth}p{0.4\textwidth}p{0.1\textwidth}@{}}
\toprule
\textbf{Risk ID} & \textbf{Risk Name} & \textbf{Description} & \textbf{Severity} \\ \midrule
RISK-001 & Lack of MFA on Email & The absence of MFA on email accounts greatly increases the risk of account takeover via credential stuffing or phishing. This can lead to data breaches, financial fraud, and Business Email Compromise (BEC). & \textbf{Critical} \\
\addlinespace
RISK-002 & Inadequate Security Awareness Training & Failing to provide annual security training for all staff leaves the organization vulnerable to social engineering attacks. Employees are less likely to recognize and report phishing attempts, increasing the likelihood of a successful breach. & \textbf{High} \\ \bottomrule
\end{tabular}
\caption{Identified Security Risks}
\label{tab:risks}
\end{table}

% --- Section 6: Recommendations ---
\section{Recommendations}
\label{sec:recommendations}

The following prioritized recommendations are provided to mitigate the identified risks and improve the overall security posture of \textbf{[Organization Name]}.

\subsection{Priority 1: Critical}
\begin{description}
    \item[Remediate RISK-001:] \textbf{Enforce MFA for all Email Accounts.}
    \begin{itemize}
        \item \textbf{Action:} Immediately enable and enforce MFA for all user mailboxes. Prioritize accounts with access to sensitive information or financial systems, such as executives and finance department personnel.
        \item \textbf{Justification:} This is the single most effective control to prevent unauthorized access to email accounts and mitigate the threat of Business Email Compromise (BEC).
    \end{itemize}
\end{description}

\subsection{Priority 2: High}
\begin{description}
    \item[Remediate RISK-002:] \textbf{Implement Annual Security Awareness Training.}
    \begin{itemize}
        \item \textbf{Action:} Establish a mandatory security awareness training program for all employees, to be completed annually. The training should cover key topics such as phishing, password security, and acceptable use policies.
        \item \textbf{Justification:} A well-informed workforce is a critical layer of defense. Regular training reinforces security best practices and reduces the organization's susceptibility to human-targeted attacks.
    \end{itemize}
\end{description}

\subsection{Priority 3: Informational}
\begin{description}
    \item[Enhance Technical Visibility:] \textbf{Conduct Authenticated Vulnerability Scanning.}
    \begin{itemize}
        \item \textbf{Action:} Schedule regular internal and external authenticated vulnerability scans against critical infrastructure. An authenticated scan logs into systems to provide a much deeper and more accurate view of software vulnerabilities and misconfigurations.
        \item \textbf{Justification:} While the external scan was clean, it only provides an outside-in view. Authenticated scans are necessary to identify vulnerabilities that are not exposed to the public internet but could be exploited by an attacker who gains internal access.
    \end{itemize}
\end{description}

\end{document}
```