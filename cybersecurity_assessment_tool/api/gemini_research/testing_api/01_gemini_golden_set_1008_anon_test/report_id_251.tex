```latex
\documentclass[12pt, a4paper]{article}

% Preamble: Required Packages
\usepackage[margin=1in]{geometry}
\usepackage{pifont} % For check and cross marks
\usepackage{booktabs} % For professional tables
\usepackage{hyperref} % For clickable links and references
\usepackage{url} % For URL formatting
\usepackage{seqsplit} % To split long strings without breaking
\usepackage{xcolor} % For colors
\usepackage{graphicx} % For potential logos/images
\usepackage{fancyhdr} % For headers and footers

% --- Document Setup ---
\hypersetup{
    colorlinks=true,
    linkcolor=blue,
    filecolor=magenta,      
    urlcolor=cyan,
    pdftitle={Cybersecurity Assessment Report},
    pdfauthor={Cybersecurity Analyst},
    pdfsubject={Security Assessment},
    pdfkeywords={Security, Analysis, Report},
    bookmarks=true
}

% --- Custom Colors ---
\definecolor{severitycritical}{HTML}{990000}
\definecolor{severityhigh}{HTML}{D14302}
\definecolor{severitymedium}{HTML}{EFA900}
\definecolor{severitylow}{HTML}{3A7A2A}

% --- Header and Footer ---
\pagestyle{fancy}
\fancyhf{} % clear all header and footer fields
\fancyhead[L]{Cybersecurity Assessment Report}
\fancyhead[R]{\textbf{[Organization Name]}}
\fancyfoot[C]{\thepage}
\renewcommand{\headrulewidth}{0.4pt}
\renewcommand{\footrulewidth}{0.4pt}

% --- Document Start ---
\begin{document}

% --- Title Page ---
\begin{titlepage}
    \centering
    \vspace*{1cm}
    
    \Huge
    \textbf{Cybersecurity Assessment Report}
    
    \vspace{1.5cm}
    
    \Large
    Prepared for: \\
    \vspace{0.5cm}
    \textbf{[Organization Name]}
    
    \vspace{2cm}
    
    \large
    Date of Report: \today
    
    \vfill
    
    \large
    \textbf{CONFIDENTIAL}
    
    \vspace{0.5cm}
    
    \normalsize
    This document contains sensitive information. Access is restricted to authorized personnel only. Do not distribute without explicit permission.
    
\end{titlepage}

\tableofcontents
\newpage

% --- Section 1: Executive Overview ---
\section{Executive Overview}
This report details the findings of a cybersecurity assessment conducted for \textbf{[Organization Name]}. The analysis is based on a combination of an external network scan, a review of organizational security controls via a questionnaire, and a list of pre-existing known risks.

The assessment identified several critical and high-risk security gaps that require immediate attention. Key findings include:

\begin{itemize}
    \item \textbf{Inadequate Multi-Factor Authentication (MFA):} The organization does not enforce MFA for accessing email or for logging into employee computers. This represents a critical vulnerability, significantly increasing the risk of account compromise and unauthorized access.
    \item \textbf{Publicly Exposed Administrative Service:} The external network scan revealed that Secure Shell (SSH) on port 22 is open to the public internet on host \texttt{[Target IP]}. This service is a primary target for automated brute-force attacks.
    \item \textbf{Gaps in Security Training:} While new employees receive security awareness training, there is no program for annual refresher training for all staff. This can lead to a degradation of security consciousness over time.
    \item \textbf{Pre-existing Critical Risk:} The organization has a documented critical risk, "Localhost Exposed," with a CVSS score of 10.0. This requires immediate investigation and remediation.
\end{itemize}

The combination of these findings indicates a security posture that is vulnerable to common cyberattacks. This report provides a detailed breakdown of each finding and offers actionable recommendations to mitigate the identified risks and strengthen the organization's overall security posture.

% --- Section 2: Organizational Information ---
\section{Organizational Information}
The following details were used as the basis for this assessment. Due to the anonymized nature of the input data, placeholders have been used where necessary.

\begin{tabular}{@{}ll}
    \toprule
    \textbf{Attribute} & \textbf{Value} \\
    \midrule
    Organization Name & \textbf{[Organization Name]} \\
    Primary Email Domain & \texttt{[Domain]} \\
    External IP Address Scanned & \texttt{[Client IP]} \\
    \bottomrule
\end{tabular}

% --- Section 3: Security Control Review ---
\section{Security Control Review}
The following table summarizes the organization's responses to a security controls questionnaire. A \textcolor{green}{\ding{51}} indicates a positive control is in place, while a \textcolor{red}{\ding{55}} indicates a potential security gap.

\begin{table}[h!]
\centering
\caption{Security Controls Questionnaire Results}
\begin{tabular}{@{}p{0.6\textwidth}cc@{}}
    \toprule
    \textbf{Control Question} & \textbf{Response} & \textbf{Status} \\
    \midrule
    Do you require MFA to access email? & No & \textcolor{red}{\ding{55}} \\
    Do you require MFA to log into computers? & No & \textcolor{red}{\ding{55}} \\
    Do you require MFA to access sensitive data systems? & Yes & \textcolor{green}{\ding{51}} \\
    Does your organization have an employee acceptable use policy? & Yes & \textcolor{green}{\ding{51}} \\
    Does your organization do security awareness training for new employees? & Yes & \textcolor{green}{\ding{51}} \\
    Does your organization do security awareness training for all employees at least once per year? & No & \textcolor{red}{\ding{55}} \\
    \bottomrule
\end{tabular}
\end{table}

\subsection*{Analysis of Control Gaps}
The questionnaire revealed three significant control gaps:
\begin{enumerate}
    \item \textbf{Lack of MFA on Email and Computers:} Email is the primary target for phishing attacks leading to business email compromise (BEC). Lack of MFA on endpoints allows attackers who have stolen credentials to gain direct access to the internal network.
    \item \textbf{Incomplete Security Awareness Training:} Cyber threats evolve constantly. Failing to provide annual refresher training to all employees leaves the organization vulnerable to new social engineering tactics and phishing campaigns.
\end{enumerate}

% --- Section 4: Technical Scan Results ---
\section{Technical Scan Results}
An external network scan was performed using Nmap against the target IP address. The scan identified the following open ports and services.

\begin{itemize}
    \item \textbf{Target IP Address:} \texttt{[Target IP]}
    \item \textbf{Host Status:} Up
\end{itemize}

\begin{table}[h!]
\centering
\caption{Open Ports Identified on \texttt{[Target IP]}}
\begin{tabular}{@{}llll@{}}
    \toprule
    \textbf{Port} & \textbf{State} & \textbf{Service (Inferred)} & \textbf{Notes} \\
    \midrule
    22/tcp & open & SSH (Secure Shell) & Exposing SSH to the public internet is a high-risk configuration. \\
    & & & It is a common target for brute-force and credential stuffing attacks. \\
    \bottomrule
\end{tabular}
\end{table}

% --- Section 5: Consolidated Risk Assessment ---
\section{Consolidated Risk Assessment}
The following table consolidates findings from the questionnaire, technical scan, and pre-existing risk list into a prioritized summary.

\begin{table}[h!]
\centering
\caption{Summary of Identified Risks}
\begin{tabular}{@{}lp{0.45\textwidth}ll@{}}
    \toprule
    \textbf{Risk ID} & \textbf{Description} & \textbf{Source} & \textbf{Severity} \\
    \midrule
    RISK-001 & Pre-existing "Localhost Exposed" vulnerability with a CVSS score of 10.0. & Existing Risks & \textcolor{severitycritical}{\textbf{Critical}} \\
    \addlinespace
    RISK-002 & Lack of MFA for email access, exposing the organization to account takeovers. & Questionnaire & \textcolor{severitycritical}{\textbf{Critical}} \\
    \addlinespace
    RISK-003 & Publicly exposed SSH service on \texttt{[Target IP]}, inviting automated attacks. & Network Scan & \textcolor{severityhigh}{\textbf{High}} \\
    \addlinespace
    RISK-004 & Lack of MFA for computer logins, weakening endpoint security. & Questionnaire & \textcolor{severityhigh}{\textbf{High}} \\
    \addlinespace
    RISK-005 & Security awareness training is not conducted annually for all employees. & Questionnaire & \textcolor{severitymedium}{\textbf{Medium}} \\
    \bottomrule
\end{tabular}
\end{table}

% --- Section 6: Recommendations ---
\section{Recommendations}
The following actions are recommended to mitigate the identified risks and improve the overall security posture of \textbf{[Organization Name]}.

\subsection{RISK-001: Remediate Pre-existing Critical Vulnerability}
\textbf{Action:} Immediately investigate the "Localhost Exposed" risk. The CVSS score of 10.0 indicates a severe vulnerability that could be easily exploited. The technical team must identify the affected assets and apply the necessary patches or configuration changes to remediate this risk without delay.

\subsection{RISK-002 \& RISK-004: Implement Comprehensive MFA}
\textbf{Action:} Mandate the use of Multi-Factor Authentication (MFA) across the organization.
\begin{itemize}
    \item \textbf{Priority 1 (Email):} Enable MFA for all email accounts immediately. This is the single most effective control to prevent business email compromise.
    \item \textbf{Priority 2 (Endpoints):} Deploy MFA for all computer and remote access (VPN) logins. This adds a critical layer of defense against stolen credentials being used to access the internal network.
\end{itemize}

\subsection{RISK-003: Secure the Exposed SSH Service}
\textbf{Action:} Restrict access to the SSH service on \texttt{[Target IP]}.
\begin{itemize}
    \item \textbf{Immediate Fix:} If external SSH access is not required, disable the service or block port 22 at the network firewall.
    \item \textbf{Long-Term Fix:} If access is required, implement strict firewall rules to only allow connections from trusted, whitelisted IP addresses. Furthermore, enforce the use of public key authentication and disable password-based logins for SSH.
\end{itemize}

\subsection{RISK-005: Establish Annual Security Training}
\textbf{Action:} Develop and implement a mandatory annual security awareness training program for all employees, including management. The training should cover current threats such as phishing, social engineering, and proper data handling. Track completion to ensure compliance.

\end{document}
```