```latex
\documentclass[12pt]{article}

% === PACKAGES ===
\usepackage[margin=1in]{geometry}
\usepackage{pifont} % For \ding{51} (checkmark) and \ding{55} (cross)
\usepackage{booktabs} % For professional-looking tables
\usepackage{hyperref}
\usepackage{url}
\usepackage{seqsplit} % To break long strings in \texttt
\usepackage{graphicx}
\usepackage{xcolor}
\usepackage{fancyhdr}
\usepackage{array}

% === DOCUMENT CONFIGURATION ===
\hypersetup{
    colorlinks=true,
    linkcolor=blue,
    filecolor=magenta,
    urlcolor=cyan,
}

% Define colors for severity
\definecolor{criticalred}{HTML}{990000}
\definecolor{highorange}{HTML}{E69138}
\definecolor{mediumyellow}{HTML}{F1C232}
\definecolor{lowgreen}{HTML}{6AA84F}

% Custom column type for wrapping text in tables
\newcolumntype{L}[1]{>{\raggedright\let\newline\\\arraybackslash\hspace{0pt}}m{#1}}

% === HEADER & FOOTER ===
\pagestyle{fancy}
\fancyhf{}
\lhead{Cybersecurity Assessment Report}
\rhead{\textbf{[Organization Name]}}
\cfoot{Page \thepage}
\renewcommand{\headrulewidth}{0.4pt}
\renewcommand{\footrulewidth}{0.4pt}

% === DOCUMENT START ===
\begin{document}

% === TITLE PAGE ===
\begin{titlepage}
    \centering
    \vspace*{1cm}
    \Huge\textbf{Cybersecurity Posture Assessment Report}
    \vspace{1.5cm}
    \
    \large
    \textbf{Prepared for:} \\
    \vspace{0.2cm}
    \textbf{[Organization Name]}
    \
    \vspace{2cm}
    \textbf{Date of Report:} \\
    \vspace{0.2cm}
    \today
    \
    \vfill
    \large
    \textbf{Generated by:} \\
    \vspace{0.2cm}
    Cybersecurity Analysis Division
\end{titlepage}

\tableofcontents
\newpage

% === SECTION 1: EXECUTIVE SUMMARY ===
\section{Executive Summary}

This report provides a comprehensive analysis of the cybersecurity posture for \textbf{[Organization Name]}, based on a synthesis of network scan data, a security controls questionnaire, and a review of pre-existing risks. The assessment identified several critical and high-severity risks that require immediate attention to mitigate potential security breaches.

The most significant findings are:
\begin{itemize}
    \item \textbf{Critical MFA Gap:} Multi-Factor Authentication (MFA) is not enforced for email access. This exposes the organization to a high risk of business email compromise (BEC), phishing attacks, and unauthorized account access.
    \item \textbf{Critical Database Exposure:} A MySQL database is publicly exposed to the internet. The service is running version 5.7.33, which reached its official End-of-Life (EOL) in October 2023. EOL software no longer receives security updates, making it an easy target for attackers leveraging known vulnerabilities.
\end{itemize}

While the organization has implemented several positive security controls, such as MFA for computer logins and security awareness training, the identified gaps represent significant threats. This report outlines these findings in detail and provides actionable recommendations to strengthen the organization's security defenses.

% === SECTION 2: ORGANIZATIONAL & SCAN INFORMATION ===
\section{Organizational \& Scan Information}

This section details the organizational context and the scope of the technical assessment. The information provided is based on the data supplied for this analysis.

\begin{table}[h!]
\centering
\begin{tabular}{@{}ll@{}}
\toprule
\textbf{Item} & \textbf{Detail} \\ \midrule
Organization Name & \textbf{[Organization Name]} \\
Primary Domain & \texttt{[Domain]} \\
Client External IP & \texttt{[Client IP]} \\
Scanned Target IP & \texttt{[Target IP]} \\ \bottomrule
\end{tabular}
\caption{Assessment Scope and Target Information.}
\label{tab:org_info}
\end{table}

% === SECTION 3: SECURITY CONTROL REVIEW ===
\section{Security Control Review}

The following table summarizes the organization's responses to a security controls questionnaire. This review helps identify policy and procedure gaps that can impact overall security. A green checkmark (\textcolor{lowgreen}{\ding{51}}) indicates a positive control, while a red cross (\textcolor{criticalred}{\ding{55}}) indicates a significant gap.

\begin{table}[h!]
\centering
\begin{tabular}{@{}L{10cm}cc@{}}
\toprule
\textbf{Control Question} & \textbf{Response} & \textbf{Status} \\ \midrule
Do you require MFA to access email? & No & \textcolor{criticalred}{\ding{55}} \\
Do you require MFA to log into computers? & Yes & \textcolor{lowgreen}{\ding{51}} \\
Do you require MFA to access sensitive data systems? & Yes & \textcolor{lowgreen}{\ding{51}} \\
Does your organization have an employee acceptable use policy? & Yes & \textcolor{lowgreen}{\ding{51}} \\
Does your organization do security awareness training for new employees? & Yes & \textcolor{lowgreen}{\ding{51}} \\
Does your organization do security awareness training for all employees at least once per year? & Yes & \textcolor{lowgreen}{\ding{51}} \\ \bottomrule
\end{tabular}
\caption{Security Controls Questionnaire Results.}
\label{tab:controls}
\end{table}

\subsection*{Analysis}
The questionnaire reveals a critical weakness in the access control policy: the lack of mandatory MFA for email. Email is a primary vector for cyberattacks, and its compromise can lead to data breaches, financial fraud, and further infiltration of the network. While other controls are in place, this single gap significantly elevates the organization's risk profile.

% === SECTION 4: TECHNICAL SCAN RESULTS ===
\section{Technical Scan Results}

An external network scan was performed to identify open ports and exposed services. The results below detail the findings for the target system.

\begin{table}[h!]
\centering
\begin{tabular}{@{}lllll@{}}
\toprule
\textbf{Port} & \textbf{State} & \textbf{Service} & \textbf{Product} & \textbf{Version} \\ \midrule
3306/tcp & open & mysql & MySQL & 5.7.33 \\ \bottomrule
\end{tabular}
\caption{Open Ports and Services Detected via Nmap Scan.}
\label{tab:nmap_results}
\end{table}

\subsection*{Analysis}
The scan confirms that port 3306 is open, which is the default port for the MySQL database service. Exposing a database directly to the public internet is a highly dangerous practice.
\
\
\textbf{End-of-Life Software Detected:} The identified MySQL version, \textbf{5.7.33}, is particularly alarming. The MySQL 5.7 series reached its End-of-Life (EOL) in \textbf{October 2023}. This means it no longer receives security patches from the vendor, and any vulnerabilities discovered since that date will remain unpatched. Attackers actively scan for EOL software as it represents a stable and reliable entry point into a network.

% === SECTION 5: CONSOLIDATED RISK ASSESSMENT ===
\section{Consolidated Risk Assessment}

This section synthesizes all findings from the questionnaire, technical scan, and pre-existing risk data into a unified list of security risks. Each risk is assigned a severity level to guide prioritization.

\begin{table}[h!]
\centering
\begin{tabular}{@{}p{0.1\linewidth} p{0.3\linewidth} p{0.15\linewidth} p{0.35\linewidth}@{}}
\toprule
\textbf{Risk ID} & \textbf{Risk Name} & \textbf{Severity} & \textbf{Description} \\ \midrule
RISK-001 & \textbf{Publicly Exposed End-of-Life Database} & \textcolor{criticalred}{\textbf{Critical}} & A MySQL 5.7.33 database (EOL) is exposed on port 3306. This allows attackers to exploit known, unpatched vulnerabilities to gain unauthorized access, exfiltrate data, or compromise the underlying server. This finding confirms and elevates the pre-existing "Database Exposure" risk. \\
\addlinespace
RISK-002 & \textbf{MFA Not Enforced for Email} & \textcolor{criticalred}{\textbf{Critical}} & The absence of MFA on email accounts makes them highly susceptible to compromise via phishing or credential stuffing attacks. A compromised email account is a gateway to widespread organizational damage. \\
\bottomrule
\end{tabular}
\caption{Summary of Identified Security Risks.}
\label{tab:risks}
\end{table}

% === SECTION 6: RECOMMENDATIONS ===
\section{Recommendations}

The following actionable recommendations are provided to address the identified risks. They are prioritized based on severity and potential impact.

\subsection*{RISK-001: Publicly Exposed End-of-Life Database (Critical)}
\begin{itemize}
    \item \textbf{Immediate Action (Containment):} Implement strict firewall rules to block all public access to TCP port 3306. Access should be restricted to a whitelist of specific, trusted IP addresses only.
    \item \textbf{Short-Term Action (Remediation):} Develop and execute a plan to upgrade the MySQL database from the EOL version 5.7 to a currently supported version (e.g., MySQL 8.x). This is crucial for receiving security patches.
    \item \textbf{Long-Term Strategy (Prevention):} Re-architect the application infrastructure to ensure databases are not directly exposed to the internet. Access should be mediated through an application layer or restricted to a private network accessible only via a Virtual Private Network (VPN). Implement a formal patch and vulnerability management program.
\end{itemize}

\subsection*{RISK-002: MFA Not Enforced for Email (Critical)}
\begin{itemize}
    \item \textbf{Immediate Action (Remediation):} Enforce MFA for all user accounts on the organization's email platform (e.g., Microsoft 365, Google Workspace) without delay.
    \item \textbf{Long-Term Strategy (Prevention):} Establish a formal access control policy that mandates MFA for all critical systems, including remote access solutions, administrative interfaces, and cloud services. Conduct user training to ensure a smooth rollout and adoption.
\end{itemize}

\end{document}
```