```latex
\documentclass[12pt]{article}

% Preamble: Required Packages and Document Setup
\usepackage[margin=1in]{geometry}
\usepackage{pifont} % For checkmarks and crosses
\usepackage{booktabs} % For professional tables
\usepackage{hyperref} % For hyperlinks
\usepackage{url} % For URL formatting
\usepackage{seqsplit} % For splitting long strings
\usepackage{graphicx} % For potential logo
\usepackage{xcolor} % For colors

% Define colors for severity
\definecolor{critical}{HTML}{990000}
\definecolor{high}{HTML}{D13F00}
\definecolor{medium}{HTML}{E89900}
\definecolor{low}{HTML}{339900}

% Hyperref setup
\hypersetup{
    colorlinks=true,
    linkcolor=blue,
    filecolor=magenta,      
    urlcolor=cyan,
    pdftitle={Cybersecurity Assessment Report},
    pdfpagemode=FullScreen,
}

% Title Page Information
\title{Cybersecurity Assessment Report \\ \large For: \textbf{[Organization Name]}}
\author{Cybersecurity Analysis Division}
\date{\today}

\begin{document}

\maketitle
\thispagestyle{empty}
\newpage

\tableofcontents
\newpage

% --- 1. Executive Summary ---
\section{Executive Summary}

This report provides a cybersecurity assessment for \textbf{[Organization Name]}, conducted on \textbf{[Scan Date]}. The analysis is based on a combination of an external network scan, a review of organizational security controls via a questionnaire, and an evaluation of pre-existing risk data.

The assessment reveals a mixed security posture. On the positive side, the external network scan of the target IP address \texttt{[Target IP]} did not identify any open ports or exposed services. This suggests a strong network perimeter configuration for the scanned asset, minimizing the external attack surface.

However, significant gaps were identified in the organization's procedural and administrative controls. The two most critical findings are:
\begin{itemize}
    \item \textbf{Lack of an Employee Acceptable Use Policy (AUP):} This absence creates ambiguity regarding the proper use of company assets and data, increasing the risk of insider threats and unintentional data breaches.
    \item \textbf{Insufficient Security Awareness Training:} While new employees receive training, there is no mandatory annual training for all staff. This lapse allows security knowledge to become outdated, making employees more susceptible to social engineering attacks like phishing.
\end{itemize}

These policy and training deficiencies represent a high level of risk to the organization. Recommendations in this report focus on establishing these foundational security controls to mitigate human-related risks and strengthen the overall security culture.

% --- 2. Organizational Information ---
\section{Organizational Information}

This section details the organizational information used as the basis for this assessment. The data has been anonymized as per the engagement protocol.

\begin{tabular}{@{}ll}
\toprule
\textbf{Attribute} & \textbf{Value} \\
\midrule
Organization Name & \textbf{[Organization Name]} \\
Primary Email Domain & \texttt{[Domain]} \\
Client External IP & \texttt{[Client IP]} \\
\bottomrule
\end{tabular}

% --- 3. Security Control Review (Questionnaire Analysis) ---
\section{Security Control Review}

The following table summarizes the organization's responses to a security controls questionnaire. "No" answers indicate potential gaps in the security framework and are highlighted as areas for improvement.

\begin{table}[h!]
\centering
\caption{Security Controls Questionnaire Results}
\begin{tabular}{p{0.6\textwidth} c p{0.2\textwidth}}
\toprule
\textbf{Control Question} & \textbf{Response} & \textbf{Assessment} \\
\midrule
Do you require MFA to access email? & \ding{51} & Best Practice Met \\
Do you require MFA to log into computers? & \ding{51} & Best Practice Met \\
Do you require MFA to access sensitive data systems? & \ding{51} & Best Practice Met \\
\addlinespace
Does your organization have an employee acceptable use policy? & \textbf{\color{red}\ding{55}} & \textbf{Critical Gap} \\
\addlinespace
Does your organization do security awareness training for new employees? & \ding{51} & Good Practice \\
\addlinespace
Does your organization do security awareness training for all employees at least once per year? & \textbf{\color{red}\ding{55}} & \textbf{High Risk} \\
\bottomrule
\end{tabular}
\end{table}

\paragraph{Analysis:} The organization has successfully implemented Multi-Factor Authentication (MFA) across key systems, which is a commendable and highly effective control. However, the absence of a formal Acceptable Use Policy and the lack of recurring security awareness training for all staff are significant vulnerabilities that undermine the technical controls in place.

% --- 4. Technical Scan Results ---
\section{Technical Scan Results}

An external network vulnerability scan was performed to identify exposed services and potential vulnerabilities on the organization's perimeter.

\begin{itemize}
    \item \textbf{Target IP Address:} \texttt{[Target IP]}
    \item \textbf{Scan Date:} \textbf{[Scan Date]}
\end{itemize}

\subsection{Scan Findings}
The scan completed successfully. \textbf{No open ports or services were detected on the target system.} This indicates a well-configured firewall or that the scanned system does not host public-facing services, which is a strong defensive posture from a network perspective.

% --- 5. Consolidated Risk Assessment ---
\section{Consolidated Risk Assessment}

This section consolidates findings from the security control review, technical scan, and pre-existing risk data. Based on the provided inputs, no pre-existing or technical vulnerabilities were identified. The primary risks are administrative and procedural.

\begin{table}[h!]
\centering
\caption{Summary of Identified Risks}
\begin{tabular}{p{0.1\textwidth} p{0.25\textwidth} p{0.4\textwidth} p{0.1\textwidth}}
\toprule
\textbf{ID} & \textbf{Risk Name} & \textbf{Description} & \textbf{Severity} \\
\midrule
RISK-001 & Lack of Acceptable Use Policy (AUP) & Without a formal AUP, employees may misuse company assets or mishandle data without clear consequences, increasing insider risk. & \textcolor{high}{\textbf{High}} \\
\addlinespace
RISK-002 & Insufficient Security Awareness Training & The lack of annual, mandatory training for all staff increases susceptibility to phishing, social engineering, and other human-centric attacks. & \textcolor{high}{\textbf{High}} \\
\bottomrule
\end{tabular}
\end{table}

% --- 6. Recommendations ---
\section{Recommendations}

The following actions are recommended to address the identified risks and improve the overall cybersecurity posture of \textbf{[Organization Name]}.

\subsection{Recommendation for RISK-001: Implement an Acceptable Use Policy}
\begin{itemize}
    \item \textbf{Action:} Develop, document, and implement a comprehensive Employee Acceptable Use Policy (AUP). This policy should clearly define the rules for using company networks, devices, software, and data.
    \item \textbf{Details:} The AUP should be reviewed by legal counsel, communicated to all employees, and integrated into the new hire onboarding process. Existing employees must read and formally acknowledge the policy.
    \item \textbf{Priority:} \textbf{Critical}
\end{itemize}

\subsection{Recommendation for RISK-002: Establish Annual Security Training}
\begin{itemize}
    \item \textbf{Action:} Institute a mandatory security awareness training program for all employees to be completed at least once per year.
    \item \textbf{Details:} The training should cover current threats such as phishing, ransomware, password security, and social engineering. The program should include a method for tracking completion to ensure 100\% compliance.
    \item \textbf{Priority:} \textbf{High}
\end{itemize}

\subsection{General Recommendation: Continuous Monitoring}
\begin{itemize}
    \item \textbf{Action:} Although the initial scan showed no open services, it is crucial to perform regular, automated external and internal vulnerability scans.
    \item \textbf{Details:} A continuous monitoring strategy ensures that new vulnerabilities or misconfigurations are detected and remediated promptly as the IT environment changes.
    \item \textbf{Priority:} \textbf{Medium}
\end{itemize}

\end{document}
```