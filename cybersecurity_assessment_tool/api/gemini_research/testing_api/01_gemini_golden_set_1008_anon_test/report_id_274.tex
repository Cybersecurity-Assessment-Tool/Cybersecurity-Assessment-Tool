```latex
\documentclass[12pt]{article}

% Required Packages
\usepackage[margin=1in]{geometry}
\usepackage{pifont} % For \ding
\usepackage{booktabs} % For professional tables
\usepackage{hyperref} % For hyperlinks
\usepackage{url} % For URL formatting
\usepackage{seqsplit} % To split long strings in tt font
\usepackage{graphicx}
\usepackage{xcolor}

% Hyperref Setup
\hypersetup{
    colorlinks=true,
    linkcolor=blue,
    filecolor=magenta,      
    urlcolor=cyan,
    pdftitle={Cybersecurity Assessment Report},
    pdfpagemode=FullScreen,
}

% Define check and cross marks for tables
\newcommand{\cmark}{\ding{51}}
\newcommand{\xmark}{\ding{55}}

% --- Document Start ---
\begin{document}

% --- Title Page ---
\begin{titlepage}
    \centering
    \vspace*{1cm}
    \Huge \textbf{Cybersecurity Assessment Report}
    \vspace{1.5cm}
    \Large \textbf{Prepared for:} \\
    \vspace{0.5cm}
    \huge \textbf{[Organization Name]}
    \vspace{2cm}
    \rule{\linewidth}{0.5mm}
    \vspace{0.5cm}
    \large \textbf{Date of Report:} \today \\
    \vspace{0.2cm}
    \large \textbf{Analysis Period:} Current Scan
    \rule{\linewidth}{0.5mm}
    \vfill
    \large \textit{This report contains sensitive information and is intended solely for the use of the recipient organization. Distribution is strictly prohibited.}
\end{titlepage}

\tableofcontents
\newpage

% --- Section 1: Executive Summary ---
\section{Executive Summary}
This report provides a comprehensive analysis of the cybersecurity posture of \textbf{[Organization Name]}, based on a review of organizational security controls, an external network vulnerability scan, and a summary of pre-existing risks.

The assessment revealed a mixed security posture. On a positive note, the external network scan of the target IP address indicated a strong perimeter defense, as no open ports or services were discovered. This suggests that the external-facing firewall is well-configured and effectively limits the organization's attack surface.

However, the review of internal security controls identified two high-risk gaps that require immediate attention. Firstly, Multi-Factor Authentication (MFA) is not enforced for accessing sensitive data systems. This creates a significant risk of unauthorized access and data breach if user credentials are compromised. Secondly, new employees do not receive mandatory security awareness training during their onboarding process, leaving a critical window of vulnerability to social engineering and unintentional policy violations.

This report details these findings and provides actionable recommendations to mitigate the identified risks and strengthen the organization's overall security framework.

% --- Section 2: Organizational Information ---
\section{Organizational Information}
This section outlines the basic information for the organization under review. This data is used to define the scope of the assessment.

\begin{itemize}
    \item \textbf{Organization Name:} \textbf{[Organization Name]}
    \item \textbf{Primary Email Domain:} \texttt{[Domain]}
    \item \textbf{Scanned External IP:} \texttt{[Client IP]}
\end{itemize}

% --- Section 3: Security Control Review ---
\section{Security Control Review}
A security questionnaire was completed to evaluate the implementation of key administrative and technical controls. The results are summarized below. A green checkmark (\textcolor{green}{\cmark}) indicates a positive control is in place, while a red cross (\textcolor{red}{\xmark}) indicates a control gap.

\begin{table}[h!]
\centering
\caption{Security Controls Questionnaire Results}
\begin{tabular}{p{0.8\linewidth} c}
\toprule
\textbf{Control Question} & \textbf{Status} \\
\midrule
Do you require MFA to access email? & \textcolor{green}{\cmark} \\
Do you require MFA to log into computers? & \textcolor{green}{\cmark} \\
\textbf{Do you require MFA to access sensitive data systems?} & \textcolor{red}{\xmark} \\
Does your organization have an employee acceptable use policy? & \textcolor{green}{\cmark} \\
\textbf{Does your organization do security awareness training for new employees?} & \textcolor{red}{\xmark} \\
Does your organization do security awareness training for all employees at least once per year? & \textcolor{green}{\cmark} \\
\bottomrule
\end{tabular}
\end{table}

\subsection*{Analysis of Control Gaps}
The review identified two significant control gaps:
\begin{enumerate}
    \item \textbf{Lack of MFA for Sensitive Systems:} While MFA is commendably used for email and computer logins, its absence on sensitive data systems exposes critical assets to credential theft attacks.
    \item \textbf{No Onboarding Security Training:} New employees are a primary target for attackers. The lack of security training during the onboarding process means they are not equipped to recognize threats like phishing or understand internal security policies from day one.
\end{enumerate}

% --- Section 4: Technical Scan Results ---
\section{Technical Scan Results}
An external network scan was conducted to identify open ports, running services, and potential vulnerabilities on the organization's public-facing infrastructure.

\begin{itemize}
    \item \textbf{Target IP Address:} \texttt{[Target IP]}
    \item \textbf{Scan Date:} Not provided in scan data.
\end{itemize}

\subsection*{Findings}
The scan completed successfully and found \textbf{no open ports}. All 1000 scanned TCP ports were in a `closed` or `filtered` state. This is a strong positive finding, indicating that the network perimeter is well-secured and does not expose any services to the public internet at this address.

% --- Section 5: Risk Assessment ---
\section{Risk Assessment}
This section synthesizes findings from the security control review, technical scan, and pre-existing risk data. The primary risks identified stem from gaps in administrative and access controls. No technical vulnerabilities were discovered during the external scan, and no pre-existing vulnerabilities were provided for this assessment.

\begin{table}[h!]
\centering
\caption{Identified Risks and Severity}
\begin{tabular}{p{0.25\linewidth} p{0.55\linewidth} p{0.1\linewidth}}
\toprule
\textbf{Risk Name} & \textbf{Overview} & \textbf{Severity} \\
\midrule
\textbf{Inadequate Access Control for Sensitive Systems} & The absence of Multi-Factor Authentication (MFA) on systems containing sensitive data significantly increases the risk of unauthorized access and data breach through compromised credentials. & \textbf{High} \\
\addlinespace
\textbf{Insufficient New Hire Security Training} & Failing to provide security awareness training during employee onboarding leaves the organization vulnerable to social engineering, phishing, and unintentional policy violations by new staff members. & \textbf{High} \\
\bottomrule
\end{tabular}
\end{table}

% --- Section 6: Recommendations ---
\section{Recommendations}
Based on the risk assessment, the following high-priority actions are recommended to enhance the organization's security posture.

\begin{enumerate}
    \item \textbf{High Priority: Implement MFA for Sensitive Systems} \\
    \textit{Description:} Procure and deploy an MFA solution for all applications and systems classified as containing sensitive or critical data. This control should be mandatory for all users, including administrators and executives. \\
    \textit{Impact:} Drastically reduces the risk of unauthorized access from compromised passwords, protecting the organization's most valuable data assets.

    \item \textbf{High Priority: Establish Mandatory Onboarding Security Training} \\
    \textit{Description:} Develop a mandatory security awareness training module to be included in the formal onboarding process for all new employees and contractors. This training should cover, at a minimum: phishing and social engineering awareness, acceptable use policies, password security, and incident reporting procedures. \\
    \textit{Impact:} Equips new hires with the knowledge to act as a human firewall from their first day, reducing susceptibility to common cyber threats and reinforcing a culture of security.
\end{enumerate}

% --- Document End ---
\end{document}
```