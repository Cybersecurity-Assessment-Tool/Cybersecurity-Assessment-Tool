```latex
\documentclass[12pt]{article}

% --- PACKAGES ---
\usepackage[margin=1in]{geometry}
\usepackage{pifont} % For checkmarks and crosses
\usepackage{booktabs} % For professional tables
\usepackage{hyperref} % For hyperlinks
\usepackage{url} % For URL formatting
\usepackage{seqsplit} % To split long strings in tt font
\usepackage{graphicx} % For logo
\usepackage{xcolor} % For colors

% --- DOCUMENT INFORMATION ---
\title{
    \vspace{-1.5cm}
    \includegraphics[width=0.3\textwidth]{cyber-logo.png} \\ % Placeholder for a logo
    \vspace{1cm}
    \textbf{Cybersecurity Posture Assessment Report} \\
    \large Prepared for: \textbf{[Organization Name]}
}
\author{Cybersecurity Analyst}
\date{\today}

% --- HYPERREF SETUP ---
\hypersetup{
    colorlinks=true,
    linkcolor=blue,
    filecolor=magenta,      
    urlcolor=cyan,
    pdftitle={Cybersecurity Posture Assessment Report},
    pdfpagemode=FullScreen,
}

\begin{document}

\maketitle
\thispagestyle{empty}
\newpage

\tableofcontents
\newpage

% --- EXECUTIVE SUMMARY ---
\section*{1.0 Executive Summary}

This report provides a cybersecurity posture assessment for \textbf{[Organization Name]}, based on an analysis of network scan data, organizational security controls, and known risks. The assessment was conducted on \today.

The analysis revealed \textbf{two critical-risk findings} that require immediate attention. The primary concern is the direct exposure of a Remote Desktop Protocol (RDP) service on port 3389 to the public internet at \texttt{[Target IP]}. This configuration is a common vector for ransomware attacks and unauthorized access.

This technical vulnerability is severely compounded by a systemic organizational weakness: the \textbf{complete absence of Multi-Factor Authentication (MFA)} for accessing email, computers, and sensitive data systems. The combination of an exposed entry point (RDP) and weak authentication controls (lack of MFA) creates a high-probability path for a malicious actor to compromise the internal network.

While the organization demonstrates a solid foundation in security policy and awareness training, the identified technical and authentication gaps currently place the organization at an unacceptably high level of risk. Immediate remediation of the exposed RDP service and a strategic rollout of MFA are strongly recommended.

% --- ORGANIZATIONAL INFORMATION ---
\section*{2.0 Organizational Information}

This section outlines the key details of the organization as understood for this assessment. Due to the anonymized nature of the provided data, placeholders have been used.

\begin{itemize}
    \item \textbf{Organization Name:} \textbf{[Organization Name]}
    \item \textbf{Primary Email Domain:} \texttt{[Domain]}
    \item \textbf{Assessed External IP:} \texttt{[Client IP]}
\end{itemize}

% --- SECURITY CONTROL REVIEW ---
\section*{3.0 Security Control Review}

A review of the organization's security controls was conducted via a questionnaire. The responses highlight a significant gap in identity and access management. While foundational policies and training are in place, the lack of MFA is a critical deficiency.

\begin{table}[h!]
\centering
\caption{Security Controls Questionnaire Analysis}
\begin{tabular}{p{0.6\linewidth} c p{0.2\linewidth}}
\toprule
\textbf{Control Question} & \textbf{Response} & \textbf{Assessment} \\
\midrule
Does your organization have an employee acceptable use policy? & \ding{51} Yes & Best Practice Met \\
Does your organization do security awareness training for new employees? & \ding{51} Yes & Best Practice Met \\
Does your organization do security awareness training for all employees at least once per year? & \ding{51} Yes & Best Practice Met \\
\midrule
Do you require MFA to access email? & {\color{red}\ding{55}} No & \textbf{Critical Gap} \\
Do you require MFA to log into computers? & {\color{red}\ding{55}} No & \textbf{Critical Gap} \\
Do you require MFA to access sensitive data systems? & {\color{red}\ding{55}} No & \textbf{Critical Gap} \\
\bottomrule
\end{tabular}
\end{table}

% --- TECHNICAL SCAN RESULTS ---
\section*{4.0 Technical Scan Results}

An external network scan was performed on the target IP address. The scan identified one open port, which presents a significant security risk.

\begin{itemize}
    \item \textbf{Target IP Address:} \texttt{[Target IP]}
    \item \textbf{Scan Tool:} Nmap
\end{itemize}

\begin{table}[h!]
\centering
\caption{Open Ports Detected on \texttt{[Target IP]}}
\begin{tabular}{c c l l}
\toprule
\textbf{Port} & \textbf{State} & \textbf{Service} & \textbf{Analysis} \\
\midrule
3389/tcp & Open & ms-wbt-server & High Risk. This is the Remote Desktop Protocol (RDP). \\
\bottomrule
\end{tabular}
\end{table}

\subsection*{Analysis of Findings}
The discovery of an open RDP port (3389) is a critical finding. RDP is a protocol that allows for direct, graphical remote control of a server or workstation. Exposing this service directly to the internet is highly discouraged as it makes the organization a prime target for:
\begin{itemize}
    \item \textbf{Brute-Force Attacks:} Automated tools constantly scan the internet for open RDP ports and attempt to guess user credentials.
    \item \textbf{Credential Stuffing:} Attackers use credentials stolen from other data breaches to attempt logins.
    \item \textbf{Exploitation of Vulnerabilities:} RDP has had numerous critical vulnerabilities over the years (e.g., BlueKeep). Even a fully patched system is at risk from password-based attacks.
\end{itemize}
This finding directly correlates with and confirms the pre-existing risk documented in the organization's risk register.

% --- RISK ASSESSMENT SUMMARY ---
\section*{5.0 Risk Assessment Summary}

The following table synthesizes findings from the security control review, technical scan, and pre-existing risk data into a prioritized list. The severity scores are based on the CVSS (Common Vulnerability Scoring System) framework where applicable, with a scale of 0-10.

\begin{table}[h!]
\centering
\caption{Synthesized Risk Register}
\begin{tabular}{p{0.1\linewidth} p{0.25\linewidth} p{0.45\linewidth} c}
\toprule
\textbf{Risk ID} & \textbf{Risk Name} & \textbf{Description} & \textbf{Severity} \\
\midrule
\textbf{RISK-001} & Public RDP Exposure & The Remote Desktop Protocol on port 3389 is exposed to the internet on \texttt{[Target IP]}, allowing attackers to attempt unauthorized access. This is a primary entry vector for ransomware. & \textbf{Critical (9.0)} \\
\addlinespace
\textbf{RISK-002} & Systemic Lack of MFA & Multi-Factor Authentication is not enforced for email, computer logins, or access to sensitive systems. This critically weakens credential security and magnifies the impact of a compromised password. & \textbf{Critical (9.0)} \\
\bottomrule
\end{tabular}
\end{table}

% --- RECOMMENDATIONS ---
\section*{6.0 Recommendations}

Based on the critical risks identified, the following actions are recommended. They are prioritized into immediate actions to mitigate the most severe threats and strategic improvements for long-term security.

\subsection*{6.1 Immediate Actions (To be completed within 72 hours)}
\begin{enumerate}
    \item \textbf{Remediate RDP Exposure (RISK-001):} Immediately close port 3389 on the external firewall for \texttt{[Target IP]}. If remote access is essential, restrict access to a list of known, trusted source IP addresses as a temporary measure until a VPN solution is implemented.
\end{enumerate}

\subsection*{6.2 Strategic Improvements (To be implemented within 3-6 months)}
\begin{enumerate}
    \item \textbf{Deploy a Virtual Private Network (VPN) (RISK-001):} Implement a corporate VPN for all remote access. This provides a secure, encrypted tunnel for remote administration and removes the need to expose services like RDP directly to the internet.
    \item \textbf{Implement Multi-Factor Authentication (MFA) (RISK-002):} Begin a phased rollout of MFA across the organization. The highest priorities are:
    \begin{itemize}
        \item Email (e.g., Office 365, Google Workspace).
        \item All remote access solutions (VPN).
        \item Access to sensitive data systems and privileged accounts.
    \end{itemize}
    \item \textbf{Conduct a Password Policy Review:} In conjunction with the MFA rollout, review and enforce a strong password policy, including complexity requirements and a ban on easily guessed or previously breached passwords.
\end{enumerate}

\end{document}
```