```latex
\documentclass[12pt]{article}

% Preamble: Required Packages
\usepackage[margin=1in]{geometry}
\usepackage{pifont} % Required for \ding{51} (checkmark) and \ding{55} (cross)
\usepackage{booktabs} % For professional-looking tables (\toprule, \midrule, \bottomrule)
\usepackage{hyperref} % For clickable links and references
\usepackage{url}      % For formatting URLs
\usepackage{seqsplit} % For splitting long, unbreakable strings like hashes or URLs

% Document Metadata and Hyperlink Setup
\hypersetup{
    colorlinks=true,
    linkcolor=black,
    filecolor=magenta,
    urlcolor=blue,
    pdftitle={Cybersecurity Posture Assessment Report},
    pdfauthor={Cybersecurity Analyst},
    pdfsubject={Security Analysis},
    pdfkeywords={Cybersecurity, Risk Assessment, Nmap, LaTeX}
}

% --- Document Start ---
\begin{document}

% Title Block
\title{Cybersecurity Posture Assessment Report}
\author{Cybersecurity Analyst}
\date{\today}
\maketitle

% --- 1. Executive Summary ---
\section*{Executive Summary}

This report details the findings of a cybersecurity posture assessment conducted for \textbf{[Organization Name]}. The assessment synthesized data from an external network scan, a security controls questionnaire, and a list of pre-existing risks.

The analysis reveals a \textbf{Critical} overall risk posture. Several severe vulnerabilities and security control gaps were identified that require immediate attention. Key findings include:
\begin{itemize}
    \item \textbf{Critically Vulnerable External Service:} An external FTP server is running a version of \texttt{vsftpd} (2.3.4) with a known, severe backdoor vulnerability (CVE-2011-2523), which could allow an attacker to gain full control of the system.
    \item \textbf{Complete Absence of Multi-Factor Authentication (MFA):} MFA is not enforced for email, computer logins, or access to sensitive data. This represents a critical gap, as a single compromised password could lead to a widespread breach.
    \item \textbf{Inadequate Security Training:} While new employees receive training, there is no recurring annual security awareness program for all staff, increasing the organization's susceptibility to phishing and other social engineering attacks.
    \item \textbf{Known End-of-Life Systems:} The organization is aware of workstations running the outdated and unsupported Windows 7 operating system.
\end{itemize}

Immediate remediation of the vulnerable FTP server and the rapid implementation of MFA are paramount to reducing the organization's exposure to attack.

% --- 2. Organizational Information ---
\section*{Organizational Information}

The following details were used as the basis for this assessment. As per the template mode, placeholders are used where data was not provided.
\begin{itemize}
    \item \textbf{Organization Name:} \textbf{[Organization Name]}
    \item \textbf{Primary Email Domain:} \texttt{[Domain]}
    \item \textbf{External IP Scanned:} \texttt{[Target IP]}
\end{itemize}

% --- 3. Security Control Review ---
\section{Security Control Review}

The following table summarizes the organization's responses to a security controls questionnaire. "No" answers indicate significant gaps in the security framework and are highlighted with analyst notes.

\vspace{1em} % Add some vertical space before the table

\begin{tabular}{p{0.6\textwidth} c p{0.25\textwidth}}
\toprule
\textbf{Control Question} & \textbf{Response} & \textbf{Analyst Note} \\
\midrule
Do you require MFA to access email? & \ding{55} & \textbf{Critical Gap.} Email is a primary target for attackers. \\
Do you require MFA to log into computers? & \ding{55} & \textbf{Critical Gap.} Lack of MFA allows for easier lateral movement post-breach. \\
Do you require MFA to access sensitive data systems? & \ding{55} & \textbf{Critical Gap.} Exposes crown jewel data to unauthorized access. \\
Does your organization do security awareness training for all employees at least once per year? & \ding{55} & \textbf{High Risk.} Security knowledge degrades over time without reinforcement. \\
Does your organization have an employee acceptable use policy? & \ding{51} & Foundational policy is in place. \\
Does your organization do security awareness training for new employees? & \ding{51} & Good practice for onboarding. \\
\bottomrule
\end{tabular}

% --- 4. Technical Scan Results ---
\section{Technical Scan Results}

An Nmap scan was performed on the external IP address \texttt{[Target IP]}. The scan identified one open port with a critically vulnerable service.

\vspace{1em}

\begin{tabular}{l l l l p{0.4\textwidth}}
\toprule
\textbf{Port} & \textbf{State} & \textbf{Service} & \textbf{Version} & \textbf{Finding} \\
\midrule
21/tcp & open & ftp & vsftpd 2.3.4 & \textbf{Critical.} This specific version contains a well-known backdoor (CVE-2011-2523) that allows for remote command execution. Furthermore, the server permits anonymous FTP login, which can be abused for data exfiltration or malware staging. \\
\bottomrule
\end{tabular}

% --- 5. Consolidated Risk Assessment ---
\section{Consolidated Risk Assessment}

The following table correlates findings from the questionnaire, technical scan, and pre-existing risk data to provide a unified view of the organization's top security risks.

\vspace{1em}

\begin{tabular}{p{0.3\textwidth} p{0.5\textwidth} l}
\toprule
\textbf{Risk Name} & \textbf{Description} & \textbf{Severity} \\
\midrule
Vulnerable FTP Server & The public-facing FTP server (\texttt{vsftpd 2.3.4}) is susceptible to a known remote code execution backdoor. & \textbf{Critical} \\
\addlinespace
Absence of MFA & No MFA is enforced for email, computers, or sensitive data access, severely increasing the risk of unauthorized access from compromised credentials. & \textbf{Critical} \\
\addlinespace
Anonymous FTP Access & Anonymous FTP login is permitted on an external server, which can be abused by attackers for data exfiltration or as a malware drop point. & \textbf{High} \\
\addlinespace
Lack of Annual Security Training & No recurring security awareness training is provided to all employees, leading to an increased susceptibility to social engineering attacks like phishing. & \textbf{High} \\
\addlinespace
Outdated Windows OS & Workstations are running Windows 7, which is end-of-life and no longer receives security updates from Microsoft, leaving them vulnerable to exploitation. & \textbf{Medium} \\
\bottomrule
\end{tabular}

% --- 6. Recommendations ---
\section{Recommendations}

Based on the consolidated risk assessment, the following actions are recommended, prioritized by severity.

\subsection*{Immediate Actions (Critical Priority)}
\begin{enumerate}
    \item \textbf{Remediate FTP Server:} Immediately take the FTP server at \texttt{[Target IP]} offline. If the service is business-critical, it must be upgraded to a patched version and reconfigured to disable anonymous access. Secure alternatives like SFTP (SSH File Transfer Protocol) should be implemented for all future file transfers.
    \item \textbf{Implement MFA:} Begin a phased rollout of MFA across the organization. Prioritize the following systems:
    \begin{itemize}
        \item Email (e.g., Office 365, Google Workspace)
        \item VPN and other remote access solutions
        \item All systems containing sensitive data
    \end{itemize}
\end{enumerate}

\subsection*{High Priority Actions}
\begin{enumerate}
    \setcounter{enumi}{2} % Continue numbering from the previous list
    \item \textbf{Establish Annual Security Training:} Implement a mandatory annual security awareness training program for all employees. This program should cover topics such as phishing identification, password hygiene, and acceptable use policies.
    \item \textbf{Plan Operating System Upgrades:} Accelerate the plan to upgrade or replace all workstations running Windows 7. Create a firm timeline to eliminate all unsupported operating systems from the network.
\end{enumerate}

\end{document}
% --- Document End ---
```