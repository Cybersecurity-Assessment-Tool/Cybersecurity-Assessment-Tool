```latex
\documentclass[12pt]{article}

% Preamble: Required Packages
\usepackage[margin=1in]{geometry} % For setting page margins
\usepackage{pifont}               % For checkmarks and crosses (\ding)
\usepackage{booktabs}             % For professional-looking tables
\usepackage{hyperref}             % For clickable links, references, and a PDF outline
\usepackage{url}                  % For formatting URLs
\usepackage{seqsplit}             % For splitting long strings without spaces
\usepackage{graphicx}             % For including logos/images
\usepackage{xcolor}               % For using colors

% --- Document Setup ---
\hypersetup{
    colorlinks=true,
    linkcolor=blue,
    filecolor=magenta,      
    urlcolor=cyan,
    pdftitle={Cybersecurity Posture Assessment Report},
    pdfauthor={Cybersecurity Analyst},
    pdfsubject={Security Analysis},
    pdfkeywords={Security, Report, Analysis},
    bookmarks=true
}

% Define custom colors
\definecolor{darkblue}{rgb}{0.0, 0.0, 0.55}
\definecolor{darkred}{rgb}{0.55, 0.0, 0.0}

% --- Document Start ---
\begin{document}

% --- Title Page ---
\begin{titlepage}
    \centering
    \vfill
    \huge\bfseries Cybersecurity Posture Assessment Report
    \vspace{1.5cm}
    \Large For
    \vspace{0.5cm}
    \huge\textbf{[Organization Name]}
    \vfill
    \large
    \begin{tabular}{ll}
        \textbf{Author:} & Cybersecurity Analyst \\
        \textbf{Date:}   & \today \\
        \textbf{Status:} & Final \\
    \end{tabular}
    \vspace{1cm}
    \rule{\textwidth}{0.4pt}
    \par
    \textit{This document contains sensitive information and is intended for the exclusive use of the recipient.}
\end{titlepage}

\tableofcontents
\newpage

% --- Section 1: Executive Summary ---
\section{Executive Summary}

This report provides a comprehensive assessment of the cybersecurity posture for \textbf{[Organization Name]}, based on an analysis of organizational security controls, a technical network scan, and a review of pre-existing risk data. The assessment was conducted on \today.

The analysis revealed several critical and high-risk security gaps originating from organizational policies and procedures. Most notably, the lack of Multi-Factor Authentication (MFA) for computer logins and access to sensitive data systems presents a significant risk of unauthorized access and potential data breach. Furthermore, the absence of a formal Acceptable Use Policy (AUP) and mandatory annual security awareness training for all employees weakens the organization's human firewall, leaving it more susceptible to social engineering and insider threats.

On a positive note, the technical network scan of the designated target IP address did not identify any open ports. This is a strong indicator of a well-configured perimeter firewall. Interestingly, this finding contradicts a pre-existing risk item concerning an "Unencrypted Web Server" on port 80. This suggests that the previously identified risk may have been successfully remediated.

Key recommendations focus on immediate implementation of MFA across all critical systems, development of foundational security policies, and establishment of a recurring security training program to address the identified deficiencies and bolster the organization's overall defensive capabilities.

% --- Section 2: Organizational Information ---
\section{Organizational Information}

This section details the information provided for the assessment. Due to the anonymized nature of the input data, placeholders have been used where necessary.

\begin{itemize}
    \item \textbf{Organization Name:} \textbf{[Organization Name]}
    \item \textbf{Primary Email Domain:} \texttt{[Domain]}
    \item \textbf{Scanned External IP:} \texttt{[Client IP]}
\end{itemize}

% --- Section 3: Security Control Review ---
\section{Security Control Review}

The following table summarizes the organization's responses to a security questionnaire. Controls marked with a red cross (\ding{55}) represent significant gaps in the security posture and are addressed in the Risk Assessment section of this report.

\vspace{1em} % Add some vertical space before the table

\begin{center}
\begin{tabular}{lc}
\toprule
\textbf{Control Question} & \textbf{Status} \\
\midrule
Do you require MFA to access email? & \ding{51} \\
Do you require MFA to log into computers? & {\color{darkred}\ding{55}} \\
Do you require MFA to access sensitive data systems? & {\color{darkred}\ding{55}} \\
Does your organization have an employee acceptable use policy? & {\color{darkred}\ding{55}} \\
Does your organization do security awareness training for new employees? & \ding{51} \\
Does your organization do security awareness training for all employees at least once per year? & {\color{darkred}\ding{55}} \\
\bottomrule
\end{tabular}
\end{center}

\vspace{1em}

The findings indicate a robust control for email access but highlight critical deficiencies in endpoint security, access control for sensitive data, and foundational employee security governance.

% --- Section 4: Technical Scan Results ---
\section{Technical Scan Results}

A network scan was performed to identify externally accessible services and potential vulnerabilities.

\subsection{Scan Details}
\begin{itemize}
    \item \textbf{Target IP:} \texttt{[Target IP]}
    \item \textbf{Scan Date:} Not provided in scan data; report generated \today.
    \item \textbf{Scanner:} Nmap
\end{itemize}

\subsection{Findings}
The scan confirmed that the host at \texttt{[Target IP]} is online and responsive. 

\textbf{No open TCP ports were discovered during this assessment.} Port 80 (HTTP), which was specifically checked, was found to be in a \textbf{`closed`} state. This is a positive security finding, indicating a well-configured firewall or the absence of listening services on the scanned ports. This result directly contradicts the pre-existing risk detailed in Section 5.

% --- Section 5: Risk Assessment ---
\section{Risk Assessment}

This section synthesizes the findings from the security control review, technical scan, and pre-existing risk data into a consolidated list of identified risks.

\begin{table}[h!]
\centering
\begin{tabular}{p{0.3\linewidth} p{0.12\linewidth} p{0.5\linewidth}}
\toprule
\textbf{Risk Title} & \textbf{Severity} & \textbf{Description \& Correlation} \\
\midrule
\textbf{Lack of MFA for Sensitive Systems} & \textbf{Critical} & The absence of MFA for systems containing sensitive data creates a high risk of unauthorized access and data exfiltration if an employee's credentials are compromised. \\
\addlinespace
\textbf{Lack of MFA for Endpoint Logins} & \textbf{High} & Without MFA on computers, a compromised password could grant an attacker full access to an employee's machine, network resources, and locally stored data. \\
\addlinespace
\textbf{Missing Acceptable Use Policy (AUP)} & \textbf{High} & The lack of a formal AUP means there are no clear, enforceable rules for employees regarding the use of company assets, which can lead to risky behavior and insider threats. \\
\addlinespace
\textbf{Inadequate Security Training} & \textbf{High} & While new employees receive training, the absence of an annual refresher for all staff allows security knowledge to degrade, increasing susceptibility to phishing and social engineering attacks. \\
\addlinespace
\textbf{Unencrypted Web Server (Port 80)} & Medium (5.0) & \textbf{Contradictory Finding:} This risk was pre-existing. However, our technical scan on \texttt{[Target IP]} found Port 80 to be \textbf{closed}. This risk may have been remediated. Verification is required. \\
\bottomrule
\end{tabular}
\caption{Consolidated Risk Summary}
\label{tab:risk_summary}
\end{table}

% --- Section 6: Recommendations ---
\section{Recommendations}

Based on the risk assessment, the following actions are recommended to improve the organization's cybersecurity posture. Recommendations are prioritized by severity.

\begin{enumerate}
    \item \textbf{[Critical] Implement MFA for Sensitive Data Systems:} Immediately deploy a robust MFA solution for all applications and systems identified as containing or providing access to sensitive organizational or customer data. This is the highest priority action to prevent a significant data breach.

    \item \textbf{[High] Enforce MFA for Endpoint Logins:} Require MFA for all employee computer logins (e.g., Windows, macOS). This drastically reduces the risk of an attacker gaining a foothold in the network via compromised credentials.

    \item \textbf{[High] Develop and Implement an Acceptable Use Policy (AUP):} Draft, approve, and disseminate a formal AUP that clearly defines the rules for using company technology and data. All employees should be required to read and acknowledge the policy.

    \item \textbf{[High] Establish Annual Security Awareness Training:} Institute a mandatory security awareness training program for all employees to be completed at least once per year. This program should cover current threats such as phishing, ransomware, and social engineering.

    \item \textbf{[Medium] Validate and Update Risk Register:} Conduct a review of the pre-existing risk, "Unencrypted Web Server," to formally confirm if it has been remediated as the recent scan suggests. Update the internal risk register to accurately reflect the current state of the control.
\end{enumerate}

% --- Document End ---
\end{document}
```