Of course. As a Cybersecurity Analyst and LaTeX expert, I will analyze the provided data and generate a complete, professional report in the required LaTeX format.

Here is the synthesized report:

```latex
\documentclass[12pt, a4paper]{article}

% Preamble: Required Packages
\usepackage[margin=1in]{geometry}
\usepackage{pifont} % For checkmarks and crosses
\usepackage{booktabs} % For professional tables
\usepackage{hyperref} % For clickable links
\usepackage{url} % For formatting URLs
\usepackage{seqsplit} % For splitting long strings
\usepackage{graphicx}
\usepackage[table]{xcolor}
\usepackage{fancyhdr}

% --- Document Setup ---
\definecolor{tablegray}{gray}{0.9}
\definecolor{criticalred}{RGB}{217, 83, 79}
\definecolor{highorange}{RGB}{240, 173, 78}

\hypersetup{
    colorlinks=true,
    linkcolor=blue,
    filecolor=magenta,      
    urlcolor=cyan,
    pdftitle={Cybersecurity Assessment Report},
    pdfpagemode=FullScreen,
}

\pagestyle{fancy}
\fancyhf{}
\fancyhead[L]{Cybersecurity Assessment Report}
\fancyhead[R]{\textbf{[Organization Name]}}
\fancyfoot[C]{\thepage}

% --- Document Start ---
\begin{document}

% --- Title Page ---
\begin{titlepage}
    \centering
    \vspace*{1cm}
    \Huge\textbf{Cybersecurity Assessment Report}
    \vspace{1.5cm}
    \
    \large
    Prepared for: \\
    \vspace{0.5cm}
    \textbf{[Organization Name]}
    
    \vfill
    
    \large
    \textbf{Date of Report:} \today \\
    \textbf{Report ID:} CYBER-2023-001
    
\end{titlepage}

\tableofcontents
\newpage

% --- Section 1: Executive Summary ---
\section{Executive Summary}

This report details the findings of a cybersecurity assessment conducted for \textbf{[Organization Name]}. The assessment included a review of organizational security controls via a questionnaire and a network vulnerability scan of the designated external target.

The overall security posture reveals a mix of strengths and critical weaknesses. On a positive note, the external network scan of the target IP address \texttt{[Target IP]} did not identify any open ports or exposed services. This suggests a strong network perimeter defense or effective firewall configuration. The organization also demonstrates a commitment to security awareness training for its employees.

However, the assessment identified significant gaps in access control and internal policy that present a high degree of risk. Key findings include:
\begin{itemize}
    \item \textbf{Lack of Multi-Factor Authentication (MFA):} MFA is not enforced for logging into computers or accessing sensitive data systems. This exposes the organization to significant risk from credential theft and unauthorized access.
    \item \textbf{Absence of an Acceptable Use Policy (AUP):} The organization lacks a formal AUP, which is a foundational document for establishing security expectations and enforcing proper conduct for employees using company resources.
\end{itemize}

Immediate action is required to address these critical deficiencies. Recommendations focus on the rapid implementation of MFA across all critical systems and the development of essential security policies to mitigate these risks and strengthen the organization's overall defensive posture.

% --- Section 2: Organizational Information ---
\section{Organizational Information}

This section contains the high-level information provided for the assessment.
\begin{table}[h!]
\centering
\caption{Client Organizational Details}
\rowcolors{2}{tablegray}{white}
\begin{tabular}{ll}
\toprule
\textbf{Attribute} & \textbf{Value} \\
\midrule
Organization Name & \textbf{[Organization Name]} \\
Primary Domain & \texttt{[Domain]} \\
External IP Scanned & \texttt{[Client IP]} \\
\bottomrule
\end{tabular}
\end{table}

% --- Section 3: Security Control Review ---
\section{Security Control Review}

A security questionnaire was completed to evaluate the implementation of key administrative and technical controls. The table below summarizes the responses. A green checkmark (\textcolor{green}{\ding{51}}) indicates a positive control, while a red cross (\textcolor{red}{\ding{55}}) indicates a control gap.

\begin{table}[h!]
\centering
\caption{Questionnaire Responses and Analysis}
\label{tab:questionnaire}
\begin{tabular}{p{0.7\textwidth}c}
\toprule
\textbf{Control Question} & \textbf{Response} \\
\midrule
Do you require MFA to access email? & \textcolor{green}{\ding{51}} \\
Do you require MFA to log into computers? & \textcolor{red}{\ding{55}} \\
Do you require MFA to access sensitive data systems? & \textcolor{red}{\ding{55}} \\
Does your organization have an employee acceptable use policy? & \textcolor{red}{\ding{55}} \\
Does your organization do security awareness training for new employees? & \textcolor{green}{\ding{51}} \\
Does your organization do security awareness training for all employees at least once per year? & \textcolor{green}{\ding{51}} \\
\bottomrule
\end{tabular}
\end{table}

\subsection*{Analysis of Control Gaps}
The responses highlight critical deficiencies in identity and access management. The absence of MFA on computer logins and sensitive systems dramatically increases the risk of unauthorized access should an employee's credentials be compromised. Furthermore, the lack of an Acceptable Use Policy means there are no formally documented rules for employees regarding the use of company assets, creating potential for misuse and legal ambiguity.

% --- Section 4: Technical Scan Results ---
\section{Technical Scan Results}

An external network scan was performed to identify vulnerabilities on the public-facing infrastructure.

\begin{itemize}
    \item \textbf{Target IP Address:} \texttt{[Target IP]}
    \item \textbf{Scan Date:} Not provided in scan data.
\end{itemize}

\subsection*{Summary of Findings}
The network scan completed successfully but found \textbf{no open TCP or UDP ports} on the target host. All ports scanned appeared to be in a `closed` or `filtered` state.

\textbf{Interpretation:} This is a positive security finding. It indicates that the external firewall is properly configured to deny unsolicited inbound traffic, effectively minimizing the external attack surface of this asset. No further technical vulnerabilities were identified from this external perspective.

% --- Section 5: Overall Risk Assessment ---
\section{Overall Risk Assessment}

This section synthesizes findings from the security control review, technical scan, and pre-existing risk data. The risks listed below are derived from the identified control gaps. No pre-existing vulnerabilities were provided for this assessment.

\begin{table}[h!]
\centering
\caption{Identified Risks}
\label{tab:risks}
\begin{tabular}{p{0.15\textwidth}p{0.3\textwidth}p{0.4\textwidth}l}
\toprule
\textbf{Risk ID} & \textbf{Risk Name} & \textbf{Description} & \textbf{Severity} \\
\midrule
RISK-001 & Lack of MFA on Endpoints and Sensitive Systems & The absence of a second authentication factor for computer and sensitive system logins allows an attacker with stolen credentials to gain direct access to internal resources. & \cellcolor{criticalred}Critical \\
\addlinespace
RISK-002 & Absence of Acceptable Use Policy (AUP) & Without a formal AUP, the organization lacks an enforceable framework for governing employee use of IT assets, increasing the risk of insider threat, data misuse, and non-compliance. & \cellcolor{highorange}High \\
\bottomrule
\end{tabular}
\end{table}

% --- Section 6: Recommendations ---
\section{Recommendations}

The following actions are recommended to mitigate the identified risks and improve the overall security posture of \textbf{[Organization Name]}.

\subsection*{RISK-001: Lack of MFA (Critical)}
\begin{itemize}
    \item \textbf{Immediate Action:} Prioritize the deployment of a robust Multi-Factor Authentication solution for all employees and privileged users.
    \item \textbf{Scope:} The MFA requirement should be enforced for:
        \begin{itemize}
            \item All remote access (VPN).
            \item All local and remote logins to workstations and servers.
            \item Access to all applications storing or processing sensitive data.
        \end{itemize}
    \item \textbf{Justification:} This is the single most effective control to prevent unauthorized access resulting from compromised credentials, which is a primary vector in most cyberattacks.
\end{itemize}

\subsection*{RISK-002: Absence of AUP (High)}
\begin{itemize}
    \item \textbf{Action:} Develop and formally ratify a comprehensive Acceptable Use Policy.
    \item \textbf{Content:} The policy should clearly define rules for:
        \begin{itemize}
            \item Use of company email, internet, and network resources.
            \item Data handling and protection responsibilities.
            \item Use of personal devices (BYOD), if applicable.
            \item Prohibited activities.
            \item Consequences for policy violations.
        \end{itemize}
    \item \textbf{Implementation:} The policy must be communicated to all current employees and integrated into the onboarding process for new hires. Require all employees to formally acknowledge they have read and understood the policy.
\end{itemize}

\end{document}
```