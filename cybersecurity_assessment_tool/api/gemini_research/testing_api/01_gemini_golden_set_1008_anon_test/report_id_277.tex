```latex
\documentclass[12pt]{article}

% --- PACKAGES ---
\usepackage[margin=1in]{geometry} % Set page margins
\usepackage{pifont}               % For symbols like checkmarks (\ding{51}) and crosses (\ding{55})
\usepackage{booktabs}             % For professional-looking tables
\usepackage{hyperref}             % For clickable links and references
\usepackage{url}                  % For formatting URLs
\usepackage{seqsplit}             % To split long strings without spaces
\usepackage{graphicx}             % For images (e.g., logo)
\usepackage{array}                % For advanced table column formatting

% --- DOCUMENT SETUP ---
\hypersetup{
    colorlinks=true,
    linkcolor=black,
    urlcolor=blue,
    pdftitle={Cybersecurity Posture Report},
    pdfauthor={Cybersecurity Analysis Division}
}

% --- COMMANDS ---
\newcommand{\yes}{\ding{51}}
\newcommand{\no}{\ding{55}}

% --- DOCUMENT START ---
\begin{document}

% --- TITLE PAGE ---
\begin{titlepage}
    \centering
    \vspace*{1cm}
    \Huge\textbf{Cybersecurity Posture Report}
    \vspace{1.5cm}
    \Large
    Prepared for: \\
    \vspace{0.5cm}
    \textbf{[Organization Name]}
    \vfill
    \large
    \textbf{Date of Report:} \today \\
    \textbf{Analysis Period:} November 2025
    \vspace{1.5cm}
    \normalsize
    This document is confidential and intended solely for the use of \textbf{[Organization Name]}.
\end{titlepage}

\tableofcontents
\newpage

% --- EXECUTIVE SUMMARY ---
\section*{1. Executive Summary}
This report provides a comprehensive analysis of the cybersecurity posture of \textbf{[Organization Name]}, based on a review of organizational security controls, an external network scan conducted on November 22, 2025, and an evaluation of existing risks.

The assessment identified several critical and high-risk findings that require immediate attention. Key weaknesses were discovered in identity and access management, specifically the lack of Multi-Factor Authentication (MFA) for computer and sensitive data access. Furthermore, foundational security policies and training programs are incomplete, increasing the organization's susceptibility to human-centric attacks.

From a technical perspective, the external network scan revealed a public-facing web server running an outdated and potentially vulnerable version of nginx. These combined findings indicate a significant risk of unauthorized access and potential system compromise. This report outlines these risks in detail and provides actionable recommendations to mitigate them and strengthen the overall security posture.

% --- ORGANIZATIONAL INFORMATION ---
\section*{2. Organizational Information}
The following details, provided or derived for this assessment, establish the context for the findings. Due to missing data, placeholders have been used.
\begin{itemize}
    \item \textbf{Organization Name:} \textbf{[Organization Name]}
    \item \textbf{Primary Email Domain:} \texttt{[Domain]}
    \item \textbf{Assessed External IP Address:} \texttt{[Client IP]}
\end{itemize}

% --- SECURITY CONTROL REVIEW ---
\section*{3. Security Control Review}
A review of organizational security controls was conducted via a questionnaire. The responses, summarized below, indicate several significant gaps in foundational security practices.

\begin{center}
\begin{tabular}{p{0.7\textwidth} c}
\toprule
\textbf{Control Question} & \textbf{Response} \\
\midrule
Do you require MFA to access email? & \yes \\
Do you require MFA to log into computers? & \no \\
Do you require MFA to access sensitive data systems? & \no \\
Does your organization have an employee acceptable use policy? & \no \\
Does your organization do security awareness training for new employees? & \yes \\
Does your organization do security awareness training for all employees at least once per year? & \no \\
\bottomrule
\end{tabular}
\end{center}

\subsection*{Analysis of Control Gaps}
The "No" responses highlight critical deficiencies. The lack of MFA for computer and sensitive data access represents a primary vector for account takeover attacks. The absence of an Acceptable Use Policy and mandatory annual security training for all employees increases the risk of insider threats, policy violations, and successful social engineering attacks like phishing.

% --- TECHNICAL SCAN RESULTS ---
\section*{4. Technical Scan Results}
An external network vulnerability scan was performed to identify exposed services and potential vulnerabilities on the perimeter.

\begin{itemize}
    \item \textbf{Scan Date:} 2025-11-22
    \item \textbf{Target IP Address:} \texttt{[Target IP]}
\end{itemize}

\subsection*{Open Ports and Services}
The following table details the open ports and services discovered on the target system.
\begin{center}
\begin{tabular}{l l l l l}
\toprule
\textbf{Port} & \textbf{State} & \textbf{Service} & \textbf{Product} & \textbf{Version} \\
\midrule
443/tcp & open & https & nginx & 1.18.0 \\
\bottomrule
\end{tabular}
\end{center}

\subsection*{Technical Analysis}
The scan identified a single open port, 443 (HTTPS), running an nginx web server, version \textbf{1.18.0}. This version was released in April 2020 and is now considered outdated. It is known to be affected by several publicly disclosed vulnerabilities (e.g., CVE-2021-23017). Running outdated software on a public-facing server presents a high risk of exploitation, which could lead to unauthorized access, data exposure, or a full system compromise.

% --- RISK ASSESSMENT ---
\section*{5. Identified Risks and Assessment}
The following table synthesizes findings from the security control review and the technical scan into a prioritized list of risks. No pre-existing vulnerabilities were reported in the provided data.

\begin{center}
\begin{tabular}{p{0.15\textwidth} p{0.6\textwidth} l}
\toprule
\textbf{Risk ID} & \textbf{Description} & \textbf{Severity} \\
\midrule
RISK-001 & \textbf{Lack of MFA on Critical Systems:} User accounts for computers and sensitive data systems are not protected by Multi-Factor Authentication, significantly increasing the risk of compromise from stolen credentials. & Critical \\
\addlinespace
RISK-002 & \textbf{Outdated Public-Facing Web Server:} The external web server runs an outdated version of nginx (1.18.0) with known vulnerabilities, exposing the organization to remote automated attacks and targeted exploitation. & High \\
\addlinespace
RISK-003 & \textbf{Inadequate Security Policies and Training:} The absence of an Acceptable Use Policy and mandatory annual security training weakens the organization's defense against human error, insider threats, and phishing. & High \\
\bottomrule
\end{tabular}
\end{center}

% --- RECOMMENDATIONS ---
\section*{6. Recommendations}
To address the identified risks and improve the overall security posture of \textbf{[Organization Name]}, the following actions are recommended with urgency.

\begin{enumerate}
    \item \textbf{Implement Comprehensive MFA (Addresses RISK-001):}
    \begin{itemize}
        \item \textbf{Action:} Prioritize the deployment of a robust Multi-Factor Authentication solution for all employees to access company computers and any systems containing sensitive or confidential data.
        \item \textbf{Impact:} This is the most critical step to prevent unauthorized access resulting from credential theft.
    \end{itemize}
    
    \item \textbf{Upgrade Web Server Software (Addresses RISK-002):}
    \begin{itemize}
        \item \textbf{Action:} Immediately schedule and perform an upgrade of the nginx server from version 1.18.0 to the latest stable version. Implement a formal patch management process to ensure all public-facing systems are kept up-to-date.
        \item \textbf{Impact:} Mitigates the risk of exploitation from known vulnerabilities.
    \end{itemize}
    
    \item \textbf{Develop and Implement Security Policies (Addresses RISK-003):}
    \begin{itemize}
        \item \textbf{Action:} Create a formal Acceptable Use Policy (AUP) that all employees must read and acknowledge upon hire and annually thereafter. This policy should clearly define the rules for using company IT assets.
        \item \textbf{Impact:} Establishes a baseline for secure behavior and provides grounds for enforcement.
    \end{itemize}
    
    \item \textbf{Establish a Security Awareness Program (Addresses RISK-003):}
    \begin{itemize}
        \item \textbf{Action:} Implement a mandatory, annual security awareness training program for all employees. This program should cover critical topics such as phishing identification, password security, and social engineering tactics.
        \item \textbf{Impact:} Reduces the likelihood of security incidents caused by human error.
    \end{itemize}
\end{enumerate}

\end{document}
```