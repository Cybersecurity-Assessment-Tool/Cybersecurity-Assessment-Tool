```latex
\documentclass[12pt]{article}

% Preamble: Required Packages
\usepackage[margin=1in]{geometry}
\usepackage{pifont} % For checkmarks and crosses
\usepackage{booktabs} % For professional tables
\usepackage{hyperref} % For clickable links
\usepackage{url} % For URL formatting
\usepackage{seqsplit} % For splitting long strings
\usepackage{graphicx}
\usepackage{xcolor}
\usepackage{fancyhdr}

% Document Metadata and Styling
\hypersetup{
    colorlinks=true,
    linkcolor=blue,
    filecolor=magenta,      
    urlcolor=cyan,
    pdftitle={Cybersecurity Posture Report},
    pdfpagemode=FullScreen,
}

\pagestyle{fancy}
\fancyhf{}
\lhead{Cybersecurity Posture Report}
\rhead{\textbf{[Organization Name]}}
\cfoot{\thepage}

\begin{document}

% --- Title Page ---
\begin{titlepage}
    \centering
    \vspace*{1cm}
    \Huge\textbf{Cybersecurity Posture Report}
    \vspace{1.5cm}
    \Large
    \textbf{Prepared for:}\\
    \vspace{0.5cm}
    \textbf{[Organization Name]}\\
    \vspace{2cm}
    \textbf{Date of Report:}\\
    \vspace{0.5cm}
    \today
    \vfill
    \large
    \textit{This report contains sensitive information and should be handled with care.}
\end{titlepage}

\tableofcontents
\newpage

% --- Section 1: Executive Summary ---
\section{Executive Summary}

This report provides a comprehensive analysis of the cybersecurity posture for \textbf{[Organization Name]}, based on a review of administrative controls, pre-existing risks, and a technical network scan conducted on \today.

The assessment reveals a mixed security posture. On the one hand, the external network perimeter appears strong, with no open ports discovered on the scanned target IP address. This suggests a well-configured firewall and adherence to the principle of least privilege for external-facing services. Additionally, the organization has successfully implemented multi-factor authentication (MFA) for critical systems like email and sensitive data access, which significantly reduces the risk of account compromise for those assets.

However, two significant gaps were identified in foundational security controls. The absence of mandatory MFA for computer logins presents a \textbf{Critical} risk, as it leaves endpoints vulnerable to unauthorized access if user credentials are compromised. Furthermore, the lack of a formal Employee Acceptable Use Policy (AUP) is a \textbf{High} risk, creating ambiguity regarding the proper use of company assets and increasing insider threat and compliance risks.

Immediate remediation should focus on closing these administrative and identity management gaps to build upon the existing technical strengths and create a more resilient security environment.

% --- Section 2: Organizational Information ---
\section{Organizational Information}

This section details the scope and target of the assessment. The information provided is based on the data supplied for this analysis.

\begin{table}[h!]
\centering
\begin{tabular}{@{}ll@{}}
\toprule
\textbf{Identifier} & \textbf{Value} \\ \midrule
Organization Name   & \textbf{[Organization Name]} \\
Primary Domain      & \texttt{[Domain]} \\
Client Public IP    & \texttt{[Client IP]} \\
Scanned Target IP   & \texttt{[Target IP]} \\ \bottomrule
\end{tabular}
\caption{Assessment Scope and Target Details.}
\end{table}

% --- Section 3: Security Control Review ---
\section{Security Control Review}

A review of key administrative and technical security controls was conducted via a questionnaire. The following table summarizes the organization's responses and provides an assessment of each control's status. "No" answers indicate significant gaps that require attention.

\begin{table}[h!]
\centering
\begin{tabular}{@{}p{0.6\linewidth}cc@{}}
\toprule
\textbf{Control Question} & \textbf{Response} & \textbf{Assessment} \\ \midrule
Do you require MFA to access email? & \ding{51} & Good Practice \\
\addlinespace
Do you require MFA to log into computers? & \textbf{\color{red}\ding{55}} & \textbf{Critical Gap} \\
\addlinespace
Do you require MFA to access sensitive data systems? & \ding{51} & Good Practice \\
\addlinespace
Does your organization have an employee acceptable use policy? & \textbf{\color{red}\ding{55}} & \textbf{High Risk} \\
\addlinespace
Does your organization do security awareness training for new employees? & \ding{51} & Good Practice \\
\addlinespace
Does your organization do security awareness training for all employees at least once per year? & \ding{51} & Good Practice \\ \bottomrule
\end{tabular}
\caption{Security Controls Questionnaire Analysis.}
\end{table}

% --- Section 4: Technical Scan Results ---
\section{Technical Scan Results}

An external network scan was performed to identify open ports and exposed services on the target system.

\subsection{External Network Scan (Nmap)}
\begin{itemize}
    \item \textbf{Target IP Address:} \texttt{[Target IP]}
    \item \textbf{Scan Summary:} The scan completed successfully and probed for common TCP ports.
    \item \textbf{Findings:}
    \begin{itemize}
        \item \textbf{Open Ports: 0.} No open TCP ports were discovered on the target system.
        \item \textbf{Port State:} All other scanned ports were reported as 'closed', meaning they are accessible but have no application listening on them.
    \end{itemize}
    \item \textbf{Assessment:} The absence of open ports indicates a strong external security posture. This configuration effectively minimizes the attack surface accessible from the public internet and suggests a properly configured firewall that denies all unsolicited inbound traffic by default.
\end{itemize}

% --- Section 5: Risk Assessment ---
\section{Risk Assessment}

This section synthesizes findings from the security control review and technical scan to provide a consolidated list of identified risks. The pre-existing risk register was empty, so the following table is composed entirely of new findings from this assessment.

\begin{table}[h!]
\centering
\begin{tabular}{@{}p{0.25\linewidth}p{0.5\linewidth}l@{}}
\toprule
\textbf{Risk Name} & \textbf{Description} & \textbf{Severity} \\ \midrule
\addlinespace
Lack of MFA on Workstations & Users can log into company computers without a second factor of authentication. This elevates the risk of unauthorized access and lateral movement resulting from a single compromised password. & \textbf{Critical} \\
\addlinespace
Missing Acceptable Use Policy (AUP) & The absence of a formal policy creates ambiguity for employees regarding the protection and use of company assets and data. This increases insider threat, compliance, and legal risks. & \textbf{High} \\
\addlinespace
\bottomrule
\end{tabular}
\caption{Consolidated Risk Summary.}
\end{table}

% --- Section 6: Recommendations ---
\section{Recommendations}

The following actions are recommended to mitigate the identified risks and improve the overall security posture of \textbf{[Organization Name]}. Recommendations are prioritized based on the severity of the associated risk.

\begin{enumerate}
    \item \textbf{Implement Mandatory MFA for Workstation Logins (Critical):}
    \begin{itemize}
        \item \textbf{Action:} Deploy a robust Multi-Factor Authentication (MFA) solution for all employee and privileged user logins to company workstations and servers.
        \item \textbf{Justification:} This is the single most effective control to prevent unauthorized access from stolen or weak credentials, which is a primary vector for ransomware and data breach incidents.
        \item \textbf{Examples:} Solutions like Windows Hello for Business, Duo Security, or other identity providers can be integrated with Active Directory or Azure AD to enforce this control.
    \end{itemize}
    \vspace{1em}
    \item \textbf{Develop and Implement an Acceptable Use Policy (High):}
    \begin{itemize}
        \item \textbf{Action:} Draft, approve, and implement a comprehensive Acceptable Use Policy (AUP) that governs the use of all company IT assets, including networks, computers, and data.
        \item \textbf{Justification:} An AUP establishes clear expectations for employees, provides a basis for disciplinary action in case of misuse, and is a foundational component of most cybersecurity and data privacy compliance frameworks (e.g., ISO 27001, SOC 2).
        \item \textbf{Implementation:} The policy should be reviewed by legal counsel, communicated to all current employees, and integrated into the onboarding process for new hires. Require all employees to sign an acknowledgment form indicating they have read and understood the policy.
    \end{itemize}
\end{enumerate}

\end{document}
```