```latex
\documentclass[12pt]{article}

% Preamble: Required Packages
\usepackage[margin=1in]{geometry}
\usepackage{pifont} % For symbols like checkmarks and crosses
\usepackage{booktabs} % For professional-looking tables
\usepackage{hyperref} % For clickable links and table of contents
\usepackage{url}      % For properly formatting URLs
\usepackage{seqsplit} % For breaking long strings in tt font
\usepackage{xcolor}   % For custom colors
\usepackage{graphicx} % For potential logos

% --- Document Setup ---
\hypersetup{
    colorlinks=true,
    linkcolor=blue,
    filecolor=magenta,      
    urlcolor=cyan,
    pdftitle={Cybersecurity Posture Assessment Report},
    pdfauthor={Cybersecurity Analyst},
}

% --- Custom Commands ---
\newcommand{\yes}{\ding{51}} % Green checkmark
\newcommand{\no}{\ding{55}}  % Red cross

% --- Document Title ---
\title{Cybersecurity Posture Assessment Report \\ \large For \textbf{[Organization Name]}}
\author{Cybersecurity Analyst}
\date{\today}


\begin{document}

\maketitle
\thispagestyle{empty}
\newpage

\tableofcontents
\newpage

% ==============================================================================
\section{Executive Summary}
% ==============================================================================

This report details the findings of a cybersecurity posture assessment conducted for \textbf{[Organization Name]}. The assessment incorporated a review of organizational security controls, an external network vulnerability scan, and an analysis of pre-existing risks.

The organization demonstrates a solid foundation in security awareness, with established policies for acceptable use and mandatory training for all employees. Furthermore, Multi-Factor Authentication (MFA) is correctly enforced for computer and sensitive data system access.

However, a critical security gap was identified: the \textbf{absence of mandatory MFA for email access}. Email is a primary vector for sophisticated attacks such as phishing, business email compromise (BEC), and account takeover. This gap exposes the organization to significant risk of unauthorized access, data breaches, and financial loss.

The external network scan of the designated target IP address did not identify any open ports. This suggests a properly configured firewall is in place, or the host was not responsive at the time of the scan. No pre-existing vulnerabilities were reported for analysis.

The primary recommendation of this report is the immediate implementation of MFA across all email accounts to mitigate this critical risk.

% ==============================================================================
\section{Organizational Information}
% ==============================================================================

The following information was used as the basis for this assessment. Due to the anonymized nature of the data provided, placeholders have been used.

\begin{itemize}
    \item \textbf{Organization Name:} \textbf{[Organization Name]}
    \item \textbf{Primary Domain:} \texttt{[Domain]}
    \item \textbf{Target IP for Scan:} \texttt{[Client IP]}
\end{itemize}

% ==============================================================================
\section{Security Control Review}
% ==============================================================================

The following table summarizes the organization's responses to the security controls questionnaire. A checkmark (\yes) indicates a positive control is in place, while a cross (\no) indicates a potential security gap that requires attention.

\begin{table}[h!]
\centering
\caption{Security Controls Questionnaire Results}
\begin{tabular}{p{0.8\textwidth}c}
\toprule
\textbf{Control Question} & \textbf{Status} \\
\midrule
Do you require MFA to access email? & \no \\
Do you require MFA to log into computers? & \yes \\
Do you require MFA to access sensitive data systems? & \yes \\
Does your organization have an employee acceptable use policy? & \yes \\
Does your organization do security awareness training for new employees? & \yes \\
Does your organization do security awareness training for all employees at least once per year? & \yes \\
\bottomrule
\end{tabular}
\end{table}

\subsection*{Analysis}
The questionnaire reveals a significant weakness. The lack of MFA on email is a critical oversight, as email accounts are high-value targets for attackers. A compromised email account can serve as a launchpad for internal phishing campaigns, unauthorized access to sensitive data shared via email, and password reset attacks on other integrated services.

% ==============================================================================
\section{Technical Scan Results}
% ==============================================================================

An external network scan was performed to identify open ports and exposed services on the organization's perimeter.

\begin{itemize}
    \item \textbf{Target IP Address:} \texttt{[Target IP]}
    \item \textbf{Scan Date:} Data not provided in scan results.
    \item \textbf{Findings:} The scan completed without discovering any open TCP or UDP ports. 
\end{itemize}

\subsection*{Analysis}
The absence of open ports is a positive security finding. It indicates that the external firewall is likely configured with a default-deny policy, only allowing traffic for specific, intended services, none of which were exposed to the public internet from the scanning source. This significantly reduces the external attack surface.

% ==============================================================================
\section{Risk Assessment}
% ==============================================================================

This section synthesizes findings from the security control review, technical scans, and pre-existing risk data. Based on the provided inputs, one new critical risk has been identified. No pre-existing risks were reported.

\begin{table}[h!]
\centering
\caption{Identified Risks}
\begin{tabular}{lp{0.6\textwidth}l}
\toprule
\textbf{Risk ID} & \textbf{Description} & \textbf{Severity} \\
\midrule
RISK-001 & \textbf{Lack of MFA on Email Accounts:} The absence of a second authentication factor for email access allows an attacker with valid credentials (e.g., from a phishing attack or password reuse) to gain full access to an employee's mailbox. & \textbf{Critical} \\
\bottomrule
\end{tabular}
\end{table}

\subsection*{Impact Analysis for RISK-001}
A compromised email account can lead to:
\begin{itemize}
    \item \textbf{Data Exfiltration:} Attackers can access and steal sensitive information contained in emails and attachments.
    \item \textbf{Business Email Compromise (BEC):} Attackers can impersonate employees to authorize fraudulent wire transfers or deceive other employees into revealing sensitive information.
    \item \textbf{Lateral Movement:} The compromised account can be used to reset passwords for other corporate applications, expanding the attacker's foothold within the network.
    \item \textbf{Reputational Damage:} A public breach originating from a compromised email account can severely damage the organization's reputation with clients and partners.
\end{itemize}

% ==============================================================================
\section{Recommendations}
% ==============================================================================

The following actions are recommended to mitigate the identified risks and improve the overall security posture of \textbf{[Organization Name]}.

\begin{enumerate}
    \item \textbf{Implement MFA for Email Access (Critical Priority):}
    \begin{itemize}
        \item \textbf{Action:} Immediately enable and enforce MFA for all user accounts on the organization's email platform (e.g., Microsoft 365, Google Workspace).
        \item \textbf{Justification:} This is the single most effective control to prevent unauthorized access to email accounts, directly mitigating RISK-001. It protects against password spraying, credential stuffing, and phishing attacks.
    \end{itemize}
    \vspace{1em}
    
    \item \textbf{Conduct Regular Security Control Audits (High Priority):}
    \begin{itemize}
        \item \textbf{Action:} Establish a quarterly or semi-annual process to review security control questionnaires and validate that policies are being enforced as intended.
        \item \textbf{Justification:} This proactive measure ensures that security gaps, like the one identified with email MFA, are discovered and remediated before they can be exploited.
    \end{itemize}
    \vspace{1em}
    
    \item \textbf{Perform Periodic External Penetration Testing (Medium Priority):}
    \begin{itemize}
        \item \textbf{Action:} While the initial scan showed no open ports, a comprehensive penetration test should be conducted annually by a third party.
        \item \textbf{Justification:} A full penetration test goes beyond a simple port scan, actively attempting to exploit potential misconfigurations or vulnerabilities that may not be immediately apparent. This provides a higher level of assurance about the security of the network perimeter.
    \end{itemize}
\end{enumerate}

\end{document}
```