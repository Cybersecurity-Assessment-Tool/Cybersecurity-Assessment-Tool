```latex
\documentclass[12pt]{article}

% Preamble: Required Packages
\usepackage[margin=1in]{geometry}
\usepackage{pifont} % For checkmarks and crosses
\usepackage{booktabs} % For professional tables
\usepackage{hyperref} % For hyperlinks
\usepackage{url} % For URL formatting
\usepackage{seqsplit} % For splitting long strings
\usepackage{graphicx}
\usepackage{xcolor}

% Hyperref Setup
\hypersetup{
    colorlinks=true,
    linkcolor=blue,
    filecolor=magenta,      
    urlcolor=cyan,
    pdftitle={Cybersecurity Posture Report},
    pdfpagemode=FullScreen,
}

% Define custom colors
\definecolor{darkred}{rgb}{0.55, 0.0, 0.0}
\definecolor{darkorange}{rgb}{0.8, 0.33, 0.0}

% Document Start
\begin{document}

% --- Title Page ---
\begin{titlepage}
    \centering
    \vspace*{1cm}
    \Huge\textbf{Cybersecurity Posture Report}
    \vspace{1.5cm}
    \Large
    \textbf{Prepared for:} \\
    \vspace{0.5cm}
    \textbf{[Organization Name]}
    \vspace{2cm}
    \large
    \textbf{Date of Report:} \today \\
    \vspace{0.5cm}
    \textbf{Scan Date:} [Scan Date]
    \vfill
    \large
    \textbf{Generated By:} \\
    Expert-Level Cybersecurity Analyst
\end{titlepage}

\tableofcontents
\newpage

% --- Section 1: Executive Summary ---
\section{Executive Summary}
This report provides a comprehensive analysis of the cybersecurity posture for \textbf{[Organization Name]}, based on network scans, a security controls questionnaire, and a review of pre-existing risks. The assessment reveals several critical and high-risk vulnerabilities that require immediate attention.

Key findings indicate significant gaps in fundamental security controls, including the lack of mandatory Multi-Factor Authentication (MFA) for computer logins and the absence of an employee Acceptable Use Policy. These policy-based weaknesses are compounded by technical findings, such as an exposed Secure Shell (SSH) service on the external network. Furthermore, a pre-existing critical vulnerability, "Localhost Exposed," with a CVSS score of 10.0, remains an immediate threat to the organization.

The overall security posture is assessed as \textbf{High-Risk}. The recommendations provided in this report are prioritized to address the most severe threats first, with a focus on strengthening access controls, formalizing security policies, and remediating known critical vulnerabilities.

% --- Section 2: Organizational Information ---
\section{Organizational Information}
This section outlines the basic information used as the basis for this assessment. Due to the anonymized nature of the input data, placeholders have been used where necessary.

\begin{itemize}
    \item \textbf{Organization Name:} \textbf{[Organization Name]}
    \item \textbf{Primary Domain:} \texttt{[Domain]}
    \item \textbf{External IP Address Scanned:} \texttt{[Client IP]}
\end{itemize}

% --- Section 3: Security Control Review ---
\section{Security Control Review (Questionnaire Analysis)}
The following table details the organization's responses to a security controls questionnaire. Each response is assessed against industry best practices. "No" answers represent significant gaps in the security framework and are highlighted as risks.

\begin{table}[h!]
\centering
\caption{Security Controls Questionnaire Analysis}
\label{tab:controls}
\begin{tabular}{@{}p{0.55\textwidth} c p{0.25\textwidth}@{}}
\toprule
\textbf{Control Question} & \textbf{Response} & \textbf{Assessment} \\
\midrule
Do you require MFA to access email? & \ding{51} & Compliant \\
\addlinespace
Do you require MFA to log into computers? & \textbf{\color{darkred}\ding{55}} & \textbf{Critical Gap}. Lack of MFA on endpoints significantly increases the risk of lateral movement after an initial compromise. \\
\addlinespace
Do you require MFA to access sensitive data systems? & \ding{51} & Compliant \\
\addlinespace
Does your organization have an employee acceptable use policy? & \textbf{\color{darkred}\ding{55}} & \textbf{High Risk}. Absence of a formal policy creates ambiguity and legal exposure regarding employee use of company assets. \\
\addlinespace
Does your organization do security awareness training for new employees? & \ding{51} & Compliant \\
\addlinespace
Does your organization do security awareness training for all employees at least once per year? & \textbf{\color{darkorange}\ding{55}} & \textbf{High Risk}. Without regular training, employees are more susceptible to evolving threats like phishing and social engineering. \\
\bottomrule
\end{tabular}
\end{table}

% --- Section 4: Technical Scan Results ---
\section{Technical Scan Results}
A network scan was performed on the target system to identify open ports and exposed services.

\subsection{Nmap Scan Findings}
\begin{itemize}
    \item \textbf{Target IP Address:} \texttt{[Target IP]}
    \item \textbf{Host Status:} Up
\end{itemize}

The following table summarizes the open ports discovered on the target system.

\begin{table}[h!]
\centering
\caption{Open Ports on \texttt{[Target IP]}}
\label{tab:ports}
\begin{tabular}{@{}llll@{}}
\toprule
\textbf{Port} & \textbf{State} & \textbf{Service (Probable)} & \textbf{Notes} \\
\midrule
22/tcp & open & SSH (Secure Shell) & Exposed to the public internet. This is a common vector for brute-force attacks. Service version information was not available from the scan data. \\
\bottomrule
\end{tabular}
\end{table}

% --- Section 5: Consolidated Risk Assessment ---
\section{Consolidated Risk Assessment}
This section synthesizes findings from the questionnaire, technical scans, and pre-existing risk data into a consolidated list. Each risk is categorized by severity to guide remediation efforts.

\begin{table}[h!]
\centering
\caption{Summary of Identified Risks}
\label{tab:risks}
\begin{tabular}{@{}p{0.4\textwidth} p{0.3\textwidth} l@{}}
\toprule
\textbf{Risk Description} & \textbf{Source} & \textbf{Severity} \\
\midrule
\textbf{Localhost Exposed} (CVSS 10.0) & Pre-existing Risk Data & \textbf{\color{darkred}Critical} \\
\addlinespace
Lack of MFA for computer logins & Questionnaire & \textbf{\color{darkred}Critical} \\
\addlinespace
Exposed SSH (Port 22) on external network & Network Scan & \textbf{\color{darkorange}High} \\
\addlinespace
No formal Acceptable Use Policy (AUP) & Questionnaire & \textbf{\color{darkorange}High} \\
\addlinespace
No annual security awareness training for all staff & Questionnaire & \textbf{\color{darkorange}High} \\
\bottomrule
\end{tabular}
\end{table}

% --- Section 6: Recommendations ---
\section{Recommendations}
The following actionable recommendations are prioritized based on the risk assessment. Addressing these items will significantly improve the organization's security posture.

\subsection{Priority 1: Critical Risks}
\begin{enumerate}
    \item \textbf{Remediate "Localhost Exposed" Vulnerability:}
    \begin{itemize}
        \item \textbf{Action:} Immediately investigate and remediate the pre-existing critical vulnerability identified as "Localhost Exposed". Given the CVSS score of 10.0, this should be the highest priority.
        \item \textbf{Impact:} Prevents catastrophic compromise of the affected system.
    \end{itemize}
    \item \textbf{Implement MFA on All Computer Logins:}
    \begin{itemize}
        \item \textbf{Action:} Deploy a mandatory MFA solution for all employee and administrative logins to workstations and servers.
        \item \textbf{Impact:} Drastically reduces the risk of unauthorized access and lateral movement resulting from compromised credentials.
    \end{itemize}
\end{enumerate}

\subsection{Priority 2: High Risks}
\begin{enumerate}
    \item \textbf{Secure the Exposed SSH Service:}
    \begin{itemize}
        \item \textbf{Action:} If SSH access is required from the internet, restrict access to known, trusted IP addresses (whitelisting). Disable password-based authentication and enforce the use of strong cryptographic keys. If external access is not required, block port 22 at the firewall.
        \item \textbf{Impact:} Protects against brute-force attacks and unauthorized access to a critical management interface.
    \end{itemize}
    \item \textbf{Develop and Implement an Acceptable Use Policy (AUP):}
    \begin{itemize}
        \item \textbf{Action:} Draft, approve, and disseminate a formal AUP that clearly defines the rules and responsibilities for all users of the organization's IT resources. Require all employees to read and acknowledge the policy.
        \item \textbf{Impact:} Establishes a clear security baseline for user behavior and reduces legal and operational risk.
    \end{itemize}
    \item \textbf{Establish an Annual Security Awareness Training Program:}
    \begin{itemize}
        \item \textbf{Action:} Implement a mandatory, recurring security awareness training program for all employees. The training should cover current threats such as phishing, malware, and social engineering.
        \item \textbf{Impact:} Strengthens the "human firewall" and reduces the likelihood of security incidents caused by human error.
    \end{itemize}
\end{enumerate}

\end{document}
```