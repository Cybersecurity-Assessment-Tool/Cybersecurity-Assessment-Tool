As an expert-level Cybersecurity Analyst and LaTeX Report Generator, I will now analyze the provided data and synthesize it into a comprehensive and professional security report.

Despite the corruption of the network scan and current risks data, a meaningful analysis can still be conducted based on the organizational questionnaire. The report will clearly state the limitations imposed by the corrupted data and focus on the actionable insights derived from the available information.

Here is the complete LaTeX report:

```latex
\documentclass[12pt]{article}

% === PACKAGES ===
\usepackage[margin=1in]{geometry} % Set margins
\usepackage{pifont}               % For checkmarks and crosses (\ding)
\usepackage{booktabs}             % For professional tables (\toprule, \midrule, \bottomrule)
\usepackage{hyperref}             % For clickable links and metadata
\usepackage{url}                  % For formatting URLs
\usepackage{seqsplit}             % For splitting long strings in texttt
\usepackage{graphicx}             % For logos, etc. (optional)
\usepackage{xcolor}               % For colors

% === DOCUMENT METADATA ===
\hypersetup{
    colorlinks=true,
    linkcolor=blue,
    filecolor=magenta,      
    urlcolor=cyan,
    pdftitle={Cybersecurity Posture Assessment Report},
    pdfauthor={Cybersecurity Analyst},
    pdfsubject={Security Assessment},
    pdfkeywords={Security, Analysis, Report},
}

% === CUSTOM COMMANDS ===
\newcommand{\yes}{\ding{51}} % Green checkmark
\newcommand{\no}{\ding{55}}  % Red cross

% === TITLE SECTION ===
\title{
    \vspace{-2cm}
    \rule{\textwidth}{1pt} \\ [0.5cm]
    \textbf{Cybersecurity Posture Assessment Report} \\ [0.2cm]
    \rule{\textwidth}{1pt}
}
\author{Cybersecurity Analyst}
\date{\today}

% ==============================================================================
% === BEGIN DOCUMENT ===
% ==============================================================================
\begin{document}

\maketitle
\thispagestyle{empty}
\newpage

\tableofcontents
\newpage

% ==============================================================================
% 1. EXECUTIVE SUMMARY
% ==============================================================================
\section{Executive Summary}

This report provides an assessment of the cybersecurity posture for \textbf{[Organization Name]}. The analysis is based on three data sources: a self-reported security controls questionnaire, a technical network scan, and a list of pre-existing risks. 

\textbf{Important Note on Data Integrity:} During the analysis process, it was determined that the data feeds for the \textbf{Technical Network Scan (Input 1)} and the \textbf{Current Risks (Input 3)} were corrupted and could not be parsed. Consequently, this assessment is primarily based on the \textbf{Security Controls Questionnaire (Input 2)}. A new technical scan is strongly recommended to identify network-based vulnerabilities.

The questionnaire revealed several critical gaps in the organization's security controls. Key findings include:
\begin{itemize}
    \item \textbf{Lack of Endpoint Multi-Factor Authentication (MFA):} User computers are not protected by MFA, creating a significant risk of unauthorized access if credentials are compromised.
    \item \textbf{Absence of an Acceptable Use Policy (AUP):} Without a formal AUP, there are no established rules for employee use of company assets, leading to potential misuse and insider threats.
    \item \textbf{Inadequate Onboarding Security Training:} New employees do not receive security awareness training, leaving them vulnerable to social engineering attacks from their first day.
\end{itemize}

These findings indicate a reactive rather than proactive security stance. The recommendations in this report focus on establishing foundational security controls to mitigate these high-priority risks and build a stronger security culture.

% ==============================================================================
% 2. ORGANIZATIONAL INFORMATION
% ==============================================================================
\section{Organizational Information}

The following details were used as the basis for this assessment. Due to the anonymized nature of the provided data, placeholders have been used where necessary.

\begin{itemize}
    \item \textbf{Organization Name:} \textbf{[Organization Name]}
    \item \textbf{Primary Email Domain:} \texttt{[Domain]}
    \item \textbf{External IP Scanned:} \texttt{[Client IP]}
\end{itemize}


% ==============================================================================
% 3. SECURITY CONTROL REVIEW (QUESTIONNAIRE)
% ==============================================================================
\section{Security Control Review (Questionnaire Analysis)}

The following table summarizes the organization's self-reported status on key security controls. Items marked with a red \no\ represent significant gaps that increase organizational risk.

\begin{table}[h!]
\centering
\caption{Security Controls Questionnaire Results}
\begin{tabular}{p{0.6\textwidth} c l}
\toprule
\textbf{Control Question} & \textbf{Response} & \textbf{Assessment} \\
\midrule
Do you require MFA to access email? & \yes & Good Practice \\
\addlinespace
Do you require MFA to log into computers? & \no & \textbf{Critical Gap} \\
\addlinespace
Do you require MFA to access sensitive data systems? & \yes & Good Practice \\
\addlinespace
Does your organization have an employee acceptable use policy? & \no & \textbf{High Risk Gap} \\
\addlinespace
Does your organization do security awareness training for new employees? & \no & \textbf{High Risk Gap} \\
\addlinespace
Does your organization do security awareness training for all employees at least once per year? & \yes & Good Practice \\
\bottomrule
\end{tabular}
\end{table}


% ==============================================================================
% 4. TECHNICAL SCAN RESULTS
% ==============================================================================
\section{Technical Scan Results}

A network port scan was intended to be performed against the target IP address \texttt{[Target IP]}.

\textbf{Data Corruption Notice:} The results file from the network scan (\texttt{Input\_1\_Network\_Scan\_JSON}) was found to be malformed or corrupted. Therefore, a technical analysis of open ports, running services, and potential vulnerabilities could not be completed. 

A properly executed scan would typically identify the following:
\begin{itemize}
    \item Open TCP/UDP ports and their associated services.
    \item Software product names and version numbers.
    \item Outdated software versions with known public vulnerabilities (CVEs).
    \item Insecure service configurations (e.g., FTP, Telnet).
\end{itemize}

\textbf{It is imperative that a new network scan be conducted to obtain this critical visibility into the organization's external attack surface.}


% ==============================================================================
% 5. RISK ASSESSMENT
% ==============================================================================
\section{Risk Assessment}

This risk assessment is based on the findings from the Security Control Review. The pre-existing risk data (\texttt{Input\_3\_Current\_Risks\_JSON}) was unavailable due to data corruption. The following table details the newly identified risks.

\begin{table}[h!]
\centering
\caption{Summary of Identified Risks}
\begin{tabular}{p{0.15\textwidth} p{0.3\textwidth} p{0.35\textwidth} p{0.1\textwidth}}
\toprule
\textbf{Risk ID} & \textbf{Risk Name} & \textbf{Description} & \textbf{Severity} \\
\midrule
RISK-001 & Lack of Endpoint MFA & The absence of MFA on computer logins means a compromised password is all an attacker needs to gain access to an employee's machine and potentially the network. & \textbf{Critical} \\
\addlinespace
RISK-002 & Missing Acceptable Use Policy (AUP) & Without a formal policy, employees are unaware of their responsibilities regarding data handling and system usage, increasing the risk of insider threats and accidental data loss. & High \\
\addlinespace
RISK-003 & Inadequate New Hire Security Training & New employees are a primary target for phishing and social engineering. Failing to train them upon hiring leaves a critical window of vulnerability open. & High \\
\bottomrule
\end{tabular}
\end{table}


% ==============================================================================
% 6. RECOMMENDATIONS
% ==============================================================================
\section{Recommendations}

The following actions are recommended to address the identified risks and improve the overall security posture of \textbf{[Organization Name]}.

\begin{enumerate}
    \item \textbf{Implement Endpoint MFA (Addresses RISK-001):}
    \begin{itemize}
        \item \textbf{Action:} Deploy a robust Multi-Factor Authentication solution for all employee computer logins (e.g., Windows Hello, Duo, Okta).
        \item \textbf{Priority:} \textbf{Critical}. This should be the immediate priority. Start with privileged users (administrators) and expand to all employees within the next quarter.
        \item \textbf{Impact:} Drastically reduces the risk of unauthorized access from compromised credentials.
    \end{itemize}
    \vspace{0.5cm}
    \item \textbf{Develop and Enforce an Acceptable Use Policy (Addresses RISK-002):}
    \begin{itemize}
        \item \textbf{Action:} Draft a formal AUP that clearly defines the rules for using company networks, devices, and data. This policy should be reviewed by HR and legal counsel.
        \item \textbf{Priority:} High. This foundational policy should be developed and implemented within the next 60 days.
        \item \textbf{Impact:} Establishes a clear security baseline for all employees, reduces legal liability, and provides a basis for disciplinary action in case of violation. All employees must sign an acknowledgement of the policy.
    \end{itemize}
    \vspace{0.5cm}
    \item \textbf{Establish a Security Onboarding Program (Addresses RISK-003):}
    \begin{itemize}
        \item \textbf{Action:} Create a mandatory security awareness training module that is a required part of the new employee onboarding process. The training should cover phishing, password security, data handling, and the new AUP.
        \item \textbf{Priority:} High. This should be implemented as soon as the AUP is finalized.
        \item \textbf{Impact:} Ensures that new employees understand their security responsibilities from day one, reducing their susceptibility to common cyberattacks.
    \end{itemize}
    \vspace{0.5cm}
    \item \textbf{Conduct a New Technical Network Scan:}
    \begin{itemize}
        \item \textbf{Action:} Commission a new, comprehensive external and internal network vulnerability scan to identify technical weaknesses that were missed due to the corrupted data.
        \item \textbf{Priority:} High.
        \item \textbf{Impact:} Provides essential visibility into the organization's technical attack surface, allowing for the remediation of software vulnerabilities and misconfigurations.
    \end{itemize}
\end{enumerate}

\end{document}
% ==============================================================================
% === END DOCUMENT ===
% ==============================================================================
```