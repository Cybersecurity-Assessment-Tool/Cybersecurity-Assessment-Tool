```latex
\documentclass[12pt]{article}

% ----------------------------------------------------------------------
% PREAMBLE
% ----------------------------------------------------------------------
\usepackage[margin=1in]{geometry}
\usepackage{pifont} % For checkmarks and crosses
\usepackage{booktabs} % For professional tables
\usepackage{hyperref} % For hyperlinks
\usepackage{url} % For URL formatting
\usepackage{seqsplit} % For splitting long text strings to prevent overflow
\usepackage{graphicx}
\usepackage[utf8]{inputenc}

\hypersetup{
    colorlinks=true,
    linkcolor=black,
    urlcolor=blue,
    pdftitle={Cybersecurity Assessment Report},
    pdfauthor={Cybersecurity Analyst},
}

% Custom commands
\newcommand{\yes}{\ding{51}}
\newcommand{\no}{\ding{55}}

% ----------------------------------------------------------------------
% DOCUMENT START
% ----------------------------------------------------------------------
\begin{document}

% --- TITLE PAGE ---
\begin{titlepage}
    \centering
    \vspace*{2cm}
    \Huge{\textbf{Cybersecurity Assessment Report}}
    \vspace{1.5cm}
    \Large{\textbf{Prepared for:}} \\
    \vspace{0.5cm}
    \huge{\textbf{[Organization Name]}}
    \vfill
    \large{\textbf{Date of Report:}} \\
    \vspace{0.2cm}
    \today
    \vspace{2cm}
    \large{This report contains sensitive information and should be handled with care.}
\end{titlepage}

\tableofcontents
\newpage

% ----------------------------------------------------------------------
% 1. EXECUTIVE SUMMARY
% ----------------------------------------------------------------------
\section{Executive Summary}

This report provides a comprehensive analysis of the security posture of \textbf{[Organization Name]}, based on a review of organizational security controls, an external network scan, and pre-existing risk data.

The assessment identified several critical and high-risk gaps in the organization's security framework. The most pressing concerns are the absence of multi-factor authentication (MFA) for sensitive data systems and a complete lack of a formal security awareness training program. These procedural and policy-based weaknesses significantly elevate the risk of a security breach originating from social engineering or compromised credentials.

Furthermore, a technical scan revealed an externally exposed Secure Shell (SSH) service on port 22. While common for remote administration, an improperly configured or unpatched SSH service can serve as a direct entry point for attackers.

Immediate and decisive action is required to address these findings. Recommendations focus on implementing foundational security controls, such as enforcing MFA universally, establishing a robust security training program, and hardening externally facing services. Addressing these issues will substantially improve the organization's resilience against common cyber threats.

% ----------------------------------------------------------------------
% 2. ORGANIZATIONAL INFORMATION
% ----------------------------------------------------------------------
\section{Organizational Information}

This section details the information provided about the organization. The assessment was conducted based on the following scope.

\begin{itemize}
    \item \textbf{Organization Name:} \textbf{[Organization Name]}
    \item \textbf{Primary Domain:} \texttt{[Domain]}
    \item \textbf{External IP Scanned:} \texttt{[Client IP]}
    \item \textbf{Scan Target IP:} \texttt{[Target IP]}
    \item \textbf{Date of Scan:} Not specified in scan data.
\end{itemize}

% ----------------------------------------------------------------------
% 3. SECURITY CONTROL REVIEW
% ----------------------------------------------------------------------
\section{Security Control Review}

A review of the organization's security controls was conducted via a questionnaire. The responses highlight significant gaps in fundamental security practices. A "No" response indicates a missing control and a potential area of high risk.

\begin{table}[h!]
\centering
\caption{Security Control Questionnaire Analysis}
\begin{tabular}{p{0.6\linewidth} c l}
\toprule
\textbf{Control Question} & \textbf{Response} & \textbf{Assessment} \\
\midrule
Do you require MFA to access email? & \yes & Best Practice Met \\
Do you require MFA to log into computers? & \yes & Best Practice Met \\
Do you require MFA to access sensitive data systems? & \no & \textbf{Critical Gap} \\
Does your organization have an employee acceptable use policy? & \no & High Risk \\
Does your organization do security awareness training for new employees? & \no & High Risk \\
Does your organization do security awareness training for all employees at least once per year? & \no & High Risk \\
\bottomrule
\end{tabular}
\end{table}

% ----------------------------------------------------------------------
% 4. TECHNICAL SCAN RESULTS
% ----------------------------------------------------------------------
\section{Technical Scan Results}

An external network scan was performed on the target IP address \texttt{[Target IP]}. The scan identified the following open ports and services.

\begin{table}[h!]
\centering
\caption{Open Port Analysis}
\begin{tabular}{l l l p{0.5\linewidth}}
\toprule
\textbf{Port} & \textbf{State} & \textbf{Service} & \textbf{Notes} \\
\midrule
22/tcp & open & SSH (Secure Shell) & This port is used for remote administrative access. Exposing SSH to the public internet is risky. The service version and configuration (e.g., password vs. key-based authentication) were not determined and must be reviewed to ensure it is secure. \\
\bottomrule
\end{tabular}
\end{table}

% ----------------------------------------------------------------------
% 5. RISK ASSESSMENT
% ----------------------------------------------------------------------
\section{Risk Assessment}

The following table synthesizes findings from the security control review and the technical scan into a prioritized list of risks. No pre-existing vulnerabilities were provided for this assessment.

\begin{table}[h!]
\centering
\caption{Summary of Identified Risks}
\begin{tabular}{p{0.1\linewidth} p{0.25\linewidth} p{0.45\linewidth} l}
\toprule
\textbf{ID} & \textbf{Risk Name} & \textbf{Description} & \textbf{Severity} \\
\midrule
RISK-001 & Lack of MFA on Sensitive Systems & The absence of MFA on systems containing sensitive data means a single compromised password could lead to a major data breach. & \textbf{Critical} \\
\addlinespace
RISK-002 & Inadequate Security Awareness Program & With no security training, employees are highly susceptible to phishing, social engineering, and other attacks that rely on human error. & High \\
\addlinespace
RISK-003 & Exposed SSH Service with Unknown Configuration & An externally accessible SSH service, if not securely configured (e.g., outdated, allows password logins), provides a direct vector for unauthorized access. & High \\
\addlinespace
RISK-004 & Missing Acceptable Use Policy (AUP) & Without a formal AUP, there is no enforceable standard for employee behavior regarding company assets, increasing the likelihood of insider threat and misuse. & High \\
\bottomrule
\end{tabular}
\end{table}

% ----------------------------------------------------------------------
% 6. RECOMMENDATIONS
% ----------------------------------------------------------------------
\section{Recommendations}

Based on the risks identified, the following actions are recommended to improve the security posture of \textbf{[Organization Name]}.

\begin{itemize}
    \item \textbf{Regarding RISK-001 (Lack of MFA):}
    \begin{itemize}
        \item \textbf{Immediate Action:} Enforce MFA across all systems that store, process, or access sensitive data. This should be treated as the highest priority remediation task.
    \end{itemize}
    \vspace{0.5cm}
    \item \textbf{Regarding RISK-002 (Security Awareness):}
    \begin{itemize}
        \item \textbf{Short-Term:} Develop and implement a mandatory security awareness training program for all new and existing employees.
        \item \textbf{Long-Term:} Conduct annual refresher training and periodic phishing simulations to maintain a high level of security consciousness.
    \end{itemize}
    \vspace{0.5cm}
    \item \textbf{Regarding RISK-003 (Exposed SSH):}
    \begin{itemize}
        \item \textbf{Immediate Action:} Conduct a configuration review of the SSH service on \texttt{[Target IP]}.
        \item \textbf{Required Hardening Steps:}
        \begin{itemize}
            \item Ensure the SSH server software is fully patched and up-to-date.
            \item Disable password-based authentication and enforce the use of strong cryptographic keys.
            \item Disable root login over SSH.
            \item If external access is not essential, restrict SSH access to an internal network or a secure VPN.
        \end{itemize}
    \end{itemize}
    \vspace{0.5cm}
    \item \textbf{Regarding RISK-004 (Missing AUP):}
    \begin{itemize}
        \item \textbf{Short-Term:} Draft a comprehensive Acceptable Use Policy that clearly defines the rules for using company networks, devices, and data.
        \item \textbf{Mid-Term:} Communicate the policy to all employees and require them to formally acknowledge it. Integrate the AUP into the new-hire onboarding process.
    \end{itemize}
\end{itemize}

% ----------------------------------------------------------------------
% DOCUMENT END
% ----------------------------------------------------------------------
\end{document}
```