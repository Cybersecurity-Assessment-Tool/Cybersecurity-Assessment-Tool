```latex
\documentclass[12pt]{article}

% ----------------------------------------------------------------------
% PREAMBLE
% ----------------------------------------------------------------------
\usepackage[margin=1in]{geometry}
\usepackage{pifont} % For checkmarks and crosses (\ding{51}, \ding{55})
\usepackage{booktabs} % For professional tables (\toprule, \midrule, \bottomrule)
\usepackage[hidelinks]{hyperref} % For clickable links without boxes
\usepackage{url} % For URL formatting
\usepackage{seqsplit} % For splitting long strings to prevent overflow

% Define checkmark and cross symbols for clarity
\newcommand{\cmark}{\ding{51}}
\newcommand{\xmark}{\ding{55}}

% ----------------------------------------------------------------------
% DOCUMENT START
% ----------------------------------------------------------------------
\begin{document}

% ----------------------------------------------------------------------
% TITLE PAGE
% ----------------------------------------------------------------------
\title{
    \vspace{2cm}
    \textbf{Cybersecurity Posture & Risk Assessment Report} \\
    \large \textit{Generated from Correlated Security Data} \\
    \vspace{1.5cm}
}
\author{\textbf{[Organization Name]}}
\date{\today}
\maketitle
\thispagestyle{empty}
\newpage

% ----------------------------------------------------------------------
% TABLE OF CONTENTS
% ----------------------------------------------------------------------
\tableofcontents
\newpage

% ----------------------------------------------------------------------
% 1. EXECUTIVE SUMMARY
% ----------------------------------------------------------------------
\section*{1. Executive Summary}

This report provides a comprehensive analysis of the cybersecurity posture for \textbf{[Organization Name]}. The assessment is based on a correlation of external network scan data, an internal security controls questionnaire, and a review of pre-existing risk documentation.

The analysis reveals a mixed security posture. The organization has successfully implemented several critical controls, including Multi-Factor Authentication (MFA) across email, computer logins, and sensitive data systems. However, two significant risks have been identified that require immediate attention:

\begin{enumerate}
    \item \textbf{Critical Technical Vulnerability:} An external network scan confirmed that a Remote Desktop Protocol (RDP) service (port 3389) is publicly exposed on the network perimeter at \texttt{[Client IP]}. This finding aligns with a known high-severity risk and presents an immediate and substantial threat of unauthorized access, potentially leading to a ransomware attack or complete system compromise.
    
    \item \textbf{High-Risk Procedural Gap:} The security controls review identified that the organization does not conduct annual security awareness training for all employees. This gap increases the organization's susceptibility to social engineering and phishing attacks, as employee knowledge of current threats is not regularly reinforced.
\end{enumerate}

This report details these findings and provides prioritized, actionable recommendations to mitigate the identified risks and strengthen the overall security posture of \textbf{[Organization Name]}.

% ----------------------------------------------------------------------
% 2. ORGANIZATIONAL INFORMATION
% ----------------------------------------------------------------------
\section*{2. Organizational Information}

The following information was used as the basis for this assessment. Where data was not provided, placeholders have been used.

\begin{itemize}
    \item \textbf{Organization Name:} \textbf{[Organization Name]}
    \item \textbf{Primary Domain:} \texttt{[Domain]}
    \item \textbf{External IP Address Scanned:} \texttt{[Client IP]}
\end{itemize}

% ----------------------------------------------------------------------
% 3. SECURITY CONTROL REVIEW (QUESTIONNAIRE ANALYSIS)
% ----------------------------------------------------------------------
\section*{3. Security Control Review}

An internal questionnaire was reviewed to assess the current state of administrative and procedural security controls. The results are summarized in the table below. "Yes" answers, indicating a control is in place, are marked with a green checkmark (\cmark). "No" answers, indicating a control gap, are marked with a red cross (\xmark).

\begin{table}[h!]
\centering
\caption{Security Controls Questionnaire Results}
\begin{tabular}{p{0.8\linewidth} c}
\toprule
\textbf{Control Question} & \textbf{Status} \\
\midrule
Do you require MFA to access email? & \cmark \\
Do you require MFA to log into computers? & \cmark \\
Do you require MFA to access sensitive data systems? & \cmark \\
Does your organization have an employee acceptable use policy? & \cmark \\
Does your organization do security awareness training for new employees? & \cmark \\
\textbf{Does your organization do security awareness training for all employees at least once per year?} & \textbf{\xmark} \\
\bottomrule
\end{tabular}
\end{table}

\subsection*{Analysis of Control Gaps}
The review indicates a significant gap in the organization's security training program. While new employees receive initial training, the lack of a mandatory annual refresher course for \textit{all} staff is a high-risk oversight. The threat landscape evolves rapidly, and without continuous education, employees are more likely to fall victim to modern phishing and social engineering tactics. This gap undermines the effectiveness of technical controls.

% ----------------------------------------------------------------------
% 4. TECHNICAL SCAN RESULTS
% ----------------------------------------------------------------------
\section*{4. Technical Scan Results}

An external network scan was performed on the target IP address to identify open ports and exposed services. The scan was conducted using Nmap.

\begin{itemize}
    \item \textbf{Target IP Address:} \texttt{[Target IP]}
    \item \textbf{Scan Date:} \today
\end{itemize}

\begin{table}[h!]
\centering
\caption{Open Ports Detected on \texttt{[Target IP]}}
\begin{tabular}{l l l l}
\toprule
\textbf{Port/Proto} & \textbf{State} & \textbf{Service} & \textbf{Notes} \\
\midrule
3389/tcp & open & ms-wbt-server & Remote Desktop Protocol (RDP) \\
\bottomrule
\end{tabular}
\end{table}

\subsection*{Analysis of Technical Findings}
The scan identified that TCP port 3389 is open to the public internet. This port is used for Microsoft's Remote Desktop Protocol (RDP). Publicly exposed RDP is a well-known and highly targeted attack vector used by malicious actors, including ransomware gangs, to gain initial access to a network. This finding represents a critical vulnerability that directly exposes the organization to a high risk of compromise. This technical finding validates the pre-existing risk documented in the organization's risk register.

% ----------------------------------------------------------------------
% 5. CONSOLIDATED RISK ASSESSMENT
% ----------------------------------------------------------------------
\section*{5. Consolidated Risk Assessment}

The following table synthesizes findings from the security questionnaire, technical scan, and pre-existing risk data into a prioritized list of current risks.

\begin{table}[h!]
\centering
\caption{Summary of Identified Risks}
\begin{tabular}{p{0.3\linewidth} p{0.15\linewidth} p{0.45\linewidth}}
\toprule
\textbf{Risk Name} & \textbf{Severity} & \textbf{Overview} \\
\midrule
\textbf{Public RDP Exposure} & \textbf{Critical (9.0)} & Port 3389 (RDP) is open on \texttt{[Client IP]}, allowing direct connection attempts from the internet. This is a primary vector for ransomware and unauthorized access. \\
\addlinespace
\textbf{Lack of Annual Security Awareness Training} & \textbf{High} & The absence of a mandatory annual security training program for all employees increases the risk of successful phishing and social engineering attacks. \\
\bottomrule
\end{tabular}
\end{table}

% ----------------------------------------------------------------------
% 6. RECOMMENDATIONS
% ----------------------------------------------------------------------
\section*{6. Recommendations}

The following actions are recommended to mitigate the identified risks. They are prioritized based on severity and potential impact.

\subsection*{Priority 1: Remediate Critical RDP Exposure (Immediate)}
This vulnerability must be addressed immediately to prevent a potential network compromise.

\begin{itemize}
    \item \textbf{Immediate Action (Containment):} Implement a firewall rule to \textbf{block all inbound traffic} to TCP port 3389 on the external interface of \texttt{[Client IP]}. This will immediately remove the public exposure.
    \item \textbf{Long-Term Solution (Secure Access):} For necessary remote administration, implement a Virtual Private Network (VPN) solution. Access to the internal network, including RDP services, should only be permitted through the VPN, which must be configured with strong encryption and Multi-Factor Authentication (MFA).
\end{itemize}

\subsection*{Priority 2: Implement Annual Security Training (High)}
This procedural gap should be closed to build a more resilient human firewall.

\begin{itemize}
    \item \textbf{Action:} Procure and implement a security awareness training platform or service.
    \item \textbf{Implementation:} Mandate that all employees, including management and executives, complete a security awareness training module at least once per year. This training should cover modern threats such as phishing, business email compromise, password security, and safe browsing habits. Track completion to ensure 100\% compliance.
\end{itemize}

% ----------------------------------------------------------------------
% DOCUMENT END
% ----------------------------------------------------------------------
\end{document}
```