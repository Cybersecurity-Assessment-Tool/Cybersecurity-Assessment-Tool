```latex
\documentclass[12pt]{article}

% === PACKAGES ===
\usepackage[margin=1in]{geometry}
\usepackage{pifont} % Required for \ding
\usepackage{booktabs} % For professional-looking tables
\usepackage[hidelinks]{hyperref} % For URLs and cross-references without boxes
\usepackage{url}
\usepackage{seqsplit} % To split long strings in \texttt
\usepackage{fancyhdr} % For headers and footers
\usepackage{lastpage} % To get the total page count
\usepackage{xcolor}   % For custom colors

% === DOCUMENT CONFIGURATION ===
% Define colors
\definecolor{darkblue}{rgb}{0.0, 0.0, 0.55}
\definecolor{darkred}{rgb}{0.6, 0.0, 0.0}
\definecolor{darkgreen}{rgb}{0.0, 0.4, 0.0}

% Hyperref setup
\hypersetup{
    colorlinks=true,
    linkcolor=darkblue,
    filecolor=darkblue,      
    urlcolor=darkblue,
    citecolor=darkblue,
}

% Custom commands for Yes/No symbols
\newcommand{\yes}{\textcolor{darkgreen}{\ding{51}}}
\newcommand{\no}{\textcolor{darkred}{\ding{55}}}

% Header and Footer configuration
\pagestyle{fancy}
\fancyhf{} % Clear all header and footer fields
\lhead{Cybersecurity Assessment Report}
\rhead{\textbf{[Organization Name]}}
\cfoot{Page \thepage\ of \pageref{LastPage}}
\renewcommand{\headrulewidth}{0.4pt}
\renewcommand{\footrulewidth}{0.4pt}

% === DOCUMENT START ===
\begin{document}

\title{
    \vspace{-1.5cm} % Adjust vertical space
    \rule{\textwidth}{1pt} \\ [0.5cm]
    \textbf{Cybersecurity Posture Assessment Report} \\ [0.2cm]
    \large For: \textbf{[Organization Name]} \\ [0.5cm]
    \rule{\textwidth}{1pt}
}
\author{Cybersecurity Analyst}
\date{\today}

\maketitle
\thispagestyle{empty}
\newpage

\tableofcontents
\newpage

% ==============================================================================
\section{Executive Summary}
% ==============================================================================

This report details the findings of a cybersecurity posture assessment conducted for \textbf{[Organization Name]}. The assessment combined an automated network scan, a review of existing risks, and an analysis of organizational security controls via a questionnaire.

The assessment identified several critical and high-severity risks that require immediate attention. The most significant finding is a publicly exposed MySQL database (\texttt{5.7.33}) on port 3306. This version is confirmed to be **End-of-Life (EOL)** as of October 2023, meaning it no longer receives security updates and is highly susceptible to known exploits.

This technical vulnerability is compounded by critical gaps in organizational controls. The lack of Multi-Factor Authentication (MFA) for computer logins and access to sensitive data systems creates a significant risk of unauthorized access. Should an attacker compromise a user's credentials, they could potentially access the exposed database and other internal resources without facing a secondary authentication challenge.

Furthermore, the absence of a formal Acceptable Use Policy indicates a gap in foundational cybersecurity governance.

In summary, the organization's current security posture is considered high-risk due to the combination of a critical, internet-facing vulnerability and insufficient access controls. Prioritized recommendations are provided in Section \ref{sec:recommendations} to mitigate these risks effectively.

% ==============================================================================
\section{Organizational Information}
% ==============================================================================

The following information was used as the basis for this assessment. Anonymized placeholders are used where data was not provided.

\begin{itemize}
    \item \textbf{Organization Name:} \textbf{[Organization Name]}
    \item \textbf{Primary Email Domain:} \texttt{[Domain]}
    \item \textbf{Assessed External IP:} \texttt{[Client IP]}
    \item \textbf{Target of Network Scan:} \texttt{[Target IP]}
    \item \textbf{Assessment Date:} \today
\end{itemize}

% ==============================================================================
\section{Security Control Review}
% ==============================================================================

The following table summarizes the organization's responses to the security controls questionnaire. "No" answers indicate potential gaps in the security framework and are highlighted for review.

\begin{table}[h!]
\centering
\caption{Security Controls Questionnaire Analysis}
\label{tab:controls}
\begin{tabular}{p{0.55\textwidth} c p{0.25\textwidth}}
\toprule
\textbf{Control Question} & \textbf{Response} & \textbf{Analyst Note} \\
\midrule
Do you require MFA to access email? & \yes & Good practice. \\
\addlinespace
Do you require MFA to log into computers? & \no & \textbf{Critical Gap.} Compromised credentials could lead to direct endpoint access. \\
\addlinespace
Do you require MFA to access sensitive data systems? & \no & \textbf{Critical Gap.} Increases the risk of a data breach. \\
\addlinespace
Does your organization have an employee acceptable use policy? & \no & \textbf{High Risk.} Lack of a foundational policy for security governance. \\
\addlinespace
Does your organization do security awareness training for new employees? & \yes & Good practice. \\
\addlinespace
Does your organization do security awareness training for all employees at least once per year? & \yes & Good practice. \\
\bottomrule
\end{tabular}
\end{table}

% ==============================================================================
\section{Technical Scan Results}
% ==============================================================================

An external network scan was performed against the target IP address \texttt{[Target IP]}. The scan identified one open port with a service that presents a significant security risk.

\begin{table}[h!]
\centering
\caption{Open Port Analysis}
\label{tab:nmap}
\begin{tabular}{l l l l p{0.3\textwidth}}
\toprule
\textbf{Port} & \textbf{State} & \textbf{Service} & \textbf{Product \& Version} & \textbf{Notes} \\
\midrule
3306/tcp & Open & mysql & MySQL 5.7.33 & \textbf{CRITICAL:} This version is \textbf{End-of-Life (EOL)} and no longer receives security updates. Public exposure is highly discouraged. \\
\bottomrule
\end{tabular}
\end{table}

% ==============================================================================
\section{Correlated Risk Assessment}
% ==============================================================================

The following table synthesizes findings from the technical scan, control review, and pre-existing risk data. Risks are prioritized by severity.

\begin{table}[h!]
\centering
\caption{Summary of Identified Risks}
\label{tab:risks}
\begin{tabular}{p{0.1\textwidth} p{0.3\textwidth} l p{0.4\textwidth}}
\toprule
\textbf{Risk ID} & \textbf{Risk Title} & \textbf{Severity} & \textbf{Description} \\
\midrule
RISK-001 & Exposed End-of-Life Database & \textbf{Critical} & A MySQL 5.7.33 database is open to the internet. This version is EOL and vulnerable to known exploits. \\
\addlinespace
RISK-002 & Lack of Multi-Factor Authentication (MFA) & \textbf{Critical} & MFA is not enforced for computer or sensitive system access, making the organization highly vulnerable to credential theft and unauthorized access. \\
\addlinespace
RISK-003 & Database Exposure & High & The pre-existing risk of an exposed MySQL port (3306) is confirmed by the technical scan. This allows direct connection attempts from any attacker. \\
\addlinespace
RISK-004 & Missing Acceptable Use Policy & High & The absence of a foundational AUP creates ambiguity regarding security responsibilities and acceptable employee behavior, weakening the overall security culture. \\
\bottomrule
\end{tabular}
\end{table}

% ==============================================================================
\section{Recommendations}
\label{sec:recommendations}
% ==============================================================================

The following prioritized actions are recommended to mitigate the identified risks and improve the organization's security posture.

\subsection{Immediate Actions (To Be Completed in 0-7 Days)}
\begin{enumerate}
    \item \textbf{Restrict Access to Database Port:} Immediately implement firewall rules to block all public internet access to TCP port 3306 on \texttt{[Target IP]}. Access should only be permitted from trusted internal IP addresses or via a secure VPN connection.
\end{enumerate}

\subsection{Short-Term Actions (To Be Completed in 1-3 Months)}
\begin{enumerate}
    \setcounter{enumi}{1} % Continue numbering
    \item \textbf{Upgrade End-of-Life Software:} Develop and execute a plan to migrate the MySQL 5.7.33 database to a fully supported version (e.g., MySQL 8.x). This is critical to ensure the system receives ongoing security patches.
    \item \textbf{Implement Multi-Factor Authentication (MFA):} Deploy MFA across all critical systems. Prioritize the following:
    \begin{itemize}
        \item All user computer logins (desktops and laptops).
        \item All systems containing sensitive or regulated data.
        \item All remote access solutions (e.g., VPNs).
    \end{itemize}
\end{enumerate}

\subsection{Mid-Term Actions (To Be Completed in 3-6 Months)}
\begin{enumerate}
    \setcounter{enumi}{3} % Continue numbering
    \item \textbf{Develop and Implement an Acceptable Use Policy (AUP):} Create a formal AUP that clearly defines the rules and responsibilities for all employees when using company IT assets. Ensure all employees read and acknowledge the policy.
    \item \textbf{Conduct Comprehensive Vulnerability Scanning:} Schedule regular, authenticated vulnerability scans for all external and internal systems to proactively identify and remediate security weaknesses.
\end{enumerate}

\end{document}
```