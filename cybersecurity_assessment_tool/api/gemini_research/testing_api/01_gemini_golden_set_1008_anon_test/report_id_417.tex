```latex
\documentclass[12pt]{article}

% --- PACKAGES ---
\usepackage[margin=1in]{geometry}
\usepackage{pifont} % For checkmarks and crosses
\usepackage{booktabs} % For professional tables
\usepackage{hyperref} % For hyperlinks
\usepackage{url} % For URL formatting
\usepackage{seqsplit} % For splitting long strings
\usepackage{graphicx}
\usepackage{xcolor}
\usepackage{fancyhdr}
\usepackage{lastpage}
\usepackage{longtable}

% --- DOCUMENT SETUP ---
\definecolor{darkblue}{rgb}{0.0, 0.0, 0.55}
\definecolor{darkred}{rgb}{0.55, 0.0, 0.0}

\hypersetup{
    colorlinks=true,
    linkcolor=darkblue,
    filecolor=magenta,      
    urlcolor=darkblue,
    pdftitle={Cybersecurity Assessment Report},
    pdfpagemode=FullScreen,
}

\pagestyle{fancy}
\fancyhf{}
\fancyhead[L]{Cybersecurity Assessment Report}
\fancyhead[R]{\textbf{[Organization Name]}}
\fancyfoot[C]{\thepage\ of \pageref{LastPage}}
\renewcommand{\headrulewidth}{0.4pt}
\renewcommand{\footrulewidth}{0.4pt}

% --- COMMANDS ---
\newcommand{\yes}{\ding{51}}
\newcommand{\no}{\ding{55}}

% --- DOCUMENT START ---
\begin{document}

% --- TITLE PAGE ---
\begin{titlepage}
    \centering
    \vspace*{1cm}
    
    \includegraphics[width=0.4\textwidth]{example-image-a} % Placeholder for a logo
    
    \vspace{1.5cm}
    
    {\Huge\bfseries Cybersecurity Assessment Report\par}
    
    \vspace{1.5cm}
    
    {\Large Prepared for:\par}
    \vspace{0.5cm}
    {\Huge\bfseries \textbf{[Organization Name]}}\par
    
    \vfill
    
    {\large \today\par}
\end{titlepage}

\tableofcontents
\newpage

% --- EXECUTIVE SUMMARY ---
\section*{1.0 Executive Summary}

This report details the findings of a cybersecurity assessment conducted for \textbf{[Organization Name]}. The assessment combined a review of organizational security controls, an external network scan, and an analysis of pre-existing risk data.

The overall security posture is determined to be at a \textbf{CRITICAL} risk level. This conclusion is based on two primary factors:
\begin{enumerate}
    \item \textbf{Systemic Lack of Foundational Security Controls:} The security questionnaire revealed a complete absence of fundamental controls, including Multi-Factor Authentication (MFA) for all critical access points (email, computers, sensitive systems), employee security policies, and security awareness training. These gaps represent a significant failure in security governance and expose the organization to a wide range of common cyberattacks.
    
    \item \textbf{Discovery of a Highly Sensitive Exposed Service:} The technical network scan identified an open service on port 8080 with the title \texttt{"TOP SECRET DB"}. The public exposure of a system with such a name, likely without authentication, is an immediate and severe threat. This finding directly contradicts previous risk assessments which had marked this port as secure, indicating a significant and unmitigated change in the organization's risk landscape.
\end{enumerate}

Immediate remediation is required to address the exposed service. Following this, a strategic initiative must be launched to implement the missing foundational security controls to reduce the organization's overall attack surface and improve its defensive capabilities.

% --- ORGANIZATIONAL INFORMATION ---
\section*{2.0 Organizational Information}

This section contains the high-level information used as the basis for this assessment.
\begin{itemize}
    \item \textbf{Organization Name:} \textbf{[Organization Name]}
    \item \textbf{Primary Domain:} \texttt{[Domain]}
    \item \textbf{External IP Address Scanned:} \texttt{[Client IP]}
\end{itemize}

% --- SECURITY CONTROL REVIEW ---
\section*{3.0 Security Control Review}

A security questionnaire was completed to evaluate the administrative and policy-based controls in place. The results indicate critical deficiencies across all reviewed domains. Each "No" response highlights a missing control that is considered standard practice for modern cybersecurity defense.

\begin{longtable}{p{0.6\linewidth} c c}
    \toprule
    \textbf{Control Question} & \textbf{Response} & \textbf{Status} \\
    \midrule
    \endfirsthead
    \toprule
    \textbf{Control Question} & \textbf{Response} & \textbf{Status} \\
    \midrule
    \endhead
    \bottomrule
    \endfoot
    Do you require MFA to access email? & \no & \textcolor{darkred}{\textbf{Critical Gap}} \\
    \midrule
    Do you require MFA to log into computers? & \no & \textcolor{darkred}{\textbf{Critical Gap}} \\
    \midrule
    Do you require MFA to access sensitive data systems? & \no & \textcolor{darkred}{\textbf{Critical Gap}} \\
    \midrule
    Does your organization have an employee acceptable use policy? & \no & \textcolor{darkred}{\textbf{High Risk}} \\
    \midrule
    Does your organization do security awareness training for new employees? & \no & \textcolor{darkred}{\textbf{High Risk}} \\
    \midrule
    Does your organization do security awareness training for all employees at least once per year? & \no & \textcolor{darkred}{\textbf{High Risk}} \\
\end{longtable}

% --- TECHNICAL SCAN RESULTS ---
\section*{4.0 Technical Scan Results}

An external network scan was performed against the organization's public-facing infrastructure to identify open ports and exposed services.

\begin{itemize}
    \item \textbf{Target IP Address:} \texttt{[Target IP]}
    \item \textbf{Scan Date:} \today
\end{itemize}

\subsection*{4.1 Open Ports Discovered}
The scan revealed one open port. The details are listed below.

\begin{table}[h!]
    \centering
    \begin{tabular}{l l l p{0.5\linewidth}}
        \toprule
        \textbf{Port} & \textbf{Protocol} & \textbf{State} & \textbf{Service Information} \\
        \midrule
        8080 & TCP & Open & HTTP service discovered. The title of the web page is: \textbf{\textcolor{darkred}{"TOP SECRET DB"}}. \\
        \bottomrule
    \end{tabular}
    \caption{Open Ports and Services on \texttt{[Target IP]}.}
\end{table}

\subsection*{4.2 Analysis of Technical Findings}
The discovery on port 8080 is a finding of the highest criticality. A service with a title suggesting it contains top-secret data should never be exposed to the public internet without robust authentication and access controls. This finding invalidates the previous risk assessment (\textit{Input\_3\_Current\_Risks\_JSON}) which stated, "Port 8080 is confirmed secure and false positive." The service is active, exposed, and presents an immediate and severe risk of a data breach.

% --- RISK ASSESSMENT ---
\section*{5.0 Synthesized Risk Assessment}

This section correlates findings from the security control review, technical scan, and previous risk data to provide a holistic view of the current risk posture.

\begin{longtable}{p{0.25\linewidth} p{0.5\linewidth} l}
    \toprule
    \textbf{Risk Name} & \textbf{Description} & \textbf{Severity} \\
    \midrule
    \endfirsthead
    \toprule
    \textbf{Risk Name} & \textbf{Description} & \textbf{Severity} \\
    \midrule
    \endhead
    \bottomrule
    \endfoot
    Exposed Sensitive Data Service & A service running on port 8080 is publicly accessible and titled "TOP SECRET DB". This suggests highly sensitive data is exposed without proper authentication, creating a direct path for data exfiltration. This finding invalidates previous assessments. & \textcolor{darkred}{\textbf{Critical}} \\
    \midrule
    Complete Lack of Multi-Factor Authentication (MFA) & MFA is not enforced for email, computer logins, or sensitive systems. This significantly increases the risk of account compromise via phishing or password spraying, which could lead to unauthorized access to critical data. & \textcolor{darkred}{\textbf{Critical}} \\
    \midrule
    Absence of Security Policies and Training & The organization lacks an acceptable use policy and a security awareness training program. This results in employees being unaware of their security responsibilities, making them more susceptible to social engineering and accidental data leakage. & \textcolor{darkred}{\textbf{High}} \\
\end{longtable}

% --- RECOMMENDATIONS ---
\section*{6.0 Recommendations}

The following actions are recommended to mitigate the identified risks. They are prioritized based on urgency and potential impact.

\subsection*{6.1 Immediate Actions (0-72 Hours)}
\begin{itemize}
    \item \textbf{Secure Exposed Service on Port 8080:}
    \begin{enumerate}
        \item Immediately place the service on port 8080 behind a firewall and restrict all public access.
        \item Conduct an internal investigation to identify the system owner, the nature of the data it contains, and why it was exposed.
        \item If the system is required, implement strong, MFA-enabled authentication before allowing any access.
    \end{enumerate}
\end{itemize}

\subsection*{6.2 High-Priority Actions (1-3 Months)}
\begin{itemize}
    \item \textbf{Deploy Multi-Factor Authentication (MFA):}
    \begin{enumerate}
        \item Phase 1: Enforce MFA on all email accounts (e.g., Office 365, Google Workspace).
        \item Phase 2: Enforce MFA on all remote access solutions (VPNs) and administrative accounts.
        \item Phase 3: Roll out MFA for all computer logins and access to sensitive data systems.
    \end{enumerate}
    \item \textbf{Develop Foundational Security Policies:}
    \begin{enumerate}
        \item Create and enforce an \textit{Employee Acceptable Use Policy} that defines rules for using company assets.
        \item Create a \textit{Data Classification and Handling Policy} to guide employees on how to protect sensitive information.
    \end{enumerate}
\end{itemize}

\subsection*{6.3 Medium-Priority Actions (3-6 Months)}
\begin{itemize}
    \item \textbf{Establish Security Awareness Training Program:}
    \begin{enumerate}
        \item Implement a mandatory training module for all new hires during their onboarding process.
        \item Conduct annual, mandatory security awareness training for all employees, covering topics like phishing, password security, and data handling.
    \end{enumerate}
\end{itemize}

\end{document}
```