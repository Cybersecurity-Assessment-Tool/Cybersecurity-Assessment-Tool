```latex
\documentclass[12pt]{article}

% Preamble: Required Packages
\usepackage[margin=1in]{geometry}
\usepackage{pifont} % For \ding{51} (checkmark) and \ding{55} (cross)
\usepackage{booktabs} % For professional-looking tables
\usepackage[hidelinks]{hyperref}
\usepackage{url}
\usepackage{seqsplit} % For breaking long strings like hashes or URLs
\usepackage{xcolor}   % For coloring text

% Document Metadata
\title{Cybersecurity Posture Assessment Report}
\author{Cybersecurity Analysis Division}
\date{\today}

\begin{document}

\maketitle
\thispagestyle{empty}
\newpage
\tableofcontents
\newpage

\section{Executive Summary}

This report provides a comprehensive cybersecurity assessment for \textbf{[Organization Name]}, based on an analysis of network scan data, organizational security controls, and known risks. The assessment identified several critical and high-risk vulnerabilities that require immediate attention.

The primary finding is the direct exposure of a Remote Desktop Protocol (RDP) service on port 3389 to the public internet at \texttt{[Client IP]}. This is a critical vulnerability (CVSS 9.0) that significantly increases the risk of unauthorized access, ransomware attacks, and data breaches.

This technical vulnerability is compounded by critical gaps in organizational security controls. The lack of Multi-Factor Authentication (MFA) for computer and sensitive system access, the absence of a formal Acceptable Use Policy, and the failure to conduct annual security awareness training for all employees create an environment where an attack is more likely to succeed.

Immediate remediation should focus on securing the exposed RDP service. Concurrently, the organization must prioritize the implementation of MFA and the development of foundational security policies and training programs to build a more resilient security posture.

\section{Organizational Information}

This section details the organizational information used for this assessment. As the provided data was anonymized, placeholders have been used.

\begin{itemize}
    \item \textbf{Organization Name:} \textbf{[Organization Name]}
    \item \textbf{Primary Domain:} \texttt{[Domain]}
    \item \textbf{External IP Address Assessed:} \texttt{[Client IP]}
\end{itemize}

\section{Security Control Review}

The following table summarizes the organization's responses to a security controls questionnaire. Items marked with a red 'X' (\textcolor{red}{\ding{55}}) indicate significant gaps in the security framework and represent areas of high risk.

\begin{center}
\begin{tabular}{p{0.7\linewidth}c}
\toprule
\textbf{Control Question} & \textbf{Status} \\
\midrule
Do you require MFA to access email? & \textcolor{green!80!black}{\ding{51}} \\
Do you require MFA to log into computers? & \textcolor{red}{\ding{55}} \\
Do you require MFA to access sensitive data systems? & \textcolor{red}{\ding{55}} \\
Does your organization have an employee acceptable use policy? & \textcolor{red}{\ding{55}} \\
Does your organization do security awareness training for new employees? & \textcolor{green!80!black}{\ding{51}} \\
Does your organization do security awareness training for all employees at least once per year? & \textcolor{red}{\ding{55}} \\
\bottomrule
\end{tabular}
\end{center}

\subsection*{Analysis of Gaps}
\begin{itemize}
    \item \textbf{Lack of MFA:} The absence of MFA on computer logins and sensitive systems is a critical weakness. It means that a single compromised password could grant an attacker significant access to the internal network.
    \item \textbf{Missing Policies \& Training:} The lack of an Acceptable Use Policy and annual security training indicates a low level of security maturity. Employees may be unaware of their responsibilities, making them more susceptible to social engineering and phishing attacks.
\end{itemize}

\section{Technical Scan Results}

A network scan was performed on the target IP address. The results confirm the presence of an externally accessible service, detailed below.

\begin{itemize}
    \item \textbf{Target IP Address:} \texttt{[Target IP]}
    \item \textbf{Scan Tool:} Nmap
\end{itemize}

\begin{center}
\begin{tabular}{lllll}
\toprule
\textbf{Port} & \textbf{State} & \textbf{Service} & \textbf{Product} & \textbf{Version} \\
\midrule
3389/tcp & open & ms-wbt-server & N/A & N/A \\
\bottomrule
\end{tabular}
\end{center}

\subsection*{Analysis of Findings}
The scan confirms that port \textbf{3389/tcp}, the standard port for Microsoft's Remote Desktop Protocol (RDP), is open to the public internet. RDP is a frequent target for attackers who use brute-force password attacks, credential stuffing, and exploitation of known vulnerabilities (e.g., BlueKeep) to gain initial access to a network. Exposing this service directly is a critical security risk.

\section{Correlated Risk Assessment}

This section synthesizes the findings from the security control review, technical scan, and pre-existing risk data into a consolidated view of the organization's current risk posture.

\begin{center}
\begin{tabular}{p{0.25\linewidth}p{0.55\linewidth}l}
\toprule
\textbf{Risk / Vulnerability} & \textbf{Description} & \textbf{Severity} \\
\midrule
\textbf{External RDP Exposure} & Port 3389 (RDP) is open to the public internet, inviting brute-force and exploit-based attacks. This was confirmed by the technical scan and aligns with pre-existing risk data. & \textbf{Critical} \\
\addlinespace
\textbf{Insufficient MFA} & The lack of MFA for computer and sensitive system logins means a single compromised password could lead to a full system compromise, greatly increasing the impact of the exposed RDP. & \textbf{High} \\
\addlinespace
\textbf{Policy \& Training Gaps} & The absence of an Acceptable Use Policy and recurring annual training weakens the human firewall, making credential harvesting via phishing more likely to succeed. & \textbf{High} \\
\bottomrule
\end{tabular}
\end{center}

\section{Prioritized Recommendations}

The following actionable recommendations are provided to mitigate the identified risks. They are prioritized based on severity and potential impact.

\begin{enumerate}
    \item \textbf{[Immediate] Remediate RDP Exposure:} The exposed RDP service on \texttt{[Target IP]} must be secured immediately.
        \begin{itemize}
            \item \textbf{Short-Term Fix:} Implement a firewall rule to block all inbound traffic to TCP port 3389 from the internet. Access should only be permitted from specific, trusted IP addresses if absolutely necessary.
            \item \textbf{Long-Term Solution:} Implement a Virtual Private Network (VPN) solution. All remote access to internal resources, including RDP, should require users to connect to the VPN first.
        \end{itemize}

    \item \textbf{[High Priority] Implement Comprehensive MFA:} Deploy MFA across all critical access points to protect against credential compromise.
        \begin{itemize}
            \item Prioritize enabling MFA for all computer logins (e.g., via Windows Hello for Business, Duo, etc.).
            \item Enforce MFA for access to all systems containing sensitive or critical data.
        \end{itemize}

    \item \textbf{[Medium Priority] Develop and Implement Foundational Policies:} Establish a formal security policy framework.
        \begin{itemize}
            \item Create and enforce an \textit{Employee Acceptable Use Policy} that clearly defines rules for using company assets and data.
        \end{itemize}

    \item \textbf{[Medium Priority] Establish a Security Awareness Program:} A robust training program is essential for building a strong security culture.
        \begin{itemize}
            \item Implement a mandatory security awareness training program for all employees to be completed at least once per year.
            \item Conduct periodic phishing simulations to test and improve employee resilience.
        \end{itemize}
\end{enumerate}

\end{document}
```