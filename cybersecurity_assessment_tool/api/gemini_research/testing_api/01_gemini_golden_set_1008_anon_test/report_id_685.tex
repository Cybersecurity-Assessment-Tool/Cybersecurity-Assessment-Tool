```latex
\documentclass[12pt]{article}

% Preamble: Required Packages
\usepackage[margin=1in]{geometry}
\usepackage{pifont} % For checkmarks and crosses
\usepackage{booktabs} % For professional tables
\usepackage{hyperref} % For clickable links
\usepackage{url} % For formatting URLs
\usepackage{seqsplit} % For splitting long strings in texttt

% Document Metadata
\title{Cybersecurity Posture Assessment Report}
\author{Cybersecurity Analysis Division}
\date{November 22, 2025}

\begin{document}

\maketitle
\thispagestyle{empty}
\newpage

\tableofcontents
\newpage

% --- 1. Executive Summary ---
\section{Executive Summary}

This report details the findings of a cybersecurity posture assessment conducted on November 22, 2025, for \textbf{[Organization Name]}. The assessment combined an analysis of organizational security controls, an external network scan, and a review of pre-existing risks.

The overall security posture of \textbf{[Organization Name]} requires immediate attention. Several critical and high-risk vulnerabilities were identified that expose the organization to significant threats, including data breaches, account compromise, and service disruption.

Key findings include:
\begin{itemize}
    \item \textbf{Critical Gaps in Access Control:} Multi-factor authentication (MFA) is not enforced for accessing email or other sensitive data systems. This is a critical vulnerability that greatly increases the risk of unauthorized access through phishing or credential theft.
    \item \textbf{Vulnerable External Service:} The external-facing web server at \texttt{[Target IP]} is running an outdated and unsupported version of Nginx (1.18.0). This software is known to have multiple publicly disclosed vulnerabilities, presenting a direct target for attackers.
    \item \textbf{Inadequate Employee Onboarding:} The organization does not provide security awareness training to new employees. This gap makes the organization more susceptible to social engineering attacks, as new hires are often prime targets.
\end{itemize}

Immediate remediation of these issues is strongly recommended to reduce the organization's attack surface and protect its critical assets. Detailed recommendations are provided in Section 6 of this report.

% --- 2. Organizational Information ---
\section{Organizational Information}

This section provides a summary of the organizational details relevant to this assessment. As the provided data was anonymized, placeholders are used where necessary.

\begin{tabular}{@{}ll}
    \toprule
    \textbf{Detail} & \textbf{Information} \\
    \midrule
    Organization Name & \textbf{[Organization Name]} \\
    Primary Email Domain & \texttt{[Domain]} \\
    External IP Scanned & \texttt{[Client IP]} \\
    Assessment Date & November 22, 2025 \\
    \bottomrule
\end{tabular}

% --- 3. Security Control Review ---
\section{Security Control Review}

The following table summarizes the organization's responses to a security controls questionnaire. A green checkmark (\ding{51}) indicates a positive control is in place, while a red cross (\ding{55}) indicates a control gap that introduces risk.

\begin{table}[h!]
\centering
\begin{tabular}{@{}lc@{}}
    \toprule
    \textbf{Control Question} & \textbf{Response} \\
    \midrule
    Do you require MFA to access email? & \ding{55} \\
    Do you require MFA to log into computers? & \ding{51} \\
    Do you require MFA to access sensitive data systems? & \ding{55} \\
    Does your organization have an employee acceptable use policy? & \ding{51} \\
    Does your organization do security awareness training for new employees? & \ding{55} \\
    Does your organization do security awareness training for all employees at least once per year? & \ding{51} \\
    \bottomrule
\end{tabular}
\caption{Organizational Security Controls Questionnaire Results.}
\end{table}

The identified gaps in MFA for email and sensitive systems, along with the lack of security training for new hires, represent significant weaknesses in the organization's defensive posture.

% --- 4. Technical Scan Results ---
\section{Technical Scan Results}

An external network scan was performed on the target IP address \texttt{[Target IP]} to identify open ports and exposed services.

\subsection{Open Ports and Services}
The scan revealed the following open port:

\begin{table}[h!]
\centering
\begin{tabular}{@{}llll@{}}
    \toprule
    \textbf{Port} & \textbf{State} & \textbf{Service} & \textbf{Product \& Version} \\
    \midrule
    443/tcp & open & https & nginx 1.18.0 \\
    \bottomrule
\end{tabular}
\caption{Open Ports Identified on \texttt{[Target IP]}.}
\end{table}

\subsection{Analysis of Findings}
The scan identified an Nginx web server, version \textbf{1.18.0}, exposed to the internet. This version was released in April 2020 and is now considered outdated and no longer receives security updates from the developer. 

Running outdated software on internet-facing systems is a high-risk practice. Nginx 1.18.0 is known to be vulnerable to several publicly disclosed security flaws (CVEs), which could allow an attacker to cause a denial of service, bypass security restrictions, or potentially execute arbitrary code. This finding represents a significant and exploitable weakness in the organization's external perimeter.

% --- 5. Risk Assessment ---
\section{Risk Assessment}

This section synthesizes the findings from the security control review and the technical scan into a consolidated list of identified risks. No pre-existing risks were reported.

\begin{table}[h!]
\centering
\begin{tabular}{@{}llll@{}}
    \toprule
    \textbf{Risk ID} & \textbf{Risk Description} & \textbf{Affected Asset(s)} & \textbf{Severity} \\
    \midrule
    RISK-001 & No MFA on email accounts. & Email System, User Accounts & \textbf{Critical} \\
    RISK-002 & No MFA on sensitive data systems. & Sensitive Data \& Systems & \textbf{Critical} \\
    RISK-003 & Outdated Nginx web server. & External Web Server (\texttt{[Target IP]}) & \textbf{High} \\
    RISK-004 & No security training for new hires. & Employees, Internal Network & \textbf{High} \\
    \bottomrule
\end{tabular}
\caption{Summary of Identified Risks.}
\end{table}

% --- 6. Recommendations ---
\section{Recommendations}

The following actions are recommended to mitigate the identified risks and improve the overall security posture of \textbf{[Organization Name]}. Recommendations are prioritized by severity.

\begin{description}
    \item[\textbf{RISK-001: Implement MFA for Email (Critical)}] Immediately enforce multi-factor authentication (MFA) for all user and administrative email accounts. This is the single most effective control to prevent account takeovers resulting from phishing and credential theft.

    \item[\textbf{RISK-002: Implement MFA for Sensitive Systems (Critical)}] Enforce MFA for all access to systems containing sensitive or critical business data. This includes databases, financial applications, and administrative interfaces.

    \item[\textbf{RISK-003: Remediate Vulnerable Web Server (High)}] Upgrade the Nginx server on \texttt{[Target IP]} from version 1.18.0 to the latest stable version. Implement a formal patch management program to ensure all internet-facing systems receive security updates in a timely manner.

    \item[\textbf{RISK-004: Implement Onboarding Security Training (High)}] Develop and implement a mandatory security awareness training module for all new employees as part of the standard onboarding process. This training should cover, at a minimum, phishing identification, password hygiene, and the organization's acceptable use policy.
\end{description}

\end{document}
```