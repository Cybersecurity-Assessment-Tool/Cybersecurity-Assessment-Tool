```latex
\documentclass[12pt, a4paper]{article}

% Preamble: Required Packages
\usepackage[margin=1in]{geometry}
\usepackage{pifont} % For checkmarks and crosses
\usepackage{booktabs} % For professional tables
\usepackage{hyperref} % For clickable links
\usepackage{url} % For URL formatting
\usepackage{seqsplit} % For splitting long strings
\usepackage{graphicx}
\usepackage[table]{xcolor}
\usepackage{fancyhdr}
\usepackage{lastpage}

% --- Document Setup ---
\hypersetup{
    colorlinks=true,
    linkcolor=blue,
    filecolor=magenta,      
    urlcolor=cyan,
    pdftitle={Cybersecurity Assessment Report},
    pdfauthor={Cybersecurity Analyst},
    pdfsubject={Security Posture Analysis},
    pdfkeywords={Security, Nmap, Risk, Assessment},
}

% --- Header and Footer ---
\pagestyle{fancy}
\fancyhf{} % clear all header and footer fields
\fancyhead[L]{Cybersecurity Assessment Report}
\fancyhead[R]{\textbf{[Organization Name]}}
\fancyfoot[C]{\thepage\ of \pageref{LastPage}}
\renewcommand{\headrulewidth}{0.4pt}
\renewcommand{\footrulewidth}{0.4pt}

% --- Document Start ---
\begin{document}

% --- Title Page ---
\begin{titlepage}
    \centering
    \vfill
    {\Huge\bfseries Cybersecurity Assessment Report\par}
    \vspace{1.5cm}
    {\Large Prepared for:\par}
    \vspace{0.5cm}
    {\Huge\bfseries \textbf{[Organization Name]}}\par
    \vfill
    {\large \today\par}
    \vspace{0.5cm}
    {\large Report generated by Expert Cybersecurity Analyst Bot\par}
\end{titlepage}

\tableofcontents
\newpage

% --- Section 1: Executive Overview ---
\section{Executive Overview}
This report details the findings of a cybersecurity assessment conducted for \textbf{[Organization Name]}. The assessment combined an external network scan, a review of organizational security controls, and an analysis of pre-existing risks to evaluate the overall security posture.

The analysis revealed several critical and high-risk vulnerabilities that require immediate attention. A key technical finding is an externally exposed FTP server running a dangerously outdated and vulnerable version of \texttt{vsftpd} (2.3.4), which is known to contain a critical backdoor vulnerability (CVE-2011-2523). This is exacerbated by a misconfiguration allowing anonymous, unauthenticated access.

From a policy and procedural standpoint, significant gaps were identified. The organization lacks a formal Acceptable Use Policy and does not conduct security awareness training for its employees. These deficiencies significantly increase the risk of insider threats and susceptibility to social engineering attacks.

While the organization has successfully implemented Multi-Factor Authentication (MFA) across key systems, the combination of a critical external vulnerability and major policy gaps places the organization at a high risk of a security breach. Immediate remediation of the technical vulnerabilities and development of foundational security policies are strongly recommended.

% --- Section 2: Organizational Information ---
\section{Organizational Information}
This section provides the organizational details used as a basis for this assessment. Due to the anonymized nature of the input data, placeholders have been used.

\begin{table}[h!]
\centering
\caption{Client Details}
\label{tab:client-details}
\begin{tabular}{@{}ll@{}}
\toprule
\textbf{Attribute} & \textbf{Value} \\ \midrule
Organization Name & \textbf{[Organization Name]} \\
Primary Domain & \texttt{[Domain]} \\
External IP Scanned & \texttt{[Client IP]} \\
Target of Scan & \texttt{[Target IP]} \\ \bottomrule
\end{tabular}
\end{table}

% --- Section 3: Security Control Review ---
\section{Security Control Review}
A review of foundational security controls was conducted based on a questionnaire. The results highlight a strong implementation of Multi-Factor Authentication (MFA) but reveal critical deficiencies in employee policy and training.

\begin{table}[h!]
\centering
\caption{Security Questionnaire Results}
\label{tab:questionnaire}
\renewcommand{\arraystretch}{1.2}
\begin{tabular}{@{}lcc@{}}
\toprule
\textbf{Control Question} & \textbf{Response} & \textbf{Status} \\ \midrule
Do you require MFA to access email? & Yes & \ding{51} \\
Do you require MFA to log into computers? & Yes & \ding{51} \\
Do you require MFA to access sensitive data systems? & Yes & \ding{51} \\
\rowcolor{red!15} Does your organization have an employee acceptable use policy? & No & \ding{55} \\
\rowcolor{red!15} Does your organization do security awareness training for new employees? & No & \ding{55} \\
\rowcolor{red!15} Does your organization do security awareness training for all employees? & No & \ding{55} \\ \bottomrule
\end{tabular}
\end{table}

\subsection*{Analysis of Gaps}
\begin{itemize}
    \item \textbf{Acceptable Use Policy (AUP):} The absence of an AUP is a critical governance gap. This policy is essential for setting clear expectations for employees regarding the use of company assets, data handling, and online behavior, thereby reducing insider risk.
    \item \textbf{Security Awareness Training:} The lack of both initial and ongoing training is a high-risk finding. Untrained employees are significantly more vulnerable to phishing, malware, and social engineering attacks, which are primary vectors for initial compromise.
\end{itemize}

% --- Section 4: Technical Scan Results ---
\section{Technical Scan Results}
An external network scan was performed on the target IP address \texttt{[Target IP]}. The scan identified one open port with a critically vulnerable service.

\begin{table}[h!]
\centering
\caption{Open Ports and Services}
\label{tab:nmap-results}
\renewcommand{\arraystretch}{1.2}
\begin{tabular}{@{}lllll@{}}
\toprule
\textbf{Port} & \textbf{State} & \textbf{Service} & \textbf{Product} & \textbf{Version} \\ \midrule
\rowcolor{red!25} 21/tcp & open & ftp & vsftpd & 2.3.4 \\ \bottomrule
\end{tabular}
\end{table}

\subsection*{Vulnerability Analysis}
The scan revealed the following critical issues associated with port 21:
\begin{itemize}
    \item \textbf{Vulnerable Software (CVE-2011-2523):} The FTP server is running \textbf{vsftpd version 2.3.4}. This specific version is widely known to contain a backdoor that was inserted into the source code. If exploited, this vulnerability allows an attacker to gain a command shell on the server with root privileges, leading to a full system compromise.
    \item \textbf{Insecure Configuration:} The scan confirmed that \textbf{Anonymous FTP login is allowed}. This configuration permits any user on the internet to connect to the FTP server without authentication and potentially access, upload, or download files. This poses a severe risk of data leakage, malware distribution, and unauthorized system access.
\end{itemize}

% --- Section 5: Consolidated Risk Assessment ---
\section{Consolidated Risk Assessment}
This section synthesizes findings from the technical scan, policy review, and pre-existing risk data into a prioritized list.

\begin{table}[h!]
\centering
\caption{Summary of Identified Risks}
\label{tab:risk-summary}
\renewcommand{\arraystretch}{1.3}
\begin{tabular}{@{}p{0.1\linewidth}p{0.55\linewidth}p{0.2\linewidth}@{}}
\toprule
\textbf{Risk ID} & \textbf{Description} & \textbf{Severity} \\ \midrule
\rowcolor{red!25} R-001 & \textbf{Critical FTP Vulnerability (vsftpd 2.3.4):} The FTP server version contains a known backdoor (CVE-2011-2523) allowing remote command execution. & Critical (9.8) \\
\rowcolor{red!25} R-002 & \textbf{Insecure FTP Configuration:} The server allows anonymous, unauthenticated access, posing a severe risk of data breach and unauthorized access. & Critical (9.1) \\
\rowcolor{orange!25} R-003 & \textbf{Lack of Employee Acceptable Use Policy:} Absence of a formal AUP increases the risk of insider threat and improper data handling. & High (7.5) \\
\rowcolor{orange!25} R-004 & \textbf{Lack of Security Awareness Training:} Employees are not trained to recognize or respond to common cyber threats like phishing, increasing susceptibility to attack. & High (7.2) \\
\rowcolor{yellow!25} R-005 & \textbf{Outdated Operating Systems (Windows 7):} Workstations are running an unsupported OS that no longer receives security updates, exposing them to known exploits. & Medium (5.0) \\ \bottomrule
\end{tabular}
\end{table}

% --- Section 6: Recommendations ---
\section{Recommendations}
Based on the identified risks, the following actions are recommended. They are prioritized to address the most critical threats first.

\subsection{Immediate Actions (Due within 72 hours)}
\begin{enumerate}
    \item \textbf{Remediate FTP Server (R-001, R-002):}
    \begin{itemize}
        \item Immediately take the FTP server offline to prevent exploitation.
        \item If FTP is a business necessity, upgrade the \texttt{vsftpd} software to the latest stable version. If not, decommission the service entirely.
        \item Disable anonymous FTP login. Enforce strong, unique credentials for all authorized users.
        \item Consider replacing FTP with a more secure protocol like SFTP (SSH File Transfer Protocol).
    \end{itemize}
\end{enumerate}

\subsection{High-Priority Actions (Due within 30 days)}
\begin{enumerate}
    \item \textbf{Develop and Implement an Acceptable Use Policy (R-003):}
    \begin{itemize}
        \item Draft a comprehensive AUP that covers data handling, internet usage, password security, and incident reporting.
        \item Require all current and new employees to read and formally acknowledge the policy.
    \end{itemize}
    \item \textbf{Establish a Security Awareness Training Program (R-004):}
    \begin{itemize}
        \item Procure and implement a training solution for all employees.
        \item Ensure the program includes modules on phishing, password security, and social engineering.
        \item Make training mandatory for all new hires and require annual refresher courses for all staff.
    \end{itemize}
\end{enumerate}

\subsection{Medium-Priority Actions (Due within 90-180 days)}
\begin{enumerate}
    \item \textbf{Plan and Execute OS Upgrades (R-005):}
    \begin{itemize}
        \item Develop a phased plan to upgrade or replace all workstations running Windows 7.
        \item Prioritize machines used by employees with access to sensitive data.
        \item Ensure all new systems are deployed with a modern, supported operating system.
    \end{itemize}
\end{enumerate}

\end{document}
```