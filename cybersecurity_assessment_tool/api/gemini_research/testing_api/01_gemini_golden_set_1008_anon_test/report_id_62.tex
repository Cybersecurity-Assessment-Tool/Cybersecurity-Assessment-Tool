```latex
\documentclass[12pt, a4paper]{article}

% Preamble: Required Packages and Document Setup
\usepackage[margin=1in]{geometry}
\usepackage{pifont} % For checkmarks and crosses (\ding)
\usepackage{booktabs} % For professional tables
\usepackage{hyperref} % For clickable links
\usepackage{url} % For URL formatting
\usepackage{seqsplit} % To split long strings in \texttt
\usepackage{graphicx}
\usepackage[table]{xcolor} % For coloring table cells
\usepackage{fancyhdr} % For custom headers/footers

% --- Document Metadata and Hyperref Setup ---
\hypersetup{
    colorlinks=true,
    linkcolor=blue,
    filecolor=magenta,      
    urlcolor=cyan,
    pdftitle={Cybersecurity Posture Assessment Report},
    pdfpagemode=FullScreen,
}

% --- Custom Commands and Colors for Severity ---
\definecolor{criticalred}{HTML}{D10000}
\definecolor{highorange}{HTML}{E57300}
\definecolor{mediumyellow}{HTML}{FFBF00}
\newcommand{\sevCRITICAL}{\colorbox{criticalred}{\color{white}\textbf{\strut CRITICAL}}}
\newcommand{\sevHIGH}{\colorbox{highorange}{\color{white}\textbf{\strut HIGH}}}
\newcommand{\sevMEDIUM}{\colorbox{mediumyellow}{\color{black}\textbf{\strut MEDIUM}}}
\newcommand{\checkMark}{\ding{51}} % Green checkmark
\newcommand{\crossMark}{\ding{55}} % Red X

% --- Header and Footer ---
\pagestyle{fancy}
\fancyhf{} % Clear all header and footer fields
\fancyhead[L]{Cybersecurity Posture Assessment}
\fancyhead[R]{\textbf{[Organization Name]}}
\fancyfoot[C]{\thepage}
\renewcommand{\headrulewidth}{0.4pt}
\renewcommand{\footrulewidth}{0.4pt}

% ==============================================================================
% --- BEGIN DOCUMENT ---
% ==============================================================================
\begin{document}

% --- Title Page ---
\begin{titlepage}
    \centering
    \vspace*{1cm}
    \includegraphics[width=0.4\textwidth]{example-image-a} % Placeholder for company logo
    
    \vspace{2cm}
    
    {\Huge\bfseries Cybersecurity Posture Assessment Report\par}
    
    \vspace{1.5cm}
    
    {\Large Prepared for:\par}
    {\Large\bfseries \textbf{[Organization Name]}}\par
    
    \vspace{2cm}
    
    {\large Report Date: \today\par}
    
    \vfill
    
    {\large This report contains sensitive information and should be handled with the utmost confidentiality.\par}
\end{titlepage}

\tableofcontents
\newpage

% ==============================================================================
% Section 1: Executive Summary
% ==============================================================================
\section{Executive Summary}

This report provides a comprehensive assessment of the cybersecurity posture for \textbf{[Organization Name]}. The analysis is based on a combination of technical network scanning, a review of administrative security controls via a questionnaire, and an evaluation of previously identified risks.

The assessment revealed several areas of concern requiring immediate attention. The most critical finding is an externally exposed FTP service (\texttt{vsftpd 2.3.4}) on host \texttt{[Target IP]}. This specific version is known to contain a critical backdoor vulnerability (CVE-2011-2523) and is dangerously configured to allow anonymous logins. This presents a direct and immediate threat of unauthorized access and potential system compromise.

Furthermore, a significant administrative gap was identified: the absence of a formal Employee Acceptable Use Policy. This lack of documented policy creates ambiguity regarding secure practices and exposes the organization to insider threats and unintentional security breaches.

These new findings, combined with the pre-existing risk of outdated Windows 7 workstations, indicate a security posture with significant vulnerabilities. We strongly recommend prioritizing the remediation actions outlined in Section \ref{sec:recommendations} to mitigate these risks and strengthen the organization's overall defense.

% ==============================================================================
% Section 2: Organizational Information
% ==============================================================================
\section{Organizational Information}

This section contains the high-level information used as the basis for this assessment. Due to the anonymized nature of the data provided, placeholders have been used where necessary.

\begin{table}[h!]
\centering
\caption{Client and Assessment Scope}
\begin{tabular}{@{}ll@{}}
\toprule
\textbf{Attribute} & \textbf{Value} \\ \midrule
Organization Name & \textbf{[Organization Name]} \\
Primary Domain & \texttt{[Domain]} \\
External IP Scanned & \texttt{[Client IP]} \\
Target of Technical Scan & \texttt{[Target IP]} \\ \bottomrule
\end{tabular}
\end{table}

% ==============================================================================
% Section 3: Security Control Review
% ==============================================================================
\section{Security Control Review}

The following table summarizes the organization's responses to a security controls questionnaire. Answers marked with a \crossMark\ indicate a potential gap in the security framework that may increase risk.

\begin{table}[h!]
\centering
\caption{Security Questionnaire Analysis}
\label{tab:questionnaire}
\begin{tabular}{@{}llc@{}}
\toprule
\textbf{Control Area} & \textbf{Question} & \textbf{Status} \\ \midrule
\rowcolor{gray!10}
Access Control & Do you require MFA to access email? & \checkMark \\
Access Control & Do you require MFA to log into computers? & \checkMark \\
\rowcolor{gray!10}
Access Control & Do you require MFA to access sensitive data systems? & \checkMark \\
\textbf{Policy \& Governance} & \textbf{Does your organization have an employee acceptable use policy?} & \textbf{\crossMark} \\
\rowcolor{gray!10}
Training & Does your organization do security awareness training for new employees? & \checkMark \\
Training & Do you do security awareness training for all employees annually? & \checkMark \\ \bottomrule
\end{tabular}
\end{table}

\paragraph{Analysis:} The organization has implemented strong Multi-Factor Authentication (MFA) controls across key systems, which is commendable. However, the absence of an \textbf{Employee Acceptable Use Policy} is a critical administrative gap. This policy is fundamental for setting clear expectations for employees regarding the safe and acceptable use of company assets, thereby reducing the risk of insider threats and accidental data exposure.

% ==============================================================================
% Section 4: Technical Scan Results
% ==============================================================================
\section{Technical Scan Results}

A network scan was performed on the target system to identify open ports and exposed services.

\subsection{Host: \texttt{[Target IP]}}
\begin{itemize}
    \item \textbf{Status:} Host is Up
\end{itemize}

\begin{table}[h!]
\centering
\caption{Open Ports and Services on \texttt{[Target IP]}}
\label{tab:nmap}
\begin{tabular}{@{}lllll@{}}
\toprule
\textbf{Port} & \textbf{State} & \textbf{Service} & \textbf{Version} & \textbf{Notes} \\ \midrule
\rowcolor{red!20}
21/tcp & Open & ftp & vsftpd 2.3.4 & \begin{tabular}[c]{@{}l@{}}Anonymous FTP login allowed.\\ \textbf{Critical Vulnerability (CVE-2011-2523)}\end{tabular} \\ \bottomrule
\end{tabular}
\end{table}

\paragraph{Analysis:} The scan identified a single open port, 21/tcp, running a File Transfer Protocol (FTP) service.
\begin{itemize}
    \item \textbf{Insecure Configuration:} The service allows for anonymous login, which permits any user on the internet to access files on the server without authentication. This is a severe security risk.
    \item \textbf{Critical Vulnerability:} The identified version, \textbf{vsftpd 2.3.4}, is widely known to contain a critical backdoor vulnerability (CVE-2011-2523). An attacker can exploit this vulnerability by sending a specific sequence of characters in a username field, which opens a command shell on port 6200, granting the attacker remote control over the system.
\end{itemize}

% ==============================================================================
% Section 5: Consolidated Risk Assessment
% ==============================================================================
\section{Consolidated Risk Assessment}

This section synthesizes findings from the security control review, technical scan, and pre-existing risk data into a prioritized list.

\begin{table}[h!]
\centering
\caption{Summary of Identified Risks}
\label{tab:risks}
\begin{tabular}{@{}p{0.05\linewidth} p{0.3\linewidth} p{0.15\linewidth} p{0.4\linewidth}@{}}
\toprule
\textbf{ID} & \textbf{Risk Name} & \textbf{Severity} & \textbf{Description} \\ \midrule
\rowcolor{criticalred!25}
R-01 & Vulnerable FTP Service with Anonymous Login & \sevCRITICAL & The server at \texttt{[Target IP]} is running vsftpd 2.3.4, which has a known remote command execution backdoor (CVE-2011-2523). Anonymous login is also enabled, making it trivial to exploit. \\
\addlinespace
\rowcolor{highorange!25}
R-02 & Missing Employee Acceptable Use Policy & \sevHIGH & The organization lacks a formal policy defining the rules for using company IT assets. This increases the risk of insider threat, data leakage, and non-compliance. \\
\addlinespace
\rowcolor{mediumyellow!25}
R-03 & Outdated Windows 7 Workstations & \sevMEDIUM & (Pre-existing risk) Workstations are running Windows 7, which is an end-of-life operating system no longer receiving security updates from Microsoft, leaving them vulnerable to known exploits. \\ \bottomrule
\end{tabular}
\end{table}

% ==============================================================================
% Section 6: Recommendations
% ==============================================================================
\section{Recommendations}
\label{sec:recommendations}

Based on the risk assessment, the following actions are recommended to mitigate the identified vulnerabilities and improve the overall security posture.

\subsection{R-01: Vulnerable FTP Service (Critical)}
\begin{itemize}
    \item \textbf{Immediate Action:} If the FTP service is not essential for business operations, disable and block port 21 at the firewall immediately.
    \item \textbf{Required Action:} If the service is required, immediately upgrade the \texttt{vsftpd} software to the latest stable version to patch the CVE-2011-2523 vulnerability.
    \item \textbf{Configuration Hardening:} Disable anonymous FTP login. All access should require strong, unique user credentials.
    \item \textbf{Best Practice:} For all future file transfer needs, migrate to a secure protocol such as SFTP (SSH File Transfer Protocol) or FTPS (FTP over SSL/TLS).
\end{itemize}

\subsection{R-02: Missing Acceptable Use Policy (High)}
\begin{itemize}
    \item \textbf{Immediate Action:} Draft and implement a comprehensive Employee Acceptable Use Policy (AUP). This policy should clearly define rules for email, internet usage, software installation, data handling, and remote access.
    \item \textbf{Implementation:} Ensure the policy is formally approved by management and communicated to all current employees.
    \item \textbf{Ongoing Action:} Incorporate the AUP into the new employee onboarding process, requiring all new hires to read and sign an acknowledgment form. Review and update the policy annually.
\end{itemize}

\subsection{R-03: Outdated Windows 7 Workstations (Medium)}
\begin{itemize}
    \item \textbf{Short-Term Action:} Prioritize the existing plan to upgrade all Windows 7 workstations to a currently supported operating system, such as Windows 10 or Windows 11.
    \item \textbf{Compensating Controls:} Until upgrades are complete, ensure these machines have up-to-date antivirus software, are isolated on the network from critical assets if possible, and have limited internet access.
\end{itemize}

\end{document}
% ==============================================================================
% --- END DOCUMENT ---
% ==============================================================================
```