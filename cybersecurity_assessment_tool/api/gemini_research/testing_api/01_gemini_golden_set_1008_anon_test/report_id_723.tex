```latex
\documentclass[12pt]{article}

% Preamble: Required Packages
\usepackage[utf8]{inputenc}
\usepackage[margin=1in]{geometry}
\usepackage{pifont} % For checkmarks and crosses (\ding)
\usepackage{booktabs} % For professional tables
\usepackage{hyperref} % For clickable links and better PDF navigation
\usepackage{url} % For formatting URLs
\usepackage{seqsplit} % For splitting long strings without spaces
\usepackage{xcolor} % For colors

% Document Information
\title{Cybersecurity Posture and Risk Assessment Report}
\author{Cybersecurity Analysis Division}
\date{\today}

% Hyperref Setup
\hypersetup{
    colorlinks=true,
    linkcolor=blue,
    filecolor=magenta,      
    urlcolor=cyan,
    pdftitle={Cybersecurity Posture and Risk Assessment Report},
    pdfpagemode=FullScreen,
}

\begin{document}

\maketitle

\begin{abstract}
\noindent This report provides a comprehensive analysis of the cybersecurity posture for \textbf{[Organization Name]}. The assessment is based on a synthesis of network scan data, an organizational security controls questionnaire, and a review of pre-existing risks. The analysis identified a critical risk related to a publicly exposed, end-of-life database service. Additionally, a high-risk gap was identified in the organization's security awareness training program. This document details these findings and provides actionable recommendations to mitigate the identified risks.
\end{abstract}

\section*{1. Overview and Executive Summary}

The primary objective of this assessment was to evaluate the external security posture and internal security controls of the organization. Data was gathered from three distinct sources: an external network scan, a security questionnaire, and a list of known risks.

\textbf{Key Findings:}
\begin{itemize}
    \item \textbf{Critical Risk:} A MySQL database server (version 5.7.33) was found to be publicly accessible from the internet on port 3306. This version is officially End-of-Life (EOL) as of October 2023 and no longer receives security updates, making it highly susceptible to exploitation. This finding directly confirms and elevates a pre-existing identified risk.
    \item \textbf{High Risk:} The organization does not conduct mandatory annual security awareness training for all employees. This represents a significant gap in organizational security hygiene, increasing susceptibility to social engineering and phishing attacks.
\end{itemize}

Overall, while the organization has implemented several key security controls, such as Multi-Factor Authentication (MFA), the critical exposure of an unsupported database service requires immediate attention to prevent a potential data breach.

\section*{2. Organizational Information}

The following details were used as the basis for this assessment. Due to the anonymized nature of the input data, placeholders have been used where necessary.

\begin{table}[h!]
\centering
\begin{tabular}{@{}ll@{}}
\toprule
\textbf{Attribute} & \textbf{Value} \\ \midrule
Organization Name    & \textbf{[Organization Name]} \\
Primary Domain       & \texttt{[Domain]} \\
External IP Address  & \texttt{[Client IP]} \\ \bottomrule
\end{tabular}
\caption{Client Organizational Details.}
\end{label{tab:org_info}
\end{table}

\section*{3. Security Control Review}

The following table summarizes the organization's responses to a security controls questionnaire. The status indicates whether the control is in place (\ding{51}) or not (\ding{55}).

\begin{table}[h!]
\centering
\begin{tabular}{@{}lc@{}}
\toprule
\textbf{Security Control Question} & \textbf{Status} \\ \midrule
Do you require MFA to access email? & \ding{51} \\
Do you require MFA to log into computers? & \ding{51} \\
Do you require MFA to access sensitive data systems? & \ding{51} \\
Does your organization have an employee acceptable use policy? & \ding{51} \\
Does your organization do security awareness training for new employees? & \ding{51} \\
\textbf{Does your organization do security awareness training for all employees at least once per year?} & \textcolor{red}{\ding{55}} \\ \bottomrule
\end{tabular}
\caption{Security Controls Questionnaire Results.}
\label{tab:controls}
\end{table}

\subsection*{Analysis of Control Gaps}
The questionnaire reveals a significant gap: the lack of annual security awareness training for all staff. While training new hires is a good first step, the threat landscape evolves continuously. Regular, recurring training is a fundamental best practice for mitigating human-centric risks like phishing, which remains a primary vector for initial network compromise. This gap is classified as a \textbf{High Risk}.

\section*{4. Technical Scan Results}

An external network scan was performed to identify open ports and exposed services. The scan targeted the client's external-facing infrastructure.

\begin{itemize}
    \item \textbf{Target IP Address:} \texttt{[Target IP]}
    \item \textbf{Scan Status:} Host is UP.
\end{itemize}

\begin{table}[h!]
\centering
\begin{tabular}{@{}lllll@{}}
\toprule
\textbf{Port} & \textbf{State} & \textbf{Service} & \textbf{Product} & \textbf{Version} \\ \midrule
3306/tcp      & open           & mysql            & MySQL            & 5.7.33           \\ \bottomrule
\end{tabular}
\caption{Open Ports Detected on \texttt{[Target IP]}.}
\label{tab:nmap_results}
\end{table}

\subsection*{Analysis of Technical Findings}
The scan confirms the pre-existing risk of "Database Exposure". Port 3306 is open to the public internet, which is a highly insecure configuration. Attackers can directly attempt to brute-force credentials, exploit vulnerabilities, or perform denial-of-service attacks against the database.

\textbf{Crucially, the detected MySQL version 5.7.33 reached its official End-of-Life (EOL) in October 2023.} This means Oracle no longer provides security patches for this version. Any vulnerabilities discovered after this date will remain unpatched, exposing the server to significant and unacceptable risk. The combination of public exposure and EOL software constitutes a \textbf{Critical Risk}.

\section*{5. Consolidated Risk Assessment}

This section synthesizes findings from the security questionnaire, technical scan, and pre-existing risk data into a consolidated list of prioritized risks.

\begin{table}[h!]
\centering
\begin{tabular}{@{}p{0.2\linewidth}p{0.1\linewidth}p{0.6\linewidth}@{}}
\toprule
\textbf{Risk Name} & \textbf{Severity} & \textbf{Description} \\ \midrule
\textbf{Publicly Exposed End-of-Life Database Service} & \textbf{Critical} & A MySQL 5.7.33 database is accessible on port 3306 from the internet. The software is EOL and no longer receives security updates, making it a prime target for automated exploits. This could lead to a full data breach. \\
\addlinespace
\textbf{Lack of Annual Security Awareness Training} & \textbf{High} & The absence of a recurring security training program for all employees increases the organization's vulnerability to phishing and social engineering attacks, which could lead to credential theft and initial access for attackers. \\ \bottomrule
\end{tabular}
\caption{Summary of Identified Risks.}
\label{tab:risk_summary}
\end{table}

\section*{6. Recommendations}

The following actionable recommendations are provided to mitigate the identified risks. They are prioritized based on severity and potential impact.

\begin{enumerate}
    \item \textbf{Immediate (Priority 1): Restrict Access to Database Server.}
    \begin{itemize}
        \item \textbf{Action:} Implement strict firewall rules to deny all public access to port 3306 on host \texttt{[Target IP]}. Access should only be permitted from specific, trusted internal IP addresses.
        \item \textbf{Justification:} This is the fastest way to mitigate the immediate threat of external attack against the exposed database.
    \end{itemize}

    \item \textbf{High Priority (Priority 2): Upgrade End-of-Life Database Software.}
    \begin{itemize}
        \item \textbf{Action:} Plan and execute the migration of the MySQL 5.7.33 database to a currently supported version (e.g., MySQL 8.x).
        \item \textbf{Justification:} Running EOL software is unsustainable. An upgrade is necessary to receive security patches and protect against future vulnerabilities.
    \end{itemize}

    \item \textbf{Medium Priority (Priority 3): Implement Annual Security Training.}
    \begin{itemize}
        \item \textbf{Action:} Establish a mandatory security awareness training program for all employees, to be completed annually. The program should cover topics such as phishing identification, password hygiene, and acceptable use policies.
        \item \textbf{Justification:} This strengthens the "human firewall" and reduces the likelihood of a security incident originating from employee error.
    \end{itemize}
    
    \item \textbf{Long-Term (Priority 4): Implement Secure Remote Access.}
    \begin{itemize}
        \item \textbf{Action:} For any required remote database administration, implement a Virtual Private Network (VPN) or a bastion host solution.
        \item \textbf{Justification:} This provides a secure, encrypted, and authenticated channel for administrative access, eliminating the need for any direct public exposure of sensitive services.
    \end{itemize}
\end{enumerate}

\end{document}
```