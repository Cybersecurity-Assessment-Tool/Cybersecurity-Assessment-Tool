```latex
\documentclass[12pt]{article}

% Required Packages
\usepackage[margin=1in]{geometry}
\usepackage{pifont} % For checkmarks and crosses
\usepackage{booktabs} % For professional tables
\usepackage{hyperref} % For hyperlinks
\usepackage{url} % For URL formatting
\usepackage{seqsplit} % For splitting long text strings
\usepackage{xcolor} % For colors
\usepackage{graphicx} % For potential logos/images
\usepackage{fancyhdr} % For headers and footers

% --- Document Setup ---

% Define colors for risk levels
\definecolor{critical}{HTML}{990000}
\definecolor{high}{HTML}{D14900}
\definecolor{medium}{HTML}{E89900}
\definecolor{low}{HTML}{339900}

% Hyperlink setup
\hypersetup{
    colorlinks=true,
    linkcolor=blue,
    filecolor=magenta,      
    urlcolor=cyan,
    pdftitle={Cybersecurity Posture Assessment Report},
    pdfpagemode=FullScreen,
}

% Header and Footer
\pagestyle{fancy}
\fancyhf{}
\lhead{\textbf{Cybersecurity Posture Assessment}}
\rhead{\textbf{[Organization Name]}}
\cfoot{\thepage}

% --- Document Start ---

\begin{document}

% --- Title Page ---
\begin{titlepage}
    \centering
    \vspace*{1cm}
    
    \Huge
    \textbf{Cybersecurity Posture Assessment Report}
    
    \vspace{1.5cm}
    
    \Large
    \textbf{Prepared for:} \\
    \vspace{0.5cm}
    \textbf{[Organization Name]}
    
    \vspace{2cm}
    
    \Large
    \textbf{Date of Report:} \\
    \vspace{0.5cm}
    \today
    
    \vfill
    
    \large
    \textit{This report contains sensitive and confidential information intended only for the recipient. Distribution is strictly prohibited.}
    
\end{titlepage}

\tableofcontents
\newpage

% --- Section 1: Executive Summary ---
\section{Executive Summary}
This report provides a comprehensive analysis of the cybersecurity posture for \textbf{[Organization Name]}. The assessment is based on a correlation of external network scan data, a security controls questionnaire, and a review of pre-existing risk documentation.

The overall security posture is critically weak and requires immediate remediation. A key technical finding confirmed the direct exposure of a Remote Desktop Protocol (RDP) service on port 3389 to the public internet. This vulnerability, with a CVSS score of 9.0, represents a severe and immediate threat, as it is a primary vector for ransomware attacks and unauthorized network access.

This technical flaw is dangerously amplified by significant gaps in organizational security controls. The lack of Multi-Factor Authentication (MFA) for computer and sensitive data system access, combined with the absence of a formal employee acceptable use policy and security awareness training, creates an environment highly susceptible to compromise. An attacker who obtains a single user's credentials could potentially gain direct, unimpeded access to the internal network.

Urgent action is required to address the RDP exposure, implement MFA, and establish foundational security policies and training programs. Detailed recommendations are provided in Section \ref{sec:recommendations}.

% --- Section 2: Organizational Information ---
\section{Organizational Information}
The following details were used as the basis for this assessment. Where information was not provided, placeholders have been used.

\begin{table}[h!]
\centering
\begin{tabular}{@{}ll@{}}
\toprule
\textbf{Attribute} & \textbf{Value} \\ \midrule
Organization Name & \textbf{[Organization Name]} \\
Primary Domain & \texttt{[Domain]} \\
Assessed External IP & \texttt{[Client IP]} \\
Scan Target IP & \texttt{[Target IP]} \\ \bottomrule
\end{tabular}
\caption{Client and Assessment Scope Details}
\end{table}

% --- Section 3: Security Control Review ---
\section{Security Control Review}
A review of the organization's security controls was conducted via a questionnaire. The responses reveal critical deficiencies in fundamental security practices. A summary is provided below.

\begin{table}[h!]
\centering
\begin{tabular}{@{}p{0.55\linewidth}ccp{0.2\linewidth}@{}}
\toprule
\textbf{Control Question} & \textbf{Response} & \textbf{Assessment} \\ \midrule
Do you require MFA to access email? & \ding{51} Yes & Good Practice \\
\addlinespace
Do you require MFA to log into computers? & \ding{55} No & \textcolor{high}{\textbf{High Risk}} \\
\addlinespace
Do you require MFA to access sensitive data systems? & \ding{55} No & \textcolor{critical}{\textbf{Critical Gap}} \\
\addlinespace
Does your organization have an employee acceptable use policy? & \ding{55} No & \textcolor{high}{\textbf{High Risk}} \\
\addlinespace
Does your organization do security awareness training for new employees? & \ding{55} No & \textcolor{critical}{\textbf{Critical Gap}} \\
\addlinespace
Does your organization do security awareness training for all employees at least once per year? & \ding{55} No & \textcolor{critical}{\textbf{Critical Gap}} \\ \bottomrule
\end{tabular}
\caption{Security Controls Questionnaire Analysis}
\end{table}

% --- Section 4: Technical Scan Results ---
\section{Technical Scan Results}
An external network scan was performed against the target IP address \texttt{[Target IP]}. The scan identified one open port, which presents a significant security risk.

\begin{table}[h!]
\centering
\begin{tabular}{@{}llll@{}}
\toprule
\textbf{Port} & \textbf{State} & \textbf{Service} & \textbf{Notes} \\ \midrule
3389/tcp & Open & ms-wbt-server & This is the Microsoft Remote Desktop Protocol (RDP). \\
& & & Exposing RDP directly to the internet is a critical risk \\
& & & and a common target for attackers. \\ \bottomrule
\end{tabular}
\caption{Open Ports Identified on \texttt{[Target IP]}}
\end{table}

% --- Section 5: Correlated Risk Assessment ---
\section{Correlated Risk Assessment}
The following table synthesizes findings from the technical scan, security control review, and pre-existing risk data to provide a holistic view of the current risk landscape.

\begin{table}[h!]
\centering
\begin{tabular}{@{}p{0.25\linewidth}p{0.5\linewidth}p{0.15\linewidth}@{}}
\toprule
\textbf{Risk Name} & \textbf{Description} & \textbf{Severity} \\ \midrule
\addlinespace
\textbf{Critical RDP Exposure} & The network scan confirmed that Remote Desktop Protocol (port 3389) is open on \texttt{[Target IP]}. This aligns with pre-existing risk data and provides a direct path for attackers into the internal network. & \textcolor{critical}{\textbf{Critical (9.0)}} \\
\addlinespace
\textbf{Lack of Multi-Factor Authentication (MFA)} & The absence of MFA for computer and sensitive system access, combined with the exposed RDP service, means a single compromised password could lead to a full network breach. & \textcolor{critical}{\textbf{Critical}} \\
\addlinespace
\textbf{Deficient Security Policies and Training} & The lack of an Acceptable Use Policy and any form of security awareness training significantly increases the organization's susceptibility to phishing and social engineering, which are primary methods for credential theft. & \textcolor{high}{\textbf{High}} \\
\addlinespace
\bottomrule
\end{tabular}
\caption{Summary of Identified and Correlated Risks}
\end{table}

% --- Section 6: Recommendations ---
\section{Recommendations}
\label{sec:recommendations}
Based on the correlated findings, the following actions are recommended to mitigate the identified risks and improve the overall security posture of \textbf{[Organization Name]}. Recommendations are prioritized by urgency.

\subsection{Immediate Actions (Remediate within 24-48 hours)}
\begin{enumerate}
    \item \textbf{Remediate RDP Exposure:} Immediately implement a firewall rule to block all inbound traffic to TCP port 3389 on \texttt{[Target IP]}. If remote access is a business requirement, it must be placed behind a Virtual Private Network (VPN) that requires Multi-Factor Authentication.
\end{enumerate}

\subsection{Short-Term Actions (Implement within 30 days)}
\begin{enumerate}
    \item \textbf{Deploy Multi-Factor Authentication (MFA):} Prioritize the rollout of MFA for all remote access (VPN), logins to company computers, and access to all systems storing sensitive data. This is the single most effective control to prevent unauthorized access from compromised credentials.
    \item \textbf{Develop and Implement an Acceptable Use Policy (AUP):} Create a formal AUP that defines the rules and expectations for employees using company IT resources. All employees must be required to read and acknowledge this policy.
\end{enumerate}

\subsection{Long-Term Actions (Implement within 90 days)}
\begin{enumerate}
    \item \textbf{Establish Security Awareness Training Program:} Implement a mandatory security awareness training program for all employees. This should include initial training for new hires and at least one annual refresher course for all staff. The training should cover topics such as phishing, password security, and social engineering.
    \item \textbf{Implement a Vulnerability Management Program:} Establish a regular schedule for external and internal network vulnerability scanning. This will help proactively identify and remediate technical vulnerabilities before they can be exploited.
\end{enumerate}

\end{document}
```