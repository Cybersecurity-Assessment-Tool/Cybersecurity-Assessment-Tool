```latex
\documentclass[12pt]{article}

% --- PACKAGES ---
\usepackage[margin=1in]{geometry}
\usepackage{pifont} % For checkmarks and crosses
\usepackage{booktabs} % For professional tables
\usepackage{hyperref} % For clickable links
\usepackage{url} % For URL formatting
\usepackage{seqsplit} % To split long strings without breaking
\usepackage{xcolor} % For colors

% --- DOCUMENT SETUP ---
\hypersetup{
    colorlinks=true,
    linkcolor=blue,
    filecolor=magenta,      
    urlcolor=cyan,
    pdftitle={Cybersecurity Posture Report},
    pdfpagemode=FullScreen,
}

\newcommand{\yes}{\ding{51}}
\newcommand{\no}{\ding{55}}

% --- DOCUMENT START ---
\begin{document}

% --- TITLE PAGE ---
\begin{titlepage}
    \centering
    \vspace*{1cm}
    \Huge{\textbf{Cybersecurity Posture Report}}
    \vspace{1.5cm}
    \
    \large{Prepared for:} \\
    \vspace{0.5cm}
    \textbf{[Organization Name]}
    \vfill
    \large{Generated by:} \\
    \vspace{0.5cm}
    Cybersecurity Analyst
    \vspace{1cm}
    \
    \large{\today}
\end{titlepage}

\tableofcontents
\newpage

% --- EXECUTIVE SUMMARY ---
\section{Executive Summary}
This report provides a comprehensive analysis of the cybersecurity posture for \textbf{[Organization Name]}, based on a combination of organizational data, network scanning, and a review of current risks. The assessment was conducted on [Scan Date], though this was not provided in the scan data.

The analysis reveals several critical and high-risk security gaps that require immediate attention. The most significant findings are the absence of Multi-Factor Authentication (MFA) for computer and sensitive data system access, and a complete lack of a security awareness training program for employees. These administrative control failures create a substantial risk of unauthorized access and compromise through social engineering or credential theft.

Technically, an exposed Secure Shell (SSH) service (Port 22) was identified on the network perimeter at \texttt{[Target IP]}. While necessary for remote administration, this service becomes a high-value target for attackers, especially when foundational controls like MFA are missing.

This report outlines the identified risks and provides prioritized, actionable recommendations to mitigate them and strengthen the organization's overall security posture.

% --- ORGANIZATIONAL INFORMATION ---
\section{Organizational Information}
The following details were used as the basis for this assessment. Note that key identifying information was not provided and has been replaced with placeholders.

\begin{itemize}
    \item \textbf{Organization Name:} \textbf{[Organization Name]}
    \item \textbf{Primary Domain:} \texttt{[Domain]}
    \item \textbf{External IP Address Scanned:} \texttt{[Client IP]}
\end{itemize}

% --- SECURITY CONTROL REVIEW ---
\section{Security Control Review}
An assessment of the organization's administrative security controls was performed based on a supplied questionnaire. The results indicate significant gaps in access control and employee security training. "No" answers represent a deviation from security best practices and are flagged as risks.

\begin{table}[h!]
\centering
\caption{Organizational Security Control Questionnaire}
\label{tab:controls}
\begin{tabular}{@{}p{0.6\linewidth} c l@{}}
\toprule
\textbf{Control Question} & \textbf{Response} & \textbf{Assessment} \\
\midrule
Do you require MFA to access email? & \yes & Compliant \\
\addlinespace
Do you require MFA to log into computers? & \no & \textcolor{red}{\textbf{Critical Gap}} \\
\addlinespace
Do you require MFA to access sensitive data systems? & \no & \textcolor{red}{\textbf{Critical Gap}} \\
\addlinespace
Does your organization have an employee acceptable use policy? & \yes & Compliant \\
\addlinespace
Does your organization do security awareness training for new employees? & \no & \textcolor{orange}{High Risk} \\
\addlinespace
Does your organization do security awareness training for all employees at least once per year? & \no & \textcolor{orange}{High Risk} \\
\bottomrule
\end{tabular}
\end{table}

% --- TECHNICAL SCAN RESULTS ---
\section{Technical Scan Results}
A network scan was performed on the organization's external infrastructure to identify exposed services. The scan was limited in scope and further authenticated scanning is recommended.

\begin{itemize}
    \item \textbf{Target IP Address:} \texttt{[Target IP]}
    \item \textbf{Scan Date:} Not Provided
\end{itemize}

\subsection{Open Ports}
The following ports were found to be open and accessible from the public internet.

\begin{table}[h!]
\centering
\caption{Open Port Analysis}
\label{tab:ports}
\begin{tabular}{@{}l l l l@{}}
\toprule
\textbf{Port} & \textbf{State} & \textbf{Service (Inferred)} & \textbf{Product / Version} \\
\midrule
22/tcp & open & SSH (Secure Shell) & Not Identified \\
\bottomrule
\end{tabular}
\end{table}

\subsection{Technical Analysis}
The scan identified an open SSH port (22), which is commonly used for remote system administration. While this service is often required for operational purposes, its exposure to the internet makes it a primary target for brute-force and credential-stuffing attacks. The risk associated with this finding is significantly amplified by the lack of MFA for computer logins, as confirmed in the security control review. A single compromised password could lead to direct administrative access to a critical system.

% --- RISK ASSESSMENT ---
\section{Risk Assessment}
The following table summarizes the key risks identified by correlating the organizational data, technical scan results, and pre-existing risk information. As no pre-existing vulnerabilities were provided, all risks below are newly identified during this assessment.

\begin{table}[h!]
\centering
\caption{Summary of Identified Risks}
\label{tab:risks}
\begin{tabular}{@{}p{0.1\linewidth} p{0.25\linewidth} p{0.4\linewidth} l@{}}
\toprule
\textbf{ID} & \textbf{Risk Name} & \textbf{Description} & \textbf{Severity} \\
\midrule
RISK-001 & Lack of MFA on Critical Systems & The absence of MFA on computer logins and sensitive data systems exposes the organization to account takeover and unauthorized access. & \textcolor{red}{\textbf{Critical}} \\
\addlinespace
RISK-002 & Inadequate Security Awareness Program & Without training, employees are more likely to fall victim to phishing, malware, and social engineering, leading to data breaches. & \textcolor{orange}{\textbf{High}} \\
\addlinespace
RISK-003 & Exposed SSH Service without MFA & The publicly accessible SSH service, combined with a lack of MFA on endpoints, provides a direct path for attackers to gain system access with stolen credentials. & \textcolor{yellow!80!black}{\textbf{Medium}} \\
\bottomrule
\end{tabular}
\end{table}

% --- RECOMMENDATIONS ---
\section{Recommendations}
Based on the findings, the following prioritized actions are recommended to mitigate the identified risks and improve the overall security posture.

\subsection{Priority 1: Remediate Critical Risks}
\begin{itemize}
    \item \textbf{Implement MFA Everywhere (RISK-001):} Immediately deploy a robust MFA solution for all employee computer logins and for all access to systems containing sensitive or critical data. This is the single most effective control to prevent unauthorized access.
\end{itemize}

\subsection{Priority 2: Remediate High Risks}
\begin{itemize}
    \item \textbf{Establish a Security Awareness Program (RISK-002):} Develop and implement a mandatory security awareness training program. This program must be included in the onboarding process for all new employees and conducted at least annually for all staff.
\end{itemize}

\subsection{Priority 3: Remediate Medium Risks}
\begin{itemize}
    \item \textbf{Harden SSH Service (RISK-003):} Review the business need for the exposed SSH service on \texttt{[Target IP]}. If it is required:
    \begin{enumerate}
        \item Restrict access to a specific IP allowlist of trusted administrative locations.
        \item Disable password-based authentication and enforce the use of strong cryptographic keys (e.g., ED25519).
        \item Ensure the SSH service is patched and running an up-to-date version.
    \end{enumerate}
\end{itemize}

% --- CONCLUSION ---
\section{Conclusion}
The assessment for \textbf{[Organization Name]} has identified significant and actionable gaps in its cybersecurity defenses. While foundational elements like an acceptable use policy are in place, the lack of critical access controls (MFA) and a formal security training program presents a clear and present danger to the organization. The exposed SSH service further exacerbates this risk.

By implementing the prioritized recommendations outlined in this report, the organization can substantially reduce its attack surface, improve its resilience against common cyber threats, and build a stronger security culture.

\end{document}
```