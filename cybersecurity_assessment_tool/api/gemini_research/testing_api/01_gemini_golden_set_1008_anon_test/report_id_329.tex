```latex
\documentclass[12pt]{article}

% === PACKAGES ===
\usepackage[margin=1in]{geometry}
\usepackage{pifont}                 % For check and cross marks (\ding)
\usepackage{booktabs}               % For professional-looking tables
\usepackage{hyperref}               % For hyperlinks
\usepackage{url}                    % For URL formatting
\usepackage{seqsplit}               % For splitting long strings without spaces
\usepackage{fancyhdr}               % For headers and footers
\usepackage{graphicx}               % To include images/logos
\usepackage{xcolor}                 % For custom colors
\usepackage{lastpage}               % To reference the last page number

% === DOCUMENT SETUP ===
\hypersetup{
    colorlinks=true,
    linkcolor=blue,
    filecolor=magenta,      
    urlcolor=cyan,
    pdftitle={Cybersecurity Posture Assessment Report},
    pdfauthor={Cybersecurity Analysis Division},
}

% === CUSTOM COMMANDS & SETTINGS ===
% Define colors for risk severity
\definecolor{sev_critical}{HTML}{990000}
\definecolor{sev_high}{HTML}{DD4B39}
\definecolor{sev_medium}{HTML}{F4B400}
\definecolor{sev_low}{HTML}{4285F4}

% Shorthand for check and cross marks
\newcommand{\yes}{\ding{51}}
\newcommand{\no}{\ding{55}}

% Header and Footer configuration
\pagestyle{fancy}
\fancyhf{} % Clear all header and footer fields
\fancyhead[L]{Cybersecurity Posture Assessment}
\fancyhead[R]{\textbf{[Organization Name]}}
\fancyfoot[C]{Page \thepage\ of \pageref{LastPage}}
\renewcommand{\headrulewidth}{0.4pt}
\renewcommand{\footrulewidth}{0.4pt}

% === DOCUMENT START ===
\begin{document}

% --- TITLE PAGE ---
\begin{titlepage}
    \centering
    \vspace*{2cm}
    
    \Huge \textbf{Cybersecurity Posture Assessment Report}
    
    \vspace{1.5cm}
    
    \Large Prepared For: \\
    \vspace{0.5cm}
    \textbf{[Organization Name]}
    
    \vfill
    
    \large
    Report Generated: \today \\
    Analysis Division
    
    \vspace{1cm}
    
    \small
    \textit{This document contains sensitive information and should be handled with care.}
    
\end{titlepage}

\newpage

% --- TABLE OF CONTENTS ---
\tableofcontents
\newpage

% --- EXECUTIVE SUMMARY ---
\section{Executive Summary}

This report details the findings of a cybersecurity posture assessment conducted for \textbf{[Organization Name]}. The analysis correlates data from a network perimeter scan, a security controls questionnaire, and a review of pre-existing risk documentation.

The assessment identified a \textbf{\textcolor{sev_critical}{CRITICAL}} risk posture. A primary finding is the direct exposure of a Remote Desktop Protocol (RDP) service on port 3389 to the public internet. This configuration is a well-known and highly targeted vector for ransomware attacks and unauthorized network access.

This technical vulnerability is severely compounded by critical gaps in organizational security controls. Specifically, the complete lack of Multi-Factor Authentication (MFA) for email, computer logins, and sensitive data systems means that a single compromised password could lead to a full network breach. Furthermore, the absence of security awareness training for new employees significantly increases the likelihood of such a credential compromise via phishing or other social engineering tactics.

Immediate remediation is required to address the exposed RDP service. Strategic initiatives to implement MFA and enhance the security training program are strongly recommended to build a more resilient security posture.

% --- ORGANIZATIONAL INFORMATION ---
\section{Organizational Information}
This section provides the context for the assessment based on the provided data.
\begin{itemize}
    \item \textbf{Organization Name:} \textbf{[Organization Name]}
    \item \textbf{Primary Email Domain:} \texttt{[Domain]}
    \item \textbf{External IP Scanned:} \texttt{[Client IP]}
\end{itemize}

% --- SECURITY CONTROL REVIEW ---
\section{Security Control Review}
The following table summarizes the organization's responses to a security controls questionnaire. "No" answers indicate significant gaps that increase organizational risk.

\begin{table}[h!]
\centering
\caption{Security Controls Questionnaire Analysis}
\label{tab:controls}
\begin{tabular}{p{0.6\linewidth} c p{0.25\linewidth}}
\toprule
\textbf{Control Question} & \textbf{Response} & \textbf{Analyst Finding} \\
\midrule
Do you require MFA to access email? & \no & \textcolor{sev_critical}{Critical Gap} \\
Do you require MFA to log into computers? & \no & \textcolor{sev_critical}{Critical Gap} \\
Do you require MFA to access sensitive data systems? & \no & \textcolor{sev_critical}{Critical Gap} \\
\addlinespace
Does your organization have an employee acceptable use policy? & \yes & Positive Control \\
\addlinespace
Does your organization do security awareness training for new employees? & \no & \textcolor{sev_high}{High Risk} \\
\addlinespace
Does your organization do security awareness training for all employees at least once per year? & \yes & Positive Control \\
\bottomrule
\end{tabular}
\end{table}

The lack of MFA across all critical access points is the most severe finding from this review. It effectively removes a fundamental layer of defense against credential-based attacks.

% --- TECHNICAL SCAN RESULTS ---
\section{Technical Scan Results}
An Nmap scan was performed on the organization's external perimeter. The results below highlight open ports and services accessible from the public internet.

\begin{table}[h!]
\centering
\caption{Nmap Scan Findings for Target: \texttt{[Target IP]}}
\label{tab:nmap}
\begin{tabular}{l l l l}
\toprule
\textbf{Port} & \textbf{State} & \textbf{Service} & \textbf{Analysis} \\
\midrule
3389/tcp & open & ms-wbt-server & Remote Desktop Protocol (RDP) \\
\bottomrule
\end{tabular}
\end{table}

\paragraph{Analysis:} The scan confirms that port 3389 is open, exposing the Microsoft Remote Desktop Protocol service directly to the internet. RDP is a frequent target for brute-force password attacks and exploitation of known vulnerabilities (e.g., BlueKeep). This finding aligns with the pre-existing risk documented in Input 3 and represents an immediate and severe threat to the network.

% --- CORRELATED RISK ASSESSMENT ---
\section{Correlated Risk Assessment}
This section synthesizes all data points into a prioritized list of identified risks.

\begin{table}[h!]
\centering
\caption{Summary of Identified Risks}
\label{tab:risks}
\begin{tabular}{p{0.3\linewidth} p{0.15\linewidth} p{0.45\linewidth}}
\toprule
\textbf{Risk Name} & \textbf{Severity} & \textbf{Description} \\
\midrule
\textbf{Publicly Exposed RDP Service} & \textcolor{sev_critical}{Critical} & Port 3389 (RDP) is open on target \texttt{[Target IP]}. This is a primary vector for ransomware and unauthorized access. This risk is confirmed by both the technical scan and existing risk documentation. \\
\addlinespace
\textbf{Lack of Multi-Factor Authentication (MFA)} & \textcolor{sev_critical}{Critical} & MFA is not enforced for email, computer logins, or sensitive systems. This allows an attacker with a single valid password to gain full access, dramatically increasing the impact of the exposed RDP service. \\
\addlinespace
\textbf{Insufficient Security Awareness Training} & \textcolor{sev_high}{High} & New employees do not receive security training, making them highly susceptible to phishing attacks that could compromise credentials needed to access the exposed RDP service or other internal resources. \\
\bottomrule
\end{tabular}
\end{table}

% --- RECOMMENDATIONS ---
\section{Recommendations}
The following actionable recommendations are provided to mitigate the identified risks. They are prioritized by urgency.

\subsection{Immediate Actions (Remediation - Complete within 24 hours)}
\begin{enumerate}
    \item \textbf{Block Port 3389:} Immediately configure the perimeter firewall to \textbf{deny all inbound traffic} to TCP port 3389 for the affected IP address (\texttt{[Target IP]}). This is the most critical step to prevent an imminent compromise.
\end{enumerate}

\subsection{Short-Term Actions (Hardening - Complete within 30 days)}
\begin{enumerate}
    \item \textbf{Implement MFA:} Prioritize the deployment of a robust MFA solution. The rollout should be staged:
    \begin{itemize}
        \item \textbf{Phase 1:} All remote access solutions and privileged accounts.
        \item \textbf{Phase 2:} All user email accounts (e.g., Office 365, Google Workspace).
        \item \textbf{Phase 3:} All systems containing sensitive data.
    \end{itemize}
    \item \textbf{Mandate New Hire Training:} Develop and integrate a mandatory security awareness training module into the employee onboarding process. This training should cover phishing, password hygiene, and acceptable use policies.
\end{enumerate}

\subsection{Long-Term Actions (Strategy - Complete within 6-12 months)}
\begin{enumerate}
    \item \textbf{Deploy Secure Remote Access Solution:} Implement a modern, secure remote access solution such as a Virtual Private Network (VPN) with MFA or a Zero Trust Network Access (ZTNA) platform. This will provide authenticated and encrypted access for remote employees without exposing services like RDP directly to the internet.
    \item \textbf{Establish a Vulnerability Management Program:} Implement a recurring, automated vulnerability scanning process for all external and internal assets. This will help proactively identify and remediate new security weaknesses before they can be exploited.
\end{enumerate}

\end{document}
```