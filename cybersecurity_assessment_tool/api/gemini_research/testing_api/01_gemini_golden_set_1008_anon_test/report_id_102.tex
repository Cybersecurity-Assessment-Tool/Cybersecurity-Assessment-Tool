```latex
\documentclass[12pt]{article}

% -----------------------------------------------------------------------------
% |                                 PACKAGES                                  |
% -----------------------------------------------------------------------------
\usepackage[margin=1in]{geometry}
\usepackage{pifont}                 % For checkmarks and crosses (\ding)
\usepackage{booktabs}               % For professional-looking tables
\usepackage[utf8]{inputenc}
\usepackage{graphicx}
\usepackage{xcolor}
\usepackage{fancyhdr}
\usepackage{hyperref}               % For clickable links and references
\usepackage{url}                    % For formatting URLs
\usepackage{seqsplit}               % To split long strings in \texttt

% -----------------------------------------------------------------------------
% |                             DOCUMENT SETUP                                |
% -----------------------------------------------------------------------------
\hypersetup{
    colorlinks=true,
    linkcolor=black,
    filecolor=magenta,      
    urlcolor=blue,
    pdftitle={Cybersecurity Posture Assessment Report},
    pdfpagemode=FullScreen,
}

% Define colors for table rows
\definecolor{tableheadcolor}{rgb}{0.1, 0.2, 0.4}
\definecolor{tablerowcolor}{gray}{0.95}

% Header and Footer
\pagestyle{fancy}
\fancyhf{}
\fancyhead[L]{\textbf{[Organization Name]} -- Cybersecurity Assessment}
\fancyfoot[C]{Confidential}
\fancyfoot[R]{\thepage}
\renewcommand{\headrulewidth}{0.4pt}
\renewcommand{\footrulewidth}{0.4pt}

% Check and cross symbols
\newcommand{\cmark}{\ding{51}}
\newcommand{\xmark}{\ding{55}}

% -----------------------------------------------------------------------------
% |                           START OF DOCUMENT                               |
% -----------------------------------------------------------------------------
\begin{document}

\title{Cybersecurity Posture Assessment Report}
\author{Cybersecurity Analysis Division}
\date{\today}
\maketitle

\begin{abstract}
\noindent This report provides a comprehensive cybersecurity assessment for \textbf{[Organization Name]}, synthesizing data from technical network scans, an organizational security controls questionnaire, and a review of pre-existing risks. The analysis identifies key strengths and critical vulnerabilities in the current security posture and offers prioritized, actionable recommendations to mitigate identified risks.
\end{abstract}

\newpage
\tableofcontents
\newpage

% =============================================================================
% |                           EXECUTIVE SUMMARY                               |
% =============================================================================
\section{Executive Summary}

This assessment was conducted to evaluate the overall security posture of \textbf{[Organization Name]}. Our analysis correlated external network scan data with internal security control policies to provide a holistic view of the organization's resilience against common cyber threats.

\paragraph{Key Findings:} The assessment revealed a mixed security posture. On a positive note, the external network scan of the target system did not identify any open ports, indicating a strong perimeter defense. The previously reported risk of an unencrypted web server on Port 80 appears to have been remediated, as our scan found this port to be closed.

However, significant gaps were identified in organizational security controls. The two most critical findings are:
\begin{itemize}
    \item \textbf{Lack of Multi-Factor Authentication (MFA) for Email:} The absence of MFA on the \texttt{[Domain]} email system represents a critical vulnerability. Email is a primary target for attackers, and password-only authentication is insufficient to protect against phishing and credential theft.
    \item \textbf{Absence of Annual Security Training:} While new employees receive security training, there is no mandatory annual refresher for all staff. This can lead to a decay in security awareness over time, increasing the organization's susceptibility to social engineering attacks.
\end{itemize}

\paragraph{Strategic Recommendation:} We strongly advise that \textbf{[Organization Name]} prioritizes the implementation of MFA for all email accounts. Concurrently, an annual security awareness training program should be established to reinforce a culture of security throughout the organization. Addressing these two control gaps will yield the most significant improvement in the organization's overall defensive capabilities.

% =============================================================================
% |                         ORGANIZATIONAL INFORMATION                        |
% =============================================================================
\section{Organizational and Assessment Information}

The following details provide context for this assessment. Information has been derived from the provided datasets. Placeholders are used where data was not available.

\begin{tabular}{@{}ll}
\toprule
\textbf{Item} & \textbf{Detail} \\
\midrule
Organization Name & \textbf{[Organization Name]} \\
Email Domain & \texttt{[Domain]} \\
Client External IP & \texttt{[Client IP]} \\
Target Scanned IP & \texttt{[Target IP]} \\
Assessment Date & \today \\
\bottomrule
\end{tabular}

% =============================================================================
% |                         SECURITY CONTROL REVIEW                           |
% =============================================================================
\section{Security Control Review}

The following table summarizes the organization's responses to a security controls questionnaire. Each response is assessed against industry best practices to identify potential policy and procedure gaps.

\begin{table}[h!]
\centering
\caption{Security Controls Questionnaire Analysis}
\begin{tabular}{p{0.6\linewidth} c p{0.2\linewidth}}
\toprule
\rowcolor{tableheadcolor}
\textcolor{white}{\textbf{Control Question}} & \textcolor{white}{\textbf{Response}} & \textcolor{white}{\textbf{Assessment}} \\
\midrule
Do you require MFA to access email? & \xmark & \textcolor{red}{\textbf{Critical Gap}} \\
\rowcolor{tablerowcolor}
Do you require MFA to log into computers? & \cmark & Control in Place \\
Do you require MFA to access sensitive data systems? & \cmark & Control in Place \\
\rowcolor{tablerowcolor}
Does your organization have an employee acceptable use policy? & \cmark & Control in Place \\
Does your organization do security awareness training for new employees? & \cmark & Control in Place \\
\rowcolor{tablerowcolor}
Does your organization do security awareness training for all employees at least once per year? & \xmark & \textcolor{orange}{\textbf{High Risk}} \\
\bottomrule
\end{tabular}
\end{table}

% =============================================================================
% |                           TECHNICAL SCAN RESULTS                          |
% =============================================================================
\section{Technical Scan Results}

An external network scan was performed using Nmap to identify accessible services on the organization's perimeter.

\begin{itemize}
    \item \textbf{Target IP:} \texttt{[Target IP]}
    \item \textbf{Scan Date:} \today
    \item \textbf{Summary:} The scan confirmed the target host is online. However, \textbf{no open ports were detected}. This indicates a well-configured firewall and a minimal external attack surface for this host, which is a significant security strength.
\end{itemize}

\begin{table}[h!]
\centering
\caption{Port Scan Details for Host \texttt{[Target IP]}}
\begin{tabular}{cccc}
\toprule
\rowcolor{tableheadcolor}
\textcolor{white}{\textbf{Port}} & \textcolor{white}{\textbf{State}} & \textcolor{white}{\textbf{Service}} & \textcolor{white}{\textbf{Product / Version}} \\
\midrule
80 & closed & http & N/A \\
\bottomrule
\end{tabular}
\end{table}

\paragraph{Analyst Note:} A pre-existing risk indicated that Port 80 was open. This scan's finding that Port 80 is `closed` suggests that the previously identified vulnerability has been successfully remediated.

% =============================================================================
% |                             RISK ASSESSMENT                               |
% =============================================================================
\section{Overall Risk Assessment}

This section synthesizes findings from all data sources into a prioritized list of risks.

\begin{table}[h!]
\centering
\caption{Synthesized Risk Register}
\begin{tabular}{p{0.1\linewidth} p{0.2\linewidth} p{0.5\linewidth} p{0.1\linewidth}}
\toprule
\rowcolor{tableheadcolor}
\textcolor{white}{\textbf{Risk ID}} & \textcolor{white}{\textbf{Risk Name}} & \textcolor{white}{\textbf{Description}} & \textcolor{white}{\textbf{Severity}} \\
\midrule
ORG-001 & \textbf{Email Account Compromise} & The lack of MFA on email accounts makes them highly susceptible to takeover via phishing or credential stuffing attacks. Compromised email can lead to data breaches and further network intrusion. & \textcolor{red}{\textbf{Critical}} \\
\midrule
\rowcolor{tablerowcolor}
ORG-002 & \textbf{Degraded Security Awareness} & Without mandatory annual training, employees are more likely to fall victim to social engineering tactics. This elevates the risk of malware infection and unauthorized access. & \textcolor{orange}{\textbf{High}} \\
\midrule
NET-001 & \textbf{Unencrypted Web Server} & A previous finding noted Port 80 was open. The current scan shows this port is closed. The risk is now considered remediated and requires no further action beyond validation. & \textcolor{green}{Resolved} \\
\bottomrule
\end{tabular}
\end{table}

% =============================================================================
% |                              RECOMMENDATIONS                              |
% =============================================================================
\section{Recommendations}

The following prioritized recommendations are provided to address the risks identified in this report.

\subsection{Critical Priority: Remediate Immediately}
\subsubsection*{Risk ID ORG-001: Implement MFA for Email Access}
\begin{description}
    \item[Action:] Enforce Multi-Factor Authentication (MFA) for all user access to the \texttt{[Domain]} email system. This should apply to webmail, desktop clients, and mobile devices.
    \item[Justification:] This is the single most effective control to prevent email account compromise. It provides a critical layer of defense even if user credentials are stolen, mitigating the primary vector for business email compromise (BEC) and phishing-based intrusions.
\end{description}

\subsection{High Priority: Remediate within 90 Days}
\subsubsection*{Risk ID ORG-002: Establish Annual Security Awareness Training}
\begin{description}
    \item[Action:] Procure or develop a security awareness training program and mandate its completion by all employees on an annual basis. The training should cover phishing, password security, social engineering, and acceptable use policies.
    \item[Justification:] A well-trained workforce is a critical component of a defense-in-depth strategy. Regular training reinforces security best practices, reduces the likelihood of human error, and fosters a strong security culture.
\end{description}

\subsection{Informational: For Validation}
\subsubsection*{Risk ID NET-001: Validate Remediation of Port 80}
\begin{description}
    \item[Action:] Confirm with the network or system administration team that the closure of Port 80 on host \texttt{[Target IP]} was an intentional and documented change.
    \item[Justification:] While the current state is secure, it is important to validate that security improvements are deliberate and permanent, rather than the result of a temporary service outage or misconfiguration.
\end{description}

\end{document}
% -----------------------------------------------------------------------------
% |                            END OF DOCUMENT                                |
% -----------------------------------------------------------------------------
```