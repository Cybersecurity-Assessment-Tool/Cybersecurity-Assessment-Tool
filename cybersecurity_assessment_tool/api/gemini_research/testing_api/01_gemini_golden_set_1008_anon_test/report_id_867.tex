```latex
\documentclass[12pt]{article}

% Preamble: Required Packages
\usepackage[margin=1in]{geometry}
\usepackage{pifont} % For checkmarks and crosses
\usepackage{booktabs} % For professional tables
\usepackage{hyperref} % For hyperlinks and metadata
\usepackage{url} % For formatting URLs
\usepackage{seqsplit} % For splitting long strings to prevent overflow
\usepackage{graphicx}
\usepackage{xcolor}

% Document Metadata
\hypersetup{
    colorlinks=true,
    linkcolor=blue,
    filecolor=magenta,      
    urlcolor=cyan,
    pdftitle={Cybersecurity Posture Assessment Report},
    pdfauthor={Cybersecurity Analyst},
    pdfsubject={Security Analysis},
    pdfkeywords={Cybersecurity, Nmap, Risk Assessment},
}

% Title Information
\title{Cybersecurity Posture Assessment Report}
\author{Cybersecurity Analyst}
\date{\today}

\begin{document}

\maketitle
\thispagestyle{empty}
\newpage

\tableofcontents
\thispagestyle{empty}
\newpage

\setcounter{page}{1}

% --- SECTION 1: EXECUTIVE OVERVIEW ---
\section{Executive Overview}

This report provides a comprehensive cybersecurity assessment for \textbf{[Organization Name]}, based on an analysis of network scan data, organizational security controls, and pre-existing risk information. The assessment was conducted to identify vulnerabilities, evaluate the current security posture, and provide actionable recommendations to mitigate identified risks.

The overall security posture is assessed as \textbf{High-Risk}. This is primarily due to the convergence of three critical findings:
\begin{enumerate}
    \item \textbf{Direct Public Exposure of a Critical Service:} A MySQL database server is directly accessible from the public internet. This exposes the organization to data breaches, denial-of-service attacks, and ransomware.
    \item \textbf{Use of End-of-Life Software:} The exposed MySQL database is running version 5.7.33, which reached its official End-of-Life (EOL) in October 2023. This version no longer receives security patches, leaving it perpetually vulnerable to newly discovered exploits.
    \item \textbf{Significant Policy and Training Gaps:} The organization lacks a formal Acceptable Use Policy and does not provide security awareness training to new employees. These foundational gaps create a permissive environment for human error and insider threats, significantly increasing the likelihood of a security incident.
\end{enumerate}

Immediate action is required to address the exposed database. Strategic initiatives must be undertaken to remediate the policy gaps and upgrade unsupported software.

% --- SECTION 2: ORGANIZATIONAL INFORMATION ---
\section{Organizational Information}

The following information was used as a baseline for this assessment. As key identifying data was not provided, placeholders have been used.

\begin{table}[h!]
\centering
\begin{tabular}{@{}ll@{}}
\toprule
\textbf{Attribute} & \textbf{Value} \\ \midrule
Organization Name    & \textbf{[Organization Name]} \\
Primary Domain       & \texttt{[Domain]} \\
External IP Address  & \texttt{[Client IP]} \\ \bottomrule
\end{tabular}
\caption{Client Organizational Details.}
\label{tab:org_info}
\end{table}

% --- SECTION 3: SECURITY CONTROL REVIEW ---
\section{Security Control Review (Questionnaire Analysis)}

An evaluation of the organization's administrative and procedural security controls was conducted via a questionnaire. The results, detailed in Table \ref{tab:controls}, highlight both strengths and critical weaknesses. While the organization has successfully implemented Multi-Factor Authentication (MFA) across key systems, there are alarming deficiencies in policy and employee training.

\begin{table}[h!]
\centering
\begin{tabular}{@{}lc@{}}
\toprule
\textbf{Control Question} & \textbf{Response} \\ \midrule
Do you require MFA to access email? & \ding{51} \\ % Yes
Do you require MFA to log into computers? & \ding{51} \\ % Yes
Do you require MFA to access sensitive data systems? & \ding{51} \\ % Yes
Does your organization have an employee acceptable use policy? & \textbf{\color{red}\ding{55}} \\ % No
Does your organization do security awareness training for new employees? & \textbf{\color{red}\ding{55}} \\ % No
Does your organization do security awareness training for all employees annually? & \ding{51} \\ % Yes
\bottomrule
\end{tabular}
\caption{Security Control Questionnaire Results (\ding{51}=Yes, \ding{55}=No).}
\label{tab:controls}
\end{table}

\subsection*{Analysis of Gaps}
The two "No" responses represent significant risks:
\begin{itemize}
    \item \textbf{Lack of Acceptable Use Policy (AUP):} An AUP is a foundational document that defines how employees may use company IT assets. Without it, there are no clear rules or consequences for misuse, potentially leading to data leakage, unauthorized software installation, and legal liabilities.
    \item \textbf{No Security Training for New Employees:} New hires are often targeted by social engineering attacks. Failing to provide immediate security training during onboarding leaves the organization highly vulnerable, as new employees are not equipped to recognize phishing, malware, or other common threats.
\end{itemize}

% --- SECTION 4: TECHNICAL SCAN RESULTS ---
\section{Technical Scan Results}

An external network scan was performed against the target IP address \texttt{[Target IP]} to identify open ports and exposed services.

\subsection*{Open Ports and Services}
The scan identified one open port, which indicates a publicly accessible database service. Details are provided in Table \ref{tab:scan_results}.

\begin{table}[h!]
\centering
\begin{tabular}{@{}lllll@{}}
\toprule
\textbf{Port} & \textbf{State} & \textbf{Service} & \textbf{Product} & \textbf{Version} \\ \midrule
3306/tcp      & open           & mysql            & MySQL            & 5.7.33           \\ \bottomrule
\end{tabular}
\caption{Nmap Scan Findings for Target: \texttt{[Target IP]}.}
\label{tab:scan_results}
\end{table}

\subsection*{Technical Analysis}
The presence of an open MySQL port (3306) is a critical security risk. This configuration allows any attacker on the internet to attempt to connect to the database, making it a prime target for:
\begin{itemize}
    \item \textbf{Brute-force attacks} to guess credentials.
    \item \textbf{Exploitation of known vulnerabilities} in the MySQL software.
    \item \textbf{Denial-of-Service (DoS) attacks}.
\end{itemize}

Compounding this risk, the identified version, \textbf{MySQL 5.7.33}, is \textbf{End-of-Life (EOL)} as of October 2023. EOL software no longer receives security updates from the vendor, meaning any vulnerabilities discovered after this date will remain unpatched. This transforms the exposed database from a high-risk configuration into a critical, actively exploitable vulnerability.

% --- SECTION 5: RISK ASSESSMENT SUMMARY ---
\section{Risk Assessment Summary}

The following table synthesizes findings from the security control review, technical scan, and pre-existing risk data into a consolidated list of key risks facing the organization.

\begin{table}[h!]
\centering
\begin{tabular}{@{}p{0.3\textwidth}p{0.15\textwidth}p{0.45\textwidth}@{}}
\toprule
\textbf{Risk Name} & \textbf{Severity} & \textbf{Overview} \\ \midrule
\textbf{Database Exposure} & \textbf{Critical} & The MySQL database on port 3306 is publicly accessible, inviting brute-force attacks and direct exploitation. This was confirmed by both the network scan and pre-existing risk data. \\
\addlinespace
\textbf{End-of-Life Software} & \textbf{Critical} & The exposed MySQL service is running version 5.7.33, which is no longer supported with security patches. This creates an unfixable attack surface for any future vulnerabilities discovered in that version. \\
\addlinespace
\textbf{Inadequate New Hire Training} & \textbf{High} & New employees are not provided with security awareness training, making them highly susceptible to social engineering and phishing attacks from day one. \\
\addlinespace
\textbf{Lack of Acceptable Use Policy} & \textbf{High} & The absence of a formal AUP creates ambiguity regarding the proper use of IT assets, increasing the risk of insider threat, data misuse, and non-compliance. \\ \bottomrule
\end{tabular}
\caption{Consolidated Risk Register.}
\label{tab:risk_summary}
\end{table}

% --- SECTION 6: RECOMMENDATIONS ---
\section{Recommendations}

The following recommendations are prioritized to provide a clear, actionable path toward improving the organization's security posture.

\subsection*{Priority 1: Critical (Immediate Action Required)}
\begin{enumerate}
    \item \textbf{Restrict Access to Port 3306:}
    \begin{itemize}
        \item \textbf{Immediate Fix:} Implement strict firewall rules to deny all public access to TCP port 3306. Access should only be permitted from trusted, internal IP addresses.
        \item \textbf{Long-Term Solution:} For remote or administrative access, implement a Virtual Private Network (VPN). All database connections from outside the local network should be routed exclusively through the VPN.
    \end{itemize}
    \item \textbf{Plan Database Upgrade:}
    \begin{itemize}
        \item \textbf{Action:} Initiate a project to migrate the MySQL 5.7.33 database to a currently supported version (e.g., MySQL 8.x). This is essential to ensure the system can receive future security patches.
    \end{itemize}
\end{enumerate}

\subsection*{Priority 2: High (Action Within 90 Days)}
\begin{enumerate}
    \item \textbf{Develop and Implement an Acceptable Use Policy (AUP):}
    \begin{itemize}
        \item \textbf{Action:} Draft a formal AUP that clearly outlines the rules for using company technology and data. The policy should be reviewed by management, communicated to all employees, and formally acknowledged via signature.
    \end{itemize}
    \item \textbf{Establish a New Hire Security Training Program:}
    \begin{itemize}
        \item \textbf{Action:} Integrate a mandatory security awareness training module into the new employee onboarding process. This training should cover key topics such as phishing, password security, data handling, and the new AUP.
    \end{itemize}
\end{enumerate}

\end{document}
```