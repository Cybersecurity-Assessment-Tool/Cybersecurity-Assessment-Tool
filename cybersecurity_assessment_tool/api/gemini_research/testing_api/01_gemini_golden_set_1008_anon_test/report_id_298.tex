```latex
\documentclass[12pt]{article}

% --- PACKAGES ---
\usepackage[margin=1in]{geometry}
\usepackage{pifont}         % For symbols like checkmarks
\usepackage{booktabs}       % For professional-looking tables
\usepackage{xcolor}         % For colored text
\usepackage{hyperref}       % For hyperlinks and document metadata
\usepackage{url}            % For formatting URLs
\usepackage{seqsplit}       % For breaking long strings in tt font

% --- DOCUMENT SETUP ---
\hypersetup{
    colorlinks=true,
    linkcolor=blue,
    filecolor=magenta,      
    urlcolor=cyan,
    pdftitle={Cybersecurity Posture Report},
    pdfauthor={Cybersecurity Analyst},
    pdfpagemode=FullScreen,
}

% --- CUSTOM COMMANDS ---
\newcommand{\cmark}{\textcolor{green}{\ding{51}}} % Green checkmark
\newcommand{\xmark}{\textcolor{red}{\ding{55}}}   % Red cross

% --- DOCUMENT START ---
\begin{document}

\title{Cybersecurity Posture Report \\ \large For \textbf{[Organization Name]}}
\author{Cybersecurity Analyst}
\date{\today}
\maketitle

\section*{1. Executive Summary}
This report provides an analysis of the cybersecurity posture for \textbf{[Organization Name]}, based on a review of organizational security controls, an external network scan, and pre-existing risk data. The assessment identified several critical and high-risk gaps in security policies and procedures, despite a positive finding from the external network scan.

The external network scan of the target IP address \texttt{[Target IP]} revealed no exposed services, indicating a well-configured perimeter firewall for that specific host. However, significant risks were identified from the organizational data review. Critical deficiencies include the lack of multi-factor authentication (MFA) for sensitive data systems, the absence of a formal Acceptable Use Policy (AUP), and a failure to provide security awareness training to new employees during onboarding. These gaps expose the organization to significant risks, including unauthorized access, data breaches, and insider threats.

Immediate remediation is recommended, focusing on the implementation of MFA, development of foundational security policies, and enhancement of the employee training program.

\section*{2. Organizational Information}
This assessment was conducted based on the following organizational details. As identity data was not provided, placeholders have been used.
\begin{itemize}
    \item \textbf{Organization Name:} \textbf{[Organization Name]}
    \item \textbf{Primary Email Domain:} \texttt{[Domain]}
    \item \textbf{Scanned External IP:} \texttt{[Client IP]}
\end{itemize}

\section*{3. Security Control Review}
The following table summarizes the organization's responses to a security controls questionnaire. Items marked with a red \xmark\ indicate a deviation from security best practices and represent a potential risk.

\begin{table}[h!]
\centering
\begin{tabular}{p{0.75\linewidth} c}
\toprule
\textbf{Control Question} & \textbf{Response} \\
\midrule
Do you require MFA to access email? & \cmark \\
Do you require MFA to log into computers? & \cmark \\
Do you require MFA to access sensitive data systems? & \xmark \\
Does your organization have an employee acceptable use policy? & \xmark \\
Does your organization do security awareness training for new employees? & \xmark \\
Does your organization do security awareness training for all employees at least once per year? & \cmark \\
\bottomrule
\end{tabular}
\caption{Organizational Security Controls Questionnaire Results.}
\label{tab:controls}
\end{table}

\section*{4. Technical Scan Results}
An external network vulnerability scan was conducted to identify exposed services and potential vulnerabilities on the organization's perimeter.

\begin{itemize}
    \item \textbf{Target IP Address:} \texttt{[Target IP]}
    \item \textbf{Scan Date:} The scan date was not provided in the input data.
\end{itemize}

\subsection*{Findings}
The scan completed successfully. \textbf{No open ports or exposed services were detected on the target system.} This is a positive security finding, suggesting that the host is either offline or protected by a properly configured firewall that denies all unsolicited inbound traffic.

\section*{5. Risk Assessment}
The following table synthesizes findings from the security control review and technical scan. No pre-existing vulnerabilities were provided for this assessment. The risks identified below are derived directly from the current analysis.

\begin{table}[h!]
\centering
\begin{tabular}{p{0.25\linewidth} p{0.5\linewidth} l}
\toprule
\textbf{Risk Name} & \textbf{Overview} & \textbf{Severity} \\
\midrule
Lack of MFA on Sensitive Systems & The absence of MFA on systems containing sensitive data significantly increases the risk of unauthorized access via compromised credentials. An attacker with a valid username and password can gain direct access. & \textcolor{red}{Critical} \\
\addlinespace
Missing Foundational Security Policies & The organization lacks a formal Acceptable Use Policy (AUP). This creates ambiguity for employees regarding safe technology use and limits the organization's ability to enforce security standards. & \textcolor{orange}{High} \\
\addlinespace
Inadequate Employee Onboarding Security Training & New employees do not receive security awareness training upon being hired. This oversight leaves the organization vulnerable to immediate human error, as new staff are not equipped with knowledge to avoid common threats like phishing. & \textcolor{orange}{High} \\
\bottomrule
\end{tabular}
\caption{Identified Risks and Severity.}
\label{tab:risks}
\end{table}

\section*{6. Recommendations}
To mitigate the identified risks and improve the overall security posture, the following actions are recommended with urgency:

\begin{enumerate}
    \item \textbf{Implement MFA for Sensitive Systems (Critical Priority):} Immediately deploy mandatory multi-factor authentication for all user accounts (including administrative and service accounts) that have access to systems storing or processing sensitive or critical data.
    
    \item \textbf{Develop and Implement an Acceptable Use Policy (AUP) (High Priority):} Create a formal AUP that clearly defines the rules and expectations for employee use of company technology, data, and network resources. This policy should be communicated to all employees and signed as a condition of employment.
    
    \item \textbf{Integrate Security Training into Onboarding (High Priority):} Develop a mandatory security awareness training module and integrate it into the new employee onboarding process. This ensures that all staff understand their security responsibilities from their first day.
\end{enumerate}

\section*{7. Conclusion}
While \textbf{[Organization Name]} demonstrates a strong perimeter defense on the scanned asset, critical administrative and procedural controls are lacking. The identified gaps in MFA, policy, and training represent significant and immediate risks to the organization's data and systems. We strongly advise prioritizing the recommendations outlined in this report to build a more resilient and comprehensive security program.

\end{document}
```