```latex
\documentclass[12pt]{article}

% Preamble: Required Packages
\usepackage[margin=1in]{geometry}
\usepackage{pifont} % For checkmarks and crosses
\usepackage{booktabs} % For professional tables
\usepackage{hyperref} % For clickable links
\usepackage{url} % For URL formatting
\usepackage{seqsplit} % For splitting long strings
\usepackage[utf8]{inputenc}
\usepackage{graphicx}
\usepackage{xcolor}

% Document Metadata
\title{Cybersecurity Posture Assessment Report}
\author{Cybersecurity Analysis Division}
\date{\today}

% Hyperref Setup
\hypersetup{
    colorlinks=true,
    linkcolor=blue,
    filecolor=magenta,      
    urlcolor=cyan,
    pdftitle={Cybersecurity Posture Assessment Report},
    pdfpagemode=FullScreen,
}

\begin{document}

\maketitle
\hrule
\vspace{1em}

% Executive Overview Section
\section*{Executive Overview}
This report provides a comprehensive analysis of the cybersecurity posture for \textbf{[Organization Name]}, based on a review of organizational security controls, an external network scan, and pre-existing risk data.

The assessment reveals a mixed security posture. On the technical front, the external network scan of the target IP address showed a strong defensive configuration, with no open ports detected. This significantly reduces the external attack surface and is a commendable finding.

However, the review of organizational security controls identified several critical administrative and access control gaps. The most severe finding is the lack of multi-factor authentication (MFA) for accessing sensitive data systems. Additionally, the absence of a formal Acceptable Use Policy (AUP) and the lack of security awareness training for new employees represent foundational weaknesses that increase the risk of insider threats and policy violations.

Immediate remediation should focus on implementing MFA for sensitive systems and establishing core security policies and training programs to address these high-priority risks.

% Organizational Information Section
\section{Organizational Information}
The following details were used as the basis for this assessment.
\begin{itemize}
    \item \textbf{Organization Name:} \textbf{[Organization Name]}
    \item \textbf{Primary Domain:} \texttt{[Domain]}
    \item \textbf{Scanned External IP:} \texttt{[Client IP]}
\end{itemize}

% Security Control Review Section
\section{Security Control Review}
An assessment of administrative and procedural controls was conducted via a security questionnaire. The responses indicate key areas of strength and weakness in the organization's security policies. "No" answers highlight significant gaps that require immediate attention.

\begin{table}[h!]
\centering
\caption{Security Controls Questionnaire Analysis}
\begin{tabular}{p{0.6\linewidth} c p{0.2\linewidth}}
\toprule
\textbf{Control Question} & \textbf{Response} & \textbf{Assessment} \\
\midrule
Do you require MFA to access email? & \ding{51} Yes & Best Practice Met \\
Do you require MFA to log into computers? & \ding{51} Yes & Best Practice Met \\
Do you require MFA to access sensitive data systems? & \ding{55} No & \textbf{Critical Gap} \\
Does your organization have an employee acceptable use policy? & \ding{55} No & \textbf{High Risk} \\
Does your organization do security awareness training for new employees? & \ding{55} No & \textbf{High Risk} \\
Does your organization do security awareness training for all employees at least once per year? & \ding{51} Yes & Best Practice Met \\
\bottomrule
\end{tabular}
\end{table}

% Technical Scan Results Section
\section{Technical Scan Results}
An external network vulnerability scan was performed to identify potential weaknesses visible from the public internet.

\begin{itemize}
    \item \textbf{Target IP Address:} \texttt{[Target IP]}
    \item \textbf{Scan Date:} \today
    \item \textbf{Scan Tool:} Nmap
\end{itemize}

\subsection*{Key Findings}
The scan reported that all 1000 of the most common TCP ports were in a \textbf{closed} state.

\textbf{Analysis:} No open ports were detected on the target system. This is a positive security finding, indicating a strong firewall configuration that effectively minimizes the external attack surface. It significantly reduces the likelihood of an external attacker discovering and exploiting network services.

% Risk Assessment Section
\section{Risk Assessment}
The following table synthesizes findings from the security control review and technical scan. Since no pre-existing vulnerabilities were provided, this assessment is based solely on new findings from this engagement.

\begin{table}[h!]
\centering
\caption{Identified Risks and Severity}
\begin{tabular}{p{0.1\linewidth} p{0.25\linewidth} p{0.4\linewidth} l}
\toprule
\textbf{Risk ID} & \textbf{Risk Name} & \textbf{Description} & \textbf{Severity} \\
\midrule
RISK-001 & Lack of MFA on Sensitive Systems & The absence of MFA for accessing critical data stores allows for unauthorized access via compromised credentials alone. & \textbf{Critical} \\
\addlinespace
RISK-002 & Missing Acceptable Use Policy (AUP) & Without a formal AUP, employees lack clear guidelines on the acceptable use of company assets, increasing the risk of misuse and insider threat. & \textbf{High} \\
\addlinespace
RISK-003 & No Security Training for New Hires & New employees are not provided with security awareness training during onboarding, leaving a critical window where they are more susceptible to social engineering and policy violations. & \textbf{High} \\
\bottomrule
\end{tabular}
\end{table}

% Recommendations Section
\section{Recommendations}
Based on the risks identified in this report, the following actions are recommended to improve the organization's cybersecurity posture. Recommendations are prioritized by severity.

\begin{description}
    \item[\textbf{1. (RISK-001 - Critical)}] \textbf{Implement MFA for Sensitive Systems:}
    Deploy a mandatory multi-factor authentication solution for all user accounts (including administrative and service accounts) that have access to systems containing sensitive, confidential, or proprietary data. This is the single most effective control to mitigate credential theft.

    \item[\textbf{2. (RISK-002 - High)}] \textbf{Develop and Implement an Acceptable Use Policy (AUP):}
    Create a formal AUP that clearly defines the rules and responsibilities for all users regarding the use of company networks, systems, and data. This policy should be reviewed by legal counsel, communicated to all employees, and formally acknowledged via signature.

    \item[\textbf{3. (RISK-003 - High)}] \textbf{Integrate Security Training into Onboarding:}
    Establish a mandatory security awareness training module as part of the new employee onboarding process. This training should cover key topics such as phishing, password security, data handling, and the new AUP. This ensures a baseline of security knowledge from day one.
\end{description}

\end{document}
```