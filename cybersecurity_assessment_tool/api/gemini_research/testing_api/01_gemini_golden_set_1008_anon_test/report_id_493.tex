```latex
\documentclass[12pt]{article}

% Preamble: Required Packages
\usepackage[margin=1in]{geometry}
\usepackage{pifont} % For checkmarks and crosses
\usepackage{booktabs} % For professional tables
\usepackage{hyperref} % For hyperlinks
\usepackage{url} % For URL formatting
\usepackage{seqsplit} % To split long strings in tt font

% Document Metadata
\title{Cybersecurity Posture Assessment Report}
\author{Cybersecurity Analysis Division}
\date{November 22, 2025}

\begin{document}

\maketitle
\thispagestyle{empty}
\newpage
\tableofcontents
\newpage

% ==============================================================================
\section{Executive Summary}
% ==============================================================================

This report provides a comprehensive analysis of the cybersecurity posture for \textbf{[Organization Name]}, based on data collected on November 22, 2025. The assessment combines a review of organizational security controls, an external network scan, and an evaluation of pre-existing risks.

The analysis has identified several critical and high-risk security gaps that require immediate attention. The most significant findings include:
\begin{itemize}
    \item \textbf{Critical Control Gaps:} The absence of Multi-Factor Authentication (MFA) for email access and computer logins presents a critical risk of account compromise and unauthorized access to internal systems.
    \item \textbf{High-Risk Technical Vulnerability:} The external-facing web server at \texttt{[Target IP]} is running an outdated and unsupported version of Nginx (1.18.0). This software is known to have multiple security vulnerabilities, exposing the organization to potential compromise.
    \item \textbf{High-Risk Policy Gaps:} The lack of a formal Acceptable Use Policy and the absence of annual security awareness training for all employees significantly weaken the organization's human firewall, making it more susceptible to social engineering and phishing attacks.
\end{itemize}

The overall security posture is assessed as high-risk. This report details these findings and provides prioritized, actionable recommendations to mitigate the identified risks and strengthen the organization's defenses.

% ==============================================================================
\section{Organizational Information}
% ==============================================================================

The following information was used as the basis for this assessment. Due to the anonymized nature of the provided data, placeholders have been used where necessary.

\begin{table}[h!]
\centering
\begin{tabular}{@{}ll@{}}
\toprule
\textbf{Attribute} & \textbf{Value} \\ \midrule
Organization Name & \textbf{[Organization Name]} \\
Primary Email Domain & \texttt{[Domain]} \\
Monitored External IP & \texttt{[Client IP]} \\
Assessment Date & 2025-11-22 \\ \bottomrule
\end{tabular}
\caption{Client and Assessment Details.}
\end{table}

% ==============================================================================
\section{Security Control Review}
% ==============================================================================

A review of the organization's security controls was conducted via a standardized questionnaire. The responses indicate significant gaps in fundamental security practices. A summary of the findings is presented below.

\begin{table}[h!]
\centering
\begin{tabular}{@{}p{0.6\linewidth}cc@{}}
\toprule
\textbf{Control Question} & \textbf{Response} & \textbf{Assessment} \\ \midrule
Do you require MFA to access email? & \ding{55} No & Critical Gap \\
Do you require MFA to log into computers? & \ding{55} No & Critical Gap \\
Do you require MFA to access sensitive data systems? & \ding{51} Yes & Good Practice \\
Does your organization have an employee acceptable use policy? & \ding{55} No & High Risk \\
Does your organization do security awareness training for new employees? & \ding{51} Yes & Good Practice \\
Does your organization do security awareness training for all employees at least once per year? & \ding{55} No & High Risk \\ \bottomrule
\end{tabular}
\caption{Security Controls Questionnaire Analysis.}
\end{table}

% ==============================================================================
\section{Technical Scan Results}
% ==============================================================================

An external network scan was performed against the target IP address \texttt{[Target IP]} on November 22, 2025. The scan identified one open port with a service running an outdated software version.

\subsection{Open Ports and Services}
The following table details the services exposed to the public internet.

\begin{table}[h!]
\centering
\begin{tabular}{@{}llll@{}}
\toprule
\textbf{Port} & \textbf{State} & \textbf{Service} & \textbf{Product \& Version} \\ \midrule
443/tcp & Open & https & Nginx 1.18.0 \\ \bottomrule
\end{tabular}
\caption{Network Scan Findings for Target: \texttt{[Target IP]}.}
\end{table}

\subsection{Analysis of Findings}
The scan identified an Nginx web server, version 1.18.0, exposed on port 443 (HTTPS). This version was released in April 2020 and reached its end-of-life in May 2021. It is no longer supported with security patches and has several known vulnerabilities, including but not limited to CVE-2021-23017. Running outdated software on internet-facing systems poses a high risk of exploitation by automated and targeted attacks.

% ==============================================================================
\section{Overall Risk Assessment}
% ==============================================================================

This section synthesizes the findings from the security control review and the technical scan. No pre-existing vulnerabilities were reported. The following new risks have been identified and prioritized.

\begin{table}[h!]
\centering
\begin{tabular}{@{}lp{0.6\linewidth}l@{}}
\toprule
\textbf{ID} & \textbf{Risk Description} & \textbf{Severity} \\ \midrule
RISK-001 & Lack of MFA for email and endpoint access allows for straightforward account takeovers if credentials are stolen, phished, or brute-forced. & \textbf{Critical} \\
\addlinespace
RISK-002 & The external web server runs an outdated and vulnerable version of Nginx (1.18.0), making it a prime target for automated exploitation. & \textbf{High} \\
\addlinespace
RISK-003 & Lack of annual security awareness training for all staff increases susceptibility to phishing and other social engineering attacks. & \textbf{High} \\
\addlinespace
RISK-004 & Absence of a formal Acceptable Use Policy (AUP) creates ambiguity regarding secure employee behavior and limits the organization's ability to enforce security standards. & \textbf{High} \\ \bottomrule
\end{tabular}
\caption{Summary of Identified Risks.}
\end{table}

% ==============================================================================
\section{Recommendations}
% ==============================================================================

The following actions are recommended to mitigate the identified risks. Recommendations are prioritized based on severity and potential impact.

\subsection{Priority 1: Remediate Critical Risks}
\begin{description}
    \item[RISK-001:] \textbf{Implement MFA Immediately.}
    \begin{itemize}
        \item Enforce MFA for all user accounts across all critical platforms, starting with email (e.g., Office 365, Google Workspace) and endpoint logins (e.g., Windows Hello, Duo).
        \item This is the single most effective control to prevent unauthorized access and should be treated as an emergency priority.
    \end{itemize}
\end{description}

\subsection{Priority 2: Remediate High Risks}
\begin{description}
    \item[RISK-002:] \textbf{Upgrade Web Server Software.}
    \begin{itemize}
        \item Plan and execute an upgrade of the Nginx server at \texttt{[Target IP]} from version 1.18.0 to the latest stable version.
        \item Establish a patch management policy to ensure all internet-facing systems are updated in a timely manner.
    \end{itemize}
    \item[RISK-003:] \textbf{Establish Annual Security Training.}
    \begin{itemize}
        \item Develop or procure a security awareness training program and make it mandatory for all employees to complete it annually.
        \item The training should cover key topics such as phishing, password security, and data handling.
    \end{itemize}
    \item[RISK-004:] \textbf{Develop and Implement an AUP.}
    \begin{itemize}
        \item Draft a formal Acceptable Use Policy that clearly defines the rules for using company IT assets.
        \item Require all employees to read and acknowledge the policy as a condition of their employment.
    \end{itemize}
\end{description}

\end{document}
```