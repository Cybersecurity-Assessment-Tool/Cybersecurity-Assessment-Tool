```latex
\documentclass[12pt]{article}

% Required Packages
\usepackage[margin=1in]{geometry}
\usepackage{pifont}         % For symbols like checkmarks and crosses
\usepackage{booktabs}       % For professional-quality tables
\usepackage{hyperref}       % For creating hyperlinks within the document
\usepackage{url}            % For formatting URLs
\usepackage{seqsplit}       % For splitting long, unbreakable strings
\usepackage{xcolor}         % For text coloring

% --- Document Setup ---
\hypersetup{
    colorlinks=true,
    linkcolor=blue,
    filecolor=magenta,
    urlcolor=cyan,
    pdftitle={Cybersecurity Posture Assessment Report},
    pdfauthor={Cybersecurity Analyst},
}

% --- Custom Commands ---
\newcommand{\yes}{\ding{51}} % Checkmark
\newcommand{\no}{\ding{55}}  % Cross mark

% --- Document Start ---
\begin{document}

\title{Cybersecurity Posture Assessment Report \\ \large For: \textbf{[Organization Name]}}
\author{Cybersecurity Analyst}
\date{\today}
\maketitle

\section*{Executive Summary}
This report provides a cybersecurity posture assessment for \textbf{[Organization Name]}, based on a review of organizational security controls, an external network scan, and pre-existing risk data. The analysis reveals several critical and high-risk security gaps that expose the organization to significant threats, including account compromise, unauthorized access, and data breaches.

Key findings include a systemic lack of Multi-Factor Authentication (MFA) across all critical services, the absence of foundational security policies, and an incomplete security awareness training program. Compounding these issues, a technical scan identified an exposed Secure Shell (SSH) port on an external-facing asset.

Immediate and decisive action is required to remediate these vulnerabilities. This report outlines prioritized recommendations to strengthen the organization's security posture and mitigate the identified risks.

\section{Organizational Information}
The following details were used as the basis for this assessment. As identity data was not provided, placeholders have been used.
\begin{itemize}
    \item \textbf{Organization Name:} \textbf{[Organization Name]}
    \item \textbf{Primary Domain:} \texttt{[Domain]}
    \item \textbf{Scanned Asset IP:} \texttt{[Client IP]}
\end{itemize}

\section{Security Control Review}
A review of the organization's security controls was conducted via a standardized questionnaire. The responses indicate significant gaps in fundamental security practices, particularly concerning identity and access management and corporate policy.

\begin{table}[h!]
\centering
\begin{tabular}{p{0.6\textwidth} c l}
\toprule
\textbf{Control Question} & \textbf{Response} & \textbf{Assessment} \\
\midrule
Do you require MFA to access email? & \no & \textcolor{red}{Critical Gap} \\
Do you require MFA to log into computers? & \no & \textcolor{orange}{High Risk} \\
Do you require MFA to access sensitive data systems? & \no & \textcolor{red}{Critical Gap} \\
Does your organization have an employee acceptable use policy? & \no & \textcolor{orange}{High Risk} \\
Does your organization do security awareness training for new employees? & \yes & Control in Place \\
Does your organization do security awareness training for all employees at least once per year? & \no & \textcolor{orange}{High Risk} \\
\bottomrule
\end{tabular}
\caption{Security Controls Questionnaire Analysis}
\end{table}

\section{Technical Scan Results}
An external network scan was performed on the client's provided IP address to identify exposed services. The scan data did not include a specific scan date.

\subsection{Target Information}
\begin{itemize}
    \item \textbf{Target IP:} \texttt{[Target IP]}
    \item \textbf{Status:} Host is Up
\end{itemize}

\subsection{Open Ports}
The following ports were found to be open and accessible from the public internet. Exposing management services like SSH is a significant security risk.

\begin{table}[h!]
\centering
\begin{tabular}{c c l p{0.5\textwidth}}
\toprule
\textbf{Port} & \textbf{State} & \textbf{Service} & \textbf{Notes} \\
\midrule
22/tcp & Open & SSH & The Secure Shell service is exposed. This is a common vector for brute-force and credential stuffing attacks. Version information was not available from the provided scan data. \\
\bottomrule
\end{tabular}
\caption{Open Port Findings}
\end{table}

\section{Risk Assessment}
Based on the correlation of the security control review and technical scan results, the following risks have been identified. The provided data indicated no pre-existing risks.

\begin{table}[h!]
\centering
\begin{tabular}{p{0.1\textwidth} p{0.25\textwidth} p{0.45\textwidth} l}
\toprule
\textbf{Risk ID} & \textbf{Risk Name} & \textbf{Description} & \textbf{Severity} \\
\midrule
RISK-001 & Widespread Lack of Multi-Factor Authentication & The absence of MFA for email, endpoints, and sensitive data systems creates a high risk of account takeover and unauthorized access if user credentials are compromised. & \textcolor{red}{Critical} \\
\addlinespace
RISK-002 & Exposed SSH Management Port & The SSH port (22) is open to the internet, making the server a target for automated brute-force attacks. This risk is amplified by the lack of MFA controls. & \textcolor{orange}{High} \\
\addlinespace
RISK-003 & Missing Foundational Security Policies & The lack of an Acceptable Use Policy results in an undefined security baseline for employees, increasing the likelihood of insecure practices. & \textcolor{orange}{High} \\
\addlinespace
RISK-004 & Inadequate Security Awareness Program & While new hires receive training, the lack of annual refresher training for all employees allows security knowledge to become outdated, making them more susceptible to phishing and social engineering. & \textcolor{orange}{High} \\
\bottomrule
\end{tabular}
\caption{Summary of Identified Risks}
\end{table}

\section{Recommendations}
The following actions are recommended to mitigate the identified risks and improve the overall security posture of \textbf{[Organization Name]}. Recommendations are prioritized by severity.

\begin{enumerate}
    \item \textbf{[Critical] Immediately Implement Multi-Factor Authentication (MFA):}
    \begin{itemize}
        \item Prioritize the deployment of MFA on all externally-facing services, especially email (e.g., Office 365, Google Workspace).
        \item Extend MFA requirements to all systems handling sensitive data and for all remote access solutions (e.g., VPNs).
        \item Develop a roadmap for deploying MFA for all employee computer logins.
    \end{itemize}

    \item \textbf{[High] Secure Exposed Network Services:}
    \begin{itemize}
        \item Restrict access to the SSH port (22) on asset \texttt{[Target IP]} to a whitelist of trusted administrative IP addresses.
        \item If SSH access is required, enforce the use of strong, key-based authentication and disable password-based authentication entirely.
        \item Conduct a comprehensive vulnerability scan on the asset to identify any other potential weaknesses.
    \end{itemize}

    \item \textbf{[High] Develop and Enforce Foundational Security Policies:}
    \begin{itemize}
        \item Create and formally adopt an Employee Acceptable Use Policy (AUP) that defines rules for using company assets, data handling, and internet usage.
        \item Establish a strong password policy that mandates complexity, length, and regular rotation.
    \end{itemize}

    \item \textbf{[High] Enhance the Security Awareness Program:}
    \begin{itemize}
        \item Implement a mandatory annual security awareness training program for all employees, covering topics such as phishing, password hygiene, and social engineering.
        \item Supplement annual training with regular phishing simulations to test and reinforce employee knowledge.
    \end{itemize}
\end{enumerate}

\end{document}
```