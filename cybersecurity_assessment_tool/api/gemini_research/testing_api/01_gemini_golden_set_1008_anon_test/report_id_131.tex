```latex
\documentclass[12pt]{article}

% 1. Preamble & Package Inclusion
\usepackage[margin=1in]{geometry}
\usepackage{pifont} % For checkmarks and crosses
\usepackage{booktabs} % For professional tables
\usepackage{hyperref} % For hyperlinks
\usepackage{url}      % For URL formatting
\usepackage{seqsplit} % For splitting long text strings
\usepackage{graphicx}
\usepackage{xcolor}

% --- Document Metadata ---
\title{Cybersecurity Posture Assessment Report for \textbf{[Organization Name]}}
\author{Cybersecurity Analysis Division}
\date{\today}

% --- Hyperref Setup ---
\hypersetup{
    colorlinks=true,
    linkcolor=blue,
    filecolor=magenta,      
    urlcolor=cyan,
    pdftitle={Cybersecurity Posture Assessment Report},
    pdfpagemode=FullScreen,
}

\begin{document}

\maketitle
\thispagestyle{empty}
\newpage

\tableofcontents
\newpage

% 2. Executive Overview
\section{Executive Overview}
This report provides a comprehensive analysis of the cybersecurity posture for \textbf{[Organization Name]}. The assessment is based on a correlation of network scan data, a security controls questionnaire, and a review of pre-existing risks.

The overall security posture is determined to be \textbf{critically weak}. The analysis revealed a complete absence of fundamental security controls, including Multi-Factor Authentication (MFA) across all critical systems and a lack of foundational security policies and employee training. These organizational gaps are compounded by technical vulnerabilities, such as an exposed management service (SSH on port 22), which is a common target for automated attacks.

Immediate and decisive action is required to mitigate the high probability of a security breach. The recommendations outlined in this report prioritize the most critical vulnerabilities to establish a baseline level of security and protect key organizational assets.

% 3. Organizational Information
\section{Organizational Information}
The following details were used as the basis for this assessment. Due to missing data in the provided inputs, placeholders have been used.

\begin{itemize}
    \item \textbf{Organization Name:} \textbf{[Organization Name]}
    \item \textbf{Primary Domain:} \texttt{[Domain]}
    \item \textbf{Scanned IP Address:} \texttt{[Client IP]}
\end{itemize}

% 4. Security Control Review (from Questionnaire)
\section{Security Control Review}
The following table summarizes the organization's responses to a security controls questionnaire. A green checkmark (\ding{51}) indicates a positive control is in place, while a red cross (\ding{55}) indicates a significant gap.

\vspace{1em}
\begin{center}
\begin{tabular}{p{0.7\linewidth} c}
\toprule
\textbf{Control Question} & \textbf{Response} \\
\midrule
Do you require MFA to access email? & \textcolor{red}{\ding{55}} \\
Do you require MFA to log into computers? & \textcolor{red}{\ding{55}} \\
Do you require MFA to access sensitive data systems? & \textcolor{red}{\ding{55}} \\
Does your organization have an employee acceptable use policy? & \textcolor{red}{\ding{55}} \\
Does your organization do security awareness training for new employees? & \textcolor{red}{\ding{55}} \\
Does your organization do security awareness training for all employees at least once per year? & \textcolor{red}{\ding{55}} \\
\bottomrule
\end{tabular}
\end{center}

\subsection*{Analysis of Controls}
The review indicates a complete failure to implement basic, yet essential, security controls. The lack of MFA for email, workstations, and sensitive data systems presents a critical risk of account takeover. Furthermore, the absence of an acceptable use policy and any form of security awareness training leaves the organization highly susceptible to human error, phishing attacks, and insider threats.

% 5. Technical Scan Results
\section{Technical Scan Results}
An external network scan was performed to identify exposed services on the public-facing infrastructure.

\begin{itemize}
    \item \textbf{Scan Target:} \texttt{[Target IP]}
    \item \textbf{Host Status:} Up
\end{itemize}

\subsection*{Open Ports}
The following table details the ports found to be open and accessible from the internet.

\vspace{1em}
\begin{center}
\begin{tabular}{l l l l}
\toprule
\textbf{Port} & \textbf{State} & \textbf{Service} & \textbf{Product / Version} \\
\midrule
22/tcp & open & ssh & Not Disclosed \\
\bottomrule
\end{tabular}
\end{center}

\subsection*{Analysis of Technical Findings}
The scan identified that port 22 (SSH - Secure Shell) is open to the public internet. While SSH is a secure protocol for remote administration, its exposure is a significant security risk. This service is a primary target for automated brute-force attacks, where threat actors attempt to guess usernames and passwords to gain unauthorized access to the server. Without proper controls, such as IP whitelisting or fail2ban, this exposed service represents a direct vector for a network compromise.

% 6. Consolidated Risk Assessment
\section{Consolidated Risk Assessment}
The following table synthesizes findings from the security questionnaire, technical scans, and pre-existing risk data into a prioritized list of security risks.

\vspace{1em}
\begin{center}
\begin{tabular}{p{0.25\linewidth} p{0.5\linewidth} p{0.15\linewidth}}
\toprule
\textbf{Risk Name} & \textbf{Description} & \textbf{Severity} \\
\midrule
\textbf{Localhost Exposed} & Pre-existing critical vulnerability identified. Specifics require further investigation. Affected Element: \texttt{[Target IP]}. & \textbf{Critical (10.0)} \\
\addlinespace
\textbf{Absence of Multi-Factor Authentication (MFA)} & The lack of MFA on email, computers, and data systems allows for account compromise with only a single factor (a password), which can be easily stolen or guessed. & \textbf{Critical} \\
\addlinespace
\textbf{Exposed SSH Management Port} & Port 22 is open to the internet, inviting automated brute-force and credential stuffing attacks that can lead to a full server compromise. & \textbf{High} \\
\addlinespace
\textbf{Lack of Security Policies and Training} & Without an Acceptable Use Policy or security training, employees are unaware of their responsibilities and are more likely to fall victim to phishing and social engineering attacks. & \textbf{High} \\
\bottomrule
\end{tabular}
\end{center}

% 7. Recommendations
\section{Recommendations}
To address the identified risks, the following actions are recommended, prioritized by severity.

\subsection*{Immediate Actions (Critical Priority)}
\begin{itemize}
    \item \textbf{Implement MFA:} Immediately enable MFA for all users on all critical systems, starting with email, VPN access, and access to sensitive data repositories. This is the single most effective control to prevent account takeovers.
    \item \textbf{Restrict SSH Access:} Configure the network firewall to restrict access to port 22. Access should only be permitted from a limited set of trusted IP addresses (an IP whitelist). If public access is required, implement additional controls like key-based authentication and disable password-based logins.
\end{itemize}

\subsection*{Short-Term Actions (High Priority)}
\begin{itemize}
    \item \textbf{Develop an Acceptable Use Policy (AUP):} Draft and implement an AUP that clearly defines the rules for using company IT assets, data, and internet access. All employees must read and acknowledge this policy.
    \item \textbf{Deploy Security Awareness Training:} Enroll all employees in a mandatory security awareness training program. This program should cover key topics such as phishing, password security, and data handling. Training should be required for all new hires and conducted annually for all staff.
\end{itemize}

\subsection*{Long-Term Strategic Actions}
\begin{itemize}
    \item \textbf{Establish a Vulnerability Management Program:} Conduct regular, automated vulnerability scans of all internal and external assets to proactively identify and remediate security weaknesses.
    \item \textbf{Implement Centralized Logging:} Deploy a Security Information and Event Management (SIEM) solution or a centralized logging server to collect and monitor security events from across the network. This provides visibility and enables incident detection and response.
\end{itemize}

\end{document}
```