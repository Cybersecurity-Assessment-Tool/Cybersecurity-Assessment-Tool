```latex
\documentclass[12pt]{article}

% Preamble: Required Packages
\usepackage[margin=1in]{geometry}
\usepackage{pifont} % For checkmarks and crosses
\usepackage{booktabs} % For professional tables
\usepackage{hyperref} % For hyperlinks
\usepackage{url} % For URL formatting
\usepackage{seqsplit} % For splitting long strings
\usepackage{graphicx}
\usepackage{xcolor}

% Document Metadata
\title{Cybersecurity Assessment Report}
\author{Cybersecurity Analysis Division}
\date{\today}

% Hyperref Setup
\hypersetup{
    colorlinks=true,
    linkcolor=blue,
    filecolor=magenta,      
    urlcolor=cyan,
    pdftitle={Cybersecurity Assessment Report},
    pdfpagemode=FullScreen,
}

\begin{document}

\maketitle
\thispagestyle{empty}
\newpage

\tableofcontents
\newpage

% --- 1. Executive Summary ---
\section{Executive Summary}

This report details the findings of a cybersecurity assessment conducted for \textbf{[Organization Name]}. The analysis correlates data from an external network scan, a security controls questionnaire, and a list of pre-existing known risks.

The assessment has identified a \textbf{critical-risk} security posture. A primary finding is the direct exposure of Remote Desktop Protocol (RDP) on port 3389 to the public internet on host \texttt{[Target IP]}. This vulnerability, with a CVSS score of 9.0, presents a significant and immediate threat of unauthorized access, potentially leading to a full system compromise or ransomware attack.

Furthermore, significant gaps were identified in the organization's internal security controls. The absence of Multi-Factor Authentication (MFA) for email and sensitive data systems, coupled with a lack of a formal security awareness training program, drastically increases the risk of a successful phishing or credential theft attack.

Immediate remediation of the exposed RDP service is paramount. Following this, the implementation of MFA and the establishment of a comprehensive security awareness program are strongly recommended to mitigate the identified high-risk vulnerabilities.

% --- 2. Organizational Information ---
\section{Organizational Information}
This section provides the key identification details for the organization under review. The data has been anonymized for this report.

\begin{itemize}
    \item \textbf{Organization Name:} \textbf{[Organization Name]}
    \item \textbf{Primary Email Domain:} \texttt{[Domain]}
    \item \textbf{Scanned External IP Address:} \texttt{[Client IP]}
\end{itemize}

% --- 3. Security Control Review ---
\section{Security Control Review (Questionnaire Analysis)}
An analysis of the security questionnaire reveals several areas of concern where current practices do not align with security best practices. "No" answers indicate significant control gaps that increase organizational risk.

\begin{table}[h!]
\centering
\caption{Security Controls Questionnaire Results}
\begin{tabular}{p{0.6\linewidth} c l}
\toprule
\textbf{Control Question} & \textbf{Response} & \textbf{Assessment} \\
\midrule
Do you require MFA to access email? & \ding{55} & \textcolor{red}{\textbf{Critical Gap}} \\
Do you require MFA to log into computers? & \ding{51} & Good Practice \\
Do you require MFA to access sensitive data systems? & \ding{55} & \textcolor{red}{\textbf{Critical Gap}} \\
Does your organization have an employee acceptable use policy? & \ding{51} & Good Practice \\
Does your organization do security awareness training for new employees? & \ding{55} & \textcolor{orange}{High Risk} \\
Does your organization do security awareness training for all employees at least once per year? & \ding{55} & \textcolor{orange}{High Risk} \\
\bottomrule
\end{tabular}
\end{table}

% --- 4. Technical Scan Results ---
\section{Technical Scan Results}
An external network scan was performed using Nmap to identify open ports and exposed services on the organization's public-facing infrastructure.

\begin{itemize}
    \item \textbf{Target IP Address:} \texttt{[Target IP]}
    \item \textbf{Scan Status:} Host is UP.
\end{itemize}

The following table details the open ports discovered during the scan.

\begin{table}[h!]
\centering
\caption{Open Port Analysis}
\begin{tabular}{c c l l}
\toprule
\textbf{Port} & \textbf{State} & \textbf{Service Name} & \textbf{Notes} \\
\midrule
3389/tcp & OPEN & ms-wbt-server & Remote Desktop Protocol (RDP). \\
 & & & \textcolor{red}{\textbf{Critical Risk Exposure.}} \\
\bottomrule
\end{tabular}
\end{table}

\subsection{Analysis of Findings}
The scan confirms that port 3389 (RDP) is open and accessible from the public internet. RDP is a frequent target for brute-force attacks and exploitation of known vulnerabilities (e.g., BlueKeep). Exposing this service directly without mitigating controls like a VPN or firewall restrictions is a critical security flaw.

% --- 5. Correlated Risk Assessment ---
\section{Correlated Risk Assessment}
This section synthesizes findings from all data sources to provide a holistic view of the organization's risk profile.

\begin{table}[h!]
\centering
\caption{Summary of Identified Risks}
\begin{tabular}{p{0.1\linewidth} p{0.25\linewidth} p{0.4\linewidth} l}
\toprule
\textbf{Risk ID} & \textbf{Risk Name} & \textbf{Description} & \textbf{Severity} \\
\midrule
RISK-001 & \textbf{Public RDP Exposure} & The network scan confirms that RDP (port 3389) is exposed on \texttt{[Target IP]}. This aligns with pre-existing risk data and presents an immediate threat of compromise. & \textbf{Critical (9.0)} \\
\addlinespace
RISK-002 & \textbf{Lack of MFA on Critical Systems} & MFA is not enforced for accessing email or sensitive data. A compromised password would grant an attacker direct access to these key systems. & \textbf{High} \\
\addlinespace
RISK-003 & \textbf{Insufficient Security Awareness} & The organization does not conduct security awareness training for new or existing employees, increasing susceptibility to phishing and social engineering attacks. & \textbf{High} \\
\bottomrule
\end{tabular}
\end{table}

% --- 6. Recommendations ---
\section{Recommendations}
The following prioritized actions are recommended to mitigate the identified risks and improve the overall security posture of \textbf{[Organization Name]}.

\subsection{Priority 1: Remediate RDP Exposure (Immediate)}
This is the most critical finding and must be addressed immediately to prevent an imminent breach.
\begin{itemize}
    \item \textbf{Short-Term Fix:} Immediately implement a firewall rule to block all inbound traffic to TCP port 3389 on \texttt{[Target IP]} from the public internet.
    \item \textbf{Long-Term Solution:} For necessary remote access, deploy a Virtual Private Network (VPN) solution. All remote administration should be performed exclusively through the secure VPN tunnel.
\end{itemize}

\subsection{Priority 2: Implement Multi-Factor Authentication (High)}
To protect against credential theft, MFA is essential.
\begin{itemize}
    \item \textbf{Action:} Procure and deploy an MFA solution for all users.
    \item \textbf{Implementation Priority:}
    \begin{enumerate}
        \item All email accounts (e.g., Office 365, Google Workspace).
        \item All systems containing sensitive data.
        \item All privileged/administrator accounts.
    \end{enumerate}
\end{itemize}

\subsection{Priority 3: Establish Security Awareness Training Program (High)}
A well-trained workforce is a critical layer of defense.
\begin{itemize}
    \item \textbf{Action:} Develop and implement a formal security awareness training program.
    \item \textbf{Requirements:}
    \begin{enumerate}
        \item The training must be mandatory for all new employees during their onboarding process.
        \item All existing employees must complete the training at least once per year.
        \item The program should cover key topics such as phishing identification, password security, and the acceptable use policy.
    \end{enumerate}
\end{itemize}

\end{document}
```