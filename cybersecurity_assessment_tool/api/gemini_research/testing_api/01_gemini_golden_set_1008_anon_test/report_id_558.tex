```latex
\documentclass[12pt]{article}

% --- PACKAGES ---
\usepackage[margin=1in]{geometry}
\usepackage{pifont} % For checkmarks and crosses
\usepackage{booktabs} % For professional tables
\usepackage{hyperref} % For hyperlinks
\usepackage{url} % For URL formatting
\usepackage{seqsplit} % For splitting long strings to prevent overflow
\usepackage{graphicx}
\usepackage{xcolor}

% --- DOCUMENT METADATA ---
\title{Cybersecurity Posture Assessment Report}
\author{Cybersecurity Analysis Division}
\date{\today}

% --- HYPERREF SETUP ---
\hypersetup{
    colorlinks=true,
    linkcolor=blue,
    filecolor=magenta,      
    urlcolor=cyan,
    pdftitle={Cybersecurity Posture Assessment Report},
    pdfpagemode=FullScreen,
}

\begin{document}

\maketitle
\thispagestyle{empty}
\newpage

\tableofcontents
\newpage

% --- EXECUTIVE SUMMARY ---
\section{Executive Summary}

This report provides a comprehensive analysis of the cybersecurity posture for \textbf{[Organization Name]}, based on a review of organizational security controls, an external network scan, and pre-existing risk data. The assessment was conducted to identify vulnerabilities, evaluate current security practices, and provide actionable recommendations to mitigate identified risks.

\paragraph{Key Findings:}
The overall security posture presents a mixed landscape. On a positive note, the external network perimeter appears to be well-secured. The technical scan did not identify any open ports on the target system \texttt{[Client IP]}, indicating a properly configured firewall. Notably, a previously identified risk concerning an open unencrypted web server on Port 80 appears to have been remediated, as the port was found to be closed during this assessment.

However, significant and critical gaps were identified in the organization's identity and access management (IAM) policies. The lack of mandatory Multi-Factor Authentication (MFA) for accessing email and other sensitive data systems constitutes a high-risk exposure. These gaps leave the organization highly vulnerable to account compromise through common attack vectors like phishing and credential stuffing, which could lead to a significant data breach.

\paragraph{Primary Recommendations:}
Immediate action is required to address the identified IAM weaknesses. The highest priority is to enforce MFA across all email accounts and sensitive data systems. Secondly, the organization should formally validate the closure of Port 80 and update its internal risk register to reflect this remediation.

% --- ORGANIZATIONAL INFORMATION ---
\section{Organizational Information}

This section details the information provided by the client for this assessment. Due to the anonymized nature of the data provided, placeholders have been used where necessary.

\begin{itemize}
    \item \textbf{Organization Name:} \textbf{[Organization Name]}
    \item \textbf{Primary Email Domain:} \texttt{[Domain]}
    \item \textbf{External IP Address Scanned:} \texttt{[Client IP]}
\end{itemize}

% --- SECURITY CONTROL REVIEW ---
\section{Security Control Review (Questionnaire)}

The following table summarizes the organization's responses to a security controls questionnaire. The assessment column highlights areas that deviate from established best practices and represent potential security gaps.

\begin{table}[h!]
\centering
\caption{Security Controls Questionnaire Analysis}
\begin{tabular}{p{0.6\linewidth} c l}
\toprule
\textbf{Control Question} & \textbf{Response} & \textbf{Assessment} \\
\midrule
Do you require MFA to access email? & \ding{55} No & \textcolor{red}{\textbf{High Risk}} \\
Do you require MFA to log into computers? & \ding{51} Yes & Meets Best Practice \\
Do you require MFA to access sensitive data systems? & \ding{55} No & \textcolor{red}{\textbf{Critical Gap}} \\
Does your organization have an employee acceptable use policy? & \ding{51} Yes & Meets Best Practice \\
Does your organization do security awareness training for new employees? & \ding{51} Yes & Meets Best Practice \\
Does your organization do security awareness training for all employees at least once per year? & \ding{51} Yes & Meets Best Practice \\
\bottomrule
\end{tabular}
\end{table}

The most critical findings from this review are the absence of MFA for email and sensitive systems. Email is a primary target for attackers, and its compromise often serves as a gateway to further infiltration. The lack of MFA on sensitive data systems directly exposes the organization's most valuable assets to unauthorized access.

% --- TECHNICAL SCAN RESULTS ---
\section{Technical Scan Results}

An external network scan was performed on the provided IP address to identify accessible services and potential vulnerabilities.

\begin{itemize}
    \item \textbf{Target IP Address:} \texttt{[Target IP]}
    \item \textbf{Scan Summary:} No open ports were detected.
\end{itemize}

\begin{table}[h!]
\centering
\caption{Nmap Scan Port Summary}
\begin{tabular}{l l l l l}
\toprule
\textbf{Port} & \textbf{State} & \textbf{Service} & \textbf{Product} & \textbf{Version} \\
\midrule
80/tcp & closed & http & N/A & N/A \\
\bottomrule
\end{tabular}
\end{table}

\paragraph{Analysis:} The scan results indicate a strong network perimeter defense. The fact that no ports were found open to the public internet significantly reduces the external attack surface. The finding that port 80 is closed directly contradicts a pre-existing risk, suggesting that the vulnerability has been successfully remediated.

% --- CONSOLIDATED RISK ASSESSMENT ---
\section{Consolidated Risk Assessment}

This section synthesizes findings from the security control review, technical scan, and pre-existing risk data into a consolidated list of current risks.

\begin{table}[h!]
\centering
\caption{Summary of Identified Risks}
\begin{tabular}{p{0.25\linewidth} p{0.45\linewidth} l l}
\toprule
\textbf{Risk Name} & \textbf{Overview} & \textbf{Severity} & \textbf{Status} \\
\midrule
\textbf{Lack of MFA on Sensitive Systems} & Access to systems containing critical or sensitive data is protected only by a username and password, increasing the risk of a data breach. & \textbf{Critical} & \textcolor{red}{Active Gap} \\
\textbf{Lack of MFA on Email} & Employee email accounts are vulnerable to takeover via phishing or credential stuffing, which can lead to business email compromise and further internal attacks. & \textbf{High} & \textcolor{red}{Active Gap} \\
\textbf{Unencrypted Web Server} & A pre-existing risk noted that Port 80 was open, exposing unencrypted HTTP traffic. & Medium & \textcolor{olive}{Remediated} \\
\bottomrule
\end{tabular}
\end{table}

% --- RECOMMENDATIONS ---
\section{Recommendations}

The following recommendations are prioritized based on the risk and potential impact to the organization.

\subsection{Priority 1: Enforce MFA on Sensitive Data Systems (Critical)}
\textbf{Action:} Immediately deploy and enforce a mandatory Multi-Factor Authentication policy for all user accounts (including administrative and service accounts) that have access to sensitive data systems.
\textbf{Justification:} This is the most critical step to protect the organization's "crown jewels." A successful password attack against a privileged user could lead to a catastrophic data breach.

\subsection{Priority 2: Enforce MFA on All Email Accounts (High)}
\textbf{Action:} Enable and enforce MFA for all user email accounts.
\textbf{Justification:} Email is the primary vector for phishing attacks and is often the first point of compromise. Securing email with MFA drastically reduces the risk of account takeovers, business email compromise (BEC), and subsequent malware infections.

\subsection{Priority 3: Update Internal Risk Register (Informational)}
\textbf{Action:} Formally document the remediation of the "Unencrypted Web Server" risk. The team responsible for closing Port 80 should confirm the change was intentional and documented.
\textbf{Justification:} Maintaining an accurate and up-to-date risk register is crucial for effective cybersecurity governance and ensures that resources are focused on current, active threats.

\end{document}
```