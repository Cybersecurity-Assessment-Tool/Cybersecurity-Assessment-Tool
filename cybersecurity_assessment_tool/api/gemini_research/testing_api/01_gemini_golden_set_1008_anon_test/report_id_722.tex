```latex
\documentclass[12pt]{article}

% Required Packages
\usepackage[margin=1in]{geometry}
\usepackage{pifont} % For checkmarks and crosses
\usepackage{booktabs} % For professional tables
\usepackage[hidelinks]{hyperref} % For clickable links without boxes
\usepackage{url}
\usepackage{seqsplit} % For splitting long strings in tt font

% Document Information
\title{Cybersecurity Posture Assessment Report}
\author{Cybersecurity Analyst}
\date{\today}

\begin{document}

\maketitle
\thispagestyle{empty}
\newpage
\tableofcontents
\newpage

\section{Executive Summary}

This report provides a comprehensive cybersecurity assessment for \textbf{[Organization Name]}, based on an analysis of network scan data, organizational security controls, and pre-existing risk information. The assessment reveals several critical and high-risk vulnerabilities that require immediate attention.

Key findings indicate a significant risk of unauthorized access and data breach. A publicly accessible FTP server was identified running a dangerously outdated version of \texttt{vsftpd} (2.3.4), which is known to contain a critical backdoor vulnerability. This is compounded by the server's configuration, which permits anonymous logins.

Furthermore, a review of administrative controls identified a critical gap in identity and access management: Multi-Factor Authentication (MFA) is not enforced for email, computer logins, or access to sensitive data systems. The absence of a formal Acceptable Use Policy exacerbates these risks. These issues, combined with an existing risk of outdated Windows 7 workstations, create a fragile security posture that could be easily compromised.

Immediate remediation of the vulnerable FTP server and the swift implementation of MFA are the highest priorities.

\section{Organizational Information}

This report is prepared for the following organization:

\begin{itemize}
    \item \textbf{Organization Name:} \textbf{[Organization Name]}
    \item \textbf{Primary Domain:} \texttt{[Domain]}
    \item \textbf{External IP Scanned:} \texttt{[Client IP]}
\end{itemize}

\section{Security Control Review}

The following table summarizes the organization's responses to a security controls questionnaire. Items marked with a cross (\ding{55}) represent significant gaps in the current security posture and are discussed in the Risk Assessment section.

\begin{table}[h!]
\centering
\caption{Security Controls Questionnaire Results}
\begin{tabular}{p{0.8\linewidth} c}
\toprule
\textbf{Control Question} & \textbf{Status} \\
\midrule
Do you require MFA to access email? & \ding{55} \\
Do you require MFA to log into computers? & \ding{55} \\
Do you require MFA to access sensitive data systems? & \ding{55} \\
Does your organization have an employee acceptable use policy? & \ding{55} \\
Does your organization do security awareness training for new employees? & \ding{51} \\
Does your organization do security awareness training for all employees at least once per year? & \ding{51} \\
\bottomrule
\end{tabular}
\end{table}

\section{Technical Scan Results}

An external network scan was performed against the target IP address \texttt{[Target IP]}. The scan identified the following open ports and services.

\begin{table}[h!]
\centering
\caption{Open Port and Service Information}
\begin{tabular}{l l l l}
\toprule
\textbf{Port} & \textbf{State} & \textbf{Service} & \textbf{Version Details} \\
\midrule
21/tcp & Open & ftp & vsftpd 2.3.4 \\
\bottomrule
\end{tabular}
\end{table}

\subsection{Critical Findings from Technical Scan}
\begin{itemize}
    \item \textbf{Vulnerable FTP Version:} The service \texttt{vsftpd 2.3.4} is a specific version known to contain a critical backdoor vulnerability (CVE-2011-2523). An attacker can gain a command shell on the server by sending a specific string as the username.
    \item \textbf{Anonymous FTP Login:} The scan confirmed that "Anonymous FTP login" is allowed. This configuration permits any user on the internet to connect to the server and potentially access, upload, or download files without authentication, posing a severe data breach risk.
\end{itemize}

\section{Consolidated Risk Assessment}

The following table synthesizes findings from the security control review, the technical scan, and pre-existing risk data. Risks are prioritized by severity to guide remediation efforts.

\begin{table}[h!]
\centering
\caption{Synthesized Risk Summary}
\begin{tabular}{p{0.25\linewidth} p{0.5\linewidth} l}
\toprule
\textbf{Risk Name} & \textbf{Overview} & \textbf{Severity} \\
\midrule
\textbf{Vulnerable FTP Server with Anonymous Access} & An external FTP server is running \texttt{vsftpd 2.3.4}, which has a known remote code execution backdoor. It is also configured to allow anonymous login, permitting unauthorized file access. & \textbf{Critical} \\
\addlinespace
\textbf{Lack of Multi-Factor Authentication (MFA)} & MFA is not enforced for email, computer logins, or sensitive systems. This makes the organization highly susceptible to credential theft and unauthorized access. & \textbf{Critical} \\
\addlinespace
\textbf{No Acceptable Use Policy} & The absence of a formal policy defining the proper use of company assets creates ambiguity and increases the risk of insider threat and accidental data exposure. & \textbf{High} \\
\addlinespace
\textbf{Outdated Windows Policy} & Workstations are running Windows 7, an end-of-life operating system that no longer receives security updates, leaving them vulnerable to known exploits. & Medium \\
\bottomrule
\end{tabular}
\end{table}

\section{Recommendations}

Based on the analysis, the following actions are recommended, ordered by priority:

\begin{enumerate}
    \item \textbf{[Immediate] Remediate Vulnerable FTP Server:}
    \begin{itemize}
        \item Immediately take the FTP server offline if it is not business-critical.
        \item If required, upgrade \texttt{vsftpd} to the latest stable version and reconfigure it to disable anonymous access.
        \item Enforce strong, unique passwords for all FTP accounts and utilize FTP over TLS (FTPS) for encrypted connections.
    \end{itemize}
    \item \textbf{[High Priority] Implement Multi-Factor Authentication (MFA):}
    \begin{itemize}
        \item Prioritize the rollout of MFA for all user accounts, starting with email access and privileged accounts.
        \item Extend MFA enforcement to all computer logins and access to systems containing sensitive data.
    \end{itemize}
    \item \textbf{[Medium Priority] Develop and Implement an Acceptable Use Policy (AUP):}
    \begin{itemize}
        \item Draft a formal AUP that clearly outlines the rules and responsibilities for all employees when using company technology and data.
        \item Integrate the AUP into the new employee onboarding process and require all current employees to review and acknowledge it.
    \end{itemize}
    \item \textbf{[Planned] Upgrade Outdated Operating Systems:}
    \begin{itemize}
        \item Accelerate the project to upgrade all Windows 7 workstations to a supported operating system, such as Windows 10 or 11, to ensure they receive critical security patches.
    \end{itemize}
\end{enumerate}

\end{document}
```