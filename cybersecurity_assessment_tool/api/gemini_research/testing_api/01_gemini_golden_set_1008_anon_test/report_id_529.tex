```latex
\documentclass[12pt]{article}

% Preamble: Required Packages
\usepackage[margin=1in]{geometry}
\usepackage{pifont} % For checkmarks and crosses
\usepackage{booktabs} % For professional tables
\usepackage{hyperref} % For clickable links
\usepackage{url} % For URL formatting
\usepackage{seqsplit} % To split long strings in tt font
\usepackage[T1]{fontenc}
\usepackage{xcolor}

% Document Metadata
\title{Cybersecurity Assessment Report}
\author{Cybersecurity Analysis Division}
\date{November 22, 2025}

% Hyperref Setup
\hypersetup{
    colorlinks=true,
    linkcolor=blue,
    filecolor=magenta,      
    urlcolor=cyan,
    pdftitle={Cybersecurity Assessment Report},
    pdfauthor={Cybersecurity Analysis Division},
    pdfsubject={Security Posture Analysis},
    pdfkeywords={Security, Nmap, Risk, Assessment},
    bookmarks=true
}

\begin{document}

\maketitle
\thispagestyle{empty}
\newpage

\tableofcontents
\thispagestyle{empty}
\newpage

\setcounter{page}{1}

% --- 1. Executive Summary ---
\section{Executive Summary}
This report details the findings of a cybersecurity assessment conducted on \textbf{[Organization Name]}. The analysis is based on a network vulnerability scan, a review of organizational security controls, and an evaluation of pre-existing risks.

The assessment reveals a \textbf{critical risk posture} that requires immediate remediation. Key findings include:
\begin{itemize}
    \item \textbf{Widespread Lack of Multi-Factor Authentication (MFA):} The organization has not implemented MFA for email, computer logins, or access to sensitive data systems. This represents a critical vulnerability, exposing the organization to significant risks of account compromise and unauthorized access.
    \item \textbf{Vulnerable External Services:} The external network scan identified a web server running an outdated and vulnerable version of Nginx (1.18.0). This service is a prime target for external attackers seeking to compromise the network perimeter.
    \item \textbf{Deficient Security Governance:} The organization lacks a formal employee Acceptable Use Policy and does not conduct annual security awareness training for all staff. These gaps indicate a weak security culture and increase the likelihood of human error leading to a security incident.
\end{itemize}

Immediate action is required to address these findings. Recommendations focus on the rapid deployment of MFA, patching of vulnerable systems, and the development of foundational security policies and training programs.

% --- 2. Organizational Information ---
\section{Organizational Information}
This section provides the organizational details as understood for the scope of this assessment. Due to the anonymized nature of the provided data, placeholders are used where necessary.

\begin{table}[h!]
\centering
\caption{Client Details}
\begin{tabular}{@{}ll@{}}
\toprule
\textbf{Attribute} & \textbf{Value} \\ \midrule
Organization Name & \textbf{[Organization Name]} \\
Primary Domain & \texttt{[Domain]} \\
External IP Address Scanned & \texttt{[Client IP]} \\
Assessment Date & November 22, 2025 \\ \bottomrule
\end{tabular}
\end{table}

% --- 3. Security Control Review ---
\section{Security Control Review (Questionnaire Analysis)}
An analysis of the organization's security questionnaire responses highlights significant gaps in fundamental security controls. The following table summarizes the responses and provides an initial assessment of their impact. A checkmark (\ding{51}) indicates an affirmative response, while a cross (\ding{55}) indicates a negative response, often corresponding to a control gap.

\begin{table}[h!]
\centering
\caption{Security Questionnaire Analysis}
\begin{tabular}{@{}p{0.55\linewidth} c p{0.2\linewidth}@{}}
\toprule
\textbf{Control Question} & \textbf{Response} & \textbf{Assessment} \\ \midrule
Do you require MFA to access email? & \ding{55} & \textcolor{red}{Critical Gap} \\
Do you require MFA to log into computers? & \ding{55} & \textcolor{red}{Critical Gap} \\
Do you require MFA to access sensitive data systems? & \ding{55} & \textcolor{red}{Critical Gap} \\
Does your organization have an employee acceptable use policy? & \ding{55} & \textcolor{orange}{High Risk} \\
Does your organization do security awareness training for new employees? & \ding{51} & Foundational Control Met \\
Does your organization do security awareness training for all employees at least once per year? & \ding{55} & \textcolor{orange}{High Risk} \\ \bottomrule
\end{tabular}
\end{table}

% --- 4. Technical Scan Results ---
\section{Technical Scan Results}
A network scan was performed against the organization's external infrastructure to identify open ports and exposed services.

\begin{itemize}
    \item \textbf{Target IP Address:} \texttt{[Target IP]}
    \item \textbf{Scan Date:} 2025-11-22T10:00:00Z
\end{itemize}

The scan revealed the following open port:

\begin{table}[h!]
\centering
\caption{Open Port Analysis}
\begin{tabular}{@{}lllll@{}}
\toprule
\textbf{Port} & \textbf{State} & \textbf{Service} & \textbf{Product \& Version} & \textbf{Notes} \\ \midrule
443/tcp & open & https & nginx 1.18.0 & \parbox{4cm}{\textcolor{red}{Outdated version.} Known vulnerabilities exist (e.g., CVE-2021-23017). Immediate patching is required.} \\ \bottomrule
\end{tabular}
\end{table}

\subsection{Analysis of Findings}
The primary technical finding is the presence of Nginx version 1.18.0, which is exposed to the internet. This version was released in April 2020 and is no longer supported. It is susceptible to multiple publicly disclosed vulnerabilities that could allow an attacker to gain unauthorized access, cause a denial of service, or otherwise compromise the web server.

% --- 5. Overall Risk Assessment ---
\section{Overall Risk Assessment}
This section correlates the findings from the security control review and the technical scan to provide a consolidated list of identified risks. No pre-existing vulnerabilities were reported.

\begin{table}[h!]
\centering
\caption{Consolidated Risk Register}
\begin{tabular}{@{}lp{0.5\linewidth}ll@{}}
\toprule
\textbf{Risk ID} & \textbf{Description} & \textbf{Source} & \textbf{Severity} \\ \midrule
RISK-001 & Widespread lack of Multi-Factor Authentication (MFA) on all critical systems, including email and sensitive data access points. & Questionnaire & \textcolor{red}{Critical} \\
\addlinespace
RISK-002 & Publicly accessible web server running outdated and vulnerable Nginx software (v1.18.0). & Network Scan & \textcolor{orange}{High} \\
\addlinespace
RISK-003 & Inadequate security governance, evidenced by the lack of an Acceptable Use Policy and mandatory annual security training. & Questionnaire & \textcolor{orange}{High} \\ \bottomrule
\end{tabular}
\end{table}

% --- 6. Recommendations ---
\section{Recommendations}
The following actions are recommended to mitigate the identified risks and improve the overall security posture of \textbf{[Organization Name]}.

\subsection{RISK-001: Remediate Lack of MFA (Priority: Immediate)}
\begin{itemize}
    \item \textbf{Action:} Procure and deploy an MFA solution across the organization.
    \item \textbf{Details:} Prioritize the rollout of MFA in the following order:
    \begin{enumerate}
        \item All external access points (e.g., VPN, remote desktop).
        \item Email systems (e.g., Office 365, Google Workspace).
        \item Access to all systems containing sensitive or regulated data.
        \item Standard employee computer logins.
    \end{enumerate}
    \item \textbf{Justification:} MFA is one of the most effective controls for preventing unauthorized access resulting from compromised credentials.
\end{itemize}

\subsection{RISK-002: Patch Vulnerable Web Server (Priority: Immediate)}
\begin{itemize}
    \item \textbf{Action:} Upgrade the Nginx server on host \texttt{[Target IP]} to the latest stable version.
    \item \textbf{Details:} Develop a patch management plan that includes testing the new version in a staging environment before deploying to production. This will ensure compatibility with existing web applications.
    \item \textbf{Justification:} Patching this vulnerability closes a direct avenue for external attackers to compromise the organization's network perimeter.
\end{itemize}

\subsection{RISK-003: Establish Security Governance (Priority: High)}
\begin{itemize}
    \item \textbf{Action:} Develop and implement foundational security policies and training programs.
    \item \textbf{Details:}
    \begin{enumerate}
        \item Draft and ratify a formal Acceptable Use Policy (AUP) that all employees must read and sign.
        \item Implement a mandatory annual security awareness training program for all employees. The training should cover topics such as phishing, password security, and data handling.
    \end{enumerate}
    \item \textbf{Justification:} A strong security culture, reinforced by clear policies and regular training, reduces the risk of security incidents caused by human error.
\end{itemize}

\end{document}
```