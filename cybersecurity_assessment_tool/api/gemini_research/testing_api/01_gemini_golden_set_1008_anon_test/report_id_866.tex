```latex
\documentclass[12pt]{article}

% Preamble: Required Packages and Document Setup
\usepackage[margin=1in]{geometry}
\usepackage{pifont} % For checkmarks and crosses
\usepackage{booktabs} % For professional tables
\usepackage{hyperref} % For clickable links
\usepackage{url} % For URL formatting
\usepackage{seqsplit} % To split long strings in texttt
\usepackage{graphicx}
\usepackage{xcolor}
\usepackage{fancyhdr}
\usepackage{lastpage}
\usepackage{datetime}

% --- Document Metadata and Hyperref Setup ---
\hypersetup{
    colorlinks=true,
    linkcolor=blue,
    filecolor=magenta,      
    urlcolor=cyan,
    pdftitle={Cybersecurity Assessment Report},
    pdfauthor={Automated Security Analysis System},
    pdfsubject={Security Posture Analysis},
    pdfkeywords={Cybersecurity, Risk, Assessment, Nmap, Policy},
    bookmarks=true
}

% --- Custom Colors for Severity ---
\definecolor{criticalred}{HTML}{D12727}
\definecolor{highorange}{HTML}{E46D1A}
\definecolor{mediumyellow}{HTML}{F5C242}
\definecolor{lowblue}{HTML}{3A97D1}
\definecolor{infogray}{HTML}{808080}

% --- Header and Footer Configuration ---
\pagestyle{fancy}
\fancyhf{} % Clear all header and footer fields
\fancyhead[L]{Cybersecurity Assessment Report}
\fancyhead[R]{\textbf{[Organization Name]}}
\fancyfoot[C]{Page \thepage\ of \pageref{LastPage}}
\renewcommand{\headrulewidth}{0.4pt}
\renewcommand{\footrulewidth}{0.4pt}

% --- Document Start ---
\begin{document}

% --- Title Page ---
\begin{titlepage}
    \centering
    \vspace*{2cm}
    
    {\Huge \textbf{Cybersecurity Assessment Report}\par}
    \vspace{1.5cm}
    
    {\Large Prepared for:\par}
    \vspace{0.5cm}
    {\Huge \textbf{[Organization Name]}}\par
    
    \vspace{3cm}
    
    {\large \textbf{Date of Report:} \today\par}
    
    \vfill
    
    {\large \textit{This report contains sensitive information and should be handled with care. Access is restricted to authorized personnel only.}\par}
\end{titlepage}

\newpage
\tableofcontents
\newpage

% --- Section 1: Executive Summary ---
\section{Executive Summary}

This report provides a comprehensive analysis of the security posture for \textbf{[Organization Name]}, based on technical network scans, a review of organizational security controls, and pre-existing risk data. The assessment was conducted to identify vulnerabilities, policy gaps, and misconfigurations that could expose the organization to cyber threats.

The overall security posture is rated as \textbf{CRITICAL}. Several significant, high-impact risks were identified that require immediate attention.

Key findings include:
\begin{itemize}
    \item \textbf{Exposed Vulnerable Service:} A publicly accessible FTP server was discovered running a critically outdated version of \texttt{vsftpd} (2.3.4), which is known to contain a backdoor. The service is also misconfigured to allow anonymous logins, presenting an immediate and severe risk of unauthorized access and system compromise.
    \item \textbf{Inadequate Access Controls:} Multi-Factor Authentication (MFA) is not enforced for accessing critical assets, including email and sensitive data systems. This significantly increases the risk of account compromise through phishing or credential theft.
    \item \textbf{Policy and Training Deficiencies:} The organization lacks a formal Acceptable Use Policy (AUP) and does not provide security awareness training for new employees. These gaps in foundational security practices increase the likelihood of human error leading to a security incident.
    \item \textbf{Pre-existing Unmitigated Risk:} The continued use of Windows 7 workstations, an end-of-life operating system, remains an unaddressed medium-level risk.
\end{itemize}

Urgent remediation of the identified critical vulnerabilities is strongly recommended to prevent potential exploitation and a subsequent security breach.

% --- Section 2: Organizational Information ---
\section{Organizational Information}

This section details the information provided about the organization. The placeholders indicate that this data was not available at the time of the assessment.

\begin{tabular}{@{}ll}
    \toprule
    \textbf{Attribute} & \textbf{Value} \\
    \midrule
    Organization Name & \textbf{[Organization Name]} \\
    Primary Email Domain & \texttt{[Domain]} \\
    External IP Address Scanned & \texttt{[Client IP]} \\
    \bottomrule
\end{tabular}

% --- Section 3: Security Control Review ---
\section{Security Control Review}

The following table summarizes the organization's responses to a security controls questionnaire. Items marked with \ding{55} represent significant gaps in the security framework and are directly correlated with identified risks.

\begin{table}[h!]
\centering
\caption{Security Controls Questionnaire Analysis}
\begin{tabular}{@{}p{0.6\linewidth}cc@{}}
    \toprule
    \textbf{Control Question} & \textbf{Response} & \textbf{Status} \\
    \midrule
    Do you require MFA to access email? & \textcolor{criticalred}{\ding{55}} & \textbf{Critical Gap} \\
    Do you require MFA to log into computers? & \textcolor{green}{\ding{51}} & Implemented \\
    Do you require MFA to access sensitive data systems? & \textcolor{criticalred}{\ding{55}} & \textbf{Critical Gap} \\
    Does your organization have an employee acceptable use policy? & \textcolor{highorange}{\ding{55}} & \textbf{High Risk Gap} \\
    Does your organization do security awareness training for new employees? & \textcolor{highorange}{\ding{55}} & \textbf{High Risk Gap} \\
    Does your organization do security awareness training for all employees at least once per year? & \textcolor{green}{\ding{51}} & Implemented \\
    \bottomrule
\end{tabular}
\end{table}

% --- Section 4: Technical Scan Results ---
\section{Technical Scan Results}

A network scan was performed on the target system to identify open ports and exposed services.

\begin{itemize}
    \item \textbf{Target IP Address:} \texttt{[Target IP]}
    \item \textbf{Scan Date:} Not specified in scan data.
\end{itemize}

\begin{table}[h!]
\centering
\caption{Open Port Analysis}
\begin{tabular}{@{}llllll@{}}
    \toprule
    \textbf{Port} & \textbf{State} & \textbf{Service} & \textbf{Product} & \textbf{Version} & \textbf{Notes} \\
    \midrule
    21/tcp & Open & ftp & vsftpd & 2.3.4 & \begin{tabular}[t]{@{}l@{}}\textbf{CRITICAL FINDING:}\\ Anonymous FTP login allowed. \\ Version vulnerable to backdoor \\ (CVE-2011-2523).\end{tabular} \\
    \bottomrule
\end{tabular}
\end{table}

\subsection{Analysis of Findings}
The scan identified a \texttt{vsftpd} service, version 2.3.4, running on port 21. This specific version, released in 2011, contains a well-documented critical backdoor vulnerability (\href{https://nvd.nist.gov/vuln/detail/CVE-2011-2523}{CVE-2011-2523}). If exploited, this vulnerability allows an attacker to execute arbitrary commands on the server with root privileges.

Furthermore, the server is configured to allow \textbf{anonymous FTP login}. This misconfiguration permits any unauthenticated user on the internet to connect to the server, and potentially upload or download files, which could lead to data exfiltration or the introduction of malware.

% --- Section 5: Risk Assessment Summary ---
\section{Risk Assessment Summary}

The following table consolidates findings from the technical scan, control review, and pre-existing data into a prioritized list of risks.

\begin{table}[h!]
\centering
\caption{Consolidated Risk Register}
\begin{tabular}{@{}lp{0.25\linewidth}p{0.15\linewidth}p{0.4\linewidth}@{}}
    \toprule
    \textbf{ID} & \textbf{Risk Title} & \textbf{Severity} & \textbf{Description} \\
    \midrule
    RISK-001 & Vulnerable Anonymous FTP Server & \textcolor{criticalred}{\textbf{Critical}} & An outdated and vulnerable FTP server (vsftpd 2.3.4) is publicly exposed and allows anonymous access. This could lead to immediate system compromise. \\
    \addlinespace
    RISK-002 & Inadequate Multi-Factor Authentication & \textcolor{criticalred}{\textbf{Critical}} & Lack of MFA on email and sensitive data systems exposes the organization to account takeover via phishing and credential stuffing attacks. \\
    \addlinespace
    RISK-003 & Deficient Security Policies \& Training & \textcolor{highorange}{\textbf{High}} & The absence of an Acceptable Use Policy and security training for new hires increases the risk of insider threat and human error. \\
    \addlinespace
    RISK-004 & Outdated Windows Policy & \textcolor{mediumyellow}{\textbf{Medium}} & (Existing Risk) Workstations are running Windows 7, an unsupported OS that no longer receives security updates, leaving them vulnerable to known exploits. \\
    \bottomrule
\end{tabular}
\end{table}

% --- Section 6: Recommendations ---
\section{Recommendations}

The following actions are recommended to mitigate the identified risks. They are prioritized based on severity and potential impact.

\subsection{Immediate Actions (Priority: Critical)}
\begin{enumerate}
    \item \textbf{Remediate Vulnerable FTP Server (RISK-001):}
    \begin{itemize}
        \item Immediately take the FTP service on \texttt{[Target IP]} offline.
        \item If a file transfer service is required, decommission the \texttt{vsftpd} server and replace it with a modern, secure alternative such as SFTP (SSH File Transfer Protocol).
        \item If FTP must be used, upgrade to the latest stable version and disable anonymous access. Access should be restricted by a firewall to trusted IP addresses only.
    \end{itemize}
    \item \textbf{Implement MFA (RISK-002):}
    \begin{itemize}
        \item Enforce MFA for all user accounts on the primary email system (\texttt{[Domain]}) immediately.
        \item Prioritize the rollout of MFA for all systems identified as containing sensitive data.
    \end{itemize}
\end{enumerate}

\subsection{High-Priority Actions}
\begin{enumerate}
    \setcounter{enumi}{2}
    \item \textbf{Develop and Implement Security Policies (RISK-003):}
    \begin{itemize}
        \item Create a formal Acceptable Use Policy (AUP) that all employees must read and sign.
        \item Develop a security awareness training module and integrate it into the onboarding process for all new employees.
    \end{itemize}
\end{enumerate}

\subsection{Medium-Priority Actions}
\begin{enumerate}
    \setcounter{enumi}{3}
    \item \textbf{Address End-of-Life Systems (RISK-004):}
    \begin{itemize}
        \item Accelerate the plan to upgrade or replace all workstations running Windows 7.
        \item Isolate any remaining Windows 7 machines from the main network until they can be decommissioned.
    \end{itemize}
\end{enumerate}

\end{document}
```