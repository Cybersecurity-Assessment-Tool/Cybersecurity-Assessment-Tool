```latex
\documentclass[12pt]{article}

% Preamble: Required Packages
\usepackage[margin=1in]{geometry}
\usepackage{pifont} % For checkmarks and crosses
\usepackage{booktabs} % For professional tables
\usepackage{hyperref} % For clickable links
\usepackage{url} % For URL formatting
\usepackage{seqsplit} % For splitting long strings
\usepackage{graphicx}
\usepackage[table]{xcolor}
\usepackage{fancyhdr}
\usepackage{lastpage}

% --- Document Setup ---
\hypersetup{
    colorlinks=true,
    linkcolor=black,
    urlcolor=blue,
    pdftitle={Cybersecurity Posture Report},
    pdfauthor={Cybersecurity Analyst},
}

% --- Header and Footer ---
\pagestyle{fancy}
\fancyhf{} % Clear all header and footer fields
\fancyhead[L]{Cybersecurity Posture Report}
\fancyhead[R]{\textbf{[Organization Name]}}
\fancyfoot[C]{Page \thepage\ of \pageref{LastPage}}
\renewcommand{\headrulewidth}{0.4pt}
\renewcommand{\footrulewidth}{0.4pt}

% --- Custom Commands ---
\newcommand{\yes}{\ding{51}}
\newcommand{\no}{\ding{55}}
\newcommand{\riskcritical}{\color{red!80!black}\textbf{Critical}}
\newcommand{\riskhigh}{\color{orange!90!black}\textbf{High}}
\newcommand{\riskmedium}{\color{yellow!90!black}\textbf{Medium}}
\newcommand{\risklow}{\color{green!70!black}\textbf{Low}}

\begin{document}

% --- Title Page ---
\begin{titlepage}
    \centering
    \vspace*{2cm}
    \Huge \textbf{Cybersecurity Posture Report}
    \vspace{1.5cm}
    \Large Prepared for:
    \vspace{0.5cm}
    \huge \textbf{[Organization Name]}
    \vfill
    \large
    \today
    \vspace{1cm}
    \normalsize
    \textit{This report contains sensitive information and should be handled with care.}
\end{titlepage}

\newpage
\tableofcontents
\newpage

% --- Section 1: Executive Summary ---
\section{Executive Summary}
This report provides a comprehensive analysis of the cybersecurity posture for \textbf{[Organization Name]}. The assessment combines a review of organizational security controls, an external network vulnerability scan, and an evaluation of pre-existing risks.

The overall security posture is assessed as being at a \textbf{high risk level}. This is primarily due to several critical deficiencies in fundamental security controls. Key findings include:
\begin{itemize}
    \item \textbf{Lack of Multi-Factor Authentication (MFA):} MFA is not enforced for logging into computers or accessing sensitive data systems. This represents a critical vulnerability, as a single compromised password could lead to a significant data breach.
    \item \textbf{Absence of Security Awareness Program:} The organization does not conduct security awareness training for new or existing employees, nor does it have an Acceptable Use Policy. This significantly increases the susceptibility to human-error-based attacks like phishing and social engineering.
    \item \textbf{Exposed Management Services:} The external network scan identified an open Secure Shell (SSH) port (\texttt{22/tcp}). While necessary for remote administration, its exposure to the public internet creates a direct vector for brute-force attacks and exploitation if not properly secured.
\end{itemize}

Immediate and decisive action is required to address these gaps. This report outlines specific, actionable recommendations to mitigate the identified risks and strengthen the organization's defenses against common cyber threats.

% --- Section 2: Organizational Information ---
\section{Organizational Information}
The following details were used as the basis for this assessment. Due to the anonymized nature of the provided data, placeholders have been used where necessary.

\begin{table}[h!]
\centering
\begin{tabular}{@{}ll@{}}
\toprule
\textbf{Attribute} & \textbf{Value} \\ \midrule
Organization Name & \textbf{[Organization Name]} \\
Email Domain & \texttt{[Domain]} \\
External IP Address & \texttt{[Client IP]} \\ \bottomrule
\end{tabular}
\caption{Client Organizational Details.}
\label{tab:org_info}
\end{table}

% --- Section 3: Security Control Review ---
\section{Security Control Review}
A review of organizational security controls was conducted via a questionnaire. The responses reveal significant gaps in foundational security practices. A "No" answer indicates a missing control and a potential area of high risk.

\begin{table}[h!]
\centering
\rowcolors{2}{gray!10}{white}
\begin{tabular}{@{}p{0.6\textwidth}cc@{}}
\toprule
\textbf{Control Question} & \textbf{Response} & \textbf{Status} \\ \midrule
Do you require MFA to access email? & \yes & Implemented \\
Do you require MFA to log into computers? & \no & \riskhigh \\
Do you require MFA to access sensitive data systems? & \no & \riskcritical \\
Does your organization have an employee acceptable use policy? & \no & \riskhigh \\
Does your organization do security awareness training for new employees? & \no & \riskcritical \\
Does your organization do security awareness training for all employees at least once per year? & \no & \riskcritical \\ \bottomrule
\end{tabular}
\caption{Security Controls Questionnaire Results.}
\label{tab:controls}
\end{table}

% --- Section 4: Technical Scan Results ---
\section{Technical Scan Results}
An external network scan was performed against the target IP address \texttt{[Target IP]}. The scan identified one open port, which indicates a service exposed to the public internet.

\subsection{Open Ports}
The following table details the open port discovered during the scan.

\begin{table}[h!]
\centering
\begin{tabular}{@{}llll@{}}
\toprule
\textbf{Port} & \textbf{State} & \textbf{Service (Inferred)} & \textbf{Notes} \\ \midrule
22/tcp & open & SSH & Secure Shell for remote administration. \\ \bottomrule
\end{tabular}
\caption{Open Ports on Target: \texttt{[Target IP]}.}
\label{tab:ports}
\end{table}

\subsection{Analysis of Findings}
The presence of an open SSH port (22) is a notable finding. While essential for remote management, exposing SSH directly to the internet is a significant security risk. It becomes a primary target for automated brute-force attacks, where attackers attempt to guess usernames and passwords. Without robust security measures such as IP whitelisting, strong password policies, disabled root login, and key-based authentication, this service can become a gateway for unauthorized access into the network.

This technical finding is especially concerning when correlated with the lack of MFA on computer and sensitive system logins. An attacker who successfully compromises credentials for the SSH service could potentially gain deep access to the internal network without facing any further authentication challenges.

% --- Section 5: Risk Assessment ---
\section{Risk Assessment}
This section synthesizes the findings from the security control review and the technical scan. No pre-existing risks were documented. The following new risks have been identified and prioritized.

\begin{table}[h!]
\centering
\rowcolors{2}{gray!10}{white}
\begin{tabular}{@{}p{0.25\textwidth}p{0.55\textwidth}l@{}}
\toprule
\textbf{Risk Name} & \textbf{Overview} & \textbf{Severity} \\ \midrule
\textbf{Lack of MFA on Critical Systems} & User accounts for computers and sensitive data systems are protected only by passwords. A single compromised credential could grant an attacker full access. & \riskcritical \\
\textbf{Inadequate Security Awareness Program} & The absence of an Acceptable Use Policy and security training leaves employees unprepared to identify and resist social engineering, phishing, and other common attacks. & \riskcritical \\
\textbf{Exposed SSH Management Port} & The SSH service is open to the public internet, making it a target for brute-force attacks and exploitation of potential vulnerabilities. & \riskhigh \\
\textbf{Missing Acceptable Use Policy} & Without a formal policy, there are no established rules for how employees should use company technology and data, leading to inconsistent and insecure practices. & \riskhigh \\
\bottomrule
\end{tabular}
\caption{Summary of Identified Risks.}
\label{tab:risks}
\end{table}

% --- Section 6: Recommendations ---
\section{Recommendations}
The following actions are recommended to mitigate the identified risks. Recommendations are prioritized based on severity and potential impact.

\subsection{Immediate Priorities (Critical Risks)}
\begin{enumerate}
    \item \textbf{Implement Multi-Factor Authentication (MFA):}
    \begin{itemize}
        \item \textbf{Action:} Deploy a robust MFA solution for all user logins to computers, servers, and any system containing sensitive data.
        \item \textbf{Impact:} Drastically reduces the risk of unauthorized access from compromised credentials. This is the single most effective control to implement.
    \end{itemize}
    \item \textbf{Establish a Security Awareness Program:}
    \begin{itemize}
        \item \textbf{Action:} Develop and implement a mandatory security awareness training program for all new and existing employees. Training should be conducted annually and cover topics such as phishing, password security, and data handling.
        \item \textbf{Impact:} Creates a "human firewall" by empowering employees to recognize and report security threats, reducing the likelihood of successful social engineering attacks.
    \end{itemize}
\end{enumerate}

\subsection{Secondary Priorities (High Risks)}
\begin{enumerate}
    \setcounter{enumi}{2} % Continue numbering from previous list
    \item \textbf{Secure the Exposed SSH Port:}
    \begin{itemize}
        \item \textbf{Action:} Restrict access to the SSH port (22/tcp) to only trusted IP addresses via a firewall rule. If remote access is needed from dynamic locations, implement a Virtual Private Network (VPN) and require users to connect to the VPN before accessing SSH.
        \item \textbf{Action (Hardening):} Additionally, ensure SSH is securely configured by disabling root login, disabling password-based authentication in favor of cryptographic keys, and keeping the SSH server software up-to-date.
        \item \textbf{Impact:} Reduces the attack surface and protects a critical administrative entry point from external threats.
    \end{itemize}
    \item \textbf{Develop an Acceptable Use Policy (AUP):}
    \begin{itemize}
        \item \textbf{Action:} Create and enforce a formal AUP that clearly defines the rules and responsibilities for all employees when using company assets, data, and network resources.
        \item \textbf{Impact:} Establishes a clear security baseline for employee behavior and provides a framework for enforcing security standards.
    \end{itemize}
\end{enumerate}

\end{document}
```