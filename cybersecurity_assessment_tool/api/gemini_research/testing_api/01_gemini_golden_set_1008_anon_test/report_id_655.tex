```latex
\documentclass[12pt]{article}

% Required Packages
\usepackage[margin=1in]{geometry}
\usepackage{pifont} % For checkmarks and crosses
\usepackage{booktabs} % For professional tables
\usepackage{hyperref} % For hyperlinks
\usepackage{url} % For URL formatting
\usepackage{seqsplit} % For splitting long strings
\usepackage{graphicx}
\usepackage{xcolor}

% Document Metadata and Styling
\hypersetup{
    colorlinks=true,
    linkcolor=blue,
    filecolor=magenta,      
    urlcolor=cyan,
    pdftitle={Cybersecurity Posture Report},
    pdfauthor={Cybersecurity Analyst},
    pdfsubject={Security Assessment},
    pdfkeywords={Security, Report, Analysis},
}

\newcommand{\yes}{\ding{51}} % Checkmark
\newcommand{\no}{\ding{55}}  % X-mark

% --- Document Start ---
\begin{document}

\title{
    \textbf{Cybersecurity Posture Report} \\
    \large \textit{Generated on \today}
}
\author{Cybersecurity Analyst}
\date{}
\maketitle

\begin{abstract}
\noindent This report provides a comprehensive analysis of the cybersecurity posture for \textbf{[Organization Name]}. The assessment is based on a synthesis of data from a network vulnerability scan, an organizational security questionnaire, and a review of pre-existing risks. The analysis identifies key vulnerabilities, security control gaps, and provides actionable recommendations to mitigate the identified risks and improve the overall security posture.
\end{abstract}

\tableofcontents
\newpage

% --- Section 1: Overview ---
\section{Executive Summary}
This assessment was conducted to evaluate the external security posture and internal security controls of \textbf{[Organization Name]}. While the organization demonstrates a solid foundation in security awareness and endpoint multi-factor authentication (MFA), two significant risks were identified that require immediate attention.

A critical security gap was found in the access control mechanisms for sensitive data systems, which currently lack MFA. This significantly increases the risk of unauthorized access in the event of a credential compromise.

Furthermore, the external network scan revealed an exposed Secure Shell (SSH) port (22/TCP). Publicly accessible management services like SSH are primary targets for automated brute-force attacks and can serve as an entry point for threat actors.

The combination of these findings presents a tangible risk to the confidentiality and integrity of the organization's sensitive data. This report details these findings and provides prioritized, actionable recommendations to address them.

% --- Section 2: Organizational Information ---
\section{Organizational Information}
The following details were used as the basis for this assessment. Based on the provided data, some identifying information has been anonymized.

\begin{itemize}
    \item \textbf{Organization Name:} \textbf{[Organization Name]}
    \item \textbf{Primary Domain:} \texttt{[Domain]}
    \item \textbf{External IP Scanned:} \texttt{[Client IP]}
\end{itemize}

% --- Section 3: Security Control Review ---
\section{Security Control Review}
An internal security questionnaire was reviewed to assess the current state of administrative and policy-based controls. The responses are summarized below. A checkmark (\yes) indicates a positive control is in place, while a cross (\no) indicates a potential security gap.

\begin{table}[h!]
\centering
\caption{Security Questionnaire Analysis}
\label{tab:questionnaire}
\begin{tabular}{p{0.75\linewidth} c}
\toprule
\textbf{Control Question} & \textbf{Response} \\
\midrule
Do you require MFA to access email? & \yes \\
Do you require MFA to log into computers? & \yes \\
\textbf{Do you require MFA to access sensitive data systems?} & \textcolor{red}{\no} \\
Does your organization have an employee acceptable use policy? & \yes \\
Does your organization do security awareness training for new employees? & \yes \\
Does your organization do security awareness training for all employees at least once per year? & \yes \\
\bottomrule
\end{tabular}
\end{table}

\subsection*{Analysis of Controls}
The organization has implemented several key security controls effectively, including MFA for email and computer logins, and a robust security awareness training program. However, the absence of MFA for accessing sensitive data systems is a \textbf{critical weakness}. Should an attacker compromise a user's credentials, they could potentially gain direct access to the organization's most valuable data without needing a second authentication factor.

% --- Section 4: Technical Scan Results ---
\section{Technical Scan Results}
An external network scan was performed on the organization's public-facing IP address to identify open ports and exposed services.

\begin{itemize}
    \item \textbf{Target IP Address:} \texttt{[Target IP]}
    \item \textbf{Scan Date:} Data from latest available scan.
\end{itemize}

The following table details the open ports discovered during the scan.

\begin{table}[h!]
\centering
\caption{Open Port Analysis}
\label{tab:nmap}
\begin{tabular}{l l l l}
\toprule
\textbf{Port} & \textbf{State} & \textbf{Service} & \textbf{Product / Version} \\
\midrule
22/TCP & open & ssh (assumed) & Not enumerated \\
\bottomrule
\end{tabular}
\end{table}

\subsection*{Analysis of Technical Findings}
The scan identified that port 22 (TCP), the standard port for the Secure Shell (SSH) protocol, is open to the public internet. SSH is a common administrative service used for remote server management. Exposing this service directly to the internet presents a \textbf{high risk} for the following reasons:
\begin{itemize}
    \item \textbf{Brute-Force Attacks:} Automated bots constantly scan the internet for open SSH ports and attempt to guess credentials.
    \item \textbf{Credential Stuffing:} If user credentials are leaked from another service, attackers will try them against this exposed SSH port.
    \item \textbf{Vulnerability Exploitation:} If the SSH server software is outdated or misconfigured, it could be vulnerable to remote code execution exploits.
\end{itemize}

% --- Section 5: Risk Assessment ---
\section{Risk Assessment}
This section correlates the findings from the security control review, technical scan, and any pre-existing known risks. The vulnerabilities are prioritized based on their potential impact and exploitability.

\begin{table}[h!]
\centering
\caption{Summary of Identified Risks}
\label{tab:risks}
\begin{tabular}{p{0.1\linewidth} p{0.45\linewidth} l p{0.2\linewidth}}
\toprule
\textbf{Risk ID} & \textbf{Description} & \textbf{Severity} & \textbf{Data Source} \\
\midrule
RISK-001 & Lack of Multi-Factor Authentication (MFA) on systems containing sensitive data. & \textbf{Critical} & Questionnaire \\
\addlinespace
RISK-002 & Exposed SSH management port (22/TCP) on an external-facing asset. & \textbf{High} & Network Scan \\
\addlinespace
\multicolumn{4}{l}{\textit{No pre-existing vulnerabilities were reported in the input data.}} \\
\bottomrule
\end{tabular}
\end{table}

% --- Section 6: Recommendations ---
\section{Recommendations}
The following actions are recommended to mitigate the identified risks. Recommendations are prioritized from most to least critical.

\subsection*{RISK-001: Lack of MFA on Sensitive Systems (Critical)}
\begin{itemize}
    \item \textbf{Immediate Action:} Enforce mandatory MFA for all user accounts, especially administrative and privileged accounts, that can access sensitive data systems.
    \item \textbf{Strategic Action:} Conduct an audit of all systems containing sensitive data to ensure that MFA is consistently applied. Develop a policy that mandates MFA as a baseline for any new system that will store or process sensitive information.
\end{itemize}

\subsection*{RISK-002: Exposed SSH Management Port (High)}
\begin{itemize}
    \item \textbf{Immediate Action:} If remote access via SSH is not required, block port 22 at the network firewall immediately.
    \item \textbf{Alternative Action:} If remote access is necessary, implement a bastion host or a Virtual Private Network (VPN) and require users to connect to it before accessing internal resources via SSH.
    \item \textbf{Hardening:} At a minimum, restrict access to port 22 to a whitelist of trusted IP addresses. Furthermore, ensure that the SSH service is configured securely by disabling password-based authentication in favor of public key cryptography and disabling root login.
\end{itemize}

\end{document}
% --- Document End ---
```