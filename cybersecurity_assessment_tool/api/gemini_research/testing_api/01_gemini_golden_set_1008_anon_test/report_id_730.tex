```latex
\documentclass[12pt]{article}

% Preamble: Required Packages
\usepackage[margin=1in]{geometry}
\usepackage{pifont} % For checkmarks and crosses
\usepackage{booktabs} % For professional tables
\usepackage{hyperref} % For hyperlinks
\usepackage{url} % For URL formatting
\usepackage{seqsplit} % To split long strings
\usepackage{graphicx} % For logo (optional)
\usepackage{xcolor} % For colors

% Document Information
\title{Cybersecurity Assessment Report}
\author{Cybersecurity Analysis Division}
\date{\today}

% Hyperref Setup
\hypersetup{
    colorlinks=true,
    linkcolor=blue,
    filecolor=magenta,      
    urlcolor=cyan,
    pdftitle={Cybersecurity Assessment Report},
    pdfpagemode=FullScreen,
}

\begin{document}

\maketitle
\thispagestyle{empty}
\newpage

\tableofcontents
\newpage

% --- 1. Executive Summary ---
\section{Executive Summary}

This report details the findings of a cybersecurity assessment conducted for \textbf{[Organization Name]}. The analysis is based on a network vulnerability scan, a review of organizational security controls, and an evaluation of pre-existing risks.

The assessment identified several critical and high-risk security gaps. The most significant findings are the complete lack of Multi-Factor Authentication (MFA) for email and computer access, and the absence of a formal security awareness training program and an acceptable use policy. These deficiencies expose the organization to a high likelihood of account compromise, phishing attacks, and insider threats.

The external network scan of the target IP address \texttt{[Target IP]} did not identify any open ports or services. While this is a positive finding, it does not guarantee security, as the scan may have been blocked by a firewall.

Immediate remediation is required to address the identified control gaps. Recommendations focus on implementing foundational security controls to significantly reduce the organization's risk exposure.

% --- 2. Organizational Information ---
\section{Organizational Information}

This section provides the organizational details used as the basis for this assessment. Due to missing data in the provided inputs, placeholders have been used.

\begin{table}[h!]
\centering
\begin{tabular}{@{}ll@{}}
\toprule
\textbf{Attribute} & \textbf{Value} \\ \midrule
Organization Name & \textbf{[Organization Name]} \\
Primary Domain & \texttt{[Domain]} \\
External IP Scanned & \texttt{[Client IP]} \\ 
Target IP from Scan Data & \texttt{[Target IP]} \\
\bottomrule
\end{tabular}
\caption{Organizational and Scoping Details.}
\label{tab:org_info}
\end{table}

% --- 3. Security Control Review ---
\section{Security Control Review}

A review of administrative and technical security controls was conducted via a standardized questionnaire. The responses indicate significant gaps in fundamental security practices. A "No" response (\ding{55}) highlights a missing control that increases organizational risk.

\begin{table}[h!]
\centering
\begin{tabular}{@{}p{0.75\linewidth}c@{}}
\toprule
\textbf{Control Question} & \textbf{Response} \\ \midrule
Do you require MFA to access email? & \ding{55} \\
Do you require MFA to log into computers? & \ding{55} \\
Do you require MFA to access sensitive data systems? & \ding{51} \\
Does your organization have an employee acceptable use policy? & \ding{55} \\
Does your organization do security awareness training for new employees? & \ding{55} \\
Does your organization do security awareness training for all employees at least once per year? & \ding{55} \\
\bottomrule
\end{tabular}
\caption{Security Controls Questionnaire Results (\ding{51}=Yes, \ding{55}=No).}
\label{tab:controls}
\end{table}

% --- 4. Technical Scan Results ---
\section{Technical Scan Results}

An external network scan was performed to identify open ports, running services, and potential vulnerabilities on the public-facing infrastructure.

\begin{itemize}
    \item \textbf{Target IP Address:} \texttt{[Target IP]}
    \item \textbf{Scan Date:} \today
\end{itemize}

\subsection{Scan Summary}
The network scan completed successfully but did not identify any open TCP or UDP ports on the target system. 

\textbf{Conclusion:} While no services were exposed, this result could indicate that a firewall or intrusion prevention system is effectively blocking scan probes. This is a positive security posture from an external perspective, but it does not preclude the existence of vulnerabilities that may be exploitable under different conditions or from an internal vantage point.

% --- 5. Risk Assessment ---
\section{Risk Assessment}

This section synthesizes the findings from the security control review and technical scan. No pre-existing vulnerabilities were provided for this assessment. The following risks have been identified based on the new findings.

\begin{table}[h!]
\centering
\begin{tabular}{@{}p{0.1\linewidth}p{0.2\linewidth}p{0.5\linewidth}l@{}}
\toprule
\textbf{Risk ID} & \textbf{Risk Name} & \textbf{Description} & \textbf{Severity} \\ \midrule
RISK-001 & Lack of MFA on Critical Systems & The absence of MFA on email and endpoint logins means a compromised password provides an attacker with direct access, facilitating data breaches and lateral movement. & \textbf{Critical} \\
\addlinespace
RISK-002 & No Security Awareness Program & Without training, employees are significantly more likely to fall victim to phishing and social engineering attacks, rendering technical controls less effective. & \textbf{High} \\
\addlinespace
RISK-003 & No Acceptable Use Policy (AUP) & The lack of a formal AUP creates ambiguity regarding the proper use of company assets, increasing the risk of insider threat, data leakage, and non-compliance. & \textbf{High} \\
\bottomrule
\end{tabular}
\caption{Identified Risks and Severity.}
\label{tab:risks}
\end{table}

% --- 6. Recommendations ---
\section{Recommendations}

The following actions are recommended to mitigate the identified risks and improve the overall security posture of \textbf{[Organization Name]}.

\subsection{Immediate Actions (0-30 Days)}
\begin{description}
    \item[For RISK-001:] \textbf{Implement Multi-Factor Authentication.}
    \begin{itemize}
        \item Prioritize the immediate rollout and enforcement of MFA for all users on email systems (e.g., Microsoft 365, Google Workspace).
        \item Begin planning the deployment of MFA for all computer/endpoint logins.
    \end{itemize}
\end{description}

\subsection{Short-Term Actions (30-90 Days)}
\begin{description}
    \item[For RISK-002:] \textbf{Establish a Security Awareness Program.}
    \begin{itemize}
        \item Develop and deliver mandatory security awareness training for all new and existing employees. Key topics must include phishing identification, password security, and data handling.
        \item Subscribe to a phishing simulation service to test and reinforce employee training.
    \end{itemize}
    \item[For RISK-003:] \textbf{Develop and Implement an Acceptable Use Policy.}
    \begin{itemize}
        \item Draft a formal AUP that clearly outlines the rules for using company networks, devices, and data.
        \item Require all employees to read and formally acknowledge the policy.
    \end{itemize}
\end{description}

\end{document}
```