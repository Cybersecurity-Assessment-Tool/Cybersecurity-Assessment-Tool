```latex
\documentclass[12pt]{article}

% Preamble: Required Packages
\usepackage[margin=1in]{geometry}
\usepackage{pifont} % For checkmarks and crosses
\usepackage{booktabs} % For professional tables
\usepackage{hyperref} % For clickable links and better PDF navigation
\usepackage{url}      % For formatting URLs
\usepackage{seqsplit} % To split long strings without breaking
\usepackage{graphicx} % For logo (placeholder)
\usepackage{fancyhdr} % For header/footer

% --- Document Metadata ---
\title{Cybersecurity Posture Assessment Report}
\author{Cybersecurity Analysis Division}
\date{\today}

% --- Header and Footer ---
\pagestyle{fancy}
\fancyhf{} % Clear all header and footer fields
\fancyhead[L]{\textbf{[Organization Name]} Cybersecurity Report}
\fancyfoot[C]{\thepage}
\renewcommand{\headrulewidth}{0.4pt}
\renewcommand{\footrulewidth}{0.4pt}

\begin{document}

\maketitle
\thispagestyle{empty}
\newpage

\tableofcontents
\newpage

% --- Section 1: Executive Overview ---
\section{Executive Overview}

This report details the findings of a cybersecurity assessment conducted for \textbf{[Organization Name]}. The assessment incorporated a review of organizational security controls, an external network scan, and an analysis of pre-existing risk data.

The overall security posture reveals several critical and high-risk gaps that require immediate attention. The most significant finding is the \textbf{lack of Multi-Factor Authentication (MFA) for email access}, which exposes the organization to a high likelihood of account compromise and subsequent data breaches.

Furthermore, foundational security policies, such as an Acceptable Use Policy and mandatory annual security training for all staff, are not in place. These policy gaps create an environment where security incidents are more likely to occur due to human error.

Technically, the external network scan identified an open port for unencrypted web traffic (HTTP), which poses a risk of data interception. Addressing these correlated findings should be the top priority to materially improve the organization's defensive posture.

% --- Section 2: Organizational Information ---
\section{Organizational Information}

The following details were used as the basis for this assessment. Due to the anonymized nature of the provided data, placeholders have been used where necessary.

\begin{itemize}
    \item \textbf{Organization Name:} \textbf{[Organization Name]}
    \item \textbf{Primary Domain:} \texttt{[Domain]}
    \item \textbf{External IP Address Scanned:} \texttt{[Client IP]}
\end{itemize}

% --- Section 3: Security Control Review ---
\section{Security Control Review}

A review of organizational security controls was conducted based on a standardized questionnaire. The responses indicate a mix of implemented controls and significant gaps. The table below summarizes the findings. A (\ding{51}) indicates a positive control is in place, while a (\ding{55}) indicates a control gap.

\begin{table}[h!]
\centering
\caption{Organizational Security Control Status}
\label{tab:controls}
\begin{tabular}{@{}lc@{}}
\toprule
\textbf{Control Question} & \textbf{Status} \\ \midrule
Do you require MFA to access email? & \ding{55} \\
Do you require MFA to log into computers? & \ding{51} \\
Do you require MFA to access sensitive data systems? & \ding{51} \\
Does your organization have an employee acceptable use policy? & \ding{55} \\
Does your organization do security awareness training for new employees? & \ding{51} \\
Does your organization do security awareness training for all employees at least once per year? & \ding{55} \\ \bottomrule
\end{tabular}
\end{table}

\subsection*{Analysis of Control Gaps}
The identified gaps represent significant risks:
\begin{itemize}
    \item \textbf{No MFA for Email:} \textbf{(Critical Risk)} Email is a primary target for attackers. Without MFA, a compromised password is all that is needed to gain access to sensitive communications, impersonate employees, and launch further attacks against the organization and its partners.
    \item \textbf{No Acceptable Use Policy (AUP):} \textbf{(High Risk)} An AUP sets clear expectations for how employees may use company resources. Its absence can lead to misuse of assets and creates ambiguity during incident response or HR actions.
    \item \textbf{No Annual Security Training:} \textbf{(High Risk)} The threat landscape evolves constantly. Failing to provide annual refresher training means employees' security awareness degrades over time, making them more susceptible to phishing and social engineering attacks.
\end{itemize}

% --- Section 4: Technical Scan Results ---
\section{Technical Scan Results}

An external network vulnerability scan was performed on the target IP address. The scan was non-intrusive and aimed to identify open ports and exposed services.

\begin{itemize}
    \item \textbf{Target IP Address:} \texttt{[Target IP]}
    \item \textbf{Scan Date:} The scan date was not provided in the scan metadata.
\end{itemize}

\subsection*{Open Ports and Services}
The scan identified the following open port:

\begin{table}[h!]
\centering
\caption{Open Port Findings}
\label{tab:ports}
\begin{tabular}{@{}llll@{}}
\toprule
\textbf{Port} & \textbf{Protocol} & \textbf{State} & \textbf{Service/Notes} \\ \midrule
80 & TCP & open & HTTP (Hypertext Transfer Protocol) \\ \bottomrule
\end{tabular}
\end{table}

\subsection*{Technical Analysis}
The presence of an open Port 80/HTTP is a significant finding. HTTP is an unencrypted protocol, meaning any data transmitted between a user's browser and the web server can be intercepted and read by an attacker on the same network (e.g., public Wi-Fi). This includes login credentials, session cookies, and other sensitive information. Best practice dictates that all web traffic should be encrypted using HTTPS (Port 443).

% --- Section 5: Consolidated Risk Assessment ---
\section{Consolidated Risk Assessment}

This section correlates the findings from the security control review and the technical scan. No legitimate, pre-existing vulnerabilities were provided for analysis in the input data. The following table summarizes the newly identified risks, prioritized by severity.

\begin{table}[h!]
\centering
\caption{Summary of Identified Risks}
\label{tab:risks}
\begin{tabular}{@{}p{0.25\linewidth}p{0.15\linewidth}p{0.5\linewidth}@{}}
\toprule
\textbf{Risk Name} & \textbf{Severity} & \textbf{Description} \\ \midrule
\textbf{Email Account Compromise via Password Attack} & \textbf{Critical} & The lack of MFA on email accounts makes them highly susceptible to takeover if user credentials are leaked or guessed. This can lead to a full-scale data breach. \\
\addlinespace
\textbf{Unencrypted Web Traffic Interception} & \textbf{High} & The active HTTP service on port 80 allows for man-in-the-middle attacks, where sensitive user data, including credentials, can be stolen in transit. \\
\addlinespace
\textbf{Inadequate Security Policies and Training} & \textbf{High} & The absence of an AUP and recurring annual security training increases the likelihood of security incidents caused by uninformed or negligent employee behavior. \\ \bottomrule
\end{tabular}
\end{table}

% --- Section 6: Recommendations ---
\section{Recommendations}

The following actionable recommendations are provided to mitigate the identified risks. They are prioritized to address the most critical issues first.

\subsection*{Immediate Priority (Implement within 7 days)}
\begin{enumerate}
    \item \textbf{Enforce MFA on All Email Accounts:} Immediately enable and enforce MFA for all user mailboxes. This is the single most effective control to prevent unauthorized access to email.
\end{enumerate}

\subsection*{High Priority (Implement within 30 days)}
\begin{enumerate}
    \item \textbf{Implement HTTPS and Disable HTTP:}
        \begin{itemize}
            \item Obtain and install a TLS/SSL certificate for the service running on \texttt{[Target IP]}.
            \item Configure the web server to redirect all HTTP traffic to HTTPS (Port 443).
            \item If possible, block Port 80 at the firewall once redirection is confirmed to be working.
        \end{itemize}
\end{enumerate}

\subsection*{Medium Priority (Implement within 90 days)}
\begin{enumerate}
    \item \textbf{Develop and Implement an Acceptable Use Policy (AUP):} Create a formal AUP document that all employees must read and sign. This policy should clearly define the rules for using company IT assets, data, and internet access.
    \item \textbf{Establish an Annual Security Awareness Training Program:} Implement a mandatory, recurring security awareness training program for all employees. The training should be conducted at least once per year and cover current threats such as phishing, malware, and social engineering.
\end{enumerate}

\end{document}
```