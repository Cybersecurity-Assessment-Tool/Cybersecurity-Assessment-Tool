```latex
\documentclass[12pt]{article}

% Preamble: Required Packages
\usepackage[margin=1in]{geometry}
\usepackage{pifont} % For \ding symbols (checkmarks and crosses)
\usepackage{booktabs} % For professional-looking tables
\usepackage{hyperref} % For clickable links
\usepackage{url}      % For formatting URLs
\usepackage{seqsplit} % To split long strings without breaking
\usepackage{xcolor}   % For coloring text
\usepackage{graphicx} % For potential logos or diagrams
\usepackage{fancyhdr} % For headers and footers

% --- Document Setup ---
\hypersetup{
    colorlinks=true,
    linkcolor=blue,
    filecolor=magenta,
    urlcolor=cyan,
}

% --- Header and Footer ---
\pagestyle{fancy}
\fancyhf{} % Clear all header and footer fields
\fancyhead[L]{Cybersecurity Posture Assessment}
\fancyhead[R]{\textbf{[Organization Name]}}
\fancyfoot[C]{\thepage}
\renewcommand{\headrulewidth}{0.4pt}
\renewcommand{\footrulewidth}{0.4pt}

% --- Begin Document ---
\begin{document}

% --- Title Page ---
\begin{titlepage}
    \centering
    \vspace*{1cm}
    \Huge\textbf{Cybersecurity Posture Assessment Report}
    \vspace{1.5cm}
    \Large
    \begin{tabular}{ll}
        \textbf{Client:} & \textbf{[Organization Name]} \\
        \textbf{Date of Report:} & \today \\
        \textbf{Author:} & Cybersecurity Analyst \\
    \end{tabular}
    \vfill
    \small\textit{This report contains sensitive information and is intended solely for the use of \textbf{[Organization Name]}. Distribution is strictly prohibited.}
\end{titlepage}

\tableofcontents
\newpage

% --- Section 1: Executive Summary ---
\section{Executive Summary}
This report provides a comprehensive analysis of the cybersecurity posture for \textbf{[Organization Name]}, based on technical network scans, a review of existing security controls, and an assessment of previously identified risks.

The assessment has identified several critical and high-severity risks that require immediate attention. The most critical finding is the direct exposure of Remote Desktop Protocol (RDP) on port 3389 to the public internet. This vulnerability, with a CVSS score of 9.0, presents a significant and immediate threat of ransomware attacks or unauthorized system access.

This technical vulnerability is severely compounded by critical gaps in organizational security controls. Specifically, the absence of Multi-Factor Authentication (MFA) for computer logins means that a single compromised password could lead to a full system breach. Furthermore, the lack of a formal Acceptable Use Policy and mandatory security training for new employees indicates a foundational weakness in the organization's security culture.

Immediate remediation is required to address the exposed RDP service and enforce MFA. Strategic improvements to security policies and training programs are necessary to build a more resilient and secure operational environment.

% --- Section 2: Organizational Information ---
\section{Organizational Information}
This section details the information provided by the client organization.
\begin{description}
    \item[Organization Name:] \textbf{[Organization Name]}
    \item[Email Domain:] \texttt{[Domain]}
    \item[External IP Scanned:] \texttt{[Client IP]}
\end{description}

% --- Section 3: Security Control Review ---
\section{Security Control Review}
A review of the organization's security controls was conducted via a questionnaire. The responses are summarized below. Gaps identified by a "No" answer are marked in red and represent significant areas of risk.

\begin{table}[h!]
\centering
\caption{Security Controls Questionnaire Results}
\begin{tabular}{p{0.75\textwidth} c}
\toprule
\textbf{Control Question} & \textbf{Response} \\
\midrule
Do you require MFA to access email? & \ding{51} \\
Do you require MFA to log into computers? & \textcolor{red}{\ding{55}} \\
Do you require MFA to access sensitive data systems? & \ding{51} \\
Does your organization have an employee acceptable use policy? & \textcolor{red}{\ding{55}} \\
Does your organization do security awareness training for new employees? & \textcolor{red}{\ding{55}} \\
Does your organization do security awareness training for all employees at least once per year? & \ding{51} \\
\bottomrule
\end{tabular}
\end{table}

\subsection*{Analysis of Gaps}
\begin{itemize}
    \item \textbf{No MFA for Computer Logins:} This is a critical gap. Without MFA, compromised credentials (e.g., from a phishing attack) can be used directly to gain access to an employee's computer and, potentially, the entire network.
    \item \textbf{No Acceptable Use Policy:} The absence of a formal policy creates ambiguity regarding employee responsibilities for protecting company data and systems.
    \item \textbf{No Security Training for New Hires:} New employees are a common target for social engineering attacks. Failing to provide security training during onboarding leaves the organization vulnerable.
\end{itemize}

% --- Section 4: Technical Scan Results ---
\section{Technical Scan Results}
A network scan was performed on the client's external infrastructure to identify open ports and exposed services.

\begin{description}
    \item[Target IP Address:] \texttt{[Target IP]}
    \item[Scan Date:] Corresponds with the date of this report.
\end{description}

\begin{table}[h!]
\centering
\caption{Open Ports Detected on \texttt{[Target IP]}}
\begin{tabular}{llll}
\toprule
\textbf{Port} & \textbf{State} & \textbf{Service Name} & \textbf{Product/Version} \\
\midrule
3389/tcp & open & ms-wbt-server & (Not specified) \\
\bottomrule
\end{tabular}
\end{table}

\subsection*{Analysis of Findings}
The scan confirms that port \textbf{3389}, used for Microsoft's Remote Desktop Protocol (RDP), is open to the public internet. Exposing RDP directly is an extremely high-risk configuration. This service is a primary target for threat actors who use automated tools to scan for open RDP ports, which they then attack via brute-force password guessing, credential stuffing, or exploitation of known vulnerabilities (e.g., BlueKeep). A successful attack can lead to a full system compromise and is a common entry point for ransomware deployment.

% --- Section 5: Correlated Risk Assessment ---
\section{Correlated Risk Assessment}
The following table synthesizes findings from the security control review, technical scan, and pre-existing risk data into a prioritized list of current risks.

\begin{table}[h!]
\centering
\caption{Summary of Identified Risks}
\begin{tabular}{p{0.25\textwidth} p{0.5\textwidth} l}
\toprule
\textbf{Risk Name} & \textbf{Description} & \textbf{Severity} \\
\midrule
\textbf{Public RDP Exposure} & Port 3389 (RDP) is exposed on \texttt{[Target IP]}, allowing direct remote access attempts from the internet. This is a primary vector for ransomware. & \textbf{Critical} \\
\addlinespace
\textbf{Lack of Endpoint MFA} & The absence of MFA on computer logins, combined with the exposed RDP service, means a single stolen password could grant an attacker full remote control. & \textbf{Critical} \\
\addlinespace
\textbf{Missing Acceptable Use Policy} & Without a formal policy, there is no enforceable standard for employee behavior regarding data handling and system security, increasing insider threat risk. & \textbf{High} \\
\addlinespace
\textbf{Inadequate New Hire Security Training} & New employees are not trained on security best practices, making them highly susceptible to phishing and social engineering attacks that could lead to credential compromise. & \textbf{High} \\
\bottomrule
\end{tabular}
\end{table}

% --- Section 6: Recommendations ---
\section{Recommendations}
The following actions are recommended to mitigate the identified risks. They are prioritized based on severity and the potential for impact.

\subsection*{Priority 1: Immediate Actions (Within 24-48 Hours)}
\begin{enumerate}
    \item \textbf{Isolate the Exposed RDP Service:}
    \begin{itemize}
        \item \textbf{Action:} Immediately implement a firewall rule to block all inbound traffic to TCP port 3389 on \texttt{[Target IP]}.
        \item \textbf{Justification:} This is the single most effective step to prevent an imminent breach via the exposed RDP service.
        \item \textbf{Alternative:} If remote access is business-critical, it must be placed behind a secure Virtual Private Network (VPN) with MFA enabled.
    \end{itemize}
\end{enumerate}

\subsection*{Priority 2: High-Priority Actions (Within 30 Days)}
\begin{enumerate}
    \setcounter{enumi}{1} % Continue numbering
    \item \textbf{Implement Multi-Factor Authentication for All Endpoints:}
    \begin{itemize}
        \item \textbf{Action:} Deploy and enforce an MFA solution for all computer and server logins across the organization.
        \item \textbf{Justification:} This creates a critical defense layer, ensuring that a compromised password alone is not sufficient for an attacker to gain access.
    \end{itemize}
    \item \textbf{Develop and Implement an Acceptable Use Policy (AUP):}
    \begin{itemize}
        \item \textbf{Action:} Draft a formal AUP that clearly defines rules for all employees regarding the use of company technology, data handling, and security responsibilities. All employees must read and acknowledge the policy.
        \item \textbf{Justification:} An AUP establishes a baseline for secure behavior and provides a framework for enforcement.
    \end{itemize}
    \item \textbf{Integrate Security Training into Onboarding:}
    \begin{itemize}
        \item \textbf{Action:} Create a mandatory security awareness training module for all new hires. This should cover topics such as phishing, password security, and the new AUP.
        \item \textbf{Justification:} This ensures that new employees are aware of security threats and their role in protecting the organization from day one.
    \end{itemize}
\end{enumerate}

\subsection*{Priority 3: Strategic Long-Term Improvements}
\begin{enumerate}
    \setcounter{enumi}{4} % Continue numbering
    \item \textbf{Adopt a Zero Trust Security Model:}
    \begin{itemize}
        \item \textbf{Action:} Begin planning a transition away from perimeter-based security towards a Zero Trust architecture, where all access requests (both internal and external) are continuously verified.
        \item \textbf{Justification:} This modern approach significantly reduces the attack surface and minimizes the impact of a potential breach.
    \end{itemize}
\end{enumerate}

\end{document}
```