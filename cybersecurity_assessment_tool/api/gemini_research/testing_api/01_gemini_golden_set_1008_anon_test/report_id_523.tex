```latex
\documentclass[12pt]{article}

% Preamble: Required Packages
\usepackage[margin=1in]{geometry}
\usepackage{pifont} % For checkmarks and crosses (dingbats)
\usepackage{booktabs} % For professional-looking tables
\usepackage{hyperref} % For hyperlinks
\usepackage{url} % For URL formatting
\usepackage{seqsplit} % To split long text sequences without breaking words
\usepackage{graphicx}
\usepackage{xcolor}
\usepackage{fancyhdr}

% --- Document Setup ---
\hypersetup{
    colorlinks=true,
    linkcolor=blue,
    filecolor=magenta,      
    urlcolor=cyan,
    pdftitle={Cybersecurity Posture Assessment Report},
    pdfpagemode=FullScreen,
}

% Define colors for severity levels
\definecolor{critical}{HTML}{990000}
\definecolor{high}{HTML}{D14302}
\definecolor{medium}{HTML}{E5A500}
\definecolor{low}{HTML}{339900}

% Header and Footer
\pagestyle{fancy}
\fancyhf{}
\fancyhead[L]{Cybersecurity Assessment Report}
\fancyhead[R]{\textbf{[Organization Name]}}
\fancyfoot[C]{\thepage}

% --- Document Start ---
\begin{document}

% --- Title Page ---
\begin{titlepage}
    \centering
    \vspace*{1cm}
    
    \Huge
    \textbf{Cybersecurity Posture Assessment Report}
    
    \vspace{1.5cm}
    
    \Large
    Prepared for: \\
    \vspace{0.5cm}
    \textbf{[Organization Name]}
    
    \vspace{2cm}
    
    \large
    Report Date: \today
    
    \vfill
    
    \large
    \textit{This report contains sensitive information and should be handled with care.}
    
\end{titlepage}

\tableofcontents
\newpage

% --- Section 1: Executive Summary ---
\section{Executive Summary}
This report provides a comprehensive analysis of the cybersecurity posture for \textbf{[Organization Name]}, based on a review of organizational security controls, an external network scan, and pre-existing risk data.

The assessment reveals a \textbf{critically weak security posture} characterized by fundamental control gaps. The complete absence of Multi-Factor Authentication (MFA) across all critical systems, including email, endpoints, and sensitive data repositories, represents an immediate and severe threat. This is compounded by the lack of foundational security policies and employee awareness training.

Furthermore, technical analysis identified an externally exposed Secure Shell (SSH) service on port 22. When combined with the lack of MFA, this creates a high-impact attack vector for credential-based attacks, potentially leading to a full network compromise.

Immediate and decisive action is required to remediate these critical vulnerabilities. The recommendations outlined in this report are prioritized to address the most significant risks first, with a focus on implementing MFA, securing network services, and establishing a baseline of security governance.

% --- Section 2: Organizational Information ---
\section{Organizational Information}
The following details were used as the basis for this assessment. Due to missing data in the provided inputs, placeholders have been used.

\begin{itemize}
    \item \textbf{Organization Name:} \textbf{[Organization Name]}
    \item \textbf{Primary Domain:} \texttt{[Domain]}
    \item \textbf{External IP Address Scanned:} \texttt{[Client IP]}
\end{itemize}

% --- Section 3: Security Control Review ---
\section{Security Control Review}
The following table summarizes the responses from the organizational security questionnaire. "No" answers indicate significant gaps in the organization's defensive capabilities and are marked with \ding{55}.

\begin{table}[h!]
\centering
\caption{Organizational Security Control Questionnaire Analysis}
\begin{tabular}{p{0.6\linewidth} c p{0.25\linewidth}}
\toprule
\textbf{Control Question} & \textbf{Response} & \textbf{Assessment} \\
\midrule
Do you require MFA to access email? & \ding{55} & \textcolor{critical}{\textbf{Critical Control Gap}} \\
Do you require MFA to log into computers? & \ding{55} & \textcolor{critical}{\textbf{Critical Control Gap}} \\
Do you require MFA to access sensitive data systems? & \ding{55} & \textcolor{critical}{\textbf{Critical Control Gap}} \\
Does your organization have an employee acceptable use policy? & \ding{55} & \textcolor{high}{\textbf{High-Risk Policy Gap}} \\
Does your organization do security awareness training for new employees? & \ding{55} & \textcolor{high}{\textbf{High-Risk Process Gap}} \\
Does your organization do security awareness training for all employees at least once per year? & \ding{55} & \textcolor{high}{\textbf{High-Risk Process Gap}} \\
\bottomrule
\end{tabular}
\end{table}

The consistent "No" responses across all questions highlight a lack of fundamental security controls. The absence of MFA is the most pressing issue, leaving the organization highly vulnerable to account takeover attacks. The lack of policies and training exacerbates this risk by failing to establish a security-conscious culture.

% --- Section 4: Technical Scan Results ---
\section{Technical Scan Results}
An external network scan was performed on the target IP address. The scan identified the following open ports and services.

\begin{itemize}
    \item \textbf{Target IP Address:} \texttt{[Target IP]}
    \item \textbf{Scan Date:} Not specified in scan data.
\end{itemize}

\begin{table}[h!]
\centering
\caption{Open Port Analysis}
\begin{tabular}{l l l p{0.5\linewidth}}
\toprule
\textbf{Port} & \textbf{State} & \textbf{Service} & \textbf{Analyst Notes} \\
\midrule
22/tcp & open & SSH & The Secure Shell service is exposed to the public internet. This is a common target for brute-force and credential stuffing attacks. Without version information, specific vulnerabilities cannot be confirmed, but exposure itself is a significant risk. \\
\bottomrule
\end{tabular}
\end{table}

% --- Section 5: Risk Assessment ---
\section{Risk Assessment}
This section synthesizes findings from the security control review and the technical scan. The pre-existing risk data provided (\texttt{Input\_3\_Current\_Risks\_JSON}) was empty, indicating a potential gap in the organization's existing risk management program.

\begin{table}[h!]
\centering
\caption{Synthesized Risk Summary}
\begin{tabular}{p{0.2\linewidth} p{0.6\linewidth} c}
\toprule
\textbf{Risk ID} & \textbf{Finding} & \textbf{Severity} \\
\midrule
RISK-001 & \textbf{Absence of Multi-Factor Authentication (MFA):} No MFA is enforced for email, computer logins, or access to sensitive data, leaving all accounts vulnerable to simple password compromise. & \textcolor{critical}{\textbf{Critical}} \\
\addlinespace
RISK-002 & \textbf{Exposed SSH Management Service:} Port 22 (SSH) is open to the internet, providing a direct vector for attackers to attempt unauthorized access to internal systems. This risk is amplified by RISK-001. & \textcolor{critical}{\textbf{Critical}} \\
\addlinespace
RISK-003 & \textbf{Lack of Security Policies and Procedures:} The absence of an Acceptable Use Policy means there are no formal guidelines for employees regarding the secure use of company assets. & \textcolor{high}{\textbf{High}} \\
\addlinespace
RISK-004 & \textbf{No Security Awareness Training Program:} Employees are not trained on security best practices, making them highly susceptible to phishing, social engineering, and other common attack methods. & \textcolor{high}{\textbf{High}} \\
\bottomrule
\end{tabular}
\end{table}

% --- Section 6: Recommendations ---
\section{Recommendations}
The following actionable recommendations are provided to mitigate the identified risks. They are prioritized based on severity and potential impact.

\subsection{Immediate Priority (Critical Risks)}
\begin{enumerate}
    \item \textbf{Implement Multi-Factor Authentication (MFA) Universally:}
    \begin{itemize}
        \item \textbf{Action:} Deploy a robust MFA solution (e.g., authenticator app, hardware token) for all user accounts.
        \item \textbf{Scope:} Prioritize email (e.g., Office 365, Google Workspace), VPN access, endpoint logins (computers), and all systems containing sensitive data.
        \item \textbf{Justification:} This is the single most effective control to prevent account takeovers, mitigating RISK-001.
    \end{itemize}
    \item \textbf{Secure the Exposed SSH Service:}
    \begin{itemize}
        \item \textbf{Action:} If remote access is required, place the SSH service behind a VPN. If it must remain exposed, restrict access via a firewall to only trusted IP addresses.
        \item \textbf{Hardening:} Additionally, disable password-based authentication in favor of public key cryptography and implement an intrusion prevention tool like \texttt{fail2ban}.
        \item \textbf{Justification:} This action directly mitigates RISK-002 by removing or hardening the exposed attack surface.
    \end{itemize}
\end{enumerate}

\subsection{High Priority (High Risks)}
\begin{enumerate}
    \setcounter{enumi}{2} % Continue numbering from previous list
    \item \textbf{Develop and Implement Foundational Security Policies:}
    \begin{itemize}
        \item \textbf{Action:} Draft, approve, and communicate an organization-wide Acceptable Use Policy (AUP). This policy should clearly define rules for handling data, using company equipment, and accessing the internet.
        \item \textbf{Justification:} Establishes a baseline for secure behavior and provides a framework for enforcement, mitigating RISK-003.
    \end{itemize}
    \item \textbf{Establish a Security Awareness Training Program:}
    \begin{itemize}
        \item \textbf{Action:} Implement a mandatory security awareness training module for all new hires during onboarding. Subsequently, conduct annual refresher training for all employees covering topics like phishing, password hygiene, and social engineering.
        \item \textbf{Justification:} Creates a human firewall, reducing the likelihood of successful phishing and other social engineering attacks, mitigating RISK-004.
    \end{itemize}
\end{enumerate}

\subsection{Strategic Recommendation}
\begin{enumerate}
    \setcounter{enumi}{4}
    \item \textbf{Implement a Formal Vulnerability Management Program:}
    \begin{itemize}
        \item \textbf{Action:} Establish a process to regularly scan for, identify, track, and remediate vulnerabilities across all organizational assets.
        \item \textbf{Justification:} The empty pre-existing risk list suggests this capability is missing. A formal program will ensure that risks are managed proactively rather than reactively.
    \end{itemize}
\end{enumerate}

% --- Document End ---
\end{document}
```