```latex
\documentclass[12pt, a4paper]{article}

% Preamble: Required Packages
\usepackage[margin=1in]{geometry}
\usepackage{pifont} % For checkmarks and crosses
\usepackage{booktabs} % For professional tables
\usepackage{hyperref} % For hyperlinks
\usepackage{url} % For URL formatting
\usepackage{seqsplit} % For splitting long strings
\usepackage{graphicx}
\usepackage{xcolor}
\usepackage{fancyhdr}

% Document Metadata
\title{Cybersecurity Posture Assessment Report}
\author{Cybersecurity Analysis Division}
\date{\today}

% Hyperref Setup
\hypersetup{
    colorlinks=true,
    linkcolor=blue,
    filecolor=magenta,      
    urlcolor=cyan,
    pdftitle={Cybersecurity Posture Assessment Report},
    pdfpagemode=FullScreen,
}

% Header and Footer
\pagestyle{fancy}
\fancyhf{}
\lhead{Security Report for \textbf{[Organization Name]}}
\rfoot{Page \thepage}

\begin{document}

\maketitle
\thispagestyle{empty}
\newpage

\tableofcontents
\newpage

% --- 1. Executive Summary ---
\section{Executive Summary}

This report details the findings of a cybersecurity posture assessment conducted for \textbf{[Organization Name]}. The assessment synthesized data from an external network scan, a security controls questionnaire, and a list of pre-existing risks.

The analysis revealed a \textbf{Critical} external vulnerability. A publicly accessible FTP server was identified running \texttt{vsftpd version 2.3.4}, a version known to contain a critical backdoor vulnerability (CVE-2011-2523) that allows for remote code execution. This server is also misconfigured to allow anonymous logins, posing a severe and immediate threat to the organization's data and network integrity.

Furthermore, significant internal control gaps were identified. The lack of Multi-Factor Authentication (MFA) for computer logins and the absence of a mandatory annual security awareness training program represent high-risk deficiencies. These gaps, coupled with the existing risk of outdated Windows 7 workstations, create an environment susceptible to both external attacks and internal threats.

Immediate remediation of the vulnerable FTP server is paramount. Following this, we strongly recommend implementing the high-priority recommendations outlined in this report to strengthen internal security controls and reduce the overall attack surface.

% --- 2. Organizational Information ---
\section{Organizational Information}

This section provides the organizational details used as the basis for this assessment. As per the provided data, some information has been anonymized and is represented by placeholders.

\begin{itemize}
    \item \textbf{Organization Name:} \textbf{[Organization Name]}
    \item \textbf{Primary Domain:} \texttt{[Domain]}
    \item \textbf{External IP Scanned:} \texttt{[Client IP]}
\end{itemize}

% --- 3. Security Control Review ---
\section{Security Control Review (Questionnaire)}

An internal security questionnaire was reviewed to assess the current state of administrative and technical controls. The responses are summarized below. A checkmark (\ding{51}) indicates a positive control, while a cross (\ding{55}) indicates a control gap.

\begin{table}[h!]
\centering
\caption{Security Controls Questionnaire Results}
\begin{tabular}{p{0.8\linewidth} c}
\toprule
\textbf{Control Question} & \textbf{Status} \\
\midrule
Do you require MFA to access email? & \ding{51} \\
Do you require MFA to log into computers? & \textcolor{red}{\ding{55}} \\
Do you require MFA to access sensitive data systems? & \ding{51} \\
Does your organization have an employee acceptable use policy? & \ding{51} \\
Does your organization do security awareness training for new employees? & \ding{51} \\
Does your organization do security awareness training for all employees at least once per year? & \textcolor{red}{\ding{55}} \\
\bottomrule
\end{tabular}
\end{table}

\subsection*{Analysis of Control Gaps}
Two significant control gaps were identified from the questionnaire:
\begin{enumerate}
    \item \textbf{No MFA for Computer Logins:} The absence of MFA on endpoint devices is a critical weakness. If an employee's credentials are stolen (e.g., through phishing), an attacker can gain direct access to their computer and potentially move laterally across the network.
    \item \textbf{No Annual Security Awareness Training:} Security is an ongoing process. While training new hires is a good first step, the threat landscape evolves continuously. Without annual refresher training, employees are more likely to fall victim to modern phishing, social engineering, and malware attacks.
\end{enumerate}

% --- 4. Technical Scan Results ---
\section{Technical Penetration Test Results}

An external network scan was performed to identify publicly accessible services and potential vulnerabilities.

\subsection*{External Network Scan on \texttt{[Target IP]}}
The scan, conducted on \today, identified one host as "up" with the following open port:

\begin{table}[h!]
\centering
\caption{Open Ports and Services Detected}
\begin{tabular}{l l l l}
\toprule
\textbf{Port} & \textbf{State} & \textbf{Service} & \textbf{Version Details} \\
\midrule
21/tcp & Open & ftp & vsftpd 2.3.4 \\
\bottomrule
\end{tabular}
\end{table}

\subsection*{Analysis of Technical Findings}
The technical scan revealed two major security issues associated with the open FTP port:

\begin{enumerate}
    \item \textbf{Critical Vulnerability (CVE-2011-2523):} The identified version, \textbf{\texttt{vsftpd 2.3.4}}, contains a well-known, critical backdoor. This vulnerability was intentionally added to the source code and allows an unauthenticated remote attacker to execute arbitrary commands with root privileges on the server. This represents a direct path for a complete system compromise.
    
    \item \textbf{Insecure Configuration:} The scan confirmed that \textbf{Anonymous FTP login is allowed}. This configuration permits any user on the internet to connect to the FTP server without credentials, granting them access to list, download, and potentially upload files. This poses a significant risk of data leakage and could allow an attacker to stage malicious files on the server.
\end{enumerate}

% --- 5. Consolidated Risk Assessment ---
\section{Consolidated Risk Assessment}

This table synthesizes findings from the questionnaire, the technical scan, and pre-existing risks to provide a holistic view of the organization's risk profile.

\begin{table}[h!]
\centering
\caption{Summary of Identified Risks}
\begin{tabular}{p{0.3\linewidth} p{0.5\linewidth} l}
\toprule
\textbf{Risk Name} & \textbf{Overview} & \textbf{Severity} \\
\midrule
\textbf{RCE in vsftpd 2.3.4} & A critical backdoor in the public-facing FTP server allows for remote command execution and full system compromise. & \textbf{Critical} \\
\addlinespace
\textbf{Anonymous FTP Access} & The FTP server is configured to allow unauthenticated access, risking unauthorized data access and exposure. & \textbf{High} \\
\addlinespace
\textbf{No MFA on Endpoints} & The lack of MFA for computer logins significantly increases the risk of unauthorized access via compromised credentials. & \textbf{High} \\
\addlinespace
\textbf{No Annual Security Training} & Employees do not receive recurring security training, increasing susceptibility to social engineering and phishing attacks. & \textbf{Medium} \\
\addlinespace
\textbf{Outdated Windows Policy} & Workstations are running Windows 7, an unsupported OS that no longer receives security updates. & \textbf{Medium} \\
\bottomrule
\end{tabular}
\end{table}

% --- 6. Recommendations ---
\section{Recommendations}

The following actionable recommendations are provided to mitigate the identified risks. They are prioritized based on severity and potential impact.

\subsection*{Critical Priority}
\begin{itemize}
    \item \textbf{Remediate Vulnerable FTP Server Immediately:}
    \begin{enumerate}
        \item Take the server at \texttt{[Target IP]} offline immediately to prevent exploitation.
        \item If FTP is a business necessity, rebuild the server using a modern, patched version of an FTP service (\texttt{vsftpd 3.0.3} or later).
        \item If FTP is not required, decommission the server permanently.
    \end{enumerate}
\end{itemize}

\subsection*{High Priority}
\begin{itemize}
    \item \textbf{Disable Anonymous FTP Access:} Once the FTP server is patched and back online (if required), ensure that anonymous logins are explicitly disabled in its configuration file. Access should be restricted to authenticated users only.
    \item \textbf{Implement MFA for Endpoint Logins:} Deploy an MFA solution for all employee computer logins. Prioritize implementation for administrative accounts and users with access to sensitive data.
\end{itemize}

\subsection*{Medium Priority}
\begin{itemize}
    \item \textbf{Establish Annual Security Training:} Develop and enforce a mandatory security awareness training program for all employees to be completed at least once per year. This program should cover topics like phishing, password hygiene, and acceptable use.
    \item \textbf{Accelerate OS Upgrade Plan:} Prioritize the project to upgrade or replace all workstations running Windows 7. Operating on an unsupported OS exposes the organization to a wide range of unpatchable vulnerabilities.
\end{itemize}

\end{document}
```