```latex
\documentclass[12pt]{article}

% Preamble: Required Packages
\usepackage[margin=1in]{geometry}
\usepackage{pifont} % Required for \ding{51} (checkmark) and \ding{55} (cross)
\usepackage{booktabs} % For professional-looking tables (\toprule, \midrule, \bottomrule)
\usepackage{hyperref} % For clickable links and references
\usepackage{url}      % For formatting URLs
\usepackage{seqsplit} % For splitting long strings in \texttt
\usepackage{fancyhdr} % For custom headers and footers
\usepackage{lastpage} % To get the total number of pages

% --- Document Setup ---
\hypersetup{
    colorlinks=true,
    linkcolor=black,
    urlcolor=blue,
    pdftitle={Cybersecurity Posture Assessment Report},
    pdfauthor={Cybersecurity Analysis Division},
}

% --- Header and Footer ---
\pagestyle{fancy}
\fancyhf{} % Clear all header and footer fields
\fancyhead[L]{Cybersecurity Posture Assessment}
\fancyhead[R]{\textbf{[Organization Name]}}
\fancyfoot[C]{Page \thepage\ of \pageref{LastPage}}
\renewcommand{\headrulewidth}{0.4pt}
\renewcommand{\footrulewidth}{0.4pt}

% --- Document Start ---
\begin{document}

% --- Title Page ---
\begin{titlepage}
    \centering
    \vspace*{2cm}
    {\Huge \bfseries Cybersecurity Posture Assessment Report}
    \vspace{1.5cm}
    \rule{\linewidth}{0.5mm}
    \vspace{0.5cm}
    {\Large Prepared For:}
    \vspace{0.5cm}
    {\Huge \textbf{[Organization Name]}}
    \vspace{2cm}
    {\large Report Date: \today}
    \vfill
    {\large \bfseries Confidential}
    \vspace{1cm}
\end{titlepage}

\tableofcontents
\newpage

% --- Section 1: Executive Summary ---
\section{Executive Summary}
This report provides a comprehensive analysis of the cybersecurity posture for \textbf{[Organization Name]}, based on a review of organizational security controls, an external network scan, and an assessment of current risks. The assessment was conducted on \today.

The analysis reveals a mixed security posture. The organization demonstrates a commitment to security through the implementation of security awareness training and multi-factor authentication (MFA) for email access. The external network scan of the target host indicates a strong perimeter defense, as no open ports or services were discovered.

However, several critical and high-risk gaps were identified in internal security controls. The absence of mandatory MFA for computer logins and access to sensitive data systems presents a significant risk of unauthorized access and potential data breach. Furthermore, the lack of a formal Employee Acceptable Use Policy creates ambiguity regarding security responsibilities and acceptable behavior, weakening the overall security culture.

Immediate remediation should focus on implementing robust MFA across all critical systems and developing foundational security policies to mitigate these identified risks.

% --- Section 2: Organizational Information ---
\section{Organizational Information}
This section details the information provided for the assessment. Placeholders are used where data was not available.
\begin{itemize}
    \item \textbf{Organization Name:} \textbf{[Organization Name]}
    \item \textbf{Primary Email Domain:} \texttt{[Domain]}
    \item \textbf{Scanned External IP:} \texttt{[Client IP]}
\end{itemize}

% --- Section 3: Security Control Review ---
\section{Security Control Review}
The following table summarizes the organization's responses to a security controls questionnaire. This review provides insight into the current state of implemented policies and procedures. A green checkmark (\ding{51}) indicates a positive control, while a red cross (\ding{55}) highlights a potential gap.

\begin{table}[h!]
\centering
\caption{Organizational Security Controls Questionnaire}
\label{tab:controls}
\begin{tabular}{p{0.7\linewidth} c}
\toprule
\textbf{Control Question} & \textbf{Response} \\
\midrule
Do you require MFA to access email? & \ding{51} \\
Do you require MFA to log into computers? & \ding{55} \\
Do you require MFA to access sensitive data systems? & \ding{55} \\
Does your organization have an employee acceptable use policy? & \ding{55} \\
Does your organization do security awareness training for new employees? & \ding{51} \\
Does your organization do security awareness training for all employees at least once per year? & \ding{51} \\
\bottomrule
\end{tabular}
\end{table}

\paragraph{Analysis:} The questionnaire reveals critical gaps in access control. While email is protected by MFA, the failure to extend this protection to computer logins and sensitive data systems exposes the organization to significant risk from compromised credentials. Additionally, the absence of an Acceptable Use Policy is a foundational governance gap.

% --- Section 4: Technical Scan Results ---
\section{Technical Scan Results}
An external network vulnerability scan was performed to identify exposed services and potential vulnerabilities on the organization's perimeter.

\subsection{Nmap Scan Findings}
\begin{itemize}
    \item \textbf{Target IP Address:} \texttt{[Target IP]}
    \item \textbf{Scan Date:} Not provided in scan data.
    \item \textbf{Host Status:} Up
    \item \textbf{Key Finding:} The scan confirmed the host is online but discovered \textbf{no open TCP ports}. All 1000 scanned ports were in a 'closed' state.
\end{itemize}

\paragraph{Analysis:} This result is a positive security finding. It indicates that the target system has a well-configured firewall that is effectively blocking unsolicited inbound connection attempts from the internet. This significantly reduces the external attack surface of the scanned asset.

% --- Section 5: Risk Assessment ---
\section{Risk Assessment}
This section synthesizes findings from the security control review and technical scan. The following risks have been identified and prioritized based on their potential impact on the organization. No pre-existing vulnerabilities were provided for this assessment.

\begin{table}[h!]
\centering
\caption{Identified Risks and Severity}
\label{tab:risks}
\begin{tabular}{p{0.25\linewidth} p{0.5\linewidth} l}
\toprule
\textbf{Risk Name} & \textbf{Overview} & \textbf{Severity} \\
\midrule
\textbf{Lack of MFA on Sensitive Systems} & The absence of MFA for accessing systems containing sensitive data creates a high risk of data breach. A single compromised password could grant an attacker direct access to critical information. & \textbf{Critical} \\
\addlinespace
\textbf{Lack of MFA on Workstations} & User computers (endpoints) are not protected by MFA. If an employee's credentials are stolen, an attacker could gain full access to their workstation, establishing a foothold within the internal network. & \textbf{High} \\
\addlinespace
\textbf{No Acceptable Use Policy (AUP)} & Without a formal AUP, employees may be unaware of their security responsibilities. This policy gap can lead to inconsistent security practices and complicates enforcement of security standards. & \textbf{Medium} \\
\bottomrule
\end{tabular}
\end{table}

% --- Section 6: Recommendations ---
\section{Recommendations}
The following actionable recommendations are provided to address the identified risks. They are prioritized based on severity to guide remediation efforts effectively.

\begin{enumerate}
    \item \textbf{Deploy MFA for All Sensitive Systems (Critical):}
    \begin{itemize}
        \item \textbf{Action:} Immediately prioritize the implementation of a robust MFA solution for all applications, databases, and systems that store or process sensitive or critical organizational data.
        \item \textbf{Justification:} This is the most effective single control to prevent unauthorized access to the organization's most valuable assets in the event of a credential compromise.
    \end{itemize}
    \vspace{0.5cm}
    \item \textbf{Enforce MFA for All Computer Logins (High):}
    \begin{itemize}
        \item \textbf{Action:} Implement MFA for all employee and privileged user logins to company-managed workstations and servers (Windows, macOS, Linux).
        \item \textbf{Justification:} This measure protects the primary entry point to the corporate network and prevents attackers from easily moving laterally after stealing a user's password.
    \end{itemize}
    \vspace{0.5cm}
    \item \textbf{Develop and Implement an Acceptable Use Policy (Medium):}
    \begin{itemize}
        \item \textbf{Action:} Draft, approve, and disseminate a formal AUP that clearly defines the rules for using company IT assets, data handling responsibilities, and the consequences of non-compliance.
        \item \textbf{Justification:} An AUP is a foundational governance document that establishes a baseline for secure user behavior and empowers the organization to enforce its security standards.
    \end{itemize}
\end{enumerate}

\end{document}
```