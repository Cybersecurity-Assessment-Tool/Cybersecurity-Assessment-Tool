```latex
\documentclass[12pt, a4paper]{article}

% ----------------------------------------------------------------------
% PREAMBLE
% ----------------------------------------------------------------------
\usepackage[margin=1in]{geometry}
\usepackage{pifont} % Required for \ding
\usepackage{booktabs} % For professional-looking tables
\usepackage{hyperref} % For hyperlinks
\usepackage{url} % For formatting URLs
\usepackage{seqsplit} % To split long strings without breaking
\usepackage{graphicx}
\usepackage{xcolor}
\usepackage{array}

% Define custom colors for risk levels
\definecolor{critical}{RGB}{217, 83, 79}
\definecolor{high}{RGB}{240, 173, 78}
\definecolor{medium}{RGB}{91, 192, 222}
\definecolor{low}{RGB}{92, 184, 92}

% Hyperlink setup
\hypersetup{
    colorlinks=true,
    linkcolor=blue,
    filecolor=magenta,
    urlcolor=cyan,
    pdftitle={Cybersecurity Posture Assessment Report},
    pdfauthor={Cybersecurity Analysis Division},
}

% Custom commands for Yes/No symbols
\newcommand{\yes}{\ding{51}} % Checkmark
\newcommand{\no}{\ding{55}}  % Cross

% Table column type for wrapping text
\newcolumntype{L}[1]{>{\raggedright\let\newline\\\arraybackslash\hspace{0pt}}m{#1}}

% ----------------------------------------------------------------------
% DOCUMENT START
% ----------------------------------------------------------------------
\begin{document}

\title{Cybersecurity Posture Assessment Report}
\author{Cybersecurity Analysis Division}
\date{\today}
\maketitle

\begin{abstract}
This report provides a comprehensive analysis of the cybersecurity posture of \textbf{[Organization Name]}, based on network scans, a security controls questionnaire, and a review of existing risks. The assessment synthesizes technical findings with organizational policies to identify key vulnerabilities and provide actionable recommendations for remediation.
\end{abstract}

\tableofcontents
\newpage

% ----------------------------------------------------------------------
% 1. EXECUTIVE SUMMARY
% ----------------------------------------------------------------------
\section{Executive Summary}
This assessment has identified several critical and high-risk findings that expose \textbf{[Organization Name]} to significant cybersecurity threats. The analysis reveals foundational gaps in security controls, which are exacerbated by a publicly exposed network service.

\textbf{Critical Findings} include the complete absence of a security awareness training program and the lack of Multi-Factor Authentication (MFA) for accessing sensitive data systems. These deficiencies create a high probability of credential compromise and subsequent unauthorized access to the organization's most valuable assets.

\textbf{High-Risk Findings} include an exposed Secure Shell (SSH) service on the network perimeter, the absence of MFA for computer logins, and the lack of an employee acceptable use policy. This combination of technical vulnerability and policy gaps provides a clear path for attackers to breach the network and move laterally with minimal resistance.

The pre-existing risk register was empty, indicating that all findings in this report are newly identified. Immediate and decisive action is required to address these vulnerabilities and mitigate the risk of a significant security incident.

% ----------------------------------------------------------------------
% 2. ORGANIZATIONAL INFORMATION
% ----------------------------------------------------------------------
\section{Organizational Information}
The following information was used as the basis for this assessment. Due to the anonymized nature of the provided data, placeholders have been used where necessary.

\begin{table}[h!]
\centering
\begin{tabular}{@{}ll@{}}
\toprule
\textbf{Attribute} & \textbf{Value} \\ \midrule
Organization Name    & \textbf{[Organization Name]} \\
Primary Email Domain & \texttt{[Domain]} \\
External IP Address  & \texttt{[Client IP]} \\ \bottomrule
\end{tabular}
\caption{Client Organizational Details.}
\label{tab:org_info}
\end{table}

% ----------------------------------------------------------------------
% 3. SECURITY CONTROL REVIEW
% ----------------------------------------------------------------------
\section{Security Control Review}
A review of the organization's security controls was conducted via a questionnaire. The responses highlight significant gaps in identity and access management, policy, and employee training. A "No" response indicates a missing control and a potential security risk.

\begin{table}[h!]
\centering
\begin{tabular}{@{}L{12cm}c@{}}
\toprule
\textbf{Control Question} & \textbf{Response} \\ \midrule
Do you require MFA to access email? & \yes \\
Do you require MFA to log into computers? & \no \\
Do you require MFA to access sensitive data systems? & \no \\
Does your organization have an employee acceptable use policy? & \no \\
Does your organization do security awareness training for new employees? & \no \\
Does your organization do security awareness training for all employees at least once per year? & \no \\ \bottomrule
\end{tabular}
\caption{Security Controls Questionnaire Results.}
\label{tab:controls}
\end{table}

% ----------------------------------------------------------------------
% 4. TECHNICAL SCAN RESULTS
% ----------------------------------------------------------------------
\section{Technical Scan Results}
An external network scan was performed to identify exposed services.
\begin{itemize}
    \item \textbf{Target IP Address:} \texttt{[Target IP]}
    \item \textbf{Scan Date:} \textbf{[Scan Date]}
\end{itemize}

The scan identified the following open port on the network perimeter:

\begin{table}[h!]
\centering
\begin{tabular}{@{}llll@{}}
\toprule
\textbf{Port} & \textbf{State} & \textbf{Service} & \textbf{Notes} \\ \midrule
22/tcp & Open & SSH (Presumed) & Exposed to the public internet. Increases risk of \\
       &      &                & brute-force attacks and unauthorized access. \\ \bottomrule
\end{tabular}
\caption{Open Ports Detected on External Scan.}
\label{tab:scan_results}
\end{table}

\textbf{Analysis:} The presence of an open SSH port is a significant finding. While necessary for remote administration, public exposure without proper controls (e.g., IP whitelisting, key-based authentication, intrusion prevention) presents a high-value target for attackers.

% ----------------------------------------------------------------------
% 5. RISK ASSESSMENT
% ----------------------------------------------------------------------
\section{Risk Assessment}
The following table synthesizes findings from the security control review, technical scan, and pre-existing risk data. Each risk is assigned a severity level based on its potential impact and likelihood.

\begin{table}[h!]
\centering
\begin{tabular}{@{}lL{7cm}l@{}}
\toprule
\textbf{ID} & \textbf{Risk Description} & \textbf{Severity} \\ \midrule
RISK-001 & \textbf{Lack of MFA on Sensitive Systems:} No MFA requirement for & \colorbox{critical}{\color{white}\textbf{CRITICAL}} \\
         & sensitive data access allows a single compromised password to lead to a major data breach. & \\
\addlinespace
RISK-002 & \textbf{No Security Awareness Training:} Employees are not trained to & \colorbox{critical}{\color{white}\textbf{CRITICAL}} \\
         & recognize or report phishing and other social engineering attacks, making them a primary target. & \\
\addlinespace
RISK-003 & \textbf{Exposed SSH Service:} Port 22 is open to the internet, inviting & \colorbox{high}{\color{white}\textbf{HIGH}} \\
         & automated brute-force attacks against user credentials to gain a network foothold. & \\
\addlinespace
RISK-004 & \textbf{Lack of MFA on Computers:} A compromised password allows an & \colorbox{high}{\color{white}\textbf{HIGH}} \\
         & attacker to log into an employee's computer, gaining access to local data and network resources. & \\
\addlinespace
RISK-005 & \textbf{No Acceptable Use Policy (AUP):} The absence of a formal & \colorbox{high}{\color{white}\textbf{HIGH}} \\
         & AUP creates ambiguity regarding secure practices and limits the organization's ability to enforce them. & \\
\bottomrule
\end{tabular}
\caption{Summary of Identified Risks.}
\label{tab:risk_summary}
\end{table}

% ----------------------------------------------------------------------
% 6. RECOMMENDATIONS
% ----------------------------------------------------------------------
\section{Recommendations}
The following actions are recommended to mitigate the identified risks. Recommendations are prioritized based on risk severity.

\begin{table}[h!]
\centering
\begin{tabular}{@{}lL{10cm}@{}}
\toprule
\textbf{Risk ID} & \textbf{Recommended Action} \\ \midrule
\textbf{RISK-001} & \textbf{Implement MFA for Sensitive Systems:} Immediately deploy and enforce a strong MFA solution (e.g., authenticator app, hardware token) for all access to systems containing sensitive or critical data. \\
\addlinespace
\textbf{RISK-002} & \textbf{Establish a Security Training Program:} Procure and implement a mandatory security awareness training program for all employees. This must be completed by new hires upon onboarding and by all staff at least annually. \\
\addlinespace
\textbf{RISK-003} & \textbf{Secure the SSH Service:}
\begin{itemize}
    \item If remote SSH access is not required, block port 22 at the firewall.
    \item If required, restrict access to a whitelist of trusted IP addresses.
    \item Enforce strong, key-based authentication and disable password-based logins.
\end{itemize} \\
\addlinespace
\textbf{RISK-004} & \textbf{Enforce MFA for Endpoint Login:} Deploy MFA for all workstation and laptop logins to protect against credential theft and unauthorized local access. \\
\addlinespace
\textbf{RISK-005} & \textbf{Develop an Acceptable Use Policy:} Draft, approve, and communicate a formal AUP that clearly defines the rules for using company IT assets, data, and network resources. \\
\bottomrule
\end{tabular}
\caption{Remediation Actions.}
\label{tab:recommendations}
\end{table}

\end{document}
```