```latex
\documentclass[12pt]{article}

% --- PACKAGE IMPORTS ---
\usepackage[margin=1in]{geometry}
\usepackage{pifont} % For checkmarks and crosses
\usepackage{booktabs} % For professional tables
\usepackage{hyperref} % For hyperlinks
\usepackage{url} % For URL formatting
\usepackage{seqsplit} % For splitting long strings
\usepackage{graphicx}
\usepackage{xcolor}

% --- DOCUMENT METADATA ---
\title{Cybersecurity Posture Assessment Report}
\author{Cybersecurity Analyst}
\date{November 22, 2025}

% --- HYPERREF SETUP ---
\hypersetup{
    colorlinks=true,
    linkcolor=blue,
    filecolor=magenta,      
    urlcolor=cyan,
    pdftitle={Cybersecurity Posture Assessment Report},
    pdfpagemode=FullScreen,
}

% --- BEGIN DOCUMENT ---
\begin{document}

\maketitle
\thispagestyle{empty}
\newpage

\tableofcontents
\newpage

% ==============================================================================
% 1. EXECUTIVE SUMMARY
% ==============================================================================
\section{Executive Summary}

This report provides a comprehensive assessment of the cybersecurity posture for \textbf{[Organization Name]}, based on an analysis of organizational security controls, a technical network scan, and a review of existing risks. The assessment was conducted on November 22, 2025.

The analysis reveals a mixed security posture. The organization has successfully implemented strong Multi-Factor Authentication (MFA) controls across key access points, which is a commendable foundational strength. However, this is offset by critical deficiencies in administrative and human-centric security controls. Specifically, the lack of a formal employee Acceptable Use Policy (AUP) and the absence of any security awareness training program present high-severity risks. These gaps significantly increase the organization's susceptibility to social engineering, insider threats, and policy violations.

Furthermore, the external network scan identified a critically outdated Nginx web server (\texttt{1.18.0}) exposed to the internet. This version is known to have multiple publicly disclosed vulnerabilities, making it a prime target for automated attacks and posing an immediate and severe threat to the organization's data and infrastructure.

Immediate remediation is required to address the outdated web server. Concurrently, efforts should be made to develop and implement the missing security policies and training programs to build a more resilient, defense-in-depth security posture.

% ==============================================================================
% 2. ORGANIZATIONAL INFORMATION
% ==============================================================================
\section{Organizational and Scan Information}

The following table summarizes the key information used for this assessment. The data has been sourced from provided organizational records and technical scan metadata.

\begin{table}[h!]
\centering
\begin{tabular}{@{}ll@{}}
\toprule
\textbf{Item} & \textbf{Detail} \\
\midrule
Organization Name & \textbf{[Organization Name]} \\
Email Domain & \texttt{[Domain]} \\
External IP Scanned & \texttt{[Client IP]} \\
Target IP Address & \texttt{[Target IP]} \\
Scan Date & 2025-11-22T10:00:00Z \\
\bottomrule
\end{tabular}
\caption{Assessment Subject Information}
\label{tab:org_info}
\end{table}

% ==============================================================================
% 3. SECURITY CONTROL REVIEW
% ==============================================================================
\section{Security Control Review}

A review of the organization's security controls was conducted via a questionnaire. The responses indicate strong controls in user authentication but critical gaps in policy and employee education. Answers of "No" represent significant weaknesses that require immediate attention.

\begin{table}[h!]
\centering
\begin{tabular}{@{}p{0.75\textwidth}c@{}}
\toprule
\textbf{Control Question} & \textbf{Response} \\
\midrule
Do you require MFA to access email? & \textcolor{green}{\ding{51}} \\
Do you require MFA to log into computers? & \textcolor{green}{\ding{51}} \\
Do you require MFA to access sensitive data systems? & \textcolor{green}{\ding{51}} \\
\midrule
Does your organization have an employee acceptable use policy? & \textcolor{red}{\ding{55}} \\
Does your organization do security awareness training for new employees? & \textcolor{red}{\ding{55}} \\
Does your organization do security awareness training for all employees at least once per year? & \textcolor{red}{\ding{55}} \\
\bottomrule
\end{tabular}
\caption{Organizational Security Control Questionnaire}
\label{tab:controls}
\end{table}

% ==============================================================================
% 4. TECHNICAL SCAN RESULTS
% ==============================================================================
\section{Technical Scan Results}

An external network scan was performed against the target IP address \texttt{[Target IP]}. The scan identified one open port, which is hosting a public-facing web service.

\subsection{Open Ports and Services}
The following service was identified as accessible from the public internet.

\begin{table}[h!]
\centering
\begin{tabular}{@{}lllll@{}}
\toprule
\textbf{Port} & \textbf{State} & \textbf{Service} & \textbf{Product} & \textbf{Version} \\
\midrule
443/tcp & open & https & nginx & 1.18.0 \\
\bottomrule
\end{tabular}
\caption{Identified Open Ports and Services}
\label{tab:nmap_results}
\end{table}

\subsection{Analysis of Findings}
The scan revealed an Nginx web server, version \texttt{1.18.0}, listening on port 443 (HTTPS). This version was released in April 2020 and is now considered severely outdated. The Nginx mainline version is significantly newer, and version 1.18.0 is no longer supported with security patches.

Running this outdated software exposes the organization to numerous publicly known vulnerabilities (CVEs), which can be exploited by attackers to achieve remote code execution, denial of service, or information disclosure. This finding represents a \textbf{critical risk}.

% ==============================================================================
% 5. RISK ASSESSMENT SUMMARY
% ==============================================================================
\section{Risk Assessment Summary}

This section synthesizes the findings from the security control review and the technical scan. As no pre-existing risks were provided, the following table details the newly identified vulnerabilities.

\begin{table}[h!]
\centering
\begin{tabular}{@{}p{0.1\textwidth}p{0.3\textwidth}p{0.15\textwidth}p{0.35\textwidth}@{}}
\toprule
\textbf{Risk ID} & \textbf{Finding} & \textbf{Severity} & \textbf{Description} \\
\midrule
R-01 & Outdated Public-Facing Web Server & \textbf{Critical} & The external web server at \texttt{[Target IP]} is running Nginx 1.18.0, an unsupported version with multiple known critical vulnerabilities. \\
\addlinespace
R-02 & Lack of Security Awareness Training & \textbf{High} & The organization does not conduct security awareness training for new or existing employees, increasing susceptibility to phishing, malware, and social engineering attacks. \\
\addlinespace
R-03 & Missing Acceptable Use Policy (AUP) & \textbf{High} & The absence of a formal AUP creates ambiguity regarding the secure use of company assets and leaves the organization without a formal mechanism to enforce security standards. \\
\bottomrule
\end{tabular}
\caption{Summary of Identified Risks}
\label{tab:risk_summary}
\end{table}

% ==============================================================================
% 6. RECOMMENDATIONS
% ==============================================================================
\section{Recommendations}

Based on the identified risks, the following prioritized actions are recommended to improve the organization's cybersecurity posture.

\begin{itemize}
    \item \textbf{[Critical] R-01: Remediate Outdated Nginx Server}
    \begin{itemize}
        \item \textbf{Immediate Action:} Upgrade the Nginx server on host \texttt{[Target IP]} to the latest stable version immediately to mitigate known vulnerabilities.
        \item \textbf{Strategic Action:} Implement a formal patch management and vulnerability management program. Regularly scan external and internal assets for outdated software and apply security patches in a timely manner.
    \end{itemize}
    \vspace{0.5cm}
    \item \textbf{[High] R-02: Implement Security Awareness Training}
    \begin{itemize}
        \item \textbf{Immediate Action:} Enroll all current employees in a foundational security awareness training course.
        \item \textbf{Strategic Action:} Develop and implement a continuous security awareness training program. This program must be mandatory for all new hires during onboarding and required for all staff annually. The training should cover topics such as phishing, password security, and data handling.
    \end{itemize}
    \vspace{0.5cm}
    \item \textbf{[High] R-03: Establish an Acceptable Use Policy}
    \begin{itemize}
        \item \textbf{Immediate Action:} Draft and approve a formal Acceptable Use Policy (AUP) that clearly defines the rules and responsibilities for all users of the organization's IT resources.
        \item \textbf{Strategic Action:} Communicate the policy to all employees and require them to formally acknowledge it. Integrate the AUP into the new employee onboarding process.
    \end{itemize}
\end{itemize}

\end{document}
```