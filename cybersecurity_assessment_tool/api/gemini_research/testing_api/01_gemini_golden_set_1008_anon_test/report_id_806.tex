```latex
\documentclass[12pt]{article}

% Preamble: Required Packages
\usepackage[margin=1in]{geometry}
\usepackage{pifont} % For \ding symbols (checkmarks and crosses)
\usepackage{booktabs} % For professional-looking tables
\usepackage{hyperref} % For creating hyperlinks
\usepackage{url}      % For proper URL formatting
\usepackage{seqsplit} % To split long strings without breaking
\usepackage{graphicx} % For potential logos or diagrams
\usepackage{xcolor}   % For custom colors

% Hyperlink Setup
\hypersetup{
    colorlinks=true,
    linkcolor=blue,
    filecolor=magenta,
    urlcolor=cyan,
    pdftitle={Cybersecurity Posture Assessment Report},
    pdfpagemode=FullScreen,
}

% Define custom colors for severity
\definecolor{criticalred}{HTML}{D7263D}
\definecolor{highorange}{HTML}{F49D40}
\definecolor{mediumyellow}{HTML}{F4D440}

% Document Start
\begin{document}

% --- Title Page ---
\begin{titlepage}
    \centering
    \vspace*{1cm}
    \Huge\textbf{Cybersecurity Posture Assessment Report}
    \vspace{1.5cm}
    \Large
    \textbf{Prepared for:}\\
    \vspace{0.5cm}
    \textbf{[Organization Name]}
    \vspace{2cm}
    \large
    \textbf{Date of Report:}\\
    \today
    \vfill
    \large
    \textbf{Generated By:}\\
    Cybersecurity Analysis Division
\end{titlepage}

\tableofcontents
\newpage

% --- 1. Executive Summary ---
\section{Executive Summary}
This report provides a comprehensive assessment of the cybersecurity posture for \textbf{[Organization Name]}, based on an analysis of network scan data, organizational security controls, and pre-existing risk registers. The assessment was conducted on \today.

The overall security posture is determined to be at a \textbf{high-risk level}. This is due to the discovery of several critical vulnerabilities and significant gaps in administrative security controls.

Key findings include:
\begin{itemize}
    \item \textbf{Critical External Vulnerability:} An externally facing FTP server was identified running a dangerously outdated and vulnerable version of \texttt{vsftpd 2.3.4}. This specific version contains a known backdoor (CVE-2011-2523), which could allow an attacker to gain complete control of the system. This risk is compounded by the server's configuration, which permits anonymous, unauthenticated access.
    \item \textbf{Critical Access Control Gaps:} Multi-Factor Authentication (MFA) is not enforced for accessing sensitive data systems, leaving critical assets vulnerable to compromise via stolen credentials.
    \item \textbf{Significant Policy Deficiencies:} The organization lacks a formal Acceptable Use Policy (AUP) and does not provide recurring annual security awareness training for all employees. These gaps increase the likelihood of human error leading to a security incident.
    \item \textbf{Pre-existing Endpoint Risk:} An existing issue regarding outdated Windows 7 workstations remains, which exposes the internal network to numerous known and unpatched vulnerabilities.
\end{itemize}

Immediate remediation of the external FTP server is strongly recommended, followed by the implementation of robust access controls and the development of foundational security policies and training programs.

% --- 2. Organizational Information ---
\section{Organizational Information}
This section details the organizational data used for this assessment. As the provided data was anonymized, placeholders are used where necessary.

\begin{tabular}{@{}ll}
\toprule
\textbf{Attribute} & \textbf{Value} \\
\midrule
Organization Name & \textbf{[Organization Name]} \\
Primary Email Domain & \texttt{[Domain]} \\
External IP Scanned & \texttt{[Client IP]} \\
\bottomrule
\end{tabular}

% --- 3. Security Control Review ---
\section{Security Control Review}
The following table summarizes the organization's responses to a security controls questionnaire. A red 'X' (\textcolor{red}{\ding{55}}) indicates a potential gap in security posture that requires attention.

\begin{table}[h!]
\centering
\caption{Security Controls Questionnaire Analysis}
\begin{tabular}{@{}p{0.75\linewidth}c@{}}
\toprule
\textbf{Control Question} & \textbf{Response} \\
\midrule
Do you require MFA to access email? & \textcolor{green}{\ding{51}} \\
Do you require MFA to log into computers? & \textcolor{green}{\ding{51}} \\
Do you require MFA to access sensitive data systems? & \textcolor{red}{\ding{55}} \\
Does your organization have an employee acceptable use policy? & \textcolor{red}{\ding{55}} \\
Does your organization do security awareness training for new employees? & \textcolor{green}{\ding{51}} \\
Does your organization do security awareness training for all employees at least once per year? & \textcolor{red}{\ding{55}} \\
\bottomrule
\end{tabular}
\end{table}

The identified gaps in MFA for sensitive systems, the lack of an acceptable use policy, and the absence of recurring security training are significant findings that directly contribute to the organization's risk profile.

% --- 4. Technical Scan Results ---
\section{Technical Scan Results}
An external network scan was performed against the target IP address to identify open ports and exposed services.

\begin{itemize}
    \item \textbf{Target IP Address:} \texttt{[Target IP]}
    \item \textbf{Scan Date:} Not specified in scan data.
\end{itemize}

\subsection{Open Ports and Services}
The following table details the services discovered to be accessible from the public internet.

\begin{table}[h!]
\centering
\caption{Discovered Open Ports}
\begin{tabular}{@{}lllll@{}}
\toprule
\textbf{Port} & \textbf{State} & \textbf{Service} & \textbf{Version} & \textbf{Notes} \\
\midrule
21/tcp & Open & FTP & vsftpd 2.3.4 & \textbf{Anonymous FTP login allowed.} \\
\bottomrule
\end{tabular}
\end{table}

\subsection{Analysis of Technical Findings}
The presence of an open FTP port with \textbf{\texttt{vsftpd version 2.3.4}} is a \textbf{critical security risk}. This specific version was compromised, and a backdoor was inserted into the source code, which was active from June 30th, 2011 to July 3rd, 2011. If this version was downloaded and compiled during that timeframe, it contains a command execution backdoor (CVE-2011-2523).

Furthermore, the configuration allowing \textbf{anonymous FTP login} is highly insecure. It permits any user on the internet to connect to the server and potentially access, upload, or download files without any authentication, which can lead to data breaches or the hosting of malicious content.

% --- 5. Consolidated Risk Assessment ---
\section{Consolidated Risk Assessment}
This section synthesizes findings from the security control review, technical scan, and pre-existing risk data into a consolidated list of identified risks.

\begin{table}[h!]
\centering
\caption{Summary of Identified Risks}
\begin{tabular}{@{}p{0.25\linewidth}p{0.5\linewidth}p{0.15\linewidth}@{}}
\toprule
\textbf{Risk Name} & \textbf{Overview} & \textbf{Severity} \\
\midrule
\textbf{Vulnerable FTP Service (CVE-2011-2523)} & An external FTP server runs \texttt{vsftpd 2.3.4}, which contains a known remote command execution backdoor. & \textcolor{criticalred}{\textbf{Critical}} \\
\addlinespace
\textbf{Anonymous FTP Access} & The FTP server is configured to allow unauthenticated access, enabling potential data exfiltration or malicious file uploads. & \textcolor{criticalred}{\textbf{Critical}} \\
\addlinespace
\textbf{No MFA on Sensitive Systems} & Lack of MFA on systems holding sensitive data exposes them to credential theft and unauthorized access. & \textcolor{criticalred}{\textbf{Critical}} \\
\addlinespace
\textbf{Missing Acceptable Use Policy} & The absence of a formal AUP means there are no clear guidelines for employees on the proper use of company assets, increasing insider risk. & \textcolor{highorange}{\textbf{High}} \\
\addlinespace
\textbf{Inadequate Security Training} & Without recurring annual training, employees' awareness of evolving threats diminishes, making them more susceptible to phishing and social engineering. & \textcolor{highorange}{\textbf{High}} \\
\addlinespace
\textbf{Outdated Windows Policy} & Workstations are running Windows 7, an end-of-life OS that no longer receives security updates, exposing them to numerous vulnerabilities. & \textcolor{mediumyellow}{\textbf{Medium}} \\
\bottomrule
\end{tabular}
\end{table}

% --- 6. Recommendations ---
\section{Recommendations}
Based on the risk assessment, the following actions are recommended to mitigate the identified vulnerabilities and improve the overall security posture. Recommendations are prioritized based on severity.

\begin{enumerate}
    \item \textbf{[Critical] Remediate Vulnerable FTP Server:} Immediately take the FTP server offline. If the service is business-critical, it must be updated to the latest stable version of \texttt{vsftpd} or replaced with a secure file transfer alternative like SFTP (SSH File Transfer Protocol).
    
    \item \textbf{[Critical] Disable Anonymous FTP Access:} If the FTP server is reinstated, its configuration must be modified to disable anonymous logins. All access should require a unique username and a strong password.

    \item \textbf{[Critical] Implement MFA on Sensitive Systems:} Prioritize the deployment of a robust MFA solution across all systems identified as containing sensitive data. This should be the highest access control priority.

    \item \textbf{[High] Develop and Enforce an Acceptable Use Policy (AUP):} Create a formal AUP that clearly defines the rules and responsibilities for all employees when using company technology and data. Ensure all employees read and acknowledge the policy.

    \item \textbf{[High] Establish Annual Security Awareness Training:} Implement a mandatory, recurring security awareness training program for all employees. The training should be conducted at least annually and cover topics such as phishing, password security, and social engineering.

    \item \textbf{[Medium] Plan and Execute OS Upgrades:} Accelerate the plan to upgrade all workstations from Windows 7 to a modern, supported operating system like Windows 10 or 11 to ensure they receive critical security patches.

\end{enumerate}

\end{document}
```