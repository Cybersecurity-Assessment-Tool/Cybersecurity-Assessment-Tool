```latex
\documentclass[12pt, a4paper]{article}

% Preamble: Required Packages
\usepackage[margin=1in]{geometry}
\usepackage{pifont} % For checkmarks and crosses
\usepackage{booktabs} % For professional tables
\usepackage{hyperref} % For clickable links and better PDF navigation
\usepackage{url} % For formatting URLs
\usepackage{seqsplit} % For splitting long strings without breaking
\usepackage{graphicx} % For logo
\usepackage{xcolor} % For colors

% Hyperref Setup
\hypersetup{
    colorlinks=true,
    linkcolor=blue,
    filecolor=magenta,      
    urlcolor=cyan,
    pdftitle={Cybersecurity Assessment Report},
    pdfpagemode=FullScreen,
}

% Define colors for severity
\definecolor{critical}{HTML}{990000}
\definecolor{high}{HTML}{D14302}
\definecolor{medium}{HTML}{E5A50A}
\definecolor{low}{HTML}{3E8E41}

% Document Information
\title{Cybersecurity Assessment Report}
\author{Cybersecurity Analysis Division}
\date{\today}

\begin{document}

\maketitle
\thispagestyle{empty}
\newpage

\tableofcontents
\newpage

% --- 1. Executive Overview ---
\section{Executive Overview}

This report details the findings of a cybersecurity assessment conducted for \textbf{[Organization Name]}. The evaluation combined a review of organizational security controls via a questionnaire, an external network vulnerability scan, and an analysis of pre-existing risks.

The primary objective was to identify security gaps, assess the current risk posture, and provide actionable recommendations to enhance the organization's defenses against cyber threats.

\paragraph{Key Findings:} The assessment revealed several critical and high-risk gaps in the organization's security controls. Most notably:
\begin{itemize}
    \item \textbf{Critical Risk:} Multi-Factor Authentication (MFA) is not enforced for email access. This represents a significant vulnerability to account takeover and business email compromise (BEC) attacks.
    \item \textbf{High Risk:} MFA is not required for computer logins, weakening endpoint security and increasing the risk of unauthorized access if credentials are stolen.
    \item \textbf{High Risk:} The organization lacks a formal Employee Acceptable Use Policy (AUP), creating ambiguity regarding security responsibilities and increasing insider threat risks.
\end{itemize}

\paragraph{Positive Controls:} The organization has implemented positive security controls, including MFA for sensitive data systems and a security awareness training program for all employees. The external network scan of the designated target IP address did not reveal any open ports, suggesting a strong perimeter firewall configuration.

\paragraph{Conclusion:} While some foundational security measures are in place, the identified gaps in authentication and policy require immediate attention. The recommendations provided in this report are prioritized to address the most severe risks first, strengthening the overall security posture of \textbf{[Organization Name]}.

% --- 2. Organizational Information ---
\section{Organizational Information}

This section contains the high-level information used as the basis for this assessment. Due to the anonymized nature of the data provided, placeholders have been used where necessary.

\begin{table}[h!]
\centering
\begin{tabular}{@{}ll@{}}
\toprule
\textbf{Attribute} & \textbf{Value} \\ \midrule
Organization Name & \textbf{[Organization Name]} \\
Primary Email Domain & \texttt{[Domain]} \\
Assessed External IP & \texttt{[Client IP]} \\
Assessment Date & \today \\ \bottomrule
\end{tabular}
\caption{Client and Assessment Details.}
\end{table}

% --- 3. Security Control Review ---
\section{Security Control Review}

The following table summarizes the organization's responses to a security controls questionnaire. A green checkmark (\textcolor{green}{\ding{51}}) indicates a positive control, while a red cross (\textcolor{red}{\ding{55}}) indicates a potential security gap that requires attention.

\begin{table}[h!]
\centering
\begin{tabular}{@{}p{0.7\linewidth}c@{}}
\toprule
\textbf{Control Question} & \textbf{Response} \\ \midrule
Do you require MFA to access email? & \textcolor{red}{\ding{55}} \\
Do you require MFA to log into computers? & \textcolor{red}{\ding{55}} \\
Do you require MFA to access sensitive data systems? & \textcolor{green}{\ding{51}} \\
Does your organization have an employee acceptable use policy? & \textcolor{red}{\ding{55}} \\
Does your organization do security awareness training for new employees? & \textcolor{green}{\ding{51}} \\
Does your organization do security awareness training for all employees at least once per year? & \textcolor{green}{\ding{51}} \\ \bottomrule
\end{tabular}
\caption{Security Controls Questionnaire Results.}
\end{table}

% --- 4. Technical Scan Results ---
\section{Technical Scan Results}

An external network scan was performed to identify open ports and exposed services on the public-facing infrastructure.

\begin{itemize}
    \item \textbf{Target IP Address:} \texttt{[Target IP]}
    \item \textbf{Scan Date:} [Scan Date]
\end{itemize}

\paragraph{Findings:} The network scan did not identify any open TCP ports on the target host. This is a positive security finding, suggesting that a well-configured perimeter firewall is in place, which effectively limits the external attack surface. No vulnerabilities associated with exposed services could be identified as a result.

% --- 5. Risk Assessment & Findings ---
\section{Risk Assessment \& Findings}

This section synthesizes the information from the security control review, technical scan, and pre-existing risk data. The following table details the identified risks, their severity, and a brief description. The risks are derived primarily from the gaps identified in the security questionnaire. No pre-existing vulnerabilities were provided for this assessment.

\begin{table}[h!]
\centering
\begin{tabular}{@{}p{0.1\linewidth}p{0.25\linewidth}p{0.15\linewidth}p{0.4\linewidth}@{}}
\toprule
\textbf{ID} & \textbf{Finding} & \textbf{Severity} & \textbf{Description} \\ \midrule
RISK-001 & No MFA on Email & \textcolor{critical}{\textbf{Critical}} & The absence of MFA on email accounts makes them highly susceptible to phishing, credential stuffing, and account takeover attacks, which can lead to data breaches and financial fraud. \\
\addlinespace
RISK-002 & No MFA on Endpoints & \textcolor{high}{\textbf{High}} & Without MFA for computer logins, a single compromised password can grant an attacker full access to an employee's workstation and potentially the internal network. \\
\addlinespace
RISK-003 & No Acceptable Use Policy (AUP) & \textcolor{high}{\textbf{High}} & The lack of a formal AUP creates ambiguity for employees regarding the secure use of company assets. This increases the risk of insider threats, data mishandling, and non-compliance with regulations. \\ \bottomrule
\end{tabular}
\caption{Summary of Identified Risks.}
\end{table}

% --- 6. Recommendations ---
\section{Recommendations}

Based on the findings in the previous section, the following actionable recommendations are provided to mitigate the identified risks and improve the overall security posture of \textbf{[Organization Name]}.

\subsection{RISK-001: Implement MFA for Email (Critical)}
\begin{itemize}
    \item \textbf{Action:} Immediately enable and enforce MFA for all user accounts with access to the organization's email system (e.g., Microsoft 365, Google Workspace).
    \item \textbf{Justification:} This is the single most effective control to prevent unauthorized access to email accounts, which are a primary target for attackers.
\end{itemize}

\subsection{RISK-002: Implement MFA for Endpoints (High)}
\begin{itemize}
    \item \textbf{Action:} Deploy an MFA solution for all employee computer logins (desktops and laptops). Solutions like Windows Hello for Business, Duo Security, or other third-party identity providers can be used.
    \item \textbf{Justification:} This adds a critical layer of security to protect endpoints, ensuring that stolen credentials alone are not enough to compromise a user's machine and gain a foothold on the network.
\end{itemize}

\subsection{RISK-003: Develop an Acceptable Use Policy (High)}
\begin{itemize}
    \item \textbf{Action:} Develop, approve, and implement a formal Acceptable Use Policy (AUP). This policy should clearly define the rules and responsibilities for all employees when using company-owned IT assets, data, and internet services.
    \item \textbf{Justification:} An AUP establishes a baseline for secure behavior, reduces organizational risk, and provides a framework for enforcing security standards. All employees should be required to read and acknowledge the policy.
\end{itemize}

\end{document}
```