```latex
\documentclass[12pt]{article}

% Preamble: Required Packages
\usepackage[margin=1in]{geometry}
\usepackage{pifont} % For checkmarks and crosses
\usepackage{booktabs} % For professional-looking tables
\usepackage{hyperref} % For hyperlinks and metadata
\usepackage{url}      % For formatting URLs
\usepackage{seqsplit} % For splitting long strings in tt font

% Document Metadata
\hypersetup{
    colorlinks=true,
    linkcolor=black,
    urlcolor=blue,
    pdftitle={Cybersecurity Posture Report},
    pdfauthor={Cybersecurity Analyst},
    pdfsubject={Security Assessment},
    pdfkeywords={Security, Risk, Assessment, Nmap, Policy}
}

% Checkmark and Cross definitions
\newcommand{\cmark}{\ding{51}}%
\newcommand{\xmark}{\ding{55}}%

\begin{document}

% --- Title Page ---
\begin{titlepage}
    \centering
    \vspace*{\fill}
    \huge\textbf{Cybersecurity Posture Report}
    \vspace{1.5cm}
    \Large For: \textbf{[Organization Name]}
    \vspace{2cm}
    \normalsize
    \begin{tabular}{ll}
        \textbf{Date of Report:} & November 22, 2025 \\
        \textbf{Author:} & Cybersecurity Analyst \\
        \textbf{Classification:} & Confidential \\
    \end{tabular}
    \vspace*{\fill}
\end{titlepage}

\tableofcontents
\newpage

% --- 1. Executive Summary ---
\section{Executive Summary}
This report details the findings of a cybersecurity assessment conducted on November 22, 2025. The assessment combined a review of organizational security controls, an external network scan, and an analysis of pre-existing risks.

The overall security posture of \textbf{[Organization Name]} is critically deficient and requires immediate remediation. Several foundational security controls are absent, creating significant exposure to common cyber threats such as phishing, ransomware, and unauthorized access.

Key findings include:
\begin{itemize}
    \item \textbf{Critical Gaps in Access Control:} Multi-Factor Authentication (MFA) is not enforced for email or computer logins, leaving the organization's primary communication and endpoint assets highly vulnerable to compromise.
    \item \textbf{Lack of Security Governance:} The organization lacks a formal Acceptable Use Policy and does not conduct any security awareness training for its employees. This indicates a poor security culture and a high risk of human error leading to a security incident.
    \item \textbf{Vulnerable External Services:} The external-facing web server is running an outdated and vulnerable version of nginx (1.18.0), which is susceptible to publicly known exploits.
\end{itemize}

Immediate and decisive action is required to address these high-risk findings. Recommendations are provided in Section \ref{sec:recommendations} to guide remediation efforts.

% --- 2. Organizational Information ---
\section{Organizational Information}
This section provides a summary of the organizational details used as the basis for this assessment. Due to the anonymized nature of the provided data, placeholders have been used where necessary.

\begin{tabular}{@{}ll}
    \toprule
    \textbf{Attribute} & \textbf{Value} \\
    \midrule
    Organization Name & \textbf{[Organization Name]} \\
    Primary Email Domain & \texttt{[Domain]} \\
    External IP Address & \texttt{[Client IP]} \\
    \bottomrule
\end{tabular}

% --- 3. Security Control Review ---
\section{Security Control Review}
A review of administrative and policy-based security controls was conducted via a questionnaire. The responses reveal critical gaps in fundamental security practices. A summary of the findings is presented in Table \ref{tab:controls}.

\begin{table}[h!]
    \centering
    \caption{Organizational Security Control Questionnaire}
    \label{tab:controls}
    \begin{tabular}{@{}p{0.6\linewidth} c l@{}}
        \toprule
        \textbf{Control Question} & \textbf{Response} & \textbf{Assessment} \\
        \midrule
        Do you require MFA to access email? & \xmark & \textbf{Critical Gap} \\
        Do you require MFA to log into computers? & \xmark & \textbf{Critical Gap} \\
        Do you require MFA to access sensitive data systems? & \cmark & Good Practice \\
        Does your organization have an employee acceptable use policy? & \xmark & High Risk / Policy Gap \\
        Does your organization do security awareness training for new employees? & \xmark & High Risk / Training Gap \\
        Does your organization do security awareness training for all employees at least once per year? & \xmark & High Risk / Training Gap \\
        \bottomrule
    \end{tabular}
\end{table}

The absence of MFA on email and workstations, coupled with a complete lack of security policies and training, exposes the organization to a very high likelihood of a successful phishing attack or credential compromise event.

% --- 4. Technical Scan Results ---
\section{Technical Scan Results}
An external network vulnerability scan was performed against the organization's public-facing infrastructure.

\begin{itemize}
    \item \textbf{Target IP Address:} \texttt{[Target IP]}
    \item \textbf{Scan Date:} 2025-11-22
\end{itemize}

The scan identified one open port, as detailed in Table \ref{tab:nmap}.

\begin{table}[h!]
    \centering
    \caption{Open Port Analysis}
    \label{tab:nmap}
    \begin{tabular}{@{}llll@{}}
        \toprule
        \textbf{Port} & \textbf{State} & \textbf{Service} & \textbf{Product \& Version} \\
        \midrule
        443/tcp & open & https & nginx 1.18.0 \\
        \bottomrule
    \end{tabular}
\end{table}

\subsection{Analysis of Findings}
The primary technical finding is the use of \textbf{nginx version 1.18.0}. This version was released in April 2020 and is now significantly outdated. It is known to be vulnerable to several security issues, including but not limited to CVE-2021-23017, a DNS resolver vulnerability which can lead to denial-of-service or information disclosure. Running outdated software on internet-facing systems presents a tangible risk of exploitation by automated and targeted attacks.

% --- 5. Integrated Risk Assessment ---
\section{Integrated Risk Assessment}
This section correlates the findings from the security control review and the technical scan. No pre-existing risks were provided for this assessment. The following new risks have been identified and prioritized.

\begin{table}[h!]
    \centering
    \caption{Summary of Identified Risks}
    \label{tab:risks}
    \begin{tabular}{@{}lp{0.3\linewidth}p{0.4\linewidth}l@{}}
        \toprule
        \textbf{ID} & \textbf{Risk Name} & \textbf{Description} & \textbf{Severity} \\
        \midrule
        RISK-001 & Lack of Comprehensive MFA & No MFA on email or endpoints allows for simple credential compromise to grant attackers broad access. & \textbf{Critical} \\
        \addlinespace
        RISK-002 & Deficient Security Policies \& Training & The absence of an AUP and security training programs results in a high likelihood of security incidents caused by human error. & \textbf{High} \\
        \addlinespace
        RISK-003 & Outdated Web Server Software & The public-facing web server runs an old version of nginx with known vulnerabilities, exposing it to external attack. & \textbf{Medium} \\
        \bottomrule
    \end{tabular}
\end{table}

% --- 6. Recommendations ---
\section{Recommendations}
\label{sec:recommendations}
Based on the integrated risk assessment, the following actions are recommended to improve the organization's security posture. These are prioritized based on severity and impact.

\begin{enumerate}
    \item \textbf{[Critical] Implement Comprehensive MFA (RISK-001):}
    \begin{itemize}
        \item Immediately enforce MFA for all users on the primary email system (\texttt{[Domain]}).
        \item Deploy an MFA solution for all workstation and server logins, for both remote and on-premise access.
    \end{itemize}

    \item \textbf{[High] Establish Security Governance Program (RISK-002):}
    \begin{itemize}
        \item Develop and implement a formal Acceptable Use Policy (AUP) that all employees must read and acknowledge.
        \item Institute a mandatory security awareness training program for all new and existing employees. Training should be conducted upon hiring and at least annually thereafter, covering topics like phishing, password security, and data handling.
    \end{itemize}

    \item \textbf{[Medium] Remediate Vulnerable Software (RISK-003):}
    \begin{itemize}
        \item Create a patch management plan for the external web server at \texttt{[Target IP]}.
        \item Upgrade the nginx 1.18.0 instance to the latest stable version recommended by the vendor to mitigate known vulnerabilities.
        \item Implement a regular vulnerability scanning schedule to proactively identify and address future software vulnerabilities.
    \end{itemize}
\end{enumerate}

% --- 7. Conclusion ---
\section{Conclusion}
The assessment reveals a security posture with fundamental weaknesses that place \textbf{[Organization Name]} at a high risk of a significant security breach. The lack of basic controls like Multi-Factor Authentication and security training must be addressed as a top priority. By implementing the recommendations outlined in this report, the organization can substantially reduce its risk exposure and build a more resilient security foundation.

\end{document}
```