```latex
\documentclass[12pt]{article}

% --- PACKAGES ---
\usepackage[margin=1in]{geometry}
\usepackage{pifont} % For checkmarks and crosses
\usepackage{booktabs} % For professional tables
\usepackage{hyperref} % For hyperlinks
\usepackage{url} % For URL formatting
\usepackage{seqsplit} % For splitting long strings in texttt
\usepackage[T1]{fontenc}

% --- DOCUMENT METADATA ---
\title{Cybersecurity Posture Assessment Report}
\author{Cybersecurity Analyst}
\date{\today}

% --- HYPERREF SETUP ---
\hypersetup{
    colorlinks=true,
    linkcolor=black,
    urlcolor=blue,
    pdftitle={Cybersecurity Posture Assessment Report},
    pdfauthor={Cybersecurity Analyst},
    pdfsubject={Security Assessment},
    pdfkeywords={Security, Assessment, Report}
}

% --- BEGIN DOCUMENT ---
\begin{document}

\maketitle
\thispagestyle{empty}
\newpage
\tableofcontents
\newpage

% ==============================================================================
\section{Executive Summary}
% ==============================================================================

This report details the findings of a cybersecurity posture assessment conducted for \textbf{[Organization Name]}. The assessment combined a review of organizational security controls, an external network scan, and an analysis of pre-existing risks to provide a comprehensive overview of the organization's security posture.

The external network scan of the designated target IP address revealed a strong perimeter defense, with no open ports detected. This indicates a well-configured firewall or a lack of exposed services at the scanned address, which is a positive security finding.

However, the review of organizational security controls identified two significant gaps that present a high level of risk. A \textbf{Critical} risk was identified due to the absence of Multi-Factor Authentication (MFA) for computer logins. Additionally, a \textbf{High} risk was identified from the lack of a formal Employee Acceptable Use Policy (AUP). These policy and procedural weaknesses could expose the organization to significant threats, such as unauthorized access and insider threats, despite the strong network perimeter.

This report provides detailed findings and actionable recommendations to mitigate the identified risks and enhance the overall security posture of \textbf{[Organization Name]}.

% ==============================================================================
\section{Organizational Information}
% ==============================================================================

The following information was used as the basis for this assessment.

\begin{itemize}
    \item \textbf{Organization Name:} \textbf{[Organization Name]}
    \item \textbf{Primary Domain:} \seqsplit{\texttt{[Domain]}}
    \item \textbf{Scanned External IP:} \seqsplit{\texttt{[Client IP]}}
\end{itemize}

% ==============================================================================
\section{Security Control Review}
% ==============================================================================

A review of key organizational security controls was conducted based on a standardized questionnaire. The results are summarized below. "Yes" responses, indicating a control is in place, are marked with \ding{51}. "No" responses, indicating a control gap, are marked with \ding{55}.

\begin{table}[h!]
\centering
\caption{Organizational Security Control Status}
\label{tab:controls}
\begin{tabular}{@{}p{0.6\textwidth}cc@{}}
\toprule
\textbf{Control Question} & \textbf{Response} & \textbf{Assessment} \\
\midrule
Do you require MFA to access email? & \ding{51} & Best Practice Met \\
\addlinespace
Do you require MFA to log into computers? & \textbf{\ding{55}} & \textbf{Critical Gap} \\
\addlinespace
Do you require MFA to access sensitive data systems? & \ding{51} & Best Practice Met \\
\addlinespace
Does your organization have an employee acceptable use policy? & \textbf{\ding{55}} & \textbf{High-Risk Gap} \\
\addlinespace
Does your organization do security awareness training for new employees? & \ding{51} & Best Practice Met \\
\addlinespace
Does your organization do security awareness training for all employees at least once per year? & \ding{51} & Best Practice Met \\
\bottomrule
\end{tabular}
\end{table}

% ==============================================================================
\section{Technical Scan Results}
% ==============================================================================

An external network scan was performed to identify exposed services and potential vulnerabilities on the organization's public-facing infrastructure.

\subsection{Nmap Scan Findings}
\begin{itemize}
    \item \textbf{Target IP:} \seqsplit{\texttt{[Target IP]}}
    \item \textbf{Host Status:} Up
    \item \textbf{Scan Summary:} The scan completed successfully and determined the host was online. No open ports were discovered on the target system. All 1000 scanned ports were reported as \texttt{closed}.
    \item \textbf{Analyst Notes:} A finding of no open ports is a strong indicator of a well-configured firewall and a hardened external posture for this specific asset. This significantly reduces the external attack surface.
\end{itemize}

% ==============================================================================
\section{Consolidated Risk Assessment}
% ==============================================================================

This section synthesizes findings from the security control review, technical scans, and pre-existing risk data. The following table summarizes the key risks identified during this assessment.

\begin{table}[h!]
\centering
\caption{Summary of Identified Risks}
\label{tab:risks}
\begin{tabular}{@{}p{0.25\textwidth}p{0.5\textwidth}l@{}}
\toprule
\textbf{Risk Name} & \textbf{Overview} & \textbf{Severity} \\
\midrule
\addlinespace
Lack of MFA for Endpoint Logins & User credentials (username/password) are the only factor required for computer access. A compromised password could lead to direct endpoint and potential network compromise. & \textbf{Critical} \\
\addlinespace
No Employee Acceptable Use Policy (AUP) & The absence of a formal AUP creates ambiguity regarding safe and acceptable use of company systems, increasing the risk of insider threats and unintentional data exposure. & \textbf{High} \\
\addlinespace
\bottomrule
\end{tabular}
\end{table}

\textit{Note: No pre-existing vulnerabilities were provided for this assessment.}

% ==============================================================================
\section{Recommendations}
% ==============================================================================

Based on the findings of this assessment, the following actions are recommended to mitigate identified risks and improve the overall security posture.

\subsection{Priority: Critical}
\begin{itemize}
    \item \textbf{Implement MFA for Computer Logins:}
    \begin{itemize}
        \item \textbf{Action:} Enforce Multi-Factor Authentication (MFA) for all user logins to company-managed computers (desktops and laptops).
        \item \textbf{Justification:} This is one of the most effective controls to prevent unauthorized access resulting from compromised credentials. It protects endpoints, which are the gateway to the internal network and sensitive data.
        \item \textbf{Suggested Solutions:} Windows Hello for Business, Duo Security, Okta, or other enterprise-grade MFA providers.
    \end{itemize}
\end{itemize}

\subsection{Priority: High}
\begin{itemize}
    \item \textbf{Develop and Implement an Acceptable Use Policy (AUP):}
    \begin{itemize}
        \item \textbf{Action:} Create a formal AUP that clearly defines the rules and expectations for all employees when using company technology, network resources, and data.
        \item \textbf{Justification:} An AUP is a foundational governance document that reduces organizational risk, sets a baseline for user behavior, and provides a framework for enforcing security policies.
        \item \textbf{Implementation:} The policy should be reviewed by management, published, and distributed to all employees. A signed acknowledgment should be collected from each employee to ensure awareness.
    \end{itemize}
\end{itemize}

\end{document}
```