```latex
\documentclass[12pt]{article}

% Preamble: Required Packages
\usepackage[a4paper, margin=1in]{geometry}
\usepackage{pifont} % For \ding
\usepackage{booktabs} % For professional tables
\usepackage{hyperref} % For hyperlinks and PDF metadata
\usepackage{url} % For formatting URLs
\usepackage{seqsplit} % For splitting long strings in texttt
\usepackage{graphicx}
\usepackage[table]{xcolor} % For coloring table cells
\usepackage{tcolorbox} % For highlighted boxes

% Document Metadata
\hypersetup{
    colorlinks=true,
    linkcolor=blue,
    filecolor=magenta,      
    urlcolor=cyan,
    pdftitle={Cybersecurity Assessment Report},
    pdfauthor={Cybersecurity Analyst},
    pdfsubject={Security Analysis},
    pdfkeywords={Security, Report, Analysis},
}

% Custom Commands for Severity
\newcommand{\sevCRITICAL}{\textcolor{red!80!black}{\textbf{CRITICAL}}}
\newcommand{\sevHIGH}{\textcolor{orange!90!black}{\textbf{HIGH}}}
\newcommand{\sevMEDIUM}{\textcolor{yellow!80!black}{\textbf{MEDIUM}}}
\newcommand{\sevLOW}{\textcolor{green!70!black}{\textbf{LOW}}}

% Check and Cross marks
\newcommand{\cmark}{\ding{51}}
\newcommand{\xmark}{\ding{55}}

\begin{document}

% --- Title Page ---
\begin{titlepage}
    \centering
    \vspace*{1cm}
    \Huge\textbf{Cybersecurity Assessment Report}
    \vspace{1.5cm}
    \Large
    Prepared for: \textbf{[Organization Name]}
    \vspace{2cm}
    \includegraphics[width=0.4\textwidth]{example-image-a} % Placeholder for a logo
    \vfill
    \large
    \textbf{Date of Report:} \today \\
    \textbf{Generated By:} Expert Cybersecurity Analyst
\end{titlepage}

\tableofcontents
\newpage

% --- Section 1: Executive Summary ---
\section{Executive Summary}
This report provides a comprehensive cybersecurity assessment for \textbf{[Organization Name]}, based on an analysis of network scan data, organizational security controls, and pre-existing risk information. The assessment synthesizes these data points to provide a holistic view of the organization's current security posture.

The analysis revealed several significant risks that require immediate attention. Key findings include:
\begin{itemize}
    \item \textbf{\sevCRITICAL{} Risk:} A publicly accessible FTP server was identified running a dangerously outdated version of \texttt{vsftpd} (2.3.4). This version contains a well-known critical backdoor vulnerability (CVE-2011-2523). Furthermore, the server is configured to allow anonymous logins, posing an immediate and severe threat of unauthorized access and system compromise.
    \item \textbf{\sevHIGH{} Risk:} A critical gap in endpoint security was identified. The organization does not enforce Multi-Factor Authentication (MFA) for computer logins. This significantly increases the risk of unauthorized access to workstations and internal resources should an employee's credentials be compromised.
    \item \textbf{\sevMEDIUM{} Risk:} The organization is aware of an existing issue with outdated Windows 7 workstations. This End-of-Life operating system no longer receives security updates, leaving it vulnerable to a wide range of exploits and compounding the risk associated with the lack of MFA.
\end{itemize}

The combination of these findings indicates a fragile security posture. An external attacker could readily exploit the FTP server to gain a foothold in the network. The lack of fundamental security controls like MFA on endpoints would then facilitate lateral movement and deeper compromise. Immediate remediation of the identified critical and high-severity risks is strongly recommended.

% --- Section 2: Organizational Information ---
\section{Organizational Information}
This section details the information provided about the organization.
\begin{itemize}
    \item \textbf{Organization Name:} \textbf{[Organization Name]}
    \item \textbf{Primary Email Domain:} \texttt{[Domain]}
    \item \textbf{Scanned External IP:} \texttt{[Client IP]}
\end{itemize}

% --- Section 3: Security Control Review ---
\section{Security Control Review}
The following table summarizes the organization's responses to a security controls questionnaire. These answers provide insight into the documented security policies and procedures currently in place.

\begin{table}[h!]
\centering
\caption{Security Controls Questionnaire Results}
\label{tab:controls}
\begin{tabular}{p{0.75\linewidth} c}
\toprule
\textbf{Control Question} & \textbf{Response} \\
\midrule
Do you require MFA to access email? & \cmark \\
Do you require MFA to log into computers? & \cellcolor{red!25}\xmark \\
Do you require MFA to access sensitive data systems? & \cmark \\
Does your organization have an employee acceptable use policy? & \cmark \\
Does your organization do security awareness training for new employees? & \cmark \\
Does your organization do security awareness training for all employees at least once per year? & \cmark \\
\bottomrule
\end{tabular}
\end{table}

\subsection*{Analysis of Control Gaps}
While the organization has implemented several important security controls, including MFA for email and sensitive systems, a significant gap exists. The failure to require MFA for computer logins (\textbf{highlighted in red}) is a major weakness. This control is fundamental to modern endpoint security and protects against credential theft attacks (e.g., phishing, password spraying). Without it, a compromised password directly translates to workstation access for an attacker.

% --- Section 4: Technical Scan Results ---
\section{Technical Scan Results}
An external network scan was performed to identify open ports and exposed services.
\begin{itemize}
    \item \textbf{Target IP Address:} \texttt{[Target IP]}
    \item \textbf{Scan Date:} Scan date not provided in source data.
\end{itemize}

\begin{table}[h!]
\centering
\caption{Open Ports and Services Detected}
\label{tab:nmap}
\begin{tabular}{l l l l p{0.3\linewidth}}
\toprule
\textbf{Port} & \textbf{State} & \textbf{Service} & \textbf{Product \& Version} & \textbf{Notes} \\
\midrule
21/tcp & Open & ftp & vsftpd 2.3.4 & \sevCRITICAL{} Anonymous FTP login allowed. This version is vulnerable to a critical backdoor (CVE-2011-2523). \\
\bottomrule
\end{tabular}
\end{table}

\subsection*{Analysis of Technical Findings}
The technical scan revealed a \textbf{critical vulnerability}. The FTP service (\texttt{vsftpd 2.3.4}) is dangerously outdated (circa 2011) and contains a known backdoor that allows for remote command execution. The configuration allowing anonymous login further exacerbates this risk by permitting any external entity to connect and potentially interact with the vulnerable service or access stored files. This service should be considered compromised and must be addressed immediately.

% --- Section 5: Consolidated Risk Assessment ---
\section{Consolidated Risk Assessment}
This section correlates findings from the security control review, technical scan, and pre-existing risk data to provide a unified view of the top security risks.

\begin{table}[h!]
\centering
\caption{Summary of Identified Risks}
\label{tab:risks}
\begin{tabular}{p{0.1\linewidth} p{0.25\linewidth} p{0.4\linewidth} l}
\toprule
\textbf{Risk ID} & \textbf{Risk Name} & \textbf{Description} & \textbf{Severity} \\
\midrule
RISK-001 & Vulnerable Public FTP Server & An outdated and misconfigured FTP server (\texttt{vsftpd 2.3.4}) is exposed to the internet, allowing anonymous access and containing a critical backdoor vulnerability. & \sevCRITICAL{} \\
\addlinespace
RISK-002 & No MFA on Workstation Logins & Lack of Multi-Factor Authentication on computer logins allows an attacker with valid credentials to gain direct access to endpoints and the internal network. & \sevHIGH{} \\
\addlinespace
RISK-003 & Outdated Operating Systems & Workstations are running Windows 7, an End-of-Life OS that no longer receives security updates, leaving them susceptible to known exploits. & \sevMEDIUM{} \\
\bottomrule
\end{tabular}
\end{table}

% --- Section 6: Recommendations ---
\section{Recommendations}
Based on the consolidated risk assessment, the following actions are recommended to mitigate the identified vulnerabilities and improve the overall security posture.

\begin{tcolorbox}[colback=red!5!white,colframe=red!75!black,title=\textbf{Immediate Priority: Remediate RISK-001}]
\subsubsection*{Vulnerable Public FTP Server}
\begin{enumerate}
    \item \textbf{Disconnect:} Immediately take the server at \texttt{[Target IP]} offline by blocking port 21 at the firewall or shutting down the machine.
    \item \textbf{Investigate:} Perform a forensic analysis of the server to determine if it has already been compromised.
    \item \textbf{Decommission:} Do not bring the current service back online. If FTP functionality is required, deploy a new, fully patched server using a secure protocol like SFTP (SSH File Transfer Protocol).
    \item \textbf{Configure Securely:} Ensure any replacement service requires strong authentication and does not allow anonymous access unless explicitly and safely required for public data distribution.
\end{enumerate}
\end{tcolorbox}

\begin{tcolorbox}[colback=orange!5!white,colframe=orange!75!black,title=\textbf{High Priority: Remediate RISK-002}]
\subsubsection*{No MFA on Workstation Logins}
\begin{enumerate}
    \item \textbf{Procure Solution:} Select and procure an MFA solution for endpoint authentication. Options include Windows Hello for Business, Duo Security, or other identity providers.
    \item \textbf{Deploy:} Roll out MFA enforcement for all employee workstation and laptop logins.
    \item \textbf{Train Users:} Provide clear instructions and training to all employees on how to use the new MFA system to ensure a smooth transition.
\end{enumerate}
\end{tcolorbox}

\begin{tcolorbox}[colback=yellow!5!white,colframe=yellow!60!black,title=\textbf{Scheduled Priority: Remediate RISK-003}]
\subsubsection*{Outdated Operating Systems}
\begin{enumerate}
    \item \textbf{Accelerate Upgrade Plan:} Expedite the existing project to upgrade or replace all workstations running Windows 7.
    \item \textbf{Prioritize Critical Users:} Prioritize the upgrade for users with access to sensitive data and system administrators.
    \item \textbf{Isolate (If Necessary):} If immediate upgrades are not possible, isolate remaining Windows 7 machines on a restricted network segment to limit their exposure and potential impact in a compromise.
\end{enumerate}
\end{tcolorbox}

\end{document}
```