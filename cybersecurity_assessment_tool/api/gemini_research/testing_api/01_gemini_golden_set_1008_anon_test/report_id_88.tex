```latex
\documentclass[12pt]{article}

% --- PACKAGES ---
\usepackage[margin=1in]{geometry} % Set page margins
\usepackage{pifont}               % For checkmarks and crosses (\ding)
\usepackage{booktabs}             % For professional tables (\toprule, \midrule, \bottomrule)
\usepackage{hyperref}             % For hyperlinks, metadata
\usepackage{url}                  % For formatting URLs
\usepackage{seqsplit}             % For splitting long strings in texttt
\usepackage{graphicx}             % For logos (optional, but good practice)
\usepackage{xcolor}               % For colors

% --- DOCUMENT METADATA ---
\hypersetup{
    colorlinks=true,
    linkcolor=blue,
    filecolor=magenta,      
    urlcolor=cyan,
    pdftitle={Cybersecurity Posture Assessment Report},
    pdfauthor={Cybersecurity Analyst},
    pdfsubject={Security Assessment},
    pdfkeywords={Cybersecurity, Nmap, Risk, Assessment},
}

% --- COMMANDS ---
\newcommand{\yes}{\ding{51}}
\newcommand{\no}{\ding{55}}

% --- DOCUMENT START ---
\begin{document}

% --- TITLE PAGE ---
\begin{titlepage}
    \centering
    \vspace*{1cm}
    \Huge\textbf{Cybersecurity Posture Assessment Report}
    \vspace{1.5cm}
    \Large
    \textbf{Prepared for:} \\
    \vspace{0.5cm}
    \textbf{[Organization Name]}
    \vspace{2cm}
    \large
    \textbf{Date of Report:} \\
    \vspace{0.5cm}
    \today
    \vfill
    \large
    \textbf{Generated By:} \\
    \vspace{0.5cm}
    Expert Cybersecurity Analyst
\end{titlepage}

\tableofcontents
\newpage

% --- EXECUTIVE SUMMARY ---
\section{Executive Summary}
This report provides a comprehensive assessment of the cybersecurity posture for \textbf{[Organization Name]}, based on an analysis of organizational security controls, external network scan results, and pre-existing risk data.

The assessment identified a significant disparity between the organization's external and internal security postures. The external network perimeter appears strong, with a network scan revealing no open ports on the target system. This indicates a well-configured firewall and is a commendable security practice.

However, the review of internal security controls revealed several critical and high-risk gaps. The absence of Multi-Factor Authentication (MFA) for email and computer access represents a \textbf{critical vulnerability}, exposing the organization to severe risks such as business email compromise and unauthorized system access. Furthermore, the lack of a formal Acceptable Use Policy and mandatory security training for new employees creates a high-risk environment susceptible to insider threats and human error.

This report outlines the detailed findings and provides actionable recommendations to mitigate these identified risks and strengthen the overall security framework.

% --- ORGANIZATIONAL INFORMATION ---
\section{Organizational Information}
The following information was used as the basis for this assessment. Due to the anonymized nature of the input data, placeholders have been used where necessary.

\begin{table}[h!]
\centering
\begin{tabular}{@{}ll@{}}
\toprule
\textbf{Attribute} & \textbf{Value} \\ \midrule
Organization Name & \textbf{[Organization Name]} \\
Email Domain & \texttt{[Domain]} \\
External IP Address & \texttt{[Client IP]} \\ \bottomrule
\end{tabular}
\caption{Client Organizational Details}
\end{table}

% --- SECURITY CONTROL REVIEW ---
\section{Security Control Review}
A review of the organization's security policies and procedures was conducted via a questionnaire. The responses highlight key areas of strength and weakness in the current security framework. Answers marked with \no\ indicate significant gaps that require immediate attention.

\begin{table}[h!]
\centering
\begin{tabular}{@{}p{0.75\textwidth}c@{}}
\toprule
\textbf{Control Question} & \textbf{Response} \\ \midrule
Do you require MFA to access email? & \no \\
Do you require MFA to log into computers? & \no \\
Do you require MFA to access sensitive data systems? & \yes \\
Does your organization have an employee acceptable use policy? & \no \\
Does your organization do security awareness training for new employees? & \no \\
Does your organization do security awareness training for all employees at least once per year? & \yes \\ \bottomrule
\end{tabular}
\caption{Security Controls Questionnaire Results}
\end{table}

\subsection*{Analysis of Controls}
The questionnaire results reveal critical deficiencies in identity and access management and foundational security policies. While it is positive that MFA is used for sensitive data systems and annual training is conducted, the lack of MFA on core services like email and endpoint logins, coupled with the absence of an Acceptable Use Policy and new-hire training, creates an environment where security controls can be easily bypassed.

% --- TECHNICAL SCAN RESULTS ---
\section{Technical Scan Results}
An external network scan was performed to identify open ports and exposed services on the organization's perimeter.

\begin{itemize}
    \item \textbf{Target IP Address:} \texttt{[Target IP]}
    \item \textbf{Scan Date:} [Scan Date Not Provided]
    \item \textbf{Scanner Used:} Nmap
\end{itemize}

\subsection*{Findings}
The scan concluded that the target host is online, but all scanned ports were found to be in a \textbf{closed} state.

\begin{verbatim}
Host is up.
All 1000 scanned ports on [Target IP] are in state: closed
\end{verbatim}

\subsection*{Analysis of Technical Findings}
\textbf{No open ports were discovered.} This is a strong positive finding. It indicates that the external firewall is properly configured to deny unsolicited inbound traffic, significantly reducing the external attack surface. This practice effectively protects against opportunistic, automated attacks that scan for common vulnerable services.

% --- RISK ASSESSMENT ---
\section{Risk Assessment}
This section synthesizes the findings from the security control review, technical scan, and pre-existing risk data. The primary risks identified are procedural and policy-based rather than technical vulnerabilities on the network perimeter.

\begin{table}[h!]
\centering
\begin{tabular}{@{}p{0.2\textwidth}p{0.55\textwidth}l@{}}
\toprule
\textbf{Risk Name} & \textbf{Overview} & \textbf{Severity} \\ \midrule
\textbf{Lack of MFA on Core Systems} & Email accounts and computer logins are protected only by passwords. This makes them highly vulnerable to credential stuffing, phishing, and brute-force attacks, which could lead to a full network compromise. & \textcolor{red}{\textbf{Critical}} \\
\addlinespace
\textbf{Missing Acceptable Use Policy (AUP)} & Without a formal AUP, there are no defined rules for employee use of company assets. This increases the risk of data misuse, unauthorized software installation, and insider threats, while also creating legal and compliance challenges. & \textcolor{orange}{\textbf{High}} \\
\addlinespace
\textbf{Incomplete Security Awareness Training} & New employees are not provided with security training during onboarding. This is a critical missed opportunity to establish a security-first mindset, leaving the organization's newest and often most vulnerable members unaware of policies and threats. & \textcolor{orange}{\textbf{High}} \\
\bottomrule
\end{tabular}
\caption{Summary of Identified Risks}
\end{table}

% --- RECOMMENDATIONS ---
\section{Recommendations}
The following actionable recommendations are provided to address the identified risks. They are prioritized based on severity to guide remediation efforts effectively.

\subsection*{1. Implement Comprehensive Multi-Factor Authentication (Critical)}
\begin{itemize}
    \item \textbf{Immediate Action:} Enforce MFA for all user access to email systems (e.g., Office 365, Google Workspace). This is the single most effective control to prevent business email compromise.
    \item \textbf{Next Step:} Deploy MFA for all remote access solutions (VPN) and for logging into company-managed computers (e.g., via Windows Hello for Business, Duo, or a similar solution).
    \item \textbf{Goal:} Ensure that no critical system relies solely on a password for authentication.
\end{itemize}

\subsection*{2. Develop and Enforce an Acceptable Use Policy (High)}
\begin{itemize}
    \item \textbf{Immediate Action:} Draft a formal Acceptable Use Policy (AUP) that clearly defines the rules for using company networks, computers, email, and data.
    \item \textbf{Next Step:} Require all current employees to read and formally acknowledge the new policy.
    \item \textbf{Goal:} Integrate the AUP acknowledgement into the new employee onboarding process to establish clear expectations from day one.
\end{itemize}

\subsection*{3. Enhance the Security Awareness Training Program (High)}
\begin{itemize}
    \item \textbf{Immediate Action:} Develop a security awareness training module specifically for new hires. This module should be a mandatory part of the onboarding process.
    \item \textbf{Content to Include:} The training should cover key topics such as phishing identification, password security, data handling, and the newly created AUP.
    \item \textbf{Goal:} Foster a culture of security throughout the organization, starting from an employee's first day. Continue with annual refresher training to keep security top-of-mind.
\end{itemize}

% --- DOCUMENT END ---
\end{document}
```