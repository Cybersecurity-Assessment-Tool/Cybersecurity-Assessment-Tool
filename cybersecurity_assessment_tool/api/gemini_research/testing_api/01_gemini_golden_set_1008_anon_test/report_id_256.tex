```latex
\documentclass[12pt]{article}

% Preamble: Required Packages
\usepackage[margin=1in]{geometry} % For setting page margins
\usepackage{pifont}               % For checkmarks and crosses (dingbats)
\usepackage{booktabs}             % For professional-looking tables
\usepackage{graphicx}             % For including logos, etc.
\usepackage{xcolor}               % For custom colors
\usepackage[hidelinks]{hyperref}  % For hyperlinks without ugly boxes
\usepackage{url}                  % For formatting URLs
\usepackage{seqsplit}             % For splitting long strings in texttt
\usepackage{fancyhdr}             % For headers and footers

% --- Document Setup ---
\pagestyle{fancy}
\fancyhf{} % Clear all header and footer fields
\fancyhead[L]{Cybersecurity Assessment Report}
\fancyhead[R]{\textbf{[Organization Name]}}
\fancyfoot[C]{\thepage}
\renewcommand{\headrulewidth}{0.4pt}
\renewcommand{\footrulewidth}{0.4pt}

% Define colors for severity
\definecolor{criticalred}{HTML}{D7263D}
\definecolor{highorange}{HTML}{F49D40}
\definecolor{mediumyellow}{HTML}{F4D440}
\definecolor{lowblue}{HTML}{5C9EAD}
\definecolor{infogray}{HTML}{939597}

% --- Document Start ---
\begin{document}

% --- Title Page ---
\begin{titlepage}
    \centering
    \vspace*{1cm}
    
    \Huge
    \textbf{Cybersecurity Posture Assessment Report}
    
    \vspace{1.5cm}
    
    \Large
    Prepared for: \\
    \vspace{0.5cm}
    \textbf{[Organization Name]}
    
    \vspace{2cm}
    
    \normalsize
    \textbf{Date of Report:} \today \\
    \textbf{Scan Date:} [Scan Date]
    
    \vfill
    
    \normalsize
    \textbf{CONFIDENTIAL} \\
    This document contains sensitive information. Access is restricted to authorized personnel only.
    
\end{titlepage}

% --- Table of Contents ---
\tableofcontents
\newpage

% --- Section 1: Executive Summary ---
\section{Executive Summary}
This report provides a comprehensive assessment of the cybersecurity posture for \textbf{[Organization Name]}. The analysis is based on a correlation of network scan data, a security controls questionnaire, and a review of known risks.

The overall security posture is currently weak, with several critical and high-risk gaps identified in administrative and policy-based controls. While the external network scan of the target asset revealed no open ports—a positive indicator of a well-configured network perimeter—this strength is significantly undermined by internal policy deficiencies.

Key findings include a critical lack of Multi-Factor Authentication (MFA) for email and sensitive data systems, exposing the organization to a high risk of account takeover and data breach. Furthermore, the complete absence of an employee acceptable use policy and any form of security awareness training creates a vulnerable environment where employees are more susceptible to social engineering and phishing attacks.

Immediate remediation is required to address these foundational security gaps. Recommendations focus on implementing critical access controls, establishing a formal security policy framework, and launching a comprehensive employee training program.

% --- Section 2: Organizational Information ---
\section{Organizational Information}
This section details the information provided by the client organization. The data is used to establish the context for the assessment.
\begin{center}
\begin{tabular}{ll}
\toprule
\textbf{Attribute} & \textbf{Value} \\
\midrule
Organization Name & \textbf{[Organization Name]} \\
Primary Email Domain & \texttt{[Domain]} \\
External IP Address & \texttt{[Client IP]} \\
\bottomrule
\end{tabular}
\end{center}

% --- Section 3: Security Control Review ---
\section{Security Control Review}
The following table summarizes the organization's responses to a security controls questionnaire. Each response is evaluated against industry best practices to identify potential gaps. A green checkmark (\textcolor{green}{\ding{51}}) indicates an aligned practice, while a red 'X' (\textcolor{criticalred}{\ding{55}}) signifies a significant control gap.

\begin{center}
\begin{tabular}{p{0.5\linewidth} c p{0.35\linewidth}}
\toprule
\textbf{Control Question} & \textbf{Status} & \textbf{Finding / Best Practice} \\
\midrule
Do you require MFA to access email? & \textcolor{criticalred}{\ding{55}} & \textbf{Critical Gap.} Email is a primary target for attackers. Lack of MFA exposes accounts to compromise via credential theft. \\
\addlinespace
Do you require MFA to log into computers? & \textcolor{green}{\ding{51}} & Control is in place. Endpoint access is protected with strong authentication. \\
\addlinespace
Do you require MFA to access sensitive data systems? & \textcolor{criticalred}{\ding{55}} & \textbf{Critical Gap.} Sensitive data is not adequately protected from unauthorized access resulting from compromised credentials. \\
\addlinespace
Does your organization have an employee acceptable use policy? & \textcolor{criticalred}{\ding{55}} & \textbf{High Risk.} Without a formal policy, there is no clear guidance for employees on the secure and acceptable use of company assets. \\
\addlinespace
Does your organization do security awareness training for new employees? & \textcolor{criticalred}{\ding{55}} & \textbf{High Risk.} New employees are a common target for attackers. They are not being equipped to identify or respond to threats. \\
\addlinespace
Does your organization do security awareness training for all employees at least once per year? & \textcolor{criticalred}{\ding{55}} & \textbf{High Risk.} Security is a continuous process. Without regular training, employee knowledge degrades, increasing susceptibility to phishing and social engineering. \\
\bottomrule
\end{tabular}
\end{center}

% --- Section 4: Technical Scan Results ---
\section{Technical Scan Results}
An external network vulnerability scan was performed on the designated target IP address to identify open ports and exposed services.

\subsection{Scan Summary}
\begin{itemize}
    \item \textbf{Target IP:} \texttt{[Target IP]}
    \item \textbf{Host Status:} Up
    \item \textbf{Finding:} The scan confirmed that the host is online but did not identify any open TCP ports. All 1000 scanned ports were reported as being in a "closed" state. This indicates a strong network firewall configuration that denies unsolicited inbound traffic, which is a positive security control.
\end{itemize}

\subsection{Detailed Port Information}
No open ports were discovered during the scan.

% --- Section 5: Identified Risks and Vulnerabilities ---
\section{Identified Risks and Vulnerabilities}
This section synthesizes findings from the security control review and technical scan to provide a consolidated list of identified risks. Based on the provided data, there were no pre-existing vulnerabilities, so all risks listed below are derived from this assessment.

\begin{center}
\begin{tabular}{p{0.4\linewidth} p{0.15\linewidth} p{0.4\linewidth}}
\toprule
\textbf{Risk / Vulnerability} & \textbf{Severity} & \textbf{Overview} \\
\midrule
\textbf{Lack of MFA on Critical Systems} & \textcolor{criticalred}{\textbf{Critical}} & The absence of MFA on email and sensitive data systems presents a severe risk of account takeover, data exfiltration, and business email compromise. \\
\addlinespace
\textbf{No Security Awareness Training Program} & \textcolor{highorange}{\textbf{High}} & Employees are the first line of defense. Without training, they are highly susceptible to phishing, malware, and social engineering attacks, which can bypass technical controls. \\
\addlinespace
\textbf{Absence of Acceptable Use Policy (AUP)} & \textcolor{highorange}{\textbf{High}} & The lack of a formal AUP creates ambiguity regarding security responsibilities and acceptable behavior, increasing the risk of insider threat and non-compliance. \\
\bottomrule
\end{tabular}
\end{center}

% --- Section 6: Recommendations ---
\section{Recommendations}
The following actionable recommendations are prioritized based on the severity of the associated risks.

\subsection{Immediate Priority (Critical Risks)}
\begin{itemize}
    \item \textbf{Implement MFA on All Email Accounts:} Immediately enforce MFA for all user access to the email system (\texttt{[Domain]}). This is the single most effective control to prevent account compromise.
    \item \textbf{Enforce MFA on Sensitive Systems:} Identify all systems containing sensitive data and enforce MFA for all user access. Prioritize systems with financial, customer, or proprietary information.
\end{itemize}

\subsection{High Priority}
\begin{itemize}
    \item \textbf{Develop and Implement a Security Awareness Program:}
    \begin{itemize}
        \item Create a mandatory training module for all new hires as part of their onboarding process.
        \item Schedule and require annual security awareness training for all existing employees to keep them informed of evolving threats.
    \end{itemize}
    \item \textbf{Establish an Acceptable Use Policy (AUP):}
    \begin{itemize}
        \item Draft a formal AUP that clearly defines the rules for using company networks, devices, and data.
        \item Require all employees to read and acknowledge the policy.
    \end{itemize}
\end{itemize}

% --- Section 7: Conclusion ---
\section{Conclusion}
The assessment for \textbf{[Organization Name]} reveals a significant disparity between its technical and administrative security controls. While the external network perimeter appears secure, the organization is highly vulnerable to attacks that target its employees, such as phishing and credential theft. The identified gaps in MFA implementation, security training, and formal policies represent a substantial risk to the confidentiality, integrity, and availability of its data and systems.

By implementing the prioritized recommendations in this report, \textbf{[Organization Name]} can drastically reduce its attack surface and build a more resilient and mature security posture.

\end{document}
```