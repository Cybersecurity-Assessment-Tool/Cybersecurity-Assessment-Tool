```latex
\documentclass[12pt]{article}

% Preamble: Required Packages
\usepackage[margin=1in]{geometry}
\usepackage{pifont} % For checkmarks and crosses (\ding)
\usepackage{booktabs} % For professional-looking tables
\usepackage{hyperref} % For hyperlinks and metadata
\usepackage{url} % For formatting URLs
\usepackage{seqsplit} % For breaking long monospaced strings
\usepackage{xcolor} % For colors
\usepackage{graphicx}

% Document Metadata and Hyperlink Setup
\hypersetup{
    colorlinks=true,
    linkcolor=black,
    urlcolor=blue,
    pdftitle={Cybersecurity Assessment Report},
    pdfauthor={Cybersecurity Analysis Division},
    pdfsubject={Security Posture Analysis},
    pdfkeywords={Security, RDP, MFA, Risk Assessment}
}

% Custom Commands for Readability
\newcommand{\yes}{\textcolor{green}{\ding{51}}}
\newcommand{\no}{\textcolor{red}{\ding{55}}}
\newcommand{\orgname}{\textbf{[Organization Name]}}
\newcommand{\clientip}{\texttt{[Client IP]}}
\newcommand{\targetip}{\texttt{[Target IP]}}
\newcommand{\domain}{\texttt{[Domain]}}

% --- DOCUMENT START ---
\begin{document}

% Title Page
\begin{titlepage}
    \centering
    \vspace*{1cm}
    \Huge\textbf{Cybersecurity Assessment Report}
    \vspace{1.5cm}
    \Large
    Prepared for: \\
    \vspace{0.5cm}
    \orgname
    \vfill
    \large
    \textbf{Date of Report:} \today \\
    \textbf{Analysis Period:} Current Assessment
    \vspace{1cm}
    \rule{\textwidth}{0.4pt}
    \vspace{0.4cm}
    \textit{This report contains sensitive information and should be handled with discretion. Distribution is restricted to authorized personnel only.}
\end{titlepage}

\tableofcontents
\newpage

% --- EXECUTIVE SUMMARY ---
\section{Executive Summary}
This report provides a comprehensive analysis of the current cybersecurity posture for \orgname, based on network scans, a security controls questionnaire, and a review of pre-existing risks.

The assessment reveals a \textbf{Critical} overall risk level. The primary finding is the direct exposure of a Remote Desktop Protocol (RDP) service on port 3389 to the public internet. This vulnerability is severely compounded by a systemic lack of Multi-Factor Authentication (MFA) across all critical systems, including email, computer logins, and sensitive data access. Furthermore, the absence of a consistent security awareness training program for employees significantly increases the likelihood of a successful social engineering or phishing attack, which could lead to credential compromise.

The combination of these factors creates a direct and high-probability pathway for a ransomware attack or significant data breach. Immediate remediation is required to mitigate these risks and protect organizational assets.

% --- ORGANIZATIONAL INFORMATION ---
\section{Organizational Information}
This section outlines the basic information used for this assessment. Due to the anonymized nature of the input data, placeholders have been used.

\begin{tabular}{@{}ll}
    \textbf{Organization Name:} & \orgname \\
    \textbf{Primary Email Domain:} & \domain \\
    \textbf{Known External IP Address:} & \clientip \\
\end{tabular}

% --- SECURITY CONTROL REVIEW ---
\section{Security Control Review}
The following table summarizes the organization's responses to a security controls questionnaire. Answers marked with \no{} represent significant gaps in the security framework and are directly correlated with the risks identified in this report.

\begin{table}[h!]
\centering
\caption{Security Controls Questionnaire Results}
\begin{tabular}{p{0.75\textwidth}c}
\toprule
\textbf{Control Question} & \textbf{Status} \\
\midrule
Do you require MFA to access email? & \no \\
Do you require MFA to log into computers? & \no \\
Do you require MFA to access sensitive data systems? & \no \\
Does your organization have an employee acceptable use policy? & \yes \\
Does your organization do security awareness training for new employees? & \no \\
Does your organization do security awareness training for all employees at least once per year? & \no \\
\bottomrule
\end{tabular}
\end{table}

\subsection*{Analysis of Control Gaps}
The lack of MFA for email, computer, and data access is a critical deficiency. It means that a single compromised password is all an attacker needs to gain significant access to the network. The absence of security awareness training makes employees highly susceptible to phishing attacks designed to steal those passwords.

% --- TECHNICAL SCAN RESULTS ---
\section{Technical Scan Results}
An external network scan was performed to identify exposed services. The scan results corroborate and validate the pre-existing risk data.

\begin{itemize}
    \item \textbf{Scan Target:} \targetip
    \item \textbf{Scan Date:} \today
    \item \textbf{Scanner Used:} Nmap
\end{itemize}

\subsection*{Open Ports and Services}
A single critical port was found open to the public internet.

\begin{table}[h!]
\centering
\caption{Exposed Network Services}
\begin{tabular}{llll}
\toprule
\textbf{Port} & \textbf{State} & \textbf{Service Name} & \textbf{Description} \\
\midrule
3389/tcp & open & \texttt{ms-wbt-server} & Microsoft Remote Desktop Protocol (RDP) \\
\bottomrule
\end{tabular}
\end{table}

\subsection*{Technical Analysis}
The exposure of RDP (port 3389) is a well-known and highly dangerous configuration. It is one of the most common attack vectors for ransomware groups, who continuously scan the internet for open RDP ports. Attackers can use brute-force methods, credential stuffing, or credentials stolen via phishing to gain direct, interactive control over the exposed server. This provides a foothold to move laterally within the internal network.

% --- RISK ASSESSMENT & CORRELATION ---
\section{Risk Assessment \& Correlation}
This section synthesizes the findings from the security questionnaire, technical scan, and pre-existing risk register. The identified risks are not isolated; they are interconnected and amplify each other.

\begin{table}[h!]
\centering
\caption{Synthesized Risk Summary}
\begin{tabular}{p{0.2\textwidth}p{0.5\textwidth}p{0.2\textwidth}}
\toprule
\textbf{Risk Name} & \textbf{Overview} & \textbf{Severity} \\
\midrule
\textbf{RDP Exposure} & The Remote Desktop Protocol service on \targetip{} is publicly accessible. This is confirmed by both the Nmap scan and pre-existing risk data. & \textbf{Critical (9.0)} \\
\addlinespace
\textbf{Lack of MFA} & The absence of MFA on all critical systems means a single compromised password grants an attacker full access. This directly elevates the threat of the exposed RDP service. & \textbf{Critical} \\
\addlinespace
\textbf{No Security Training} & Employees are not trained to identify or report phishing attempts. This makes credential theft trivial for an attacker, providing them the keys to exploit the exposed RDP and lack of MFA. & \textbf{High} \\
\bottomrule
\end{tabular}
\end{table}

% --- RECOMMENDATIONS ---
\section{Recommendations}
The following actionable recommendations are provided to mitigate the identified risks. They are prioritized based on urgency and impact.

\subsection*{Immediate Actions (Within 24 Hours)}
\begin{itemize}
    \item \textbf{Close Port 3389:} Immediately configure the firewall protecting \targetip{} to block all inbound traffic on TCP port 3389 from the internet. This is the single most important step to prevent an imminent breach.
\end{itemize}

\subsection*{High-Priority Actions (Within 30 Days)}
\begin{itemize}
    \item \textbf{Implement MFA:} Deploy Multi-Factor Authentication across all remote access solutions, email platforms (e.g., Office 365, Google Workspace), and systems containing sensitive data.
    \item \textbf{Secure Remote Access:} If remote access is required, replace direct RDP exposure with a secure solution such as a Virtual Private Network (VPN) or a Zero Trust Network Access (ZTNA) gateway. All such solutions must be protected with MFA.
\end{itemize}

\subsection*{Medium-Priority Actions (Within 90 Days)}
\begin{itemize}
    \item \textbf{Launch Security Awareness Training:} Implement a mandatory security awareness training program for all employees. This program should be conducted upon hiring and at least annually thereafter. It should cover topics like phishing, password hygiene, and acceptable use.
    \item \textbf{Conduct Vulnerability Scanning:} Establish a regular vulnerability scanning program for all external and internal assets to proactively identify and remediate security weaknesses.
\end{itemize}

% --- CONCLUSION ---
\section{Conclusion}
The current security posture of \orgname{} is precarious and requires immediate attention. The combination of an exposed RDP server, a lack of multi-factor authentication, and an untrained workforce presents a clear and present danger to the organization's data, operations, and reputation.

By implementing the prioritized recommendations in this report, \orgname{} can significantly reduce its attack surface and build a more resilient and defensible security foundation.

% --- DOCUMENT END ---
\end{document}
```