```latex
\documentclass[12pt]{article}

% Preamble: Required Packages
\usepackage[margin=1in]{geometry}
\usepackage{pifont} % For \ding
\usepackage{booktabs} % For professional tables
\usepackage{hyperref} % For hyperlinks
\usepackage{url} % For URL formatting
\usepackage{seqsplit} % For splitting long strings in tt font

% Hyperref Setup
\hypersetup{
    colorlinks=true,
    linkcolor=black,
    urlcolor=blue,
    pdftitle={Cybersecurity Posture Report},
    pdfauthor={Cybersecurity Analyst},
    pdfsubject={Security Assessment},
    pdfkeywords={Security, Nmap, Risk, Assessment},
}

% Document Start
\begin{document}

% --- Title Page ---
\begin{titlepage}
    \centering
    \vspace*{\stretch{1.0}}
    \Huge\textbf{Cybersecurity Posture Report}
    \vspace{0.5cm}
    \Large For
    \vspace{0.5cm}
    \huge\textbf{[Organization Name]}
    \vspace*{\stretch{2.0}}
    \large
    \textbf{Date of Report:} \today \\
    \textbf{Date of Assessment:} 2025-11-22 \\
    \textbf{Author:} Cybersecurity Analyst
    \vspace*{\stretch{1.0}}
\end{titlepage}

\tableofcontents
\newpage

% --- Section 1: Executive Overview ---
\section{Executive Overview}
This report details the findings of a cybersecurity assessment conducted for \textbf{[Organization Name]}. The assessment combined a review of organizational security controls, an external network scan, and an analysis of pre-existing risks. The scan was performed on \textbf{November 22, 2025}.

The organization demonstrates a solid foundation in security awareness training and policy. However, two significant risks were identified that require immediate attention. A critical gap in endpoint security was noted due to the absence of Multi-Factor Authentication (MFA) for computer logins. Additionally, the external-facing web server at \seqsplit{\texttt{[Target IP]}} is running an outdated version of Nginx (1.18.0), exposing the organization to publicly known vulnerabilities.

Addressing these issues is crucial to mitigating the risk of unauthorized access, data breaches, and service disruption. Detailed findings and actionable recommendations are provided in the subsequent sections.

% --- Section 2: Organizational Information ---
\section{Organizational Information}
The following details were used as the basis for this assessment. As per the provided data, some identifying information has been anonymized.

\begin{itemize}
    \item \textbf{Organization Name:} \textbf{[Organization Name]}
    \item \textbf{Primary Email Domain:} \seqsplit{\texttt{[Domain]}}
    \item \textbf{External IP Address Scanned:} \seqsplit{\texttt{[Client IP]}}
\end{itemize}

% --- Section 3: Security Control Review ---
\section{Security Control Review}
A review of organizational security controls was conducted based on a standardized questionnaire. The responses indicate the current state of implemented policies and procedures. A checkmark (\ding{51}) indicates a positive control, while a cross (\ding{55}) signifies a potential security gap.

\begin{table}[h!]
\centering
\caption{Organizational Security Control Questionnaire}
\label{tab:controls}
\begin{tabular}{p{0.75\linewidth} c}
\toprule
\textbf{Control Question} & \textbf{Response} \\
\midrule
Do you require MFA to access email? & \ding{51} \\
Do you require MFA to log into computers? & \textbf{\color{red}\ding{55}} \\
Do you require MFA to access sensitive data systems? & \ding{51} \\
Does your organization have an employee acceptable use policy? & \ding{51} \\
Does your organization do security awareness training for new employees? & \ding{51} \\
Does your organization do security awareness training for all employees at least once per year? & \ding{51} \\
\bottomrule
\end{tabular}
\end{table}

The primary finding from this review is the lack of MFA for computer logins. This represents a critical vulnerability, as a single compromised password could grant an attacker direct access to an employee's workstation and, potentially, the internal network.

% --- Section 4: Technical Scan Results ---
\section{Technical Scan Results}
An external network scan was performed against the public-facing IP address to identify open ports and exposed services.

\begin{itemize}
    \item \textbf{Target IP Address:} \seqsplit{\texttt{[Target IP]}}
    \item \textbf{Scan Date:} 2025-11-22T10:00:00Z
\end{itemize}

\begin{table}[h!]
\centering
\caption{Open Ports and Services Detected}
\label{tab:nmap}
\begin{tabular}{l l l l l}
\toprule
\textbf{Port} & \textbf{State} & \textbf{Service} & \textbf{Product} & \textbf{Version} \\
\midrule
443/tcp & open & https & nginx & 1.18.0 \\
\bottomrule
\end{tabular}
\end{table}

The scan identified a web server running Nginx version 1.18.0. This version was released in April 2020 and is now considered outdated. It is known to be vulnerable to several security exploits (e.g., CVE-2021-23017), which could allow an attacker to compromise the server.

% --- Section 5: Risk Assessment ---
\section{Risk Assessment}
This section synthesizes findings from the security control review, technical scan, and pre-existing risk data. The pre-existing risk log was empty. The following new risks have been identified.

\begin{table}[h!]
\centering
\caption{Identified Risks}
\label{tab:risks}
\begin{tabular}{p{0.1\linewidth} p{0.3\linewidth} p{0.15\linewidth} p{0.35\linewidth}}
\toprule
\textbf{Risk ID} & \textbf{Risk Name} & \textbf{Severity} & \textbf{Description} \\
\midrule
RISK-001 & Lack of MFA on Workstation Logins & \textbf{Critical} & The absence of MFA on computer logins means that a compromised password is the only barrier to entry. This significantly increases the risk of unauthorized access, lateral movement, and ransomware deployment. \\
\addlinespace
RISK-002 & Outdated Nginx Web Server & \textbf{High} & The web server is running Nginx 1.18.0, a version with publicly disclosed vulnerabilities. This exposes the server to automated attacks and potential compromise, leading to data breaches or service unavailability. \\
\bottomrule
\end{tabular}
\end{table}

% --- Section 6: Recommendations ---
\section{Recommendations}
Based on the identified risks, the following actions are recommended to improve the organization's security posture.

\begin{itemize}
    \item \textbf{Recommendation for RISK-001 (Critical):}
    \begin{itemize}
        \item \textbf{Action:} Implement a mandatory Multi-Factor Authentication (MFA) policy for all employee and privileged user computer/workstation logins.
        \item \textbf{Justification:} This will create a critical defense-in-depth layer, ensuring that even if user credentials are stolen, attackers cannot gain access to endpoints without a second authentication factor.
    \end{itemize}
    \vspace{1em}
    \item \textbf{Recommendation for RISK-002 (High):}
    \begin{itemize}
        \item \textbf{Action:} Upgrade the Nginx web server on host \seqsplit{\texttt{[Target IP]}} from version 1.18.0 to the latest stable version recommended by the vendor.
        \item \textbf{Justification:} Patching and updating software is a fundamental security control. Upgrading Nginx will remediate known vulnerabilities and protect the server from being an easy target for attackers. A regular patch management schedule should be implemented for all external-facing systems.
    \end{itemize}
\end{itemize}

\end{document}
```