```latex
\documentclass[12pt]{article}

% --- PACKAGES ---
\usepackage[margin=1in]{geometry}
\usepackage{pifont} % For checkmarks and crosses
\usepackage{booktabs} % For professional tables
\usepackage{hyperref} % For hyperlinks
\usepackage{url} % For URL formatting
\usepackage{seqsplit} % For splitting long strings in texttt
\usepackage[T1]{fontenc}

% --- DOCUMENT INFORMATION ---
\title{Cybersecurity Posture Assessment Report}
\author{Cybersecurity Analyst}
\date{November 22, 2025}

% --- HYPERREF SETUP ---
\hypersetup{
    colorlinks=true,
    linkcolor=black,
    urlcolor=blue,
    pdftitle={Cybersecurity Posture Assessment Report},
    pdfauthor={Cybersecurity Analyst},
}

% --- BEGIN DOCUMENT ---
\begin{document}

\maketitle

\begin{abstract}
This report provides a comprehensive cybersecurity assessment for \textbf{[Organization Name]}. The analysis is based on a synthesis of network scan data, a security controls questionnaire, and a review of pre-existing risks. The assessment identifies critical gaps in administrative controls, including the absence of an acceptable use policy and inadequate security training. Additionally, a significant technical vulnerability was discovered in the form of outdated web server software. This report details these findings and provides actionable recommendations to mitigate the identified risks and improve the overall security posture.
\end{abstract}

\tableofcontents
\newpage

% ===================================================================
\section{Overview and Scope}
% ===================================================================

This assessment was conducted to evaluate the current security posture of \textbf{[Organization Name]}. The scope of this evaluation includes:
\begin{itemize}
    \item A review of administrative and policy-based security controls via a questionnaire.
    \item A technical network scan of the organization's external-facing infrastructure.
    \item A correlation of findings to produce a prioritized list of risks and remediation steps.
\end{itemize}

The primary goal is to provide a clear and actionable summary of security weaknesses that require immediate attention.

% ===================================================================
\section{Organizational Information}
% ===================================================================

The following information was used as the basis for this assessment. Due to the anonymized nature of the provided data, placeholders have been used where necessary.

\begin{tabular}{@{}ll}
    \toprule
    \textbf{Attribute} & \textbf{Value} \\
    \midrule
    Organization Name & \textbf{[Organization Name]} \\
    Primary Domain & \texttt{[Domain]} \\
    External IP Scanned & \texttt{[Client IP]} \\
    \bottomrule
\end{tabular}

% ===================================================================
\section{Security Control Review}
% ===================================================================

An administrative control review was performed based on a security questionnaire. The responses indicate several critical gaps in foundational security practices. A "No" response highlights a missing control that significantly increases organizational risk.

\subsection{Questionnaire Responses}

\begin{table}[h!]
\centering
\caption{Security Controls Questionnaire Results}
\label{tab:controls}
\begin{tabular}{@{}p{0.7\linewidth}cc@{}}
\toprule
\textbf{Control Question} & \textbf{Response} & \textbf{Status} \\
\midrule
Do you require MFA to access email? & Yes & \ding{51} \\
Do you require MFA to log into computers? & No & \ding{55} \\
Do you require MFA to access sensitive data systems? & Yes & \ding{51} \\
Does your organization have an employee acceptable use policy? & No & \ding{55} \\
Does your organization do security awareness training for new employees? & Yes & \ding{51} \\
Does your organization do security awareness training for all employees at least once per year? & No & \ding{55} \\
\bottomrule
\end{tabular}
\end{table}

\subsection{Analysis of Control Gaps}
The review identified three major control deficiencies:
\begin{itemize}
    \item \textbf{Lack of Endpoint MFA:} While MFA is enforced for email and sensitive systems, the absence of MFA for computer logins leaves endpoints vulnerable. If an attacker compromises user credentials, they can gain direct access to a workstation, establishing a foothold within the network.
    \item \textbf{No Acceptable Use Policy (AUP):} An AUP is a foundational policy that defines the rules for using company IT assets. Without it, there is no formal guidance for employees, increasing the risk of insider threat, data misuse, and non-compliance.
    \item \textbf{Inadequate Security Training:} Security training for new hires is a good start, but it is insufficient. The threat landscape evolves continuously. The lack of mandatory, annual refresher training for all employees means the workforce is likely unprepared for modern phishing, social engineering, and ransomware attacks.
\end{itemize}

% ===================================================================
\section{Technical Scan Results}
% ===================================================================

A network scan was performed to identify open ports and services exposed on the organization's external infrastructure.

\subsection{Scan Details}
\begin{itemize}
    \item \textbf{Target IP Address:} \texttt{[Target IP]}
    \item \textbf{Scan Date:} 2025-11-22
\end{itemize}

\subsection{Open Ports and Services}

\begin{table}[h!]
\centering
\caption{Discovered Open Ports}
\label{tab:nmap}
\begin{tabular}{@{}lllll@{}}
\toprule
\textbf{Port} & \textbf{State} & \textbf{Service} & \textbf{Product} & \textbf{Version} \\
\midrule
443/tcp & open & https & nginx & 1.18.0 \\
\bottomrule
\end{tabular}
\end{table}

\subsection{Technical Analysis}
The scan identified an Nginx web server, version \textbf{1.18.0}, running on port 443. This version was released in April 2020 and is now considered outdated. It is known to be affected by several security vulnerabilities, including but not limited to CVE-2021-23017. Running outdated software on internet-facing systems presents a high risk of compromise, as attackers can exploit known vulnerabilities to gain unauthorized access.

% ===================================================================
\section{Consolidated Risk Assessment}
% ===================================================================

The following table synthesizes the findings from the administrative and technical reviews into a prioritized list of risks. Currently, there are no pre-existing risks on record.

\begin{table}[h!]
\centering
\caption{Identified Security Risks}
\label{tab:risks}
\begin{tabular}{@{}lp{0.3\linewidth}p{0.4\linewidth}l@{}}
\toprule
\textbf{ID} & \textbf{Risk Name} & \textbf{Description} & \textbf{Severity} \\
\midrule
RISK-001 & Outdated Web Server Software & The public-facing web server runs Nginx 1.18.0, which has known vulnerabilities that can be exploited remotely. & \textbf{High} \\
\addlinespace
RISK-002 & Lack of Endpoint MFA & No MFA is required for computer logins, exposing endpoints to takeover if credentials are stolen. & \textbf{High} \\
\addlinespace
RISK-003 & Inadequate Security Awareness Program & Lack of annual training for all staff increases susceptibility to phishing and social engineering attacks. & \textbf{High} \\
\addlinespace
RISK-004 & Missing Acceptable Use Policy & The absence of a formal AUP creates ambiguity and increases the risk of misuse of corporate assets. & \textbf{Medium} \\
\bottomrule
\end{tabular}
\end{table}

% ===================================================================
\section{Recommendations}
% ===================================================================

To address the identified risks and strengthen the organization's security posture, the following actions are recommended with high priority.

\begin{enumerate}
    \item \textbf{Upgrade Web Server Software (RISK-001):}
    \begin{itemize}
        \item Plan and execute an upgrade of the Nginx server on host \texttt{[Target IP]} from version 1.18.0 to the latest stable version.
        \item Implement a patch management policy to ensure all internet-facing systems are updated in a timely manner.
    \end{itemize}

    \item \textbf{Implement Endpoint MFA (RISK-002):}
    \begin{itemize}
        \item Deploy and enforce a multi-factor authentication solution for all employee computer and laptop logins (e.g., Windows Hello, Duo, etc.).
        \item Prioritize deployment for privileged users, such as administrators and executives.
    \end{itemize}

    \item \textbf{Establish Annual Security Training (RISK-003):}
    \begin{itemize}
        \item Develop or procure a security awareness training program that is mandatory for all employees on an annual basis.
        \item The training should cover modern threats such as phishing, ransomware, and proper data handling.
        \item Conduct periodic phishing simulations to test and reinforce the training.
    \end{itemize}
    
    \item \textbf{Develop and Implement an AUP (RISK-004):}
    \begin{itemize}
        \item Create a formal Acceptable Use Policy that clearly outlines the rules for using company networks, devices, and data.
        \item Ensure all current and new employees read and acknowledge the policy.
    \end{itemize}
\end{enumerate}

\end{document}
```