```latex
\documentclass[12pt]{article}

% Preamble: Required Packages
\usepackage[margin=1in]{geometry}
\usepackage{pifont}                 % For symbols like checkmarks (\ding{51}) and crosses (\ding{55})
\usepackage{booktabs}               % For professional-looking tables
\usepackage{hyperref}               % For clickable links and references
\usepackage{url}                    % For formatting URLs
\usepackage{seqsplit}               % To split long strings without breaking
\usepackage{xcolor}                 % For custom colors
\usepackage{fancyhdr}               % For custom headers and footers

% --- Document Configuration ---

% Color definitions
\definecolor{darkblue}{rgb}{0.0, 0.0, 0.55}
\definecolor{darkred}{rgb}{0.55, 0.0, 0.0}

% Hyperref setup for better navigation and appearance
\hypersetup{
    colorlinks=true,
    linkcolor=darkblue,
    urlcolor=darkblue,
    citecolor=darkblue,
    pdftitle={Cybersecurity Posture Assessment Report},
    pdfauthor={Cybersecurity Analysis Division},
    pdfsubject={Security Assessment}
}

% Header and Footer Configuration
\pagestyle{fancy}
\fancyhf{} % Clear all header and footer fields
\lhead{Cybersecurity Assessment Report}
\rhead{\textbf{[Organization Name]}}
\cfoot{\thepage}
\renewcommand{\headrulewidth}{0.4pt}
\renewcommand{\footrulewidth}{0.4pt}

% --- Document Start ---
\begin{document}

\title{
    \vspace{-1.5cm} % Adjust title position
    \textbf{Cybersecurity Posture Assessment Report}\\
    \large For: \textbf{[Organization Name]}
}
\author{Cybersecurity Analysis Division}
\date{\today}

\maketitle
\thispagestyle{fancy} % Apply fancy style to the first page as well

\begin{abstract}
\noindent This report provides a comprehensive analysis of the cybersecurity posture for \textbf{[Organization Name]}. The assessment is based on a synthesis of technical network scans, a review of organizational security controls, and an evaluation of pre-existing risk data. The analysis identified critical gaps in identity and access management, specifically the lack of Multi-Factor Authentication (MFA) for email and sensitive systems. Furthermore, foundational policy and training deficiencies were noted. On a positive note, technical scanning indicates that a previously identified risk related to an open web port has been successfully remediated. This report outlines these findings in detail and provides actionable recommendations to mitigate identified risks and strengthen the overall security posture.
\end{abstract}

\tableofcontents
\newpage

% ==============================================================================
\section{Overview and Scope}
% ==============================================================================

This assessment aims to provide a clear and concise snapshot of the organization's current security state. The scope of this analysis includes:
\begin{itemize}
    \item \textbf{Organizational Controls:} A review of self-reported security practices via a standardized questionnaire.
    \item \textbf{Technical Perimeter Scan:} An external network scan to identify open ports and exposed services.
    \item \textbf{Risk Correlation:} A synthesis of the above data points with previously documented risks to create a unified view of the threat landscape.
\end{itemize}

% ==============================================================================
\section{Organizational Information}
% ==============================================================================

The following information was used as the basis for this assessment. As per the template mode, placeholders are used where data was not provided.

\begin{tabular}{@{}ll}
    \toprule
    \textbf{Attribute} & \textbf{Value} \\
    \midrule
    Organization Name & \textbf{[Organization Name]} \\
    Primary Domain & \texttt{[Domain]} \\
    External IP Scanned & \texttt{[Client IP]} \\
    Target IP in Scan & \texttt{[Target IP]} \\
    Scan Date & \today \\
    \bottomrule
\end{tabular}

% ==============================================================================
\section{Security Control Review}
% ==============================================================================

The following table details the responses from the organizational security questionnaire. "No" answers indicate significant gaps in the security framework and are flagged as high-priority areas for remediation.

\begin{table}[h!]
\centering
\caption{Organizational Security Control Questionnaire Results}
\begin{tabular}{@{}p{0.6\linewidth} c p{0.2\linewidth}@{}}
    \toprule
    \textbf{Control Question} & \textbf{Response} & \textbf{Analyst Assessment} \\
    \midrule
    Do you require MFA to access email? & 
    \textcolor{darkred}{\ding{55}} & \textcolor{darkred}{\textbf{Critical Gap}} \\
    
    Do you require MFA to log into computers? & 
    \textcolor{darkgreen}{\ding{51}} & Good Practice \\
    
    Do you require MFA to access sensitive data systems? & 
    \textcolor{darkred}{\ding{55}} & \textcolor{darkred}{\textbf{Critical Gap}} \\
    
    Does your organization have an employee acceptable use policy? & 
    \textcolor{darkred}{\ding{55}} & \textcolor{darkred}{High Risk} \\
    
    Does your organization do security awareness training for new employees? & 
    \textcolor{darkgreen}{\ding{51}} & Good Practice \\
    
    Does your organization do security awareness training for all employees at least once per year? & 
    \textcolor{darkred}{\ding{55}} & \textcolor{darkred}{High Risk} \\
    \bottomrule
\end{tabular}
\end{table}

\subsection*{Analysis of Control Gaps}
The lack of enforced MFA on email and sensitive data systems represents a critical vulnerability. Email is a primary vector for phishing and account takeover attacks, which can lead to widespread compromise. Similarly, sensitive data systems without MFA are prime targets for unauthorized access. The absence of an Acceptable Use Policy (AUP) and annual security training creates a high-risk environment where employees may be unaware of their security responsibilities, increasing the likelihood of human error leading to a security incident.

% ==============================================================================
\section{Technical Scan Results}
% ==============================================================================

An external network scan was performed against the target IP address to identify listening services.

\subsection*{Scan Summary}
\begin{itemize}
    \item \textbf{Target IP:} \texttt{[Target IP]}
    \item \textbf{Host Status:} UP
    \item \textbf{Key Finding:} The scan confirmed that \textbf{port 80 (HTTP) is closed}. 
\end{itemize}

\subsection*{Analyst Notes}
The finding that port 80 is closed is a significant positive development. It indicates that the external firewall is properly configured to prevent unencrypted web traffic, mitigating risks such as man-in-the-middle attacks. This technical result directly contradicts a pre-existing risk entry (see Section 5), suggesting that successful remediation has occurred since the last assessment. No other open ports were discovered during this scan.

% ==============================================================================
\section{Consolidated Risk Assessment}
% ==============================================================================

The following table synthesizes findings from the security control review, technical scan, and pre-existing risk data into a prioritized list.

\begin{table}[h!]
\centering
\caption{Prioritized Risk Register}
\begin{tabular}{@{}p{0.4\linewidth} p{0.15\linewidth} p{0.35\linewidth}@{}}
    \toprule
    \textbf{Risk Description} & \textbf{Severity} & \textbf{Affected Elements / Notes} \\
    \midrule
    \textbf{No MFA for Email Access} & \textbf{Critical} & All user accounts with email access. Increases risk of business email compromise (BEC) and phishing success. \\
    \addlinespace
    \textbf{No MFA for Sensitive Data Systems} & \textbf{Critical} & All systems housing sensitive or critical data. Increases risk of data breach and unauthorized access. \\
    \addlinespace
    \textbf{Lack of Annual Security Awareness Training} & High & All employees. A stale security culture increases susceptibility to social engineering and policy violations. \\
    \addlinespace
    \textbf{No Employee Acceptable Use Policy (AUP)} & High & Entire organization. Lack of a formal policy creates ambiguity regarding proper use of company assets and data handling. \\
    \addlinespace
    Unencrypted Web Server (Port 80) & Informational & External IP: \texttt{[Client IP]}. \textbf{Status: Remediated.} The current scan confirms port 80 is closed, resolving this risk. \\
    \bottomrule
\end{tabular}
\end{table}

% ==============================================================================
\section{Recommendations}
% ==============================================================================

Based on the consolidated risk assessment, the following actions are recommended to improve the security posture of \textbf{[Organization Name]}. Recommendations are prioritized by severity.

\begin{enumerate}
    \item \textbf{[Critical] Enforce MFA for All Email and Sensitive Systems:}
    \begin{itemize}
        \item \textbf{Immediate Action:} Enable and enforce MFA for all users on the primary email platform (e.g., Microsoft 365, Google Workspace).
        \item \textbf{Short-Term Action:} Identify all systems classified as containing sensitive data and develop a phased rollout plan to enforce MFA for access within the next 60 days.
    \end{itemize}

    \item \textbf{[High] Develop and Implement Foundational Security Policies:}
    \begin{itemize}
        \item \textbf{Immediate Action:} Draft and ratify an official Employee Acceptable Use Policy (AUP). This policy should be distributed to all employees, and acknowledgement of receipt should be tracked.
    \end{itemize}

    \item \textbf{[High] Establish a Continuous Security Awareness Program:}
    \begin{itemize}
        \item \textbf{Short-Term Action:} Procure and implement a security awareness training platform. Ensure that all employees complete a baseline training module within the next 90 days.
        \item \textbf{Long-Term Action:} Establish a formal, annual security awareness training program for all staff to ensure continuous education on emerging threats.
    \end{itemize}

    \item \textbf{[Informational] Validate and Close Previous Risks:}
    \begin{itemize}
        \item \textbf{Immediate Action:} Formally update the internal risk register to mark the "Unencrypted Web Server" risk as resolved. Continue periodic external scanning to ensure the control remains effective.
    \end{itemize}
\end{enumerate}

\end{document}
```