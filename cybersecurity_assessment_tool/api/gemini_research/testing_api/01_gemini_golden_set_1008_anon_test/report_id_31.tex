```latex
\documentclass[12pt]{article}

% Preamble: Required Packages
\usepackage[margin=1in]{geometry}
\usepackage{pifont} % For checkmarks and crosses
\usepackage{booktabs} % For professional tables
\usepackage{hyperref} % For clickable links
\usepackage{url}      % For URL formatting
\usepackage{seqsplit} % For splitting long strings in texttt
\usepackage[utf8]{inputenc}

% Document Metadata
\title{Cybersecurity Posture Assessment Report}
\author{Cybersecurity Analyst}
\date{\today}

% Hyperref Setup
\hypersetup{
    colorlinks=true,
    linkcolor=black,
    filecolor=magenta,      
    urlcolor=blue,
    pdftitle={Cybersecurity Posture Assessment Report},
    pdfpagemode=FullScreen,
}

\begin{document}

\maketitle
\thispagestyle{empty}
\newpage
\tableofcontents
\newpage

% --- 1. Executive Summary ---
\section{Executive Summary}

This report details the findings of a cybersecurity assessment for \textbf{[Organization Name]}. The analysis is based on a network scan, a review of existing risks, and a security controls questionnaire.

The assessment has identified several critical and high-risk security gaps that require immediate attention. The most significant finding is the systemic lack of Multi-Factor Authentication (MFA) across all key systems, including email, computer logins, and sensitive data repositories. This exposes the organization to a high risk of account takeover and subsequent data breaches.

Furthermore, the absence of a structured security awareness training program for both new and existing employees significantly increases the organization's susceptibility to social engineering and phishing attacks. Technical findings revealed an exposed Secure Shell (SSH) service on the network perimeter, which, if not properly hardened, could serve as an entry point for attackers.

Immediate remediation should focus on implementing MFA, establishing a comprehensive security awareness program, and hardening externally-facing services.

% --- 2. Organizational Information ---
\section{Organizational Information}

This section provides the high-level details of the organization under review. The data has been anonymized for this report.

\begin{itemize}
    \item \textbf{Organization Name:} \textbf{[Organization Name]}
    \item \textbf{Primary Email Domain:} \texttt{[Domain]}
    \item \textbf{External IP Address Scanned:} \texttt{[Client IP]}
\end{itemize}

% --- 3. Security Control Review ---
\section{Security Control Review}

A security controls questionnaire was completed to evaluate existing administrative and procedural safeguards. The results indicate significant gaps in identity and access management and employee security training. "No" answers represent a failure of a key security control and are flagged as risks.

\begin{table}[h!]
\centering
\caption{Security Controls Questionnaire Results}
\begin{tabular}{p{0.6\linewidth} c l}
\toprule
\textbf{Control Question} & \textbf{Response} & \textbf{Assessment} \\
\midrule
Do you require MFA to access email? & \ding{55} & Critical Gap \\
Do you require MFA to log into computers? & \ding{55} & Critical Gap \\
Do you require MFA to access sensitive data systems? & \ding{55} & Critical Gap \\
Does your organization do security awareness training for new employees? & \ding{55} & High Risk \\
Does your organization do security awareness training for all employees at least once per year? & \ding{55} & High Risk \\
Does your organization have an employee acceptable use policy? & \ding{51} & Control in Place \\
\bottomrule
\end{tabular}
\end{table}

% --- 4. Technical Scan Results ---
\section{Technical Scan Results}

An external network scan was performed to identify open ports and exposed services on the organization's perimeter.

\begin{itemize}
    \item \textbf{Target IP Address:} \texttt{[Target IP]}
    \item \textbf{Scan Date:} Scan date not provided in source data.
\end{itemize}

The scan identified the following open port. Exposing management services like SSH to the public internet increases the attack surface and requires robust security configurations.

\begin{table}[h!]
\centering
\caption{Open Ports Detected on \texttt{[Target IP]}}
\begin{tabular}{c c c p{0.4\linewidth}}
\toprule
\textbf{Port} & \textbf{State} & \textbf{Service (Inferred)} & \textbf{Notes} \\
\midrule
22/tcp & open & SSH & Service version information was not available. Exposed SSH is a common target for brute-force attacks. Access should be restricted and hardened. \\
\bottomrule
\end{tabular}
\end{table}

% --- 5. Risk Assessment Summary ---
\section{Risk Assessment Summary}

This section synthesizes findings from the security control review, technical scan, and pre-existing risk data. The following risks have been identified and prioritized based on their potential impact on the organization. No pre-existing vulnerabilities were reported in the input data.

\begin{table}[h!]
\centering
\caption{Identified Risks}
\begin{tabular}{p{0.1\linewidth} p{0.25\linewidth} p{0.4\linewidth} c}
\toprule
\textbf{Risk ID} & \textbf{Risk Name} & \textbf{Description} & \textbf{Severity} \\
\midrule
RISK-001 & No Multi-Factor Authentication (MFA) & The absence of MFA for email, computers, and sensitive data systems allows an attacker with stolen credentials to gain unauthorized access. & \textbf{Critical} \\
\addlinespace
RISK-002 & Inadequate Security Awareness Training & Employees are not trained to recognize or respond to phishing, social engineering, or other common cyber threats, making them a vulnerable entry point. & \textbf{High} \\
\addlinespace
RISK-003 & Exposed SSH Management Port & Port 22 (SSH) is open to the internet. Without proper hardening (e.g., key-only auth, IP filtering), it is vulnerable to brute-force and credential stuffing attacks. & \textbf{Medium} \\
\bottomrule
\end{tabular}
\end{table}

% --- 6. Recommendations ---
\section{Recommendations}

The following prioritized recommendations are provided to mitigate the identified risks and improve the overall security posture of \textbf{[Organization Name]}.

\subsection*{Priority 1: Implement Multi-Factor Authentication (Critical)}
\begin{itemize}
    \item \textbf{Action:} Deploy a robust MFA solution across the entire organization.
    \item \textbf{Details:} Prioritize enabling MFA on all email accounts (e.g., Office 365, Google Workspace) and any systems hosting sensitive data. Subsequently, enforce MFA for all remote access (VPN) and local computer logins.
    \item \textbf{Risk Mitigated:} RISK-001
\end{itemize}

\subsection*{Priority 2: Establish a Security Awareness Program (High)}
\begin{itemize}
    \item \textbf{Action:} Procure and implement a security awareness training platform.
    \item \textbf{Details:} Mandate foundational training for all new employees during onboarding. Require all staff to complete annual refresher training and conduct regular phishing simulations to reinforce learning and measure effectiveness.
    \item \textbf{Risk Mitigated:} RISK-002
\end{itemize}

\subsection*{Priority 3: Harden External Services (Medium)}
\begin{itemize}
    \item \textbf{Action:} Secure the exposed SSH service on \texttt{[Target IP]}.
    \item \textbf{Details:} If SSH access is required from the internet, restrict access to a whitelist of trusted IP addresses. Disable password-based authentication and enforce the use of strong SSH keys. Implement an intrusion prevention tool like Fail2ban to automatically block malicious login attempts.
    \item \textbf{Risk Mitigated:} RISK-003
\end{itemize}

% --- 7. Conclusion ---
\section{Conclusion}

The assessment reveals that while \textbf{[Organization Name]} has foundational policies in place, there are critical deficiencies in technical and administrative controls. The lack of MFA and security training represents an unacceptable level of risk. We strongly advise the organization to prioritize the implementation of the recommendations outlined in this report to significantly strengthen its defenses against common cyber threats.

\end{document}
```