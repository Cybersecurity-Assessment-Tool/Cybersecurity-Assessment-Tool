```latex
\documentclass[12pt]{article}

% Preamble: Required Packages
\usepackage[margin=1in]{geometry}
\usepackage{pifont} % For checkmarks and crosses
\usepackage{booktabs} % For professional tables
\usepackage{hyperref} % For clickable links and references
\usepackage{url} % For formatting URLs
\usepackage{seqsplit} % For splitting long strings without spaces
\usepackage{xcolor} % For colors
\usepackage{graphicx} % For logos, etc.
\usepackage{fancyhdr} % For headers and footers

% --- Document Setup ---

% Define colors for risk levels
\definecolor{critical}{HTML}{990000}
\definecolor{high}{HTML}{D14124}
\definecolor{medium}{HTML}{E89923}
\definecolor{low}{HTML}{3A7D44}

% Hyperref setup
\hypersetup{
    colorlinks=true,
    linkcolor=blue,
    filecolor=magenta,      
    urlcolor=cyan,
    pdftitle={Cybersecurity Assessment Report},
    pdfpagemode=FullScreen,
}

% Header and Footer
\pagestyle{fancy}
\fancyhf{} % Clear all header and footer fields
\fancyhead[L]{Cybersecurity Assessment Report}
\fancyhead[R]{\textbf{[Organization Name]}}
\fancyfoot[C]{\thepage}
\renewcommand{\headrulewidth}{0.4pt}
\renewcommand{\footrulewidth}{0.4pt}

% --- Document Start ---

\begin{document}

% --- Title Page ---
\begin{titlepage}
    \centering
    \vspace*{1cm}
    \Huge \textbf{Cybersecurity Assessment Report}
    \vspace{1.5cm}
    
    \Large Prepared for: \\
    \vspace{0.5cm}
    \textbf{[Organization Name]}
    
    \vspace{2cm}
    
    \Large Prepared by: \\
    \vspace{0.5cm}
    Cybersecurity Analyst
    
    \vfill
    
    \Large \today
\end{titlepage}

\tableofcontents
\newpage

% --- Section 1: Executive Summary ---
\section{Executive Summary}
This report details the findings of a cybersecurity assessment conducted for \textbf{[Organization Name]}. The assessment combined a technical network scan, a review of organizational security controls, and an analysis of pre-existing risks.

The overall security posture is considered critically weak and requires immediate remediation. Several high-impact vulnerabilities were discovered that expose the organization to significant threats, including data breaches, unauthorized access, and malware infection.

\textbf{Key Critical Findings Include:}
\begin{itemize}
    \item \textbf{Exposed Vulnerable FTP Server:} A publicly accessible FTP server (\texttt{[Target IP]}) is running a dangerously outdated version of vsftpd (2.3.4), which is known to contain a critical backdoor vulnerability (CVE-2011-2523). Furthermore, the server is configured to allow anonymous logins, permitting unauthenticated access to files.
    \item \textbf{Lack of Multi-Factor Authentication (MFA) on Email:} The organization does not require MFA for email access. This represents a severe security gap, as email accounts are a primary target for attackers seeking to gain an initial foothold in a network.
    \item \textbf{Insufficient Security Training:} While new employees receive training, there is no mandatory annual security awareness training for all staff. This increases the organization's susceptibility to social engineering and phishing attacks over time.
\end{itemize}

This report provides a detailed breakdown of all identified risks and offers actionable recommendations to mitigate them. We urge management to prioritize the remediation of the critical findings outlined herein.

% --- Section 2: Organizational Information ---
\section{Organizational Information}
The following information was used as the basis for this assessment. Note that where data was not provided, placeholders have been used.

\begin{table}[h!]
\centering
\begin{tabular}{@{}ll@{}}
\toprule
\textbf{Attribute} & \textbf{Value} \\ \midrule
Organization Name    & \textbf{[Organization Name]} \\
Primary Email Domain & \texttt{[Domain]} \\
External IP Address  & \texttt{[Client IP]} \\ \bottomrule
\end{tabular}
\caption{Client Organizational Data}
\end{table}

% --- Section 3: Security Control Review ---
\section{Security Control Review}
A review of organizational security controls was conducted based on a standardized questionnaire. The responses indicate significant gaps in the security framework, particularly concerning access control and ongoing employee education.

\begin{table}[h!]
\centering
\begin{tabular}{@{}p{0.75\textwidth}c@{}}
\toprule
\textbf{Control Question} & \textbf{Response} \\ \midrule
Do you require MFA to access email? & \textcolor{red}{\ding{55}} \\
Do you require MFA to log into computers? & \textcolor{green}{\ding{51}} \\
Do you require MFA to access sensitive data systems? & \textcolor{green}{\ding{51}} \\
Does your organization have an employee acceptable use policy? & \textcolor{green}{\ding{51}} \\
Does your organization do security awareness training for new employees? & \textcolor{green}{\ding{51}} \\
Does your organization do security awareness training for all employees at least once per year? & \textcolor{red}{\ding{55}} \\ \bottomrule
\end{tabular}
\caption{Security Controls Questionnaire Results}
\end{table}

\subsection*{Analysis of Gaps}
The two "No" responses are major points of concern:
\begin{itemize}
    \item \textbf{No MFA for Email:} Email is the gateway to an organization's data and services. Without MFA, a compromised password is all an attacker needs to gain full access to an employee's mailbox, leading to data exfiltration, phishing of other employees, and password resets for other services.
    \item \textbf{No Annual Security Training:} The threat landscape evolves constantly. Failing to provide regular, updated training leaves employees vulnerable to modern attack techniques, undermining other security investments.
\end{itemize}

% --- Section 4: Technical Scan Results ---
\section{Technical Scan Results}
An external network scan was performed to identify exposed services and potential vulnerabilities.

\begin{itemize}
    \item \textbf{Target IP Address:} \texttt{[Target IP]}
    \item \textbf{Scan Date:} \today
\end{itemize}

The scan revealed one open port with a critical misconfiguration.

\begin{table}[h!]
\centering
\begin{tabular}{@{}llllll@{}}
\toprule
\textbf{Port} & \textbf{State} & \textbf{Service} & \textbf{Product} & \textbf{Version} & \textbf{Notes} \\ \midrule
21/tcp & Open & ftp & vsftpd & 2.3.4 & \begin{tabular}[c]{@{}l@{}}Anonymous login allowed.\\ \textbf{CRITICAL:} Known backdoor\\ vulnerability (CVE-2011-2523).\end{tabular} \\ \bottomrule
\end{tabular}
\caption{Open Port Analysis}
\end{table}

\subsection*{Analysis of Technical Findings}
The presence of an open FTP port running \textbf{vsftpd version 2.3.4} is a severe and immediate threat. This specific version was compromised in 2011, and a malicious backdoor was inserted into the source code. An attacker can gain a command shell on the server by sending a specific sequence of characters as the username. Compounding this, the configuration allowing \textbf{anonymous FTP login} enables any unauthenticated user on the internet to access, upload, or download files, making it a prime target for data theft or for use as a malware distribution point.

% --- Section 5: Consolidated Risk Assessment ---
\section{Consolidated Risk Assessment}
The following table synthesizes findings from the technical scan, control review, and pre-existing risk data into a consolidated list. Risks are prioritized by severity to guide remediation efforts.

\begin{table}[h!]
\centering
\begin{tabular}{@{}lp{0.5\textwidth}l@{}}
\toprule
\textbf{Risk ID} & \textbf{Risk Description} & \textbf{Severity} \\ \midrule
RISK-001 & \textbf{Vulnerable FTP Server with Anonymous Access:} An outdated FTP service (vsftpd 2.3.4) with a known backdoor (CVE-2011-2523) is exposed to the internet. Anonymous login is enabled. & \textcolor{critical}{\textbf{Critical}} \\
\addlinespace
RISK-002 & \textbf{No MFA for Email:} Email accounts are secured only by passwords, making them highly susceptible to takeover via phishing or credential stuffing attacks. & \textcolor{critical}{\textbf{Critical}} \\
\addlinespace
RISK-003 & \textbf{Lack of Annual Security Training:} Employees do not receive recurring security awareness training, leading to skill decay and increased susceptibility to social engineering. & \textcolor{high}{\textbf{High}} \\
\addlinespace
RISK-004 & \textbf{Outdated Windows Policy:} Workstations are running Windows 7, an end-of-life operating system that no longer receives security updates, leaving them vulnerable to exploitation. & \textcolor{medium}{\textbf{Medium}} \\ \bottomrule
\end{tabular}
\caption{Consolidated Risk Register}
\end{table}

% --- Section 6: Recommendations ---
\section{Recommendations}
The following actions are recommended to mitigate the identified risks. They are ordered by priority, addressing critical risks first.

\subsection{RISK-001: Remediate Vulnerable FTP Server (Immediate)}
\begin{itemize}
    \item \textbf{Immediate Action:} Take the FTP server offline immediately by stopping the vsftpd service or blocking port 21 at the firewall.
    \item \textbf{Short-Term Fix:} If FTP is a business necessity, upgrade the vsftpd package to the latest stable version and explicitly disable anonymous access in its configuration file.
    \item \textbf{Long-Term Solution:} Decommission the FTP service entirely. Migrate all file transfer workflows to a secure protocol such as SFTP (SSH File Transfer Protocol) or a managed file transfer solution.
\end{itemize}

\subsection{RISK-002: Implement MFA for Email (Immediate)}
\begin{itemize}
    \item \textbf{Immediate Action:} Procure and enable MFA for your email provider (e.g., Microsoft 365, Google Workspace).
    \item \textbf{Phased Rollout:} Begin with a pilot group of IT and executive staff, then enforce MFA for all remaining employees within 30 days.
    \item \textbf{Policy Update:} Update the organization's access control policy to mandate MFA for all externally-facing services.
\end{itemize}

\subsection{RISK-003: Establish Annual Security Training (High Priority)}
\begin{itemize}
    \item \textbf{Action:} Procure a security awareness training platform or service.
    \item \textbf{Implementation:} Develop a mandatory annual training curriculum covering key topics like phishing, password security, and acceptable use.
    \item \textbf{Tracking:} Implement a system to track completion and ensure 100\% of employees complete the training each year.
\end{itemize}

\subsection{RISK-004: Address Outdated Windows Policy (Medium Priority)}
\begin{itemize}
    \item \textbf{Action:} Reiterate and accelerate the existing plan to upgrade all Windows 7 workstations.
    \item \textbf{Short-Term Mitigation:} Until upgrades are complete, ensure these machines are isolated on the network as much as possible and have endpoint protection installed.
    \item \textbf{Long-Term Solution:} Develop a formal hardware and software lifecycle management policy to prevent operating systems from becoming unsupported in the future.
\end{itemize}

\end{document}
```