```latex
\documentclass[12pt]{article}

% --- PACKAGES ---
\usepackage[margin=1in]{geometry}
\usepackage{pifont} % For checkmarks and crosses
\usepackage{booktabs} % For professional tables
\usepackage{hyperref} % For clickable links
\usepackage{url} % For formatting URLs
\usepackage{seqsplit} % For splitting long strings in texttt
\usepackage{xcolor} % For colors

% --- DOCUMENT METADATA ---
\title{Cybersecurity Posture Assessment Report}
\author{Cybersecurity Analyst}
\date{\today}

% --- HYPERREF SETUP ---
\hypersetup{
    colorlinks=true,
    linkcolor=blue,
    filecolor=magenta,      
    urlcolor=cyan,
    pdftitle={Cybersecurity Posture Assessment Report},
    pdfpagemode=FullScreen,
}

% --- COMMANDS ---
\newcommand{\yes}{\ding{51}}
\newcommand{\no}{\ding{55}}

% --- DOCUMENT START ---
\begin{document}

\maketitle
\thispagestyle{empty}
\newpage

\tableofcontents
\newpage

% ===================================================================
\section{Executive Summary}
% ===================================================================

This report provides a cybersecurity assessment for \textbf{[Organization Name]}, based on a review of organizational security controls, an external network scan, and an analysis of known risks.

The assessment reveals a mixed security posture. The organization demonstrates strong identity and access management practices by enforcing Multi-Factor Authentication (MFA) across key systems. However, two critical gaps were identified in the security awareness program. The complete absence of security training for both new and existing employees represents a \textbf{High} risk, as it leaves the organization vulnerable to social engineering and phishing attacks.

From a technical standpoint, an external scan identified an open administrative port (SSH on port 22). While necessary for remote management, its exposure to the public internet creates a significant attack vector if not properly hardened. The combination of this technical exposure and the lack of employee security training elevates the overall risk profile.

This report outlines these findings in detail and provides actionable recommendations to mitigate the identified risks, prioritizing the implementation of a comprehensive security awareness training program and the hardening of the exposed SSH service.

% ===================================================================
\section{Organizational Information}
% ===================================================================

The following information was used as the basis for this assessment. Due to the anonymized nature of the provided data, placeholders are used where necessary.

\begin{itemize}
    \item \textbf{Organization Name:} \textbf{[Organization Name]}
    \item \textbf{Primary Domain:} \texttt{[Domain]}
    \item \textbf{Scanned Public IP:} \texttt{[Client IP]}
\end{itemize}

% ===================================================================
\section{Security Control Review}
% ===================================================================

A review of the organization's security controls was conducted via a standardized questionnaire. The responses indicate strong controls in identity management but critical deficiencies in security awareness training. "No" answers highlight areas requiring immediate attention.

\begin{table}[h!]
\centering
\caption{Security Controls Questionnaire Results}
\label{tab:controls}
\begin{tabular}{p{0.75\linewidth} c}
\toprule
\textbf{Control Question} & \textbf{Response} \\
\midrule
Do you require MFA to access email? & \yes \\
Do you require MFA to log into computers? & \yes \\
Do you require MFA to access sensitive data systems? & \yes \\
Does your organization have an employee acceptable use policy? & \yes \\
\midrule
\textcolor{red}{Does your organization do security awareness training for new employees?} & \textcolor{red}{\no} \\
\textcolor{red}{Does your organization do security awareness training for all employees at least once per year?} & \textcolor{red}{\no} \\
\bottomrule
\end{tabular}
\end{table}

% ===================================================================
\section{Technical Scan Results}
% ===================================================================

An external network scan was performed on the organization's provided public IP address. The scan identified the following open ports.

\begin{itemize}
    \item \textbf{Target IP Address:} \texttt{[Target IP]}
    \item \textbf{Scan Date:} Not provided in scan data. Report generated on \today.
\end{itemize}

\begin{table}[h!]
\centering
\caption{Open Port Analysis}
\label{tab:ports}
\begin{tabular}{llll}
\toprule
\textbf{Port} & \textbf{State} & \textbf{Service (Inferred)} & \textbf{Product/Version} \\
\midrule
22 & open & SSH & Not Available \\
\bottomrule
\end{tabular}
\end{table}

\subsection{Analysis of Findings}
The scan revealed that port 22, the standard port for the Secure Shell (SSH) protocol, is open to the internet. SSH is a critical tool for remote server administration. However, its public exposure makes it a prime target for attackers who may attempt to gain unauthorized access through various methods, including:
\begin{itemize}
    \item \textbf{Brute-force attacks:} Automated attempts to guess usernames and passwords.
    \item \textbf{Credential stuffing:} Using credentials stolen from other data breaches.
    \item \textbf{Exploitation of vulnerabilities:} Targeting outdated or misconfigured SSH server software.
\end{itemize}
The scan did not retrieve version information, so it is not possible to determine if the running SSH service is vulnerable to any known exploits. This finding is classified as a \textbf{Medium} risk, with the potential to become High if not properly configured.

% ===================================================================
\section{Consolidated Risk Assessment}
% ===================================================================

The following table synthesizes findings from the security control review, technical scan, and pre-existing risk data. The pre-existing risk log was empty, so all identified risks are new findings from this assessment.

\begin{table}[h!]
\centering
\caption{Summary of Identified Risks}
\label{tab:risks}
\begin{tabular}{p{0.1\linewidth} p{0.25\linewidth} p{0.45\linewidth} p{0.1\linewidth}}
\toprule
\textbf{ID} & \textbf{Risk Name} & \textbf{Description} & \textbf{Severity} \\
\midrule
\textbf{RISK-01} & Lack of Security Awareness Training & The absence of a formal training program for new and existing employees increases susceptibility to phishing, social engineering, and policy violations. This represents a significant gap in the human element of defense. & \textbf{High} \\
\addlinespace
\textbf{RISK-02} & Exposed Administrative Service (SSH) & Port 22 (SSH) is open to the public internet. If not hardened, this service is a primary target for brute-force attacks and exploitation, potentially leading to a full system compromise. & \textbf{Medium} \\
\bottomrule
\end{tabular}
\end{table}

% ===================================================================
\section{Recommendations}
% ===================================================================

The following actionable recommendations are provided to mitigate the risks identified in this report. They are prioritized based on severity and potential impact.

\subsection{High Priority}
\begin{description}
    \item[R-01: Implement Security Awareness Training Program (Mitigates RISK-01)] \\
    Immediately develop and deploy a mandatory security awareness training program.
    \begin{itemize}
        \item \textbf{Onboarding:} All new employees must complete foundational training before being granted access to systems.
        \item \textbf{Annual Refresher:} All staff must complete an annual training course covering current threats such as phishing, ransomware, and proper data handling.
        \item \textbf{Phishing Simulations:} Conduct regular, simulated phishing campaigns to test and reinforce employee vigilance.
    \end{itemize}
\end{description}

\subsection{Medium Priority}
\begin{description}
    \item[R-02: Harden Exposed SSH Service (Mitigates RISK-02)] \\
    Apply security best practices to the SSH server running on \texttt{[Target IP]}.
    \begin{itemize}
        \item \textbf{Disable Password Authentication:} Enforce the use of cryptographic keys (public key authentication) for all SSH access and disable direct password logins.
        \item \textbf{Restrict Access:} Configure firewall rules to allow SSH connections only from trusted, whitelisted IP addresses (e.g., corporate offices, VPN).
        \item \textbf{Implement Rate Limiting:} Use tools like \texttt{fail2ban} to automatically block IP addresses that generate multiple failed login attempts.
        \item \textbf{Perform Vulnerability Scan:} Conduct an authenticated, in-depth vulnerability scan of the host to identify the specific SSH version and check for known vulnerabilities.
    \end{itemize}
\end{description}

\end{document}
```