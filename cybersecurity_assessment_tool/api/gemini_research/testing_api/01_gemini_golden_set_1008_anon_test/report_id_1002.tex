```latex
\documentclass[12pt]{article}

% Preamble: Required Packages
\usepackage[margin=1in]{geometry}
\usepackage{pifont}
\usepackage{booktabs}
\usepackage{hyperref}
\usepackage{url}
\usepackage{seqsplit}
\usepackage{graphicx}
\usepackage{array}

% Document Metadata
\title{Cybersecurity Posture Assessment Report}
\author{Cybersecurity Analyst}
\date{\today}

% Hyperref Setup
\hypersetup{
    colorlinks=true,
    linkcolor=black,
    urlcolor=blue,
    pdftitle={Cybersecurity Posture Assessment Report},
    pdfauthor={Cybersecurity Analyst},
}

% Custom command for table cell formatting
\newcolumntype{L}[1]{>{\raggedright\let\newline\\\arraybackslash\hspace{0pt}}m{#1}}
\newcolumntype{C}[1]{>{\centering\let\newline\\\arraybackslash\hspace{0pt}}m{#1}}

\begin{document}

\maketitle
\thispagestyle{empty}
\newpage

\tableofcontents
\newpage

% --- 1. Executive Summary ---
\section{Executive Summary}
This report provides a comprehensive cybersecurity assessment for \textbf{[Organization Name]}, synthesizing data from a network vulnerability scan, a security controls questionnaire, and a review of pre-existing risks. The analysis was conducted to identify security gaps, evaluate the current risk posture, and provide actionable recommendations for remediation.

The assessment revealed several critical administrative control deficiencies, primarily a widespread lack of Multi-Factor Authentication (MFA) across email, workstations, and sensitive data systems. Another significant gap was identified in the employee onboarding process, which currently omits security awareness training for new hires. These issues present a high risk of account compromise and unauthorized access.

On a positive note, the external network scan of the target IP address \texttt{[Client IP]} did not identify any open ports. This indicates a strong network perimeter security posture. Specifically, the scan found that port 80 (HTTP) is closed, which contradicts a pre-existing risk item. This suggests that a previously identified vulnerability has been successfully remediated.

Our primary recommendations focus on the immediate implementation of MFA and the integration of security training into the employee onboarding process to mitigate the most severe risks.

% --- 2. Organizational Information ---
\section{Organizational Information}
This section details the information provided by the client organization for the scope of this assessment.
\begin{itemize}
    \item \textbf{Organization Name:} \textbf{[Organization Name]}
    \item \textbf{Primary Domain:} \texttt{[Domain]}
    \item \textbf{External IP Scanned:} \texttt{[Client IP]}
\end{itemize}

% --- 3. Security Control Review ---
\section{Security Control Review}
The following table summarizes the organization's responses to a security controls questionnaire. The responses are benchmarked against standard cybersecurity best practices. Answers marked with \ding{55} (No) represent significant gaps in the current security posture.

\begin{table}[h!]
\centering
\caption{Security Controls Questionnaire Analysis}
\label{tab:controls}
\begin{tabular}{L{0.7\textwidth} C{0.2\textwidth}}
\toprule
\textbf{Control Question} & \textbf{Response} \\
\midrule
Do you require MFA to access email? & \ding{55} \\
Do you require MFA to log into computers? & \ding{55} \\
Do you require MFA to access sensitive data systems? & \ding{55} \\
Does your organization have an employee acceptable use policy? & \ding{51} \\
Does your organization do security awareness training for new employees? & \ding{55} \\
Does your organization do security awareness training for all employees at least once per year? & \ding{51} \\
\bottomrule
\end{tabular}
\end{table}

\subsection*{Analysis of Gaps}
The questionnaire reveals critical deficiencies in access control and employee training:
\begin{itemize}
    \item \textbf{Lack of Multi-Factor Authentication (MFA):} The absence of MFA for email, computer logins, and sensitive systems is a critical vulnerability. This significantly increases the risk of unauthorized access via compromised credentials (e.g., through phishing or password spraying attacks).
    \item \textbf{Onboarding Training Gap:} The lack of security awareness training for new employees leaves the organization vulnerable. New hires are often targeted by social engineering attacks and may be unaware of internal security policies and procedures.
\end{itemize}

% --- 4. Technical Scan Results ---
\section{Technical Scan Results}
An external network scan was performed to identify open ports and exposed services on the organization's public-facing infrastructure.

\begin{itemize}
    \item \textbf{Target IP Address:} \texttt{[Target IP]}
    \item \textbf{Scan Date:} Not specified in scan data.
    \item \textbf{Scanner Used:} Nmap
\end{itemize}

\subsection*{Findings}
The scan confirmed that the target host is online (\textbf{status: up}). However, no open ports were discovered during the scan. The status of common ports is listed below.

\begin{table}[h!]
\centering
\caption{Port Scan Results for \texttt{[Target IP]}}
\label{tab:ports}
\begin{tabular}{c c l}
\toprule
\textbf{Port} & \textbf{State} & \textbf{Service} \\
\midrule
80 & closed & http \\
\bottomrule
\end{tabular}
\end{table}

\subsection*{Technical Assessment}
The absence of open ports on the scanned IP address indicates a strong network perimeter configuration. This significantly reduces the external attack surface. Notably, the finding that port 80 is closed contradicts a pre-existing risk documented in the organization's risk register, suggesting successful remediation has occurred.

% --- 5. Consolidated Risk Assessment ---
\section{Consolidated Risk Assessment}
This section synthesizes findings from the security questionnaire, technical scan, and pre-existing risk data into a consolidated list of current risks.

\begin{table}[h!]
\centering
\caption{Summary of Identified Risks}
\label{tab:risks}
\begin{tabular}{L{0.3\textwidth} L{0.45\textwidth} C{0.15\textwidth}}
\toprule
\textbf{Risk Name} & \textbf{Overview} & \textbf{Severity} \\
\midrule
\textbf{No MFA on Email} & Lack of a second authentication factor for email access exposes the organization to account takeover and business email compromise. & \textbf{Critical} \\
\addlinespace
\textbf{No MFA on Workstations} & Absence of MFA for computer logins allows an attacker with valid credentials to gain direct access to endpoint systems and the internal network. & \textbf{Critical} \\
\addlinespace
\textbf{No MFA on Sensitive Data Systems} & Critical data is accessible without MFA, creating a high risk of a significant data breach if credentials are compromised. & \textbf{Critical} \\
\addlinespace
\textbf{Inadequate New Hire Training} & New employees are not trained on security policies, making them more susceptible to social engineering and accidental policy violations. & \textbf{High} \\
\addlinespace
\textbf{Unencrypted Web Server} \textit{(Pre-existing)} & Port 80 was believed to be open, exposing unencrypted web traffic. \textbf{Note:} The current scan shows this port is closed. & \textbf{Low (Remediated)} \\
\bottomrule
\end{tabular}
\end{table}

% --- 6. Recommendations ---
\section{Recommendations}
The following recommendations are prioritized based on risk severity to help \textbf{[Organization Name]} improve its cybersecurity posture.

\subsection*{Priority 1: Implement Multi-Factor Authentication (Critical)}
\begin{itemize}
    \item \textbf{Action:} Deploy a robust MFA solution across all critical platforms immediately.
    \item \textbf{Details:}
        \begin{enumerate}
            \item \textbf{Email:} Enforce MFA for all email accounts (e.g., via Microsoft 365, Google Workspace). This is the highest priority to prevent business email compromise.
            \item \textbf{Sensitive Systems:} Enforce MFA for all applications and systems that store or process sensitive or regulated data.
            \item \textbf{Workstations \& VPN:} Require MFA for all remote access (VPN) and local computer logins to protect endpoints.
        \end{enumerate}
\end{itemize}

\subsection*{Priority 2: Enhance Security Awareness Program (High)}
\begin{itemize}
    \item \textbf{Action:} Integrate mandatory security awareness training into the new employee onboarding process.
    \item \textbf{Details:} Develop a training module for all new hires that covers, at a minimum, the acceptable use policy, phishing identification, password hygiene, and incident reporting procedures. This should be a prerequisite for gaining access to company systems.
\end{itemize}

\subsection*{Priority 3: Update Risk Register (Informational)}
\begin{itemize}
    \item \textbf{Action:} Review and update the internal risk register based on the findings of this report.
    \item \textbf{Details:} The risk titled "Unencrypted Web Server" related to Port 80 appears to be remediated, as confirmed by our technical scan. We recommend formally closing this risk item after internal verification.
\end{itemize}

\end{document}
```