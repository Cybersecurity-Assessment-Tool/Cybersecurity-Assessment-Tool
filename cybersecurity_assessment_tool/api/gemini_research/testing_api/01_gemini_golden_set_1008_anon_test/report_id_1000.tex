```latex
\documentclass[12pt, a4paper]{article}

% Preamble: Required Packages
\usepackage[margin=1in]{geometry}
\usepackage{pifont} % For checkmarks and crosses
\usepackage{booktabs} % For professional tables
\usepackage{hyperref} % For clickable links and references
\usepackage{url} % For formatting URLs
\usepackage{seqsplit} % To split long strings without breaking
\usepackage{graphicx}
\usepackage[table]{xcolor}
\usepackage{tocloft}

% --- Document Setup ---
\definecolor{darkblue}{rgb}{0.0, 0.0, 0.55}
\definecolor{darkred}{rgb}{0.55, 0.0, 0.0}
\definecolor{tablehead}{gray}{0.85}

\hypersetup{
    colorlinks=true,
    linkcolor=darkblue,
    filecolor=darkblue,      
    urlcolor=darkblue,
    citecolor=darkblue,
}

% --- Custom Commands ---
\newcommand{\yes}{\ding{51}}
\newcommand{\no}{\ding{55}}

% --- Title Page Information ---
\title{
    \vspace{2cm}
    \textbf{Cybersecurity Posture Assessment Report} \\
    \vspace{0.5cm}
    \large \textbf{[Organization Name]} \\
    \vspace{2cm}
}
\author{Cybersecurity Analysis Division}
\date{\today}

% ==============================================================================
% --- BEGIN DOCUMENT ---
% ==============================================================================
\begin{document}

\maketitle
\thispagestyle{empty}
\newpage

\tableofcontents
\thispagestyle{empty}
\newpage

\pagestyle{headings}

% ==============================================================================
\section{Executive Summary}
% ==============================================================================
This report provides a cybersecurity posture assessment for \textbf{[Organization Name]}, based on a combination of self-reported organizational controls, an external network scan, and a review of pre-existing risks.

The external network scan of the designated target IP address revealed a strong perimeter security posture. No open ports were detected, indicating that the firewall is effectively configured to block unsolicited inbound traffic. This significantly reduces the external attack surface.

However, the review of organizational security controls identified several critical and high-risk gaps. The most significant findings are the absence of Multi-Factor Authentication (MFA) for computer logins and the lack of a formal security awareness training program for employees. These internal control deficiencies create a substantial risk of compromise through credential theft or social engineering, which could bypass the strong network perimeter.

This report details these findings and provides prioritized, actionable recommendations to mitigate the identified risks and strengthen the overall security posture of the organization.

% ==============================================================================
\section{Organizational Information}
% ==============================================================================
The following information was used as the basis for this assessment. As some identifying data was not provided, placeholders have been used in accordance with standard reporting procedures.

\begin{table}[h!]
\centering
\begin{tabular}{@{}ll@{}}
\toprule
\textbf{Attribute} & \textbf{Value} \\ \midrule
Organization Name & \textbf{[Organization Name]} \\
Email Domain & \texttt{[Domain]} \\
External IP Address & \texttt{[Client IP]} \\
Assessment Date & \today \\ \bottomrule
\end{tabular}
\caption{Client and Assessment Details.}
\label{tab:org_info}
\end{table}

% ==============================================================================
\section{Security Control Review}
% ==============================================================================
A review of the organization's security controls was conducted via a questionnaire. The responses highlight critical areas for improvement, particularly concerning user access controls and security awareness. The results are summarized in Table \ref{tab:controls}.

\begin{table}[h!]
\centering
\rowcolors{2}{gray!10}{white}
\begin{tabular}{@{}p{0.7\textwidth}cc@{}}
\toprule
\rowcolor{tablehead}
\textbf{Control Question} & \textbf{Response} & \textbf{Status} \\ \midrule
Do you require MFA to access email? & Yes & \yes \\
Do you require MFA to log into computers? & \textbf{No} & \textcolor{darkred}{\no} \\
Do you require MFA to access sensitive data systems? & Yes & \yes \\
Does your organization have an employee acceptable use policy? & Yes & \yes \\
Does your organization do security awareness training for new employees? & \textbf{No} & \textcolor{darkred}{\no} \\
Does your organization do security awareness training for all employees at least once per year? & \textbf{No} & \textcolor{darkred}{\no} \\
\bottomrule
\end{tabular}
\caption{Security Control Questionnaire Results.}
\label{tab:controls}
\end{table}

\subsection*{Analysis of Findings}
\begin{itemize}
    \item \textbf{Positive Controls:} The implementation of MFA for email and sensitive data systems, along with an established acceptable use policy, are commendable foundational security measures.
    \item \textbf{Critical Gap - Endpoint MFA:} The absence of MFA for computer logins is a critical vulnerability. If an employee's credentials are compromised (e.g., through a phishing attack), an attacker could gain direct access to an endpoint on the internal network, providing a foothold for lateral movement and further attacks.
    \item \textbf{High Risk - Security Training:} The complete lack of a security awareness training program for both new and existing employees is a high-risk gap. This leaves the organization highly susceptible to social engineering, phishing, and malware attacks that prey on human error.
\end{itemize}

% ==============================================================================
\section{Technical Scan Results}
% ==============================================================================
An external network scan was performed using Nmap to identify open ports and services on the public-facing IP address.

\begin{table}[h!]
\centering
\begin{tabular}{@{}ll@{}}
\toprule
\textbf{Scan Parameter} & \textbf{Value} \\ \midrule
Target IP Address & \texttt{[Target IP]} \\
Scan Date & \today \\
Host Status & Up \\
Open Ports Found & 0 \\
Filtered/Closed Ports & All other ports \\ \bottomrule
\end{tabular}
\caption{Nmap Scan Summary.}
\label{tab:nmap_summary}
\end{table}

\subsection*{Scan Details and Interpretation}
The scan results indicate that the target host \texttt{[Target IP]} is online but has no discoverable open TCP/UDP ports. All scanned ports were found to be in a `closed` state.

This is a positive security finding. It suggests that a firewall or similar network security device is properly configured to deny all unsolicited incoming traffic from the internet, effectively minimizing the external network attack surface.

% ==============================================================================
\section{Consolidated Risk Assessment}
% ==============================================================================
This section synthesizes findings from the security control review, technical scan, and pre-existing risk data. No pre-existing vulnerabilities were documented. The following new risks have been identified.

\begin{table}[h!]
\centering
\rowcolors{2}{gray!10}{white}
\begin{tabular}{@{}p{0.15\textwidth}p{0.25\textwidth}p{0.4\textwidth}p{0.1\textwidth}@{}}
\toprule
\rowcolor{tablehead}
\textbf{Risk ID} & \textbf{Risk Name} & \textbf{Overview} & \textbf{Severity} \\ \midrule
RISK-001 & Lack of MFA on Endpoints & The absence of MFA on computer logins allows an attacker with stolen credentials to gain unauthorized access to the internal network, bypassing perimeter defenses. & \textbf{Critical} \\
\addlinespace
RISK-002 & Insufficient Security Awareness Training & Without training, employees are more likely to fall victim to phishing and social engineering attacks, leading to credential compromise, malware infection, or data breaches. & \textbf{High} \\
\bottomrule
\end{tabular}
\caption{Identified Security Risks.}
\label{tab:risks}
\end{table}

% ==============================================================================
\section{Recommendations}
% ==============================================================================
The following prioritized recommendations are provided to address the identified risks and improve the organization's security posture.

\subsection*{Priority 1: Critical}
\begin{description}
    \item[Implement MFA for Endpoint Access (RISK-001):]
    \begin{itemize}
        \item \textbf{Action:} Deploy a Multi-Factor Authentication solution for all employee computer logins (Windows, macOS, etc.). This should apply to both local and remote access.
        \item \textbf{Justification:} This measure will prevent unauthorized access even if user credentials are stolen, effectively mitigating the highest-priority risk identified in this assessment.
    \end{itemize}
\end{description}

\subsection*{Priority 2: High}
\begin{description}
    \item[Establish a Security Awareness Program (RISK-002):]
    \begin{itemize}
        \item \textbf{Action:} Procure and implement a security awareness training platform. Mandate that all new hires complete foundational training upon onboarding. Require all employees to complete annual refresher training and periodic phishing simulations.
        \item \textbf{Justification:} A well-trained workforce is the first line of defense against common cyberattacks. This program will reduce the likelihood of human error leading to a security incident.
    \end{itemize}
\end{description}

\end{document}
% ==============================================================================
% --- END DOCUMENT ---
% ==============================================================================
```