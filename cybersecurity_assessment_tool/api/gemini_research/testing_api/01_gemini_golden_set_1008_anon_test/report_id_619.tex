```latex
\documentclass[12pt, a4paper]{article}

% Preamble: Required Packages
\usepackage[margin=1in]{geometry}
\usepackage{pifont} % For checkmarks and crosses
\usepackage{booktabs} % For professional tables
\usepackage{hyperref} % For clickable links
\usepackage{url} % For formatting URLs
\usepackage{seqsplit} % For splitting long strings in texttt
\usepackage{graphicx} % For logo
\usepackage{xcolor} % For colors

% Document Information
\title{Cybersecurity Posture Assessment Report}
\author{Cybersecurity Analysis Division}
\date{\today}

% Hyperref Setup
\hypersetup{
    colorlinks=true,
    linkcolor=blue,
    filecolor=magenta,      
    urlcolor=cyan,
    pdftitle={Cybersecurity Posture Assessment Report},
    pdfpagemode=FullScreen,
}

\begin{document}

\begin{titlepage}
    \centering
    \vfill
    \begin{center}
        \Huge \textbf{Cybersecurity Posture Assessment Report}
    \end{center}
    \vspace{1.5cm}
    \begin{center}
        \Large Prepared for: \\
        \vspace{0.5cm}
        \textbf{[Organization Name]}
    \end{center}
    \vfill
    \begin{center}
        \today
    \end{center}
\end{titlepage}

\tableofcontents
\newpage

% --- Executive Summary ---
\section{Executive Summary}
This report details the findings of a cybersecurity assessment conducted for \textbf{[Organization Name]}. The assessment incorporated a review of organizational security controls, an external network scan, and an analysis of pre-existing risks.

The overall security posture requires immediate attention. Two significant gaps were identified in the organization's security controls:
\begin{itemize}
    \item \textbf{Critical Risk:} The absence of Multi-Factor Authentication (MFA) for email access presents a critical vulnerability. This significantly increases the risk of business email compromise, data breaches, and phishing-related incidents.
    \item \textbf{High Risk:} The lack of mandatory annual security awareness training for all employees weakens the human firewall, making the organization more susceptible to social engineering and other user-targeted attacks.
\end{itemize}

The technical scan revealed an open Secure Shell (SSH) port on an external-facing system. While often necessary for administration, this service is a common target for automated brute-force attacks and requires robust security configurations.

This report provides a detailed breakdown of these findings and offers actionable recommendations to mitigate the identified risks and strengthen the organization's overall defensive capabilities.

% --- Organizational Information ---
\section{Organizational Information}
The following information was used as the basis for this assessment. Due to the anonymized nature of the provided data, placeholders have been used where necessary.

\begin{table}[h!]
\centering
\begin{tabular}{@{}ll@{}}
\toprule
\textbf{Attribute} & \textbf{Value} \\ \midrule
Organization Name & \textbf{[Organization Name]} \\
Primary Email Domain & \texttt{[Domain]} \\
External IP Address & \texttt{[Client IP]} \\ \bottomrule
\end{tabular}
\caption{Client Organizational Details}
\end{table}

% --- Security Control Review ---
\section{Security Control Review}
A review of the organization's security controls was conducted via a questionnaire. The responses indicate a solid foundation in some areas, such as MFA for computer and sensitive system access. However, critical gaps were identified, as detailed in the table below.

\begin{table}[h!]
\centering
\begin{tabular}{@{}p{0.6\linewidth}cc@{}}
\toprule
\textbf{Control Question} & \textbf{Response} & \textbf{Status} \\ \midrule
Do you require MFA to access email? & No & \ding{55} \\
Do you require MFA to log into computers? & Yes & \ding{51} \\
Do you require MFA to access sensitive data systems? & Yes & \ding{51} \\
Does your organization have an employee acceptable use policy? & Yes & \ding{51} \\
Does your organization do security awareness training for new employees? & Yes & \ding{51} \\
Does your organization do security awareness training for all employees at least once per year? & No & \ding{55} \\ \bottomrule
\end{tabular}
\caption{Security Controls Questionnaire Results}
\end{table}

The items marked with \ding{55} represent significant deviations from security best practices and are addressed in the Risk Assessment section of this report.

% --- Technical Scan Results ---
\section{Technical Scan Results}
An external network scan was performed on the target system to identify open ports and exposed services.

\begin{itemize}
    \item \textbf{Target IP Address:} \texttt{[Target IP]}
    \item \textbf{Scan Date:} Information not provided in scan data.
\end{itemize}

\subsection{Open Ports}
The scan identified the following port as open and accessible from the public internet:

\begin{table}[h!]
\centering
\begin{tabular}{@{}llll@{}}
\toprule
\textbf{Port} & \textbf{State} & \textbf{Service} & \textbf{Notes} \\ \midrule
22/tcp & open & ssh & Secure Shell remote administration \\ \bottomrule
\end{tabular}
\caption{Open Port Findings for \texttt{[Target IP]}}
\end{table}

\subsection{Analysis}
The presence of an open SSH port (22) indicates that remote administrative access is enabled on this system. While necessary for management, it is a high-value target for attackers who use automated tools to guess or brute-force credentials. The scan data did not include service version information, which prevents an analysis for known vulnerabilities in the SSH software itself.

% --- Pre-existing Risks ---
\section{Pre-existing Risk Review}
An analysis of the provided list of current organizational risks was performed. The dataset contained no pre-existing vulnerabilities. All risks documented in this report are new findings based on the current assessment.

% --- Risk Assessment ---
\section{Risk Assessment Summary}
The following table synthesizes findings from the security control review and the technical scan into a prioritized list of risks.

\begin{table}[h!]
\centering
\begin{tabular}{@{}p{0.1\linewidth}p{0.5\linewidth}p{0.15\linewidth}p{0.15\linewidth}@{}}
\toprule
\textbf{Risk ID} & \textbf{Description} & \textbf{Severity} & \textbf{Source} \\ \midrule
RISK-001 & Lack of Multi-Factor Authentication (MFA) on email allows for account takeover with only a compromised password, leading to potential data breaches and phishing. & \textbf{Critical} & Questionnaire \\
\addlinespace
RISK-002 & Lack of annual security awareness training for all employees increases susceptibility to social engineering, phishing, and malware infections. & \textbf{High} & Questionnaire \\
\addlinespace
RISK-003 & The SSH service is exposed to the public internet, creating a risk of unauthorized access via brute-force or credential stuffing attacks. & \textbf{Medium} & Network Scan \\ \bottomrule
\end{tabular}
\caption{Summary of Identified Risks}
\end{table}

% --- Recommendations ---
\section{Recommendations}
The following actions are recommended to mitigate the identified risks and improve the overall security posture of \textbf{[Organization Name]}.

\subsection{RISK-001: Implement MFA for Email (Critical)}
\begin{itemize}
    \item \textbf{Action:} Immediately enable and enforce MFA for all user accounts across the organization's email platform (e.g., Microsoft 365, Google Workspace).
    \item \textbf{Justification:} This is the single most effective control to prevent unauthorized access to email accounts, even if user credentials are stolen. It directly mitigates the risk of business email compromise.
\end{itemize}

\subsection{RISK-002: Establish Annual Security Training (High)}
\begin{itemize}
    \item \textbf{Action:} Develop and implement a mandatory annual security awareness training program for all employees. The training should cover current threats, including phishing, social engineering, and password hygiene.
    \item \textbf{Justification:} Regular training reinforces security concepts and keeps employees informed about evolving threats, strengthening the organization's "human firewall."
\end{itemize}

\subsection{RISK-003: Harden Exposed SSH Service (Medium)}
\begin{itemize}
    \item \textbf{Action:}
        \begin{enumerate}
            \item If possible, restrict SSH access to a list of approved IP addresses (allow-listing).
            \item Disable password-based authentication and enforce the use of public/private key pairs for all SSH logins.
            \item Implement an intrusion prevention tool, such as Fail2Ban, to automatically block IP addresses that exhibit brute-force attack patterns.
        \end{enumerate}
    \item \textbf{Justification:} These layered controls significantly reduce the attack surface of the SSH service, making a successful brute-force attack highly improbable.
\end{itemize}

\end{document}
```