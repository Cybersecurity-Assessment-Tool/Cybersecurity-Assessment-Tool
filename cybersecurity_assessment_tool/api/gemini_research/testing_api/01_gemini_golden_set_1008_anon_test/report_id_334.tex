```latex
\documentclass[12pt]{article}

% Preamble: Required Packages
\usepackage[margin=1in]{geometry}
\usepackage{pifont} % For checkmarks and crosses
\usepackage{booktabs} % For professional tables
\usepackage{hyperref} % For clickable links
\usepackage{url}      % For URL formatting
\usepackage{seqsplit} % For splitting long strings
\usepackage{xcolor}   % For colors

% Document Information
\title{Cybersecurity Posture Assessment Report}
\author{Cybersecurity Analysis Division}
\date{\today}

% Hyperref Setup
\hypersetup{
    colorlinks=true,
    linkcolor=blue,
    filecolor=magenta,      
    urlcolor=cyan,
    pdftitle={Cybersecurity Posture Assessment Report},
    pdfpagemode=FullScreen,
}

\begin{document}

\maketitle
\thispagestyle{empty}
\newpage

\tableofcontents
\thispagestyle{empty}
\newpage

\setcounter{page}{1}

% ==============================================================================
% SECTION 1: EXECUTIVE SUMMARY
% ==============================================================================
\section{Executive Summary}

This report details the findings of a cybersecurity posture assessment conducted for \textbf{[Organization Name]}. The assessment incorporated an analysis of organizational security controls via a questionnaire, a technical network scan of the designated external asset, and a review of pre-existing risks.

The overall security posture presents a mixed landscape. The organization has implemented critical controls such as Multi-Factor Authentication (MFA) for email and sensitive data systems. The external network scan of the target IP address revealed no open ports, suggesting a well-hardened perimeter firewall.

However, two significant gaps in foundational security controls were identified that introduce a high level of risk. Firstly, the absence of MFA for computer logins leaves endpoints vulnerable to unauthorized access via compromised credentials. Secondly, the lack of mandatory, annual security awareness training for all employees increases the organization's susceptibility to social engineering and phishing attacks.

Immediate remediation of these control gaps is strongly recommended to mitigate significant threats and strengthen the organization's defense-in-depth strategy. Detailed findings and actionable recommendations are provided in the subsequent sections.

% ==============================================================================
% SECTION 2: ORGANIZATIONAL INFORMATION
% ==============================================================================
\section{Organizational Information}

The following details were used as the basis for this assessment. Due to the anonymized nature of the provided data, placeholders have been used where necessary.

\begin{itemize}
    \item \textbf{Organization Name:} \textbf{[Organization Name]}
    \item \textbf{Primary Domain:} \texttt{[Domain]}
    \item \textbf{Assessed External IP:} \texttt{[Client IP]}
\end{itemize}

% ==============================================================================
% SECTION 3: SECURITY CONTROL REVIEW
% ==============================================================================
\section{Security Control Review}

An assessment of internal security policies and procedures was conducted via a standardized questionnaire. The responses indicate areas of both strength and weakness in the current security framework. A summary of the responses is provided in Table \ref{tab:controls}.

\begin{table}[h!]
\centering
\caption{Organizational Security Control Questionnaire}
\label{tab:controls}
\begin{tabular}{p{0.8\linewidth} c}
\toprule
\textbf{Control Question} & \textbf{Response} \\
\midrule
Do you require MFA to access email? & \textcolor{green!80!black}{\ding{51}} \\
Do you require MFA to log into computers? & \textcolor{red}{\ding{55}} \\
Do you require MFA to access sensitive data systems? & \textcolor{green!80!black}{\ding{51}} \\
Does your organization have an employee acceptable use policy? & \textcolor{green!80!black}{\ding{51}} \\
Does your organization do security awareness training for new employees? & \textcolor{green!80!black}{\ding{51}} \\
Does your organization do security awareness training for all employees at least once per year? & \textcolor{red}{\ding{55}} \\
\bottomrule
\end{tabular}
\end{table}

\paragraph{Analysis:} The presence of MFA for email and sensitive systems is a commendable control. However, the "No" responses highlight critical vulnerabilities. The lack of MFA on computer logins is a primary concern, as it is a fundamental defense against credential theft and lateral movement. Similarly, failing to provide annual security training for all staff allows security knowledge to become outdated, increasing human-related risk factors.

% ==============================================================================
% SECTION 4: TECHNICAL SCAN RESULTS
% ==============================================================================
\section{Technical Scan Results}

An unauthenticated network scan was performed on the target system to identify open ports and exposed services.

\begin{itemize}
    \item \textbf{Target IP Address:} \texttt{[Target IP]}
    \item \textbf{Scan Date:} Not provided in scan data.
\end{itemize}

\subsection{Scan Summary}
The network scan completed successfully but did not identify any open TCP or UDP ports on the target host. 

\paragraph{Interpretation:} This result indicates that the host is not exposing any services to the public internet from the scanning source location. This is a positive security finding, suggesting that a perimeter firewall is in place and configured to deny unsolicited inbound traffic. This significantly reduces the external attack surface of the asset. No vulnerabilities could be identified as no services were accessible.

% ==============================================================================
% SECTION 5: RISK ASSESSMENT
% ==============================================================================
\section{Risk Assessment}

This section synthesizes findings from the security control review, technical scan, and any pre-existing risk data. The provided input for current risks was empty. The following risks have been identified based on the analysis conducted in this assessment.

\begin{table}[h!]
\centering
\caption{Identified Risks}
\label{tab:risks}
\begin{tabular}{p{0.15\linewidth} p{0.25\linewidth} p{0.4\linewidth} p{0.1\linewidth}}
\toprule
\textbf{Risk ID} & \textbf{Risk Name} & \textbf{Description} & \textbf{Severity} \\
\midrule
RISK-001 & Lack of Endpoint MFA & Failure to enforce MFA on computer logins exposes the organization to significant risk from compromised credentials, enabling unauthorized access and lateral movement within the network. & \textbf{Critical} \\
\addlinespace
RISK-002 & Inadequate Security Training Program & The absence of mandatory, annual security awareness training for all employees increases organizational susceptibility to phishing, social engineering, and other human-targeted attacks. & \textbf{High} \\
\bottomrule
\end{tabular}
\end{table}

% ==============================================================================
% SECTION 6: RECOMMENDATIONS
% ==============================================================================
\section{Recommendations}

Based on the identified risks, the following prioritized recommendations are provided to enhance the security posture of \textbf{[Organization Name]}.

\subsection{Priority 1: Implement Endpoint MFA (Critical)}
\begin{itemize}
    \item \textbf{Action:} Deploy a Multi-Factor Authentication solution for all user logins to company-managed computers (desktops and laptops). This should apply to both local and remote access.
    \item \textbf{Justification:} This action directly mitigates RISK-001. Even if an attacker compromises a user's password, they will be unable to access the endpoint without the second authentication factor, effectively stopping a common attack vector.
\end{itemize}

\subsection{Priority 2: Establish Annual Security Training (High)}
\begin{itemize}
    \item \textbf{Action:} Develop and implement a mandatory security awareness training program for all employees, to be completed annually. The training should cover current threats such as phishing, ransomware, and proper data handling.
    \item \textbf{Justification:} This action directly mitigates RISK-002. A well-informed workforce is a critical layer of defense. Regular training ensures that security remains a top-of-mind concern and equips employees to recognize and report threats effectively.
\end{itemize}

\subsection{Priority 3: Internal Vulnerability Scanning (Informational)}
\begin{itemize}
    \item \textbf{Action:} While the external scan was clean, it is recommended to implement a program for regular, authenticated vulnerability scanning of internal network assets.
    \item \textbf{Justification:} An external scan provides an attacker's view of the perimeter but offers no visibility into internal systems. Authenticated internal scans can identify missing patches, misconfigurations, and other vulnerabilities that could be exploited by an attacker who has already gained a foothold in the network.
\end{itemize}

\end{document}
```