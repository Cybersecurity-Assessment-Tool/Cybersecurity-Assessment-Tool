```latex
\documentclass[12pt]{article}

% Preamble: Required Packages
\usepackage[a4paper, margin=1in]{geometry}
\usepackage{pifont} % For \ding symbols (checkmarks and crosses)
\usepackage{booktabs} % For professional-looking tables (\toprule, \midrule, \bottomrule)
\usepackage{hyperref} % For clickable links and references
\usepackage{url} % For formatting URLs
\usepackage{seqsplit} % To split long strings without breaking words
\usepackage{xcolor} % For custom colors
\usepackage{graphicx} % To include images, if needed

% --- Document Setup ---
\hypersetup{
    colorlinks=true,
    linkcolor=blue,
    filecolor=magenta,
    urlcolor=cyan,
    pdftitle={Cybersecurity Posture Assessment Report},
    pdfauthor={Cybersecurity Analysis Division},
}

% --- Custom Commands ---
\newcommand{\yes}{\textcolor{green}{\ding{51}}} % Green checkmark for "Yes"
\newcommand{\no}{\textcolor{red}{\ding{55}}}   % Red cross for "No"

% --- Document Body ---
\begin{document}

% --- Title Page ---
\begin{titlepage}
    \centering
    \vspace*{\stretch{1.0}}
    \Huge \textbf{Cybersecurity Posture Assessment Report} \\
    \vspace{0.5cm}
    \LARGE For \\
    \vspace{0.5cm}
    \Huge \textbf{[Organization Name]} \\
    \vspace{\stretch{2.0}}
    \large \textbf{Date of Report:} \today \\
    \large \textbf{Prepared by:} Cybersecurity Analysis Division \\
    \vspace*{\stretch{1.0}}
\end{titlepage}

\tableofcontents
\newpage

% --- Section 1: Executive Summary ---
\section*{1. Executive Summary}

This report provides a comprehensive analysis of the cybersecurity posture for \textbf{[Organization Name]}, based on a review of organizational security controls, an external network scan, and pre-existing risk data.

The assessment identified several positive security controls, including the mandatory use of Multi-Factor Authentication (MFA) for email and sensitive data systems, as well as a robust security awareness training program for all employees. These measures significantly strengthen the organization's defense against common cyber threats like phishing and account compromise.

However, two critical gaps were identified in the organization's policies and technical controls:
\begin{itemize}
    \item \textbf{Lack of MFA for Computer Logins:} The absence of MFA on employee workstations presents a high risk, as a single compromised password could grant an attacker direct access to the internal network.
    \item \textbf{Absence of an Acceptable Use Policy (AUP):} Without a formal AUP, there is no clear guidance for employees on the secure and appropriate use of company assets, increasing the risk of insider threats and unintentional data breaches.
\end{itemize}

The technical network scan of the target IP address \texttt{[Target IP]} did not identify any open ports. Notably, port 80, which was associated with a pre-existing risk ("Unencrypted Web Server"), was found to be closed. This is a positive finding and suggests that the previously identified risk may have been remediated.

This report outlines these findings in detail and provides actionable recommendations to mitigate the identified risks and enhance the overall security posture.

% --- Section 2: Organizational Information ---
\section*{2. Organizational Information}
This assessment was conducted for the following entity:
\begin{itemize}
    \item \textbf{Organization Name:} \textbf{[Organization Name]}
    \item \textbf{Primary Email Domain:} \texttt{[Domain]}
    \item \textbf{External IP Scanned:} \texttt{[Client IP]}
\end{itemize}

% --- Section 3: Security Control Review ---
\section*{3. Security Control Review (Questionnaire Analysis)}
The following table summarizes the organization's responses to a security controls questionnaire. "No" answers indicate a deviation from security best practices and represent potential areas of risk.

\begin{table}[h!]
\centering
\caption{Security Controls Questionnaire Results}
\begin{tabular}{p{9cm}c}
\toprule
\textbf{Control Question} & \textbf{Status} \\
\midrule
Do you require MFA to access email? & \yes \\
Do you require MFA to log into computers? & \no \\
Do you require MFA to access sensitive data systems? & \yes \\
Does your organization have an employee acceptable use policy? & \no \\
Does your organization do security awareness training for new employees? & \yes \\
Does your organization do security awareness training for all employees at least once per year? & \yes \\
\bottomrule
\end{tabular}
\end{table}

\subsection*{Analysis of Gaps}
\begin{itemize}
    \item \textbf{MFA for Computers:} The lack of MFA for workstation logins is a critical vulnerability. If an employee's password is stolen, an attacker could potentially log in to their machine and gain a foothold within the internal network.
    \item \textbf{Acceptable Use Policy (AUP):} An AUP is a foundational policy that sets expectations for employee behavior when using company technology. Its absence creates ambiguity and increases the risk of misuse of assets and data.
\end{itemize}

% --- Section 4: Technical Scan Results ---
\section*{4. Technical Scan Results}
An external network scan was performed using Nmap against the designated target IP address.

\begin{itemize}
    \item \textbf{Target IP:} \texttt{[Target IP]}
    \item \textbf{Scan Date:} Data not provided in scan results.
    \item \textbf{Summary of Findings:} The scan completed successfully and found the host to be online. However, \textbf{no open ports were discovered}. All scanned ports, including common ports like 80 (HTTP) and 443 (HTTPS), were reported as 'closed'.
\end{itemize}

\subsection*{Correlation with Existing Risks}
The scan results contradict a pre-existing risk titled "Unencrypted Web Server," which stated that Port 80 was open. The current scan indicates that Port 80 is closed, suggesting this vulnerability has been successfully remediated or was a false positive in a previous assessment.

% --- Section 5: Consolidated Risk Assessment ---
\section*{5. Consolidated Risk Assessment}
The following table synthesizes findings from the security questionnaire, technical scan, and pre-existing risk data into a prioritized list.

\begin{table}[h!]
\centering
\caption{Summary of Identified Risks}
\begin{tabular}{p{4.5cm}p{6.5cm}l}
\toprule
\textbf{Risk Name} & \textbf{Overview} & \textbf{Severity} \\
\midrule
\textbf{Lack of MFA on Workstations} & A compromised password could lead to direct workstation and internal network access. This bypasses a critical layer of defense. & \textbf{High} \\
\addlinespace
\textbf{Missing Acceptable Use Policy} & Without a formal policy, the organization lacks an enforceable standard for technology use, increasing insider and legal risks. & \textbf{High} \\
\addlinespace
\textbf{Unencrypted Web Server (Historical)} & A previously identified risk indicated that port 80 was open. The recent scan found this port to be closed. & \textbf{Low (Resolved)} \\
\bottomrule
\end{tabular}
\end{table}

% --- Section 6: Recommendations ---
\section*{6. Recommendations}
The following actions are recommended to address the identified risks and improve the security posture of \textbf{[Organization Name]}.

\subsection*{Recommendation 1: Implement MFA for Workstation Logins (High Priority)}
\begin{itemize}
    \item \textbf{Action:} Deploy a mandatory Multi-Factor Authentication solution for all employee computer and server logins.
    \item \textbf{Justification:} This will prevent unauthorized access even if user credentials are stolen, significantly reducing the risk of lateral movement and ransomware attacks.
    \item \textbf{Examples:} Solutions include Windows Hello for Business, Duo Security, or other third-party identity providers that integrate with your operating systems.
\end{itemize}

\subsection*{Recommendation 2: Develop and Enforce an Acceptable Use Policy (High Priority)}
\begin{itemize}
    \item \textbf{Action:} Draft, approve, and distribute a formal Acceptable Use Policy (AUP) to all employees. Require employees to read and acknowledge the policy.
    \item \textbf{Justification:} An AUP establishes clear rules for using company IT assets, reduces the risk of misuse, and provides a basis for disciplinary action in case of violations.
    \item \textbf{Resources:} Templates are available from organizations like SANS (\url{https://www.sans.org/information-security-policy/}).
\end{itemize}

\subsection*{Recommendation 3: Verify and Close Historical Risk (Low Priority)}
\begin{itemize}
    \item \textbf{Action:} Formally document that the "Unencrypted Web Server" risk related to port 80 has been remediated, based on the recent network scan results.
    \item \textbf{Justification:} Maintaining an accurate risk register is crucial for focusing resources on current, active threats. This action ensures the risk register reflects the current state of the network.
\end{itemize}

\end{document}
```