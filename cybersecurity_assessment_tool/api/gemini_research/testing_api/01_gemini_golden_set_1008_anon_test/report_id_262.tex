```latex
\documentclass[12pt]{article}

% Required Packages
\usepackage[margin=1in]{geometry}
\usepackage{pifont} % For checkmarks and crosses
\usepackage{booktabs} % For professional tables
\usepackage{hyperref} % For hyperlinks
\usepackage{url}      % For URL formatting
\usepackage{seqsplit} % For splitting long strings
\usepackage{xcolor}   % For colors

% Document Metadata
\title{Cybersecurity Posture Assessment Report}
\author{Cybersecurity Analysis Division}
\date{\today}

% Hyperref Setup
\hypersetup{
    colorlinks=true,
    linkcolor=blue,
    filecolor=magenta,      
    urlcolor=cyan,
    pdftitle={Cybersecurity Posture Assessment Report},
    pdfpagemode=FullScreen,
}

\begin{document}

\maketitle
\thispagestyle{empty}
\newpage

\tableofcontents
\newpage

% --- Section 1: Executive Overview ---
\section{Executive Overview}
This report provides a comprehensive cybersecurity assessment for \textbf{[Organization Name]}. The analysis is based on a correlation of organizational security control data, an external network scan, and a review of pre-existing risks.

The assessment reveals a mixed security posture. On a positive note, the external network scan of the target IP address (\texttt{[Target IP]}) did not identify any open ports, suggesting a strong firewall configuration and a minimal external attack surface for that specific asset.

However, significant and critical gaps were identified in the organization's internal security controls. The lack of Multi-Factor Authentication (MFA) for computer logins presents a high-impact risk, as a single compromised password could lead to widespread system access. Furthermore, the absence of a formal Acceptable Use Policy and a structured Security Awareness Training program indicates a low level of security maturity. These procedural and policy-based deficiencies expose the organization to substantial risks from phishing, social engineering, and insider threats.

Immediate action is required to address these foundational security gaps to mitigate the risk of a significant security incident.

% --- Section 2: Organizational Information ---
\section{Organizational Information}
This section details the information provided for the assessment. Placeholders are used where data was not available.

\begin{itemize}
    \item \textbf{Organization Name:} \textbf{[Organization Name]}
    \item \textbf{Primary Email Domain:} \texttt{[Domain]}
    \item \textbf{Scanned External IP:} \texttt{[Client IP]}
    \item \textbf{Network Scan Target:} \texttt{[Target IP]}
\end{itemize}

% --- Section 3: Security Control Review ---
\section{Security Control Review}
The following table summarizes the organization's responses to a security controls questionnaire. "No" answers indicate significant deviations from security best practices and represent potential areas of high risk.

\begin{table}[h!]
\centering
\caption{Security Controls Questionnaire Analysis}
\label{tab:controls}
\begin{tabular}{@{}p{0.6\linewidth} c l@{}}
\toprule
\textbf{Control Question} & \textbf{Response} & \textbf{Assessment} \\
\midrule
Do you require MFA to access email? & \ding{51} & Best Practice Met \\
Do you require MFA to log into computers? & \textbf{\color{red}\ding{55}} & \textbf{Critical Gap Identified} \\
Do you require MFA to access sensitive data systems? & \ding{51} & Best Practice Met \\
Does your organization have an employee acceptable use policy? & \textbf{\color{red}\ding{55}} & \textbf{High-Risk Gap} \\
Does your organization do security awareness training for new employees? & \textbf{\color{red}\ding{55}} & \textbf{Critical Gap Identified} \\
Does your organization do security awareness training for all employees at least once per year? & \textbf{\color{red}\ding{55}} & \textbf{Critical Gap Identified} \\
\bottomrule
\end{tabular}
\end{table}

% --- Section 4: Technical Scan Results ---
\section{Technical Scan Results}
An external network vulnerability scan was conducted to identify open ports and exposed services.

\begin{itemize}
    \item \textbf{Target IP Address:} \texttt{[Target IP]}
    \item \textbf{Scan Date:} Data not provided in scan results.
\end{itemize}

\subsection{Summary of Findings}
The scan completed successfully and found \textbf{no open TCP or UDP ports} on the target system. This indicates a properly configured firewall that denies unsolicited inbound traffic, which is a strong security posture for this specific asset. No vulnerabilities could be identified as no services were exposed.

% --- Section 5: Risk Assessment ---
\section{Risk Assessment}
This section synthesizes findings from the security control review, technical scan, and pre-existing risk data. The primary risks identified are related to organizational and administrative controls. No pre-existing vulnerabilities were reported.

\begin{table}[h!]
\centering
\caption{Identified Risks and Severity}
\label{tab:risks}
\begin{tabular}{@{}p{0.2\linewidth} p{0.5\linewidth} l@{}}
\toprule
\textbf{Risk Name} & \textbf{Description} & \textbf{Severity} \\
\midrule
\textbf{Lack of Endpoint MFA} & The absence of MFA on computer logins means that a single compromised password could grant an attacker full access to an employee's workstation and any connected network resources. & \textbf{Critical} \\
\addlinespace
\textbf{No Security Awareness Program} & Without onboarding or annual training, employees are more likely to fall victim to phishing, social engineering, and malware attacks, making them the weakest link in the organization's defense. & \textbf{Critical} \\
\addlinespace
\textbf{Absence of Acceptable Use Policy} & Without a formal policy, there are no clear guidelines for employees on the safe and acceptable use of company assets, increasing the risk of misuse, data leakage, and non-compliance. & \textbf{High} \\
\bottomrule
\end{tabular}
\end{table}

% --- Section 6: Recommendations ---
\section{Recommendations}
The following actionable recommendations are provided to address the identified risks and improve the organization's overall security posture.

\subsection{Critical Priority}
\begin{enumerate}
    \item \textbf{Implement MFA for All Computer Logins:}
    \begin{itemize}
        \item \textbf{Action:} Deploy a robust MFA solution (e.g., authenticator app, hardware token, biometrics) for all employee and privileged user accounts for logging into workstations and servers.
        \item \textbf{Justification:} This is the single most effective control to prevent unauthorized access resulting from stolen credentials. It mitigates the risk of lateral movement and ransomware deployment.
    \end{itemize}
    
    \item \textbf{Establish a Security Awareness Training Program:}
    \begin{itemize}
        \item \textbf{Action:} Develop and mandate a security awareness training program. This must include a module for all new hires during onboarding and an annual refresher course for all staff. Topics should cover phishing identification, password hygiene, and data handling.
        \item \textbf{Justification:} A well-trained workforce is a critical layer of defense. This reduces the likelihood of human error leading to a security breach.
    \end{itemize}
\end{enumerate}

\subsection{High Priority}
\begin{enumerate}
    \setcounter{enumi}{2} % Continue numbering
    \item \textbf{Develop and Enforce an Acceptable Use Policy (AUP):}
    \begin{itemize}
        \item \textbf{Action:} Create a formal AUP document that clearly defines the rules for using company networks, computers, and data. This policy should be distributed to all employees, who must formally acknowledge their understanding and agreement.
        \item \textbf{Justification:} An AUP establishes a baseline for secure behavior, reduces legal liability, and provides a framework for enforcing security rules.
    \end{itemize}
\end{enumerate}

\end{document}
```