```latex
\documentclass[12pt]{article}

% Preamble: Required Packages
\usepackage[margin=1in]{geometry}
\usepackage{pifont} % For \ding
\usepackage{booktabs} % For professional tables
\usepackage{hyperref} % For clickable links
\usepackage{url} % For URL formatting
\usepackage{seqsplit} % For splitting long strings
\usepackage{graphicx}
\usepackage{xcolor}
\usepackage{datetime}

% Document Metadata
\title{Cybersecurity Posture Assessment Report}
\author{Cybersecurity Analysis Division}
\date{November 22, 2025}

% Hyperref Setup
\hypersetup{
    colorlinks=true,
    linkcolor=blue,
    filecolor=magenta,      
    urlcolor=cyan,
    pdftitle={Cybersecurity Posture Assessment Report},
    pdfpagemode=FullScreen,
}

\begin{document}

\maketitle
\thispagestyle{empty}
\newpage

\tableofcontents
\newpage

% --- 1. EXECUTIVE OVERVIEW ---
\section{Executive Overview}

This report details the findings of a cybersecurity posture assessment conducted on \textbf{[Organization Name]}. The assessment combined a review of organizational security controls, an external network vulnerability scan, and an analysis of pre-existing risks.

The overall security posture requires immediate attention. Several high-impact risks were identified that could expose the organization to significant threats, including account compromise, data breaches, and service disruption.

Key findings include:
\begin{itemize}
    \item \textbf{Critical Risk - Lack of Email MFA:} The absence of Multi-Factor Authentication (MFA) on email accounts represents a critical vulnerability. Email is a primary target for attackers, and a compromised account can lead to widespread system access and data exfiltration.
    \item \textbf{High Risk - Vulnerable Web Server:} The external-facing web server is running an outdated and vulnerable version of Nginx (1.18.0). This exposes the organization to numerous publicly known exploits that could lead to a server compromise.
    \item \textbf{High Risk - Inadequate Employee Onboarding:} New employees do not receive security awareness training, creating a significant gap in the human firewall. New hires are often prime targets for social engineering attacks.
\end{itemize}

This report provides a detailed breakdown of these findings and offers actionable recommendations to mitigate the identified risks and strengthen the organization's overall security posture.

% --- 2. ORGANIZATIONAL INFORMATION ---
\section{Organizational Information}
This section provides the context for the assessment based on the information provided.

\begin{table}[h!]
\centering
\begin{tabular}{@{}ll@{}}
\toprule
\textbf{Attribute} & \textbf{Value} \\ \midrule
Organization Name & \textbf{[Organization Name]} \\
Primary Email Domain & \texttt{[Domain]} \\
External IP Scanned & \texttt{[Client IP]} \\ 
Assessment Date & November 22, 2025 \\
\bottomrule
\end{tabular}
\caption{Client Organizational Data.}
\label{tab:org_info}
\end{table}

% --- 3. SECURITY CONTROL REVIEW ---
\section{Security Control Review (Questionnaire)}
The following table summarizes the organization's responses to a security controls questionnaire. Items marked with a red 'X' (\textcolor{red}{\ding{55}}) indicate a deviation from security best practices and represent a potential risk.

\begin{table}[h!]
\centering
\begin{tabular}{@{}p{0.75\linewidth}c@{}}
\toprule
\textbf{Control Question} & \textbf{Response} \\ \midrule
Do you require MFA to access email? & \textcolor{red}{\ding{55}} \\
Do you require MFA to log into computers? & \textcolor{green}{\ding{51}} \\
Do you require MFA to access sensitive data systems? & \textcolor{green}{\ding{51}} \\
Does your organization have an employee acceptable use policy? & \textcolor{green}{\ding{51}} \\
Does your organization do security awareness training for new employees? & \textcolor{red}{\ding{55}} \\
Does your organization do security awareness training for all employees at least once per year? & \textcolor{green}{\ding{51}} \\
\bottomrule
\end{tabular}
\caption{Security Controls Questionnaire Results.}
\label{tab:controls}
\end{table}

% --- 4. TECHNICAL SCAN RESULTS ---
\section{Technical Scan Results}
An external network scan was performed to identify open ports, running services, and potential vulnerabilities on the organization's public-facing infrastructure.

\subsection{Nmap Scan Findings}
\begin{itemize}
    \item \textbf{Target IP:} \texttt{[Target IP]}
    \item \textbf{Scan Date:} 2025-11-22T10:00:00Z
\end{itemize}

The scan identified one open port, detailed in Table \ref{tab:nmap_results}.

\begin{table}[h!]
\centering
\begin{tabular}{@{}lllll@{}}
\toprule
\textbf{Port} & \textbf{State} & \textbf{Service} & \textbf{Product} & \textbf{Version} \\ \midrule
443/tcp & open & https & nginx & 1.18.0 \\
\bottomrule
\end{tabular}
\caption{Open Ports and Services Detected.}
\label{tab:nmap_results}
\end{table}

\paragraph{Analysis:} The scan confirms a web server is running on port 443 (HTTPS). The server software is identified as \textbf{Nginx version 1.18.0}. This version was released in April 2020 and is now considered outdated. It is known to be vulnerable to multiple Common Vulnerabilities and Exposures (CVEs), some of which are of high or critical severity. Running outdated software on an internet-facing server presents a high risk of compromise.

% --- 5. CONSOLIDATED RISK ASSESSMENT ---
\section{Consolidated Risk Assessment}
This section synthesizes findings from the security control review, technical scan, and pre-existing risk data to provide a consolidated view of the current risk landscape. No pre-existing vulnerabilities were reported.

\begin{table}[h!]
\centering
\begin{tabular}{@{}llll@{}}
\toprule
\textbf{ID} & \textbf{Risk Description} & \textbf{Source} & \textbf{Severity} \\ \midrule
R-01 & Lack of Multi-Factor Authentication on email. & Questionnaire & \textcolor{red}{\textbf{Critical}} \\
R-02 & Vulnerable Nginx web server version (1.18.0). & Network Scan & \textcolor{orange}{\textbf{High}} \\
R-03 & No security awareness training for new employees. & Questionnaire & \textcolor{orange}{\textbf{High}} \\
\bottomrule
\end{tabular}
\caption{Summary of Identified Risks.}
\label{tab:risks}
\end{table}

% --- 6. RECOMMENDATIONS ---
\section{Recommendations}
The following actions are recommended to mitigate the identified risks and improve the overall security posture of \textbf{[Organization Name]}.

\subsection{R-01: Implement MFA on Email (Critical)}
\textbf{Action:} Immediately enforce Multi-Factor Authentication (MFA) for all user accounts across the email system (\texttt{[Domain]}). This is the single most effective control to prevent unauthorized account access resulting from credential theft or phishing attacks.
\paragraph{Priority:} Immediate.

\subsection{R-02: Patch Web Server (High)}
\textbf{Action:} The Nginx server at \texttt{[Target IP]} must be upgraded from version 1.18.0 to the latest stable version. This will patch numerous known vulnerabilities, significantly reducing the server's attack surface. A patch and vulnerability management program should be implemented to prevent such issues in the future.
\paragraph{Priority:} High.

\subsection{R-03: Institute Onboarding Security Training (High)}
\textbf{Action:} Develop and mandate a security awareness training module as a required part of the onboarding process for all new employees. This training should cover, at a minimum, phishing identification, acceptable use policies, and password hygiene. This will help integrate new team members into the organization's security culture from day one.
\paragraph{Priority:} High.

\end{document}
```