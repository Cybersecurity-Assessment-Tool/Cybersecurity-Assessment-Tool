```latex
\documentclass[12pt, a4paper]{article}

% Preamble: Required Packages and Document Setup
\usepackage[margin=1in]{geometry}
\usepackage{pifont} % For checkmarks and crosses
\usepackage{booktabs} % For professional-looking tables
\usepackage{graphicx}
\usepackage[table]{xcolor}
\usepackage{hyperref}
\usepackage{url}
\usepackage{seqsplit} % For splitting long strings in tt font

% Color and Hyperlink Setup
\definecolor{darkblue}{rgb}{0.0, 0.0, 0.55}
\hypersetup{
    colorlinks=true,
    linkcolor=darkblue,
    filecolor=darkblue,      
    urlcolor=darkblue,
    citecolor=darkblue,
}

% Document Title and Author Information
\title{Cybersecurity Posture Assessment Report}
\author{Cybersecurity Analysis Division}
\date{\today}

\begin{document}

\maketitle
\thispagestyle{empty}
\newpage

\tableofcontents
\newpage

% --- 1. Executive Summary ---
\section{Executive Summary}
This report provides a cybersecurity posture assessment for \textbf{[Organization Name]}, conducted on \today. The analysis is based on a combination of self-reported organizational security controls, an external network vulnerability scan, and a review of pre-existing risks.

The assessment identified two significant areas of concern requiring immediate attention. The most critical finding is the absence of Multi-Factor Authentication (MFA) for email access, which exposes the organization to a high risk of Business Email Compromise (BEC) and unauthorized account access. A second high-risk finding is the lack of mandatory, annual security awareness training for all employees, which increases susceptibility to phishing and other social engineering attacks.

On a positive note, the external network scan of the target system \texttt{[Client IP]} did not identify any open ports or exposed services. This suggests a well-configured network perimeter for the scanned asset. No pre-existing vulnerabilities were reported for inclusion in this assessment.

Overall, while the organization has implemented some foundational security controls, the identified gaps in authentication and employee training present a substantial risk. We strongly recommend prioritizing the remediation steps outlined in Section 6 to mitigate these threats and improve the organization's overall security posture.

% --- 2. Organizational Information ---
\section{Organizational Information}
The following information was used as the basis for this assessment. The data has been anonymized as per the engagement agreement.

\begin{table}[h!]
\centering
\caption{Client Details}
\label{tab:client-details}
\begin{tabular}{@{}ll@{}}
\toprule
\textbf{Attribute} & \textbf{Value} \\ \midrule
Organization Name & \textbf{[Organization Name]} \\
Primary Domain & \texttt{[Domain]} \\
Scanned External IP & \texttt{[Client IP]} \\ \bottomrule
\end{tabular}
\end{table}

% --- 3. Security Control Review ---
\section{Security Control Review}
The following table summarizes the organization's responses to a security controls questionnaire. This review provides insight into the current policies and procedures governing the organization's security environment. Answers marked with \ding{55} indicate a potential control gap.

\begin{table}[h!]
\centering
\caption{Security Controls Questionnaire Results}
\label{tab:controls}
\begin{tabular}{@{}p{0.8\linewidth}c@{}}
\toprule
\textbf{Question} & \textbf{Response} \\ \midrule
Do you require MFA to access email? & \ding{55} \\
Do you require MFA to log into computers? & \ding{51} \\
Do you require MFA to access sensitive data systems? & \ding{51} \\
Does your organization have an employee acceptable use policy? & \ding{51} \\
Does your organization do security awareness training for new employees? & \ding{51} \\
Does your organization do security awareness training for all employees at least once per year? & \ding{55} \\ \bottomrule
\end{tabular}
\end{table}

\subsection*{Analysis of Control Gaps}
Two significant control gaps were identified from the questionnaire:
\begin{itemize}
    \item \textbf{No MFA for Email:} This is a critical security weakness. Email accounts are primary targets for attackers seeking to gain an initial foothold, conduct phishing campaigns, or execute Business Email Compromise (BEC) attacks.
    \item \textbf{No Annual Security Awareness Training:} While training for new hires is in place, the lack of ongoing, annual training for all staff means that cybersecurity knowledge is not reinforced. The threat landscape evolves continuously, and so must employee awareness.
\end{itemize}

% --- 4. Technical Scan Results ---
\section{Technical Scan Results}
An external network scan was performed to identify any exposed services or potential vulnerabilities on the public-facing infrastructure.

\begin{itemize}
    \item \textbf{Target IP Address:} \texttt{[Target IP]}
    \item \textbf{Scan Date:} [Scan Date]
\end{itemize}

\subsection*{Findings}
The network scan performed on the target IP address revealed \textbf{no open ports}. This indicates a strong network perimeter security posture for the scanned host, as no services were found to be exposed to the public internet. This significantly reduces the external attack surface of the asset.

% --- 5. Consolidated Risk Assessment ---
\section{Consolidated Risk Assessment}
This section consolidates findings from the security control review, technical scan, and pre-existing risk data. Each identified risk is assigned a severity level to aid in prioritization.

\begin{table}[h!]
\centering
\caption{Summary of Identified Risks}
\label{tab:risks}
\begin{tabular}{@{}p{0.25\linewidth}p{0.5\linewidth}l@{}}
\toprule
\textbf{Risk Name} & \textbf{Overview} & \textbf{Severity} \\ \midrule
\rowcolor{red!25}
Lack of MFA on Email & The absence of MFA on email accounts allows for takeover with a single compromised password, enabling data breaches and BEC. & \textbf{Critical} \\
\rowcolor{orange!25}
Lack of Annual Security Training & Without regular training, employees are more likely to fall victim to phishing and social engineering, leading to security incidents. & \textbf{High} \\
\midrule
\multicolumn{3}{@{}l@{}}{\textit{Note: No vulnerabilities were discovered in the technical scan.}} \\
\multicolumn{3}{@{}l@{}}{\textit{Note: No pre-existing risks were provided for this assessment.}} \\
\bottomrule
\end{tabular}
\end{table}

% --- 6. Recommendations ---
\section{Recommendations}
The following actions are recommended to address the identified risks. Recommendations are prioritized based on the severity of the corresponding risk.

\subsection*{Recommendation 1: Implement MFA for Email Access (Critical)}
\textbf{Description:} Enforce mandatory Multi-Factor Authentication (MFA) for all user access to the organization's email system. This is the single most effective control to prevent unauthorized account access.
\begin{itemize}
    \item \textbf{Action 1:} Select and configure an MFA solution compatible with your email provider (e.g., Microsoft Authenticator, Google Authenticator, hardware tokens).
    \item \textbf{Action 2:} Develop a phased rollout plan, starting with IT staff and executives, then extending to all employees.
    \item \textbf{Action 3:} Provide clear instructions and support to all users during the transition to minimize disruption.
\end{itemize}

\subsection*{Recommendation 2: Establish Annual Security Awareness Training (High)}
\textbf{Description:} Develop and implement a mandatory security awareness training program that all employees must complete at least once per year.
\begin{itemize}
    \item \textbf{Action 1:} Procure or create training content that covers modern threats, including phishing, ransomware, password security, and social engineering.
    \item \textbf{Action 2:} Implement a system to track training completion for all employees to ensure compliance.
    \item \textbf{Action 3:} Supplement annual training with periodic phishing simulation campaigns to test and reinforce employee knowledge in a practical setting.
\end{itemize}

\end{document}
```