```latex
\documentclass[12pt, a4paper]{article}

% Preamble: Required Packages
\usepackage[margin=1in]{geometry}
\usepackage{pifont} % For checkmarks and crosses
\usepackage{booktabs} % For professional tables
\usepackage{hyperref} % For clickable links and references
\usepackage{url} % For formatting URLs
\usepackage{seqsplit} % To split long strings in tt font
\usepackage{graphicx} % For logo
\usepackage{xcolor} % For colors

% Document Information
\title{Cybersecurity Posture Assessment Report}
\author{Cybersecurity Analysis Division}
\date{November 22, 2025}

% Hyperref Setup
\hypersetup{
    colorlinks=true,
    linkcolor=blue,
    filecolor=magenta,      
    urlcolor=cyan,
    pdftitle={Cybersecurity Posture Assessment Report},
    pdfpagemode=FullScreen,
}

% Custom Commands
\newcommand{\yes}{\ding{51}}
\newcommand{\no}{\ding{55}}

\begin{document}

\begin{titlepage}
    \centering
    \vfill
    {\Huge\bfseries Cybersecurity Posture Assessment Report\par}
    \vspace{1.5cm}
    {\Large Prepared for:\par}
    \vspace{0.5cm}
    {\Huge\bfseries \textbf{[Organization Name]}}\par
    \vfill
    \vspace{1cm}
    {\large \today\par}
    \vspace{1.5cm}
    {\large \textit{This report contains sensitive information and should be handled with care.}\par}
\end{titlepage}

\tableofcontents
\newpage

\section{Executive Summary}
This report provides a comprehensive analysis of the cybersecurity posture for \textbf{[Organization Name]}, based on data gathered on November 22, 2025. The assessment combines a review of organizational security controls, an external network scan, and an evaluation of pre-existing risks.

The analysis revealed several critical and high-risk findings that require immediate attention. Key issues include:
\begin{itemize}
    \item \textbf{Critical Control Gap:} The absence of Multi-Factor Authentication (MFA) on email accounts presents a critical risk, leaving the organization highly susceptible to Business Email Compromise (BEC) and phishing attacks.
    \item \textbf{High-Risk Human Factor:} A complete lack of security awareness training for both new and existing employees significantly increases the likelihood of successful social engineering attacks.
    \item \textbf{High-Risk Technical Vulnerability:} The external-facing web server is running an outdated and vulnerable version of Nginx (1.18.0), which could be exploited by attackers to compromise the system.
\end{itemize}

The overall security posture is considered weak due to these fundamental gaps. This report outlines actionable recommendations to mitigate these risks and strengthen the organization's defenses.

\section{Organizational Information}
The following details were used as the basis for this assessment. Due to the anonymized nature of the provided data, placeholders have been used where necessary.

\begin{itemize}
    \item \textbf{Organization Name:} \textbf{[Organization Name]}
    \item \textbf{Primary Email Domain:} \texttt{[Domain]}
    \item \textbf{Known External IP Address:} \texttt{[Client IP]}
\end{itemize}

\section{Security Control Review}
A review of the organization's security controls was conducted via a standardized questionnaire. The responses highlight significant gaps in foundational security practices. "No" responses indicate a lack of a critical control and are flagged as high-impact findings.

\begin{table}[h!]
\centering
\caption{Security Controls Questionnaire Results}
\begin{tabular}{p{0.75\linewidth} c}
\toprule
\textbf{Control Question} & \textbf{Response} \\
\midrule
Do you require MFA to access email? & \no \\
Do you require MFA to log into computers? & \yes \\
Do you require MFA to access sensitive data systems? & \yes \\
Does your organization have an employee acceptable use policy? & \yes \\
Does your organization do security awareness training for new employees? & \no \\
Does your organization do security awareness training for all employees at least once per year? & \no \\
\bottomrule
\end{tabular}
\end{table}

The two most concerning findings from this review are the lack of MFA for email and the absence of a security awareness training program. Email is the number one vector for cyberattacks, and failing to protect it with MFA is a critical oversight. Furthermore, without training, employees are ill-equipped to identify and report phishing attempts, making them the weakest link in the security chain.

\section{Technical Scan Results}
An external network scan was performed to identify open ports and exposed services.

\begin{itemize}
    \item \textbf{Scan Date:} 2025-11-22T10:00:00Z
    \item \textbf{Target IP:} \texttt{[Target IP]}
\end{itemize}

\begin{table}[h!]
\centering
\caption{Open Ports and Services on \texttt{[Target IP]}}
\begin{tabular}{l l l l}
\toprule
\textbf{Port} & \textbf{State} & \textbf{Service} & \textbf{Product \& Version} \\
\midrule
443/tcp & open & https & nginx 1.18.0 \\
\bottomrule
\end{tabular}
\end{table}

\subsection{Analysis of Technical Findings}
The scan identified a single open port, 443 (HTTPS), running an Nginx web server. The detected version, \textbf{Nginx 1.18.0}, was released in April 2020 and is now considered severely outdated. This version is known to be vulnerable to multiple security exploits, including but not limited to CVE-2021-23017, which can lead to request smuggling and potential system compromise. Exposing a vulnerable web server to the internet presents a direct and high-risk attack vector.

\section{Correlated Risk Assessment}
This section synthesizes findings from the control review and the technical scan. No pre-existing risks were provided, so all items listed below are newly identified during this assessment.

\begin{table}[h!]
\centering
\caption{Summary of Identified Risks}
\begin{tabular}{p{0.1\linewidth} p{0.25\linewidth} p{0.4\linewidth} p{0.1\linewidth}}
\toprule
\textbf{ID} & \textbf{Risk Name} & \textbf{Description} & \textbf{Severity} \\
\midrule
RISK-001 & Lack of MFA on Email & The absence of MFA on email accounts allows for account takeover with only a compromised password, facilitating phishing and data breaches. & \textbf{Critical} \\
\addlinespace
RISK-002 & Vulnerable Web Server Software & The public-facing web server runs an outdated version of Nginx (1.18.0) with known vulnerabilities, exposing it to remote exploitation. & \textbf{High} \\
\addlinespace
RISK-003 & Inadequate Security Awareness Training & Employees are not trained to recognize or respond to security threats, making the organization highly vulnerable to social engineering and phishing. & \textbf{High} \\
\bottomrule
\end{tabular}
\end{table}

\section{Recommendations}
The following actions are recommended to mitigate the identified risks and improve the overall security posture of \textbf{[Organization Name]}. Recommendations are prioritized by severity.

\begin{enumerate}
    \item \textbf{[Critical] Implement MFA for Email (RISK-001):}
    \begin{itemize}
        \item Immediately enforce MFA for all user access to the email system (\texttt{[Domain]}).
        \item Prioritize phishing-resistant MFA methods such as FIDO2 security keys or authenticator apps over less secure methods like SMS.
        \item Audit all accounts for signs of existing compromise before and after MFA deployment.
    \end{itemize}
    \vspace{0.5cm}
    \item \textbf{[High] Upgrade Web Server Software (RISK-002):}
    \begin{itemize}
        \item Develop a patch management plan to upgrade the Nginx server on host \texttt{[Target IP]} to the latest stable version.
        \item Before deploying to production, test the upgrade in a staging environment to ensure application compatibility.
        \item Implement a regular patch management cycle to ensure all public-facing software is kept up-to-date.
    \end{itemize}
    \vspace{0.5cm}
    \item \textbf{[High] Establish a Security Awareness Program (RISK-003):}
    \begin{itemize}
        \item Procure and deploy a security awareness training platform for all employees.
        \item Mandate foundational training for all new hires as part of the onboarding process.
        \item Conduct annual refresher training and regular phishing simulations for all staff to build a culture of security consciousness.
    \end{itemize}
\end{enumerate}

\end{document}
```