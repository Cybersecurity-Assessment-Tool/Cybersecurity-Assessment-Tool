```latex
\documentclass[12pt]{article}

% Preamble: Required Packages
\usepackage[margin=1in]{geometry}
\usepackage{pifont} % Required for \ding
\usepackage{booktabs} % For professional-looking tables
\usepackage{hyperref} % For clickable links and references
\usepackage{url} % For formatting URLs
\usepackage{seqsplit} % To split long strings without breaking
\usepackage{fancyhdr} % For custom headers and footers
\usepackage{xcolor} % For custom colors

% --- Document Setup ---

% Hyperlink setup for a clean look
\hypersetup{
    colorlinks=true,
    linkcolor=black,
    filecolor=magenta,
    urlcolor=blue,
    pdftitle={Cybersecurity Posture Assessment Report},
    pdfpagemode=FullScreen,
}

% Header and Footer Configuration
\pagestyle{fancy}
\fancyhf{} % Clear all header and footer fields
\rhead{Cybersecurity Assessment Report}
\lhead{\textbf{[Organization Name]}}
\cfoot{Page \thepage}
\renewcommand{\headrulewidth}{0.4pt}
\renewcommand{\footrulewidth}{0.4pt}

% --- Document Start ---

\begin{document}

% --- Title Page ---
\begin{titlepage}
    \centering
    \vspace*{1cm}
    \Huge\textbf{Cybersecurity Posture Assessment Report}
    \vspace{1.5cm}
    \Large
    \textbf{Prepared for:}\\
    \vspace{0.5cm}
    \textbf{[Organization Name]}
    \vspace{2cm}
    \large
    \textbf{Date of Report:}\\
    \vspace{0.5cm}
    \today
    \vfill
    \large
    \textbf{Confidentiality Notice:}\\
    \vspace{0.5cm}
    \parbox{0.8\textwidth}{\small This document contains confidential and sensitive information. It is intended solely for the use of the individual or entity to whom it is addressed. Dissemination, distribution, or copying of this document, or any of its contents, by anyone other than the intended recipient is strictly prohibited.}
\end{titlepage}

\tableofcontents
\newpage

% --- Section 1: Executive Summary ---
\section*{1. Executive Summary}

This report details the findings of a cybersecurity assessment conducted for \textbf{[Organization Name]}. The analysis combines a review of organizational security controls, an external network vulnerability scan, and a summary of pre-existing risks.

The assessment has identified several \textbf{critical and high-risk vulnerabilities} that require immediate attention. The most significant findings include:

\begin{itemize}
    \item \textbf{Complete Lack of Multi-Factor Authentication (MFA):} The organization does not enforce MFA for email, computer logins, or access to sensitive data systems. This represents a critical control gap that dramatically increases the risk of unauthorized access and account compromise.
    \item \textbf{Publicly Exposed End-of-Life Database:} An external scan identified a MySQL database (port 3306) directly accessible from the internet. The running version, MySQL 5.7.33, reached its official End-of-Life (EOL) in October 2023 and no longer receives security updates, leaving it exposed to known and future exploits.
    \item \textbf{Insufficient Security Awareness Program:} New employees do not receive mandatory security awareness training, creating a significant vulnerability to social engineering and phishing attacks from their first day.
\end{itemize}

The combination of these vulnerabilities places the organization at a high risk of a significant security breach. The recommendations provided in this report are prioritized to address the most critical issues first. We strongly advise immediate action to remediate these findings.

% --- Section 2: Organizational Information ---
\section*{2. Organizational Information}
This section provides the context for the assessment based on the information provided.

\begin{tabular}{@{}ll}
\toprule
\textbf{Attribute} & \textbf{Value} \\
\midrule
Organization Name & \textbf{[Organization Name]} \\
Primary Email Domain & \texttt{[Domain]} \\
External IP Scanned & \texttt{[Client IP]} \\
\bottomrule
\end{tabular}

% --- Section 3: Security Control Review ---
\section*{3. Security Control Review (Questionnaire Analysis)}
The following table summarizes the organization's security posture based on a self-assessment questionnaire. "No" answers indicate significant gaps in security controls.

\begin{table}[h!]
\centering
\begin{tabular}{@{}p{0.6\linewidth} c p{0.25\linewidth}@{}}
\toprule
\textbf{Control Question} & \textbf{Status} & \textbf{Analyst Assessment} \\
\midrule
Do you require MFA to access email? & \textcolor{red}{\ding{55}} & \textbf{Critical Gap.} Lack of MFA on email is a primary vector for account takeover. \\
\addlinespace
Do you require MFA to log into computers? & \textcolor{red}{\ding{55}} & \textbf{Critical Gap.} Compromised credentials could lead to direct network access. \\
\addlinespace
Do you require MFA to access sensitive data systems? & \textcolor{red}{\ding{55}} & \textbf{Critical Gap.} Directly exposes sensitive data to credential theft. \\
\addlinespace
Does your organization have an employee acceptable use policy? & \textcolor{green}{\ding{51}} & Foundational policy is in place. \\
\addlinespace
Does your organization do security awareness training for new employees? & \textcolor{red}{\ding{55}} & \textbf{High Risk.} New hires are a common target and are left unprepared. \\
\addlinespace
Does your organization do security awareness training for all employees at least once per year? & \textcolor{green}{\ding{51}} & Good practice for existing staff. \\
\bottomrule
\end{tabular}
\caption{Analysis of Organizational Security Controls.}
\end{table}

% --- Section 4: Technical Scan Results ---
\section*{4. Technical Scan Results (Nmap)}
An external network scan was performed against the target IP address to identify open ports and exposed services.

\begin{itemize}
    \item \textbf{Target IP Address:} \texttt{[Target IP]}
    \item \textbf{Scan Status:} Host is UP.
\end{itemize}

\begin{table}[h!]
\centering
\begin{tabular}{@{}llllll@{}}
\toprule
\textbf{Port} & \textbf{State} & \textbf{Service} & \textbf{Product} & \textbf{Version} & \textbf{Analyst Notes} \\
\midrule
3306/tcp & open & mysql & MySQL & 5.7.33 & \parbox{4cm}{\textbf{Critical Finding.} Version is End-of-Life (EOL) and publicly exposed.} \\
\bottomrule
\end{tabular}
\caption{Open Ports and Services Identified.}
\end{table}

The scan confirms the pre-existing risk "Database Exposure" and elevates its severity due to the EOL status of the software. EOL software does not receive security patches from the vendor, making it a prime target for attackers who can exploit publicly known vulnerabilities without fear of them being fixed.

% --- Section 5: Correlated Risk Assessment ---
\section*{5. Correlated Risk Assessment}
This section synthesizes findings from the security control review, technical scan, and pre-existing risk data into a prioritized list.

\begin{table}[h!]
\centering
\begin{tabular}{@{}p{0.25\linewidth} p{0.15\linewidth} p{0.5\linewidth}@{}}
\toprule
\textbf{Risk Title} & \textbf{Severity} & \textbf{Description} \\
\midrule
\textbf{Exposed End-of-Life Database} & \textbf{Critical (9.8)} & A MySQL database (v5.7.33) is publicly accessible on port 3306. This version is EOL and unpatched. This directly confirms and elevates the pre-existing "Database Exposure" risk. \\
\addlinespace
\textbf{Lack of Multi-Factor Authentication} & \textbf{Critical (9.1)} & The absence of MFA on all critical systems (email, logins, sensitive data) makes the organization highly susceptible to account compromise via credential theft or phishing. This risk is amplified by the exposed database. \\
\addlinespace
\textbf{Insufficient Security Awareness Training} & \textbf{High (7.2)} & New employees do not receive security training, making them vulnerable to social engineering. This increases the likelihood of an initial compromise that could exploit other weaknesses. \\
\bottomrule
\end{tabular}
\caption{Summary of Identified and Correlated Risks.}
\end{table}

% --- Section 6: Recommendations ---
\section*{6. Recommendations}
The following actionable steps are recommended to mitigate the identified risks. They are prioritized based on severity and potential impact.

\subsection*{Priority 1: Immediate Actions (Within 72 Hours)}
\begin{enumerate}
    \item \textbf{Restrict Database Access:} Immediately apply firewall rules to deny all public access to TCP port 3306 on \texttt{[Target IP]}. Access should only be permitted from trusted internal IP addresses or through a secure VPN connection. This directly addresses the "Database Exposure" risk.
\end{enumerate}

\subsection*{Priority 2: High-Priority Actions (Within 30 Days)}
\begin{enumerate}
    \item \textbf{Deploy Multi-Factor Authentication (MFA):} Begin a phased rollout of MFA across the organization, starting with the most critical assets:
    \begin{itemize}
        \item \textbf{Phase 1 (Week 1):} Email and administrative accounts.
        \item \textbf{Phase 2 (Week 2-3):} Access to all systems containing sensitive data.
        \item \textbf{Phase 3 (Week 4):} All employee computer logins.
    \end{itemize}
    \item \textbf{Plan Database Upgrade:} Develop a project plan to migrate the MySQL 5.7.33 database to a fully supported version (e.g., MySQL 8.x). This plan should include data backup, testing, and a scheduled maintenance window for the upgrade. This is the long-term fix for the EOL software risk.
\end{enumerate}

\subsection*{Priority 3: Medium-Priority Actions (Within 90 Days)}
\begin{enumerate}
    \item \textbf{Implement New Hire Security Training:} Develop and enforce a mandatory security awareness training module as part of the new employee onboarding process. This training should cover phishing, acceptable use, password hygiene, and incident reporting.
\end{enumerate}

% --- Section 7: Conclusion ---
\section*{7. Conclusion}
The security posture of \textbf{[Organization Name]} currently contains critical vulnerabilities that expose it to a high likelihood of a successful cyberattack. The public-facing, end-of-life database, combined with the systemic lack of multi-factor authentication, creates a "perfect storm" scenario for a data breach.

By implementing the prioritized recommendations outlined in this report, the organization can significantly reduce its attack surface and build a more resilient security foundation. We urge the management to allocate the necessary resources to address these findings without delay.

\end{document}
```