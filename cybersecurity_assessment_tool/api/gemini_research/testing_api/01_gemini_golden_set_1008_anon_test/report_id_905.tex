```latex
\documentclass[12pt]{article}

% Preamble: Required Packages
\usepackage[margin=1in]{geometry}
\usepackage{pifont} % For checkmarks and crosses
\usepackage{booktabs} % For professional tables
\usepackage{hyperref} % For hyperlinks
\usepackage{url}      % For URL formatting
\usepackage{seqsplit} % For splitting long strings in tt font

% Document Metadata
\title{Cybersecurity Posture Assessment Report}
\author{Cybersecurity Analysis Division}
\date{\today}

\begin{document}

\maketitle
\thispagestyle{empty}
\newpage
\tableofcontents
\newpage

% --- 1. Executive Overview ---
\section{Executive Overview}

This report provides a cybersecurity assessment for \textbf{[Organization Name]}, based on an analysis of network scan data, organizational security controls, and pre-existing risk information. The assessment reveals a high-risk security posture requiring immediate attention.

The most critical finding is the direct exposure of Remote Desktop Protocol (RDP) on port 3389 to the public internet at \texttt{[Target IP]}. This vulnerability, with a CVSS score of 9.0, presents a severe and immediate threat, as it is a primary vector for ransomware attacks and unauthorized access.

Furthermore, significant gaps were identified in the organization's access control and administrative policies. The lack of mandatory Multi-Factor Authentication (MFA) for email and sensitive data systems critically weakens defenses against credential theft. The absence of an acceptable use policy and mandatory annual security training for all staff exacerbates the risk of human error leading to a security incident.

Immediate remediation of the exposed RDP service and the implementation of MFA are paramount to reducing the organization's attack surface and mitigating the most severe risks.

% --- 2. Organizational Information ---
\section{Organizational Information}

This section details the organizational information used as the basis for this assessment. The data provided was anonymized.

\begin{itemize}
    \item \textbf{Organization Name:} \textbf{[Organization Name]}
    \item \textbf{Primary Email Domain:} \texttt{[Domain]}
    \item \textbf{External IP Address Scanned:} \texttt{[Client IP]}
\end{itemize}

% --- 3. Security Control Review (Questionnaire Analysis) ---
\section{Security Control Review}

An analysis of the organization's security questionnaire responses highlights critical deficiencies in foundational security controls. "No" answers indicate significant gaps that increase organizational risk.

\begin{table}[h!]
\centering
\caption{Security Controls Questionnaire Analysis}
\begin{tabular}{p{8cm} c l}
\toprule
\textbf{Control Question} & \textbf{Status} & \textbf{Assessment} \\
\midrule
Do you require MFA to access email? & \ding{55} & \textbf{Critical Gap} \\
Do you require MFA to log into computers? & \ding{51} & Implemented \\
Do you require MFA to access sensitive data systems? & \ding{55} & \textbf{Critical Gap} \\
Does your organization have an employee acceptable use policy? & \ding{55} & \textbf{High Risk} \\
Does your organization do security awareness training for new employees? & \ding{51} & Implemented \\
Does your organization do security awareness training for all employees at least once per year? & \ding{55} & \textbf{High Risk} \\
\bottomrule
\end{tabular}
\end{table}

The lack of MFA for email and sensitive systems, combined with missing policies and recurring training, creates a permissive environment for security incidents to occur and escalate.

% --- 4. Technical Scan Results ---
\section{Technical Scan Results}

A network scan was performed on the organization's external IP address. The scan identified one open port, which confirms a known high-risk exposure.

\begin{itemize}
    \item \textbf{Target IP Address:} \texttt{[Target IP]}
    \item \textbf{Scan Date:} Not provided in scan metadata.
\end{itemize}

\begin{table}[h!]
\centering
\caption{Open Port Analysis}
\begin{tabular}{c c l l}
\toprule
\textbf{Port} & \textbf{State} & \textbf{Service} & \textbf{Notes} \\
\midrule
3389/tcp & Open & ms-wbt-server & Remote Desktop Protocol (RDP). \\
 & & & A primary target for attackers. \\
\bottomrule
\end{tabular}
\end{table}

\subsection*{Analysis}
The open RDP port directly corresponds to the pre-existing risk documented in the organization's risk register. This finding is critical, as exposed RDP is frequently exploited by threat actors for initial access, leading to ransomware deployment and data exfiltration. No version information was available from the scan, but any version of RDP exposed directly to the internet is considered a critical vulnerability.

% --- 5. Consolidated Risk Assessment ---
\section{Consolidated Risk Assessment}

This section synthesizes findings from the technical scan, control review, and existing risk data into a prioritized list of security risks.

\begin{table}[h!]
\centering
\caption{Summary of Identified Risks}
\begin{tabular}{p{2cm} p{6cm} p{2.5cm} p{3cm}}
\toprule
\textbf{Risk ID} & \textbf{Description} & \textbf{Severity} & \textbf{Affected Systems} \\
\midrule
RISK-001 & \textbf{Public RDP Exposure:} The Remote Desktop Protocol service is exposed to the internet, allowing for brute-force attacks and exploitation of RDP vulnerabilities. & \textbf{Critical (9.0)} & External Server at \texttt{[Target IP]} \\
\addlinespace
RISK-002 & \textbf{Lack of MFA on Critical Systems:} Email and sensitive data systems do not require Multi-Factor Authentication, leaving them vulnerable to credential compromise. & \textbf{Critical} & Email infrastructure, sensitive data repositories \\
\addlinespace
RISK-003 & \textbf{Insufficient Security Policies \& Training:} The absence of an acceptable use policy and mandatory annual training increases the likelihood of security incidents caused by human error. & \textbf{High} & Organization-wide \\
\bottomrule
\end{tabular}
\end{table}

% --- 6. Recommendations ---
\section{Recommendations}

The following actions are recommended to mitigate the identified risks. They are prioritized based on severity and potential impact.

\subsection*{Immediate Priority (Remediate within 72 hours)}
\begin{enumerate}
    \item \textbf{Remediate RDP Exposure (RISK-001):}
    \begin{itemize}
        \item \textbf{Short-Term Fix:} Immediately implement a firewall rule to block all inbound traffic to TCP port 3389 on \texttt{[Target IP]}.
        \item \textbf{Long-Term Solution:} For necessary remote access, deploy a Virtual Private Network (VPN) with strong authentication (MFA). Access to RDP should only be permitted through the VPN tunnel.
    \end{itemize}
\end{enumerate}

\subsection*{High Priority (Remediate within 30 days)}
\begin{enumerate}
    \setcounter{enumi}{1} % Continue numbering
    \item \textbf{Implement Comprehensive MFA (RISK-002):}
    \begin{itemize}
        \item Enable and enforce MFA for all user accounts across the email system (e.g., Office 365, Google Workspace).
        \item Enable and enforce MFA for all systems hosting or providing access to sensitive organizational data.
    \end{itemize}
    \item \textbf{Develop and Implement Foundational Policies (RISK-003):}
    \begin{itemize}
        \item Create and formally adopt an Employee Acceptable Use Policy that defines rules for using company assets and data.
        \item Require all employees to read and acknowledge the policy upon hire and annually thereafter.
    \end{itemize}
    \item \textbf{Establish a Security Training Program (RISK-003):}
    \begin{itemize}
        \item Implement a mandatory security awareness training program for all employees to be completed at least once per year.
        \item Topics should include phishing identification, password hygiene, and data handling best practices.
    \end{itemize}
\end{enumerate}

\end{document}
```