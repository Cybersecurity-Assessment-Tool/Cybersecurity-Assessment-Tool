```latex
\documentclass[12pt]{article}

% ----------------------------------------------------------------------
% PREAMBLE
% ----------------------------------------------------------------------
\usepackage[margin=1in]{geometry}
\usepackage{pifont} % For checkmarks and crosses
\usepackage{booktabs} % For professional tables
\usepackage{hyperref} % For clickable links
\usepackage{url} % For URL formatting
\usepackage{seqsplit} % For splitting long strings in tt font
\usepackage{graphicx} % For logo (optional)
\usepackage{xcolor} % For colors

% Define colors for severity levels
\definecolor{critical}{HTML}{990000}
\definecolor{high}{HTML}{D14302}
\definecolor{medium}{HTML}{E69500}
\definecolor{low}{HTML}{339900}

% Hyperref setup
\hypersetup{
    colorlinks=true,
    linkcolor=blue,
    filecolor=magenta,      
    urlcolor=cyan,
    pdftitle={Cybersecurity Posture Report},
    pdfauthor={Cybersecurity Analysis Cell},
    pdfsubject={Security Assessment},
    pdfkeywords={Security, Assessment, Report},
    bookmarks=true
}

% Define checkmark and cross symbols for clarity
\newcommand{\cmark}{\ding{51}}
\newcommand{\xmark}{\ding{55}}

% ----------------------------------------------------------------------
% DOCUMENT START
% ----------------------------------------------------------------------
\begin{document}

% ----------------------------------------------------------------------
% TITLE PAGE
% ----------------------------------------------------------------------
\begin{titlepage}
    \centering
    \vspace*{2cm}
    
    \Huge \textbf{Cybersecurity Posture Report}
    
    \vspace{1.5cm}
    
    \Large Prepared for:
    
    \vspace{0.5cm}
    
    \huge \textbf{[Organization Name]}
    
    \vspace{3cm}
    
    {\large \today}
    
    \vfill
    
    \large \textbf{CONFIDENTIAL}
    
    \vspace{1cm}
    
    \normalsize This document contains sensitive information and is intended solely for the use of the designated recipient. Unauthorized distribution is prohibited.
\end{titlepage}

% ----------------------------------------------------------------------
% TABLE OF CONTENTS
% ----------------------------------------------------------------------
\tableofcontents
\newpage

% ----------------------------------------------------------------------
% 1. EXECUTIVE SUMMARY
% ----------------------------------------------------------------------
\section{Executive Summary}

This report details the findings of a cybersecurity assessment conducted on \textbf{November 22, 2025}. The evaluation combined a review of organizational security controls via a questionnaire, an external network scan of public-facing assets, and an analysis of pre-existing risks.

The assessment identified several critical and high-risk security gaps that require immediate attention. Key findings include:

\begin{itemize}
    \item \textbf{Critical Control Gap:} Multi-Factor Authentication (MFA) is not enforced for accessing sensitive data systems. This significantly increases the risk of unauthorized access and data breaches.
    \item \textbf{High-Risk Technical Vulnerability:} The external-facing web server at \texttt{[Target IP]} is running an outdated version of Nginx (1.18.0). This version is no longer supported and is susceptible to numerous publicly known vulnerabilities.
    \item \textbf{High-Risk Policy Gap:} The organization lacks a formal employee Acceptable Use Policy (AUP). This absence creates ambiguity regarding security responsibilities and acceptable behavior, increasing the likelihood of insider threats and policy violations.
\end{itemize}

While the organization has implemented some positive security controls, such as MFA for email and computer logins, the identified vulnerabilities present a significant threat to the confidentiality, integrity, and availability of its data and systems. This report provides a detailed analysis of these findings and offers actionable recommendations to mitigate the identified risks.

% ----------------------------------------------------------------------
% 2. ORGANIZATIONAL INFORMATION
% ----------------------------------------------------------------------
\section{Organizational Information}

This section provides the organizational details used as the basis for this assessment. As the provided data was anonymized, placeholders have been used.

\begin{tabular}{@{}ll}
    \toprule
    \textbf{Attribute} & \textbf{Value} \\
    \midrule
    Organization Name & \textbf{[Organization Name]} \\
    Primary Email Domain & \texttt{[Domain]} \\
    Assessed External IP & \texttt{[Client IP]} \\
    \bottomrule
\end{tabular}

% ----------------------------------------------------------------------
% 3. SECURITY CONTROL REVIEW (QUESTIONNAIRE)
% ----------------------------------------------------------------------
\section{Security Control Review}

A review of the organization's security controls was conducted based on a standardized questionnaire. The responses indicate foundational gaps in policy and access control. A "No" response highlights a missing control and a potential area of high risk.

\begin{table}[h!]
\centering
\caption{Security Controls Questionnaire Results}
\begin{tabular}{@{}p{0.8\textwidth}c@{}}
    \toprule
    \textbf{Control Question} & \textbf{Response} \\
    \midrule
    Do you require MFA to access email? & \cmark \\
    Do you require MFA to log into computers? & \cmark \\
    \textbf{Do you require MFA to access sensitive data systems?} & \textbf{\xmark} \\
    \textbf{Does your organization have an employee acceptable use policy?} & \textbf{\xmark} \\
    Does your organization do security awareness training for new employees? & \cmark \\
    Does your organization do security awareness training for all employees at least once per year? & \cmark \\
    \bottomrule
\end{tabular}
\end{table}

% ----------------------------------------------------------------------
% 4. TECHNICAL SCAN RESULTS
% ----------------------------------------------------------------------
\section{Technical Scan Results}

An external network scan was performed to identify open ports and exposed services on the organization's public-facing infrastructure.

\begin{itemize}
    \item \textbf{Scan Date:} November 22, 2025
    \item \textbf{Target IP:} \texttt{[Target IP]}
\end{itemize}

\subsection{Open Ports and Services}
The scan identified one open port, which is detailed below.

\begin{table}[h!]
\centering
\caption{Discovered Open Ports}
\begin{tabular}{@{}lllll@{}}
    \toprule
    \textbf{Port} & \textbf{State} & \textbf{Service} & \textbf{Product} & \textbf{Version} \\
    \midrule
    443/TCP & Open & https & nginx & 1.18.0 \\
    \bottomrule
\end{tabular}
\end{table}

\subsection{Analysis of Findings}
The scan revealed that the web server is running \textbf{Nginx version 1.18.0}. This version was released in April 2020 and has since been superseded by many newer releases. Running outdated software, especially on an internet-facing server, is a significant security risk. Nginx 1.18.0 is known to be vulnerable to multiple Common Vulnerabilities and Exposures (CVEs), which could allow an attacker to cause a denial of service, bypass security restrictions, or potentially execute arbitrary code.

% ----------------------------------------------------------------------
% 5. CONSOLIDATED RISK ASSESSMENT
% ----------------------------------------------------------------------
\section{Consolidated Risk Assessment}

This section synthesizes the findings from the security control review and the technical scan into a prioritized list of risks. No pre-existing vulnerabilities were reported.

\begin{table}[h!]
\centering
\caption{Summary of Identified Risks}
\begin{tabular}{@{}lp{0.5\textwidth}l@{}}
    \toprule
    \textbf{Risk ID} & \textbf{Risk Name \& Description} & \textbf{Severity} \\
    \midrule
    RISK-001 & \textbf{Lack of MFA on Sensitive Systems} \newline A critical control gap that exposes high-value data to unauthorized access if credentials are compromised. & \textcolor{critical}{\textbf{Critical}} \\
    \addlinespace
    RISK-002 & \textbf{Outdated Web Server Software} \newline The public-facing Nginx server (v1.18.0) is outdated and vulnerable to known exploits, posing a risk of system compromise. & \textcolor{high}{\textbf{High}} \\
    \addlinespace
    RISK-003 & \textbf{Missing Acceptable Use Policy (AUP)} \newline The absence of a formal AUP leads to inconsistent security practices and a lack of enforceable guidelines for employees. & \textcolor{high}{\textbf{High}} \\
    \bottomrule
\end{tabular}
\end{table}

% ----------------------------------------------------------------------
% 6. RECOMMENDATIONS
% ----------------------------------------------------------------------
\section{Recommendations}

The following actions are recommended to mitigate the identified risks and improve the overall security posture of \textbf{[Organization Name]}.

\begin{enumerate}
    \item \textbf{Implement MFA for Sensitive Systems (RISK-001):}
    \begin{itemize}
        \item \textbf{Action:} Immediately prioritize the deployment and enforcement of multi-factor authentication (MFA) for all user accounts, especially administrative and privileged accounts, that can access sensitive data systems.
        \item \textbf{Impact:} Drastically reduces the risk of unauthorized access from compromised credentials.
        \item \textbf{Priority:} \textcolor{critical}{\textbf{Critical}}
    \end{itemize}
    \vspace{0.5cm}
    
    \item \textbf{Upgrade Web Server Software (RISK-002):}
    \begin{itemize}
        \item \textbf{Action:} Plan and execute an upgrade of the Nginx server at \texttt{[Target IP]} from version 1.18.0 to the latest stable version. Before deployment, test the new version in a non-production environment to ensure compatibility.
        \item \textbf{Impact:} Mitigates numerous known vulnerabilities, protecting the server and its data from public exploits.
        \item \textbf{Priority:} \textcolor{high}{\textbf{High}}
    \end{itemize}
    \vspace{0.5cm}
    
    \item \textbf{Develop and Implement an AUP (RISK-003):}
    \begin{itemize}
        \item \textbf{Action:} Draft a comprehensive Acceptable Use Policy (AUP) that clearly defines the rules and responsibilities for employees when using company IT assets. This policy should be communicated to all staff and formally acknowledged.
        \item \textbf{Impact:} Establishes a clear security baseline for user behavior, reduces insider risk, and provides a basis for disciplinary action in case of violations.
        \item \textbf{Priority:} \textcolor{high}{\textbf{High}}
    \end{itemize}
\end{enumerate}

\end{document}
```