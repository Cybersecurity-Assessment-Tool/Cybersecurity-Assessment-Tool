```latex
\documentclass[12pt]{article}

% Required Packages
\usepackage[margin=1in]{geometry}
\usepackage{pifont} % For checkmarks and crosses (\ding)
\usepackage{booktabs} % For professional-looking tables
\usepackage{hyperref} % For clickable links
\usepackage{url}      % For formatting URLs
\usepackage{seqsplit} % For breaking long strings in \texttt
\usepackage{fancyhdr} % For headers and footers
\usepackage{lastpage} % To get the total number of pages

% Document Metadata
\title{Cybersecurity Posture Assessment Report \\ \large for \textbf{[Organization Name]}}
\author{Cybersecurity Analysis Division}
\date{\today}

% Header and Footer Configuration
\pagestyle{fancy}
\fancyhf{} % Clear all header and footer fields
\fancyhead[L]{Cybersecurity Assessment for \textbf{[Organization Name]}}
\fancyfoot[C]{\thepage\ of \pageref{LastPage}}
\renewcommand{\headrulewidth}{0.4pt}
\renewcommand{\footrulewidth}{0.4pt}

\begin{document}

\maketitle
\thispagestyle{empty}

\begin{abstract}
    This report provides a detailed analysis of the cybersecurity posture of \textbf{[Organization Name]}, based on a review of organizational security controls, an external network scan, and pre-existing risk data. The assessment identifies critical deficiencies in foundational security practices and a significant, high-risk technical vulnerability. Immediate and strategic remediation actions are required to mitigate the identified risks and protect organizational assets.
\end{abstract}

\tableofcontents
\newpage

\section{Executive Summary}

This assessment reveals a \textbf{Critical} overall risk posture for \textbf{[Organization Name]}. The analysis identified a complete absence of fundamental security controls, including Multi-Factor Authentication (MFA), employee security policies, and security awareness training. These administrative gaps create a permissive environment for security incidents.

Furthermore, a technical network scan confirmed a pre-existing risk: a database server is directly exposed to the public internet. Analysis shows this database is running an \textbf{End-of-Life (EOL) version of MySQL}, which no longer receives security updates. This combination of poor administrative controls and a vulnerable, exposed critical system presents a significant and immediate threat of a data breach.

Urgent remediation is required to address the exposed database, followed by the systematic implementation of foundational security controls as detailed in the Recommendations section of this report.

\section{Organizational Information}

The following details were used as the basis for this assessment. As per the template mode for this report, placeholders are used where specific data was not provided.

\begin{table}[h!]
\centering
\begin{tabular}{@{}ll@{}}
\toprule
\textbf{Item} & \textbf{Detail} \\ \midrule
Organization Name & \textbf{[Organization Name]} \\
Email Domain & \texttt{[Domain]} \\
Primary External IP & \texttt{[Client IP]} \\
Target of Network Scan & \texttt{[Target IP]} \\
Assessment Date & \today \\ \bottomrule
\end{tabular}
\caption{Assessment Scope and Target Information.}
\end{table}

\section{Security Control Review}

A review of administrative and policy-based security controls was conducted based on a standardized questionnaire. The results indicate critical gaps in essential security practices. A "No" answer (\ding{55}) represents a failure to meet a baseline security requirement.

\begin{table}[h!]
\centering
\begin{tabular}{@{}p{0.7\textwidth}cp{0.15\textwidth}@{}}
\toprule
\textbf{Control Question} & \textbf{Status} & \textbf{Assessment} \\ \midrule
Do you require MFA to access email? & \ding{55} & Critical Gap \\
Do you require MFA to log into computers? & \ding{55} & High Risk \\
Do you require MFA to access sensitive data systems? & \ding{55} & Critical Gap \\
Does your organization have an employee acceptable use policy? & \ding{55} & High Risk \\
Does your organization do security awareness training for new employees? & \ding{55} & High Risk \\
Does your organization do security awareness training for all employees at least once per year? & \ding{55} & High Risk \\
\bottomrule
\end{tabular}
\caption{Organizational Security Control Status.}
\end{table}

\subsection{Analysis}
The complete absence of Multi-Factor Authentication is a severe weakness. MFA is a foundational control for preventing unauthorized access resulting from credential theft. The lack of security policies and awareness training significantly increases the likelihood of human error leading to a security incident, such as falling victim to phishing attacks.

\section{Technical Scan Results}

An external network scan was performed on the target IP address \texttt{[Target IP]} to identify open ports and exposed services.

\begin{table}[h!]
\centering
\begin{tabular}{@{}lllll@{}}
\toprule
\textbf{Port} & \textbf{State} & \textbf{Service} & \textbf{Product} & \textbf{Version} \\ \midrule
3306/tcp & open & mysql & MySQL & 5.7.33 \\ \bottomrule
\end{tabular}
\caption{Open Ports and Services on \texttt{[Target IP]}.}
\end{table}

\subsection{Analysis}
The scan confirms the finding from the pre-existing risk data: a MySQL database on port 3306 is open to the internet. This is a highly dangerous configuration, as it exposes the database to brute-force attacks, credential stuffing, and direct exploitation from any attacker worldwide.

Critically, the identified version, \textbf{MySQL 5.7.33}, reached its official \textbf{End-of-Life (EOL) in October 2023}. EOL software no longer receives security patches from the vendor. This means that any vulnerabilities discovered after this date will remain unpatched, making the server a prime target for exploitation.

\section{Consolidated Risk Assessment}

The following table synthesizes findings from the security control review, the technical scan, and the provided current risks data into a prioritized list.

\begin{table}[h!]
\centering
\begin{tabular}{@{}p{0.1\textwidth}p{0.5\textwidth}p{0.15\textwidth}p{0.15\textwidth}@{}}
\toprule
\textbf{Risk ID} & \textbf{Risk Description} & \textbf{Severity} & \textbf{Source} \\ \midrule
\textbf{RISK-001} & \textbf{Exposed End-of-Life Database Service.} MySQL 5.7.33 is publicly accessible on port 3306. The software is EOL and unpatched, exposing it to known exploits. & \textbf{Critical} & Network Scan, Current Risks \\
\addlinespace
\textbf{RISK-002} & \textbf{Absence of Multi-Factor Authentication.} Lack of MFA for email and sensitive systems allows for account takeover via compromised credentials. & \textbf{Critical} & Questionnaire \\
\addlinespace
\textbf{RISK-003} & \textbf{Lack of Security Policies and Training.} No acceptable use policy or security awareness training increases the risk of human-related security incidents like phishing. & \textbf{High} & Questionnaire \\
\bottomrule
\end{tabular}
\caption{Summary of Identified Risks.}
\end{table}

\section{Recommendations}

The following actions are recommended to mitigate the identified risks. They are prioritized based on severity and potential impact.

\subsection{Immediate Actions (Required within 72 hours)}
\begin{itemize}
    \item \textbf{Restrict Database Access (RISK-001):} Immediately implement firewall rules to block all public access to port 3306 on \texttt{[Target IP]}. Access should be restricted to a whitelist of specific, trusted IP addresses (e.g., application servers) or require a Virtual Private Network (VPN).
    \item \textbf{Plan Emergency Upgrade (RISK-001):} Initiate an emergency plan to upgrade the MySQL 5.7.33 instance to a fully supported version (e.g., MySQL 8.x). This is critical to ensure the system can receive security patches.
\end{itemize}

\subsection{High-Priority Actions (Required within 30 days)}
\begin{itemize}
    \item \textbf{Deploy Multi-Factor Authentication (RISK-002):} Implement and enforce MFA for all users across all critical platforms. Prioritize the following:
    \begin{enumerate}
        \item Email services (e.g., Office 365, Google Workspace).
        \item Remote access solutions (VPNs).
        \item Access to all systems containing sensitive data.
    \end{enumerate}
    \item \textbf{Develop Foundational Policies (RISK-003):} Draft and ratify an Employee Acceptable Use Policy. This policy should clearly define the rules for using company IT assets and data.
\end{itemize}

\subsection{Strategic Actions (Required within 90 days)}
\begin{itemize}
    \item \textbf{Implement Security Awareness Training (RISK-003):} Procure and roll out a security awareness training program for all employees. This program should be mandatory for new hires and conducted at least annually for all staff. Training should cover topics such as phishing, password hygiene, and data handling.
\end{itemize}

\end{document}
```