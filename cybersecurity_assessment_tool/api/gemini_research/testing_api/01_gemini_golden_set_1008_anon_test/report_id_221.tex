```latex
\documentclass[12pt, a4paper]{article}

% === PACKAGE IMPORTS ===
\usepackage[margin=1in]{geometry}
\usepackage{pifont} % For checkmarks and crosses
\usepackage{booktabs} % For professional-looking tables
\usepackage[hidelinks]{hyperref} % For clickable links without boxes
\usepackage{url} % For URL formatting
\usepackage{seqsplit} % For splitting long strings without spaces
\usepackage{graphicx}
\usepackage{xcolor}

% === DOCUMENT METADATA ===
\title{Cybersecurity Posture and Risk Assessment Report}
\author{Cybersecurity Analysis Division}
\date{\today}

% === DOCUMENT START ===
\begin{document}

\maketitle
\thispagestyle{empty}
\newpage

\tableofcontents
\newpage

% === EXECUTIVE SUMMARY ===
\section{Executive Summary}
This report provides a comprehensive cybersecurity assessment for \textbf{[Organization Name]}. The analysis is based on a correlation of external network scans, a review of internal security controls via a questionnaire, and an evaluation of previously identified risks.

The assessment has identified a \textbf{critical-risk vulnerability} that requires immediate attention. An external scan confirmed that a Remote Desktop Protocol (RDP) service is directly exposed to the public internet on the primary external IP address \texttt{[Client IP]}. This technical vulnerability is severely compounded by an organizational policy gap: the lack of mandatory Multi-Factor Authentication (MFA) for computer logins. This combination creates a high-impact, high-likelihood scenario for a network breach via credential theft or brute-force attacks.

Furthermore, significant gaps were identified in foundational security governance, including the absence of an employee acceptable use policy and a lack of recurring, annual security awareness training. These deficiencies increase the organization's overall susceptibility to a wide range of cyber threats, particularly social engineering and phishing attacks.

Immediate remediation of the exposed RDP service is paramount. Subsequently, the organization must prioritize the implementation of mandatory MFA and the development of core security policies and training programs to build a more resilient security posture.

% === ORGANIZATIONAL INFORMATION ===
\section{Organizational Information}
This section outlines the key identifying information for the organization under review. As this data was not provided, placeholders have been used.

\begin{itemize}
    \item \textbf{Organization Name:} \textbf{[Organization Name]}
    \item \textbf{Primary Email Domain:} \texttt{[Domain]}
    \item \textbf{Primary External IP Address:} \texttt{[Client IP]}
\end{itemize}

% === SECURITY CONTROL REVIEW ===
\section{Security Control Review}
The following table summarizes the organization's responses to a security controls questionnaire. "No" answers indicate significant gaps in the security framework and are discussed below.

\begin{table}[h!]
\centering
\caption{Security Controls Questionnaire Results}
\begin{tabular}{p{0.7\linewidth} c c}
\toprule
\textbf{Control Question} & \textbf{Response} & \textbf{Status} \\
\midrule
Do you require MFA to access email? & Yes & \ding{51} \\
\textbf{Do you require MFA to log into computers?} & \textbf{No} & \textcolor{red}{\ding{55}} \\
Do you require MFA to access sensitive data systems? & Yes & \ding{51} \\
\textbf{Does your organization have an employee acceptable use policy?} & \textbf{No} & \textcolor{red}{\ding{55}} \\
Does your organization do security awareness training for new employees? & Yes & \ding{51} \\
\textbf{Does your organization do security awareness training for all employees at least once per year?} & \textbf{No} & \textcolor{red}{\ding{55}} \\
\bottomrule
\end{tabular}
\end{table}

\subsection{Analysis of Control Gaps}
The questionnaire reveals three critical areas of concern:
\begin{itemize}
    \item \textbf{No MFA for Computer Logins:} This is a critical weakness. Without MFA, a compromised password is all an attacker needs to gain direct access to an employee's workstation and, potentially, the entire network. This gap directly elevates the risk associated with the exposed RDP service found in the technical scan.
    \item \textbf{No Acceptable Use Policy (AUP):} An AUP is a foundational governance document that sets clear expectations for employee behavior regarding company assets and data. Its absence leads to inconsistent security practices and a lack of enforceable consequences for policy violations.
    \item \textbf{No Annual Security Awareness Training:} While training for new hires is in place, the lack of a recurring annual program for all employees is a major oversight. The threat landscape evolves constantly, and so do tactics used by attackers. Regular training is essential to keep security top-of-mind and to educate staff on new threats like advanced phishing techniques.
\end{itemize}

% === TECHNICAL SCAN RESULTS ===
\section{Technical Scan Results}
An external network scan was performed to identify open ports and exposed services on the organization's perimeter.

\begin{itemize}
    \item \textbf{Target IP Address:} \texttt{[Target IP]} (Correlated to \texttt{[Client IP]})
    \item \textbf{Scan Tool:} Nmap
    \item \textbf{Host Status:} Up
\end{itemize}

\subsection{Open Ports and Services}
A single open port was discovered, which presents a significant risk.

\begin{table}[h!]
\centering
\caption{Exposed Network Services}
\begin{tabular}{l l l l}
\toprule
\textbf{Port} & \textbf{State} & \textbf{Service Name} & \textbf{Description} \\
\midrule
3389/tcp & open & \texttt{ms-wbt-server} & Microsoft Remote Desktop Protocol (RDP) \\
\bottomrule
\end{tabular}
\end{table}

\subsection{Analysis of Technical Findings}
The scan confirms that the Remote Desktop Protocol (RDP) service is directly accessible from the public internet. RDP is a frequent target for attackers who use brute-force password guessing, credential stuffing, and exploitation of known vulnerabilities (e.g., BlueKeep) to gain unauthorized access to internal networks. Exposing this service without mitigating controls like a VPN or strict IP whitelisting is a critical security misconfiguration.

% === RISK ASSESSMENT SUMMARY ===
\section{Risk Assessment Summary}
This section synthesizes the findings from the security control review, technical scan, and pre-existing risk data into a prioritized list of risks.

\begin{table}[h!]
\centering
\caption{Consolidated Risk Register}
\begin{tabular}{p{0.25\linewidth} p{0.5\linewidth} l}
\toprule
\textbf{Risk Name} & \textbf{Overview} & \textbf{Severity} \\
\midrule
\textbf{Exposed RDP Service with No MFA} & Port 3389 (RDP) is open to the internet. The lack of MFA on computer logins means a single compromised password could lead to a full network breach. This correlates directly with the pre-existing "RDP Exposure" risk. & \textbf{Critical (9.0)} \\
\addlinespace
\textbf{Inadequate Security Training Program} & Security training is not conducted annually for all employees, increasing the organization's susceptibility to phishing, social engineering, and human error. & High \\
\addlinespace
\textbf{Lack of Foundational Governance Policy} & The absence of an Acceptable Use Policy (AUP) creates an environment of ambiguity regarding security responsibilities and acceptable user behavior. & High \\
\bottomrule
\end{tabular}
\end{table}

% === RECOMMENDATIONS ===
\section{Recommendations}
The following actions are recommended to mitigate the identified risks and improve the overall security posture of \textbf{[Organization Name]}.

\subsection{Immediate Priority (Remediate within 24-48 hours)}
\begin{enumerate}
    \item \textbf{Close Port 3389 on the External Firewall:} Immediately implement a firewall rule to \textbf{block all inbound traffic} to TCP port 3389 on the external IP address \texttt{[Client IP]}. This will eliminate the direct exposure.
    \item \textbf{Review for Active Compromise:} Investigate system logs for the host associated with \texttt{[Target IP]} for any signs of unauthorized or suspicious login activity.
\end{enumerate}

\subsection{High Priority (Remediate within 30 days)}
\begin{enumerate}
    \item \textbf{Implement a Secure Remote Access Solution:} For necessary remote access, deploy a Virtual Private Network (VPN) solution. Access to the RDP service should only be possible after a user has successfully authenticated to the VPN, which must be configured to require Multi-Factor Authentication.
    \item \textbf{Enforce MFA for All Computer Logins:} Procure and deploy an MFA solution for all workstation and server logins. This is a critical compensating control that protects against password compromise.
\end{enumerate}

\subsection{Medium Priority (Remediate within 90 days)}
\begin{enumerate}
    \item \textbf{Develop and Implement an Acceptable Use Policy (AUP):} Create a formal AUP document that all employees must read and acknowledge. This policy should define the rules for using company networks, systems, and data.
    \item \textbf{Establish an Annual Security Awareness Training Program:} Implement a mandatory security awareness training program for all employees to be completed annually. The training should cover current threats such as phishing, ransomware, and proper data handling.
\end{enumerate}

% === DOCUMENT END ===
\end{document}
```