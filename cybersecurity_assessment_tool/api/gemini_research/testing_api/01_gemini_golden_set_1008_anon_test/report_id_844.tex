```latex
\documentclass[12pt]{article}

% Preamble: Required Packages
\usepackage[margin=1in]{geometry}
\usepackage{pifont} % For checkmarks and crosses
\usepackage{booktabs} % For professional tables
\usepackage{hyperref} % For hyperlinks
\usepackage{url} % For URL formatting
\usepackage{seqsplit} % For splitting long strings without breaking
\usepackage{graphicx}
\usepackage{xcolor}

% Document Metadata
\title{Cybersecurity Posture Assessment Report}
\author{Cybersecurity Analysis Division}
\date{\today}

% Hyperref Setup
\hypersetup{
    colorlinks=true,
    linkcolor=blue,
    filecolor=magenta,      
    urlcolor=cyan,
    pdftitle={Cybersecurity Posture Assessment Report},
    pdfpagemode=FullScreen,
}

% Custom Commands
\newcommand{\yes}{\ding{51}} % Green checkmark
\newcommand{\no}{\ding{55}}  % Red X

\begin{document}

\maketitle

\begin{abstract}
This report provides a comprehensive analysis of the cybersecurity posture for \textbf{[Organization Name]}. The assessment is based on a synthesis of data from an external network scan, a security controls questionnaire, and a review of pre-existing risks. The analysis reveals a mixed security posture: while the external network perimeter appears hardened with no open ports detected, significant internal security gaps exist. Critical deficiencies were identified in endpoint authentication and foundational security policies. This report details these findings and provides actionable recommendations to mitigate the identified risks.
\end{abstract}

\tableofcontents
\newpage

% ===================================================================
% Section 1: Overview and Organizational Information
% ===================================================================
\section{Overview and Organizational Information}

This assessment was conducted to evaluate the current security controls and identify potential vulnerabilities within the organization's IT environment. The information presented below was used as the basis for this analysis.

\begin{center}
\begin{tabular}{ll}
\toprule
\textbf{Attribute} & \textbf{Value} \\
\midrule
Organization Name & \textbf{[Organization Name]} \\
Primary Domain & \texttt{[Domain]} \\
External IP Address (Scanned) & \texttt{[Client IP]} \\
\bottomrule
\end{tabular}
\end{center}

% ===================================================================
% Section 2: Security Control Review (Questionnaire)
% ===================================================================
\section{Security Control Review}

A security controls questionnaire was completed to evaluate existing policies and procedures. The responses indicate several areas of concern where security best practices are not being met. "No" answers represent significant gaps that increase organizational risk.

\begin{center}
\begin{tabular}{p{0.75\textwidth} c}
\toprule
\textbf{Control Question} & \textbf{Response} \\
\midrule
Do you require MFA to access email? & \yes \\
Do you require MFA to log into computers? & \textcolor{red}{\no} \\
Do you require MFA to access sensitive data systems? & \yes \\
Does your organization have an employee acceptable use policy? & \textcolor{red}{\no} \\
Does your organization do security awareness training for new employees? & \yes \\
Does your organization do security awareness training for all employees at least once per year? & \yes \\
\bottomrule
\end{tabular}
\end{center}

\subsection{Analysis of Control Gaps}
\begin{itemize}
    \item \textbf{No MFA on Computers:} The absence of Multi-Factor Authentication (MFA) for computer logins is a critical vulnerability. If an employee's password is stolen or guessed, an attacker can gain direct access to their workstation and, potentially, the entire internal network.
    \item \textbf{No Acceptable Use Policy (AUP):} Lacking a formal AUP means there are no defined rules for how employees should use company IT assets. This can lead to unsafe behavior, insider threats, and legal ambiguity in the event of a security incident.
\end{itemize}

% ===================================================================
% Section 3: Technical Scan Results
% ===================================================================
\section{Technical Scan Results}

An external network scan was performed to identify open ports and exposed services on the public-facing infrastructure.

\subsection{Scan Summary}
\begin{itemize}
    \item \textbf{Target IP Address:} \texttt{[Target IP]}
    \item \textbf{Scan Date:} Not available in provided data.
    \item \textbf{Scanner Used:} Nmap
\end{itemize}

\subsection{Findings}
The scan results were positive from an external security perspective. The target host was responsive, but no open TCP ports were detected. The scanner reported that all other scanned ports were in a 'closed' state.

\textbf{Conclusion:} This finding indicates a strong firewall configuration or network access control list (ACL) is in place. A minimal external attack surface is a significant security strength, as it drastically reduces the number of potential entry points for external attackers.

% ===================================================================
% Section 4: Risk Assessment
% ===================================================================
\section{Risk Assessment}

This section synthesizes the findings from the security control review, technical scan, and pre-existing risk data. The following table prioritizes the most significant risks identified during this assessment.

\begin{center}
\begin{tabular}{p{0.25\textwidth} p{0.5\textwidth} l}
\toprule
\textbf{Risk Name} & \textbf{Overview} & \textbf{Severity} \\
\midrule
\textbf{Lack of Endpoint MFA} & User workstations do not require MFA for login. This exposes the organization to significant risk from compromised credentials, as an attacker with a valid password can gain direct access to the internal network and its resources. & \textbf{Critical} \\
\addlinespace
\textbf{Absence of Acceptable Use Policy (AUP)} & The organization lacks a formal AUP, which defines rules for employee use of IT resources. This creates ambiguity and increases the risk of misuse, insider threats, and legal liabilities. & \textbf{High} \\
\addlinespace
\multicolumn{3}{l}{\textit{No pre-existing vulnerabilities were provided for this assessment.}} \\
\bottomrule
\end{tabular}
\end{center}

% ===================================================================
% Section 5: Recommendations
% ===================================================================
\section{Recommendations}

Based on the risk assessment, the following actions are recommended to improve the organization's security posture. Recommendations are prioritized by severity.

\begin{enumerate}
    \item \textbf{[Critical] Implement MFA for All Computer Logins:}
    Deploy a mandatory Multi-Factor Authentication solution for all employee and privileged user computer logins. This is the most effective control to mitigate the risk of unauthorized access via compromised credentials.
    \begin{itemize}
        \item \textit{Action:} Evaluate and implement solutions such as Windows Hello for Business, Duo Security, or other third-party MFA providers compatible with your environment.
    \end{itemize}
    
    \item \textbf{[High] Develop and Enforce an Acceptable Use Policy (AUP):}
    Create a comprehensive AUP that clearly outlines the rules and responsibilities for all users of the organization's IT systems. This is a foundational policy for cybersecurity governance.
    \begin{itemize}
        \item \textit{Action:} Draft an AUP covering topics such as data handling, internet usage, password security, and incident reporting. Ensure the policy is distributed to all employees and that their acknowledgment is formally tracked.
    \end{itemize}
    
    \item \textbf{[Informational] Maintain Strong Perimeter Security:}
    The external network scan showed a well-configured perimeter. Continue this practice by performing regular firewall rule audits and vulnerability scans to ensure no unnecessary services are exposed to the internet.
    \begin{itemize}
        \item \textit{Action:} Schedule quarterly reviews of firewall rules and perform monthly external network scans to validate the security posture.
    \end{itemize}
\end{enumerate}

\end{document}
```