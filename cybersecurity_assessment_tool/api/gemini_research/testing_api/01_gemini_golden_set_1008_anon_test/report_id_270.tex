```latex
\documentclass[12pt, a4paper]{article}

% Preamble: Required Packages
\usepackage[margin=1in]{geometry}
\usepackage{pifont} % For checkmarks and crosses
\usepackage{booktabs} % For professional tables
\usepackage{hyperref} % For clickable links and TOC
\usepackage{url} % For formatting URLs
\usepackage{seqsplit} % For splitting long strings in texttt
\usepackage{xcolor} % For colors
\usepackage{graphicx} % For logo (placeholder)
\usepackage{fancyhdr} % For header/footer

% --- Document Metadata ---
\hypersetup{
    colorlinks=true,
    linkcolor=blue,
    filecolor=magenta,      
    urlcolor=cyan,
    pdftitle={Cybersecurity Posture Assessment Report},
    pdfauthor={Cybersecurity Analyst},
    pdfsubject={Security Analysis},
    pdfkeywords={Security, Assessment, Report},
    bookmarks=true
}

% --- Page Style ---
\pagestyle{fancy}
\fancyhf{} % clear all header and footer fields
\fancyhead[L]{Cybersecurity Posture Assessment}
\fancyhead[R]{\textbf{[Organization Name]}}
\fancyfoot[C]{\thepage}
\renewcommand{\headrulewidth}{0.4pt}
\renewcommand{\footrulewidth}{0.4pt}

% --- Custom Commands ---
\newcommand{\yes}{\ding{51}}
\newcommand{\no}{\ding{55}}

\begin{document}

% --- Title Page ---
\begin{titlepage}
    \centering
    \vspace*{1cm}
    
    \Huge
    \textbf{Cybersecurity Posture Assessment Report}
    
    \vspace{1.5cm}
    
    \Large
    Prepared for: \\
    \vspace{0.5cm}
    \textbf{[Organization Name]}
    
    \vspace{2cm}
    
    \large
    \textbf{Date of Report:} \today \\
    \textbf{Date of Scan:} [Scan Date]
    
    \vfill
    
    \large
    \textbf{Generated By:} \\
    Expert Cybersecurity Analyst
    
\end{titlepage}

\tableofcontents
\newpage

% --- Section 1: Executive Summary ---
\section{Executive Summary}

This report details the findings of a cybersecurity posture assessment conducted for \textbf{[Organization Name]}. The assessment combined an analysis of organizational security controls, a technical network scan, and a review of pre-existing risk data.

The overall security posture is assessed as \textbf{High Risk}. While the organization has implemented some key controls, such as requiring Multi-Factor Authentication (MFA) for sensitive data systems, critical deficiencies were identified that significantly elevate the risk of a security incident.

\textbf{Key Findings:}
\begin{itemize}
    \item \textbf{Critical Gaps in Access Control:} MFA is not required for accessing email or for logging into employee computers. This exposes the organization to a high likelihood of account compromise through phishing and credential theft, which could lead to Business Email Compromise (BEC) and ransomware attacks.
    \item \textbf{Lack of Security Awareness:} The organization does not provide security awareness training to new or existing employees. This lack of training makes staff highly susceptible to social engineering attacks, which are the primary vector for most cyber intrusions.
    \item \textbf{Risk Register Discrepancy:} A previously documented risk, "Unencrypted Web Server" on Port 80, appears to have been remediated. Our technical scan of the target host \texttt{[Target IP]} found that Port 80 is closed, contradicting the existing risk data. This is a positive development but indicates a potential need for improved risk register management.
\end{itemize}

Immediate and decisive action is required to address the identified gaps in MFA and security training to reduce the organization's attack surface and mitigate the most probable threats.

% --- Section 2: Organizational Information ---
\section{Organizational Information}

The following details were used as the basis for this assessment. As identity information was not provided, placeholders have been used.

\begin{table}[h!]
\centering
\begin{tabular}{@{}ll@{}}
\toprule
\textbf{Item} & \textbf{Detail} \\ \midrule
Organization Name & \textbf{[Organization Name]} \\
Primary Email Domain & \texttt{[Domain]} \\
Assessed External IP & \texttt{[Client IP]} \\
Target of Network Scan & \texttt{[Target IP]} \\ \bottomrule
\end{tabular}
\caption{Assessment Scope Information}
\end{table}

% --- Section 3: Security Control Review ---
\section{Security Control Review}

An assessment of internal security controls was conducted based on a standardized questionnaire. The responses reveal significant gaps in fundamental security practices.

\begin{table}[h!]
\centering
\begin{tabular}{@{}p{0.6\textwidth} c l@{}}
\toprule
\textbf{Control Question} & \textbf{Response} & \textbf{Assessment} \\ \midrule
Do you require MFA to access email? & \no & \textcolor{red}{\textbf{Critical Gap}} \\
Do you require MFA to log into computers? & \no & \textcolor{red}{\textbf{Critical Gap}} \\
Do you require MFA to access sensitive data systems? & \yes & Strength \\
Does your organization have an employee acceptable use policy? & \yes & Strength \\
Does your organization do security awareness training for new employees? & \no & \textcolor{orange}{High Risk} \\
Does your organization do security awareness training for all employees at least once per year? & \no & \textcolor{orange}{High Risk} \\ \bottomrule
\end{tabular}
\caption{Security Controls Questionnaire Analysis}
\end{table}

The absence of MFA on email and computer logins, combined with a lack of security training, creates a high-risk environment for phishing, credential harvesting, and subsequent unauthorized access to corporate resources.

% --- Section 4: Technical Scan Results ---
\section{Technical Scan Results}

A network scan was performed on the target host to identify open ports and exposed services.

\begin{itemize}
    \item \textbf{Target Host:} \texttt{[Target IP]}
    \item \textbf{Scan Tool:} Nmap
\end{itemize}

\subsection{Scan Findings}
The scan revealed no open ports on the target host. The status of common web ports is detailed below.

\begin{table}[h!]
\centering
\begin{tabular}{@{}llll@{}}
\toprule
\textbf{Port} & \textbf{State} & \textbf{Service} & \textbf{Notes} \\ \midrule
80/tcp & closed & http & No unencrypted web service detected. \\ \bottomrule
\end{tabular}
\caption{Nmap Port Scan Results}
\end{table}

\subsection{Analysis of Findings}
The technical scan results are positive, indicating a minimal external network attack surface on the scanned host. The fact that port 80 is closed directly contradicts a pre-existing risk entry (see Section 5), suggesting that the risk has been successfully remediated or was a false positive. It is recommended to validate this finding and update the internal risk register accordingly.

% --- Section 5: Consolidated Risk Assessment ---
\section{Consolidated Risk Assessment}

This section correlates findings from the security control review, the technical scan, and pre-existing risk data into a unified risk summary.

\begin{table}[h!]
\centering
\begin{tabular}{@{}p{0.25\textwidth} p{0.45\textwidth} l l@{}}
\toprule
\textbf{Risk Name} & \textbf{Description} & \textbf{Severity} & \textbf{Status} \\ \midrule
\textbf{Widespread Lack of MFA} & Email and computer logins are protected only by passwords, making them highly vulnerable to compromise. & \textbf{Critical} & \textbf{Active} \\
\textbf{No Security Awareness Program} & Employees are not trained to recognize or report security threats like phishing, increasing susceptibility to social engineering. & \textbf{High} & \textbf{Active} \\
\textbf{Unencrypted Web Server} & Pre-existing risk stated Port 80 was open. Our scan confirms this port is closed on the target host. & Medium & \textbf{Remediated} \\ \bottomrule
\end{tabular}
\caption{Summary of Identified Risks}
\end{table}

% --- Section 6: Recommendations ---
\section{Recommendations}

Based on the consolidated risk assessment, the following actions are recommended to improve the security posture of \textbf{[Organization Name]}. Recommendations are prioritized by severity.

\subsection{Immediate Priority (Critical)}
\begin{enumerate}
    \item \textbf{Enforce MFA for Email Access:} Immediately enable and enforce MFA for all user accounts on the \texttt{[Domain]} email system. This is the single most effective control to prevent Business Email Compromise.
    \item \textbf{Enforce MFA for Endpoint Logins:} Deploy and require MFA for all employee computer and laptop logins. This mitigates the risk of an attacker gaining network access with stolen credentials.
\end{enumerate}

\subsection{High Priority}
\begin{enumerate}
    \item \textbf{Implement Security Awareness Training:}
    \begin{itemize}
        \item Develop a mandatory security training module for all new employees as part of their onboarding process.
        \item Institute a mandatory annual security awareness training program for all staff.
        \item Conduct periodic phishing simulation campaigns to test and reinforce employee awareness.
    \end{itemize}
\end{enumerate}

\subsection{Administrative Recommendations}
\begin{enumerate}
    \item \textbf{Update Risk Register:} Formally review the "Unencrypted Web Server" risk. Confirm that Port 80 is intentionally closed across all relevant external systems and update the internal risk register to reflect its remediated status.
\end{enumerate}

\end{document}
```