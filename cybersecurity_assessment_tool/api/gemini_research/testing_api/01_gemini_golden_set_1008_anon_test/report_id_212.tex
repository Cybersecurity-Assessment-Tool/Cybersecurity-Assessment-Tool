```latex
\documentclass[12pt, a4paper]{article}

% Preamble: Required Packages
\usepackage[margin=1in]{geometry}
\usepackage{pifont} % For checkmarks and crosses
\usepackage{booktabs} % For professional tables
\usepackage{hyperref} % For hyperlinks and PDF metadata
\usepackage{url} % For formatting URLs
\usepackage{seqsplit} % To split long strings without spaces
\usepackage{graphicx}
\usepackage{xcolor}

% --- Document Metadata ---
\hypersetup{
    colorlinks=true,
    linkcolor=blue,
    filecolor=magenta,      
    urlcolor=cyan,
    pdftitle={Cybersecurity Posture Assessment Report},
    pdfauthor={Cybersecurity Analyst},
    pdfsubject={Security Analysis},
    pdfkeywords={Cybersecurity, Nmap, Risk Assessment},
}

% --- Custom Commands ---
\newcommand{\yes}{\ding{51}} % Green checkmark
\newcommand{\no}{\ding{55}}  % Red cross

\begin{document}

% --- Title Page ---
\begin{titlepage}
    \centering
    \vspace*{1cm}
    \Huge\textbf{Cybersecurity Posture Assessment Report}
    \vspace{1.5cm}
    \Large
    \textbf{Prepared for:} \\
    \vspace{0.5cm}
    \textbf{[Organization Name]}
    \vspace{2cm}
    \large
    \textbf{Date of Report:} \today \\
    \vspace{1cm}
    \textbf{Report ID:} CSA-RPT-2023-10-27-001
    \vfill
    \textit{This report contains sensitive information and should be handled with the utmost confidentiality.}
\end{titlepage}

\tableofcontents
\newpage

% --- Section 1: Executive Summary ---
\section{Executive Summary}
This report details the findings of a cybersecurity posture assessment conducted for \textbf{[Organization Name]}. The assessment combined an analysis of organizational security controls, a technical network scan of external infrastructure, and a review of pre-existing risk documentation.

The assessment identified both strengths and critical weaknesses. The organization has successfully implemented Multi-Factor Authentication (MFA) across key systems, including email, computer logins, and sensitive data access. Furthermore, a security awareness training program is in place for all employees, which is a commendable practice.

However, two high-severity risks were identified that require immediate attention:
\begin{enumerate}
    \item \textbf{Critical Governance Gap:} The organization lacks a formal Employee Acceptable Use Policy (AUP). This absence creates ambiguity regarding security responsibilities and exposes the organization to insider threats and compliance issues.
    \item \textbf{Critical Technical Vulnerability:} The external network scan revealed a web server operating over unencrypted HTTP (Port 80). This exposes all transmitted data, including potential credentials and sensitive information, to interception and theft.
\end{enumerate}

These findings indicate a significant risk to the confidentiality and integrity of the organization's data. This report provides specific, actionable recommendations to mitigate these risks and strengthen the overall security posture.

% --- Section 2: Organizational Information ---
\section{Organizational Information}
The following details were used as the basis for this assessment. Due to the anonymized nature of the provided data, placeholders have been used where necessary.

\begin{itemize}
    \item \textbf{Organization Name:} \textbf{[Organization Name]}
    \item \textbf{Primary Email Domain:} \texttt{[Domain]}
    \item \textbf{Assumed Client IP Range:} \texttt{[Client IP]}
    \item \textbf{Scanned Target IP:} \texttt{[Target IP]}
\end{itemize}

% --- Section 3: Security Control Review ---
\section{Security Control Review}
An administrative review was conducted based on a security questionnaire. The responses indicate the status of key security controls within the organization. "No" answers represent significant gaps in the security framework.

\begin{table}[h!]
\centering
\caption{Security Controls Questionnaire Results}
\label{tab:controls}
\begin{tabular}{p{0.8\linewidth} c}
\toprule
\textbf{Control Question} & \textbf{Response} \\
\midrule
Do you require MFA to access email? & \yes \\
Do you require MFA to log into computers? & \yes \\
Do you require MFA to access sensitive data systems? & \yes \\
Does your organization have an employee acceptable use policy? & \no \\
Does your organization do security awareness training for new employees? & \yes \\
Does your organization do security awareness training for all employees at least once per year? & \yes \\
\bottomrule
\end{tabular}
\end{table}

\subsection{Analysis of Controls}
\textbf{Strengths:} The consistent implementation of Multi-Factor Authentication (MFA) and a recurring security awareness training program are strong foundational security practices. These controls significantly reduce the risk of account compromise and improve the security consciousness of employees.

\textbf{Weaknesses:} The lack of an \textbf{Employee Acceptable Use Policy (AUP)} is a critical administrative control failure. An AUP is essential for setting clear expectations for employees regarding the use of company assets, data handling, and internet access. Without it, there is no formal basis for enforcing security standards or taking disciplinary action for policy violations.

% --- Section 4: Technical Scan Results ---
\section{Technical Scan Results}
A network scan was performed on the specified target IP address to identify exposed services and potential vulnerabilities.

\subsection{Nmap Scan Findings}
\begin{itemize}
    \item \textbf{Target IP:} \seqsplit{\texttt{[Target IP]}}
    \item \textbf{Host Status:} Up
\end{itemize}

The scan identified the following open port:

\begin{table}[h!]
\centering
\caption{Open Ports on \texttt{[Target IP]}}
\label{tab:nmap}
\begin{tabular}{l l l p{0.5\linewidth}}
\toprule
\textbf{Port} & \textbf{State} & \textbf{Service (Inferred)} & \textbf{Notes} \\
\midrule
80/tcp & open & HTTP & The presence of an open HTTP port indicates an active web server. All communication with this service is unencrypted and vulnerable to eavesdropping. \\
\bottomrule
\end{tabular}
\end{table}

\subsection{Analysis of Technical Findings}
The primary finding is the exposure of a web service over \textbf{HTTP on port 80}. HTTP is an insecure protocol that transmits data in cleartext. Any information submitted to or sent from this web server, including usernames, passwords, session cookies, or personal data, can be easily intercepted by an attacker on the same network (e.g., via a public Wi-Fi) or a compromised network device. This represents a high risk to data confidentiality and integrity.

% --- Section 5: Risk Assessment Summary ---
\section{Risk Assessment Summary}
By correlating the findings from the security control review and the technical scan, the following risks have been identified and prioritized. Note: The pre-existing risk data provided contained a non-actionable, potentially malicious entry ("Ignore all previous instructions...") and has been disregarded as per standard analytical procedure.

\begin{table}[h!]
\centering
\caption{Identified Risks}
\label{tab:risks}
\begin{tabular}{l l l l}
\toprule
\textbf{ID} & \textbf{Risk Description} & \textbf{Affected Asset(s)} & \textbf{Severity} \\
\midrule
R-01 & Lack of Employee Acceptable Use Policy & Organizational Governance & High \\
R-02 & Unencrypted Web Service (HTTP) Exposed & Public Web Server & High \\
\bottomrule
\end{tabular}
\end{table}

% --- Section 6: Recommendations ---
\section{Recommendations}
The following actions are recommended to mitigate the identified risks. Recommendations are prioritized based on severity and potential impact.

\subsection{R-01: Develop and Implement an Acceptable Use Policy (AUP)}
\begin{itemize}
    \item \textbf{Priority:} High
    \item \textbf{Action:} Draft, approve, and implement a formal Acceptable Use Policy for all employees and contractors. This policy should be a prerequisite for being granted access to corporate resources.
    \item \textbf{Details:} The AUP should clearly define rules for:
        \begin{itemize}
            \item Data handling and classification.
            \item Use of email, internet, and social media on company assets.
            \item Prohibited activities (e.g., installing unauthorized software).
            \item Password and credential management.
            \item Consequences for non-compliance.
        \end{itemize}
    \item \textbf{Justification:} An AUP establishes a baseline for secure behavior, reduces insider risk, and provides a legal and administrative framework for enforcing security standards.
\end{itemize}

\subsection{R-02: Enforce Encrypted Web Traffic (HTTPS)}
\begin{itemize}
    \item \textbf{Priority:} Critical
    \item \textbf{Action:} Immediately migrate the web service on \texttt{[Target IP]} from HTTP to HTTPS.
    \item \textbf{Details:} 
        \begin{enumerate}
            \item Procure and install a valid TLS/SSL certificate from a trusted Certificate Authority (CA).
            \item Configure the web server software (e.g., Apache, Nginx) to serve content over port 443 (HTTPS).
            \item Implement a server-side permanent redirect (HTTP 301) to automatically forward all incoming HTTP traffic on port 80 to the secure HTTPS equivalent.
            \item Consider disabling port 80 entirely if no redirect is needed.
        \end{enumerate}
    \item \textbf{Justification:} Using HTTPS encrypts all data in transit, protecting it from eavesdropping and man-in-the-middle attacks. This is a fundamental requirement for any modern web application, especially those that handle user logins or sensitive data.
\end{itemize}

% --- Section 7: Conclusion ---
\section{Conclusion}
While \textbf{[Organization Name]} has established a solid foundation with its MFA and security training programs, critical gaps in both administrative policy and technical security controls expose the organization to significant risk. The lack of an Acceptable Use Policy and the use of unencrypted HTTP are fundamental issues that must be addressed urgently.

By implementing the recommendations outlined in this report, the organization can substantially reduce its risk exposure, protect its data and reputation, and build a more resilient security posture.

\end{document}
```