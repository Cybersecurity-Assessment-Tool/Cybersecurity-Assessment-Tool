```latex
\documentclass[12pt, a4paper]{article}

% Preamble: Required Packages
\usepackage[margin=1in]{geometry}
\usepackage{pifont} % For checkmarks and crosses
\usepackage{booktabs} % For professional tables
\usepackage{hyperref} % For clickable links
\usepackage{url} % For URL formatting
\usepackage{seqsplit} % To split long strings without breaking
\usepackage{graphicx}
\usepackage{xcolor}
\usepackage{fancyhdr}

% --- Document Setup ---
\hypersetup{
    colorlinks=true,
    linkcolor=blue,
    filecolor=magenta,      
    urlcolor=cyan,
    pdftitle={Cybersecurity Posture Report},
    pdfpagemode=FullScreen,
}

\pagestyle{fancy}
\fancyhf{}
\lhead{Cybersecurity Posture Report}
\rhead{\textbf{[Organization Name]}}
\cfoot{\thepage}

% --- Custom Commands ---
\newcommand{\yes}{\ding{51}} % Checkmark
\newcommand{\no}{\ding{55}}  % X-mark
\definecolor{darkgreen}{rgb}{0.0, 0.5, 0.0}
\definecolor{darkred}{rgb}{0.8, 0.0, 0.0}
\newcommand{\pass}{\textcolor{darkgreen}{\yes}}
\newcommand{\fail}{\textcolor{darkred}{\no}}

% =============================================================================
% --- BEGIN DOCUMENT ---
% =============================================================================
\begin{document}

% --- Title Page ---
\begin{titlepage}
    \centering
    \vspace*{1cm}
    \Huge \textbf{Cybersecurity Posture Report}
    \vspace{1.5cm}
    \vfill
    \large
    \textbf{Prepared for:}\\
    \vspace{0.5cm}
    \textbf{[Organization Name]}
    \vspace{2cm}
    \textbf{Date of Report:}\\
    \vspace{0.5cm}
    \today
    \vfill
    \textit{This report contains sensitive information and should be handled with care.}
\end{titlepage}

\tableofcontents
\newpage

% --- Section 1: Executive Summary ---
\section{Executive Summary}
This report provides a comprehensive analysis of the current cybersecurity posture for \textbf{[Organization Name]}. The assessment is based on a correlation of a technical network scan, a security controls questionnaire, and a review of previously identified risks.

The analysis reveals a mixed security posture. The organization demonstrates a commitment to security awareness training and has successfully implemented Multi-Factor Authentication (MFA) for email access. A recent network scan confirms that a previously identified risk concerning an unencrypted web server (Port 80) has been remediated, as the port is now closed.

However, critical gaps exist in access control and administrative policies. The absence of MFA for computer logins and, more importantly, for access to sensitive data systems, presents a significant and immediate risk. Furthermore, the lack of a formal Acceptable Use Policy (AUP) weakens the organization's ability to enforce secure employee behavior.

This report outlines these findings in detail and provides prioritized, actionable recommendations to mitigate the identified risks and strengthen the overall security framework.

% --- Section 2: Organizational Information ---
\section{Organizational Information}
This assessment was conducted based on the information provided by the client. The key identifiers used for this report are placeholders, as per the anonymized data provided.

\begin{tabular}{@{}ll}
\toprule
\textbf{Identifier} & \textbf{Value} \\
\midrule
Organization Name & \textbf{[Organization Name]} \\
Primary Email Domain & \texttt{[Domain]} \\
External IP Address (Scanned) & \texttt{[Client IP]} \\
\bottomrule
\end{tabular}

% --- Section 3: Security Control Review ---
\section{Security Control Review}
The following table summarizes the organization's responses to a security controls questionnaire. This review highlights existing strengths and identifies significant gaps in administrative and technical controls.

\begin{table}[h!]
\centering
\caption{Security Controls Questionnaire Analysis}
\begin{tabular}{p{0.6\linewidth} c l}
\toprule
\textbf{Control Question} & \textbf{Response} & \textbf{Assessment} \\
\midrule
Do you require MFA to access email? & \pass & Strength \\
Do you require MFA to log into computers? & \fail & \textbf{Critical Gap} \\
Do you require MFA to access sensitive data systems? & \fail & \textbf{Critical Gap} \\
\addlinespace
Does your organization have an employee acceptable use policy? & \fail & \textbf{High Risk Gap} \\
\addlinespace
Does your organization do security awareness training for new employees? & \pass & Strength \\
Does your organization do security awareness training for all employees at least once per year? & \pass & Strength \\
\bottomrule
\end{tabular}
\end{table}

\paragraph{Analysis:} The lack of MFA on computer logins and sensitive systems is a primary concern. Should an employee's credentials be compromised, an attacker could potentially gain broad access to internal resources and critical data without needing a second authentication factor. The absence of an Acceptable Use Policy creates ambiguity regarding security responsibilities for employees.

% --- Section 4: Technical Scan Results ---
\section{Technical Scan Results}
A network scan was performed on the target IP address to identify open ports and exposed services.

\begin{itemize}
    \item \textbf{Target IP Address:} \texttt{[Target IP]}
    \item \textbf{Scan Date:} \today
    \item \textbf{Scanner Used:} Nmap
\end{itemize}

\begin{table}[h!]
\centering
\caption{Network Port Scan Findings}
\begin{tabular}{l l l l}
\toprule
\textbf{Port} & \textbf{State} & \textbf{Service} & \textbf{Product / Version} \\
\midrule
80/tcp & closed & http & N/A \\
\bottomrule
\end{tabular}
\end{table}

\paragraph{Analysis:} The scan results are positive. The target host is responsive, but Port 80 (HTTP) is confirmed to be \textbf{closed}. This prevents unencrypted web traffic and indicates that a previously identified risk, "Unencrypted Web Server," has been successfully remediated. No other open ports were discovered during this scan.

% --- Section 5: Consolidated Risk Assessment ---
\section{Consolidated Risk Assessment}
The following table synthesizes findings from the security questionnaire, technical scan, and pre-existing risk data into a consolidated list of current risks.

\begin{table}[h!]
\centering
\caption{Risk Summary}
\begin{tabular}{p{0.15\linewidth} p{0.55\linewidth} l}
\toprule
\textbf{Risk Name} & \textbf{Description} & \textbf{Severity} \\
\midrule
\textbf{Inadequate Access Control (Sensitive Systems)} & Lack of MFA on systems containing sensitive data exposes critical assets to unauthorized access if primary credentials are compromised. & \textbf{Critical} \\
\addlinespace
\textbf{Inadequate Access Control (Endpoints)} & Lack of MFA on computer logins allows for easier lateral movement within the network for an attacker with stolen credentials. & \textbf{High} \\
\addlinespace
\textbf{Lack of Acceptable Use Policy} & Without a formal AUP, there is no enforceable standard for employee behavior regarding company assets and data, increasing the risk of insider threat. & \textbf{Medium} \\
\addlinespace
Unencrypted Web Server & \textit{Port 80 was previously identified as open. The recent scan confirms this port is now closed.} & \textbf{Remediated} \\
\bottomrule
\end{tabular}
\end{table}

% --- Section 6: Recommendations ---
\section{Recommendations}
Based on the consolidated risk assessment, the following prioritized actions are recommended to improve the security posture of \textbf{[Organization Name]}.

\subsection{Priority 1: Critical}
\begin{itemize}
    \item \textbf{Implement MFA for Sensitive Systems:} Immediately deploy mandatory Multi-Factor Authentication for all user accounts (including administrative and service accounts) that have access to sensitive data systems. This is the single most effective control to mitigate the highest-impact risk.
\end{itemize}

\subsection{Priority 2: High}
\begin{itemize}
    \item \textbf{Enforce MFA for Endpoint Logins:} Roll out MFA for all employee computer logins (e.g., Windows, macOS). This hardens endpoints against unauthorized use and contains the impact of credential theft.
\end{itemize}

\subsection{Priority 3: Medium}
\begin{itemize}
    \item \textbf{Develop and Implement an Acceptable Use Policy (AUP):} Draft a formal AUP that clearly defines the rules and responsibilities for all employees when using company technology and data. This policy should be reviewed by legal counsel, communicated to all staff, and formally acknowledged via signature.
\end{itemize}

\subsection{Administrative Recommendation}
\begin{itemize}
    \item \textbf{Update Risk Register:} Formally update the internal risk register to mark the "Unencrypted Web Server" vulnerability as remediated. This ensures that security efforts are focused on current, active risks.
\end{itemize}

% --- Section 7: Conclusion ---
\section{Conclusion}
\textbf{[Organization Name]} has established a solid foundation with its security awareness training program and has taken positive steps to secure email communications and remediate past network vulnerabilities.

However, the current focus must shift to addressing the critical gaps in access control. The implementation of Multi-Factor Authentication across endpoints and sensitive systems is paramount. Complementing these technical controls with the development of a formal Acceptable Use Policy will significantly mature the organization's overall cybersecurity posture and resilience against common threats.

We recommend addressing the proposed recommendations in the order of their assigned priority to achieve the greatest risk reduction in the shortest amount of time.

% =============================================================================
% --- END DOCUMENT ---
% =============================================================================
\end{document}
```