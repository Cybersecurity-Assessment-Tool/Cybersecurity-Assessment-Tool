```latex
\documentclass[12pt]{article}

% Preamble: Required Packages
\usepackage[margin=1in]{geometry}
\usepackage{pifont} % For checkmarks and crosses
\usepackage{booktabs} % For professional tables
\usepackage{hyperref} % For clickable links and references
\usepackage{url} % For formatting URLs
\usepackage{seqsplit} % For splitting long strings without breaking
\usepackage{graphicx}
\usepackage{xcolor}
\usepackage{fancyhdr}

% --- Document Setup ---

% Define colors for risk levels
\definecolor{critical}{HTML}{990000}
\definecolor{high}{HTML}{D14302}
\definecolor{medium}{HTML}{E69500}
\definecolor{low}{HTML}{339900}

% Hyperref setup
\hypersetup{
    colorlinks=true,
    linkcolor=blue,
    filecolor=magenta,      
    urlcolor=cyan,
    pdftitle={Cybersecurity Posture Assessment Report},
    pdfpagemode=FullScreen,
}

% Header and Footer
\pagestyle{fancy}
\fancyhf{}
\fancyhead[L]{\textbf{Cybersecurity Posture Assessment}}
\fancyhead[R]{\textbf{[Organization Name]}}
\fancyfoot[C]{\thepage}

% --- Document Start ---

\begin{document}

% --- Title Page ---
\begin{titlepage}
    \centering
    \vspace*{1cm}
    
    \Huge
    \textbf{Cybersecurity Posture Assessment Report}
    
    \vspace{1.5cm}
    
    \Large
    Prepared for: \\
    \vspace{0.5cm}
    \textbf{[Organization Name]}
    
    \vspace{2cm}
    
    \large
    Generated by: \\
    \vspace{0.5cm}
    Cybersecurity Analyst & Report Generation System
    
    \vfill
    
    \large
    \today
    
\end{titlepage}

% --- Table of Contents ---
\tableofcontents
\newpage

% --- Section 1: Executive Summary ---
\section{Executive Summary}
This report provides a comprehensive cybersecurity posture assessment for \textbf{[Organization Name]}. The analysis is based on a synthesis of an external network scan, a review of internal security controls via a questionnaire, and an evaluation of pre-existing risk data.

The overall security posture is determined to be weak, with several critical and high-risk gaps identified. The most significant concerns stem from foundational policy and training deficiencies. The organization currently lacks a mandatory security awareness training program and an employee acceptable use policy. These gaps create a high susceptibility to social engineering, phishing, and insider threats.

Furthermore, a critical security control, Multi-Factor Authentication (MFA), is not enforced for computer logins, exposing the organization to significant risk of lateral movement should an attacker compromise a user's credentials.

Technical findings revealed an exposed Secure Shell (SSH) service on the external network perimeter. While common for remote administration, this service presents a direct vector for attack if not properly secured and monitored. The combination of policy gaps and technical exposures creates a heightened risk environment that requires immediate attention.

\section{Organizational Information}
This section details the organizational data used as the basis for this assessment.
\begin{itemize}
    \item \textbf{Organization Name:} \textbf{[Organization Name]}
    \item \textbf{Primary Email Domain:} \texttt{[Domain]}
    \item \textbf{External IP Address Scanned:} \texttt{[Client IP]}
\end{itemize}

\section{Security Control Review}
The following table summarizes the organization's responses to a security controls questionnaire. A green checkmark (\textcolor{green}{\ding{51}}) indicates a positive control is in place, while a red cross (\textcolor{red}{\ding{55}}) indicates a control gap that introduces risk.

\begin{table}[h!]
\centering
\caption{Security Controls Questionnaire Results}
\label{tab:controls}
\begin{tabular}{@{}lc@{}}
\toprule
\textbf{Control Question} & \textbf{Response} \\ \midrule
Do you require MFA to access email? & \textcolor{green}{\ding{51}} \\
Do you require MFA to log into computers? & \textcolor{red}{\ding{55}} \\
Do you require MFA to access sensitive data systems? & \textcolor{green}{\ding{51}} \\
Does your organization have an employee acceptable use policy? & \textcolor{red}{\ding{55}} \\
Does your organization do security awareness training for new employees? & \textcolor{red}{\ding{55}} \\
Does your organization do security awareness training for all employees at least once per year? & \textcolor{red}{\ding{55}} \\ \bottomrule
\end{tabular}
\end{table}

\subsection*{Analysis of Control Gaps}
The questionnaire reveals three major areas of concern:
\begin{itemize}
    \item \textbf{Lack of MFA on Workstations:} The absence of MFA for computer logins is a high-risk vulnerability. It significantly lowers the barrier for an attacker to move laterally within the network after a single credential compromise.
    \item \textbf{No Acceptable Use Policy (AUP):} An AUP is a foundational governance document. Without it, employees lack clear guidelines on the secure and acceptable use of company assets, increasing the risk of unintentional policy violations and insider threats.
    \item \textbf{No Security Awareness Training:} The complete absence of a security training program is a critical failure. It leaves the workforce, the first line of defense, unprepared to identify and resist phishing, social engineering, and other common cyberattacks.
\end{itemize}

\section{Technical Scan Results}
An external network scan was performed against the target IP address \texttt{[Target IP]}. The scan identified the following open ports and services accessible from the public internet.

\begin{table}[h!]
\centering
\caption{Open Ports on Target: \texttt{[Target IP]}}
\label{tab:nmap}
\begin{tabular}{@{}llll@{}}
\toprule
\textbf{Port} & \textbf{State} & \textbf{Service (Inferred)} & \textbf{Product / Version} \\ \midrule
22 & open & ssh & \textit{Not Available} \\ \bottomrule
\end{tabular}
\end{table}

\subsection*{Analysis of Technical Findings}
The scan identified that port 22, the standard port for the Secure Shell (SSH) protocol, is open. SSH is a powerful tool for remote system administration. However, its exposure to the public internet presents a significant risk.
\begin{itemize}
    \item It provides a direct target for automated brute-force password guessing attacks.
    \item If the SSH service software contains any vulnerabilities, it could be exploited for remote code execution.
\end{itemize}
The scan did not retrieve version information, which prevents a specific check for known vulnerabilities. However, the presence of the open port itself constitutes a notable risk that must be mitigated.

\section{Consolidated Risk Assessment}
This section correlates the findings from the security control review and the technical scan. The following risks have been identified and prioritized based on their potential impact and likelihood. The list of pre-existing vulnerabilities was empty.

\begin{table}[h!]
\centering
\caption{Identified Risks and Severity}
\label{tab:risks}
\begin{tabular}{@{}p{0.1\linewidth}p{0.3\linewidth}p{0.15\linewidth}p{0.35\linewidth}@{}}
\toprule
\textbf{Risk ID} & \textbf{Risk Name} & \textbf{Severity} & \textbf{Description} \\ \midrule
RISK-001 & Lack of Security Awareness Training & \textbf{\textcolor{critical}{Critical}} & The absence of employee training makes the organization highly vulnerable to phishing and social engineering, which are primary vectors for initial compromise. \\
\noalign{\medskip}
RISK-002 & No Employee Acceptable Use Policy & \textbf{\textcolor{critical}{Critical}} & A foundational policy gap that creates ambiguity regarding secure practices, increasing the likelihood of insecure employee behavior and insider threats. \\
\noalign{\medskip}
RISK-003 & No MFA for Computer Logins & \textbf{\textcolor{high}{High}} & Allows for trivial lateral movement and privilege escalation within the network if an attacker compromises a single user's credentials. \\
\noalign{\medskip}
RISK-004 & Exposed SSH Service (Port 22) & \textbf{\textcolor{medium}{Medium}} & The SSH port is open to the internet, making it a constant target for brute-force attacks and exploitation of potential software vulnerabilities. This risk is exacerbated by the lack of security training. \\ \bottomrule
\end{tabular}
\end{table}

\section{Recommendations}
The following actionable recommendations are provided to mitigate the identified risks. They are prioritized to address the most critical issues first.

\subsection{Immediate Priority (Critical Risks)}
\begin{description}
    \item[R-1: Implement Security Awareness Training (Mitigates RISK-001)] \\
    Immediately establish a mandatory security awareness training program. This program must include onboarding training for all new hires and annual refresher courses for all staff. Content should focus on identifying phishing emails, password hygiene, and reporting security incidents.
    
    \item[R-2: Develop and Enforce an Acceptable Use Policy (Mitigates RISK-002)] \\
    Develop a comprehensive Employee Acceptable Use Policy (AUP) that clearly defines the rules for using company networks, devices, and data. Require all employees to read and formally acknowledge the policy.
\end{description}

\subsection{High Priority Recommendations}
\begin{description}
    \item[R-3: Enforce MFA on All Workstations (Mitigates RISK-003)] \\
    Deploy and enforce Multi-Factor Authentication (MFA) for all computer and system logins. This single control dramatically increases the difficulty for an attacker to pivot within the network after a credential compromise.
    
    \item[R-4: Harden Network Perimeter (Mitigates RISK-004)] \\
    Restrict access to the external SSH service on port 22. Access should be limited to specific, authorized IP addresses via firewall rules. The preferred method is to place the service behind a Virtual Private Network (VPN), requiring authentication before the port is accessible. Additionally, ensure the SSH service is securely configured (e.g., disable root login, enforce key-based authentication).
\end{description}

\end{document}
```