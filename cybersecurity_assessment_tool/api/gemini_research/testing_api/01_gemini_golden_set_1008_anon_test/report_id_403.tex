```latex
\documentclass[12pt]{article}

% Preamble: Required Packages
\usepackage[margin=1in]{geometry}
\usepackage{pifont} % For checkmarks and crosses
\usepackage{booktabs} % For professional tables
\usepackage{hyperref} % For clickable links
\usepackage{url}      % For proper URL formatting
\usepackage{seqsplit} % For splitting long strings in texttt

% Document Metadata
\title{Cybersecurity Posture Assessment Report}
\author{Cybersecurity Analysis Division}
\date{\today}

\begin{document}

\maketitle
\thispagestyle{empty}
\newpage
\tableofcontents
\newpage

% ==============================================================================
\section{Executive Overview}
% ==============================================================================

This report details the findings of a cybersecurity assessment for \textbf{[Organization Name]}. The analysis is based on a combination of technical network scanning, a review of organizational security controls, and an evaluation of pre-existing risks.

The assessment has identified several critical and high-risk security gaps. The most significant finding is a complete lack of Multi-Factor Authentication (MFA) for email, computer logins, and sensitive data systems. This represents a critical vulnerability, as it leaves the organization highly susceptible to unauthorized access via compromised credentials.

Furthermore, the absence of annual security awareness training for all employees constitutes a high risk, increasing the likelihood of successful phishing and social engineering attacks. A technical scan revealed an externally exposed Secure Shell (SSH) service on port 22. When combined with the lack of MFA, this exposed service presents a significant and immediate threat of brute-force attacks.

Immediate remediation is required to address these findings and reduce the organization's attack surface.

% ==============================================================================
\section{Organizational Information}
% ==============================================================================

The following information was used as the basis for this assessment. Due to the anonymized nature of the provided data, placeholders have been used where necessary.

\begin{itemize}
    \item \textbf{Organization Name:} \textbf{[Organization Name]}
    \item \textbf{Primary Domain:} \texttt{[Domain]}
    \item \textbf{External IP Address Scanned:} \texttt{[Client IP]}
\end{itemize}

% ==============================================================================
\section{Security Control Review}
% ==============================================================================

A review of the organization's security policies and procedures was conducted via a questionnaire. The responses indicate significant gaps in fundamental security controls, particularly regarding access management and employee training. "No" answers highlight areas requiring immediate attention.

\begin{table}[h!]
\centering
\caption{Security Controls Questionnaire Results}
\begin{tabular}{p{0.8\linewidth} c}
\toprule
\textbf{Control Question} & \textbf{Status} \\
\midrule
Do you require MFA to access email? & \ding{55} \\
Do you require MFA to log into computers? & \ding{55} \\
Do you require MFA to access sensitive data systems? & \ding{55} \\
Does your organization have an employee acceptable use policy? & \ding{51} \\
Does your organization do security awareness training for new employees? & \ding{51} \\
Does your organization do security awareness training for all employees at least once per year? & \ding{55} \\
\bottomrule
\end{tabular}
\end{table}

\vspace{1em}
\noindent \textbf{Key:} \ding{51} = Yes (Control in place) \quad \ding{55} = No (Control gap identified)

% ==============================================================================
\section{Technical Scan Results}
% ==============================================================================

An external network scan was performed on the target IP address to identify open ports and exposed services.

\begin{itemize}
    \item \textbf{Target IP:} \texttt{[Target IP]}
    \item \textbf{Scan Date:} Not provided in scan data.
\end{itemize}

The scan identified the following open port:

\begin{table}[h!]
\centering
\caption{Open Ports Detected}
\begin{tabular}{l l l l}
\toprule
\textbf{Port} & \textbf{Protocol} & \textbf{Service} & \textbf{State} \\
\midrule
22 & TCP & SSH (Secure Shell) & Open \\
\bottomrule
\end{tabular}
\end{table}

\subsection*{Analysis of Technical Findings}
The presence of an open SSH port (22/TCP) indicates that a remote management service is exposed to the public internet. While necessary for remote administration, it is a primary target for attackers. Without proper hardening—such as disallowing password-based authentication, implementing fail2ban, and restricting source IPs—this service is vulnerable to brute-force and credential stuffing attacks.

% ==============================================================================
\section{Risk Assessment}
% ==============================================================================

The following table synthesizes findings from the security control review and the technical scan. No pre-existing vulnerabilities were reported. The severity level is assigned based on the potential impact and likelihood of exploitation.

\begin{table}[h!]
\centering
\caption{Summary of Identified Risks}
\begin{tabular}{p{0.25\linewidth} p{0.5\linewidth} p{0.15\linewidth}}
\toprule
\textbf{Risk Name} & \textbf{Overview} & \textbf{Severity} \\
\midrule
\textbf{Critical Lack of Multi-Factor Authentication (MFA)} & MFA is not enforced for email, computer logins, or sensitive systems. This allows an attacker with valid credentials (e.g., from a phishing attack) to gain full access without a second factor of authentication. & \textbf{Critical} \\
\addlinespace
\textbf{Inadequate Security Awareness Training} & Security training is not conducted annually for all staff. This increases organizational susceptibility to phishing, social engineering, and other human-targeted attacks, making credential compromise more likely. & \textbf{High} \\
\addlinespace
\textbf{Exposed SSH Management Port} & The SSH service on port 22 is open to the public internet. This service is a constant target for automated brute-force attacks. The risk is amplified by the lack of MFA across the organization. & \textbf{High} \\
\bottomrule
\end{tabular}
\end{table}

% ==============================================================================
\section{Recommendations}
% ==============================================================================

Based on the identified risks, the following prioritized actions are recommended to improve the security posture of \textbf{[Organization Name]}.

\subsection*{Immediate Priority (Critical Risks)}
\begin{enumerate}
    \item \textbf{Implement Multi-Factor Authentication (MFA):} Immediately deploy a mandatory MFA policy for all users across all critical systems. Prioritize the following:
    \begin{itemize}
        \item Email (e.g., Office 365, Google Workspace)
        \item Access to sensitive data systems and applications
        \item VPN and other remote access solutions
        \item Privileged user accounts
    \end{itemize}
\end{enumerate}

\subsection*{High Priority}
\begin{enumerate}
    \setcounter{enumi}{1}
    \item \textbf{Harden the Exposed SSH Service:}
    \begin{itemize}
        \item If possible, restrict access to the SSH port (22) to only known, trusted IP addresses using a firewall.
        \item Disable password-based authentication and enforce the use of public key authentication for SSH.
        \item Implement an intrusion prevention tool like \texttt{fail2ban} to automatically block IPs that exhibit malicious behavior.
    \end{itemize}
    \item \textbf{Establish an Annual Security Training Program:} Develop and implement a mandatory security awareness training program for all employees, to be completed at least once per year. This program should cover topics such as phishing identification, password hygiene, and acceptable use policies.
\end{enumerate}

\end{document}
```