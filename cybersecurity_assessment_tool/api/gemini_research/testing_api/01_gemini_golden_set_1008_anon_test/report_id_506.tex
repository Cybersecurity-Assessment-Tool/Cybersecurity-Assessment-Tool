```latex
\documentclass[12pt]{article}

% --- PACKAGES ---
\usepackage[margin=1in]{geometry}
\usepackage{pifont} % For checkmarks and crosses (\ding)
\usepackage{booktabs} % For professional tables
\usepackage{hyperref} % For clickable links
\usepackage{url} % For URL formatting
\usepackage{seqsplit} % To split long strings in texttt
\usepackage{graphicx}
\usepackage{xcolor}
\usepackage{fancyhdr}

% --- DOCUMENT METADATA ---
\title{Cybersecurity Posture Assessment Report}
\author{Cybersecurity Analysis Division}
\date{\today}

% --- HYPERREF SETUP ---
\hypersetup{
    colorlinks=true,
    linkcolor=blue,
    filecolor=magenta,      
    urlcolor=cyan,
    pdftitle={Cybersecurity Posture Assessment Report},
    pdfpagemode=FullScreen,
}

% --- HEADER & FOOTER ---
\pagestyle{fancy}
\fancyhf{}
\lhead{Confidential Report}
\rhead{\textbf{[Organization Name]}}
\cfoot{\thepage}

% --- DOCUMENT START ---
\begin{document}

\maketitle
\thispagestyle{empty}
\newpage

\tableofcontents
\newpage

% ======================================================================
\section{Executive Summary}
% ======================================================================

This report details the findings of a cybersecurity posture assessment conducted for \textbf{[Organization Name]}. The assessment combined an analysis of organizational security controls, a technical network scan, and a review of pre-existing risks.

The analysis revealed several \textbf{critical-risk vulnerabilities} and significant gaps in foundational security controls that expose the organization to a high likelihood of compromise. Key findings include:

\begin{itemize}
    \item \textbf{Exposed Vulnerable Service:} A publicly accessible FTP server was identified running a severely outdated version of vsftpd (2.3.4), which is known to contain a critical backdoor vulnerability (CVE-2011-2523). The server also permits anonymous login, allowing unauthorized access to files.
    \item \textbf{Lack of Multi-Factor Authentication (MFA):} MFA is not enforced for accessing core systems like email and employee computers. This represents a critical gap in identity and access management, significantly increasing the risk of account takeover and unauthorized access.
    \item \textbf{Inadequate Employee Training:} New employees do not receive security awareness training, creating a persistent vulnerability to social engineering attacks like phishing.
    \item \textbf{Outdated Systems:} Pre-existing risks confirm that the organization is running unsupported operating systems (Windows 7), which no longer receive security updates.
\end{itemize}

Immediate and decisive action is required to remediate these issues. The recommendations outlined in this report prioritize the most critical risks to reduce the immediate threat of a significant security incident, such as a data breach or ransomware attack.

% ======================================================================
\section{Organizational Information}
% ======================================================================

The following information was used as the basis for this assessment.

\begin{itemize}
    \item \textbf{Organization Name:} \textbf{[Organization Name]}
    \item \textbf{Primary Email Domain:} \texttt{[Domain]}
    \item \textbf{External IP Scanned:} \texttt{[Client IP]}
\end{itemize}

% ======================================================================
\section{Security Control Review}
% ======================================================================

A review of the organization's security controls was conducted via a questionnaire. The responses highlight significant gaps in fundamental security practices. A "No" response indicates a deviation from security best practices and a potential risk.

\begin{table}[h!]
\centering
\caption{Security Controls Questionnaire Analysis}
\begin{tabular}{p{0.6\linewidth} c p{0.2\linewidth}}
\toprule
\textbf{Control Question} & \textbf{Response} & \textbf{Assessment} \\
\midrule
Do you require MFA to access email? & {\color{red}\ding{55}} & Critical Gap \\
Do you require MFA to log into computers? & {\color{red}\ding{55}} & Critical Gap \\
Do you require MFA to access sensitive data systems? & {\color{green}\ding{51}} & Best Practice Met \\
Does your organization have an employee acceptable use policy? & {\color{green}\ding{51}} & Best Practice Met \\
Does your organization do security awareness training for new employees? & {\color{red}\ding{55}} & High Risk \\
Does your organization do security awareness training for all employees at least once per year? & {\color{green}\ding{51}} & Best Practice Met \\
\bottomrule
\end{tabular}
\end{table}

% ======================================================================
\section{Technical Scan Results}
% ======================================================================

An external network scan was performed against the target IP address \texttt{[Target IP]}. The scan identified one open port with a critically vulnerable service.

\begin{table}[h!]
\centering
\caption{Open Port Analysis for Target: \texttt{[Target IP]}}
\begin{tabular}{l l l l p{0.3\linewidth}}
\toprule
\textbf{Port} & \textbf{State} & \textbf{Service} & \textbf{Product \& Version} & \textbf{Notes} \\
\midrule
21/tcp & Open & ftp & vsftpd 2.3.4 & \textbf{Critical Vulnerability.} Anonymous FTP login is allowed. This version contains a known backdoor (CVE-2011-2523). \\
\bottomrule
\end{tabular}
\end{table}

% ======================================================================
\section{Overall Risk Assessment}
% ======================================================================

The following table synthesizes findings from the technical scan, control review, and pre-existing risk data into a prioritized list.

\begin{table}[h!]
\centering
\caption{Synthesized Risk Summary}
\begin{tabular}{p{0.25\linewidth} p{0.5\linewidth} l}
\toprule
\textbf{Risk / Vulnerability} & \textbf{Description} & \textbf{Severity} \\
\midrule
\textbf{Vulnerable FTP Server (CVE-2011-2523)} & An internet-facing FTP server is running a version with a known backdoor. Anonymous login is enabled, allowing attackers to easily gain access and potentially execute arbitrary code on the server. & \textbf{Critical} \\
\addlinespace
\textbf{Lack of Multi-Factor Authentication (MFA)} & MFA is not required for email or computer logins. This makes the organization highly susceptible to credential theft, phishing, and brute-force attacks, which could lead to a full network compromise. & \textbf{Critical} \\
\addlinespace
\textbf{Inadequate Employee Onboarding} & New employees are not provided with security awareness training. This makes them a prime target for social engineering attacks, which can serve as an initial entry point for attackers. & \textbf{High} \\
\addlinespace
\textbf{Outdated Windows Policy} & Workstations are running the unsupported Windows 7 operating system. These systems no longer receive security patches, leaving them vulnerable to a wide range of known exploits. & \textbf{Medium} \\
\bottomrule
\end{tabular}
\end{table}

% ======================================================================
\section{Recommendations}
% ======================================================================

The following actions are recommended to mitigate the identified risks. They are prioritized based on severity and potential impact.

% --- Subsection for Critical Risks ---
\subsection{Immediate Actions (Critical Risks)}
\begin{enumerate}
    \item \textbf{Remediate Vulnerable FTP Server:}
    \begin{itemize}
        \item \textbf{Action:} Immediately take the public-facing FTP server at \texttt{[Target IP]} offline.
        \item \textbf{Justification:} The server is exposed and contains a known backdoor, representing an active and immediate threat.
        \item \textbf{Long-Term Solution:} If file transfer is a business requirement, decommission the current server and replace it with a secure alternative such as SFTP or a modern cloud-based file-sharing service that enforces strong authentication.
    \end{itemize}
    \item \textbf{Implement Multi-Factor Authentication (MFA):}
    \begin{itemize}
        \item \textbf{Action:} Begin the phased rollout of MFA across the organization.
        \item \textbf{Prioritization:}
            \begin{enumerate}
                \item All external-facing services (email, VPN).
                \item All administrator and privileged accounts.
                \item All employee computer logins.
            \end{enumerate}
        \item \textbf{Justification:} MFA is one of the most effective controls for preventing unauthorized access resulting from compromised credentials.
    \end{itemize}
\end{enumerate}

% --- Subsection for High Risks ---
\subsection{High-Priority Actions}
\begin{enumerate}
    \setcounter{enumi}{2} % Continue numbering
    \item \textbf{Establish Security Training for New Hires:}
    \begin{itemize}
        \item \textbf{Action:} Integrate a mandatory security awareness training module into the new employee onboarding process.
        \item \textbf{Content:} The training should cover, at a minimum, phishing awareness, password hygiene, and the company's acceptable use policy.
        \item \textbf{Justification:} This closes a significant gap and reduces the risk of human error leading to a security incident.
    \end{itemize}
\end{enumerate}

% --- Subsection for Medium Risks ---
\subsection{Medium-Priority Actions}
\begin{enumerate}
    \setcounter{enumi}{3} % Continue numbering
    \item \textbf{Accelerate Operating System Upgrades:}
    \begin{itemize}
        \item \textbf{Action:} Prioritize and accelerate the existing project to upgrade or replace all workstations and servers running Windows 7.
        \item \textbf{Justification:} Using unsupported operating systems exposes the organization to numerous vulnerabilities that will not be patched.
    \end{itemize}
\end{enumerate}

\end{document}
```