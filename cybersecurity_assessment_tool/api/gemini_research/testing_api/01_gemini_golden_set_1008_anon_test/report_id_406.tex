```latex
\documentclass[12pt]{article}

% Preamble: Required Packages
\usepackage[margin=1in]{geometry}
\usepackage{pifont} % For checkmarks and crosses
\usepackage{booktabs} % For professional tables
\usepackage{hyperref} % For clickable links and TOC
\usepackage{url}
\usepackage{seqsplit} % To split long strings in texttt
\usepackage{graphicx}
\usepackage[table]{xcolor}
\usepackage{fancyhdr}

% --- Document Setup ---
\hypersetup{
    colorlinks=true,
    linkcolor=blue,
    filecolor=magenta,      
    urlcolor=cyan,
    pdftitle={Cybersecurity Posture Assessment Report},
    pdfpagemode=FullScreen,
}

\pagestyle{fancy}
\fancyhf{}
\fancyhead[L]{\textbf{Cybersecurity Posture Assessment}}
\fancyfoot[C]{\thepage}
\renewcommand{\headrulewidth}{0.4pt}
\renewcommand{\footrulewidth}{0.4pt}

% --- Custom Commands ---
\newcommand{\yes}{\ding{51}} % Green checkmark
\newcommand{\no}{\ding{55}}  % Red X

%========================================================================================
% --- DOCUMENT START ---
%========================================================================================
\begin{document}

% --- Title Page ---
\begin{titlepage}
    \centering
    \vspace*{1cm}
    
    \includegraphics[width=0.4\textwidth]{example-image-a} % Placeholder logo
    
    \vspace{1.5cm}
    
    \Huge
    \textbf{Cybersecurity Posture Assessment Report}
    
    \vspace{1.5cm}
    
    \Large
    Prepared for: \textbf{[Organization Name]}
    
    \vspace{2cm}
    
    \large
    Report Date: \today
    
    \vfill
    
    \normalsize
    \textit{This report contains sensitive information and should be handled with the utmost confidentiality. Distribution is restricted to authorized personnel only.}
    
\end{titlepage}

\tableofcontents
\newpage

%========================================================================================
% --- 1. Executive Summary ---
%========================================================================================
\section{Executive Summary}

This report details the findings of a cybersecurity posture assessment conducted for \textbf{[Organization Name]}. The assessment combined a review of organizational security controls via a questionnaire, an external network vulnerability scan, and an analysis of pre-existing risks.

The overall security posture presents several \textbf{critical areas for improvement}, primarily related to administrative and access controls. While the external network scan of the specified target did not reveal any open ports or services—a positive finding—this does not mitigate the significant risks identified in the organization's policies and procedures.

Key findings include critical gaps in the enforcement of Multi-Factor Authentication (MFA) for email and sensitive data systems. Furthermore, the absence of a formal Acceptable Use Policy and a recurring, comprehensive security awareness training program for all employees indicates a foundational weakness in the organization's security culture and governance.

Immediate action is recommended to address these gaps to reduce the risk of unauthorized access, data breaches, and social engineering attacks. The recommendations outlined in this report are prioritized to address the most severe risks first.

%========================================================================================
% --- 2. Organizational Information ---
%========================================================================================
\section{Organizational Information}

This section contains the high-level information used as the basis for this assessment. As per the provided data, placeholders have been used where specific details were not available.

\begin{table}[h!]
\centering
\begin{tabular}{@{}ll@{}}
\toprule
\textbf{Attribute} & \textbf{Value} \\ \midrule
Organization Name & \textbf{[Organization Name]} \\
Primary Email Domain & \texttt{[Domain]} \\
Scanned External IP & \texttt{[Client IP]} \\
Assessment Date & \today \\ \bottomrule
\end{tabular}
\caption{Client Organizational Details.}
\end{table}

%========================================================================================
% --- 3. Security Control Review ---
%========================================================================================
\section{Security Control Review (Questionnaire Analysis)}

The following table summarizes the organization's responses to a security controls questionnaire. "No" answers indicate significant gaps in the security framework and are correlated with identified risks in Section 5.

\begin{table}[h!]
\centering
\renewcommand{\arraystretch}{1.3}
\begin{tabular}{@{}p{0.6\linewidth}cp{0.2\linewidth}@{}}
\toprule
\textbf{Control Question} & \textbf{Response} & \textbf{Assessment} \\ \midrule
Do you require MFA to access email? & \no & \cellcolor{red!25}Critical Gap \\
Do you require MFA to log into computers? & \yes & \cellcolor{green!25}Positive Control \\
Do you require MFA to access sensitive data systems? & \no & \cellcolor{red!25}Critical Gap \\
Does your organization have an employee acceptable use policy? & \no & \cellcolor{orange!25}High Risk \\
Does your organization do security awareness training for new employees? & \yes & \cellcolor{green!25}Positive Control \\
Does your organization do security awareness training for all employees at least once per year? & \no & \cellcolor{orange!25}High Risk \\ \bottomrule
\end{tabular}
\caption{Analysis of Security Control Questionnaire.}
\end{table}

%========================================================================================
% --- 4. Technical Scan Results ---
%========================================================================================
\section{Technical Scan Results}

An external network scan was performed to identify potential vulnerabilities visible from the public internet.

\subsection{Scan Details}
\begin{itemize}
    \item \textbf{Target IP Address:} \texttt{[Target IP]}
    \item \textbf{Scan Date:} \today
    \item \textbf{Scan Type:} TCP Port Scan (Top 1000 ports)
\end{itemize}

\subsection{Findings}
The scan completed successfully and \textbf{found no open TCP ports or exposed services} on the target system.

\subsubsection{Interpretation}
This result indicates that the target system is either offline, not in use, or protected by a well-configured firewall that drops or rejects unsolicited incoming traffic. While a clean external scan is a positive sign from a network perimeter perspective, it does not provide insight into the security of internal systems, web applications, or the risks associated with phishing and credential theft, which are heavily influenced by the administrative controls detailed in Section 3.

%========================================================================================
% --- 5. Risk Assessment ---
%========================================================================================
\section{Risk Assessment}

This section synthesizes findings from the security control review and technical scan. The risks below are derived directly from the identified control gaps. No pre-existing vulnerabilities were provided for this assessment.

\begin{table}[h!]
\centering
\renewcommand{\arraystretch}{1.3}
\begin{tabular}{@{}lp{0.5\linewidth}ll@{}}
\toprule
\textbf{Risk ID} & \textbf{Description} & \textbf{Severity} & \textbf{Source} \\ \midrule
RISK-001 & Lack of MFA on email exposes the organization to account takeover via credential theft, leading to data breaches and phishing attacks. & \textbf{Critical} & Questionnaire \\
RISK-002 & Lack of MFA on sensitive data systems removes a critical layer of defense, increasing the risk of unauthorized access to confidential data. & \textbf{Critical} & Questionnaire \\
RISK-003 & The absence of an Acceptable Use Policy leads to inconsistent employee behavior and a lack of enforceable security standards. & \textbf{High} & Questionnaire \\
RISK-004 & Failure to provide annual security training for all staff increases susceptibility to social engineering, phishing, and malware infections. & \textbf{High} & Questionnaire \\
\bottomrule
\end{tabular}
\caption{Summary of Identified Risks.}
\end{table}

%========================================================================================
% --- 6. Recommendations ---
%========================================================================================
\section{Recommendations}

The following prioritized, actionable recommendations are provided to mitigate the identified risks and improve the overall security posture of \textbf{[Organization Name]}.

\subsection{Immediate Priority (Critical Risks)}

\begin{enumerate}
    \item \textbf{Implement and Enforce MFA (RISK-001 \& RISK-002):}
    \begin{itemize}
        \item \textbf{Action:} Immediately deploy and mandate the use of Multi-Factor Authentication (MFA) for all user access to email (e.g., Office 365, Google Workspace) and all systems identified as containing sensitive data.
        \item \textbf{Justification:} This is the single most effective control to prevent account takeovers resulting from stolen credentials. It directly mitigates the two highest-severity risks identified.
    \end{itemize}
\end{enumerate}

\subsection{High Priority (High Risks)}

\begin{enumerate}
    \setcounter{enumi}{1} % Continue numbering
    \item \textbf{Develop an Acceptable Use Policy (AUP) (RISK-003):}
    \begin{itemize}
        \item \textbf{Action:} Create a formal AUP that clearly defines the rules and expectations for the use of company technology, networks, and data. This policy must be communicated to all employees and acknowledgement of receipt should be tracked.
        \item \textbf{Justification:} An AUP establishes a baseline for secure behavior, reduces insider threats, and provides a legal framework for enforcing security rules.
    \end{itemize}
    
    \item \textbf{Establish a Recurring Security Awareness Program (RISK-004):}
    \begin{itemize}
        \item \textbf{Action:} Implement a mandatory security awareness training program for all employees to be completed at least annually. The program should cover key topics such as phishing, password security, and data handling. Consider regular phishing simulations to reinforce training.
        \item \textbf{Justification:} A well-trained workforce is the first line of defense against common cyberattacks. Continuous training ensures that security remains a top-of-mind concern for all staff.
    \end{itemize}
\end{enumerate}

\subsection{General Recommendations}

\begin{enumerate}
    \setcounter{enumi}{3} % Continue numbering
    \item \textbf{Enhance Vulnerability Management:}
    \begin{itemize}
        \item \textbf{Action:} While the external scan was clean, it is recommended to establish a regular cadence of both external and internal vulnerability scans. Internal scans can provide a more comprehensive view of vulnerabilities that are not visible from the internet.
        \item \textbf{Justification:} Proactive scanning helps identify and remediate technical vulnerabilities before they can be exploited by attackers.
    \end{itemize}
\end{enumerate}

\end{document}
```