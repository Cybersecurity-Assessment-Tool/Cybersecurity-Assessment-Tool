```latex
\documentclass[12pt]{article}

% Preamble: Required Packages
\usepackage[margin=1in]{geometry}
\usepackage{pifont} % For \ding symbols (checkmarks)
\usepackage{booktabs} % For professional tables
\usepackage{hyperref} % For clickable links
\usepackage{url} % For formatting URLs
\usepackage{seqsplit} % For splitting long strings in texttt
\usepackage{xcolor} % For colors

% Document Information
\title{Cybersecurity Posture Assessment Report}
\author{Cybersecurity Analysis Division}
\date{\today}

% Hyperref Setup
\hypersetup{
    colorlinks=true,
    linkcolor=blue,
    filecolor=magenta,      
    urlcolor=cyan,
    pdftitle={Cybersecurity Posture Assessment Report},
    pdfpagemode=FullScreen,
}

\begin{document}

\maketitle
\thispagestyle{empty}
\newpage
\tableofcontents
\newpage

% --- 1. Executive Summary ---
\section{Executive Summary}

This report provides a comprehensive analysis of the cybersecurity posture for \textbf{[Organization Name]}. The assessment combines a review of organizational security controls, an external network scan, and an analysis of pre-existing risks.

The most critical finding of this assessment is the direct exposure of a Remote Desktop Protocol (RDP) service on port 3389 to the public internet at \texttt{[Client IP]}. This configuration represents a severe and immediate threat, as exposed RDP is a primary vector for ransomware attacks and unauthorized network access.

This critical technical vulnerability is dangerously amplified by systemic organizational weaknesses. The complete absence of Multi-Factor Authentication (MFA) for email, computer logins, and sensitive data access means that a single compromised password could lead to a full network breach. Furthermore, the lack of an employee acceptable use policy and annual security awareness training indicates a need for foundational improvements in the organization's security culture.

Immediate remediation is required to close the exposed RDP port. Strategic initiatives must be launched to implement MFA across the enterprise and develop core security policies and training programs.

% --- 2. Organizational Information ---
\section{Organizational Information}

This section details the information provided by the client organization. The placeholders indicate that this information was not available at the time of the assessment.

\begin{itemize}
    \item \textbf{Organization Name:} \textbf{[Organization Name]}
    \item \textbf{Primary Email Domain:} \texttt{[Domain]}
    \item \textbf{External IP Address Scanned:} \texttt{[Client IP]}
\end{itemize}

% --- 3. Security Control Review ---
\section{Security Control Review}

The following table summarizes the organization's responses to a security controls questionnaire. Answers marked with \textcolor{red}{\ding{55}} indicate significant gaps in security posture and represent areas of high risk.

\begin{table}[h!]
\centering
\caption{Security Controls Questionnaire Analysis}
\begin{tabular}{p{0.7\textwidth}c}
\toprule
\textbf{Control Question} & \textbf{Status} \\
\midrule
Do you require MFA to access email? & \textcolor{red}{\ding{55}} \\
Do you require MFA to log into computers? & \textcolor{red}{\ding{55}} \\
Do you require MFA to access sensitive data systems? & \textcolor{red}{\ding{55}} \\
Does your organization have an employee acceptable use policy? & \textcolor{red}{\ding{55}} \\
Does your organization do security awareness training for new employees? & \textcolor{green}{\ding{51}} \\
Does your organization do security awareness training for all employees at least once per year? & \textcolor{red}{\ding{55}} \\
\bottomrule
\end{tabular}
\end{table}

\subsection*{Analysis of Gaps}
\begin{itemize}
    \item \textbf{Lack of MFA:} The absence of MFA for email, computer, and data access is a critical vulnerability. This significantly lowers the barrier for attackers, making credential stuffing and phishing attacks highly effective.
    \item \textbf{Policy Deficiencies:} Lacking an Acceptable Use Policy (AUP) and annual security training creates an environment where employees are unaware of their security responsibilities, increasing the likelihood of human error leading to a security incident.
\end{itemize}

% --- 4. Technical Scan Results ---
\section{Technical Scan Results}

An external network scan was performed on the target IP address to identify open ports and exposed services.

\begin{itemize}
    \item \textbf{Target IP Address:} \texttt{[Target IP]}
    \item \textbf{Scan Tool:} Nmap
\end{itemize}

The scan revealed the following open port, which is accessible from the public internet:

\begin{table}[h!]
\centering
\caption{Open Port Discovery}
\begin{tabular}{llll}
\toprule
\textbf{Port} & \textbf{State} & \textbf{Service Name} & \textbf{Description} \\
\midrule
3389/tcp & open & ms-wbt-server & Microsoft Remote Desktop Protocol (RDP) \\
\bottomrule
\end{tabular}
\end{table}

\subsection*{Technical Analysis}
The discovery of an open RDP port is a critical finding. RDP is a common target for brute-force password attacks and exploitation of vulnerabilities (e.g., BlueKeep). Exposing this service directly to the internet without mitigating controls like a VPN or IP whitelisting is highly discouraged and places the organization at extreme risk of a security breach. This technical finding directly corroborates the pre-existing risk identified as "RDP Exposure".

% --- 5. Consolidated Risk Assessment ---
\section{Consolidated Risk Assessment}

This section synthesizes findings from the security control review, technical scan, and pre-existing risk data into a prioritized list of risks.

\begin{table}[h!]
\centering
\caption{Synthesized Risk Register}
\begin{tabular}{p{0.25\textwidth}p{0.5\textwidth}l}
\toprule
\textbf{Risk Title} & \textbf{Description} & \textbf{Severity} \\
\midrule
\textbf{Exposed RDP Service} & Port 3389 (RDP) is open to the internet on \texttt{[Target IP]}, allowing attackers to attempt unauthorized access. This is a primary vector for ransomware. & \textbf{Critical} \\
\addlinespace
\textbf{Systemic Lack of MFA} & The absence of Multi-Factor Authentication for all critical access points (email, login, data) makes password-based attacks trivial to execute. & \textbf{Critical} \\
\addlinespace
\textbf{Insufficient Security Policies \& Training} & The lack of an Acceptable Use Policy and recurring annual security training weakens the human firewall and increases susceptibility to social engineering. & \textbf{High} \\
\bottomrule
\end{tabular}
\end{table}

% --- 6. Recommendations ---
\section{Recommendations}

The following actions are recommended to mitigate the identified risks and improve the overall security posture of \textbf{[Organization Name]}. Recommendations are prioritized based on risk severity and ease of implementation.

\subsection{Immediate Actions (Triage - To Be Completed in < 24 Hours)}
\begin{enumerate}
    \item \textbf{Block RDP Access:} Immediately configure the perimeter firewall to block all inbound traffic to TCP port 3389 on \texttt{[Client IP]}. If remote access is required, it must be placed behind a secure VPN.
\end{enumerate}

\subsection{Short-Term Actions (To Be Completed in < 30 Days)}
\begin{enumerate}
    \item \textbf{Implement MFA:} Prioritize the deployment of MFA for all employees and contractors, focusing on:
    \begin{itemize}
        \item Email access (e.g., Office 365, Google Workspace).
        \item VPN and remote access solutions.
        \item Access to all systems storing sensitive or critical data.
    \end{itemize}
    \item \textbf{Develop an Acceptable Use Policy (AUP):} Create and distribute a formal AUP that defines the rules and responsibilities for all users of the organization's IT systems.
\end{enumerate}

\subsection{Strategic Actions (To Be Completed in < 90 Days)}
\begin{enumerate}
    \item \textbf{Establish a Secure Remote Access Solution:} Decommission all direct-to-internet access for management protocols like RDP. Implement a corporate Virtual Private Network (VPN) with MFA as the sole method for remote network access.
    \item \textbf{Launch Annual Security Awareness Training:} Implement a mandatory security awareness training program for all employees to be completed annually. This program should cover topics such as phishing, password security, and corporate policies.
\end{enumerate}

\end{document}
```