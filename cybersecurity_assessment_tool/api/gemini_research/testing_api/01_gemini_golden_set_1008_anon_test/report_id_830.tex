```latex
\documentclass[12pt]{article}

% --- PACKAGES ---
\usepackage[margin=1in]{geometry} % Set page margins
\usepackage{pifont}               % For checkmarks and crosses (\ding)
\usepackage{booktabs}             % For professional-looking tables
\usepackage{graphicx}             % For logos (optional)
\usepackage[hidelinks]{hyperref}  % For clickable links without boxes
\usepackage{url}                  % For formatting URLs
\usepackage{seqsplit}             % For splitting long strings in texttt
\usepackage{array}                % For table column specifications

% --- DOCUMENT INFORMATION ---
\title{Cybersecurity Posture Assessment Report}
\author{Cybersecurity Analyst}
\date{\today}

% --- BEGIN DOCUMENT ---
\begin{document}

\maketitle
\thispagestyle{empty}
\newpage

\tableofcontents
\newpage

% ===================================================================
\section{Executive Summary}
% ===================================================================

This report provides a comprehensive analysis of the cybersecurity posture for \textbf{[Organization Name]}. The assessment is based on a correlation of external network scan data, a review of internal security controls via a questionnaire, and an evaluation of previously identified risks.

The analysis reveals several areas of significant concern requiring immediate attention. A critical vulnerability was identified on an external-facing system: an outdated and misconfigured FTP server (\texttt{vsftpd 2.3.4}) that allows anonymous access and is susceptible to remote code execution. This presents a direct and severe threat to the organization's network integrity and data confidentiality.

Furthermore, critical gaps were identified in the organization's administrative controls. The absence of a formal Acceptable Use Policy and the lack of mandatory annual security awareness training for all employees create a high-risk environment where human error can be easily exploited by threat actors. These policy-level deficiencies, combined with the technical vulnerabilities, significantly elevate the organization's overall risk profile.

This report outlines these findings in detail and provides a prioritized list of actionable recommendations to mitigate the identified risks and strengthen the overall security posture.

% ===================================================================
\section{Organizational Information}
% ===================================================================

The following information was used as the basis for this assessment. Due to the anonymized nature of the input data, placeholders have been used where necessary.

\begin{itemize}
    \item \textbf{Organization Name:} \textbf{[Organization Name]}
    \item \textbf{Primary Domain:} \texttt{[Domain]}
    \item \textbf{External IP Scanned:} \texttt{[Client IP]}
\end{itemize}

% ===================================================================
\section{Security Control Review}
% ===================================================================

An internal security questionnaire was reviewed to assess the maturity of existing administrative and technical controls. The responses are summarized below. Answers marked with \ding{55} indicate significant gaps in the security framework.

\begin{table}[h!]
\centering
\caption{Security Controls Questionnaire Responses}
\begin{tabular}{p{0.75\textwidth} c}
\toprule
\textbf{Control Question} & \textbf{Response} \\
\midrule
Do you require MFA to access email? & \ding{51} \\
Do you require MFA to log into computers? & \ding{51} \\
Do you require MFA to access sensitive data systems? & \ding{51} \\
\textbf{Does your organization have an employee acceptable use policy?} & \textbf{\ding{55}} \\
Does your organization do security awareness training for new employees? & \ding{51} \\
\textbf{Does your organization do security awareness training for all employees at least once per year?} & \textbf{\ding{55}} \\
\bottomrule
\end{tabular}
\end{table}

\subsection{Analysis of Control Gaps}
While the organization has implemented strong multi-factor authentication (MFA) controls, two critical administrative gaps were identified:
\begin{itemize}
    \item \textbf{No Acceptable Use Policy (AUP):} The absence of an AUP means there are no formally documented rules for employees regarding the use of company systems, data, and internet access. This increases the risk of insider threat, data leakage, and legal liability.
    \item \textbf{No Annual Security Awareness Training:} Security threats evolve rapidly. Without regular, ongoing training, employees are more susceptible to phishing, social engineering, and other common attack vectors. This gap undermines the effectiveness of technical controls.
\end{itemize}

% ===================================================================
\section{Technical Scan Results}
% ===================================================================

An external network scan was performed on the target IP address to identify open ports and exposed services.

\begin{itemize}
    \item \textbf{Target IP:} \texttt{[Target IP]}
    \item \textbf{Scan Tool:} Nmap
\end{itemize}

The scan identified one open port with a critically vulnerable service.

\begin{table}[h!]
\centering
\caption{Open Port Analysis}
\begin{tabular}{l l l l p{0.4\textwidth}}
\toprule
\textbf{Port} & \textbf{State} & \textbf{Service} & \textbf{Version} & \textbf{Notes} \\
\midrule
21/tcp & Open & ftp & vsftpd 2.3.4 & \textbf{Critical Risk.} Anonymous FTP login is allowed. This version is known to be vulnerable to a backdoor command execution flaw (CVE-2011-2523). \\
\bottomrule
\end{tabular}
\end{table}

\subsection{Analysis of Technical Findings}
The presence of an open FTP port running \texttt{vsftpd 2.3.4} is a severe security risk. This specific version contains a well-documented backdoor that was inserted into the source code, allowing an unauthenticated attacker to gain a command shell on the server. The risk is compounded by the fact that anonymous login is enabled, which allows any external entity to connect and potentially upload malicious files or exfiltrate sensitive data. This vulnerability requires immediate remediation.

% ===================================================================
\section{Consolidated Risk Assessment}
% ===================================================================

The following table synthesizes findings from the technical scan, control review, and pre-existing risk data into a consolidated list.

\begin{table}[h!]
\centering
\caption{Summary of Identified Risks}
\begin{tabular}{p{0.1\textwidth} p{0.4\textwidth} l p{0.25\textwidth}}
\toprule
\textbf{Risk ID} & \textbf{Risk Description} & \textbf{Severity} & \textbf{Affected Systems} \\
\midrule
RISK-001 & Vulnerable external FTP server (\texttt{vsftpd 2.3.4}) allows for remote code execution and anonymous access. & \textbf{Critical} & External Server (\texttt{[Target IP]}) \\
\addlinespace
RISK-002 & Lack of a formal Acceptable Use Policy (AUP) for employees. & High & All Employees, Company Assets \\
\addlinespace
RISK-003 & Lack of mandatory annual security awareness training for all staff. & High & All Employees \\
\addlinespace
RISK-004 & Outdated Windows 7 operating system in use on workstations. & Medium & Workstations \\
\bottomrule
\end{tabular}
\end{table}

% ===================================================================
\section{Recommendations}
% ===================================================================

The following actions are recommended to mitigate the identified risks, prioritized by severity.

\subsection{Immediate Actions (Critical Priority)}
\begin{enumerate}
    \item \textbf{Remediate Vulnerable FTP Server (RISK-001):}
        \begin{itemize}
            \item Immediately take the FTP service on \texttt{[Target IP]} offline or firewall it from public access.
            \item If FTP is a business requirement, upgrade the \texttt{vsftpd} software to the latest stable version and explicitly disable anonymous access.
            \item Evaluate the need for FTP and consider replacing it with a secure alternative such as SFTP (SSH File Transfer Protocol).
        \end{itemize}
\end{enumerate}

\subsection{High Priority Actions}
\begin{enumerate}
    \setcounter{enumi}{1}
    \item \textbf{Develop and Implement an Acceptable Use Policy (RISK-002):}
        \begin{itemize}
            \item Draft a formal AUP that clearly defines the rules for using company technology and data.
            \item Have the policy reviewed by legal and HR departments.
            \item Require all employees to read and formally acknowledge the policy.
        \end{itemize}
    \item \textbf{Establish an Annual Security Training Program (RISK-003):}
        \begin{itemize}
            \item Procure or develop a security awareness training program covering topics like phishing, password security, and social engineering.
            \item Make this training mandatory for all employees on an annual basis and track completion.
        \end{itemize}
\end{enumerate}

\subsection{Medium Priority Actions}
\begin{enumerate}
    \setcounter{enumi}{3}
    \item \textbf{Plan Operating System Upgrades (RISK-004):}
        \begin{itemize}
            \item Develop a phased plan to upgrade all workstations from Windows 7, which is end-of-life and no longer receives security updates, to a supported version like Windows 10 or 11.
        \end{itemize}
\end{enumerate}

\end{document}
```