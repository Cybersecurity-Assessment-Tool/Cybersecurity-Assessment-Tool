```latex
\documentclass[12pt, a4paper]{article}

% Preamble: Required Packages
\usepackage[margin=1in]{geometry}
\usepackage{pifont} % For checkmarks and crosses (\ding)
\usepackage{booktabs} % For professional tables
\usepackage{hyperref} % For hyperlinks
\usepackage{url} % For URL formatting
\usepackage{seqsplit} % To split long strings without spaces
\usepackage{graphicx}
\usepackage{xcolor}
\usepackage{fancyhdr}

% Document Metadata
\title{Cybersecurity Posture Assessment Report}
\author{Cybersecurity Analysis Division}
\date{November 22, 2025}

% Define colors for use in the report
\definecolor{darkblue}{rgb}{0.0, 0.0, 0.55}
\definecolor{darkred}{rgb}{0.55, 0.0, 0.0}

% Hyperref setup
\hypersetup{
    colorlinks=true,
    linkcolor=darkblue,
    filecolor=darkblue,      
    urlcolor=darkblue,
    citecolor=darkblue,
}

% Header and Footer
\pagestyle{fancy}
\fancyhf{}
\lhead{\textbf{Cybersecurity Assessment Report}}
\rhead{\textbf{[Organization Name]}}
\cfoot{\thepage}

\begin{document}

\maketitle
\thispagestyle{empty}
\newpage

\tableofcontents
\newpage

% --- 1. Executive Summary ---
\section{Executive Summary}
This report details the findings of a cybersecurity posture assessment conducted for \textbf{[Organization Name]} on November 22, 2025. The assessment combined a review of organizational security controls, an external network scan, and an analysis of known risks.

The analysis revealed several high-risk and critical vulnerabilities that require immediate attention. Key findings include:
\begin{itemize}
    \item \textbf{Critical Gaps in Access Control:} Multi-Factor Authentication (MFA) is not enforced for accessing email or other sensitive data systems. This exposes the organization to a significant risk of account compromise and data breach.
    \item \textbf{Outdated Internet-Facing Software:} The external web server is running an outdated version of Nginx (1.18.0), which is no longer supported and contains known, exploitable vulnerabilities.
    \item \textbf{Deficient Security Awareness Program:} The organization lacks a formal security awareness training program for both new and existing employees, increasing susceptibility to phishing and other social engineering attacks.
\end{itemize}

The combination of these findings indicates a high-risk security posture. Immediate remediation of the identified issues is strongly recommended to reduce the likelihood of a security incident.

% --- 2. Organizational Information ---
\section{Organizational Information}
The following information was used as the basis for this assessment.
\begin{itemize}
    \item \textbf{Organization Name:} \textbf{[Organization Name]}
    \item \textbf{Primary Domain:} \texttt{[Domain]}
    \item \textbf{Assessed External IP:} \texttt{[Client IP]}
\end{itemize}

% --- 3. Security Control Review ---
\section{Security Control Review}
A review of organizational security controls was conducted based on a standardized questionnaire. The responses indicate significant gaps in fundamental security practices, particularly concerning identity and access management and employee security training.

\begin{table}[h!]
\centering
\caption{Organizational Security Control Questionnaire Results}
\begin{tabular}{p{0.6\linewidth} c p{0.2\linewidth}}
\toprule
\textbf{Control Question} & \textbf{Response} & \textbf{Assessment} \\
\midrule
Do you require MFA to access email? & \ding{55} & \textcolor{darkred}{\textbf{Critical Gap}} \\
Do you require MFA to log into computers? & \ding{51} & Best Practice Met \\
Do you require MFA to access sensitive data systems? & \ding{55} & \textcolor{darkred}{\textbf{Critical Gap}} \\
Does your organization have an employee acceptable use policy? & \ding{51} & Best Practice Met \\
Does your organization do security awareness training for new employees? & \ding{55} & \textcolor{darkred}{\textbf{High Risk}} \\
Does your organization do security awareness training for all employees at least once per year? & \ding{55} & \textcolor{darkred}{\textbf{High Risk}} \\
\bottomrule
\end{tabular}
\end{table}

% --- 4. Technical Scan Results ---
\section{Technical Scan Results}
An external network vulnerability scan was performed on the date specified in this report's header. The scan targeted the organization's public-facing infrastructure to identify open ports and exposed services.

\subsection{Scan Details}
\begin{itemize}
    \item \textbf{Target IP Address:} \texttt{[Target IP]}
    \item \textbf{Scan Date:} 2025-11-22T10:00:00Z
\end{itemize}

\subsection{Open Ports and Services}
The scan identified the following open port and service.

\begin{table}[h!]
\centering
\caption{Identified Open Ports}
\begin{tabular}{l l l l}
\toprule
\textbf{Port} & \textbf{State} & \textbf{Service} & \textbf{Product / Version} \\
\midrule
443/TCP & Open & HTTPS & Nginx / 1.18.0 \\
\bottomrule
\end{tabular}
\end{table}

\subsection{Technical Analysis}
The primary finding from the technical scan is the presence of \textbf{Nginx version 1.18.0}. This version was released in April 2020 and is significantly outdated. Current stable versions are several major releases ahead. Running outdated software on internet-facing systems poses a severe security risk, as it is likely susceptible to numerous publicly known vulnerabilities that have been patched in more recent versions. This system should be considered highly vulnerable to exploitation.

% --- 5. Risk Assessment ---
\section{Risk Assessment}
Based on the correlation of organizational and technical findings, the following risks have been identified. As no pre-existing risks were provided, this list is based solely on the current assessment.

\begin{table}[h!]
\centering
\caption{Summary of Identified Risks}
\begin{tabular}{p{0.1\linewidth} p{0.25\linewidth} p{0.45\linewidth} p{0.1\linewidth}}
\toprule
\textbf{ID} & \textbf{Risk Name} & \textbf{Description} & \textbf{Severity} \\
\midrule
\textbf{RISK-001} & Lack of MFA on Critical Systems & Email and sensitive data systems are accessible with only a username and password, making them highly vulnerable to credential stuffing and phishing attacks. & \textcolor{darkred}{\textbf{Critical}} \\
\textbf{RISK-002} & Outdated Web Server Software & The public-facing web server runs an unsupported version of Nginx, which is known to have multiple security vulnerabilities. & \textcolor{darkred}{\textbf{High}} \\
\textbf{RISK-003} & Inadequate Security Awareness Program & The absence of employee security training significantly increases the risk of human error leading to security incidents, such as successful phishing attacks. & \textcolor{darkred}{\textbf{High}} \\
\bottomrule
\end{tabular}
\end{table}

% --- 6. Recommendations ---
\section{Recommendations}
To mitigate the identified risks and improve the overall security posture of \textbf{[Organization Name]}, the following actions are recommended with urgency.

\begin{enumerate}
    \item \textbf{Implement Comprehensive MFA (Addresses: RISK-001):}
    \begin{itemize}
        \item \textbf{Immediate Priority:} Enforce MFA for all user accounts, especially for remote access to the corporate network, email (\texttt{[Domain]}), and all systems designated as containing sensitive data.
        \item \textbf{Policy:} Update the access control policy to mandate MFA for all new systems and services.
    \end{itemize}
    
    \item \textbf{Remediate Outdated Software (Addresses: RISK-002):}
    \begin{itemize}
        \item \textbf{Immediate Priority:} Upgrade the Nginx server on \texttt{[Target IP]} from version 1.18.0 to the latest stable version immediately.
        \item \textbf{Policy:} Establish and implement a formal patch management policy. This policy should mandate regular scanning and timely patching of all systems, with a priority on internet-facing infrastructure.
    \end{itemize}
    
    \item \textbf{Establish a Security Awareness Program (Addresses: RISK-003):}
    \begin{itemize}
        \item \textbf{Immediate Priority:} Procure and deploy a security awareness training solution.
        \item \textbf{Policy:} Develop a program that includes mandatory security training for all new hires during onboarding and annual refresher training for all employees. This program should cover topics such as phishing, password security, and acceptable use.
    \end{itemize}
\end{enumerate}

% --- 7. Conclusion ---
\section{Conclusion}
The assessment for \textbf{[Organization Name]} has identified critical and high-severity risks that require immediate action. The lack of fundamental security controls like Multi-Factor Authentication and a security training program, combined with a vulnerable, internet-facing server, creates a significant risk of a data breach.

We strongly advise that the recommendations outlined in this report are prioritized and implemented without delay to strengthen the organization's defenses against common cyber threats.

\end{document}
```