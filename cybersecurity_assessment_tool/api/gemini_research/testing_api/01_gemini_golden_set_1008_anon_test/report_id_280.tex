```latex
\documentclass[12pt]{article}

% --- PACKAGES ---
\usepackage[margin=1in]{geometry}
\usepackage{pifont}         % For \ding symbols (checkmarks/crosses)
\usepackage{booktabs}       % For professional-looking tables
\usepackage{hyperref}       % For clickable links
\usepackage{url}            % For URL formatting
\usepackage{seqsplit}       % To split long strings without breaking words
\usepackage{fancyhdr}       % For custom headers and footers
\usepackage{xcolor}         % For defining colors
\usepackage{graphicx}

% --- DOCUMENT SETUP ---
\definecolor{darkblue}{rgb}{0.0, 0.0, 0.55}
\definecolor{darkred}{rgb}{0.55, 0.0, 0.0}

\hypersetup{
    colorlinks=true,
    linkcolor=darkblue,
    urlcolor=darkblue,
    citecolor=darkblue
}

% --- HEADER & FOOTER ---
\pagestyle{fancy}
\fancyhf{} % Clear all header and footer fields
\lhead{Cybersecurity Assessment Report}
\rhead{\textbf{[Organization Name]}}
\cfoot{Page \thepage}
\renewcommand{\headrulewidth}{0.4pt}
\renewcommand{\footrulewidth}{0.4pt}

% --- DOCUMENT START ---
\begin{document}

% --- TITLE PAGE ---
\begin{titlepage}
    \centering
    \vspace*{2cm}
    
    {\Huge \textbf{Cybersecurity Assessment Report}}
    
    \vspace{1.5cm}
    
    {\Large Prepared for:} \\
    \vspace{0.5cm}
    {\huge \textbf{[Organization Name]}}
    
    \vfill
    
    {\large \today}
    
    \vspace{1cm}
    
    \hrule
    \vspace{0.2cm}
    {\large \textbf{CONFIDENTIAL}}
    \vspace{0.2cm}
    \hrule
    
\end{titlepage}

\newpage

% --- TABLE OF CONTENTS ---
\tableofcontents
\newpage

% --- EXECUTIVE SUMMARY ---
\section*{Executive Summary}

This report provides a comprehensive analysis of the security posture of \textbf{[Organization Name]}, based on a review of organizational security controls, an external network scan, and pre-existing risk data.

The assessment identified a notable contrast between the organization's external network security and its internal security policies and controls. The external network scan of the target host revealed a strong security posture, with no open ports detected. This significantly reduces the external attack surface and is a commendable finding.

However, the review of organizational security controls uncovered several critical gaps that present a high level of risk. Key findings include:
\begin{itemize}
    \item \textbf{Lack of Multi-Factor Authentication (MFA)} for sensitive data systems, exposing critical assets to unauthorized access.
    \item \textbf{Absence of an Employee Acceptable Use Policy (AUP)}, leading to ambiguity in security responsibilities and acceptable user behavior.
    \item \textbf{Complete lack of a Security Awareness Training program} for both new and existing employees, increasing susceptibility to social engineering attacks like phishing.
\end{itemize}

These policy and procedure-based vulnerabilities currently outweigh the strong technical posture of the scanned external asset. Immediate action is recommended to address these internal control deficiencies to build a more resilient and holistic security program.

% --- ORGANIZATIONAL INFORMATION ---
\section*{Organizational Information}
This section details the information provided for the assessment. As the data was anonymized, placeholders are used where necessary.

\begin{tabular}{@{}ll}
    \toprule
    \textbf{Attribute} & \textbf{Value} \\
    \midrule
    Organization Name & \textbf{[Organization Name]} \\
    Email Domain & \texttt{[Domain]} \\
    Known External IP & \texttt{[Client IP]} \\
    \bottomrule
\end{tabular}

% --- SECURITY CONTROL REVIEW ---
\section*{Security Control Review}
The following table summarizes the responses to the security questionnaire. The status column indicates whether the control aligns with standard security best practices. A checkmark (\ding{51}) indicates alignment, while a cross (\ding{55}) signifies a critical gap.

\begin{table}[h!]
\centering
\begin{tabular}{@{}p{0.6\textwidth}cc@{}}
    \toprule
    \textbf{Control Question} & \textbf{Response} & \textbf{Status} \\
    \midrule
    Do you require MFA to access email? & Yes & \textcolor{green}{\ding{51}} \\
    Do you require MFA to log into computers? & Yes & \textcolor{green}{\ding{51}} \\
    Do you require MFA to access sensitive data systems? & No & \textcolor{red}{\ding{55}} \\
    Does your organization have an employee acceptable use policy? & No & \textcolor{red}{\ding{55}} \\
    Does your organization do security awareness training for new employees? & No & \textcolor{red}{\ding{55}} \\
    Does your organization do security awareness training for all employees at least once per year? & No & \textcolor{red}{\ding{55}} \\
    \bottomrule
\end{tabular}
\caption{Organizational Security Control Questionnaire Results}
\end{table}

\subsection*{Analysis of Control Gaps}
The questionnaire reveals significant weaknesses in administrative and access controls. While MFA is commendably enforced for email and computer logins, its absence on sensitive data systems is a critical oversight. Furthermore, the lack of both an Acceptable Use Policy and any form of security awareness training creates a high-risk environment where employees are unaware of their security responsibilities and are more vulnerable to attacks.

% --- TECHNICAL SCAN RESULTS ---
\section*{Technical Scan Results}
An external network vulnerability scan was performed to assess the perimeter security of the designated target.

\subsection*{Scan Details}
\begin{tabular}{@{}ll}
    \toprule
    \textbf{Parameter} & \textbf{Value} \\
    \midrule
    Scanner Used & Nmap \\
    Target IP & \texttt{[Target IP]} \\
    Scan Date & Not specified in scan data \\
    \bottomrule
\end{tabular}

\subsection*{Findings}
The scan results indicate that the target host \texttt{[Target IP]} is online and responsive. However, no open TCP or UDP ports were discovered. The scanner reported that the state of all other (non-scanned) ports was "closed".

\textbf{Conclusion:} This is a positive security finding. A host with no open ports presents a minimal attack surface to external threats, indicating effective firewall configuration and network segmentation.

% --- RISK ASSESSMENT ---
\section*{Risk Assessment}
This section synthesizes the findings from the security control review and technical scan. No pre-existing vulnerabilities were provided in the input data. The following risks have been identified during this assessment.

\begin{table}[h!]
\centering
\begin{tabular}{@{}p{0.2\textwidth}p{0.55\textwidth}l@{}}
    \toprule
    \textbf{Risk Name} & \textbf{Overview} & \textbf{Severity} \\
    \midrule
    \textbf{Lack of MFA on Sensitive Systems} & The absence of MFA on systems containing sensitive data significantly increases the risk of unauthorized access via compromised credentials. & \textcolor{red}{\textbf{High}} \\
    \addlinespace
    \textbf{Absence of Acceptable Use Policy (AUP)} & Without a formal AUP, employees lack clear guidance on the secure and acceptable use of company assets, increasing the likelihood of unintentional policy violations and insider threats. & \textcolor{red}{\textbf{High}} \\
    \addlinespace
    \textbf{Inadequate Security Awareness Training} & A lack of security training makes employees primary targets for phishing, social engineering, and malware attacks, as they are unable to recognize and appropriately respond to threats. & \textcolor{red}{\textbf{High}} \\
    \bottomrule
\end{tabular}
\caption{Identified Risks and Severity}
\end{table}

% --- RECOMMENDATIONS ---
\section*{Recommendations}
The following actions are recommended to mitigate the identified risks and improve the overall security posture of \textbf{[Organization Name]}.

\subsection*{Recommendation 1: Enforce MFA on All Sensitive Systems}
\textbf{Action:} Immediately prioritize and implement Multi-Factor Authentication (MFA) across all applications, databases, and systems that store, process, or transmit sensitive organizational or customer data. This control should be mandatory for all users, including administrators and service accounts where feasible.

\subsection*{Recommendation 2: Develop and Implement an Acceptable Use Policy}
\textbf{Action:} Create a formal Acceptable Use Policy (AUP) that clearly defines the rules and responsibilities for all employees when using company IT assets. The policy should be reviewed by legal counsel, communicated to all staff, and formally acknowledged via signature. It should be reviewed and updated annually.

\subsection*{Recommendation 3: Establish a Security Awareness Training Program}
\textbf{Action:} Implement a comprehensive security awareness training program. This program must include:
\begin{itemize}
    \item \textbf{Onboarding Training:} Mandatory training for all new employees before they are granted access to systems.
    \item \textbf{Annual Refresher Training:} Required for all employees to keep them updated on the latest threats and policies.
    \item \textbf{Phishing Simulations:} Regular, simulated phishing campaigns to test and improve employee resilience to social engineering.
\end{itemize}

\end{document}
```