```latex
\documentclass[12pt]{article}

% Preamble: Required Packages
\usepackage[a4paper, margin=1in]{geometry}
\usepackage{pifont} % For checkmarks and crosses
\usepackage{booktabs} % For professional tables
\usepackage{hyperref} % For clickable links and better PDF navigation
\usepackage{url} % For formatting URLs
\usepackage{seqsplit} % For splitting long strings in texttt
\usepackage{graphicx}
\usepackage{xcolor}

% Document Metadata
\title{Cybersecurity Assessment Report}
\author{Cybersecurity Analyst}
\date{November 22, 2025}

% Hyperref Setup
\hypersetup{
    colorlinks=true,
    linkcolor=blue,
    filecolor=magenta,      
    urlcolor=cyan,
    pdftitle={Cybersecurity Assessment Report},
    pdfpagemode=FullScreen,
}

\begin{document}

\maketitle
\hrule
\begin{center}
    \textbf{Prepared for: \textbf{[Organization Name]}}
\end{center}
\hrule
\vspace{2em}

\tableofcontents
\newpage

% --- 1. Executive Summary ---
\section{Executive Summary}
This report details the findings of a cybersecurity assessment conducted on November 22, 2025. The evaluation combined a technical network scan, a review of administrative security controls, and an analysis of pre-existing risks to provide a holistic view of the organization's security posture.

The assessment identified two high-risk issues requiring immediate attention. Firstly, the external-facing web server at \texttt{[Target IP]} is running an outdated version of Nginx (1.18.0), which is known to have multiple vulnerabilities. This exposes the organization to potential compromise. Secondly, a critical gap was identified in the organization's security program: the absence of a formal security awareness training program for both new and existing employees. This significantly increases the risk of successful phishing and social engineering attacks.

While the organization demonstrates strong security practices in implementing Multi-Factor Authentication (MFA) across key systems, the identified vulnerabilities could undermine these controls. This report provides specific, actionable recommendations to mitigate these risks and strengthen the overall security posture.

% --- 2. Organizational Information ---
\section{Organizational Information}
This assessment pertains to the following entity. As the provided data was anonymized, placeholders have been used where necessary.

\begin{itemize}
    \item \textbf{Organization Name:} \textbf{[Organization Name]}
    \item \textbf{Primary Domain:} \texttt{[Domain]}
    \item \textbf{External IP Scanned:} \texttt{[Client IP]}
\end{itemize}

% --- 3. Security Control Review ---
\section{Security Control Review}
A review of administrative and policy-based security controls was conducted via a standardized questionnaire. The responses indicate a strong foundation in identity and access management but reveal critical deficiencies in employee security training. The results are summarized in Table \ref{tab:controls}.

\begin{table}[h!]
\centering
\caption{Administrative Security Control Questionnaire}
\label{tab:controls}
\begin{tabular}{p{0.75\textwidth} c}
\toprule
\textbf{Control Question} & \textbf{Response} \\
\midrule
Do you require MFA to access email? & \ding{51} \\
Do you require MFA to log into computers? & \ding{51} \\
Do you require MFA to access sensitive data systems? & \ding{51} \\
Does your organization have an employee acceptable use policy? & \ding{51} \\
Does your organization do security awareness training for new employees? & \textcolor{red}{\ding{55}} \\
Does your organization do security awareness training for all employees at least once per year? & \textcolor{red}{\ding{55}} \\
\bottomrule
\end{tabular}
\end{table}

\subsection*{Analysis}
The two "No" responses (\textcolor{red}{\ding{55}}) are significant findings. The lack of a structured security awareness training program for new hires and the absence of annual refresher training for all staff create a substantial vulnerability. Employees are the first line of defense, and without proper training, they are more susceptible to phishing, malware, and social engineering attacks, which could bypass technical controls.

% --- 4. Technical Scan Results ---
\section{Technical Scan Results}
An external network scan was performed against the target IP address to identify open ports and exposed services. The scan was conducted on November 22, 2025.

\begin{itemize}
    \item \textbf{Target IP:} \texttt{[Target IP]}
    \item \textbf{Scan Date:} 2025-11-22T10:00:00Z
\end{itemize}

The scan identified one open port, detailed in Table \ref{tab:scanresults}.

\begin{table}[h!]
\centering
\caption{Open Ports and Services}
\label{tab:scanresults}
\begin{tabular}{llll}
\toprule
\textbf{Port} & \textbf{State} & \textbf{Service} & \textbf{Product \& Version} \\
\midrule
443/tcp & open & https & Nginx 1.18.0 \\
\bottomrule
\end{tabular}
\end{table}

\subsection*{Analysis}
The primary finding from the technical scan is the version of the Nginx web server software: \textbf{1.18.0}. This version was released in April 2020 and is now significantly outdated. It is missing numerous security patches and is known to be vulnerable to several publicly disclosed exploits (e.g., related to request smuggling and improper input validation). Running outdated software on a public-facing service constitutes a high-risk vulnerability that could lead to a system compromise, data breach, or denial of service.

% --- 5. Consolidated Risk Assessment ---
\section{Consolidated Risk Assessment}
Based on the analysis of security controls and technical scan data, the following risks have been identified. No pre-existing vulnerabilities were provided for this assessment.

\begin{table}[h!]
\centering
\caption{Identified Risks}
\label{tab:risks}
\begin{tabular}{p{0.1\textwidth} p{0.6\textwidth} p{0.15\textwidth}}
\toprule
\textbf{Risk ID} & \textbf{Description} & \textbf{Severity} \\
\midrule
RISK-001 & The public-facing web server is running an outdated and vulnerable version of Nginx (1.18.0), exposing it to remote exploitation. & \textbf{High} \\
\addlinespace
RISK-002 & The organization lacks a formal security awareness training program for new and existing employees, increasing susceptibility to human-centric attacks like phishing. & \textbf{High} \\
\bottomrule
\end{tabular}
\end{table}

% --- 6. Recommendations ---
\section{Recommendations}
The following actions are recommended to mitigate the identified risks and improve the organization's overall security posture.

\subsection{RISK-001: Upgrade Web Server and Implement Patch Management}
\begin{itemize}
    \item \textbf{Immediate Action:} Plan and execute an upgrade of the Nginx server from version 1.18.0 to the latest stable version recommended by the vendor. This should be done in a test environment before deploying to production to ensure compatibility.
    \item \textbf{Strategic Action:} Develop and implement a formal patch management policy. This policy should mandate regular scanning for vulnerabilities on all external-facing systems and define a clear timeline for applying security patches based on severity.
\end{itemize}

\subsection{RISK-002: Implement a Comprehensive Security Awareness Program}
\begin{itemize}
    \item \textbf{Immediate Action:} Procure and deploy a security awareness training solution. Enroll all current employees in a foundational training module covering key topics such as phishing, password security, and acceptable use.
    \item \textbf{Strategic Action:} Integrate security awareness training into the new employee onboarding process. Furthermore, establish a mandatory annual refresher training program for all staff to ensure their knowledge remains current. This program should be supplemented with periodic phishing simulation campaigns to test and reinforce learning.
\end{itemize}

\end{document}
```