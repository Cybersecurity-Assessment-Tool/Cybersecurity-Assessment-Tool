```latex
\documentclass[12pt]{article}

% Preamble: Required Packages
\usepackage[margin=1in]{geometry}
\usepackage{pifont} % For checkmarks and crosses
\usepackage{booktabs} % For professional tables
\usepackage{hyperref} % For hyperlinks and PDF metadata
\usepackage{url}      % For URL formatting
\usepackage{seqsplit} % For splitting long strings to prevent overflow

% Document Metadata
\hypersetup{
    colorlinks=true,
    linkcolor=black,
    filecolor=magenta,      
    urlcolor=blue,
    pdftitle={Cybersecurity Posture Assessment Report},
    pdfauthor={Cybersecurity Analysis Division},
    pdfsubject={Security Assessment},
    pdfkeywords={Security, Nmap, Risk, Assessment},
    bookmarks=true,
    pdfpagemode=FullScreen,
}

% --- Document Start ---
\begin{document}

% Title Page
\title{
    \vspace{2cm}
    \textbf{Cybersecurity Posture Assessment Report} \\
    \large For: \textbf{[Organization Name]}
    \vspace{1cm}
}
\author{Cybersecurity Analysis Division}
\date{November 22, 2025}
\maketitle
\thispagestyle{empty}

\newpage

% Table of Contents
\tableofcontents
\thispagestyle{empty}

\newpage

% --- Main Content ---
\pagestyle{headings}

\section{Executive Summary}
This report details the findings of a cybersecurity posture assessment conducted for \textbf{[Organization Name]}. The assessment combined an automated network scan, a review of organizational security controls via a questionnaire, and an analysis of pre-existing risks. The scan was performed on November 22, 2025.

The assessment identified several areas of concern that require immediate attention. Key findings include critical gaps in access control policies, specifically the lack of Multi-Factor Authentication (MFA) for sensitive data systems. Furthermore, the organization lacks a formal Acceptable Use Policy (AUP), which is a foundational element of a mature security program.

From a technical perspective, the external scan identified an outdated version of the Nginx web server (\texttt{1.18.0}) exposed to the internet. This version is no longer supported and has multiple known vulnerabilities, presenting a significant risk of compromise.

Overall, the organization's security posture is considered to be at a developing stage. While some essential controls like MFA for email are in place, critical policy and technical vulnerabilities must be remediated to reduce the risk of a security incident. Recommendations for remediation are detailed in Section 6 of this report.

\section{Organizational Information}
The following details were used as the basis for this assessment. As per the provided data, some identifying information has been templated.

\begin{table}[h!]
\centering
\begin{tabular}{@{}ll@{}}
\toprule
\textbf{Attribute} & \textbf{Value} \\
\midrule
Organization Name & \textbf{[Organization Name]} \\
Primary Domain & \texttt{[Domain]} \\
External IP Scanned & \texttt{[Client IP]} \\
Assessment Date & November 22, 2025 \\
\bottomrule
\end{tabular}
\caption{Client and Assessment Details.}
\end{table}

\section{Security Control Review}
The following table summarizes the organization's responses to a security controls questionnaire. A checkmark (\ding{51}) indicates a positive control is in place, while a cross (\ding{55}) indicates a control gap that introduces risk.

\begin{table}[h!]
\centering
\begin{tabular}{@{} p{0.8\textwidth} c @{}}
\toprule
\textbf{Control Question} & \textbf{Response} \\
\midrule
Do you require MFA to access email? & \ding{51} \\
Do you require MFA to log into computers? & \ding{51} \\
\textbf{Do you require MFA to access sensitive data systems?} & \textbf{\ding{55}} \\
\textbf{Does your organization have an employee acceptable use policy?} & \textbf{\ding{55}} \\
Does your organization do security awareness training for new employees? & \ding{51} \\
Does your organization do security awareness training for all employees at least once per year? & \ding{51} \\
\bottomrule
\end{tabular}
\caption{Organizational Security Controls Questionnaire Results.}
\end{table}

\subsection*{Analysis of Control Gaps}
Two critical control gaps were identified:
\begin{itemize}
    \item \textbf{Lack of MFA for Sensitive Data:} Failure to enforce MFA on systems containing sensitive data significantly increases the risk of unauthorized access through compromised credentials. This is considered a critical finding.
    \item \textbf{Missing Acceptable Use Policy (AUP):} An AUP is a foundational policy that sets clear expectations for employees regarding the use of company assets and data. Its absence can lead to inconsistent security practices and insider threats.
\end{itemize}

\section{Technical Scan Results}
An external network scan was performed against the target IP address \texttt{[Target IP]} on November 22, 2025. The scan identified one open port.

\subsection{Open Ports and Services}
The following table details the services exposed to the public internet.

\begin{table}[h!]
\centering
\begin{tabular}{@{} l l l l l @{}}
\toprule
\textbf{Port} & \textbf{State} & \textbf{Service} & \textbf{Product} & \textbf{Version} \\
\midrule
443/tcp & open & https & nginx & 1.18.0 \\
\bottomrule
\end{tabular}
\caption{Discovered Network Services.}
\end{table}

\subsection{Technical Findings Analysis}
The scan revealed that the web server is running \textbf{Nginx version 1.18.0}. This version was released in April 2020 and is now considered outdated and end-of-life. Publicly available databases list multiple vulnerabilities for this version, including Cross-Site Scripting (XSS) and request smuggling vulnerabilities. Running outdated software on internet-facing systems presents a high risk of exploitation by automated and targeted attacks.

\section{Risk Assessment}
This section synthesizes the findings from the security control review and the technical scan. No pre-existing risks were provided for correlation. The following new risks have been identified and prioritized.

\begin{table}[h!]
\centering
\begin{tabular}{@{} l p{0.5\textwidth} l l @{}}
\toprule
\textbf{Risk ID} & \textbf{Description} & \textbf{Severity} & \textbf{Source} \\
\midrule
RISK-001 & Lack of MFA on sensitive data systems allows for credential-based attacks. & \textbf{Critical} & Questionnaire \\
\addlinespace
RISK-002 & Outdated Nginx web server is exposed to the internet, vulnerable to known exploits. & \textbf{High} & Network Scan \\
\addlinespace
RISK-003 & Absence of an Acceptable Use Policy creates ambiguity and increases insider risk. & \textbf{High} & Questionnaire \\
\bottomrule
\end{tabular}
\caption{Summary of Identified Risks.}
\end{table}

\section{Recommendations}
The following actions are recommended to mitigate the identified risks and improve the overall security posture of \textbf{[Organization Name]}.

\subsection{Critical Priority}
\subsubsection{Implement MFA for Sensitive Data (RISK-001)}
\begin{itemize}
    \item \textbf{Action:} Enforce a mandatory Multi-Factor Authentication policy for all user accounts (including administrative and service accounts) that have access to systems storing or processing sensitive information.
    \item \textbf{Impact:} Drastically reduces the risk of unauthorized access via stolen or weak credentials.
    \item \textbf{Timeline:} Immediate (within 30 days).
\end{itemize}

\subsection{High Priority}
\subsubsection{Upgrade Outdated Web Server (RISK-002)}
\begin{itemize}
    \item \textbf{Action:} Plan and execute an upgrade of the Nginx server from version 1.18.0 to the latest stable version recommended by the vendor. The upgrade path should be tested in a non-production environment first.
    \item \textbf{Impact:} Mitigates numerous publicly known vulnerabilities, protecting the server and its data from compromise.
    \item \textbf{Timeline:} High (within 60 days).
\end{itemize}

\subsubsection{Establish an Acceptable Use Policy (RISK-003)}
\begin{itemize}
    \item \textbf{Action:} Develop, approve, and disseminate a formal Acceptable Use Policy (AUP) for all employees and contractors. This policy should clearly define the rules for using company networks, devices, and data.
    \item \textbf{Impact:} Establishes a clear security baseline for user behavior, reduces insider risk, and provides a basis for disciplinary action in case of policy violations.
    \item \textbf{Timeline:} High (within 90 days).
\end{itemize}

\end{document}
```