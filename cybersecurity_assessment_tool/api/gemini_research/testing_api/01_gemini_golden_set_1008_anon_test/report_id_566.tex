```latex
\documentclass[12pt]{article}

% --- PACKAGES ---
\usepackage[margin=1in]{geometry}
\usepackage{pifont}                 % For dingbats (checkmarks, crosses)
\usepackage{booktabs}               % For professional-looking tables
\usepackage{hyperref}               % For clickable links and references
\usepackage{url}                    % For formatting URLs
\usepackage{seqsplit}               % For splitting long strings in \texttt
\usepackage{graphicx}
\usepackage{fancyhdr}
\usepackage{xcolor}
\usepackage[T1]{fontenc}

% --- DOCUMENT SETUP ---
\hypersetup{
    colorlinks=true,
    linkcolor=black,
    filecolor=magenta,
    urlcolor=blue,
    pdftitle={Cybersecurity Posture Assessment Report},
    pdfauthor={Cybersecurity Analysis Division},
}

% --- HEADER & FOOTER ---
\pagestyle{fancy}
\fancyhf{} % Clear all header and footer fields
\lhead{Cybersecurity Assessment Report}
\rhead{\textbf{[Organization Name]}}
\cfoot{Page \thepage}
\renewcommand{\headrulewidth}{0.4pt}
\renewcommand{\footrulewidth}{0.4pt}

% --- DOCUMENT START ---
\begin{document}

% --- TITLE PAGE ---
\begin{titlepage}
    \centering
    \vspace*{1cm}
    \Huge\textbf{Cybersecurity Posture Assessment Report}
    \vspace{1.5cm}
    \large
    \begin{center}
        \includegraphics[width=0.4\textwidth]{https://i.imgur.com/2Y2L7jM.png} % Placeholder shield logo
    \end{center}
    \vspace{1.5cm}
    \textbf{Prepared for:}\\
    \large\textbf{[Organization Name]}\\
    \vspace{2cm}
    \textbf{Date of Report:}\\
    \large\today\\
    \vfill
    \textbf{Generated by:}\\
    \large Cybersecurity Analysis Division
\end{titlepage}

\newpage
\tableofcontents
\newpage

% --- EXECUTIVE SUMMARY ---
\section{Executive Summary}
This report details the findings of a cybersecurity assessment conducted for \textbf{[Organization Name]}. The analysis synthesized data from a network perimeter scan, a security controls questionnaire, and a review of pre-existing risks.

The assessment reveals several \textbf{critical and high-severity risks} that require immediate attention. Key findings include:
\begin{itemize}
    \item \textbf{Critical Network Exposure:} A public-facing FTP server was identified running a dangerously outdated version of \texttt{vsftpd} (2.3.4), which is known to contain a critical backdoor vulnerability (CVE-2011-2523). Furthermore, the server is misconfigured to allow anonymous, unauthenticated access. This represents a direct and immediate threat of system compromise and data breach.
    \item \textbf{Critical Identity and Access Weakness:} Multi-Factor Authentication (MFA) is not enforced for email access. As email is a primary target for attackers, this gap significantly increases the risk of business email compromise, phishing success, and unauthorized access to sensitive communications.
    \item \textbf{High-Risk Policy and Training Gaps:} The organization lacks a formal Acceptable Use Policy (AUP) and does not provide security awareness training to employees. This "human firewall" weakness makes the organization highly susceptible to social engineering attacks and insider threats.
\end{itemize}

These findings, combined with the pre-existing risk of outdated Windows workstations, indicate a reactive security posture. Immediate and decisive action is required to remediate these vulnerabilities and establish a foundational security program. Recommendations are prioritized in Section 6 to address the most critical issues first.

% --- ORGANIZATIONAL INFORMATION ---
\section{Organizational Information}
This section provides the context for the assessment based on the information provided.
\begin{center}
\begin{tabular}{@{}ll}
\toprule
\textbf{Attribute} & \textbf{Value} \\
\midrule
Organization Name    & \textbf{[Organization Name]} \\
Primary Email Domain & \texttt{[Domain]} \\
Client External IP   & \texttt{[Client IP]} \\
Scanned Target IP    & \texttt{[Target IP]} \\
\bottomrule
\end{tabular}
\end{center}

% --- SECURITY CONTROL REVIEW ---
\section{Security Control Review}
The following table summarizes the organization's responses to a security controls questionnaire. "No" answers indicate significant gaps in foundational security practices.

\begin{center}
\begin{tabular}{p{0.7\linewidth} c}
\toprule
\textbf{Control Question} & \textbf{Response} \\
\midrule
Do you require MFA to access email? & \textcolor{red}{\ding{55}} \\
Do you require MFA to log into computers? & \textcolor{green}{\ding{51}} \\
Do you require MFA to access sensitive data systems? & \textcolor{green}{\ding{51}} \\
Does your organization have an employee acceptable use policy? & \textcolor{red}{\ding{55}} \\
Does your organization do security awareness training for new employees? & \textcolor{red}{\ding{55}} \\
Does your organization do security awareness training for all employees at least once per year? & \textcolor{red}{\ding{55}} \\
\bottomrule
\end{tabular}
\end{center}

\subsection*{Analysis of Gaps}
The lack of MFA on email is a critical oversight. The absence of both an Acceptable Use Policy and any form of security awareness training creates an environment where employees are unaware of security best practices and organizational expectations, dramatically increasing the likelihood of human error leading to a security incident.

% --- TECHNICAL SCAN RESULTS ---
\section{Technical Scan Results}
An external network scan was performed on the target IP address \texttt{[Target IP]}. The scan identified one open port with a critically vulnerable service.

\subsection*{Host: \texttt{[Target IP]}}
\begin{itemize}
    \item \textbf{Status:} Host is up and responsive.
    \item \textbf{Open Ports Found:} 1
\end{itemize}

\subsubsection*{Port 21/tcp (FTP)}
\begin{description}
    \item[State:] \textbf{Open}
    \item[Service:] FTP (File Transfer Protocol)
    \item[Product:] vsftpd
    \item[Version:] \textbf{2.3.4}
    \item[Finding:] The Nmap script \texttt{ftp-anon} confirmed that \textbf{Anonymous FTP login is allowed}.
\end{description}

\subsection*{Technical Analysis}
The version of the FTP server software, \textbf{vsftpd 2.3.4}, is extremely old (released in 2011) and contains a well-known, critical backdoor vulnerability (\href{https://nvd.nist.gov/vuln/detail/CVE-2011-2523}{CVE-2011-2523}). An attacker can exploit this vulnerability to gain a command shell on the underlying server, leading to a full system compromise.

Compounding this issue, the server is configured to allow anonymous login. This allows any attacker on the internet to connect, upload, download, or view files without any authentication, posing a severe data leakage and malware delivery risk.

% --- RISK ASSESSMENT ---
\section{Risk Assessment Summary}
This table consolidates all identified risks from the technical scan, control review, and pre-existing risk data. Risks are categorized by severity to guide remediation efforts.

\begin{center}
\begin{tabular}{p{0.25\linewidth} p{0.5\linewidth} l}
\toprule
\textbf{Risk Name} & \textbf{Overview} & \textbf{Severity} \\
\midrule
\textbf{Exposed Vulnerable FTP Server} & A public-facing server is running vsftpd 2.3.4, which has a known remote code execution backdoor. Anonymous login is enabled. & \textbf{Critical} \\
\addlinespace
\textbf{No MFA for Email Access} & The lack of a second authentication factor for email accounts makes them highly susceptible to takeover via credential theft or phishing. & \textbf{Critical} \\
\addlinespace
\textbf{Lack of Security Awareness Training} & Employees are not trained on security topics, making them vulnerable to social engineering and phishing attacks. & \textbf{High} \\
\addlinespace
\textbf{No Acceptable Use Policy (AUP)} & The absence of a formal policy defining acceptable use of company assets leads to inconsistent security practices and potential liability. & \textbf{High} \\
\addlinespace
\textbf{Outdated Windows Policy} & Workstations are running the unsupported Windows 7 operating system, which no longer receives security updates. & \textbf{Medium} \\
\bottomrule
\end{tabular}
\end{center}

% --- RECOMMENDATIONS ---
\section{Recommendations}
The following actionable recommendations are prioritized to address the identified risks, starting with the most critical threats.

\subsection*{Priority 1: Immediate Remediation (Within 24-48 Hours)}
\begin{enumerate}
    \item \textbf{Isolate the FTP Server:} Immediately take the server at \texttt{[Target IP]} offline by blocking port 21 at the firewall. This is the fastest way to mitigate the immediate threat.
    \item \textbf{Decommission or Replace FTP Service:} Conduct an urgent review to determine if the FTP service is business-critical.
        \begin{itemize}
            \item If not needed, permanently decommission the server.
            \item If needed, replace it with a secure alternative like SFTP (SSH File Transfer Protocol). Ensure the new service requires authentication and is fully patched.
        \end{itemize}
    \item \textbf{Enforce MFA on All Email Accounts:} Immediately enable and enforce MFA for all user accounts on the \texttt{[Domain]} email system. This is the single most effective control to prevent email account takeovers.
\end{enumerate}

\subsection*{Priority 2: Short-Term Remediation (Within 30 Days)}
\begin{enumerate}
    \item \textbf{Develop and Implement an Acceptable Use Policy (AUP):} Draft a clear AUP that covers topics such as data handling, password requirements, and appropriate use of company systems. Require all employees to read and formally acknowledge the policy.
    \item \textbf{Deploy Foundational Security Awareness Training:} Enroll all employees in a baseline security awareness training program. The training must cover, at a minimum:
        \begin{itemize}
            \item Phishing and social engineering identification.
            \item Strong password creation and management.
            \item The importance of reporting security incidents.
        \end{itemize}
\end{enumerate}

\subsection*{Priority 3: Long-Term Strategy (Within 6-12 Months)}
\begin{enumerate}
    \item \textbf{Establish a Vulnerability Management Program:} Implement a formal process to regularly scan internal and external assets for vulnerabilities. This program should include processes for tracking, prioritizing, and remediating findings in a timely manner.
    \item \textbf{Execute System Upgrade Plan:} Follow through on the existing recommendation to upgrade all Windows 7 workstations to a modern, supported operating system like Windows 10 or 11. This closes known security holes and ensures ongoing patch availability.
\end{enumerate}

\end{document}
```