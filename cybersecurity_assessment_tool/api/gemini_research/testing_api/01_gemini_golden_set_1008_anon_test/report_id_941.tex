```latex
\documentclass[12pt]{article}

% --- PACKAGES ---
\usepackage[margin=1in]{geometry}
\usepackage{pifont} % For checkmarks and crosses
\usepackage{booktabs} % For professional tables
\usepackage{hyperref} % For links and references
\usepackage{url} % For URL formatting
\usepackage{seqsplit} % To split long monospaced strings
\usepackage{xcolor} % For colored text
\usepackage{graphicx} % For potential logos/images
\usepackage{fancyhdr} % For headers and footers

% --- DOCUMENT METADATA ---
\hypersetup{
    colorlinks=true,
    linkcolor=blue,
    filecolor=magenta,      
    urlcolor=cyan,
    pdftitle={Cybersecurity Posture Assessment Report},
    pdfauthor={Cybersecurity Analysis Division},
    pdfsubject={Security Assessment},
    pdfkeywords={Cybersecurity, Risk, Assessment},
}

% --- CUSTOM COMMANDS & SETTINGS ---
\newcommand{\yes}{\ding{51}} % Green checkmark
\newcommand{\no}{\ding{55}}  % Red X
\definecolor{severitycritical}{HTML}{940000}
\definecolor{severityhigh}{HTML}{D14000}
\definecolor{severitymedium}{HTML}{E09100}

% --- HEADER & FOOTER ---
\pagestyle{fancy}
\fancyhf{}
\lhead{Cybersecurity Assessment Report}
\rhead{\textbf{[Organization Name]}}
\cfoot{\thepage}

% --- START OF DOCUMENT ---
\begin{document}

\title{
    \vspace{2cm}
    \textbf{Cybersecurity Posture Assessment Report} \\
    \large For: \textbf{[Organization Name]}
    \vspace{1cm}
}
\author{Cybersecurity Analysis Division}
\date{\today}
\maketitle
\thispagestyle{empty}

\newpage
\tableofcontents
\newpage

% ==============================================================================
\section{Executive Summary}
% ==============================================================================
This report details the findings of a cybersecurity posture assessment conducted for \textbf{[Organization Name]}. The assessment combined an analysis of organizational security controls, a technical network scan, and a review of pre-existing risks.

The overall security posture is assessed as \textbf{Critical}. This is primarily driven by two key findings:
\begin{itemize}
    \item \textbf{Critical External Exposure:} The network scan confirmed that a Remote Desktop Protocol (RDP) service on port 3389 is directly exposed to the public internet. This represents a significant and immediate threat, as RDP is a common target for ransomware gangs and other malicious actors.
    \item \textbf{Significant Internal Control Gaps:} The security questionnaire revealed critical deficiencies in foundational security practices. There is a lack of multi-factor authentication (MFA) for computer logins, no formal employee acceptable use policy, and a complete absence of security awareness training.
\end{itemize}

These weaknesses, particularly when combined, create a high-risk environment susceptible to compromise through brute-force attacks, credential theft, and social engineering. Immediate remediation of the RDP exposure is paramount, followed by a systematic effort to address the identified policy and training gaps.

% ==============================================================================
\section{Organizational Information}
% ==============================================================================
The following information was used as the basis for this assessment.
\begin{itemize}
    \item \textbf{Organization Name:} \textbf{[Organization Name]}
    \item \textbf{Primary Domain:} \texttt{[Domain]}
    \item \textbf{External IP Scanned:} \seqsplit{\texttt{[Client IP]}}
    \item \textbf{Assessment Date:} \today
\end{itemize}

% ==============================================================================
\section{Security Control Review}
% ==============================================================================
The following table summarizes the organization's responses to a security controls questionnaire. "No" answers indicate significant gaps in the security framework and are highlighted as areas requiring immediate attention.

\begin{table}[h!]
\centering
\caption{Security Controls Questionnaire Analysis}
\begin{tabular}{p{0.6\linewidth} c l}
\toprule
\textbf{Control Question} & \textbf{Response} & \textbf{Assessment} \\
\midrule
Do you require MFA to access email? & \yes & Meets best practice. \\
\addlinespace
Do you require MFA to log into computers? & \no & \textbf{Critical Gap.} Lack of endpoint MFA increases risk from stolen credentials. \\
\addlinespace
Do you require MFA to access sensitive data systems? & \yes & Meets best practice. \\
\addlinespace
Does your organization have an employee acceptable use policy? & \no & \textbf{High Risk.} Absence of policy leads to inconsistent and insecure user behavior. \\
\addlinespace
Does your organization do security awareness training for new employees? & \no & \textbf{High Risk.} New hires are not equipped to identify or report security threats. \\
\addlinespace
Does your organization do security awareness training for all employees at least once per year? & \no & \textbf{High Risk.} The organization is highly vulnerable to phishing and social engineering. \\
\bottomrule
\end{tabular}
\end{table}

% ==============================================================================
\section{Technical Scan Results}
% ==============================================================================
An external network scan was performed on the target IP address to identify open ports and exposed services.

\begin{itemize}
    \item \textbf{Target IP:} \seqsplit{\texttt{[Target IP]}}
    \item \textbf{Scan Date:} \textbf{[Scan Date]}
\end{itemize}

\begin{table}[h!]
\centering
\caption{Open Ports Detected on \seqsplit{\texttt{[Target IP]}}}
\begin{tabular}{l l l p{0.4\linewidth}}
\toprule
\textbf{Port} & \textbf{State} & \textbf{Service} & \textbf{Notes} \\
\midrule
3389/tcp & Open & ms-wbt-server & \textbf{Critical Finding.} This is the standard port for Microsoft Remote Desktop Protocol (RDP). Public exposure is highly discouraged and is a primary vector for ransomware attacks. \\
\bottomrule
\end{tabular}
\end{table}

% ==============================================================================
\section{Consolidated Risk Assessment}
% ==============================================================================
The following table consolidates findings from the technical scan, control review, and pre-existing risk data into a prioritized list.

\begin{table}[h!]
\centering
\caption{Summary of Identified Risks}
\begin{tabular}{p{0.15\linewidth} p{0.55\linewidth} p{0.2\linewidth}}
\toprule
\textbf{Risk Name} & \textbf{Description} & \textbf{Severity} \\
\midrule
\textbf{Public RDP Exposure} & Port 3389 (RDP) is open to the internet, allowing attackers to attempt brute-force or credential-stuffing attacks to gain remote control of the system. This was confirmed by both the scan and existing risk data. & \textcolor{severitycritical}{\textbf{Critical (9.0)}} \\
\addlinespace
\textbf{Lack of Endpoint MFA} & User computers do not require MFA for login. If an attacker compromises a user's password, they can gain full access to that user's machine and potentially move laterally across the network. & \textcolor{severityhigh}{\textbf{High}} \\
\addlinespace
\textbf{No Security Awareness Training} & Employees are not trained to recognize or respond to security threats like phishing, malware, or social engineering, making them a vulnerable entry point for attackers. & \textcolor{severityhigh}{\textbf{High}} \\
\addlinespace
\textbf{No Acceptable Use Policy (AUP)} & The absence of a formal AUP means there are no clear guidelines for employees on the secure use of company assets, data handling, or internet usage. & \textcolor{severitymedium}{\textbf{Medium}} \\
\bottomrule
\end{tabular}
\end{table}

% ==============================================================================
\section{Recommendations}
% ==============================================================================
The following actions are recommended to mitigate the identified risks and improve the overall security posture of \textbf{[Organization Name]}.

\subsection{Remediate Public RDP Exposure (Immediate Priority)}
\begin{itemize}
    \item \textbf{Immediate Action:} Implement a firewall rule to block all inbound traffic to TCP port 3389 on \seqsplit{\texttt{[Target IP]}} from the public internet. Access should be restricted to trusted IP addresses only, if absolutely necessary.
    \item \textbf{Long-Term Solution:} Decommission public RDP access entirely. Implement a Virtual Private Network (VPN) solution with MFA for all remote administrative access.
\end{itemize}

\subsection{Implement Multi-Factor Authentication (MFA)}
\begin{itemize}
    \item \textbf{Immediate Action:} Begin a pilot program to deploy and enforce MFA on all IT administrator and executive workstations.
    \item \textbf{Long-Term Solution:} Procure and deploy an MFA solution for all employee computer logins to prevent unauthorized access via compromised credentials.
\end{itemize}

\subsection{Establish a Security Awareness Program}
\begin{itemize}
    \item \textbf{Immediate Action:} Enroll all current employees in a foundational security awareness training course that covers phishing, password hygiene, and threat reporting.
    \item \textbf{Long-Term Solution:} Develop a formal security program that includes mandatory training for all new hires and annual refresher training for all staff. Supplement this with periodic phishing simulation campaigns.
\end{itemize}

\subsection{Develop and Implement an Acceptable Use Policy (AUP)}
\begin{itemize}
    \item \textbf{Immediate Action:} Draft a formal AUP that clearly defines the rules for using company technology, data, and internet services.
    \item \textbf{Long-Term Solution:} Integrate the AUP into the employee onboarding process, requiring all new hires to read and acknowledge it. Require all existing staff to review and acknowledge the policy annually.
\end{itemize}

\end{document}
```