```latex
\documentclass[12pt]{article}

% Preamble: Required Packages
\usepackage[margin=1in]{geometry}
\usepackage{pifont} % For checkmarks and crosses
\usepackage{booktabs} % For professional tables
\usepackage{hyperref} % For clickable links and TOC
\usepackage{url} % For URL formatting
\usepackage{seqsplit} % For splitting long strings in tt font

% Document Metadata
\title{Cybersecurity Posture Assessment Report}
\author{Cybersecurity Analysis Division}
\date{November 22, 2025}

\hypersetup{
    colorlinks=true,
    linkcolor=black,
    urlcolor=blue,
    pdftitle={Cybersecurity Posture Assessment Report},
    pdfauthor={Cybersecurity Analysis Division},
}

\begin{document}

\maketitle
\thispagestyle{empty}
\newpage
\tableofcontents
\newpage

% --- Section 1: Executive Overview ---
\section{Executive Overview}
This report details the findings of a cybersecurity posture assessment conducted on \textbf{[Organization Name]}. The assessment combined an external network scan, a review of existing risks, and an analysis of organizational security controls based on a questionnaire.

The overall security posture requires immediate attention. Several critical and high-risk gaps were identified. Key findings include:
\begin{itemize}
    \item \textbf{Critical Control Gap:} Multi-Factor Authentication (MFA) is not enforced for email access, exposing the organization to significant risk from phishing and credential compromise.
    \item \textbf{High-Risk Technical Finding:} The external-facing web server at \texttt{[Target IP]} is running an outdated version of Nginx (1.18.0), which is known to have multiple publicly disclosed vulnerabilities.
    \item \textbf{High-Risk Procedural Gap:} New employees do not receive mandatory security awareness training, making them susceptible to social engineering attacks from their first day.
\end{itemize}
This report provides a detailed breakdown of these findings and offers actionable recommendations to mitigate the identified risks and strengthen the organization's overall security posture.

% --- Section 2: Organizational Information ---
\section{Organizational Information}
The following information was used as the basis for this assessment. Where data was not provided, placeholders have been used.

\begin{tabular}{@{}ll}
    \toprule
    \textbf{Attribute} & \textbf{Value} \\
    \midrule
    Organization Name & \textbf{[Organization Name]} \\
    Primary Email Domain & \texttt{[Domain]} \\
    External IP Address Assessed & \texttt{[Client IP]} \\
    Scan Target IP & \texttt{[Target IP]} \\
    Assessment Date & 2025-11-22 \\
    \bottomrule
\end{tabular}

% --- Section 3: Security Control Review ---
\section{Security Control Review}
The following table summarizes the organization's responses to a security controls questionnaire. A green checkmark (\ding{51}) indicates a positive control is in place, while a red cross (\ding{55}) indicates a control gap that introduces risk.

\begin{table}[h!]
\centering
\caption{Security Controls Questionnaire Analysis}
\begin{tabular}{@{}lc@{}}
    \toprule
    \textbf{Control Question} & \textbf{Response} \\
    \midrule
    Do you require MFA to access email? & \ding{55} \\
    Do you require MFA to log into computers? & \ding{51} \\
    Do you require MFA to access sensitive data systems? & \ding{51} \\
    Does your organization have an employee acceptable use policy? & \ding{51} \\
    Does your organization do security awareness training for new employees? & \ding{55} \\
    Does your organization do security awareness training for all employees at least once per year? & \ding{51} \\
    \bottomrule
\end{tabular}
\end{table}

The two "No" responses represent significant weaknesses in the organization's defenses against common attack vectors like phishing and social engineering.

% --- Section 4: Technical Scan Results ---
\section{Technical Scan Results}
An external network scan was performed against the target IP address \texttt{[Target IP]} on 2025-11-22. The following table details the open ports and services discovered.

\begin{table}[h!]
\centering
\caption{Open Ports and Services on \texttt{[Target IP]}}
\begin{tabular}{@{}lllll@{}}
    \toprule
    \textbf{Port} & \textbf{State} & \textbf{Service} & \textbf{Product} & \textbf{Version} \\
    \midrule
    443/tcp & open & https & nginx & 1.18.0 \\
    \bottomrule
\end{tabular}
\end{table}

\subsection{Analysis of Technical Findings}
The scan identified one open port (443/tcp) running an Nginx web server. The detected version, \textbf{1.18.0}, was released in April 2020. This version is significantly outdated and has reached its end-of-life. It is susceptible to numerous publicly disclosed vulnerabilities (CVEs) that could be exploited by attackers to compromise the server, potentially leading to a full system breach. This finding constitutes a high-risk vulnerability.

% --- Section 5: Consolidated Risk Assessment ---
\section{Consolidated Risk Assessment}
This section synthesizes findings from the security control review, technical scan, and any pre-existing risks provided. No pre-existing risks were documented in the input data for this assessment.

\begin{table}[h!]
\centering
\caption{Summary of Identified Risks}
\begin{tabular}{@{}llll@{}}
    \toprule
    \textbf{ID} & \textbf{Risk Description} & \textbf{Source} & \textbf{Severity} \\
    \midrule
    RISK-001 & No MFA for email access. & Questionnaire & \textbf{Critical} \\
    RISK-002 & Outdated Nginx web server. & Network Scan & \textbf{High} \\
    RISK-003 & No security training for new hires. & Questionnaire & \textbf{High} \\
    \bottomrule
\end{tabular}
\end{table}

% --- Section 6: Recommendations ---
\section{Recommendations}
Based on the consolidated risk assessment, the following prioritized actions are recommended to mitigate the identified vulnerabilities and improve the overall security posture.

\subsection{Priority 1: Critical}
\begin{itemize}
    \item \textbf{RISK-001: Enforce MFA for Email Access.}
    \begin{itemize}
        \item \textbf{Action:} Immediately enable and enforce Multi-Factor Authentication (MFA) for all user accounts across the email system (\texttt{[Domain]}).
        \item \textbf{Justification:} This is the single most effective control to prevent account takeovers resulting from phishing and credential theft, which are primary initial access vectors for attackers.
    \end{itemize}
\end{itemize}

\subsection{Priority 2: High}
\begin{itemize}
    \item \textbf{RISK-002: Upgrade Web Server Software.}
    \begin{itemize}
        \item \textbf{Action:} Plan and execute an upgrade of the Nginx server running on \texttt{[Target IP]} from version 1.18.0 to the latest stable version. Establish a regular patch management cycle for all internet-facing systems.
        \item \textbf{Justification:} Patching this server will remediate numerous known vulnerabilities, reducing the external attack surface and preventing potential system compromise.
    \end{itemize}
    \item \textbf{RISK-003: Implement New Hire Security Training.}
    \begin{itemize}
        \item \textbf{Action:} Develop and integrate a mandatory security awareness training module into the new employee onboarding process. This training should cover phishing, acceptable use, password hygiene, and incident reporting.
        \item \textbf{Justification:} New employees are high-value targets for attackers. Providing immediate training reduces the likelihood of them falling victim to social engineering and other common attacks.
    \end{itemize}
\end{itemize}

\end{document}
```