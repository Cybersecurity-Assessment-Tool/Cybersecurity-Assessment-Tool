```latex
\documentclass[12pt]{article}

% Preamble: Required Packages
\usepackage[margin=1in]{geometry}
\usepackage{pifont} % Required for \ding
\usepackage{booktabs} % For professional-looking tables
\usepackage{hyperref} % For clickable links and references
\usepackage{url}      % For formatting URLs
\usepackage{seqsplit} % For splitting long strings to prevent overflow

% Document Settings
\hypersetup{
    colorlinks=true,
    linkcolor=black,
    urlcolor=blue,
    pdftitle={Cybersecurity Posture Assessment Report},
    pdfauthor={Cybersecurity Analyst},
}

\newcommand{\yes}{\ding{51}} % Green checkmark
\newcommand{\no}{\ding{55}}  % Red cross

% --- Document Start ---
\begin{document}

\title{Cybersecurity Posture Assessment Report \\ \large For \textbf{[Organization Name]}}
\author{Cybersecurity Analyst}
\date{\today}
\maketitle

\begin{abstract}
    This report provides a comprehensive analysis of the cybersecurity posture of \textbf{[Organization Name]}. The assessment is based on a synthesis of network scan data, an organizational security controls questionnaire, and a review of pre-existing risk documentation. The findings indicate several critical and high-risk vulnerabilities that require immediate attention. Key issues include a publicly exposed database service running End-of-Life software, significant gaps in identity and access management controls, and a lack of foundational security policies. This document details these findings and provides prioritized, actionable recommendations to mitigate the identified risks.
\end{abstract}

\tableofcontents
\newpage

% --- Section 1: Organizational Information ---
\section{Organizational Information}
This section provides a summary of the organizational details used as a basis for this assessment. Due to the anonymized nature of the input data, placeholders have been used where necessary.

\begin{tabular}{@{}ll}
    \toprule
    \textbf{Attribute} & \textbf{Value} \\
    \midrule
    Organization Name & \textbf{[Organization Name]} \\
    Primary Domain & \texttt{[Domain]} \\
    External IP Scanned & \texttt{[Client IP]} \\
    \bottomrule
\end{tabular}

% --- Section 2: Security Control Review ---
\section{Security Control Review}
The following table summarizes the organization's responses to a security controls questionnaire. These answers highlight the current state of administrative and procedural security measures. Gaps identified here often represent significant organizational risk.

\begin{center}
\begin{tabular}{p{0.75\textwidth}c}
    \toprule
    \textbf{Control Question} & \textbf{Response} \\
    \midrule
    Do you require MFA to access email? & \yes \\
    Do you require MFA to log into computers? & \no \\
    Do you require MFA to access sensitive data systems? & \no \\
    Does your organization have an employee acceptable use policy? & \no \\
    Does your organization do security awareness training for new employees? & \no \\
    Does your organization do security awareness training for all employees at least once per year? & \yes \\
    \bottomrule
\end{tabular}
\end{center}

\subsection*{Analysis of Control Gaps}
The review reveals several critical gaps in security controls:
\begin{itemize}
    \item \textbf{Lack of MFA for Sensitive Systems \& Computers:} The absence of Multi-Factor Authentication (MFA) on computers and, more critically, on sensitive data systems, represents a severe security risk. A compromised password would be sufficient for an attacker to gain unauthorized access.
    \item \textbf{Missing Acceptable Use Policy (AUP):} An AUP is a foundational document that sets clear expectations for employees regarding the use of company assets. Its absence can lead to inconsistent security practices and a weakened legal standing in case of insider misuse.
    \item \textbf{No Security Training for New Employees:} Failing to train new hires on security best practices from day one exposes the organization to unnecessary risk. New employees are often prime targets for social engineering attacks.
\end{itemize}

% --- Section 3: Technical Scan Results ---
\section{Technical Scan Results}
A network scan was performed to identify open ports and exposed services on the organization's external infrastructure.

\begin{itemize}
    \item \textbf{Target IP Address:} \texttt{[Target IP]}
    \item \textbf{Scan Date:} Not provided in scan metadata.
\end{itemize}

\begin{center}
\begin{tabular}{llll}
    \toprule
    \textbf{Port} & \textbf{State} & \textbf{Service} & \textbf{Product \& Version} \\
    \midrule
    3306/tcp & open & mysql & MySQL 5.7.33 \\
    \bottomrule
\end{tabular}
\end{center}

\subsection*{Technical Analysis}
The scan identified a single open port, which presents a critical risk:
\begin{itemize}
    \item \textbf{Exposed MySQL Database (Port 3306):} Exposing a database service directly to the public internet is highly discouraged. It provides a direct target for attackers to attempt brute-force attacks, exploit vulnerabilities, or perform denial-of-service attacks.
    \item \textbf{End-of-Life (EOL) Software:} The identified version, \textbf{MySQL 5.7}, reached its official End of Life in \textbf{October 2023}. This means it no longer receives security patches from the vendor, and any newly discovered vulnerabilities will remain unpatched, leaving the system perpetually at risk.
\end{itemize}
This technical finding directly corroborates the pre-existing risk titled "Database Exposure" and elevates its severity due to the EOL status of the software.

% --- Section 4: Correlated Risk Assessment ---
\section{Correlated Risk Assessment}
This section synthesizes findings from the security control review, technical scan, and pre-existing risk data into a unified risk summary.

\begin{center}
\begin{tabular}{lp{0.6\textwidth}l}
    \toprule
    \textbf{Risk Title} & \textbf{Description} & \textbf{Severity} \\
    \midrule
    Exposed End-of-Life Database & A MySQL 5.7 database is publicly accessible on port 3306. The software is no longer supported with security updates. & \textbf{Critical} \\
    \addlinespace
    Insufficient Access Controls & MFA is not enforced on sensitive data systems, creating a single point of failure (password) for protecting critical assets. & \textbf{Critical} \\
    \addlinespace
    Lack of Foundational Policies & The absence of an Acceptable Use Policy and security training for new hires indicates a weak security culture and governance. & High \\
    \addlinespace
    Workstation Security Gap & The lack of MFA on employee computers increases the risk of lateral movement within the network if credentials are stolen. & High \\
    \bottomrule
\end{tabular}
\end{center}

% --- Section 5: Recommendations ---
\section{Recommendations}
Based on the correlated risk assessment, the following prioritized actions are recommended to improve the organization's security posture.

\subsection{Immediate Actions (To Be Completed in 0-7 Days)}
\begin{enumerate}
    \item \textbf{Restrict Access to Port 3306:} Immediately implement strict firewall rules to block all public access to TCP port 3306. Access should only be permitted from specific, trusted IP addresses. This is the most critical first step to mitigate the database exposure risk.
\end{enumerate}

\subsection{Short-Term Actions (To Be Completed in 1-3 Months)}
\begin{enumerate}
    \item \textbf{Implement MFA:} Deploy and enforce MFA for all users on all sensitive data systems and for logging into company computers. This action directly addresses two major identified risks.
    \item \textbf{Develop and Implement an AUP:} Draft and formally adopt an employee Acceptable Use Policy. Ensure all current employees read and acknowledge the policy.
    \item \textbf{Establish New Hire Security Training:} Create a mandatory security awareness training module for all new employees as part of the onboarding process.
\end{enumerate}

\subsection{Long-Term Strategy (To Be Completed in 3-12 Months)}
\begin{enumerate}
    \item \textbf{Upgrade the MySQL Database:} Plan and execute a migration from the EOL MySQL 5.7 to a currently supported version (e.g., MySQL 8.x) or a managed cloud database service. This will ensure the system receives ongoing security patches.
    \item \textbf{Implement a VPN for Remote Access:} For long-term secure database administration, deploy a Virtual Private Network (VPN). This will eliminate the need for any direct public exposure of the database port, as administrators will connect to the private network first.
\end{enumerate}

% --- Document End ---
\end{document}
```