```latex
\documentclass[12pt]{article}

% Preamble: Required Packages
\usepackage[margin=1in]{geometry}
\usepackage{pifont} % Required for \ding
\usepackage{booktabs} % For professional-looking tables
\usepackage{hyperref} % For clickable links and references
\usepackage{url}      % For formatting URLs
\usepackage{seqsplit} % For breaking long strings in \texttt
\usepackage{xcolor}   % For custom colors
\usepackage{graphicx} % For potential logos or diagrams

% --- Document Setup ---
\hypersetup{
    colorlinks=true,
    linkcolor=blue,
    filecolor=magenta,      
    urlcolor=cyan,
    pdftitle={Cybersecurity Posture Assessment Report},
    pdfpagemode=FullScreen,
}

% Define colors for risk levels
\definecolor{criticalrisk}{HTML}{D7263D}
\definecolor{highrisk}{HTML}{F46036}
\definecolor{mediumrisk}{HTML}{F9C80E}
\definecolor{lowrisk}{HTML}{2E7D32}
\definecolor{ok}{HTML}{2E7D32}
\definecolor{bad}{HTML}{D7263D}

% --- Document Start ---
\begin{document}

\title{Cybersecurity Posture Assessment Report \\ \large For: \textbf{[Organization Name]}}
\author{Cybersecurity Analysis Division}
\date{\today}
\maketitle

\begin{abstract}
This report provides a comprehensive analysis of the cybersecurity posture for \textbf{[Organization Name]}. The assessment is based on a synthesis of external network scan data, a review of organizational security controls, and an evaluation of pre-existing risk information. The findings indicate a critical risk due to an exposed administrative service, compounded by significant gaps in foundational security policies and user training. Immediate remediation is strongly advised to mitigate the high probability of a security breach.
\end{abstract}

\tableofcontents
\newpage

% ===================================================================
\section{Executive Summary}
% ===================================================================

A security assessment was conducted to evaluate the external security posture and internal security controls of \textbf{[Organization Name]}. The analysis revealed several areas of significant concern that require immediate attention.

The most critical finding is the public exposure of a Remote Desktop Protocol (RDP) service on port 3389 at the target IP address \texttt{[Target IP]}. This finding directly confirms a known high-severity risk and presents a direct and immediate threat to the organization. RDP is a primary attack vector for ransomware gangs and other malicious actors for gaining initial access to a network.

This technical vulnerability is exacerbated by critical deficiencies in administrative controls identified through the security questionnaire. Key gaps include:
\begin{itemize}
    \item \textbf{Lack of Multi-Factor Authentication (MFA)} for email access, which is a primary target for credential theft.
    \item \textbf{Absence of an Acceptable Use Policy}, leaving employees without clear guidelines on secure behavior.
    \item \textbf{No formal security awareness training program} for new or existing employees, significantly increasing susceptibility to phishing and social engineering attacks.
\end{itemize}

The combination of an exposed, high-value service with weak user-level controls places the organization at an elevated risk of a major security incident, including data breach, financial loss, and operational disruption. This report outlines detailed findings and provides prioritized, actionable recommendations to address these risks.

% ===================================================================
\section{Organizational Information}
% ===================================================================

The following information was used as the basis for this assessment. Placeholder data is used where specific information was not provided.

\begin{table}[h!]
\centering
\begin{tabular}{@{}ll@{}}
\toprule
\textbf{Attribute} & \textbf{Value} \\ \midrule
Organization Name & \textbf{[Organization Name]} \\
Primary Domain & \texttt{[Domain]} \\
External IP Address (Client) & \texttt{[Client IP]} \\
Target IP Address (Scanned) & \texttt{[Target IP]} \\
Scan Date & 2023-10-27 (Assumed from context) \\ \bottomrule
\end{tabular}
\caption{Assessment Subject Information.}
\end{table}

% ===================================================================
\section{Security Control Review}
% ===================================================================

A review of the organization's security controls was conducted via a questionnaire. The results highlight significant gaps in foundational security practices. A "No" response indicates a missing control and a potential area of high risk.

\begin{table}[h!]
\centering
\begin{tabular}{@{}p{0.65\textwidth} c c@{}}
\toprule
\textbf{Control Question} & \textbf{Response} & \textbf{Status} \\ \midrule
Do you require MFA to access email? & No & \textcolor{bad}{\ding{55}} \\
Do you require MFA to log into computers? & Yes & \textcolor{ok}{\ding{51}} \\
Do you require MFA to access sensitive data systems? & Yes & \textcolor{ok}{\ding{51}} \\
Does your organization have an employee acceptable use policy? & No & \textcolor{bad}{\ding{55}} \\
Does your organization do security awareness training for new employees? & No & \textcolor{bad}{\ding{55}} \\
Does your organization do security awareness training for all employees at least once per year? & No & \textcolor{bad}{\ding{55}} \\ \bottomrule
\end{tabular}
\caption{Organizational Security Control Status.}
\end{table}

% ===================================================================
\section{Technical Scan Results}
% ===================================================================

An external network scan was performed against the target IP address \texttt{[Target IP]} to identify open ports and exposed services.

\subsection{Summary of Findings}
The scan identified one open port, which is detailed below. The presence of this service exposed to the public internet represents a critical vulnerability.

\begin{table}[h!]
\centering
\begin{tabular}{@{}llll@{}}
\toprule
\textbf{Port} & \textbf{State} & \textbf{Service Name} & \textbf{Description} \\ \midrule
3389/tcp & open & \texttt{ms-wbt-server} & Microsoft Remote Desktop Protocol (RDP) \\ \bottomrule
\end{tabular}
\caption{Open Ports Detected on \texttt{[Target IP]}.}
\end{table}

\subsection{Analysis of Exposed Services}
\textbf{Port 3389 (RDP):} The Remote Desktop Protocol is used for remote administrative access to Windows systems. When exposed directly to the internet, it becomes a prime target for attackers. Malicious actors continuously scan the internet for open RDP ports to exploit via:
\begin{itemize}
    \item \textbf{Brute-force attacks:} Attempting to guess usernames and passwords.
    \item \textbf{Credential stuffing:} Using credentials stolen from other data breaches.
    \item \textbf{Exploitation of vulnerabilities:} Targeting unpatched or outdated RDP clients/servers.
\end{itemize}
Successful exploitation almost always leads to a full system compromise and is a common entry point for ransomware deployment.

% ===================================================================
\section{Correlated Risk Assessment}
% ===================================================================

This section synthesizes the findings from the technical scan, control review, and pre-existing risk data into a prioritized list of security risks.

\begin{table}[h!]
\centering
\begin{tabular}{@{}p{0.25\textwidth}p{0.15\textwidth}p{0.5\textwidth}@{}}
\toprule
\textbf{Risk Title} & \textbf{Severity} & \textbf{Description} \\ \midrule
\textbf{Public RDP Exposure} & \textcolor{criticalrisk}{\textbf{Critical (9.0)}} & The network scan confirms that RDP is exposed on \texttt{[Target IP]}. This is a well-known, high-impact vulnerability that is actively targeted by threat actors for network intrusion and ransomware deployment. \\
\addlinespace
\textbf{Insufficient Access Controls} & \textcolor{highrisk}{\textbf{High}} & The lack of MFA on email makes the organization highly vulnerable to phishing and credential compromise. A compromised email account can be leveraged to gain further access or conduct business email compromise (BEC) fraud. \\
\addlinespace
\textbf{Deficient Security Policies and Training} & \textcolor{highrisk}{\textbf{High}} & The absence of an Acceptable Use Policy and any form of security awareness training leaves the organization's employees—the first line of defense—unprepared to identify and respond to threats like phishing, significantly increasing overall risk. \\ \bottomrule
\end{tabular}
\caption{Summary of Identified Security Risks.}
\end{table}

% ===================================================================
\section{Recommendations}
% ===================================================================

The following prioritized recommendations are provided to address the identified risks and improve the overall security posture of \textbf{[Organization Name]}.

\subsection{Immediate Actions (Priority 1: Remediate within 72 hours)}
\begin{enumerate}
    \item \textbf{Remediate RDP Exposure:} Immediately close port 3389 on the external firewall for \texttt{[Target IP]}. If remote access is required, it \textbf{must} be placed behind a Virtual Private Network (VPN) with strong authentication (MFA). At a minimum, restrict access to a whitelist of trusted source IP addresses.
\end{enumerate}

\subsection{Short-Term Actions (Priority 2: Remediate within 30 days)}
\begin{enumerate}
    \setcounter{enumi}{1}
    \item \textbf{Implement MFA for Email:} Enforce mandatory MFA for all user access to the email system. This is one of the most effective controls for preventing account takeovers.
    \item \textbf{Deploy a VPN Solution:} If not already in place, deploy a reputable VPN solution to provide secure, encrypted remote access for all employees and administrators. Publicly exposing services like RDP is not an acceptable practice.
\end{enumerate}

\subsection{Long-Term Actions (Priority 3: Implement within 90 days)}
\begin{enumerate}
    \setcounter{enumi}{3}
    \item \textbf{Develop an Acceptable Use Policy (AUP):} Create and distribute a formal AUP that clearly defines the rules for using company assets, data, and networks. All employees must read and acknowledge this policy.
    \item \textbf{Establish a Security Awareness Training Program:} Implement a mandatory security training program for all employees. This should include initial training for new hires and at least one annual refresher course. Topics must cover phishing, password security, and social engineering.
\end{enumerate}

% --- Document End ---
\end{document}
```