```latex
\documentclass[12pt, a4paper]{article}

% Preamble: Required Packages
\usepackage[margin=1in]{geometry} % Sets page margins
\usepackage{pifont}               % For checkmarks and crosses (\ding)
\usepackage{booktabs}             % For professional-looking tables
\usepackage{hyperref}             % For hyperlinks and metadata
\usepackage{url}                  % For formatting URLs
\usepackage{seqsplit}             % For splitting long strings without spaces
\usepackage[T1]{fontenc}          % Font encoding
\usepackage{xcolor}               % For custom colors

% --- Document Metadata ---
\hypersetup{
    colorlinks=true,
    linkcolor=blue,
    filecolor=magenta,      
    urlcolor=cyan,
    pdftitle={Cybersecurity Posture Assessment Report},
    pdfauthor={Cybersecurity Analyst},
    pdfsubject={Security Analysis},
    pdfkeywords={Cybersecurity, Risk Assessment, Nmap, Policy},
    bookmarks=true,
    pdfpagemode=FullScreen,
}

% --- Custom Commands ---
\newcommand{\yes}{\ding{51}} % Green checkmark
\newcommand{\no}{\ding{55}}  % Red X

% --- Document Start ---
\begin{document}

% --- Title Page ---
\begin{titlepage}
    \centering
    \vspace*{2cm}
    \Huge \textbf{Cybersecurity Posture Assessment Report}
    \vspace{1.5cm}
    \Large \textbf{Prepared For:} \\
    \vspace{0.5cm}
    \Huge \textbf{[Organization Name]}
    \vspace{2cm}
    \Large \textbf{Date of Report:} \\
    \vspace{0.5cm}
    \Large \today
    \vfill
    \normalsize This report contains sensitive information and is intended solely for the designated recipient. Unauthorized distribution is strictly prohibited.
\end{titlepage}

\tableofcontents
\newpage

% --- Executive Summary ---
\section*{Executive Summary}
This report provides a comprehensive assessment of the cybersecurity posture for \textbf{[Organization Name]}, based on an analysis of organizational security controls, an external network scan, and a review of pre-existing risks.

The assessment identified several \textbf{critical} and \textbf{high-severity} risks related to organizational policy and access control. The complete absence of Multi-Factor Authentication (MFA) for email, computer logins, and sensitive systems represents a significant vulnerability. This exposes the organization to a high risk of account compromise and unauthorized access. Furthermore, the lack of a formal employee acceptable use policy creates ambiguity regarding security responsibilities and acceptable behavior.

On a positive note, the external network scan of the provided IP address revealed no open ports, suggesting a properly configured firewall at the network perimeter. The organization also demonstrates a commitment to security awareness training for all employees.

Immediate action is required to address the identified MFA and policy gaps. Prioritized, actionable recommendations are detailed in the final section of this report to guide risk mitigation efforts.

% --- Organizational Information ---
\section{Organizational Information}
This section details the information provided by the client for this assessment.
\begin{center}
\begin{tabular}{ll}
\toprule
\textbf{Attribute} & \textbf{Value} \\
\midrule
Organization Name & \textbf{[Organization Name]} \\
Email Domain & \texttt{[Domain]} \\
External IP Address Scanned & \texttt{[Client IP]} \\
\bottomrule
\end{tabular}
\end{center}

% --- Security Control Review ---
\section{Security Control Review}
The following table summarizes the organization's responses to a security controls questionnaire. "No" answers indicate potential gaps in the security framework and are highlighted for review.

\begin{center}
\begin{tabular}{p{0.6\linewidth} c l}
\toprule
\textbf{Control Question} & \textbf{Response} & \textbf{Status} \\
\midrule
Do you require MFA to access email? & \no & \textcolor{red}{\textbf{Critical Gap}} \\
Do you require MFA to log into computers? & \no & \textcolor{red}{\textbf{Critical Gap}} \\
Do you require MFA to access sensitive data systems? & \no & \textcolor{red}{\textbf{Critical Gap}} \\
Does your organization have an employee acceptable use policy? & \no & \textcolor{orange}{\textbf{High Risk}} \\
Does your organization do security awareness training for new employees? & \yes & Implemented \\
Does your organization do security awareness training for all employees at least once per year? & \yes & Implemented \\
\bottomrule
\end{tabular}
\end{center}

% --- Technical Scan Results ---
\section{Technical Scan Results}
An external network scan was performed to identify exposed services and potential vulnerabilities on the organization's public-facing infrastructure.

\subsection{Scan Details}
\begin{itemize}
    \item \textbf{Target IP Address:} \texttt{[Target IP]}
    \item \textbf{Scan Type:} Nmap TCP Port Scan
    \item \textbf{Date of Scan:} Scan data provided on \today
\end{itemize}

\subsection{Findings}
The scan concluded that the target host was online, but reported that all scanned ports were in a \textbf{`closed`} state. No open ports or exposed services were discovered.

\textbf{Interpretation:} This is a positive finding. It indicates that a firewall is likely in place and configured to deny unsolicited inbound traffic, which is a fundamental security best practice. This significantly reduces the external attack surface.

% --- Risk Assessment ---
\section{Risk Assessment}
This section synthesizes findings from the security control review and technical scan to provide a summary of identified risks. No pre-existing vulnerabilities were provided for this assessment.

\begin{center}
\begin{tabular}{p{0.25\linewidth} p{0.15\linewidth} p{0.5\linewidth}}
\toprule
\textbf{Risk Name} & \textbf{Severity} & \textbf{Overview} \\
\midrule
\textbf{Lack of Multi-Factor Authentication (MFA)} & \textcolor{red}{\textbf{Critical}} & The absence of MFA for email, computer, and sensitive system access makes the organization highly susceptible to credential theft and account takeover attacks. A single compromised password could grant an attacker broad access. \\
\addlinespace
\textbf{No Employee Acceptable Use Policy (AUP)} & \textcolor{orange}{\textbf{High}} & Without a formal AUP, there are no established rules for employees regarding the use of company assets, data handling, and security responsibilities. This leads to inconsistent security practices and a lack of enforceability. \\
\bottomrule
\end{tabular}
\end{center}

% --- Recommendations ---
\section{Recommendations}
The following prioritized recommendations are provided to address the identified risks and improve the overall security posture of \textbf{[Organization Name]}.

\subsection{Immediate Priority (Critical)}
\begin{enumerate}
    \item \textbf{Implement Multi-Factor Authentication (MFA) Immediately:}
    \begin{itemize}
        \item \textbf{Action:} Deploy a robust MFA solution across all critical systems.
        \item \textbf{Details:} Prioritize enabling MFA for (1) email access (e.g., Office 365, Google Workspace), (2) remote access solutions (VPNs), (3) privileged/administrator accounts, and (4) all systems containing sensitive data.
        \item \textbf{Impact:} Drastically reduces the risk of unauthorized access from compromised credentials.
    \end{itemize}
\end{enumerate}

\subsection{High Priority}
\begin{enumerate}
    \setcounter{enumi}{1} % Continue numbering from previous list
    \item \textbf{Develop and Enforce an Acceptable Use Policy (AUP):}
    \begin{itemize}
        \item \textbf{Action:} Create a formal AUP that clearly defines the rules and responsibilities for all employees when using company technology and data.
        \item \textbf{Details:} The policy should cover topics such as data privacy, password security, internet usage, and incident reporting. All employees must be required to read and formally acknowledge the policy.
        \item \textbf{Impact:} Establishes a clear security baseline for employee behavior and provides a framework for enforcing security standards.
    \end{itemize}
\end{enumerate}

\subsection{Informational}
\begin{enumerate}
    \setcounter{enumi}{2}
    \item \textbf{Maintain Strong Perimeter Security:}
    \begin{itemize}
        \item \textbf{Action:} Continue to maintain the current firewall configuration that denies all non-essential inbound traffic.
        \item \textbf{Details:} Regularly review firewall rules to ensure they align with business needs and adhere to the principle of least privilege.
        \item \textbf{Impact:} Preserves a minimal external attack surface.
    \end{itemize}
    \item \textbf{Continue Security Awareness Program:}
    \begin{itemize}
        \item \textbf{Action:} Maintain the existing security awareness training program for new and existing employees.
        \item \textbf{Details:} Consider enhancing the program with periodic phishing simulations to test and reinforce employee knowledge.
        \item \textbf{Impact:} Strengthens the human element of the security program, which is a critical defense layer.
    \end{itemize}
\end{enumerate}

\end{document}
```