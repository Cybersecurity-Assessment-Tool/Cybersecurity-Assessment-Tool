```latex
\documentclass[12pt]{article}

% Preamble: Required Packages
\usepackage[margin=1in]{geometry}
\usepackage{pifont} % For checkmarks and crosses
\usepackage{booktabs} % For professional tables
\usepackage{hyperref} % For clickable links
\usepackage{url}      % For URL formatting
\usepackage{seqsplit} % For splitting long strings to prevent overflow
\usepackage{xcolor}   % For colors

% Document Metadata
\title{Cybersecurity Posture Assessment Report}
\author{Cybersecurity Analysis Division}
\date{\today}

% Hyperref Setup
\hypersetup{
    colorlinks=true,
    linkcolor=blue,
    filecolor=magenta,      
    urlcolor=cyan,
    pdftitle={Cybersecurity Posture Assessment Report},
    pdfpagemode=FullScreen,
}

\begin{document}

\maketitle
\thispagestyle{empty}
\newpage

\tableofcontents
\thispagestyle{empty}
\newpage

\setcounter{page}{1}

% ==============================================================================
% SECTION 1: EXECUTIVE SUMMARY
% ==============================================================================
\section{Executive Summary}

This report provides a comprehensive analysis of the cybersecurity posture for \textbf{[Organization Name]}, based on network scan data, a security controls questionnaire, and a review of pre-existing risks. The assessment reveals a \textbf{critical risk posture} that requires immediate attention and remediation.

Key findings indicate a systemic failure to implement fundamental security controls. The complete absence of Multi-Factor Authentication (MFA) across email, computer logins, and sensitive data systems represents a critical vulnerability. This is dangerously compounded by an externally exposed Secure Shell (SSH) service on port 22, creating a direct and high-risk pathway for unauthorized access.

Furthermore, inconsistent security training practices, specifically the lack of training for new employees, perpetuates a high-risk environment. A pre-existing vulnerability, "Localhost Exposed," with a CVSS score of 10.0, underscores the urgency of the situation.

Immediate action is required to implement MFA, secure the exposed SSH service, and address the identified critical vulnerability. A detailed remediation plan is outlined in the Recommendations section of this report.

% ==============================================================================
% SECTION 2: ORGANIZATIONAL INFORMATION
% ==============================================================================
\section{Organizational Information}

The following details were used as the basis for this assessment. Due to the anonymized nature of the provided data, placeholders have been used where necessary.

\begin{itemize}
    \item \textbf{Organization Name:} \textbf{[Organization Name]}
    \item \textbf{Primary Domain:} \texttt{[Domain]}
    \item \textbf{Assessed External IP:} \texttt{[Client IP]}
\end{itemize}

% ==============================================================================
% SECTION 3: SECURITY CONTROL REVIEW
% ==============================================================================
\section{Security Control Review (Questionnaire Analysis)}

An analysis of the organization's security practices was conducted via a questionnaire. The responses highlight significant gaps in foundational security controls. A summary of the findings is presented in Table \ref{tab:controls}. The symbols \ding{51} (Yes) and \ding{55} (No) indicate the provided response.

\begin{table}[h!]
\centering
\caption{Security Controls Questionnaire Results}
\label{tab:controls}
\begin{tabular}{p{0.6\linewidth} c l}
\toprule
\textbf{Control Question} & \textbf{Response} & \textbf{Assessment} \\
\midrule
Do you require MFA to access email? & \ding{55} & \textcolor{red}{\textbf{Critical Gap}} \\
Do you require MFA to log into computers? & \ding{55} & \textcolor{red}{\textbf{Critical Gap}} \\
Do you require MFA to access sensitive data systems? & \ding{55} & \textcolor{red}{\textbf{Critical Gap}} \\
Does your organization have an employee acceptable use policy? & \ding{51} & Control in Place \\
Does your organization do security awareness training for new employees? & \ding{55} & \textcolor{orange}{High Risk} \\
Does your organization do security awareness training for all employees at least once per year? & \ding{51} & Control in Place \\
\bottomrule
\end{tabular}
\end{table}

The lack of MFA across all critical systems is the most severe finding from this review. MFA is a non-negotiable baseline control for preventing account takeover attacks. The failure to train new employees on security best practices from the outset creates a persistent vulnerability within the workforce.

% ==============================================================================
% SECTION 4: TECHNICAL SCAN RESULTS
% ==============================================================================
\section{Technical Scan Results}

A network scan was performed on the target system. The scan date was not provided in the input data.

\begin{itemize}
    \item \textbf{Target IP Address:} \texttt{[Target IP]}
    \item \textbf{Host Status:} Up
\end{itemize}

The scan identified the following open port, detailed in Table \ref{tab:scan}.

\begin{table}[h!]
\centering
\caption{Open Port Analysis}
\label{tab:scan}
\begin{tabular}{c c c c}
\toprule
\textbf{Port} & \textbf{State} & \textbf{Service (Inferred)} & \textbf{Product/Version} \\
\midrule
22 & Open & SSH & Not Provided by Scan \\
\bottomrule
\end{tabular}
\end{table}

\subsection*{Analysis}
The presence of an open SSH port (22) on an external-facing system is a significant finding. SSH is a common vector for brute-force and credential-stuffing attacks. When correlated with the findings from the security control review (specifically, the lack of MFA for computer logins), the risk level of this exposed service is elevated to \textbf{Critical}. An attacker who compromises a user's credentials would face no additional barriers to gaining shell access to this system. The scan's inability to determine the service version is also a concern, as it prevents analysis for known software vulnerabilities.

% ==============================================================================
% SECTION 5: CONSOLIDATED RISK ASSESSMENT
% ==============================================================================
\section{Consolidated Risk Assessment}

This section synthesizes findings from the organizational questionnaire, technical scan, and pre-existing risk data into a unified risk profile. The primary risks are summarized in Table \ref{tab:risks}.

\begin{table}[h!]
\centering
\caption{Summary of Identified Risks}
\label{tab:risks}
\begin{tabular}{p{0.3\linewidth} p{0.5\linewidth} l}
\toprule
\textbf{Risk / Finding} & \textbf{Description} & \textbf{Severity} \\
\midrule
\textbf{Pre-existing Vulnerability: "Localhost Exposed"} & A critical vulnerability with a CVSS score of 10.0 was already documented for the target system. & \textcolor{red}{\textbf{Critical}} \\
\addlinespace
\textbf{Absence of Multi-Factor Authentication (MFA)} & MFA is not enforced for email, computer logins, or access to sensitive data, leaving accounts vulnerable to takeover. & \textcolor{red}{\textbf{Critical}} \\
\addlinespace
\textbf{Exposed SSH Service without MFA} & Port 22 (SSH) is open to the internet. This, combined with the lack of MFA, creates a direct path for system compromise. & \textcolor{red}{\textbf{Critical}} \\
\addlinespace
\textbf{Inadequate Onboarding Security Training} & New employees are not receiving security awareness training, creating a gap in the human firewall from day one. & \textcolor{orange}{\textbf{High}} \\
\bottomrule
\end{tabular}
\end{table}

% ==============================================================================
% SECTION 6: RECOMMENDATIONS
% ==============================================================================
\section{Recommendations}

The following recommendations are prioritized to address the identified risks in a logical and effective manner.

\subsection*{Immediate Actions (Priority 1: Remediate within 72 hours)}
\begin{enumerate}
    \item \textbf{Address Critical Vulnerability:} Immediately investigate and remediate the "Localhost Exposed" vulnerability (CVSS 10.0). This is the highest priority.
    \item \textbf{Secure Exposed SSH Service:}
        \begin{itemize}
            \item If SSH access is not required from the public internet, block port 22 at the network firewall immediately.
            \item If external SSH access is required, restrict access to a whitelist of trusted IP addresses.
            \item Enforce public key authentication and disable password-based authentication for SSH.
        \end{itemize}
\end{enumerate}

\subsection*{High Priority Actions (Priority 2: Remediate within 30 days)}
\begin{enumerate}
    \item \textbf{Deploy MFA Universally:} Implement and enforce MFA for all users across the following services:
        \begin{itemize}
            \item Email (e.g., Office 365, Google Workspace)
            \item VPN and other remote access solutions
            \item All systems containing sensitive data
            \item Workstation and server logins
        \end{itemize}
    \item \textbf{Enhance Security Training:} Integrate mandatory security awareness training into the new employee onboarding process. This training must occur before a new employee is granted access to company systems.
\end{enumerate}

\subsection*{Medium Priority Actions (Priority 3: Remediate within 90 days)}
\begin{enumerate}
    \item \textbf{Conduct Comprehensive Vulnerability Scanning:} Commission a full, credentialed vulnerability scan of the external and internal networks. This will provide detailed information on software versions and specific vulnerabilities that the initial Nmap scan could not.
    \item \textbf{Review and Update Policies:} Review the existing Acceptable Use Policy to ensure it explicitly requires MFA and covers modern threats.
\end{enumerate}

\end{document}
```