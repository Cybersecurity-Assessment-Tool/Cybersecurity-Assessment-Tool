```latex
\documentclass[12pt]{article}

% --- PACKAGE IMPORTS ---
\usepackage[margin=1in]{geometry} % Set page margins
\usepackage{pifont}               % For checkmarks and crosses (dingbats)
\usepackage{booktabs}             % For professional-looking tables
\usepackage{hyperref}             % For hyperlinks
\usepackage{url}                  % For formatting URLs
\usepackage{seqsplit}             % To split long strings without breaking words
\usepackage{graphicx}             % For logos (optional)
\usepackage{xcolor}               % For custom colors

% --- DOCUMENT METADATA ---
\title{Cybersecurity Posture Assessment Report}
\author{Cybersecurity Analysis Division}
\date{\today}

% --- HYPERREF SETUP ---
\hypersetup{
    colorlinks=true,
    linkcolor=blue,
    filecolor=magenta,      
    urlcolor=cyan,
    pdftitle={Cybersecurity Posture Assessment Report},
    pdfpagemode=FullScreen,
}

% --- MAIN DOCUMENT ---
\begin{document}

\maketitle
\hrule
\vspace{1cm}

% =============================================================================
% 1. EXECUTIVE SUMMARY
% =============================================================================
\section*{1. Executive Summary}

This report provides a comprehensive cybersecurity assessment for \textbf{[Organization Name]}. The analysis is based on a correlation of network scan data, a security controls questionnaire, and a review of pre-existing risks.

The assessment reveals a mixed security posture. While the organization has implemented foundational controls such as Multi-Factor Authentication (MFA) for email and computer access, critical gaps were identified. The absence of MFA for sensitive data systems and the lack of a formal Acceptable Use Policy represent significant risks.

Technical scans identified an exposed management service (SSH) on a target system, which could be a vector for unauthorized access. This finding, combined with a pre-existing critical vulnerability, underscores the urgent need for remediation. This report outlines actionable recommendations to address these findings and strengthen the overall security posture.

\vspace{1cm}

% =============================================================================
% 2. ORGANIZATIONAL INFORMATION
% =============================================================================
\section*{2. Organizational Information}

This section details the information provided for the assessment. Due to the anonymized nature of the input data, placeholders have been used where necessary.

\begin{tabular}{@{}ll}
    \toprule
    \textbf{Attribute} & \textbf{Value} \\
    \midrule
    Organization Name & \textbf{[Organization Name]} \\
    Email Domain & \texttt{[Domain]} \\
    External IP Address & \texttt{[Client IP]} \\
    \bottomrule
\end{tabular}

\vspace{1cm}

% =============================================================================
% 3. SECURITY CONTROL REVIEW (QUESTIONNAIRE ANALYSIS)
% =============================================================================
\section*{3. Security Control Review (Questionnaire Analysis)}

The following table summarizes the organization's responses to the security controls questionnaire. "No" answers indicate significant gaps in the security framework and are highlighted as areas for immediate improvement.

\begin{tabular}{@{}p{0.6\linewidth} c p{0.25\linewidth}@{}}
    \toprule
    \textbf{Control Question} & \textbf{Status} & \textbf{Analyst Note} \\
    \midrule
    Do you require MFA to access email? & \ding{51} & Good practice. Protects primary communication. \\
    \addlinespace
    Do you require MFA to log into computers? & \ding{51} & Strong control against unauthorized endpoint access. \\
    \addlinespace
    Do you require MFA to access sensitive data systems? & \textcolor{red}{\ding{55}} & \textbf{Critical Gap.} Leaves high-value data vulnerable. \\
    \addlinespace
    Does your organization have an employee acceptable use policy? & \textcolor{red}{\ding{55}} & \textbf{High Risk.} Lack of clear policy creates legal and security risks. \\
    \addlinespace
    Does your organization do security awareness training for new employees? & \ding{51} & Excellent. Establishes a security mindset from day one. \\
    \addlinespace
    Does your organization do security awareness training for all employees at least once per year? & \ding{51} & Strong recurring control to maintain vigilance. \\
    \bottomrule
\end{tabular}

\vspace{1cm}

% =============================================================================
% 4. TECHNICAL SCAN RESULTS
% =============================================================================
\section*{4. Technical Scan Results}

A network scan was conducted to identify externally visible services. The following findings were observed on the target system.

\begin{itemize}
    \item \textbf{Target IP Address:} \texttt{[Target IP]}
    \item \textbf{Host Status:} Up
\end{itemize}

\subsubsection*{Open Ports}
The table below details the open ports discovered during the scan. Exposed services, especially for system management, can be significant vectors for attack.

\begin{tabular}{@{}llll@{}}
    \toprule
    \textbf{Port} & \textbf{State} & \textbf{Service} & \textbf{Analysis} \\
    \midrule
    22/tcp & Open & ssh & The Secure Shell (SSH) service is exposed. This is a common \\
           &        &     & target for brute-force attacks. Product/version information \\
           &        &     & was not available in the provided scan data. \\
    \bottomrule
\end{tabular}

\vspace{1cm}

% =============================================================================
% 5. CONSOLIDATED RISK ASSESSMENT
% =============================================================================
\section*{5. Consolidated Risk Assessment}

This table correlates findings from the questionnaire, technical scans, and pre-existing risk data to provide a unified view of the organization's security risks.

\begin{tabular}{@{}p{0.3\linewidth} p{0.45\linewidth} p{0.15\linewidth}@{}}
    \toprule
    \textbf{Risk Name} & \textbf{Description} & \textbf{Severity} \\
    \midrule
    \textbf{Localhost Exposed} & A pre-existing vulnerability identified with a CVSS score of 10.0. This represents a maximum severity risk that requires immediate attention. & \textbf{Critical} \\
    \addlinespace
    \textbf{No MFA on Sensitive Systems} & Lack of multi-factor authentication for systems holding sensitive data. A compromised password could lead to a major data breach. & \textbf{Critical} \\
    \addlinespace
    \textbf{Exposed SSH Service} & The SSH management port (22/tcp) is open to the public internet, increasing the risk of brute-force attacks and unauthorized access. & High \\
    \addlinespace
    \textbf{No Acceptable Use Policy} & The absence of a formal policy defining the acceptable use of company assets creates ambiguity and increases the risk of insider threats and misuse. & High \\
    \bottomrule
\end{tabular}

\vspace{1cm}

% =============================================================================
% 6. RECOMMENDATIONS
% =============================================================================
\section*{6. Recommendations}

The following actions are recommended to mitigate the identified risks and improve the overall security posture of \textbf{[Organization Name]}.

\begin{enumerate}
    \item \textbf{Remediate Critical Vulnerability (Localhost Exposed):}
        \begin{itemize}
            \item \textbf{Immediate Action:} The pre-existing "Localhost Exposed" risk (CVSS 10.0) must be investigated and remediated as the highest priority. Follow internal procedures or engage a security specialist to address this immediately.
        \end{itemize}
    
    \item \textbf{Implement MFA for Sensitive Systems:}
        \begin{itemize}
            \item \textbf{Immediate Action:} Enforce MFA on all systems classified as containing sensitive or critical data. This is the single most effective control to prevent unauthorized access due to credential compromise.
        \end{itemize}

    \item \textbf{Restrict SSH Access:}
        \begin{itemize}
            \item \textbf{Immediate Action:} Configure firewall rules to restrict access to port 22/tcp on host \texttt{[Target IP]}. Access should only be permitted from a whitelist of trusted IP addresses (e.g., corporate office, VPN gateway). If external access is not required, block it entirely from the internet.
        \end{itemize}

    \item \textbf{Develop and Implement an Acceptable Use Policy (AUP):}
        \begin{itemize}
            \item \textbf{Short-Term Action:} Draft a formal AUP that clearly outlines the rules and responsibilities for all employees when using company technology and data. The policy should be reviewed by legal counsel, approved by management, and communicated to all staff.
        \end{itemize}
\end{enumerate}

\end{document}
```