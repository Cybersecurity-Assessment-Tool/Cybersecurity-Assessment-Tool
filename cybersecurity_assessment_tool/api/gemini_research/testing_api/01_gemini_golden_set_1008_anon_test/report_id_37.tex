```latex
\documentclass[12pt]{article}

% --- PACKAGE IMPORTS ---
\usepackage[margin=1in]{geometry}
\usepackage{pifont} % For checkmarks and crosses
\usepackage{booktabs} % For professional tables
\usepackage{hyperref} % For clickable links
\usepackage{url} % For URL formatting
\usepackage{seqsplit} % For splitting long strings
\usepackage{graphicx} % For logo (optional)
\usepackage{xcolor} % For colors

% --- DOCUMENT METADATA & STYLING ---
\hypersetup{
    colorlinks=true,
    linkcolor=blue,
    filecolor=magenta,      
    urlcolor=cyan,
    pdftitle={Cybersecurity Assessment Report},
    pdfpagemode=FullScreen,
}

\newcommand{\yes}{\ding{51}} % Checkmark
\newcommand{\no}{\ding{55}}  % Cross

% Define severity colors
\definecolor{critical}{HTML}{990000}
\definecolor{high}{HTML}{D14302}
\definecolor{medium}{HTML}{EFAF00}
\definecolor{low}{HTML}{3B8021}

\newcommand{\sevbox}[2]{\colorbox{#1}{\textcolor{white}{\textbf{\phantom{i}#2\phantom{i}}}}}

\begin{document}

% --- TITLE PAGE ---
\begin{titlepage}
    \centering
    \vspace*{2cm}
    
    \Huge \textbf{Cybersecurity Assessment Report}
    \vspace{1.5cm}
    
    \Large Prepared for: \\
    \vspace{0.5cm}
    \textbf{[Organization Name]}
    
    \vspace{2cm}
    
    \large \textbf{Date of Report:} \today \\
    \textbf{Date of Assessment:} November 22, 2025
    
    \vfill
    
    \large \textit{This report contains sensitive information and should be handled with care.}
    
\end{titlepage}

\tableofcontents
\newpage

% --- EXECUTIVE SUMMARY ---
\section{Executive Summary}
This report details the findings of a cybersecurity assessment conducted on November 22, 2025. The assessment combined a review of organizational security controls, an external network scan, and an analysis of pre-existing risks.

The overall security posture of \textbf{[Organization Name]} is considered to be at a \textbf{high risk level}. This is primarily due to several critical deficiencies in fundamental security controls and the presence of a vulnerable, internet-facing service.

Key findings include:
\begin{itemize}
    \item \textbf{Critical Gaps in Access Control:} Multi-Factor Authentication (MFA) is not enforced for email or computer logins, exposing the organization to significant risks of account compromise and unauthorized access.
    \item \textbf{Vulnerable External Service:} The external-facing web server is running an outdated and unsupported version of Nginx (1.18.0), which is known to have multiple security vulnerabilities.
    \item \textbf{Deficient Administrative Controls:} The absence of a formal Acceptable Use Policy and a mandatory annual security awareness training program for all staff weakens the organization's human firewall and overall security culture.
\end{itemize}

Immediate remediation of these issues is strongly recommended to reduce the organization's attack surface and mitigate the risk of a significant security incident. Actionable recommendations are provided in Section \ref{sec:recommendations}.

% --- ORGANIZATIONAL INFORMATION ---
\section{Organizational Information}
This section provides the organizational details used as a basis for this assessment.
\begin{itemize}
    \item \textbf{Organization Name:} \textbf{[Organization Name]}
    \item \textbf{Primary Email Domain:} \texttt{[Domain]}
    \item \textbf{External IP Address Scanned:} \texttt{[Client IP]}
\end{itemize}

% --- SECURITY CONTROL REVIEW ---
\section{Security Control Review}
The following table summarizes the organization's responses to a security controls questionnaire. "No" answers indicate significant gaps in the security framework and are highlighted as areas of concern.

\begin{table}[h!]
\centering
\caption{Security Controls Questionnaire Analysis}
\label{tab:controls}
\begin{tabular}{@{}p{0.6\linewidth} c p{0.2\linewidth}@{}}
\toprule
\textbf{Control Question} & \textbf{Response} & \textbf{Assessment} \\
\midrule
Do you require MFA to access email? & \no & \sevbox{critical}{Critical Gap} \\
Do you require MFA to log into computers? & \no & \sevbox{critical}{Critical Gap} \\
Do you require MFA to access sensitive data systems? & \yes & Best Practice \\
Does your organization have an employee acceptable use policy? & \no & \sevbox{high}{High Risk} \\
Does your organization do security awareness training for new employees? & \yes & Good Practice \\
Does your organization do security awareness training for all employees at least once per year? & \no & \sevbox{high}{High Risk} \\
\bottomrule
\end{tabular}
\end{table}

% --- TECHNICAL SCAN RESULTS ---
\section{Technical Scan Results}
An external network scan was performed to identify open ports and exposed services on the organization's public-facing infrastructure.

\begin{itemize}
    \item \textbf{Target IP Address:} \texttt{[Target IP]}
    \item \textbf{Scan Date:} November 22, 2025
\end{itemize}

\begin{table}[h!]
\centering
\caption{Open Ports and Services Detected}
\label{tab:nmap}
\begin{tabular}{@{}lllll@{}}
\toprule
\textbf{Port} & \textbf{State} & \textbf{Service} & \textbf{Product} & \textbf{Version} \\
\midrule
443/tcp & open & https & nginx & 1.18.0 \\
\bottomrule
\end{tabular}
\end{table}

\subsection*{Analysis of Technical Findings}
The scan identified a single open port, 443 (HTTPS), running an Nginx web server. The detected version, \textbf{Nginx 1.18.0}, is outdated and reached its end-of-life in April 2022. This version is no longer supported with security patches and has several known vulnerabilities (e.g., CVE-2021-23017). Running unsupported software on an internet-facing server presents a high risk of compromise.

% --- RISK ASSESSMENT ---
\section{Risk Assessment}
This section synthesizes findings from the security control review and technical scan into a prioritized list of risks. No pre-existing vulnerabilities were reported.

\begin{table}[h!]
\centering
\caption{Summary of Identified Risks}
\label{tab:risks}
\begin{tabular}{@{}p{0.1\linewidth} p{0.4\linewidth} p{0.25\linewidth} l@{}}
\toprule
\textbf{Risk ID} & \textbf{Risk Description} & \textbf{Affected Asset(s)} & \textbf{Severity} \\
\midrule
RISK-001 & Lack of MFA on critical systems allows for account takeover via stolen credentials. & Email System, Employee Computers, User Accounts & \sevbox{critical}{Critical} \\
\addlinespace
RISK-002 & Outdated and vulnerable web server software is exposed to the internet. & Public Web Server (\texttt{[Target IP]}) & \sevbox{high}{High} \\
\addlinespace
RISK-003 & Inadequate security policies and training increase susceptibility to human-centric attacks. & All Employees, Organizational Data & \sevbox{high}{High} \\
\bottomrule
\end{tabular}
\end{table}

% --- RECOMMENDATIONS ---
\section{Recommendations}
\label{sec:recommendations}
The following actionable recommendations are provided to mitigate the identified risks. They are prioritized based on severity.

\subsection*{RISK-001: Lack of MFA (Critical)}
\begin{itemize}
    \item \textbf{Immediate Action:} Procure and deploy an MFA solution for all user accounts.
    \item \textbf{Policy:} Mandate the use of MFA for accessing all corporate resources, especially email (e.g., O365, Google Workspace) and endpoint logins (e.g., Windows Hello, Duo).
    \item \textbf{Timeline:} This should be treated as the highest priority remediation project.
\end{itemize}

\subsection*{RISK-002: Outdated Web Server (High)}
\begin{itemize}
    \item \textbf{Immediate Action:} Plan and execute an upgrade of the Nginx server from version 1.18.0 to the latest stable release.
    \item \textbf{Process:} Before upgrading the production server, test the new version in a staging environment to ensure application compatibility.
    \item \textbf{Ongoing:} Implement a patch management process to ensure all internet-facing systems are updated in a timely manner.
\end{itemize}

\subsection*{RISK-003: Inadequate Policies and Training (High)}
\begin{itemize}
    \item \textbf{Policy Development:} Draft and formally ratify an Acceptable Use Policy (AUP) that all employees must read and acknowledge.
    \item \textbf{Training Program:} Establish a mandatory, annual security awareness training program for all employees. This program should cover topics such as phishing, password security, and social engineering.
    \item \textbf{Culture:} Foster a culture of security by communicating the importance of these controls to all staff.
\end{itemize}

\end{document}
```