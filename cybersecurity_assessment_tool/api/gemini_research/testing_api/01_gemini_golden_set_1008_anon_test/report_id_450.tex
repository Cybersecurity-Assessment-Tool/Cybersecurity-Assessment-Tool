```latex
\documentclass[12pt]{article}

% Preamble: Required Packages
\usepackage[margin=1in]{geometry}
\usepackage{pifont} % For checkmarks and crosses
\usepackage{booktabs} % For professional tables
\usepackage{hyperref} % For clickable links
\usepackage{url} % For URL formatting
\usepackage{seqsplit} % For splitting long strings
\usepackage{xcolor} % For colors

% Hyperref Setup
\hypersetup{
    colorlinks=true,
    linkcolor=blue,
    filecolor=magenta,      
    urlcolor=cyan,
    pdftitle={Cybersecurity Assessment Report},
    pdfpagemode=FullScreen,
}

% Document Title
\title{Cybersecurity Assessment Report}
\author{Cybersecurity Analyst}
\date{\today}

\begin{document}

\maketitle
\thispagestyle{empty}
\newpage

\tableofcontents
\newpage

% --- 1. Executive Summary ---
\section{Executive Summary}

This report provides a comprehensive cybersecurity assessment for \textbf{[Organization Name]}, based on an analysis of network scan data, organizational security controls, and pre-existing risk documentation. The assessment synthesizes these data points to provide a holistic view of the organization's current security posture.

The key findings indicate critical gaps in administrative controls, despite a positive development in the technical security posture. Specifically, two high-priority risks were identified:
\begin{itemize}
    \item \textbf{Lack of Multi-Factor Authentication (MFA) on Sensitive Systems:} This represents a critical vulnerability, as a single compromised credential could lead to a significant data breach.
    \item \textbf{Absence of Security Training for New Employees:} New hires are a primary target for social engineering attacks. Without initial training, the organization is highly susceptible to phishing and other human-targeted threats.
\end{itemize}

On a positive note, a recent network scan confirms that a previously identified risk, an open and unencrypted web server port (Port 80), has been successfully remediated. This demonstrates proactive risk management.

Recommendations in this report focus on immediate remediation of the identified control gaps to significantly strengthen the organization's defense against common cyber threats.

% --- 2. Organizational Information ---
\section{Organizational Information}

This section details the information provided by the client for this assessment.
\begin{itemize}
    \item \textbf{Organization Name:} \textbf{[Organization Name]}
    \item \textbf{Primary Domain:} \texttt{[Domain]}
    \item \textbf{External IP Scanned:} \texttt{[Client IP]}
\end{itemize}

% --- 3. Security Control Review ---
\section{Security Control Review}

The following table summarizes the organization's responses to a security controls questionnaire. Answers marked with a red cross (\ding{55}) indicate a deviation from security best practices and represent a potential risk.

\begin{table}[h!]
\centering
\caption{Security Controls Questionnaire Analysis}
\label{tab:controls}
\begin{tabular}{p{0.7\linewidth} c c}
\toprule
\textbf{Control Question} & \textbf{Response} & \textbf{Status} \\
\midrule
Do you require MFA to access email? & Yes & \ding{51} \\
Do you require MFA to log into computers? & Yes & \ding{51} \\
\textbf{Do you require MFA to access sensitive data systems?} & \textbf{No} & \textcolor{red}{\ding{55}} \\
Does your organization have an employee acceptable use policy? & Yes & \ding{51} \\
\textbf{Does your organization do security awareness training for new employees?} & \textbf{No} & \textcolor{red}{\ding{55}} \\
Does your organization do security awareness training for all employees at least once per year? & Yes & \ding{51} \\
\bottomrule
\end{tabular}
\end{table}

\subsection{Analysis of Control Gaps}
\begin{itemize}
    \item \textbf{MFA on Sensitive Systems:} The absence of MFA on systems handling sensitive data is a critical oversight. Stolen or weak credentials are a primary vector for data breaches, and MFA is one of the most effective controls to mitigate this threat.
    \item \textbf{New Employee Security Training:} Failing to train new employees on security best practices from day one leaves a significant window of vulnerability. New hires are often eager to be helpful and may be more susceptible to phishing or social engineering attacks before they are fully integrated into the company's security culture.
\end{itemize}

% --- 4. Technical Scan Results ---
\section{Technical Scan Results}

An external network scan was performed to identify open ports and exposed services.
\begin{itemize}
    \item \textbf{Target IP Address:} \texttt{[Target IP]}
    \item \textbf{Scan Date:} Scan date not specified in data.
\end{itemize}

\begin{table}[h!]
\centering
\caption{Nmap Scan Results for \texttt{[Target IP]}}
\label{tab:nmap}
\begin{tabular}{l l l l l}
\toprule
\textbf{Port} & \textbf{Protocol} & \textbf{State} & \textbf{Service} & \textbf{Version} \\
\midrule
80 & tcp & closed & http & N/A \\
\bottomrule
\end{tabular}
\end{table}

\subsection{Technical Findings Analysis}
The scan revealed that the target host \texttt{[Target IP]} is online, but no open ports were discovered. The state of port 80 as `closed` is a strong security practice.

This finding is particularly noteworthy as it directly contradicts a previously identified risk ("Unencrypted Web Server"). This indicates that the organization has successfully remediated the issue, thereby strengthening its external network perimeter.

% --- 5. Consolidated Risk Assessment ---
\section{Consolidated Risk Assessment}

This section correlates findings from the security control review, technical scan, and pre-existing risk data. Risks are prioritized based on their potential impact on the organization.

\begin{table}[h!]
\centering
\caption{Risk Summary}
\label{tab:risks}
\begin{tabular}{p{0.1\linewidth} p{0.3\linewidth} p{0.4\linewidth} l}
\toprule
\textbf{ID} & \textbf{Risk Name} & \textbf{Description} & \textbf{Severity} \\
\midrule
RISK-001 & Lack of MFA on Sensitive Systems & Failure to implement MFA on critical systems exposes sensitive data to unauthorized access via compromised credentials. & \textbf{Critical} \\
\addlinespace
RISK-002 & Inadequate Employee Onboarding Security & New employees are not receiving security awareness training, making them a high-value target for social engineering attacks. & \textbf{High} \\
\addlinespace
RISK-003 & Unencrypted Web Server & \textit{(From previous assessment)} Port 80 was identified as open, exposing the organization to unencrypted traffic interception. & \textcolor{green}{Resolved} \\
\bottomrule
\end{tabular}
\end{table}

% --- 6. Recommendations ---
\section{Recommendations}

The following actions are recommended to address the identified risks.

\subsection{RISK-001: Lack of MFA on Sensitive Systems (Critical)}
\begin{itemize}
    \item \textbf{Immediate Action:} Identify all systems, applications, and databases that store, process, or transmit sensitive organizational or customer data.
    \item \textbf{Remediation:} Enforce Multi-Factor Authentication (MFA) for all user access, especially privileged access, to these identified systems. Prioritize systems with the most critical data.
\end{itemize}

\subsection{RISK-002: Inadequate Onboarding Security (High)}
\begin{itemize}
    \item \textbf{Immediate Action:} Develop a mandatory security awareness training module specifically for the new employee onboarding process.
    \item \textbf{Remediation:} The training should cover, at a minimum: acceptable use policies, phishing and social engineering identification, password hygiene, and procedures for reporting security incidents.
\end{itemize}

\subsection{General Recommendations}
\begin{itemize}
    \item \textbf{Acknowledge Positive Change:} The remediation of the open Port 80 (RISK-003) is commended. This proactive security measure should be recognized.
    \item \textbf{Continuous Monitoring:} Continue to perform regular, automated network scans to ensure no new, unauthorized services are exposed to the internet.
\end{itemize}

% --- 7. Conclusion ---
\section{Conclusion}

While \textbf{[Organization Name]} has demonstrated the ability to remediate technical vulnerabilities, significant gaps in administrative controls currently expose the organization to high-impact security risks. The lack of MFA on sensitive systems and the absence of security training for new hires are critical issues that require immediate attention.

By implementing the recommendations outlined in this report, the organization can substantially improve its resilience to credential theft and social engineering, which are among the most common and damaging cyber-attack vectors today.

\end{document}
```