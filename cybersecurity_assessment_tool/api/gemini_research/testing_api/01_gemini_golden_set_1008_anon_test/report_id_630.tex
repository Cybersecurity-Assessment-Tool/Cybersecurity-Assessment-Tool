```latex
\documentclass[12pt]{article}

% Preamble: Required Packages
\usepackage[margin=1in]{geometry}
\usepackage{pifont} % For checkmarks and crosses
\usepackage{booktabs} % For professional tables
\usepackage{hyperref} % For clickable links
\usepackage{url} % For formatting URLs
\usepackage{seqsplit} % For splitting long strings
\usepackage{xcolor} % For colors

% Document Information
\title{Cybersecurity Posture Assessment Report}
\author{Cybersecurity Analysis Division}
\date{\today}

% Hyperref Setup
\hypersetup{
    colorlinks=true,
    linkcolor=blue,
    filecolor=magenta,      
    urlcolor=cyan,
    pdftitle={Cybersecurity Posture Assessment Report},
    pdfpagemode=FullScreen,
}

\begin{document}

\maketitle
\thispagestyle{empty}
\newpage

\tableofcontents
\newpage

% ======================================================================
% Section 1: Executive Overview
% ======================================================================
\section{Executive Overview}

This report provides a comprehensive cybersecurity assessment for \textbf{[Organization Name]}, synthesizing data from a network vulnerability scan, a security controls questionnaire, and a review of pre-existing risks.

The assessment reveals a mixed security posture. The organization demonstrates strong technical access controls, with Multi-Factor Authentication (MFA) widely implemented across key systems. This significantly reduces the risk of unauthorized access via compromised credentials.

However, critical gaps exist in foundational administrative controls. The absence of an employee Acceptable Use Policy and a formal security awareness training program presents a high risk. These gaps leave the organization vulnerable to insider threats, social engineering, and phishing attacks, which can bypass even strong technical defenses.

On a positive note, a technical scan of the external IP address \texttt{[Client IP]} indicates that a previously identified risk—an unencrypted web server on Port 80—appears to have been remediated, as the port was found to be closed. This proactive or coincidental remediation is a positive security development that should be verified and documented.

Immediate focus should be placed on developing and implementing the missing policies and training programs to address the most significant areas of exposure.

% ======================================================================
% Section 2: Organizational Information
% ======================================================================
\section{Organizational Information}

This section details the organizational information used as the basis for this assessment. Due to the anonymized nature of the provided data, placeholders have been used where necessary.

\begin{tabular}{@{}ll}
\toprule
\textbf{Attribute} & \textbf{Value} \\
\midrule
Organization Name & \textbf{[Organization Name]} \\
Primary Domain & \texttt{[Domain]} \\
External IP Scanned & \texttt{[Client IP]} \\
\bottomrule
\end{tabular}

% ======================================================================
% Section 3: Security Control Review
% ======================================================================
\section{Security Control Review}

The following table summarizes the organization's responses to a security controls questionnaire. These answers provide insight into the current state of implemented policies and procedures. A checkmark (\ding{51}) indicates an affirmative response (control in place), while a cross (\ding{55}) indicates a negative response (control gap).

\begin{table}[h!]
\centering
\begin{tabular}{@{}lc}
\toprule
\textbf{Control Question} & \textbf{Status} \\
\midrule
Do you require MFA to access email? & \textcolor{green}{\ding{51}} \\
Do you require MFA to log into computers? & \textcolor{green}{\ding{51}} \\
Do you require MFA to access sensitive data systems? & \textcolor{green}{\ding{51}} \\
Does your organization have an employee acceptable use policy? & \textcolor{red}{\ding{55}} \\
Does your organization do security awareness training for new employees? & \textcolor{red}{\ding{55}} \\
Does your organization do security awareness training for all employees annually? & \textcolor{red}{\ding{55}} \\
\bottomrule
\end{tabular}
\caption{Security Controls Questionnaire Results}
\end{table}

\subsection{Analysis of Control Gaps}
The questionnaire reveals critical deficiencies in administrative and policy-based controls:
\begin{itemize}
    \item \textbf{No Acceptable Use Policy (AUP):} The lack of an AUP means there are no formally documented rules for how employees should use company assets, data, and networks. This increases the risk of data misuse, unauthorized software installation, and legal liability.
    \item \textbf{No Security Awareness Training:} The complete absence of a security awareness training program for both new and existing employees is a significant vulnerability. Employees are the first line of defense, and without training, they are far more likely to fall victim to phishing, malware, and other social engineering attacks.
\end{itemize}

% ======================================================================
% Section 4: Technical Scan Results
% ======================================================================
\section{Technical Scan Results}

An Nmap scan was conducted against the target IP address to identify open ports and exposed services.

\begin{itemize}
    \item \textbf{Target IP:} \texttt{[Target IP]}
    \item \textbf{Scan Date:} Assumed to be current as of report generation.
\end{itemize}

The scan revealed a very limited attack surface, which is a positive security finding. The status of notable ports is detailed below.

\begin{table}[h!]
\centering
\begin{tabular}{@{}llll}
\toprule
\textbf{Port} & \textbf{Protocol} & \textbf{State} & \textbf{Service/Notes} \\
\midrule
80 & TCP & \textbf{Closed} & HTTP. Port is not listening. \\
\bottomrule
\end{tabular}
\caption{Nmap Scan Port Summary}
\end{table}

\subsection{Analysis of Technical Findings}
The key finding is that Port 80 (HTTP) is \textbf{closed}. This directly contradicts a pre-existing risk entry (see Section 5) that indicated an unencrypted web server was active on this port. This suggests that the vulnerability has been recently remediated. Verification is recommended to ensure this closure was intentional and is a permanent configuration.

% ======================================================================
% Section 5: Consolidated Risk Assessment
% ======================================================================
\section{Consolidated Risk Assessment}

This section correlates findings from the security questionnaire, technical scan, and pre-existing risk data into a unified risk register.

\begin{table}[h!]
\centering
\begin{tabular}{@{}p{0.3\linewidth}p{0.45\linewidth}p{0.15\linewidth}}
\toprule
\textbf{Risk Name} & \textbf{Overview} & \textbf{Severity} \\
\midrule
\textbf{Insufficient Security Awareness Training} & Employees are not trained to recognize or respond to phishing, malware, or social engineering attacks, making them a primary target for threat actors. & \textbf{High} \\
\hline
\textbf{Lack of Acceptable Use Policy} & Without a formal policy, the organization has limited recourse against internal misuse of assets and is exposed to increased insider and legal risks. & \textbf{High} \\
\hline
\textbf{Unencrypted Web Server} & Port 80 was previously identified as open, exposing an unencrypted web service. The latest scan shows this port is now closed. & \textbf{Remediated} \\
\bottomrule
\end{tabular}
\caption{Summary of Identified Risks}
\end{table}

% ======================================================================
% Section 6: Recommendations
% ======================================================================
\section{Recommendations}

Based on the analysis, the following actions are recommended to mitigate the identified risks and improve the overall security posture of \textbf{[Organization Name]}.

\subsection{High Priority}
\begin{enumerate}
    \item \textbf{Develop and Implement an Acceptable Use Policy (AUP):}
    \begin{itemize}
        \item \textbf{Action:} Draft a formal AUP that clearly defines the rules for using company technology, networks, and data. The policy should be reviewed by legal counsel, approved by management, and communicated to all employees.
        \item \textbf{Impact:} Reduces insider risk, establishes a baseline for acceptable behavior, and provides a legal framework for enforcement.
    \end{itemize}
    
    \item \textbf{Establish a Security Awareness Training Program:}
    \begin{itemize}
        \item \textbf{Action:} Procure or develop a security awareness training module covering key topics like phishing, password security, and data handling. This training must be mandatory for all new hires upon onboarding and for all existing staff on an annual basis.
        \item \textbf{Impact:} Creates a "human firewall" by empowering employees to identify and report security threats, drastically reducing the likelihood of a successful phishing or social engineering attack.
    \end{itemize}
\end{enumerate}

\subsection{Verification and Documentation}
\begin{enumerate}
    \setcounter{enumi}{2} % Continue numbering
    \item \textbf{Verify Remediation of Port 80:}
    \begin{itemize}
        \item \textbf{Action:} Confirm with the IT or network team that the closure of Port 80 on \texttt{[Target IP]} was an intentional act to remediate the "Unencrypted Web Server" risk.
        \item \textbf{Impact:} Ensures the risk is formally closed in the risk register and that the secure configuration is maintained.
    \end{itemize}
\end{enumerate}

\end{document}
```