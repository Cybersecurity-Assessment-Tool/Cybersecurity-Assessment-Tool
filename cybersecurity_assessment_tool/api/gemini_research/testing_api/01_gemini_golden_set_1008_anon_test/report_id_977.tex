```latex
\documentclass[12pt]{article}

% Preamble: Required Packages
\usepackage[margin=1in]{geometry}
\usepackage{pifont}                   % For checkmarks (\ding{51}) and crosses (\ding{55})
\usepackage{booktabs}                 % For professional-looking tables
\usepackage{graphicx}                 % For including logos, etc.
\usepackage{fancyhdr}                 % For headers and footers
\usepackage{xcolor}                   % For colors
\usepackage{hyperref}                 % For clickable links and table of contents
\usepackage{url}                      % For formatting URLs
\usepackage{seqsplit}                 % To split long strings without breaking words

% --- Document Setup ---
\hypersetup{
    colorlinks=true,
    linkcolor=blue,
    filecolor=magenta,      
    urlcolor=cyan,
    pdftitle={Cybersecurity Assessment Report},
    pdfpagemode=FullScreen,
}

\pagestyle{fancy}
\fancyhf{}
\lhead{Cybersecurity Assessment Report}
\rhead{\textbf{[Organization Name]}}
\cfoot{\thepage}

% --- Document Start ---
\begin{document}

% --- Title Page ---
\begin{titlepage}
    \centering
    \vspace*{1cm}
    \Huge\textbf{Cybersecurity Assessment Report}
    \vspace{1.5cm}
    \Large
    \textbf{Prepared For:} \\
    \vspace{0.5cm}
    \textbf{[Organization Name]}
    \vspace{2cm}
    \large
    \textbf{Date of Report:} \\
    \today
    \vfill
    \large
    \textbf{Generated By:} \\
    Cybersecurity Analysis Division
\end{titlepage}

\tableofcontents
\newpage

% --- Section 1: Executive Summary ---
\section{Executive Summary}
This report details the findings of a cybersecurity assessment conducted for \textbf{[Organization Name]}. The analysis synthesized data from an external network scan, a security controls questionnaire, and a review of pre-existing risks.

The assessment identified a \textbf{critical risk}: the direct exposure of a Remote Desktop Protocol (RDP) service on port 3389/TCP to the public internet. This finding, identified in the technical scan, directly correlates with a known high-severity risk and presents a significant and immediate threat to the organization. RDP exposure is a primary vector for ransomware attacks and unauthorized access.

Furthermore, the security controls review revealed significant gaps in organizational policy and access control. Specifically, the lack of mandatory Multi-Factor Authentication (MFA) for sensitive data systems and the absence of an employee Acceptable Use Policy represent high-risk deficiencies. These gaps weaken the organization's defense-in-depth posture and increase the potential impact of a security breach.

Immediate remediation is required to address the exposed RDP service. Strategic initiatives to implement comprehensive MFA and establish foundational security policies are strongly recommended to improve the organization's overall security posture.

% --- Section 2: Organizational Information ---
\section{Organizational Information}
This section provides the key identification details for the organization under review. As the provided data was anonymized, placeholders are used.

\begin{itemize}
    \item \textbf{Organization Name:} \textbf{[Organization Name]}
    \item \textbf{Primary Email Domain:} \texttt{[Domain]}
    \item \textbf{External IP Address Scanned:} \texttt{[Client IP]}
\end{itemize}

% --- Section 3: Security Control Review ---
\section{Security Control Review}
The following table summarizes the organization's responses to a security controls questionnaire. "No" answers indicate significant gaps in security posture and are marked with a red cross.

\begin{table}[h!]
\centering
\caption{Security Controls Questionnaire Analysis}
\begin{tabular}{p{0.6\linewidth} c c}
\toprule
\textbf{Control Question} & \textbf{Response} & \textbf{Assessment} \\
\midrule
Do you require MFA to access email? & Yes & \ding{51} \\
Do you require MFA to log into computers? & Yes & \ding{51} \\
Do you require MFA to access sensitive data systems? & No & \textcolor{red}{\ding{55}} \\
Does your organization have an employee acceptable use policy? & No & \textcolor{red}{\ding{55}} \\
Does your organization do security awareness training for new employees? & Yes & \ding{51} \\
Does your organization do security awareness training for all employees at least once per year? & Yes & \ding{51} \\
\bottomrule
\end{tabular}
\end{table}

\subsection*{Analysis of Gaps}
\begin{itemize}
    \item \textbf{MFA for Sensitive Systems:} The absence of MFA on systems containing sensitive data is a critical vulnerability. Should an attacker compromise a user's credentials, they would have direct access to high-value assets without a secondary authentication challenge.
    \item \textbf{Acceptable Use Policy (AUP):} The lack of a formal AUP creates ambiguity regarding employee responsibilities for protecting company assets. An AUP is a foundational governance document that sets clear expectations for behavior and security practices.
\end{itemize}

% --- Section 4: Technical Scan Results ---
\section{Technical Scan Results}
An external network scan was performed to identify open ports and exposed services on the organization's perimeter.

\begin{itemize}
    \item \textbf{Target IP Address:} \texttt{[Target IP]}
    \item \textbf{Scan Tool:} Nmap
\end{itemize}

The scan revealed the following open port:

\begin{table}[h!]
\centering
\caption{Open Port Findings}
\begin{tabular}{l l l l}
\toprule
\textbf{Port/Proto} & \textbf{State} & \textbf{Service} & \textbf{Notes} \\
\midrule
3389/tcp & open & ms-wbt-server & Microsoft Remote Desktop Protocol (RDP) \\
\bottomrule
\end{tabular}
\end{table}

\subsection*{Technical Analysis}
The discovery of port 3389/TCP open to the internet is a critical finding. This port is used for RDP, a protocol that allows for remote administration of Windows systems. Exposing RDP directly is highly discouraged by all security frameworks and vendors, as it is a frequent target for brute-force credential attacks, credential theft, and exploitation of vulnerabilities like BlueKeep (CVE-2019-0708). This finding directly confirms the pre-existing risk detailed in Input 3.

% --- Section 5: Consolidated Risk Assessment ---
\section{Consolidated Risk Assessment}
This section correlates findings from the security questionnaire, the technical scan, and pre-existing risk data into a unified view of the organization's risk profile.

\begin{table}[h!]
\centering
\caption{Summary of Identified Risks}
\begin{tabular}{p{0.35\linewidth} p{0.45\linewidth} l}
\toprule
\textbf{Risk Title} & \textbf{Description} & \textbf{Severity} \\
\midrule
\textbf{Exposed RDP Service} & Port 3389 (RDP) is open to the public internet, allowing attackers to attempt brute-force logins or exploit protocol vulnerabilities. & \textbf{Critical (9.0)} \\
\addlinespace
\textbf{Lack of MFA on Sensitive Systems} & No secondary authentication factor is required to access high-value data, making credential compromise highly impactful. & \textbf{High} \\
\addlinespace
\textbf{Missing Acceptable Use Policy} & Absence of a foundational policy creates an undefined security culture and a lack of enforceable rules for employees. & \textbf{High} \\
\bottomrule
\end{tabular}
\end{table}

% --- Section 6: Recommendations ---
\section{Recommendations}
Based on the consolidated risk assessment, the following actions are recommended. They are prioritized by urgency.

\subsection*{Immediate Actions (Within 24 Hours)}
\begin{enumerate}
    \item \textbf{Close Port 3389 on the External Firewall:} Immediately block all inbound traffic to TCP port 3389 on the network perimeter firewall for IP \texttt{[Target IP]}. This will mitigate the most critical risk of RDP exposure.
\end{enumerate}

\subsection*{Short-Term Actions (Next 30 Days)}
\begin{enumerate}
    \item \textbf{Implement a Secure Remote Access Solution:} Replace direct RDP access with a Virtual Private Network (VPN) or a Zero Trust Network Access (ZTNA) solution. All remote administrative access should require connecting through this secure channel first.
    \item \textbf{Enforce MFA on All Sensitive Systems:} Prioritize the deployment and enforcement of MFA across all applications and systems identified as containing sensitive data.
\end{enumerate}

\subsection*{Long-Term Actions (Next 90 Days)}
\begin{enumerate}
    \item \textbf{Develop and Implement an Acceptable Use Policy (AUP):} Draft, approve, and disseminate an AUP to all employees. Require employees to read and acknowledge the policy as part of their ongoing employment.
    \item \textbf{Conduct a Comprehensive Vulnerability Assessment:} Perform a more in-depth internal and external vulnerability assessment to identify any other potential security weaknesses.
\end{enumerate}

% --- Document End ---
\end{document}
```