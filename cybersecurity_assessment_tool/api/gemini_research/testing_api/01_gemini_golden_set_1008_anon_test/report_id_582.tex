```latex
\documentclass[12pt, a4paper]{article}

% Preamble: Required Packages
\usepackage[margin=1in]{geometry}
\usepackage{pifont} % For checkmarks and crosses
\usepackage{booktabs} % For professional tables
\usepackage{hyperref} % For hyperlinks
\usepackage{url} % For URL formatting
\usepackage{seqsplit} % For splitting long strings
\usepackage{xcolor} % For colors

% Document Information
\title{Cybersecurity Posture Assessment Report}
\author{Cybersecurity Analysis Division}
\date{\today}

% Hyperref Setup
\hypersetup{
    colorlinks=true,
    linkcolor=blue,
    filecolor=magenta,      
    urlcolor=cyan,
    pdftitle={Cybersecurity Posture Assessment Report},
    pdfpagemode=FullScreen,
}

\begin{document}

\maketitle
\hrule
\vspace{1em}

\section*{1. Executive Summary}

This report provides a comprehensive cybersecurity assessment for \textbf{[Organization Name]}, based on an analysis of network scan data, organizational security controls, and pre-existing risk information. The assessment was conducted on \today.

The analysis reveals several critical and high-risk security gaps. Most notably, the lack of Multi-Factor Authentication (MFA) for accessing email and logging into computers presents a significant and immediate risk of account compromise and unauthorized access. Furthermore, the absence of mandatory annual security awareness training for all employees weakens the organization's human firewall, leaving it susceptible to social engineering attacks like phishing.

On a positive note, a technical scan of the external IP address \texttt{[Client IP]} indicates that a previously identified risk, "Unencrypted Web Server" on Port 80, appears to have been remediated, as the port was found to be closed.

This report outlines these findings in detail and provides prioritized, actionable recommendations to mitigate the identified risks and strengthen the overall security posture of \textbf{[Organization Name]}.

\section*{2. Organizational Information}

The following information was used as the basis for this assessment. Due to the anonymized nature of the input data, placeholders have been used where necessary.

\begin{table}[h!]
\centering
\begin{tabular}{@{}ll@{}}
\toprule
\textbf{Attribute} & \textbf{Value} \\ \midrule
Organization Name & \textbf{[Organization Name]} \\
Primary Domain & \texttt{[Domain]} \\
External IP Scanned & \texttt{[Client IP]} \\ \bottomrule
\end{tabular}
\caption{Client Organizational Details}
\end{table}

\section*{3. Security Control Review (Questionnaire)}

An assessment of organizational security policies and procedures was conducted via a questionnaire. The responses indicate several areas requiring immediate attention. A "No" response highlights a significant gap in security controls.

\begin{table}[h!]
\centering
\begin{tabular}{@{}p{0.7\textwidth}c@{}}
\toprule
\textbf{Control Question} & \textbf{Response} \\ \midrule
Do you require MFA to access email? & \textcolor{red}{\ding{55}} \\
Do you require MFA to log into computers? & \textcolor{red}{\ding{55}} \\
Do you require MFA to access sensitive data systems? & \textcolor{green}{\ding{51}} \\
Does your organization have an employee acceptable use policy? & \textcolor{green}{\ding{51}} \\
Does your organization do security awareness training for new employees? & \textcolor{green}{\ding{51}} \\
Does your organization do security awareness training for all employees at least once per year? & \textcolor{red}{\ding{55}} \\ \bottomrule
\end{tabular}
\caption{Security Controls Questionnaire Results (\ding{51}=Yes, \ding{55}=No)}
\end{table}

\section*{4. Technical Scan Results}

A network scan was performed on the target IP address to identify accessible services and potential vulnerabilities.

\subsection*{Scan Target}
\begin{itemize}
    \item \textbf{Target IP:} \texttt{[Target IP]} (Note: Target IP was not provided in scan data; this is a placeholder).
    \item \textbf{Host Status:} Up
\end{itemize}

\subsection*{Port Scan Details}
The scan revealed the following port status. No open ports were detected.

\begin{table}[h!]
\centering
\begin{tabular}{@{}ccc@{}}
\toprule
\textbf{Port} & \textbf{State} & \textbf{Service} \\ \midrule
80/tcp & closed & http \\ \bottomrule
\end{tabular}
\caption{Nmap Scan Port Summary}
\end{table}

\subsection*{Analysis and Correlation}
The current scan shows that port 80 (HTTP) is \textbf{closed}. This finding directly contradicts a pre-existing risk entry titled "Unencrypted Web Server," which stated that "Port 80 is open." This suggests that the previously identified vulnerability has been successfully remediated. It is recommended to verify this remediation internally and formally close the associated risk.

\section*{5. Consolidated Risk Assessment}

The following table synthesizes findings from the security questionnaire, the technical scan, and pre-existing risk data into a consolidated list. Risks are prioritized by severity.

\begin{table}[h!]
\centering
\begin{tabular}{@{}p{0.25\textwidth}p{0.45\textwidth}ll@{}}
\toprule
\textbf{Risk Name} & \textbf{Description} & \textbf{Severity} & \textbf{Source} \\ \midrule
\textbf{No MFA for Email Access} & Lack of MFA on email accounts allows for account takeover with only a compromised password, leading to data breaches and phishing. & \textbf{Critical} & Questionnaire \\
\textbf{No MFA for Endpoint Login} & Lack of MFA on computer logins allows an attacker with stolen credentials to gain full access to an endpoint and the corporate network. & \textbf{Critical} & Questionnaire \\
\textbf{No Annual Security Training} & Without regular training, employees are more likely to fall victim to phishing and other social engineering attacks. & \textbf{High} & Questionnaire \\
Unencrypted Web Server & Port 80 was previously reported as open, exposing the organization to unencrypted web traffic interception. & Medium & Pre-existing Risk (Appears Remediated) \\ \bottomrule
\end{tabular}
\caption{Summary of Identified Risks}
\end{table}

\section*{6. Recommendations}

Based on the consolidated risk assessment, the following actions are recommended to improve the security posture of \textbf{[Organization Name]}. Recommendations are listed in order of priority.

\begin{enumerate}
    \item \textbf{Implement MFA for Email and Endpoints (Priority: Critical):}
    \begin{itemize}
        \item \textbf{Action:} Immediately enable and enforce MFA for all user accounts across the primary email system (e.g., Office 365, Google Workspace) and for all computer/endpoint logins.
        \item \textbf{Justification:} This is the single most effective control to prevent unauthorized access resulting from credential theft.
    \end{itemize}

    \item \textbf{Establish Annual Security Awareness Training (Priority: High):}
    \begin{itemize}
        \item \textbf{Action:} Develop and mandate a security awareness training program for all employees to be completed annually. The program should cover key topics such as phishing, password security, and acceptable use.
        \item \textbf{Justification:} A well-informed workforce is a critical layer of defense against common cyberattacks.
    \end{itemize}

    \item \textbf{Verify and Close Remediated Risk (Priority: Low):}
    \begin{itemize}
        \item \textbf{Action:} Internally confirm that the closure of port 80 was intentional and part of a remediation plan. Update the internal risk register to reflect that the "Unencrypted Web Server" risk is closed.
        \item \textbf{Justification:} Maintaining an accurate risk register is crucial for focusing security efforts on current and relevant threats.
    \end{itemize}
\end{enumerate}

\end{document}
```