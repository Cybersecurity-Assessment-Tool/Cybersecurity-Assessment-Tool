```latex
\documentclass[12pt, a4paper]{article}

% Preamble: Required Packages
\usepackage[margin=1in]{geometry}
\usepackage{pifont} % For checkmarks and crosses
\usepackage{booktabs} % For professional tables
\usepackage{hyperref} % For clickable links
\usepackage{url} % For URL formatting
\usepackage{seqsplit} % For splitting long strings to prevent overflow
\usepackage{xcolor} % For colors

% Document Information
\title{Cybersecurity Posture Assessment Report}
\author{Cybersecurity Analysis Division}
\date{\today}

% Hyperref Setup
\hypersetup{
    colorlinks=true,
    linkcolor=blue,
    filecolor=magenta,      
    urlcolor=cyan,
    pdftitle={Cybersecurity Posture Assessment Report},
    pdfpagemode=FullScreen,
}

\begin{document}

\maketitle
\thispagestyle{empty}
\newpage
\tableofcontents
\thispagestyle{empty}
\newpage
\setcounter{page}{1}

% --- 1. Executive Summary ---
\section{Executive Summary}

This report provides a comprehensive cybersecurity assessment for \textbf{[Organization Name]}, synthesizing data from a network vulnerability scan, a security controls questionnaire, and a review of pre-existing risk documentation.

The assessment revealed a mixed security posture. On the technical front, the external network scan of the designated target IP address did not identify any open ports, which is a positive security finding. This result suggests that a previously identified risk concerning an unencrypted web server may have been remediated.

However, significant and critical gaps were identified in the organization's procedural and access control security layers. The lack of mandatory Multi-Factor Authentication (MFA) for computer and sensitive data system access represents a high-impact vulnerability. Furthermore, the absence of a structured security awareness training program for employees exposes the organization to a high likelihood of social engineering and phishing attacks.

Immediate remediation efforts should focus on implementing robust access controls (MFA) and establishing a comprehensive security awareness training program to mitigate these critical risks.

% --- 2. Organizational Information ---
\section{Organizational Information}

This assessment was conducted for the following entity. The information below is based on the data provided for this analysis.

\begin{itemize}
    \item \textbf{Organization Name:} \textbf{[Organization Name]}
    \item \textbf{Primary Domain:} \texttt{[Domain]}
    \item \textbf{Assessed External IP:} \texttt{[Client IP]}
\end{itemize}

% --- 3. Security Control Review ---
\section{Security Control Review}

The following table summarizes the organization's responses to a security controls questionnaire. "No" answers indicate significant gaps in the security framework and are flagged as areas of concern.

\begin{table}[h!]
\centering
\caption{Security Controls Questionnaire Analysis}
\label{tab:controls}
\begin{tabular}{@{}p{0.6\linewidth} c c@{}}
\toprule
\textbf{Control Question} & \textbf{Response} & \textbf{Assessment} \\
\midrule
Do you require MFA to access email? & \textcolor{green}{\ding{51}} Yes & Best Practice Met \\
\addlinespace
Do you require MFA to log into computers? & \textcolor{red}{\ding{55}} No & \textbf{Critical Gap} \\
\addlinespace
Do you require MFA to access sensitive data systems? & \textcolor{red}{\ding{55}} No & \textbf{Critical Gap} \\
\addlinespace
Does your organization have an employee acceptable use policy? & \textcolor{green}{\ding{51}} Yes & Best Practice Met \\
\addlinespace
Does your organization do security awareness training for new employees? & \textcolor{red}{\ding{55}} No & \textbf{High Risk} \\
\addlinespace
Does your organization do security awareness training for all employees at least once per year? & \textcolor{red}{\ding{55}} No & \textbf{High Risk} \\
\bottomrule
\end{tabular}
\end{table}

The analysis of the security controls highlights critical deficiencies in identity and access management and security awareness. The absence of MFA on endpoints and sensitive systems drastically increases the risk of unauthorized access from compromised credentials. The lack of security training leaves the organization's staff, the "human firewall," vulnerable to common attack vectors.

% --- 4. Technical Scan Results ---
\section{Technical Scan Results}

An external network scan was performed to identify open ports and exposed services on the organization's perimeter.

\begin{itemize}
    \item \textbf{Target IP Address:} \texttt{[Target IP]}
    \item \textbf{Scan Status:} The target host was found to be online ('up').
    \item \textbf{Findings:} The scan completed successfully and found \textbf{no open ports}. Port 80 (HTTP) was explicitly checked and found to be \textbf{closed}.
\end{itemize}

\subsection*{Analysis of Technical Findings}
The results of the network scan are positive. A minimal external attack surface significantly reduces the risk of network-based intrusions. 

Notably, the finding that port 80 is closed contradicts a pre-existing risk documented in Input 3 ("Unencrypted Web Server"). This suggests that the risk has been successfully remediated or was based on outdated information. This is a positive development that should be formally verified and documented.

% --- 5. Consolidated Risk Assessment ---
\section{Consolidated Risk Assessment}

This section correlates findings from the security questionnaire, the technical scan, and pre-existing risk data to provide a unified view of the current risk landscape.

\begin{table}[h!]
\centering
\caption{Summary of Identified Risks}
\label{tab:risks}
\begin{tabular}{@{}p{0.3\linewidth} p{0.5\linewidth} l@{}}
\toprule
\textbf{Risk Name} & \textbf{Description} & \textbf{Severity} \\
\midrule
\textbf{Insufficient Access Control} & MFA is not enforced for computer logins or access to sensitive data systems. A single compromised password could lead to a significant breach. & \textbf{Critical} \\
\addlinespace
\textbf{Lack of Security Awareness} & No security training is provided to new or existing employees, making them highly susceptible to phishing, malware, and social engineering attacks. & \textbf{High} \\
\addlinespace
\textbf{Unencrypted Web Server} & \textit{(From Input 3)} Port 80 was previously reported as open, exposing unencrypted traffic. The current scan shows this port is closed on the target IP. & \textbf{Low / Remediated} \\
\bottomrule
\end{tabular}
\end{table}

% --- 6. Recommendations ---
\section{Recommendations}

Based on the consolidated risk assessment, the following prioritized actions are recommended to improve the organization's cybersecurity posture.

\subsection*{Priority 1: Remediate Critical Risks}
\begin{enumerate}
    \item \textbf{Implement Comprehensive MFA:} Enforce MFA across all employee computer logins and for all applications and systems that store or process sensitive data. This is the single most effective control to prevent unauthorized access.
\end{enumerate}

\subsection*{Priority 2: Address High-Severity Risks}
\begin{enumerate}
    \item \textbf{Establish a Security Awareness Program:}
    \begin{itemize}
        \item Develop a mandatory security awareness training module for all new employees during onboarding.
        \item Implement a recurring, annual security awareness training program for all staff.
        \item Conduct periodic phishing simulations to test and reinforce training.
    \end{itemize}
\end{enumerate}

\subsection*{Priority 3: Validate and Document}
\begin{enumerate}
    \item \textbf{Verify Remediation of Port 80:} Confirm with system administrators that the closure of Port 80 on asset \texttt{[Target IP]} was intentional. Conduct a broader review to ensure no other public-facing systems are exposing unencrypted services, and formally update risk documentation to reflect the remediation.
\end{enumerate}

\end{document}
```