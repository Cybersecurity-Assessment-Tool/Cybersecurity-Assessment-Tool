```latex
\documentclass[12pt, a4paper]{article}

% === PACKAGES ===
\usepackage[margin=1in]{geometry}
\usepackage{pifont} % For checkmarks and crosses
\usepackage{booktabs} % For professional tables
\usepackage{hyperref} % For hyperlinks
\usepackage{url} % For URL formatting
\usepackage{seqsplit} % For splitting long strings in tt font
\usepackage{xcolor} % For colors
\usepackage{graphicx} % For logo (placeholder)
\usepackage{fancyhdr} % For header/footer

% === DOCUMENT SETUP ===
\hypersetup{
    colorlinks=true,
    linkcolor=blue,
    filecolor=magenta,      
    urlcolor=cyan,
    pdftitle={Cybersecurity Assessment Report},
    pdfpagemode=FullScreen,
}

% Define severity colors
\definecolor{sev_critical}{HTML}{940000}
\definecolor{sev_high}{HTML}{D14000}
\definecolor{sev_medium}{HTML}{E8A100}

% === HEADER & FOOTER ===
\pagestyle{fancy}
\fancyhf{} % clear all header and footer fields
\fancyhead[L]{\textbf{Cybersecurity Assessment Report}}
\fancyhead[R]{\textbf{[Organization Name]}}
\fancyfoot[C]{\thepage}
\renewcommand{\headrulewidth}{0.4pt}
\renewcommand{\footrulewidth}{0.4pt}

% === TITLE SECTION ===
\title{
    \vspace{2cm}
    \textbf{Cybersecurity Assessment Report} \\
    \large \textit{Confidential}
    \vspace{1.5cm}
}
\author{Lead Cybersecurity Analyst}
\date{\today}

% === BEGIN DOCUMENT ===
\begin{document}

\begin{titlepage}
    \centering
    \vspace*{1cm}
    
    \Huge{\textbf{Cybersecurity Assessment Report}}
    
    \vspace{1.5cm}
    
    \Large{Prepared for:} \\
    \vspace{0.5cm}
    \textbf{[Organization Name]}
    
    \vspace{2cm}
    
    \large{Date Issued:} \\
    \vspace{0.5cm}
    \today
    
    \vfill
    
    \large{\textit{This document contains sensitive and confidential information. Distribution is restricted to authorized personnel only.}}
    
\end{titlepage}

\newpage
\tableofcontents
\newpage

% ==================================================================
\section{Executive Summary}
% ==================================================================

This report details the findings of a cybersecurity assessment conducted for \textbf{[Organization Name]}. The assessment combined a technical network scan, a review of existing risks, and an analysis of organizational security controls via a questionnaire.

The overall security posture is determined to be at a \textbf{critical risk level}. The assessment uncovered significant, fundamental gaps in security controls that expose the organization to a high likelihood of compromise.

Key findings include:
\begin{itemize}
    \item \textbf{Critical Database Exposure:} A MySQL database server on port 3306 was found to be directly exposed to the public internet. The running version, MySQL 5.7.33, is \textbf{End-of-Life (EOL)} and no longer receives security updates, posing a severe and immediate risk.
    \item \textbf{Complete Lack of Multi-Factor Authentication (MFA):} MFA is not enforced for any system, including email, computer logins, and sensitive data systems. This represents a critical vulnerability to account takeover and credential-based attacks.
    \item \textbf{Absence of Foundational Policies and Training:} The organization lacks a formal Acceptable Use Policy and does not conduct security awareness training. This leaves the organization highly vulnerable to human error, phishing, and social engineering attacks.
\end{itemize}

Immediate and decisive action is required to remediate these vulnerabilities. This report provides prioritized, actionable recommendations to mitigate the identified risks and strengthen the organization's defensive posture.

% ==================================================================
\section{Organizational Information}
% ==================================================================

This section provides the context for the assessment based on the information provided.
\begin{itemize}
    \item \textbf{Organization Name:} \textbf{[Organization Name]}
    \item \textbf{Primary Domain:} \texttt{[Domain]}
    \item \textbf{Asset IP Address Scanned:} \texttt{[Client IP]}
\end{itemize}

% ==================================================================
\section{Security Control Review}
% ==================================================================

The following table summarizes the organization's responses to a security controls questionnaire. The "Analyst Assessment" column highlights the risk associated with each negative response. A \textcolor{red}{\ding{55}} indicates a control is not in place, representing a security gap.

\begin{table}[h!]
\centering
\caption{Security Controls Questionnaire Analysis}
\label{tab:controls}
\begin{tabular}{p{0.5\textwidth} c p{0.3\textwidth}}
\toprule
\textbf{Control Question} & \textbf{Response} & \textbf{Analyst Assessment} \\
\midrule
Do you require MFA to access email? & \textcolor{red}{\ding{55}} & \textbf{Critical Gap.} High risk of business email compromise (BEC) and phishing success. \\
\addlinespace
Do you require MFA to log into computers? & \textcolor{red}{\ding{55}} & \textbf{High Risk.} Allows for lateral movement if credentials are stolen. \\
\addlinespace
Do you require MFA to access sensitive data systems? & \textcolor{red}{\ding{55}} & \textbf{Critical Gap.} Direct path to exfiltration of sensitive data. \\
\addlinespace
Does your organization have an employee acceptable use policy? & \textcolor{red}{\ding{55}} & \textbf{High Risk.} Lack of clear rules for employees creates legal and security liabilities. \\
\addlinespace
Does your organization do security awareness training for new employees? & \textcolor{red}{\ding{55}} & \textbf{Critical Gap.} New staff are not equipped to identify or report threats. \\
\addlinespace
Does your organization do security awareness training for all employees at least once per year? & \textcolor{red}{\ding{55}} & \textbf{Critical Gap.} The primary defense against social engineering is missing. \\
\bottomrule
\end{tabular}
\end{table}

% ==================================================================
\section{Technical Scan Results}
% ==================================================================

An external network scan was performed on the target IP address to identify open ports and exposed services.

\begin{itemize}
    \item \textbf{Scan Target:} \texttt{[Target IP]}
    \item \textbf{Scan Date:} \today
\end{itemize}

\begin{table}[h!]
\centering
\caption{Open Ports and Services Detected}
\label{tab:nmap}
\begin{tabular}{l l l l p{0.3\textwidth}}
\toprule
\textbf{Port} & \textbf{State} & \textbf{Service} & \textbf{Product \& Version} & \textbf{Analyst Notes} \\
\midrule
3306/tcp & Open & mysql & MySQL 5.7.33 & \textbf{Critical Risk.} The service is publicly exposed. The detected version is \textbf{End-of-Life (EOL)} as of October 2023 and is no longer supported with security patches. \\
\bottomrule
\end{tabular}
\end{table}

% ==================================================================
\section{Consolidated Risk Assessment}
% ==================================================================

This table consolidates all identified risks from the questionnaire, technical scan, and pre-existing risk data. Risks are rated based on their potential impact and likelihood of exploitation.

\begin{table}[h!]
\centering
\caption{Summary of Identified Risks}
\label{tab:risks}
\begin{tabular}{p{0.25\textwidth} l p{0.35\textwidth} p{0.2\textwidth}}
\toprule
\textbf{Risk Name} & \textbf{Severity} & \textbf{Overview} & \textbf{Affected Elements} \\
\midrule
\textbf{End-of-Life Database Software} & \textcolor{sev_critical}{\textbf{Critical (9.8)}} & The MySQL service is running version 5.7.33, which is past its End-of-Life date and is vulnerable to numerous unpatched exploits. & MySQL on Port 3306 \\
\addlinespace
\textbf{Lack of Multi-Factor Authentication} & \textcolor{sev_critical}{\textbf{Critical (9.1)}} & MFA is not enforced for email, computer logins, or access to sensitive systems, allowing for trivial account takeovers. & All user accounts and systems \\
\addlinespace
\textbf{Database Exposure} & \textcolor{sev_high}{\textbf{High (7.5)}} & The MySQL database port (3306) is open to the public internet, inviting brute-force attacks and direct exploitation. & Port 3306 on \texttt{[Client IP]} \\
\addlinespace
\textbf{Lack of Security Policies \& Training} & \textcolor{sev_high}{\textbf{High (8.2)}} & The organization lacks a formal Acceptable Use Policy and does not conduct security awareness training, increasing susceptibility to human-centric attacks. & All employees and IT assets \\
\bottomrule
\end{tabular}
\end{table}

% ==================================================================
\section{Recommendations}
% ==================================================================

The following actionable recommendations are prioritized to address the most critical risks first.

\subsection{Priority 1: Immediate Actions (Due within 72 hours)}
\begin{enumerate}
    \item \textbf{Restrict Database Access:} Immediately implement firewall rules to deny all public access to TCP port 3306 on \texttt{[Client IP]}. Access should only be permitted from specific, trusted IP addresses. This directly mitigates the \textit{Database Exposure} risk.
    \item \textbf{Plan Database Upgrade:} Formulate an emergency plan to migrate the MySQL 5.7.33 database to a fully supported version (e.g., MySQL 8.x or a managed cloud equivalent). This is the only way to mitigate the \textit{End-of-Life Software} risk.
\end{enumerate}

\subsection{Priority 2: High-Impact Actions (Due within 30 days)}
\begin{enumerate}
    \setcounter{enumi}{2} % Continue numbering
    \item \textbf{Implement Multi-Factor Authentication (MFA):} Begin a phased rollout of MFA across the organization.
    \begin{itemize}
        \item \textbf{Phase 1 (Week 1):} Enforce MFA for all administrative accounts and access to sensitive data systems.
        \item \textbf{Phase 2 (Week 2):} Enforce MFA for email access (e.g., Office 365, Google Workspace).
        \item \textbf{Phase 3 (Week 4):} Enforce MFA for all employee computer logins.
    \end{itemize}
\end{enumerate}

\subsection{Priority 3: Foundational Improvements (Due within 90 days)}
\begin{enumerate}
    \setcounter{enumi}{3} % Continue numbering
    \item \textbf{Develop an Acceptable Use Policy (AUP):} Draft and ratify a formal AUP that clearly defines the rules for using company technology and data. All employees must read and acknowledge this policy.
    \item \textbf{Deploy Security Awareness Training:} Procure and deploy a security awareness training program. All new hires must complete the training upon onboarding, and all existing staff must complete it annually. This training should focus on identifying phishing, password hygiene, and reporting security incidents.
\end{enumerate}

\end{document}
```