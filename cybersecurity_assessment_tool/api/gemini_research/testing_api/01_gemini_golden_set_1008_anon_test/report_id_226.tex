```latex
\documentclass[12pt]{article}

% Preamble: Required Packages
\usepackage[margin=1in]{geometry}
\usepackage{pifont} % For checkmarks and crosses
\usepackage{booktabs} % For professional tables
\usepackage{hyperref} % For clickable links and metadata
\usepackage{url} % For formatting URLs
\usepackage{seqsplit} % To split long strings in tt font
\usepackage{graphicx}
\usepackage{xcolor}
\usepackage[utf8]{inputenc}

% Document Metadata
\hypersetup{
    colorlinks=true,
    linkcolor=blue,
    filecolor=magenta,      
    urlcolor=cyan,
    pdftitle={Cybersecurity Posture Assessment Report},
    pdfauthor={Cybersecurity Analysis Division},
    pdfsubject={Security Assessment},
    pdfkeywords={Cybersecurity, Risk, Assessment, Scan},
    bookmarks=true
}

% --- Document Start ---
\begin{document}

% Title Page
\title{
    \vspace{2cm}
    \textbf{Cybersecurity Posture Assessment Report} \\
    \large \textit{For: \textbf{[Organization Name]}}
    \vspace{1cm}
}
\author{Cybersecurity Analysis Division}
\date{\today}
\maketitle
\newpage

% Table of Contents
\tableofcontents
\newpage

% --- Section 1: Executive Summary ---
\section{Executive Summary}
This report details the findings of a cybersecurity posture assessment conducted for \textbf{[Organization Name]}. The assessment synthesizes data from a network perimeter scan, a security controls questionnaire, and a review of pre-existing risks.

The primary finding of this assessment is the presence of critical gaps in fundamental administrative and identity-related security controls. Specifically, the lack of mandatory Multi-Factor Authentication (MFA) for email access represents a critical and immediate threat, exposing the organization to significant risks such as Business Email Compromise (BEC) and account takeovers.

Furthermore, the complete absence of a security awareness training program and an employee acceptable use policy indicates a systemic weakness in security governance and culture. These gaps substantially increase the organization's susceptibility to social engineering, phishing attacks, and insider threats.

On a technical level, the external network scan of the designated target did not identify any open ports. This suggests a properly configured firewall is in place, which is a positive security control. However, the strength of this perimeter defense is undermined by the aforementioned administrative control deficiencies.

Immediate remediation efforts should focus on implementing MFA for email and establishing a foundational security awareness program.

% --- Section 2: Organizational Information ---
\section{Organizational Information}
This section provides the high-level details of the organization under review. The information has been anonymized as per the engagement requirements.

\begin{itemize}
    \item \textbf{Organization Name:} \textbf{[Organization Name]}
    \item \textbf{Primary Domain:} \texttt{[Domain]}
    \item \textbf{Assessed External IP:} \texttt{[Client IP]}
\end{itemize}

% --- Section 3: Security Control Review ---
\section{Security Control Review}
The following table summarizes the organization's responses to a security controls questionnaire. A green checkmark (\textcolor{green}{\ding{51}}) indicates a positive control is in place, while a red cross (\textcolor{red}{\ding{55}}) highlights a control gap that introduces risk.

\begin{table}[h!]
\centering
\caption{Security Controls Questionnaire Results}
\begin{tabular}{p{0.8\linewidth} c}
\toprule
\textbf{Control Question} & \textbf{Response} \\
\midrule
Do you require MFA to access email? & \textcolor{red}{\ding{55}} \\
Do you require MFA to log into computers? & \textcolor{green}{\ding{51}} \\
Do you require MFA to access sensitive data systems? & \textcolor{green}{\ding{51}} \\
Does your organization have an employee acceptable use policy? & \textcolor{red}{\ding{55}} \\
Does your organization do security awareness training for new employees? & \textcolor{red}{\ding{55}} \\
Does your organization do security awareness training for all employees at least once per year? & \textcolor{red}{\ding{55}} \\
\bottomrule
\end{tabular}
\end{table}

\subsection*{Analysis of Control Gaps}
The questionnaire reveals several significant control deficiencies:
\begin{itemize}
    \item \textbf{MFA for Email:} The absence of MFA on email is a critical vulnerability. Email accounts are primary targets for attackers seeking to launch phishing campaigns, commit financial fraud, or gain a foothold for lateral movement within the network.
    \item \textbf{Acceptable Use Policy (AUP):} Lacking an AUP means there are no formally documented rules for employees regarding the use of company assets. This creates legal and security ambiguities and makes it difficult to enforce security standards.
    \item \textbf{Security Awareness Training:} The lack of any security training for new or existing employees is a major gap. This leaves the workforce unprepared to identify and report common threats like phishing, malware, and social engineering, making them the weakest link in the organization's defense.
\end{itemize}

% --- Section 4: Technical Scan Results ---
\section{Technical Scan Results}
An external, unauthenticated network scan was performed to identify listening services on the organization's public-facing infrastructure.

\begin{itemize}
    \item \textbf{Target IP Address:} \texttt{[Target IP]}
    \item \textbf{Scan Date:} Scan date not provided in source data.
\end{itemize}

\subsection*{Findings}
\textbf{No open ports were detected on the target system.}

This result is positive, suggesting that a well-configured firewall or security group is in place, effectively blocking unsolicited inbound connection attempts from the internet. This practice, known as "default deny," is a security best practice and significantly reduces the external attack surface. While this is a strong indicator of a secure perimeter, it does not provide visibility into internal vulnerabilities or misconfigurations.

% --- Section 5: Risk Assessment ---
\section{Risk Assessment}
This section correlates the findings from the security control review and technical scan. The risks are prioritized based on their potential impact and likelihood. No pre-existing vulnerabilities were provided for this assessment.

\begin{table}[h!]
\centering
\caption{Identified Risks and Severity}
\begin{tabular}{p{0.15\linewidth} p{0.25\linewidth} p{0.4\linewidth} l}
\toprule
\textbf{Risk ID} & \textbf{Risk Name} & \textbf{Description} & \textbf{Severity} \\
\midrule
RISK-001 & Lack of MFA on Email & The absence of a second authentication factor for email access makes user accounts highly susceptible to compromise via stolen or weak passwords. & \textbf{Critical} \\
\addlinespace
RISK-002 & Inadequate Security Awareness Program & With no training, employees are likely to fall victim to phishing and other social engineering attacks, potentially leading to data breaches or malware infections. & \textbf{High} \\
\addlinespace
RISK-003 & Missing Acceptable Use Policy & The lack of a formal AUP creates ambiguity regarding employee responsibilities for protecting company data and systems, increasing the risk of insider threat and non-compliance. & \textbf{High} \\
\bottomrule
\end{tabular}
\end{table}

% --- Section 6: Recommendations ---
\section{Recommendations}
Based on the analysis, the following actions are recommended to mitigate the identified risks and improve the overall security posture of \textbf{[Organization Name]}.

\subsection*{Immediate Priority (Critical Risk)}
\begin{itemize}
    \item \textbf{RISK-001 Mitigation:} Enforce Multi-Factor Authentication (MFA) across all email accounts immediately. This is the single most effective control to prevent unauthorized account access and should be treated as an urgent priority.
\end{itemize}

\subsection*{High Priority}
\begin{itemize}
    \item \textbf{RISK-002 Mitigation:} Develop and implement a comprehensive Security Awareness Training program. This program should include:
        \begin{itemize}
            \item Mandatory training for all new hires during onboarding.
            \item Annual mandatory training for all existing employees.
            \item Regular phishing simulations to test and reinforce learning.
        \end{itemize}
    \item \textbf{RISK-003 Mitigation:} Draft, approve, and implement a formal employee Acceptable Use Policy (AUP). This policy should be communicated to all employees, and acknowledgement of its review should be tracked.
\end{itemize}

\subsection*{General Recommendations}
\begin{itemize}
    \item \textbf{Continuous Monitoring:} While the external scan was clean, it is crucial to perform regular vulnerability scans, including authenticated internal scans, to identify vulnerabilities that are not visible from the internet.
    \item \textbf{Policy Review:} Conduct a broader review of all IT and security policies to identify other potential gaps in governance.
\end{itemize}

% --- Document End ---
\end{document}
```