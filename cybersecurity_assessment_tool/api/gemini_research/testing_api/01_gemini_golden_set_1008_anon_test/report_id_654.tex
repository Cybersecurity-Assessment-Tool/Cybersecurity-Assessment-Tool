```latex
\documentclass[12pt]{article}

% Preamble: Required Packages
\usepackage[margin=1in]{geometry}
\usepackage{pifont} % For checkmarks and crosses
\usepackage{booktabs} % For professional tables
\usepackage{hyperref} % For clickable links
\usepackage{url} % For formatting URLs
\usepackage{seqsplit} % To split long strings in tt font
\usepackage{xcolor} % For colors

% Document Information
\title{Cybersecurity Posture Assessment Report}
\author{Cybersecurity Analyst}
\date{\today}

% Hyperref Setup
\hypersetup{
    colorlinks=true,
    linkcolor=blue,
    filecolor=magenta,      
    urlcolor=cyan,
    pdftitle={Cybersecurity Posture Assessment Report},
    pdfpagemode=FullScreen,
}

\begin{document}

\maketitle
\thispagestyle{empty}
\newpage
\tableofcontents
\newpage

% --- 1. Executive Summary ---
\section{Executive Summary}
This report provides a comprehensive analysis of the cybersecurity posture for \textbf{[Organization Name]}, based on a review of organizational security controls, technical network scan data, and pre-existing risk documentation. The assessment was conducted to identify key vulnerabilities, policy gaps, and areas for security improvement.

The analysis reveals a mixed security posture. The organization has implemented crucial controls such as Multi-Factor Authentication (MFA) for email and sensitive systems, along with a security awareness training program. These are commendable foundational security measures.

However, two critical gaps were identified that present a high level of risk:
\begin{itemize}
    \item \textbf{Lack of MFA for Computer Logins:} The absence of MFA on endpoint devices significantly increases the risk of unauthorized access via compromised credentials.
    - \textbf{Absence of an Acceptable Use Policy (AUP):} This policy gap leaves the organization without a formal framework to govern employee use of IT assets, creating potential legal and operational risks.
\end{itemize}

On a positive note, a recent network scan indicates that the previously identified risk of an open, unencrypted web server port (Port 80) may have been remediated, as the port was found to be closed. This should be verified and the risk register updated accordingly.

Immediate action is recommended to address the identified high-risk findings to bolster the organization's defenses against common cyber threats.

% --- 2. Organizational Information ---
\section{Organizational Information}
The following details were used as the basis for this assessment. Due to the anonymized nature of the provided data, placeholders have been used where necessary.

\begin{itemize}
    \item \textbf{Organization Name:} \textbf{[Organization Name]}
    \item \textbf{Primary Domain:} \texttt{[Domain]}
    \item \textbf{External IP Scanned:} \texttt{[Client IP]}
\end{itemize}

% --- 3. Security Control Review ---
\section{Security Control Review (Questionnaire Analysis)}
A review of the organization's security controls was conducted via a questionnaire. The responses highlight both strengths and weaknesses in the current security framework. "No" answers indicate significant gaps that require immediate attention.

\begin{table}[h!]
\centering
\caption{Security Controls Questionnaire Results}
\begin{tabular}{p{0.8\linewidth} c}
\toprule
\textbf{Control Question} & \textbf{Response} \\
\midrule
Do you require MFA to access email? & \ding{51} \\ % Yes
Do you require MFA to log into computers? & \textcolor{red}{\ding{55}} \\ % No
Do you require MFA to access sensitive data systems? & \ding{51} \\ % Yes
Does your organization have an employee acceptable use policy? & \textcolor{red}{\ding{55}} \\ % No
Does your organization do security awareness training for new employees? & \ding{51} \\ % Yes
Does your organization do security awareness training for all employees at least once per year? & \ding{51} \\ % Yes
\bottomrule
\end{tabular}
\end{table}

\subsection{Analysis of Gaps}
\begin{itemize}
    \item \textbf{MFA for Computer Logins:} The lack of MFA on workstations and laptops is a critical vulnerability. If an employee's credentials are stolen (e.g., through a phishing attack), an attacker could gain direct access to the network, potentially leading to data theft or a ransomware incident.
    \item \textbf{Acceptable Use Policy (AUP):} An AUP is a foundational governance document. Without it, there are no clear rules for employees regarding the use of company technology, data handling, and internet access. This creates ambiguity and increases the risk of insider threats, whether malicious or accidental.
\end{itemize}

% --- 4. Technical Scan Results ---
\section{Technical Scan Results}
A network scan was performed to identify open ports and exposed services on the organization's external infrastructure.

\begin{itemize}
    \item \textbf{Target IP Address:} \texttt{[Target IP]}
    \item \textbf{Scan Date:} Not Provided in Scan Data
\end{itemize}

The scan results were minimal, with only one port status identified.

\begin{table}[h!]
\centering
\caption{Nmap Scan Port Summary}
\begin{tabular}{llll}
\toprule
\textbf{Port} & \textbf{State} & \textbf{Service} & \textbf{Version} \\
\midrule
80/tcp & closed & http & N/A \\
\bottomrule
\end{tabular}
\end{table}

\subsection{Technical Analysis}
The scan indicates that port 80 (HTTP) is \textbf{closed}. This finding directly contradicts a pre-existing risk documented in Input 3 ("Unencrypted Web Server," which assumes Port 80 is open). This suggests one of two possibilities:
\begin{enumerate}
    \item The previously identified risk has been successfully remediated.
    \item The scan was incomplete, blocked by a firewall, or targeted the wrong asset.
\end{enumerate}
Assuming the scan is accurate, this is a positive development. It is recommended to verify this finding internally and, if confirmed, formally close the associated risk in the risk register.

% --- 5. Consolidated Risk Assessment ---
\section{Consolidated Risk Assessment}
The following table synthesizes findings from the security questionnaire, technical scans, and existing risk documentation into a prioritized list.

\begin{table}[h!]
\centering
\caption{Summary of Identified Risks}
\begin{tabular}{p{0.15\linewidth} p{0.45\linewidth} p{0.15\linewidth} p{0.1\linewidth}}
\toprule
\textbf{Risk ID} & \textbf{Description} & \textbf{Source} & \textbf{Severity} \\
\midrule
\textbf{RISK-001} & \textbf{No MFA for Computer Logins.} An attacker with valid credentials can gain direct endpoint and network access. & Questionnaire & \textbf{High} \\
\addlinespace
\textbf{RISK-002} & \textbf{No Employee Acceptable Use Policy.} Lack of a governing policy for IT asset usage creates legal and operational risks. & Questionnaire & \textbf{High} \\
\addlinespace
\textbf{OBS-001} & \textbf{Conflicting Port 80 Status.} Existing risk data states Port 80 is open, but a recent scan shows it is closed. & Scan vs. Risk Log & Informational \\
\bottomrule
\end{tabular}
\end{table}

% --- 6. Recommendations ---
\section{Recommendations}
The following actions are recommended to mitigate the identified risks and improve the overall security posture of \textbf{[Organization Name]}.

\subsection{High Priority: Remediate RISK-001}
\textbf{Action:} Implement and enforce Multi-Factor Authentication (MFA) for all computer (endpoint) logins.
\begin{itemize}
    \item \textbf{Immediate Steps:} Prioritize enabling MFA for all administrative and privileged user accounts.
    \item \textbf{Long-Term Strategy:} Roll out MFA to all users across the organization. Leverage solutions such as Windows Hello for Business, Duo Security, or other identity providers that integrate with your existing environment.
\end{itemize}

\subsection{High Priority: Remediate RISK-002}
\textbf{Action:} Develop, approve, and implement a formal Acceptable Use Policy (AUP).
\begin{itemize}
    \item \textbf{Immediate Steps:} Draft an AUP that clearly defines rules for computer, network, email, and internet usage, as well as data handling responsibilities.
    \item \textbf{Long-Term Strategy:} Integrate the AUP into the new employee onboarding process. Require all current employees to read and formally acknowledge the policy. Schedule annual reviews to ensure the policy remains current.
\end{itemize}

\subsection{Verification: Address OBS-001}
\textbf{Action:} Verify the status of Port 80 and update the risk register.
\begin{itemize}
    \item \textbf{Immediate Steps:} Internally confirm that Port 80 is closed on all external web-facing infrastructure. If it is confirmed to be closed, update the risk register to resolve the "Unencrypted Web Server" vulnerability.
    \item \textbf{Contingency:} If Port 80 is found to be open on any system, prioritize its closure or implement a TLS/SSL certificate to enforce HTTPS encryption.
\end{itemize}

\end{document}
```