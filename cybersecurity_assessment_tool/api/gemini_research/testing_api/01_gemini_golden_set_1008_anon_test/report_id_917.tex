```latex
\documentclass[12pt]{article}

% Preamble: Required Packages and Document Setup
\usepackage[margin=1in]{geometry}
\usepackage{pifont} % For checkmarks and crosses
\usepackage{booktabs} % For professional tables
\usepackage{hyperref} % For clickable links
\usepackage{url} % For formatting URLs
\usepackage{seqsplit} % For splitting long strings in tt font
\usepackage{graphicx}
\usepackage{xcolor}

% Document Metadata
\title{Cybersecurity Assessment Report for \textbf{[Organization Name]}}
\author{Cybersecurity Analyst}
\date{\today}

% Hyperref Setup
\hypersetup{
    colorlinks=true,
    linkcolor=blue,
    filecolor=magenta,      
    urlcolor=cyan,
    pdftitle={Cybersecurity Assessment Report},
    pdfpagemode=FullScreen,
}

\begin{document}

\maketitle
\thispagestyle{empty}
\newpage

\tableofcontents
\newpage

% --- 1. Executive Summary ---
\section{Executive Summary}
This report details the findings of a cybersecurity assessment conducted for \textbf{[Organization Name]}. The analysis correlates data from an external network scan, a security controls questionnaire, and a review of pre-existing risks.

The assessment identified a \textbf{critical risk}: the direct exposure of a Remote Desktop Protocol (RDP) service on port 3389 to the public internet. This configuration is a primary vector for ransomware attacks and unauthorized access. This technical vulnerability is further compounded by significant procedural gaps, including the lack of a formal employee Acceptable Use Policy and the absence of mandatory security awareness training for new hires.

While the organization has implemented Multi-Factor Authentication (MFA) for key systems, the combination of the exposed RDP service and foundational policy weaknesses creates a significant and immediate threat. We strongly recommend prioritizing the remediation actions outlined in Section \ref{sec:recommendations} to mitigate these risks.

% --- 2. Organizational Information ---
\section{Organizational Information}
This section contains the high-level information for the organization under review. As identity data was not provided, placeholders are used.

\begin{itemize}
    \item \textbf{Organization Name:} \textbf{[Organization Name]}
    \item \textbf{Primary Domain:} \texttt{[Domain]}
    \item \textbf{Scanned External IP:} \texttt{[Client IP]}
\end{itemize}

% --- 3. Security Control Review ---
\section{Security Control Review}
The following table summarizes the organization's responses to a security controls questionnaire. A green checkmark (\ding{51}) indicates a positive control is in place, while a red cross (\ding{55}) indicates a control gap that presents a risk.

\begin{table}[h!]
\centering
\caption{Security Controls Questionnaire Results}
\label{tab:controls}
\begin{tabular}{p{0.7\linewidth} c}
\toprule
\textbf{Control Question} & \textbf{Response} \\
\midrule
Do you require MFA to access email? & \textcolor{green}{\ding{51}} \\
Do you require MFA to log into computers? & \textcolor{green}{\ding{51}} \\
Do you require MFA to access sensitive data systems? & \textcolor{green}{\ding{51}} \\
Does your organization have an employee acceptable use policy? & \textcolor{red}{\ding{55}} \\
Does your organization do security awareness training for new employees? & \textcolor{red}{\ding{55}} \\
Does your organization do security awareness training for all employees at least once per year? & \textcolor{green}{\ding{51}} \\
\bottomrule
\end{tabular}
\end{table}

\paragraph{Analysis:} The organization demonstrates a strong commitment to identity security through the consistent implementation of MFA. However, two critical administrative control gaps were identified:
\begin{enumerate}
    \item \textbf{Lack of Acceptable Use Policy (AUP):} Without a formal AUP, employees lack clear guidelines on the secure and appropriate use of company assets, increasing the risk of unintentional data exposure or system misuse.
    \item \textbf{No Security Training for New Hires:} New employees are often prime targets for social engineering and phishing attacks. Failing to provide immediate security training upon hiring leaves a critical window of vulnerability.
\end{enumerate}

% --- 4. Technical Scan Results ---
\section{Technical Scan Results}
An external network scan was performed on the target IP address. The results below detail the open ports and services discovered.

\begin{itemize}
    \item \textbf{Target IP Address:} \texttt{[Target IP]}
\end{itemize}

\begin{table}[h!]
\centering
\caption{Open Ports Detected on \texttt{[Target IP]}}
\label{tab:nmap}
\begin{tabular}{llll}
\toprule
\textbf{Port} & \textbf{State} & \textbf{Service Name} & \textbf{Common Use} \\
\midrule
3389/tcp & open & ms-wbt-server & Microsoft Remote Desktop Protocol (RDP) \\
\bottomrule
\end{tabular}
\end{table}

\paragraph{Analysis:} The scan confirms that TCP port 3389 is open to the public internet. This port is used for Microsoft's Remote Desktop Protocol (RDP), which allows for direct administrative control of a Windows system. Exposing RDP directly to the internet is extremely dangerous and is a well-known tactic used by threat actors, particularly ransomware groups, to gain initial access to a network. This finding represents a critical technical vulnerability.

% --- 5. Correlated Risk Assessment ---
\section{Correlated Risk Assessment}
This section synthesizes the findings from the security control review, technical scan, and pre-existing risk data into a prioritized list of risks.

\begin{table}[h!]
\centering
\caption{Summary of Identified Risks}
\label{tab:risks}
\begin{tabular}{p{0.25\linewidth} p{0.5\linewidth} l}
\toprule
\textbf{Risk Name} & \textbf{Description} & \textbf{Severity} \\
\midrule
\textbf{Exposed Remote Desktop Protocol (RDP)} & The RDP service on \texttt{[Target IP]} is publicly accessible, creating a direct path for attackers to attempt brute-force logins or exploit RDP vulnerabilities to compromise the system. & \textbf{Critical (9.0)} \\
\addlinespace
\textbf{Lack of Employee Acceptable Use Policy} & The absence of a formal AUP creates ambiguity regarding security responsibilities and acceptable behavior, weakening the overall security posture and compliance framework. & \textbf{High} \\
\addlinespace
\textbf{Inadequate New-Hire Security Training} & New employees are not trained on security best practices upon joining, making them more susceptible to phishing and social engineering attacks before they are familiar with company policies. & \textbf{High} \\
\bottomrule
\end{tabular}
\end{table}

% --- 6. Recommendations ---
\section{Recommendations}
\label{sec:recommendations}
The following actionable steps are recommended to mitigate the identified risks. Recommendations are prioritized based on severity.

\subsection{Critical Risk: Exposed RDP Service}
\paragraph{Immediate Action (Containment):}
Immediately implement a firewall rule to \textbf{block all inbound traffic to TCP port 3389} on \texttt{[Target IP]}. Access should be denied from all external IP addresses. This is the single most important action to take from this report.

\paragraph{Long-Term Strategy (Remediation):}
For necessary remote access, implement a secure solution such as a \textbf{Virtual Private Network (VPN)} or a Zero-Trust Network Access (ZTNA) gateway. All remote administrative access must:
\begin{itemize}
    \item Require strong, unique passwords.
    \item Be protected by Multi-Factor Authentication (MFA).
    \item Be restricted to only authorized personnel.
\end{itemize}

\subsection{High Risk: Policy and Training Gaps}
\paragraph{Develop an Acceptable Use Policy (AUP):}
Draft and implement a formal AUP that clearly defines the rules for using company networks, systems, and data. This policy should be a mandatory part of the employee handbook and must be acknowledged in writing by all current and new employees.

\paragraph{Implement Onboarding Security Training:}
Integrate a mandatory cybersecurity awareness training module into the new-hire onboarding process. This training must occur within the first week of employment and cover, at a minimum:
\begin{itemize}
    \item Phishing and social engineering awareness.
    \item Password security and MFA requirements.
    \item The new Acceptable Use Policy.
\end{itemize}

% --- 7. Conclusion ---
\section{Conclusion}
The assessment for \textbf{[Organization Name]} reveals a solid foundation in identity management with widespread MFA adoption. However, this is critically undermined by an exposed RDP service and significant gaps in security governance and training.

The immediate priority must be to eliminate the public RDP exposure to prevent a potentially catastrophic security breach. Concurrently, addressing the policy and training deficiencies will build a more resilient and security-conscious culture, reducing the likelihood of future incidents.

\end{document}
```