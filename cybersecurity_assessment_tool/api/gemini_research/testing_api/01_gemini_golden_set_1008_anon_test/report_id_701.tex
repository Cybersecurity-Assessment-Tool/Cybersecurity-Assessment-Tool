```latex
\documentclass[12pt]{article}

% Preamble: Required Packages
\usepackage[margin=1in]{geometry}
\usepackage{pifont} % For checkmarks and crosses
\usepackage{booktabs} % For professional tables
\usepackage{hyperref} % For clickable links
\usepackage{url} % For URL formatting
\usepackage{seqsplit} % For splitting long strings
\usepackage{graphicx}
\usepackage{xcolor}

% Document Information
\title{Cybersecurity Assessment Report}
\author{Cybersecurity Analysis Division}
\date{\today}

% Hyperref Setup
\hypersetup{
    colorlinks=true,
    linkcolor=blue,
    filecolor=magenta,      
    urlcolor=cyan,
    pdftitle={Cybersecurity Assessment Report},
    pdfpagemode=FullScreen,
}

\begin{document}

\maketitle
\thispagestyle{empty}
\newpage

\tableofcontents
\thispagestyle{empty}
\newpage

\setcounter{page}{1}

% --- SECTION 1: EXECUTIVE SUMMARY ---
\section{Executive Summary}
This report details the findings of a cybersecurity assessment for \textbf{[Organization Name]}. The assessment combined an external network scan, a review of existing risks, and an analysis of organizational security controls based on a questionnaire.

The analysis revealed a critical and immediate risk: the direct exposure of a Remote Desktop Protocol (RDP) service on port 3389 to the public internet at \texttt{[Target IP]}. This finding, correlated with pre-existing risk data, represents a severe threat that could lead to a full network compromise via ransomware or data exfiltration.

Furthermore, significant gaps were identified in internal security controls. The lack of Multi-Factor Authentication (MFA) for computer logins and the absence of a formal Acceptable Use Policy (AUP) create substantial vulnerabilities. While the organization has implemented some positive controls, such as MFA for email and security awareness training, the identified critical flaws require immediate remediation to reduce the risk of a significant security incident.

\vspace{1cm}

\begin{tabular}{@{}ll}
\toprule
\textbf{Overall Risk Level:} & \textcolor{red}{\textbf{CRITICAL}} \\
\textbf{Key Findings:} & 1. Publicly Exposed RDP Service \\
                       & 2. No MFA for Endpoint/Computer Logins \\
                       & 3. Missing Acceptable Use Policy \\
\bottomrule
\end{tabular}

% --- SECTION 2: ORGANIZATIONAL INFORMATION ---
\section{Organizational Information}
This report is based on the information provided and gathered during the assessment period. The following details have been used for this analysis.

\begin{itemize}
    \item \textbf{Organization Name:} \textbf{[Organization Name]}
    \item \textbf{Primary Domain:} \texttt{[Domain]}
    \item \textbf{Assessed External IP:} \texttt{[Client IP]}
\end{itemize}

% --- SECTION 3: SECURITY CONTROL REVIEW ---
\section{Security Control Review}
The following table summarizes the organization's responses to the security controls questionnaire. Each response has been assessed against industry best practices. Items marked with \ding{55} indicate a control gap that increases organizational risk.

\begin{table}[h!]
\centering
\caption{Security Controls Questionnaire Analysis}
\begin{tabular}{p{0.5\textwidth} c p{0.3\textwidth}}
\toprule
\textbf{Control Question} & \textbf{Response} & \textbf{Assessment} \\
\midrule
Do you require MFA to access email? & \ding{51} & Meets best practice. \\
\addlinespace
Do you require MFA to log into computers? & \textcolor{red}{\ding{55}} & \textbf{Critical Gap.} Increases risk of unauthorized endpoint access via stolen credentials. \\
\addlinespace
Do you require MFA to access sensitive data systems? & \ding{51} & Meets best practice. \\
\addlinespace
Does your organization have an employee acceptable use policy? & \textcolor{red}{\ding{55}} & \textbf{High Risk.} Lack of a foundational policy creates ambiguity and increases insider risk. \\
\addlinespace
Does your organization do security awareness training for new employees? & \ding{51} & Meets best practice. \\
\addlinespace
Does your organization do security awareness training for all employees at least once per year? & \ding{51} & Meets best practice. \\
\bottomrule
\end{tabular}
\end{table}

% --- SECTION 4: TECHNICAL SCAN RESULTS ---
\section{Technical Scan Results}
An external network scan was performed on the target IP address to identify open ports and exposed services.

\begin{itemize}
    \item \textbf{Target IP Address:} \texttt{[Target IP]}
    \item \textbf{Scan Date:} Assumed to be recent, as per input data.
\end{itemize}

\subsection{Open Ports}
The scan identified the following open port, which is accessible from the public internet.

\begin{table}[h!]
\centering
\caption{Open Port Findings}
\begin{tabular}{l l l l}
\toprule
\textbf{Port} & \textbf{State} & \textbf{Service Name} & \textbf{Product / Version} \\
\midrule
3389/tcp & open & ms-wbt-server & Not specified \\
\bottomrule
\end{tabular}
\end{table}

\subsection{Analysis of Findings}
The service \texttt{ms-wbt-server} on port 3389 is the Microsoft Remote Desktop Protocol (RDP). Exposing RDP directly to the internet is extremely dangerous. It is a primary target for attackers who use brute-force password attacks, credential stuffing, and exploits for known vulnerabilities (e.g., BlueKeep) to gain unauthorized access to internal networks. This finding aligns directly with the pre-existing risk data provided.

% --- SECTION 5: CORRELATED RISK ASSESSMENT ---
\section{Correlated Risk Assessment}
This section synthesizes the findings from the technical scan, control review, and pre-existing risk data into a prioritized list of identified risks.

\begin{table}[h!]
\centering
\caption{Summary of Identified Risks}
\begin{tabular}{p{0.1\textwidth} p{0.25\textwidth} p{0.4\textwidth} p{0.1\textwidth}}
\toprule
\textbf{Risk ID} & \textbf{Risk Name} & \textbf{Description} & \textbf{Severity} \\
\midrule
RISK-001 & Public RDP Exposure & The Remote Desktop Protocol on port 3389 is exposed on \texttt{[Target IP]}. This allows attackers to attempt remote access and is a common vector for ransomware attacks. & \textbf{Critical} \\
\addlinespace
RISK-002 & Lack of Endpoint MFA & The absence of MFA for computer logins means that a single compromised password could grant an attacker full access to an employee's workstation and potentially the network. & \textbf{High} \\
\addlinespace
RISK-003 & Missing Acceptable Use Policy & Without a formal AUP, there are no defined rules for employee use of company assets. This increases the likelihood of unintentional insider threats and risky behavior. & \textbf{Medium} \\
\bottomrule
\end{tabular}
\end{table}

% --- SECTION 6: RECOMMENDATIONS ---
\section{Recommendations}
The following actions are recommended to mitigate the identified risks. They are prioritized based on severity and potential impact.

\subsection{Immediate Priority (Remediate within 24 hours)}
\begin{itemize}
    \item \textbf{Mitigate RISK-001 (Public RDP Exposure):}
    \begin{enumerate}
        \item \textbf{Short-Term Fix:} Immediately implement a firewall rule to block all inbound traffic to TCP port 3389 on \texttt{[Target IP]} from the internet. Access should only be permitted from trusted internal IP addresses, if required.
        \item \textbf{Long-Term Fix:} For necessary remote access, deploy a Virtual Private Network (VPN) solution. The VPN should be configured with strong encryption and require Multi-Factor Authentication for all users.
    \end{enumerate}
\end{itemize}

\subsection{High Priority (Remediate within 30 days)}
\begin{itemize}
    \item \textbf{Mitigate RISK-002 (Lack of Endpoint MFA):}
    \begin{enumerate}
        \item Procure and deploy an MFA solution for all Windows, macOS, and Linux endpoint logins (e.g., Windows Hello for Business, Duo, etc.).
        \item Enforce this policy for all users, including administrative and privileged accounts.
    \end{enumerate}
\end{itemize}

\subsection{Medium Priority (Remediate within 90 days)}
\begin{itemize}
    \item \textbf{Mitigate RISK-003 (Missing Acceptable Use Policy):}
    \begin{enumerate}
        \item Develop a formal Acceptable Use Policy (AUP) that clearly defines the rules for using company computers, networks, and data.
        \item Distribute the policy to all current employees and incorporate it into the onboarding process for new hires.
        \item Require all employees to formally acknowledge that they have read and understood the policy.
    \end{enumerate}
\end{itemize}

\end{document}
```