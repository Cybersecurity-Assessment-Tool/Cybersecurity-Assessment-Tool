```latex
\documentclass[12pt]{article}

% Preamble: Required Packages
\usepackage[margin=1in]{geometry}
\usepackage{pifont} % For checkmarks and crosses (\ding)
\usepackage{booktabs} % For professional-looking tables
\usepackage{hyperref} % For clickable links and references
\usepackage{url} % For formatting URLs
\usepackage{seqsplit} % To split long strings without breaking
\usepackage{graphicx}
\usepackage{xcolor}

% Document Metadata
\title{Cybersecurity Posture Assessment Report}
\author{Cybersecurity Analysis Division}
\date{\today}

% Hyperref Setup
\hypersetup{
    colorlinks=true,
    linkcolor=blue,
    filecolor=magenta,      
    urlcolor=cyan,
    pdftitle={Cybersecurity Posture Assessment Report},
    pdfpagemode=FullScreen,
}

\begin{document}

\maketitle
\thispagestyle{empty}
\newpage

\tableofcontents
\newpage

% --- 1. Executive Overview ---
\section{Executive Overview}
This report details the findings of a cybersecurity posture assessment conducted for \textbf{[Organization Name]}. The assessment combined an external network scan, a review of existing risks, and an analysis of organizational security controls based on a questionnaire.

The analysis revealed several critical and high-risk vulnerabilities that require immediate attention. The most severe finding is the direct exposure of a Remote Desktop Protocol (RDP) service to the public internet on host \texttt{[Target IP]}. This vulnerability, rated with a CVSS score of 9.0 (Critical), presents a significant risk of unauthorized access and potential system compromise.

Furthermore, critical gaps were identified in the organization's security controls. The absence of Multi-Factor Authentication (MFA) on sensitive data systems, coupled with the lack of a formal Acceptable Use Policy and annual security awareness training, significantly weakens the overall defense-in-depth strategy.

Immediate remediation of the exposed RDP service is paramount. Following this, we strongly recommend implementing the corrective actions outlined in Section 6 to address the identified policy and procedure gaps, thereby enhancing the organization's resilience against cyber threats.

% --- 2. Organizational Information ---
\section{Organizational Information}
The following information was used as the basis for this assessment. Due to the anonymized nature of the provided data, placeholders have been used where necessary.

\begin{itemize}
    \item \textbf{Organization Name:} \textbf{[Organization Name]}
    \item \textbf{Primary Email Domain:} \seqsplit{\texttt{[Domain]}}
    \item \textbf{External IP Address Scanned:} \seqsplit{\texttt{[Client IP]}}
\end{itemize}

% --- 3. Security Control Review ---
\section{Security Control Review}
An assessment of internal security controls was conducted based on a questionnaire. The responses indicate several areas of concern where current practices do not align with security best practices. The table below summarizes the findings.

\begin{table}[h!]
\centering
\caption{Security Controls Questionnaire Analysis}
\label{tab:controls}
\begin{tabular}{@{}p{0.6\linewidth}ccp{0.2\linewidth}@{}}
\toprule
\textbf{Control Question} & \textbf{Response} & \textbf{Status} & \textbf{Assessment} \\
\midrule
Do you require MFA to access email? & Yes & \ding{51} & Compliant \\
Do you require MFA to log into computers? & Yes & \ding{51} & Compliant \\
Do you require MFA to access sensitive data systems? & No & \textbf{\color{red}\ding{55}} & \textbf{Critical Gap} \\
Does your organization have an employee acceptable use policy? & No & \textbf{\color{red}\ding{55}} & \textbf{High Risk} \\
Does your organization do security awareness training for new employees? & Yes & \ding{51} & Compliant \\
Does your organization do security awareness training for all employees at least once per year? & No & \textbf{\color{red}\ding{55}} & \textbf{High Risk} \\
\bottomrule
\end{tabular}
\end{table}

The "No" responses highlight significant weaknesses:
\begin{itemize}
    \item \textbf{Lack of MFA on Sensitive Systems:} This is a critical vulnerability. In the event of a credential compromise, there is no secondary control to prevent an attacker from accessing the organization's most valuable data.
    \item \textbf{Missing Acceptable Use Policy (AUP):} Without a formal AUP, there are no clear guidelines for employees regarding the secure use of company assets, which can lead to inconsistent security practices and increased risk.
    \item \textbf{Insufficient Security Training:} Security knowledge is perishable. Failing to provide annual refresher training means employees are less likely to recognize and appropriately respond to evolving threats like phishing and social engineering.
\end{itemize}

% --- 4. Technical Scan Results ---
\section{Technical Scan Results}
An external network scan was performed on the target IP address to identify open ports and exposed services.

\begin{itemize}
    \item \textbf{Target IP Address:} \seqsplit{\texttt{[Target IP]}}
\end{itemize}

The scan identified the following open port:

\begin{table}[h!]
\centering
\caption{Open Port Analysis}
\label{tab:ports}
\begin{tabular}{@{}llll@{}}
\toprule
\textbf{Port} & \textbf{State} & \textbf{Service} & \textbf{Notes} \\
\midrule
3389/tcp & open & ms-wbt-server & \textbf{Critical Risk.} Microsoft Remote Desktop Protocol (RDP). \\
\bottomrule
\end{tabular}
\end{table}

\textbf{Analysis:} The presence of an open RDP port (3389) is a severe security risk. This service is a primary target for attackers who use brute-force techniques to guess credentials or exploit known vulnerabilities (e.g., BlueKeep) to gain unauthorized remote access to the network. This finding directly correlates with and validates the pre-existing risk documented in the organization's risk register.

% --- 5. Consolidated Risk Assessment ---
\section{Consolidated Risk Assessment}
The following table synthesizes findings from the technical scan, control review, and pre-existing risk data into a consolidated list of key risks facing the organization.

\begin{table}[h!]
\centering
\caption{Summary of Identified Risks}
\label{tab:risks}
\begin{tabular}{@{}p{0.25\linewidth}p{0.5\linewidth}l@{}}
\toprule
\textbf{Risk Title} & \textbf{Description} & \textbf{Severity} \\
\midrule
\textbf{Public RDP Exposure} & The Remote Desktop Protocol service on port 3389 is exposed to the public internet on host \texttt{[Target IP]}, inviting brute-force and exploit-based attacks. & \textbf{Critical (9.0)} \\
\addlinespace
\textbf{Lack of MFA on Sensitive Systems} & Sensitive data systems are accessible with only a username and password, leaving them highly vulnerable to credential compromise. & \textbf{Critical} \\
\addlinespace
\textbf{Missing Acceptable Use Policy} & The absence of a formal policy creates ambiguity for employees regarding secure behavior and limits the organization's ability to enforce security standards. & \textbf{High} \\
\addlinespace
\textbf{Insufficient Security Awareness Training} & Training is not conducted annually for all staff, leading to a degradation of security awareness and increased susceptibility to social engineering. & \textbf{High} \\
\bottomrule
\end{tabular}
\end{table}

% --- 6. Recommendations ---
\section{Recommendations}
Based on the consolidated risk assessment, we recommend the following actions, prioritized by severity.

\subsection{Immediate Actions (Critical Risks)}
\begin{enumerate}
    \item \textbf{Remediate RDP Exposure Immediately:}
        \begin{itemize}
            \item \textbf{Short-Term Fix:} Block all inbound traffic to TCP port 3389 on the external firewall for host \texttt{[Target IP]}. This should be done immediately to mitigate the risk.
            \item \textbf{Long-Term Solution:} If remote access is required, implement a Virtual Private Network (VPN) solution that requires Multi-Factor Authentication (MFA). All RDP access should occur exclusively through the secure VPN tunnel.
        \end{itemize}
    \item \textbf{Implement MFA on Sensitive Systems:}
        \begin{itemize}
            \item Prioritize and deploy an MFA solution for all systems identified as containing sensitive or critical data. This is the single most effective control to prevent unauthorized access resulting from stolen credentials.
        \end{itemize}
\end{enumerate}

\subsection{High-Priority Actions}
\begin{enumerate}
    \setcounter{enumi}{2} % Continue numbering from previous list
    \item \textbf{Develop and Implement an Acceptable Use Policy (AUP):}
        \begin{itemize}
            \item Draft a comprehensive AUP that clearly defines the rules and expectations for the use of company technology, data, and network resources.
            \item Ensure all employees read and formally acknowledge the policy.
        \end{itemize}
    \item \textbf{Establish an Annual Security Awareness Training Program:}
        \begin{itemize}
            \item Procure or develop a security awareness training module and mandate its completion by all employees on an annual basis.
            \item Training should cover current threats such as phishing, password security, and social engineering.
        \end{itemize}
\end{enumerate}

\end{document}
```