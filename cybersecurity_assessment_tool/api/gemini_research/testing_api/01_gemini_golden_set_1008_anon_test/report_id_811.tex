```latex
\documentclass[12pt]{article}

% Preamble: Required Packages
\usepackage[margin=1in]{geometry}
\usepackage{pifont} % For checkmarks and crosses
\usepackage{booktabs} % For professional tables
\usepackage{hyperref} % For hyperlinks
\usepackage{url} % For URL formatting
\usepackage{seqsplit} % To split long strings in tt font
\usepackage{xcolor} % For colors in tables

% Document Metadata
\title{Cybersecurity Posture Assessment Report}
\author{Cybersecurity Analysis Division}
\date{\today}

% Hyperref Setup
\hypersetup{
    colorlinks=true,
    linkcolor=blue,
    filecolor=magenta,      
    urlcolor=cyan,
    pdftitle={Cybersecurity Posture Assessment Report},
    pdfpagemode=FullScreen,
}

\begin{document}

\maketitle
\thispagestyle{empty}
\newpage
\tableofcontents
\newpage

% --- 1. Executive Summary ---
\section{Executive Summary}
This report provides a comprehensive analysis of the cybersecurity posture for \textbf{[Organization Name]}. The assessment is based on a correlation of organizational data, a network vulnerability scan, and a review of pre-existing risks.

The analysis revealed several critical and high-risk security gaps that require immediate attention. Key findings include a lack of Multi-Factor Authentication (MFA) for email and computer access, and the absence of a formal security awareness training program. These administrative control deficiencies significantly increase the organization's susceptibility to phishing, credential theft, and subsequent unauthorized access.

Technically, an externally exposed Secure Shell (SSH) service was identified. When combined with the lack of MFA and security training, this creates a high-probability attack vector for threat actors attempting to compromise the network perimeter.

This report outlines the identified risks and provides actionable recommendations to mitigate them, thereby strengthening the overall security posture of the organization.

% --- 2. Organizational Information ---
\section{Organizational Information}
The following details were used as the basis for this assessment. Due to the anonymized nature of the provided data, placeholders have been used where necessary.

\begin{itemize}
    \item \textbf{Organization Name:} \textbf{[Organization Name]}
    \item \textbf{Primary Email Domain:} \seqsplit{\texttt{[Domain]}}
    \item \textbf{External IP Scanned:} \seqsplit{\texttt{[Client IP]}}
\end{itemize}

% --- 3. Security Control Review ---
\section{Security Control Review}
A security questionnaire was conducted to evaluate the implementation of fundamental administrative and technical controls. The responses indicate significant gaps in user access controls and security training. A summary of the findings is presented in Table \ref{tab:controls}.

\begin{table}[h!]
\centering
\caption{Security Controls Questionnaire Results}
\label{tab:controls}
\begin{tabular}{p{0.75\linewidth} c}
\toprule
\textbf{Control Question} & \textbf{Response} \\
\midrule
Do you require MFA to access email? & \textcolor{red}{\ding{55}} \\
Do you require MFA to log into computers? & \textcolor{red}{\ding{55}} \\
Do you require MFA to access sensitive data systems? & \textcolor{green}{\ding{51}} \\
Does your organization have an employee acceptable use policy? & \textcolor{green}{\ding{51}} \\
Does your organization do security awareness training for new employees? & \textcolor{red}{\ding{55}} \\
Does your organization do security awareness training for all employees at least once per year? & \textcolor{red}{\ding{55}} \\
\bottomrule
\end{tabular}
\end{table}

The absence of MFA for email and computer logins, coupled with a lack of security awareness training, represents a critical risk. These gaps make the organization highly vulnerable to social engineering and credential-based attacks.

% --- 4. Technical Scan Results ---
\section{Technical Scan Results}
An external network scan was performed against the organization's public-facing infrastructure. The scan was initiated on the date of this report.

\begin{itemize}
    \item \textbf{Target IP Address:} \seqsplit{\texttt{[Target IP]}}
\end{itemize}

The scan identified one open port, detailed in Table \ref{tab:scan}.

\begin{table}[h!]
\centering
\caption{Open Port Analysis}
\label{tab:scan}
\begin{tabular}{l l l p{0.5\linewidth}}
\toprule
\textbf{Port/Proto} & \textbf{State} & \textbf{Service (Inferred)} & \textbf{Notes} \\
\midrule
22/tcp & OPEN & SSH & The Secure Shell service is exposed to the public internet. This service is a primary target for automated brute-force attacks seeking to gain unauthorized remote access. No version information was available from the scan results. \\
\bottomrule
\end{tabular}
\end{table}

% --- 5. Risk Assessment ---
\section{Risk Assessment}
The following table synthesizes findings from the security control review and the technical scan. No pre-existing vulnerabilities were reported. The risks identified below are prioritized based on their potential impact and likelihood of exploitation.

\begin{table}[h!]
\centering
\caption{Synthesized Risk Register}
\label{tab:risks}
\begin{tabular}{p{0.1\linewidth} p{0.4\linewidth} p{0.15\linewidth} p{0.25\linewidth}}
\toprule
\textbf{ID} & \textbf{Risk Description} & \textbf{Severity} & \textbf{Affected Asset(s)} \\
\midrule
RISK-001 & \textbf{Lack of MFA for Email:} Absence of MFA on email accounts greatly increases the risk of Business Email Compromise (BEC), data exfiltration, and phishing propagation. & \textbf{Critical} & Email System, Corporate Data, User Credentials \\
\midrule
RISK-002 & \textbf{No Security Awareness Training:} Employees are not trained to identify or report phishing and other social engineering attacks, making them the weakest link in the security chain. & \textbf{Critical} & All Employees, Endpoints, Corporate Data \\
\midrule
RISK-003 & \textbf{Exposed SSH Service:} The SSH port is open to the internet, creating a direct vector for brute-force attacks and potential unauthorized remote access to the internal network. & \textbf{High} & Network Perimeter, Internal Servers \\
\midrule
RISK-004 & \textbf{Lack of MFA for Endpoints:} The absence of MFA for computer logins allows an attacker with stolen credentials to easily access workstations and potentially move laterally within the network. & \textbf{High} & Employee Workstations, Servers, Internal Network \\
\bottomrule
\end{tabular}
\end{table}

% --- 6. Recommendations ---
\section{Recommendations}
Based on the identified risks, the following prioritized actions are recommended to improve the organization's security posture.

\begin{enumerate}
    \item \textbf{[Critical] Implement MFA for Email:} Immediately enforce MFA for all user mailboxes. This is the single most effective control to prevent account takeovers and Business Email Compromise.
    
    \item \textbf{[Critical] Establish a Security Awareness Program:} Develop and deploy a mandatory security awareness training program. This must include initial training for all new hires and at least one annual refresher course for all staff. Training should focus on identifying phishing, password hygiene, and reporting security incidents.
    
    \item \textbf{[High] Secure the Exposed SSH Service:} Review the business need for external SSH access.
    \begin{itemize}
        \item If not required, block port 22 at the network firewall.
        \item If required, restrict access to a whitelist of trusted source IP addresses.
        \item Enforce the use of public key authentication and disable password-based logins for SSH.
    \end{itemize}
    
    \item \textbf{[High] Enforce MFA for Endpoint and System Logins:} Deploy MFA for all computer and server logins. This provides a critical layer of defense against attacks that rely on compromised credentials for initial access and lateral movement.
\end{enumerate}

\end{document}
```