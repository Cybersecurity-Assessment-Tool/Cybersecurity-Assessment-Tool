```latex
\documentclass[12pt]{article}

% Preamble: Required Packages
\usepackage[margin=1in]{geometry}
\usepackage{pifont} % For checkmarks and crosses
\usepackage{booktabs} % For professional tables
\usepackage{hyperref} % For clickable links
\usepackage{url} % For URL formatting
\usepackage{seqsplit} % For splitting long strings
\usepackage{graphicx}
\usepackage[table]{xcolor}
\usepackage{fancyhdr}

% --- Document Setup ---
\hypersetup{
    colorlinks=true,
    linkcolor=blue,
    filecolor=magenta,      
    urlcolor=cyan,
    pdftitle={Cybersecurity Risk Assessment Report},
    pdfpagemode=FullScreen,
}

\pagestyle{fancy}
\fancyhf{}
\lhead{Cybersecurity Risk Assessment}
\rhead{\textbf{[Organization Name]}}
\cfoot{\thepage}

% --- Custom Commands & Colors ---
\newcommand{\yes}{\ding{51}}
\newcommand{\no}{\ding{55}}
\definecolor{critical}{HTML}{990000}
\definecolor{high}{HTML}{D14124}
\definecolor{medium}{HTML}{E89923}
\definecolor{low}{HTML}{3A7D34}

% ==============================================================================
% --- BEGIN DOCUMENT ---
% ==============================================================================
\begin{document}

% --- Title Page ---
\begin{titlepage}
    \centering
    \vspace*{1cm}
    \includegraphics[width=0.4\textwidth]{example-image-a} % Placeholder for company logo
    \vfill
    \huge\textbf{Cybersecurity Risk Assessment Report}
    \vspace{1.5cm}
    \Large
    \textbf{Prepared for:} \textbf{[Organization Name]}\\
    \vspace{0.5cm}
    \textbf{Date of Report:} \today\\
    \vspace{0.5cm}
    \textbf{Scan Date:} November 22, 2025\\
    \vfill
    \textit{This report contains sensitive information and should be handled with care. Distribution is restricted to authorized personnel only.}
\end{titlepage}

\tableofcontents
\newpage

% ==============================================================================
% --- 1. Executive Summary ---
% ==============================================================================
\section{Executive Summary}

This report provides a comprehensive cybersecurity risk assessment for \textbf{[Organization Name]}, based on an analysis of organizational security controls, an external network scan, and a review of pre-existing risks. The assessment was conducted to identify vulnerabilities, evaluate the current security posture, and provide actionable recommendations to mitigate identified risks.

The analysis reveals a mixed security posture. The organization demonstrates a solid foundation in security awareness training and has successfully implemented Multi-Factor Authentication (MFA) for email access. However, two critical gaps were identified that significantly increase the organization's risk profile:

\begin{enumerate}
    \item \textbf{Critical Lack of MFA:} Multi-Factor Authentication is not enforced for logging into computers or accessing sensitive data systems. This exposes the organization to a high risk of unauthorized access through credential compromise.
    \item \textbf{Outdated Web Server Software:} The external-facing web server is running an outdated version of Nginx (\texttt{1.18.0}), which has multiple publicly known vulnerabilities. This could allow an attacker to compromise the server, leading to data breaches or service disruption.
\end{enumerate}

No pre-existing vulnerabilities were documented. The recommendations in this report focus on immediately addressing these high-priority issues to strengthen the overall security posture of \textbf{[Organization Name]}.

% ==============================================================================
% --- 2. Organizational Information ---
% ==============================================================================
\section{Organizational Information}

The following details were used as the basis for this assessment. As per our template mode protocol, placeholders are used where specific data was not provided.

\begin{table}[h!]
\centering
\begin{tabular}{@{}ll@{}}
\toprule
\textbf{Attribute} & \textbf{Value} \\ \midrule
Organization Name & \textbf{[Organization Name]} \\
Primary Domain & \texttt{[Domain]} \\
External IP Address (Scanned) & \texttt{[Client IP]} \\ \bottomrule
\end{tabular}
\caption{Client Organizational Details.}
\end{label{tab:org_info}
\end{table}

% ==============================================================================
% --- 3. Security Control Review (Questionnaire Analysis) ---
% ==============================================================================
\section{Security Control Review}

An assessment of the organization's security policies and procedures was conducted via a standardized questionnaire. The responses indicate key areas of strength and significant opportunities for improvement. "No" answers represent gaps in security controls that must be addressed.

\begin{table}[h!]
\centering
\renewcommand{\arraystretch}{1.3}
\begin{tabular}{@{}p{0.6\linewidth}cp{0.2\linewidth}@{}}
\toprule
\textbf{Control Question} & \textbf{Response} & \textbf{Assessment} \\ \midrule
Do you require MFA to access email? & \yes & Best Practice Met \\
\rowcolor{red!15} Do you require MFA to log into computers? & \no & \textbf{High Risk} \\
\rowcolor{red!25} Do you require MFA to access sensitive data systems? & \no & \textbf{Critical Gap} \\
Does your organization have an employee acceptable use policy? & \yes & Best Practice Met \\
Does your organization do security awareness training for new employees? & \yes & Best Practice Met \\
Does your organization do security awareness training for all employees at least once per year? & \yes & Best Practice Met \\ \bottomrule
\end{tabular}
\caption{Security Controls Questionnaire Results.}
\label{tab:controls}
\end{table}

The failure to enforce MFA on computer logins and sensitive systems is a critical vulnerability. An attacker with stolen credentials could gain deep access to the network and its most valuable data assets without needing a second authentication factor.

% ==============================================================================
% --- 4. Technical Scan Results ---
% ==============================================================================
\section{Technical Scan Results}

An external network scan was performed on \textbf{November 22, 2025}, against the target IP address \texttt{[Target IP]}. The scan identified the following open ports and services.

\begin{table}[h!]
\centering
\renewcommand{\arraystretch}{1.3}
\begin{tabular}{@{}clllll@{}}
\toprule
\textbf{Port} & \textbf{State} & \textbf{Service} & \textbf{Product} & \textbf{Version} & \textbf{Finding} \\ \midrule
\rowcolor{high!15} 443/tcp & Open & https & nginx & \texttt{1.18.0} & \begin{tabular}[c]{@{}l@{}}\textbf{Outdated Software:}\\ This version is from 2020 and has\\ known public vulnerabilities (CVEs).\end{tabular} \\ \bottomrule
\end{tabular}
\caption{Open Ports and Services Detected on \texttt{[Target IP]}.}
\label{tab:scan_results}
\end{table}

\subsection{Analysis of Findings}
The primary technical finding is the presence of Nginx version \texttt{1.18.0} exposed to the internet. This version is significantly out of date, with the stable branch being much more recent. Outdated software is a primary target for automated attacks, as exploits for known vulnerabilities are widely available. This exposes the organization to risks such as remote code execution, denial of service, and information disclosure.

% ==============================================================================
% --- 5. Synthesized Risk Assessment ---
% ==============================================================================
\section{Synthesized Risk Assessment}

This section correlates the findings from the security control review and the technical scan to provide a summarized list of identified risks.

\begin{table}[h!]
\centering
\renewcommand{\arraystretch}{1.5}
\begin{tabular}{@{}lp{0.3\linewidth}cp{0.45\linewidth}@{}}
\toprule
\textbf{ID} & \textbf{Risk Name} & \textbf{Severity} & \textbf{Description} \\ \midrule
RISK-001 & Lack of MFA on Critical Systems & \cellcolor{critical}\color{white}Critical & The absence of MFA on endpoints and sensitive data systems allows for single-factor authentication, making successful credential theft attacks highly probable and impactful. \\ \addlinespace
RISK-002 & Outdated \& Vulnerable Web Server & \cellcolor{high}\color{white}High & The public-facing web server runs Nginx \texttt{1.18.0}, a version with multiple known high-severity vulnerabilities. This could lead to a server compromise. \\ \bottomrule
\end{tabular}
\caption{Summary of Identified Risks.}
\label{tab:risks}
\end{table}

% ==============================================================================
% --- 6. Recommendations ---
% ==============================================================================
\section{Recommendations}
The following actions are recommended to mitigate the identified risks and improve the overall security posture of \textbf{[Organization Name]}. Recommendations are prioritized based on risk severity.

\begin{enumerate}
    \item \textbf{[Critical] Implement Comprehensive MFA (RISK-001):}
    \begin{itemize}
        \item \textbf{Immediate Action:} Prioritize the deployment of a robust MFA solution for all user endpoints (laptops, desktops) and remote access systems (VPN).
        \item \textbf{Secondary Action:} Extend MFA enforcement to all applications and systems that store or process sensitive data. This is crucial for protecting the organization's most valuable assets.
    \end{itemize}
    \vspace{0.5cm}
    \item \textbf{[High] Establish a Patch Management Program (RISK-002):}
    \begin{itemize}
        \item \textbf{Immediate Action:} Plan and execute an upgrade of the Nginx server from version \texttt{1.18.0} to the latest stable version. This should be done in a staging environment first to ensure compatibility.
        \item \textbf{Strategic Action:} Implement a formal patch management policy and procedure. This should include regular scanning for outdated software on all systems, a defined testing process, and a schedule for applying security patches.
    \end{itemize}
    \vspace{0.5cm}
    \item \textbf{[Medium] Conduct Regular Vulnerability Assessments:}
    \begin{itemize}
        \item \textbf{Strategic Action:} Schedule regular, automated vulnerability scans (at least quarterly) of both external and internal networks. This will help proactively identify and remediate issues like outdated software and misconfigurations before they can be exploited.
    \end{itemize}
\end{enumerate}

% ==============================================================================
% --- 7. Conclusion ---
% ==============================================================================
\section{Conclusion}
This assessment has identified significant risks to \textbf{[Organization Name]}, primarily related to inadequate access controls and the use of vulnerable, outdated software. While the organization has positive security practices in place, such as security awareness training, the identified gaps could undermine these efforts.

By implementing the recommendations outlined in this report—especially the immediate enforcement of MFA and the upgrade of the public-facing web server—the organization can substantially reduce its attack surface and improve its resilience against common cyber threats. We recommend a follow-up assessment in six months to validate the implementation of these controls.

\end{document}
% ==============================================================================
% --- END OF DOCUMENT ---
% ==============================================================================
```