```latex
\documentclass[12pt]{article}

% Preamble: Required Packages
\usepackage[margin=1in]{geometry}
\usepackage{pifont} % For checkmarks and crosses
\usepackage{booktabs} % For professional tables
\usepackage{hyperref} % For clickable links
\usepackage{url}      % For URL formatting
\usepackage{seqsplit} % For splitting long strings
\usepackage{graphicx} % For potential logos
\usepackage{xcolor}   % For colors in text

% Document Metadata
\title{Cybersecurity Assessment Report}
\author{Cybersecurity Analysis Division}
\date{\today}

% Hyperref Setup
\hypersetup{
    colorlinks=true,
    linkcolor=blue,
    filecolor=magenta,      
    urlcolor=cyan,
    pdftitle={Cybersecurity Assessment Report},
    pdfpagemode=FullScreen,
}

\begin{document}

\maketitle
\thispagestyle{empty}
\newpage

\tableofcontents
\newpage

% --- 1. Executive Summary ---
\section{Executive Summary}

This report details the findings of a cybersecurity assessment conducted for \textbf{[Organization Name]}. The assessment combined a review of organizational security controls, an external network vulnerability scan, and an analysis of pre-existing risks.

The overall security posture presents a mixed landscape. On a positive note, the external network scan of the target host \texttt{[Target IP]} revealed no open ports, suggesting a well-configured network perimeter. The organization also enforces Multi-Factor Authentication (MFA) for email access and conducts annual security training, which are commendable baseline practices.

However, several critical and high-risk gaps were identified in the organization's security controls. The absence of MFA for computer logins and access to sensitive data systems constitutes a significant vulnerability. An attacker who compromises a user's credentials could gain direct access to endpoints and critical information. Furthermore, the lack of mandatory security awareness training for new employees leaves a crucial window of vulnerability, as new hires are often targeted by social engineering attacks.

This report provides a detailed breakdown of these findings and offers actionable recommendations to mitigate the identified risks and strengthen the organization's overall defense-in-depth strategy.

% --- 2. Organizational Information ---
\section{Organizational Information}

This assessment was conducted for the following organization. The information provided has been used to contextualize the findings within this report.

\begin{table}[h!]
\centering
\begin{tabular}{@{}ll@{}}
\toprule
\textbf{Attribute} & \textbf{Value} \\ \midrule
Organization Name & \textbf{[Organization Name]} \\
Email Domain & \texttt{[Domain]} \\
External IP Scanned & \texttt{[Client IP]} \\
Assessment Date & \today \\ \bottomrule
\end{tabular}
\caption{Client Organizational Details}
\label{tab:org_info}
\end{table}

% --- 3. Security Control Review ---
\section{Security Control Review}

A security questionnaire was completed to evaluate the implementation of key administrative and technical controls. The responses are summarized below. Items marked with a red cross (\ding{55}) indicate significant gaps in security posture.

\begin{table}[h!]
\centering
\begin{tabular}{@{}p{0.7\textwidth}c@{}}
\toprule
\textbf{Control Question} & \textbf{Status} \\ \midrule
Do you require MFA to access email? & \textcolor{green}{\ding{51}} \\
Do you require MFA to log into computers? & \textcolor{red}{\ding{55}} \\
Do you require MFA to access sensitive data systems? & \textcolor{red}{\ding{55}} \\
Does your organization have an employee acceptable use policy? & \textcolor{green}{\ding{51}} \\
Does your organization do security awareness training for new employees? & \textcolor{red}{\ding{55}} \\
Does your organization do security awareness training for all employees at least once per year? & \textcolor{green}{\ding{51}} \\ \bottomrule
\end{tabular}
\caption{Security Controls Questionnaire Results}
\label{tab:controls}
\end{table}

\subsection*{Analysis of Control Gaps}
The review identified three primary areas of concern:
\begin{itemize}
    \item \textbf{Endpoint Access Control:} The lack of MFA for computer logins significantly increases the risk of unauthorized access following a credential compromise.
    \item \textbf{Sensitive Data Protection:} Failure to protect sensitive data systems with MFA removes a critical layer of defense for the organization's most valuable information assets.
    \item \textbf{Employee Onboarding:} New employees are not receiving security awareness training upon being hired. This group is often highly susceptible to phishing and social engineering attacks, creating an immediate risk to the organization.
\end{itemize}

% --- 4. Technical Scan Results ---
\section{Technical Scan Results}

An external network scan was performed to identify open ports, running services, and potential vulnerabilities on the public-facing infrastructure.

\begin{itemize}
    \item \textbf{Target IP Address:} \texttt{[Target IP]}
    \item \textbf{Scan Date:} [Scan Date]
\end{itemize}

\subsection*{Findings}
The scan of the target system completed successfully. \textbf{No open ports or active services were discovered.}

This is a positive security finding, indicating that the network firewall and perimeter security controls are effectively configured to deny unsolicited inbound traffic. This greatly reduces the external attack surface of the scanned host.

% --- 5. Consolidated Risk Assessment ---
\section{Consolidated Risk Assessment}

This section correlates findings from the security control review, technical scan, and any pre-existing risks. No pre-existing risks were provided for this assessment. The primary risks identified stem from gaps in administrative and identity management controls.

\begin{table}[h!]
\centering
\begin{tabular}{@{}p{0.25\textwidth}p{0.5\textwidth}l@{}}
\toprule
\textbf{Risk Name} & \textbf{Overview} & \textbf{Severity} \\ \midrule
\textbf{Lack of MFA on Sensitive Systems} & The absence of a second authentication factor for systems holding sensitive data exposes critical assets to unauthorized access if user credentials are stolen. & \textbf{Critical} \\
\addlinespace
\textbf{Lack of MFA on Endpoints} & User computers and workstations are not protected by MFA. A compromised password could lead to direct endpoint access, facilitating lateral movement and data exfiltration. & \textbf{High} \\
\addlinespace
\textbf{Inadequate Onboarding Security Training} & New employees are not trained on security policies and threat identification upon joining. This makes them a prime target for phishing and social engineering attacks. & \textbf{High} \\
\bottomrule
\end{tabular}
\caption{Summary of Identified Risks}
\label{tab:risks}
\end{table}

% --- 6. Recommendations ---
\section{Recommendations}

The following actions are recommended to mitigate the identified risks and improve the overall security posture of \textbf{[Organization Name]}. Recommendations are prioritized based on risk severity.

\subsection*{Critical Priority}
\begin{enumerate}
    \item \textbf{Implement MFA for All Sensitive Systems:}
    \begin{itemize}
        \item \textbf{Action:} Enforce a non-bypassable MFA policy for all user accounts (including administrative and service accounts) that can access systems designated as containing sensitive or critical data.
        \item \textbf{Impact:} Drastically reduces the risk of a data breach resulting from stolen credentials.
    \end{itemize}
\end{enumerate}

\subsection*{High Priority}
\begin{enumerate}
    \setcounter{enumi}{1} % Continue numbering
    \item \textbf{Deploy MFA for Endpoint Logon:}
    \begin{itemize}
        \item \textbf{Action:} Integrate an MFA solution (e.g., Windows Hello for Business, Duo, etc.) for all Windows, macOS, and Linux endpoint logins.
        \item \textbf{Impact:} Protects against unauthorized physical and remote access to user workstations, preventing an attacker's initial foothold from becoming a network-wide compromise.
    \end{itemize}
    \item \textbf{Mandate Security Training During Onboarding:}
    \begin{itemize}
        \item \textbf{Action:} Develop or procure a security awareness training module and integrate it into the mandatory onboarding checklist for all new hires, to be completed within their first week of employment.
        \item \textbf{Impact:} Equips new employees with the knowledge to recognize and report security threats from day one, reducing the organization's susceptibility to social engineering.
    \end{itemize}
\end{enumerate}

\end{document}
```