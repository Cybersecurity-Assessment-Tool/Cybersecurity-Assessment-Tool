```latex
\documentclass[12pt]{article}

% Preamble: Required Packages
\usepackage[margin=1in]{geometry}
\usepackage{pifont} % For checkmarks and crosses
\usepackage{booktabs} % For professional tables
\usepackage{hyperref} % For clickable links
\usepackage{url} % For URL formatting
\usepackage{seqsplit} % For splitting long strings
\usepackage{graphicx}
\usepackage{xcolor}

% Document Metadata
\title{Cybersecurity Posture Assessment Report}
\author{Cybersecurity Analysis Division}
\date{\today}

\begin{document}

\maketitle
\thispagestyle{empty}
\newpage
\tableofcontents
\newpage

% --- 1. EXECUTIVE OVERVIEW ---
\section{Executive Overview}

This report details a cybersecurity posture assessment for \textbf{[Organization Name]}. The analysis is based on a combination of a network perimeter scan, a review of organizational security controls, and an evaluation of pre-existing risk data.

The organization demonstrates a strong commitment to identity and access management, with Multi-Factor Authentication (MFA) consistently enforced across email, computer logins, and sensitive data systems. This significantly reduces the risk of unauthorized access through compromised credentials.

However, the assessment identified \textbf{critical deficiencies in administrative controls}. The absence of a formal Acceptable Use Policy (AUP) and a structured security awareness training program for employees presents a high level of risk. These gaps leave the organization highly susceptible to human error, social engineering, and insider threats.

On the technical front, the external network scan of the provided IP address (\texttt{[Client IP]}) revealed a secure perimeter with no open ports. This is a positive finding and indicates that a previously identified risk concerning an open web server on port 80 has been successfully remediated.

In summary, while technical access controls are robust, the primary areas of concern are foundational security policies and employee training. Immediate action is recommended to address these administrative gaps to build a more resilient and comprehensive security posture.

% --- 2. ORGANIZATIONAL INFORMATION ---
\section{Organizational Information}

The following details were used as the basis for this assessment. Due to the anonymized nature of the provided data, placeholders have been used where necessary.

\begin{itemize}
    \item \textbf{Organization Name:} \textbf{[Organization Name]}
    \item \textbf{Primary Domain:} \texttt{[Domain]}
    \item \textbf{External IP Scanned:} \texttt{[Client IP]}
\end{itemize}

% --- 3. SECURITY CONTROL REVIEW ---
\section{Security Control Review}

A review of key administrative and technical security controls was conducted via a questionnaire. The results below highlight areas of strength and significant weakness. "No" answers indicate a control gap that increases organizational risk.

\begin{table}[h!]
\centering
\caption{Organizational Security Control Status}
\label{tab:controls}
\begin{tabular}{p{0.8\linewidth} c}
\toprule
\textbf{Control Question} & \textbf{Status} \\
\midrule
Do you require MFA to access email? & \textcolor{green}{\ding{51}} \\
Do you require MFA to log into computers? & \textcolor{green}{\ding{51}} \\
Do you require MFA to access sensitive data systems? & \textcolor{green}{\ding{51}} \\
\midrule
Does your organization have an employee acceptable use policy? & \textcolor{red}{\ding{55}} \\
Does your organization do security awareness training for new employees? & \textcolor{red}{\ding{55}} \\
Does your organization do security awareness training for all employees at least once per year? & \textcolor{red}{\ding{55}} \\
\bottomrule
\end{tabular}
\end{table}

\textbf{Analysis:} The organization has effectively implemented MFA, a critical control for preventing account takeover attacks. However, the complete lack of a security training program or an acceptable use policy represents a critical failure in establishing a security-conscious culture.

% --- 4. TECHNICAL SCAN RESULTS ---
\section{Technical Scan Results}

An external network scan was performed to identify exposed services and potential vulnerabilities on the organization's network perimeter.

\begin{itemize}
    \item \textbf{Target IP:} \texttt{[Target IP]}
    \item \textbf{Scan Date:} \today
\end{itemize}

The scan results are summarized in the table below.

\begin{table}[h!]
\centering
\caption{Nmap Scan Results for \texttt{[Target IP]}}
\label{tab:nmap}
\begin{tabular}{l l l l}
\toprule
\textbf{Port} & \textbf{State} & \textbf{Service} & \textbf{Version} \\
\midrule
80 & closed & http & N/A \\
\bottomrule
\end{tabular}
\end{table}

\textbf{Analysis:} The scan of the target IP address revealed no open ports. The finding that port 80 is closed is a positive security outcome. This result directly contradicts a pre-existing risk entry (see Section 5), which stated that port 80 was open. This suggests that the previously identified vulnerability has been successfully remediated. The current configuration indicates a well-hardened network perimeter.

% --- 5. CONSOLIDATED RISK ASSESSMENT ---
\section{Consolidated Risk Assessment}

This section synthesizes findings from the security control review, the technical scan, and pre-existing risk data into a consolidated list of current risks.

\begin{table}[h!]
\centering
\caption{Risk Summary}
\label{tab:risks}
\begin{tabular}{p{0.1\linewidth} p{0.3\linewidth} p{0.4\linewidth} l}
\toprule
\textbf{ID} & \textbf{Risk Name} & \textbf{Description} & \textbf{Severity} \\
\midrule
RISK-001 & \textbf{Lack of Security Awareness Training} & Employees are not trained to identify or respond to security threats like phishing or social engineering. This significantly increases the likelihood of a breach originating from human error. & \textbf{High} \\
\addlinespace
RISK-002 & \textbf{No Acceptable Use Policy (AUP)} & Without a formal AUP, there are no established rules for employee use of company assets. This exposes the organization to insider threat, data misuse, and legal liabilities. & \textbf{High} \\
\addlinespace
RISK-003 & Unencrypted Web Server & \textit{Previous Finding:} Port 80 was believed to be open. \textbf{Current Status:} Our scan confirms Port 80 is closed. This risk is considered remediated. & \textbf{Closed} \\
\bottomrule
\end{tabular}
\end{table}

% --- 6. RECOMMENDATIONS ---
\section{Recommendations}

Based on the consolidated risk assessment, the following actions are recommended to improve the organization's security posture. Recommendations are prioritized by severity.

\subsection{Priority 1: High Severity Risks}

\begin{enumerate}
    \item \textbf{Establish a Security Awareness Training Program (RISK-001):}
    \begin{itemize}
        \item \textbf{Immediate Action:} Develop and deliver mandatory security awareness training for all current employees. This training must cover, at a minimum: phishing identification, password hygiene, and social engineering tactics.
        \item \textbf{Ongoing Action:} Integrate this training into the onboarding process for all new hires. Mandate annual refresher training for all staff to ensure continuous awareness.
    \end{itemize}
    \vspace{1em}
    \item \textbf{Develop and Implement an Acceptable Use Policy (RISK-002):}
    \begin{itemize}
        \item \textbf{Immediate Action:} Draft a formal AUP that clearly defines the rules for using company networks, devices, and data.
        \item \textbf{Ongoing Action:} Require all employees to read and formally acknowledge the policy. Review and update the AUP annually or as significant organizational changes occur.
    \end{itemize}
\end{enumerate}

\subsection{Priority 2: Informational / Administrative}

\begin{enumerate}
    \setcounter{enumi}{2}
    \item \textbf{Update Risk Register (RISK-003):}
    \begin{itemize}
        \item \textbf{Action:} Formally update the internal risk register to mark the "Unencrypted Web Server" vulnerability as \textbf{Closed/Remediated} based on the technical scan results dated \today. This ensures risk documentation remains accurate.
    \end{itemize}
\end{enumerate}

\end{document}
```