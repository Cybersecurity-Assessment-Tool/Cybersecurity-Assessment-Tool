```latex
\documentclass[12pt]{article}

% --- PACKAGES ---
\usepackage[margin=1in]{geometry}
\usepackage{pifont} % For checkmarks and crosses
\usepackage{booktabs} % For professional tables
\usepackage{hyperref} % For hyperlinks
\usepackage{url} % For URL formatting
\usepackage{seqsplit} % For splitting long strings in tt font
\usepackage{xcolor} % For colors

% --- DOCUMENT SETUP ---
\hypersetup{
    colorlinks=true,
    linkcolor=blue,
    filecolor=magenta,
    urlcolor=cyan,
    pdftitle={Cybersecurity Assessment Report},
    pdfauthor={Cybersecurity Analysis Division},
}

\newcommand{\yes}{\ding{51}}
\newcommand{\no}{\ding{55}}

% --- TITLE ---
\title{Cybersecurity Assessment Report}
\author{Cybersecurity Analysis Division}
\date{\today}

% --- BEGIN DOCUMENT ---
\begin{document}

\maketitle
\thispagestyle{empty}
\newpage
\tableofcontents
\newpage

% ===================================================================
\section{Executive Summary}
% ===================================================================
This report details the findings of a cybersecurity assessment for \textbf{[Organization Name]}. The analysis combines a review of organizational security controls, a technical network scan, and an evaluation of pre-existing risks.

The assessment identified a critical-risk vulnerability: the direct exposure of the Remote Desktop Protocol (RDP) on port 3389 to the public internet. This finding, confirmed by our technical scan, correlates with a known risk and presents a significant and immediate threat of unauthorized access and ransomware attacks.

Furthermore, significant gaps were identified in the organization's foundational security governance. The absence of a formal Acceptable Use Policy (AUP) and the lack of mandatory security awareness training for new employees create a high-risk environment susceptible to human error and policy violations.

While the organization has implemented Multi-Factor Authentication (MFA) across key systems, which is a commendable strength, the identified critical vulnerabilities in network configuration and security governance must be addressed immediately to reduce the overall risk profile to an acceptable level. This report provides actionable recommendations to remediate these findings.

% ===================================================================
\section{Organizational Information}
% ===================================================================
The following information was used as the basis for this assessment. Due to the anonymized nature of the provided data, placeholders have been used where necessary.

\begin{table}[h!]
\centering
\begin{tabular}{@{}ll@{}}
\toprule
\textbf{Attribute} & \textbf{Value} \\ \midrule
Organization Name & \textbf{[Organization Name]} \\
Primary Domain & \texttt{[Domain]} \\
External IP Address & \texttt{[Client IP]} \\ \bottomrule
\end{tabular}
\caption{Client Profile}
\end{table}

% ===================================================================
\section{Security Control Review}
% ===================================================================
A review of the organization's security controls was conducted via a questionnaire. The responses indicate a mixed level of maturity. While strong authentication controls are in place, there are critical deficiencies in policy and training.

\begin{table}[h!]
\centering
\begin{tabular}{@{}p{0.5\textwidth}cp{0.25\textwidth}@{}}
\toprule
\textbf{Control Question} & \textbf{Response} & \textbf{Assessment} \\ \midrule
Do you require MFA to access email? & \yes & Strong Control \\
Do you require MFA to log into computers? & \yes & Strong Control \\
Do you require MFA to access sensitive data systems? & \yes & Strong Control \\
\addlinespace
Does your organization have an employee acceptable use policy? & \no & \textcolor{red}{\textbf{Critical Gap}} \\
Does your organization do security awareness training for new employees? & \no & \textcolor{red}{\textbf{High Risk}} \\
Does your organization do security awareness training for all employees at least once per year? & \yes & Good Practice \\ \bottomrule
\end{tabular}
\caption{Security Controls Questionnaire Analysis}
\end{table}

% ===================================================================
\section{Technical Scan Results}
% ===================================================================
A network scan was performed to identify open ports and exposed services on the organization's external infrastructure. The scan confirmed the presence of a high-risk service exposed to the internet.

\begin{itemize}
    \item \textbf{Target IP Address:} \seqsplit{\texttt{[Target IP]}}
    \item \textbf{Scan Date:} Not provided in scan data.
\end{itemize}

\begin{table}[h!]
\centering
\begin{tabular}{@{}llll@{}}
\toprule
\textbf{Port} & \textbf{State} & \textbf{Service Name} & \textbf{Notes} \\ \midrule
3389/tcp & open & \texttt{ms-wbt-server} & Remote Desktop Protocol (RDP). \\
& & & A primary vector for ransomware. \\ \bottomrule
\end{tabular}
\caption{Open Ports Detected on \seqsplit{\texttt{[Target IP]}}}
\end{table}

The scan identified that port 3389, used for Microsoft's Remote Desktop Protocol (RDP), is open and accessible from the public internet. Exposing RDP directly is a highly dangerous practice and is actively exploited by threat actors to gain initial access to corporate networks for deploying ransomware and exfiltrating data.

% ===================================================================
\section{Consolidated Risk Assessment}
% ===================================================================
The following table synthesizes findings from the security control review, technical scan, and pre-existing risk data into a consolidated list of key risks facing the organization.

\begin{table}[h!]
\centering
\begin{tabular}{@{}p{0.2\textwidth}p{0.45\textwidth}p{0.1\textwidth}p{0.15\textwidth}@{}}
\toprule
\textbf{Risk Name} & \textbf{Description} & \textbf{Severity} & \textbf{Affected Systems} \\ \midrule
\textbf{Critical RDP Exposure} & The technical scan confirms that RDP (port 3389) is exposed to the internet, creating a direct path for attackers. This aligns with a known risk with a CVSS score of 9.0. & \textcolor{red}{Critical} & \seqsplit{\texttt{[Target IP]}} \\
\addlinespace
\textbf{Lack of Foundational Security Policy} & The organization lacks a formal Acceptable Use Policy (AUP), leading to an undefined security baseline for employee behavior and system usage. & \textcolor{orange}{High} & Organization-wide \\
\addlinespace
\textbf{Inadequate Employee Onboarding} & New employees do not receive security awareness training upon joining, leaving a critical window of vulnerability before the annual training cycle. & \textcolor{orange}{High} & Organization-wide (Human Element) \\ \bottomrule
\end{tabular}
\caption{Summary of Identified Risks}
\end{table}

% ===================================================================
\section{Recommendations}
% ===================================================================
The following actions are recommended to mitigate the identified risks. Recommendations are prioritized based on severity and potential impact.

\subsection{Immediate Priority: Remediate RDP Exposure}
The exposed RDP service poses an immediate and critical threat and must be addressed without delay.

\begin{itemize}
    \item \textbf{Short-Term Fix (Immediate):} Implement a firewall rule to \textbf{block all inbound traffic} to TCP port 3389 on \seqsplit{\texttt{[Target IP]}} from the public internet. Access should only be permitted from trusted, internal IP addresses if absolutely necessary.
    \item \textbf{Long-Term Solution (Urgent):} Deploy a secure remote access solution, such as a Virtual Private Network (VPN) or a Zero Trust Network Access (ZTNA) gateway. Access to internal resources, including RDP, should require users to connect through this secure channel, which must be protected with Multi-Factor Authentication (MFA).
\end{itemize}

\subsection{High Priority: Establish Security Governance}
A strong security posture begins with clear policies and guidelines.

\begin{itemize}
    \item \textbf{Develop an Acceptable Use Policy (AUP):} Draft and ratify a formal AUP that clearly defines the rules for using company assets, data, and network resources. This policy should be a prerequisite for system access.
    \item \textbf{Policy Acknowledgement:} Ensure all current and new employees read and formally acknowledge the AUP.
\end{itemize}

\subsection{High Priority: Enhance Security Awareness Program}
The human element is a critical component of the organization's defense.

\begin{itemize}
    \item \textbf{Implement Onboarding Training:} Create a mandatory security awareness training module that is a required part of the onboarding process for all new hires. This training should occur \textit{before} the employee is granted significant access to systems.
    \item \textbf{Training Content:} The onboarding module should cover, at a minimum: phishing awareness, password security, data handling, and key points from the new Acceptable Use Policy.
\end{itemize}

\end{document}
```