```latex
\documentclass[12pt]{article}

% Preamble: Required Packages
\usepackage[margin=1in]{geometry}
\usepackage{pifont} % For checkmarks and crosses
\usepackage{booktabs} % For professional tables
\usepackage{hyperref} % For hyperlinks
\usepackage{url} % For URL formatting
\usepackage{seqsplit} % To split long strings in texttt
\usepackage{xcolor} % For colors
\usepackage{graphicx} % For potential logos/images

% Hyperref Setup
\hypersetup{
    colorlinks=true,
    linkcolor=blue,
    filecolor=magenta,      
    urlcolor=cyan,
    pdftitle={Cybersecurity Assessment Report},
    pdfauthor={Cybersecurity Analyst},
    pdfsubject={Security Assessment},
    pdfkeywords={Security, Risk, Assessment},
}

% Define colors for severity
\definecolor{criticalred}{HTML}{D73B3E}
\definecolor{highorange}{HTML}{E48A52}
\definecolor{mediumyellow}{HTML}{F3D675}
\definecolor{lowblue}{HTML}{88A9C3}

% Checkmark and Cross definitions
\newcommand{\cmark}{\ding{51}}
\newcommand{\xmark}{\ding{55}}

\begin{document}

% --- Title Page ---
\begin{titlepage}
    \centering
    \vspace*{1cm}
    \Huge\textbf{Cybersecurity Assessment Report}
    \vspace{1.5cm}
    \Large
    \textbf{Prepared for:} \\
    \vspace{0.5cm}
    \Huge\textbf{[Organization Name]}
    \vspace{2cm}
    \large
    \textbf{Date of Report:} \today \\
    \vspace{0.5cm}
    \textbf{Report ID:} CSA-2023-001
    \vfill
    \large
    \textbf{Confidentiality Notice:} \\
    \textit{This document contains confidential and sensitive information. It is intended solely for the use of the individual or entity to whom it is addressed. Dissemination, distribution, or copying of this document, or any part of it, is strictly prohibited.}
\end{titlepage}

\tableofcontents
\newpage

% --- Executive Summary ---
\section*{1.0 Executive Summary}

This report provides a comprehensive cybersecurity assessment for \textbf{[Organization Name]}, conducted on \today. The analysis synthesizes data from an external network scan, a review of existing risks, and a security controls questionnaire.

The external network perimeter appears secure, with a network scan of the target IP address revealing no open ports. This indicates a strong firewall configuration and a minimal external attack surface, which is a significant security strength.

However, the internal security posture reveals critical gaps. The analysis of the security questionnaire identified three high-impact areas of concern:
\begin{itemize}
    \item \textbf{Lack of Multi-Factor Authentication (MFA) for Sensitive Data Systems:} This is a critical deficiency that exposes the organization's most valuable data to unauthorized access.
    \item \textbf{Absence of an Employee Acceptable Use Policy (AUP):} Without a formal AUP, there is no clear guidance for employees on the proper use of company assets, increasing the risk of insider threats and unintentional data breaches.
    \item \textbf{Lack of Annual Security Awareness Training:} While new hires receive training, the absence of a recurring, annual program for all employees allows security knowledge to decay, making the organization more susceptible to social engineering attacks like phishing.
\end{itemize}

Immediate action is recommended to address these policy and access control weaknesses to bolster the organization's overall defense-in-depth strategy.

% --- Organizational Information ---
\section*{2.0 Organizational Information}

This section details the information provided for the assessment.
\begin{table}[h!]
    \centering
    \begin{tabular}{@{}ll@{}}
        \toprule
        \textbf{Attribute} & \textbf{Value} \\
        \midrule
        Organization Name & \textbf{[Organization Name]} \\
        Primary Email Domain & \texttt{[Domain]} \\
        External IP Address Scanned & \texttt{[Client IP]} \\
        \bottomrule
    \end{tabular}
    \caption{Client Organizational Data}
    \label{tab:org_info}
\end{table}

% --- Security Control Review ---
\section*{3.0 Security Control Review}

The following table summarizes the organization's responses to the security controls questionnaire. Items marked with an \xmark\ represent significant gaps in the current security posture and are discussed in the Risk Assessment section.

\begin{table}[h!]
    \centering
    \begin{tabular}{@{}p{0.7\textwidth}c@{}}
        \toprule
        \textbf{Control Question} & \textbf{Status} \\
        \midrule
        Do you require MFA to access email? & \textcolor{green}{\cmark} \\
        Do you require MFA to log into computers? & \textcolor{green}{\cmark} \\
        Do you require MFA to access sensitive data systems? & \textcolor{red}{\xmark} \\
        Does your organization have an employee acceptable use policy? & \textcolor{red}{\xmark} \\
        Does your organization do security awareness training for new employees? & \textcolor{green}{\cmark} \\
        Does your organization do security awareness training for all employees at least once per year? & \textcolor{red}{\xmark} \\
        \bottomrule
    \end{tabular}
    \caption{Security Controls Questionnaire Results}
    \label{tab:controls}
\end{table}

% --- Technical Scan Results ---
\section*{4.0 Technical Scan Results}

An external network vulnerability scan was performed to identify potential weaknesses in the organization's internet-facing infrastructure.

\subsection*{4.1 Scan Summary}
\begin{itemize}
    \item \textbf{Target IP Address:} \texttt{[Target IP]}
    \item \textbf{Scan Date:} Not specified in scan data.
    \item \textbf{Result:} The scan completed successfully and found \textbf{zero open ports}. All 1000 scanned TCP ports were in a 'closed' state.
\end{itemize}

\subsection*{4.2 Analysis}
The absence of open ports is an excellent security finding. It indicates that the external firewall is properly configured to deny unsolicited inbound traffic, significantly reducing the external attack surface. No vulnerabilities were identified from this external scan.

% --- Pre-existing Risks ---
\section*{5.0 Review of Pre-existing Risks}
A review of the provided data on current, known vulnerabilities was conducted. The data indicated that there were \textbf{no pre-existing vulnerabilities} documented for this assessment.

% --- Risk Assessment ---
\section*{6.0 Risk Assessment Summary}

This section correlates the findings from the security control review and technical scans to present a summary of identified risks. The risks below are derived primarily from the policy and procedure gaps identified in Section 3.0.

\begin{table}[h!]
    \centering
    \begin{tabular}{@{}p{0.3\textwidth}p{0.5\textwidth}l@{}}
        \toprule
        \textbf{Risk Name} & \textbf{Overview} & \textbf{Severity} \\
        \midrule
        \textbf{Inadequate Access Control for Sensitive Data} & Sensitive data systems lack Multi-Factor Authentication (MFA), making them reliant on single-factor (password) authentication. A compromised password could lead directly to a major data breach. & \colorbox{criticalred}{\color{white}\textbf{CRITICAL}} \\
        \addlinespace
        \textbf{Lack of Formal Acceptable Use Policy (AUP)} & Without a documented AUP, employees lack clear rules regarding the use of company IT assets. This increases the likelihood of misuse, data leakage, and non-compliance with regulations. & \colorbox{highorange}{\color{white}\textbf{HIGH}} \\
        \addlinespace
        \textbf{Insufficient Security Awareness Training} & Security training is not conducted annually for all staff. This leads to a decline in security awareness, making employees more vulnerable to phishing, social engineering, and other common attacks. & \colorbox{highorange}{\color{white}\textbf{HIGH}} \\
        \bottomrule
    \end{tabular}
    \caption{Summary of Identified Risks}
    \label{tab:risks}
\end{table}

% --- Recommendations ---
\section*{7.0 Recommendations}

The following actionable recommendations are provided to mitigate the identified risks and improve the overall security posture of \textbf{[Organization Name]}.

\subsection*{7.1 Critical Recommendation}
\begin{itemize}
    \item \textbf{Implement MFA on All Sensitive Systems (Risk: Inadequate Access Control):}
    \begin{itemize}
        \item \textbf{Action:} Prioritize the deployment of a robust MFA solution across all applications, databases, and administrative interfaces that process or store sensitive data.
        \item \textbf{Impact:} Drastically reduces the risk of unauthorized access via compromised credentials. This is the single most effective control to implement to protect critical assets.
    \end{itemize}
\end{itemize}

\subsection*{7.2 High Priority Recommendations}
\begin{itemize}
    \item \textbf{Develop and Implement an Acceptable Use Policy (Risk: Lack of AUP):}
    \begin{itemize}
        \item \textbf{Action:} Draft a comprehensive AUP that clearly defines the rules for using company networks, devices, email, and internet access. The policy should be reviewed by legal counsel, approved by management, and distributed to all employees for acknowledgement.
        \item \textbf{Impact:} Establishes a clear security baseline for employee behavior, reduces insider risk, and provides a basis for disciplinary action in case of violation.
    \end{itemize}
    \vspace{1em}
    \item \textbf{Establish an Annual Security Awareness Training Program (Risk: Insufficient Training):}
    \begin{itemize}
        \item \textbf{Action:} Procure or develop a security awareness training program to be completed by all employees on an annual basis. The training should cover modern threats such as phishing, ransomware, and social engineering. Consider periodic phishing simulations to test and reinforce the training.
        \item \textbf{Impact:} Creates a more resilient "human firewall" by keeping security top-of-mind for all staff, reducing the likelihood of successful social engineering attacks.
    \end{itemize}
\end{itemize}

\end{document}
```