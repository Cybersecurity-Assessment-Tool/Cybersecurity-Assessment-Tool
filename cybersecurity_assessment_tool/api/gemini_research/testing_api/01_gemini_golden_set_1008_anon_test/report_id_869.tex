```latex
\documentclass[12pt]{article}

% Preamble: Required Packages
\usepackage[margin=1in]{geometry}
\usepackage{pifont} % For checkmarks (\ding{51}) and crosses (\ding{55})
\usepackage{booktabs} % For professional-looking tables
\usepackage{hyperref}
\usepackage{url}
\usepackage{seqsplit} % For breaking long, unbreakable strings
\usepackage{graphicx}
\usepackage{xcolor}
\usepackage{array}

% Hyperlink and Color Configuration
\hypersetup{
    colorlinks=true,
    linkcolor=blue,
    filecolor=magenta,
    urlcolor=cyan,
    pdftitle={Cybersecurity Posture Assessment Report},
    pdfauthor={Automated Security Analysis Engine},
}

% Custom Commands for Readability
\newcommand{\yes}{\textcolor{green}{\ding{51}}}
\newcommand{\no}{\textcolor{red}{\ding{55}}}
\newcommand{\severitycritical}{\textcolor{red}{\textbf{Critical}}}
\newcommand{\severityhigh}{\textcolor{orange}{\textbf{High}}}

% Document Information
\title{Cybersecurity Posture Assessment Report}
\author{Automated Security Analysis Engine}
\date{\today}

\begin{document}

\maketitle
\thispagestyle{empty}
\newpage

\tableofcontents
\newpage

% --- 1. Executive Summary ---
\section{Executive Summary}

This report provides a cybersecurity posture assessment for \textbf{[Organization Name]}. The analysis is based on a correlation of network scan data, a security controls questionnaire, and a list of pre-existing risks.

The assessment identified a \severitycritical{} risk posture, primarily driven by the direct exposure of Remote Desktop Protocol (RDP) on the public internet. This technical vulnerability is significantly amplified by critical gaps in organizational security controls, including the lack of Multi-Factor Authentication (MFA) for computer and sensitive system access, and the absence of security awareness training for new employees.

This combination of findings creates a high-likelihood attack path for unauthorized access, data breach, and ransomware deployment. Immediate remediation is required to mitigate these risks. This report details the specific findings and provides prioritized, actionable recommendations.

% --- 2. Organizational Information ---
\section{Organizational Information}

The following details were used as the basis for this assessment. Anonymized placeholders are used where data was not provided.

\begin{tabular}{@{}ll}
\toprule
\textbf{Detail} & \textbf{Value} \\
\midrule
Organization Name & \textbf{[Organization Name]} \\
Email Domain & \texttt{[Domain]} \\
External IP Scanned & \texttt{[Target IP]} \\
\bottomrule
\end{tabular}

% --- 3. Security Control Review (Questionnaire Analysis) ---
\section{Security Control Review}

A review of the organization's security controls was conducted via a questionnaire. The responses indicate significant gaps in identity and access management and employee security training. "No" answers represent deviations from security best practices and introduce substantial risk.

\begin{tabular}{p{0.7\linewidth} >{\centering\arraybackslash}p{0.2\linewidth}}
\toprule
\textbf{Control Question} & \textbf{Status} \\
\midrule
Do you require MFA to access email? & \yes \\
Do you require MFA to log into computers? & \no \\
Do you require MFA to access sensitive data systems? & \no \\
Does your organization have an employee acceptable use policy? & \yes \\
Does your organization do security awareness training for new employees? & \no \\
Does your organization do security awareness training for all employees at least once per year? & \yes \\
\bottomrule
\end{tabular}

\subsection*{Analysis of Gaps}
\begin{itemize}
    \item \textbf{MFA for Computers \& Sensitive Systems:} The absence of MFA on computer logins and sensitive data systems is a \severitycritical{} gap. It allows an attacker with a single set of stolen credentials to gain significant access to the internal network and confidential data.
    \item \textbf{New Employee Training:} The lack of security training during employee onboarding is a \severityhigh{} risk. New hires are often targeted by phishing and social engineering attacks and are less familiar with company security policies, making them a vulnerable entry point.
\end{itemize}

% --- 4. Technical Scan Results ---
\section{Technical Scan Results}

An external network scan was performed on the target IP address to identify exposed services. The scan confirmed the presence of an open RDP port, which aligns with the pre-existing risk data.

\begin{itemize}
    \item \textbf{Target IP Address:} \texttt{[Target IP]}
    \item \textbf{Scan Date:} Assumed from report generation date.
\end{itemize}

\begin{tabular}{lllll}
\toprule
\textbf{Port} & \textbf{State} & \textbf{Service} & \textbf{Product/Version} & \textbf{Notes} \\
\midrule
3389/tcp & open & ms-wbt-server & (Not Detected) & Remote Desktop Protocol (RDP) \\
\bottomrule
\end{tabular}

\subsection*{Analysis of Findings}
Exposing RDP (Port 3389) directly to the internet is a well-known and highly dangerous configuration. This service is a primary target for attackers who use brute-force password attacks, credential stuffing, and exploitation of RDP vulnerabilities (e.g., BlueKeep) to gain direct administrative control over a server or workstation. This is a common entry vector for ransomware attacks.

% --- 5. Correlated Risk Assessment ---
\section{Correlated Risk Assessment}

The following table synthesizes findings from the security questionnaire, the technical scan, and pre-existing risk data into a prioritized list.

\begin{tabular}{p{0.05\linewidth} p{0.45\linewidth} p{0.15\linewidth} p{0.25\linewidth}}
\toprule
\textbf{ID} & \textbf{Risk Description} & \textbf{Severity} & \textbf{Source} \\
\midrule
R-01 & \textbf{Direct RDP Exposure:} The Remote Desktop Protocol service is exposed to the public internet, allowing direct login attempts from any attacker. & \severitycritical{} & Technical Scan, Pre-existing Risk Data \\
\addlinespace
R-02 & \textbf{No MFA on Sensitive Systems:} Lack of a secondary authentication factor for sensitive data systems allows access with a single compromised password. & \severitycritical{} & Questionnaire \\
\addlinespace
R-03 & \textbf{No MFA on Computer Logins:} Lack of MFA on endpoint logins weakens the defense against stolen credentials, especially when combined with R-01. & \severityhigh{} & Questionnaire \\
\addlinespace
R-04 & \textbf{No Onboarding Security Training:} New employees are not trained on security best practices, increasing susceptibility to phishing and social engineering. & \severityhigh{} & Questionnaire \\
\bottomrule
\end{tabular}

% --- 6. Recommendations ---
\section{Recommendations}

The following actionable recommendations are provided to mitigate the identified risks. They are prioritized based on severity and ease of implementation.

\subsection{Immediate Actions (Critical Priority)}
\begin{enumerate}
    \item \textbf{Close Port 3389 Immediately:} The most critical action is to block all public internet access to TCP port 3389 on the firewall for the asset at \texttt{[Target IP]}. This will instantly remove the direct RDP exposure.
    \item \textbf{Enforce MFA on Sensitive Systems:} Immediately begin the rollout of MFA for all access to systems identified as containing sensitive data. This is a crucial compensating control against credential theft.
\end{enumerate}

\subsection{High Priority Actions}
\begin{enumerate}
    \item \textbf{Implement a Secure Remote Access Solution:} Replace direct RDP access with a Virtual Private Network (VPN). A VPN provides an encrypted tunnel and a single, managed point of entry that can be properly secured with MFA and logging.
    \item \textbf{Enforce MFA for all Computer Logins:} Deploy MFA for all user logins to company workstations and servers. This dramatically reduces the risk of lateral movement and unauthorized access from compromised endpoints.
    \item \textbf{Implement New Hire Security Training:} Integrate mandatory security awareness training into the employee onboarding process. This training should cover phishing, password hygiene, and acceptable use policies.
\end{enumerate}

\end{document}
```