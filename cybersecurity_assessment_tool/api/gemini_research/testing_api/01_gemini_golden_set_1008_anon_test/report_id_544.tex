```latex
\documentclass[12pt]{article}

% Preamble: Required Packages
\usepackage[margin=1in]{geometry}
\usepackage{pifont} % For symbols like checkmarks (\ding{51}) and crosses (\ding{55})
\usepackage{booktabs} % For professional-looking tables
\usepackage[hidelinks]{hyperref}
\usepackage{url}
\usepackage{seqsplit} % To split long strings in \texttt
\usepackage{graphicx}
\usepackage{fancyhdr}
\usepackage{lastpage}
\usepackage{xcolor}

% --- Document Setup ---

% Define colors for hyperlinks
\definecolor{darkblue}{rgb}{0.0, 0.0, 0.55}
\hypersetup{
    colorlinks=true,
    linkcolor=darkblue,
    filecolor=darkblue,      
    urlcolor=darkblue,
    citecolor=darkblue,
}

% Header and Footer Configuration
\pagestyle{fancy}
\fancyhf{} % Clear all header and footer fields
\lhead{Cybersecurity Assessment Report}
\rhead{\textbf{[Organization Name]}}
\cfoot{Page \thepage\ of \pageref{LastPage}}
\renewcommand{\headrulewidth}{0.4pt}
\renewcommand{\footrulewidth}{0.4pt}

% --- Document Body ---

\begin{document}

% --- Title Page ---
\begin{titlepage}
    \centering
    \vspace*{2cm}
    
    {\Huge \textbf{Cybersecurity Posture Assessment Report}\par}
    \vspace{1.5cm}
    
    {\Large Prepared for:\par}
    \vspace{0.5cm}
    {\Huge \textbf{[Organization Name]}\par}
    
    \vfill
    
    {\large \today\par}
    \vspace{0.5cm}
    {\large Report Version: 1.0\par}
    
\end{titlepage}

\newpage

% --- Table of Contents ---
\tableofcontents
\newpage

% --- Section 1: Executive Overview ---
\section*{1. Executive Overview}

This report provides a cybersecurity assessment for \textbf{[Organization Name]}, synthesizing data from organizational questionnaires, technical network scans, and a review of pre-existing risks. The assessment was conducted to identify security gaps and provide actionable recommendations to enhance the organization's security posture.

\paragraph{Key Findings:} The analysis reveals a mixed security posture. On one hand, the organization demonstrates strong foundational controls in identity and access management, with Multi-Factor Authentication (MFA) widely implemented across key systems. Furthermore, the external network perimeter appears hardened, as a network scan revealed no open ports, suggesting a well-configured firewall.

However, a critical deficiency was identified in the area of security awareness. The organization currently does not provide security training for new or existing employees. This represents a significant risk, as it leaves the organization highly vulnerable to human-centric attacks such as phishing, social engineering, and business email compromise, which can bypass even strong technical controls.

\paragraph{Primary Recommendation:} The immediate priority should be the development and implementation of a comprehensive security awareness training program.

% --- Section 2: Organizational & Scan Information ---
\section*{2. Organizational \& Scan Information}

This section details the organizational context and the scope of the technical scan. The information provided is based on the data supplied for this assessment.

\begin{table}[h!]
\centering
\begin{tabular}{@{}ll@{}}
\toprule
\textbf{Attribute} & \textbf{Value} \\
\midrule
Organization Name & \textbf{[Organization Name]} \\
Email Domain & \texttt{[Domain]} \\
External IP Address (Scanned) & \texttt{[Client IP]} \\
Technical Scan Target & \texttt{[Target IP]} \\
\bottomrule
\end{tabular}
\caption{Assessment Scope and Organizational Details.}
\end{table}

% --- Section 3: Security Control Review ---
\section*{3. Security Control Review}

A review of administrative and policy-based security controls was conducted via a questionnaire. The responses indicate areas of both strength and weakness. The absence of a security awareness program is the most critical finding from this review.

\begin{table}[h!]
\centering
\begin{tabular}{@{}p{0.7\linewidth}cc@{}}
\toprule
\textbf{Control Question} & \textbf{Response} & \textbf{Status} \\
\midrule
Do you require MFA to access email? & Yes & \ding{51} \\
Do you require MFA to log into computers? & Yes & \ding{51} \\
Do you require MFA to access sensitive data systems? & Yes & \ding{51} \\
Does your organization have an employee acceptable use policy? & Yes & \ding{51} \\
\addlinespace
\color{red}Does your organization do security awareness training for new employees? & \color{red}No & \color{red}\ding{55} \\
\color{red}Does your organization do security awareness training for all employees at least once per year? & \color{red}No & \color{red}\ding{55} \\
\bottomrule
\end{tabular}
\caption{Organizational Security Controls Questionnaire Results.}
\end{table}

% --- Section 4: Technical Scan Results ---
\section*{4. Technical Scan Results}

A network port scan was performed on the target host to identify exposed services. The results are summarized below.

\paragraph{Target:} \texttt{[Target IP]}
\paragraph{Scan Summary:} The scan confirmed that the target host is online and responsive. However, \textbf{no open TCP ports were discovered}. All scanned ports were found to be in a 'closed' state.

\paragraph{Analysis:} This is a positive security finding. It indicates a strong network perimeter defense, likely due to a properly configured firewall that denies unsolicited inbound traffic. This configuration significantly reduces the external attack surface of the organization.

% --- Section 5: Risk Assessment ---
\section*{5. Risk Assessment}

This section correlates the findings from the security control review and technical scans to identify and prioritize risks. Based on the assessment, the primary risks are not technical but are related to the human element of security. No pre-existing vulnerabilities were reported.

\begin{table}[h!]
\centering
\begin{tabular}{@{}p{0.1\linewidth}p{0.25\linewidth}p{0.4\linewidth}l@{}}
\toprule
\textbf{Risk ID} & \textbf{Risk Name} & \textbf{Overview} & \textbf{Severity} \\
\midrule
RISK-001 & Lack of Onboarding Security Training & New employees are not trained on security policies, safe computing habits, or threat identification. This makes them highly susceptible to phishing and social engineering from their first day. & \textbf{High} \\
\addlinespace
RISK-002 & Lack of Annual Security Training & Without regular training, employees' awareness of evolving threats diminishes. This increases the likelihood of successful attacks over time and fails to reinforce a security-conscious culture. & \textbf{High} \\
\bottomrule
\end{tabular}
\caption{Identified Security Risks.}
\end{table}

% --- Section 6: Recommendations ---
\section*{6. Recommendations}

The following recommendations are provided to address the identified risks and improve the overall security posture of \textbf{[Organization Name]}.

\begin{enumerate}
    \item \textbf{Implement Mandatory Onboarding Security Training (RISK-001):}
    \begin{itemize}
        \item \textbf{Action:} Develop a security awareness training module and integrate it into the new-hire onboarding process.
        \item \textbf{Details:} This training should cover, at a minimum: the acceptable use policy, data handling procedures, password security, and how to identify and report phishing attempts.
        \item \textbf{Impact:} Greatly reduces the risk of new employees falling victim to common cyberattacks.
    \end{itemize}
    \vspace{0.5cm}
    
    \item \textbf{Establish an Annual Security Awareness Program (RISK-002):}
    \begin{itemize}
        \item \textbf{Action:} Procure or develop an annual security training program for all staff, including periodic phishing simulations.
        \item \textbf{Details:} The program should provide a yearly refresher on key security topics and introduce new threat trends. Phishing tests should be used to measure effectiveness and provide targeted follow-up training to those who need it.
        \item \textbf{Impact:} Fosters a strong security culture and ensures that the "human firewall" remains a resilient defense layer against evolving threats.
    \end{itemize}
\end{enumerate}

\end{document}
```