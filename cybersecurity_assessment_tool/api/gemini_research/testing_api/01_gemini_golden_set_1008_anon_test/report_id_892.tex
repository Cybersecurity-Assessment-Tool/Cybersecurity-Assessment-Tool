```latex
\documentclass[12pt]{article}

% Preamble: Required Packages
\usepackage[margin=1in]{geometry}
\usepackage{pifont} % For check and cross marks (\ding)
\usepackage{booktabs} % For professional tables
\usepackage{hyperref}
\usepackage{url}
\usepackage{seqsplit} % To break long strings in \texttt

% Hyperlink Setup
\hypersetup{
    colorlinks=true,
    linkcolor=black,
    filecolor=magenta,
    urlcolor=blue,
}

% Custom Commands for Readability
\newcommand{\yes}{\ding{51}} % Green checkmark
\newcommand{\no}{\ding{55}}  % Red X

% ------------------- DOCUMENT START -------------------

\begin{document}

\title{Cybersecurity Posture Assessment Report}
\author{Cybersecurity Analysis Division}
\date{\today}
\maketitle

\hrule\vspace{1em}

% ===================================================================
\section*{Executive Summary}
% ===================================================================

This report provides a cybersecurity assessment for \textbf{[Organization Name]}, synthesizing data from a network perimeter scan, a security controls questionnaire, and a review of known risks.

The external network scan of the target IP address \texttt{[Client IP]} revealed a strong security posture; the host was online, but no open ports were detected. This indicates a well-configured firewall that effectively limits external attack surfaces.

However, the security controls review identified two high-risk internal gaps. The absence of Multi-Factor Authentication (MFA) on employee computers and the lack of a formal Acceptable Use Policy (AUP) present significant security risks. These gaps could expose the organization to threats from compromised credentials and insider misuse of assets.

While the network perimeter is secure, immediate action is required to address these critical internal control deficiencies to build a defense-in-depth security strategy.

% ===================================================================
\section{Organizational Information}
% ===================================================================

The following information was used for this assessment. Placeholders are used where data was not provided.

\begin{itemize}
    \item \textbf{Organization Name:} \textbf{[Organization Name]}
    \item \textbf{Primary Email Domain:} \texttt{[Domain]}
    \item \textbf{Scanned External IP:} \texttt{[Client IP]}
\end{itemize}

% ===================================================================
\section{Security Control Review}
% ===================================================================

The following table summarizes the organization's responses to a security controls questionnaire. Items marked with \no\ represent significant gaps in the current security framework.

\begin{table}[h!]
\centering
\begin{tabular}{p{0.8\textwidth}c}
\toprule
\textbf{Control Question} & \textbf{Response} \\
\midrule
Do you require MFA to access email? & \yes \\
Do you require MFA to log into computers? & \no \\
Do you require MFA to access sensitive data systems? & \yes \\
Does your organization have an employee acceptable use policy? & \no \\
Does your organization do security awareness training for new employees? & \yes \\
Does your organization do security awareness training for all employees at least once per year? & \yes \\
\bottomrule
\end{tabular}
\caption{Security Controls Questionnaire Results}
\end{table}

% ===================================================================
\section{Technical Scan Results}
% ===================================================================

An external network scan was performed on the provided target IP address.

\begin{itemize}
    \item \textbf{Target IP:} \texttt{[Target IP]}
    \item \textbf{Host Status:} Up
    \item \textbf{Scan Summary:} The scan confirmed the host was responsive. However, \textbf{no open TCP ports were detected}. All 1000 scanned ports were in a 'closed' state.
\end{itemize}

\noindent\textbf{Analysis:} This is a positive finding. A lack of open ports on an external-facing IP address significantly reduces the attack surface available to external adversaries and suggests a properly configured stateful firewall is in place.

% ===================================================================
\section{Risk Assessment}
% ===================================================================

The following table correlates findings from the security control review and technical scans. No pre-existing vulnerabilities were reported. The primary risks identified are related to internal policy and identity management.

\begin{table}[h!]
\centering
\begin{tabular}{p{0.25\textwidth}p{0.5\textwidth}l}
\toprule
\textbf{Risk Name} & \textbf{Overview} & \textbf{Severity} \\
\midrule
\textbf{Lack of MFA on Workstations} & User computers do not require Multi-Factor Authentication for login. A single compromised password could grant an attacker full access to an employee's workstation and any connected network resources. & \textbf{High} \\
\addlinespace
\textbf{Missing Acceptable Use Policy (AUP)} & The organization lacks a formal policy defining the acceptable use of company assets (computers, network, data). This creates ambiguity and increases the risk of insider threat, accidental data leakage, and non-compliant behavior. & \textbf{High} \\
\bottomrule
\end{tabular}
\caption{Identified Security Risks}
\end{table}

% ===================================================================
\section{Recommendations}
% ===================================================================

To mitigate the identified risks and improve the overall security posture, we recommend the following actions:

\begin{enumerate}
    \item \textbf{Implement MFA for Workstation Logins (High Priority):}
    \begin{itemize}
        \item \textbf{Action:} Deploy and enforce a mandatory MFA solution for all employee computer logins. This is a critical compensating control that protects against password theft and reuse.
        \item \textbf{Examples:} Solutions include Windows Hello for Business, YubiKey hardware tokens, or third-party identity providers like Duo or Okta.
    \end{itemize}
    \vspace{1em}
    \item \textbf{Develop and Enforce an Acceptable Use Policy (High Priority):}
    \begin{itemize}
        \item \textbf{Action:} Draft a comprehensive AUP that clearly outlines the rules and responsibilities for all users of company technology and data. The policy should cover topics such as data handling, internet usage, personal device usage, and security incident reporting.
        \item \textbf{Implementation:} Ensure the policy is formally communicated to all current and new employees, and require a signed acknowledgment of their understanding and compliance.
    \end{itemize}
\end{enumerate}

\end{document}
% -------------------- DOCUMENT END --------------------
```