```latex
\documentclass[12pt]{article}

% --- PACKAGE IMPORTS ---
\usepackage[margin=1in]{geometry} % Set page margins
\usepackage{pifont}               % For checkmarks and crosses (dingbats)
\usepackage{booktabs}             % For professional-looking tables
\usepackage{hyperref}             % For hyperlinks
\usepackage{url}                  % For formatting URLs
\usepackage{seqsplit}             % For splitting long strings without spaces

% --- DOCUMENT METADATA ---
\title{Cybersecurity Posture and Risk Assessment Report}
\author{Cybersecurity Analysis Division}
\date{\today}

% --- DOCUMENT START ---
\begin{document}

\maketitle
\hrule
\vspace{1em}

% ===================================================================
% SECTION 1: EXECUTIVE OVERVIEW
% ===================================================================
\section*{1.0 Executive Overview}

This report provides a comprehensive cybersecurity assessment for \textbf{[Organization Name]}, based on an analysis of network scan data, organizational security controls, and pre-existing risk registers. The assessment reveals a mixed security posture with several critical and high-risk gaps that require immediate attention.

While the organization has implemented important controls such as Multi-Factor Authentication (MFA) for email and sensitive data systems, significant weaknesses were identified in endpoint security and employee security awareness. Specifically, the absence of MFA for computer logins and the lack of a formal security training program present substantial risks.

Technical analysis identified an exposed management service (SSH) on the external network perimeter. When correlated with the identified policy gaps, this finding elevates the risk of unauthorized access and potential system compromise. This report details these findings and provides actionable recommendations to mitigate the identified risks and strengthen the overall security posture.

% ===================================================================
% SECTION 2: ORGANIZATIONAL INFORMATION
% ===================================================================
\section*{2.0 Organizational Information}

The following details have been used for this assessment. Based on the provided data, placeholder values have been used where specific information was not available.

\begin{itemize}
    \item \textbf{Organization Name:} \textbf{[Organization Name]}
    \item \textbf{Primary Domain:} \texttt{[Domain]}
    \item \textbf{Assessed External IP:} \texttt{[Client IP]}
\end{itemize}

% ===================================================================
% SECTION 3: SECURITY CONTROL REVIEW (QUESTIONNAIRE)
% ===================================================================
\section*{3.0 Security Control Review}

An assessment of the organization's security policies and controls was conducted via a standardized questionnaire. The results are summarized below. Items marked with a cross (\ding{55}) indicate a significant gap in security controls.

\begin{table}[h!]
\centering
\begin{tabular}{p{0.75\linewidth} c}
\toprule
\textbf{Control Question} & \textbf{Status} \\
\midrule
Do you require MFA to access email? & \ding{51} \\
Do you require MFA to log into computers? & \textbf{\color{red}\ding{55}} \\
Do you require MFA to access sensitive data systems? & \ding{51} \\
Does your organization have an employee acceptable use policy? & \ding{51} \\
Does your organization do security awareness training for new employees? & \textbf{\color{red}\ding{55}} \\
Does your organization do security awareness training for all employees at least once per year? & \textbf{\color{red}\ding{55}} \\
\bottomrule
\end{tabular}
\caption{Security Control Questionnaire Results}
\end{table}

\subsection*{3.1 Analysis of Control Gaps}
The review identified two primary areas of concern:

\begin{itemize}
    \item \textbf{Endpoint Security:} The lack of MFA for computer logins is a critical weakness. If an attacker compromises an employee's credentials (e.g., through phishing), they can gain direct access to the network and company resources without a second authentication factor.
    \item \textbf{Security Awareness:} The absence of a security awareness training program for both new and existing employees is a high-risk gap. This leaves the organization vulnerable to social engineering attacks, as employees may not be equipped to identify and respond to threats like phishing, malware, or business email compromise.
\end{itemize}

% ===================================================================
% SECTION 4: TECHNICAL SCAN RESULTS
% ===================================================================
\section*{4.0 Technical Scan Results}

A network scan was performed on the target system to identify open ports and exposed services.

\begin{itemize}
    \item \textbf{Target IP Address:} \texttt{[Target IP]}
    \item \textbf{Scan Status:} Host is UP
\end{itemize}

\begin{table}[h!]
\centering
\begin{tabular}{c c l l}
\toprule
\textbf{Port} & \textbf{State} & \textbf{Service (Inferred)} & \textbf{Notes} \\
\midrule
22/TCP & OPEN & SSH (Secure Shell) & Remote administrative access. \\
\bottomrule
\end{tabular}
\caption{Open Ports Detected on \texttt{[Target IP]}}
\end{table}

\subsection*{4.1 Analysis of Technical Findings}
The scan detected that port 22 (SSH) is open to the public internet. SSH is a common protocol for remote system administration. While necessary for management, its exposure on the external perimeter presents a significant risk. Attackers frequently scan for open SSH ports to perform brute-force password attacks or exploit potential vulnerabilities in the SSH server software. This finding is especially concerning when combined with the lack of MFA on computer logins.

% ===================================================================
% SECTION 5: CORRELATED RISK ASSESSMENT
% ===================================================================
\section*{5.0 Correlated Risk Assessment}

This section synthesizes findings from the security control review, technical scan, and pre-existing risk data to provide a holistic view of the organization's risk profile.

\begin{table}[h!]
\centering
\begin{tabular}{p{0.3\linewidth} p{0.5\linewidth} l}
\toprule
\textbf{Risk / Vulnerability} & \textbf{Description} & \textbf{Severity} \\
\midrule
\textbf{Localhost Exposed} & Pre-existing finding. A critical service intended for internal use only is exposed to the external network. & \textbf{Critical} \\
\addlinespace
\textbf{Lack of MFA on Endpoints} & User credentials are the single point of failure for computer access. A credential compromise could lead to a full network breach. & \textbf{Critical} \\
\addlinespace
\textbf{Exposed SSH Service} & The primary remote administration port is open to the internet, inviting automated attacks and targeted intrusion attempts. & \textbf{High} \\
\addlinespace
\textbf{No Security Awareness Training} & Employees are not trained to recognize or report security threats, making the organization highly susceptible to phishing and social engineering. & \textbf{High} \\
\bottomrule
\end{tabular}
\caption{Summary of Identified Risks}
\end{table}

% ===================================================================
% SECTION 6: RECOMMENDATIONS
% ===================================================================
\section*{6.0 Recommendations}

The following actions are recommended to mitigate the identified risks and improve the security posture of \textbf{[Organization Name]}.

\subsection*{6.1 Critical Priority}
\begin{enumerate}
    \item \textbf{Implement MFA for All Computer Logins:}
    \begin{itemize}
        \item \textbf{Action:} Deploy a robust MFA solution (e.g., Windows Hello for Business, Duo, Okta) for all workstation and server logins, both local and remote.
        \item \textbf{Impact:} Drastically reduces the risk of unauthorized access from compromised credentials.
    \end{itemize}
    \item \textbf{Remediate "Localhost Exposed" Vulnerability:}
    \begin{itemize}
        \item \textbf{Action:} Identify the service associated with this critical finding and reconfigure it to bind only to the loopback interface (127.0.0.1), making it inaccessible from the network.
        \item \textbf{Impact:} Eliminates a critical attack vector.
    \end{itemize}
\end{enumerate}

\subsection*{6.2 High Priority}
\begin{enumerate}
    \setcounter{enumi}{2} % Continue numbering from previous list
    \item \textbf{Establish a Security Awareness Training Program:}
    \begin{itemize}
        \item \textbf{Action:} Implement a mandatory training program for all new hires upon onboarding. Conduct annual refresher training and periodic phishing simulations for all employees.
        \item \textbf{Impact:} Creates a "human firewall" by empowering employees to identify and report threats, reducing the likelihood of a successful social engineering attack.
    \end{itemize}
    \item \textbf{Harden and Restrict SSH Access:}
    \begin{itemize}
        \item \textbf{Action:} If external SSH access is not required, block port 22 at the firewall. If it is required, restrict access to a whitelist of trusted IP addresses. Furthermore, enforce key-based authentication (disabling passwords) and, if possible, add an MFA layer to SSH logins.
        \item \textbf{Impact:} Minimizes the attack surface and protects a critical administrative entry point.
    \end{itemize}
\end{enumerate}

\end{document}
```