```latex
\documentclass[12pt]{article}

% Preamble: Required Packages
\usepackage[margin=1in]{geometry}
\usepackage{pifont} % For checkmarks and crosses
\usepackage{booktabs} % For professional tables
\usepackage{hyperref} % For clickable links
\usepackage{url} % For URL formatting
\usepackage{seqsplit} % To split long strings in tt font
\usepackage{graphicx}
\usepackage[table]{xcolor}
\usepackage{fancyhdr}

% --- Document Setup ---
\hypersetup{
    colorlinks=true,
    linkcolor=blue,
    filecolor=magenta,      
    urlcolor=cyan,
    pdftitle={Cybersecurity Posture Assessment Report},
    pdfpagemode=FullScreen,
}

\pagestyle{fancy}
\fancyhf{}
\lhead{Cybersecurity Posture Assessment}
\rhead{\textbf{[Organization Name]}}
\cfoot{\thepage}

\newcommand{\yes}{\ding{51}}
\newcommand{\no}{\ding{55}}

% --- Document Start ---
\begin{document}

% --- Title Page ---
\begin{titlepage}
    \centering
    \vspace*{1cm}
    \includegraphics[width=0.4\textwidth]{example-image-a} % Placeholder logo
    \vfill
    \Huge\bfseries
    Cybersecurity Posture Assessment Report
    \vspace{1cm}
    \Large
    \par
    \textbf{Prepared for:} \textbf{[Organization Name]}
    \vspace{1.5cm}
    \par
    \textbf{Date of Report:} \today
    \vfill
    \textit{This report contains sensitive information and should be handled with care.}
\end{titlepage}

\tableofcontents
\clearpage

% --- Section 1: Executive Summary ---
\section{Executive Summary}

This report provides a comprehensive assessment of the cybersecurity posture for \textbf{[Organization Name]}. The analysis is based on a combination of technical network scanning, a review of existing risks, and an evaluation of organizational security controls via a questionnaire.

\paragraph{Key Findings:}
The assessment revealed a mixed security posture. On the one hand, the external network scan of the target system \texttt{[Target IP]} showed a strong perimeter defense, with no open ports detected. This is a positive indicator of a well-configured firewall.

However, significant and critical gaps were identified in the organization's administrative and access control policies. The absence of Multi-Factor Authentication (MFA) for email and sensitive data systems represents a critical vulnerability. These gaps, coupled with the lack of a formal Acceptable Use Policy and security training for new employees, expose the organization to a high risk of account compromise, data breaches, and social engineering attacks.

\paragraph{Overall Assessment:}
While the network perimeter appears secure, the primary risks to \textbf{[Organization Name]} are currently internal and policy-related. Immediate action is required to address the identified gaps in access control and employee security awareness to mitigate the risk of a significant security incident.

\clearpage

% --- Section 2: Organizational Information ---
\section{Organizational Information}

This section details the information provided for the assessment.

\begin{tabular}{@{}ll}
    \toprule
    \textbf{Attribute} & \textbf{Value} \\
    \midrule
    Organization Name & \textbf{[Organization Name]} \\
    Primary Email Domain & \texttt{[Domain]} \\
    Client External IP & \texttt{[Client IP]} \\
    \bottomrule
\end{tabular}

\clearpage

% --- Section 3: Security Control Review ---
\section{Security Control Review (Questionnaire Analysis)}

The following table summarizes the responses to the security questionnaire and provides an assessment of each control. "No" answers indicate a gap in security controls and are highlighted as risks.

\begin{table}[h!]
\centering
\caption{Security Controls Questionnaire Results}
\begin{tabular}{@{}p{0.6\linewidth} c p{0.25\linewidth}@{}}
    \toprule
    \textbf{Control Question} & \textbf{Response} & \textbf{Assessment} \\
    \midrule
    Do you require MFA to access email? & \no & \textcolor{red}{\textbf{Critical Gap.}} Email is a primary target for attackers. \\
    \addlinespace
    Do you require MFA to log into computers? & \yes & Good Practice. \\
    \addlinespace
    Do you require MFA to access sensitive data systems? & \no & \textcolor{red}{\textbf{Critical Gap.}} Direct risk to confidential data. \\
    \addlinespace
    Does your organization have an employee acceptable use policy? & \no & \textcolor{orange}{\textbf{High Risk.}} Lack of clear guidelines for employees. \\
    \addlinespace
    Does your organization do security awareness training for new employees? & \no & \textcolor{orange}{\textbf{High Risk.}} New hires are a prime target for social engineering. \\
    \addlinespace
    Does your organization do security awareness training for all employees at least once per year? & \yes & Good Practice. \\
    \bottomrule
\end{tabular}
\end{table}

\clearpage

% --- Section 4: Technical Scan Results ---
\section{Technical Scan Results}

A network scan was performed to identify open ports and services exposed to the internet.

\begin{itemize}
    \item \textbf{Target IP Address:} \texttt{[Target IP]}
    \item \textbf{Scan Date:} \textbf{[Scan Date]} % Placeholder as it was not in the input
\end{itemize}

\subsection{Scan Summary}
The scan confirmed that the host at \texttt{[Target IP]} was online and responsive. However, \textbf{no open ports were discovered}. All other scanned ports were found to be in a 'closed' state, meaning they are accessible but have no application listening on them.

\subsection{Interpretation}
This is a positive security finding. It indicates that a robust firewall or security group policy is in place, effectively minimizing the external attack surface of the target system. No vulnerabilities related to exposed services could be identified.

\clearpage

% --- Section 5: Consolidated Risk Assessment ---
\section{Consolidated Risk Assessment}

This section consolidates findings from the security control review and technical scan into a prioritized list of risks. No pre-existing vulnerabilities were reported.

\begin{table}[h!]
\centering
\caption{Identified Risks}
\begin{tabular}{@{}p{0.1\linewidth} p{0.3\linewidth} p{0.4\linewidth} l@{}}
    \toprule
    \textbf{Risk ID} & \textbf{Risk Name} & \textbf{Description} & \textbf{Severity} \\
    \midrule
    R-01 & No MFA on Email & The absence of MFA on email accounts greatly increases the risk of unauthorized access via phishing or credential stuffing. & \cellcolor{red!25}Critical \\
    \addlinespace
    R-02 & No MFA on Sensitive Data Systems & Lack of MFA for sensitive systems allows an attacker with stolen credentials to directly access and exfiltrate core business data. & \cellcolor{red!25}Critical \\
    \addlinespace
    R-03 & No Acceptable Use Policy (AUP) & Without a formal AUP, employees may engage in risky behavior unknowingly, and the organization has no formal basis for enforcement. & \cellcolor{orange!25}High \\
    \addlinespace
    R-04 & No Onboarding Security Training & New employees are not trained on security best practices, making them highly susceptible to social engineering and policy violations. & \cellcolor{orange!25}High \\
    \bottomrule
\end{tabular}
\end{table}

\clearpage

% --- Section 6: Recommendations ---
\section{Recommendations}

Based on the risk assessment, the following actions are recommended to improve the security posture of \textbf{[Organization Name]}. Recommendations are prioritized by severity.

\subsection{Priority 1: Critical}
\begin{enumerate}
    \item \textbf{Implement MFA for Email Access (Risk R-01):} Immediately enforce MFA for all user access to the email system (e.g., O365, Google Workspace). This is the single most effective control to prevent email account takeovers.
    \item \textbf{Enforce MFA for Sensitive Systems (Risk R-02):} Identify all systems containing sensitive or critical data and enforce MFA for all user access, including administrative and remote access.
\end{enumerate}

\subsection{Priority 2: High}
\begin{enumerate}
    \setcounter{enumi}{2}
    \item \textbf{Develop and Implement an Acceptable Use Policy (Risk R-03):} Create a formal AUP that clearly defines the rules for using company assets, data, and networks. Require all employees to read and acknowledge the policy.
    \item \textbf{Integrate Security Training into Onboarding (Risk R-04):} Develop a mandatory security awareness training module for all new employees. This training should be completed during their first week and cover topics such as phishing, password security, and the new AUP.
\end{enumerate}

\subsection{Priority 3: Informational}
\begin{enumerate}
    \setcounter{enumi}{4}
    \item \textbf{Maintain Strong Network Perimeter:} Continue to maintain the current firewall configuration that limits exposed services on external-facing assets. Regularly review firewall rules to ensure they remain effective.
\end{enumerate}

\end{document}
```