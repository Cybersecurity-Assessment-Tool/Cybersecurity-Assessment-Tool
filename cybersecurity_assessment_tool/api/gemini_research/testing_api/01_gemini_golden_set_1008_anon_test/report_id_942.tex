```latex
\documentclass[12pt]{article}

% --- PACKAGES ---
\usepackage[margin=1in]{geometry}
\usepackage{pifont} % For checkmarks and crosses
\usepackage{booktabs} % For professional tables
\usepackage[hidelinks]{hyperref} % For clickable links
\usepackage{url} % For typesetting URLs
\usepackage{seqsplit} % To split long strings in tt font
\usepackage{graphicx}
\usepackage{xcolor}

% --- DOCUMENT DEFINITIONS ---
\newcommand{\yes}{\ding{51}}
\newcommand{\no}{\ding{55}}
\definecolor{darkred}{rgb}{0.55, 0.0, 0.0}
\definecolor{darkorange}{rgb}{1.0, 0.55, 0.0}
\definecolor{darkgreen}{rgb}{0.0, 0.39, 0.0}

% --- DOCUMENT START ---
\begin{document}

% --- TITLE PAGE ---
\begin{titlepage}
    \centering
    \vspace*{1cm}
    \Huge
    \textbf{Cybersecurity Posture Assessment Report}
    \vspace{1.5cm}
    \Large
    Prepared for: \\
    \vspace{0.5cm}
    \textbf{[Organization Name]}
    \vspace{2cm}
    \large
    Report Date: \today \\
    \vspace{0.5cm}
    Generated By: \\
    Expert Cybersecurity Analyst
    \vfill
    \textit{This report contains sensitive information and should be handled with care.}
\end{titlepage}

\tableofcontents
\newpage

% --- EXECUTIVE SUMMARY ---
\section{Executive Summary}

This report provides a comprehensive analysis of the cybersecurity posture for \textbf{[Organization Name]}, based on network scans, a security controls questionnaire, and a review of existing risks. The assessment was conducted on \today.

The external network scan of the target IP address \texttt{[Target IP]} revealed a positive security posture, with no open or vulnerable ports detected. Notably, Port 80, which was previously identified as a risk, was found to be closed. This indicates that the "Unencrypted Web Server" risk has likely been remediated.

However, the security controls review identified several critical gaps in internal security policies that significantly weaken the organization's defense against common cyber threats. The most severe findings include:
\begin{itemize}
    \item \textbf{Critical Risk:} The absence of Multi-Factor Authentication (MFA) on employee computers. This exposes the organization to a high risk of unauthorized access and lateral movement should an employee's credentials be compromised.
    \item \textbf{High Risk:} A complete lack of a security awareness training program for both new and existing employees. This makes the organization highly susceptible to phishing, social engineering, and other human-targeted attacks.
\end{itemize}

While the external perimeter appears secure, the internal policy gaps present an urgent call to action. We recommend prioritizing the implementation of endpoint MFA and the development of a comprehensive security awareness training program to mitigate these substantial risks.

% --- ORGANIZATIONAL INFORMATION ---
\section{Organizational Information}
The following details were used as the basis for this assessment. Due to the anonymized nature of the provided data, placeholders have been used where necessary.

\begin{table}[h!]
\centering
\begin{tabular}{@{}ll@{}}
\toprule
\textbf{Attribute} & \textbf{Value} \\ \midrule
Organization Name & \textbf{[Organization Name]} \\
Email Domain & \texttt{[Domain]} \\
External IP Scanned & \texttt{[Client IP]} \\ \bottomrule
\end{tabular}
\caption{Client Organizational Details}
\end{table}

% --- SECURITY CONTROL REVIEW ---
\section{Security Control Review}
The following table summarizes the organization's responses to a security controls questionnaire. Items marked with a red \no\ represent significant gaps in the security framework and are addressed in the Risk Assessment section.

\begin{table}[h!]
\centering
\begin{tabular}{@{}p{0.75\linewidth}c@{}}
\toprule
\textbf{Control Question} & \textbf{Status} \\ \midrule
Do you require MFA to access email? & \textcolor{darkgreen}{\yes} \\
\textbf{Do you require MFA to log into computers?} & \textcolor{darkred}{\no} \\
Do you require MFA to access sensitive data systems? & \textcolor{darkgreen}{\yes} \\
Does your organization have an employee acceptable use policy? & \textcolor{darkgreen}{\yes} \\
\textbf{Does your organization do security awareness training for new employees?} & \textcolor{darkred}{\no} \\
\textbf{Does your organization do security awareness training for all employees at least once per year?} & \textcolor{darkred}{\no} \\ \bottomrule
\end{tabular}
\caption{Security Controls Questionnaire Results}
\end{table}

% --- TECHNICAL SCAN RESULTS ---
\section{Technical Scan Results}
An Nmap scan was performed on the target host to identify open ports and exposed services. The target for this scan was \texttt{[Target IP]}.

\subsection{Scan Summary}
The scan results were positive, indicating a well-configured external firewall. No open ports were discovered. The status of a key port is detailed below.

\begin{table}[h!]
\centering
\begin{tabular}{@{}llll@{}}
\toprule
\textbf{Port} & \textbf{State} & \textbf{Service} & \textbf{Version} \\ \midrule
80/tcp & Closed & http & N/A \\ \bottomrule
\end{tabular}
\caption{Nmap Port Scan Details}
\end{table}

\subsection{Analysis}
The scan shows that Port 80 (HTTP) is closed. This is a strong security practice, as it prevents unencrypted web traffic. This finding directly contradicts a pre-existing risk entry ("Unencrypted Web Server"), suggesting that this vulnerability has been successfully remediated since it was last assessed. No other vulnerabilities were identified from this external scan.

% --- RISK ASSESSMENT ---
\section{Risk Assessment}
This section correlates findings from the security control review, the technical scan, and pre-existing risk data. The primary risks facing the organization are currently policy- and procedure-based rather than technical vulnerabilities on the external perimeter.

\begin{table}[h!]
\centering
\begin{tabular}{@{}p{0.25\linewidth}p{0.5\linewidth}l@{}}
\toprule
\textbf{Risk Name} & \textbf{Description} & \textbf{Severity} \\ \midrule
\textbf{Lack of Endpoint MFA} & User computers do not require MFA for login. If an employee's password is stolen (e.g., via phishing), an attacker can gain direct access to the endpoint and internal network. & \textcolor{darkred}{\textbf{Critical}} \\
\addlinespace
\textbf{Insufficient Security Training} & Employees receive no security awareness training. This significantly increases the likelihood of successful phishing, malware infection, and social engineering attacks, as staff are not equipped to identify threats. & \textcolor{darkorange}{\textbf{High}} \\
\addlinespace
\textbf{Unencrypted Web Server (Mitigated)} & A pre-existing risk stated Port 80 was open. Our scan on \today~found this port to be closed. This indicates the risk has been mitigated and the register should be updated. & Informational \\ \bottomrule
\end{tabular}
\caption{Consolidated Risk Summary}
\end{table}

% --- RECOMMENDATIONS ---
\section{Recommendations}
Based on the analysis, we provide the following actionable recommendations to improve the security posture of \textbf{[Organization Name]}. Recommendations are prioritized by severity.

\begin{enumerate}
    \item \textbf{Implement MFA for All Endpoints (Critical):}
    \begin{itemize}
        \item \textbf{Action:} Deploy a Multi-Factor Authentication solution for all employee computer logins (Windows, macOS, etc.). This is the single most effective control to prevent unauthorized access from compromised credentials.
        \item \textbf{Impact:} Drastically reduces the risk of account takeover and subsequent lateral movement within the network.
    \end{itemize}
    \vspace{0.5cm}
    \item \textbf{Establish a Security Awareness Training Program (High):}
    \begin{itemize}
        \item \textbf{Action:} Develop and mandate a security awareness training program. This program should be required for all new hires during onboarding and for all existing employees on an annual basis.
        \item \textbf{Content:} Training should cover, at a minimum, phishing identification, password hygiene, acceptable use policies, and incident reporting procedures.
        \item \textbf{Impact:} Creates a "human firewall" and reduces the organization's susceptibility to the most common cyber-attack vectors.
    \end{itemize}
    \vspace{0.5cm}
    \item \textbf{Update Risk Register (Informational):}
    \begin{itemize}
        \item \textbf{Action:} Formally validate that the closure of Port 80 is a permanent configuration. Update the internal risk register to reflect that the "Unencrypted Web Server" risk is now mitigated.
        \item \textbf{Impact:} Ensures risk documentation is accurate and reflects the current security posture.
    \end{itemize}
\end{enumerate}

\end{document}
```