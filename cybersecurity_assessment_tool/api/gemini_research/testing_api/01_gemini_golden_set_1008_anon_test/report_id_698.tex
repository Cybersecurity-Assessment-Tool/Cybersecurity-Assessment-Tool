```latex
\documentclass[12pt]{article}

% Preamble: Required Packages
\usepackage[margin=1in]{geometry}
\usepackage{pifont} % For checkmarks and crosses
\usepackage{booktabs} % For professional tables
\usepackage[hidelinks]{hyperref} % For clickable links
\usepackage{url} % For formatting URLs
\usepackage{seqsplit} % For splitting long strings in tt font
\usepackage{graphicx}
\usepackage{xcolor}
\usepackage{array}

% Define custom colors for severity
\definecolor{critical}{HTML}{990000}
\definecolor{high}{HTML}{DD4B39}
\definecolor{medium}{HTML}{F4B400}
\definecolor{low}{HTML}{4285F4}

% Document Metadata
\title{Cybersecurity Posture Assessment Report}
\author{Cybersecurity Analysis Division}
\date{\today}

\begin{document}

\maketitle
\thispagestyle{empty}
\newpage
\tableofcontents
\newpage

% --- 1. Executive Summary ---
\section{Executive Summary}
This report details the findings of a cybersecurity posture assessment conducted for \textbf{[Organization Name]}. The assessment combined an external network scan, a review of existing risks, and an analysis of a security controls questionnaire.

The overall security posture is considered weak, with several critical and high-risk vulnerabilities identified that require immediate attention. Key findings include a publicly accessible FTP server running a critically outdated and vulnerable version of vsftpd, which also permits anonymous logins. This poses a significant risk of unauthorized access and potential system compromise.

Furthermore, critical policy and procedural gaps were identified. The lack of Multi-Factor Authentication (MFA) on email accounts exposes the organization to a high risk of Business Email Compromise (BEC). The absence of an Acceptable Use Policy (AUP) and mandatory annual security training for all staff exacerbates the human element of risk, leaving the organization vulnerable to both insider threats and external social engineering attacks.

Immediate remediation of the technical vulnerabilities and implementation of the recommended security controls are crucial to mitigate these risks and improve the organization's defensive capabilities.

% --- 2. Organizational Information ---
\section{Organizational Information}
This section provides the context for the assessment based on the information provided.
\begin{itemize}
    \item \textbf{Organization Name:} \textbf{[Organization Name]}
    \item \textbf{Primary Email Domain:} \texttt{[Domain]}
    \item \textbf{Scanned External IP:} \texttt{[Client IP]}
\end{itemize}

% --- 3. Security Control Review ---
\section{Security Control Review}
The following table summarizes the organization's responses to a security controls questionnaire. "No" answers indicate significant gaps in the security framework and are highlighted as either Critical Gaps or High Risks.

\begin{table}[h!]
\centering
\caption{Security Controls Questionnaire Analysis}
\label{tab:controls}
\begin{tabular}{>{\raggedright\arraybackslash}p{6cm} >{\centering\arraybackslash}p{1.5cm} >{\raggedright\arraybackslash}p{6cm}}
\toprule
\textbf{Control Question} & \textbf{Response} & \textbf{Assessment} \\
\midrule
Do you require MFA to access email? & \ding{55} & \textbf{Critical Gap.} Email is a primary vector for account takeover and phishing. Lack of MFA significantly increases this risk. \\
\addlinespace
Do you require MFA to log into computers? & \ding{51} & \textbf{Good Practice.} Protects against unauthorized local access to workstations. \\
\addlinespace
Do you require MFA to access sensitive data systems? & \ding{51} & \textbf{Good Practice.} An essential control for protecting critical organizational data. \\
\addlinespace
Does your organization have an employee acceptable use policy? & \ding{55} & \textbf{High Risk.} Lack of a formal AUP creates ambiguity regarding the proper use of IT assets and increases insider threat risk. \\
\addlinespace
Does your organization do security awareness training for new employees? & \ding{51} & \textbf{Good Practice.} Establishes a security baseline for new hires. \\
\addlinespace
Does your organization do security awareness training for all employees at least once per year? & \ding{55} & \textbf{High Risk.} Security knowledge degrades over time. Without recurring training, staff are more susceptible to evolving threats like phishing. \\
\bottomrule
\end{tabular}
\end{table}

% --- 4. Technical Vulnerability Scan Results ---
\section{Technical Vulnerability Scan Results}
An external network scan was performed against the target IP address \texttt{[Target IP]}. The scan identified the following open ports and services.

\begin{table}[h!]
\centering
\caption{Open Port Analysis}
\label{tab:nmap}
\begin{tabular}{l l l l l}
\toprule
\textbf{Port} & \textbf{State} & \textbf{Service} & \textbf{Version} & \textbf{Details} \\
\midrule
21/tcp & open & ftp & vsftpd 2.3.4 & \parbox[t]{6cm}{\textbf{CRITICAL FINDING:}\\ Anonymous FTP login is allowed. This version is dangerously outdated and contains a known backdoor vulnerability (CVE-2011-2523) allowing remote command execution.} \\
\bottomrule
\end{tabular}
\end{table}

% --- 5. Consolidated Risk Assessment ---
\section{Consolidated Risk Assessment}
This section synthesizes findings from the technical scan, control review, and pre-existing risk data into a consolidated list of identified risks.

\begin{table}[h!]
\centering
\caption{Summary of Identified Risks}
\label{tab:risks}
\begin{tabular}{p{2.5cm} p{8.5cm} c}
\toprule
\textbf{Risk Name} & \textbf{Description} & \textbf{Severity} \\
\midrule
\addlinespace
Vulnerable FTP Service & A public-facing FTP server is running a version with a known remote code execution vulnerability (CVE-2011-2523). & \textcolor{critical}{\textbf{Critical}} \\
\addlinespace
Anonymous FTP Access & The FTP server allows anonymous login, enabling unauthorized users to access, upload, or download files, potentially leading to data leakage or malware distribution. & \textcolor{critical}{\textbf{Critical}} \\
\addlinespace
No MFA for Email & Lack of Multi-Factor Authentication on email accounts significantly increases the risk of Business Email Compromise (BEC) and account takeover. & \textcolor{critical}{\textbf{Critical}} \\
\addlinespace
Missing Acceptable Use Policy & The absence of a formal AUP leaves the organization legally and operationally exposed to misuse of IT assets by employees. & \textcolor{high}{\textbf{High}} \\
\addlinespace
Lack of Annual Security Training & Without recurring training, employees are more susceptible to phishing and social engineering attacks, making them a weak link in the security chain. & \textcolor{high}{\textbf{High}} \\
\addlinespace
Outdated Windows Policy (Pre-existing) & Workstations are running an unsupported OS (Windows 7), which no longer receives security updates and is vulnerable to exploitation. & \textcolor{medium}{\textbf{Medium}} \\
\addlinespace
\bottomrule
\end{tabular}
\end{table}

% --- 6. Recommendations ---
\section{Recommendations}
The following actions are recommended to mitigate the identified risks. Recommendations are prioritized based on severity.

\subsection{Critical Risk Remediation}
\begin{itemize}
    \item \textbf{Vulnerable FTP Server (vsftpd 2.3.4):}
    \begin{itemize}
        \item \textbf{Immediate:} Take the FTP server offline immediately to prevent exploitation.
        \item \textbf{Short-Term:} If FTP is a business necessity, migrate the service to a new server with the latest, patched version of a secure file transfer application (e.g., a modern vsftpd or ProFTPD).
        \item \textbf{Long-Term:} Evaluate the business need for FTP. If possible, replace it with a more secure protocol like SFTP (SSH File Transfer Protocol).
    \end{itemize}
    \item \textbf{Anonymous FTP Access:}
    \begin{itemize}
        \item \textbf{Immediate:} When rebuilding the FTP service, ensure anonymous login is disabled. All access must be authenticated with strong, unique credentials.
    \end{itemize}
    \item \textbf{No MFA for Email:}
    \begin{itemize}
        \item \textbf{Immediate:} Procure and enforce MFA for all user accounts on the email system. Prioritize deployment for administrative and executive accounts.
    \end{itemize}
\end{itemize}

\subsection{High Risk Remediation}
\begin{itemize}
    \item \textbf{Missing Acceptable Use Policy (AUP):}
    \begin{itemize}
        \item \textbf{Short-Term:} Draft and implement a comprehensive AUP that clearly defines the rules for using company IT assets. Require all employees to read and formally acknowledge the policy.
    \end{itemize}
    \item \textbf{Lack of Annual Security Training:}
    \begin{itemize}
        \item \textbf{Short-Term:} Schedule and conduct mandatory security awareness training for all current employees. Incorporate this training into an annual requirement for all staff.
    \end{itemize}
\end{itemize}

\subsection{Medium Risk Remediation}
\begin{itemize}
    \item \textbf{Outdated Windows Policy:}
    \begin{itemize}
        \item \textbf{Short-Term:} Ensure all Windows 7 machines are isolated from critical network segments to limit their potential impact if compromised.
        \item \textbf{Long-Term:} Execute the existing plan to upgrade or replace all hardware running Windows 7 with a modern, supported operating system (e.g., Windows 11).
    \end{itemize}
\end{itemize}

\end{document}
```