```latex
\documentclass[12pt]{article}

% --- PACKAGE IMPORTS ---
\usepackage[margin=1in]{geometry} % Set page margins
\usepackage{pifont}               % For checkmark and X symbols (\ding)
\usepackage{booktabs}             % For professional-looking tables
\usepackage{hyperref}             % For hyperlinks, adds metadata
\usepackage{url}                  % For formatting URLs
\usepackage{seqsplit}             % For splitting long strings in texttt
\usepackage[T1]{fontenc}          % For proper font encoding

% --- DOCUMENT METADATA ---
\hypersetup{
    colorlinks=true,
    linkcolor=black,
    urlcolor=blue,
    pdftitle={Cybersecurity Posture Assessment Report},
    pdfauthor={Cybersecurity Analysis Cell},
    pdfsubject={Security Assessment},
    pdfkeywords={Cybersecurity, Nmap, Risk Assessment}
}

% --- DOCUMENT START ---
\begin{document}

% --- TITLE PAGE ---
\title{
    \vspace{2cm}
    \textbf{Cybersecurity Posture Assessment Report} \\
    \large \textit{Confidential}
    \vspace{1cm}
}
\author{Cybersecurity Analysis Cell}
\date{\today}
\maketitle
\thispagestyle{empty}
\newpage

% --- TABLE OF CONTENTS ---
\tableofcontents
\newpage

% --- EXECUTIVE SUMMARY ---
\section*{Executive Summary}

This report details the findings of a cybersecurity posture assessment conducted for \textbf{[Organization Name]}. The analysis correlates data from an external network scan, a security controls questionnaire, and a list of pre-existing risks.

The assessment identified several critical and high-risk vulnerabilities that require immediate attention. Key findings include:
\begin{itemize}
    \item \textbf{Critical Database Exposure:} A MySQL database service is publicly exposed to the internet. The running version, MySQL 5.7.33, is past its End-of-Life (EOL) and contains known vulnerabilities, significantly increasing the risk of a data breach.
    \item \textbf{Critical Identity and Access Gap:} Multi-Factor Authentication (MFA) is not enforced for email access. This represents a severe weakness, as compromised credentials could lead to Business Email Compromise (BEC), data exfiltration, and further network intrusion.
    \item \textbf{High-Risk Onboarding Process:} New employees do not receive mandatory security awareness training, leaving the organization vulnerable to social engineering and phishing attacks from its inception.
\end{itemize}

Immediate remediation of the exposed database and the MFA gap is strongly recommended to mitigate the most severe risks. Further recommendations are provided to address all identified weaknesses and improve the organization's overall security posture.

% --- ORGANIZATIONAL INFORMATION ---
\section*{Organizational Information}
This report is based on the information provided by the client and discovered during the assessment.
\begin{itemize}
    \item \textbf{Organization Name:} \textbf{[Organization Name]}
    \item \textbf{Primary Domain:} \texttt{[Domain]}
    \item \textbf{Scanned External IP:} \texttt{[Client IP]}
\end{itemize}

% --- SECURITY CONTROL REVIEW ---
\section*{Security Control Review}
The following table summarizes the organization's responses to a security controls questionnaire. "No" answers indicate significant gaps in the security framework.

\begin{table}[h!]
\centering
\caption{Security Controls Questionnaire Analysis}
\begin{tabular}{p{0.6\linewidth} c l}
\toprule
\textbf{Control Question} & \textbf{Response} & \textbf{Assessment} \\
\midrule
Do you require MFA to access email? & \ding{55} & \textbf{Critical Gap} \\
Do you require MFA to log into computers? & \ding{51} & Satisfactory \\
Do you require MFA to access sensitive data systems? & \ding{51} & Satisfactory \\
Does your organization have an employee acceptable use policy? & \ding{51} & Satisfactory \\
Does your organization do security awareness training for new employees? & \ding{55} & \textbf{High Risk} \\
Does your organization do security awareness training for all employees at least once per year? & \ding{51} & Satisfactory \\
\bottomrule
\end{tabular}
\end{table}

% --- TECHNICAL SCAN RESULTS ---
\section*{Technical Scan Results}
An external network scan was performed on the target IP address \texttt{[Target IP]}. The scan identified the following open port and running service.

\subsection*{Open Ports and Services}
\begin{table}[h!]
\centering
\caption{Nmap Scan Findings}
\begin{tabular}{l l l l}
\toprule
\textbf{Port} & \textbf{State} & \textbf{Service} & \textbf{Version} \\
\midrule
3306/tcp & Open & mysql & MySQL 5.7.33 \\
\bottomrule
\end{tabular}
\end{table}

\subsection*{Technical Analysis}
The scan confirms that a MySQL database on port 3306 is directly accessible from the public internet. This configuration is highly discouraged as it exposes the database to brute-force attacks, credential stuffing, and exploitation of vulnerabilities.

Furthermore, the identified version, \textbf{MySQL 5.7.33}, reached its official End-of-Life (EOL) in October 2023. EOL software no longer receives security patches from the vendor, meaning any newly discovered vulnerabilities will remain unpatched. This version is known to be vulnerable to multiple CVEs, making its public exposure a critical risk. This finding directly corroborates the pre-existing risk "Database Exposure".

% --- SYNTHESIZED RISK ASSESSMENT ---
\section*{Synthesized Risk Assessment}
The following table synthesizes findings from the security questionnaire, technical scan, and pre-existing risk data into a prioritized list of security risks.

\begin{table}[h!]
\centering
\caption{Summary of Identified Risks}
\begin{tabular}{p{0.1\linewidth} p{0.25\linewidth} p{0.45\linewidth} l}
\toprule
\textbf{ID} & \textbf{Risk Title} & \textbf{Description} & \textbf{Severity} \\
\midrule
\textbf{RISK-001} & Publicly Exposed \& Outdated Database & A MySQL 5.7.33 database (CVSS 7.5) is exposed on port 3306. The software is End-of-Life, unpatched, and vulnerable to remote attack, posing a severe risk of data breach. & \textbf{Critical} \\
\addlinespace
\textbf{RISK-002} & No MFA for Email Access & Lack of MFA on email accounts makes them highly susceptible to compromise via phishing or password reuse. This can lead to Business Email Compromise (BEC) and data loss. & \textbf{Critical} \\
\addlinespace
\textbf{RISK-003} & Insufficient Onboarding Security Training & New employees are not trained on security best practices, making them prime targets for social engineering and phishing attacks from their first day. & \textbf{High} \\
\bottomrule
\end{tabular}
\end{table}

% --- RECOMMENDATIONS ---
\section*{Recommendations}
The following actions are recommended to mitigate the identified risks and strengthen the organization's security posture.

\subsection*{Immediate Actions (Critical Priority)}
\begin{enumerate}
    \item \textbf{Restrict Database Access (RISK-001):} Immediately implement firewall rules to block all public ingress traffic to TCP port 3306 on \texttt{[Target IP]}. Access should only be permitted from trusted internal IP addresses.
    \item \textbf{Enforce MFA on Email (RISK-002):} Immediately enable and enforce Multi-Factor Authentication for all user accounts accessing the email system.
\end{enumerate}

\subsection*{Short-Term Actions (High Priority)}
\begin{enumerate}
    \item \textbf{Upgrade Database Server (RISK-001):} Plan and execute an upgrade of the MySQL 5.7.33 server to a currently supported version (e.g., MySQL 8.x). This is essential to receive security patches.
    \item \textbf{Implement New Hire Training (RISK-003):} Develop and integrate a mandatory security awareness training module into the new employee onboarding process. This should cover phishing, acceptable use, and password hygiene.
\end{enumerate}

\subsection*{Long-Term Strategic Actions}
\begin{enumerate}
    \item \textbf{Implement Secure Database Access (RISK-001):} For remote database administration, establish a secure access method such as a Virtual Private Network (VPN) or a bastion host (jump box) instead of direct exposure.
    \item \textbf{Establish a Vulnerability Management Program:} Implement a regular schedule of internal and external vulnerability scanning to proactively identify and remediate security weaknesses before they can be exploited.
\end{enumerate}

\end{document}
```