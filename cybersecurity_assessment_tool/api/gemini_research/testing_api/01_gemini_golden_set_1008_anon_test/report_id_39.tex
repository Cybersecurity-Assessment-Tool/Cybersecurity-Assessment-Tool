```latex
\documentclass[12pt]{article}

% -----------------------------------------------------------------------------
% PREAMBLE
% -----------------------------------------------------------------------------
\usepackage[margin=1in]{geometry}
\usepackage[T1]{fontenc}
\usepackage{pifont} % For checkmarks and crosses (\ding)
\usepackage{booktabs} % For professional tables
\usepackage{xcolor} % For coloring text
\usepackage{graphicx} % For potential logos/images
\usepackage{hyperref} % For hyperlinks and metadata
\usepackage{url} % For formatting URLs
\usepackage{seqsplit} % To split long strings in tt font

% --- Custom Commands & Colors ---
\newcommand{\yes}{\ding{51}} % Green checkmark
\newcommand{\no}{\ding{55}}  % Red X
\definecolor{darkred}{rgb}{0.55, 0.0, 0.0}
\definecolor{darkorange}{rgb}{0.8, 0.33, 0.0}
\definecolor{darkyellow}{rgb}{0.7, 0.5, 0.0}

% --- Hyperref Setup ---
\hypersetup{
    colorlinks=true,
    linkcolor=blue,
    filecolor=magenta,      
    urlcolor=cyan,
    pdftitle={Cybersecurity Assessment Report},
    pdfauthor={Cybersecurity Analyst},
    pdfsubject={Security Assessment},
    pdfkeywords={Security, Assessment, Report},
    bookmarks=true,
    breaklinks=true
}

% -----------------------------------------------------------------------------
% DOCUMENT START
% -----------------------------------------------------------------------------
\begin{document}

% --- TITLE PAGE ---
\begin{titlepage}
    \centering
    \vspace*{1cm}
    \Huge\textbf{Cybersecurity Assessment Report}
    \vspace{1.5cm}
    \Large
    \textbf{Prepared for:} \\
    \vspace{0.5cm}
    \textbf{[Organization Name]}
    \vspace{2.5cm}
    \large
    \textbf{Date of Report:} \today \\
    \vspace{0.5cm}
    \textbf{Generated By:} Cybersecurity Analyst
    \vfill
    \textit{This report contains sensitive information and is intended solely for the use of the recipient organization. Distribution without prior consent is prohibited.}
\end{titlepage}

\tableofcontents
\newpage

% =============================================================================
\section{Executive Summary}
% =============================================================================
This report outlines the findings of a cybersecurity assessment conducted for \textbf{[Organization Name]}. The assessment combined an automated network scan, a review of existing risk documentation, and an analysis of organizational security controls via a questionnaire.

The analysis revealed a \textbf{critical risk posture}. Key findings include:
\begin{itemize}
    \item \textbf{Exposed End-of-Life Database:} An externally accessible MySQL database (\texttt{5.7.33}) was identified. This version is End-of-Life (EOL) and no longer receives security updates, making it a prime target for attackers. This finding directly correlates with a known risk, "Database Exposure."
    \item \textbf{Systemic Lack of Multi-Factor Authentication (MFA):} MFA is not enforced for email, computer logins, or access to sensitive data systems. This represents a severe control gap that significantly increases the risk of account compromise and unauthorized access.
    \item \textbf{Absence of Foundational Policies:} The organization lacks a formal Acceptable Use Policy, which is a fundamental component of a mature security program.
\end{itemize}

Immediate and decisive action is required to remediate these vulnerabilities. This report provides a detailed breakdown of the findings and a prioritized list of actionable recommendations to improve the organization's security posture.

% =============================================================================
\section{Organizational Information}
% =============================================================================
The following information was used as the basis for this assessment. As per the provided data, placeholder values are used where specific details were not available.

\begin{itemize}
    \item \textbf{Organization Name:} \textbf{[Organization Name]}
    \item \textbf{Primary Domain:} \texttt{[Domain]}
    \item \textbf{External IP Scanned:} \texttt{[Client IP]}
\end{itemize}

% =============================================================================
\section{Security Control Review}
% =============================================================================
The following table summarizes the organization's responses to the security controls questionnaire. Each "No" response indicates a significant gap in the security framework and has been flagged as a high or critical risk.

\begin{table}[h!]
\centering
\caption{Security Controls Questionnaire Analysis}
\label{tab:controls}
\begin{tabular}{@{}p{0.55\textwidth} c p{0.25\textwidth}@{}}
\toprule
\textbf{Control Question} & \textbf{Response} & \textbf{Analyst Assessment} \\
\midrule
Do you require MFA to access email? & \textcolor{darkred}{\no} & \textbf{Critical Risk}. Email is a primary target for phishing and account takeover. \\
\addlinespace
Do you require MFA to log into computers? & \textcolor{darkred}{\no} & \textbf{High Risk}. Lack of MFA on endpoints allows lateral movement after a credential compromise. \\
\addlinespace
Do you require MFA to access sensitive data systems? & \textcolor{darkred}{\no} & \textbf{Critical Risk}. This is a direct threat to the confidentiality and integrity of core data assets. \\
\addlinespace
Does your organization have an employee acceptable use policy? & \textcolor{darkred}{\no} & \textbf{High Risk}. Lack of a formal policy creates ambiguity and increases insider threat risk. \\
\addlinespace
Does your organization do security awareness training for new employees? & \textcolor{green}{\yes} & Good Practice. Foundational training is in place. \\
\addlinespace
Does your organization do security awareness training for all employees at least once per year? & \textcolor{green}{\yes} & Good Practice. Ongoing training reinforces security concepts. \\
\bottomrule
\end{tabular}
\end{table}

% =============================================================================
\section{Technical Scan Results}
% =============================================================================
An external network scan was performed against the target IP address \texttt{[Target IP]}. The scan identified one open port with a publicly accessible service.

\subsection{Open Ports and Services}
The following service was found to be exposed to the public internet.

\begin{table}[h!]
\centering
\caption{Identified Open Ports}
\label{tab:ports}
\begin{tabular}{@{}lllll@{}}
\toprule
\textbf{Port} & \textbf{State} & \textbf{Service} & \textbf{Product} & \textbf{Version} \\
\midrule
3306/tcp & open & mysql & MySQL & 5.7.33 \\
\bottomrule
\end{tabular}
\end{table}

\subsection{Vulnerability Analysis}
\textbf{End-of-Life Software Detected:} The identified MySQL version, \texttt{5.7.33}, reached its official End-of-Life (EOL) in October 2023. EOL software no longer receives security patches, bug fixes, or technical support from the vendor. Any vulnerabilities discovered in this version will remain unpatched, leaving the system perpetually vulnerable to exploitation.

Exposing any database directly to the internet is a highly dangerous practice. Exposing an EOL database presents an extreme and unacceptable level of risk.

% =============================================================================
\section{Consolidated Risk Assessment}
% =============================================================================
This section synthesizes findings from the security questionnaire, the technical scan, and pre-existing risk documentation into a consolidated list of identified risks.

\begin{table}[h!]
\centering
\caption{Summary of Identified Risks}
\label{tab:risks}
\begin{tabular}{@{}p{0.3\textwidth} p{0.55\textwidth} l@{}}
\toprule
\textbf{Risk Title} & \textbf{Description} & \textbf{Severity} \\
\midrule
\textbf{Exposed End-of-Life Database} & The MySQL database on port 3306 is publicly accessible and runs on an unsupported, EOL version (5.7.33). This exposes the organization to unpatchable vulnerabilities. & \textcolor{darkred}{\textbf{Critical}} \\
\addlinespace
\textbf{Lack of Multi-Factor Authentication} & MFA is not enforced on any systems, including email and sensitive data platforms. This makes credential theft a high-impact event, leading to unauthorized access. & \textcolor{darkred}{\textbf{Critical}} \\
\addlinespace
\textbf{Missing Acceptable Use Policy (AUP)} & The absence of a formal AUP means there are no clear guidelines for employees regarding the protection of company assets, data handling, and acceptable online behavior. & \textcolor{darkorange}{\textbf{High}} \\
\bottomrule
\end{tabular}
\end{table}

% =============================================================================
\section{Recommendations}
% =============================================================================
To address the identified risks, the following remediation actions are recommended, prioritized by urgency.

\subsection{Immediate Actions (To Be Completed Within 72 Hours)}
\begin{enumerate}
    \item \textbf{Restrict Access to MySQL Port:} Immediately apply firewall rules to block all public access to TCP port 3306 on host \texttt{[Target IP]}. Access should only be permitted from trusted internal IP addresses or through a secure VPN connection.
    \item \textbf{Develop an Upgrade Plan for MySQL:} Initiate a project to migrate the MySQL 5.7 database to a currently supported version (e.g., MySQL 8.0 or later). This plan should include data backup, testing, and a scheduled migration window.
\end{enumerate}

\subsection{High-Priority Actions (To Be Completed Within 30 Days)}
\begin{enumerate}
    \setcounter{enumi}{2}
    \item \textbf{Implement Multi-Factor Authentication (MFA):}
        \begin{itemize}
            \item \textbf{Phase 1 (Critical Systems):} Enforce MFA for all users on email services (e.g., Office 365, Google Workspace) and any systems identified as containing sensitive data.
            \item \textbf{Phase 2 (All Systems):} Roll out MFA for all remaining systems, including endpoint logins and VPN access.
        \end{itemize}
    \item \textbf{Develop and Implement an Acceptable Use Policy (AUP):} Draft a formal AUP that clearly defines rules for employees regarding data handling, internet usage, and security responsibilities. This policy must be communicated to all staff and acknowledged in writing.
\end{enumerate}

\subsection{Medium-Priority Actions (To Be Completed Within 90 Days)}
\begin{enumerate}
    \setcounter{enumi}{4}
    \item \textbf{Enhance Security Awareness Training:} Update the existing security awareness training program to include specific modules on the new AUP, the importance of MFA, and the dangers of phishing attacks that aim to steal credentials.
\end{enumerate}

% =============================================================================
\section{Conclusion}
% =============================================================================
The current security posture of \textbf{[Organization Name]} presents a significant and immediate risk of a data breach. The combination of an exposed, outdated database and a complete lack of multi-factor authentication creates a direct path for malicious actors to compromise sensitive information.

While the organization has a baseline security awareness program, it is undermined by the absence of critical technical and administrative controls. We strongly urge the management to allocate the necessary resources to implement the recommendations outlined in this report without delay. Proactive remediation is essential to protect the organization's assets, reputation, and operational continuity.

% -----------------------------------------------------------------------------
% DOCUMENT END
% -----------------------------------------------------------------------------
\end{document}
```