```latex
\documentclass[12pt, a4paper]{article}

% Preamble: Required Packages
\usepackage[margin=1in]{geometry}
\usepackage{pifont} % For checkmarks and crosses
\usepackage{booktabs} % For professional tables
\usepackage{hyperref} % For clickable links
\usepackage{url} % For URL formatting
\usepackage{seqsplit} % To split long strings without breaking
\usepackage{graphicx}
\usepackage{xcolor}
\usepackage{fancyhdr}
\usepackage{lastpage}

% --- Document Setup ---
\hypersetup{
    colorlinks=true,
    linkcolor=blue,
    filecolor=magenta,      
    urlcolor=cyan,
    pdftitle={Cybersecurity Posture Report},
    pdfauthor={Automated Security Analyst},
    pdfsubject={Security Assessment},
    pdfkeywords={Security, Report, Analysis},
}

% Define colors for risk levels
\definecolor{riskcritical}{HTML}{990000}
\definecolor{riskhigh}{HTML}{DD4B39}
\definecolor{riskmedium}{HTML}{F4B400}
\definecolor{risklow}{HTML}{4285F4}
\definecolor{riskinformational}{HTML}{999999}
\definecolor{riskok}{HTML}{0F9D58}

% --- Header and Footer ---
\pagestyle{fancy}
\fancyhf{} % Clear all header and footer fields
\fancyhead[L]{Cybersecurity Posture Report}
\fancyhead[R]{\textbf{[Organization Name]}}
\fancyfoot[C]{Page \thepage\ of \pageref{LastPage}}
\renewcommand{\headrulewidth}{0.4pt}
\renewcommand{\footrulewidth}{0.4pt}

% --- Document Start ---
\begin{document}

% --- Title Page ---
\begin{titlepage}
    \centering
    \vfill
    {\Huge \textbf{Cybersecurity Posture Report}\par}
    \vspace{1.5cm}
    {\Large \textbf{Prepared for:}}
    \vspace{0.5cm}
    {\huge \textbf{[Organization Name]}}
    \vfill
    {\large \today\par}
\end{titlepage}

\tableofcontents
\newpage

% --- Section 1: Executive Summary ---
\section*{Executive Summary}

This report provides a comprehensive analysis of the cybersecurity posture for \textbf{[Organization Name]}, based on a review of organizational security controls, a technical network scan, and pre-existing risk data. The assessment identified several critical and high-risk security gaps that require immediate attention.

Key findings indicate significant weaknesses in identity and access management, specifically the lack of Multi-Factor Authentication (MFA) for email and sensitive data systems. Another high-risk area is the absence of a formal security awareness training program for new employees, leaving the organization vulnerable to social engineering and human error from day one.

On a positive note, the technical network scan confirmed that a previously identified risk concerning an open unencrypted web port (Port 80) has been mitigated, as the port is now closed. This demonstrates progress in hardening the external network perimeter.

The following sections detail these findings and provide prioritized, actionable recommendations to remediate the identified risks and strengthen the overall security posture.

% --- Section 2: Organizational Information ---
\section*{Organizational Information}

This assessment was conducted for the following entity and associated assets:
\begin{itemize}
    \item \textbf{Organization Name:} \textbf{[Organization Name]}
    \item \textbf{Primary Domain:} \texttt{[Domain]}
    \item \textbf{External IP Address Scanned:} \texttt{[Client IP]}
\end{itemize}

% --- Section 3: Security Control Review ---
\section*{Security Control Review}

A review of administrative and organizational security controls was conducted via a questionnaire. The results highlight critical gaps in access control and employee training policies. A "No" response indicates a deviation from security best practices and a potential area of high risk.

\begin{table}[h!]
\centering
\caption{Organizational Security Controls Questionnaire}
\label{tab:controls}
\begin{tabular}{p{0.75\linewidth} c}
\toprule
\textbf{Control Question} & \textbf{Status} \\
\midrule
Do you require MFA to access email? & \textcolor{riskcritical}{\ding{55}} \\
Do you require MFA to log into computers? & \textcolor{riskok}{\ding{51}} \\
Do you require MFA to access sensitive data systems? & \textcolor{riskcritical}{\ding{55}} \\
Does your organization have an employee acceptable use policy? & \textcolor{riskok}{\ding{51}} \\
Does your organization do security awareness training for new employees? & \textcolor{riskhigh}{\ding{55}} \\
Does your organization do security awareness training for all employees at least once per year? & \textcolor{riskok}{\ding{51}} \\
\bottomrule
\end{tabular}
\end{table}

\subsection*{Analysis of Control Gaps}
\begin{itemize}
    \item \textbf{MFA Gaps (Critical):} The absence of MFA on email and sensitive data systems represents a critical vulnerability. Email is a primary target for account takeover attacks, which can lead to widespread compromise. Similarly, sensitive data systems without MFA are susceptible to unauthorized access via stolen credentials.
    \item \textbf{New Hire Training (High):} Failing to provide security awareness training during employee onboarding leaves a critical window of vulnerability. New employees are often unfamiliar with company policies and are prime targets for phishing and other social engineering attacks.
\end{itemize}

% --- Section 4: Technical Scan Results ---
\section*{Technical Scan Results}

A network scan was performed on the target IP address to identify externally accessible services and potential vulnerabilities.

\begin{itemize}
    \item \textbf{Target IP Address:} \texttt{[Target IP]}
    \item \textbf{Scan Date:} \today
    \item \textbf{Scanner Used:} Nmap
\end{itemize}

\subsection*{Scan Findings}
The scan revealed that the target host is online, but no open ports were detected. The status of a key port is listed below:

\begin{table}[h!]
\centering
\caption{Notable Port Scan Results}
\label{tab:scan}
\begin{tabular}{l l l}
\toprule
\textbf{Port} & \textbf{Protocol} & \textbf{State} \\
\midrule
80 & TCP & \textbf{Closed} \\
\bottomrule
\end{tabular}
\end{table}

\textbf{Analysis:} The scan indicates a strong perimeter security posture for the tested asset. Notably, Port 80 (HTTP) is closed. This is a positive finding, as it mitigates risks associated with unencrypted web traffic. This result directly contradicts a previously documented risk, indicating that remediation has occurred.

% --- Section 5: Correlated Risk Assessment ---
\section*{Correlated Risk Assessment}

This section synthesizes findings from the security control review, technical scan, and pre-existing risk data into a unified risk summary.

\begin{table}[h!]
\centering
\caption{Summary of Identified Risks}
\label{tab:risks}
\begin{tabular}{p{0.3\linewidth} p{0.45\linewidth} l}
\toprule
\textbf{Risk Name} & \textbf{Overview} & \textbf{Severity} \\
\midrule
\textbf{Lack of MFA on Email} & Threat actors can gain unauthorized access to email accounts using compromised credentials, leading to data breaches and further attacks. & \textcolor{riskcritical}{\textbf{Critical}} \\
\rule{0pt}{4ex} % Add vertical space
\textbf{Lack of MFA on Sensitive Data Systems} & Critical business systems are vulnerable to unauthorized access, potentially resulting in significant data exfiltration or operational disruption. & \textcolor{riskcritical}{\textbf{Critical}} \\
\rule{0pt}{4ex}
\textbf{No Security Onboarding for New Hires} & New employees are not trained on security policies and threats, making them a high-risk group for social engineering and policy violations. & \textcolor{riskhigh}{\textbf{High}} \\
\rule{0pt}{4ex}
\textbf{Unencrypted Web Server} \textit{(Historical)} & Port 80 was believed to be open, exposing the organization to risks of unencrypted data transmission. \textbf{Status: Mitigated.} & \textcolor{riskok}{\textbf{Closed}} \\
\bottomrule
\end{tabular}
\end{table}

% --- Section 6: Recommendations ---
\section*{Recommendations}

The following actions are recommended to address the identified risks. They are prioritized based on severity and potential impact.

\subsection*{Priority 1: Critical Risks}
\begin{enumerate}
    \item \textbf{Implement MFA for Email Access:}
    \begin{itemize}
        \item \textbf{Action:} Enforce mandatory MFA for all user access to the organization's email system (e.g., O365, Google Workspace).
        \item \textbf{Impact:} Drastically reduces the risk of email account takeovers, phishing success, and subsequent business email compromise (BEC).
    \end{itemize}
    \item \textbf{Implement MFA for Sensitive Systems:}
    \begin{itemize}
        \item \textbf{Action:} Identify all systems containing sensitive, confidential, or regulated data. Enforce MFA for all user and administrative access to these systems.
        \item \textbf{Impact:} Protects the organization's most valuable data assets from unauthorized access via compromised credentials.
    \end{itemize}
\end{enumerate}

\subsection*{Priority 2: High Risks}
\begin{enumerate}
    \setcounter{enumi}{2} % Continue numbering
    \item \textbf{Establish Security Training for New Hires:}
    \begin{itemize}
        \item \textbf{Action:} Develop a mandatory security awareness training module to be included in the standard onboarding process for all new employees and contractors. This should cover acceptable use, phishing identification, and incident reporting.
        \item \textbf{Impact:} Reduces the "human firewall" weakness and creates a security-conscious culture from the start of employment.
    \end{itemize}
\end{enumerate}

\subsection*{Priority 3: Administrative Actions}
\begin{enumerate}
    \setcounter{enumi}{3}
    \item \textbf{Update Internal Risk Register:}
    \begin{itemize}
        \item \textbf{Action:} Review the internal risk register and formally close the risk titled "Unencrypted Web Server" related to Port 80.
        \item \textbf{Impact:} Ensures the risk register accurately reflects the current security posture and allows resources to be focused on active threats.
    \end{itemize}
\end{enumerate}

\end{document}
```