```latex
\documentclass[12pt]{article}

% --- PACKAGE IMPORTS ---
\usepackage[margin=1in]{geometry} % For setting page margins
\usepackage{pifont}               % For checkmark and X symbols (\ding)
\usepackage{booktabs}             % For professional-looking tables
\usepackage{hyperref}             % For hyperlinks (e.g., in the footer)
\usepackage{url}                  % For formatting URLs
\usepackage{seqsplit}             % To split long strings without breaking words
\usepackage{graphicx}             % For including logos (optional, but good practice)

% --- DOCUMENT METADATA ---
\title{Cybersecurity Posture Assessment Report \\ \large For: \textbf{[Organization Name]}}
\author{Cybersecurity Analyst}
\date{\today}

% --- HYPERREF SETUP ---
\hypersetup{
    colorlinks=true,
    linkcolor=black,
    urlcolor=blue,
    pdftitle={Cybersecurity Posture Assessment Report},
    pdfauthor={Cybersecurity Analyst},
    pdfsubject={Security Assessment},
    pdfkeywords={Security, Assessment, Report, Vulnerability}
}

% --- DOCUMENT START ---
\begin{document}

\maketitle
\thispagestyle{empty}
\newpage

\tableofcontents
\newpage

% ==============================================================================
% SECTION 1: EXECUTIVE OVERVIEW
% ==============================================================================
\section{Executive Overview}

This report details the findings of a cybersecurity posture assessment conducted for \textbf{[Organization Name]}. The assessment combined a review of organizational security controls via a questionnaire, an external network scan, and an analysis of pre-existing risks.

\paragraph{Key Findings:} The overall security posture presents a mixed landscape. On one hand, the organization demonstrates a strong external network defense, with no open ports discovered on the scanned target (\texttt{[Target IP]}). This indicates a well-configured firewall and a minimal external attack surface, which is a significant strength.

However, critical gaps were identified in administrative and access controls. The two most significant risks are:
\begin{itemize}
    \item \textbf{Lack of Multi-Factor Authentication (MFA) on Email:} This is a critical vulnerability that exposes the organization to a high risk of account compromise, phishing, and Business Email Compromise (BEC).
    \item \textbf{Absence of an Employee Acceptable Use Policy (AUP):} This high-risk administrative gap means there are no formal guidelines governing the use of company IT assets, which can lead to insider threats and data misuse.
\end{itemize}

No previously known vulnerabilities were reported. The recommendations in this report are prioritized to address the most critical findings first to achieve the greatest immediate improvement in the organization's security posture.

% ==============================================================================
% SECTION 2: ORGANIZATIONAL INFORMATION
% ==============================================================================
\section{Organizational Information}

The following details were used as the basis for this assessment. As per our template mode for anonymized data, placeholders are used where information was not provided.

\begin{tabular}{@{}ll}
    \toprule
    \textbf{Attribute} & \textbf{Value} \\
    \midrule
    Organization Name & \textbf{[Organization Name]} \\
    Primary Email Domain & \texttt{[Domain]} \\
    External IP Address (Target) & \texttt{[Client IP]} \\
    \bottomrule
\end{tabular}

% ==============================================================================
% SECTION 3: SECURITY CONTROL REVIEW
% ==============================================================================
\section{Security Control Review}

The following table summarizes the organization's responses to a security controls questionnaire. Items marked with a green checkmark (\ding{51}) are positive controls, while items marked with a red X (\ding{55}) represent significant security gaps that require immediate attention.

\begin{table}[h!]
\centering
\begin{tabular}{@{}p{0.75\linewidth}c@{}}
    \toprule
    \textbf{Control Question} & \textbf{Response} \\
    \midrule
    Do you require MFA to access email? & \ding{55} \\
    Do you require MFA to log into computers? & \ding{51} \\
    Do you require MFA to access sensitive data systems? & \ding{51} \\
    Does your organization have an employee acceptable use policy? & \ding{55} \\
    Does your organization do security awareness training for new employees? & \ding{51} \\
    Does your organization do security awareness training for all employees at least once per year? & \ding{51} \\
    \bottomrule
\end{tabular}
\caption{Security Controls Questionnaire Results}
\end{label{tab:controls}
\end{table}

\paragraph{Analysis:} The questionnaire reveals strong practices in security awareness training and the application of MFA for computer and sensitive system access. However, the two "No" responses are critical weaknesses. Email is often the primary entry point for attackers, and the lack of an Acceptable Use Policy creates ambiguity and potential for misuse of company resources.

% ==============================================================================
% SECTION 4: TECHNICAL SCAN RESULTS
% ==============================================================================
\section{Technical Scan Results}

An external network scan was performed using Nmap to identify the attack surface of the designated target.

\begin{itemize}
    \item \textbf{Target IP Address:} \texttt{[Target IP]}
    \item \textbf{Host Status:} Up
    \item \textbf{Scan Findings:} The scan confirmed that the host was online and responsive. However, a comprehensive port scan revealed \textbf{no open TCP or UDP ports}. All 65,535 TCP ports and a common set of UDP ports were found to be in a "closed" state.
\end{itemize}

\paragraph{Conclusion:} This is an excellent security finding. A host with no open external ports presents a minimal attack surface to external threats. This configuration suggests that a properly configured stateful firewall is in place, correctly dropping or rejecting unsolicited incoming traffic. This practice significantly reduces the risk of external network-based attacks.

% ==============================================================================
% SECTION 5: RISK ASSESSMENT
% ==============================================================================
\section{Risk Assessment}

This section synthesizes findings from the security control review and technical scan. While no pre-existing vulnerabilities were reported in the input data, the assessment has identified the following new risks.

\begin{table}[h!]
\centering
\begin{tabular}{@{}p{0.25\linewidth}p{0.5\linewidth}p{0.15\linewidth}@{}}
    \toprule
    \textbf{Risk Name} & \textbf{Overview} & \textbf{Severity} \\
    \midrule
    Lack of MFA on Email & The absence of Multi-Factor Authentication on email accounts drastically increases the risk of account compromise via phishing or credential theft. This can lead to data breaches, financial fraud, and Business Email Compromise (BEC). & \textbf{Critical} \\
    \addlinespace
    Missing Acceptable Use Policy (AUP) & Without a formal AUP, employees lack clear guidelines on the proper use of IT assets. This increases the risk of insider threat (both malicious and accidental), data leakage, and potential legal or compliance violations. & \textbf{High} \\
    \bottomrule
\end{tabular}
\caption{Identified Risks and Severity}
\label{tab:risks}
\end{table}

% ==============================================================================
% SECTION 6: RECOMMENDATIONS
% ==============================================================================
\section{Recommendations}

Based on the risk assessment, the following prioritized actions are recommended to enhance the organization's cybersecurity posture.

\begin{enumerate}
    \item \textbf{[Critical] Deploy MFA for Email Access:}
    This is the highest priority recommendation. The organization must enforce MFA for all users accessing the email system, whether via webmail or client applications. This single control is the most effective defense against email account takeovers.
    
    \item \textbf{[High] Develop and Implement an Acceptable Use Policy (AUP):}
    Create a formal AUP document that clearly defines the rules and responsibilities for all employees when using company networks, computers, and data. This policy should be integrated into the new employee onboarding process and reviewed annually by all staff.
    
    \item \textbf{[Informational] Maintain and Audit Firewall Configuration:}
    The current external network posture is a major strength. Continue to maintain this "default deny" firewall configuration. We recommend conducting periodic (e.g., quarterly) reviews of firewall rules to ensure they remain aligned with business needs and the principle of least privilege.
\end{enumerate}

\end{document}
% --- DOCUMENT END ---
```