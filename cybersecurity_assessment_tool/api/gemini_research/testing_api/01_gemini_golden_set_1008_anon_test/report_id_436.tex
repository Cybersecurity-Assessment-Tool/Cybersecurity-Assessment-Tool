```latex
\documentclass[12pt]{article}

% Preamble: Required Packages
\usepackage[margin=1in]{geometry}
\usepackage{pifont} % For \ding symbols (checkmarks and crosses)
\usepackage{booktabs} % For professional-looking tables
\usepackage{hyperref}
\usepackage{url}
\usepackage{seqsplit} % To break long strings in \texttt
\usepackage{graphicx}
\usepackage{xcolor}
\usepackage{datetime}

% Document Metadata and Hyperlink Setup
\hypersetup{
    colorlinks=true,
    linkcolor=blue,
    filecolor=magenta,
    urlcolor=cyan,
    pdftitle={Cybersecurity Posture Assessment Report},
    pdfauthor={Cybersecurity Analysis Division},
    pdfsubject={Security Assessment},
    pdfkeywords={Cybersecurity, Risk, Assessment, Nmap, Policy}
}

% Custom Command for placeholder text
\newcommand{\placeholder}[1]{\textcolor{red!80!black}{\textbf{#1}}}

\begin{document}

% --- Title Page ---
\begin{titlepage}
    \centering
    \vspace*{1cm}
    \Huge \textbf{Cybersecurity Posture Assessment Report}
    \vspace{1.5cm}
    \Large
    Prepared for: \\
    \vspace{0.5cm}
    \textbf{[Organization Name]}
    \vspace{2cm}
    \normalsize
    \textbf{Analysis Conducted By:} \\
    Cybersecurity Analysis Division
    \vspace{1cm}
    \textbf{Date of Report:} \\
    \today
    \vfill
    \small
    \textit{This report contains sensitive information and is intended solely for the use of the recipient organization. Unauthorized distribution is strictly prohibited.}
\end{titlepage}

\tableofcontents
\newpage

% --- Section 1: Executive Summary ---
\section{Executive Summary}
This report details the findings of a cybersecurity posture assessment conducted for \textbf{[Organization Name]}. The assessment combined an external network scan, a review of existing risks, and an analysis of organizational security controls based on a provided questionnaire.

The overall security posture presents a mixed landscape. On a positive note, the external network scan of the target asset at \texttt{[Target IP]} revealed no open ports, suggesting a robust and well-configured perimeter firewall. This significantly reduces the external attack surface.

However, the review of organizational security controls identified two critical gaps that introduce substantial risk:
\begin{itemize}
    \item \textbf{Lack of Multi-Factor Authentication (MFA) for Email:} This is a critical vulnerability, as email is a primary target for attackers. Without MFA, user accounts are susceptible to compromise via phishing, credential stuffing, and other common attacks, which could lead to a significant data breach.
    \item \textbf{Absence of an Employee Acceptable Use Policy (AUP):} This policy gap creates ambiguity regarding the proper use of company assets, increasing the risk of insider threats, unintentional data leakage, and non-compliance with potential regulatory requirements.
\end{itemize}

No pre-existing vulnerabilities were reported. The immediate priority should be to address the identified policy and access control deficiencies to mitigate the high-impact risks they represent. Recommendations for remediation are detailed in Section \ref{sec:recommendations}.

% --- Section 2: Organizational Information ---
\section{Organizational Information}
The following details were used as the basis for this assessment. As identity information was not provided, standard placeholders are in use.

\begin{description}
    \item[Organization Name:] \textbf{[Organization Name]}
    \item[Primary Email Domain:] \texttt{[Domain]}
    \item[Target IP Address Scanned:] \texttt{[Target IP]}
\end{description}

% --- Section 3: Security Control Review ---
\section{Security Control Review}
An analysis of the organization's security practices was performed based on a questionnaire. The results below highlight current controls. Gaps identified with a \ding{55} represent areas of significant risk.

\begin{table}[h!]
\centering
\caption{Security Controls Questionnaire Analysis}
\label{tab:controls}
\begin{tabular}{p{0.8\linewidth} c}
\toprule
\textbf{Control Question} & \textbf{Status} \\
\midrule
Do you require MFA to access email? & \ding{55} \\
Do you require MFA to log into computers? & \ding{51} \\
Do you require MFA to access sensitive data systems? & \ding{51} \\
Does your organization have an employee acceptable use policy? & \ding{55} \\
Does your organization do security awareness training for new employees? & \ding{51} \\
Does your organization do security awareness training for all employees at least once per year? & \ding{51} \\
\bottomrule
\end{tabular}
\end{table}

\noindent \textit{Note: \ding{51} indicates 'Yes' (Control in place), \ding{55} indicates 'No' (Control gap).}

% --- Section 4: Technical Scan Results ---
\section{Technical Scan Results}
An external network vulnerability scan was performed using Nmap against the designated target IP address.

\begin{description}
    \item[Target IP:] \texttt{[Target IP]}
    \item[Scan Date:] \today
    \item[Host Status:] Up
    \item[Summary:] The scan confirmed that the target host is online and responsive. However, no open TCP ports were discovered within the top 1000 scanned ports. All ports reported a 'closed' state.
    \item[Conclusion:] This result is a strong indicator of a properly configured firewall that denies unsolicited inbound traffic. The organization's external network perimeter appears secure from common, opportunistic attacks.
\end{description}

% --- Section 5: Risk Assessment ---
\section{Risk Assessment}
The following table synthesizes findings from the security control review and technical scan to identify and prioritize risks to the organization.

\begin{table}[h!]
\centering
\caption{Identified Risks and Severity}
\label{tab:risks}
\begin{tabular}{p{0.1\linewidth} p{0.25\linewidth} p{0.45\linewidth} p{0.1\linewidth}}
\toprule
\textbf{ID} & \textbf{Risk Name} & \textbf{Description} & \textbf{Severity} \\
\midrule
RISK-001 & Lack of MFA for Email Access & Email accounts are protected only by a single factor (password), making them highly vulnerable to phishing, credential stuffing, and brute-force attacks. A compromised email account is a gateway to data breaches and further network intrusion. & \textbf{Critical} \\
\addlinespace
RISK-002 & Missing Acceptable Use Policy (AUP) & Without a formal AUP, employees lack clear guidelines on the secure and acceptable use of company systems and data. This increases the risk of insider threat, accidental data exposure, and legal liability. & \textbf{High} \\
\bottomrule
\end{tabular}
\end{table}

% --- Section 6: Recommendations ---
\section{Recommendations}
\label{sec:recommendations}
The following actionable recommendations are provided to address the identified risks and improve the overall security posture of \textbf{[Organization Name]}.

\subsection{RISK-001: Implement MFA for Email Access (Critical)}
\begin{itemize}
    \item \textbf{Immediate Action:} Enforce mandatory MFA for all user accounts accessing the corporate email system (e.g., Microsoft 365, Google Workspace). This should apply to all access methods, including webmail, desktop clients, and mobile devices.
    \item \textbf{Implementation:} Select an MFA method that balances security and user convenience, such as authenticator apps (preferred) or SMS-based codes.
    \item \textbf{User Training:} Communicate the change to all employees, provide clear instructions on how to set up MFA, and explain its importance in protecting their accounts and company data.
\end{itemize}

\subsection{RISK-002: Develop and Implement an AUP (High)}
\begin{itemize}
    \item \textbf{Immediate Action:} Draft a formal Acceptable Use Policy that clearly defines the rules and responsibilities for all employees when using company-owned IT assets.
    \item \textbf{Policy Content:} The AUP should cover, at a minimum: data handling procedures, internet usage restrictions, use of personal devices (BYOD), software installation rules, and consequences for non-compliance.
    \item \textbf{Deployment:} The policy should be reviewed by management and legal counsel before being distributed to all current employees for review and acknowledgment. This process must be integrated into the onboarding procedure for all new hires.
\end{itemize}

% --- Section 7: Conclusion ---
\section{Conclusion}
The assessment for \textbf{[Organization Name]} reveals a security posture with notable strengths and critical weaknesses. The secure configuration of the external network perimeter is commendable and provides a solid first line of defense.

However, the internal policy and identity management controls require immediate attention. The absence of MFA on a critical service like email and the lack of a foundational Acceptable Use Policy expose the organization to severe risks that could undermine its technical defenses.

By implementing the recommendations outlined in this report, \textbf{[Organization Name]} can significantly reduce its risk exposure and build a more resilient and mature cybersecurity program.

\end{document}
```