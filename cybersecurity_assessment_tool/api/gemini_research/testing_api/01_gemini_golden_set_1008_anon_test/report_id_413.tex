```latex
\documentclass[12pt]{article}

% Preamble: Required Packages
\usepackage[margin=1in]{geometry} % Set page margins
\usepackage{pifont}               % For checkmarks and crosses (\ding)
\usepackage{booktabs}             % For professional-looking tables
\usepackage{hyperref}             % For clickable links and references
\usepackage{url}                  % For formatting URLs
\usepackage{seqsplit}             % For splitting long strings in texttt
\usepackage[utf8]{inputenc}       % For UTF-8 input encoding

% Document Metadata
\title{Cybersecurity Posture Assessment Report}
\author{Cybersecurity Analyst}
\date{\today}

% Hyperref Setup
\hypersetup{
    colorlinks=true,
    linkcolor=black,
    urlcolor=blue,
    pdftitle={Cybersecurity Posture Assessment Report},
    pdfauthor={Cybersecurity Analyst},
    pdfsubject={Security Analysis},
    pdfkeywords={Security, Assessment, Nmap, Risk}
}

\begin{document}

\maketitle
\thispagestyle{empty} % No page number on the title page
\newpage

\tableofcontents
\newpage

% --- 1. Executive Summary ---
\section{Executive Summary}

This report details the findings of a cybersecurity assessment conducted for \textbf{[Organization Name]}. The analysis synthesizes results from a network vulnerability scan, a review of existing risks, and an organizational security controls questionnaire.

The assessment reveals a mixed security posture. While foundational controls like security awareness training and multi-factor authentication (MFA) for computer logins are in place, several critical vulnerabilities and policy gaps present a significant and immediate risk to the organization.

Key findings include:
\begin{itemize}
    \item \textbf{Critical Network Exposure:} The network scan confirmed that Remote Desktop Protocol (RDP) on port 3389 is publicly exposed on the external IP address \texttt{[Client IP]}. This is a high-risk vulnerability commonly exploited by attackers for initial access, leading to data breaches and ransomware deployment.
    \item \textbf{Insufficient Access Controls:} Multi-factor authentication is not enforced for accessing email or other sensitive data systems. This critical gap dramatically increases the likelihood of a successful account compromise via phishing or credential theft.
    \item \textbf{Policy Deficiencies:} The organization lacks a formal Employee Acceptable Use Policy (AUP). This absence creates ambiguity regarding security responsibilities and increases insider and legal risks.
\end{itemize}

Immediate remediation of the exposed RDP service is strongly recommended, followed by the swift implementation of comprehensive MFA and the development of key security policies. Addressing these issues will substantially improve the organization's resilience against common cyber threats.

% --- 2. Organizational Information ---
\section{Organizational Information}

This section provides the high-level details of the assessed organization. The information was derived from the provided data; placeholders are used where data was not available.

\begin{table}[h!]
\centering
\begin{tabular}{@{}ll@{}}
\toprule
\textbf{Attribute} & \textbf{Value} \\ \midrule
Organization Name & \textbf{[Organization Name]} \\
Primary Email Domain & \texttt{[Domain]} \\
Assessed External IP & \texttt{[Client IP]} \\ \bottomrule
\end{tabular}
\caption{Client Organizational Details.}
\label{tab:org_info}
\end{table}

% --- 3. Security Control Review ---
\section{Security Control Review}

The following table summarizes the organization's responses to a security controls questionnaire. The analysis highlights existing strengths and identifies significant gaps in the current security framework. A \ding{51} indicates a positive control, while a \ding{55} indicates a gap requiring attention.

\begin{table}[h!]
\centering
\begin{tabular}{@{}p{0.5\textwidth}cp{0.3\textwidth}@{}}
\toprule
\textbf{Control Question} & \textbf{Response} & \textbf{Analyst Notes} \\ \midrule
Do you require MFA to access email? & \ding{55} & \textbf{Critical Gap.} Email is a primary target for account takeover. \\
Do you require MFA to log into computers? & \ding{51} & Good practice for endpoint security. \\
Do you require MFA to access sensitive data systems? & \ding{55} & \textbf{Critical Gap.} High-value data is inadequately protected. \\
Does your organization have an employee acceptable use policy? & \ding{55} & \textbf{High Risk.} Lack of formal policy increases insider and legal risk. \\
Does your organization do security awareness training for new employees? & \ding{51} & Positive control for onboarding. \\
Does your organization do security awareness training for all employees at least once per year? & \ding{51} & Strong practice for maintaining security awareness. \\ \bottomrule
\end{tabular}
\caption{Security Controls Questionnaire Analysis.}
\label{tab:controls_review}
\end{table}

% --- 4. Technical Scan Results ---
\section{Technical Scan Results}

An external network scan was performed against the target IP address \texttt{[Target IP]}. The scan identified the following open ports and services, indicating systems that are accessible from the public internet.

\begin{table}[h!]
\centering
\begin{tabular}{@{}llll@{}}
\toprule
\textbf{Port} & \textbf{State} & \textbf{Service Name} & \textbf{Analyst Notes} \\ \midrule
3389/tcp & Open & ms-wbt-server & Remote Desktop Protocol (RDP). This is a \textbf{critical finding}. \\
& & & Publicly exposed RDP is a primary vector for ransomware attacks. \\ \bottomrule
\end{tabular}
\caption{Open Ports Detected on \texttt{[Target IP]}.}
\label{tab:nmap_results}
\end{table}

The presence of an open RDP port confirms the pre-existing risk documented in the organization's risk register and poses an immediate threat that must be addressed.

% --- 5. Consolidated Risk Assessment ---
\section{Consolidated Risk Assessment}

This section synthesizes findings from the technical scan, control review, and existing risk data into a consolidated list of key risks facing the organization.

\begin{table}[h!]
\centering
\begin{tabular}{@{}p{0.15\textwidth}p{0.65\textwidth}l@{}}
\toprule
\textbf{Risk Title} & \textbf{Description} & \textbf{Severity} \\ \midrule
\textbf{Public RDP Exposure} & The network scan confirmed that Remote Desktop Protocol (port 3389) is open to the internet on \texttt{[Target IP]}. This allows attackers to attempt brute-force or credential stuffing attacks to gain direct access to the internal network. & \textbf{Critical} \\
\addlinespace
\textbf{Lack of MFA on Critical Systems} & Multi-factor authentication is not enforced for email or sensitive data systems. This allows an attacker with stolen credentials (e.g., from a phishing attack) to gain unauthorized access to critical communications and data. & \textbf{Critical} \\
\addlinespace
\textbf{Missing Acceptable Use Policy (AUP)} & The absence of a formal AUP leaves security expectations undefined for employees. This increases the risk of unintentional data exposure, misuse of company assets, and complicates incident response. & \textbf{High} \\ \bottomrule
\end{tabular}
\caption{Summary of Identified Risks.}
\label{tab:risk_summary}
\end{table}

% --- 6. Recommendations ---
\section{Recommendations}

Based on the consolidated risk assessment, the following prioritized actions are recommended to mitigate the identified vulnerabilities and improve the overall security posture of \textbf{[Organization Name]}.

\subsection{Priority 1: Immediate Actions (0-7 Days)}
\begin{enumerate}
    \item \textbf{Remediate RDP Exposure:} Immediately block all inbound traffic to TCP port 3389 on the external firewall for \texttt{[Client IP]}.
    \begin{itemize}
        \item \textbf{Long-Term Solution:} If remote access is required, implement a secure Virtual Private Network (VPN) solution that requires multi-factor authentication for all connections.
    \end{itemize}
\end{enumerate}

\subsection{Priority 2: High-Impact Actions (1-4 Weeks)}
\begin{enumerate}
    \setcounter{enumi}{1} % Continue numbering from previous list
    \item \textbf{Implement Comprehensive MFA:} Procure and deploy an MFA solution for all employees. Prioritize enforcement on the following systems:
    \begin{itemize}
        \item Email (e.g., Office 365, Google Workspace).
        \item All systems containing sensitive or regulated data.
        \item All remote access solutions (e.g., VPN).
    \end{itemize}
    \item \textbf{Develop and Implement an Acceptable Use Policy (AUP):} Create a formal AUP that clearly defines the rules and expectations for employee use of company technology, data, and internet access. Ensure all employees read and acknowledge the policy.
\end{enumerate}

\subsection{Priority 3: Foundational Improvements (Next 90 Days)}
\begin{enumerate}
    \setcounter{enumi}{3} % Continue numbering
    \item \textbf{Conduct a Full Access Control Review:} Perform a comprehensive review of user access rights across all sensitive systems to ensure the principle of least privilege is being followed. Remove any unnecessary or excessive permissions.
\end{enumerate}

\end{document}
```