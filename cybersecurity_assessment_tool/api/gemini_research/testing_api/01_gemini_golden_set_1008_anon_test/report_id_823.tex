```latex
\documentclass[12pt]{article}

% Preamble: Required Packages
\usepackage[margin=1in]{geometry}
\usepackage{pifont} % For checkmarks and crosses
\usepackage{booktabs} % For professional tables
\usepackage{hyperref} % For clickable links
\usepackage{url}      % For URL formatting
\usepackage{seqsplit} % To split long strings in tt font
\usepackage{graphicx} % For potential logo inclusion
\usepackage{xcolor}   % For color definitions

% --- Document Information ---
\title{Cybersecurity Posture Assessment Report \\ \large For: \textbf{[Organization Name]}}
\author{Cybersecurity Analyst}
\date{\today}

% --- Custom Commands & Settings ---
\hypersetup{
    colorlinks=true,
    linkcolor=blue,
    filecolor=magenta,      
    urlcolor=cyan,
    pdftitle={Cybersecurity Posture Assessment Report},
    pdfpagemode=FullScreen,
}
\newcommand{\yes}{\ding{51}} % Checkmark
\newcommand{\no}{\ding{55}}  % Cross

\begin{document}

\maketitle
\thispagestyle{empty}
\newpage

\tableofcontents
\newpage

% ==============================================================================
\section{Executive Summary}
% ==============================================================================

This report details the findings of a cybersecurity posture assessment conducted for \textbf{[Organization Name]}. The assessment combined an analysis of organizational security controls, a technical network scan, and a review of pre-existing risks.

The overall security posture is assessed as \textbf{High Risk}. This is primarily due to several critical security control gaps. Key findings include:

\begin{itemize}
    \item \textbf{Critical Lack of Multi-Factor Authentication (MFA):} MFA is not enforced for accessing email, computer logins, or sensitive data systems. This significantly increases the risk of unauthorized access through compromised credentials.
    \item \textbf{Inadequate Security Training:} New employees do not receive security awareness training as part of their onboarding process, leaving the organization vulnerable to social engineering and human error from the outset.
    \item \textbf{Exposed Administrative Services:} The external network scan identified an open SSH port (22), which is a common target for brute-force and credential-stuffing attacks by malicious actors.
\end{itemize}

Immediate remediation of these issues is strongly recommended to reduce the organization's attack surface and mitigate the risk of a significant security incident. Detailed recommendations are provided in Section \ref{sec:recommendations}.

% ==============================================================================
\section{Organizational Information}
% ==============================================================================

The following information was used as the basis for this assessment. As identity data was not provided, placeholders are used.

\begin{itemize}
    \item \textbf{Organization Name:} \textbf{[Organization Name]}
    \item \textbf{Primary Email Domain:} \texttt{[Domain]}
    \item \textbf{Monitored External IP:} \texttt{[Client IP]}
\end{itemize}

% ==============================================================================
\section{Security Control Review}
% ==============================================================================

The following table summarizes the organization's responses to a security controls questionnaire. A red \no\ indicates a potential security gap that deviates from best practices.

\begin{table}[h!]
\centering
\caption{Security Controls Questionnaire Results}
\begin{tabular}{p{0.8\textwidth} c}
\toprule
\textbf{Control Question} & \textbf{Response} \\
\midrule
Do you require MFA to access email? & \textcolor{red}{\no} \\
Do you require MFA to log into computers? & \textcolor{red}{\no} \\
Do you require MFA to access sensitive data systems? & \textcolor{red}{\no} \\
Does your organization have an employee acceptable use policy? & \textcolor{green}{\yes} \\
Does your organization do security awareness training for new employees? & \textcolor{red}{\no} \\
Does your organization do security awareness training for all employees at least once per year? & \textcolor{green}{\yes} \\
\bottomrule
\end{tabular}
\end{table}

The review highlights critical deficiencies in access control (MFA) and employee onboarding, which are major contributors to the overall risk profile.

% ==============================================================================
\section{Technical Scan Results}
% ==============================================================================

An external network scan was performed to identify open ports and exposed services.

\begin{itemize}
    \item \textbf{Scan Target:} \texttt{[Target IP]}
    \item \textbf{Scan Date:} \textbf{[Scan Date]}
\end{itemize}

\subsection{Open Ports}
The following table details the ports found to be open and accessible from the public internet.

\begin{table}[h!]
\centering
\caption{Open Port Analysis}
\begin{tabular}{l l l p{0.5\textwidth}}
\toprule
\textbf{Port} & \textbf{State} & \textbf{Service} & \textbf{Notes} \\
\midrule
22/tcp & Open & SSH (Inferred) & The Secure Shell service is exposed. This is a common vector for brute-force and credential-based attacks. No version information was available from the scan. \\
\bottomrule
\end{tabular}
\end{table}

% ==============================================================================
\section{Risk Assessment}
% ==============================================================================

This section synthesizes the findings from the security control review and technical scan into a prioritized list of risks. No pre-existing vulnerabilities were reported.

\begin{table}[h!]
\centering
\caption{Identified Risks}
\begin{tabular}{p{0.1\textwidth} p{0.25\textwidth} p{0.4\textwidth} p{0.1\textwidth}}
\toprule
\textbf{ID} & \textbf{Risk Name} & \textbf{Description} & \textbf{Severity} \\
\midrule
RISK-001 & Lack of Multi-Factor Authentication (MFA) & The absence of MFA for email, endpoints, and sensitive systems makes the organization highly susceptible to account takeover attacks. & \textbf{Critical} \\
\addlinespace
RISK-002 & Exposed SSH Service & Port 22 (SSH) is open to the internet on \texttt{[Target IP]}, increasing the risk of unauthorized access attempts and brute-force attacks against administrative interfaces. & \textbf{High} \\
\addlinespace
RISK-003 & Inadequate Security Awareness Training & New employees do not receive security training, creating a persistent vulnerability to phishing, social engineering, and policy violations. & \textbf{High} \\
\bottomrule
\end{tabular}
\end{table}

% ==============================================================================
\section{Recommendations}
\label{sec:recommendations}
% ==============================================================================

The following actionable recommendations are provided to address the identified risks. They are prioritized by severity.

\subsection{RISK-001: Lack of MFA (Critical)}
\begin{itemize}
    \item \textbf{Action:} Implement and enforce a mandatory MFA policy for all users across all critical systems.
    \item \textbf{Details:} Prioritize the rollout for:
        \begin{enumerate}
            \item All privileged/administrator accounts.
            \item Email access (e.g., Office 365, Google Workspace).
            \item Remote access solutions (VPNs).
            \item All systems containing sensitive or regulated data.
        \end{enumerate}
    \item \textbf{Impact:} Drastically reduces the risk of account compromise and unauthorized access.
\end{itemize}

\subsection{RISK-002: Exposed SSH Service (High)}
\begin{itemize}
    \item \textbf{Action:} Restrict access to the SSH service on \texttt{[Target IP]}.
    \item \textbf{Details:} The preferred method is to place the service behind a Virtual Private Network (VPN). If direct access is required, implement a strict firewall rule to allow connections only from specific, trusted IP addresses. Public access should be disabled immediately.
    \item \textbf{Impact:} Significantly reduces the attack surface and protects a key administrative entry point.
\end{itemize}

\subsection{RISK-003: Inadequate Security Training (High)}
\begin{itemize}
    \item \textbf{Action:} Develop and implement a mandatory security awareness training program for all new employees.
    \item \textbf{Details:} This training should be a required part of the employee onboarding process and should cover, at a minimum, acceptable use, phishing identification, password hygiene, and incident reporting procedures.
    \item \textbf{Impact:} Strengthens the "human firewall" and reduces the likelihood of security incidents caused by employee error.
\end{itemize}

\end{document}
```