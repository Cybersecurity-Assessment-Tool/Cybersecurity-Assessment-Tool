```latex
\documentclass[12pt]{article}

% --- PACKAGES ---
\usepackage[a4paper, margin=1in]{geometry} % Page layout
\usepackage{pifont}                       % For checkmarks and crosses
\usepackage{booktabs}                     % For professional tables
\usepackage{hyperref}                     % For hyperlinks and document metadata
\usepackage{url}                          % For URL formatting
\usepackage{seqsplit}                     % For splitting long strings without spaces
\usepackage{graphicx}                     % For logos (optional)
\usepackage{xcolor}                       % For colors

% --- DOCUMENT METADATA ---
\hypersetup{
    colorlinks=true,
    linkcolor=blue,
    filecolor=magenta,      
    urlcolor=cyan,
    pdftitle={Cybersecurity Posture Assessment Report},
    pdfauthor={Cybersecurity Analysis Division},
    pdfsubject={Security Assessment},
    pdfkeywords={Cybersecurity, Nmap, Risk, Assessment},
    pdftoolbar=true,
}

% --- TITLE ---
\title{
    \vspace{-2cm}
    \rule{\textwidth}{2pt} \\ [0.5cm]
    \textbf{Cybersecurity Posture Assessment Report} \\ [0.2cm]
    \large For: \textbf{[Organization Name]} \\ [0.5cm]
    \rule{\textwidth}{2pt}
}
\author{Cybersecurity Analysis Division}
\date{\today}

% --- DOCUMENT START ---
\begin{document}

\maketitle
\thispagestyle{empty}
\newpage

\tableofcontents
\newpage

% ==============================================================================
\section{Executive Overview}
% ==============================================================================

This report details the findings of a cybersecurity assessment conducted for \textbf{[Organization Name]}. The analysis is based on a combination of technical network scanning, a review of existing risk documentation, and an evaluation of organizational security controls via a questionnaire.

The assessment reveals a mixed security posture. The organization has implemented several important controls, such as Multi-Factor Authentication (MFA) for email and computer access. However, critical vulnerabilities and control gaps were identified that expose the organization to a high risk of significant data compromise.

The most critical findings are:
\begin{itemize}
    \item \textbf{Publicly Exposed and Unsupported Database:} A MySQL database (version 5.7.33) is directly accessible from the internet. This version is End-of-Life (EOL) and no longer receives security updates, making it an easy target for known exploits.
    \item \textbf{Lack of MFA on Sensitive Systems:} While MFA is used for email, it is not enforced for accessing sensitive data systems. This gap allows a single compromised password to potentially lead to a major data breach.
    \item \textbf{Inadequate Employee Onboarding:} New employees do not receive security awareness training, making them highly susceptible to phishing and social engineering attacks, which are common initial access vectors.
\end{itemize}

These findings, when correlated, present a clear and immediate threat. Urgent remediation is required to mitigate the risk of a security incident. This report provides specific, actionable recommendations to address these vulnerabilities.

% ==============================================================================
\section{Organizational Information}
% ==============================================================================

The following information was used as the basis for this assessment. As per our analysis protocol, certain identifying details have been templated due to missing data in the provided inputs.

\begin{itemize}
    \item \textbf{Organization Name:} \textbf{[Organization Name]}
    \item \textbf{Primary Domain:} \texttt{[Domain]}
    \item \textbf{External IP Address Scanned:} \texttt{[Client IP]}
\end{itemize}

% ==============================================================================
\section{Security Control Review}
% ==============================================================================

The following table summarizes the organization's responses to a security controls questionnaire. A checkmark (\ding{51}) indicates a positive control is in place, while a cross (\ding{55}) indicates a control gap.

\begin{table}[h!]
\centering
\caption{Security Controls Questionnaire Results}
\begin{tabular}{p{0.7\textwidth} c c}
\toprule
\textbf{Control Question} & \textbf{Response} & \textbf{Status} \\
\midrule
Do you require MFA to access email? & Yes & \textcolor{green}{\ding{51}} \\
Do you require MFA to log into computers? & Yes & \textcolor{green}{\ding{51}} \\
\textbf{Do you require MFA to access sensitive data systems?} & \textbf{No} & \textcolor{red}{\ding{55}} \\
Does your organization have an employee acceptable use policy? & Yes & \textcolor{green}{\ding{51}} \\
\textbf{Does your organization do security awareness training for new employees?} & \textbf{No} & \textcolor{red}{\ding{55}} \\
Does your organization do security awareness training for all employees at least once per year? & Yes & \textcolor{green}{\ding{51}} \\
\bottomrule
\end{tabular}
\end{table}

\subsection*{Analysis of Control Gaps}
The two "No" responses represent significant weaknesses in the organization's defensive posture:
\begin{itemize}
    \item \textbf{MFA on Sensitive Systems:} The absence of MFA on systems holding sensitive data is a critical failure. It negates many other security efforts, as a single compromised credential could grant an attacker direct access to the organization's most valuable information.
    \item \textbf{New Hire Training:} Failing to train new employees on security best practices leaves a critical window of vulnerability. New hires are often targeted by attackers as they are less familiar with company policies and more likely to fall for social engineering tactics.
\end{itemize}

% ==============================================================================
\section{Technical Scan Results}
% ==============================================================================

An external network scan was performed against the target IP address \texttt{[Target IP]}. The scan identified the following open ports and services.

\begin{table}[h!]
\centering
\caption{Open Port Scan Results for \texttt{[Target IP]}}
\begin{tabular}{l l l l l}
\toprule
\textbf{Port} & \textbf{State} & \textbf{Service} & \textbf{Product} & \textbf{Version} \\
\midrule
3306/tcp & open & mysql & MySQL & 5.7.33 \\
\bottomrule
\end{tabular}
\end{table}

\subsection*{Technical Analysis}
The scan revealed that TCP port 3306 is open to the public internet. This port is used by the MySQL database service.
\begin{itemize}
    \item \textbf{Direct Exposure:} Exposing a database server directly to the internet is highly discouraged. It allows attackers worldwide to attempt brute-force password attacks, exploit vulnerabilities, and potentially exfiltrate data.
    \item \textbf{Unsupported Version:} The identified MySQL version, \textbf{5.7.33}, reached its official End-of-Life (EOL) in October 2023. This means it no longer receives security patches from the vendor. Any vulnerabilities discovered since that date remain unpatched, making this service a prime target for exploitation.
\end{itemize}

% ==============================================================================
\section{Correlated Risk Assessment}
% ==============================================================================

By synthesizing the questionnaire results, technical scan data, and pre-existing risk information, we have compiled a prioritized list of correlated risks.

\begin{table}[h!]
\centering
\caption{Summary of Key Risks}
\begin{tabular}{p{0.25\textwidth} p{0.15\textwidth} p{0.5\textwidth}}
\toprule
\textbf{Risk Name} & \textbf{Severity} & \textbf{Description \& Correlation} \\
\midrule
\textbf{Exposed \& Unsupported Database} & \textbf{Critical (9.8)} & The Nmap scan confirms that an unsupported MySQL 5.7.33 database is exposed (Port 3306). This directly validates the pre-existing risk. The lack of MFA on sensitive systems means a compromised user password could grant an attacker full access to this database. \\
\addlinespace
\textbf{Lack of MFA on Sensitive Systems} & \textbf{Critical (9.1)} & This administrative control gap, identified in the questionnaire, is a critical vulnerability. It is the primary compensating control that should be protecting critical assets like the exposed database. Its absence dramatically increases the likelihood of a successful breach. \\
\addlinespace
\textbf{Inadequate Employee Onboarding} & \textbf{High (7.5)} & The lack of security training for new hires creates a high-risk entry point for attackers via phishing or social engineering. A successful phishing attack could yield credentials that are then used to access the exposed database or other sensitive systems lacking MFA. \\
\bottomrule
\end{tabular}
\end{table}

% ==============================================================================
\section{Recommendations}
% ==============================================================================

The following actions are recommended to mitigate the identified risks. They are prioritized based on severity and potential impact.

\subsection*{Priority 1: Critical (Remediate Immediately)}
\begin{enumerate}
    \item \textbf{Isolate the Database Server:}
    \begin{itemize}
        \item \textbf{Action:} Immediately implement firewall rules to block all public ingress traffic to TCP port 3306 on \texttt{[Target IP]}.
        \item \textbf{Justification:} This is the fastest way to remove the immediate threat of external attack against the database. Access should only be permitted from trusted internal IP addresses or through a secure Virtual Private Network (VPN).
    \end{itemize}
    \vspace{0.5cm}
    \item \textbf{Implement MFA on All Sensitive Systems:}
    \begin{itemize}
        \item \textbf{Action:} Identify all systems classified as containing sensitive data and enforce MFA for all user access.
        \item \textbf{Justification:} This closes the most significant administrative control gap and protects critical data even if user credentials are compromised.
    \end{itemize}
    \vspace{0.5cm}
    \item \textbf{Plan Database Upgrade:}
    \begin{itemize}
        \item \textbf{Action:} Develop a plan to migrate the MySQL 5.7.33 database to a fully supported version (e.g., MySQL 8.x).
        \item \textbf{Justification:} Running EOL software is unsustainable. Upgrading is essential to receive security patches and protect against future vulnerabilities.
    \end{itemize}
\end{enumerate}

\subsection*{Priority 2: High (Remediate within 30-60 days)}
\begin{enumerate}
    \setcounter{enumi}{3} % Continue numbering
    \item \textbf{Mandate New Hire Security Training:}
    \begin{itemize}
        \item \textbf{Action:} Develop or procure a security awareness training module and integrate it into the mandatory onboarding process for all new employees.
        \item \textbf{Justification:} This strengthens the "human firewall" and reduces the risk of initial compromise through phishing and other social engineering tactics.
    \end{itemize}
\end{enumerate}

\end{document}
```