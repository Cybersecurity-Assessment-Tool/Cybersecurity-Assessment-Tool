```latex
\documentclass[12pt]{article}

% --- PACKAGES ---
\usepackage[margin=1in]{geometry}
\usepackage{pifont} % For checkmarks and crosses
\usepackage{booktabs} % For professional tables
\usepackage{hyperref} % For hyperlinks
\usepackage{url} % For URL formatting
\usepackage{seqsplit} % To split long strings in texttt
\usepackage{graphicx}
\usepackage{xcolor}

% --- DOCUMENT METADATA ---
\title{Cybersecurity Posture Assessment Report}
\author{Cybersecurity Analysis Division}
\date{\today}

% --- HYPERREF SETUP ---
\hypersetup{
    colorlinks=true,
    linkcolor=blue,
    filecolor=magenta,      
    urlcolor=cyan,
    pdftitle={Cybersecurity Posture Assessment Report},
    pdfpagemode=FullScreen,
}

% --- COMMANDS ---
\newcommand{\yes}{\ding{51}}
\newcommand{\no}{\ding{55}}

\begin{document}

\maketitle
\thispagestyle{empty}
\newpage
\tableofcontents
\newpage

% ==============================================================================
\section{Executive Overview}
% ==============================================================================

This report provides a comprehensive analysis of the cybersecurity posture for \textbf{[Organization Name]}. The assessment is based on a correlation of data from an external network scan, a security controls questionnaire, and a review of pre-existing risks.

The organization has implemented several strong foundational security controls, including the mandatory use of Multi-Factor Authentication (MFA) for email, computer logins, and access to sensitive data systems. The external network scan of the provided IP address, \texttt{[Client IP]}, revealed no open ports, indicating a robust network perimeter defense against external threats.

However, a critical gap was identified in the organization's security awareness program. The failure to conduct annual security awareness training for all employees represents a \textbf{High} risk. This deficiency leaves the organization vulnerable to human-centric attacks such as phishing and social engineering, which are leading causes of security breaches.

The primary recommendation is to immediately establish a mandatory, annual security awareness training program for all personnel to mitigate this significant risk.

% ==============================================================================
\section{Organizational Information}
% ==============================================================================

The following information was used as the basis for this assessment.

\begin{itemize}
    \item \textbf{Organization Name:} \textbf{[Organization Name]}
    \item \textbf{Primary Email Domain:} \texttt{[Domain]}
    \item \textbf{Scanned External IP:} \texttt{[Client IP]}
\end{itemize}

% ==============================================================================
\section{Security Control Review}
% ==============================================================================

The following table details the responses from the security controls questionnaire. Each response is assessed against industry best practices.

\begin{table}[h!]
\centering
\begin{tabular}{p{0.6\linewidth} c p{0.2\linewidth}}
\toprule
\textbf{Control Question} & \textbf{Response} & \textbf{Assessment} \\
\midrule
Do you require MFA to access email? & \yes & Compliant \\
\addlinespace
Do you require MFA to log into computers? & \yes & Compliant \\
\addlinespace
Do you require MFA to access sensitive data systems? & \yes & Compliant \\
\addlinespace
Does your organization have an employee acceptable use policy? & \yes & Compliant \\
\addlinespace
Does your organization do security awareness training for new employees? & \yes & Compliant \\
\addlinespace
Does your organization do security awareness training for all employees at least once per year? & \no & \textbf{Critical Gap} \\
\bottomrule
\end{tabular}
\caption{Security Controls Questionnaire Analysis}
\label{tab:controls}
\end{table}

The analysis highlights a significant weakness: the lack of ongoing, annual security training. While onboarding training is a good first step, the threat landscape evolves continuously. Annual training is essential to keep employees informed about new tactics used by malicious actors.

% ==============================================================================
\section{Technical Scan Results}
% ==============================================================================

An external network scan was performed against the target IP address to identify exposed services and potential vulnerabilities.

\begin{itemize}
    \item \textbf{Target IP Address:} \texttt{[Target IP]}
    \item \textbf{Scan Date:} Not provided in scan data.
\end{itemize}

\subsection{Scan Summary}
The network scan completed successfully but did not detect any open TCP or UDP ports on the target system. 

\textbf{Finding:} This is a positive security finding. It suggests that the organization's firewall and network perimeter configurations are effectively blocking unsolicited inbound traffic, significantly reducing the external attack surface. No further technical analysis is possible based on these results.

% ==============================================================================
\section{Consolidated Risk Assessment}
% ==============================================================================

The following table synthesizes findings from the security control review, technical scan, and any pre-existing risks provided.

\begin{table}[h!]
\centering
\begin{tabular}{p{0.1\linewidth} p{0.3\linewidth} p{0.4\linewidth} p{0.1\linewidth}}
\toprule
\textbf{Risk ID} & \textbf{Risk Name} & \textbf{Description} & \textbf{Severity} \\
\midrule
RISK-001 & Lack of Annual Security Awareness Training & The organization does not conduct security training for all employees on an annual basis. This increases susceptibility to phishing, social engineering, and other human-centered attacks, as employee knowledge of current threats is not refreshed. & \textbf{High} \\
\addlinespace
\multicolumn{4}{l}{\textit{No pre-existing vulnerabilities were provided for this assessment.}} \\
\bottomrule
\end{tabular}
\caption{Consolidated Risk Register}
\label{tab:risks}
\end{table}

% ==============================================================================
\section{Recommendations}
% ==============================================================================

Based on the consolidated risk assessment, the following actions are recommended to enhance the organization's cybersecurity posture.

\subsection{High Priority}
\begin{itemize}
    \item \textbf{Implement Annual Security Awareness Training (RISK-001):} Immediately develop and deploy a mandatory annual security awareness training program for all employees. This program should be updated yearly to cover current threats such as sophisticated phishing campaigns, ransomware tactics, social engineering, and best practices for data handling and incident reporting. Management must ensure 100\% completion and track metrics to gauge effectiveness.
\end{itemize}

\subsection{Medium Priority}
\begin{itemize}
    \item \textbf{Internal Scan Verification:} While the external scan showed a strong perimeter, it provides no visibility into the internal network. It is recommended to conduct regular, authenticated internal vulnerability scans to identify and remediate potential security weaknesses within the network that could be exploited if the perimeter is breached.
\end{itemize}

\end{document}
```