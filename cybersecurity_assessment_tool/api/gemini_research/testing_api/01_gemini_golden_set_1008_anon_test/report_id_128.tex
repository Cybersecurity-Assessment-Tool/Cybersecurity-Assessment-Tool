```latex
\documentclass[12pt]{article}

% Preamble: Required Packages
\usepackage[a4paper, margin=1in]{geometry}
\usepackage{pifont} % For checkmarks and crosses
\usepackage{booktabs} % For professional tables
\usepackage{hyperref} % For clickable links
\usepackage{url} % For URL formatting
\usepackage{seqsplit} % For splitting long strings in tt font
\usepackage[T1]{fontenc}

% Document Metadata
\title{Cybersecurity Posture Assessment Report}
\author{Cybersecurity Analyst Group}
\date{\today}

% Hyperref Setup
\hypersetup{
    colorlinks=true,
    linkcolor=black,
    urlcolor=blue,
    pdftitle={Cybersecurity Posture Assessment Report},
    pdfauthor={Cybersecurity Analyst Group},
    pdfsubject={Security Analysis},
    pdfkeywords={Cybersecurity, Nmap, Risk Assessment}
}

\begin{document}

\maketitle
\thispagestyle{empty}
\newpage
\tableofcontents
\thispagestyle{empty}
\newpage
\setcounter{page}{1}

% --- 1. Executive Summary ---
\section{Executive Summary}

This report details the findings of a cybersecurity posture assessment conducted for \textbf{[Organization Name]}. The analysis is based on a combination of technical network scanning, a review of administrative security controls via a questionnaire, and an evaluation of pre-existing risk data.

The assessment reveals a \textbf{critical risk posture} due to fundamental deficiencies in both technical and administrative security controls. The complete absence of Multi-Factor Authentication (MFA) across all critical systems, including email and sensitive data access, represents an immediate and severe threat. This is compounded by the lack of foundational security policies and employee awareness training.

Furthermore, technical scanning identified the use of the unencrypted HTTP protocol on an external-facing system, exposing the organization to data interception and credential theft. Immediate and decisive action is required to remediate these high-impact vulnerabilities and establish a baseline of security for the organization.

% --- 2. Organizational Information ---
\section{Organizational Information}

This section contains the high-level, anonymized information for the organization under review.

\begin{itemize}
    \item \textbf{Organization Name:} \textbf{[Organization Name]}
    \item \textbf{Primary Email Domain:} \texttt{[Domain]}
    \item \textbf{External IP Address Scanned:} \texttt{[Client IP]}
\end{itemize}

% --- 3. Security Control Review (Questionnaire Analysis) ---
\section{Security Control Review}

An administrative review was conducted based on a security questionnaire. The results indicate critical gaps in essential security policies and procedures. A "No" answer (\ding{55}) signifies a missing control that exposes the organization to significant risk.

\begin{table}[h!]
\centering
\caption{Security Questionnaire Analysis}
\begin{tabular}{p{0.6\linewidth} c p{0.25\linewidth}}
\toprule
\textbf{Control Question} & \textbf{Status} & \textbf{Analyst Note} \\
\midrule
Do you require MFA to access email? & \ding{55} & Critical Gap. Email is a primary target for account takeover. \\
\addlinespace
Do you require MFA to log into computers? & \ding{55} & High Risk. Lack of MFA allows for easier lateral movement after a breach. \\
\addlinespace
Do you require MFA to access sensitive data systems? & \ding{55} & Critical Gap. The organization's most valuable data is not adequately protected. \\
\addlinespace
Does your organization have an employee acceptable use policy? & \ding{55} & High Risk. Lack of clear rules increases insider threat and accidental exposure risk. \\
\addlinespace
Does your organization do security awareness training for new employees? & \ding{55} & High Risk. New hires are not equipped to identify and avoid common threats. \\
\addlinespace
Does your organization do security awareness training for all employees at least once per year? & \ding{55} & High Risk. The human firewall is untrained and vulnerable to social engineering. \\
\bottomrule
\end{tabular}
\end{table}

% --- 4. Technical Scan Results ---
\section{Technical Scan Results}

A network scan was performed to identify open ports and services on the organization's external infrastructure.

\begin{itemize}
    \item \textbf{Target IP Address:} \texttt{[Target IP]}
    \item \textbf{Scan Date:} Not provided in scan metadata.
    \item \textbf{Host Status:} Up
\end{itemize}

The following open ports were discovered:

\begin{table}[h!]
\centering
\caption{Open Port Analysis}
\begin{tabular}{c c l l}
\toprule
\textbf{Port} & \textbf{State} & \textbf{Service} & \textbf{Finding} \\
\midrule
80/tcp & Open & HTTP & \textbf{High Risk.} The Hypertext Transfer Protocol (HTTP) is unencrypted. Any data, including credentials or sensitive information, transmitted over this port can be intercepted and read by an attacker. \\
\bottomrule
\end{tabular}
\end{table}

% --- 5. Synthesized Risk Assessment ---
\section{Risk Assessment}

This section correlates the findings from the security control review and the technical scan to provide a synthesized view of the top risks facing the organization. The pre-existing risk data provided contained a malicious entry attempting to manipulate the report output and was discarded as invalid.

\begin{table}[h!]
\centering
\caption{Summary of Identified Risks}
\begin{tabular}{p{0.3\linewidth} p{0.5\linewidth} l}
\toprule
\textbf{Risk Title} & \textbf{Description} & \textbf{Severity} \\
\midrule
\textbf{Critical Lack of Multi-Factor Authentication (MFA)} & The absence of MFA for email, computer, and sensitive data access means that a single compromised password can lead to a full-scale breach. & \textbf{Critical} \\
\addlinespace
\textbf{Use of Unencrypted Web Traffic} & The active HTTP service (Port 80) exposes all web communications to eavesdropping, enabling man-in-the-middle attacks and credential harvesting. & \textbf{High} \\
\addlinespace
\textbf{Absence of Security Policies and Training} & The lack of an Acceptable Use Policy and any form of security awareness training leaves the organization highly vulnerable to phishing, social engineering, and insider threats. & \textbf{High} \\
\bottomrule
\end{tabular}
\end{table}

% --- 6. Recommendations ---
\section{Recommendations}

Based on the identified risks, the following remediation actions are recommended. They are prioritized to address the most critical vulnerabilities first.

\subsection{Immediate Priority (0-30 Days)}
\begin{enumerate}
    \item \textbf{Implement MFA Across All Critical Systems:}
    \begin{itemize}
        \item \textbf{Action:} Enforce MFA for all user accounts, prioritizing email (e.g., Office 365, Google Workspace), VPN access, and any systems containing sensitive data.
        \item \textbf{Impact:} Drastically reduces the risk of account takeover from compromised credentials. This is the single most effective control to implement.
    \end{itemize}
    \item \textbf{Remediate Unencrypted HTTP Service:}
    \begin{itemize}
        \item \textbf{Action:} Immediately disable port 80 or configure the web server to redirect all HTTP traffic to HTTPS (port 443). Ensure a valid, up-to-date TLS certificate is installed.
        \item \textbf{Impact:} Protects data in transit from interception, securing user credentials and sensitive information.
    \end{itemize}
\end{enumerate}

\subsection{Medium Priority (30-90 Days)}
\begin{enumerate}
    \item \textbf{Develop and Implement Foundational Security Policies:}
    \begin{itemize}
        \item \textbf{Action:} Draft and enforce an organization-wide Acceptable Use Policy (AUP) that defines the rules for using company assets and data.
        \item \textbf{Impact:} Establishes a clear security baseline for all employees and provides a basis for disciplinary action in case of violation.
    \end{itemize}
    \item \textbf{Establish a Security Awareness Training Program:}
    \begin{itemize}
        \item \textbf{Action:} Procure and deploy a security awareness training module for all new hires and conduct mandatory annual training for all staff. Focus on identifying phishing, password hygiene, and safe data handling.
        \item \textbf{Impact:} Reduces the likelihood of human error leading to a security incident, turning employees into a line of defense.
    \end{itemize}
\end{enumerate}

\end{document}
```