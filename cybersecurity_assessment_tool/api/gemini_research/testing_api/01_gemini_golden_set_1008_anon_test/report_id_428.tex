```latex
\documentclass[12pt, a4paper]{article}

% Preamble: Required Packages
\usepackage[margin=1in]{geometry}
\usepackage{pifont} % For checkmarks and crosses
\usepackage{booktabs} % For professional tables
\usepackage{hyperref} % For clickable links
\usepackage{url} % For URL formatting
\usepackage{seqsplit} % To split long strings in tt font
\usepackage{graphicx}
\usepackage[table]{xcolor}
\usepackage{fancyhdr}
\usepackage{lastpage}
\usepackage{datetime}

% --- Document Setup ---
\hypersetup{
    colorlinks=true,
    linkcolor=blue,
    filecolor=magenta,      
    urlcolor=cyan,
    pdftitle={Cybersecurity Posture Report},
    pdfpagemode=FullScreen,
}

% --- Header and Footer ---
\pagestyle{fancy}
\fancyhf{} % Clear all header and footer fields
\fancyhead[L]{Cybersecurity Posture Report}
\fancyhead[R]{\textbf{[Organization Name]}}
\fancyfoot[C]{\thepage\ of \pageref{LastPage}}
\renewcommand{\headrulewidth}{0.4pt}
\renewcommand{\footrulewidth}{0.4pt}

% --- Custom Commands ---
\newcommand{\yes}{\ding{51}}
\newcommand{\no}{\ding{55}}
\newcommand{\riskHigh}[1]{\colorbox{red!80}{\color{white}\textbf{#1}}}
\newcommand{\riskMedium}[1]{\colorbox{orange!80}{\color{white}\textbf{#1}}}
\newcommand{\riskLow}[1]{\colorbox{yellow!80}{\color{black}\textbf{#1}}}
\newcommand{\riskInfo}[1]{\colorbox{blue!60}{\color{white}\textbf{#1}}}

\begin{document}

% --- Title Page ---
\begin{titlepage}
    \centering
    \vspace*{2cm}
    
    \includegraphics[width=0.4\textwidth]{example-image-a} % Placeholder logo
    
    \vspace{1.5cm}
    
    {\Huge \textbf{Cybersecurity Posture Report}\par}
    
    \vspace{1.5cm}
    
    {\Large \textbf{Prepared for:}\\ \vspace{0.5cm} \textbf{[Organization Name]}}
    
    \vspace{2cm}
    
    {\large \today}
    
    \vfill
    
    {\large \textit{This report contains sensitive information and should be handled with care. Distribution is restricted to authorized personnel only.}}
    
\end{titlepage}

\tableofcontents
\newpage

% --- Section 1: Executive Summary ---
\section{Executive Summary}
This report provides a comprehensive analysis of the cybersecurity posture for \textbf{[Organization Name]}, based on a review of organizational security controls, an external network scan, and pre-existing risk data. The assessment was conducted on \today.

Overall, the organization demonstrates a foundational level of security awareness, particularly with security training and MFA for email and sensitive systems. However, several critical gaps were identified that significantly increase the organization's risk profile.

Key findings include:
\begin{itemize}
    \item \textbf{Critical Policy Gap:} The absence of an employee Acceptable Use Policy (AUP) creates ambiguity regarding the secure use of company assets and data.
    \item \textbf{Critical Endpoint Security Gap:} The lack of mandatory Multi-Factor Authentication (MFA) for computer logins exposes the organization to significant risk from compromised credentials, potentially leading to unauthorized access and lateral movement within the network.
    \item \textbf{High-Risk Network Exposure:} The external network scan revealed an open port 80 (HTTP) on a public-facing asset. This service transmits data in cleartext, exposing any information, including potential credentials, to interception.
\end{itemize}

These findings indicate a need for immediate remediation to strengthen endpoint security, formalize security policies, and secure external-facing services. This report provides detailed analysis and actionable recommendations to address these identified risks.

% --- Section 2: Organizational Information ---
\section{Organizational Information}
The following details were used as the basis for this assessment. As per the provided data, placeholder values are used where specific information was not available.

\begin{table}[h!]
\centering
\begin{tabular}{@{}ll@{}}
\toprule
\textbf{Attribute} & \textbf{Value} \\ \midrule
Organization Name & \textbf{[Organization Name]} \\
Primary Domain & \texttt{[Domain]} \\
External IP Address (Source) & \texttt{[Client IP]} \\
Report Date & \today \\ \bottomrule
\end{tabular}
\caption{Client Organizational Details.}
\label{tab:org_info}
\end{table}

% --- Section 3: Security Control Review ---
\section{Security Control Review (Questionnaire)}
An assessment of internal security controls was conducted via a questionnaire. The responses indicate a mixed implementation of security best practices. While security awareness training and MFA for high-value assets are in place, significant gaps exist in endpoint and policy controls.

\begin{table}[h!]
\centering
\rowcolors{2}{gray!10}{white}
\begin{tabular}{@{}p{0.7\linewidth}cc@{}}
\toprule
\textbf{Control Question} & \textbf{Response} & \textbf{Status} \\ \midrule
Do you require MFA to access email? & Yes & \yes \\
\rowcolor{red!15} Do you require MFA to log into computers? & No & \no \\
Do you require MFA to access sensitive data systems? & Yes & \yes \\
\rowcolor{red!15} Does your organization have an employee acceptable use policy? & No & \no \\
Does your organization do security awareness training for new employees? & Yes & \yes \\
Does your organization do security awareness training for all employees at least once per year? & Yes & \yes \\ \bottomrule
\end{tabular}
\caption{Security Control Questionnaire Results. Highlighted rows indicate significant security gaps.}
\label{tab:controls}
\end{table}

The "No" responses for MFA on computers and the lack of an Acceptable Use Policy are primary contributors to the organization's risk profile and are detailed further in Section \ref{sec:risks}.

% --- Section 4: Technical Scan Results ---
\section{Technical Scan Results}
An external network scan was performed using Nmap against the target IP address. The scan was non-intrusive and aimed to identify open ports and publicly exposed services.

\begin{itemize}
    \item \textbf{Target IP Address:} \texttt{[Target IP]}
    \item \textbf{Host Status:} Up
\end{itemize}

The scan identified the following open port:

\begin{table}[h!]
\centering
\begin{tabular}{@{}llll@{}}
\toprule
\textbf{Port} & \textbf{State} & \textbf{Inferred Service} & \textbf{Notes} \\ \midrule
80/tcp & open & HTTP & Unencrypted web traffic. High risk. \\ \bottomrule
\end{tabular}
\caption{Open Ports Identified on \texttt{[Target IP]}.}
\label{tab:nmap}
\end{table}

\subsection{Analysis of Findings}
The presence of an open port 80 (HTTP) is a significant security concern. The HTTP protocol does not encrypt data in transit. This means that any information exchanged between a user and the server, including usernames, passwords, or other sensitive data, can be easily intercepted and read by an attacker on the same network (e.g., public Wi-Fi) or in a man-in-the-middle position. Best practice dictates that all web traffic should be encrypted using HTTPS (port 443) with a valid TLS certificate.

% --- Section 5: Identified Risks & Vulnerabilities ---
\section{Identified Risks \& Vulnerabilities}
\label{sec:risks}
This section synthesizes findings from the security control review, technical scan, and pre-existing risk data. Each risk is assigned a severity level to aid in prioritization.

\begin{table}[h!]
\centering
\begin{tabular}{@{}p{0.1\linewidth}p{0.3\linewidth}p{0.15\linewidth}p{0.35\linewidth}@{}}
\toprule
\textbf{ID} & \textbf{Risk Title} & \textbf{Severity} & \textbf{Description} \\ \midrule
RISK-001 & Lack of MFA on Employee Computers & \riskHigh{High} & The absence of MFA on workstations allows an attacker with stolen credentials to gain direct access to the internal network, bypassing a critical security layer. \\
\addlinespace
RISK-002 & Unencrypted Web Service (HTTP) Exposed & \riskHigh{High} & Port 80 is open to the internet, serving content over HTTP. This exposes all transmitted data to interception and manipulation. \\
\addlinespace
RISK-003 & Missing Employee Acceptable Use Policy & \riskMedium{Medium} & Without a formal AUP, there are no established rules for employee use of IT assets, increasing the risk of insider threats and unsafe practices. \\
\addlinespace
RISK-004 & Pre-existing Risk: System Overridden & \riskInfo{Info} & A pre-existing risk was noted with a CVSS score of 0.0. Overview: "System Overridden". This is logged for tracking but is considered non-critical. \\ \bottomrule
\end{tabular}
\caption{Summary of Identified Risks.}
\label{tab:risks}
\end{table}

% --- Section 6: Recommendations ---
\section{Recommendations}
The following actionable recommendations are provided to mitigate the identified risks. They are prioritized based on severity.

\subsection{RISK-001: Lack of MFA on Employee Computers (High)}
\begin{itemize}
    \item \textbf{Immediate Action:} Develop a project plan to deploy Multi-Factor Authentication for all employee computer logins (Windows, macOS, Linux).
    \item \textbf{Long-Term Strategy:} Procure and implement an MFA solution compatible with your environment. Options include Windows Hello for Business, Duo Security, Okta, or other identity providers that support endpoint MFA. Prioritize deployment for privileged users (administrators) and executives.
\end{itemize}

\subsection{RISK-002: Unencrypted Web Service (HTTP) Exposed (High)}
\begin{itemize}
    \item \textbf{Immediate Action:} Identify the service running on port 80 of \texttt{[Target IP]}. If it is a web server, acquire and install a valid TLS certificate from a trusted Certificate Authority (e.g., Let's Encrypt).
    \item \textbf{Long-Term Strategy:} Reconfigure the web server to redirect all HTTP traffic to HTTPS (port 443). Implement HTTP Strict Transport Security (HSTS) to ensure that browsers only connect to the site over a secure connection. Disable port 80 at the firewall if it is not needed for the HTTPS redirect.
\end{itemize}

\subsection{RISK-003: Missing Employee Acceptable Use Policy (Medium)}
\begin{itemize}
    \item \textbf{Immediate Action:} Draft a comprehensive Acceptable Use Policy (AUP) that clearly defines the rules and expectations for using company computers, networks, and data.
    \item \textbf{Long-Term Strategy:} Ratify the AUP with management and legal counsel. Implement a process to ensure all new and existing employees read and formally acknowledge the policy. Schedule annual reviews to keep the policy current.
\end{itemize}

\end{document}
```