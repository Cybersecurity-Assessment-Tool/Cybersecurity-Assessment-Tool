```latex
\documentclass[12pt]{article}

% Preamble: Required Packages
\usepackage[margin=1in]{geometry} % Sets page margins
\usepackage{pifont}               % For dingbats symbols like checkmarks
\usepackage{booktabs}             % For professional-looking tables
\usepackage[hidelinks]{hyperref}  % For clickable links without boxes
\usepackage{url}                  % For formatting URLs
\usepackage{seqsplit}             % To split long strings in texttt
\usepackage{graphicx}             % For potential logos
\usepackage{xcolor}               % For colors

% Define custom colors for severity
\definecolor{criticalred}{HTML}{D12B2B}
\definecolor{highorange}{HTML}{E67E22}
\definecolor{mediumyellow}{HTML}{F1C40F}

% Document Information
\title{Cybersecurity Posture Assessment Report}
\author{Cybersecurity Analysis Division}
\date{\today}

\begin{document}

\maketitle
\thispagestyle{empty}
\newpage

\tableofcontents
\newpage

%======================================================================
% 1. EXECUTIVE SUMMARY
%======================================================================
\section{Executive Summary}

This report provides a comprehensive analysis of the cybersecurity posture for \textbf{[Organization Name]}, based on technical network scans, a review of organizational security controls, and an assessment of pre-existing risks. The assessment was conducted on \today.

The analysis revealed a \textbf{critical risk} related to the direct exposure of a Remote Desktop Protocol (RDP) service on port 3389 to the public internet at \texttt{[Target IP]}. This configuration represents a significant and immediate threat, as it is a primary vector for ransomware attacks and unauthorized access.

Furthermore, the organizational review identified a \textbf{high-risk gap} in the security awareness training program. The lack of mandatory training for new and existing employees significantly increases the organization's susceptibility to social engineering and phishing attacks, which could lead to credential compromise and subsequent exploitation of the exposed RDP service.

While the organization has implemented commendable Multi-Factor Authentication (MFA) controls for email, computer logins, and sensitive systems, these measures do not fully mitigate the immediate threat posed by the exposed RDP service.

Immediate remediation of the exposed service and the implementation of a comprehensive security awareness program are strongly recommended to reduce the organization's risk profile.

%======================================================================
% 2. ORGANIZATIONAL INFORMATION
%======================================================================
\section{Organizational Information}

This section details the information provided for the assessment. The data has been anonymized as per the engagement's template mode.

\begin{tabular}{@{}ll}
\toprule
\textbf{Attribute} & \textbf{Value} \\
\midrule
Organization Name & \textbf{[Organization Name]} \\
Primary Domain & \texttt{[Domain]} \\
External IP Address (Scanned) & \texttt{[Client IP]} \\
\bottomrule
\end{tabular}

%======================================================================
% 3. SECURITY CONTROL REVIEW
%======================================================================
\section{Security Control Review}

An assessment of administrative and organizational security controls was conducted via a standardized questionnaire. The responses indicate a strong foundation in identity and access management but reveal critical deficiencies in employee security training.

\begin{tabular}{@{}p{0.7\linewidth}cc@{}}
\toprule
\textbf{Control Question} & \textbf{Response} & \textbf{Status} \\
\midrule
Do you require MFA to access email? & Yes & \ding{51} \\
Do you require MFA to log into computers? & Yes & \ding{51} \\
Do you require MFA to access sensitive data systems? & Yes & \ding{51} \\
Does your organization have an employee acceptable use policy? & Yes & \ding{51} \\
Does your organization do security awareness training for new employees? & No & {\color{criticalred}\ding{55}} \\
Does your organization do security awareness training for all employees at least once per year? & No & {\color{criticalred}\ding{55}} \\
\bottomrule
\end{tabular}

\subsection*{Analysis}
The "No" responses for security awareness training are significant findings. A lack of a formal training program leaves the organization vulnerable to phishing, business email compromise, and other social engineering tactics. This gap is particularly concerning when combined with any externally facing services.

%======================================================================
% 4. TECHNICAL SCAN RESULTS
%======================================================================
\section{Technical Scan Results}

A network scan was performed against the target host \texttt{[Target IP]} to identify open ports and exposed services.

\subsection*{Host Status: UP}
The target host was responsive at the time of the scan.

\subsection*{Open Ports}
The following ports were found to be open and accessible from the public internet:

\begin{tabular}{@{}llll@{}}
\toprule
\textbf{Port} & \textbf{State} & \textbf{Service Name} & \textbf{Product / Version} \\
\midrule
3389/tcp & open & ms-wbt-server & Not Disclosed \\
\bottomrule
\end{tabular}

\subsection*{Analysis of Findings}
The scan confirms that port \textbf{3389/tcp}, the standard port for Microsoft's Remote Desktop Protocol (RDP), is open. Exposing RDP directly to the internet is a highly discouraged practice due to its frequent targeting by attackers for:
\begin{itemize}
    \item \textbf{Brute-force attacks:} Automated attempts to guess user credentials.
    \item \textbf{Credential stuffing:} Using credentials stolen from other breaches.
    \item \textbf{Exploitation of vulnerabilities:} Such as the "BlueKeep" (CVE-2019-0708) vulnerability.
\end{itemize}
This finding directly corroborates the pre-existing risk documented in the subsequent section and poses an immediate threat to the organization's network integrity.

%======================================================================
% 5. RISK ASSESSMENT SUMMARY
%======================================================================
\section{Risk Assessment Summary}

This section synthesizes findings from the security control review, technical scan, and pre-existing risk data into a consolidated risk register.

\begin{tabular}{@{}p{0.2\linewidth}p{0.15\linewidth}p{0.6\linewidth}@{}}
\toprule
\textbf{Risk Name} & \textbf{Severity} & \textbf{Description} \\
\midrule
\textbf{RDP Exposure} & {\color{criticalred}\textbf{Critical (9.0)}} & The technical scan confirmed that the Remote Desktop Protocol (RDP) service is exposed on port 3389 at \texttt{[Target IP]}. This aligns with the pre-existing risk data and presents a critical entry point for unauthorized access and ransomware deployment. \\
\addlinespace
\textbf{Lack of Security Awareness Training} & {\color{highorange}\textbf{High}} & The organization does not provide security training to new or existing employees. This administrative gap creates a high-risk human factor, making the organization highly susceptible to phishing and social engineering attacks that could lead to credential compromise. \\
\bottomrule
\end{tabular}

%======================================================================
% 6. RECOMMENDATIONS
%======================================================================
\section{Recommendations}

The following prioritized recommendations are provided to mitigate the identified risks and improve the overall security posture of \textbf{[Organization Name]}.

\subsection*{Priority 1: Remediate Exposed RDP Service (Critical)}
\begin{enumerate}
    \item \textbf{Immediate Action:} If RDP access is not required, immediately block port 3389 at the network firewall for all inbound traffic from the internet.
    \item \textbf{Short-Term Action:} If remote access is required, implement network-level access controls to restrict RDP access to only trusted source IP addresses.
    \item \textbf{Long-Term Solution:} Decommission direct RDP access. Implement a secure remote access solution, such as a Virtual Private Network (VPN) or a Zero Trust Network Access (ZTNA) gateway, that requires Multi-Factor Authentication (MFA).
\end{enumerate}

\subsection*{Priority 2: Establish a Security Awareness Program (High)}
\begin{enumerate}
    \item \textbf{Immediate Action:} Procure and implement a security awareness training platform.
    \item \textbf{Short-Term Action:} Develop a mandatory training module for all new employees as part of the onboarding process.
    \item \textbf{Long-Term Solution:} Establish a formal policy requiring all employees to complete security awareness training annually. Supplement this with periodic phishing simulation campaigns to measure effectiveness and provide targeted training.
\end{enumerate}

\end{document}
```