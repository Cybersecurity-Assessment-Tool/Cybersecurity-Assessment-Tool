```latex
\documentclass[12pt]{article}

% Preamble: Required Packages
\usepackage[margin=1in]{geometry}
\usepackage{pifont} % For checkmarks and crosses
\usepackage{booktabs} % For professional tables
\usepackage{hyperref} % For hyperlinks and PDF metadata
\usepackage{url} % For formatting URLs
\usepackage{seqsplit} % For splitting long strings without breaking
\usepackage{xcolor} % For colors in text

% Document Metadata
\hypersetup{
    colorlinks=true,
    linkcolor=blue,
    filecolor=magenta,      
    urlcolor=cyan,
    pdftitle={Cybersecurity Posture Assessment Report},
    pdfauthor={Cybersecurity Analyst},
    pdfsubject={Security Analysis},
    pdfkeywords={Cybersecurity, Report, Analysis},
}

% Define colors for severity
\definecolor{criticalred}{HTML}{D10000}
\definecolor{highorange}{HTML}{E25F00}
\definecolor{mediumyellow}{HTML}{F0C200}
\definecolor{lowblue}{HTML}{0073E6}

% Custom commands for severity
\newcommand{\sevCRITICAL}{\textcolor{criticalred}{\textbf{Critical}}}
\newcommand{\sevHIGH}{\textcolor{highorange}{\textbf{High}}}
\newcommand{\sevMEDIUM}{\textcolor{mediumyellow}{\textbf{Medium}}}
\newcommand{\sevLOW}{\textcolor{lowblue}{\textbf{Low}}}

\begin{document}

% --- TITLE PAGE ---
\begin{titlepage}
    \centering
    \vspace*{1cm}
    \Huge\textbf{Cybersecurity Posture Assessment Report}
    \vspace{1.5cm}
    \vfill
    \large
    \textbf{Prepared for:}\\
    \vspace{0.5cm}
    \textbf{[Organization Name]}
    \vspace{2cm}
    \textbf{Date of Report:}\\
    \today
    \vspace{1.5cm}
    \textbf{Author:}\\
    Cybersecurity Analyst
\end{titlepage}

\tableofcontents
\newpage

% --- EXECUTIVE SUMMARY ---
\section{Executive Summary}
This report provides a comprehensive analysis of the cybersecurity posture for \textbf{[Organization Name]}. The assessment is based on a synthesis of an external network scan, a review of internal security controls via a questionnaire, and an evaluation of pre-existing risks.

The external network scan of the target IP address (\texttt{[Client IP]}) revealed a strong perimeter security posture, with no open ports detected. This indicates a minimal external attack surface, which is a significant strength.

However, the security control review identified several \sevCRITICAL{} and \sevHIGH{} risk gaps in internal policies and procedures. The most pressing concerns are the lack of Multi-Factor Authentication (MFA) for email and computer access, and the complete absence of a security awareness training program for employees. These deficiencies expose the organization to significant threats, including phishing, business email compromise, and ransomware, despite the strong network perimeter.

This report details these findings and provides actionable recommendations to mitigate the identified risks and strengthen the organization's overall security resilience.

% --- ORGANIZATIONAL INFORMATION ---
\section{Organizational Information}
The following details were used as the basis for this assessment. Where information was not provided, placeholders have been used.

\begin{table}[h!]
\centering
\begin{tabular}{@{}ll@{}}
\toprule
\textbf{Attribute} & \textbf{Value} \\ \midrule
Organization Name & \textbf{[Organization Name]} \\
Primary Email Domain & \texttt{[Domain]} \\
External IP Scanned & \texttt{[Client IP]} \\ \bottomrule
\end{tabular}
\caption{Client Organizational Data}
\label{tab:org_data}
\end{table}

% --- SECURITY CONTROL REVIEW ---
\section{Security Control Review}
An assessment of internal security controls was conducted based on a questionnaire. The responses highlight critical areas where security policies and enforcement are lacking. A "No" response indicates a potential security gap that increases organizational risk.

\begin{table}[h!]
\centering
\begin{tabular}{@{}p{0.75\linewidth}c@{}}
\toprule
\textbf{Control Question} & \textbf{Response} \\ \midrule
Do you require MFA to access email? & \ding{55} \\
Do you require MFA to log into computers? & \ding{55} \\
Do you require MFA to access sensitive data systems? & \ding{51} \\
Does your organization have an employee acceptable use policy? & \ding{55} \\
Does your organization do security awareness training for new employees? & \ding{55} \\
Does your organization do security awareness training for all employees at least once per year? & \ding{55} \\ \bottomrule
\end{tabular}
\caption{Security Controls Questionnaire Results (\ding{51}=Yes, \ding{55}=No)}
\label{tab:controls}
\end{table}

\textbf{Analysis:} The lack of MFA on email and endpoints, combined with the absence of any security awareness training, represents a critical vulnerability. While MFA is enforced for sensitive systems, primary communication and access channels remain unprotected, making them prime targets for credential theft and social engineering attacks.

% --- TECHNICAL SCAN RESULTS ---
\section{Technical Scan Results}
An external network vulnerability scan was performed using Nmap against the provided target IP address.

\begin{itemize}
    \item \textbf{Target IP:} \texttt{[Target IP]}
    \item \textbf{Scan Type:} Standard TCP Port Scan
    \item \textbf{Scan Date:} Not specified in scan data.
\end{itemize}

\subsection{Key Findings}
The scan results were positive from an external security perspective. The target host was responsive, but all scanned ports were found to be in a \textbf{closed} state.

\begin{itemize}
    \item \textbf{Host Status:} Up
    \item \textbf{Open Ports:} 0
    \item \textbf{Filtered/Closed Ports:} All scanned ports.
\end{itemize}

\textbf{Conclusion:} A host with no open ports presents a minimal attack surface to external threats. This suggests a well-configured firewall or network access control list (ACL) is in place, effectively blocking unsolicited inbound traffic. This is a commendable security practice.

% --- RISK ASSESSMENT ---
\section{Risk Assessment}
This section correlates the findings from the security control review and the technical scan. While no technical vulnerabilities were discovered externally, significant policy and procedural risks were identified. No pre-existing vulnerabilities were reported.

\begin{table}[h!]
\centering
\begin{tabular}{@{}p{0.1\linewidth}p{0.5\linewidth}p{0.2\linewidth}@{}}
\toprule
\textbf{Risk ID} & \textbf{Finding} & \textbf{Severity} \\ \midrule
RISK-001 & \textbf{Lack of MFA on Email and Endpoints:} User accounts for email and computer logins are protected only by passwords, making them highly susceptible to compromise via phishing, brute-force attacks, or credential stuffing. & \sevCRITICAL \\
\addlinespace
RISK-002 & \textbf{Absence of Security Awareness Training:} Without training, employees are unlikely to recognize and report phishing attempts or other social engineering tactics, making them the weakest link in the organization's defense. & \sevCRITICAL \\
\addlinespace
RISK-003 & \textbf{No Employee Acceptable Use Policy (AUP):} The lack of a formal AUP creates ambiguity regarding the proper use of company assets and data, increasing the risk of insider threat, data leakage, and non-compliance. & \sevHIGH \\
\bottomrule
\end{tabular}
\caption{Summary of Identified Risks}
\label{tab:risks}
\end{table}

% --- RECOMMENDATIONS ---
\section{Recommendations}
The following actions are recommended to mitigate the identified risks and improve the overall security posture of \textbf{[Organization Name]}.

\subsection{Remediation for RISK-001: Implement MFA}
\begin{itemize}
    \item \textbf{Immediate Action:} Enable and enforce MFA for all users on the primary email system (e.g., Microsoft 365, Google Workspace). This is the single most effective control to prevent business email compromise.
    \item \textbf{Short-Term Action:} Deploy an MFA solution for all computer logins (e.g., Windows Hello for Business, Duo, Okta). This protects against unauthorized access to endpoints if credentials are stolen.
    \item \textbf{Policy:} Update IT policies to mandate MFA for all new systems and services that support it.
\end{itemize}

\subsection{Remediation for RISK-002: Establish Security Training}
\begin{itemize}
    \item \textbf{Immediate Action:} Enroll all current employees in a foundational security awareness training course. This should cover, at a minimum, phishing, password security, and safe browsing habits.
    \item \textbf{Ongoing Action:} Implement a formal security awareness program that includes:
    \begin{itemize}
        \item Mandatory training for all new hires during onboarding.
        \item Annual refresher training for all staff.
        \item Regular (e.g., quarterly) simulated phishing campaigns to test and reinforce learning.
    \end{itemize}
\end{itemize}

\subsection{Remediation for RISK-003: Develop an AUP}
\begin{itemize}
    \item \textbf{Short-Term Action:} Draft a clear and concise Acceptable Use Policy (AUP) that governs the use of all company-owned IT assets, including computers, networks, and data.
    \item \textbf{Implementation:} Require all employees to read and formally acknowledge the AUP. Integrate this acknowledgment process into the new hire onboarding checklist.
    \item \textbf{Review:} The AUP should be reviewed and updated annually by management to ensure it remains relevant.
\end{itemize}

\end{document}
```