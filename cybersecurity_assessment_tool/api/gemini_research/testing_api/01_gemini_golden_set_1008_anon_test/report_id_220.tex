```latex
\documentclass[12pt]{article}

% --- PACKAGE IMPORTS ---
\usepackage[a4paper, margin=1in]{geometry}
\usepackage{pifont} % For checkmarks and crosses
\usepackage{booktabs} % For professional tables
\usepackage{hyperref} % For hyperlinks
\usepackage{url} % For URL formatting
\usepackage{seqsplit} % To split long strings without breaking
\usepackage{graphicx}
\usepackage{fancyhdr}

% --- DOCUMENT METADATA & SETUP ---
\hypersetup{
    colorlinks=true,
    linkcolor=black,
    urlcolor=blue,
    pdftitle={Cybersecurity Assessment Report},
    pdfauthor={Automated Security Analysis System},
    pdfsubject={Security Posture Review},
    pdfkeywords={Cybersecurity, Nmap, Risk Assessment}
}

\pagestyle{fancy}
\fancyhf{}
\lhead{Cybersecurity Assessment Report}
\rhead{\textbf{[Organization Name]}}
\cfoot{\thepage}

% --- DOCUMENT START ---
\begin{document}

% --- TITLE PAGE ---
\begin{titlepage}
    \centering
    \vspace*{2cm}
    
    \Huge
    \textbf{Cybersecurity Assessment Report}
    
    \vspace{1.5cm}
    
    \Large
    Prepared for: \\
    \vspace{0.5cm}
    \textbf{[Organization Name]}
    
    \vspace{2cm}
    
    \large
    Date of Report: \today
    
    \vfill
    
    \normalsize
    \textit{This report contains sensitive information. Distribution should be limited to authorized personnel only.}
    
\end{titlepage}

\tableofcontents
\newpage

% --- SECTION 1: EXECUTIVE SUMMARY ---
\section{Executive Summary}

This report details the findings of a cybersecurity assessment conducted for \textbf{[Organization Name]}. The assessment combined an external network perimeter scan with a review of internal security controls and pre-existing risks.

The external network scan of the target IP address (\texttt{[Target IP]}) revealed a positive security posture, with \textbf{no open ports or exposed services detected}. This suggests a well-configured firewall and a strong network perimeter.

However, the review of organizational security controls identified \textbf{several critical and high-risk deficiencies}. The most significant gaps include:
\begin{itemize}
    \item \textbf{Lack of Multi-Factor Authentication (MFA):} MFA is not enforced for accessing email or for computer logins. This exposes the organization to a high risk of account compromise through phishing or password theft.
    \item \textbf{Absence of an Acceptable Use Policy (AUP):} Without a formal AUP, employees lack clear guidelines on the secure use of company assets, increasing the likelihood of insider threats and unintentional data breaches.
    \item \textbf{Incomplete Security Awareness Training:} While annual training is in place, new employees do not receive security training upon being hired, leaving a critical window of vulnerability.
\end{itemize}

While the network perimeter appears secure, the identified policy and procedural gaps represent a significant threat to the organization's overall security. Immediate action is recommended to address these vulnerabilities, focusing on the implementation of MFA and the development of foundational security policies.

% --- SECTION 2: ORGANIZATIONAL INFORMATION ---
\section{Assessment Scope and Information}

The information below was used as the basis for this assessment. Where data was not provided, placeholders have been used.

\begin{tabular}{@{}ll}
    \toprule
    \textbf{Attribute} & \textbf{Value} \\
    \midrule
    Organization Name & \textbf{[Organization Name]} \\
    Primary Email Domain & \seqsplit{\texttt{[Domain]}} \\
    Client External IP & \seqsplit{\texttt{[Client IP]}} \\
    Scanned Target IP & \seqsplit{\texttt{[Target IP]}} \\
    Scan Date & \today \\
    \bottomrule
\end{tabular}

% --- SECTION 3: SECURITY CONTROL REVIEW ---
\section{Security Control Review}

The following table summarizes the organization's responses to a security controls questionnaire. A checkmark (\ding{51}) indicates a positive control is in place, while a cross (\ding{55}) indicates a control gap that introduces risk.

\begin{table}[h!]
\centering
\begin{tabular}{@{}p{0.8\linewidth}c@{}}
    \toprule
    \textbf{Control Question} & \textbf{Response} \\
    \midrule
    Do you require MFA to access email? & \ding{55} \\
    Do you require MFA to log into computers? & \ding{55} \\
    Do you require MFA to access sensitive data systems? & \ding{51} \\
    Does your organization have an employee acceptable use policy? & \ding{55} \\
    Does your organization do security awareness training for new employees? & \ding{55} \\
    Does your organization do security awareness training for all employees at least once per year? & \ding{51} \\
    \bottomrule
\end{tabular}
\caption{Organizational Security Controls Questionnaire Results.}
\end{label{tab:controls}
\end{table}

The identified gaps in MFA implementation and foundational security policies are the primary drivers of risk identified in this report.

% --- SECTION 4: TECHNICAL SCAN RESULTS ---
\section{Technical Scan Results}

An external network scan was performed against the designated target IP address to identify any exposed services or potential vulnerabilities.

\begin{itemize}
    \item \textbf{Target IP:} \texttt{[Target IP]}
    \item \textbf{Scan Tool:} Nmap
    \item \textbf{Overall Status:} The host was responsive (status: up).
    \item \textbf{Findings:} The scan confirmed that all 1000 most common TCP ports were in a \textbf{closed} state. No open ports or running services were discovered. This is a strong indicator of a properly configured firewall that denies unsolicited inbound traffic.
\end{itemize}

There were no technical vulnerabilities identified from the external network scan.

% --- SECTION 5: RISK ASSESSMENT ---
\section{Risk Assessment}

This section correlates the findings from the security control review and technical scan to provide a consolidated list of identified risks. No pre-existing vulnerabilities were reported.

\begin{table}[h!]
\centering
\begin{tabular}{@{}p{0.25\linewidth}p{0.5\linewidth}p{0.15\linewidth}@{}}
    \toprule
    \textbf{Identified Risk} & \textbf{Description} & \textbf{Severity} \\
    \midrule
    \textbf{Account Compromise via Lack of MFA} & The absence of MFA for email and computer access makes user accounts highly susceptible to takeover from stolen credentials. A single compromised password could grant an attacker broad access. & \textbf{Critical} \\
    \addlinespace
    \textbf{Insider Threat and Policy Non-Compliance} & Without a formal Acceptable Use Policy (AUP), there is no enforceable standard for employee behavior on corporate systems. This increases the risk of data misuse, unauthorized software installation, and unintentional breaches. & \textbf{High} \\
    \addlinespace
    \textbf{New Employee Vulnerability} & Failing to provide security awareness training during onboarding leaves new hires highly vulnerable to phishing and social engineering attacks before they are familiar with company security practices. & \textbf{High} \\
    \bottomrule
\end{tabular}
\caption{Summary of Identified Risks.}
\label{tab:risks}
\end{table}

% --- SECTION 6: RECOMMENDATIONS ---
\section{Recommendations}

The following actions are recommended to mitigate the identified risks and improve the overall security posture of \textbf{[Organization Name]}.

\begin{itemize}
    \item[\textbf{1.}] \textbf{(Critical) Implement Comprehensive MFA:}
    \begin{itemize}
        \item Immediately enforce MFA for all user access to email systems (e.g., Office 365, Google Workspace).
        \item Deploy an MFA solution for all computer and remote access (VPN) logins.
        \item This single control is the most effective defense against the majority of account compromise attacks.
    \end{itemize}
    \vspace{0.5cm}
    \item[\textbf{2.}] \textbf{(High) Develop and Implement an Acceptable Use Policy (AUP):}
    \begin{itemize}
        \item Create a formal AUP that clearly defines the rules for using company networks, computers, and data.
        \item Require all current and new employees to read and formally acknowledge the policy.
        \item Review and update the AUP at least annually.
    \end{itemize}
    \vspace{0.5cm}
    \item[\textbf{3.}] \textbf{(High) Enhance Security Awareness Training Program:}
    \begin{itemize}
        \item Integrate mandatory security awareness training into the new employee onboarding process, to be completed within the first week of employment.
        \item This training should cover key topics such as phishing identification, password hygiene, and the newly created AUP.
    \end{itemize}
    \vspace{0.5cm}
    \item[\textbf{4.}] \textbf{(Informational) Maintain Network Security Vigilance:}
    \begin{itemize}
        \item Continue the practice of maintaining a hardened network perimeter.
        \item Schedule regular, automated external vulnerability scans to ensure no new services are inadvertently exposed over time.
    \end{itemize}
\end{itemize}

\end{document}
```