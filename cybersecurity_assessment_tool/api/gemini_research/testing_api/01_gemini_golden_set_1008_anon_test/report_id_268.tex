```latex
\documentclass[12pt]{article}

% Preamble: Required Packages
\usepackage[margin=1in]{geometry}
\usepackage{pifont} % For checkmarks and crosses
\usepackage{booktabs} % For professional tables
\usepackage{hyperref} % For clickable links and metadata
\usepackage{url}      % For formatting URLs
\usepackage{seqsplit} % For splitting long strings without spaces
\usepackage{graphicx} % For potential logos/images
\usepackage{xcolor}   % For custom colors

% Hyperref Setup
\hypersetup{
    colorlinks=true,
    linkcolor=blue,
    filecolor=magenta,      
    urlcolor=cyan,
    pdftitle={Cybersecurity Assessment Report},
    pdfauthor={Cybersecurity Analyst},
    pdfsubject={Security Analysis},
    pdfkeywords={Security, Risk, Assessment},
    bookmarks=true
}

% Document Start
\begin{document}

% --- Title Page ---
\begin{titlepage}
    \centering
    \vspace*{1cm}
    
    \Huge
    \textbf{Cybersecurity Assessment Report}
    
    \vspace{1.5cm}
    
    \Large
    Prepared for: \\
    \vspace{0.5cm}
    \textbf{[Organization Name]}
    
    \vspace{2cm}
    
    \large
    \textbf{Date of Report:} \today
    
    \vfill
    
    \large
    \textbf{Generated By:} \\
    Cybersecurity Analyst
    
\end{titlepage}

\tableofcontents
\newpage

% --- Section 1: Executive Summary ---
\section{Executive Summary}

This report details the findings of a cybersecurity assessment conducted for \textbf{[Organization Name]}. The evaluation combined a technical network scan with a review of organizational security controls and pre-existing risks.

The primary finding is a significant disparity between the organization's technical and procedural security postures. The external network scan of the target system revealed a strong configuration with no open ports detected, which effectively minimizes the external attack surface.

However, the review of security controls identified several critical and high-risk deficiencies in administrative and policy-based safeguards. The most severe gaps include:
\begin{itemize}
    \item \textbf{Lack of Multi-Factor Authentication (MFA) for email access}, which exposes the organization to a high risk of business email compromise (BEC) and account takeovers.
    \item \textbf{Absence of a formal employee Acceptable Use Policy (AUP)}, leading to a lack of clear guidelines for system usage and data handling.
    \item \textbf{Complete lack of a security awareness training program}, leaving employees vulnerable to phishing and social engineering attacks.
\end{itemize}

While the current network perimeter appears secure, these procedural gaps represent a significant threat to the organization's overall security. Immediate action is recommended to address these findings, focusing first on implementing MFA for email services.

% --- Section 2: Organizational Information ---
\section{Organizational Information}

The following details were used as the basis for this assessment. As identity data was not provided, placeholders have been used.

\begin{itemize}
    \item \textbf{Organization Name:} \textbf{[Organization Name]}
    \item \textbf{Primary Email Domain:} \texttt{[Domain]}
    \item \textbf{External IP Address Assessed:} \texttt{[Client IP]}
\end{itemize}

% --- Section 3: Security Control Review ---
\section{Security Control Review}

A review of internal security controls was conducted via a standardized questionnaire. The responses indicate critical gaps in foundational security practices. A summary of the responses is provided in Table \ref{tab:controls}. The checkmark (\ding{51}) indicates a positive control is in place, while the cross (\ding{55}) indicates a control gap.

\begin{table}[h!]
\centering
\caption{Security Controls Questionnaire Results}
\label{tab:controls}
\begin{tabular}{@{}lc@{}}
\toprule
\textbf{Control Question} & \textbf{Response} \\
\midrule
Do you require MFA to access email? & \ding{55} \\
Do you require MFA to log into computers? & \ding{51} \\
Do you require MFA to access sensitive data systems? & \ding{51} \\
Does your organization have an employee acceptable use policy? & \ding{55} \\
Does your organization do security awareness training for new employees? & \ding{55} \\
Does your organization do security awareness training for all employees annually? & \ding{55} \\
\bottomrule
\end{tabular}
\end{table}

% --- Section 4: Technical Scan Results ---
\section{Technical Scan Results}

An external network scan was performed using Nmap to identify open ports and services on the designated target system.

\begin{itemize}
    \item \textbf{Target IP Address:} \texttt{[Target IP]}
    \item \textbf{Scan Findings:} The scan confirmed that the host is online and responsive. However, no open TCP or UDP ports were discovered within the scanned range. All non-open ports were reported as `closed`.
    \item \textbf{Analysis:} This result indicates a strong network perimeter security posture for the scanned host. By not exposing any services to the public internet, the organization significantly reduces its vulnerability to external network-based attacks. This is an excellent security practice.
\end{itemize}

% --- Section 5: Risk Assessment ---
\section{Risk Assessment}

This section correlates the findings from the security control review and technical scan. While no pre-existing vulnerabilities were reported and no technical vulnerabilities were found, the policy and procedural gaps create significant organizational risks. These are detailed in Table \ref{tab:risks}.

\begin{table}[h!]
\centering
\caption{Identified Risks and Severity}
\label{tab:risks}
\begin{tabular}{@{}lp{4cm}ll@{}}
\toprule
\textbf{Risk ID} & \textbf{Finding} & \textbf{Severity} & \textbf{Description} \\
\midrule
R-01 & No MFA for Email Access & \textbf{Critical} & The absence of MFA on email accounts makes them highly susceptible to compromise via phishing or credential stuffing. A compromised email account is a primary vector for data breaches and further attacks. \\
\addlinespace
R-02 & No Security Awareness Training Program & \textbf{High} & Without training, employees are the weakest link in security. They are more likely to click malicious links, fall for phishing scams, or mishandle sensitive data, creating an entry point for attackers. \\
\addlinespace
R-03 & No Employee Acceptable Use Policy (AUP) & \textbf{High} & Lack of a formal AUP leads to inconsistent and potentially unsafe user behavior. It also creates challenges in enforcing security rules and holding individuals accountable for misuse of company assets. \\
\bottomrule
\end{tabular}
\end{table}

% --- Section 6: Recommendations ---
\section{Recommendations}

Based on the identified risks, the following prioritized recommendations are provided to strengthen the security posture of \textbf{[Organization Name]}.

\subsection{Priority 1: Remediate Immediately}
\begin{description}
    \item[Implement MFA for Email (Risk R-01):] This is the most critical action to take. Enforce MFA for all users accessing the email system, whether through a web client or a mobile application. This single control is highly effective at preventing account takeovers, even if passwords are stolen.
\end{description}

\subsection{Priority 2: Remediate within 30-60 Days}
\begin{description}
    \item[Establish a Security Awareness Training Program (Risk R-02):]
    \begin{itemize}
        \item Develop or procure a training module for all new hires that covers fundamental topics like phishing, password security, and data handling.
        \item Implement an annual refresher training program for all employees to ensure their knowledge remains current.
        \item Consider periodic phishing simulations to test and reinforce employee awareness.
    \end{itemize}
\end{description}

\subsection{Priority 3: Remediate within 60-90 Days}
\begin{description}
    \item[Develop and Enforce an Acceptable Use Policy (Risk R-03):]
    \begin{itemize}
        \item Draft a clear and concise AUP that outlines the rules for using company networks, computers, and data.
        \item Have the policy reviewed by management and/or legal counsel.
        \item Distribute the policy to all employees and require them to formally acknowledge that they have read and understood it.
    \end{itemize}
\end{description}

\end{document}
```