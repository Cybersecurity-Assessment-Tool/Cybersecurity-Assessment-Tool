```latex
\documentclass[12pt]{article}

% Preamble: Required Packages
\usepackage[margin=1in]{geometry}
\usepackage{pifont} % For checkmarks and crosses
\usepackage{booktabs} % For professional tables
\usepackage{hyperref} % For clickable links
\usepackage{url}      % For URL formatting
\usepackage{seqsplit} % To split long strings in tt font
\usepackage[T1]{fontenc}

% Document Metadata
\hypersetup{
    colorlinks=true,
    linkcolor=black,
    urlcolor=blue,
    pdftitle={Cybersecurity Assessment Report},
    pdfauthor={Cybersecurity Analyst},
    pdfsubject={Security Analysis},
    pdfkeywords={Cybersecurity, Risk Assessment, Network Scan}
}

\newcommand{\yes}{\ding{51}} % Green checkmark
\newcommand{\no}{\ding{55}}  % Red X

\begin{document}

% --- Title Page ---
\begin{titlepage}
    \centering
    \vspace*{1cm}
    \Huge\textbf{Cybersecurity Assessment Report}
    \vspace{1.5cm}
    \Large
    Prepared for: \textbf{[Organization Name]}
    \vspace{2cm}
    \normalsize
    \begin{center}
        \begin{tabular}{ll}
            \textbf{Report Date:} & \today \\
            \textbf{Author:} & Cybersecurity Analyst \\
            \textbf{Version:} & 1.0 \\
        \end{tabular}
    \end{center}
    \vfill
    \small\textit{This report contains sensitive information and should be handled with the utmost confidentiality. Distribution is restricted to authorized personnel only.}
\end{titlepage}

\tableofcontents
\newpage

% --- Executive Summary ---
\section{Executive Summary}
This report provides a comprehensive cybersecurity assessment for \textbf{[Organization Name]}, based on an analysis of organizational security controls, an external network scan, and a review of pre-existing risks.

The assessment identified several critical and high-risk security gaps related to organizational policies and identity management. The most severe findings are the complete absence of Multi-Factor Authentication (MFA) for email, computer logins, and sensitive data systems. This exposes the organization to significant risks of account compromise and unauthorized access. Additionally, the lack of an employee Acceptable Use Policy and mandatory annual security training for all staff weakens the human element of the security posture.

On a positive note, the external network scan of the target IP address revealed a strong perimeter defense. No open ports were detected, indicating a well-configured firewall that effectively limits the external attack surface.

Immediate action is recommended to address the identified policy and MFA deficiencies to mitigate the most critical risks. Detailed findings and actionable recommendations are provided in the subsequent sections of this report.

% --- Organizational Information ---
\section{Organizational Information}
This section details the organizational data used as a basis for this assessment. Due to the anonymized nature of the input, placeholders have been used where necessary.

\begin{itemize}
    \item \textbf{Organization Name:} \textbf{[Organization Name]}
    \item \textbf{Primary Email Domain:} \texttt{[Domain]}
    \item \textbf{Scanned External IP:} \texttt{[Client IP]}
\end{itemize}

% --- Security Control Review ---
\section{Security Control Review}
The following table summarizes the organization's responses to a security controls questionnaire. Answers marked with \no\ represent significant gaps in the current security posture.

\begin{table}[h!]
\centering
\caption{Security Controls Questionnaire Analysis}
\label{tab:controls}
\begin{tabular}{p{0.7\linewidth} c c}
\toprule
\textbf{Control Question} & \textbf{Response} & \textbf{Status} \\
\midrule
Do you require MFA to access email? & No & \no \\
Do you require MFA to log into computers? & No & \no \\
Do you require MFA to access sensitive data systems? & No & \no \\
Does your organization have an employee acceptable use policy? & No & \no \\
Does your organization do security awareness training for new employees? & Yes & \yes \\
Does your organization do security awareness training for all employees at least once per year? & No & \no \\
\bottomrule
\end{tabular}
\end{table}

\subsection*{Analysis of Controls}
The review of security controls reveals critical deficiencies in identity and access management. The absence of MFA across all key systems is a primary concern. While it is commendable that new employees receive security training, the lack of a mandatory annual refresher course for all staff allows for knowledge decay and reduces resilience against evolving threats.

% --- Technical Scan Results ---
\section{Technical Scan Results}
An external network scan was performed to identify open ports and exposed services on the organization's perimeter.

\begin{itemize}
    \item \textbf{Target IP Address:} \texttt{[Target IP]}
    \item \textbf{Scan Date:} \today
    \item \textbf{Scanner Used:} Nmap
\end{itemize}

\begin{table}[h!]
\centering
\caption{Nmap Scan Findings for \texttt{[Target IP]}}
\label{tab:nmap}
\begin{tabular}{p{0.2\linewidth} p{0.6\linewidth}}
\toprule
\textbf{Port / Protocol} & \textbf{Service Details} \\
\midrule
\multicolumn{2}{c}{\textit{No open ports were detected.}} \\
\midrule
\textbf{Overall Status} & The host is up, but all scanned ports are in a 'closed' state. This indicates a strong firewall configuration that denies unsolicited inbound connections, significantly reducing the external attack surface. \\
\bottomrule
\end{tabular}
\end{table}

% --- Risk Assessment ---
\section{Risk Assessment}
This section synthesizes findings from the security control review and technical scan. No pre-existing vulnerabilities were provided for this assessment. The following risks have been identified and prioritized based on their potential impact.

\begin{table}[h!]
\centering
\caption{Identified Risks and Severity}
\label{tab:risks}
\begin{tabular}{p{0.15\linewidth} p{0.25\linewidth} p{0.4\linewidth} c}
\toprule
\textbf{Risk ID} & \textbf{Risk Name} & \textbf{Description} & \textbf{Severity} \\
\midrule
RISK-001 & Lack of Multi-Factor Authentication (MFA) & The absence of MFA for email, endpoints, and sensitive systems makes user accounts highly susceptible to takeover via credential theft or phishing. & \textbf{Critical} \\
\addlinespace
RISK-002 & Inadequate Security Policies & The lack of a formal Acceptable Use Policy (AUP) creates ambiguity for employees and increases the risk of insider threats and unintentional data breaches. & \textbf{High} \\
\addlinespace
RISK-003 & Insufficient Security Awareness Training & Without mandatory annual training, employees are less likely to recognize and appropriately respond to modern phishing, social engineering, and malware attacks. & \textbf{High} \\
\bottomrule
\end{tabular}
\end{table}

% --- Recommendations ---
\section{Recommendations}
Based on the identified risks, the following actionable recommendations are provided to enhance the organization's cybersecurity posture.

\subsection{Immediate Actions (1-30 Days)}
\begin{description}
    \item[For RISK-001 (Critical):] \textbf{Implement MFA Across All Critical Systems.}
    \begin{itemize}
        \item Prioritize the deployment of MFA for email access (e.g., Office 365, Google Workspace) immediately.
        \item Subsequently, enforce MFA for all remote access solutions (VPNs), administrative accounts, and systems containing sensitive data.
        \item Develop a roadmap to enable MFA for all employee computer logins.
    \end{itemize}
\end{description}

\subsection{Short-Term Actions (30-90 Days)}
\begin{description}
    \item[For RISK-002 (High):] \textbf{Develop and Implement an Acceptable Use Policy (AUP).}
    \begin{itemize}
        \item Draft a clear and comprehensive AUP that governs the use of company assets, including computers, networks, and data.
        \item Communicate the policy to all employees and require them to formally acknowledge their understanding and agreement.
    \end{itemize}
    \item[For RISK-003 (High):] \textbf{Establish a Mandatory Annual Security Training Program.}
    \begin{itemize}
        \item Select and deploy a security awareness training platform for all employees.
        \item The training should cover key topics such as phishing, password security, social engineering, and the new AUP.
        \item Track completion to ensure 100\% participation.
    \end{itemize}
\end{description}

\subsection{Ongoing Recommendations}
\begin{description}
    \item[Maintain Strong Perimeter Security:] The external network scan showed excellent firewall hygiene. Continue to enforce a "default deny" policy and conduct regular vulnerability scans to ensure this posture is maintained.
\end{description}

\end{document}
```