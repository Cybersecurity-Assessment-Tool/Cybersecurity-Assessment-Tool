```latex
\documentclass[12pt]{article}

% Preamble: Required Packages
\usepackage[margin=1in]{geometry}
\usepackage{pifont} % For checkmarks and crosses
\usepackage{booktabs} % For professional tables
\usepackage{hyperref} % For hyperlinks
\usepackage{url} % For URL formatting
\usepackage{seqsplit} % For splitting long strings
\usepackage{graphicx} % For logo
\usepackage{xcolor} % For colors

% Document Information
\title{Cybersecurity Posture Assessment Report}
\author{Cybersecurity Analyst}
\date{\today}

% Hyperref Setup
\hypersetup{
    colorlinks=true,
    linkcolor=blue,
    filecolor=magenta,      
    urlcolor=cyan,
    pdftitle={Cybersecurity Posture Assessment Report},
    pdfpagemode=FullScreen,
}

\begin{document}

\begin{titlepage}
    \centering
    \vfill
    \huge\textbf{Cybersecurity Posture Assessment Report}\\[0.5cm]
    \Large\textbf{Prepared for: \textbf{[Organization Name]}}\\[2.0cm]
    \normalsize
    \begin{tabular}{ll}
        \textbf{Report Date:} & \today \\
        \textbf{Prepared By:} & Expert Cybersecurity Analyst \\
    \end{tabular}
    \vfill
    \small
    \textit{This report is confidential and intended solely for the use of \textbf{[Organization Name]}. It contains a detailed analysis of security controls, technical vulnerabilities, and existing risks. The findings and recommendations herein are based on data provided as of the report date.}
\end{titlepage}

\maketitle
\tableofcontents
\newpage

% --- 1. Executive Summary ---
\section{Executive Summary}
This report provides a comprehensive assessment of the cybersecurity posture for \textbf{[Organization Name]}. The analysis is based on a review of organizational security controls, a technical network scan of external infrastructure, and a summary of pre-existing risks.

The assessment reveals several critical and high-risk gaps in the organization's security framework. Key findings include:
\begin{itemize}
    \item \textbf{Critical Control Gaps:} Multi-Factor Authentication (MFA) is not enforced for accessing email or sensitive data systems. This represents a significant vulnerability to account takeover and data breaches.
    \item \textbf{Policy Deficiencies:} The organization lacks a formal employee Acceptable Use Policy and does not provide security awareness training to new hires. These omissions create a high-risk environment susceptible to human error and insider threats.
    \item \textbf{Technical Exposure:} The external network scan identified an open Secure Shell (SSH) port. While common for remote administration, an improperly configured or unmonitored SSH service is a primary target for attackers.
\end{itemize}

The combination of these policy, procedural, and technical weaknesses places the organization at a high risk of a security incident. Immediate and decisive action is required to remediate these findings. This report provides specific, actionable recommendations to mitigate the identified risks and strengthen the overall security posture.

% --- 2. Organizational Information ---
\section{Organizational Information}
This section details the organizational data used as the basis for this assessment. The information was provided by the client.

\begin{tabular}{@{}ll}
    \toprule
    \textbf{Attribute} & \textbf{Value} \\
    \midrule
    Organization Name & \textbf{[Organization Name]} \\
    Primary Email Domain & \texttt{[Domain]} \\
    External IP Address Scanned & \texttt{[Client IP]} \\
    \bottomrule
\end{tabular}

% --- 3. Security Control Review ---
\section{Security Control Review}
The following table summarizes the organization's responses to a security controls questionnaire. A green checkmark (\textcolor{green}{\ding{51}}) indicates a positive control is in place, while a red cross (\textcolor{red}{\ding{55}}) indicates a control gap that presents a risk.

\begin{table}[h!]
\centering
\caption{Security Controls Questionnaire Analysis}
\begin{tabular}{@{}p{0.8\linewidth}c@{}}
    \toprule
    \textbf{Control Question} & \textbf{Response} \\
    \midrule
    Do you require MFA to access email? & \textcolor{red}{\ding{55}} \\
    Do you require MFA to log into computers? & \textcolor{green}{\ding{51}} \\
    Do you require MFA to access sensitive data systems? & \textcolor{red}{\ding{55}} \\
    Does your organization have an employee acceptable use policy? & \textcolor{red}{\ding{55}} \\
    Does your organization do security awareness training for new employees? & \textcolor{red}{\ding{55}} \\
    Does your organization do security awareness training for all employees at least once per year? & \textcolor{green}{\ding{51}} \\
    \bottomrule
\end{tabular}
\end{table}

\subsection*{Analysis of Control Gaps}
The questionnaire reveals critical deficiencies in access control and security governance. The absence of MFA for email and sensitive data are the most severe findings, as these are primary targets for cybercriminals. Furthermore, the lack of an Acceptable Use Policy and new-hire training indicates a foundational weakness in security culture and governance.

% --- 4. Technical Scan Results ---
\section{Technical Scan Results}
A network port scan was conducted against the organization's external-facing infrastructure to identify exposed services.

\begin{itemize}
    \item \textbf{Target IP Address:} \texttt{[Target IP]}
    \item \textbf{Scan Date:} \today
\end{itemize}

\begin{table}[h!]
\centering
\caption{Open Ports Detected on \texttt{[Target IP]}}
\begin{tabular}{@{}llll@{}}
    \toprule
    \textbf{Port} & \textbf{State} & \textbf{Service} & \textbf{Product / Version} \\
    \midrule
    22/tcp & open & ssh (inferred) & Information Not Available \\
    \bottomrule
\end{tabular}
\end{table}

\subsection*{Analysis of Technical Findings}
The scan identified that port 22 (SSH) is open to the public internet. SSH is a common protocol for remote server administration. However, if not properly secured, it can be a significant entry point for attackers. Common attack vectors against SSH include:
\begin{itemize}
    \item \textbf{Brute-force attacks:} Automated attempts to guess usernames and passwords.
    \item \textbf{Credential stuffing:} Using credentials stolen from other data breaches.
    \item \textbf{Exploitation of vulnerabilities:} Targeting outdated or vulnerable versions of the SSH server software.
\end{itemize}
As the scan could not determine the software version, it is not possible to check for specific vulnerabilities. However, the service's exposure alone constitutes a risk that must be addressed.

% --- 5. Correlated Risk Assessment ---
\section{Correlated Risk Assessment}
This section synthesizes the findings from the security control review and the technical scan. The pre-existing risk register was empty, so all identified risks are new findings.

\begin{table}[h!]
\centering
\caption{Summary of Identified Risks}
\begin{tabular}{@{}lp{0.5\linewidth}l@{}}
    \toprule
    \textbf{Risk Name} & \textbf{Overview} & \textbf{Severity} \\
    \midrule
    \textbf{Lack of MFA on Critical Systems} & The absence of MFA on email and sensitive data systems exposes the organization to a high likelihood of account compromise and subsequent data breach. & \textbf{Critical} \\
    \addlinespace
    \textbf{Inadequate Security Policies} & The lack of an Acceptable Use Policy and new-hire security training creates an environment where employees are unaware of security best practices, increasing the risk of insider threats and human error. & \textbf{High} \\
    \addlinespace
    \textbf{Exposed SSH Service} & An externally-facing SSH port is a target for automated attacks. If compromised, it could grant an attacker administrative access to a key server. This risk is amplified by the lack of MFA on backend systems. & \textbf{Medium} \\
    \bottomrule
\end{tabular}
\end{table}

% --- 6. Recommendations ---
\section{Recommendations}
The following actions are recommended to mitigate the identified risks and improve the overall security posture of \textbf{[Organization Name]}. Recommendations are prioritized based on risk severity.

\begin{enumerate}
    \item \textbf{Implement and Enforce MFA (Critical):}
    \begin{itemize}
        \item Immediately enable MFA for all user accounts on the primary email system.
        \item Prioritize the deployment of MFA on all systems classified as containing sensitive data.
        \item Develop a policy mandating the use of MFA for all remote access and access to critical internal systems.
    \end{itemize}

    \item \textbf{Develop Foundational Security Policies (High):}
    \begin{itemize}
        \item Create and ratify a formal Acceptable Use Policy (AUP) that clearly defines the rules for using company IT assets. Ensure all employees read and acknowledge the policy.
        \item Institute a mandatory security awareness training module for all new employees as part of the onboarding process.
    \end{itemize}

    \item \textbf{Harden Exposed SSH Service (Medium):}
    \begin{itemize}
        \item If remote access is required, restrict SSH access to a list of trusted IP addresses (whitelisting).
        \item Disable password-based authentication and enforce the use of public/private key pairs for all SSH logins.
        \item Implement an intrusion detection and prevention tool, such as `fail2ban`, to automatically block IP addresses that exhibit malicious behavior (e.g., multiple failed login attempts).
        \item Ensure the SSH server software is kept up-to-date with the latest security patches.
    \end{itemize}
\end{enumerate}

\end{document}
```