```latex
\documentclass[12pt]{article}

% --- PACKAGES ---
\usepackage[margin=1in]{geometry}
\usepackage{pifont} % For checkmarks and crosses
\usepackage{booktabs} % For professional tables
\usepackage{hyperref} % For clickable links
\usepackage{url}      % For proper URL formatting
\usepackage{seqsplit} % For splitting long strings in texttt

% --- DOCUMENT METADATA ---
\title{Cybersecurity Posture Assessment Report \\ \large For \textbf{[Organization Name]}}
\author{Cybersecurity Analysis Division}
\date{\today}

\begin{document}

\maketitle
\tableofcontents
\newpage

% --- EXECUTIVE SUMMARY ---
\section{Executive Summary}
This report details the findings of a cybersecurity posture assessment conducted for \textbf{[Organization Name]}. The analysis combines a review of organizational security controls, a technical network scan of the external perimeter, and an evaluation of pre-existing risk documentation.

The assessment identified several critical and high-risk vulnerabilities that require immediate attention. The most severe finding is the direct public exposure of the Remote Desktop Protocol (RDP) service on port 3389. This technical vulnerability is critically amplified by organizational policy gaps, namely the systemic lack of Multi-Factor Authentication (MFA) for logging into computers and accessing email. This combination of factors creates a significant risk of unauthorized access and ransomware infection.

Furthermore, the absence of annual security awareness training for all employees constitutes a high-risk gap, increasing the organization's susceptibility to social engineering and phishing attacks.

Immediate remediation of the exposed RDP service and a strategic rollout of MFA are strongly recommended to mitigate these risks and improve the overall security posture.

% --- ORGANIZATIONAL INFORMATION ---
\section{Organizational Information}
The following details were used as the basis for this assessment. Due to the anonymized nature of the provided data, placeholders have been used where necessary.

\begin{itemize}
    \item \textbf{Organization Name:} \textbf{[Organization Name]}
    \item \textbf{Primary Domain:} \texttt{[Domain]}
    \item \textbf{External IP Scanned:} \texttt{[Client IP]}
\end{itemize}

% --- SECURITY CONTROL REVIEW ---
\section{Security Control Review (Questionnaire Analysis)}
A review of the organization's security controls was conducted via a questionnaire. The responses highlight significant gaps in identity and access management, which directly impact the risk level of technical findings.

\begin{table}[h!]
\centering
\caption{Security Controls Questionnaire Results}
\begin{tabular}{p{0.6\linewidth} c l}
\toprule
\textbf{Control Question} & \textbf{Response} & \textbf{Assessment} \\
\midrule
Do you require MFA to access email? & \ding{55} & \textbf{Critical Gap} \\
Do you require MFA to log into computers? & \ding{55} & \textbf{Critical Gap} \\
Do you require MFA to access sensitive data systems? & \ding{55} & \textbf{Critical Gap} \\
\addlinespace
Does your organization have an employee acceptable use policy? & \ding{51} & Control in Place \\
Does your organization do security awareness training for new employees? & \ding{51} & Control in Place \\
\addlinespace
Does your organization do security awareness training for all employees at least once per year? & \ding{55} & \textbf{High Risk} \\
\bottomrule
\end{tabular}
\end{table}

The lack of MFA across all critical access points (email, computer login, sensitive data) is a severe deficiency. MFA is a foundational security control that protects against credential theft and brute-force attacks. The absence of annual security training for all staff indicates that employees may not be equipped to recognize and respond to modern cyber threats.

% --- TECHNICAL SCAN RESULTS ---
\section{Technical Scan Results}
An external network scan was performed to identify accessible services on the organization's public-facing infrastructure.

\begin{itemize}
    \item \textbf{Target IP Address:} \texttt{[Target IP]}
    \item \textbf{Scan Date:} \today
\end{itemize}

The scan revealed the following open port:

\begin{table}[h!]
\centering
\caption{Open Port Analysis}
\begin{tabular}{l l l p{0.4\linewidth}}
\toprule
\textbf{Port} & \textbf{State} & \textbf{Service} & \textbf{Notes} \\
\midrule
3389/tcp & open & ms-wbt-server & This is the standard port for Microsoft Remote Desktop Protocol (RDP). Direct public exposure of RDP is a common vector for ransomware attacks and unauthorized access. \\
\bottomrule
\end{tabular}
\end{table}

\textbf{Analysis:} The technical scan confirms the pre-existing risk identified as "RDP Exposure". This finding, when correlated with the lack of MFA for computer logins, elevates the risk to a critical level. An attacker could potentially compromise a user's password through phishing or a brute-force attack and gain direct, interactive access to the internal network.

% --- CONSOLIDATED RISK ASSESSMENT ---
\section{Consolidated Risk Assessment}
The following table synthesizes findings from the security control review, technical scan, and pre-existing risk data into a prioritized list.

\begin{table}[h!]
\centering
\caption{Summary of Key Risks}
\begin{tabular}{p{0.2\linewidth} p{0.45\linewidth} l p{0.15\linewidth}}
\toprule
\textbf{Risk Name} & \textbf{Description} & \textbf{Severity} & \textbf{Affected Systems} \\
\midrule
\textbf{Exposed RDP without MFA} & The RDP service is publicly accessible, and the organization does not enforce MFA for computer logins. This allows an attacker with a valid password to gain full remote control of a system. & \textbf{Critical (9.0)} & \texttt{[Target IP]} \\
\addlinespace
\textbf{Lack of Annual Security Training} & Employees do not receive recurring security awareness training, increasing the likelihood of successful phishing and social engineering attacks that could lead to credential compromise. & \textbf{High} & All Personnel \\
\bottomrule
\end{tabular}
\end{table}

% --- RECOMMENDATIONS ---
\section{Recommendations}
Based on the consolidated risk assessment, the following actions are recommended to strengthen the organization's security posture.

\subsection{Immediate Actions (Priority 1)}
\begin{enumerate}
    \item \textbf{Remediate RDP Exposure:} Access to TCP port 3389 on \texttt{[Target IP]} must be removed from the public internet immediately.
        \begin{itemize}
            \item \textbf{Best Practice:} Implement a Virtual Private Network (VPN) with MFA for all remote access. RDP should only be accessible through the VPN tunnel.
            \item \textbf{Alternative:} If a VPN cannot be deployed immediately, create a strict firewall rule that only allows access to port 3389 from a limited set of trusted, static IP addresses.
            \item \textbf{Emergency Fix:} If remote access is not business-critical, close the port entirely.
        \end{itemize}
\end{enumerate}

\subsection{Strategic Actions (Priority 2)}
\begin{enumerate}
    \item \textbf{Implement Multi-Factor Authentication (MFA):} Begin a project to deploy MFA across the entire organization. Prioritize the following systems:
        \begin{itemize}
            \item Email (e.g., Office 365, Google Workspace)
            \item VPN and other remote access solutions
            \item All systems containing sensitive data
            \item Workstation and server logins
        \end{itemize}
    \item \textbf{Establish Annual Security Awareness Training:} Implement a mandatory annual security training program for all employees. This program should cover current threats such as phishing, ransomware, and proper data handling to foster a security-conscious culture.
\end{enumerate}

\end{document}
```