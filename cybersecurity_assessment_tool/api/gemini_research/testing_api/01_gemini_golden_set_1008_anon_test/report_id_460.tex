```latex
\documentclass[12pt]{article}

% --- PACKAGES ---
\usepackage[margin=1in]{geometry}
\usepackage{pifont} % For check and cross marks
\usepackage{booktabs} % For professional tables
\usepackage{hyperref} % For clickable links
\usepackage{url} % For URL formatting
\usepackage{seqsplit} % For splitting long strings
\usepackage{graphicx} % For potential logos/images
\usepackage{fancyhdr} % For headers and footers

% --- DOCUMENT SETUP ---
\hypersetup{
    colorlinks=true,
    linkcolor=black,
    urlcolor=blue,
    pdftitle={Cybersecurity Posture Assessment Report},
    pdfauthor={Cybersecurity Analyst},
}

\pagestyle{fancy}
\fancyhf{}
\lhead{Confidential Security Report}
\rhead{\textbf{[Organization Name]}}
\cfoot{\thepage}

% --- DOCUMENT START ---
\begin{document}

% --- TITLE PAGE ---
\begin{titlepage}
    \centering
    \vspace*{2cm}
    
    \Huge
    \textbf{Cybersecurity Posture Assessment Report}
    
    \vspace{1.5cm}
    
    \Large
    Prepared for: \\
    \vspace{0.5cm}
    \textbf{[Organization Name]}
    
    \vfill
    
    \large
    Report Date: \today \\
    Analysis Period: [Scan Date]
    
\end{titlepage}

\tableofcontents
\newpage

% --- EXECUTIVE OVERVIEW ---
\section{Executive Overview}
This report details the findings of a cybersecurity posture assessment conducted for \textbf{[Organization Name]}. The assessment combined a review of organizational security controls, an external network scan, and an analysis of known vulnerabilities.

The overall security posture is considered \textbf{HIGH RISK}. While the external network perimeter appears to be well-configured with no open ports detected on the scanned target, there are critical deficiencies in fundamental administrative and identity-based security controls. 

The most significant risks stem from the lack of Multi-Factor Authentication (MFA) for email and sensitive data systems. This exposes the organization to a severe risk of account compromise through phishing and other credential-based attacks. Furthermore, the absence of an acceptable use policy and security awareness training programs leaves the organization vulnerable to insider threats and social engineering, as employees are not equipped with the knowledge or guidelines to operate securely.

Immediate remediation of these policy and identity management gaps is strongly recommended to reduce the likelihood of a significant security breach.

% --- ORGANIZATIONAL INFORMATION ---
\section{Organizational Information}
The following details were used as the basis for this assessment.
\begin{itemize}
    \item \textbf{Organization Name:} \textbf{[Organization Name]}
    \item \textbf{Primary Email Domain:} \texttt{[Domain]}
    \item \textbf{Client External IP:} \texttt{[Client IP]}
    \item \textbf{Target IP Address Scanned:} \texttt{[Target IP]}
\end{itemize}

% --- SECURITY CONTROL REVIEW ---
\section{Security Control Review}
A security questionnaire was completed to evaluate the organization's current administrative and policy-based controls. The responses revealed several significant gaps in security best practices. A checkmark (\ding{51}) indicates a positive control is in place, while a cross mark (\ding{55}) indicates a control gap.

\begin{table}[h!]
\centering
\caption{Security Controls Questionnaire Results}
\begin{tabular}{p{0.75\linewidth} c}
\toprule
\textbf{Control Question} & \textbf{Response} \\
\midrule
Do you require MFA to access email? & \ding{55} \\
Do you require MFA to log into computers? & \ding{51} \\
Do you require MFA to access sensitive data systems? & \ding{55} \\
Does your organization have an employee acceptable use policy? & \ding{55} \\
Does your organization do security awareness training for new employees? & \ding{55} \\
Does your organization do security awareness training for all employees at least once per year? & \ding{55} \\
\bottomrule
\end{tabular}
\end{table}

% --- TECHNICAL SCAN RESULTS ---
\section{Technical Scan Results}
An external network scan was performed on the designated target IP address to identify accessible services and potential vulnerabilities.

\begin{itemize}
    \item \textbf{Target IP:} \texttt{[Target IP]}
    \item \textbf{Scan Date:} [Scan Date]
\end{itemize}

\subsection{Summary of Findings}
The scan results were positive, indicating a strong network perimeter. \textbf{No open ports were discovered.} All 1000 scanned TCP ports were found to be in a `closed` state. This configuration significantly reduces the external attack surface and is commendable.

% --- RISK ASSESSMENT ---
\section{Risk Assessment}
This section synthesizes findings from the security control review, technical scan, and pre-existing risk data. No pre-existing vulnerabilities were reported. The following new risks were identified during this assessment, primarily from gaps in administrative controls.

\begin{table}[h!]
\centering
\caption{Identified Risks}
\begin{tabular}{p{0.25\linewidth} p{0.5\linewidth} p{0.15\linewidth}}
\toprule
\textbf{Risk Name} & \textbf{Overview} & \textbf{Severity} \\
\midrule
\textbf{Email Account Compromise} & The absence of MFA on email accounts creates a critical vulnerability. If user credentials are stolen via phishing or other means, attackers can gain full access to email, leading to data breaches, financial fraud, and further internal attacks. & \textbf{Critical} \\
\addlinespace
\textbf{Sensitive Data Exposure} & Critical data systems lack MFA protection. A single compromised credential could grant an attacker access to the organization's most sensitive information, posing a severe threat to confidentiality and integrity. & \textbf{Critical} \\
\addlinespace
\textbf{Lack of Employee Security Governance} & Without an Acceptable Use Policy, there are no defined rules for employee behavior regarding company assets and data. This increases the risk of insider threat (both malicious and accidental) and potential non-compliance with regulations. & \textbf{High} \\
\addlinespace
\textbf{Untrained Workforce} & The lack of a security awareness training program means employees are likely unable to recognize or properly respond to common threats like phishing and social engineering, making them the weakest link in the organization's defense. & \textbf{High} \\
\bottomrule
\end{tabular}
\end{table}

% --- RECOMMENDATIONS ---
\section{Recommendations}
Based on the identified risks, the following actions are recommended to improve the organization's cybersecurity posture. Recommendations are prioritized by severity.

\subsection{Immediate Priority (Critical Risks)}
\begin{enumerate}
    \item \textbf{Implement MFA for Email:} Immediately enforce MFA for all user access to the email system (\texttt{[Domain]}). This is the single most effective control to prevent account takeovers.
    \item \textbf{Enforce MFA for Sensitive Systems:} Deploy MFA on all systems identified as containing sensitive or critical data without delay. Access to these systems should require more than just a username and password.
\end{enumerate}

\subsection{High Priority}
\begin{enumerate}
    \item \textbf{Develop an Acceptable Use Policy (AUP):} Create and implement a formal AUP that clearly outlines the rules and responsibilities for all employees when using company technology and data. All employees must read and acknowledge this policy.
    \item \textbf{Establish a Security Awareness Training Program:} Launch a mandatory security awareness training program for all employees. This program should be required for new hires during onboarding and conducted annually for all staff. Topics must include phishing, password security, and data handling.
\end{enumerate}

\subsection{General Recommendations}
\begin{enumerate}
    \item \textbf{Continue Network Monitoring:} Maintain the current strong network perimeter configuration. Continue to perform regular external and internal vulnerability scans to ensure no new weaknesses are introduced.
\end{enumerate}

\end{document}
```