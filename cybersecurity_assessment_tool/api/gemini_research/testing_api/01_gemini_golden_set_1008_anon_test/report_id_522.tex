```latex
\documentclass[12pt]{article}

% --- PACKAGES ---
\usepackage[margin=1in]{geometry}
\usepackage{pifont} % Required for \ding{51} and \ding{55}
\usepackage{booktabs} % For professional-looking tables
\usepackage[hidelinks]{hyperref}
\usepackage{url}
\usepackage{seqsplit} % For splitting long strings in \texttt
\usepackage{graphicx}
\usepackage{fancyhdr}
\usepackage[utf8]{inputenc}
\usepackage{xcolor}

% --- DOCUMENT METADATA ---
\title{Cybersecurity Posture Assessment Report}
\author{Cybersecurity Analysis Division}
\date{\today}

% --- HEADER & FOOTER ---
\pagestyle{fancy}
\fancyhf{}
\lhead{\textbf{[Organization Name]} - Cybersecurity Assessment}
\rhead{\today}
\cfoot{\thepage}

% --- COLOR DEFINITIONS FOR SEVERITY ---
\definecolor{sev_critical}{HTML}{940000}
\definecolor{sev_high}{HTML}{D14124}
\definecolor{sev_medium}{HTML}{E8A12C}
\definecolor{sev_low}{HTML}{2C88D1}
\definecolor{sev_info}{HTML}{4CAF50}

\newcommand{\sevtext}[2]{\colorbox{#1}{\textcolor{white}{\textbf{\phantom{i}#2\phantom{i}}}}}

\begin{document}

\maketitle
\thispagestyle{empty}
\newpage

\tableofcontents
\newpage

% ==============================================================================
% 1. EXECUTIVE SUMMARY
% ==============================================================================
\section{Executive Summary}

This report provides a cybersecurity posture assessment for \textbf{[Organization Name]}, based on an analysis of network scan data, a security controls questionnaire, and a review of pre-existing risks.

The assessment reveals critical gaps in administrative and access controls that require immediate attention. The two most significant findings are the lack of Multi-Factor Authentication (MFA) for sensitive data systems and the absence of an employee Acceptable Use Policy (AUP). These deficiencies expose the organization to a high risk of unauthorized access, data breaches, and insider threats.

On a positive note, the external network scan of the target IP address \texttt{[Target IP]} did not identify any open ports. Specifically, port 80 (HTTP) was found to be closed. This finding contradicts a previously identified risk concerning an unencrypted web server, suggesting that remediation may have already occurred. This discrepancy should be formally verified and the risk register updated accordingly.

The primary recommendations focus on implementing foundational security controls: mandating MFA across all critical systems and establishing a formal AUP to govern employee use of company assets.

% ==============================================================================
% 2. ORGANIZATIONAL INFORMATION
% ==============================================================================
\section{Organizational Information}

The following details were used as the basis for this assessment. Due to the anonymized nature of the provided data, placeholders have been used where necessary.

\begin{table}[h!]
\centering
\begin{tabular}{@{}ll@{}}
\toprule
\textbf{Item} & \textbf{Detail} \\ \midrule
Organization Name & \textbf{[Organization Name]} \\
Primary Email Domain & \texttt{[Domain]} \\
Assessed External IP & \texttt{[Client IP]} \\
Network Scan Target & \texttt{[Target IP]} \\ \bottomrule
\end{tabular}
\caption{Organizational and Assessment Scope.}
\end{table}

% ==============================================================================
% 3. SECURITY CONTROL REVIEW
% ==============================================================================
\section{Security Control Review}

A review of the organization's security controls was conducted via a questionnaire. The responses indicate several areas of concern where standard security best practices are not being met. The results are summarized in Table \ref{tab:controls}.

\begin{table}[h!]
\centering
\begin{tabular}{@{}lc@{}}
\toprule
\textbf{Control Question} & \textbf{Response} \\ \midrule
Do you require MFA to access email? & \ding{51} \\
Do you require MFA to log into computers? & \ding{51} \\
Do you require MFA to access sensitive data systems? & \textbf{\color{red}\ding{55}} \\
Does your organization have an employee acceptable use policy? & \textbf{\color{red}\ding{55}} \\
Does your organization do security awareness training for new employees? & \ding{51} \\
Does your organization do security awareness training for all employees annually? & \ding{51} \\ \bottomrule
\end{tabular}
\caption{Security Controls Questionnaire Results.}
\label{tab:controls}
\end{table}

\subsection{Analysis of Control Gaps}
The responses marked with a \textbf{\color{red}\ding{55}} (No) represent significant security gaps:
\begin{itemize}
    \item \textbf{No MFA for Sensitive Data Systems:} This is a critical vulnerability. Without MFA, sensitive systems are protected only by passwords, which are susceptible to phishing, brute-force attacks, and credential stuffing. A single compromised password could lead to a major data breach.
    \item \textbf{No Acceptable Use Policy (AUP):} An AUP is a foundational administrative control. Its absence means there are no formal rules governing how employees may use company technology and data. This increases the risk of both unintentional and malicious insider threats and creates legal and compliance challenges.
\end{itemize}

% ==============================================================================
% 4. TECHNICAL SCAN RESULTS
% ==============================================================================
\section{Technical Scan Results}

An external network scan was performed against the target IP address \texttt{[Target IP]} to identify accessible services.

\subsection{Scan Summary}
\begin{itemize}
    \item \textbf{Scanner:} Nmap
    \item \textbf{Target Host:} \texttt{[Target IP]}
    \item \textbf{Host Status:} Up (Online)
    \item \textbf{Key Finding:} No open ports were discovered on the target host.
\end{itemize}

\begin{table}[h!]
\centering
\begin{tabular}{@{}llll@{}}
\toprule
\textbf{Port} & \textbf{State} & \textbf{Service} & \textbf{Product / Version} \\ \midrule
80/tcp & closed & http & N/A \\ \bottomrule
\end{tabular}
\caption{Port Scan Details for \texttt{[Target IP]}.}
\label{tab:scan}
\end{table}

\subsection{Analysis of Scan Results}
The scan results are positive from a network security perspective, indicating a minimal external attack surface. The fact that port 80 (HTTP) is explicitly reported as \texttt{closed} is a significant finding. This result directly conflicts with the pre-existing risk data (see Section 5), which states that Port 80 is open. This suggests that the previously identified vulnerability may have been remediated. It is recommended to confirm internally that this port was closed intentionally as part of a remediation effort.

% ==============================================================================
% 5. CONSOLIDATED RISK ASSESSMENT
% ==============================================================================
\section{Consolidated Risk Assessment}

This section synthesizes findings from the security questionnaire, technical scan, and pre-existing risk data into a consolidated list of current risks.

\begin{table}[h!]
\centering
\begin{tabular}{@{}p{0.3\linewidth}p{0.5\linewidth}l@{}}
\toprule
\textbf{Risk Name} & \textbf{Description} & \textbf{Severity} \\ \midrule
\textbf{Lack of MFA for Sensitive Systems} & MFA is not enforced for systems containing sensitive data, creating a high risk of unauthorized access via compromised credentials. & \sevtext{sev_critical}{Critical} \\
\addlinespace
\textbf{Missing Acceptable Use Policy (AUP)} & The absence of a formal AUP increases the risk of insider threats, misuse of assets, and non-compliance. & \sevtext{sev_high}{High} \\
\addlinespace
\textbf{Unencrypted Web Server (Potentially Remediated)} & A pre-existing risk indicated an open Port 80. The current scan found this port closed. This risk should be re-validated and potentially closed. & \sevtext{sev_info}{Informational} \\
\bottomrule
\end{tabular}
\caption{Summary of Identified Risks.}
\label{tab:risks}
\end{table}

% ==============================================================================
% 6. RECOMMENDATIONS
% ==============================================================================
\section{Recommendations}

Based on the consolidated risk assessment, the following actions are recommended to improve the security posture of \textbf{[Organization Name]}.

\begin{enumerate}
    \item \textbf{[Critical] Implement MFA on Sensitive Systems:}
    \begin{itemize}
        \item \textbf{Action:} Immediately prioritize and deploy MFA across all applications and systems that store, process, or transmit sensitive data.
        \item \textbf{Impact:} Drastically reduces the risk of unauthorized access and data breaches resulting from compromised credentials.
    \end{itemize}
    \vspace{1em}
    \item \textbf{[High] Develop and Implement an Acceptable Use Policy:}
    \begin{itemize}
        \item \textbf{Action:} Draft a formal AUP that clearly defines the rules for using company networks, devices, and data. Require all current and new employees to read and formally acknowledge the policy.
        \item \textbf{Impact:} Establishes a clear security baseline for employee behavior, reduces insider risk, and provides a basis for disciplinary action if violated.
    \end{itemize}
    \vspace{1em}
    \item \textbf{[Informational] Validate and Update Risk Register:}
    \begin{itemize}
        \item \textbf{Action:} Formally verify that the closure of Port 80 on \texttt{[Target IP]} was intentional. If confirmed, update the internal risk register to mark the "Unencrypted Web Server" risk as remediated.
        \item \textbf{Impact:} Ensures the risk register is accurate and reflects the current security posture.
    \end{itemize}
\end{enumerate}

\end{document}
```