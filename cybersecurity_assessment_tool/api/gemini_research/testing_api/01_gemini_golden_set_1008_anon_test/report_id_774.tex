```latex
\documentclass[12pt]{article}

% --- PACKAGE IMPORTS ---
\usepackage[margin=1in]{geometry} % Set page margins
\usepackage{pifont}               % For checkmarks and crosses (\ding)
\usepackage{booktabs}             % For professional tables (\toprule, \midrule, \bottomrule)
\usepackage{hyperref}             % For hyperlinks and document metadata
\usepackage{url}                  % For formatting URLs
\usepackage{seqsplit}             % For splitting long strings without spaces
\usepackage{xcolor}               % For colors

% --- DOCUMENT METADATA ---
\hypersetup{
    colorlinks=true,
    linkcolor=blue,
    filecolor=magenta,      
    urlcolor=cyan,
    pdftitle={Cybersecurity Assessment Report},
    pdfauthor={Cybersecurity Analysis Unit},
    pdfsubject={Security Assessment},
    pdfkeywords={Cybersecurity, Nmap, Risk, Assessment},
}

% --- CUSTOM COMMANDS ---
\newcommand{\yes}{\ding{51}} % Green checkmark
\newcommand{\no}{\ding{55}}  % Red X

% --- TITLE ---
\title{Cybersecurity Assessment Report}
\author{Cybersecurity Analysis Unit}
\date{\today}

% --- DOCUMENT BEGIN ---
\begin{document}

\maketitle
\thispagestyle{empty}
\newpage

\tableofcontents
\newpage

% ==============================================================================
% 1. EXECUTIVE SUMMARY
% ==============================================================================
\section{Executive Summary}

This report details the findings of a cybersecurity assessment conducted for \textbf{[Organization Name]}. The evaluation combined a technical network scan of the external IP address \texttt{[Target IP]}, a review of organizational security controls via a questionnaire, and a correlation with pre-existing risk data.

The assessment revealed a mixed security posture. On a positive note, the external network scan did not identify any open ports or immediate technical vulnerabilities. Specifically, a previously identified risk concerning an open Port 80 (HTTP) appears to have been remediated, as our scan confirmed this port is now closed. This indicates progress in securing the network perimeter.

However, significant gaps were identified in the organization's internal security policies and controls. The two most critical findings are:
\begin{itemize}
    \item \textbf{Lack of Multi-Factor Authentication (MFA) for Email:} The absence of mandatory MFA for email access represents a critical vulnerability. Email is a primary target for attackers, and compromised accounts can lead to data breaches, financial fraud, and further system intrusions.
    \item \textbf{Absence of Annual Security Awareness Training:} The organization does not conduct mandatory security awareness training for all employees on an annual basis. This lapse allows security knowledge to degrade over time, increasing the organization's susceptibility to phishing, social engineering, and other human-centric attacks.
\end{itemize}

While the technical perimeter is currently secure, the identified policy gaps expose the organization to significant risk. Recommendations in this report focus on immediately addressing these critical control deficiencies to build a more resilient and defense-in-depth security posture.

% ==============================================================================
% 2. ORGANIZATIONAL INFORMATION
% ==============================================================================
\section{Organizational Information}

This section provides the context for the assessment based on the information provided.

\begin{table}[h!]
\centering
\begin{tabular}{@{}ll@{}}
\toprule
\textbf{Attribute} & \textbf{Value} \\ \midrule
Organization Name    & \textbf{[Organization Name]} \\
Primary Domain       & \texttt{[Domain]} \\
External IP Scanned  & \texttt{[Target IP]} \\ \bottomrule
\end{tabular}
\caption{Client Organizational Details}
\end{table}

% ==============================================================================
% 3. SECURITY CONTROL REVIEW
% ==============================================================================
\section{Security Control Review}

The following table summarizes the organization's responses to the security controls questionnaire. Each response is evaluated against industry best practices. "No" answers indicate significant gaps in the security framework.

\begin{table}[h!]
\centering
\begin{tabular}{@{}p{0.6\textwidth}ccp{0.2\textwidth}@{}}
\toprule
\textbf{Control Question} & \textbf{Response} & \textbf{Assessment} \\ \midrule
Do you require MFA to access email? & \no & \textbf{Critical Gap} \\
Do you require MFA to log into computers? & \yes & Best Practice Met \\
Do you require MFA to access sensitive data systems? & \yes & Best Practice Met \\
Does your organization have an employee acceptable use policy? & \yes & Best Practice Met \\
Does your organization do security awareness training for new employees? & \yes & Best Practice Met \\
Does your organization do security awareness training for all employees at least once per year? & \no & \textbf{High Risk Gap} \\ \bottomrule
\end{tabular}
\caption{Security Controls Questionnaire Analysis}
\end{table}

% ==============================================================================
% 4. TECHNICAL SCAN RESULTS
% ==============================================================================
\section{Technical Scan Results}

An Nmap scan was performed on the designated external target to identify open ports and exposed services.

\begin{itemize}
    \item \textbf{Target IP:} \texttt{[Target IP]}
    \item \textbf{Scan Date:} [Scan Date]
\end{itemize}

\subsection{Scan Findings}
The scan revealed that the host is up, but no open ports were detected. The state of commonly checked ports is detailed below.

\begin{table}[h!]
\centering
\begin{tabular}{@{}cccc@{}}
\toprule
\textbf{Port} & \textbf{State} & \textbf{Service} & \textbf{Product / Version} \\ \midrule
80/tcp        & closed         & http             & N/A                        \\ \bottomrule
\end{tabular}
\caption{Network Scan Port Details}
\end{table}

\subsection{Technical Analysis}
The scan indicates a strong external network posture for the assessed IP address. The absence of open ports significantly reduces the attack surface available to external threats. 

Notably, the scan confirms that Port 80 (HTTP) is \textbf{closed}. This finding contradicts a pre-existing risk entry ("Unencrypted Web Server") which stated the port was open. This suggests that the previously identified vulnerability has been successfully remediated.

% ==============================================================================
% 5. CONSOLIDATED RISK ASSESSMENT
% ==============================================================================
\section{Consolidated Risk Assessment}

This section synthesizes findings from the security control review, technical scan, and pre-existing risk data into a consolidated list of current risks.

\begin{table}[h!]
\centering
\begin{tabular}{@{}lp{0.5\textwidth}ll@{}}
\toprule
\textbf{Risk ID} & \textbf{Risk Description} & \textbf{Severity} & \textbf{Affected Asset(s)} \\ \midrule
RISK-001 & \textbf{Lack of MFA on Email Accounts:} User email accounts can be compromised with only a password, leading to data breaches and phishing. & \textbf{Critical} & Email System, User Accounts \\
RISK-002 & \textbf{Insufficient Security Training:} Employees are not receiving regular, annual security training, increasing susceptibility to social engineering. & \textbf{High} & All Employees \\
RISK-003 & \textbf{Unencrypted Web Server:} Port 80 was previously identified as open, exposing the organization to unencrypted traffic interception. & \textbf{Mitigated} & \texttt{[Target IP]} \\ \bottomrule
\end{tabular}
\caption{Summary of Identified Risks}
\end{table}

% ==============================================================================
% 6. RECOMMENDATIONS
% ==============================================================================
\section{Recommendations}

Based on the analysis, the following actions are recommended to mitigate the identified risks and improve the overall security posture of \textbf{[Organization Name]}.

\begin{enumerate}
    \item \textbf{[Critical] Implement Mandatory MFA for Email Access:}
    \begin{itemize}
        \item \textbf{Action:} Enforce MFA for all users accessing the email system, both internally and externally. This is the single most effective control to prevent unauthorized account access.
        \item \textbf{Justification:} Mitigates RISK-001. Protects against credential stuffing, password spraying, and phishing attacks that successfully harvest user credentials.
        \item \textbf{Suggested Solutions:} Utilize authenticator apps (e.g., Google Authenticator, Microsoft Authenticator), hardware tokens, or biometric options available through your email provider.
    \end{itemize}
    \vspace{1em}
    
    \item \textbf{[High] Establish an Annual Security Awareness Training Program:}
    \begin{itemize}
        \item \textbf{Action:} Develop and implement a mandatory security awareness training program for all employees, to be completed annually.
        \item \textbf{Justification:} Mitigates RISK-002. A well-trained workforce is a critical layer of defense. Regular training ensures that security remains a top-of-mind concern and that employees are aware of the latest threats and policies.
        \item \textbf{Suggested Topics:} Phishing identification, password hygiene, acceptable use policies, incident reporting procedures, and physical security.
    \end{itemize}
    \vspace{1em}

    \item \textbf{[Informational] Update and Validate the Risk Register:}
    \begin{itemize}
        \item \textbf{Action:} Review the organization's risk register to formally close the risk associated with the open Port 80 (RISK-003).
        \item \textbf{Justification:} The technical scan confirmed that this risk has been mitigated. Maintaining an accurate and up-to-date risk register is crucial for effective risk management and for prioritizing future security efforts.
    \end{itemize}
\end{enumerate}

\end{document}
```