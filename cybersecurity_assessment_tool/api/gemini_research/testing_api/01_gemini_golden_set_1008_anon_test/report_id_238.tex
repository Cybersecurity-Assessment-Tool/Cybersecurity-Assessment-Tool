```latex
\documentclass[12pt]{article}

% Preamble: Required Packages
\usepackage[margin=1in]{geometry} % Set page margins
\usepackage{pifont}                 % For checkmarks and crosses (\ding{51}, \ding{55})
\usepackage{booktabs}               % For professional-looking tables
\usepackage{hyperref}               % For clickable links and better PDF metadata
\usepackage{url}                    % For typesetting URLs
\usepackage{seqsplit}               % To split long strings in \texttt
\usepackage[utf8]{inputenc}         % For UTF-8 input

% Document Metadata
\title{Cybersecurity Posture Assessment Report}
\author{Cybersecurity Analysis Division}
\date{\today}

\begin{document}

\maketitle
\thispagestyle{empty}
\newpage
\tableofcontents
\newpage

% -----------------------------------------------------------------------------
% Section 1: Executive Summary
% -----------------------------------------------------------------------------
\section*{Executive Summary}

This report provides a cybersecurity posture assessment for \textbf{[Organization Name]}, conducted on \today. The analysis is based on a review of organizational security controls, an external network vulnerability scan, and a summary of pre-existing risks.

The assessment reveals a mixed security posture. On a positive note, the external network perimeter scan of the target IP address \texttt{[Client IP]} showed no open ports, suggesting a well-configured firewall that effectively limits external exposure. Additionally, the organization enforces Multi-Factor Authentication (MFA) for computer logins and conducts annual security awareness training for all employees.

However, several critical and high-risk security gaps were identified through the organizational questionnaire. The most severe findings include the lack of MFA for accessing email and sensitive data systems. These gaps expose the organization to significant risks of account compromise and data breaches. Furthermore, the absence of a formal Acceptable Use Policy (AUP) and a security training program for new employees indicates foundational weaknesses in security governance and culture.

Immediate remediation is required to address the identified MFA and policy gaps to reduce the risk of a significant security incident. Detailed findings and actionable recommendations are provided in the subsequent sections of this report.

% -----------------------------------------------------------------------------
% Section 2: Organizational Information
% -----------------------------------------------------------------------------
\section*{Organizational Information}

This section details the information provided by the client organization. Due to the anonymized nature of the input data, placeholders are used where specific details were not available.

\begin{itemize}
    \item \textbf{Organization Name:} \textbf{[Organization Name]}
    \item \textbf{Primary Domain:} \texttt{[Domain]}
    \item \textbf{External IP Scanned:} \texttt{[Client IP]}
\end{itemize}

% -----------------------------------------------------------------------------
% Section 3: Security Control Review
% -----------------------------------------------------------------------------
\section*{Security Control Review}

The following table summarizes the organization's responses to a security controls questionnaire. A green checkmark (\ding{51}) indicates a positive control is in place, while a red cross (\ding{55}) indicates a security gap that requires attention.

\begin{table}[h!]
\centering
\caption{Organizational Security Controls Questionnaire}
\label{tab:controls}
\begin{tabular}{p{0.75\linewidth} c}
\toprule
\textbf{Control Question} & \textbf{Status} \\
\midrule
Do you require MFA to access email? & \ding{55} \\
Do you require MFA to log into computers? & \ding{51} \\
Do you require MFA to access sensitive data systems? & \ding{55} \\
Does your organization have an employee acceptable use policy? & \ding{55} \\
Does your organization do security awareness training for new employees? & \ding{55} \\
Does your organization do security awareness training for all employees at least once per year? & \ding{51} \\
\bottomrule
\end{tabular}
\end{table}

\subsection*{Analysis of Gaps}
The review identified four significant gaps:
\begin{itemize}
    \item \textbf{No MFA for Email:} Email is a primary target for phishing and account takeover attacks. The lack of MFA is a critical vulnerability.
    \item \textbf{No MFA for Sensitive Data Systems:} Failure to protect sensitive systems with MFA significantly increases the risk of a data breach.
    \item \textbf{No Acceptable Use Policy (AUP):} An AUP is a foundational policy that sets clear expectations for employee behavior and use of company assets. Its absence can lead to inconsistent security practices and insider threats.
    \item \textbf{No Onboarding Security Training:} New employees are often prime targets for social engineering. Failing to provide immediate security training leaves a critical window of vulnerability.
\end{itemize}

% -----------------------------------------------------------------------------
% Section 4: Technical Scan Results
% -----------------------------------------------------------------------------
\section*{Technical Scan Results}

An external network scan was performed to identify open ports and exposed services.

\begin{itemize}
    \item \textbf{Target IP:} \texttt{[Target IP]}
    \item \textbf{Scan Date:} \today
\end{itemize}

\subsection*{Findings}
The scan completed successfully and found \textbf{no open ports} on the target system. This is a positive security finding, indicating that the perimeter firewall is likely configured to deny all unsolicited inbound traffic, adhering to the principle of least privilege. This significantly reduces the external attack surface of the organization.

% -----------------------------------------------------------------------------
% Section 5: Consolidated Risk Assessment
% -----------------------------------------------------------------------------
\section*{Consolidated Risk Assessment}

This section correlates findings from the security control review and technical scans. The pre-existing risk register was empty, so all identified risks are new findings from this assessment.

\begin{table}[h!]
\centering
\caption{Summary of Identified Risks}
\label{tab:risks}
\begin{tabular}{p{0.25\linewidth} p{0.55\linewidth} l}
\toprule
\textbf{Risk Name} & \textbf{Overview} & \textbf{Severity} \\
\midrule
\textbf{Email Account Compromise} & The absence of MFA on email accounts makes them highly susceptible to takeover via phishing or credential stuffing attacks. & \textbf{Critical} \\
\addlinespace
\textbf{Sensitive Data Breach} & Lack of MFA on systems holding sensitive data allows an attacker with stolen credentials to gain direct access, leading to a potential data breach. & \textbf{Critical} \\
\addlinespace
\textbf{Lack of Security Governance} & The absence of a formal Acceptable Use Policy results in unclear security expectations for employees, increasing the risk of both accidental and malicious insider threats. & High \\
\addlinespace
\textbf{New Employee Vulnerability} & Without security training during onboarding, new hires are more likely to fall victim to social engineering attacks before they are integrated into the annual training cycle. & High \\
\bottomrule
\end{tabular}
\end{table}

% -----------------------------------------------------------------------------
% Section 6: Recommendations
% -----------------------------------------------------------------------------
\section*{Recommendations}

Based on the consolidated risk assessment, the following actions are recommended to improve the security posture of \textbf{[Organization Name]}. Recommendations are prioritized based on severity.

\subsection*{Critical Priority}
\begin{enumerate}
    \item \textbf{Implement MFA for Email and Sensitive Systems:}
    \begin{itemize}
        \item \textbf{Action:} Immediately enable and enforce MFA for all user access to the corporate email system (e.g., Microsoft 365, Google Workspace).
        \item \textbf{Action:} Identify all systems containing sensitive or critical data and enforce MFA for all access, both privileged and standard.
        \item \textbf{Justification:} This is the single most effective control to prevent account takeovers and subsequent data breaches.
    \end{itemize}
\end{enumerate}

\subsection*{High Priority}
\begin{enumerate}
    \setcounter{enumi}{1} % Continue numbering from previous list
    \item \textbf{Develop and Implement an Acceptable Use Policy (AUP):}
    \begin{itemize}
        \item \textbf{Action:} Draft a formal AUP that clearly defines the rules for using company networks, systems, and data. The policy should be reviewed by management, communicated to all employees, and formally acknowledged via signature.
        \item \textbf{Justification:} An AUP establishes a baseline for secure behavior and is a foundational component of a security governance program.
    \end{itemize}
    \item \textbf{Integrate Security Training into Employee Onboarding:}
    \begin{itemize}
        \item \textbf{Action:} Create a mandatory security awareness training module for all new employees to be completed within their first week of employment. This should cover key topics like phishing, password security, and the new AUP.
        \item \textbf{Justification:} This closes a critical vulnerability window and helps build a security-conscious culture from day one.
    \end{itemize}
\end{enumerate}

\end{document}
```