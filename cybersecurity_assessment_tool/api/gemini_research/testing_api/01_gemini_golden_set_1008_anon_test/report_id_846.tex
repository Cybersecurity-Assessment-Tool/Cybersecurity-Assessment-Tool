```latex
\documentclass[12pt]{article}

% --- PACKAGES ---
\usepackage[margin=1in]{geometry}
\usepackage{pifont} % For checkmarks and crosses
\usepackage{booktabs} % For professional tables
\usepackage{hyperref} % For hyperlinks
\usepackage{url} % For URL formatting
\usepackage{seqsplit} % For splitting long strings
\usepackage[utf8]{inputenc}

% --- DOCUMENT METADATA ---
\title{Cybersecurity Posture Assessment Report}
\author{Cybersecurity Analysis Division}
\date{\today}

% --- HYPERREF SETUP ---
\hypersetup{
    colorlinks=true,
    linkcolor=blue,
    filecolor=magenta,      
    urlcolor=cyan,
    pdftitle={Cybersecurity Posture Assessment Report},
    pdfpagemode=FullScreen,
}

% --- DOCUMENT START ---
\begin{document}

\maketitle
\thispagestyle{empty}
\newpage

\tableofcontents
\newpage

% ===================================================================
\section{Executive Summary}
% ===================================================================

This report provides a comprehensive analysis of the cybersecurity posture for \textbf{[Organization Name]}. The assessment is based on a correlation of a network vulnerability scan, a review of existing risks, and an organizational security controls questionnaire.

The primary findings indicate significant gaps in fundamental administrative and policy-based security controls. The lack of mandatory Multi-Factor Authentication (MFA) for email and computer access represents a \textbf{critical risk}, exposing the organization to account compromise and subsequent data breaches. Furthermore, the absence of an employee Acceptable Use Policy and a structured security awareness training program creates a high-risk environment susceptible to human error and insider threats.

On a positive note, the technical network scan of the external IP address \texttt{[Client IP]} revealed a secure perimeter with no open ports detected. This finding suggests that a previously identified risk, "Unencrypted Web Server" on Port 80, has been successfully remediated.

Immediate action is required to address the identified policy and access control deficiencies to reduce the organization's attack surface and improve its overall resilience against common cyber threats.

% ===================================================================
\section{Organizational Information}
% ===================================================================

The following information was used as the basis for this assessment. Due to the anonymized nature of the provided data, placeholders have been used where necessary.

\begin{itemize}
    \item \textbf{Organization Name:} \textbf{[Organization Name]}
    \item \textbf{Primary Domain:} \texttt{[Domain]}
    \item \textbf{External IP Scanned:} \texttt{[Client IP]}
\end{itemize}

% ===================================================================
\section{Security Control Review}
% ===================================================================

A review of the organization's security controls was conducted via a questionnaire. The responses highlight critical areas for improvement in identity and access management, policy enforcement, and employee security awareness. Answers marked with \ding{55} (No) are considered significant gaps in the security framework.

\begin{table}[h!]
\centering
\caption{Organizational Security Controls Questionnaire}
\begin{tabular}{p{0.8\linewidth} c}
\toprule
\textbf{Control Question} & \textbf{Response} \\
\midrule
Do you require MFA to access email? & \ding{55} \\
Do you require MFA to log into computers? & \ding{55} \\
Do you require MFA to access sensitive data systems? & \ding{51} \\
Does your organization have an employee acceptable use policy? & \ding{55} \\
Does your organization do security awareness training for new employees? & \ding{55} \\
Does your organization do security awareness training for all employees at least once per year? & \ding{55} \\
\bottomrule
\end{tabular}
\end{table}

\paragraph{Analysis:} The lack of MFA on email is a critical vulnerability, as email accounts are a primary target for attackers seeking to gain an initial foothold. The absence of a formal Acceptable Use Policy and security training program indicates a low level of security maturity and increases the risk of incidents caused by unintentional employee actions.

% ===================================================================
\section{Technical Scan Results}
% ===================================================================

An external network scan was performed on the target IP address to identify open ports and exposed services.

\begin{itemize}
    \item \textbf{Target IP:} \texttt{[Target IP]}
    \item \textbf{Scan Status:} Host is Up
\end{itemize}

The scan results below show that no open ports were discovered. This is a positive security finding, indicating a well-configured network perimeter firewall.

\begin{table}[h!]
\centering
\caption{Nmap Scan Port Summary}
\begin{tabular}{ccccc}
\toprule
\textbf{Port} & \textbf{State} & \textbf{Service} & \textbf{Product} & \textbf{Version} \\
\midrule
80 & closed & http & N/A & N/A \\
\bottomrule
\end{tabular}
\end{table}

\paragraph{Analysis:} The scan confirmed that port 80 is closed. This directly contradicts a pre-existing risk entry (see Section 5), suggesting that the vulnerability has been remediated. A secure external posture is crucial, but it does not mitigate the internal and policy-based risks identified in other sections.

% ===================================================================
\section{Correlated Risk Assessment}
% ===================================================================

This section synthesizes findings from the security questionnaire, the technical scan, and pre-existing risk data into a prioritized list of current risks.

\begin{table}[h!]
\centering
\caption{Summary of Identified Risks}
\begin{tabular}{p{0.25\linewidth} p{0.45\linewidth} l l}
\toprule
\textbf{Risk Name} & \textbf{Description} & \textbf{Severity} & \textbf{Status} \\
\midrule
\textbf{Inadequate Access Controls} & MFA is not enforced for email or computer logins, exposing the organization to account takeover and unauthorized access. & \textbf{Critical} & Active \\
\addlinespace
\textbf{Lack of Security Policy \& Training} & The absence of an Acceptable Use Policy and security awareness training increases the likelihood of security incidents caused by human error. & High & Active \\
\addlinespace
\textbf{Unencrypted Web Server} & A previously identified risk where port 80 was believed to be open. The current scan shows this port is now closed. & Medium & \textbf{Remediated} \\
\bottomrule
\end{tabular}
\end{table}

% ===================================================================
\section{Recommendations}
% ===================================================================

Based on the correlated risk assessment, the following actions are recommended to improve the security posture of \textbf{[Organization Name]}. Recommendations are prioritized by severity.

\begin{enumerate}
    \item \textbf{[Critical] Implement Multi-Factor Authentication (MFA):}
    \begin{itemize}
        \item Immediately enforce MFA for all user accounts on the primary email system.
        \item Develop a plan to roll out MFA for all computer logins, starting with administrative and privileged user accounts.
    \end{itemize}

    \item \textbf{[High] Establish a Security Awareness Program:}
    \begin{itemize}
        \item Develop and implement a formal security awareness training module for all new employees as part of their onboarding process.
        \item Mandate annual refresher security training for all staff, covering topics like phishing, password hygiene, and data handling.
    \end{itemize}

    \item \textbf{[High] Develop and Enforce an Acceptable Use Policy (AUP):}
    \begin{itemize}
        \item Create a clear and concise AUP that governs the use of company assets, including computers, networks, and data.
        \item Require all employees to read and formally acknowledge the policy.
    \end{itemize}
    
    \item \textbf{[Informational] Confirm and Document Risk Remediation:}
    \begin{itemize}
        \item Internally validate that the closure of port 80 was an intentional and documented change.
        \item Update the internal risk register to formally close the "Unencrypted Web Server" risk item.
    \end{itemize}
\end{enumerate}

\end{document}
```