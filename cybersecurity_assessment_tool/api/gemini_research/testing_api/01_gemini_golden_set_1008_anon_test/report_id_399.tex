```latex
\documentclass[12pt]{article}

% Required Packages
\usepackage[margin=1in]{geometry}
\usepackage{pifont} % For \ding{51} and \ding{55}
\usepackage{booktabs} % For professional tables
\usepackage{hyperref} % For clickable links and metadata
\usepackage{url} % For URL formatting
\usepackage{seqsplit} % For splitting long strings in \texttt
\usepackage{xcolor} % For colors

% Document Metadata
\hypersetup{
    colorlinks=true,
    linkcolor=blue,
    filecolor=magenta,      
    urlcolor=cyan,
    pdftitle={Cybersecurity Posture Assessment Report},
    pdfauthor={Cybersecurity Analyst},
    pdfsubject={Security Analysis},
    pdfkeywords={Cybersecurity, Risk, Assessment},
}

% Title and Author
\title{Cybersecurity Posture Assessment Report}
\author{Generated by Expert Cybersecurity Analyst}
\date{\today}

\begin{document}

\maketitle
\tableofcontents
\newpage

% --- 1. Executive Summary ---
\section{Executive Summary}

This report provides a comprehensive cybersecurity assessment for \textbf{[Organization Name]}, based on an analysis of network scan data, organizational security controls, and pre-existing risk documentation. The assessment reveals several critical and high-severity risks that require immediate attention to mitigate potential security breaches.

Key findings indicate a significant external exposure of a core database service running an End-of-Life (EOL) software version, which no longer receives security updates. This is compounded by critical gaps in identity and access management, specifically the absence of Multi-Factor Authentication (MFA) for email and endpoint access.

The overall security posture is considered weak due to these fundamental control failures. The recommendations outlined in this report provide a clear, actionable roadmap for remediating the identified vulnerabilities and strengthening the organization's defenses against common cyber threats.

% --- 2. Organizational Information ---
\section{Organizational Information}

This section details the organizational data used for this assessment. The information has been anonymized as per the engagement protocol.

\begin{itemize}
    \item \textbf{Organization Name:} \textbf{[Organization Name]}
    \item \textbf{Primary Email Domain:} \texttt{[Domain]}
    \item \textbf{Scanned External IP:} \texttt{[Client IP]}
\end{itemize}

% --- 3. Security Control Review ---
\section{Security Control Review}

An assessment of the organization's self-reported security controls was conducted via a questionnaire. The following table summarizes the responses and highlights significant gaps in foundational security practices. A checkmark (\ding{51}) indicates a positive control is in place, while an X (\ding{55}) indicates a control gap.

\begin{table}[h!]
\centering
\caption{Organizational Security Control Questionnaire}
\begin{tabular}{p{8cm} c c}
\toprule
\textbf{Control Question} & \textbf{Response} & \textbf{Status} \\
\midrule
Do you require MFA to access email? & No & \textcolor{red}{\ding{55}} \\
Do you require MFA to log into computers? & No & \textcolor{red}{\ding{55}} \\
Do you require MFA to access sensitive data systems? & Yes & \textcolor{green}{\ding{51}} \\
Does your organization have an employee acceptable use policy? & Yes & \textcolor{green}{\ding{51}} \\
Does your organization do security awareness training for new employees? & Yes & \textcolor{green}{\ding{51}} \\
Does your organization do security awareness training for all employees at least once per year? & Yes & \textcolor{green}{\ding{51}} \\
\bottomrule
\end{tabular}
\end{table}

\subsection*{Analysis of Control Gaps}
The lack of MFA for email and computer logins represents a critical vulnerability. Email is a primary target for phishing attacks leading to account compromise, while unprotected endpoint logins can lead to unauthorized access and lateral movement within the network.

% --- 4. Technical Scan Results ---
\section{Technical Scan Results}

An external network scan was performed on the target IP address to identify open ports and exposed services.

\subsection*{Scan Target: \texttt{[Target IP]}}
The scan revealed the following open port:

\begin{table}[h!]
\centering
\caption{Open Ports and Services}
\begin{tabular}{l l l l}
\toprule
\textbf{Port} & \textbf{Service} & \textbf{Product} & \textbf{Version} \\
\midrule
3306/tcp & mysql & MySQL & 5.7.33 \\
\bottomrule
\end{tabular}
\end{table}

\subsection*{Technical Analysis}
The scan confirms that a MySQL database server is directly exposed to the public internet on port 3306. This configuration is highly discouraged as it exposes the database to brute-force attacks, credential stuffing, and exploitation of potential vulnerabilities.

\textbf{Critical Finding:} The identified MySQL version, \textbf{5.7.33}, reached its official End-of-Life (EOL) in October 2023. This means it no longer receives security patches from the vendor, and any newly discovered vulnerabilities will remain unpatched, posing a severe and unmitigable risk to the data it contains.

% --- 5. Consolidated Risk Assessment ---
\section{Consolidated Risk Assessment}

This section synthesizes findings from the security control review, technical scan, and pre-existing risk data into a consolidated list of key risks.

\begin{table}[h!]
\centering
\caption{Summary of Identified Risks}
\begin{tabular}{p{1.5cm} p{4cm} p{6.5cm} l}
\toprule
\textbf{Risk ID} & \textbf{Risk Name} & \textbf{Description} & \textbf{Severity} \\
\midrule
RISK-001 & Public Exposure of End-of-Life Database & A MySQL 5.7.33 database (EOL) is accessible from the internet on port 3306, exposing it to attack and exploitation. & \textbf{Critical} \\
\addlinespace
RISK-002 & Lack of MFA for Email Access & The absence of MFA on email accounts allows for account takeover via credential theft or phishing, leading to data breaches and further attacks. & \textbf{Critical} \\
\addlinespace
RISK-003 & Lack of MFA for Endpoint Logins & The absence of MFA on computer logins allows attackers with stolen credentials to gain direct access to internal systems. & \textbf{High} \\
\bottomrule
\end{tabular}
\end{table}

% --- 6. Recommendations ---
\section{Recommendations}

The following actionable recommendations are provided to address the identified risks. Risks should be remediated in order of severity.

\subsection*{RISK-001: Public Exposure of End-of-Life Database (Critical)}
\begin{itemize}
    \item \textbf{Immediate Action:} Implement strict firewall rules to block all public access to TCP port 3306. Access should only be permitted from trusted internal IP addresses.
    \item \textbf{Short-Term Plan:} Develop a migration plan to upgrade the MySQL 5.7 database to a currently supported version (e.g., MySQL 8.x) to ensure security patches are available.
    \item \textbf{Long-Term Strategy:} Enforce a policy where no database servers are directly exposed to the internet. All administrative or application access should be routed through a secure channel, such as a VPN or an authenticated bastion host.
\end{itemize}

\subsection*{RISK-002: Lack of MFA for Email Access (Critical)}
\begin{itemize}
    \item \textbf{Immediate Action:} Procure and enforce an MFA solution for all user accounts on the primary email system. This is the single most effective control to prevent email account compromise.
    \item \textbf{Supporting Action:} Conduct a phishing awareness campaign to educate users on the importance of MFA and how to identify and report suspicious emails.
\end{itemize}

\subsection*{RISK-003: Lack of MFA for Endpoint Logins (High)}
\begin{itemize}
    \item \textbf{Immediate Action:} Deploy and mandate MFA for all user logins to company-managed computers (desktops and laptops). This significantly raises the difficulty for an attacker to use stolen credentials to access the internal network.
\end{itemize}

\end{document}
```