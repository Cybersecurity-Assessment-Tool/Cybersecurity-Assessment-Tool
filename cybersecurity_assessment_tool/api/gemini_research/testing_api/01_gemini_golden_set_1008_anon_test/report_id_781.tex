```latex
\documentclass[12pt]{article}

% Preamble: Required Packages
\usepackage[margin=1in]{geometry}
\usepackage{pifont} % For checkmarks and crosses (\ding)
\usepackage{booktabs} % For professional-looking tables
\usepackage{hyperref} % For hyperlinks and document metadata
\usepackage{url} % For formatting URLs
\usepackage{seqsplit} % For splitting long strings without breaking
\usepackage[utf8]{inputenc}

% Hyperref Setup
\hypersetup{
    colorlinks=true,
    linkcolor=black,
    urlcolor=blue,
    pdftitle={Cybersecurity Assessment Report},
    pdfauthor={Cybersecurity Analyst},
    pdfsubject={Security Analysis},
    pdfkeywords={Security, Nmap, Risk Assessment},
    bookmarks=true,
    bookmarksopen=true
}

% Document Start
\begin{document}

% Title Page
\title{
    \textbf{Cybersecurity Assessment Report} \\
    \large For: \textbf{[Organization Name]}
}
\author{Cybersecurity Analyst}
\date{November 22, 2025}
\maketitle

\hrule
\vspace{1cm}

% Table of Contents
\tableofcontents

\newpage

% --- Executive Summary ---
\section{Executive Summary}
This report provides a comprehensive analysis of the cybersecurity posture of \textbf{[Organization Name]}, based on data collected on November 22, 2025. The assessment combines a review of organizational security controls, an external network scan, and an evaluation of pre-existing risks.

The organization demonstrates a solid foundation in security awareness and general Multi-Factor Authentication (MFA) adoption for email and computer access. However, two high-risk vulnerabilities were identified that require immediate attention:

\begin{enumerate}
    \item \textbf{Inadequate MFA on Sensitive Systems:} The lack of mandatory MFA for accessing sensitive data systems represents a critical control gap. This significantly increases the risk of a data breach resulting from compromised user credentials.
    \item \textbf{Vulnerable Web Server Software:} The external-facing web server is running an outdated and vulnerable version of Nginx (1.18.0). This exposes the organization to numerous publicly known exploits that could lead to server compromise, data exfiltration, or denial of service.
\end{enumerate}

This report details these findings and provides actionable recommendations to mitigate the identified risks and strengthen the overall security posture.

% --- Organizational Information ---
\section{Organizational Information}
This section provides the key identification details for the organization under review. As this data was not provided, placeholders are used.

\begin{tabular}{@{}ll}
    \toprule
    \textbf{Detail} & \textbf{Value} \\
    \midrule
    Organization Name & \textbf{[Organization Name]} \\
    Primary Domain & \texttt{[Domain]} \\
    External IP Address Scanned & \texttt{[Client IP]} \\
    \bottomrule
\end{tabular}

% --- Security Control Review ---
\section{Security Control Review}
A review of the organization's security controls was conducted via a questionnaire. The responses indicate a good baseline but highlight a critical gap in access control for sensitive data.

\begin{table}[h!]
\centering
\caption{Security Questionnaire Responses}
\begin{tabular}{@{}lc@{}}
\toprule
\textbf{Control Question} & \textbf{Response} \\
\midrule
Do you require MFA to access email? & \ding{51} \\ % Yes
Do you require MFA to log into computers? & \ding{51} \\ % Yes
\textbf{Do you require MFA to access sensitive data systems?} & \textbf{\color{red}\ding{55}} \\ % No
Does your organization have an employee acceptable use policy? & \ding{51} \\ % Yes
Does your organization do security awareness training for new employees? & \ding{51} \\ % Yes
Does your organization do security awareness training for all employees? & \ding{51} \\ % Yes
\bottomrule
\end{tabular}
\end{table}

\paragraph{Analysis:} The failure to enforce MFA on sensitive data systems is a significant finding. While MFA is correctly applied to email and workstations, sensitive data remains the ultimate target for attackers. Accessing this data with only a username and password combination is a high-risk practice that does not align with industry best practices.

% --- Technical Scan Results ---
\section{Technical Scan Results}
An external network scan was performed against the target IP address \texttt{[Target IP]} to identify open ports and exposed services.

\begin{table}[h!]
\centering
\caption{Nmap Scan Results for \texttt{[Target IP]}}
\begin{tabular}{@{}lllll@{}}
\toprule
\textbf{Port} & \textbf{State} & \textbf{Service} & \textbf{Product} & \textbf{Version} \\
\midrule
443/tcp & Open & https & nginx & 1.18.0 \\
\bottomrule
\end{tabular}
\end{table}

\paragraph{Analysis:} The scan identified a single open port (443/tcp) running an Nginx web server, version \textbf{1.18.0}. This version was released in April 2020 and is now considered outdated. It is associated with multiple Common Vulnerabilities and Exposures (CVEs), including but not limited to CVE-2021-23017. Running outdated software on internet-facing systems presents a high risk of compromise.

% --- Pre-existing Risks ---
\section{Pre-existing Risks}
A review of the provided list of current organizational risks was conducted.
\begin{itemize}
    \item No pre-existing vulnerabilities were reported in the input data.
\end{itemize}

% --- Risk Assessment Summary ---
\section{Risk Assessment Summary}
The following table synthesizes the findings from the security control review and the technical scan into a prioritized list of risks.

\begin{table}[h!]
\centering
\caption{Identified Risks and Severity}
\begin{tabular}{@{}p{0.3\linewidth}p{0.5\linewidth}l@{}}
\toprule
\textbf{Risk Name} & \textbf{Overview} & \textbf{Severity} \\
\midrule
\textbf{Inadequate MFA on Sensitive Systems} & The absence of MFA on systems storing or processing sensitive data exposes critical assets to unauthorized access via compromised credentials. & \textbf{High} \\
\addlinespace
\textbf{Vulnerable Nginx Web Server} & The public-facing web server is running Nginx 1.18.0, an outdated version with known vulnerabilities. This could be exploited to compromise the server. & \textbf{High} \\
\bottomrule
\end{tabular}
\end{table}

% --- Recommendations ---
\section{Recommendations}
Based on the risk assessment, the following actions are recommended to mitigate the identified vulnerabilities.

\subsection{Remediation for: Inadequate MFA on Sensitive Systems}
\begin{itemize}
    \item \textbf{Immediate Action:} Implement and enforce a non-bypassable MFA policy for all user accounts (including administrative and service accounts) that have access to sensitive data systems.
    \item \textbf{Strategic Action:} Conduct a full audit of all systems to classify data sensitivity and ensure that access controls are commensurate with the data's value and risk.
\end{itemize}

\subsection{Remediation for: Vulnerable Nginx Web Server}
\begin{itemize}
    \item \textbf{Immediate Action:} Upgrade the Nginx server on host \texttt{[Target IP]} from version 1.18.0 to the latest stable version recommended by the vendor. Prior to the upgrade, perform testing in a staging environment to ensure compatibility.
    \item \textbf{Strategic Action:} Establish a formal patch management and vulnerability management policy. This policy should mandate regular scanning and timely patching of all internet-facing systems and software.
\end{itemize}

% --- Conclusion ---
\section{Conclusion}
\textbf{[Organization Name]} has established some positive security practices, particularly in security awareness and MFA for initial access points. However, the critical vulnerabilities identified in this report—the MFA gap for sensitive data and the outdated web server software—pose a significant and immediate threat to the organization's security and data integrity.

We strongly advise prioritizing the remediation actions outlined in the Recommendations section to reduce the attack surface and enhance the overall cybersecurity resilience of the organization.

% Document End
\end{document}
```