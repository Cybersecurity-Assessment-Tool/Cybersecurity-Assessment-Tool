```latex
\documentclass[12pt]{article}

% --- PACKAGES ---
\usepackage[margin=1in]{geometry} % Set page margins
\usepackage{pifont}               % For checkmarks and crosses (\ding{51}, \ding{55})
\usepackage{booktabs}             % For professional tables (\toprule, \midrule, \bottomrule)
\usepackage[hidelinks]{hyperref}  % For clickable links without boxes
\usepackage{url}                  % For formatting URLs
\usepackage{seqsplit}             % For splitting long strings in texttt
\usepackage{graphicx}             % For logos (optional, but good practice)
\usepackage{xcolor}               % For colors

% --- DOCUMENT METADATA ---
\title{Cybersecurity Posture Assessment Report\\ \large For \textbf{[Organization Name]}}
\author{Cybersecurity Analysis Division}
\date{\today}

% --- DOCUMENT START ---
\begin{document}

\maketitle
\thispagestyle{empty}
\newpage

\tableofcontents
\newpage

% ==============================================================================
\section*{1. Executive Summary}
% ==============================================================================

This report provides a comprehensive analysis of the cybersecurity posture for \textbf{[Organization Name]}. The assessment is based on a correlation of external network scan data, a review of internal security controls via a questionnaire, and an evaluation of pre-existing risk documentation.

The overall security posture is determined to be \textbf{CRITICAL}. Several severe vulnerabilities and procedural gaps were identified that expose the organization to significant risk of data breach, unauthorized access, and service disruption.

Key critical findings include:
\begin{itemize}
    \item \textbf{Exposed End-of-Life Database:} A MySQL database (version 5.7.33) is directly exposed to the public internet. This version is past its End-of-Life (EOL) as of October 2023 and no longer receives security updates, making it an easy target for attackers.
    \item \textbf{Systemic Lack of Multi-Factor Authentication (MFA):} MFA is not enforced for email, computer logins, or access to sensitive data systems. This represents a critical control failure, as a single compromised password could grant an attacker widespread access.
    \item \textbf{Deficient Security Policies and Training:} The absence of a formal Acceptable Use Policy and a mandatory annual security awareness training program for all employees creates a culture that is highly susceptible to human error and social engineering attacks.
\end{itemize}

Immediate remediation of these issues is strongly recommended to reduce the organization's attack surface and mitigate the high probability of a security incident.

% ==============================================================================
\section*{2. Organizational Information}
% ==============================================================================

The following information was used as the basis for this assessment. Due to the anonymized nature of the provided data, placeholders have been used where necessary.

\begin{table}[h!]
\centering
\begin{tabular}{@{}ll@{}}
\toprule
\textbf{Attribute} & \textbf{Value} \\ \midrule
Organization Name  & \textbf{[Organization Name]} \\
Primary Domain     & \texttt{[Domain]} \\
External IP Scanned & \texttt{[Client IP]} \\ \bottomrule
\end{tabular}
\caption{Client Organizational Data}
\end{table}

% ==============================================================================
\section*{3. Security Control Review}
% ==============================================================================

A security questionnaire was reviewed to assess the maturity of internal security controls. The results indicate significant gaps in fundamental security practices. A "No" answer highlights a missing control and a potential area of high risk.

\begin{table}[h!]
\centering
\begin{tabular}{@{}p{0.8\textwidth}c@{}}
\toprule
\textbf{Control Question} & \textbf{Status} \\ \midrule
Do you require MFA to access email? & \textcolor{red}{\ding{55}} \\
Do you require MFA to log into computers? & \textcolor{red}{\ding{55}} \\
Do you require MFA to access sensitive data systems? & \textcolor{red}{\ding{55}} \\
Does your organization have an employee acceptable use policy? & \textcolor{red}{\ding{55}} \\
Does your organization do security awareness training for new employees? & \textcolor{green}{\ding{51}} \\
Does your organization do security awareness training for all employees at least once per year? & \textcolor{red}{\ding{55}} \\ \bottomrule
\end{tabular}
\caption{Security Controls Questionnaire Results}
\end{table}

\subsection*{Analysis of Control Gaps}
The lack of MFA across all critical access points (email, endpoints, data systems) is a critical vulnerability. This single point of failure dramatically increases the risk of account takeover attacks. Furthermore, the absence of an acceptable use policy and annual security training indicates a reactive, rather than proactive, approach to managing human-related risk.

% ==============================================================================
\section*{4. Technical Scan Results}
% ==============================================================================

An external network scan was performed against the target IP address \texttt{[Target IP]}. The scan identified one open port, which presents a significant and immediate risk.

\begin{table}[h!]
\centering
\begin{tabular}{@{}lllll@{}}
\toprule
\textbf{Port} & \textbf{State} & \textbf{Service} & \textbf{Product} & \textbf{Version} \\ \midrule
3306/tcp      & open           & mysql            & MySQL            & 5.7.33           \\ \bottomrule
\end{tabular}
\caption{Open Port Findings}
\end{table}

\subsection*{Analysis of Technical Findings}
The scan confirms that a MySQL database server is directly accessible from the public internet on port 3306. This configuration is highly discouraged as it exposes the database to brute-force attacks, credential stuffing, and exploitation of known vulnerabilities.

Crucially, the identified version, \textbf{MySQL 5.7.33}, reached its official End-of-Life (EOL) in October 2023. EOL software no longer receives security patches from the vendor, meaning any newly discovered vulnerabilities will remain unpatched. This elevates the risk of compromise from High to \textbf{Critical}. This finding directly validates the pre-existing risk documented in "Database Exposure."

% ==============================================================================
\section*{5. Consolidated Risk Assessment}
% ==============================================================================

By correlating the security control gaps, technical findings, and pre-existing risk data, we have synthesized a prioritized list of organizational risks.

\begin{table}[h!]
\centering
\begin{tabular}{@{}p{0.2\textwidth}p{0.5\textwidth}p{0.2\textwidth}@{}}
\toprule
\textbf{Severity} & \textbf{Risk Title \& Description} & \textbf{Affected Assets} \\ \midrule
\textbf{CRITICAL} & \textbf{Exposed End-of-Life Database Service} \newline An EOL MySQL database is directly exposed to the internet, making it a prime target for automated attacks and exploitation of unpatched vulnerabilities. & Database Server, All Stored Data, \texttt{[Target IP]}:3306 \\
\addlinespace
\textbf{CRITICAL} & \textbf{Lack of Multi-Factor Authentication} \newline The absence of MFA on all critical systems means a single compromised password could lead to a full-scale breach of email, internal networks, and sensitive data. & User Accounts, Email System, Endpoints, Data Repositories \\
\addlinespace
\textbf{HIGH} & \textbf{Insufficient Security Policies \& Training} \newline Without an acceptable use policy or recurring security training, employees are more likely to fall victim to phishing, misuse assets, or mishandle data, increasing overall organizational risk. & All Employees, All Corporate Data \\
\bottomrule
\end{tabular}
\caption{Synthesized Risk Summary}
\end{table}

% ==============================================================================
\section*{6. Recommendations}
% ==============================================================================

The following actionable recommendations are provided to address the identified risks. They are prioritized based on severity and the potential impact of a successful attack.

\subsection*{Immediate Priority (Remediate within 72 hours)}
\begin{enumerate}
    \item \textbf{Firewall the Exposed Database:} Immediately implement a firewall rule to block all public access to TCP port 3306 on \texttt{[Target IP]}. Access should only be permitted from trusted internal IP addresses or through a secure VPN connection.
    \item \textbf{Develop a Migration Plan:} Begin planning the immediate migration of the MySQL 5.7.33 database to a fully supported version (e.g., MySQL 8.0 or later) to ensure security patches are available.
\end{enumerate}

\subsection*{High Priority (Remediate within 30-60 days)}
\begin{enumerate}
    \item \textbf{Deploy Multi-Factor Authentication (MFA):}
        \begin{itemize}
            \item Phase 1: Enable MFA for email access for all users.
            \item Phase 2: Enable MFA for VPN and access to all systems containing sensitive data.
            \item Phase 3: Enable MFA for all computer/endpoint logins.
        \end{itemize}
    \item \textbf{Establish Foundational Policies:} Draft and implement a formal Employee Acceptable Use Policy that all employees must read and acknowledge.
    \item \textbf{Implement Annual Security Training:} Procure and deploy a security awareness training program that is mandatory for all employees on an annual basis.
\end{enumerate}

\subsection*{Medium Priority (Remediate within 90 days)}
\begin{enumerate}
    \item \textbf{Conduct Comprehensive Vulnerability Scanning:} Perform an authenticated and unauthenticated vulnerability scan across all external and internal assets to identify other potential weaknesses.
\end{enumerate}

\end{document}
```