```latex
\documentclass[12pt]{article}

% Preamble: Required Packages
\usepackage[margin=1in]{geometry}
\usepackage{pifont} % For checkmarks and crosses
\usepackage{booktabs} % For professional tables
\usepackage{hyperref} % For hyperlinks
\usepackage{url}      % For URL formatting
\usepackage{seqsplit} % For splitting long strings without breaking
\usepackage{graphicx}
\usepackage{xcolor}

% Document Information
\title{Cybersecurity Assessment Report}
\author{Cybersecurity Analysis Division}
\date{\today}

% Hyperref Setup
\hypersetup{
    colorlinks=true,
    linkcolor=blue,
    filecolor=magenta,      
    urlcolor=cyan,
    pdftitle={Cybersecurity Assessment Report},
    pdfpagemode=FullScreen,
}

% Custom Commands
\newcommand{\yes}{\ding{51}}
\newcommand{\no}{\ding{55}}
\newcommand{\orgname}{\textbf{[Organization Name]}}
\newcommand{\clientip}{\texttt{[Client IP]}}
\newcommand{\clientdomain}{\texttt{[Domain]}}
\newcommand{\targetip}{\texttt{[Target IP]}}

\begin{document}

\maketitle
\thispagestyle{empty}
\newpage

\tableofcontents
\newpage

% --- 1. Executive Summary ---
\section{Executive Summary}

This report provides a cybersecurity assessment for \orgname, based on a combination of organizational data review, an external network scan, and an analysis of pre-existing risks.

The external network posture appears strong; a network scan performed against the primary external IP address (\targetip) revealed \textbf{no open ports}. This indicates a well-configured perimeter firewall that effectively limits external exposure, which is a significant security strength.

However, the internal security control review identified several critical and high-risk gaps. The most pressing concerns are the lack of Multi-Factor Authentication (MFA) for accessing email and logging into company computers. These gaps expose the organization to significant risks, including business email compromise, ransomware attacks, and unauthorized data access, should user credentials be compromised. Furthermore, the absence of a formal Employee Acceptable Use Policy creates ambiguity regarding the secure and appropriate use of company assets.

While security awareness training practices are commendable, immediate action is required to address the identified identity and access management weaknesses. Recommendations focus on the swift implementation of MFA across critical systems and the formalization of internal security policies to mitigate these risks and bolster the organization's overall security posture.

% --- 2. Organizational Information ---
\section{Organizational Information}

This section details the information provided by the client organization. The data has been anonymized as per the engagement requirements.

\begin{tabular}{@{}ll}
\toprule
\textbf{Attribute} & \textbf{Value} \\
\midrule
Organization Name & \orgname \\
Primary Email Domain & \clientdomain \\
Primary External IP & \clientip \\
\bottomrule
\end{tabular}

% --- 3. Security Control Review ---
\section{Security Control Review}

The following table summarizes the organization's responses to a security controls questionnaire. Each response is evaluated against standard cybersecurity best practices. Items marked with \no\ represent significant gaps in the current security posture.

\begin{tabular}{@{}p{0.6\linewidth}cp{0.25\linewidth}@{}}
\toprule
\textbf{Control Question} & \textbf{Status} & \textbf{Analyst Note} \\
\midrule
Do you require MFA to access email? & \no & \textcolor{red}{\textbf{Critical Risk.}} Lack of MFA on email is a primary vector for account takeovers and phishing attacks. \\
\addlinespace
Do you require MFA to log into computers? & \no & \textcolor{red}{\textbf{Critical Risk.}} Compromised credentials could lead directly to device and network access. \\
\addlinespace
Do you require MFA to access sensitive data systems? & \yes & Good. Protects the most critical data stores. \\
\addlinespace
Does your organization have an employee acceptable use policy? & \no & \textcolor{orange}{\textbf{High Risk.}} Lack of a formal policy creates legal and operational risks. \\
\addlinespace
Does your organization do security awareness training for new employees? & \yes & Good. Establishes a security baseline for new hires. \\
\addlinespace
Does your organization do security awareness training for all employees at least once per year? & \yes & Good. Reinforces security concepts annually. \\
\bottomrule
\end{tabular}

% --- 4. Technical Scan Results ---
\section{Technical Scan Results}

An external network vulnerability scan was conducted to identify open ports and exposed services on the organization's public-facing infrastructure.

\begin{itemize}
    \item \textbf{Target IP Address:} \targetip
    \item \textbf{Scan Date:} Not Specified in Scan Data
    \item \textbf{Scan Summary:} The scan completed successfully and determined the host was online.
\end{itemize}

\subsection{Findings}
The scan results were positive, indicating a strong network perimeter defense.
\begin{itemize}
    \item \textbf{Open Ports:} \textbf{None.} No open TCP or UDP ports were discovered.
    \item \textbf{Port State:} All scanned ports were found to be in a `closed` state, meaning they are accessible but have no application listening on them. This is the desired state for ports that are not explicitly required for business operations.
\end{itemize}

This result suggests that the organization's firewall is correctly configured to deny unsolicited inbound traffic, significantly reducing the external attack surface.

% --- 5. Consolidated Risk Assessment ---
\section{Consolidated Risk Assessment}

This section synthesizes findings from the security control review, technical scan, and pre-existing risk data. No pre-existing vulnerabilities were provided for this assessment. The primary risks identified stem from internal policy and access control gaps.

\begin{tabular}{@{}p{0.25\linewidth}p{0.5\linewidth}l@{}}
\toprule
\textbf{Risk Name} & \textbf{Overview} & \textbf{Severity} \\
\midrule
\textbf{Lack of MFA on Email} & Threat actors can gain full access to an employee's mailbox using only stolen credentials. This enables business email compromise, data exfiltration, and further phishing attacks against internal and external contacts. & \textcolor{red}{\textbf{Critical}} \\
\addlinespace
\textbf{Lack of MFA on Workstations} & If an employee's password is stolen (e.g., from a third-party breach), an attacker can log directly into their computer, gaining access to local files and potentially the internal network. This is a common entry point for ransomware. & \textcolor{red}{\textbf{Critical}} \\
\addlinespace
\textbf{No Acceptable Use Policy (AUP)} & Without a formal AUP, there are no clear, enforceable rules for employees regarding the use of company technology. This can lead to unintentional data exposure, misuse of resources, and a weakened legal standing in the event of an insider incident. & \textcolor{orange}{\textbf{High}} \\
\bottomrule
\end{tabular}

% --- 6. Recommendations ---
\section{Recommendations}

Based on the consolidated risk assessment, the following actions are recommended to improve the security posture of \orgname. Recommendations are prioritized based on risk severity.

\begin{enumerate}
    \item \textbf{[Critical] Implement MFA for Email Access:}
    \begin{itemize}
        \item \textbf{Action:} Enforce MFA for all users accessing the email system (e.g., Office 365, Google Workspace).
        \item \textbf{Justification:} This is the single most effective control to prevent email account takeovers. It immediately mitigates the risk of business email compromise and phishing-related breaches.
    \end{itemize}
    
    \item \textbf{[Critical] Enforce MFA for Computer Logons:}
    \begin{itemize}
        \item \textbf{Action:} Deploy an MFA solution for all employee workstation and laptop logins (e.g., Windows Hello, Duo, etc.).
        \item \textbf{Justification:} This prevents unauthorized access to devices and the internal network, even if a user's password is stolen. It serves as a crucial defense against ransomware deployment.
    \end{itemize}
    
    \item \textbf{[High] Develop and Implement an Acceptable Use Policy (AUP):}
    \begin{itemize}
        \item \textbf{Action:} Draft a formal AUP that clearly defines the rules for using company networks, computers, email, and data. Require all employees to read and formally acknowledge the policy.
        \item \textbf{Justification:} An AUP establishes clear expectations for behavior, reduces insider risk, and provides a framework for disciplinary action in case of policy violations.
    \end{itemize}
\end{enumerate}

\end{document}
```