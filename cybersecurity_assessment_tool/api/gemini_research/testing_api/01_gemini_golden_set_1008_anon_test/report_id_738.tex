```latex
\documentclass[12pt]{article}

% Preamble: Required Packages
\usepackage[margin=1in]{geometry}
\usepackage{pifont} % For checkmarks and crosses
\usepackage{booktabs} % For professional tables
\usepackage{hyperref} % For clickable links
\usepackage{url} % For URL formatting
\usepackage{seqsplit} % For splitting long strings
\usepackage{graphicx}
\usepackage{xcolor}
\usepackage{datetime}

% --- Document Metadata ---
\title{Cybersecurity Posture Assessment Report}
\author{Lead Cybersecurity Analyst}
\date{\today}

% --- Hyperref Setup ---
\hypersetup{
    colorlinks=true,
    linkcolor=blue,
    filecolor=magenta,      
    urlcolor=cyan,
    pdftitle={Cybersecurity Posture Assessment Report},
    pdfpagemode=FullScreen,
}

% --- Custom Commands ---
\newcommand{\yes}{\ding{51}}
\newcommand{\no}{\ding{55}}
\newcommand{\placeholder}[1]{\textbf{[#1]}}
\newcommand{\code}[1]{\texttt{#1}}

\begin{document}

\maketitle
\thispagestyle{empty}
\newpage

\tableofcontents
\newpage

% ==============================================================================
% 1. Executive Summary
% ==============================================================================
\section{Executive Summary}

This report provides a comprehensive cybersecurity assessment for \placeholder{Organization Name}, based on a correlation of organizational data, an external network scan, and a review of pre-existing risks.

The assessment reveals a mixed security posture. The organization has implemented some strong security controls, such as requiring Multi-Factor Authentication (MFA) for email and sensitive data systems. The external network scan of the target IP address, \code{[Target IP]}, showed a secure perimeter with no open ports discovered, which is a significant positive finding.

However, several critical and high-risk gaps were identified in procedural and endpoint security controls. The most pressing issues include:
\begin{itemize}
    \item \textbf{Lack of Endpoint MFA:} The absence of MFA for computer logins presents a significant risk, as a single compromised password could grant an attacker direct access to an endpoint.
    \item \textbf{Missing Foundational Policies:} The organization lacks a formal Employee Acceptable Use Policy (AUP), creating ambiguity regarding secure practices.
    \item \textbf{Inadequate Employee Onboarding:} New employees do not receive security awareness training upon hiring, leaving a critical window of vulnerability.
\end{itemize}

The technical scan results contradict a previously identified risk concerning an "Unencrypted Web Server" on port 80, as this port was found to be closed. This suggests the risk has been remediated or was based on outdated information.

Immediate action is recommended to address the identified policy and endpoint security gaps to reduce the overall risk profile and enhance the organization's resilience against common cyber threats.

% ==============================================================================
% 2. Organizational Information
% ==============================================================================
\section{Organizational Information}

This section contains the high-level information provided for the assessment. As this report was generated in template mode due to missing identity data, placeholders are used.

\begin{table}[h!]
\centering
\begin{tabular}{@{}ll@{}}
\toprule
\textbf{Attribute} & \textbf{Value} \\ \midrule
Organization Name & \placeholder{Organization Name} \\
Primary Domain & \code{[Domain]} \\
External IP Scanned & \code{[Client IP]} \\
Report Date & \today \\ \bottomrule
\end{tabular}
\caption{Client Organizational Details}
\end{table}

% ==============================================================================
% 3. Security Control Review (Questionnaire Analysis)
% ==============================================================================
\section{Security Control Review (Questionnaire Analysis)}

The following table summarizes the organization's responses to a security controls questionnaire. Items marked with a red cross (\no) indicate a deviation from security best practices and represent a control gap.

\begin{table}[h!]
\centering
\begin{tabular}{@{}p{0.7\textwidth}cc@{}}
\toprule
\textbf{Control Question} & \textbf{Response} & \textbf{Status} \\ \midrule
Do you require MFA to access email? & Yes & \textcolor{green}{\yes} \\
\textbf{Do you require MFA to log into computers?} & \textbf{No} & \textcolor{red}{\no} \\
Do you require MFA to access sensitive data systems? & Yes & \textcolor{green}{\yes} \\
\textbf{Does your organization have an employee acceptable use policy?} & \textbf{No} & \textcolor{red}{\no} \\
\textbf{Does your organization do security awareness training for new employees?} & \textbf{No} & \textcolor{red}{\no} \\
Does your organization do security awareness training for all employees at least once per year? & Yes & \textcolor{green}{\yes} \\ \bottomrule
\end{tabular}
\caption{Security Controls Questionnaire Results}
\end{table}

\subsection*{Analysis of Gaps}
\begin{itemize}
    \item \textbf{MFA on Computers:} The lack of MFA on endpoints is a critical vulnerability. If an employee's password is stolen (e.g., via phishing), an attacker could gain direct access to their computer and the corporate network.
    \item \textbf{Acceptable Use Policy (AUP):} An AUP is a foundational policy that sets clear expectations for employees on how to use company assets securely. Its absence can lead to unintentional misuse and security incidents.
    \item \textbf{New Hire Training:} New employees are often prime targets for social engineering attacks. Failing to provide immediate security training during onboarding exposes the organization to unnecessary risk.
\end{itemize}

% ==============================================================================
% 4. Technical Scan Results
% ==============================================================================
\section{Technical Scan Results}

An external network scan was performed using Nmap to identify open ports and services visible from the public internet.

\begin{table}[h!]
\centering
\begin{tabular}{@{}ll@{}}
\toprule
\textbf{Scan Parameter} & \textbf{Value} \\ \midrule
Target IP Address & \code{[Target IP]} \\
Host Status & Up \\ \bottomrule
\end{tabular}
\caption{Scan Metadata}
\end{table}

\subsection*{Port Scan Findings}
The scan revealed a strong external security posture. \textbf{No open ports were detected.} This significantly reduces the external attack surface. Notably, port 80 (HTTP), which was listed as a pre-existing risk, was found to be closed.

\begin{table}[h!]
\centering
\begin{tabular}{@{}lll@{}}
\toprule
\textbf{Port} & \textbf{Protocol} & \textbf{State} \\ \midrule
80 & TCP & Closed \\ \bottomrule
\end{tabular}
\caption{Key Port Scan Results}
\end{table}

\subsection*{Validation of Prior Risk}
The pre-existing risk list (Input 3) mentioned an "Unencrypted Web Server" on port 80. The current scan results \textbf{do not validate} this risk. The finding that port 80 is closed indicates this vulnerability has either been successfully remediated or the prior information was inaccurate.

% ==============================================================================
% 5. Correlated Risk Assessment
% ==============================================================================
\section{Correlated Risk Assessment}

This section synthesizes findings from the questionnaire, technical scan, and pre-existing risk data into a prioritized list of current risks.

\begin{table}[h!]
\centering
\begin{tabular}{@{}lp{0.5\textwidth}l@{}}
\toprule
\textbf{Risk Name} & \textbf{Description} & \textbf{Severity} \\ \midrule
\textbf{Lack of Endpoint MFA} & No MFA on computer logins allows for endpoint compromise via stolen credentials. This could lead to data theft, lateral movement, or ransomware deployment. & \textbf{High} \\
\textbf{Missing Acceptable Use Policy} & Without a formal AUP, employees may unknowingly engage in risky behavior, such as using unauthorized software or mishandling sensitive data. & Medium \\
\textbf{Inadequate New-Hire Onboarding} & New employees are not trained on security policies and threats upon hiring, making them highly susceptible to phishing and social engineering attacks. & Medium \\
\textit{Unencrypted Web Server (Mitigated)} & \textit{A previously identified risk of an open port 80. The current scan confirms this port is closed, indicating the risk is mitigated.} & \textit{Informational} \\ \bottomrule
\end{tabular}
\caption{Summary of Identified Risks}
\end{table}

% ==============================================================================
% 6. Recommendations
% ==============================================================================
\section{Recommendations}

The following actionable recommendations are provided to address the identified risks and improve the overall security posture.

\subsection*{High Priority: Implement Endpoint MFA}
\begin{itemize}
    \item \textbf{Action:} Deploy Multi-Factor Authentication for all employee computer and laptop logins. This control acts as a critical barrier against credential theft-based attacks.
    \item \textbf{Implementation:} Prioritize deployment for privileged users (administrators) and executives. Solutions include Windows Hello for Business, Duo Security, or hardware tokens like YubiKeys.
\end{itemize}

\subsection*{Medium Priority: Develop and Implement an AUP}
\begin{itemize}
    \item \textbf{Action:} Create a formal Employee Acceptable Use Policy (AUP) that clearly defines the rules and expectations for using company technology and data.
    \item \textbf{Implementation:} The policy should be reviewed by HR and legal, communicated to all employees, and require a signed acknowledgment.
    \item \textbf{Resource:} A great starting point is the SANS Institute policy template library: \url{https://www.sans.org/information-security-policy/}
\end{itemize}

\subsection*{Medium Priority: Integrate Security into Onboarding}
\begin{itemize}
    \item \textbf{Action:} Make security awareness training a mandatory component of the new employee onboarding process.
    \item \textbf{Implementation:} This training should occur within the first week of employment and cover key topics such as phishing, password security, data handling, and the new AUP.
\end{itemize}

% ==============================================================================
% 7. Conclusion
% ==============================================================================
\section{Conclusion}

\placeholder{Organization Name} has established a commendable security perimeter, as evidenced by the clean external network scan. However, significant internal and procedural weaknesses currently expose the organization to substantial risk. The lack of endpoint MFA, a formal AUP, and security training for new hires are critical gaps that undermine other security investments.

By implementing the high-priority recommendations outlined in this report, the organization can significantly strengthen its defenses against prevalent cyber threats, reduce its attack surface, and foster a more security-conscious culture.

\end{document}
```