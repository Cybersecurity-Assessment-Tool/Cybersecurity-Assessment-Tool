```latex
\documentclass[12pt]{article}

% Preamble: Required Packages
\usepackage[margin=1in]{geometry}
\usepackage{pifont} % For checkmarks and crosses (\ding)
\usepackage{booktabs} % For professional-looking tables
\usepackage{hyperref} % For clickable links and document metadata
\usepackage{url} % For formatting URLs
\usepackage{seqsplit} % To break long strings like IPs or hashes
\usepackage[T1]{fontenc}

% Document Metadata and Hyperlink Setup
\hypersetup{
    colorlinks=true,
    linkcolor=black,
    filecolor=magenta,
    urlcolor=blue,
    pdftitle={Cybersecurity Posture Report},
    pdfauthor={Cybersecurity Analysis Engine},
    pdfsubject={Security Assessment},
    pdfkeywords={Cybersecurity, Risk, Assessment, Nmap},
    pdftitle={Cybersecurity Posture Report},
    pdfpagemode=FullScreen,
}

% --- Document Start ---
\begin{document}

\title{Cybersecurity Posture Report}
\author{Automated Security Analysis Engine}
\date{\today}
\maketitle

\begin{abstract}
This report provides a comprehensive analysis of the cybersecurity posture for \textbf{[Organization Name]}. The assessment is based on a synthesis of network scan data, a security controls questionnaire, and a review of pre-existing documented risks. The analysis identifies critical security gaps, technical vulnerabilities, and provides prioritized, actionable recommendations to mitigate identified risks and enhance the organization's overall security resilience.
\end{abstract}

\tableofcontents
\newpage

% ===================================================================
\section{Executive Summary}
% ===================================================================

This assessment reveals a mixed security posture for \textbf{[Organization Name]}. The organization has successfully implemented several important foundational controls, including mandatory Multi-Factor Authentication (MFA) for email access, an employee acceptable use policy, and annual security awareness training. These measures significantly reduce the risk of account compromise and establish a baseline for secure employee conduct.

However, several critical and high-risk gaps were identified that require immediate attention. The most significant concerns are the absence of MFA for computer logins and access to sensitive data systems. This exposes the organization to substantial risk from credential theft, potentially leading to unauthorized access and data breaches. Furthermore, the lack of security awareness training for new employees during their onboarding process leaves a window of vulnerability where new staff may be more susceptible to social engineering attacks.

A technical network scan of the external IP address \seqsplit{\texttt{[Client IP]}} revealed a minimal attack surface, with port 80 (HTTP) found to be closed. This finding contradicts a pre-existing documented risk concerning an "Unencrypted Web Server." This discrepancy suggests the risk may have been remediated.

Our primary recommendations are to prioritize the deployment of MFA across all workstations and sensitive systems, integrate mandatory security training into the employee onboarding process, and formally verify and update the status of the "Unencrypted Web Server" risk in the organization's risk register.

% ===================================================================
\section{Organizational Information}
% ===================================================================

The following information was used as the basis for this assessment. Due to the anonymized nature of the input data, placeholders have been used where necessary.

\begin{itemize}
    \item \textbf{Organization Name:} \textbf{[Organization Name]}
    \item \textbf{Primary Domain:} \seqsplit{\texttt{[Domain]}}
    \item \textbf{External IP Scanned:} \seqsplit{\texttt{[Client IP]}}
\end{itemize}

% ===================================================================
\section{Security Control Review}
% ===================================================================

The following table summarizes the organization's responses to the security controls questionnaire. A green checkmark (\ding{51}) indicates a positive control is in place, while a red cross (\ding{55}) indicates a potential security gap.

\begin{table}[h!]
\centering
\caption{Security Controls Questionnaire Results}
\begin{tabular}{@{}lc@{}}
\toprule
\textbf{Control Question} & \textbf{Status} \\
\midrule
Do you require MFA to access email? & \ding{51} \\
Do you require MFA to log into computers? & \ding{55} \\
Do you require MFA to access sensitive data systems? & \ding{55} \\
Does your organization have an employee acceptable use policy? & \ding{51} \\
Does your organization do security awareness training for new employees? & \ding{55} \\
Does your organization do security awareness training for all employees at least once per year? & \ding{51} \\
\bottomrule
\end{tabular}
\end{table}

\paragraph{Analysis:} The questionnaire reveals critical gaps in access control. The lack of MFA on computer and sensitive system logins significantly increases the risk of unauthorized access via compromised credentials. The absence of security training during onboarding is a high-risk gap, as new employees are often prime targets for phishing and other social engineering attacks.

% ===================================================================
\section{Technical Scan Results}
% ===================================================================

An external network scan was performed on the target IP address. The results indicate a minimal external footprint at the time of the scan.

\begin{itemize}
    \item \textbf{Target IP:} \seqsplit{\texttt{[Target IP]}}
    \item \textbf{Scan Status:} Host was detected as "up".
\end{itemize}

\begin{table}[h!]
\centering
\caption{Port Scan Details}
\begin{tabular}{@{}llll@{}}
\toprule
\textbf{Port} & \textbf{State} & \textbf{Service} & \textbf{Version} \\
\midrule
80/tcp & closed & http & N/A \\
\bottomrule
\end{tabular}
\end{table}

\paragraph{Analysis:} The scan found no open ports on the target system. Specifically, port 80 (HTTP) was confirmed to be closed. This is a positive security finding, as it reduces the external attack surface. However, this result directly contradicts the pre-existing risk documented in Input 3, which states that port 80 is open. This discrepancy requires further investigation to determine if the risk has been resolved or if the initial report was inaccurate.

% ===================================================================
\section{Consolidated Risk Assessment}
% ===================================================================

The following table correlates findings from the security questionnaire, technical scans, and pre-existing risk documentation into a consolidated list of current risks.

\begin{table}[h!]
\centering
\caption{Summary of Identified Risks}
\begin{tabular}{@{}p{0.1\linewidth}p{0.3\linewidth}p{0.15\linewidth}p{0.35\linewidth}@{}}
\toprule
\textbf{ID} & \textbf{Risk Name} & \textbf{Severity} & \textbf{Description} \\
\midrule
\textbf{RISK-01} & No MFA on Sensitive Systems & \textbf{Critical} & Lack of MFA on systems holding sensitive data creates a high risk of a data breach from a single compromised password. \\
\addlinespace
\textbf{RISK-02} & No MFA on Workstations & High & The absence of MFA for computer logins allows an attacker with valid credentials to gain initial access to the internal network. \\
\addlinespace
\textbf{RISK-03} & No Onboarding Security Training & High & New employees are not trained on security policies and threats, making them more susceptible to social engineering attacks. \\
\addlinespace
\textbf{RISK-04} & Unencrypted Web Server (Verification Required) & Medium & A pre-existing risk states Port 80 is open. Our scan found it closed. The status of this risk is unconfirmed and must be verified. \\
\bottomrule
\end{tabular}
\end{table}

% ===================================================================
\section{Recommendations}
% ===================================================================

Based on the analysis, we provide the following prioritized recommendations to mitigate the identified risks.

\begin{enumerate}
    \item \textbf{Implement Comprehensive MFA (RISK-01, RISK-02):}
    \begin{itemize}
        \item \textbf{Immediate Priority:} Deploy a robust MFA solution for all access to systems identified as containing sensitive data. This is the most critical step to prevent a major data breach.
        \item \textbf{High Priority:} Enforce MFA for all employee logins to company workstations and laptops. This strengthens the initial line of defense against network intrusion.
    \end{itemize}

    \item \textbf{Establish Onboarding Security Training (RISK-03):}
    \begin{itemize}
        \item \textbf{Action:} Develop and integrate a mandatory security awareness training module into the new employee onboarding process. This training should cover, at a minimum: acceptable use, phishing identification, password hygiene, and incident reporting procedures.
    \end{itemize}

    \item \textbf{Verify and Update Risk Register (RISK-04):}
    \begin{itemize}
        \item \textbf{Action:} Conduct an internal review to confirm the status of the web server on Port 80. If the service has been decommissioned, formally close the risk in the risk register. If the service was moved or the initial finding was a false positive, update the documentation to reflect the accurate state.
    \end{itemize}
\end{enumerate}

% --- Document End ---
\end{document}
```