```latex
\documentclass[12pt]{article}

% Preamble: Required Packages
\usepackage[margin=1in]{geometry}
\usepackage{pifont} % For checkmarks and crosses
\usepackage{booktabs} % For professional tables
\usepackage{hyperref} % For hyperlinks
\usepackage{url} % For URL formatting
\usepackage{seqsplit} % For splitting long strings
\usepackage{xcolor} % For colors in tables

% Document Information
\title{Cybersecurity Posture Assessment Report}
\author{Cybersecurity Analysis Division}
\date{\today}

% Hyperref Setup
\hypersetup{
    colorlinks=true,
    linkcolor=blue,
    filecolor=magenta,      
    urlcolor=cyan,
    pdftitle={Cybersecurity Posture Assessment Report},
    pdfpagemode=FullScreen,
}

\begin{document}

\maketitle
\thispagestyle{empty}
\newpage
\tableofcontents
\newpage

\section{Executive Summary}

This report details the findings of a cybersecurity posture assessment conducted for \textbf{[Organization Name]}. The assessment combined a review of organizational security controls via a questionnaire, an external network scan of a designated target, and a review of previously identified risks.

The analysis revealed several areas of concern that require immediate attention. While the organization has implemented some positive security controls, such as multi-factor authentication (MFA) for computer and sensitive system access, critical gaps exist that expose the organization to significant risk.

Key findings include:
\begin{itemize}
    \item \textbf{Critical Risk:} The absence of MFA for email access. This is a primary vector for account compromise, business email compromise (BEC), and subsequent data breaches.
    \item \textbf{High Risk:} The lack of a formal Employee Acceptable Use Policy (AUP). This governance gap creates ambiguity regarding security responsibilities and acceptable behavior, hindering enforcement and a strong security culture.
    \item \textbf{Technical Finding:} An exposed Secure Shell (SSH) service (Port 22) was identified on the target system \texttt{[Target IP]}. Publicly accessible administrative services are common targets for automated brute-force attacks.
\end{itemize}

This report provides a detailed breakdown of these findings and offers actionable recommendations to mitigate the identified risks and improve the overall security posture of \textbf{[Organization Name]}.

\section{Organizational Information}

The following information was used as the basis for this assessment. Due to the anonymized nature of the provided data, placeholders have been used where necessary.

\begin{itemize}
    \item \textbf{Organization Name:} \textbf{[Organization Name]}
    \item \textbf{Primary Email Domain:} \texttt{[Domain]}
    \item \textbf{Client External IP:} \texttt{[Client IP]}
\end{itemize}

\section{Security Control Review}

The following table summarizes the organization's responses to the security controls questionnaire. "No" answers indicate a gap in security controls and are highlighted for review.

\begin{table}[h!]
\centering
\caption{Security Controls Questionnaire Analysis}
\begin{tabular}{p{0.6\linewidth} c p{0.2\linewidth}}
\toprule
\textbf{Control Question} & \textbf{Response} & \textbf{Assessment} \\
\midrule
Do you require MFA to access email? & \textcolor{red}{\ding{55}} & \textbf{Critical Gap} \\
Do you require MFA to log into computers? & \textcolor{green}{\ding{51}} & Implemented \\
Do you require MFA to access sensitive data systems? & \textcolor{green}{\ding{51}} & Implemented \\
Does your organization have an employee acceptable use policy? & \textcolor{red}{\ding{55}} & \textbf{High Risk} \\
Does your organization do security awareness training for new employees? & \textcolor{green}{\ding{51}} & Implemented \\
Does your organization do security awareness training for all employees at least once per year? & \textcolor{green}{\ding{51}} & Implemented \\
\bottomrule
\end{tabular}
\end{table}

\subsection{Analysis of Gaps}
\begin{itemize}
    \item \textbf{MFA for Email:} Email is a gateway to an organization's data and a primary target for attackers. Without MFA, a single compromised password can lead to a full account takeover.
    \item \textbf{Acceptable Use Policy (AUP):} An AUP is a foundational governance document. Its absence means there are no formally documented rules for employees regarding the use of company assets, data handling, and security responsibilities.
\end{itemize}

\section{Technical Scan Results}

An external network scan was performed to identify open ports and services on the organization's perimeter.

\begin{itemize}
    \item \textbf{Target IP Address:} \texttt{[Target IP]}
    \item \textbf{Scan Date:} \today
\end{itemize}

\begin{table}[h!]
\centering
\caption{Open Port Scan Findings}
\begin{tabular}{l l l p{0.5\linewidth}}
\toprule
\textbf{Port} & \textbf{State} & \textbf{Service (Inferred)} & \textbf{Notes} \\
\midrule
22/tcp & Open & SSH (Secure Shell) & The scan did not retrieve service version information. Exposing SSH to the public internet increases the risk of brute-force attacks and exploitation if vulnerabilities exist. \\
\bottomrule
\end{tabular}
\end{table}

\section{Risk Assessment Summary}

The following table correlates the findings from the security control review and the technical scan. No pre-existing vulnerabilities were reported.

\begin{table}[h!]
\centering
\caption{Identified Risks}
\begin{tabular}{p{0.1\linewidth} p{0.25\linewidth} p{0.45\linewidth} l}
\toprule
\textbf{ID} & \textbf{Risk Name} & \textbf{Description} & \textbf{Severity} \\
\midrule
RISK-001 & Lack of MFA on Email & The absence of MFA on email accounts allows for account takeover with only a compromised password. & \textbf{Critical} \\
\addlinespace
RISK-002 & Missing Acceptable Use Policy & Lack of a formal AUP creates inconsistent security practices and a weak foundation for security governance. & \textbf{High} \\
\addlinespace
RISK-003 & Exposed SSH Service & The SSH management port is open to the public internet, making it a target for brute-force login attempts and potential exploitation. & \textbf{Medium} \\
\bottomrule
\end{tabular}
\end{table}

\section{Recommendations}

To address the identified risks and strengthen the security posture of \textbf{[Organization Name]}, we provide the following prioritized recommendations.

\subsection{Immediate Actions (Critical \& High Risks)}

\begin{enumerate}
    \item \textbf{RISK-001: Implement MFA for Email}
    \begin{itemize}
        \item Immediately enable and enforce MFA for all user mailboxes.
        \item Prioritize deployment for privileged accounts (administrators, executives) and then roll out to all users within 30 days.
        \item Provide users with clear instructions on how to enroll in MFA.
    \end{itemize}

    \item \textbf{RISK-002: Develop and Implement an AUP}
    \begin{itemize}
        \item Draft a comprehensive Acceptable Use Policy that covers topics such as data handling, internet usage, password requirements, and incident reporting.
        \item Have the policy reviewed by legal and HR departments.
        \item Require all employees to read and formally acknowledge the policy upon implementation and annually thereafter.
    \end{itemize}
\end{enumerate}

\subsection{Secondary Actions (Medium Risks)}

\begin{enumerate}
    \setcounter{enumi}{2} % Continue numbering from the previous list
    \item \textbf{RISK-003: Secure the Exposed SSH Service}
    \begin{itemize}
        \item \textbf{Review Business Need:} Determine if public access to SSH on \texttt{[Target IP]} is necessary. If not, block it at the firewall.
        \item \textbf{Implement Controls:} If access is required, restrict it to known, trusted IP addresses (IP whitelisting).
        \item \textbf{Strengthen Authentication:} Disable password-based authentication and enforce the use of strong cryptographic keys (e.g., ED25519) for all SSH access.
        \item \textbf{Monitor:} Implement a tool like Fail2Ban to automatically block IPs that exhibit brute-force behavior.
    \end{itemize}
\end{enumerate}

\end{document}
```