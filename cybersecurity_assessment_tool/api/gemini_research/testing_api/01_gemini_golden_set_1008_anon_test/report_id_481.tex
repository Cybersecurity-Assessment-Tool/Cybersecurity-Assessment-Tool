```latex
\documentclass[12pt]{article}

% Preamble: Required Packages
\usepackage[margin=1in]{geometry}
\usepackage{pifont} % For checkmarks and crosses
\usepackage{booktabs} % For professional tables
\usepackage{hyperref} % For hyperlinks
\usepackage{url} % For URL formatting
\usepackage{seqsplit} % For splitting long strings
\usepackage{graphicx} % For logo (placeholder)
\usepackage{fancyhdr} % For headers and footers
\usepackage{xcolor} % For colors

% --- Document Metadata ---
\title{Cybersecurity Posture Assessment Report}
\author{Cybersecurity Analysis Division}
\date{November 22, 2025}

% --- Hyperref Setup ---
\hypersetup{
    colorlinks=true,
    linkcolor=blue,
    filecolor=magenta,      
    urlcolor=cyan,
    pdftitle={Cybersecurity Posture Assessment Report},
    pdfpagemode=FullScreen,
}

% --- Header and Footer Setup ---
\pagestyle{fancy}
\fancyhf{} % Clear all header and footer fields
\fancyhead[L]{\textbf{Cybersecurity Assessment Report}}
\fancyhead[R]{\textbf{[Organization Name]}}
\fancyfoot[C]{\thepage}
\renewcommand{\headrulewidth}{0.4pt}
\renewcommand{\footrulewidth}{0.4pt}

\begin{document}

\begin{titlepage}
    \centering
    \vspace*{1cm}
    
    \Huge\textbf{Cybersecurity Posture Assessment Report}
    
    \vspace{1.5cm}
    
    \Large\textbf{Prepared for:} \\
    \vspace{0.5cm}
    \textbf{[Organization Name]}
    
    \vspace{2cm}
    
    \textbf{Date of Assessment:} \\
    November 22, 2025
    
    \vfill
    
    \large
    \textbf{CONFIDENTIAL} \\
    \vspace{0.5cm}
    \small This document contains sensitive information. Distribution is restricted to authorized personnel only.
    
\end{titlepage}

\tableofcontents
\newpage

% --- Section 1: Executive Overview ---
\section{Executive Overview}
This report provides a comprehensive analysis of the cybersecurity posture for \textbf{[Organization Name]}, based on data collected on November 22, 2025. The assessment combined a review of organizational security controls, a technical network scan of external infrastructure, and an evaluation of known risks.

The analysis reveals a mixed security posture. The organization demonstrates strong identity and access management controls, with Multi-Factor Authentication (MFA) consistently enforced across email, computer logins, and sensitive data systems. This is a commendable practice that significantly reduces the risk of unauthorized access.

However, critical deficiencies were identified in foundational administrative controls. The absence of an employee Acceptable Use Policy (AUP) and a formal security awareness training program for both new and existing employees creates significant human-factor risks. These gaps leave the organization vulnerable to social engineering, insider threats, and inconsistent security practices.

Furthermore, the external network scan identified a web server running an outdated version of Nginx (\texttt{1.18.0}). This software is no longer supported and is susceptible to numerous publicly known vulnerabilities, posing a direct threat to the confidentiality, integrity, and availability of the services it hosts.

Immediate remediation efforts should focus on addressing the policy and training gaps and upgrading the vulnerable web server software to mitigate these high-impact risks.

% --- Section 2: Organizational Information ---
\section{Organizational Information}
The following details were used as the basis for this assessment. Due to the anonymized nature of the provided data, placeholders have been used where necessary.

\begin{table}[h!]
\centering
\begin{tabular}{@{}ll@{}}
\toprule
\textbf{Attribute} & \textbf{Value} \\ \midrule
Organization Name & \textbf{[Organization Name]} \\
Primary Domain & \texttt{[Domain]} \\
External IP Address (Source) & \texttt{[Client IP]} \\
Target IP Address (Scanned) & \texttt{[Target IP]} \\
Assessment Date & November 22, 2025 \\ \bottomrule
\end{tabular}
\caption{Subject Organization Details}
\label{tab:org_info}
\end{table}

% --- Section 3: Security Control Review ---
\section{Security Control Review}
A review of administrative and organizational security controls was conducted based on a standardized questionnaire. The results highlight areas of both strength and weakness in the organization's security policies and procedures.

\begin{table}[h!]
\centering
\begin{tabular}{@{}p{0.75\linewidth}c@{}}
\toprule
\textbf{Control Question} & \textbf{Response} \\ \midrule
Do you require MFA to access email? & \ding{51} \\
Do you require MFA to log into computers? & \ding{51} \\
Do you require MFA to access sensitive data systems? & \ding{51} \\
Does your organization have an employee acceptable use policy? & \textcolor{red}{\ding{55}} \\
Does your organization do security awareness training for new employees? & \textcolor{red}{\ding{55}} \\
Does your organization do security awareness training for all employees at least once per year? & \textcolor{red}{\ding{55}} \\ \bottomrule
\end{tabular}
\caption{Security Controls Questionnaire Results (\ding{51}=Yes, \ding{55}=No)}
\label{tab:controls}
\end{table}

\paragraph{Analysis:} The organization has successfully implemented MFA across key access points, which is a critical defense against credential theft. However, the "No" responses indicate major gaps in the human element of security. The lack of an Acceptable Use Policy and a structured security awareness training program are categorized as \textbf{Critical Risks}. Without these controls, employees are not formally equipped with the knowledge or guidelines to protect company assets, making them prime targets for phishing and other social engineering attacks.

% --- Section 4: Technical Scan Results ---
\section{Technical Scan Results}
An external network scan was performed against the target IP address \texttt{[Target IP]} to identify open ports and exposed services.

\begin{table}[h!]
\centering
\begin{tabular}{@{}lllll@{}}
\toprule
\textbf{Port} & \textbf{State} & \textbf{Service} & \textbf{Product} & \textbf{Version} \\ \midrule
443/tcp & Open & https & nginx & \textcolor{red}{1.18.0} \\ \bottomrule
\end{tabular}
\caption{Open Ports and Services on \texttt{[Target IP]}}
\label{tab:scan_results}
\end{table}

\paragraph{Analysis:} The scan identified a single open port, 443/tcp, which is standard for HTTPS traffic. The service is provided by an Nginx web server. The detected version, \texttt{1.18.0}, was released in April 2020 and is now considered outdated and unsupported. This version is known to be vulnerable to multiple Common Vulnerabilities and Exposures (CVEs). Exposing an outdated web server to the internet presents a \textbf{High Risk}, as it can be exploited by attackers to compromise the server, steal data, or use it as a pivot point to attack the internal network.

% --- Section 5: Risk Assessment Summary ---
\section{Risk Assessment Summary}
The following table synthesizes the findings from the security control review and technical scan into a prioritized list of risks.

\begin{table}[h!]
\centering
\begin{tabular}{@{}p{0.1\linewidth}p{0.25\linewidth}p{0.15\linewidth}p{0.4\linewidth}@{}}
\toprule
\textbf{Risk ID} & \textbf{Risk Name} & \textbf{Severity} & \textbf{Description} \\ \midrule
RISK-001 & Lack of Security Awareness Training & \textbf{Critical} & The absence of a formal training program for new and existing employees exposes the organization to a high likelihood of human error, phishing, and social engineering attacks. \\
\addlinespace
RISK-002 & Outdated Web Server Software & \textbf{High} & The public-facing Nginx server is running version 1.18.0, which is outdated and has multiple known vulnerabilities that could lead to a system compromise. \\
\addlinespace
RISK-003 & Missing Acceptable Use Policy (AUP) & \textbf{High} & Without a formal AUP, there are no clear rules for employees regarding the use of company systems and data, increasing the risk of misuse and insider threats. \\ \bottomrule
\end{tabular}
\caption{Summary of Identified Risks}
\label{tab:risk_summary}
\end{table}

% --- Section 6: Recommendations ---
\section{Recommendations}
Based on the identified risks, the following prioritized actions are recommended to improve the cybersecurity posture of \textbf{[Organization Name]}.

\subsection{RISK-001: Lack of Security Awareness Training (Critical)}
\begin{itemize}
    \item \textbf{Immediate Action:} Develop and implement a mandatory security awareness training program for all employees. This program should be part of the onboarding process for all new hires.
    \item \textbf{Ongoing Action:} Conduct annual refresher training for all staff to keep them updated on the latest threats and best practices.
    \item \textbf{Validation:} Track completion rates for all training modules and conduct periodic phishing simulations to measure the effectiveness of the program.
\end{itemize}

\subsection{RISK-002: Outdated Web Server Software (High)}
\begin{itemize}
    \item \textbf{Immediate Action:} Plan and execute an upgrade of the Nginx server on \texttt{[Target IP]} from version \texttt{1.18.0} to a current, supported stable version. The upgrade path should be tested in a non-production environment first to ensure compatibility.
    \item \textbf{Ongoing Action:} Implement a formal patch and vulnerability management process. This should include regular scanning of all external and internal systems to identify and remediate outdated software in a timely manner.
\end{itemize}

\subsection{RISK-003: Missing Acceptable Use Policy (AUP) (High)}
\begin{itemize}
    \item \textbf{Immediate Action:} Draft a comprehensive Acceptable Use Policy that clearly defines the rules for using company networks, devices, software, and data.
    \item \textbf{Ongoing Action:} Require all employees, contractors, and relevant third parties to read and formally acknowledge the AUP upon hire and annually thereafter. The policy should be reviewed and updated at least once per year.
\end{itemize}

\end{document}
```