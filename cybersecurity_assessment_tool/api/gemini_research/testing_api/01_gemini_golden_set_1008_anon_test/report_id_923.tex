```latex
\documentclass[12pt, a4paper]{article}

% Required Packages
\usepackage[margin=1in]{geometry}
\usepackage{pifont} % For \ding
\usepackage{booktabs} % For professional tables
\usepackage{hyperref} % For clickable links and ToC
\usepackage{url} % For URL formatting
\usepackage{seqsplit} % For splitting long strings to prevent overflow

% Document Metadata and Hyperref Setup
\hypersetup{
    colorlinks=true,
    linkcolor=black,
    urlcolor=blue,
    pdftitle={Cybersecurity Posture Assessment Report},
    pdfauthor={Cybersecurity Analyst},
    pdfsubject={Security Analysis},
    pdfkeywords={Cybersecurity, Risk Assessment, Network Scan}
}

% Title
\title{Cybersecurity Posture Assessment Report \\ \large for \textbf{[Organization Name]}}
\author{Cybersecurity Analyst}
\date{\today}

\begin{document}

\maketitle
\thispagestyle{empty}
\newpage

\tableofcontents
\newpage

% --- Section 1: Executive Overview ---
\section{Executive Overview}
This report details the findings of a cybersecurity posture assessment conducted for \textbf{[Organization Name]}. The assessment combines an analysis of organizational security controls, a technical network scan of external infrastructure, and a review of previously identified risks.

The analysis revealed several critical and high-risk security gaps that require immediate attention. Most notably, Multi-Factor Authentication (MFA) is not enforced for accessing email or other sensitive data systems. This exposes the organization to significant risk from account compromise and data breaches. Furthermore, the absence of a formal Acceptable Use Policy (AUP) creates ambiguity regarding employee responsibilities for security.

Technical scanning identified an exposed SSH (Secure Shell) service on the external network perimeter. While a common administrative tool, its exposure must be strictly controlled. This finding, combined with a pre-existing critical risk documented as "Localhost Exposed", indicates potential weaknesses in the network architecture and security configuration that must be remediated urgently.

This report provides a detailed breakdown of these findings and offers prioritized, actionable recommendations to mitigate the identified risks and improve the overall security posture of \textbf{[Organization Name]}.

% --- Section 2: Organizational Information ---
\section{Organizational Information}
The following details were used as the basis for this assessment. Due to the anonymized nature of the provided data, placeholders have been used where necessary.

\begin{itemize}
    \item \textbf{Organization Name:} \textbf{[Organization Name]}
    \item \textbf{Primary Email Domain:} \texttt{[Domain]}
    \item \textbf{External IP Address Scanned:} \texttt{[Client IP]}
\end{itemize}

% --- Section 3: Security Control Review ---
\section{Security Control Review}
A review of organizational security controls was conducted based on a standard security questionnaire. The responses highlight critical gaps in identity and access management and internal policy.

\begin{table}[h!]
\centering
\caption{Organizational Security Control Questionnaire}
\begin{tabular}{p{0.75\linewidth} c}
\toprule
\textbf{Control Question} & \textbf{Response} \\
\midrule
Do you require MFA to access email? & \ding{55} \\
Do you require MFA to log into computers? & \ding{51} \\
Do you require MFA to access sensitive data systems? & \ding{55} \\
Does your organization have an employee acceptable use policy? & \ding{55} \\
Does your organization do security awareness training for new employees? & \ding{51} \\
Does your organization do security awareness training for all employees at least once per year? & \ding{51} \\
\bottomrule
\end{tabular}
\end{table}

\subsection*{Analysis of Controls}
The questionnaire responses indicate a robust security awareness training program. However, there are three significant deficiencies:
\begin{itemize}
    \item \textbf{No MFA for Email:} Email is a primary target for attackers. Lack of MFA makes accounts highly vulnerable to phishing and credential stuffing attacks, which can lead to business email compromise and further network intrusion.
    \item \textbf{No MFA for Sensitive Systems:} Failure to protect sensitive data systems with MFA is a critical oversight. This directly exposes the organization's most valuable data to unauthorized access.
    \item \textbf{No Acceptable Use Policy (AUP):} An AUP is a foundational policy that defines how employees can use company resources. Its absence can lead to inconsistent security practices and a lack of accountability.
\end{itemize}

% --- Section 4: Technical Scan Results ---
\section{Technical Scan Results}
An external network scan was performed on the target IP address to identify exposed services.

\begin{itemize}
    \item \textbf{Target IP Address:} \texttt{[Target IP]}
    \item \textbf{Scan Status:} Host is UP
\end{itemize}

\begin{table}[h!]
\centering
\caption{Open Ports Detected on \texttt{[Target IP]}}
\begin{tabular}{l l l l}
\toprule
\textbf{Port} & \textbf{State} & \textbf{Service} & \textbf{Product / Version} \\
\midrule
22/tcp & open & ssh & N/A \\
\bottomrule
\end{tabular}
\end{table}

\subsection*{Analysis of Technical Findings}
The scan identified that port 22, the standard port for the Secure Shell (SSH) protocol, is open to the internet. SSH is a powerful administrative tool, but its public exposure presents a significant risk. If not properly configured, it can be a target for:
\begin{itemize}
    \item \textbf{Brute-force attacks:} Automated tools can be used to guess usernames and passwords.
    \item \textbf{Exploitation of vulnerabilities:} If the SSH server software is outdated, it may contain known vulnerabilities that could be exploited for remote code execution.
\end{itemize}
Given the identified gaps in MFA policy, it is crucial to ensure this administrative interface is protected by compensating controls, such as IP whitelisting and key-based authentication.

% --- Section 5: Consolidated Risk Assessment ---
\section{Consolidated Risk Assessment}
The following table synthesizes findings from the security control review, technical scan, and pre-existing risk data into a consolidated list of identified risks.

\begin{table}[h!]
\centering
\caption{Summary of Identified Risks}
\begin{tabular}{p{0.3\linewidth} p{0.5\linewidth} l}
\toprule
\textbf{Risk Name} & \textbf{Description} & \textbf{Severity} \\
\midrule
\textbf{Localhost Exposed} & Pre-existing documented risk with a CVSS score of 10.0. The overview states this is a critical issue. & \textbf{Critical} \\
\textbf{No MFA on Email} & Lack of MFA on email accounts exposes the organization to account takeover and phishing. & \textbf{Critical} \\
\textbf{No MFA on Sensitive Systems} & Lack of MFA on systems containing sensitive data creates a direct path for data exfiltration. & \textbf{Critical} \\
\textbf{Exposed SSH Management Port} & The SSH service is exposed to the internet, increasing the risk of brute-force attacks and unauthorized access. & \textbf{High} \\
\textbf{No Acceptable Use Policy} & The absence of a formal AUP leads to inconsistent security practices and lack of employee accountability. & \textbf{High} \\
\bottomrule
\end{tabular}
\end{table}

% --- Section 6: Recommendations ---
\section{Recommendations}
Based on the consolidated risk assessment, the following actions are recommended to mitigate the identified vulnerabilities. Recommendations are prioritized by severity.

\subsection*{Immediate Priority (Critical Risks)}
\begin{enumerate}
    \item \textbf{Remediate "Localhost Exposed" Vulnerability:} An immediate investigation must be launched to understand and remediate the pre-existing critical risk identified as "Localhost Exposed". This represents the highest priority.
    \item \textbf{Enforce MFA on All Email Accounts:} Immediately implement and enforce MFA for all user access to the email system. This is the single most effective control to prevent business email compromise.
    \item \textbf{Enforce MFA on Sensitive Systems:} Deploy MFA on all systems identified as containing sensitive or critical data to protect against unauthorized access and data breaches.
\end{enumerate}

\subsection*{High Priority}
\begin{enumerate}
    \setcounter{enumi}{3} % Continue numbering
    \item \textbf{Restrict SSH Access:} The SSH service on \texttt{[Target IP]} should be firewalled to allow access only from a limited set of trusted IP addresses (e.g., corporate VPN, administrator workstations). Public access should be denied.
    \item \textbf{Harden SSH Configuration:} Ensure the SSH server is configured according to security best practices. This includes disabling password-based authentication and enforcing the use of strong cryptographic keys.
    \item \textbf{Develop and Implement an Acceptable Use Policy (AUP):} Draft and ratify a formal AUP that clearly outlines the rules and responsibilities for all employees when using company IT assets. Communicate this policy to all staff.
\end{enumerate}

\end{document}
```