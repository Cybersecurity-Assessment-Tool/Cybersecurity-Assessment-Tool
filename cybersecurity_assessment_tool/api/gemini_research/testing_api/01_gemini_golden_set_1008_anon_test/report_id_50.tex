```latex
\documentclass[12pt]{article}

% -----------------------------------------------------------------------------
% PREAMBLE
% -----------------------------------------------------------------------------
\usepackage[margin=1in]{geometry}
\usepackage{pifont} % Required for check and cross marks
\usepackage{booktabs} % For professional-looking tables
\usepackage{hyperref} % For clickable links and references
\usepackage{url}      % For formatting URLs
\usepackage{seqsplit} % For breaking long strings in tt font
\usepackage{xcolor}   % For custom colors

% Hyperlink setup
\hypersetup{
    colorlinks=true,
    linkcolor=blue,
    filecolor=magenta,
    urlcolor=cyan,
    pdftitle={Cybersecurity Posture Assessment Report},
    pdfauthor={Cybersecurity Analysis Division},
}

% Custom commands for Yes/No indicators
\newcommand{\yes}{\textcolor{green}{\ding{51}}}
\newcommand{\no}{\textcolor{red}{\ding{55}}}

% -----------------------------------------------------------------------------
% DOCUMENT START
% -----------------------------------------------------------------------------
\begin{document}

\title{Cybersecurity Posture Assessment Report}
\author{Cybersecurity Analysis Division}
\date{\today}
\maketitle

\hrule\vspace{1em}

% -----------------------------------------------------------------------------
% 1. EXECUTIVE SUMMARY
% -----------------------------------------------------------------------------
\section*{1. Executive Summary}

This report provides a comprehensive assessment of the cybersecurity posture for \textbf{[Organization Name]}. The analysis is based on a synthesis of external network scans, a review of internal security controls via questionnaire, and an evaluation of pre-existing risks.

The assessment reveals a \textbf{critical risk posture}. Key findings include a public-facing FTP server with a known, severe remote code execution (RCE) vulnerability (CVE-2011-2523) and anonymous login enabled. Internally, there are significant gaps in access control, most notably the absence of Multi-Factor Authentication (MFA) for computer and sensitive data system access. These weaknesses, combined with an existing issue of outdated Windows 7 workstations, create multiple vectors for a potential compromise.

Immediate remediation of the external FTP vulnerability is paramount. Following this, a strategic implementation of MFA across critical assets is strongly recommended to mitigate the risk of credential-based attacks.

% -----------------------------------------------------------------------------
% 2. ORGANIZATIONAL INFORMATION
% -----------------------------------------------------------------------------
\section*{2. Organizational Information}

The following details were used as the basis for this assessment. Due to the anonymized nature of the provided data, placeholders have been used where necessary.

\begin{tabular}{@{}ll}
    \textbf{Organization Name:} & \textbf{[Organization Name]} \\
    \textbf{Primary Domain:} & \texttt{[Domain]} \\
    \textbf{Client IP Address:} & \texttt{[Client IP]} \\
\end{tabular}

% -----------------------------------------------------------------------------
% 3. SECURITY CONTROL REVIEW (QUESTIONNAIRE ANALYSIS)
% -----------------------------------------------------------------------------
\section*{3. Security Control Review}

An analysis of the organization's security questionnaire reveals several key strengths and critical gaps in administrative and technical controls. While security awareness training and MFA for email are commendable, the lack of MFA for workstation and sensitive system access represents a significant weakness.

\begin{table}[h!]
\centering
\caption{Security Controls Questionnaire Results}
\begin{tabular}{p{0.75\textwidth}c}
\toprule
\textbf{Control Question} & \textbf{Status} \\
\midrule
Do you require MFA to access email? & \yes \\
Do you require MFA to log into computers? & \no \\
Do you require MFA to access sensitive data systems? & \no \\
Does your organization have an employee acceptable use policy? & \yes \\
Does your organization do security awareness training for new employees? & \yes \\
Does your organization do security awareness training for all employees at least once per year? & \yes \\
\bottomrule
\end{tabular}
\end{table}

% -----------------------------------------------------------------------------
% 4. TECHNICAL SCAN RESULTS
% -----------------------------------------------------------------------------
\section*{4. Technical Scan Results}

An external network scan was performed against the target IP address \texttt{[Target IP]}. The scan identified one open port with a critically vulnerable service.

\subsection*{Open Ports and Services}
\begin{table}[h!]
\centering
\caption{Nmap Scan Findings}
\begin{tabular}{@{}lllll@{}}
\toprule
\textbf{Port} & \textbf{State} & \textbf{Service} & \textbf{Version} & \textbf{Key Findings} \\
\midrule
21/tcp & open & ftp & vsftpd 2.3.4 & \begin{tabular}[c]{@{}l@{}}1. Critical Vulnerability (CVE-2011-2523)\\ 2. Anonymous FTP Login Allowed\end{tabular} \\
\bottomrule
\end{tabular}
\end{table}

\subsection*{Vulnerability Analysis}
The FTP service is running \textbf{vsftpd version 2.3.4}. This specific version contains a well-documented backdoor vulnerability, cataloged as \href{https://nvd.nist.gov/vuln/detail/CVE-2011-2523}{CVE-2011-2523}. An attacker can exploit this vulnerability by sending a specific sequence of characters in the username field, which opens a command shell on port 6200, granting the attacker remote code execution capabilities on the server.

Furthermore, the server is configured to allow \textbf{Anonymous FTP login}. This allows any external user to connect and access files on the FTP server without authentication, posing a significant data leakage risk.

% -----------------------------------------------------------------------------
% 5. CONSOLIDATED RISK ASSESSMENT
% -----------------------------------------------------------------------------
\section*{5. Consolidated Risk Assessment}

The following table synthesizes findings from the technical scan, control review, and pre-existing risk data into a prioritized list of security risks.

\begin{table}[h!]
\centering
\caption{Summary of Identified Risks}
\begin{tabular}{p{0.3\textwidth} p{0.5\textwidth} l}
\toprule
\textbf{Risk Name} & \textbf{Description} & \textbf{Severity} \\
\midrule
\textbf{Critical RCE in FTP Server} & The public-facing FTP server is running vsftpd 2.3.4, which contains a known backdoor (CVE-2011-2523) allowing full system compromise. & \textbf{Critical} \\
\addlinespace
\textbf{No MFA for Sensitive Systems} & Lack of MFA on sensitive data systems allows for single-factor authentication compromise, potentially leading to a major data breach. & \textbf{Critical} \\
\addlinespace
\textbf{Anonymous FTP Access} & The FTP server allows unauthenticated logins, exposing files to the public internet and risking data exfiltration. & High \\
\addlinespace
\textbf{No MFA for Computer Login} & Workstations do not require MFA, increasing the risk of unauthorized local and network access via compromised credentials. & High \\
\addlinespace
\textbf{Outdated Windows Policy} & Computers are running Windows 7, an unsupported OS lacking modern security features and patches, making them vulnerable to known exploits. & Medium \\
\bottomrule
\end{tabular}
\end{table}

% -----------------------------------------------------------------------------
% 6. RECOMMENDATIONS
% -----------------------------------------------------------------------------
\section*{6. Recommendations}

Based on the risk assessment, the following prioritized actions are recommended to improve the organization's security posture.

\begin{enumerate}
    \item \textbf{Immediate: Remediate FTP Server Vulnerability.} The public-facing vsftpd server represents a direct and immediate threat.
    \begin{itemize}
        \item If the FTP service is not essential, \textbf{disable and block port 21} immediately.
        \item If the service is required, \textbf{upgrade vsftpd to the latest stable version} without delay.
        \item In either case, \textbf{disable anonymous FTP access} unless there is an explicit and documented business requirement.
    \end{itemize}
    \vspace{1em}

    \item \textbf{High Priority: Implement MFA on Sensitive Systems.} Protect critical data by enforcing MFA for all systems identified as containing sensitive information. This is the single most effective control to prevent data breaches from compromised credentials.
    \vspace{1em}
    
    \item \textbf{High Priority: Enforce MFA for Computer Logins.} Deploy MFA for all workstation and server logins. This strengthens initial access controls and contains the impact of stolen passwords, which is especially important given the presence of outdated operating systems.
    \vspace{1em}
    
    \item \textbf{Medium Priority: Accelerate Windows OS Upgrade Plan.} Continue with the existing plan to upgrade all Windows 7 machines. An unsupported OS is a persistent risk that cannot be fully mitigated by other controls. Prioritize upgrading systems used by privileged users or those that access sensitive data.
\end{enumerate}

\end{document}
% -----------------------------------------------------------------------------
% END OF DOCUMENT
% -----------------------------------------------------------------------------
```