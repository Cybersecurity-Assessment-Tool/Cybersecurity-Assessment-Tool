```latex
\documentclass[12pt]{article}

% Preamble: Required Packages
\usepackage[margin=1in]{geometry}
\usepackage{pifont} % For checkmarks and crosses
\usepackage{booktabs} % For professional tables
\usepackage{hyperref} % For clickable links
\usepackage{url} % For URL formatting
\usepackage{seqsplit} % For splitting long strings to prevent overflow
\usepackage{graphicx}
\usepackage{xcolor}

% Hyperref Setup
\hypersetup{
    colorlinks=true,
    linkcolor=blue,
    filecolor=magenta,      
    urlcolor=cyan,
    pdftitle={Cybersecurity Posture Report},
    pdfpagemode=FullScreen,
}

% Define custom colors
\definecolor{darkred}{rgb}{0.55, 0.0, 0.0}
\definecolor{darkorange}{rgb}{1.0, 0.55, 0.0}

% Title Information
\title{Cybersecurity Posture Report}
\author{Cybersecurity Analysis Division}
\date{\today}

\begin{document}

\maketitle
\thispagestyle{empty}
\newpage

\tableofcontents
\newpage

\section{Executive Summary}
This report provides a comprehensive analysis of the cybersecurity posture for \textbf{[Organization Name]}. The assessment is based on a synthesis of network scan data, a review of organizational security controls, and an evaluation of pre-existing risk documentation.

The analysis reveals a critical risk posture. The primary drivers of this risk are the complete absence of Multi-Factor Authentication (MFA) across all critical systems, the lack of foundational security policies and training for new employees, and the presence of a pre-existing critical vulnerability. An externally facing SSH service was also identified, which, when combined with the lack of MFA, presents a significant vector for unauthorized access.

Immediate and decisive action is required to remediate these findings. The recommendations section outlines a prioritized plan to address the most severe risks first, focusing on implementing MFA, remediating the critical vulnerability, and establishing essential security governance controls.

\section{Organizational Information}
This section details the information provided by the client organization. The data is used to establish the context for the technical and procedural analysis that follows.

\begin{itemize}
    \item \textbf{Organization Name:} \textbf{[Organization Name]}
    \item \textbf{Primary Domain:} \texttt{[Domain]}
    \item \textbf{External IP Scanned:} \texttt{[Client IP]}
\end{itemize}

\section{Security Control Review}
The following table summarizes the organization's responses to a standard security controls questionnaire. A green checkmark (\textcolor{green}{\ding{51}}) indicates a positive control is in place, while a red cross (\textcolor{darkred}{\ding{55}}) indicates a control gap.

\begin{table}[h!]
\centering
\caption{Organizational Security Controls Questionnaire}
\label{tab:controls}
\begin{tabular}{@{}lc@{}}
\toprule
\textbf{Control Question} & \textbf{Response} \\ \midrule
Do you require MFA to access email? & \textcolor{darkred}{\ding{55}} \\
Do you require MFA to log into computers? & \textcolor{darkred}{\ding{55}} \\
Do you require MFA to access sensitive data systems? & \textcolor{darkred}{\ding{55}} \\
Does your organization have an employee acceptable use policy? & \textcolor{darkred}{\ding{55}} \\
Does your organization do security awareness training for new employees? & \textcolor{darkred}{\ding{55}} \\
Does your organization do security awareness training for all employees at least once per year? & \textcolor{green}{\ding{51}} \\ \bottomrule
\end{tabular}
\end{table}

\subsection*{Analysis of Control Gaps}
The questionnaire reveals several critical gaps in the organization's defensive posture:
\begin{itemize}
    \item \textbf{Lack of Multi-Factor Authentication (MFA):} The absence of MFA for email, computer logins, and sensitive data access is a critical weakness. This significantly increases the risk of account compromise via credential theft, phishing, or password spraying attacks.
    \item \textbf{Missing Foundational Policies:} Without an Acceptable Use Policy, there are no formal guidelines for employees on the proper and secure use of company assets, creating ambiguity and risk.
    \item \textbf{Inadequate Employee Onboarding:} New employees are not receiving security awareness training. This is a missed opportunity to instill a security-conscious mindset from day one, leaving a vulnerable window before they attend the annual training.
\end{itemize}

\section{Technical Scan Results}
An external network scan was performed on the target IP address to identify open ports and exposed services.

\begin{itemize}
    \item \textbf{Target IP:} \texttt{[Target IP]}
    \item \textbf{Scan Date:} \today
\end{itemize}

\begin{table}[h!]
\centering
\caption{Open Ports Detected}
\label{tab:ports}
\begin{tabular}{@{}llll@{}}
\toprule
\textbf{Port} & \textbf{State} & \textbf{Service} & \textbf{Analysis} \\ \midrule
22/tcp & open & SSH & Secure Shell (SSH) is used for remote administration. \\
& & & Exposing this service to the internet is a high risk. \\
& & & It is a common target for brute-force attacks. \\
& & & The risk is amplified by the lack of MFA. \\
\bottomrule
\end{tabular}
\end{table}

\section{Consolidated Risk Assessment}
This section correlates findings from the security control review, technical scan, and pre-existing risk documentation into a prioritized list of identified risks.

\begin{table}[h!]
\centering
\caption{Summary of Identified Risks}
\label{tab:risks}
\begin{tabular}{@{}p{0.1\textwidth}p{0.5\textwidth}p{0.15\textwidth}p{0.15\textwidth}@{}}
\toprule
\textbf{Risk ID} & \textbf{Description} & \textbf{Severity} & \textbf{Source} \\ \midrule
\textbf{RISK-001} & A pre-existing vulnerability, "Localhost Exposed", is documented with a CVSS score of 10.0. This represents a maximum-severity risk that must be addressed immediately. & \textcolor{darkred}{\textbf{Critical}} & Current Risks \\
\addlinespace
\textbf{RISK-002} & Complete lack of Multi-Factor Authentication (MFA) for email, computer, and sensitive data access. This leaves the organization highly vulnerable to account takeover. & \textcolor{darkred}{\textbf{Critical}} & Questionnaire \\
\addlinespace
\textbf{RISK-003} & The Secure Shell (SSH) administrative service on port 22 is exposed to the public internet. This invites automated brute-force attacks and targeted intrusion attempts. & \textcolor{darkorange}{\textbf{High}} & Network Scan \\
\addlinespace
\textbf{RISK-004} & Absence of an Acceptable Use Policy and security training for new hires. This indicates a weakness in security governance and culture, increasing the likelihood of human error. & \textcolor{darkorange}{\textbf{High}} & Questionnaire \\ \bottomrule
\end{tabular}
\end{table}

\section{Recommendations}
The following actionable recommendations are prioritized based on the severity of the identified risks.

\subsection{Immediate Priority (Critical Risks)}
\begin{enumerate}
    \item \textbf{Remediate RISK-001 (Localhost Exposed):} Given a CVSS score of 10.0, this vulnerability must be treated as an emergency. Immediately apply any available patches or implement mitigation measures as per the vulnerability disclosure. Isolate affected systems if a patch is not available.
    \item \textbf{Implement Multi-Factor Authentication (RISK-002):}
        \begin{itemize}
            \item Immediately enable MFA for all user accounts on the email platform (e.g., Office 365, Google Workspace).
            \item Phase in MFA for all remote access systems (VPN, SSH) and access to systems containing sensitive data.
            \item Develop a roadmap to enforce MFA for all computer logins.
        \end{itemize}
\end{enumerate}

\subsection{High Priority Recommendations}
\begin{enumerate}
    \setcounter{enumi}{2}
    \item \textbf{Secure the Exposed SSH Service (RISK-003):}
        \begin{itemize}
            \item If remote access via SSH is not a business requirement, disable the service and block port 22 at the firewall.
            \item If SSH is required, restrict access to a whitelist of trusted IP addresses.
            \item Enforce public key authentication and disable password-based authentication for SSH.
        \end{itemize}
    \item \textbf{Establish Foundational Security Governance (RISK-004):}
        \begin{itemize}
            \item Develop and ratify a formal Employee Acceptable Use Policy. Ensure all employees read and acknowledge the policy.
            \item Create a mandatory security awareness training module for all new employees as part of the onboarding process.
        \end{itemize}
\end{enumerate}

\section{Conclusion}
The current cybersecurity posture of \textbf{[Organization Name]} contains several critical and high-severity risks that require immediate attention. The combination of an unpatched critical vulnerability, exposed administrative services, and a lack of fundamental security controls like MFA creates a high probability of a significant security incident.

By following the prioritized recommendations in this report, the organization can substantially reduce its attack surface and build a more resilient and defensible security foundation.

\end{document}
```