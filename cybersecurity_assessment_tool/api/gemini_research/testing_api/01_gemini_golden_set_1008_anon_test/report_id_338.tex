```latex
\documentclass[12pt]{article}

% --- PACKAGES ---
\usepackage[margin=1in]{geometry}
\usepackage{pifont} % Required for \ding
\usepackage{booktabs} % For professional-looking tables
\usepackage{hyperref} % For creating hyperlinks in the document
\usepackage{url}   % For proper URL formatting
\usepackage{seqsplit} % For splitting long strings in \texttt
\usepackage{graphicx}
\usepackage{fancyhdr}

% --- DOCUMENT SETUP ---
\hypersetup{
    colorlinks=true,
    linkcolor=black,
    filecolor=magenta,      
    urlcolor=blue,
    pdftitle={Cybersecurity Posture Assessment Report},
    pdfpagemode=FullScreen,
}

% --- CUSTOM COMMANDS ---
\newcommand{\yes}{\ding{51}} % Checkmark
\newcommand{\no}{\ding{55}}  % X mark

% --- HEADER & FOOTER ---
\pagestyle{fancy}
\fancyhf{}
\fancyhead[L]{Cybersecurity Assessment Report}
\fancyhead[R]{\textbf{[Organization Name]}}
\fancyfoot[C]{\thepage}

% --- DOCUMENT START ---
\begin{document}

\title{Cybersecurity Posture Assessment Report \\ \large For: \textbf{[Organization Name]}}
\author{Cybersecurity Analysis Division}
\date{\today}
\maketitle
\thispagestyle{empty}

\newpage

\tableofcontents

\newpage

\section{Executive Summary}
This report provides a comprehensive assessment of the cybersecurity posture for \textbf{[Organization Name]}, synthesizing findings from technical network scans, a security controls questionnaire, and a review of pre-existing risks.

The assessment has identified several high-impact vulnerabilities and control gaps that require immediate attention. The most critical finding is an externally accessible FTP server running a dangerously outdated and vulnerable version of \texttt{vsftpd 2.3.4}. This specific version is known to contain a backdoor (\textbf{CVE-2011-2523}) that allows for trivial remote code execution. The server is further misconfigured to allow anonymous access, posing a severe and immediate threat of data breach and system compromise.

Furthermore, significant gaps were identified in administrative and access controls. The lack of Multi-Factor Authentication (MFA) for sensitive data systems, coupled with the absence of a formal Acceptable Use Policy (AUP) and annual security training, leaves the organization highly susceptible to both internal and external threats, including credential theft and social engineering attacks.

Immediate remediation of the vulnerable FTP server is paramount. Following this, a systematic effort to close the identified access control and policy gaps is strongly recommended to improve the organization's overall defensive posture.

\section{Organizational Information}
This section outlines the basic information used as the basis for this assessment. Due to the anonymized nature of the input data, placeholders have been used.

\begin{itemize}
    \item \textbf{Organization Name:} \textbf{[Organization Name]}
    \item \textbf{Primary Domain:} \texttt{[Domain]}
    \item \textbf{External IP Scanned:} \texttt{[Client IP]}
\end{itemize}

\section{Security Control Review}
The following table summarizes the organization's self-reported security controls. A green checkmark (\yes) indicates a positive control is in place, while a red 'X' (\no) highlights a control gap that represents a potential risk.

\begin{table}[h!]
\centering
\begin{tabular}{p{0.7\linewidth}c}
\toprule
\textbf{Control Question} & \textbf{Status} \\
\midrule
Do you require MFA to access email? & \yes \\
Do you require MFA to log into computers? & \yes \\
Do you require MFA to access sensitive data systems? & \no \\
Does your organization have an employee acceptable use policy? & \no \\
Does your organization do security awareness training for new employees? & \yes \\
Does your organization do security awareness training for all employees at least once per year? & \no \\
\bottomrule
\end{tabular}
\caption{Security Controls Questionnaire Results}
\end{table}

\paragraph{Analysis:} The review indicates critical gaps in security controls. The absence of Multi-Factor Authentication (MFA) for sensitive data systems is a primary concern. Additionally, the lack of a formal Acceptable Use Policy (AUP) and annual security awareness training for all employees creates significant unaddressed human-factor risks.

\section{Technical Scan Results}
An external network scan was performed to identify exposed services and potential vulnerabilities on the perimeter.

\subsection{Host: \texttt{[Target IP]}}
The scan identified one host as active and responsive.
\begin{itemize}
    \item \textbf{Status:} Host is UP.
    \item \textbf{Open Ports Found:} 1
\end{itemize}

\begin{table}[h!]
\centering
\begin{tabular}{lllll}
\toprule
\textbf{Port} & \textbf{State} & \textbf{Service} & \textbf{Version} & \textbf{Notes} \\
\midrule
21/tcp & open & ftp & vsftpd 2.3.4 & Anonymous FTP login allowed \\
\bottomrule
\end{tabular}
\caption{Open Ports and Services for \texttt{[Target IP]}}
\end{table}

\paragraph{Critical Finding:} The version of vsftpd detected, \textbf{2.3.4}, is notoriously vulnerable. This specific version contains a critical backdoor vulnerability (\textbf{CVE-2011-2523}) that was intentionally inserted into the source code. An attacker can gain a command shell on the server by simply entering a specific character sequence as the username. Furthermore, the service is configured to allow \textbf{anonymous FTP logins}, which permits unauthenticated access to files on the server. This represents an immediate and severe threat to the organization's data and infrastructure.

\section{Consolidated Risk Assessment}
The following table synthesizes findings from the technical scan, control review, and pre-existing risk data to provide a holistic view of the organization's risk landscape.

\begin{table}[h!]
\centering
\begin{tabular}{p{0.35\linewidth}p{0.45\linewidth}l}
\toprule
\textbf{Risk / Vulnerability} & \textbf{Description} & \textbf{Severity} \\
\midrule
Vulnerable FTP Service (CVE-2011-2523) & A public-facing FTP server is running a version with a known remote code execution backdoor. & \textbf{Critical} \\
Anonymous FTP Access & The FTP server allows unauthenticated users to access, upload, or download files, posing a severe data breach risk. & \textbf{Critical} \\
No MFA on Sensitive Systems & Lack of MFA on systems holding sensitive data allows for credential-based attacks and unauthorized access. & High \\
Outdated Windows Policy & Workstations are running Windows 7, an unsupported OS lacking modern security patches. & Medium \\
No Acceptable Use Policy & Lack of a formal AUP creates ambiguity regarding employee responsibilities and acceptable system usage. & Medium \\
Inadequate Security Training & Failure to provide annual security training increases susceptibility to phishing and social engineering. & Medium \\
\bottomrule
\end{tabular}
\caption{Summary of Identified Risks}
\end{table}

\section{Recommendations}
Based on the consolidated risk assessment, the following prioritized actions are recommended to mitigate the identified threats.

\subsection{Immediate Actions (Priority 1)}
\begin{enumerate}
    \item \textbf{Decommission or Isolate Vulnerable FTP Server:} Immediately take the server running \texttt{vsftpd 2.3.4} offline. If FTP is a business necessity:
        \begin{itemize}
            \item Upgrade to the latest stable version of vsftpd or an alternative secure file transfer solution (e.g., SFTP).
            \item Disable anonymous access immediately.
            \item Place the service behind a firewall and restrict access to only known, trusted IP addresses.
        \end{itemize}
    \item \textbf{Implement MFA on Sensitive Systems:} Enforce MFA on all systems identified as containing sensitive data. This is a critical compensating control to protect against credential theft and unauthorized access.
\end{enumerate}

\subsection{High-Priority Actions (Priority 2)}
\begin{enumerate}
    \item \textbf{Develop and Implement an Acceptable Use Policy (AUP):} Create a formal AUP that all employees must read and sign. This policy should clearly define rules for using company assets, data handling, and internet usage.
    \item \textbf{Establish Annual Security Awareness Training:} Institute a mandatory security awareness training program for all employees, to be completed annually. This program should cover topics like phishing, password security, and social engineering.
    \item \textbf{Upgrade End-of-Life Operating Systems:} Begin the process of upgrading all workstations from Windows 7 to a supported operating system like Windows 10 or 11 to ensure they receive critical security updates.
\end{enumerate}

\end{document}
```