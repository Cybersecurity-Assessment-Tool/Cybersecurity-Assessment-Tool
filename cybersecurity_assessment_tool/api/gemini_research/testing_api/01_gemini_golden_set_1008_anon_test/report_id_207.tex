```latex
\documentclass[12pt]{article}

% --- PACKAGES ---
\usepackage[margin=1in]{geometry}
\usepackage{pifont} % For checkmarks and crosses
\usepackage{booktabs} % For professional tables
\usepackage{hyperref} % For clickable links
\usepackage{url} % For URL formatting
\usepackage{seqsplit} % For splitting long strings
\usepackage{graphicx}
\usepackage{xcolor}
\usepackage{fancyhdr}
\usepackage{lastpage}

% --- DOCUMENT SETUP ---
\hypersetup{
    colorlinks=true,
    linkcolor=blue,
    filecolor=magenta,      
    urlcolor=cyan,
    pdftitle={Cybersecurity Assessment Report},
    pdfpagemode=FullScreen,
}

% --- HEADER & FOOTER ---
\pagestyle{fancy}
\fancyhf{} % Clear all header and footer fields
\fancyhead[L]{Cybersecurity Assessment Report}
\fancyhead[R]{\textbf{[Organization Name]}}
\fancyfoot[C]{Page \thepage\ of \pageref{LastPage}}
\renewcommand{\headrulewidth}{0.4pt}
\renewcommand{\footrulewidth}{0.4pt}

% --- DOCUMENT START ---
\begin{document}

% --- TITLE PAGE ---
\begin{titlepage}
    \centering
    \vspace*{1cm}
    
    \Huge
    \textbf{Cybersecurity Assessment Report}
    
    \vspace{1.5cm}
    
    \Large
    Prepared for:
    
    \vspace{0.5cm}
    
    \textbf{[Organization Name]}
    
    \vfill
    
    \large
    \today
    
\end{titlepage}

\tableofcontents
\newpage

% --- EXECUTIVE SUMMARY ---
\section{Executive Summary}

This report provides a comprehensive analysis of the current cybersecurity posture of \textbf{[Organization Name]}. The assessment is based on a combination of network scanning, a review of existing risks, and an analysis of organizational security controls via a questionnaire.

The overall security posture is determined to be \textbf{CRITICAL}. Several fundamental security controls are absent, and a critical, internet-facing asset is running an unsupported, end-of-life software version.

Key findings include:
\begin{itemize}
    \item \textbf{Complete Absence of Multi-Factor Authentication (MFA):} MFA is not enforced for email, computer logins, or access to sensitive data systems. This represents a critical vulnerability, significantly increasing the risk of account compromise and unauthorized access.
    \item \textbf{Exposed End-of-Life Database:} An external scan identified a MySQL database (version 5.7.33) directly accessible from the internet. This version reached its official End of Life (EOL) in October 2023 and no longer receives security updates, making it an easy target for exploitation.
    \item \textbf{Lack of Foundational Security Policies and Training:} The organization lacks a formal Acceptable Use Policy and does not conduct security awareness training for its employees. This indicates a low level of security maturity and increases the risk of human error leading to a security incident.
\end{itemize}

Immediate and decisive action is required to remediate these high-risk vulnerabilities. The recommendations section of this report outlines a prioritized, actionable plan to mitigate these threats and improve the organization's overall defensive capabilities.

\newpage

% --- ORGANIZATIONAL INFORMATION ---
\section{Organizational Information}
This section details the information provided for this assessment. Due to the anonymized nature of the input data, placeholders are used where necessary.

\begin{tabular}{@{}ll}
    \toprule
    \textbf{Attribute} & \textbf{Value} \\
    \midrule
    Organization Name & \textbf{[Organization Name]} \\
    Primary Email Domain & \texttt{[Domain]} \\
    External IP Address Scanned & \texttt{[Client IP]} \\
    \bottomrule
\end{tabular}

% --- SECURITY CONTROL REVIEW ---
\section{Security Control Review}
The following table summarizes the responses to the security questionnaire. A "No" response indicates a significant gap in the organization's security controls and is flagged as a high or critical risk.

\begin{table}[h!]
\centering
\caption{Security Controls Questionnaire Analysis}
\begin{tabular}{@{}p{8cm}ccp{3cm}@{}}
    \toprule
    \textbf{Control Question} & \textbf{Response} & \textbf{Status} & \textbf{Assessment} \\
    \midrule
    Do you require MFA to access email? & No & \ding{55} & Critical Gap \\
    Do you require MFA to log into computers? & No & \ding{55} & Critical Gap \\
    Do you require MFA to access sensitive data systems? & No & \ding{55} & Critical Gap \\
    Does your organization have an employee acceptable use policy? & No & \ding{55} & High Risk \\
    Does your organization do security awareness training for new employees? & No & \ding{55} & High Risk \\
    Does your organization do security awareness training for all employees at least once per year? & No & \ding{55} & High Risk \\
    \bottomrule
\end{tabular}
\end{table}

% --- TECHNICAL SCAN RESULTS ---
\section{Technical Scan Results}
An external network scan was performed to identify open ports and exposed services.

\subsection{Nmap Scan Findings for \texttt{[Target IP]}}
The scan revealed one open port, which exposes a database service directly to the internet.

\begin{table}[h!]
\centering
\caption{Open Port Analysis}
\begin{tabular}{@{}llllll@{}}
    \toprule
    \textbf{Port} & \textbf{State} & \textbf{Service} & \textbf{Product} & \textbf{Version} & \textbf{Notes} \\
    \midrule
    3306 & open & mysql & MySQL & 5.7.33 & \textcolor{red}{\textbf{End-of-Life (EOL)}} \\
    \bottomrule
\end{tabular}
\end{table}

\paragraph{Analysis:} The MySQL 5.7.x branch reached its End of Life in October 2023. This means it no longer receives security patches from the vendor. Running an EOL database, especially one exposed to the public internet, constitutes a \textbf{critical risk}. Attackers can exploit known, unpatched vulnerabilities to gain unauthorized access, exfiltrate data, or compromise the underlying server.

% --- RISK ASSESSMENT ---
\section{Consolidated Risk Assessment}
This section synthesizes findings from the questionnaire, technical scans, and pre-existing risk data into a consolidated list of prioritized risks.

\begin{table}[h!]
\centering
\caption{Summary of Identified Risks}
\begin{tabular}{@{}p{4.5cm}p{7.5cm}l@{}}
    \toprule
    \textbf{Risk Name} & \textbf{Description} & \textbf{Severity} \\
    \midrule
    \textbf{Exposed End-of-Life Database} & A MySQL 5.7.33 database (EOL Oct 2023) is publicly exposed on port 3306. This allows attackers to exploit unpatched vulnerabilities. & \textcolor{red}{\textbf{Critical}} \\
    \addlinespace
    \textbf{Lack of Multi-Factor Authentication (MFA)} & The absence of MFA for email, endpoints, and sensitive systems makes user accounts highly susceptible to takeover via credential theft or phishing. & \textcolor{red}{\textbf{Critical}} \\
    \addlinespace
    \textbf{Lack of Security Policies \& Training} & The organization has no Acceptable Use Policy or security awareness training program, leading to a higher likelihood of security incidents caused by human error. & \textcolor{orange}{\textbf{High}} \\
    \bottomrule
\end{tabular}
\end{table}

% --- RECOMMENDATIONS ---
\section{Recommendations}
The following actionable recommendations are provided to mitigate the identified risks. They are prioritized based on urgency and impact.

\subsection{Immediate Priority (0-30 Days)}
\begin{enumerate}
    \item \textbf{Isolate the Exposed Database:} Immediately implement firewall rules to restrict all access to TCP port 3306 on host \texttt{[Target IP]}. Access should only be permitted from specific, trusted IP addresses. This is the fastest way to mitigate the exposure risk.
    \item \textbf{Deploy Multi-Factor Authentication (MFA):} Begin immediate rollout of MFA across the organization. Prioritize the following systems:
    \begin{itemize}
        \item Email (e.g., Office 365, Google Workspace)
        \item VPN and other remote access solutions
        \item All administrative accounts
        \item Systems containing sensitive data
    \end{itemize}
\end{enumerate}

\subsection{Medium Priority (30-90 Days)}
\begin{enumerate}
    \item \textbf{Upgrade the End-of-Life Database:} Plan and execute the migration of the MySQL 5.7.33 database to a fully supported version (e.g., MySQL 8.x). This is essential for long-term security and stability.
    \item \textbf{Develop Foundational Security Policies:} Draft and implement a basic set of security policies, starting with an \textbf{Acceptable Use Policy (AUP)} for all employees.
    \item \textbf{Implement Security Awareness Training:} Enroll all employees in a reputable security awareness training program. This should be a mandatory part of onboarding for new hires and an annual requirement for all staff.
\end{enumerate}

\subsection{Long-Term Strategy (90+ Days)}
\begin{enumerate}
    \item \textbf{Establish a Vulnerability Management Program:} Implement a formal process for regular network scanning, vulnerability identification, and timely patching of all systems.
    \item \textbf{Implement a Secure Remote Access Solution:} For long-term database access, replace direct exposure with a secure solution like a Virtual Private Network (VPN) with MFA.
\end{enumerate}

\end{document}
```