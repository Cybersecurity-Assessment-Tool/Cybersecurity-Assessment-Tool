```latex
\documentclass[12pt]{article}

% Preamble: Required Packages
\usepackage[margin=1in]{geometry}
\usepackage{pifont} % For checkmarks and crosses
\usepackage{booktabs} % For professional tables
\usepackage{hyperref} % For hyperlinks
\usepackage{url}      % For URL formatting
\usepackage{seqsplit} % For splitting long strings
\usepackage{graphicx}
\usepackage{xcolor}

% Hyperref Setup
\hypersetup{
    colorlinks=true,
    linkcolor=blue,
    filecolor=magenta,      
    urlcolor=cyan,
    pdftitle={Cybersecurity Risk Assessment Report},
    pdfpagemode=FullScreen,
}

% Document Title and Metadata
\title{Cybersecurity Risk Assessment Report \\ \large For: \textbf{[Organization Name]}}
\author{Cybersecurity Analysis Team}
\date{\today}

\begin{document}

\maketitle
\thispagestyle{empty}
\newpage

\tableofcontents
\newpage

% --- 1. Executive Summary ---
\section{Executive Summary}

This report details the findings of a cybersecurity assessment conducted for \textbf{[Organization Name]}. The analysis is based on a synthesis of external network scan data, a review of internal security controls via a questionnaire, and a correlation with pre-existing risk data.

The assessment identified a **critical risk**: the direct exposure of the Remote Desktop Protocol (RDP) service on port 3389 to the public internet. This configuration is a common target for attackers and significantly increases the risk of unauthorized access, ransomware attacks, and data breaches.

Furthermore, a **high-risk gap** was identified in the organization's security posture: the lack of mandatory annual security awareness training for all employees. This deficiency heightens the probability of human error, such as falling for phishing attacks or using weak credentials, which could directly lead to the compromise of the exposed RDP service.

Immediate remediation of the exposed service is strongly recommended, followed by the implementation of a comprehensive annual training program to strengthen the organization's human firewall.

% --- 2. Organizational Information ---
\section{Organizational Information}

The following details were used as the basis for this assessment. Due to the anonymized nature of the provided data, placeholders have been used where necessary.

\begin{itemize}
    \item \textbf{Organization Name:} \textbf{[Organization Name]}
    \item \textbf{Primary Domain:} \texttt{[Domain]}
    \item \textbf{External IP Address Scanned:} \texttt{[Client IP]}
\end{itemize}

% --- 3. Security Control Review ---
\section{Security Control Review}

An assessment of organizational security controls was performed based on the provided questionnaire. While the organization demonstrates strong adoption of Multi-Factor Authentication (MFA) and foundational policies, a significant gap in ongoing employee education was noted.

\begin{table}[h!]
\centering
\caption{Security Control Questionnaire Analysis}
\begin{tabular}{p{0.6\linewidth} c l}
\toprule
\textbf{Control Question} & \textbf{Response} & \textbf{Assessment} \\
\midrule
Do you require MFA to access email? & \ding{51} & Good Practice \\
Do you require MFA to log into computers? & \ding{51} & Good Practice \\
Do you require MFA to access sensitive data systems? & \ding{51} & Good Practice \\
Does your organization have an employee acceptable use policy? & \ding{51} & Good Practice \\
Does your organization do security awareness training for new employees? & \ding{51} & Good Practice \\
\midrule
\textbf{Does your organization do security awareness training for all employees at least once per year?} & \textbf{\color{red}\ding{55}} & \textbf{\color{red}High Risk Gap} \\
\bottomrule
\end{tabular}
\end{table}

% --- 4. Technical Scan Results ---
\section{Technical Scan Results}

An external network scan was conducted against the target IP address \texttt{[Target IP]}. The scan revealed one open port, which presents a critical attack surface.

\begin{table}[h!]
\centering
\caption{Open Ports Detected on \texttt{[Target IP]}}
\begin{tabular}{l l l l}
\toprule
\textbf{Port} & \textbf{State} & \textbf{Service Name} & \textbf{Notes} \\
\midrule
3389/tcp & open & ms-wbt-server & Remote Desktop Protocol (RDP) \\
\bottomrule
\end{tabular}
\end{table}

\subsection*{Analysis of Findings}
The service \texttt{ms-wbt-server} is Microsoft's Remote Desktop Protocol (RDP). Exposing RDP directly to the internet is extremely dangerous. It is a primary vector for ransomware attacks, where threat actors scan the internet for open RDP ports, and then attempt to gain access via brute-force password attacks or by exploiting known vulnerabilities.

% --- 5. Correlated Risk Assessment ---
\section{Correlated Risk Assessment}

This section synthesizes the findings from the security control review, technical scan, and pre-existing risk data into a prioritized list of security risks.

\begin{table}[h!]
\centering
\caption{Summary of Identified Risks}
\begin{tabular}{p{0.2\linewidth} p{0.45\linewidth} l p{0.15\linewidth}}
\toprule
\textbf{Risk Name} & \textbf{Description} & \textbf{Severity} & \textbf{Affected Elements} \\
\midrule
\textbf{RDP Exposure} & The RDP service on port 3389 is directly exposed to the internet, inviting automated brute-force attacks and exploitation attempts. This finding from the technical scan confirms a known high-severity risk. & \textbf{Critical (9.0)} & \texttt{[Target IP]} \\
\addlinespace
\textbf{Lack of Annual Security Training} & The absence of recurring security training for all staff increases the likelihood of credential compromise through phishing or poor password hygiene, which could be used to attack the exposed RDP service. & \textbf{High} & All Employees \\
\bottomrule
\end{tabular}
\end{table}

% --- 6. Recommendations ---
\section{Recommendations}

The following actions are recommended to mitigate the identified risks. Risks should be addressed in order of severity.

\subsection{Remediate RDP Exposure (Critical)}
\begin{itemize}
    \item \textbf{Immediate Action (Containment):} Implement a firewall rule to \textbf{block all inbound traffic to TCP port 3389} on the external interface for \texttt{[Target IP]}. This will immediately remove the exposure.
    \item \textbf{Long-Term Solution (Eradication):} For necessary remote access, deploy a Virtual Private Network (VPN) solution. All administrative access to internal systems, including RDP, must be enforced through the secure VPN tunnel. This follows the principle of "least privilege" and removes the service from public view.
\end{itemize}

\subsection{Implement Annual Security Training (High)}
\begin{itemize}
    \item \textbf{Action (Program Development):} Develop and mandate an annual security awareness training program for all employees. This program is essential for maintaining a security-conscious culture.
    \item \textbf{Content Suggestions:} The training should cover, at a minimum:
        \begin{itemize}
            \item Phishing and social engineering awareness.
            \item Strong password creation and management.
            \item The importance of Multi-Factor Authentication (MFA).
            \item The organization's acceptable use policy.
            \item Procedures for reporting security incidents.
        \end{itemize}
\end{itemize}

\end{document}
```