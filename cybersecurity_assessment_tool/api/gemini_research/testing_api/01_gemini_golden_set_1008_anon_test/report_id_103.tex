```latex
\documentclass[12pt]{article}

% Preamble: Required Packages
\usepackage[a4paper, margin=1in]{geometry}
\usepackage{pifont} % For checkmarks and crosses
\usepackage{booktabs} % For professional tables
\usepackage{hyperref} % For clickable links and better PDF navigation
\usepackage{url} % For formatting URLs
\usepackage{seqsplit} % For splitting long strings without spaces
\usepackage{graphicx} % For potential logos in a real report
\usepackage{xcolor} % For colors

% Hyperref Setup
\hypersetup{
    colorlinks=true,
    linkcolor=blue,
    filecolor=magenta,      
    urlcolor=cyan,
    pdftitle={Cybersecurity Posture Assessment Report},
    pdfpagemode=FullScreen,
}

% Document Metadata
\title{Cybersecurity Posture Assessment Report \\ \large for \textbf{[Organization Name]}}
\author{Cybersecurity Analyst Group}
\date{\today}

\begin{document}

\maketitle
\thispagestyle{empty}
\newpage

\tableofcontents
\thispagestyle{empty}
\newpage

\setcounter{page}{1}

% --- 1. Executive Summary ---
\section{Executive Summary}

This report provides a comprehensive analysis of the cybersecurity posture for \textbf{[Organization Name]}, based on a combination of self-reported organizational controls, an external network scan, and a review of pre-existing risks.

The assessment identified several critical and high-risk security gaps that require immediate attention. The most severe findings include a complete lack of Multi-Factor Authentication (MFA) across all key systems (email, computers, and sensitive data), which exposes the organization to significant risk of account compromise and unauthorized access.

Furthermore, a technical scan of the external IP address \texttt{[Client IP]} revealed an open Secure Shell (SSH) port (22/TCP) on host \texttt{[Target IP]}. Exposing management services like SSH directly to the public internet is a high-risk practice, making the server a prime target for automated brute-force attacks and potential exploitation.

The combination of weak access controls (no MFA) and an exposed management port creates a synergistic risk, where a single compromised credential could lead to a full server compromise. Recommendations have been provided to address these findings and strengthen the overall security posture.

% --- 2. Organizational Information ---
\section{Organizational Information}

The following information was used as the basis for this assessment. Due to the anonymized nature of the provided data, placeholders have been used where necessary.

\begin{table}[h!]
\centering
\begin{tabular}{@{}ll@{}}
\toprule
\textbf{Attribute} & \textbf{Value} \\ \midrule
Organization Name & \textbf{[Organization Name]} \\
Primary Domain & \texttt{[Domain]} \\
External IP Address & \texttt{[Client IP]} \\ \bottomrule
\end{tabular}
\caption{Client Organizational Details}
\label{tab:org_info}
\end{table}

% --- 3. Security Control Review ---
\section{Security Control Review}

A review of the organization's security controls was conducted via a standardized questionnaire. The responses indicate significant gaps in fundamental security practices, particularly concerning identity and access management. A "No" response (\ding{55}) highlights a control that is not in place and represents a potential risk.

\begin{table}[h!]
\centering
\begin{tabular}{@{}p{0.7\linewidth}c@{}}
\toprule
\textbf{Control Question} & \textbf{Response} \\ \midrule
Do you require MFA to access email? & \ding{55} \\
Do you require MFA to log into computers? & \ding{55} \\
Do you require MFA to access sensitive data systems? & \ding{55} \\
Does your organization have an employee acceptable use policy? & \ding{51} \\
Does your organization do security awareness training for new employees? & \ding{55} \\
Does your organization do security awareness training for all employees at least once per year? & \ding{51} \\ \bottomrule
\end{tabular}
\caption{Security Controls Questionnaire Results (\ding{51}=Yes, \ding{55}=No)}
\label{tab:controls}
\end{table}

\subsection*{Analysis of Control Gaps}
\begin{itemize}
    \item \textbf{Lack of Multi-Factor Authentication (MFA):} The absence of MFA for email, computer logins, and sensitive data access is a critical vulnerability. This single point of failure means that a compromised password is all an attacker needs to gain significant access to organizational systems and data.
    \item \textbf{No Security Training for New Employees:} While annual training is in place, the lack of security awareness training during employee onboarding is a high-risk gap. New hires are often more susceptible to social engineering and may be unaware of company policies, making them an attractive target for attackers.
\end{itemize}

% --- 4. Technical Scan Results ---
\section{Technical Scan Results}

An external network vulnerability scan was performed against the target IP address to identify open ports and exposed services.

\begin{itemize}
    \item \textbf{Target IP:} \texttt{[Target IP]}
    \item \textbf{Scan Date:} Scan date not provided in source data.
\end{itemize}

The scan identified the following open port:

\begin{table}[h!]
\centering
\begin{tabular}{@{}llll@{}}
\toprule
\textbf{Port} & \textbf{State} & \textbf{Service (Inferred)} & \textbf{Product / Version} \\ \midrule
22/tcp & open & SSH & Not identified by scan \\ \bottomrule
\end{tabular}
\caption{Open Ports on \texttt{[Target IP]}}
\label{tab:scan_results}
\end{table}

\subsection*{Analysis of Technical Findings}
The presence of an open SSH port (22) on the public internet is a significant security risk. This service is a common target for automated brute-force attacks that attempt to guess user credentials. If the SSH server software is unpatched or misconfigured, it could also be vulnerable to direct exploitation. Without MFA enforced on SSH logins, this port represents a direct path into the organization's network for an attacker with valid or stolen credentials.

% --- 5. Consolidated Risk Assessment ---
\section{Consolidated Risk Assessment}

The following table synthesizes findings from the security control review, technical scan, and pre-existing risk data. Each identified risk is assigned a severity level based on its potential impact and likelihood of exploitation.

\begin{table}[h!]
\centering
\begin{tabular}{@{}p{0.1\linewidth}p{0.35\linewidth}p{0.3\linewidth}p{0.1\linewidth}@{}}
\toprule
\textbf{ID} & \textbf{Risk Description} & \textbf{Affected Asset(s)} & \textbf{Severity} \\ \midrule
\textbf{RISK-01} & Complete lack of Multi-Factor Authentication (MFA) allows for account takeover with a single compromised password. & Email System, Workstations, Sensitive Data Systems & \textbf{Critical} \\
\addlinespace
\textbf{RISK-02} & Exposed SSH management port (22/tcp) is susceptible to brute-force attacks and potential exploitation. & Server at \texttt{[Target IP]} & \textbf{High} \\
\addlinespace
\textbf{RISK-03} & New employees do not receive security awareness training, creating a high-risk period for social engineering and policy violations. & New Employees, Organizational Data & \textbf{High} \\
\bottomrule
\end{tabular}
\caption{Summary of Identified Risks}
\label{tab:risk_summary}
\end{table}

% --- 6. Recommendations ---
\section{Recommendations}

The following actions are recommended to mitigate the identified risks and improve the overall security posture of \textbf{[Organization Name]}.

\subsection*{RISK-01: Lack of Multi-Factor Authentication (Critical)}
\begin{enumerate}
    \item \textbf{Immediate Action:} Prioritize and implement a mandatory MFA policy for all users across all critical systems, including:
    \begin{itemize}
        \item Email (e.g., Office 365, Google Workspace)
        \item VPN and remote access solutions
        \item Access to all systems storing sensitive data
        \item Workstation logins
    \end{itemize}
    \item \textbf{Long-Term Strategy:} Develop a formal Identity and Access Management (IAM) policy that mandates MFA for all new systems and services by default.
\end{enumerate}

\subsection*{RISK-02: Exposed SSH Management Port (High)}
\begin{enumerate}
    \item \textbf{Immediate Action:} Restrict access to port 22 on \texttt{[Target IP]}. Access should be limited to specific, trusted IP addresses via firewall rules. The preferred method is to place the server behind a Virtual Private Network (VPN) and require users to connect to the VPN before accessing SSH.
    \item \textbf{Supporting Actions:}
    \begin{itemize}
        \item Ensure the SSH service is fully patched and running the latest stable version.
        \item Enforce strong password policies and consider disabling password-based authentication in favor of public key authentication.
        \item Implement an intrusion detection system (IDS) or log monitoring to detect and alert on brute-force attempts.
    \end{itemize}
\end{enumerate}

\subsection*{RISK-03: Inadequate New Employee Onboarding (High)}
\begin{enumerate}
    \item \textbf{Immediate Action:} Develop and implement a mandatory security awareness training module as part of the new employee onboarding process. This training should occur before the employee is granted access to sensitive systems.
    \item \textbf{Content Strategy:} The training should cover, at a minimum:
    \begin{itemize}
        \item The organization's acceptable use policy.
        \item How to identify and report phishing attempts.
        \item Password security and MFA requirements.
        \item Data handling best practices.
    \end{itemize}
\end{enumerate}

\end{document}
```