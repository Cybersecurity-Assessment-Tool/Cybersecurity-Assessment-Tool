```latex
\documentclass[12pt, a4paper]{article}

% Preamble: Required Packages
\usepackage[margin=1in]{geometry}
\usepackage{pifont} % For checkmarks and crosses
\usepackage{booktabs} % For professional tables
\usepackage{hyperref} % For clickable links and metadata
\usepackage{url} % For URL formatting
\usepackage{seqsplit} % For splitting long strings in texttt
\usepackage{graphicx}
\usepackage{xcolor}
\usepackage{fancyhdr} % For headers and footers
\usepackage{lastpage} % To get the total number of pages

% --- Document Metadata ---
\hypersetup{
    colorlinks=true,
    linkcolor=blue,
    filecolor=magenta,      
    urlcolor=cyan,
    pdftitle={Cybersecurity Posture Assessment Report},
    pdfauthor={Cybersecurity Analyst},
    pdfsubject={Security Analysis},
    pdfkeywords={Security, Report, Analysis},
}

% --- Define Colors ---
\definecolor{darkblue}{rgb}{0.0, 0.0, 0.5}
\definecolor{darkred}{rgb}{0.5, 0.0, 0.0}
\definecolor{darkgreen}{rgb}{0.0, 0.5, 0.0}

% --- Header and Footer Configuration ---
\pagestyle{fancy}
\fancyhf{} % Clear all header and footer fields
\fancyhead[L]{\textbf{Cybersecurity Posture Assessment}}
\fancyhead[R]{\textbf{[Organization Name]}}
\fancyfoot[C]{\thepage\ of \pageref{LastPage}}
\renewcommand{\headrulewidth}{0.4pt}
\renewcommand{\footrulewidth}{0.4pt}

% --- Title Page ---
\title{
    \vspace{2cm}
    \textbf{Cybersecurity Posture Assessment Report}\\
    \vspace{0.5cm}
    \large{Confidential}
    \vspace{1.5cm}
}
\author{Generated by Expert Cybersecurity Analyst}
\date{\today}

% --- Document Body ---
\begin{document}

\maketitle
\thispagestyle{empty}
\newpage

\tableofcontents
\newpage

% --- Section 1: Executive Summary ---
\section{Executive Summary}

This report provides a comprehensive cybersecurity posture assessment for \textbf{[Organization Name]}, based on an analysis of network scan data, organizational security controls, and existing risk documentation. The assessment was conducted on \today.

The analysis reveals several critical and high-risk security gaps that require immediate attention. Foundational security controls, such as Multi-Factor Authentication (MFA) for email and a comprehensive security awareness training program, are not in place. These omissions expose the organization to significant threats, including business email compromise, phishing, and ransomware.

Furthermore, a technical scan identified an externally exposed Secure Shell (SSH) service. While necessary for remote administration, its exposure without robust protective measures creates a direct vector for unauthorized access attempts.

The overall security posture is considered weak due to these fundamental deficiencies in both policy and technical controls. This report outlines the identified risks and provides actionable recommendations to mitigate them, strengthen the organization's defenses, and improve its overall resilience against cyber threats.

\vspace{1cm}

% --- Section 2: Organizational Information ---
\section{Organizational Information}

The following details were used as the basis for this assessment. Due to the anonymized nature of the input data, placeholders have been used where necessary.

\begin{table}[h!]
\centering
\begin{tabular}{@{}ll@{}}
\toprule
\textbf{Attribute} & \textbf{Value} \\ \midrule
Organization Name & \textbf{[Organization Name]} \\
Primary Email Domain & \texttt{[Domain]} \\
External IP Address (Target) & \texttt{[Client IP]} \\
Scan Target IP & \texttt{[Target IP]} \\
Assessment Date & \today \\ \bottomrule
\end{tabular}
\caption{Organizational and Assessment Details}
\end{table}

% --- Section 3: Security Control Review ---
\section{Security Control Review}

A review of the organization's security controls was conducted via a questionnaire. The responses indicate significant gaps in fundamental security practices. A summary of the findings is presented below.

\begin{table}[h!]
\centering
\begin{tabular}{@{}p{0.6\linewidth}cc@{}}
\toprule
\textbf{Control Question} & \textbf{Response} & \textbf{Status} \\ \midrule
Do you require MFA to access email? & No & \textcolor{darkred}{\ding{55}} \\
Do you require MFA to log into computers? & Yes & \textcolor{darkgreen}{\ding{51}} \\
Do you require MFA to access sensitive data systems? & Yes & \textcolor{darkgreen}{\ding{51}} \\
Does your organization have an employee acceptable use policy? & No & \textcolor{darkred}{\ding{55}} \\
Does your organization do security awareness training for new employees? & No & \textcolor{darkred}{\ding{55}} \\
Does your organization do security awareness training for all employees at least once per year? & No & \textcolor{darkred}{\ding{55}} \\ \bottomrule
\end{tabular}
\caption{Security Controls Questionnaire Results}
\end{table}

\subsection*{Analysis of Control Gaps}
The "No" responses highlight critical weaknesses:
\begin{itemize}
    \item \textbf{No MFA for Email:} This is a critical vulnerability. Email accounts are a primary target for attackers seeking to conduct phishing, social engineering, and business email compromise (BEC) attacks. Without MFA, a compromised password is all an attacker needs to gain access.
    \item \textbf{No Acceptable Use Policy (AUP):} An AUP is a foundational policy that sets clear expectations for employees regarding the use of company assets. Its absence can lead to inconsistent security practices, misuse of resources, and increased insider risk.
    \item \textbf{No Security Awareness Training:} The complete lack of a security awareness training program for both new and existing employees is a critical failure. This leaves the workforce highly susceptible to social engineering attacks, which are the root cause of a majority of security breaches.
\end{itemize}

% --- Section 4: Technical Scan Results ---
\section{Technical Scan Results}

An external network scan was performed on the target IP address \texttt{[Target IP]}. The scan identified the following open ports and services accessible from the public internet.

\begin{table}[h!]
\centering
\begin{tabular}{@{}llll@{}}
\toprule
\textbf{Port} & \textbf{State} & \textbf{Service} & \textbf{Notes} \\ \midrule
22/tcp & open & SSH & Secure Shell (SSH) is used for remote administration. \\
& & & Exposing this service directly to the internet increases \\
& & & the risk of brute-force attacks and exploitation of \\
& & & potential vulnerabilities. Service version information \\
& & & was not available in the provided scan data. \\ \bottomrule
\end{tabular}
\caption{Open Ports Detected on \texttt{[Target IP]}}
\end{table}

\subsection*{Analysis of Technical Findings}
The primary finding is the open SSH port. While essential for remote management, its public exposure is a significant security concern. Without compensating controls like IP whitelisting, strong password policies, mandatory key-based authentication, and intrusion detection systems (e.g., Fail2Ban), this service presents an attractive target for attackers.

% --- Section 5: Consolidated Risk Assessment ---
\section{Consolidated Risk Assessment}

The following table synthesizes findings from the security control review, technical scan, and pre-existing risk data. Each identified risk has been assigned a severity level based on its potential impact and likelihood.

\begin{table}[h!]
\centering
\begin{tabular}{@{}p{0.1\linewidth}p{0.25\linewidth}p{0.45\linewidth}l@{}}
\toprule
\textbf{Risk ID} & \textbf{Risk Title} & \textbf{Description} & \textbf{Severity} \\ \midrule
RISK-001 & Lack of MFA on Email & The absence of MFA on email accounts allows for account takeover with only a compromised password, enabling BEC and data breaches. & \textbf{Critical} \\
\addlinespace
RISK-002 & Inadequate Security Awareness Program & With no training program, employees are unprepared to identify and resist phishing, malware, and other social engineering attacks. & \textbf{Critical} \\
\addlinespace
RISK-003 & Missing Acceptable Use Policy & The lack of a formal AUP creates ambiguity regarding secure practices and employee responsibilities, increasing insider and compliance risks. & \textbf{High} \\
\addlinespace
RISK-004 & Exposed SSH Service & The SSH administrative port is open to the internet, creating a vector for brute-force attacks and potential exploitation if not securely configured. & \textbf{Medium} \\ \bottomrule
\end{tabular}
\caption{Summary of Identified Risks}
\end{table}

% --- Section 6: Recommendations ---
\section{Recommendations}

The following actions are recommended to address the identified risks and improve the overall security posture of \textbf{[Organization Name]}. Recommendations are prioritized based on risk severity.

\subsection{RISK-001: Implement MFA for Email (Critical)}
\begin{itemize}
    \item \textbf{Action:} Immediately enable and enforce Multi-Factor Authentication (MFA) for all user email accounts.
    \item \textbf{Justification:} This is the single most effective control to prevent unauthorized access to email, mitigating the risk of business email compromise and phishing-related data breaches.
    \item \textbf{Priority:} Immediate.
\end{itemize}

\subsection{RISK-002: Establish a Security Awareness Program (Critical)}
\begin{itemize}
    \item \textbf{Action:} Develop and implement a mandatory security awareness training program. This program should include initial training for all new hires and at least one annual refresher course for all staff.
    \item \textbf{Justification:} Training empowers employees to become the first line of defense, significantly reducing the organization's susceptibility to social engineering attacks.
    \item \textbf{Priority:} Immediate.
\end{itemize}

\subsection{RISK-003: Develop an Acceptable Use Policy (High)}
\begin{itemize}
    \item \textbf{Action:} Draft, approve, and disseminate an organization-wide Acceptable Use Policy (AUP). Ensure all employees read and acknowledge the policy.
    \item \textbf{Justification:} An AUP establishes clear rules for the use of technology and data, reducing ambiguity and strengthening the organization's legal and security stance.
    \item \textbf{Priority:} High.
\end{itemize}

\subsection{RISK-004: Secure the Exposed SSH Service (Medium)}
\begin{itemize}
    \item \textbf{Action:} Review the business need for external SSH access. If required, implement the following controls:
    \begin{enumerate}
        \item Restrict access to trusted IP addresses using a firewall.
        \item Disable password-based authentication and enforce the use of strong SSH keys.
        \item Implement an intrusion prevention tool like Fail2Ban to block IPs after repeated failed login attempts.
        \item Ensure the SSH server software is patched and up-to-date.
    \end{enumerate}
    \item \textbf{Justification:} These layered controls significantly reduce the attack surface of the exposed service, protecting a critical administrative entry point.
    \item \textbf{Priority:} High.
\end{itemize}

\end{document}
```