```latex
\documentclass[12pt]{article}

% ----------------------------------------------------------------------
% PREAMBLE
% ----------------------------------------------------------------------
\usepackage[margin=1in]{geometry}
\usepackage{pifont} % For checkmarks and crosses
\usepackage{booktabs} % For professional tables
\usepackage{hyperref} % For hyperlinks
\usepackage{url}      % For URL formatting
\usepackage{seqsplit} % To split long strings without breaking
\usepackage{xcolor}   % For colors

% Define colors for severity
\definecolor{criticalred}{HTML}{D73027}
\definecolor{highorange}{HTML}{FC8D59}
\definecolor{mediumyellow}{HTML}{FEE08B}
\definecolor{lowblue}{HTML}{91BFDB}
\definecolor{infogray}{HTML}{E0E0E0}

% Hyperref Setup
\hypersetup{
    colorlinks=true,
    linkcolor=blue,
    filecolor=magenta,      
    urlcolor=cyan,
    pdftitle={Cybersecurity Posture Report},
    pdfpagemode=FullScreen,
}

% Checkmark and Cross definitions
\newcommand{\cmark}{\ding{51}}
\newcommand{\xmark}{\ding{55}}

% ----------------------------------------------------------------------
% DOCUMENT START
% ----------------------------------------------------------------------
\begin{document}

% ----------------------------------------------------------------------
% TITLE PAGE
% ----------------------------------------------------------------------
\begin{titlepage}
    \centering
    \vspace*{1cm}
    \Huge
    \textbf{Cybersecurity Posture Report}
    \vspace{1.5cm}
    \Large
    Prepared for: \\
    \vspace{0.5cm}
    \textbf{[Organization Name]}
    \vfill
    \Large
    Report Date: \today \\
    Generated by: Cybersecurity Analyst
\end{titlepage}

\tableofcontents
\newpage

% ----------------------------------------------------------------------
% SECTION 1: EXECUTIVE SUMMARY
% ----------------------------------------------------------------------
\section{Executive Summary}

This report provides a comprehensive analysis of the cybersecurity posture for \textbf{[Organization Name]}, based on a review of organizational security controls, an external network scan, and an assessment of pre-existing risks.

The assessment reveals a mixed security posture. The organization has implemented several important foundational controls, including Multi-Factor Authentication (MFA) for computer and sensitive system access, an acceptable use policy, and security training for new hires.

However, two critical gaps were identified that significantly increase the organization's risk profile:
\begin{itemize}
    \item \textbf{Lack of MFA on Email:} The absence of MFA on email accounts represents a critical vulnerability. Email is a primary vector for phishing attacks, and a compromised account could lead to data breaches, financial fraud, and further system compromise.
    \item \textbf{Lack of Annual Security Training:} Without mandatory, recurring security awareness training for all employees, the organization's human firewall weakens over time. Staff may not recognize new and evolving threats, making them more susceptible to social engineering attacks.
\end{itemize}

From a technical perspective, the external network scan identified an open SSH port (22/TCP) on the network perimeter at \texttt{[Target IP]}. While necessary for remote administration, its public exposure is a common target for brute-force and credential-stuffing attacks. This risk is amplified by the lack of MFA on email, as stolen credentials could potentially be used to gain access to this exposed service.

This report concludes with prioritized, actionable recommendations to mitigate these identified risks and strengthen the overall security posture.

% ----------------------------------------------------------------------
% SECTION 2: ORGANIZATIONAL INFORMATION
% ----------------------------------------------------------------------
\section{Organizational Information}

This section details the information provided by the client organization. Due to the anonymized nature of the data provided, placeholders are used where specific details were unavailable.

\begin{table}[h!]
\centering
\begin{tabular}{@{}ll@{}}
\toprule
\textbf{Attribute} & \textbf{Value} \\ \midrule
Organization Name & \textbf{[Organization Name]} \\
Primary Domain & \texttt{[Domain]} \\
External IP Scanned & \texttt{[Client IP]} \\ \bottomrule
\end{tabular}
\caption{Client Organizational Details}
\end{table}

% ----------------------------------------------------------------------
% SECTION 3: SECURITY CONTROL REVIEW
% ----------------------------------------------------------------------
\section{Security Control Review}

The following table summarizes the organization's responses to the security controls questionnaire. Each response is evaluated against industry best practices. "No" answers indicate significant gaps in the security framework.

\begin{table}[h!]
\centering
\begin{tabular}{@{}p{0.6\textwidth}ccp{0.15\textwidth}@{}}
\toprule
\textbf{Control Question} & \textbf{Response} & \textbf{Assessment} \\ \midrule
Do you require MFA to access email? & No & \xmark & \textcolor{criticalred}{\textbf{Critical Gap}} \\
Do you require MFA to log into computers? & Yes & \cmark & Met \\
Do you require MFA to access sensitive data systems? & Yes & \cmark & Met \\
Does your organization have an employee acceptable use policy? & Yes & \cmark & Met \\
Does your organization do security awareness training for new employees? & Yes & \cmark & Met \\
Does your organization do security awareness training for all employees at least once per year? & No & \xmark & \textcolor{highorange}{\textbf{High Risk}} \\ \bottomrule
\end{tabular}
\caption{Security Controls Questionnaire Analysis}
\end{table}

% ----------------------------------------------------------------------
% SECTION 4: TECHNICAL SCAN RESULTS
% ----------------------------------------------------------------------
\section{Technical Scan Results}

An external network vulnerability scan was performed against the target IP address. The scan identified the following open ports and services accessible from the public internet.

\begin{itemize}
    \item \textbf{Target IP:} \texttt{[Target IP]}
    \item \textbf{Scan Date:} Not provided in scan data. Report generated on \today.
\end{itemize}

\begin{table}[h!]
\centering
\begin{tabular}{@{}llll@{}}
\toprule
\textbf{Port} & \textbf{State} & \textbf{Service (Inferred)} & \textbf{Notes} \\ \midrule
22/tcp & open & SSH & Secure Shell is used for remote administration. \\
& & & Exposing this port to the internet is a risk. \\
& & & It is a common target for automated brute-force attacks. \\
& & & No version information was available from the scan. \\ \bottomrule
\end{tabular}
\caption{Open Ports Detected on \texttt{[Target IP]}}
\end{table}

% ----------------------------------------------------------------------
% SECTION 5: CONSOLIDATED RISK ASSESSMENT
% ----------------------------------------------------------------------
\section{Consolidated Risk Assessment}

This section synthesizes findings from the control review and technical scan into a consolidated list of identified risks. The pre-existing risk register was empty, so all risks listed below are new findings from this assessment.

\begin{table}[h!]
\centering
\begin{tabular}{@{}p{0.05\textwidth}p{0.4\textwidth}p{0.15\textwidth}p{0.25\textwidth}@{}}
\toprule
\textbf{ID} & \textbf{Risk Description} & \textbf{Severity} & \textbf{Affected Asset(s)} \\ \midrule
\textbf{R-01} & \textbf{Lack of MFA on Email Accounts.} A threat actor can gain full access to an employee's mailbox with only a password, leading to data exfiltration, phishing, and business email compromise. & \textcolor{criticalred}{Critical} & Email System, Sensitive Data, User Credentials \\
\addlinespace
\textbf{R-02} & \textbf{Lack of Annual Security Awareness Training.} Employees are not kept up-to-date on evolving threats, increasing the likelihood of successful phishing and social engineering attacks. & \textcolor{highorange}{High} & All Employees, All Systems \\
\addlinespace
\textbf{R-03} & \textbf{Exposed SSH Management Port.} The SSH service on \texttt{[Target IP]} is open to the public internet, making it a target for brute-force and credential stuffing attacks. & \textcolor{highorange}{High} & Network Infrastructure, Server Hosting SSH \\ \bottomrule
\end{tabular}
\caption{Summary of Identified Risks}
\end{table}

% ----------------------------------------------------------------------
% SECTION 6: RECOMMENDATIONS
% ----------------------------------------------------------------------
\section{Recommendations}

The following actionable recommendations are provided to mitigate the identified risks and improve the organization's overall security posture.

\subsection{Immediate Priority (Critical Risk)}
\begin{description}
    \item[Recommendation for R-01 (Lack of MFA on Email):] \hfill \\
    Immediately enforce Multi-Factor Authentication (MFA) for all user access to the email system. Prioritize implementation for accounts with administrative privileges and those in sensitive roles (e.g., finance, HR, executive leadership).
\end{description}

\subsection{High Priority}
\begin{description}
    \item[Recommendation for R-02 (Lack of Annual Training):] \hfill \\
    Establish a mandatory, recurring security awareness training program for all employees, to be completed at least annually. The training should cover current threats such as phishing, ransomware, and social engineering.
    
    \item[Recommendation for R-03 (Exposed SSH Port):] \hfill \\
    Review the business requirement for exposing the SSH service on \texttt{[Target IP]} to the internet.
    \begin{itemize}
        \item If it is not required, block access at the firewall.
        \item If it is required, implement compensating controls such as:
            \begin{itemize}
                \item Restricting access to a whitelist of trusted IP addresses.
                \item Disabling password-based authentication in favor of public key cryptography.
                \item Implementing an intrusion prevention tool like \texttt{fail2ban} to block malicious login attempts.
            \end{itemize}
    \end{itemize}
\end{description}

% ----------------------------------------------------------------------
% DOCUMENT END
% ----------------------------------------------------------------------
\end{document}
```