```latex
\documentclass[12pt]{article}

% Preamble: Required Packages
\usepackage[margin=1in]{geometry}
\usepackage{pifont} % For checkmarks and crosses
\usepackage{booktabs} % For professional tables
\usepackage{hyperref} % For clickable links
\usepackage{url} % For URL formatting
\usepackage{seqsplit} % For splitting long strings
\usepackage{graphicx}
\usepackage{xcolor}

% Document Metadata
\title{Cybersecurity Posture Assessment Report}
\author{Cybersecurity Analysis Division}
\date{\today}

% Hyperref Setup
\hypersetup{
    colorlinks=true,
    linkcolor=blue,
    filecolor=magenta,      
    urlcolor=cyan,
    pdftitle={Cybersecurity Posture Assessment Report},
    pdfpagemode=FullScreen,
}

\begin{document}

\maketitle
\thispagestyle{empty}
\newpage

\tableofcontents
\newpage

% --- SECTION 1: EXECUTIVE SUMMARY ---
\section{Executive Summary}

This report provides a comprehensive cybersecurity posture assessment for \textbf{[Organization Name]}. The analysis is based on a correlation of external network scan data, a review of internal security controls via a questionnaire, and an evaluation of previously identified risks.

The assessment reveals critical gaps in access control policies, which present a significant risk to the organization. Specifically, the absence of Multi-Factor Authentication (MFA) for email and sensitive data systems exposes the organization to account compromise and data breach threats. Furthermore, the lack of a formal Acceptable Use Policy (AUP) creates ambiguity regarding employee responsibilities for protecting information assets.

On a positive note, the external network scan of the target IP address \texttt{[Target IP]} showed no open ports. This finding contradicts a pre-existing risk concerning an unencrypted web server on Port 80. This suggests that the previously identified vulnerability may have been successfully remediated.

Immediate action should be focused on implementing MFA across all critical systems and establishing a formal AUP to strengthen the organization's overall security posture.

% --- SECTION 2: ORGANIZATIONAL INFORMATION ---
\section{Organizational Information}

This section details the organizational information used as the basis for this assessment. The data has been anonymized as per the engagement protocol.

\begin{table}[h!]
\centering
\caption{Client Organizational Details}
\begin{tabular}{@{}ll@{}}
\toprule
\textbf{Attribute} & \textbf{Value} \\ \midrule
Organization Name & \textbf{[Organization Name]} \\
Primary Email Domain & \texttt{[Domain]} \\
External IP Address Scanned & \texttt{[Client IP]} \\ \bottomrule
\end{tabular}
\end{table}

% --- SECTION 3: SECURITY CONTROL REVIEW ---
\section{Security Control Review (Questionnaire Analysis)}

The following table summarizes the organization's responses to a security controls questionnaire. A \textcolor{red}{\ding{55}} indicates a potential control gap that increases risk, while a \textcolor{green}{\ding{51}} indicates a security best practice is in place.

\begin{table}[h!]
\centering
\caption{Security Controls Questionnaire Results}
\begin{tabular}{@{}p{0.6\textwidth}cc@{}}
\toprule
\textbf{Control Question} & \textbf{Response} & \textbf{Status} \\ \midrule
Do you require MFA to access email? & No & \textcolor{red}{\ding{55}} \\
Do you require MFA to log into computers? & Yes & \textcolor{green}{\ding{51}} \\
Do you require MFA to access sensitive data systems? & No & \textcolor{red}{\ding{55}} \\
Does your organization have an employee acceptable use policy? & No & \textcolor{red}{\ding{55}} \\
Does your organization do security awareness training for new employees? & Yes & \textcolor{green}{\ding{51}} \\
Does your organization do security awareness training for all employees at least once per year? & Yes & \textcolor{green}{\ding{51}} \\ \bottomrule
\end{tabular}
\end{table}

\subsection*{Analysis of Control Gaps}
The questionnaire identified three significant control gaps:
\begin{itemize}
    \item \textbf{No MFA for Email:} Email is a primary target for phishing and business email compromise (BEC) attacks. The lack of MFA is a critical vulnerability.
    \item \textbf{No MFA for Sensitive Data:} Failure to protect sensitive data systems with MFA significantly increases the risk of a data breach from compromised credentials.
    \item \textbf{No Acceptable Use Policy (AUP):} An AUP is a foundational policy that sets clear expectations for employees on how to use company assets securely. Its absence can lead to inconsistent security practices and insider threats.
\end{itemize}

% --- SECTION 4: TECHNICAL SCAN RESULTS ---
\section{Technical Scan Results}

An external network scan was performed using Nmap to identify open ports and exposed services on the perimeter.

\begin{itemize}
    \item \textbf{Target IP Address:} \texttt{[Target IP]}
    \item \textbf{Scan Status:} The target host was responsive (status: up).
    \item \textbf{Findings:} The scan revealed \textbf{no open ports}. Port 80 (HTTP), which is often exposed, was explicitly identified as being in a \textbf{closed} state.
\end{itemize}

\subsection*{Analysis of Technical Findings}
The external network perimeter of the scanned target appears to be well-hardened, with no services exposed to the public internet. This is a strong security posture from a network perspective.

Notably, this scan result contradicts a previously documented risk (see Section 5) indicating that Port 80 was open. This suggests that the prior risk has been remediated or was a false positive.

% --- SECTION 5: OVERALL RISK ASSESSMENT ---
\section{Overall Risk Assessment}

This section synthesizes findings from the security control review, technical scan, and pre-existing risk register into a consolidated list of current risks.

\begin{table}[h!]
\centering
\caption{Consolidated Risk Summary}
\begin{tabular}{@{}p{0.3\textwidth}p{0.5\textwidth}l@{}}
\toprule
\textbf{Risk Name} & \textbf{Description} & \textbf{Severity} \\ \midrule
\textbf{Lack of MFA on Email Accounts} & User email accounts are protected only by passwords, making them highly vulnerable to phishing, credential stuffing, and account takeover. & \textbf{Critical} \\
\addlinespace
\textbf{Lack of MFA on Sensitive Data Systems} & Critical data systems lack a secondary authentication factor, exposing sensitive information to unauthorized access if credentials are compromised. & \textbf{Critical} \\
\addlinespace
\textbf{Missing Employee Acceptable Use Policy} & The absence of a formal AUP creates inconsistent security behavior and a lack of accountability for misuse of company IT assets. & \textbf{High} \\
\addlinespace
Unencrypted Web Server (Previously Identified) & A pre-existing risk stated that Port 80 was open. \textit{Our current scan shows this port is closed, indicating this risk is likely remediated.} & Medium \\ \bottomrule
\end{tabular}
\end{table}

% --- SECTION 6: RECOMMENDATIONS ---
\section{Recommendations}

The following actionable recommendations are provided to mitigate the identified risks and improve the overall security posture of \textbf{[Organization Name]}.

\subsection*{Priority 1: Critical Risks}
\begin{enumerate}
    \item \textbf{Implement MFA for Email:} Immediately enforce MFA for all user mailboxes. This is the single most effective control to prevent business email compromise.
    \begin{itemize}
        \item \textbf{Easy Fix:} Enable MFA for all administrator and executive accounts first, then roll out to all users within 30 days.
        \item \textbf{Long-Term Fix:} Integrate MFA into the standard employee onboarding process.
    \end{itemize}
    \item \textbf{Enforce MFA on Sensitive Systems:} Deploy MFA on all applications and systems that store, process, or transmit sensitive organizational or customer data.
    \begin{itemize}
        \item \textbf{Easy Fix:} Prioritize systems accessible from the internet or containing financial, PII, or intellectual property data.
        \item \textbf{Long-Term Fix:} Develop a data classification policy to ensure all systems are protected with access controls appropriate to the data they handle.
    \end{itemize}
\end{enumerate}

\subsection*{Priority 2: High Risks}
\begin{enumerate}
    \setcounter{enumi}{2} % Continue numbering
    \item \textbf{Develop and Implement an Acceptable Use Policy (AUP):}
    \begin{itemize}
        \item \textbf{Easy Fix:} Create a formal AUP document that outlines the rules for using company networks, computers, and data. Require all employees to read and acknowledge the policy.
        \item \textbf{Long-Term Fix:} Incorporate the AUP into the new employee onboarding process and conduct annual reviews to ensure it remains current.
    \end{itemize}
\end{enumerate}

\subsection*{Priority 3: Informational / Verification}
\begin{enumerate}
    \setcounter{enumi}{3} % Continue numbering
    \item \textbf{Verify and Close "Unencrypted Web Server" Risk:}
    \begin{itemize}
        \item \textbf{Action:} Confirm with the system administrator that Port 80 was intentionally closed on \texttt{[Target IP]}. Once confirmed, formally update the internal risk register to reflect that this vulnerability has been remediated.
    \end{itemize}
\end{enumerate}

\end{document}
```