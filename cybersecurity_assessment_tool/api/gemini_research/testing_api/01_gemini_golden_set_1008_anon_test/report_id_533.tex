An analysis of the provided data has been completed. The following is the complete and professional LaTeX report synthesizing the network scan, organizational data, and known risks.

```latex
\documentclass[12pt, a4paper]{article}

% Preamble: Required Packages
\usepackage[margin=1in]{geometry}
\usepackage{pifont} % For checkmarks and crosses
\usepackage{booktabs} % For professional tables
\usepackage{hyperref} % For hyperlinks
\usepackage{url} % For URL formatting
\usepackage{seqsplit} % To split long strings in tt font
\usepackage{xcolor} % For colors
\usepackage{graphicx} % For logo (placeholder)
\usepackage{fancyhdr} % For header/footer

% --- Document Setup ---

% Color definitions for risk levels
\definecolor{criticalrisk}{HTML}{D73027}
\definecolor{highrisk}{HTML}{F46D43}
\definecolor{mediumrisk}{HTML}{FEE08B}
\definecolor{lowrisk}{HTML}{ABDDA4}
\definecolor{infoblue}{HTML}{4575B4}

% Hyperref setup
\hypersetup{
    colorlinks=true,
    linkcolor=infoblue,
    filecolor=magenta,      
    urlcolor=infoblue,
    pdftitle={Cybersecurity Risk Assessment Report},
    pdfpagemode=FullScreen,
}

% Header and Footer
\pagestyle{fancy}
\fancyhf{} % clear all header and footer fields
\fancyhead[L]{Cybersecurity Risk Assessment Report}
\fancyhead[R]{\textbf{[Organization Name]}}
\fancyfoot[C]{\thepage}
\renewcommand{\headrulewidth}{0.4pt}
\renewcommand{\footrulewidth}{0.4pt}

% --- Document Start ---

\begin{document}

% --- Title Page ---
\begin{titlepage}
    \centering
    \vspace*{1cm}
    
    \Huge
    \textbf{Cybersecurity Risk Assessment Report}
    
    \vspace{1.5cm}
    
    \Large
    Prepared for: \textbf{[Organization Name]}
    
    \vspace{2cm}
    
    \normalsize
    \textbf{Author:} Cybersecurity Analyst\\
    \textbf{Date:} \today
    
    \vfill
    
    \small
    \textit{This report contains sensitive information and should be handled with care. Distribution is restricted to authorized personnel only.}
    
\end{titlepage}

\tableofcontents
\newpage

% --- Section 1: Executive Summary ---
\section{Executive Summary}

This report provides a comprehensive analysis of the current cybersecurity posture of \textbf{[Organization Name]}, based on technical network scans, a review of existing security controls, and pre-identified risks. The assessment was conducted to identify vulnerabilities, evaluate policy and procedure gaps, and provide actionable recommendations to enhance security.

The analysis confirmed a \textbf{critical risk}: the direct exposure of Remote Desktop Protocol (RDP) on port 3389 to the public internet. This vulnerability is a primary target for ransomware attacks and unauthorized access attempts and requires immediate remediation.

Additionally, a \textbf{high-risk governance gap} was identified: the organization lacks a formal Employee Acceptable Use Policy (AUP). This absence creates ambiguity regarding the proper use of corporate assets and data, increasing the likelihood of insider threats and accidental data breaches.

On a positive note, the organization has implemented strong controls regarding Multi-Factor Authentication (MFA) for email, computer logins, and access to sensitive systems. This significantly strengthens defenses against credential-based attacks.

Overall, the security posture is considered high-risk due to the critical network exposure. The recommendations provided in this report prioritize the immediate mitigation of this threat, followed by the implementation of foundational security policies.

% --- Section 2: Organizational Information ---
\section{Organizational Information}

The following details were used as the basis for this assessment. Due to the anonymized nature of the input data, placeholders are used where necessary.

\begin{itemize}
    \item \textbf{Organization Name:} \textbf{[Organization Name]}
    \item \textbf{Primary Domain:} \texttt{[Domain]}
    \item \textbf{External IP Scanned:} \texttt{[Client IP]}
\end{itemize}

% --- Section 3: Security Control Review ---
\section{Security Control Review}

A review of organizational security controls was conducted via a questionnaire. The responses indicate a strong foundation in identity and access management but highlight a critical gap in policy governance.

\begin{table}[h!]
\centering
\caption{Security Controls Questionnaire Results}
\begin{tabular}{p{0.6\linewidth} c l}
\toprule
\textbf{Control Question} & \textbf{Response} & \textbf{Assessment} \\
\midrule
Do you require MFA to access email? & \textcolor{green}{\ding{51}} & Strong Control \\
Do you require MFA to log into computers? & \textcolor{green}{\ding{51}} & Strong Control \\
Do you require MFA to access sensitive data systems? & \textcolor{green}{\ding{51}} & Strong Control \\
\addlinespace
Does your organization have an employee acceptable use policy? & \textcolor{red}{\ding{55}} & \textbf{Critical Policy Gap} \\
\addlinespace
Does your organization do security awareness training for new employees? & \textcolor{green}{\ding{51}} & Good Practice \\
Does your organization do security awareness training for all employees at least once per year? & \textcolor{green}{\ding{51}} & Good Practice \\
\bottomrule
\end{tabular}
\end{table}

% --- Section 4: Technical Scan Results ---
\section{Technical Scan Results}

A network scan was performed to identify open ports and exposed services on the organization's external infrastructure.

\subsection{Nmap Scan Details}
\begin{itemize}
    \item \textbf{Target IP Address:} \texttt{[Target IP]}
    \item \textbf{Scan Date:} \today
\end{itemize}

The scan revealed the following open port:

\begin{table}[h!]
\centering
\caption{Open Port Findings}
\begin{tabular}{l l l l}
\toprule
\textbf{Port} & \textbf{State} & \textbf{Service} & \textbf{Product/Version} \\
\midrule
3389/tcp & open & ms-wbt-server & Not Disclosed \\
\bottomrule
\end{tabular}
\end{table}

\subsection{Analysis of Findings}
The scan confirms that port \textbf{3389/tcp} is open. This port is used for Microsoft's Remote Desktop Protocol (RDP). Exposing RDP directly to the public internet is an extremely high-risk configuration. It allows attackers to perform brute-force password attacks, exploit unpatched RDP vulnerabilities (such as BlueKeep), and gain direct administrative access to the internal network. This finding is a common precursor to significant security incidents, including ransomware deployment.

% --- Section 5: Consolidated Risk Assessment ---
\section{Consolidated Risk Assessment}
This section synthesizes findings from the technical scan, control review, and pre-existing risk data into a consolidated list.

\begin{table}[h!]
\centering
\caption{Summary of Identified Risks}
\begin{tabular}{p{0.25\linewidth} p{0.4\linewidth} l p{0.15\linewidth}}
\toprule
\textbf{Risk Name} & \textbf{Description} & \textbf{Severity} & \textbf{Affected Asset(s)} \\
\midrule
\addlinespace
Exposed Remote Desktop Protocol (RDP) & Port 3389 is open to the internet, creating a direct vector for unauthorized access and ransomware. & \colorbox{criticalrisk}{\color{white} \textbf{Critical (9.0)}} & \texttt{[Target IP]} \\
\addlinespace
Lack of Acceptable Use Policy & The absence of a formal AUP increases insider threat risks and potential for misuse of IT assets. & \colorbox{highrisk}{\color{white} \textbf{High}} & Organization-wide \\
\addlinespace
\bottomrule
\end{tabular}
\end{table}

% --- Section 6: Recommendations ---
\section{Recommendations}
The following actions are recommended to mitigate the identified risks. Recommendations are prioritized based on severity.

\subsection{Critical Risk: Exposed RDP}
This finding requires immediate attention to prevent a potential network compromise.

\begin{itemize}
    \item \textbf{Immediate Action (Easy Fix):} Block all inbound traffic to port 3389 on the external firewall for IP address \texttt{[Target IP]}. This will immediately remove the public exposure.
    \item \textbf{Long-Term Solution:} If remote access to this system is necessary, implement a secure Virtual Private Network (VPN) solution. Configure the VPN to require Multi-Factor Authentication (MFA) for all users. RDP access should only be permitted over the secure VPN tunnel.
\end{itemize}

\subsection{High Risk: No Acceptable Use Policy}
Establishing clear policies is fundamental to a strong security program.

\begin{itemize}
    \item \textbf{Action:} Develop, approve, and implement a comprehensive Employee Acceptable Use Policy (AUP).
    \item \textbf{Details:} This policy should, at a minimum, define rules for:
        \begin{itemize}
            \item Data handling and classification.
            \item Proper use of email and internet.
            \item Use of company-owned devices.
            \item Prohibited activities.
            \item Consequences for policy violations.
        \end{itemize}
    \item \textbf{Implementation:} Ensure all current employees read and formally acknowledge the new policy. Integrate the AUP into the new-hire onboarding process and reference it during annual security awareness training.
\end{itemize}

\end{document}
```