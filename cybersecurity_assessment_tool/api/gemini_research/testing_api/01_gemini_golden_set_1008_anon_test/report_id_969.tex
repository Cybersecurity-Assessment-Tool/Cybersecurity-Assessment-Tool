```latex
\documentclass[12pt]{article}

% Preamble: Required Packages
\usepackage[margin=1in]{geometry} % Sets page margins
\usepackage{pifont}               % Provides access to dingbats like checkmarks and crosses
\usepackage{booktabs}             % For professional-looking tables
\usepackage{hyperref}             % For hyperlinks, metadata, and better PDF navigation
\usepackage{url}                  % For formatting URLs
\usepackage{seqsplit}             % For splitting long strings in \texttt
\usepackage{graphicx}             % For including logos (optional but good practice)
\usepackage{xcolor}               % For custom colors

% --- Document Metadata ---
\hypersetup{
    colorlinks=true,
    linkcolor=blue,
    filecolor=magenta,      
    urlcolor=cyan,
    pdftitle={Cybersecurity Posture Assessment Report},
    pdfauthor={Cybersecurity Analyst},
    pdfsubject={Security Analysis},
    pdfkeywords={Cybersecurity, Nmap, Risk Assessment},
    pdftoolbar=true,
}

% --- Custom Commands for Status Indicators ---
\newcommand{\cmark}{\ding{51}} % Checkmark
\newcommand{\xmark}{\ding{55}} % X-mark

\begin{document}

% --- Title Page ---
\begin{titlepage}
    \centering
    \vspace*{1cm}
    \Huge\textbf{Cybersecurity Posture Assessment Report}
    \vspace{1.5cm}
    \Large
    \textbf{Prepared for:}\\
    \vspace{0.5cm}
    \textbf{[Organization Name]}
    \vspace{2.5cm}
    
    \begin{center}
        \rule{\linewidth}{0.5mm}
        \vspace{0.5cm}
        \Large \textbf{CONFIDENTIAL}
        \vspace{0.5cm}
        \rule{\linewidth}{0.5mm}
    \end{center}
    
    \vfill
    
    \large
    \textbf{Date of Report:} \today \\
    \textbf{Author:} Cybersecurity Analyst
\end{titlepage}

\tableofcontents
\newpage

% --- Section 1: Executive Summary ---
\section*{1. Executive Summary}
This report provides a comprehensive cybersecurity assessment for \textbf{[Organization Name]}, synthesizing data from a network vulnerability scan, a security controls questionnaire, and a review of existing risk documentation.

The analysis reveals a mixed security posture. While the organization has implemented critical controls such as Multi-Factor Authentication (MFA) for email and sensitive data systems, significant and high-risk gaps were identified. 

Most critically, a network scan performed on the external IP address \texttt{[Client IP]} revealed an open service on port 8080 with the title \textbf{"TOP SECRET DB"}. This finding directly contradicts the existing risk documentation, which incorrectly classifies this port as a secure false positive. This represents a severe information disclosure and a potential vector for a data breach.

Furthermore, critical administrative and technical controls are missing. The absence of MFA for computer logins and the lack of a formal Employee Acceptable Use Policy create substantial risks that could be exploited by threat actors.

This report outlines these findings in detail and provides prioritized, actionable recommendations to mitigate the identified risks and strengthen the organization's overall security posture.

% --- Section 2: Organizational Information ---
\section*{2. Organizational Information}
This section provides high-level details about the organization based on the provided data.
\begin{itemize}
    \item \textbf{Organization Name:} \textbf{[Organization Name]}
    \item \textbf{Primary Domain:} \texttt{[Domain]}
    \item \textbf{External IP Scanned:} \texttt{[Client IP]}
\end{itemize}

% --- Section 3: Security Control Review ---
\section*{3. Security Control Review}
The following table summarizes the organization's responses to a security controls questionnaire. Items marked with an \xmark\ represent significant gaps in the security framework and are discussed in the Risk Assessment section.

\begin{table}[h!]
\centering
\caption{Security Controls Questionnaire Analysis}
\begin{tabular}{p{0.7\linewidth} c}
\toprule
\textbf{Control Question} & \textbf{Status} \\
\midrule
Do you require MFA to access email? & \cmark \\
Do you require MFA to log into computers? & \textcolor{red}{\xmark} \\
Do you require MFA to access sensitive data systems? & \cmark \\
Does your organization have an employee acceptable use policy? & \textcolor{red}{\xmark} \\
Does your organization do security awareness training for new employees? & \cmark \\
Does your organization do security awareness training for all employees at least once per year? & \cmark \\
\bottomrule
\end{tabular}
\end{table}

% --- Section 4: Technical Scan Results ---
\section*{4. Technical Scan Results}
An external network scan was conducted on the target IP address. The results indicate the presence of an open port exposing a potentially sensitive service.

\subsection*{4.1. Scan Details}
\begin{itemize}
    \item \textbf{Target IP:} \texttt{[Target IP]} (Derived from \texttt{[Client IP]})
    \item \textbf{Scan Tool:} Nmap
\end{itemize}

\subsection*{4.2. Open Ports and Services}
A single open port was identified during the scan. The details are highly concerning due to the service's descriptive title.

\begin{table}[h!]
\centering
\caption{Identified Open Ports}
\begin{tabular}{l l l p{0.5\linewidth}}
\toprule
\textbf{Port} & \textbf{State} & \textbf{Service} & \textbf{Details / Banner} \\
\midrule
8080/tcp & Open & http (inferred) & \textbf{HTTP Title:} \seqsplit{\texttt{TOP SECRET DB}} \\
\bottomrule
\end{tabular}
\end{table}

\textbf{Analysis:} The title "TOP SECRET DB" is a critical information disclosure vulnerability. It makes the service a high-value target for attackers and suggests that a sensitive database may be directly accessible from the public internet. \textbf{This finding directly contradicts the existing risk documentation (Input 3), which incorrectly states this port is secure.}

% --- Section 5: Risk Assessment & Correlation ---
\section*{5. Risk Assessment \& Correlation}
This section correlates the findings from the security control review and the technical scan to provide a synthesized view of the top risks facing the organization.

\begin{table}[h!]
\centering
\caption{Synthesized Risk Summary}
\begin{tabular}{p{0.25\linewidth} p{0.15\linewidth} p{0.5\linewidth}}
\toprule
\textbf{Risk Name} & \textbf{Severity} & \textbf{Description \& Correlation} \\
\midrule
\textbf{Exposed Sensitive Database Interface} & \textbf{Critical} & The service on port 8080 is titled "TOP SECRET DB". This contradicts the existing risk register and indicates a high probability of sensitive data exposure. This is the most urgent finding. \\
\addlinespace
\textbf{Lack of Endpoint MFA} & \textbf{High} & The "No" response for MFA on computer logins means that a single compromised password could grant an attacker full access to an employee's workstation, from which they could access internal resources, including the exposed database. \\
\addlinespace
\textbf{Missing Acceptable Use Policy (AUP)} & \textbf{Medium} & The absence of a formal AUP creates ambiguity regarding security responsibilities. It weakens the organization's ability to enforce security standards and complicates disciplinary action in case of an insider threat or accidental misconfiguration. \\
\bottomrule
\end{tabular}
\end{table}

% --- Section 6: Recommendations ---
\section*{6. Recommendations}
The following actionable recommendations are provided to address the identified risks. They are prioritized based on severity.

\subsection*{6.1. Immediate Actions (Critical Risk)}
\begin{enumerate}
    \item \textbf{Investigate and Remediate Port 8080:}
    \begin{itemize}
        \item Immediately investigate the service running on port 8080 on \texttt{[Target IP]}.
        \item If the service is business-critical and contains sensitive data, restrict access immediately via firewall rules to allow only authorized IP addresses.
        \item If the service is not intended for public access, shut it down or move it to an internal, non-routable network.
        \item Regardless of function, change the service's title to remove any descriptive or sensitive information.
        \item \textbf{Update the risk register to reflect the true state of this vulnerability.}
    \end{itemize}
\end{enumerate}

\subsection*{6.2. High-Priority Actions}
\begin{enumerate}
    \setcounter{enumi}{1} % Continue numbering
    \item \textbf{Implement Endpoint MFA:}
    \begin{itemize}
        \item Deploy a robust MFA solution (e.g., Windows Hello for Business, Duo, Okta) for all employee computer logins.
        \item Prioritize deployment for privileged users, such as system administrators and executives, before rolling out to the entire organization.
    \end{itemize}
\end{enumerate}

\subsection*{6.3. Medium-Priority Actions}
\begin{enumerate}
    \setcounter{enumi}{2} % Continue numbering
    \item \textbf{Develop and Implement an Acceptable Use Policy (AUP):}
    \begin{itemize}
        \item Draft a formal AUP that clearly defines rules for the use of company technology, data handling, and security responsibilities.
        \item Require all current and new employees to read and formally acknowledge the policy as a condition of their employment.
        \item Integrate the AUP into the security awareness training program.
    \end{itemize}
\end{enumerate}

\end{document}
```