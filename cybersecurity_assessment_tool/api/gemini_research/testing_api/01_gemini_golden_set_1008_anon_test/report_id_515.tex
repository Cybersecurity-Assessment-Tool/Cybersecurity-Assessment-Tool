```latex
\documentclass[12pt]{article}

% --- PACKAGE IMPORTS ---
\usepackage[margin=1in]{geometry}
\usepackage{pifont} % For checkmarks and crosses
\usepackage{booktabs} % For professional tables
\usepackage{graphicx}
\usepackage{xcolor}
\usepackage{hyperref}
\usepackage{url}
\usepackage{seqsplit} % For splitting long strings in tt font

% --- DOCUMENT CONFIGURATION ---
\definecolor{darkblue}{rgb}{0.0, 0.0, 0.55}
\definecolor{darkred}{rgb}{0.55, 0.0, 0.0}

\hypersetup{
    colorlinks=true,
    linkcolor=darkblue,
    filecolor=darkblue,      
    urlcolor=darkblue,
    citecolor=darkblue,
}

% --- TITLE INFORMATION ---
\title{Cybersecurity Posture Assessment Report}
\author{Cybersecurity Analyst}
\date{\today}

% --- DOCUMENT BEGIN ---
\begin{document}

\maketitle
\thispagestyle{empty}
\newpage

\tableofcontents
\thispagestyle{empty}
\newpage

% ==============================================================================
\section{Executive Summary}
% ==============================================================================

This report provides a comprehensive analysis of the cybersecurity posture for \textbf{[Organization Name]}. The assessment is based on a synthesis of network scan data, a review of organizational security controls, and an evaluation of pre-existing risks.

The overall security posture is determined to be \textbf{critically weak} and requires immediate and decisive action. Several high-impact deficiencies were identified that expose the organization to significant threats, including data breaches, account compromise, and service disruption.

\textbf{Key Findings Include:}
\begin{itemize}
    \item \textbf{Critical Pre-existing Vulnerability:} A risk identified as "Localhost Exposed" with a CVSS score of 10.0 (Critical) is present. This represents a severe and immediate threat that must be prioritized for remediation.
    \item \textbf{Lack of Email Multi-Factor Authentication (MFA):} The absence of MFA on email accounts is a critical control gap. This significantly increases the risk of Business Email Compromise (BEC), phishing success, and subsequent unauthorized access to sensitive data.
    \item \textbf{Inadequate Foundational Policies:} The organization lacks a formal Acceptable Use Policy and a structured security awareness training program. These foundational elements are essential for establishing a security-conscious culture and mitigating human-related risks.
    \item \textbf{Exposed Network Service:} The external network scan identified an open SSH port (22), a common target for brute-force and credential stuffing attacks.
\end{itemize}

Urgent remediation of these issues is paramount. Recommendations provided in Section \ref{sec:recommendations} are prioritized to address the most critical risks first.

\newpage

% ==============================================================================
\section{Organizational Information}
% ==============================================================================

The following information was used as the basis for this assessment. Due to the anonymized nature of the provided data, placeholders have been used where necessary.

\begin{table}[h!]
\centering
\begin{tabular}{@{}ll@{}}
\toprule
\textbf{Attribute} & \textbf{Value} \\ \midrule
Organization Name & \textbf{[Organization Name]} \\
Primary Email Domain & \seqsplit{\texttt{[Domain]}} \\
External IP Address Scanned & \texttt{[Client IP]} \\
Assessment Date & \today \\ \bottomrule
\end{tabular}
\caption{Subject Organization Details}
\end{table}

% ==============================================================================
\section{Security Control Review (Questionnaire Analysis)}
% ==============================================================================

A review of the organization's security controls via a questionnaire revealed significant gaps in foundational security practices. "No" answers indicate a lack of a specific control and are flagged as high-risk deficiencies.

\begin{table}[h!]
\centering
\begin{tabular}{@{}p{0.5\linewidth}ccp{0.25\linewidth}@{}}
\toprule
\textbf{Control Question} & \textbf{Response} & \textbf{Assessment} \\ \midrule
Do you require MFA to access email? & \textcolor{darkred}{\ding{55}} & \textbf{Critical Gap} \\
Do you require MFA to log into computers? & \textcolor{green}{\ding{51}} & Met \\
Do you require MFA to access sensitive data systems? & \textcolor{green}{\ding{51}} & Met \\
Does your organization have an employee acceptable use policy? & \textcolor{darkred}{\ding{55}} & \textbf{High Risk} \\
Does your organization do security awareness training for new employees? & \textcolor{darkred}{\ding{55}} & \textbf{High Risk} \\
Does your organization do security awareness training for all employees at least once per year? & \textcolor{darkred}{\ding{55}} & \textbf{High Risk} \\ \bottomrule
\end{tabular}
\caption{Security Control Questionnaire Results}
\end{table}

% ==============================================================================
\section{Technical Scan Results}
% ==============================================================================

An external network scan was performed to identify exposed services on the organization's public-facing infrastructure.

\begin{itemize}
    \item \textbf{Target IP Address:} \texttt{[Target IP]}
    \item \textbf{Host Status:} Up
\end{itemize}

The following open ports were discovered:

\begin{table}[h!]
\centering
\begin{tabular}{@{}llll@{}}
\toprule
\textbf{Port} & \textbf{State} & \textbf{Service (Presumed)} & \textbf{Notes} \\ \midrule
22/tcp & Open & SSH & Exposing SSH to the internet is a high-risk configuration. \\
& & & It is a primary target for automated brute-force attacks. \\
& & & No service version information was available from this scan. \\ \bottomrule
\end{tabular}
\caption{Open Ports Detected on \texttt{[Target IP]}}
\end{table}

% ==============================================================================
\section{Consolidated Risk Assessment}
% ==============================================================================

The following table synthesizes findings from all data sources into a prioritized list of identified risks.

\begin{table}[h!]
\centering
\begin{tabular}{@{}lp{0.4\linewidth}ll@{}}
\toprule
\textbf{Risk ID} & \textbf{Risk Description} & \textbf{Affected Asset(s)} & \textbf{Severity} \\ \midrule
RISK-001 & Pre-existing "Localhost Exposed" vulnerability with a CVSS score of 10.0. & \texttt{[Target IP]} & \textbf{Critical} \\
RISK-002 & Lack of MFA on email infrastructure, enabling account takeovers. & \texttt{[Domain]} Email System & \textbf{Critical} \\
RISK-003 & Lack of security awareness training program, increasing susceptibility to phishing. & All Employees & \textbf{High} \\
RISK-004 & SSH service exposed to the public internet, inviting brute-force attacks. & \texttt{[Target IP]} & \textbf{High} \\
RISK-005 & Absence of an Acceptable Use Policy, leading to inconsistent security behavior. & All Employees & \textbf{Medium} \\ \bottomrule
\end{tabular}
\caption{Summary of Identified Risks}
\end{table}

% ==============================================================================
\section{Recommendations}
\label{sec:recommendations}
% ==============================================================================

The following actionable recommendations are prioritized to mitigate the identified risks effectively.

\subsection*{Immediate Priority (Remediate within 72 hours)}

\begin{description}
    \item[RISK-001: Remediate "Localhost Exposed" Vulnerability] Investigate the critical vulnerability on \texttt{[Target IP]} immediately. This likely requires applying firewall rules to restrict access to the affected service from the internet, allowing local or VPN-only access. A full vulnerability scan should be conducted to confirm remediation.
    \item[RISK-002: Enforce MFA for Email] Immediately enable and enforce MFA for all user accounts on the \texttt{[Domain]} email system. This is the single most effective control to prevent Business Email Compromise (BEC) and unauthorized account access.
\end{description}

\subsection*{High Priority (Remediate within 30 days)}

\begin{description}
    \item[RISK-004: Harden Exposed SSH Service]
    \begin{itemize}
        \item If SSH access is not required from the internet, block port 22 at the firewall.
        \item If external access is required, restrict access to a whitelist of known, trusted IP addresses.
        \item Disable password-based authentication and enforce the use of public key authentication only.
        \item Disable root login over SSH.
        \item Implement an intrusion prevention tool like Fail2Ban to block IPs that exhibit brute-force behavior.
    \end{itemize}
    \item[RISK-003: Implement Security Awareness Training]
    \begin{itemize}
        \item Procure and deploy a security awareness training platform.
        \item Enroll all new hires in a mandatory introductory training module.
        \item Schedule and enforce annual security training for all existing employees, focusing on phishing, password hygiene, and data handling.
    \end{itemize}
\end{description}

\subsection*{Medium Priority (Remediate within 90 days)}

\begin{description}
    \item[RISK-005: Develop and Enforce Acceptable Use Policy (AUP)] Draft a formal AUP that clearly defines the rules for using company assets, data, and network resources. This policy should be reviewed by legal/HR, distributed to all employees, and require a signed acknowledgment of receipt.
\end{description}

\end{document}
```