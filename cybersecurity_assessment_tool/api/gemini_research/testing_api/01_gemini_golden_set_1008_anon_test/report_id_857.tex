```latex
\documentclass[12pt]{article}

% Preamble: Required Packages
\usepackage[margin=1in]{geometry}
\usepackage{pifont} % For check and cross marks
\usepackage{booktabs} % For professional tables
\usepackage{xcolor} % For colors
\usepackage{hyperref} % For hyperlinks
\usepackage{url} % For URL formatting
\usepackage{seqsplit} % For splitting long strings in tt font

% Hyperref Setup
\hypersetup{
    colorlinks=true,
    linkcolor=blue,
    filecolor=magenta,      
    urlcolor=cyan,
    pdftitle={Cybersecurity Assessment Report},
    pdfpagemode=FullScreen,
}

% Document Metadata
\title{Cybersecurity Assessment Report \\ \large for \textbf{[Organization Name]}}
\author{Cybersecurity Analyst}
\date{\today}

\begin{document}

\maketitle
\hrule
\vspace{1em}

\tableofcontents

\newpage

\section{Executive Summary}

This report details the findings of a cybersecurity assessment conducted for \textbf{[Organization Name]}. The analysis synthesized data from a network scan, a security controls questionnaire, and a list of pre-existing risks.

A \textbf{critical risk} was identified: the direct exposure of a Remote Desktop Protocol (RDP) service on the public internet at \texttt{[Client IP]}. This vulnerability is a primary vector for ransomware attacks and unauthorized access.

This technical flaw is significantly amplified by critical gaps in organizational security controls. The lack of mandatory Multi-Factor Authentication (MFA) for accessing email and sensitive data systems means that a single compromised password could lead to a full system breach. Furthermore, the absence of a formal Acceptable Use Policy indicates a gap in security governance.

Immediate action is required to remediate the exposed RDP service and implement foundational security controls to protect the organization's assets and data.

\section{Organizational Information}

The following details were used as the basis for this assessment. Anonymized placeholders are used where data was not provided.

\begin{table}[h!]
\centering
\begin{tabular}{@{}ll@{}}
\toprule
\textbf{Attribute} & \textbf{Value} \\ \midrule
Organization Name & \textbf{[Organization Name]} \\
Primary Domain & \texttt{[Domain]} \\
External IP Address Assessed & \texttt{[Client IP]} \\ \bottomrule
\end{tabular}
\caption{Client Organizational Details}
\end{table}

\section{Security Control Review}

An assessment of organizational security controls was performed based on a supplied questionnaire. The following table summarizes the responses and provides an analyst's assessment of each control's status. Gaps in these controls often increase the likelihood and impact of technical vulnerabilities.

\begin{table}[h!]
\centering
\begin{tabular}{@{}lcc@{}}
\toprule
\textbf{Control Question} & \textbf{Status} & \textbf{Assessment} \\ \midrule
Do you require MFA to access email? & \ding{55} & \textcolor{red}{\textbf{Critical Gap}} \\
Do you require MFA to log into computers? & \ding{51} & Control in Place \\
Do you require MFA to access sensitive data systems? & \ding{55} & \textcolor{red}{\textbf{Critical Gap}} \\
Does your organization have an employee AUP? & \ding{55} & \textcolor{orange}{High Risk Policy Gap} \\
Security awareness training for new employees? & \ding{51} & Control in Place \\
Annual security awareness training for all employees? & \ding{51} & Control in Place \\ \bottomrule
\end{tabular}
\caption{Security Controls Questionnaire Analysis}
\label{tab:controls}
\end{table}

\section{Technical Scan Results}

An external network scan was performed against the target IP address. The scan identified the following open ports and services.

\begin{itemize}
    \item \textbf{Target IP Address:} \texttt{[Target IP]}
    \item \textbf{Scan Date:} Assumed to be current as of this report.
\end{itemize}

\begin{table}[h!]
\centering
\begin{tabular}{@{}llll@{}}
\toprule
\textbf{Port/Proto} & \textbf{State} & \textbf{Service} & \textbf{Analysis} \\ \midrule
3389/tcp & open & ms-wbt-server & Microsoft Remote Desktop Protocol \\ \bottomrule
\end{tabular}
\caption{Open Ports Detected on \texttt{[Target IP]}}
\label{tab:nmap}
\end{table}

\subsection{Analysis of Findings}
The scan confirms that port 3389/TCP, used for Remote Desktop Protocol (RDP), is open to the public internet. RDP is a common target for brute-force password attacks and exploitation of vulnerabilities. Exposing this service directly to the internet is a critical security risk and is strongly discouraged by cybersecurity best practices. This finding corroborates the pre-existing "RDP Exposure" risk identified in Input 3.

\section{Consolidated Risk Assessment}

The following table correlates findings from the security control review, the technical scan, and pre-existing risk data to provide a consolidated view of the primary risks facing the organization.

\begin{table}[h!]
\centering
\begin{tabular}{@{}p{0.15\linewidth}p{0.5\linewidth}p{0.15\linewidth}@{}}
\toprule
\textbf{Risk ID} & \textbf{Risk Description} & \textbf{Severity} \\ \midrule
\textbf{RISK-001} & \textbf{Publicly Exposed RDP Service:} The RDP service on \texttt{[Client IP]} is accessible from the internet, creating a high risk of unauthorized access and ransomware. & \textbf{Critical (9.0)} \\
\addlinespace
\textbf{RISK-002} & \textbf{No MFA on Email:} Lack of MFA on email makes business email compromise (BEC) and account takeovers trivial if user credentials are stolen. & \textbf{Critical} \\
\addlinespace
\textbf{RISK-003} & \textbf{No MFA on Sensitive Data Systems:} The absence of MFA on critical systems means that a single compromised password could lead to a major data breach. & \textbf{Critical} \\
\addlinespace
\textbf{RISK-004} & \textbf{No Employee Acceptable Use Policy (AUP):} Without a formal AUP, there is no enforceable standard for user behavior, increasing the risk of insider threat and accidental data loss. & \textbf{High} \\ \bottomrule
\end{tabular}
\caption{Summary of Identified Risks}
\label{tab:risks}
\end{table}

\section{Recommendations}

The following prioritized recommendations are provided to mitigate the identified risks.

\subsection{Priority 1: Immediate Remediation (Now)}
\begin{enumerate}
    \item \textbf{Remediate RDP Exposure (RISK-001):} Immediately block all inbound traffic to port 3389/TCP on the external firewall for \texttt{[Client IP]}. If remote access is essential, restrict access to a whitelist of known, trusted source IP addresses as a temporary measure.
\end{enumerate}

\subsection{Priority 2: Short-Term Remediation (Next 30 Days)}
\begin{enumerate}
    \item \textbf{Implement MFA on Critical Systems (RISK-002, RISK-003):} Enforce MFA for all users on all external-facing services, especially email. Develop a plan to roll out MFA to all sensitive internal data systems.
\end{enumerate}

\subsection{Priority 3: Mid-Term Remediation (Next 90 Days)}
\begin{enumerate}
    \item \textbf{Develop and Implement an AUP (RISK-004):} Create a formal Acceptable Use Policy that defines the rules and responsibilities for all employees when using company IT assets. Ensure all employees read and acknowledge the policy.
\end{enumerate}

\subsection{Priority 4: Long-Term Strategy (Next 6 Months)}
\begin{enumerate}
    \item \textbf{Deploy a Secure Remote Access Solution (RISK-001):} Replace direct RDP access with a modern, secure remote access solution, such as a Virtual Private Network (VPN) or a Zero Trust Network Access (ZTNA) gateway. This provides an encrypted, authenticated, and audited channel for all remote administration.
\end{enumerate}

\end{document}
```