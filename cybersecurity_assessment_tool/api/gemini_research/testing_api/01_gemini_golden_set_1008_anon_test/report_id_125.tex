```latex
\documentclass[12pt]{article}

% 1. DOCUMENT SETUP & PACKAGES
\usepackage[margin=1in]{geometry}
\usepackage{pifont} % For checkmarks and crosses
\usepackage{booktabs} % For professional tables
\usepackage{hyperref} % For hyperlinks and PDF metadata
\usepackage{url} % For formatting URLs
\usepackage{seqsplit} % For splitting long strings in texttt
\usepackage{graphicx}
\usepackage{xcolor}

% --- Document Metadata ---
\hypersetup{
    colorlinks=true,
    linkcolor=blue,
    filecolor=magenta,      
    urlcolor=cyan,
    pdftitle={Cybersecurity Assessment Report},
    pdfauthor={Cybersecurity Analyst},
    pdfsubject={Security Posture Analysis},
    pdfkeywords={Security, Analysis, Report},
    bookmarks=true
}

\begin{document}

% 2. TITLE PAGE
\begin{titlepage}
    \centering
    \vspace*{1cm}
    \Huge\textbf{Cybersecurity Assessment Report}
    \vspace{1.5cm}
    \Large
    \textbf{Prepared for:} \\
    \vspace{0.5cm}
    \textbf{[Organization Name]}
    \vspace{2cm}
    \rule{\linewidth}{0.5mm}
    \vspace{0.5cm}
    {\large \today}
    \vspace{0.5cm}
    \rule{\linewidth}{0.5mm}
    \vfill
    \textit{This report contains sensitive information and is intended solely for the designated recipient. Unauthorized distribution is prohibited.}
\end{titlepage}

\tableofcontents
\newpage

% 3. EXECUTIVE SUMMARY
\section{Executive Summary}
This report details the findings of a cybersecurity assessment conducted for \textbf{[Organization Name]}. The analysis correlates data from an external network scan, a security controls questionnaire, and a review of pre-existing risk documentation.

The assessment identified two primary areas of concern that require immediate attention:

\begin{enumerate}
    \item \textbf{Critical External Exposure:} The external network scan confirmed that a Remote Desktop Protocol (RDP) service on port 3389 is publicly accessible at \texttt{[Client IP]}. This exposure represents a critical risk (CVSS 9.0), as it makes the network highly susceptible to brute-force attacks, ransomware, and exploitation of known RDP vulnerabilities. This finding validates a previously documented risk and elevates its urgency.

    \item \textbf{High-Risk Administrative Gap:} The security controls review revealed that the organization does not provide mandatory security awareness training for new employees during their onboarding process. This gap creates a significant vulnerability, as untrained new hires are prime targets for phishing and social engineering attacks, which can lead to credential theft and initial network compromise.
\end{enumerate}

While the organization has implemented strong controls in other areas, such as widespread Multi-Factor Authentication (MFA), the identified risks are severe and could undermine existing security measures. This report provides specific, actionable recommendations to mitigate these threats and improve the overall security posture.

% 4. ORGANIZATIONAL INFORMATION
\section{Organizational Information}
The following details were used as the basis for this assessment. Due to the anonymized nature of the input data, placeholders have been used.

\begin{tabular}{@{}ll}
    \toprule
    \textbf{Attribute} & \textbf{Value} \\
    \midrule
    Organization Name & \textbf{[Organization Name]} \\
    Primary Email Domain & \texttt{[Domain]} \\
    External IP Address Scanned & \texttt{[Client IP]} \\
    \bottomrule
\end{tabular}

% 5. SECURITY CONTROL REVIEW (QUESTIONNAIRE)
\section{Security Control Review}
An administrative review of security controls was conducted via a questionnaire. The responses are summarized below. A checkmark (\ding{51}) indicates a positive control is in place, while a cross (\ding{55}) indicates a control gap.

\begin{table}[h!]
\centering
\begin{tabular}{@{}lc@{}}
\toprule
\textbf{Control Question} & \textbf{Response} \\
\midrule
Do you require MFA to access email? & \ding{51} \\
Do you require MFA to log into computers? & \ding{51} \\
Do you require MFA to access sensitive data systems? & \ding{51} \\
Does your organization have an employee acceptable use policy? & \ding{51} \\
\textbf{Does your organization do security awareness training for new employees?} & \textbf{\color{red}\ding{55}} \\
Does your organization do security awareness training for all employees at least once per year? & \ding{51} \\
\bottomrule
\end{tabular}
\caption{Security Controls Questionnaire Results}
\end{table}

\subsection*{Analysis}
The organization demonstrates a strong commitment to identity and access management through its comprehensive implementation of MFA. However, the lack of mandatory security training for new employees represents a significant process gap. New hires are often unfamiliar with corporate policies and are highly susceptible to targeted attacks. This gap should be addressed to reduce the "human-element" risk.

% 6. TECHNICAL SCAN RESULTS
\section{Technical Scan Results}
An external network scan was performed on the target IP address to identify open ports and exposed services.

\begin{itemize}
    \item \textbf{Target IP Address:} \texttt{[Target IP]}
\end{itemize}

\begin{table}[h!]
\centering
\begin{tabular}{@{}llll@{}}
\toprule
\textbf{Port} & \textbf{State} & \textbf{Service Name} & \textbf{Notes} \\
\midrule
3389/tcp & open & ms-wbt-server & Microsoft Remote Desktop Protocol (RDP) \\
\bottomrule
\end{tabular}
\caption{Open Ports Detected on \texttt{[Target IP]}}
\end{table}

\subsection*{Analysis}
The scan confirms that TCP port 3389 is open, exposing the Microsoft Remote Desktop Protocol (RDP) service directly to the public internet. RDP is a primary target for attackers who use brute-force password guessing, credential stuffing, and exploitation of vulnerabilities (e.g., BlueKeep, DejaBlue) to gain unauthorized access to internal networks. This finding is of critical severity.

% 7. CONSOLIDATED RISK ASSESSMENT
\section{Consolidated Risk Assessment}
The following table synthesizes findings from the technical scan, the controls review, and pre-existing risk data into a prioritized list.

\begin{table}[h!]
\centering
\begin{tabular}{@{}p{0.25\linewidth}p{0.4\linewidth}p{0.1\linewidth}p{0.15\linewidth}@{}}
\toprule
\textbf{Risk Name} & \textbf{Description} & \textbf{Severity} & \textbf{Affected Systems} \\
\midrule
\textbf{Public RDP Exposure} & The Remote Desktop Protocol service on port 3389 is exposed to the internet, allowing attackers to attempt unauthorized access. This finding was confirmed by the technical scan. & \textbf{Critical (9.0)} & \texttt{[Client IP]} \\
\addlinespace
\textbf{Lack of New Employee Security Training} & New employees are not required to undergo security awareness training during onboarding, making them more vulnerable to phishing and social engineering attacks. & \textbf{High} & All New Hires / Human Processes \\
\bottomrule
\end{tabular}
\caption{Summary of Identified Risks}
\end{table}

% 8. RECOMMENDATIONS
\section{Recommendations}
Based on the analysis, the following actions are recommended to mitigate the identified risks. Recommendations are prioritized by severity.

\begin{enumerate}
    \item \textbf{Remediate RDP Exposure (Critical Priority)}
    \begin{itemize}
        \item \textbf{Immediate Action:} Implement a firewall rule to block all inbound traffic to TCP port 3389 on \texttt{[Client IP]} from the public internet. Access should be denied by default.
        \item \textbf{Long-Term Solution:} For necessary remote administration, deploy a secure remote access solution, such as a Virtual Private Network (VPN) or a Zero Trust Network Access (ZTNA) gateway. Access should require Multi-Factor Authentication (MFA).
    \end{itemize}
    \vspace{1em}
    \item \textbf{Implement Onboarding Security Training (High Priority)}
    \begin{itemize}
        \item \textbf{Action:} Develop and mandate a security awareness training module for all new employees as a standard part of the onboarding process.
        \item \textbf{Content:} The training should cover, at a minimum: phishing and spear-phishing recognition, strong password creation and management, acceptable use of company assets, and the process for reporting security incidents.
    \end{itemize}
\end{enumerate}

\end{document}
```