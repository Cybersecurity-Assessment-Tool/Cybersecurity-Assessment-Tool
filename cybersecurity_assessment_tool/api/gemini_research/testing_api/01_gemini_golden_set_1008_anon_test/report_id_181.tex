```latex
\documentclass[12pt]{article}

% ----------------------------------------------------------------------
%  PACKAGES
% ----------------------------------------------------------------------
\usepackage[margin=1in]{geometry}
\usepackage{pifont} % For checkmarks and crosses
\usepackage{booktabs} % For professional tables
\usepackage{hyperref} % For hyperlinks
\usepackage{url} % For URL formatting
\usepackage{seqsplit} % For splitting long strings in tt font
\usepackage[T1]{fontenc}

% ----------------------------------------------------------------------
%  DOCUMENT CONFIGURATION
% ----------------------------------------------------------------------
\hypersetup{
    colorlinks=true,
    linkcolor=black,
    urlcolor=blue,
    pdftitle={Cybersecurity Posture Assessment Report},
    pdfauthor={Cybersecurity Analyst},
    pdfsubject={Security Analysis},
    pdfkeywords={Security, Risk, Assessment, Nmap, Vulnerability}
}

\newcommand{\yes}{\ding{51}}
\newcommand{\no}{\ding{55}}

% ----------------------------------------------------------------------
%  BEGIN DOCUMENT
% ----------------------------------------------------------------------
\begin{document}

\title{
    \textbf{Cybersecurity Posture Assessment Report}\\
    \large \textbf{[Organization Name]}
}
\author{Cybersecurity Analyst}
\date{November 22, 2025}
\maketitle

\hrule
\vspace{1em}
\begin{abstract}
This report provides a comprehensive analysis of the cybersecurity posture of \textbf{[Organization Name]}. The assessment is based on a synthesis of organizational data, a technical network scan, and a review of pre-existing risks. The findings highlight critical security gaps and provide actionable recommendations to mitigate identified vulnerabilities and enhance the overall security framework.
\end{abstract}
\vspace{1em}
\hrule

\tableofcontents
\newpage

% ----------------------------------------------------------------------
%  SECTION 1: EXECUTIVE SUMMARY
% ----------------------------------------------------------------------
\section{Overview and Executive Summary}
This assessment, conducted on November 22, 2025, evaluates the security posture of \textbf{[Organization Name]}. The analysis combines a review of self-reported security controls with a technical scan of the external network perimeter.

The organization has implemented several positive security controls, including Multi-Factor Authentication (MFA) for email and sensitive data systems. However, the assessment identified significant risks that require immediate attention:

\begin{itemize}
    \item \textbf{Critical Control Gaps:} A lack of mandatory MFA for computer logins presents a high risk for unauthorized access and lateral movement. Furthermore, the absence of security awareness training for new employees leaves the organization vulnerable to social engineering and phishing attacks.
    \item \textbf{Technical Vulnerabilities:} The external-facing web server is running an outdated version of Nginx (1.18.0), which is known to have multiple security vulnerabilities. This exposes the organization to potential compromise.
\end{itemize}

While a foundational security level is present, the identified gaps substantially increase the organization's risk profile. The recommendations outlined in this report are designed to address these weaknesses systematically and should be prioritized for implementation.

% ----------------------------------------------------------------------
%  SECTION 2: ORGANIZATIONAL INFORMATION
% ----------------------------------------------------------------------
\section{Organizational Information}
The following details were used as the basis for this assessment. As per the provided data, placeholder values are used where specific information was not available.

\begin{tabular}{@{}ll}
    \toprule
    \textbf{Attribute} & \textbf{Value} \\
    \midrule
    Organization Name & \textbf{[Organization Name]} \\
    Email Domain & \seqsplit{\texttt{[Domain]}} \\
    External IP Scanned & \seqsplit{\texttt{[Client IP]}} \\
    \bottomrule
\end{tabular}

% ----------------------------------------------------------------------
%  SECTION 3: SECURITY CONTROL REVIEW
% ----------------------------------------------------------------------
\section{Security Control Review}
The following table summarizes the organization's responses to a security controls questionnaire. A red cross (\no) indicates a potential security gap that increases risk.

\begin{table}[h!]
\centering
\caption{Security Controls Questionnaire Results}
\begin{tabular}{@{}p{0.7\linewidth}cc@{}}
    \toprule
    \textbf{Control Question} & \textbf{Response} & \textbf{Status} \\
    \midrule
    Do you require MFA to access email? & Yes & \yes \\
    Do you require MFA to log into computers? & No & \no \\
    Do you require MFA to access sensitive data systems? & Yes & \yes \\
    Does your organization have an employee acceptable use policy? & Yes & \yes \\
    Does your organization do security awareness training for new employees? & No & \no \\
    Does your organization do security awareness training for all employees at least once per year? & Yes & \yes \\
    \bottomrule
\end{tabular}
\end{table}

\paragraph{Analysis:}
The lack of MFA for computer logins is a critical weakness. If an employee's credentials are stolen, an attacker could gain direct access to a corporate device and the internal network. Similarly, failing to train new employees on security best practices from day one makes them a primary target for phishing and other social engineering attacks.

% ----------------------------------------------------------------------
%  SECTION 4: TECHNICAL SCAN RESULTS
% ----------------------------------------------------------------------
\section{Technical Scan Results}
An Nmap scan was performed against the organization's external-facing IP address to identify open ports and exposed services.

\begin{itemize}
    \item \textbf{Target IP:} \texttt{[Target IP]}
    \item \textbf{Scan Date:} 2025-11-22
\end{itemize}

\begin{table}[h!]
\centering
\caption{Open Ports and Services}
\begin{tabular}{@{}lllll@{}}
    \toprule
    \textbf{Port} & \textbf{State} & \textbf{Service} & \textbf{Product} & \textbf{Version} \\
    \midrule
    443/tcp & open & https & nginx & 1.18.0 \\
    \bottomrule
\end{tabular}
\end{table}

\paragraph{Analysis:}
The scan identified an Nginx web server, version 1.18.0, exposed to the internet on port 443 (HTTPS). This version was released in April 2020 and is now considered outdated. It is susceptible to several publicly known vulnerabilities (e.g., CVE-2021-23017). Running outdated software on perimeter systems presents a significant and unnecessary risk of exploitation.

% ----------------------------------------------------------------------
%  SECTION 5: RISK ASSESSMENT
% ----------------------------------------------------------------------
\section{Risk Assessment}
This section synthesizes findings from the security control review and the technical scan. No pre-existing risks were provided for this assessment.

\begin{table}[h!]
\centering
\caption{Summary of Identified Risks}
\begin{tabular}{@{}p{0.1\linewidth}p{0.25\linewidth}p{0.4\linewidth}p{0.1\linewidth}@{}}
    \toprule
    \textbf{Risk ID} & \textbf{Risk Name} & \textbf{Description} & \textbf{Severity} \\
    \midrule
    ID-001 & Inadequate MFA Coverage & MFA is not enforced for computer logins, exposing the internal network to compromise if credentials are stolen. & \textbf{High} \\
    \addlinespace
    ID-002 & Insufficient Employee Onboarding & New employees do not receive security awareness training, making them highly susceptible to social engineering attacks. & \textbf{High} \\
    \addlinespace
    ID-003 & Vulnerable Web Server Software & The external web server runs an outdated and vulnerable version of Nginx (1.18.0), which could be exploited by attackers. & \textbf{High} \\
    \bottomrule
\end{tabular}
\end{table}

% ----------------------------------------------------------------------
%  SECTION 6: RECOMMENDATIONS
% ----------------------------------------------------------------------
\section{Recommendations}
The following actions are recommended to mitigate the identified risks and improve the overall security posture of \textbf{[Organization Name]}.

\subsection*{Recommendation for ID-001: Enforce MFA for Endpoint Access}
\textbf{Action:} Implement and enforce a policy requiring MFA for all workstation and laptop logins.
\begin{itemize}
    \item \textbf{Details:} Deploy a robust MFA solution compatible with your operating systems (e.g., Windows Hello for Business, Duo Security, Okta). This control is critical for preventing unauthorized access and containing the impact of a credential compromise.
    \item \textbf{Priority:} \textbf{Critical}.
\end{itemize}

\subsection*{Recommendation for ID-002: Integrate Security into Onboarding}
\textbf{Action:} Develop and mandate a security awareness training module as a standard part of the new employee onboarding process.
\begin{itemize}
    \item \textbf{Details:} The training should cover, at a minimum, phishing identification, password hygiene, acceptable use of company assets, and procedures for reporting security incidents. This establishes a security-conscious mindset from day one.
    \item \textbf{Priority:} \textbf{High}.
\end{itemize}

\subsection*{Recommendation for ID-003: Remediate Web Server Vulnerability}
\textbf{Action:} Plan and execute an upgrade of the Nginx web server to a current, stable, and patched version.
\begin{itemize}
    \item \textbf{Details:} Before upgrading, audit the current Nginx configuration to ensure it follows security best practices (e.g., disabling unnecessary modules, implementing strong TLS ciphers). After the upgrade, perform a new vulnerability scan to validate the remediation.
    \item \textbf{Priority:} \textbf{High}.
\end{itemize}

\end{document}
```