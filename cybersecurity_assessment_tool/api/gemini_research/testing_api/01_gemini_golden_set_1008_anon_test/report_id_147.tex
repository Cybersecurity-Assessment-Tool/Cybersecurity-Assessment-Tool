```latex
\documentclass[12pt]{article}

% Preamble: Required Packages
\usepackage[margin=1in]{geometry}
\usepackage{pifont} % For checkmarks and crosses (\ding{51} and \ding{55})
\usepackage{booktabs} % For professional-looking tables
\usepackage{hyperref}
\usepackage{url}
\usepackage{seqsplit} % For breaking long strings in \texttt
\usepackage{graphicx}
\usepackage{xcolor}
\usepackage{fancyhdr}

% Hyperref Setup
\hypersetup{
    colorlinks=true,
    linkcolor=blue,
    filecolor=magenta,
    urlcolor=cyan,
    pdftitle={Cybersecurity Posture Assessment Report},
    pdfauthor={Cybersecurity Analysis Division},
}

% Custom Commands for Status Indicators
\newcommand{\yes}{\textcolor{green}{\ding{51}}}
\newcommand{\no}{\textcolor{red}{\ding{55}}}

% Header and Footer
\pagestyle{fancy}
\fancyhf{}
\fancyhead[L]{Cybersecurity Posture Assessment}
\fancyhead[R]{\textbf{[Organization Name]}}
\fancyfoot[C]{\thepage}

% Document Start
\begin{document}

\title{Cybersecurity Posture Assessment Report}
\author{Cybersecurity Analysis Division}
\date{\today}
\maketitle

\begin{abstract}
This report provides a comprehensive analysis of the cybersecurity posture for \textbf{[Organization Name]}. The assessment is based on a synthesis of technical network scans, a review of organizational security controls, and an evaluation of pre-existing risk data. The findings indicate several critical and high-risk vulnerabilities that require immediate attention to mitigate the potential for significant security incidents. Key findings include a publicly exposed, End-of-Life (EOL) database, critical gaps in Multi-Factor Authentication (MFA) for sensitive systems, and a complete lack of a security awareness training program.
\end{abstract}

\tableofcontents
\newpage

\section{Overview and Scope}
The primary objective of this assessment was to identify and evaluate security risks within the organization's digital environment. The scope included an external network scan of key assets, a review of implemented security policies via a questionnaire, and a correlation of these findings with known risks. This report consolidates all data into a single, actionable document designed to guide remediation efforts.

\subsection{Organizational Information}
The following details were used as the basis for this assessment. Per the template requirements, placeholders are used where data was not provided.
\begin{itemize}
    \item \textbf{Organization Name:} \textbf{[Organization Name]}
    \item \textbf{Primary Email Domain:} \texttt{[Domain]}
    \item \textbf{External IP Address Scanned:} \texttt{[Client IP]}
\end{itemize}

\section{Security Control Review}
An administrative review of security controls was conducted based on a standardized questionnaire. The responses highlight significant gaps in the organization's security policies and practices, particularly concerning user access and awareness. "No" answers indicate a failure to meet baseline security best practices.

\begin{table}[h!]
\centering
\caption{Organizational Security Control Questionnaire}
\label{tab:controls}
\begin{tabular}{@{}p{0.7\linewidth}cc@{}}
\toprule
\textbf{Control Question} & \textbf{Response} & \textbf{Status} \\
\midrule
Do you require MFA to access email? & Yes & \yes \\
Do you require MFA to log into computers? & Yes & \yes \\
\textbf{Do you require MFA to access sensitive data systems?} & \textbf{No} & \no \\
Does your organization have an employee acceptable use policy? & Yes & \yes \\
\textbf{Does your organization do security awareness training for new employees?} & \textbf{No} & \no \\
\textbf{Does your organization do security awareness training for all employees at least once per year?} & \textbf{No} & \no \\
\bottomrule
\end{tabular}
\end{table}

\section{Technical Scan Results}
An external network scan was performed to identify open ports and exposed services. The scan revealed a critical exposure of a database service to the public internet.

\begin{itemize}
    \item \textbf{Target IP Address:} \texttt{[Target IP]}
    \item \textbf{Scan Status:} Host is up and responsive.
\end{itemize}

\begin{table}[h!]
\centering
\caption{Open Ports and Services Detected}
\label{tab:nmap}
\begin{tabular}{@{}llll@{}}
\toprule
\textbf{Port} & \textbf{State} & \textbf{Service} & \textbf{Product \& Version} \\
\midrule
3306/tcp & open & mysql & MySQL 5.7.33 \\
\bottomrule
\end{tabular}
\end{table}

\subsection{Analysis of Technical Findings}
The scan identified that TCP port 3306 is open, exposing a MySQL database server directly to the internet. The detected version, \textbf{MySQL 5.7.33}, reached its official End-of-Life (EOL) in October 2023. EOL software no longer receives security updates from the vendor, leaving it perpetually vulnerable to newly discovered exploits. This finding represents a critical and immediate threat to the confidentiality, integrity, and availability of the data stored within the database.

\section{Consolidated Risk Assessment}
By correlating the security control gaps, technical findings, and pre-existing risk data, we have compiled a prioritized list of security risks facing the organization.

\begin{table}[h!]
\centering
\caption{Summary of Identified Risks}
\label{tab:risks}
\begin{tabular}{@{}p{0.2\linewidth}p{0.6\linewidth}l@{}}
\toprule
\textbf{Risk Name} & \textbf{Description} & \textbf{Severity} \\
\midrule
\textbf{Exposed End-of-Life Database} & The MySQL database on port 3306 is publicly accessible and runs version 5.7.33, which is past its End-of-Life. This exposes the organization to unpatchable vulnerabilities and potential data breaches. & \textbf{Critical} \\
\addlinespace
\textbf{Insufficient Multi-Factor Authentication} & MFA is not required for sensitive data systems. A compromised password could grant an attacker direct access to critical assets, including the exposed database. & \textbf{High} \\
\addlinespace
\textbf{Lack of Security Awareness Training} & The organization does not conduct security awareness training. Employees are more susceptible to phishing and social engineering attacks, increasing the risk of credential compromise. & \textbf{High} \\
\bottomrule
\end{tabular}
\end{table}

\section{Recommendations}
The following recommendations are prioritized to address the identified risks. Immediate action should be taken to remediate critical and high-severity findings.

\subsection{Immediate Priority (Remediate within 72 hours)}
\begin{enumerate}
    \item \textbf{Restrict Database Access:} Immediately implement strict firewall rules to block all public ingress traffic to TCP port 3306 on asset \texttt{[Target IP]}. Access should be restricted to a whitelist of trusted internal IP addresses or be accessible only via a secure Virtual Private Network (VPN).
\end{enumerate}

\subsection{High Priority (Remediate within 30 days)}
\begin{enumerate}
    \setcounter{enumi}{1} % Continue numbering
    \item \textbf{Upgrade Database Software:} Plan and execute an upgrade of the MySQL 5.7.33 instance to a currently supported version (e.g., MySQL 8.x). This is essential to ensure the system receives critical security patches.
    \item \textbf{Enforce MFA on Sensitive Systems:} Procure and deploy an MFA solution for all sensitive data systems. This control is a critical defense against credential theft and unauthorized access.
    \item \textbf{Establish Security Awareness Program:} Develop and implement a mandatory security awareness training program. This program must include onboarding for new hires and annual refresher courses for all employees, covering topics such as phishing, password hygiene, and acceptable use.
\end{enumerate}

\end{document}
```