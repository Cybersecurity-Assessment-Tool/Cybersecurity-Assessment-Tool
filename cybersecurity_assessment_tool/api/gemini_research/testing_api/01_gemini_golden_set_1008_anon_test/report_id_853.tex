```latex
\documentclass[12pt]{article}

% Preamble: Required Packages
\usepackage[margin=1in]{geometry}
\usepackage{pifont} % For checkmarks and crosses (\ding)
\usepackage{booktabs} % For professional tables
\usepackage{hyperref} % For hyperlinks and metadata
\usepackage{url}
\usepackage{seqsplit} % To break long strings in \texttt
\usepackage{graphicx}
\usepackage{xcolor}
\usepackage{datetime}

% --- Document Metadata ---
\hypersetup{
    colorlinks=true,
    linkcolor=blue,
    urlcolor=cyan,
    pdftitle={Cybersecurity Posture Assessment Report},
    pdfauthor={Cybersecurity Analysis Division},
    pdfsubject={Security Assessment}
}

% --- Custom Commands ---
\newcommand{\yes}{\textcolor{green}{\ding{51}}} % Green checkmark
\newcommand{\no}{\textcolor{red}{\ding{55}}}   % Red cross

% --- Document Start ---
\begin{document}

% --- Title Page ---
\begin{titlepage}
    \centering
    \vspace*{1cm}
    \Huge\textbf{Cybersecurity Posture Assessment Report}
    \vspace{1.5cm}
    \Large
    \textbf{Prepared For:}\\
    \vspace{0.5cm}
    \Huge \textbf{[Organization Name]}
    \vspace{2cm}
    \large
    \textbf{Report Date:} \today \\
    \textbf{Assessment Date:} November 22, 2025
    \vfill
    \large
    \textbf{Generated By:}\\
    Cybersecurity Analysis Division
\end{titlepage}

\newpage

% --- Table of Contents ---
\tableofcontents
\newpage

% --- Section 1: Executive Summary ---
\section{Executive Summary}

This report details the findings of a cybersecurity posture assessment conducted for \textbf{[Organization Name]}. The assessment combined an external network scan, a review of existing risks, and an analysis of organizational security controls based on a questionnaire.

The overall security posture reveals a mixed landscape. The organization has implemented critical controls such as Multi-Factor Authentication (MFA) for email and sensitive data access. However, several significant gaps were identified that expose the organization to a high level of risk.

Key findings include:
\begin{itemize}
    \item \textbf{Critical Control Gap:} Multi-Factor Authentication is not enforced for logging into employee computers, leaving a primary access vector vulnerable to credential compromise.
    \item \textbf{Vulnerable External Service:} The external-facing web server is running an outdated version of Nginx (1.18.0), which is no longer supported and has known vulnerabilities.
    \item \textbf{Policy Deficiencies:} The absence of a formal Employee Acceptable Use Policy and the lack of security awareness training for new employees create significant human-factor risks.
\end{itemize}

This report provides a detailed breakdown of these findings and offers actionable recommendations to mitigate the identified risks and strengthen the organization's overall security posture.

% --- Section 2: Organizational Information ---
\section{Organizational Information}
This section provides the high-level details of the organization under review. As per the provided data, the following placeholders are used due to missing information.

\begin{itemize}
    \item \textbf{Organization Name:} \textbf{[Organization Name]}
    \item \textbf{Primary Email Domain:} \texttt{[Domain]}
    \item \textbf{External IP Address Scanned:} \texttt{[Client IP]}
\end{itemize}

% --- Section 3: Security Control Review ---
\section{Security Control Review}
The following table summarizes the organization's responses to the security controls questionnaire. A \yes\ indicates the control is in place, while a \no\ indicates a potential gap.

\begin{table}[h!]
\centering
\caption{Security Controls Questionnaire Results}
\begin{tabular}{p{0.7\textwidth}c}
\toprule
\textbf{Control Question} & \textbf{Response} \\
\midrule
Do you require MFA to access email? & \yes \\
Do you require MFA to log into computers? & \no \\
Do you require MFA to access sensitive data systems? & \yes \\
Does your organization have an employee acceptable use policy? & \no \\
Does your organization do security awareness training for new employees? & \no \\
Does your organization do security awareness training for all employees at least once per year? & \yes \\
\bottomrule
\end{tabular}
\end{table}

\subsection*{Analysis of Control Gaps}
The questionnaire reveals three significant control gaps:
\begin{enumerate}
    \item \textbf{Lack of MFA on Endpoints:} Failure to require MFA for computer logins is a critical weakness. If an employee's password is stolen (e.g., through phishing), an attacker could gain direct access to their workstation and the corporate network.
    \item \textbf{Missing Acceptable Use Policy (AUP):} An AUP is a foundational policy that sets clear expectations for how employees should use company assets. Its absence can lead to inconsistent security practices and misuse of resources.
    \item \textbf{No Onboarding Security Training:} New employees are often a primary target for social engineering attacks. Failing to provide security training during onboarding leaves a critical window of vulnerability.
\end{enumerate}

% --- Section 4: Technical Scan Results ---
\section{Technical Scan Results}
An external network scan was performed to identify open ports and services exposed to the internet.

\begin{itemize}
    \item \textbf{Scan Target:} \texttt{[Target IP]}
    \item \textbf{Scan Date:} 2025-11-22T10:00:00Z
\end{itemize}

\begin{table}[h!]
\centering
\caption{Open Ports and Services}
\begin{tabular}{l l l l l}
\toprule
\textbf{Port} & \textbf{State} & \textbf{Service} & \textbf{Product} & \textbf{Version} \\
\midrule
443/tcp & open & https & nginx & 1.18.0 \\
\bottomrule
\end{tabular}
\end{table}

\subsection*{Analysis of Technical Findings}
The scan identified a single open port (443/tcp) running an Nginx web server. The detected version, \textbf{Nginx 1.18.0}, was released in April 2020. This version is outdated and no longer receives security updates from the developer. Running unsupported software on an internet-facing server presents a high risk, as it may be vulnerable to numerous publicly known exploits that could lead to a system compromise.

% --- Section 5: Consolidated Risk Assessment ---
\section{Consolidated Risk Assessment}
This section correlates findings from the security control review, technical scan, and pre-existing risk register. Since the pre-existing risk register was empty, all risks listed below are new findings from this assessment.

\begin{table}[h!]
\centering
\caption{Summary of Identified Risks}
\begin{tabular}{p{0.1\textwidth} p{0.6\textwidth} p{0.15\textwidth}}
\toprule
\textbf{ID} & \textbf{Risk Description} & \textbf{Severity} \\
\midrule
RISK-001 & Lack of MFA on employee computers allows for straightforward account takeover if credentials are compromised. & \textbf{Critical} \\
\addlinespace
RISK-002 & The external web server runs outdated and unsupported Nginx 1.18.0 software, which is exposed to known vulnerabilities. & \textbf{High} \\
\addlinespace
RISK-003 & The absence of an Employee Acceptable Use Policy leads to inconsistent security practices and a weakened security culture. & \textbf{High} \\
\addlinespace
RISK-004 & New employees do not receive security awareness training, making them highly susceptible to social engineering attacks. & \textbf{High} \\
\bottomrule
\end{tabular}
\end{table}

% --- Section 6: Recommendations ---
\section{Recommendations}
The following actions are recommended to mitigate the identified risks and improve the overall security posture of \textbf{[Organization Name]}.

\subsection*{RISK-001: Lack of MFA on Endpoints (Critical)}
\begin{itemize}
    \item \textbf{Action:} Procure and implement an MFA solution for all employee computer logins (desktops and laptops). This should apply to both local and remote access.
    \item \textbf{Priority:} Immediate. This is the most critical finding and should be addressed urgently.
\end{itemize}

\subsection*{RISK-002: Outdated Web Server Software (High)}
\begin{itemize}
    \item \textbf{Action:} Plan and execute an upgrade of the Nginx server from version 1.18.0 to the latest stable version. Before upgrading the production server, test the new version in a staging environment to ensure application compatibility.
    \item \textbf{Priority:} High.
\end{itemize}

\subsection*{RISK-003: Missing Acceptable Use Policy (High)}
\begin{itemize}
    \item \textbf{Action:} Develop a formal Employee Acceptable Use Policy (AUP). This policy should be reviewed by management and legal counsel, distributed to all employees, and formally acknowledged via signature.
    \item \textbf{Priority:} High.
\end{itemize}

\subsection*{RISK-004: No Security Training for New Employees (High)}
\begin{itemize}
    \item \textbf{Action:} Integrate mandatory cybersecurity awareness training into the new employee onboarding process. This training should cover key topics such as phishing, password security, and the new AUP.
    \item \textbf{Priority:} High.
\end{itemize}

% --- Document End ---
\end{document}
```