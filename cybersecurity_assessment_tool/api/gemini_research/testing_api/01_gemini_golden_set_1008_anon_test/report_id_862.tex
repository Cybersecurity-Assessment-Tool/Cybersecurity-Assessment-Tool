```latex
\documentclass[12pt]{article}

% --- PACKAGES ---
\usepackage[margin=1in]{geometry}
\usepackage{pifont} % For checkmarks and crosses
\usepackage{booktabs} % For professional tables
\usepackage{hyperref} % For hyperlinks
\usepackage{url}      % For URL formatting
\usepackage{seqsplit} % For splitting long strings
\usepackage{graphicx} % For logo (placeholder)
\usepackage{xcolor}   % For colors

% --- DOCUMENT SETUP ---
\hypersetup{
    colorlinks=true,
    linkcolor=blue,
    filecolor=magenta,      
    urlcolor=cyan,
    pdftitle={Cybersecurity Assessment Report},
    pdfpagemode=FullScreen,
}

% --- TITLE ---
\title{
    \vspace{-2cm}
    % \includegraphics[width=4cm]{logo_placeholder.png} % Placeholder for a logo
    \begin{center}
    \rule{\textwidth}{1pt}
    \huge \textbf{Cybersecurity Assessment Report} \\
    \rule{\textwidth}{1pt}
    \end{center}
    \vspace{1cm}
    \large Prepared for: \textbf{[Organization Name]} \\
    \normalsize \today
}
\author{Cybersecurity Analysis Division}
\date{}

% --- BEGIN DOCUMENT ---
\begin{document}
\maketitle
\thispagestyle{empty}
\newpage

\tableofcontents
\newpage

% ==============================================================================
\section{Executive Summary}
% ==============================================================================

This report provides a comprehensive analysis of the cybersecurity posture of \textbf{[Organization Name]}, based on a review of organizational security controls, an external network scan, and an assessment of current risks. The assessment was conducted on \today.

The analysis revealed several critical and high-risk gaps in foundational security controls. The most significant findings stem from the organizational security questionnaire, which indicate a lack of Multi-Factor Authentication (MFA) for email and computer access, the absence of a formal employee acceptable use policy, and incomplete annual security awareness training. These deficiencies expose the organization to significant risks, including unauthorized access, data breaches, and phishing attacks.

The external network scan of the target IP address, \texttt{[Target IP]}, did not identify any open ports or services. While this is a positive finding, it could also indicate that the host was offline or protected by a firewall that blocked the scan.

Based on these findings, this report provides prioritized, actionable recommendations to mitigate the identified risks and strengthen the organization's overall security posture. Immediate focus should be placed on implementing MFA across all critical systems and developing core security policies and training programs.

% ==============================================================================
\section{Organizational Information}
% ==============================================================================

The following information was used as the basis for this assessment. Due to the anonymized nature of the data provided, placeholders have been used where necessary.

\begin{table}[h!]
\centering
\begin{tabular}{@{}ll@{}}
\toprule
\textbf{Attribute} & \textbf{Value} \\ \midrule
Organization Name & \textbf{[Organization Name]} \\
Primary Domain & \texttt{[Domain]} \\
External IP Address (Target) & \texttt{[Client IP]} \\ \bottomrule
\end{tabular}
\caption{Client Organizational Details}
\end{table}

% ==============================================================================
\section{Security Control Review (Questionnaire)}
% ==============================================================================

A review of the organization's security controls was conducted via a questionnaire. The responses highlight significant gaps in administrative and technical controls. A "No" response indicates a deviation from security best practices and represents a potential risk.

\begin{table}[h!]
\centering
\begin{tabular}{@{}p{0.6\textwidth}cp{0.2\textwidth}@{}}
\toprule
\textbf{Control Question} & \textbf{Response} & \textbf{Assessment} \\ \midrule
Do you require MFA to access email? & \textcolor{red}{\ding{55}} & \textcolor{red}{Critical Gap} \\
Do you require MFA to log into computers? & \textcolor{red}{\ding{55}} & \textcolor{red}{Critical Gap} \\
Do you require MFA to access sensitive data systems? & \textcolor{green}{\ding{51}} & Aligned \\
Does your organization have an employee acceptable use policy? & \textcolor{red}{\ding{55}} & \textcolor{red}{High Risk} \\
Does your organization do security awareness training for new employees? & \textcolor{green}{\ding{51}} & Aligned \\
Does your organization do security awareness training for all employees at least once per year? & \textcolor{red}{\ding{55}} & \textcolor{red}{High Risk} \\ \bottomrule
\end{tabular}
\caption{Security Control Questionnaire Analysis}
\end{table}

% ==============================================================================
\section{Technical Scan Results}
% ==============================================================================

An external network vulnerability scan was performed to identify exposed services and potential vulnerabilities on the public-facing infrastructure.

\begin{itemize}
    \item \textbf{Target IP Address:} \texttt{[Target IP]}
    \item \textbf{Scan Date:} \today
\end{itemize}

\subsection{Findings}
The scan completed successfully but found \textbf{no open ports or running services} on the target host.

\subsubsection{Interpretation}
This result indicates a strong network perimeter security posture for the scanned IP, as no services are exposed to the public internet. However, this could also be the result of:
\begin{itemize}
    \item A firewall (network or host-based) explicitly blocking the scan probes.
    \item The host being temporarily offline during the scan window.
    \item An incorrect IP address being provided for the scan.
\end{itemize}
While no vulnerabilities were found, the lack of response from the host should be verified to ensure it was the intended outcome.

% ==============================================================================
\section{Risk Assessment}
% ==============================================================================

This section synthesizes findings from the security control review, technical scan, and pre-existing risk data. As no pre-existing vulnerabilities or technical findings were identified, the primary risks are derived from the policy and procedure gaps identified in the questionnaire.

\begin{table}[h!]
\centering
\begin{tabular}{@{}p{0.25\textwidth}p{0.15\textwidth}p{0.5\textwidth}@{}}
\toprule
\textbf{Risk Name} & \textbf{Severity} & \textbf{Overview} \\ \midrule
\textbf{Lack of MFA on Critical Systems} & \textcolor{red}{Critical} & The absence of MFA for email and computer logins drastically increases the risk of account compromise via stolen credentials. Email is a primary target for phishing and business email compromise (BEC) attacks. \\
\addlinespace
\textbf{Inadequate Security Policies} & \textcolor{orange}{High} & Without a formal Acceptable Use Policy (AUP), employees lack clear guidelines on the secure use of company assets. This can lead to unintentional security incidents and creates challenges for enforcement. \\
\addlinespace
\textbf{Insufficient Security Training} & \textcolor{orange}{High} & Failing to provide annual security awareness training for all employees means that staff may not be able to recognize or appropriately respond to modern threats like phishing and social engineering, making them a vulnerable entry point for attackers. \\ \bottomrule
\end{tabular}
\caption{Summary of Identified Risks}
\end{table}

% ==============================================================================
\section{Recommendations}
% ==============================================================================

The following recommendations are prioritized to address the critical and high-risk findings identified during this assessment.

\subsection{Priority 1: Remediate Critical Risks}
\begin{enumerate}
    \item \textbf{Implement Multi-Factor Authentication (MFA):}
    \begin{itemize}
        \item \textbf{Action:} Immediately enable and enforce MFA for all user accounts, especially for email access (e.g., Office 365, Google Workspace) and remote access VPNs.
        \item \textbf{Justification:} This is the single most effective control to prevent unauthorized access resulting from compromised credentials. It mitigates the risk of account takeover and subsequent data breaches.
    \end{itemize}

    \item \textbf{Enforce MFA for Endpoint Logins:}
    \begin{itemize}
        \item \textbf{Action:} Deploy a solution to require MFA for logging into all company-managed computers (desktops and laptops).
        \item \textbf{Justification:} This protects against unauthorized physical or remote access to endpoints, safeguarding local data and preventing lateral movement within the network if an attacker gains a foothold.
    \end{itemize}
\end{enumerate}

\subsection{Priority 2: Remediate High Risks}
\begin{enumerate}
    \item \textbf{Develop and Implement an Acceptable Use Policy (AUP):}
    \begin{itemize}
        \item \textbf{Action:} Draft a formal AUP that clearly defines the rules for using company networks, devices, and data. Require all employees to read and acknowledge the policy.
        \item \textbf{Justification:} An AUP establishes a baseline for secure behavior, reduces organizational liability, and provides a framework for enforcing security standards.
    \end{itemize}

    \item \textbf{Establish a Mandatory Annual Security Awareness Program:}
    \begin{itemize}
        \item \textbf{Action:} Implement a recurring, mandatory security awareness training program for all employees. This program should cover key topics such as phishing, password security, and social engineering.
        \item \textbf{Justification:} Continuous training reinforces security concepts and keeps employees updated on evolving threats, strengthening the "human firewall" and reducing the likelihood of successful social engineering attacks.
    \end{itemize}
\end{enumerate}

\end{document}
```