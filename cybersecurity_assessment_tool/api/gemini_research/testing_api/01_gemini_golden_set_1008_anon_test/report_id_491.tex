```latex
\documentclass[12pt]{article}

% Preamble: Required Packages
\usepackage[a4paper, margin=1in]{geometry}
\usepackage{pifont} % Required for check and cross marks (\ding)
\usepackage{booktabs} % For professional-looking tables
\usepackage{hyperref} % For creating hyperlinks in the document
\usepackage{url}      % For formatting URLs
\usepackage{seqsplit} % For splitting long strings in \texttt
\usepackage{xcolor}   % For custom colors
\usepackage{graphicx} % For titles and logos (optional)

% --- Document Metadata and Hyperlink Setup ---
\definecolor{darkblue}{rgb}{0.0, 0.0, 0.55}
\definecolor{darkred}{rgb}{0.55, 0.0, 0.0}

\hypersetup{
    colorlinks=true,
    linkcolor=darkblue,
    filecolor=darkblue,
    urlcolor=darkblue,
    citecolor=darkblue,
    pdftitle={Cybersecurity Posture Assessment Report},
    pdfauthor={Cybersecurity Analyst},
    pdfsubject={Security Report},
    pdfkeywords={security, assessment, report, nmap, risk}
}

% --- Custom Commands ---
\newcommand{\yes}{\ding{51}}
\newcommand{\no}{\textcolor{darkred}{\ding{55}}}

% --- Document Start ---
\begin{document}

% --- Title Page ---
\begin{titlepage}
    \centering
    \vspace*{2cm}
    {\Huge \textbf{Cybersecurity Posture Assessment Report}}
    \vspace{1.5cm}
    \vfill
    {\Large \textbf{Prepared for:}} \\
    \vspace{0.5cm}
    {\huge \textbf{[Organization Name]}}
    \vfill
    \rule{0.8\textwidth}{0.4pt}
    \vspace{0.4cm}
    {\large \textbf{Date of Report:} \today} \\
    {\large \textbf{Report ID:} CSA-2024-001}
    \vspace{0.2cm}
    \rule{0.8\textwidth}{0.4pt}
\end{titlepage}

\tableofcontents
\newpage

% --- Section 1: Executive Summary ---
\section*{Executive Summary}

This report provides a comprehensive assessment of the cybersecurity posture for \textbf{[Organization Name]}, based on an analysis of organizational security controls, an external network scan, and a review of pre-existing risks.

The assessment identified several critical and high-severity risks that require immediate attention. Key findings include significant gaps in Multi-Factor Authentication (MFA) for computer and sensitive data access, the lack of a mandatory annual security awareness training program, and an externally exposed management service (SSH on port 22).

These vulnerabilities, particularly when correlated, create a substantial risk of unauthorized access, credential compromise, and potential data breach. The pre-existing critical risk, "Localhost Exposed," further elevates the overall risk profile. This report outlines these findings in detail and provides a prioritized list of actionable recommendations to mitigate the identified threats and strengthen the organization's security defenses.

% --- Section 2: Organizational Information ---
\section*{Organizational Information}

The following details were used as the basis for this assessment. Placeholder values are used where data was not provided.

\begin{itemize}
    \item \textbf{Organization Name:} \textbf{[Organization Name]}
    \item \textbf{Primary Domain:} \texttt{[Domain]}
    \item \textbf{External IP Address Scanned:} \texttt{[Client IP]}
\end{itemize}

% --- Section 3: Security Control Review (Questionnaire Analysis) ---
\section*{Security Control Review}

An analysis of the organization's security questionnaire reveals several policy and implementation gaps. "No" answers indicate a deviation from security best practices and are highlighted as significant risks.

\begin{table}[h!]
\centering
\caption{Security Controls Questionnaire Analysis}
\begin{tabular}{p{0.6\linewidth} c p{0.25\linewidth}}
\toprule
\textbf{Control Question} & \textbf{Status} & \textbf{Analyst's Note} \\
\midrule
Do you require MFA to access email? & \yes & Good Practice. \\
\addlinespace
Do you require MFA to log into computers? & \no & \textbf{Critical Gap.} Lack of MFA on endpoints significantly increases risk from stolen credentials. \\
\addlinespace
Do you require MFA to access sensitive data systems? & \no & \textbf{Critical Gap.} The organization's most valuable data is not protected by a secondary authentication factor. \\
\addlinespace
Does your organization have an employee acceptable use policy? & \yes & Good Practice. \\
\addlinespace
Does your organization do security awareness training for new employees? & \yes & Good Practice. \\
\addlinespace
Does your organization do security awareness training for all employees at least once per year? & \no & \textbf{High Risk.} Without recurring training, employees are more susceptible to evolving threats like phishing. \\
\bottomrule
\end{tabular}
\end{table}

% --- Section 4: Technical Scan Results ---
\section*{Technical Scan Results}

An external network scan was performed to identify open ports and exposed services on the organization's public-facing infrastructure.

\begin{itemize}
    \item \textbf{Target IP Address:} \texttt{[Target IP]}
    \item \textbf{Scan Date:} \today
\end{itemize}

The scan revealed the following open port:

\begin{table}[h!]
\centering
\caption{Open Ports Detected on \texttt{[Target IP]}}
\begin{tabular}{l l l l}
\toprule
\textbf{Port} & \textbf{Protocol} & \textbf{State} & \textbf{Service (Inferred)} \\
\midrule
22 & TCP & open & SSH (Secure Shell) \\
\bottomrule
\end{tabular}
\end{table}

\subsection*{Analysis of Technical Findings}
The Secure Shell (SSH) service on port 22 is exposed to the public internet. SSH is a common management protocol used for remote server administration. When exposed externally, it becomes a primary target for automated brute-force attacks, where attackers attempt to guess usernames and passwords to gain unauthorized access. This finding, combined with the lack of MFA for computer logins, constitutes a high-severity risk.

% --- Section 5: Consolidated Risk Assessment ---
\section*{Consolidated Risk Assessment}

The following table synthesizes findings from the security control review, technical scan, and pre-existing risk data into a consolidated list of identified risks.

\begin{table}[h!]
\centering
\caption{Summary of Identified Risks}
\begin{tabular}{p{0.45\linewidth} l p{0.35\linewidth}}
\toprule
\textbf{Risk Description} & \textbf{Severity} & \textbf{Source of Finding} \\
\midrule
\textbf{Pre-existing: Localhost Exposed} & \textbf{Critical} & Input 3: Current Risks JSON \\
A pre-existing critical vulnerability (CVSS 10.0) was noted. Details should be reviewed internally. & & \\
\addlinespace
\textbf{Lack of MFA for Computer Logins} & \textbf{Critical} & Input 2: Org Data JSON \\
A single compromised password could lead to direct endpoint and network access. & & \\
\addlinespace
\textbf{Lack of MFA for Sensitive Data Systems} & \textbf{Critical} & Input 2: Org Data JSON \\
The organization's most critical data assets lack a fundamental security control. & & \\
\addlinespace
\textbf{Exposed SSH Management Port} & \textbf{High} & Input 1: Network Scan JSON \\
The primary remote access service is open to attack from any location on the internet. & & \\
\addlinespace
\textbf{Lack of Annual Security Awareness Training} & \textbf{High} & Input 2: Org Data JSON \\
The human element of security is not being consistently reinforced against modern threats. & & \\
\bottomrule
\end{tabular}
\end{table}

% --- Section 6: Recommendations ---
\section*{Recommendations}

The following prioritized recommendations are provided to address the identified risks and improve the overall security posture of \textbf{[Organization Name]}.

\subsection*{Priority 1: Remediate Critical Access Control Gaps (Immediate)}
\begin{enumerate}
    \item \textbf{Implement MFA Everywhere:} Immediately deploy a mandatory MFA policy for all employee computer logins and for access to all systems containing sensitive data. This is the single most effective control to mitigate the risk of credential compromise.
    \item \textbf{Restrict SSH Access:} If SSH access is required from the internet, restrict it to a whitelist of trusted IP addresses (e.g., office or administrator home IPs). If public access is unavoidable, disable password-based authentication and enforce the use of strong cryptographic keys (key-based authentication).
\end{enumerate}

\subsection*{Priority 2: Address High-Severity Risks (Near-Term)}
\begin{enumerate}
    \item \textbf{Establish Annual Security Training:} Implement a mandatory security awareness training program for all employees to be completed at least once per year. This program should cover modern threats such as phishing, social engineering, and password hygiene.
    \item \textbf{Investigate "Localhost Exposed" Risk:} Launch an immediate internal investigation into the pre-existing "Localhost Exposed" risk. Given its critical severity score (10.0), a full technical analysis is required to understand its impact and apply the necessary remediation.
\end{enumerate}

% --- Document End ---
\end{document}
```