```latex
\documentclass[12pt]{article}

% Preamble: Required Packages
\usepackage[margin=1in]{geometry}
\usepackage{pifont} % For checkmarks and crosses
\usepackage{booktabs} % For professional tables
\usepackage{hyperref} % For hyperlinks
\usepackage{url} % For URL formatting
\usepackage{seqsplit} % For splitting long strings to prevent overflow
\usepackage{graphicx}
\usepackage{xcolor}

% Document Metadata
\title{Cybersecurity Posture Assessment Report}
\author{Cybersecurity Analysis Division}
\date{\today}

% Hyperref Setup
\hypersetup{
    colorlinks=true,
    linkcolor=blue,
    filecolor=magenta,      
    urlcolor=cyan,
    pdftitle={Cybersecurity Posture Assessment Report},
    pdfpagemode=FullScreen,
}

\begin{document}

\maketitle
\thispagestyle{empty}
\newpage

\tableofcontents
\newpage

% --- Section 1: Executive Overview ---
\section{Executive Overview}

This report details the findings of a cybersecurity posture assessment conducted for \textbf{[Organization Name]}. The assessment combined a review of organizational security controls via a questionnaire, an external network vulnerability scan, and an analysis of pre-existing risks.

The primary finding of this assessment is a \textbf{critical gap} in Identity and Access Management (IAM) controls. The organization has not implemented Multi-Factor Authentication (MFA) for email, computer logins, or access to sensitive data systems. This systemic weakness exposes the organization to a high risk of account compromise, business email compromise (BEC), and unauthorized data access. Additionally, a gap was identified in the security onboarding process, as new employees do not receive mandatory security awareness training.

On a positive note, the external network scan of the designated target IP address (\texttt{[Client IP]}) revealed no open ports. This indicates a strong perimeter firewall configuration that effectively blocks unsolicited inbound traffic, which is a commendable security practice.

Immediate remediation efforts should focus on the phased implementation of MFA across all critical systems, starting with email services. Enhancing the security awareness program to include new hires is also a high-priority recommendation.

% --- Section 2: Organizational Information ---
\section{Organizational Information}

This section provides a summary of the organizational details relevant to this assessment. The data has been anonymized for this report template.

\begin{itemize}
    \item \textbf{Organization Name:} \textbf{[Organization Name]}
    \item \textbf{Primary Email Domain:} \texttt{[Domain]}
    \item \textbf{Scanned External IP:} \texttt{[Client IP]}
\end{itemize}

% --- Section 3: Security Control Review ---
\section{Security Control Review}

The following table summarizes the organization's responses to a security controls questionnaire. Responses marked with \textcolor{red}{\ding{55}} indicate a deviation from security best practices and represent a potential risk.

\begin{table}[h!]
\centering
\caption{Security Controls Questionnaire Results}
\begin{tabular}{p{0.75\linewidth} c}
\toprule
\textbf{Control Question} & \textbf{Response} \\
\midrule
Do you require MFA to access email? & \textcolor{red}{\ding{55}} \\
Do you require MFA to log into computers? & \textcolor{red}{\ding{55}} \\
Do you require MFA to access sensitive data systems? & \textcolor{red}{\ding{55}} \\
Does your organization have an employee acceptable use policy? & \textcolor{green}{\ding{51}} \\
Does your organization do security awareness training for new employees? & \textcolor{red}{\ding{55}} \\
Does your organization do security awareness training for all employees at least once per year? & \textcolor{green}{\ding{51}} \\
\bottomrule
\end{tabular}
\end{table}

\subsection*{Analysis of Control Gaps}
The questionnaire reveals several significant control deficiencies:
\begin{itemize}
    \item \textbf{Lack of MFA:} The absence of MFA across email, endpoints, and sensitive systems is the most critical finding. This control is fundamental for preventing unauthorized access resulting from credential theft.
    \item \textbf{Onboarding Training Gap:} New employees are a common target for social engineering attacks. The lack of security awareness training during the onboarding process leaves the organization vulnerable.
\end{itemize}

% --- Section 4: Technical Scan Results ---
\section{Technical Scan Results}

An external network scan was performed to identify open ports and exposed services on the organization's perimeter.

\begin{itemize}
    \item \textbf{Target IP Address:} \texttt{[Target IP]}
    \item \textbf{Scan Date:} \today
\end{itemize}

\subsection*{Findings}
The network scan did not identify any open TCP or UDP ports on the target host. This is a positive security finding, suggesting that a well-configured firewall is in place, enforcing a principle of least privilege by denying all unsolicited inbound connections. No vulnerabilities associated with exposed services could be identified as a result.

% --- Section 5: Risk Assessment Summary ---
\section{Risk Assessment Summary}

This section synthesizes findings from the security control review and technical scan. Since no pre-existing risks were provided and the technical scan yielded no findings, the following risks are derived entirely from the identified policy and procedure gaps.

\begin{table}[h!]
\centering
\caption{Identified Risks and Severity}
\begin{tabular}{p{0.1\linewidth} p{0.25\linewidth} p{0.45\linewidth} p{0.1\linewidth}}
\toprule
\textbf{Risk ID} & \textbf{Risk Name} & \textbf{Description} & \textbf{Severity} \\
\midrule
ORG-001 & No MFA for Email Access & Lack of MFA on email accounts greatly increases the risk of Business Email Compromise (BEC), phishing success, and account takeover. & \textbf{Critical} \\
\addlinespace
ORG-002 & No MFA for Sensitive Systems & Failure to protect sensitive data systems with MFA exposes critical data to unauthorized access and exfiltration if an attacker obtains valid credentials. & \textbf{Critical} \\
\addlinespace
ORG-003 & No MFA for Endpoint Login & Unprotected endpoints allow an attacker with stolen credentials to gain direct access to the local network, user data, and potentially escalate privileges. & \textbf{High} \\
\addlinespace
ORG-004 & Inadequate New Hire Training & New employees are not trained on security best practices, making them more susceptible to social engineering and phishing attacks from day one. & \textbf{High} \\
\bottomrule
\end{tabular}
\end{table}

% --- Section 6: Recommendations ---
\section{Recommendations}

The following recommendations are provided to address the identified risks and improve the overall security posture of \textbf{[Organization Name]}.

\subsection*{Priority 1: Implement Multi-Factor Authentication (Critical)}
\begin{itemize}
    \item \textbf{Action:} Immediately begin a phased rollout of MFA across the entire organization.
    \item \textbf{Details:}
    \begin{enumerate}
        \item \textbf{Phase 1 (Email):} Enforce MFA for all user access to the email system (e.g., Microsoft 365, Google Workspace). This is the most critical first step to mitigate BEC and phishing risks.
        \item \textbf{Phase 2 (Sensitive Systems):} Deploy MFA for all applications and systems that store or process sensitive, confidential, or regulated data.
        \item \textbf{Phase 3 (Endpoints):} Implement MFA for all employee computer logins to secure devices against unauthorized local and remote access.
    \end{enumerate}
\end{itemize}

\subsection*{Priority 2: Strengthen Security Awareness Program (High)}
\begin{itemize}
    \item \textbf{Action:} Integrate mandatory security awareness training into the new employee onboarding process.
    \item \textbf{Details:} The training module should cover, at a minimum, phishing identification, acceptable use of company assets, password hygiene, and incident reporting procedures. This ensures a baseline of security knowledge for all new staff members before they are granted system access.
\end{itemize}

\subsection*{Priority 3: Maintain Perimeter Security (Informational)}
\begin{itemize}
    \item \textbf{Action:} Continue to maintain the strong firewall configuration observed during the scan.
    \item \textbf{Details:} The current practice of denying all non-essential inbound traffic is excellent. We recommend performing periodic, authenticated vulnerability scans and regular firewall rule reviews to ensure this secure posture is maintained over time.
\end{itemize}

\end{document}
```