```latex
\documentclass[12pt]{article}

% --- PACKAGES ---
\usepackage[margin=1in]{geometry}
\usepackage{pifont} % For checkmarks and crosses
\usepackage{booktabs} % For professional tables
\usepackage{hyperref} % For clickable links
\usepackage{url} % For URL formatting
\usepackage{seqsplit} % For splitting long strings without spaces
\usepackage{graphicx}
\usepackage{xcolor}

% --- DOCUMENT SETUP ---
\hypersetup{
    colorlinks=true,
    linkcolor=blue,
    filecolor=magenta,      
    urlcolor=cyan,
    pdftitle={Cybersecurity Assessment Report},
    pdfpagemode=FullScreen,
}

\newcommand{\yes}{\ding{51}}
\newcommand{\no}{\ding{55}}

% --- TITLE ---
\title{Cybersecurity Assessment Report \\ \large For \textbf{[Organization Name]}}
\author{Cybersecurity Analyst Group}
\date{\today}

% --- DOCUMENT START ---
\begin{document}

\maketitle
\tableofcontents
\newpage

% ===================================================================
\section{Executive Summary}
% ===================================================================

This report details the findings of a cybersecurity assessment conducted for \textbf{[Organization Name]}. The analysis synthesized data from a network vulnerability scan, a security controls questionnaire, and a review of pre-existing risks.

The assessment identified several \textbf{critical-risk vulnerabilities} that require immediate attention. The most severe finding is the direct exposure of a Remote Desktop Protocol (RDP) service on the organization's external IP address, \texttt{[Client IP]}. This vulnerability is listed with a CVSS score of 9.0 and presents a direct path for unauthorized remote access.

This critical technical flaw is dangerously amplified by significant organizational security gaps. Specifically, the lack of enforced Multi-Factor Authentication (MFA) for computer and email access dramatically increases the likelihood of a successful breach via credential compromise. Furthermore, the absence of an Acceptable Use Policy and recurring security awareness training indicates a need for foundational improvements in the organization's security culture.

Immediate remediation is required to address the exposed RDP service to prevent a potential security incident, such as a ransomware attack.

% ===================================================================
\section{Organizational Information}
% ===================================================================

The following information was used as the basis for this assessment. Placeholders are used where data was not provided.

\begin{itemize}
    \item \textbf{Organization Name:} \textbf{[Organization Name]}
    \item \textbf{Primary Domain:} \texttt{[Domain]}
    \item \textbf{External IP Address Assessed:} \texttt{[Client IP]}
\end{itemize}

% ===================================================================
\section{Security Control Review (Questionnaire)}
% ===================================================================

The following table summarizes the organization's responses to a security controls questionnaire. Each "No" response represents a gap in security posture and has been flagged with an associated risk level.

\begin{table}[h!]
\centering
\caption{Security Controls Questionnaire Analysis}
\begin{tabular}{p{7cm} c p{5cm}}
\toprule
\textbf{Control Question} & \textbf{Response} & \textbf{Analyst Notes} \\
\midrule
Do you require MFA to access email? & \no & \textbf{Critical Gap.} Email is a primary target for phishing. Lack of MFA exposes the organization to account takeovers. \\
\addlinespace
Do you require MFA to log into computers? & \no & \textbf{Critical Gap.} This significantly increases the risk of unauthorized access, especially when combined with exposed remote services like RDP. \\
\addlinespace
Do you require MFA to access sensitive data systems? & \yes & \textbf{Positive Control.} This is a strong practice, but its effectiveness is undermined by other systemic weaknesses. \\
\addlinespace
Does your organization have an employee acceptable use policy? & \no & \textbf{High Risk.} Without an AUP, employees lack clear guidelines on security responsibilities, data handling, and acceptable online behavior. \\
\addlinespace
Does your organization do security awareness training for new employees? & \yes & \textbf{Positive Control.} A good foundational step for onboarding. \\
\addlinespace
Does your organization do security awareness training for all employees at least once per year? & \no & \textbf{High Risk.} Security knowledge degrades over time. Annual training is essential to maintain awareness of evolving threats. \\
\bottomrule
\end{tabular}
\end{table}

% ===================================================================
\section{Technical Scan Results}
% ===================================================================

An external network scan was performed against the public IP address \texttt{[Target IP]}. The scan identified one open port, which presents a critical risk.

\begin{table}[h!]
\centering
\caption{Open Port Analysis}
\begin{tabular}{l l l p{7cm}}
\toprule
\textbf{Port} & \textbf{State} & \textbf{Service} & \textbf{Analyst Notes} \\
\midrule
3389/tcp & open & ms-wbt-server & \textbf{Critical Finding.} This port is used for Microsoft's Remote Desktop Protocol (RDP). Exposing RDP directly to the internet is extremely dangerous and is a primary vector for brute-force, credential stuffing, and ransomware attacks. This finding confirms the pre-existing risk identified in Input 3. \\
\bottomrule
\end{tabular}
\end{table}

% ===================================================================
\section{Correlated Risk Assessment}
% ===================================================================

The following table synthesizes findings from the security questionnaire, technical scans, and pre-existing risk data to provide a holistic view of the organization's security posture.

\begin{table}[h!]
\centering
\caption{Summary of Identified Risks}
\begin{tabular}{p{3cm} p{8cm} l}
\toprule
\textbf{Risk Title} & \textbf{Description} & \textbf{Severity} \\
\midrule
\textbf{Critical RDP Exposure} & Port 3389 (RDP) is open to the internet on \texttt{[Target IP]}. This allows attackers to attempt to gain direct remote control of the system. This was confirmed by both the network scan and pre-existing risk data. & \textbf{Critical} \\
\addlinespace
\textbf{Insufficient Identity and Access Management} & Multi-Factor Authentication (MFA) is not enforced for computer logins or email access. This weakness critically magnifies the risk of a successful RDP breach, as a single compromised password would be sufficient for access. & \textbf{Critical} \\
\addlinespace
\textbf{Foundational Policy \& Training Gaps} & The absence of a formal Acceptable Use Policy and a recurring annual security training program indicates a reactive, rather than proactive, security culture. This increases the likelihood of human error leading to a security incident. & \textbf{High} \\
\bottomrule
\end{tabular}
\end{table}

% ===================================================================
\section{Recommendations}
% ===================================================================

The following actionable recommendations are prioritized to address the identified risks in a logical and effective manner.

\subsection*{Priority 1: Immediate (To be completed within 24 hours)}
\begin{itemize}
    \item \textbf{Action:} Implement a firewall rule to \textbf{block all inbound traffic to TCP port 3389} on the external interface for IP address \texttt{[Client IP]}.
    \item \textbf{Justification:} This immediately mitigates the most critical threat of a remote takeover via the exposed RDP service. This is the single most important action to take.
\end{itemize}

\subsection*{Priority 2: Short-Term (To be completed within 1-4 weeks)}
\begin{itemize}
    \item \textbf{Action:} If remote access is required, deploy a Virtual Private Network (VPN) solution that requires MFA for all connections. Staff should connect to the VPN before initiating an RDP session.
    \item \textbf{Justification:} A VPN provides a secure, encrypted tunnel for remote access, eliminating the need to expose services like RDP directly to the internet.
    \item \textbf{Action:} Procure and enforce an MFA solution for all user accounts, prioritizing email (e.g., Office 365, Google Workspace) and computer logins (e.g., Duo, Windows Hello for Business).
    \item \textbf{Justification:} Protects against credential theft and provides a critical layer of defense against unauthorized access.
\end{itemize}

\subsection*{Priority 3: Mid-Term (To be completed within 1-3 months)}
\begin{itemize}
    \item \textbf{Action:} Develop, ratify, and distribute a formal Acceptable Use Policy (AUP) that all employees must read and acknowledge.
    \item \textbf{Justification:} Establishes clear security expectations and responsibilities for all employees regarding company assets and data.
    \item \textbf{Action:} Implement an annual, mandatory security awareness training program for all staff.
    \item \textbf{Justification:} Reinforces security best practices, keeps employees aware of evolving threats like phishing, and fosters a stronger security culture.
\end{itemize}

\end{document}
```