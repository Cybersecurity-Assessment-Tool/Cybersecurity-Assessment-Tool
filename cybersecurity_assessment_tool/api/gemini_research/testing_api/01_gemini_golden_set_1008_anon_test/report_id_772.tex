```latex
\documentclass[12pt]{article}

% Preamble: Required Packages
\usepackage[margin=1in]{geometry}
\usepackage{pifont} % For checkmarks and crosses
\usepackage{booktabs} % For professional tables
\usepackage{hyperref} % For clickable links
\usepackage{url} % For URL formatting
\usepackage{seqsplit} % To split long strings in tt font
\usepackage{graphicx} % For potential logo inclusion
\usepackage{xcolor} % For color definitions

% --- Document Metadata ---
\title{Cybersecurity Posture Assessment Report}
\author{Cybersecurity Analysis Division}
\date{\today}

% --- Custom Commands & Settings ---
\hypersetup{
    colorlinks=true,
    linkcolor=blue,
    filecolor=magenta,      
    urlcolor=cyan,
    pdftitle={Cybersecurity Posture Assessment Report},
    pdfpagemode=FullScreen,
}
\newcommand{\yes}{\ding{51}}
\newcommand{\no}{\ding{55}}

% ==============================================================================
% --- BEGIN DOCUMENT ---
% ==============================================================================
\begin{document}

\maketitle
\thispagestyle{empty}
\newpage

\tableofcontents
\newpage

% ==============================================================================
% 1. Executive Overview
% ==============================================================================
\section{Executive Overview}

This report details the findings of a cybersecurity posture assessment conducted for \textbf{[Organization Name]}. The assessment combined an external network scan, a review of existing risks, and an analysis of organizational security controls based on a provided questionnaire.

\paragraph{Key Findings:} The overall security posture presents a mixed landscape. On a positive note, the external network scan of the target asset revealed a strong defensive perimeter, with no open ports detected. This suggests a well-configured firewall and a minimized external attack surface for the scanned system.

However, significant and high-risk gaps were identified in internal security controls. The lack of mandatory Multi-Factor Authentication (MFA) for computer logins and the absence of a formal security awareness training program for employees represent critical vulnerabilities. These procedural and policy-based weaknesses expose the organization to substantial risks, including unauthorized access, credential theft, and susceptibility to social engineering attacks like phishing.

\paragraph{Summary:} While the external network defenses appear robust, the primary threats to the organization are internal and human-centric. Immediate action is required to address the identified gaps in endpoint security and employee training to mitigate these high-severity risks.

% ==============================================================================
% 2. Organizational Information
% ==============================================================================
\section{Organizational Information}

The following details were used as the basis for this assessment. Information not provided in the source data is marked with a placeholder.

\begin{itemize}
    \item \textbf{Organization Name:} \textbf{[Organization Name]}
    \item \textbf{Primary Domain:} \texttt{[Domain]}
    \item \textbf{Scanned IP Address:} \texttt{[Client IP]}
\end{itemize}

% ==============================================================================
% 3. Security Control Review (Questionnaire Analysis)
% ==============================================================================
\section{Security Control Review}

An analysis of the security questionnaire reveals the organization's current implementation of key security controls. "No" answers indicate significant gaps that increase organizational risk.

\begin{table}[h!]
\centering
\caption{Security Control Questionnaire Results}
\begin{tabular}{p{0.8\linewidth} c}
\toprule
\textbf{Control Question} & \textbf{Response} \\
\midrule
Do you require MFA to access email? & \yes \\
Do you require MFA to log into computers? & \textcolor{red}{\no} \\
Do you require MFA to access sensitive data systems? & \yes \\
Does your organization have an employee acceptable use policy? & \yes \\
Does your organization do security awareness training for new employees? & \textcolor{red}{\no} \\
Does your organization do security awareness training for all employees at least once per year? & \textcolor{red}{\no} \\
\bottomrule
\end{tabular}
\end{table}

\paragraph{Analysis:} The organization has successfully implemented MFA for email and sensitive systems, which is commendable. However, the absence of MFA on computer logins is a critical oversight. If an employee's credentials are stolen, an attacker could gain direct access to their workstation and potentially the internal network. Furthermore, the complete lack of a security awareness training program leaves employees unprepared to identify and respond to common cyber threats such as phishing, creating a vulnerable human perimeter.

% ==============================================================================
% 4. Technical Scan Results
% ==============================================================================
\section{Technical Scan Results}

An external network scan was performed to identify open ports and exposed services on the designated target system.

\begin{itemize}
    \item \textbf{Scan Target:} \texttt{[Target IP]}
    \item \textbf{Scan Status:} The target host was responsive (status: up).
    \item \textbf{Findings:} The scan confirmed that \textbf{zero open ports} were detected. All 1000 scanned TCP ports were in a 'closed' state.
\end{itemize}

\paragraph{Conclusion:} The absence of open ports is an excellent security posture for an external-facing asset. It indicates that the firewall is properly configured to deny all unsolicited incoming traffic, significantly reducing the external attack surface and protecting the system from network-based attacks.

% ==============================================================================
% 5. Risk Assessment Summary
% ==============================================================================
\section{Risk Assessment Summary}

The following table synthesizes findings from the security control review and technical scan. No pre-existing vulnerabilities were provided for this assessment. The identified risks are based on observed gaps.

\begin{table}[h!]
\centering
\caption{Identified Risks}
\begin{tabular}{p{0.1\linewidth} p{0.3\linewidth} p{0.4\linewidth} p{0.1\linewidth}}
\toprule
\textbf{ID} & \textbf{Risk Name} & \textbf{Description} & \textbf{Severity} \\
\midrule
\textbf{R-01} & Lack of Endpoint Multi-Factor Authentication & The absence of MFA for computer logins allows an attacker with stolen credentials to gain direct access to an endpoint and the internal network. & \textbf{High} \\
\addlinespace
\textbf{R-02} & Inadequate Security Awareness Training Program & Without training, employees are more likely to fall victim to phishing, social engineering, and other attacks, potentially leading to credential compromise or malware infection. & \textbf{High} \\
\bottomrule
\end{tabular}
\end{table}

% ==============================================================================
% 6. Recommendations
% ==============================================================================
\section{Recommendations}

The following actionable recommendations are provided to mitigate the high-severity risks identified during this assessment.

\subsection{Risk R-01: Lack of Endpoint MFA}
\begin{itemize}
    \item \textbf{Action:} Implement and enforce a mandatory Multi-Factor Authentication (MFA) policy for all employee computer logins (desktops and laptops).
    \item \textbf{Justification:} This creates a critical secondary layer of defense. Even if an employee's password is compromised, the attacker will be unable to access the system without the second factor (e.g., a code from a mobile app, a physical security key).
    \item \textbf{Priority:} \textbf{Critical}. This should be addressed immediately.
\end{itemize}

\subsection{Risk R-02: Inadequate Security Awareness Training}
\begin{itemize}
    \item \textbf{Action:} Develop and implement a formal, mandatory security awareness training program. This program should include:
    \begin{enumerate}
        \item Onboarding training for all new employees before they are granted system access.
        \item Annual refresher training for all staff covering current threats like phishing, ransomware, and proper data handling.
        \item Periodic simulated phishing campaigns to test and reinforce employee knowledge.
    \end{enumerate}
    \item \textbf{Justification:} A well-trained workforce is the first line of defense against many common cyberattacks. Training reduces the likelihood of human error leading to a security breach.
    \item \textbf{Priority:} \textbf{High}. A program should be developed and implemented within the next quarter.
\end{itemize}

\end{document}
% ==============================================================================
% --- END DOCUMENT ---
% ==============================================================================
```