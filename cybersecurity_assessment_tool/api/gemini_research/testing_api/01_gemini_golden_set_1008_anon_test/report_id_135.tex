```latex
\documentclass[12pt]{article}

% --- PACKAGES ---
\usepackage[margin=1in]{geometry} % Set page margins
\usepackage{pifont}               % For checkmarks and crosses (\ding)
\usepackage{booktabs}             % For professional-looking tables
\usepackage{hyperref}             % For hyperlinks
\usepackage{url}                  % For URL formatting
\usepackage{seqsplit}             % For splitting long strings in \texttt
\usepackage{graphicx}             % For logos (optional)
\usepackage{xcolor}               % For colors

% --- DOCUMENT INFORMATION ---
\title{Cybersecurity Posture Assessment Report}
\author{Cybersecurity Analyst}
\date{\today}

% --- HYPERREF SETUP ---
\hypersetup{
    colorlinks=true,
    linkcolor=blue,
    filecolor=magenta,      
    urlcolor=cyan,
    pdftitle={Cybersecurity Posture Assessment Report},
    pdfpagemode=FullScreen,
}

% --- DOCUMENT START ---
\begin{document}

\maketitle

% ===================================================================
% SECTION 1: EXECUTIVE OVERVIEW
% ===================================================================
\section*{1. Executive Overview}

This report details the findings of a cybersecurity assessment for \textbf{[Organization Name]}. The analysis is based on network scans, a security controls questionnaire, and a review of pre-existing risks.

The assessment has identified several critical and high-severity risks that require immediate attention. The most critical finding is the direct exposure of a MySQL database service (\texttt{5.7.33}) to the network on port \texttt{3306}. This version of MySQL is officially End-of-Life (EOL) as of October 2023 and no longer receives security updates, elevating the risk of exploitation significantly.

Furthermore, a critical gap was identified in the organization's security culture: the complete absence of a security awareness training program for both new and existing employees. This deficiency makes the organization highly susceptible to social engineering attacks, such as phishing.

Immediate remediation of the exposed database and implementation of a security training program are the highest priorities. Detailed recommendations are provided in Section 6.

% ===================================================================
% SECTION 2: ORGANIZATIONAL INFORMATION
% ===================================================================
\section*{2. Organizational Information}

The following details were used as the basis for this assessment. Information that was not provided has been marked with a placeholder.

\begin{tabular}{@{}ll}
\toprule
\textbf{Attribute} & \textbf{Value} \\
\midrule
Organization Name & \textbf{[Organization Name]} \\
Email Domain & \texttt{[Domain]} \\
External IP Scanned & \texttt{[Client IP]} \\
\bottomrule
\end{tabular}

% ===================================================================
% SECTION 3: SECURITY CONTROL REVIEW
% ===================================================================
\section*{3. Security Control Review}

A review of organizational security controls was conducted based on a questionnaire. The results below highlight key strengths and critical gaps in current security practices. The checkmark (\ding{51}) indicates a positive control is in place, while the cross (\ding{55}) indicates a control gap.

\subsection*{3.1 Questionnaire Results}

\begin{tabular}{@{}p{0.8\linewidth}c@{}}
\toprule
\textbf{Control Question} & \textbf{Status} \\
\midrule
Do you require MFA to access email? & \textcolor{green}{\ding{51}} \\
Do you require MFA to log into computers? & \textcolor{green}{\ding{51}} \\
Do you require MFA to access sensitive data systems? & \textcolor{green}{\ding{51}} \\
Does your organization have an employee acceptable use policy? & \textcolor{green}{\ding{51}} \\
\addlinespace
Does your organization do security awareness training for new employees? & \textcolor{red}{\ding{55}} \\
Does your organization do security awareness training for all employees at least once per year? & \textcolor{red}{\ding{55}} \\
\bottomrule
\end{tabular}

\subsection*{3.2 Analysis}
The organization has successfully implemented Multi-Factor Authentication (MFA) across key systems, which is a commendable strength. However, the complete lack of a security awareness training program represents a critical vulnerability. Without training, employees are significantly more likely to fall victim to phishing, malware, and other social engineering tactics, potentially bypassing other technical controls.

% ===================================================================
% SECTION 4: TECHNICAL SCAN RESULTS
% ===================================================================
\section*{4. Technical Scan Results}

An external network scan was performed on the target IP address \texttt{[Target IP]} to identify open ports and exposed services.

\subsection*{4.1 Open Ports and Services}

\begin{tabular}{@{}lllll@{}}
\toprule
\textbf{Port} & \textbf{State} & \textbf{Service} & \textbf{Product} & \textbf{Version} \\
\midrule
3306/tcp & open & mysql & MySQL & 5.7.33 \\
\bottomrule
\end{tabular}

\subsection*{4.2 Analysis}
The scan identified that TCP port \texttt{3306}, the default port for MySQL databases, is open to the public. Direct exposure of a database to the internet is a severe security risk, as it allows attackers to directly target the service with brute-force attacks, credential stuffing, and exploits for known vulnerabilities.

\textbf{Critical Finding:} The identified MySQL version, \textbf{5.7.33}, reached its official End-of-Life (EOL) in October 2023. This means it no longer receives security patches from the vendor, and any vulnerabilities discovered since that date remain unpatched. An exposed, EOL database service is a prime target for attackers and poses a critical threat to data confidentiality, integrity, and availability.

% ===================================================================
% SECTION 5: RISK ASSESSMENT SUMMARY
% ===================================================================
\section*{5. Risk Assessment Summary}

The following table correlates findings from the security control review, technical scan, and pre-existing risk data into a prioritized list.

\begin{tabular}{@{}p{0.25\linewidth}p{0.1\linewidth}p{0.6\linewidth}@{}}
\toprule
\textbf{Risk Name} & \textbf{Severity} & \textbf{Overview \& Affected Elements} \\
\midrule
\addlinespace
\textbf{Exposed End-of-Life Database Service} & \textbf{Critical (9.8)} & A MySQL database (v5.7.33) is publicly accessible on port 3306. This version is End-of-Life and unpatched, making it highly vulnerable to compromise. \newline \textit{Affected: \texttt{[Target IP]}:3306}} \\
\addlinespace
\textbf{Lack of Security Awareness Training} & \textbf{High (7.8)} & The organization does not conduct security awareness training for new or existing employees. This significantly increases the risk of successful phishing and social engineering attacks. \newline \textit{Affected: All Employees} \\
\bottomrule
\end{tabular}

% ===================================================================
% SECTION 6: RECOMMENDATIONS
% ===================================================================
\section*{6. Recommendations}

The following actions are recommended to mitigate the identified risks. Recommendations are prioritized based on severity and potential impact.

\subsection*{6.1 Critical Risk: Exposed End-of-Life Database Service}
\begin{itemize}
    \item \textbf{Immediate (0-7 days):} Implement strict firewall rules to deny all public access to port \texttt{3306}. Access should be restricted to a minimal set of trusted IP addresses, such as an office or VPN gateway.
    \item \textbf{Short-Term (1-3 months):} Plan and execute the migration of the MySQL 5.7 database to a currently supported version (e.g., MySQL 8.x). This is essential to ensure the service receives ongoing security updates.
    \item \textbf{Long-Term (3-6 months):} Re-architect database access to eliminate any direct public exposure. Implement a bastion host (jump box) or require a Virtual Private Network (VPN) connection for all administrative access.
\end{itemize}

\subsection*{6.2 High Risk: Lack of Security Awareness Training}
\begin{itemize}
    \item \textbf{Immediate (0-30 days):} Procure and deploy a reputable security awareness training platform for all employees.
    \item \textbf{Short-Term (1-3 months):} Ensure all current employees complete a baseline training module covering key topics like phishing, password security, and acceptable use. Integrate this training into the mandatory onboarding process for all new hires.
    \item \textbf{Long-Term (Ongoing):} Establish a continuous security awareness program that includes annual refresher training and periodic phishing simulations to test and reinforce employee knowledge.
\end{itemize}

\end{document}
```