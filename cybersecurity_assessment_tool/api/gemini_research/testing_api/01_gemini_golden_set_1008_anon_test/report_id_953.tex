```latex
\documentclass[12pt]{article}

% Preamble: Required Packages
\usepackage[margin=1in]{geometry}
\usepackage{pifont} % For \ding symbols (checkmarks and crosses)
\usepackage{booktabs} % For professional-quality tables
\usepackage{hyperref} % For clickable links and references
\usepackage{url}      % For formatting URLs
\usepackage{seqsplit} % To split long strings without breaking words
\usepackage{xcolor}   % For custom colors
\usepackage{graphicx}

% --- Document Setup ---
\hypersetup{
    colorlinks=true,
    linkcolor=blue,
    urlcolor=cyan,
}

% --- Custom Commands ---
\newcommand{\yes}{\textcolor{green}{\ding{51}}} % Green checkmark for "Yes"
\newcommand{\no}{\textcolor{red}{\ding{55}}}   % Red X for "No"

% --- Title Information ---
\title{Cybersecurity Posture Assessment Report}
\author{Automated Security Analysis Engine}
\date{\today}


\begin{document}

\maketitle
\hrule
\vspace{1em}
\begin{center}
    \textbf{Client:} \textbf{[Organization Name]} \\
    \textbf{Report ID:} REP-2023-451
\end{center}
\vspace{1em}
\hrule
\newpage

\tableofcontents
\newpage

% ==============================================================================
% SECTION 1: EXECUTIVE SUMMARY
% ==============================================================================
\section{Executive Summary}

This report provides a comprehensive analysis of the cybersecurity posture for \textbf{[Organization Name]}, based on technical network scans, a review of existing risks, and an organizational security controls questionnaire.

The assessment identified a \textbf{critical risk}: The direct exposure of a Remote Desktop Protocol (RDP) service on port 3389 to the public internet. This finding was confirmed by both the active network scan and pre-existing risk documentation. This configuration is highly dangerous and is a common entry point for ransomware attacks and unauthorized network access.

This critical technical vulnerability is significantly amplified by several key organizational security gaps:
\begin{itemize}
    \item \textbf{Lack of Multi-Factor Authentication (MFA):} MFA is not required for computer logins or access to sensitive data systems. This means a compromised password is all an attacker needs to gain access.
    \item \textbf{Inadequate Security Training:} New employees do not receive security awareness training, making them more susceptible to phishing attacks designed to steal credentials.
\end{itemize}

The combination of an exposed, high-value service (RDP) with weak access controls (no MFA) and a higher likelihood of credential compromise (lack of training) creates a high-impact, high-probability attack path. Immediate remediation is required to mitigate the risk of a significant security breach.

% ==============================================================================
% SECTION 2: ORGANIZATIONAL INFORMATION
% ==============================================================================
\section{Organizational Information}

This section details the information provided about the organization. The data has been anonymized as per the analysis template.

\begin{tabular}{@{}ll}
    \toprule
    \textbf{Attribute} & \textbf{Value} \\
    \midrule
    Organization Name & \textbf{[Organization Name]} \\
    Email Domain & \texttt{[Domain]} \\
    External IP Address (Client) & \texttt{[Client IP]} \\
    \bottomrule
\end{tabular}

% ==============================================================================
% SECTION 3: SECURITY CONTROL REVIEW
% ==============================================================================
\section{Security Control Review}

The following table summarizes the organization's responses to the security controls questionnaire. Answers marked with \no\ indicate significant gaps in the security framework that require attention.

\begin{tabular}{@{}p{12cm}c@{}}
    \toprule
    \textbf{Control Question} & \textbf{Status} \\
    \midrule
    Do you require MFA to access email? & \yes \\
    \textbf{Do you require MFA to log into computers?} & \no \\
    \textbf{Do you require MFA to access sensitive data systems?} & \no \\
    Does your organization have an employee acceptable use policy? & \yes \\
    \textbf{Does your organization do security awareness training for new employees?} & \no \\
    Does your organization do security awareness training for all employees at least once per year? & \yes \\
    \bottomrule
\end{tabular}

\subsection*{Analysis of Gaps}
The lack of MFA for computer and sensitive system access, combined with the absence of security training for new hires, constitutes a severe weakness in the organization's defense-in-depth strategy.

% ==============================================================================
% SECTION 4: TECHNICAL SCAN RESULTS
% ==============================================================================
\section{Technical Scan Results}

An external network scan was performed to identify open ports and exposed services on the organization's perimeter.

\begin{itemize}
    \item \textbf{Target IP Address:} \texttt{[Target IP]}
    \item \textbf{Scan Status:} Host is UP.
\end{itemize}

\subsection*{Open Ports Discovered}
The following table details the ports found to be open and accessible from the public internet.

\begin{tabular}{@{}llll@{}}
    \toprule
    \textbf{Port} & \textbf{State} & \textbf{Service Name} & \textbf{Description} \\
    \midrule
    3389/tcp & open & \texttt{ms-wbt-server} & Microsoft Remote Desktop Protocol (RDP) \\
    \bottomrule
\end{tabular}

\subsection*{Technical Finding}
The scan confirms that port 3389 is open, exposing the RDP service. This service is a primary target for brute-force password attacks and exploitation of known vulnerabilities. Exposing RDP directly to the internet is strongly discouraged by all major cybersecurity frameworks and authorities.

% ==============================================================================
% SECTION 5: CORRELATED RISK ASSESSMENT
% ==============================================================================
\section{Correlated Risk Assessment}

This section synthesizes findings from all data sources into a prioritized list of identified risks.

\begin{tabular}{@{}p{4cm}p{1.5cm}p{8.5cm}@{}}
    \toprule
    \textbf{Risk / Finding} & \textbf{Severity} & \textbf{Description} \\
    \midrule
    \textbf{Publicly Exposed RDP Service} & \textbf{Critical (9.0)} & The technical scan and pre-existing risk data confirm that RDP is exposed on \texttt{[Target IP]}. This allows attackers to directly attempt to compromise a critical entry point into the internal network. \\
    \addlinespace
    \textbf{Insufficient MFA Implementation} & \textbf{High} & The lack of MFA on computer logins and sensitive systems means that a single compromised password could lead to a full network breach, especially when combined with the exposed RDP service. \\
    \addlinespace
    \textbf{Inadequate Security Training Program} & \textbf{High} & New employees are not trained on security best practices. This increases the organization's susceptibility to phishing and social engineering, which are the primary methods for stealing the credentials needed to exploit the other identified weaknesses. \\
    \bottomrule
\end{tabular}

% ==============================================================================
% SECTION 6: RECOMMENDATIONS
% ==============================================================================
\section{Recommendations}

The following actionable recommendations are provided to mitigate the identified risks. They are prioritized based on severity and ease of implementation.

\subsection*{Priority 1: Immediate Actions (Within 24 Hours)}
\begin{enumerate}
    \item \textbf{Block RDP Access:} Immediately configure the perimeter firewall to block all inbound traffic to TCP port 3389 on host \texttt{[Target IP]}. This is the single most important step to prevent an imminent breach.
\end{enumerate}

\subsection*{Priority 2: Short-Term Actions (1-4 Weeks)}
\begin{enumerate}
    \item \textbf{Deploy MFA:} Implement and enforce a mandatory MFA policy for all remote access, all computer logins (especially for privileged accounts), and all access to systems containing sensitive data.
    \item \textbf{Implement New Hire Training:} Develop and deploy a mandatory security awareness training module for all new employees as part of the onboarding process.
\end{enumerate}

\subsection*{Priority 3: Long-Term Strategic Actions (1-3 Months)}
\begin{enumerate}
    \item \textbf{Deploy a VPN Solution:} For necessary remote access, implement a secure Virtual Private Network (VPN) solution. The VPN should be configured to require MFA for all connections. This provides a secure, encrypted tunnel for remote administration, replacing the need for direct RDP exposure.
    \item \textbf{Conduct a Credential Audit:} Perform a comprehensive audit of all user accounts and passwords. Enforce a strong password policy and force a password reset for all users, as it is unknown if any credentials have already been compromised.
\end{enumerate}

\end{document}
```