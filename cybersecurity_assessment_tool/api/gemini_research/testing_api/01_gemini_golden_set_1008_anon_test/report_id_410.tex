```latex
\documentclass[12pt]{article}

% Preamble: Required Packages
\usepackage[margin=1in]{geometry}
\usepackage{pifont} % For checkmarks (\ding{51}) and crosses (\ding{55})
\usepackage{booktabs} % For professional-looking tables
\usepackage{hyperref} % For hyperlinks and metadata
\usepackage{url}      % For proper URL formatting
\usepackage{seqsplit} % For splitting long text strings in \texttt
\usepackage{xcolor}   % For custom colors
\usepackage{graphicx}

% --- Document Setup ---
\hypersetup{
    colorlinks=true,
    linkcolor=blue,
    filecolor=magenta,
    urlcolor=cyan,
    pdftitle={Cybersecurity Posture Assessment Report},
    pdfauthor={Cybersecurity Analysis Division},
    pdfsubject={Security Assessment},
    pdfkeywords={Cybersecurity, Nmap, Risk, Analysis}
}

% --- Custom Commands ---
\newcommand{\yes}{\textcolor{green}{\ding{51}}}
\newcommand{\no}{\textcolor{red}{\ding{55}}}
\newcommand{\orgname}{\textbf{[Organization Name]}}
\newcommand{\clientip}{\texttt{[Client IP]}}
\newcommand{\clientdomain}{\texttt{[Domain]}}
\newcommand{\targetip}{\texttt{[Target IP]}}

\begin{document}

% --- Title Page ---
\title{Cybersecurity Posture Assessment Report \\ \large For: \orgname}
\author{Cybersecurity Analysis Division}
\date{\today}
\maketitle
\thispagestyle{empty}
\newpage

\tableofcontents
\newpage

% --- Section 1: Executive Overview ---
\section{Executive Overview}
This report details the findings of a cybersecurity posture assessment conducted for \orgname. The assessment combined a review of organizational security controls, an external network scan, and an analysis of pre-existing risks.

The analysis revealed several \textbf{critical and high-risk vulnerabilities} that require immediate attention. Key findings include:
\begin{itemize}
    \item \textbf{Critical Network Exposure:} An externally facing FTP server was identified running a dangerously outdated and vulnerable version of \texttt{vsftpd} (2.3.4). This specific version is widely known to contain a backdoor, allowing for remote code execution. The service is further misconfigured to allow anonymous logins, posing a severe and immediate threat of system compromise and data breach.
    \item \textbf{Critical Access Control Gap:} Multi-Factor Authentication (MFA) is not enforced for accessing sensitive data systems. This significantly increases the risk of unauthorized access through compromised credentials.
    \item \textbf{High-Risk Policy Gap:} The organization lacks a formal Employee Acceptable Use Policy, creating ambiguity regarding the secure use of company assets and limiting enforcement capabilities.
    \item \textbf{Pre-existing Medium Risk:} The organization is aware of an ongoing risk related to outdated Windows 7 workstations, which are no longer supported and do not receive security updates.
\end{itemize}

Immediate remediation of the exposed FTP server and implementation of MFA are paramount to mitigating the most severe threats to the organization. Detailed findings and actionable recommendations are provided in the subsequent sections.

% --- Section 2: Organizational Information ---
\section{Organizational Information}
This section provides general information about the organization and the scope of this assessment. The data is based on information provided prior to the engagement.

\begin{tabular}{@{}ll}
    \toprule
    \textbf{Detail} & \textbf{Information} \\
    \midrule
    Organization Name & \orgname \\
    Primary Domain & \clientdomain \\
    Client External IP & \clientip \\
    Assessment Target IP & \targetip \\
    Assessment Date & \today \\
    \bottomrule
\end{tabular}

% --- Section 3: Security Control Review ---
\section{Security Control Review}
The following table summarizes the organization's responses to a security controls questionnaire. Gaps in these controls often represent significant organizational risks.

\begin{tabular}{@{}p{0.6\linewidth} c p{0.25\linewidth}@{}}
    \toprule
    \textbf{Control Question} & \textbf{Status} & \textbf{Analyst Assessment} \\
    \midrule
    Do you require MFA to access email? & \yes & Strong control. Protects primary communication channel. \\
    \addlinespace
    Do you require MFA to log into computers? & \yes & Good practice for endpoint security. \\
    \addlinespace
    \textbf{Do you require MFA to access sensitive data systems?} & \no & \textbf{Critical Gap.} Lack of MFA on critical systems is a primary vector for data breaches. \\
    \addlinespace
    \textbf{Does your organization have an employee acceptable use policy?} & \no & \textbf{High Risk.} Absence of a foundational policy leads to inconsistent practices and lack of enforceability. \\
    \addlinespace
    Does your organization do security awareness training for new employees? & \yes & Excellent. Establishes a security baseline from day one. \\
    \addlinespace
    Does your organization do security awareness training for all employees at least once per year? & \yes & Strong ongoing practice. Maintains security awareness. \\
    \bottomrule
\end{tabular}

% --- Section 4: Technical Scan Results ---
\section{Technical Scan Results}
An external network scan was performed on the target IP address \targetip\ to identify open ports and exposed services.

\subsection{Host Status}
The target host was found to be \textbf{up} and responsive to network probes.

\subsection{Open Ports and Services}
The following table details the services discovered on the target system.

\begin{tabular}{@{}l l l l p{0.4\linewidth}@{}}
    \toprule
    \textbf{Port} & \textbf{State} & \textbf{Service} & \textbf{Version} & \textbf{Notes} \\
    \midrule
    21/tcp & Open & ftp & vsftpd 2.3.4 & \textbf{CRITICAL FINDING.} Anonymous FTP login is allowed. This version contains a well-known backdoor vulnerability (CVE-2011-2523) that can lead to remote command execution. \\
    \bottomrule
\end{tabular}

% --- Section 5: Consolidated Risk Assessment ---
\section{Consolidated Risk Assessment}
This section synthesizes findings from the questionnaire, technical scan, and pre-existing risk data into a consolidated list of identified risks.

\begin{tabular}{@{}p{0.3\linewidth} p{0.4\linewidth} l@{}}
    \toprule
    \textbf{Risk Name} & \textbf{Overview} & \textbf{Severity} \\
    \midrule
    \textbf{Exposed Vulnerable FTP Server} & An outdated FTP server (\texttt{vsftpd 2.3.4}) with a known backdoor vulnerability is exposed to the internet and allows anonymous login. & \textbf{Critical} \\
    \addlinespace
    \textbf{Lack of MFA on Sensitive Systems} & Critical data systems can be accessed with only a username and password, making them highly susceptible to credential theft and compromise. & \textbf{Critical} \\
    \addlinespace
    \textbf{Missing Acceptable Use Policy} & The absence of a formal policy creates inconsistent security behavior and removes the framework for enforcing secure practices. & \textbf{High} \\
    \addlinespace
    \textbf{Outdated Windows Policy} & Workstations are running Windows 7, an end-of-life operating system that no longer receives security updates from Microsoft. & \textbf{Medium} \\
    \bottomrule
\end{tabular}

% --- Section 6: Recommendations ---
\section{Recommendations}
The following actionable recommendations are provided to mitigate the identified risks. They are prioritized based on severity and potential impact.

\subsection*{Priority 1: Immediate Actions (Due within 48 hours)}
\begin{enumerate}
    \item \textbf{Remediate Exposed FTP Server:}
    \begin{itemize}
        \item Immediately disable or firewall port 21 on \targetip\ to remove the public-facing threat.
        \item If FTP is a business requirement, upgrade the \texttt{vsftpd} service to the latest stable version and disable anonymous access.
        \item Consider using more secure file transfer alternatives like SFTP (SSH File Transfer Protocol).
    \end{itemize}
    \item \textbf{Implement MFA on Sensitive Systems:}
    \begin{itemize}
        \item Immediately begin the process of enabling MFA on all systems identified as containing sensitive data.
        \item Prioritize administrator accounts and then expand to all user accounts with access to this data.
    \end{itemize}
\end{enumerate}

\subsection*{Priority 2: High-Priority Actions (Due within 30 days)}
\begin{enumerate}
    \setcounter{enumi}{2} % Continue numbering
    \item \textbf{Develop and Implement an Acceptable Use Policy (AUP):}
    \begin{itemize}
        \item Draft a formal AUP that clearly defines the rules for using company networks, systems, and data.
        \item Communicate the policy to all employees and require them to formally acknowledge it.
    \end{itemize}
\end{enumerate}

\subsection*{Priority 3: Medium-Priority Actions (Ongoing)}
\begin{enumerate}
    \setcounter{enumi}{3} % Continue numbering
    \item \textbf{Accelerate Operating System Upgrades:}
    \begin{itemize}
        \item Continue with the existing plan to upgrade all Windows 7 workstations to a modern, supported operating system (e.g., Windows 10/11).
        \item Isolate any remaining Windows 7 machines from the main network if they cannot be immediately upgraded.
    \end{itemize}
\end{enumerate}

\end{document}
```