```latex
\documentclass[12pt]{article}

% Preamble: Required Packages
\usepackage[margin=1in]{geometry}
\usepackage{pifont} % For checkmarks and crosses
\usepackage{booktabs} % For professional tables
\usepackage{hyperref} % For hyperlinks and metadata
\usepackage{url} % For formatting URLs
\usepackage{seqsplit} % For splitting long strings in tt font
\usepackage{graphicx} % For potential logos
\usepackage{xcolor} % For colors

% Document Metadata
\hypersetup{
    colorlinks=true,
    linkcolor=blue,
    filecolor=magenta,      
    urlcolor=cyan,
    pdftitle={Cybersecurity Assessment Report},
    pdfauthor={Cybersecurity Analyst},
    pdfsubject={Security Assessment},
    pdfkeywords={Security, Analysis, Report},
    bookmarks=true
}

% Define custom colors for severity
\definecolor{criticalred}{HTML}{D7263D}
\definecolor{highorange}{HTML}{F49D40}
\definecolor{mediumyellow}{HTML}{F4D440}

% Custom command for severity labels
\newcommand{\severity}[2]{\colorbox{#1}{\textcolor{white}{\textbf{\,#2\,}}}}

\begin{document}

% --- Title Page ---
\begin{titlepage}
    \centering
    \vspace*{1cm}
    \Huge\textbf{Cybersecurity Assessment Report}
    \vspace{1.5cm}
    \large
    \begin{tabular}{ll}
        \textbf{Client:} & \textbf{[Organization Name]} \\
        \textbf{Date of Report:} & \today \\
        \textbf{Date of Assessment:} & November 22, 2025 \\
    \end{tabular}
    \vfill
    \large
    \textit{This report contains sensitive information and is intended for the exclusive use of the client. Distribution is strictly prohibited.}
\end{titlepage}

\tableofcontents
\newpage

% --- Executive Summary ---
\section{Executive Summary}
This report details the findings of a cybersecurity assessment conducted on November 22, 2025. The assessment combined a review of organizational security controls, an external network scan, and an analysis of pre-existing risks.

The overall security posture of \textbf{[Organization Name]} is assessed as \textbf{High-Risk}. This is primarily due to several critical deficiencies in fundamental security controls. Key findings include:
\begin{itemize}
    \item \textbf{Critical Lack of Multi-Factor Authentication (MFA):} MFA is not enforced for accessing email or for logging into employee computers. This exposes the organization to a high likelihood of account compromise through phishing and credential theft.
    \item \textbf{Absence of Security Awareness Program:} The organization lacks a formal security awareness training program for new or existing employees, and does not have an Acceptable Use Policy. This significantly increases the risk posed by human error.
    \item \textbf{Vulnerable External Service:} The external network scan identified a public-facing Nginx web server running version 1.18.0. This version is outdated, no longer supported, and has multiple known public vulnerabilities, posing a direct threat of compromise.
\end{itemize}

Immediate remediation is required to address these gaps. Recommendations are provided in Section \ref{sec:recommendations} to mitigate the identified risks and improve the organization's defensive capabilities.

% --- Organizational Information ---
\section{Organizational Information}
The following details were used as the basis for this assessment. As per the provided data, placeholder values are used where specific information was not available.

\begin{table}[h!]
\centering
\begin{tabular}{@{}ll@{}}
\toprule
\textbf{Attribute} & \textbf{Value} \\ \midrule
Organization Name & \textbf{[Organization Name]} \\
Primary Email Domain & \texttt{[Domain]} \\
External IP Address (Target) & \texttt{[Client IP]} \\ \bottomrule
\end{tabular}
\caption{Client Organizational Details.}
\end{table}

% --- Security Control Review ---
\section{Security Control Review}
A review of administrative and policy-based security controls was conducted based on a standardized questionnaire. The results reveal significant gaps in foundational security practices. A "No" response indicates a missing control and a potential area of high risk.

\begin{table}[h!]
\centering
\begin{tabular}{@{}p{0.75\linewidth}c@{}}
\toprule
\textbf{Control Question} & \textbf{Response} \\ \midrule
Do you require MFA to access email? & \ding{55} \\
Do you require MFA to log into computers? & \ding{55} \\
Do you require MFA to access sensitive data systems? & \ding{51} \\
Does your organization have an employee acceptable use policy? & \ding{55} \\
Does your organization do security awareness training for new employees? & \ding{55} \\
Does your organization do security awareness training for all employees at least once per year? & \ding{55} \\ \bottomrule
\end{tabular}
\caption{Security Controls Questionnaire Results (\ding{51}=Yes, \ding{55}=No).}
\label{tab:controls}
\end{table}

\subsection*{Analysis}
The responses in Table \ref{tab:controls} highlight a systemic weakness in user and policy management. The lack of MFA for email and workstations is a critical vulnerability. Furthermore, the absence of an Acceptable Use Policy and any form of security awareness training leaves the organization highly susceptible to social engineering, phishing, and insider threats, whether malicious or unintentional.

% --- Technical Scan Results ---
\section{Technical Scan Results}
An external network scan was performed to identify open ports and services visible on the public internet.

\begin{itemize}
    \item \textbf{Scan Date:} November 22, 2025
    \item \textbf{Target IP Address:} \texttt{[Target IP]}
\end{itemize}

\begin{table}[h!]
\centering
\begin{tabular}{@{}lllll@{}}
\toprule
\textbf{Port} & \textbf{State} & \textbf{Service} & \textbf{Product} & \textbf{Version} \\ \midrule
443/tcp & open & https & nginx & 1.18.0 \\ \bottomrule
\end{tabular}
\caption{Open Ports and Services Identified.}
\label{tab:scanresults}
\end{table}

\subsection*{Analysis}
The scan identified a single open port, 443/TCP, running an Nginx web server. The detected version, \textbf{1.18.0}, was released in April 2020 and is now considered end-of-life. This version is known to be affected by several security vulnerabilities, including but not limited to:
\begin{itemize}
    \item \textbf{CVE-2021-23017:} A DNS resolver vulnerability that could allow an attacker to smuggle DNS packets, potentially leading to cache poisoning or other attacks.
\end{itemize}
Running outdated and unsupported software on an internet-facing server presents a high risk of system compromise.

% --- Risk Assessment ---
\section{Risk Assessment}
This section synthesizes the findings from the security control review and the technical scan. No pre-existing vulnerabilities were reported. The following new risks have been identified and prioritized.

\begin{table}[h!]
\centering
\begin{tabular}{@{}lp{0.5\linewidth}l@{}}
\toprule
\textbf{Risk ID} & \textbf{Finding} & \textbf{Severity} \\ \midrule
RISK-001 & \textbf{Lack of Multi-Factor Authentication (MFA)} \newline \small{MFA is not enforced on email or computer logins, allowing for trivial account takeovers if credentials are stolen.} & \severity{criticalred}{Critical} \\
\addlinespace
RISK-002 & \textbf{Inadequate Security Awareness Program} \newline \small{No AUP or security training exists, leading to a high probability of successful phishing and social engineering attacks.} & \severity{highorange}{High} \\
\addlinespace
RISK-003 & \textbf{Vulnerable External Web Server} \newline \small{The public-facing Nginx 1.18.0 server is outdated, unsupported, and has known exploitable vulnerabilities.} & \severity{highorange}{High} \\
\bottomrule
\end{tabular}
\caption{Summary of Identified Risks.}
\label{tab:risks}
\end{table}

% --- Recommendations ---
\section{Recommendations}
\label{sec:recommendations}
The following actionable recommendations are provided to mitigate the identified risks. They are prioritized based on severity and potential impact.

\subsection{Immediate Priority (Critical)}
\begin{itemize}
    \item \textbf{RISK-001: Implement MFA Immediately.} Enforce MFA across all user accounts for email access (e.g., Office 365, Google Workspace) and workstation logins (e.g., Windows Hello, Duo). This is the single most effective control to prevent account compromise.
    \item \textbf{RISK-003: Remediate Vulnerable Nginx Server.} Immediately upgrade the Nginx server to the latest stable version. If an upgrade is not possible, apply virtual patches through a Web Application Firewall (WAF) and create a plan to migrate the service to a supported platform.
\end{itemize}

\subsection{High Priority}
\begin{itemize}
    \item \textbf{RISK-002: Establish a Security Awareness Program.} 
    \begin{enumerate}
        \item Develop and enforce an employee Acceptable Use Policy (AUP) that clearly defines rules for using company assets.
        \item Implement mandatory security awareness training for all new hires during their onboarding process.
        \item Conduct annual security awareness training and regular phishing simulations for all employees to build a security-conscious culture.
    \end{enumerate}
\end{itemize}

\subsection{Medium Priority}
\begin{itemize}
    \item \textbf{Establish a Patch Management Program:} Develop and implement a formal policy for regularly identifying, testing, and deploying security patches for all systems and software to prevent future instances of vulnerable software exposure.
    \item \textbf{Conduct Regular Vulnerability Scanning:} Implement a recurring, authenticated vulnerability scanning program for both internal and external assets to proactively identify and remediate security weaknesses.
\end{itemize}

\end{document}
```