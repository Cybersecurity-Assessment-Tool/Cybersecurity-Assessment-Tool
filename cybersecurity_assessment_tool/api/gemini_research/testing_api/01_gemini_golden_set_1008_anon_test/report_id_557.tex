```latex
\documentclass[12pt]{article}

% ----------------------------------------------------------------------
% METADATA & PACKAGES
% ----------------------------------------------------------------------
\usepackage[margin=1in]{geometry}
\usepackage{pifont} % For checkmarks and crosses
\usepackage{booktabs} % For professional tables
\usepackage{xcolor}   % For coloring text
\usepackage{seqsplit} % For splitting long strings to prevent overflow
\usepackage{hyperref}
\usepackage{url}

% Hyperlink setup
\hypersetup{
    colorlinks=true,
    linkcolor=blue,
    filecolor=magenta,
    urlcolor=cyan,
    pdftitle={Cybersecurity Posture Assessment Report},
    pdfauthor={Cybersecurity Analysis Division},
}

% Custom commands for status indicators
\newcommand{\yes}{\textcolor{green}{\ding{51}}}
\newcommand{\no}{\textcolor{red}{\ding{55}}}

% ----------------------------------------------------------------------
% DOCUMENT START
% ----------------------------------------------------------------------
\begin{document}

\title{Cybersecurity Posture Assessment Report}
\author{Cybersecurity Analysis Division}
\date{\today}
\maketitle

\section*{Executive Summary}
This report provides a comprehensive analysis of the cybersecurity posture for \textbf{[Organization Name]}. The assessment is based on a correlation of network scan data, a security controls questionnaire, and a review of pre-existing risks.

The analysis reveals a mixed security posture. The organization has implemented strong multi-factor authentication (MFA) controls across key systems, which is a commendable best practice. However, this strength is severely undermined by critical deficiencies in other areas. A network scan confirmed a high-risk exposure: Remote Desktop Protocol (RDP) is open to the public internet on host \texttt{[Target IP]}. This finding directly validates a known critical risk and presents a significant and immediate threat of unauthorized access and ransomware.

Furthermore, foundational security practices are absent. The organization lacks an employee acceptable use policy and does not conduct security awareness training. This combination of a direct, high-impact technical vulnerability and a lack of user security awareness creates a high-probability, high-impact risk scenario. Immediate remediation is required to address the RDP exposure, followed by the implementation of essential security policies and training programs.

\section{Organizational Information}
\begin{itemize}
    \item \textbf{Organization Name:} \textbf{[Organization Name]}
    \item \textbf{Primary Domain:} \texttt{[Domain]}
    \item \textbf{External IP Scanned:} \texttt{[Client IP]}
\end{itemize}

\section{Security Control Review}
The following table summarizes the organization's responses to a security controls questionnaire. "No" answers indicate significant gaps in the security framework that increase organizational risk.

\begin{table}[h!]
\centering
\caption{Security Controls Questionnaire Results}
\begin{tabular}{p{0.7\linewidth}c}
\toprule
\textbf{Control Question} & \textbf{Status} \\
\midrule
Do you require MFA to access email? & \yes \\
Do you require MFA to log into computers? & \yes \\
Do you require MFA to access sensitive data systems? & \yes \\
Does your organization have an employee acceptable use policy? & \no \\
Does your organization do security awareness training for new employees? & \no \\
Does your organization do security awareness training for all employees at least once per year? & \no \\
\bottomrule
\end{tabular}
\end{table}

\subsection*{Analysis}
While the implementation of MFA is a critical strength, the complete absence of an acceptable use policy and any form of security awareness training is a critical weakness. Without these controls, employees are more likely to engage in risky behavior, fall victim to phishing attacks, or use weak credentials, directly increasing the threat to the organization.

\section{Technical Scan Results}
An external network scan was performed to identify open ports and exposed services. The scan targeted the primary external IP address \texttt{[Client IP]}.

\subsection*{Target: \texttt{[Target IP]}}
The scan identified the following open port on the target system:

\begin{table}[h!]
\centering
\caption{Open Port Analysis for \texttt{[Target IP]}}
\begin{tabular}{llll}
\toprule
\textbf{Port} & \textbf{State} & \textbf{Service} & \textbf{Notes} \\
\midrule
3389/tcp & Open & ms-wbt-server & Remote Desktop Protocol (RDP). \\
\bottomrule
\end{tabular}
\end{table}

\subsection*{Analysis}
The presence of an open RDP port (3389) on the public internet is a critical security vulnerability. This service is a primary target for attackers who use brute-force password attacks and exploit known vulnerabilities to gain unauthorized access to internal networks. This finding corroborates the pre-existing risk documented in the organization's risk register.

\section{Consolidated Risk Assessment}
The following table synthesizes findings from the security questionnaire, technical scan, and pre-existing risk data into a prioritized list of risks.

\begin{table}[h!]
\centering
\caption{Synthesized Risk Summary}
\begin{tabular}{p{0.25\linewidth}p{0.5\linewidth}l}
\toprule
\textbf{Risk} & \textbf{Description} & \textbf{Severity} \\
\midrule
\textbf{External RDP Exposure} & Port 3389 (RDP) is publicly accessible on \texttt{[Target IP]}, allowing attackers to attempt brute-force logins or exploit RDP vulnerabilities. This is a common vector for ransomware. & \textbf{Critical (9.0)} \\
\addlinespace
\textbf{Lack of Security Policies} & The absence of an Acceptable Use Policy means there are no formal guidelines for employees regarding the secure use of company assets, data, and networks. & \textbf{High} \\
\addlinespace
\textbf{Insufficient Security Training} & The lack of security awareness training for new and existing employees makes the organization highly susceptible to social engineering, phishing, and password-related attacks. & \textbf{High} \\
\bottomrule
\end{tabular}
\end{table}

\section{Recommendations}
The following actions are recommended to mitigate the identified risks, prioritized by urgency.

\begin{enumerate}
    \item \textbf{Immediate (Critical): Remediate RDP Exposure}
    \begin{itemize}
        \item Immediately block all inbound traffic to TCP port 3389 on your external firewall for the host \texttt{[Target IP]}.
        \item If remote access is required, implement a Virtual Private Network (VPN) with multi-factor authentication. All RDP access should occur exclusively through the secure VPN tunnel.
    \end{itemize}
    \item \textbf{Short-Term (High): Establish Foundational Policies}
    \begin{itemize}
        \item Develop and implement a formal Acceptable Use Policy (AUP). This policy must be read and acknowledged by all employees.
        \item The AUP should outline rules for password complexity, data handling, and appropriate use of IT resources.
    \end{itemize}
    \item \textbf{Mid-Term (High): Implement Security Awareness Training}
    \begin{itemize}
        \item Institute a mandatory security awareness training program for all new hires as part of their onboarding process.
        \item Conduct annual security awareness training for all staff to keep them informed about current threats, such as phishing and social engineering.
    \end{itemize}
\end{enumerate}

\end{document}
% ----------------------------------------------------------------------
% DOCUMENT END
% ----------------------------------------------------------------------
```