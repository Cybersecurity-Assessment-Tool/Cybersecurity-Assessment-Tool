```latex
\documentclass[12pt]{article}

% Preamble: Required Packages
\usepackage[margin=1in]{geometry}
\usepackage{pifont} % For checkmarks and crosses
\usepackage{booktabs} % For professional tables
\usepackage{hyperref} % For clickable links
\usepackage{url} % For URL formatting
\usepackage{seqsplit} % For splitting long strings
\usepackage{graphicx}
\usepackage[table]{xcolor}
\usepackage{fancyhdr}

% --- Document Setup ---
\definecolor{darkblue}{rgb}{0.0, 0.0, 0.55}
\definecolor{darkred}{rgb}{0.55, 0.0, 0.0}
\definecolor{tablehead}{gray}{0.9}

\hypersetup{
    colorlinks=true,
    linkcolor=darkblue,
    filecolor=darkblue,      
    urlcolor=darkblue,
    citecolor=darkblue,
}

\pagestyle{fancy}
\fancyhf{}
\lhead{Cybersecurity Assessment Report}
\rhead{\textbf{[Organization Name]}}
\cfoot{\thepage}
\renewcommand{\headrulewidth}{0.4pt}
\renewcommand{\footrulewidth}{0.4pt}

% --- Document Start ---
\begin{document}

% --- Title Page ---
\begin{titlepage}
    \centering
    \vspace*{1cm}
    
    \includegraphics[width=0.4\textwidth]{example-image-a} % Placeholder logo
    
    \vspace{1.5cm}
    
    {\Huge\bfseries Cybersecurity Posture Assessment Report\par}
    
    \vspace{1cm}
    
    {\Large Prepared for:\par}
    {\Large\bfseries [Organization Name]\par}
    
    \vspace{2cm}
    
    {\large Report Date: \today\par}
    {\large Assessment Date: 2025-11-22\par}
    
    \vfill
    
    {\large This report contains sensitive information and should be handled with care.\par}
    
\end{titlepage}

\tableofcontents
\newpage

% --- Executive Summary ---
\section{Executive Summary}

This report details the findings of a cybersecurity assessment conducted for \textbf{[Organization Name]} on November 22, 2025. The assessment combined a review of organizational security controls, an external network vulnerability scan, and an analysis of pre-existing risks.

The overall security posture requires immediate attention. Several critical and high-risk vulnerabilities were identified that expose the organization to significant threats, including unauthorized access, data breaches, and ransomware attacks.

Key findings include:
\begin{itemize}
    \item \textbf{Critical Gaps in Access Control:} Multi-Factor Authentication (MFA) is not enforced for employee email or computer logins. This represents a critical vulnerability, as compromised credentials could grant an attacker widespread access to internal systems and communications.
    \item \textbf{Vulnerable External Service:} The external-facing web server at \texttt{[Client IP]} is running an outdated version of Nginx (1.18.0). This version is several years old and has multiple known, publicly disclosed vulnerabilities that could be exploited to compromise the server.
    \item \textbf{Inadequate Employee Onboarding:} New employees do not receive security awareness training, creating a recurring window of vulnerability where new staff are more susceptible to phishing and social engineering attacks.
\end{itemize}

This report provides a detailed breakdown of these findings and offers actionable recommendations to mitigate the identified risks and strengthen the organization's overall security posture. We strongly advise prioritizing the implementation of MFA and the patching of the external web server.

\newpage

% --- Organizational Information ---
\section{Organizational Information}

This section provides the organizational details used as the basis for this assessment. As per the provided data, placeholders have been used where specific information was not available.

\begin{table}[h!]
\centering
\caption{Client Organizational Details}
\begin{tabular}{@{}ll@{}}
\toprule
\textbf{Attribute} & \textbf{Value} \\
\midrule
Organization Name & \textbf{[Organization Name]} \\
Primary Email Domain & \texttt{[Domain]} \\
External IP Address Scanned & \texttt{[Client IP]} \\
\bottomrule
\end{tabular}
\end{table}

% --- Security Control Review ---
\section{Security Control Review}

A review of internal security controls was conducted based on a standardized questionnaire. The responses highlight significant gaps in foundational security practices. A summary of the findings is presented in Table 2. The checkmark (\ding{51}) indicates a positive control is in place, while the cross mark (\ding{55}) indicates a control gap.

\begin{table}[h!]
\centering
\caption{Security Controls Questionnaire Analysis}
\rowcolors{2}{gray!10}{white}
\begin{tabular}{@{}p{8cm}ccp{3cm}@{}}
\toprule
\rowcolor{tablehead}
\textbf{Control Question} & \textbf{Response} & \textbf{Status} & \textbf{Assessment} \\
\midrule
Do you require MFA to access email? & No & \ding{55} & \textbf{Critical Gap} \\
Do you require MFA to log into computers? & No & \ding{55} & \textbf{Critical Gap} \\
Do you require MFA to access sensitive data systems? & Yes & \ding{51} & Best Practice \\
Does your organization have an employee acceptable use policy? & Yes & \ding{51} & Best Practice \\
Does your organization do security awareness training for new employees? & No & \ding{55} & \textbf{High Risk} \\
Does your organization do security awareness training for all employees at least once per year? & Yes & \ding{51} & Best Practice \\
\bottomrule
\end{tabular}
\end{table}

\subsection*{Analysis of Control Gaps}
\begin{itemize}
    \item \textbf{MFA for Email and Computers:} The absence of MFA on primary communication (email) and endpoint (computer) systems is a critical weakness. A single compromised password could lead to a full account takeover and lateral movement within the network.
    \item \textbf{New Hire Training:} Failing to train new employees on security best practices from day one makes them prime targets for phishing and social engineering attacks before they are integrated into the regular annual training cycle.
\end{itemize}

\newpage

% --- Technical Scan Results ---
\section{Technical Scan Results}

An external network scan was performed against the public-facing IP address \texttt{[Target IP]} on \textbf{2025-11-22}. The scan identified one open port running a vulnerable service.

\begin{table}[h!]
\centering
\caption{Open Port Analysis for Target: \texttt{[Target IP]}}
\rowcolors{2}{gray!10}{white}
\begin{tabular}{@{}ccccc@{}}
\toprule
\rowcolor{tablehead}
\textbf{Port} & \textbf{State} & \textbf{Service} & \textbf{Product} & \textbf{Version} \\
\midrule
443/tcp & open & https & nginx & 1.18.0 \\
\bottomrule
\end{tabular}
\end{table}

\subsection*{Vulnerability Finding: Outdated Nginx Server}
The scan revealed that the HTTPS service on port 443 is served by \textbf{Nginx version 1.18.0}. This version was released in April 2020 and is now considered outdated and unsupported. It is affected by several publicly known vulnerabilities (CVEs) that could allow an attacker to cause a denial of service, bypass security restrictions, or potentially achieve remote code execution, depending on the server's configuration. This finding is classified as a high-severity risk.

% --- Pre-existing Risks ---
\section{Pre-existing Risk Review}
An analysis of the provided list of current organizational risks was conducted. The data indicated that there were \textbf{no pre-existing vulnerabilities} formally tracked prior to this assessment. All risks documented in the following section are new findings.

% --- Risk Assessment Summary ---
\section{Risk Assessment Summary}

The following table synthesizes the findings from the security control review and the technical scan into a prioritized list of risks.

\begin{table}[h!]
\centering
\caption{Summary of Identified Risks}
\begin{tabular}{@{}lp{6cm}l@{}}
\toprule
\rowcolor{tablehead}
\textbf{Risk ID} & \textbf{Risk Name} & \textbf{Severity} \\
\midrule
RISK-001 & Lack of MFA on Email and Endpoints & \cellcolor{red!25}\textbf{Critical} \\
RISK-002 & Outdated and Vulnerable Nginx Web Server & \cellcolor{orange!35}\textbf{High} \\
RISK-003 & Inadequate New Hire Security Training & \cellcolor{orange!35}\textbf{High} \\
\bottomrule
\end{tabular}
\end{table}

\newpage

% --- Recommendations ---
\section{Recommendations}

The following actions are recommended to mitigate the identified risks and improve the overall security posture of \textbf{[Organization Name]}.

\subsection*{RISK-001: Lack of MFA on Email and Endpoints (Critical)}
\begin{itemize}
    \item \textbf{Immediate Action:} Procure and implement a robust Multi-Factor Authentication (MFA) solution for all user accounts.
    \item \textbf{Priority 1:} Enforce MFA for access to all email accounts (e.g., via O365 or Google Workspace security policies).
    \item \textbf{Priority 2:} Enforce MFA for all computer and remote access (VPN) logins.
    \item \textbf{Justification:} This is the single most effective control to prevent account takeovers resulting from stolen or weak passwords.
\end{itemize}

\subsection*{RISK-002: Outdated Nginx Web Server (High)}
\begin{itemize}
    \item \textbf{Immediate Action:} Develop a patch and upgrade plan for the public-facing web server at \texttt{[Target IP]}.
    \item \textbf{Priority 1:} Upgrade Nginx from version 1.18.0 to the latest stable version recommended by the vendor. This will patch numerous known security vulnerabilities.
    \item \textbf{Priority 2:} Implement a regular patch management cycle for all internet-facing systems to ensure they are updated in a timely manner.
    \item \textbf{Justification:} Unpatched external systems are a primary target for automated attacks and can serve as an initial entry point into the corporate network.
\end{itemize}

\subsection*{RISK-003: Inadequate New Hire Security Training (High)}
\begin{itemize}
    \item \textbf{Immediate Action:} Develop a mandatory security awareness training module to be included in the standard onboarding process for all new employees and contractors.
    \item \textbf{Content:} The training should cover, at a minimum: phishing awareness, password security, acceptable use of company assets, and how to report a security incident.
    \item \textbf{Justification:} Equipping new hires with security knowledge from day one reduces the organization's susceptibility to social engineering and human error, closing a significant gap in the security program.
\end{itemize}

\end{document}
```