```latex
\documentclass[12pt]{article}

% Preamble: Required Packages and Document Setup
\usepackage[margin=1in]{geometry}
\usepackage{pifont} % For checkmarks and crosses
\usepackage{booktabs} % For professional tables
\usepackage{hyperref} % For clickable links
\usepackage{url} % For formatting URLs
\usepackage{seqsplit} % For splitting long strings to prevent overflow
\usepackage{graphicx}
\usepackage{xcolor}

% Define colors for severity levels
\definecolor{critical}{HTML}{990000}
\definecolor{high}{HTML}{D14302}
\definecolor{medium}{HTML}{E5A800}
\definecolor{low}{HTML}{339900}

% Hyperref setup
\hypersetup{
    colorlinks=true,
    linkcolor=blue,
    filecolor=magenta,      
    urlcolor=cyan,
    pdftitle={Cybersecurity Posture Assessment},
    pdfpagemode=FullScreen,
}

% Document Metadata
\title{Cybersecurity Posture Assessment Report}
\author{Cybersecurity Analysis Division}
\date{\today}

\begin{document}

\maketitle
\thispagestyle{empty}
\newpage

\tableofcontents
\newpage

% --- 1. Executive Summary ---
\section{Executive Summary}

This report details the findings of a cybersecurity posture assessment conducted for \textbf{[Organization Name]}. The assessment incorporated an analysis of organizational security policies via a questionnaire, a technical network scan of the external perimeter, and a review of previously identified risks.

The analysis revealed \textbf{critical deficiencies} in foundational security controls. The complete absence of Multi-Factor Authentication (MFA) across all key systems, including email and remote access, represents an immediate and severe threat. This is compounded by a lack of a formal security awareness training program and an employee acceptable use policy. These organizational gaps significantly increase the risk of a successful cyberattack, particularly through social engineering and credential compromise.

A technical scan identified an exposed Secure Shell (SSH) service on the network perimeter. While common, an exposed SSH service without compensating controls like MFA and strong password policies becomes a high-value target for attackers.

Immediate remediation is required to address these findings. The highest priority should be the rapid deployment of MFA, followed by the implementation of a security awareness program and the formalization of security policies.

% --- 2. Organizational Information ---
\section{Organizational Information}

The following details were used as the basis for this assessment. Due to the anonymized nature of the provided data, placeholders have been used where necessary.

\begin{itemize}
    \item \textbf{Organization Name:} \textbf{[Organization Name]}
    \item \textbf{Primary Domain:} \texttt{[Domain]}
    \item \textbf{External IP Address Scanned:} \texttt{[Client IP]}
    \item \textbf{Target IP Address for Technical Scan:} \texttt{[Target IP]}
\end{itemize}

% --- 3. Security Control Review (Questionnaire Analysis) ---
\section{Security Control Review}

An assessment of the organization's security policies and controls was conducted via a standardized questionnaire. The responses indicate significant gaps in fundamental security practices. A summary of the findings is presented in Table \ref{tab:controls}. The symbol \ding{55} denotes a negative response, indicating a potential control gap.

\begin{table}[h!]
\centering
\caption{Security Control Questionnaire Analysis}
\label{tab:controls}
\begin{tabular}{p{0.5\linewidth} c p{0.3\linewidth}}
\toprule
\textbf{Control Question} & \textbf{Response} & \textbf{Analyst Notes} \\
\midrule
Do you require MFA to access email? & \ding{55} & \textbf{Critical Gap.} Email is a primary target for account takeover and phishing. \\
\addlinespace
Do you require MFA to log into computers? & \ding{55} & \textbf{High Risk.} Lack of MFA on endpoints increases risk from stolen credentials. \\
\addlinespace
Do you require MFA to access sensitive data systems? & \ding{55} & \textbf{Critical Gap.} Sensitive data is left vulnerable to single-factor authentication failures. \\
\addlinespace
Does your organization have an employee acceptable use policy? & \ding{55} & \textbf{High Risk.} Lack of a formal policy creates ambiguity and increases insider risk. \\
\addlinespace
Does your organization do security awareness training for new employees? & \ding{55} & \textbf{Critical Gap.} New employees are not equipped to identify and report security threats. \\
\addlinespace
Does your organization do security awareness training for all employees at least once per year? & \ding{55} & \textbf{Critical Gap.} The organization is highly susceptible to social engineering attacks. \\
\bottomrule
\end{tabular}
\end{table}

% --- 4. Technical Scan Results ---
\section{Technical Scan Results}

An external network scan was performed on the target IP address \texttt{[Target IP]} to identify open ports and exposed services. The scan was non-intrusive and aimed to simulate the initial reconnaissance phase of an external attacker.

\subsection{Scan Summary}
The scan identified one open port, which is detailed in Table \ref{tab:scanresults}.

\begin{table}[h!]
\centering
\caption{Open Port Analysis}
\label{tab:scanresults}
\begin{tabular}{l l l l}
\toprule
\textbf{Port/Proto} & \textbf{State} & \textbf{Service} & \textbf{Analyst Notes} \\
\midrule
22/tcp & Open & ssh & Secure Shell (SSH) is used for remote administration. \\
 & & & Exposing SSH to the internet is a common practice but \\
 & & & requires strong authentication controls. Version \\
 & & & information was not available from this scan. \\
\bottomrule
\end{tabular}
\end{table}

\subsection{Technical Findings Analysis}
The primary finding is an open SSH port (22/tcp). This service is a frequent target for brute-force and credential-stuffing attacks. In the context of the organizational control gaps (specifically, the lack of MFA), this exposed service poses a significantly elevated risk. An attacker who obtains valid credentials through other means (e.g., phishing) could gain direct, unauthorized administrative access to the network.

% --- 5. Integrated Risk Assessment ---
\section{Integrated Risk Assessment}

This section correlates the findings from the security control review, the technical scan, and any pre-existing risks. The current assessment did not find any pre-existing documented vulnerabilities. The identified risks are summarized in Table \ref{tab:risks}.

\begin{table}[h!]
\centering
\caption{Summary of Identified Risks}
\label{tab:risks}
\begin{tabular}{p{0.3\linewidth} p{0.15\linewidth} p{0.45\linewidth}}
\toprule
\textbf{Risk Name} & \textbf{Severity} & \textbf{Overview} \\
\midrule
\textbf{Absence of Multi-Factor Authentication (MFA)} & \textcolor{critical}{\textbf{Critical}} & The lack of MFA for email, endpoints, and sensitive systems makes account compromise trivial if credentials are stolen. This is the most severe risk identified. \\
\addlinespace
\textbf{Lack of Security Awareness Program} & \textcolor{critical}{\textbf{Critical}} & Without training, employees are highly likely to fall victim to phishing and other social engineering attacks, providing attackers with an initial foothold. \\
\addlinespace
\textbf{Exposed SSH Service with Weak Controls} & \textcolor{high}{\textbf{High}} & The open SSH port, combined with the lack of MFA, creates a direct path for attackers to gain remote access if they can guess, brute-force, or steal a valid password. \\
\addlinespace
\textbf{No Formal Acceptable Use Policy (AUP)} & \textcolor{high}{\textbf{High}} & The absence of a formal AUP leads to inconsistent security practices and a lack of enforceable guidelines for employees regarding the protection of company assets. \\
\bottomrule
\end{tabular}
\end{table}

% --- 6. Recommendations ---
\section{Recommendations}

Based on the identified risks, the following prioritized actions are recommended to improve the organization's cybersecurity posture.

\subsection{Immediate Priority (0-30 Days)}
\begin{enumerate}
    \item \textbf{Deploy Multi-Factor Authentication (MFA):}
    \begin{itemize}
        \item Immediately enable MFA for all users on the primary email system (e.g., Microsoft 365, Google Workspace).
        \item Enforce MFA for all remote access systems, especially the exposed SSH service.
        \item Plan the rollout of MFA for all sensitive data systems and endpoint logins.
    \end{itemize}
    
    \item \textbf{Harden Exposed SSH Service:}
    \begin{itemize}
        \item If possible, restrict access to the SSH port to known, trusted IP addresses using a firewall rule (IP whitelisting).
        \item Disable password-based authentication and enforce the use of cryptographic keys (e.g., SSH keys) for all SSH logins.
        \item Implement an intrusion detection tool like Fail2Ban to automatically block IPs that exhibit malicious behavior (e.g., multiple failed login attempts).
    \end{itemize}
\end{enumerate}

\subsection{High Priority (30-90 Days)}
\begin{enumerate}
    \item \textbf{Implement a Security Awareness Program:}
    \begin{itemize}
        \item Procure and deploy a security awareness training solution for all employees.
        \item Conduct an initial baseline training campaign covering key topics like phishing, password security, and acceptable use.
        \item Integrate this training into the onboarding process for all new hires.
        \item Schedule annual refresher training for all staff.
    \end{itemize}
    
    \item \textbf{Develop and Implement Security Policies:}
    \begin{itemize}
        \item Draft and formally approve an Employee Acceptable Use Policy (AUP) that clearly defines the rules for using company IT assets.
        \item Develop a password policy that mandates complexity, length, and regular reviews.
    \end{itemize}
\end{enumerate}

% --- 7. Conclusion ---
\section{Conclusion}

The assessment for \textbf{[Organization Name]} reveals a foundational weakness in its cybersecurity posture, primarily due to the absence of MFA and a formal security awareness program. While the technical attack surface appears small, the identified control gaps create a high-risk environment where a single compromised credential could lead to a significant security breach.

By implementing the recommendations outlined in this report, particularly the immediate deployment of MFA and the hardening of the external SSH service, the organization can take immediate and impactful steps to drastically reduce its risk exposure.

\end{document}
```