```latex
\documentclass[12pt]{article}

% Preamble: Required Packages
\usepackage[margin=1in]{geometry}
\usepackage{pifont} % For \ding
\usepackage{booktabs} % For professional tables
\usepackage{hyperref} % For hyperlinks
\usepackage{url} % For URL formatting
\usepackage{seqsplit} % For splitting long strings
\usepackage{graphicx}
\usepackage{xcolor}

% Hyperref Setup
\hypersetup{
    colorlinks=true,
    linkcolor=blue,
    filecolor=magenta,      
    urlcolor=cyan,
    pdftitle={Cybersecurity Posture Report},
    pdfpagemode=FullScreen,
}

% Document Title
\title{Cybersecurity Posture Report \\ \large For \textbf{[Organization Name]}}
\author{Cybersecurity Analysis Team}
\date{\today}

\begin{document}

\maketitle
\thispagestyle{empty}
\newpage

\tableofcontents
\newpage

% --- Section 1: Executive Overview ---
\section{Executive Overview}

This report provides a comprehensive analysis of the cybersecurity posture for \textbf{[Organization Name]}, based on network scans, a security controls questionnaire, and a review of pre-existing risks.

The assessment has identified a \textbf{critical risk exposure} that requires immediate attention. The primary finding is the direct public exposure of a Remote Desktop Protocol (RDP) service on port 3389 at the external IP address \texttt{[Client IP]}. This configuration is a well-known and highly targeted vector for ransomware attacks and unauthorized access.

This critical technical vulnerability is severely compounded by significant gaps in organizational security controls. The complete absence of Multi-Factor Authentication (MFA) for email, computer logins, and sensitive data access, combined with a lack of a formal security awareness training program, creates a high-likelihood scenario for a security breach. An attacker could leverage a single compromised password to gain direct access to the internal network.

Immediate remediation of the exposed RDP service is paramount. Following this, a strategic implementation of MFA and the development of a security awareness program are essential to elevate the organization's security posture from its current high-risk state.

% --- Section 2: Organizational Information ---
\section{Organizational Information}

This section details the information provided by the client organization. The data has been anonymized as per the engagement requirements.

\begin{itemize}
    \item \textbf{Organization Name:} \textbf{[Organization Name]}
    \item \textbf{Primary Email Domain:} \texttt{[Domain]}
    \item \textbf{External IP Address Scanned:} \texttt{[Client IP]}
\end{itemize}

% --- Section 3: Security Control Review ---
\section{Security Control Review}

The following table summarizes the organization's responses to a security controls questionnaire. Each "No" response indicates a significant gap in the defensive posture and has been flagged for review.

\begin{table}[h!]
\centering
\caption{Security Controls Questionnaire Analysis}
\begin{tabular}{p{0.6\linewidth} c l}
\toprule
\textbf{Control Question} & \textbf{Response} & \textbf{Assessment} \\
\midrule
Do you require MFA to access email? & \ding{55} & \textcolor{red}{\textbf{Critical Gap}} \\
Do you require MFA to log into computers? & \ding{55} & \textcolor{red}{\textbf{Critical Gap}} \\
Do you require MFA to access sensitive data systems? & \ding{55} & \textcolor{red}{\textbf{Critical Gap}} \\
Does your organization have an employee acceptable use policy? & \ding{51} & Good Practice \\
Does your organization do security awareness training for new employees? & \ding{55} & \textcolor{orange}{High Risk} \\
Does your organization do security awareness training for all employees at least once per year? & \ding{55} & \textcolor{orange}{High Risk} \\
\bottomrule
\end{tabular}
\end{table}

\subsection*{Analysis of Gaps}
\begin{itemize}
    \item \textbf{Lack of MFA:} The absence of MFA across all critical access points (email, computers, data systems) is a critical vulnerability. It means that a single compromised password is all an attacker needs to gain significant access.
    \item \textbf{Lack of Security Training:} Without initial and ongoing security awareness training, employees are significantly more vulnerable to phishing and social engineering attacks, which are the primary methods for initial credential compromise.
\end{itemize}

% --- Section 4: Technical Scan Results ---
\section{Technical Scan Results}

An external network scan was performed on the target IP address to identify open ports and exposed services.

\begin{itemize}
    \item \textbf{Target IP Address:} \texttt{[Target IP]}
    \item \textbf{Scan Date:} Data not provided in scan metadata.
\end{itemize}

\begin{table}[h!]
\centering
\caption{Open Ports Detected on \texttt{[Target IP]}}
\begin{tabular}{c c l l}
\toprule
\textbf{Port} & \textbf{State} & \textbf{Service Name} & \textbf{Analysis} \\
\midrule
3389/tcp & open & \texttt{ms-wbt-server} & Microsoft Remote Desktop Protocol (RDP) \\
\bottomrule
\end{tabular}
\end{table}

\subsection*{Analysis of Findings}
The scan confirms that port \textbf{3389 (RDP)} is open to the public internet. RDP is a protocol that allows for remote administrative control of a Windows system. Exposing this service directly to the internet is extremely dangerous and is a common tactic used by attackers, particularly ransomware groups, to gain an initial foothold in a network. This finding directly correlates with the pre-existing risk identified in the subsequent section.

% --- Section 5: Consolidated Risk Assessment ---
\section{Consolidated Risk Assessment}

This section synthesizes findings from the security control review, technical scan, and pre-existing risk data into a prioritized list of risks.

\begin{table}[h!]
\centering
\caption{Summary of Identified Risks}
\begin{tabular}{p{0.25\linewidth} p{0.55\linewidth} l}
\toprule
\textbf{Risk Name} & \textbf{Description} & \textbf{Severity} \\
\midrule
\textbf{Public RDP Exposure without MFA} & The technical scan confirmed RDP (port 3389) is publicly exposed. This is combined with a complete lack of MFA, meaning a brute-forced or phished password would grant an attacker direct network access. & \textcolor{red}{\textbf{Critical}} \\
\addlinespace
\textbf{Inadequate Security Awareness Program} & The organization does not conduct security awareness training for new or existing employees. This increases susceptibility to phishing and social engineering, which could lead to credential compromise. & \textcolor{orange}{\textbf{High}} \\
\addlinespace
\textbf{Systemic Lack of Multi-Factor Authentication} & MFA is not enforced for email, computer logins, or sensitive systems. This removes a critical layer of defense against account takeover attacks. & \textcolor{orange}{\textbf{High}} \\
\bottomrule
\end{tabular}
\end{table}

% --- Section 6: Recommendations ---
\section{Recommendations}

The following actionable recommendations are provided to mitigate the identified risks. They are prioritized based on severity and potential impact.

\subsection*{Immediate Priority (Critical)}
\begin{enumerate}
    \item \textbf{Remediate RDP Exposure:} Immediately close port 3389 on the external firewall for IP address \texttt{[Client IP]}.
    \begin{itemize}
        \item \textbf{If remote access is required:} Implement a Virtual Private Network (VPN) with MFA enabled. Access to the RDP service should only be permitted through the secure VPN tunnel.
        \item \textbf{If remote access is not required:} The port should remain closed and the service disabled on the host machine.
    \end{itemize}
\end{enumerate}

\subsection*{High Priority}
\begin{enumerate}
    \setcounter{enumi}{1}
    \item \textbf{Deploy Multi-Factor Authentication (MFA):} Begin a phased rollout of MFA for all users and systems.
    \begin{itemize}
        \item \textbf{Phase 1:} Enable MFA for all remote access systems (VPN) and all email accounts (e.g., Microsoft 365, Google Workspace).
        \item \textbf{Phase 2:} Enforce MFA for logins to all sensitive data systems and for all administrative accounts.
        \item \textbf{Phase 3:} Expand MFA to all employee computer logins.
    \end{itemize}
    \item \textbf{Implement Security Awareness Training:}
    \begin{itemize}
        \item Procure and deploy a security awareness training platform.
        \item Enroll all new employees in a mandatory training module as part of their onboarding process.
        \item Require all current employees to complete annual security awareness training and conduct regular phishing simulations to reinforce learning.
    \end{itemize}
\end{enumerate}

\subsection*{Medium Priority}
\begin{enumerate}
    \setcounter{enumi}{3}
    \item \textbf{Review and Enforce Acceptable Use Policy (AUP):} While an AUP exists, ensure it is reviewed annually, communicated to all employees, and that employees formally acknowledge they have read and understood it.
\end{enumerate}

\end{document}
```