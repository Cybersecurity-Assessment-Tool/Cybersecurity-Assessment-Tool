```latex
\documentclass[12pt]{article}

% Preamble: Required Packages
\usepackage[margin=1in]{geometry}
\usepackage{pifont} % For checkmarks (\ding{51}) and crosses (\ding{55})
\usepackage{booktabs} % For professional-looking tables
\usepackage{hyperref}
\usepackage{url}
\usepackage{seqsplit} % To break long strings like hashes or URLs
\usepackage{xcolor}
\usepackage{graphicx}
\usepackage{array}

% Define custom commands for Yes/No symbols
\newcommand{\yes}{\textcolor{green!80!black}{\ding{51}}}
\newcommand{\no}{\textcolor{red!90!black}{\ding{55}}}

% Hyperlink Setup
\hypersetup{
    colorlinks=true,
    linkcolor=blue,
    filecolor=magenta,      
    urlcolor=cyan,
    pdftitle={Cybersecurity Posture Assessment Report},
    pdfpagemode=FullScreen,
}

% Document Title
\title{Cybersecurity Posture Assessment Report \\ \large For \textbf{[Organization Name]}}
\author{Cybersecurity Analysis Division}
\date{\today}

\begin{document}

\maketitle
\thispagestyle{empty}
\newpage

\tableofcontents
\newpage

% --- 1. Executive Summary ---
\section{Executive Summary}

This report provides a comprehensive cybersecurity posture assessment for \textbf{[Organization Name]}, conducted on \today. The analysis is based on a synthesis of network scan data, an organizational security controls questionnaire, and a review of pre-existing risks.

The assessment reveals a mixed security posture. On one hand, the organization demonstrates a strong commitment to foundational security controls, as evidenced by the universal implementation of Multi-Factor Authentication (MFA) for critical systems and consistent security awareness training. These proactive measures significantly reduce the risk of common cyberattacks like phishing and credential compromise.

However, the technical network scan identified a critical exposure: a publicly accessible Secure Shell (SSH) service on port 22. While essential for remote administration, an internet-facing SSH port is a high-value target for automated brute-force attacks and exploitation attempts. This finding represents the most significant and immediate risk to the organization's network integrity.

Our primary recommendation is to immediately review the business necessity of this exposed service. If required, access should be strictly limited to trusted IP addresses via firewall rules, and authentication should be hardened by disabling passwords in favor of public-key cryptography.

\section{Organizational Information}

The following details were used as the basis for this assessment. As per the template mode for anonymized data, placeholders are used where information was not provided.

\begin{table}[h!]
\centering
\begin{tabular}{@{}ll@{}}
\toprule
\textbf{Attribute} & \textbf{Value} \\
\midrule
Organization Name & \textbf{[Organization Name]} \\
Primary Domain & \texttt{[Domain]} \\
External IP Address (Scanned) & \texttt{[Client IP]} \\
\bottomrule
\end{tabular}
\caption{Client Organizational Data}
\label{tab:org_data}
\end{table}

\section{Security Control Review}

The following table summarizes the organization's responses to the security controls questionnaire. The results indicate a strong and consistently applied security policy baseline.

\begin{table}[h!]
\centering
\begin{tabular}{@{} >{\raggedright\arraybackslash}p{0.75\linewidth} c @{}}
\toprule
\textbf{Control Question} & \textbf{Response} \\
\midrule
Do you require MFA to access email? & \yes \\
Do you require MFA to log into computers? & \yes \\
Do you require MFA to access sensitive data systems? & \yes \\
Does your organization have an employee acceptable use policy? & \yes \\
Does your organization do security awareness training for new employees? & \yes \\
Does your organization do security awareness training for all employees at least once per year? & \yes \\
\bottomrule
\end{tabular}
\caption{Security Controls Questionnaire Results}
\label{tab:controls}
\end{table}

\subsection{Analysis}
All responses were affirmative, indicating a mature approach to user access control and security awareness. This significantly strengthens the organization's defense against social engineering and credential-based attacks. No policy-based gaps were identified from this review.

\section{Technical Scan Results}

An external network scan was performed to identify open ports and exposed services on the client's perimeter.

\begin{itemize}
    \item \textbf{Target IP:} \texttt{[Target IP]} (Placeholder for empty target in scan data)
    \item \textbf{Scan Type:} Basic Port Scan (Nmap)
\end{itemize}

The scan revealed the following open port:

\begin{table}[h!]
\centering
\begin{tabular}{@{}lllll@{}}
\toprule
\textbf{Port} & \textbf{State} & \textbf{Service} & \textbf{Product} & \textbf{Version} \\
\midrule
22/tcp & open & ssh & (Not Detected) & (Not Detected) \\
\bottomrule
\end{tabular}
\caption{Open Ports Detected on Target Host}
\label{tab:scan_results}
\end{table}

\subsection{Analysis}
The presence of an open SSH port (22) is a significant finding. This service is used for remote system administration. When exposed to the public internet, it becomes a primary target for attackers who use automated tools to guess credentials (brute-force attacks) or exploit potential vulnerabilities in the SSH server software itself. The scan did not retrieve version information, which prevents us from checking for known public exploits (CVEs) for the specific software in use. This lack of information is, in itself, a minor risk.

\section{Risk Assessment Summary}

This section correlates the findings from the security control review, technical scan, and pre-existing risk data. The pre-existing risk register was empty. The primary risk identified during this assessment is detailed below.

\begin{table}[h!]
\centering
\begin{tabular}{@{}lp{0.5\linewidth}l@{}}
\toprule
\textbf{Risk ID} & \textbf{Description} & \textbf{Severity} \\
\midrule
\textbf{RISK-001} & \textbf{Exposed SSH Service:} Port 22 (SSH) is open to the public internet, creating a direct vector for unauthorized access attempts, brute-force attacks, and potential exploitation of the service. & \textbf{\textcolor{red!90!black}{High}} \\
\bottomrule
\end{tabular}
\caption{Identified Risks}
\label{tab:risks}
\end{table}

\section{Recommendations}

Based on the risk assessment, the following actionable recommendations are provided to improve the organization's security posture. Priority should be given to mitigating the high-severity risk.

\subsection{RISK-001: Exposed SSH Service}

\begin{itemize}
    \item \textbf{Immediate Action - Access Control:}
    \begin{enumerate}
        \item \textbf{Verify Business Need:} Confirm if remote SSH access from the public internet is strictly necessary.
        \item \textbf{Implement Firewall Whitelisting:} If access is required, configure firewall rules to restrict access to this port to only known, trusted source IP addresses (e.g., administrator's office or a bastion host). Deny all other traffic by default.
    \end{enumerate}

    \item \textbf{Hardening Authentication:}
    \begin{enumerate}
        \item \textbf{Disable Password Authentication:} Modify the SSH server configuration (`sshd_config`) to disallow password-based logins (`PasswordAuthentication no`).
        \item \textbf{Enforce Public Key Authentication:} Mandate the use of strong cryptographic keys (e.g., ED25519 or RSA 4096-bit) for all SSH access. This is significantly more secure than passwords.
    \end{enumerate}

    \item \textbf{Monitoring and Further Analysis:}
    \begin{enumerate}
        \item \textbf{Implement Intrusion Prevention:} Deploy a tool like \texttt{fail2ban} to automatically block IP addresses that exhibit malicious behavior, such as repeated failed login attempts.
        \item \textbf{Conduct a Deeper Scan:} Perform an authenticated, version-detecting vulnerability scan on the host to identify the specific SSH server software and version, and check for any associated vulnerabilities.
    \end{enumerate}
\end{itemize}

\end{document}
```