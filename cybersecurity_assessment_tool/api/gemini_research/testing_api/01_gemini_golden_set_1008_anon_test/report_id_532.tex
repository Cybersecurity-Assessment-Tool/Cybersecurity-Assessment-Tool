```latex
\documentclass[12pt, a4paper]{article}

% Preamble: Required Packages
\usepackage[margin=1in]{geometry} % Set page margins
\usepackage{pifont}               % For checkmarks and crosses (\ding)
\usepackage{booktabs}             % For professional-looking tables
\usepackage{hyperref}             % For hyperlinks, URLs, and metadata
\usepackage{url}                  % For formatting URLs
\usepackage{seqsplit}             % To split long strings in \texttt
\usepackage{graphicx}             % For including logos, etc.
\usepackage[table]{xcolor}        % For coloring table cells

% --- Document Metadata ---
\hypersetup{
    colorlinks=true,
    linkcolor=blue,
    filecolor=magenta,      
    urlcolor=cyan,
    pdftitle={Cybersecurity Posture Assessment Report},
    pdfauthor={Cybersecurity Analyst},
    pdfsubject={Security Analysis},
    pdfkeywords={Cybersecurity, Risk, Assessment},
    pdftoolbar=true,
}

% --- Custom Commands & Colors ---
\newcommand{\yes}{\ding{51}} % Green checkmark
\newcommand{\no}{\ding{55}}  % Red X

\definecolor{highseverity}{HTML}{D9534F}
\definecolor{mediumseverity}{HTML}{F0AD4E}
\definecolor{lowseverity}{HTML}{5CB85C}
\definecolor{tableheader}{gray}{0.9}

% --- Document Start ---
\begin{document}

% --- Title Page ---
\begin{titlepage}
    \centering
    \vspace*{1cm}
    
    \Huge
    \textbf{Cybersecurity Posture Assessment Report}
    
    \vspace{1.5cm}
    
    \Large
    Prepared for: \\
    \vspace{0.5cm}
    \textbf{[Organization Name]}
    
    \vfill % Pushes the following content to the bottom
    
    \large
    \textbf{Author:} Cybersecurity Analyst \\
    \textbf{Date:} \today
    
\end{titlepage}

\tableofcontents
\newpage

% --- Section 1: Executive Summary ---
\section{Executive Summary}
This report provides a comprehensive assessment of the cybersecurity posture for \textbf{[Organization Name]}, based on an analysis of organizational security controls, an external network scan, and a review of known risks.

The assessment reveals a mixed security posture. On the one hand, the organization demonstrates strong foundational controls, including the enforcement of Multi-Factor Authentication (MFA) across key systems and a well-configured external firewall that presented no open ports during the scan. These measures significantly reduce the risk of unauthorized external access.

However, a critical gap was identified in the organization's security awareness program. The absence of mandatory security training for new employees constitutes a \textbf{High} risk. New hires are a primary target for social engineering and phishing attacks, and this procedural gap leaves the organization vulnerable to human-centric threats that can bypass even strong technical controls.

Immediate action is recommended to implement a robust security onboarding program for all new personnel to mitigate this critical vulnerability and enhance the organization's overall resilience against cyber threats.

% --- Section 2: Organizational Information ---
\section{Organizational Information}
This section provides the key identifying information for the organization under review. The data provided may be anonymized for reporting purposes.

\begin{table}[h!]
\centering
\begin{tabular}{@{}ll@{}}
\toprule
\textbf{Attribute} & \textbf{Value} \\ \midrule
Organization Name  & \textbf{[Organization Name]} \\
Primary Domain     & \texttt{[Domain]} \\
External IP Scanned & \texttt{[Client IP]} \\ \bottomrule
\end{tabular}
\caption{Client Organizational Details}
\end{table}

% --- Section 3: Security Control Review ---
\section{Security Control Review}
The following table summarizes the organization's responses to a security controls questionnaire. Each response is assessed against industry best practices. A red cross (\no) indicates a significant control gap that requires immediate attention.

\begin{table}[h!]
\centering
\rowcolors{2}{gray!10}{white}
\begin{tabular}{@{}p{0.6\textwidth}ccp{0.2\textwidth}@{}}
\toprule
\rowcolor{tableheader}
\textbf{Control Question} & \textbf{Response} & \textbf{Assessment} \\ \midrule
Do you require MFA to access email? & Yes & \yes \\
Do you require MFA to log into computers? & Yes & \yes \\
Do you require MFA to access sensitive data systems? & Yes & \yes \\
Does your organization have an employee acceptable use policy? & Yes & \yes \\
\textbf{Does your organization do security awareness training for new employees?} & \textbf{No} & \textbf{\no} \\
Does your organization do security awareness training for all employees at least once per year? & Yes & \yes \\ \bottomrule
\end{tabular}
\caption{Security Controls Questionnaire Analysis}
\end{table}

\paragraph{Key Finding:} The lack of security awareness training for new employees is a critical deficiency. While an annual training program is in place, new staff remain untrained for up to a year, creating a significant window of vulnerability. This gap undermines other security controls, as untrained users are more likely to fall victim to phishing or social engineering attacks.

% --- Section 4: Technical Scan Results ---
\section{Technical Scan Results}
An external network vulnerability scan was conducted to identify exposed services and potential weaknesses visible from the public internet.

\begin{itemize}
    \item \textbf{Target IP Address:} \texttt{[Target IP]}
    \item \textbf{Scan Date:} 2023-10-27
    \item \textbf{Scanner Used:} Nmap
\end{itemize}

\paragraph{Summary of Findings:} The scan results were positive, indicating a strong network perimeter defense.
\begin{itemize}
    \item \textbf{Open Ports:} No open TCP or UDP ports were discovered on the target host.
    \item \textbf{Port State:} All scanned ports were reported as 'closed'. This suggests that a firewall is actively blocking or rejecting connection attempts, which is the desired configuration for a secure perimeter.
\end{itemize}
This finding indicates that the external-facing network infrastructure is well-configured to prevent unauthorized discovery and access to services.

% --- Section 5: Risk Assessment ---
\section{Risk Assessment}
This section synthesizes findings from the security control review and technical scan to identify and prioritize key risks. The risk below was identified during this assessment, as no pre-existing risks were provided.

\begin{table}[h!]
\centering
\begin{tabular}{@{}lp{0.5\textwidth}l@{}}
\toprule
\rowcolor{tableheader}
\textbf{Risk ID} & \textbf{Risk Name \& Overview} & \textbf{Severity} \\ \midrule
RISK-001 & \textbf{Lack of Security Training for New Employees} \newline \small New hires are not provided with security awareness training upon joining, making them highly susceptible to phishing, social engineering, and unintentional policy violations. This gap exposes the organization to initial access threats. & \cellcolor{highseverity}High \\
\bottomrule
\end{tabular}
\caption{Identified Cybersecurity Risks}
\end{table}

% --- Section 6: Recommendations ---
\section{Recommendations}
The following actionable recommendations are provided to address the identified risks and improve the overall security posture of \textbf{[Organization Name]}.

\begin{table}[h!]
\centering
\begin{tabular}{@{}lcl@{}}
\toprule
\rowcolor{tableheader}
\textbf{ID} & \textbf{Priority} & \textbf{Recommendation} \\ \midrule
\rowcolor{gray!10}
REC-001 & High & \parbox[t]{0.7\textwidth}{Implement a mandatory security awareness training module as part of the new employee onboarding process. This training should cover, at a minimum: phishing identification, acceptable use policies, password hygiene, and incident reporting procedures.} \\
\addlinespace
REC-002 & Medium & \parbox[t]{0.7\textwidth}{Review and update the annual security awareness training content to ensure it reflects current threat trends and reinforces the foundational concepts introduced during onboarding.} \\
\bottomrule
\end{tabular}
\caption{Actionable Security Recommendations}
\end{table}

\end{document}
```