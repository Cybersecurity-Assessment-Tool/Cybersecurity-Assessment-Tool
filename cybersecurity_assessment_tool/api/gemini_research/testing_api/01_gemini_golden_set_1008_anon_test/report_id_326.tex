```latex
\documentclass[12pt]{article}

% --- PACKAGE IMPORTS ---
\usepackage[margin=1in]{geometry} % For setting page margins
\usepackage{pifont}                 % For checkmarks and crosses (\ding)
\usepackage{booktabs}               % For professional-looking tables
\usepackage{hyperref}               % For hyperlinks and metadata
\usepackage{url}                    % For formatting URLs
\usepackage{seqsplit}               % For splitting long strings without spaces
\usepackage{xcolor}                 % For colors

% --- DOCUMENT METADATA ---
\hypersetup{
    colorlinks=true,
    linkcolor=blue,
    filecolor=magenta,      
    urlcolor=cyan,
    pdftitle={Cybersecurity Risk Assessment Report},
    pdfauthor={Cybersecurity Analyst},
    pdfsubject={Security Assessment},
    pdfkeywords={Security, Risk, Analysis},
}

% --- HELPER COMMANDS ---
\newcommand{\yes}{\ding{51}} % Green checkmark
\newcommand{\no}{\ding{55}}  % Red cross

\definecolor{darkgreen}{rgb}{0.0, 0.5, 0.0}
\renewcommand{\yes}{{\color{darkgreen}\ding{51}}}
\renewcommand{\no}{{\color{red}\ding{55}}}

% --- DOCUMENT START ---
\begin{document}

% --- TITLE PAGE ---
\begin{titlepage}
    \centering
    \vspace*{1cm}
    \Huge
    \textbf{Cybersecurity Risk Assessment Report}
    \vspace{1.5cm}
    \Large
    Prepared for: \\
    \vspace{0.5cm}
    \textbf{[Organization Name]}
    \vfill
    \Large
    \textbf{Date:} \today \\
    \vspace{0.8cm}
    \large
    Generated by: Expert Cybersecurity Analyst
\end{titlepage}

\tableofcontents
\newpage

% --- EXECUTIVE SUMMARY ---
\section*{1.0 Executive Summary}
This report details the findings of a cybersecurity assessment conducted for \textbf{[Organization Name]}. The analysis correlates results from an external network scan, a security controls questionnaire, and a review of pre-existing risks.

The assessment identified several critical and high-severity risks that require immediate attention. A key finding is an externally exposed FTP server running a dangerously outdated version of \texttt{vsftpd} (2.3.4), which is known to contain a critical backdoor vulnerability (CVE-2011-2523). This service also permits anonymous login, presenting a direct and immediate threat of unauthorized access and system compromise.

Furthermore, significant gaps were identified in organizational security controls. The lack of Multi-Factor Authentication (MFA) for accessing email and sensitive data systems drastically increases the risk of account takeover and data breaches. The absence of a formal Acceptable Use Policy contributes to an environment where insecure services, such as the identified FTP server, can be deployed without oversight.

These new findings, combined with the pre-existing risk of outdated Windows 7 workstations, paint a picture of a high-risk environment. Immediate remediation of the vulnerable FTP server and the implementation of MFA are the highest priorities.

% --- ORGANIZATIONAL INFORMATION ---
\section*{2.0 Organizational Information}
This section provides the key identification details for the organization under review. As this report was generated in template mode, placeholders are used where specific data was not provided.

\begin{itemize}
    \item \textbf{Organization Name:} \textbf{[Organization Name]}
    \item \textbf{Primary Domain:} \texttt{[Domain]}
    \item \textbf{Scanned External IP:} \texttt{[Client IP]}
\end{itemize}

% --- SECURITY CONTROL REVIEW ---
\section*{3.0 Security Control Review}
The following table summarizes the organization's responses to a security controls questionnaire. Answers marked with \no\ indicate significant gaps in the security posture that often correlate with technical vulnerabilities.

\begin{table}[h!]
\centering
\caption{Security Controls Questionnaire Results}
\begin{tabular}{p{0.8\linewidth} c}
\toprule
\textbf{Control Question} & \textbf{Status} \\
\midrule
Do you require MFA to access email? & \no \\
Do you require MFA to log into computers? & \yes \\
Do you require MFA to access sensitive data systems? & \no \\
Does your organization have an employee acceptable use policy? & \no \\
Does your organization do security awareness training for new employees? & \yes \\
Does your organization do security awareness training for all employees at least once per year? & \yes \\
\bottomrule
\end{tabular}
\end{table}

\subsection*{Analysis of Control Gaps}
\begin{itemize}
    \item \textbf{Lack of MFA on Email \& Sensitive Data:} The absence of MFA on email and sensitive systems represents a critical weakness. Email is a primary target for phishing attacks, and a compromised account can serve as a pivot point into the organization. Lack of MFA on sensitive data systems removes a crucial layer of defense against unauthorized access.
    \item \textbf{No Acceptable Use Policy (AUP):} Without a formal AUP, employees lack clear guidelines on the safe and acceptable use of company assets. This policy gap can lead to unintentional security risks, such as the deployment of unauthorized and insecure services.
\end{itemize}

% --- TECHNICAL SCAN RESULTS ---
\section*{4.0 Technical Scan Results}
An external network scan was performed on the target IP address \texttt{[Target IP]}. The scan identified one open port with a critically vulnerable service.

\begin{table}[h!]
\centering
\caption{Open Port Details for \texttt{[Target IP]}}
\begin{tabular}{l l l l}
\toprule
\textbf{Port} & \textbf{State} & \textbf{Service} & \textbf{Version Details} \\
\midrule
21/tcp & open & ftp & vsftpd 2.3.4 \\
\bottomrule
\end{tabular}
\end{table}

\subsection*{Detailed Findings}
\begin{itemize}
    \item \textbf{Critical Vulnerability - vsftpd 2.3.4 Backdoor (CVE-2011-2523):} The version of the FTP server, \texttt{vsftpd 2.3.4}, is extremely old and contains a well-known, high-severity backdoor. An attacker can exploit this vulnerability by sending a specific sequence of characters in the username field, which opens a command shell on port 6200. This provides the attacker with remote control over the server.
    \item \textbf{Critical Misconfiguration - Anonymous FTP Login:} The scan confirmed that anonymous FTP login is allowed. This permits any user on the internet to connect to the server and potentially access, upload, or download files without authentication. This could lead to a data breach or allow an attacker to use the server to host malicious content.
\end{itemize}

% --- RISK ASSESSMENT SUMMARY ---
\section*{5.0 Risk Assessment Summary}
This section synthesizes all findings into a prioritized list of identified risks.

\begin{table}[h!]
\centering
\caption{Consolidated Risk Register}
\begin{tabular}{p{0.1\linewidth} p{0.25\linewidth} p{0.45\linewidth} l}
\toprule
\textbf{ID} & \textbf{Risk Name} & \textbf{Description} & \textbf{Severity} \\
\midrule
R-01 & Exploitable FTP Server & An external FTP server is running \texttt{vsftpd 2.3.4}, which is vulnerable to a remote command execution backdoor (CVE-2011-2523). Anonymous login is also enabled. & \textbf{Critical} \\
\addlinespace
R-02 & Insufficient Access Controls & MFA is not enforced for access to email or sensitive data systems, leaving them vulnerable to compromise via stolen credentials. & \textbf{Critical} \\
\addlinespace
R-03 & Lack of Foundational Policies & The absence of an Acceptable Use Policy creates an uncontrolled IT environment and increases the likelihood of security incidents. & High \\
\addlinespace
R-04 & Outdated Operating Systems & The organization has workstations running Windows 7, which is an unsupported OS and no longer receives security updates. (Pre-existing risk). & Medium \\
\bottomrule
\end{tabular}
\end{table}

% --- RECOMMENDATIONS ---
\section*{6.0 Recommendations}
The following actionable recommendations are provided to mitigate the identified risks. They are prioritized based on severity and exploitability.

\subsection*{Priority 1: Immediate Actions (Within 24-48 Hours)}
\begin{enumerate}
    \item \textbf{Mitigate Vulnerable FTP Server (R-01):}
    \begin{itemize}
        \item Immediately take the FTP server offline.
        \item If the service is not required for business, decommission it permanently.
        \item If the service is required, it must be upgraded to a modern, patched version of FTP software (e.g., a recent version of vsftpd or ProFTPD) and reconfigured to disable anonymous access.
        \item Implement strict firewall rules to limit access to trusted IP addresses only.
        \item Consider replacing FTP with a more secure file transfer protocol like SFTP (SSH File Transfer Protocol).
    \end{itemize}
\end{enumerate}

\subsection*{Priority 2: High-Impact Actions (Within 1-2 Weeks)}
\begin{enumerate}
    \item \textbf{Implement Multi-Factor Authentication (R-02):}
    \begin{itemize}
        \item Enable and enforce MFA across all user accounts for email access (e.g., Office 365, Google Workspace).
        \item Identify all systems classified as containing sensitive data and enforce MFA for access.
    \end{itemize}
    \item \textbf{Develop and Implement an Acceptable Use Policy (R-03):}
    \begin{itemize}
        \item Draft a formal AUP that defines rules for technology use, data handling, and security responsibilities for all employees.
        \item Communicate the policy to all staff and require acknowledgment of receipt.
    \end{itemize}
\end{enumerate}

\subsection*{Priority 3: Ongoing/Planned Initiatives}
\begin{enumerate}
    \item \textbf{Continue Operating System Upgrades (R-04):}
    \begin{itemize}
        \item Execute the existing plan to upgrade or replace all workstations running Windows 7 to a modern, supported operating system like Windows 10 or 11.
    \end{itemize}
\end{enumerate}

% --- DOCUMENT END ---
\end{document}
```