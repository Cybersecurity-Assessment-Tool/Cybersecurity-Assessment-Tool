```latex
\documentclass[12pt, a4paper]{article}

% Preamble: Required Packages
\usepackage[margin=1in]{geometry}
\usepackage{pifont} % For checkmarks and crosses
\usepackage{booktabs} % For professional tables
\usepackage{hyperref} % For clickable links
\usepackage{url} % For URL formatting
\usepackage{seqsplit} % To split long strings in tt font
\usepackage{graphicx}
\usepackage{xcolor}
\usepackage{datetime}

% --- Document Metadata ---
\title{Cybersecurity Posture Assessment Report}
\author{Cybersecurity Analysis Division}
\date{\today}

% --- Hyperref Setup ---
\hypersetup{
    colorlinks=true,
    linkcolor=black,
    urlcolor=blue,
    pdftitle={Cybersecurity Posture Assessment Report},
    pdfauthor={Cybersecurity Analysis Division},
    pdfsubject={Security Assessment},
    pdfkeywords={Security, Nmap, Risk, Assessment}
}

% --- Document Body ---
\begin{document}

\maketitle
\thispagestyle{empty}
\newpage

\tableofcontents
\newpage

% ==============================================================================
\section{Executive Summary}
% ==============================================================================
This report provides a comprehensive cybersecurity assessment for \textbf{[Organization Name]}, based on a combination of technical network scanning, a security controls questionnaire, and a review of pre-existing risk documentation. The analysis was conducted on \today.

The assessment reveals several critical and high-risk security gaps that require immediate attention. A pre-existing vulnerability, \textbf{Localhost Exposed}, is documented with a CVSS score of 10.0 (Critical) and affects the primary external asset. Furthermore, organizational policy and access control weaknesses were identified, most notably the lack of Multi-Factor Authentication (MFA) for sensitive data systems and the absence of an employee Acceptable Use Policy (AUP).

A technical scan of the external IP address \texttt{[Client IP]} identified an open SSH port (22), which represents a significant attack vector if not securely configured.

The overall security posture is considered weak due to these converging factors. Immediate remediation of the identified critical risks is strongly recommended to reduce the likelihood of a security breach.

% ==============================================================================
\section{Organizational Information}
% ==============================================================================
The following information was used as the basis for this assessment. Due to the anonymized nature of the provided data, placeholders have been used where necessary.

\begin{table}[h!]
\centering
\begin{tabular}{@{}ll@{}}
\toprule
\textbf{Attribute} & \textbf{Value} \\ \midrule
Organization Name & \textbf{[Organization Name]} \\
Primary Domain & \texttt{[Domain]} \\
External IP Scanned & \texttt{[Client IP]} \\
Target IP in Scan & \texttt{[Target IP]} \\ \bottomrule
\end{tabular}
\caption{Client Organizational Details}
\label{tab:org_info}
\end{table}

% ==============================================================================
\section{Security Control Review (Questionnaire Analysis)}
% ==============================================================================
A security questionnaire was completed to evaluate existing administrative and technical controls. The responses are summarized below. A green checkmark (\ding{51}) indicates a positive control is in place, while a red cross (\ding{55}) indicates a security gap.

\begin{table}[h!]
\centering
\begin{tabular}{@{}p{0.8\textwidth}c@{}}
\toprule
\textbf{Control Question} & \textbf{Response} \\ \midrule
Do you require MFA to access email? & \textcolor{green}{\ding{51}} \\
Do you require MFA to log into computers? & \textcolor{green}{\ding{51}} \\
\textbf{Do you require MFA to access sensitive data systems?} & \textcolor{red}{\ding{55}} \\
\textbf{Does your organization have an employee acceptable use policy?} & \textcolor{red}{\ding{55}} \\
Does your organization do security awareness training for new employees? & \textcolor{green}{\ding{51}} \\
Does your organization do security awareness training for all employees at least once per year? & \textcolor{green}{\ding{51}} \\ \bottomrule
\end{tabular}
\caption{Security Controls Questionnaire Results}
\label{tab:controls}
\end{table}

\subsection*{Analysis of Gaps}
Two critical gaps were identified from the questionnaire:
\begin{itemize}
    \item \textbf{No MFA for Sensitive Systems:} The absence of MFA on systems housing sensitive data is a critical vulnerability. This significantly increases the risk of unauthorized access via compromised credentials, which is a primary vector in data breaches.
    \item \textbf{No Acceptable Use Policy (AUP):} The lack of a formal AUP creates ambiguity regarding the secure and appropriate use of company assets. This can lead to unintentional security incidents and makes it difficult to enforce security standards among employees.
\end{itemize}

% ==============================================================================
\section{Technical Scan Results}
% ==============================================================================
A network scan was performed on the target host \texttt{[Target IP]} to identify open ports and exposed services.

\subsection*{Host Status}
The target host was found to be \textbf{up} and responsive at the time of the scan.

\subsection*{Open Ports}
The following table details the open ports discovered on the target system.

\begin{table}[h!]
\centering
\begin{tabular}{@{}llll@{}}
\toprule
\textbf{Port} & \textbf{State} & \textbf{Service (Inferred)} & \textbf{Notes} \\ \midrule
22/tcp & open & SSH & Secure Shell service is exposed to the internet. \\
& & & No version information was available in the scan data. \\
& & & This is a common vector for brute-force attacks. \\ \bottomrule
\end{tabular}
\caption{Discovered Open Ports on \texttt{[Target IP]}}
\label{tab:nmap_results}
\end{table}

% ==============================================================================
\section{Consolidated Risk Assessment}
% ==============================================================================
The following table synthesizes findings from the security questionnaire, the technical scan, and pre-existing risk documentation into a consolidated list of identified risks.

\begin{table}[h!]
\centering
\begin{tabular}{@{}p{0.25\linewidth}p{0.4\linewidth}p{0.1\linewidth}p{0.15\linewidth}@{}}
\toprule
\textbf{Risk Name} & \textbf{Description} & \textbf{Severity} & \textbf{Affected Asset(s)} \\ \midrule
\textbf{Localhost Exposed} & A pre-existing critical vulnerability was identified. The nature of the exposure requires immediate investigation. & \textbf{Critical} & \texttt{[Target IP]} \\
\addlinespace
\textbf{No MFA on Sensitive Systems} & Lack of multi-factor authentication for critical data systems allows for single-factor credential compromise. & \textbf{Critical} & Internal Data Systems \\
\addlinespace
\textbf{Exposed SSH Service} & The Secure Shell service is publicly accessible, creating a direct vector for brute-force or credential-stuffing attacks. & High & \texttt{[Target IP]} \\
\addlinespace
\textbf{No Acceptable Use Policy} & Absence of a formal policy governing the use of IT assets increases the risk of insider threat and accidental data exposure. & High & All Employees \\ \bottomrule
\end{tabular}
\caption{Summary of Identified Risks}
\label{tab:risk_summary}
\end{table}

% ==============================================================================
\section{Recommendations}
% ==============================================================================
The following actions are recommended to mitigate the identified risks and improve the overall security posture of \textbf{[Organization Name]}.

\subsection*{1. Remediate "Localhost Exposed" (Critical)}
The pre-existing vulnerability documented as "Localhost Exposed" with a CVSS score of 10.0 must be the top priority.
\begin{itemize}
    \item \textbf{Immediate Action:} Launch an immediate investigation to understand the technical details of this vulnerability on asset \texttt{[Target IP]}.
    \item \textbf{Long-Term Fix:} Apply the necessary patches, firewall rules, or configuration changes to fully remediate the exposure.
\end{itemize}

\subsection*{2. Implement MFA on Sensitive Systems (Critical)}
To protect against unauthorized access to critical data, MFA must be enforced.
\begin{itemize}
    \item \textbf{Immediate Action:} Identify all systems classified as containing sensitive data.
    \item \textbf{Long-Term Fix:} Procure and deploy an MFA solution for these systems. Mandate its use for all user accounts, especially those with administrative privileges.
\end{itemize}

\subsection*{3. Secure the Exposed SSH Service (High)}
The publicly accessible SSH service on \texttt{[Target IP]} must be hardened.
\begin{itemize}
    \item \textbf{Immediate Action:} Review the current SSH configuration. If possible, restrict access to a whitelist of trusted IP addresses via firewall rules.
    \item \textbf{Long-Term Fix:} Enforce public key authentication and disable password-based logins. Implement an intrusion prevention tool like \texttt{fail2ban} to block brute-force attempts.
\end{itemize}

\subsection*{4. Develop and Implement an Acceptable Use Policy (High)}
To establish a baseline for security conduct, a formal AUP is required.
\begin{itemize}
    \item \textbf{Immediate Action:} Draft an AUP that clearly defines the rules for using company networks, devices, and data.
    \item \textbf{Long-Term Fix:} Ratify the policy with management, disseminate it to all employees, and integrate it into the new-hire onboarding process. Require all staff to formally acknowledge they have read and understood the policy.
\end{itemize}

\end{document}
```