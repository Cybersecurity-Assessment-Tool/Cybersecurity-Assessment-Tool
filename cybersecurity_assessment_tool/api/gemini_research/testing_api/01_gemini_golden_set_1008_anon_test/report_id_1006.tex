```latex
\documentclass[12pt]{article}

% --- PREAMBLE ---
\usepackage[margin=1in]{geometry}
\usepackage{pifont} % Required for \ding
\usepackage{booktabs} % For professional-looking tables
\usepackage{hyperref} % For clickable links and references
\usepackage{url} % For formatting URLs
\usepackage{seqsplit} % To split long monospaced strings
\usepackage{xcolor} % For colors

% --- HYPERREF SETUP ---
\hypersetup{
    colorlinks=true,
    linkcolor=black,
    urlcolor=blue,
    pdftitle={Cybersecurity Posture Assessment Report},
    pdfauthor={Cybersecurity Analyst},
}

% --- DOCUMENT METADATA ---
\title{Cybersecurity Posture Assessment Report}
\author{\textbf{[Organization Name]}}
\date{\today}

% --- DOCUMENT START ---
\begin{document}

\maketitle
\hrule
\vspace{1cm}

% ===================================================================
% SECTION 1: EXECUTIVE SUMMARY
% ===================================================================
\section*{Executive Summary}

This report provides a cybersecurity posture assessment for \textbf{[Organization Name]}, synthesizing data from a security controls questionnaire, a review of existing risks, and an external network scan.

The assessment reveals a mixed security posture. The organization demonstrates strong foundational controls in identity and access management by consistently enforcing Multi-Factor Authentication (MFA) across email, workstations, and sensitive systems. An acceptable use policy is also in place, which is a positive finding.

However, two critical gaps were identified in the security awareness program. The lack of mandatory security training for both new and existing employees constitutes a \textbf{High} risk. This deficiency significantly increases the organization's susceptibility to social engineering attacks, such as phishing, which are a primary vector for security breaches.

The external network scan conducted on the target IP address \texttt{[Target IP]} did not identify any open ports or exposed services. This suggests the host is either offline or protected by a well-configured firewall that denies all unsolicited inbound traffic, which is a strong defensive posture from an external perspective.

Immediate action is recommended to establish a comprehensive security awareness training program to mitigate the identified human-factor risks.

\vspace{1cm}

% ===================================================================
% SECTION 2: ORGANIZATIONAL INFORMATION
% ===================================================================
\section*{Organizational Information}

The following details were used as the basis for this assessment. As per the provided data, placeholder values are used where specific information was not available.

\begin{table}[h!]
\centering
\begin{tabular}{@{}ll@{}}
\toprule
\textbf{Item} & \textbf{Detail} \\ \midrule
Organization Name & \textbf{[Organization Name]} \\
Primary Email Domain & \seqsplit{\texttt{[Domain]}} \\
Scanned External IP & \seqsplit{\texttt{[Client IP]}} \\ \bottomrule
\end{tabular}
\caption{Client Profile}
\end{table}

\vspace{1cm}

% ===================================================================
% SECTION 3: SECURITY CONTROL REVIEW
% ===================================================================
\section*{Security Control Review}

The following table summarizes the organization's responses to the security controls questionnaire. A checkmark (\ding{51}) indicates a positive control is in place, while a cross (\ding{55}) indicates a control gap.

\begin{table}[h!]
\centering
\begin{tabular}{@{}lc@{}}
\toprule
\textbf{Control Question} & \textbf{Status} \\ \midrule
Do you require MFA to access email? & \textcolor{green}{\ding{51}} \\
Do you require MFA to log into computers? & \textcolor{green}{\ding{51}} \\
Do you require MFA to access sensitive data systems? & \textcolor{green}{\ding{51}} \\
Does your organization have an employee acceptable use policy? & \textcolor{green}{\ding{51}} \\
\midrule
\textbf{Does your organization do security awareness training for new employees?} & \textbf{\textcolor{red}{\ding{55}}} \\
\textbf{Does your organization do security awareness training for all employees at least once per year?} & \textbf{\textcolor{red}{\ding{55}}} \\
\bottomrule
\end{tabular}
\caption{Security Controls Questionnaire Results}
\end{table}

\textbf{Analysis:} The consistent implementation of MFA is commendable and significantly reduces the risk of account compromise. However, the complete absence of a security awareness training program is a critical vulnerability. Employees are the first line of defense, and without proper training, they are unprepared to recognize and respond to modern cyber threats.

\vspace{1cm}

% ===================================================================
% SECTION 4: TECHNICAL SCAN RESULTS
% ===================================================================
\section*{Technical Scan Results}

An external network vulnerability scan was performed against the designated target IP address.

\begin{itemize}
    \item \textbf{Target IP:} \seqsplit{\texttt{[Target IP]}}
    \item \textbf{Findings:} The scan completed without identifying any open TCP or UDP ports. No services, products, or versions were enumerated.
\end{itemize}

\textbf{Interpretation:} The absence of open ports is a positive security finding. It indicates that the target system is not exposing any services to the public internet, which may be due to a restrictive firewall policy or because the host was not online during the scan. This effectively minimizes the external attack surface of this specific asset. No further technical vulnerabilities were identified from this scan.

\vspace{1cm}

% ===================================================================
% SECTION 5: RISK ASSESSMENT
% ===================================================================
\section*{Risk Assessment}

The following table correlates findings from the security control review and technical scan. As no pre-existing or technical vulnerabilities were found, the risks listed are derived directly from the identified policy and procedure gaps.

\begin{table}[h!]
\centering
\begin{tabular}{@{}p{0.1\linewidth} p{0.3\linewidth} p{0.4\linewidth} p{0.1\linewidth}@{}}
\toprule
\textbf{Risk ID} & \textbf{Risk Name} & \textbf{Overview} & \textbf{Severity} \\ \midrule
R-01 & \textbf{Lack of New-Hire Security Training} & New employees are not equipped with fundamental security knowledge upon joining. This makes them immediately vulnerable to social engineering attacks and unintentional policy violations. & \textbf{High} \\
\addlinespace
R-02 & \textbf{Lack of Annual Security Training} & Without a recurring training program, the security awareness of all employees degrades over time. They are less likely to recognize and properly handle evolving threats like sophisticated phishing and ransomware attacks. & \textbf{High} \\
\bottomrule
\end{tabular}
\caption{Summary of Identified Risks}
\end{table}

\vspace{1cm}

% ===================================================================
% SECTION 6: RECOMMENDATIONS
% ===================================================================
\section*{Recommendations}

To address the identified risks and improve the overall cybersecurity posture, the following actions are recommended with high priority.

\begin{enumerate}
    \item \textbf{Implement Mandatory Onboarding Security Training (Risk R-01):}
    \begin{itemize}
        \item \textbf{Action:} Develop a foundational security awareness training module and integrate it into the mandatory onboarding process for all new hires, including contractors.
        \item \textbf{Content:} The training should cover, at a minimum: acceptable use policies, phishing and social engineering awareness, password security, and procedures for reporting security incidents.
        \item \textbf{Impact:} Ensures a baseline level of security knowledge across the organization from day one of employment, reducing the risk of early-tenure security incidents.
    \end{itemize}
    \vspace{0.5cm}
    \item \textbf{Establish an Annual Security Awareness Program (Risk R-02):}
    \begin{itemize}
        \item \textbf{Action:} Procure or develop a comprehensive security awareness training program to be completed by all employees annually.
        \item \textbf{Content:} This program should be updated yearly to reflect the current threat landscape. It should be supplemented with regular phishing simulation exercises to test and reinforce the training in a practical setting.
        \item \textbf{Impact:} Creates a security-conscious culture, keeps defenses sharp against evolving threats, and provides metrics on the organization's resilience to social engineering.
    \end{itemize}
\end{enumerate}

\end{document}
```