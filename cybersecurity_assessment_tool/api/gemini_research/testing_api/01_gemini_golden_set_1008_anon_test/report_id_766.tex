```latex
\documentclass[12pt]{article}

% Preamble: Required Packages
\usepackage[margin=1in]{geometry}
\usepackage{pifont} % For checkmarks and crosses
\usepackage{booktabs} % For professional tables
\usepackage{hyperref} % For hyperlinks
\usepackage{url} % For URL formatting
\usepackage{seqsplit} % To split long strings without spaces
\usepackage{xcolor} % For colored text
\usepackage{graphicx} % For potential logos/images
\usepackage{fancyhdr} % For headers and footers

% --- Document Setup ---
\hypersetup{
    colorlinks=true,
    linkcolor=blue,
    filecolor=magenta,      
    urlcolor=cyan,
    pdftitle={Cybersecurity Posture Assessment Report},
    pdfauthor={Cybersecurity Analyst},
    pdfsubject={Security Analysis},
    pdfkeywords={Security, Report, Analysis},
}

% Define colors for risk levels
\definecolor{critical}{HTML}{D7263D}
\definecolor{high}{HTML}{F49D40}
\definecolor{medium}{HTML}{F4D440}
\definecolor{low}{HTML}{88A2AA}

% --- Header and Footer ---
\pagestyle{fancy}
\fancyhf{}
\lhead{Confidential Security Report}
\rhead{\textbf{[Organization Name]}}
\cfoot{\thepage}

% --- Document Start ---
\begin{document}

% --- Title Page ---
\begin{titlepage}
    \centering
    \vspace*{1cm}
    \includegraphics[width=0.3\textwidth]{example-image-a} % Placeholder for company logo
    
    \vspace{1.5cm}
    
    \Huge
    \textbf{Cybersecurity Posture Assessment Report}
    
    \vspace{1.5cm}
    
    \Large
    Prepared for: \textbf{[Organization Name]}
    
    \vspace{2cm}
    
    \large
    Report Date: \today
    
    \vfill
    
    \normalsize
    \textit{This document contains sensitive and confidential information. Distribution is restricted to authorized personnel only.}
    
\end{titlepage}

\tableofcontents
\newpage

% --- Section 1: Executive Summary ---
\section{Executive Summary}
This report provides a comprehensive assessment of the cybersecurity posture for \textbf{[Organization Name]}, based on an analysis of organizational data, a technical network scan, and a review of pre-existing risks.

The assessment reveals a mixed security posture with several positive controls in place, such as the enforcement of Multi-Factor Authentication (MFA) for computer and sensitive system access. However, critical gaps were identified that expose the organization to significant threats, most notably Business Email Compromise (BEC) and social engineering attacks.

The most urgent findings are the lack of mandatory MFA for email, the absence of a formal employee acceptable use policy, and the failure to conduct annual security awareness training for all staff. These policy and procedural weaknesses represent a greater immediate threat than any technical vulnerabilities discovered during the external scan.

The external network scan of the target IP address revealed no open ports, which is a strong security finding, suggesting a well-configured firewall. No pre-existing vulnerabilities were provided for review.

This report outlines these findings in detail and provides actionable, prioritized recommendations to mitigate the identified risks and strengthen the overall security framework of \textbf{[Organization Name]}.

% --- Section 2: Organizational Information ---
\section{Organizational Information}
The following details were used as the basis for this assessment. Due to the anonymized nature of the provided data, placeholders have been used where necessary.

\begin{table}[h!]
\centering
\caption{Client Organizational Details}
\begin{tabular}{@{}ll@{}}
\toprule
\textbf{Attribute} & \textbf{Value} \\ \midrule
Organization Name & \textbf{[Organization Name]} \\
Primary Email Domain & \texttt{[Domain]} \\
External IP Address Scanned & \texttt{[Client IP]} \\ \bottomrule
\end{tabular}
\end{table}

% --- Section 3: Security Control Review ---
\section{Security Control Review}
An assessment of organizational security controls was conducted based on a standard security questionnaire. The responses indicate foundational gaps in policy and user-level security. A "No" answer (\ding{55}) signifies a deviation from security best practices and a potential area of high risk.

\begin{table}[h!]
\centering
\caption{Security Questionnaire Analysis}
\label{tab:questionnaire}
\begin{tabular}{@{}p{0.8\linewidth}c@{}}
\toprule
\textbf{Control Question} & \textbf{Response} \\ \midrule
Do you require MFA to access email? & \ding{55} \\
Do you require MFA to log into computers? & \ding{51} \\
Do you require MFA to access sensitive data systems? & \ding{51} \\
Does your organization have an employee acceptable use policy? & \ding{55} \\
Does your organization do security awareness training for new employees? & \ding{51} \\
Does your organization do security awareness training for all employees at least once per year? & \ding{55} \\ \bottomrule
\end{tabular}
\end{table}

\subsection*{Analysis of Findings}
\begin{itemize}
    \item \textbf{MFA on Email (Critical Gap):} The absence of MFA on email is a critical vulnerability. Email accounts are the primary target for phishing attacks, which can lead to credential theft, data breaches, and Business Email Compromise (BEC).
    \item \textbf{Acceptable Use Policy (High Risk):} Without a formal Acceptable Use Policy (AUP), there are no clear guidelines for employees on the proper use of company assets. This increases the risk of insider threats, data leakage, and legal liabilities.
    \item \textbf{Annual Security Training (High Risk):} While training new hires is a good first step, the security landscape is constantly evolving. The lack of annual refresher training for all employees leaves the organization vulnerable to modern social engineering tactics.
\end{itemize}

% --- Section 4: Technical Scan Results ---
\section{Technical Scan Results}
A network scan was conducted against the target IP address provided. The results are summarized below.

\subsection*{Scan Target}
\begin{itemize}
    \item \textbf{Target IP Address:} \texttt{[Target IP]}
    \item \textbf{Scan Date:} Not Available
\end{itemize}

\subsection*{Summary of Findings}
The network scan against the target system did not identify any open TCP ports. This is a positive security finding, indicating that the external firewall is likely configured to deny all unsolicited inbound traffic ("default deny" posture). This significantly reduces the external attack surface of the asset. No vulnerabilities related to exposed services could be identified.

% --- Section 5: Risk Assessment & Correlation ---
\section{Risk Assessment \& Correlation}
This section synthesizes the findings from the security control review and technical scan. As no pre-existing risks were provided (Input 3 was empty), all identified risks are derived from this assessment. The primary risks are procedural and policy-based rather than technical.

\begin{table}[h!]
\centering
\caption{Summary of Identified Risks}
\label{tab:risks}
\begin{tabular}{@{}p{0.2\linewidth}p{0.2\linewidth}p{0.5\linewidth}@{}}
\toprule
\textbf{Risk ID} & \textbf{Severity} & \textbf{Overview} \\ \midrule
RISK-001 & \textcolor{critical}{\textbf{Critical}} & \textbf{Lack of MFA on Email:} Exposes the organization to a high likelihood of account compromise via phishing, leading to data breaches, financial fraud (BEC), and further internal network compromise. \\
\addlinespace
RISK-002 & \textcolor{high}{\textbf{High}} & \textbf{No Acceptable Use Policy:} Creates ambiguity regarding proper use of IT assets, increasing the risk of unintentional data exposure, insider threats, and non-compliance with regulations. \\
\addlinespace
RISK-003 & \textcolor{high}{\textbf{High}} & \textbf{Inadequate Security Training:} Without annual training, employees are more susceptible to evolving phishing and social engineering tactics, making them the weakest link in the security chain. \\
\bottomrule
\end{tabular}
\end{table}

% --- Section 6: Recommendations ---
\section{Recommendations}
Based on the risk assessment, the following prioritized actions are recommended to enhance the security posture of \textbf{[Organization Name]}.

\begin{enumerate}
    \item \textbf{[Critical] Implement MFA for Email Immediately:}
    \begin{itemize}
        \item \textbf{Action:} Enforce mandatory MFA for all user and administrative email accounts.
        \item \textbf{Justification:} This is the single most effective control to prevent email account takeovers and mitigate the risk of Business Email Compromise (BEC). This directly addresses RISK-001.
    \end{itemize}
    
    \item \textbf{[High] Develop and Implement an Acceptable Use Policy (AUP):}
    \begin{itemize}
        \item \textbf{Action:} Draft a formal AUP that clearly defines the rules for using company networks, systems, and data. Require all employees to read and acknowledge the policy.
        \item \textbf{Justification:} An AUP establishes a baseline for secure employee behavior, provides a basis for disciplinary action, and reduces legal liability. This directly addresses RISK-002.
    \end{itemize}
    
    \item \textbf{[High] Establish a Mandatory Annual Security Awareness Program:}
    \begin{itemize}
        \item \textbf{Action:} Procure and roll out a security awareness training program for all employees, to be completed annually. The training should cover phishing, password security, data handling, and social engineering.
        \item \textbf{Justification:} Regular training reinforces security concepts and keeps employees informed about current threats, turning them from a potential liability into a human firewall. This directly addresses RISK-003.
    \end{itemize}
\end{enumerate}

\end{document}
```