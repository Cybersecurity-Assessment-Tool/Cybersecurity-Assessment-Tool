```latex
\documentclass[12pt]{article}

% --- PACKAGES ---
\usepackage[margin=1in]{geometry}
\usepackage{pifont} % For checkmarks and crosses
\usepackage{booktabs} % For professional tables
\usepackage{hyperref} % For clickable links
\usepackage{url} % For URL formatting
\usepackage{seqsplit} % For splitting long strings to prevent overflow
\usepackage{fancyhdr} % For headers and footers
\usepackage{graphicx}
\usepackage{xcolor}
\usepackage{lastpage}

% --- DOCUMENT SETUP ---
\hypersetup{
    colorlinks=true,
    linkcolor=blue,
    filecolor=magenta,      
    urlcolor=cyan,
    pdftitle={Cybersecurity Posture Assessment},
    pdfpagemode=FullScreen,
}

% --- HEADER & FOOTER ---
\pagestyle{fancy}
\fancyhf{} % Clear all header and footer fields
\fancyhead[L]{\textbf{Cybersecurity Posture Assessment}}
\fancyhead[R]{\textbf{[Organization Name]}}
\fancyfoot[C]{Page \thepage\ of \pageref{LastPage}}
\fancyfoot[R]{\today}
\renewcommand{\headrulewidth}{0.4pt}
\renewcommand{\footrulewidth}{0.4pt}

% --- DOCUMENT START ---
\begin{document}

% --- TITLE PAGE ---
\begin{titlepage}
    \centering
    \vspace*{1cm}
    \Huge{\textbf{Cybersecurity Posture Assessment Report}}
    \vspace{1.5cm}
    \
    \large{Prepared for:}
    \vspace{0.5cm}
    \
    \Huge{\textbf{[Organization Name]}}
    \vspace{2cm}
    \
    \large{Date of Report: \today}
    \vspace{1cm}
    \
    \large{Date of Scan: [Scan Date]}
    \vfill
    \large{\textit{This report contains sensitive information and should be handled with care.}}
\end{titlepage}

\tableofcontents
\newpage

% --- SECTION 1: EXECUTIVE SUMMARY ---
\section{Executive Summary}
This report provides a comprehensive analysis of the cybersecurity posture for \textbf{[Organization Name]}, based on a review of organizational security controls, an external network scan, and an assessment of current risks.

The assessment identified two significant areas of concern requiring immediate attention. A \textbf{critical} gap was discovered in the lack of Multi-Factor Authentication (MFA) for email access, exposing the organization to a high risk of phishing, business email compromise, and account takeover. Additionally, a \textbf{high-risk} gap was identified due to the absence of a formal employee Acceptable Use Policy (AUP), which can lead to inconsistent security practices and a lack of recourse for policy violations.

The external network scan of the target IP address did not reveal any open ports. While this can indicate a strong firewall configuration, it underscores the importance of robust internal controls, as the primary risks identified are policy and configuration-based rather than externally exposed vulnerabilities.

Recommendations focus on the immediate implementation of MFA for email and the development and enforcement of a comprehensive AUP to mitigate these identified risks and strengthen the overall security posture.

\newpage

% --- SECTION 2: ORGANIZATIONAL INFORMATION ---
\section{Organizational Information}
The following information was used as the basis for this assessment. Due to the anonymized nature of the provided data, placeholders have been used where necessary.

\begin{table}[h!]
\centering
\begin{tabular}{@{}ll@{}}
\toprule
\textbf{Attribute} & \textbf{Value} \\ \midrule
Organization Name & \textbf{[Organization Name]} \\
Primary Email Domain & \texttt{[Domain]} \\
External IP Address Scanned & \texttt{[Client IP]} \\
Target of Network Scan & \texttt{[Target IP]} \\ \bottomrule
\end{tabular}
\caption{Client and Assessment Scope Details.}
\label{tab:org_info}
\end{table}

% --- SECTION 3: SECURITY CONTROL REVIEW ---
\section{Security Control Review}
A questionnaire was conducted to evaluate the implementation of key administrative and technical security controls. The results are summarized below. Answers marked with \ding{55} (No) represent significant gaps in the security framework.

\begin{table}[h!]
\centering
\begin{tabular}{@{}lcc@{}}
\toprule
\textbf{Control Question} & \textbf{Status} & \textbf{Risk Level} \\ \midrule
Do you require MFA to access email? & \ding{55} & \textcolor{red}{\textbf{Critical}} \\
Do you require MFA to log into computers? & \ding{51} & Low \\
Do you require MFA to access sensitive data systems? & \ding{51} & Low \\
Does your organization have an employee acceptable use policy? & \ding{55} & \textcolor{orange}{\textbf{High}} \\
Does your organization do security awareness training for new employees? & \ding{51} & Low \\
Does your organization do security awareness training for all employees? & \ding{51} & Low \\ \bottomrule
\end{tabular}
\caption{Security Controls Questionnaire Results.}
\label{tab:controls}
\end{table}

\subsection*{Analysis of Gaps}
\begin{itemize}
    \item \textbf{MFA for Email (Critical):} The absence of MFA on email is the most severe finding. Email accounts are a primary target for attackers. A compromised email account can lead to data breaches, financial fraud, and further infiltration into the network.
    \item \textbf{Acceptable Use Policy (High):} Lacking a formal AUP means there are no clearly defined rules for employees regarding the use of company assets. This can lead to unintentional security incidents and creates ambiguity in enforcing security standards.
\end{itemize}

% --- SECTION 4: TECHNICAL SCAN RESULTS ---
\section{Technical Scan Results}
An external network vulnerability scan was performed on the designated target IP address to identify exposed services and potential vulnerabilities.

\begin{itemize}
    \item \textbf{Target IP Address:} \texttt{[Target IP]}
    \item \textbf{Scan Date:} [Scan Date]
\end{itemize}

\subsection*{Findings}
The scan completed successfully and \textbf{no open ports or running services were detected} on the target system.

\subsection*{Interpretation}
This result typically indicates that the target system is protected by a well-configured firewall that is effectively blocking all unsolicited inbound traffic from the internet. This is a positive security posture from an external network perspective. However, it does not provide visibility into internal network security or application-level vulnerabilities that may be accessible after authentication.

% --- SECTION 5: RISK ASSESSMENT ---
\section{Risk Assessment}
This section synthesizes findings from the security control review, technical scan, and any pre-existing risk data. The following table summarizes the key risks identified during this assessment. No pre-existing vulnerabilities were provided for this analysis.

\begin{table}[h!]
\centering
\begin{tabular}{@{}p{0.3\linewidth}p{0.5\linewidth}p{0.15\linewidth}@{}}
\toprule
\textbf{Risk Name} & \textbf{Overview} & \textbf{Severity} \\ \midrule
\textbf{MFA Not Enforced for Email} & The lack of a second authentication factor for email access makes user accounts highly susceptible to compromise via phishing or credential stuffing attacks. This could lead to data breaches and financial loss. & \textcolor{red}{\textbf{Critical}} \\
\addlinespace
\textbf{Missing Acceptable Use Policy (AUP)} & Without a formal AUP, employees lack clear guidance on the secure and acceptable use of corporate resources. This increases the risk of insider threats, data mishandling, and policy violations. & \textcolor{orange}{\textbf{High}} \\
\bottomrule
\end{tabular}
\caption{Summary of Identified Risks.}
\label{tab:risks}
\end{table}

% --- SECTION 6: RECOMMENDATIONS ---
\section{Recommendations}
Based on the findings of this assessment, the following actions are recommended to mitigate the identified risks and improve the overall security posture of \textbf{[Organization Name]}.

\begin{enumerate}
    \item \textbf{Implement MFA for Email (Priority: Immediate)}
    \begin{itemize}
        \item \textbf{Action:} Enforce mandatory Multi-Factor Authentication (MFA) for all user and administrative email accounts.
        \item \textbf{Justification:} This is the single most effective control to prevent unauthorized access to email accounts, mitigating the risk of business email compromise and phishing-related breaches.
    \end{itemize}
    \vspace{0.5cm}
    
    \item \textbf{Develop and Implement an Acceptable Use Policy (Priority: High)}
    \begin{itemize}
        \item \textbf{Action:} Draft, approve, and disseminate a formal AUP that clearly defines the rules for using company networks, devices, and data. Require all employees to read and acknowledge the policy.
        \item \textbf{Justification:} An AUP establishes a baseline for secure behavior, reduces ambiguity, and provides a framework for enforcing security standards across the organization.
    \end{itemize}
    \vspace{0.5cm}
    
    \item \textbf{Conduct Regular Security Reviews (Priority: Medium)}
    \begin{itemize}
        \item \textbf{Action:} Establish a schedule for periodic reviews of all security controls, policies, and technical configurations to ensure they remain effective and aligned with best practices.
        \item \textbf{Justification:} The threat landscape is constantly evolving. Regular reviews ensure that security measures adapt to new threats and organizational changes.
    \end{itemize}
\end{enumerate}

\end{document}
```