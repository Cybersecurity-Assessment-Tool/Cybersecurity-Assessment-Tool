```latex
\documentclass[12pt]{article}

% --- PACKAGES ---
\usepackage[a4paper, margin=1in]{geometry}
\usepackage{pifont} % For checkmarks and crosses
\usepackage{booktabs} % For professional tables
\usepackage{hyperref} % For clickable links
\usepackage{url} % For URL formatting
\usepackage{seqsplit} % For splitting long strings in \texttt
\usepackage[T1]{fontenc}

% --- DOCUMENT METADATA ---
\hypersetup{
    colorlinks=true,
    linkcolor=black,
    urlcolor=blue,
    pdftitle={Cybersecurity Posture Assessment Report},
    pdfauthor={Cybersecurity Analyst},
    pdfsubject={Security Assessment},
    pdfkeywords={Cybersecurity, Risk, Assessment, Nmap, RDP}
}

% --- DOCUMENT START ---
\begin{document}

% --- TITLE PAGE ---
\title{
    \vspace{2cm}
    \textbf{Cybersecurity Posture Assessment Report} \\
    \large For: \textbf{[Organization Name]}
    \vspace{1cm}
}
\author{Cybersecurity Analyst}
\date{\today}
\maketitle
\thispagestyle{empty}
\newpage

% --- TABLE OF CONTENTS ---
\tableofcontents
\newpage

% --- EXECUTIVE SUMMARY ---
\section*{Executive Summary}

This report provides a comprehensive assessment of the cybersecurity posture for \textbf{[Organization Name]}. The analysis is based on a correlation of external network scan data, a review of existing security controls via a questionnaire, and an evaluation of previously identified risks.

The assessment has identified a \textbf{critical risk} to the organization. An external network scan confirmed that a server at \texttt{[Target IP]} has Microsoft Remote Desktop Protocol (RDP) (port 3389) exposed directly to the public internet. This configuration is a common target for ransomware gangs and other malicious actors.

This critical technical vulnerability is significantly amplified by several organizational security gaps identified in the security controls review. Key findings include:
\begin{itemize}
    \item \textbf{Lack of Multi-Factor Authentication (MFA) on Email:} This exposes the organization to a high risk of business email compromise and credential theft.
    \item \textbf{Absence of an Acceptable Use Policy (AUP):} This creates ambiguity for employees regarding security responsibilities.
    \item \textbf{Inadequate Security Awareness Training:} The lack of annual refresher training for all employees increases susceptibility to phishing and social engineering attacks.
\end{itemize}

Immediate remediation is required to address the exposed RDP service. Further recommendations are provided to mature the organization's overall security posture and mitigate the identified risks.

% --- ORGANIZATIONAL INFORMATION ---
\section{Organizational Information}

This section details the information provided about the organization.
\begin{itemize}
    \item \textbf{Organization Name:} \textbf{[Organization Name]}
    \item \textbf{Primary Email Domain:} \texttt{[Domain]}
    \item \textbf{External IP Address Analyzed:} \texttt{[Client IP]}
\end{itemize}

% --- SECURITY CONTROL REVIEW ---
\section{Security Control Review}

A security questionnaire was completed to evaluate the status of key administrative and technical controls. The results are summarized below. Answers marked with a red cross (\ding{55}) indicate a control gap that increases organizational risk.

\subsection*{Questionnaire Results}

\begin{table}[h!]
\centering
\caption{Security Controls Questionnaire Analysis}
\begin{tabular}{p{0.8\textwidth}c}
\toprule
\textbf{Control Question} & \textbf{Status} \\
\midrule
Do you require MFA to log into computers? & \ding{51} \\ % Yes
Do you require MFA to access sensitive data systems? & \ding{51} \\ % Yes
Does your organization do security awareness training for new employees? & \ding{51} \\ % Yes
\addlinespace
Do you require MFA to access email? & \ding{55} \\ % No
Does your organization have an employee acceptable use policy? & \ding{55} \\ % No
Does your organization do security awareness training for all employees at least once per year? & \ding{55} \\ % No
\bottomrule
\end{tabular}
\end{table}

\subsection*{Analysis of Control Gaps}
The review identified three significant control gaps:
\begin{itemize}
    \item \textbf{No MFA for Email:} Email is a primary target for attackers. Without MFA, a single compromised password can lead to a full account takeover, data breaches, and further internal compromise.
    \item \textbf{No Acceptable Use Policy (AUP):} An AUP is a foundational policy that defines how employees can use company resources. Its absence can lead to inconsistent security practices and a weakened security culture.
    \item \textbf{No Annual Security Training:} The threat landscape evolves constantly. Failing to provide annual refresher training leaves employees vulnerable to modern phishing and social engineering tactics.
\end{itemize}

% --- TECHNICAL SCAN RESULTS ---
\section{Technical Scan Results}

An external network scan was performed to identify exposed services on the organization's public-facing infrastructure.

\subsection*{Nmap Scan Findings}
\begin{itemize}
    \item \textbf{Target IP Address:} \texttt{[Target IP]}
    \item \textbf{Scan Status:} Host is up.
\end{itemize}

The following open port was discovered:

\begin{table}[h!]
\centering
\caption{Open Ports Detected on \texttt{[Target IP]}}
\begin{tabular}{llll}
\toprule
\textbf{Port} & \textbf{State} & \textbf{Service Name} & \textbf{Associated Risk} \\
\midrule
3389/tcp & open & ms-wbt-server (RDP) & \textbf{Critical} \\
\bottomrule
\end{tabular}
\end{table}

\subsection*{Analysis of Technical Findings}
The scan confirms that port 3389, used for Microsoft Remote Desktop Protocol (RDP), is open to the public internet. RDP is a primary vector for ransomware attacks. Exposing this service directly without mitigating controls like a VPN or IP whitelisting presents a severe and immediate threat to the network. This finding directly validates the pre-existing risk identified as "RDP Exposure."

% --- RISK ASSESSMENT ---
\section{Synthesized Risk Assessment}

This section correlates the findings from the security control review, the technical scan, and pre-existing risk data into a unified risk register.

\begin{table}[h!]
\centering
\caption{Consolidated Risk Register}
\begin{tabular}{lp{0.5\textwidth}cl}
\toprule
\textbf{ID} & \textbf{Risk Description} & \textbf{Severity} & \textbf{Status} \\
\midrule
R-01 & \textbf{Public RDP Exposure:} The RDP service is directly accessible from the internet, inviting brute-force and vulnerability exploitation attacks. & Critical (9.0) & \textbf{Confirmed} \\
\addlinespace
R-02 & \textbf{No MFA on Email:} Lack of MFA on email accounts makes them highly susceptible to credential theft and account takeover via phishing. & Critical & \textbf{Confirmed} \\
\addlinespace
R-03 & \textbf{Inadequate Security Policies:} The absence of a foundational Acceptable Use Policy weakens the overall security posture. & High & \textbf{Confirmed} \\
\addlinespace
R-04 & \textbf{Insufficient Security Training:} Lack of recurring annual training increases the likelihood of employees falling victim to social engineering. & High & \textbf{Confirmed} \\
\bottomrule
\end{tabular}
\end{table}

% --- RECOMMENDATIONS ---
\section{Recommendations and Remediation Plan}

The following actions are recommended to mitigate the identified risks. They are prioritized based on severity and potential impact.

\begin{enumerate}
    \item \textbf{[Immediate] Remediate RDP Exposure:}
    \begin{itemize}
        \item \textbf{Short-Term:} Immediately close port 3389 on the external firewall for IP address \texttt{[Target IP]}. If remote access is required, restrict access to a whitelist of trusted source IP addresses.
        \item \textbf{Long-Term:} Implement a secure remote access solution, such as a Virtual Private Network (VPN) or a Zero Trust Network Access (ZTNA) gateway. This will ensure that all remote access is authenticated and encrypted, without exposing the RDP service directly.
    \end{itemize}
    \vspace{0.5cm}

    \item \textbf{[High Priority] Implement MFA on All Email Accounts:}
    \begin{itemize}
        \item Enforce MFA for all users accessing email, regardless of their location. This is one of the most effective controls to prevent account takeovers.
    \end{itemize}
    \vspace{0.5cm}

    \item \textbf{[Medium Priority] Develop and Implement Foundational Policies:}
    \begin{itemize}
        \item Create and enforce an Acceptable Use Policy (AUP) that clearly outlines the rules for using company IT assets, including data handling and internet usage.
        \item Ensure all employees read and acknowledge the policy.
    \end{itemize}
    \vspace{0.5cm}

    \item \textbf{[Medium Priority] Establish a Recurring Security Training Program:}
    \begin{itemize}
        \item Implement a mandatory security awareness training program for all employees to be completed at least once per year.
        \item Training should cover current threats, such as phishing, ransomware, and proper data handling.
    \end{itemize}
\end{enumerate}

% --- DOCUMENT END ---
\end{document}
```