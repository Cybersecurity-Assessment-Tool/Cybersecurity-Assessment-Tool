```latex
\documentclass[12pt]{article}

% Preamble: Required Packages and Document Setup
\usepackage[margin=1in]{geometry}
\usepackage{pifont} % For checkmarks and crosses (\ding{51} and \ding{55})
\usepackage{booktabs} % For professional-looking tables
\usepackage[hidelinks]{hyperref} % For clickable links, hidelinks removes the colored boxes
\usepackage{url} % For properly formatting URLs
\usepackage{seqsplit} % To split long strings in \texttt
\usepackage{fancyhdr} % For headers and footers
\usepackage{xcolor} % For custom colors

% Custom Commands for Readability
\newcommand{\yes}{\textcolor{green}{\ding{51}}}
\newcommand{\no}{\textcolor{red}{\ding{55}}}
\newcommand{\orgname}{\textbf{[Organization Name]}}
\newcommand{\orgdomain}{\texttt{[Domain]}}
\newcommand{\clientip}{\texttt{[Client IP]}}
\newcommand{\targetip}{\texttt{[Target IP]}}

% Document Metadata
\title{Cybersecurity Posture Assessment Report}
\author{Cybersecurity Analysis Division}
\date{\today}

% Header and Footer Configuration
\pagestyle{fancy}
\fancyhf{}
\lhead{\orgname}
\rhead{Confidential}
\cfoot{\thepage}

\begin{document}

\maketitle
\thispagestyle{empty}
\tableofcontents
\newpage

% =============================================================================
\section{Executive Summary}
% =============================================================================

This report provides a comprehensive analysis of the cybersecurity posture for \orgname, based on a combination of network scanning, a security controls questionnaire, and a review of pre-existing risks.

The assessment has identified several critical and high-severity risks that require immediate attention. The most pressing finding is an externally facing FTP server running a dangerously outdated version of \texttt{vsftpd} (2.3.4), which is known to contain a critical backdoor vulnerability (CVE-2011-2523). This vulnerability is exacerbated by the allowance of anonymous logins.

Furthermore, significant gaps in access control policies were identified. The lack of mandatory Multi-Factor Authentication (MFA) for accessing email and other sensitive data systems presents a critical risk, substantially increasing the likelihood of account compromise and unauthorized data access. The absence of a formal Acceptable Use Policy further weakens the organization's security governance.

This report outlines these findings in detail and provides a prioritized list of actionable recommendations to mitigate the identified risks and strengthen the overall security posture.

% =============================================================================
\section{Organizational Information}
% =============================================================================

The following information was used as the basis for this assessment. Due to the anonymized nature of the input data, placeholders have been used where necessary.

\begin{itemize}
    \item \textbf{Organization Name:} \orgname
    \item \textbf{Primary Email Domain:} \orgdomain
    \item \textbf{External IP Scanned:} \clientip
\end{itemize}

% =============================================================================
\section{Security Control Review}
% =============================================================================

A review of the organization's security controls was conducted via a questionnaire. The responses reveal significant gaps in fundamental security practices, particularly concerning access control. "No" answers indicate a failure to meet baseline security standards and are flagged as high-impact risks.

\begin{table}[h!]
\centering
\caption{Security Controls Questionnaire Analysis}
\begin{tabular}{p{0.6\linewidth} c p{0.25\linewidth}}
\toprule
\textbf{Control Question} & \textbf{Status} & \textbf{Assessment} \\
\midrule
Do you require MFA to access email? & \no & \textbf{Critical Gap}. Email is a primary target for attackers. \\
\addlinespace
Do you require MFA to log into computers? & \yes & Meets best practice. \\
\addlinespace
Do you require MFA to access sensitive data systems? & \no & \textbf{Critical Gap}. Leaves critical assets vulnerable to compromise. \\
\addlinespace
Does your organization have an employee acceptable use policy? & \no & \textbf{High Risk}. Lack of formal policy creates ambiguity and legal risk. \\
\addlinespace
Does your organization do security awareness training for new employees? & \yes & Meets best practice. \\
\addlinespace
Does your organization do security awareness training for all employees at least once per year? & \yes & Meets best practice. \\
\bottomrule
\end{tabular}
\end{table}

% =============================================================================
\section{Technical Scan Results}
% =============================================================================

An external network scan was performed against the organization's public-facing infrastructure. The target IP address was not specified in the scan data and is represented by a placeholder.

\begin{itemize}
    \item \textbf{Target IP:} \targetip
    \item \textbf{Host Status:} Up
\end{itemize}

The scan identified one open port with a critically vulnerable service.

\begin{table}[h!]
\centering
\caption{Open Port Analysis}
\begin{tabular}{l l l l p{0.3\linewidth}}
\toprule
\textbf{Port} & \textbf{State} & \textbf{Service} & \textbf{Version} & \textbf{Notes} \\
\midrule
21/tcp & Open & FTP & vsftpd 2.3.4 & \textbf{CRITICAL FINDING:} This version is vulnerable to a backdoor (CVE-2011-2523). Anonymous FTP login is also enabled, posing a severe risk. \\
\bottomrule
\end{tabular}
\end{table}

% =============================================================================
\section{Consolidated Risk Assessment}
% =============================================================================

The following table synthesizes findings from the technical scan, the control review, and pre-existing risk data into a prioritized list.

\begin{table}[h!]
\centering
\caption{Prioritized Risk Register}
\begin{tabular}{p{0.1\linewidth} p{0.25\linewidth} l p{0.45\linewidth}}
\toprule
\textbf{Risk ID} & \textbf{Risk Title} & \textbf{Severity} & \textbf{Description} \\
\midrule
RISK-001 & Exposed Vulnerable FTP Server & \textbf{Critical} & An externally accessible FTP server is running \texttt{vsftpd 2.3.4}, which has a known backdoor vulnerability. Anonymous login is enabled. \\
\addlinespace
RISK-002 & Lack of MFA on Email and Sensitive Systems & \textbf{Critical} & The absence of MFA on critical systems like email makes them highly susceptible to credential theft and unauthorized access. \\
\addlinespace
RISK-003 & Missing Acceptable Use Policy & \textbf{High} & The lack of a formal policy defining acceptable use of company assets exposes the organization to insider threats and legal liabilities. \\
\addlinespace
RISK-004 & Outdated Windows Policy (Pre-existing) & \textbf{Medium} & Workstations are running Windows 7, which is an unsupported OS lacking modern security features. (CVSS 5.0). \\
\bottomrule
\end{tabular}
\end{table}

% =============================================================================
\section{Recommendations}
% =============================================================================

The following actions are recommended to mitigate the identified risks. They are prioritized based on severity.

\subsection{RISK-001: Exposed Vulnerable FTP Server (Critical)}
\begin{itemize}
    \item \textbf{Immediate (0-24 hours):} Take the FTP server offline immediately to prevent exploitation. If this is not possible, firewall the server to allow access only from trusted IP addresses.
    \item \textbf{Short-Term (1-7 days):} If the FTP service is business-critical, upgrade \texttt{vsftpd} to the latest stable version and disable anonymous access.
    \item \textbf{Long-Term:} Decommission the FTP protocol in favor of a secure alternative like SFTP (SSH File Transfer Protocol) or a managed file transfer solution.
\end{itemize}

\subsection{RISK-002: Lack of MFA on Email and Sensitive Systems (Critical)}
\begin{itemize}
    \item \textbf{Immediate (0-7 days):} Procure and begin deploying an MFA solution for all users.
    \item \textbf{Short-Term (1-4 weeks):} Enforce mandatory MFA for all access to email (e.g., Office 365, G Suite) and any systems identified as containing sensitive data.
\end{itemize}

\subsection{RISK-003: Missing Acceptable Use Policy (High)}
\begin{itemize}
    \item \textbf{Short-Term (1-2 months):} Draft a formal Acceptable Use Policy (AUP) that clearly outlines the rules for using company networks, systems, and data.
    \item \textbf{Medium-Term:} Communicate the policy to all employees and require them to formally acknowledge it. Integrate the AUP into the new employee onboarding process.
\end{itemize}

\subsection{RISK-004: Outdated Windows Policy (Medium)}
\begin{itemize}
    \item \textbf{Ongoing:} Continue with the existing plan to upgrade all Windows 7 workstations to a modern, supported operating system such as Windows 10 or 11. Prioritize machines used by employees with access to sensitive data.
\end{itemize}

\end{document}
```