```latex
\documentclass[12pt]{article}

% --- PACKAGE IMPORTS ---
\usepackage[margin=1in]{geometry} % Set page margins
\usepackage{pifont}               % For checkmark and X symbols (\ding{51}, \ding{55})
\usepackage{booktabs}             % For professional-looking tables
\usepackage{hyperref}             % For clickable links and better PDF navigation
\usepackage{url}                  % For formatting URLs
\usepackage{seqsplit}             % For splitting long strings without breaking
\usepackage{xcolor}               % For custom colors

% --- DOCUMENT METADATA ---
\title{Cybersecurity Posture Assessment Report}
\author{Cybersecurity Analysis Division}
\date{\today}

% --- HYPERREF SETUP ---
\hypersetup{
    colorlinks=true,
    linkcolor=blue,
    filecolor=magenta,      
    urlcolor=cyan,
    pdftitle={Cybersecurity Posture Assessment Report},
    pdfpagemode=FullScreen,
}

% --- DOCUMENT START ---
\begin{document}

\maketitle

\begin{abstract}
    This report provides a comprehensive analysis of the cybersecurity posture for \textbf{[Organization Name]}. The assessment is based on a synthesis of external network scan data, an internal security controls questionnaire, and a review of pre-existing risks. The analysis reveals a critical external vulnerability and significant internal security control gaps that require immediate attention. Key findings include an exposed, outdated, and misconfigured FTP server, a lack of multi-factor authentication (MFA) for critical systems, and deficiencies in foundational security policies and training.
\end{abstract}

\tableofcontents
\newpage

% ===================================================================
\section{Overview and Executive Summary}
% ===================================================================

This assessment was conducted to evaluate the current security state of \textbf{[Organization Name]}. The findings indicate a high-risk environment characterized by a combination of technical vulnerabilities, procedural gaps, and known legacy issues.

\paragraph{Key Findings:}
\begin{itemize}
    \item \textbf{Critical External Exposure:} An externally facing FTP server was identified running a dangerously outdated version of \texttt{vsftpd} (2.3.4), which is known to be vulnerable to remote code execution. Furthermore, the server is misconfigured to allow anonymous logins, presenting a clear and immediate threat to the network.
    \item \textbf{Identity and Access Management Gaps:} The organization lacks mandatory multi-factor authentication (MFA) for computer logins and access to sensitive data systems. This significantly increases the risk of unauthorized access from compromised credentials.
    \item \textbf{Policy and Training Deficiencies:} The absence of a formal Acceptable Use Policy and a mandatory annual security awareness training program for all employees creates a high-risk environment susceptible to human error and insider threats.
    \item \textbf{Pre-existing Endpoint Risk:} The organization is aware of an existing risk related to outdated Windows 7 workstations, which compounds the risk posed by weak access controls.
\end{itemize}

Immediate remediation of the external FTP server is paramount. Following this, a strategic initiative to implement MFA and develop foundational security policies is strongly recommended to reduce the overall attack surface.

% ===================================================================
\section{Organizational Information}
% ===================================================================

The following information was used as the basis for this assessment. Placeholders are used where data was not provided.

\begin{itemize}
    \item \textbf{Organization Name:} \textbf{[Organization Name]}
    \item \textbf{Primary Domain:} \texttt{[Domain]}
    \item \textbf{Scanned External IP:} \texttt{[Client IP]}
\end{itemize}

% ===================================================================
\section{Security Control Review}
% ===================================================================

The following table summarizes the organization's responses to a security controls questionnaire. "No" answers indicate significant gaps in the security framework and are correlated with identified risks.

\begin{table}[h!]
\centering
\caption{Security Controls Questionnaire Analysis}
\label{tab:controls}
\begin{tabular}{@{}p{0.6\linewidth} c l@{}}
\toprule
\textbf{Control Question} & \textbf{Response} & \textbf{Assessment} \\
\midrule
Do you require MFA to access email? & \ding{51} & Good Practice \\
Do you require MFA to log into computers? & \ding{55} & \textbf{High Risk Gap} \\
Do you require MFA to access sensitive data systems? & \ding{55} & \textbf{Critical Risk Gap} \\
Does your organization have an employee acceptable use policy? & \ding{55} & \textbf{High Risk Gap} \\
Does your organization do security awareness training for new employees? & \ding{51} & Good Practice \\
Does your organization do security awareness training for all employees at least once per year? & \ding{55} & \textbf{High Risk Gap} \\
\bottomrule
\end{tabular}
\end{table}

% ===================================================================
\section{Technical Scan Results}
% ===================================================================

An external network scan was performed on the target IP address. The results indicate a critical vulnerability that requires immediate attention.

\begin{description}
    \item[Target IP:] \texttt{[Target IP]}
    \item[Status:] Host is up and responsive.
    \item[Finding 1: Exposed and Vulnerable FTP Server]
        \begin{itemize}
            \item \textbf{Port:} 21/tcp (Open)
            \item \textbf{Service:} FTP (File Transfer Protocol)
            \item \textbf{Product:} vsftpd
            \item \textbf{Version:} \textbf{2.3.4}
            \item \textbf{Details:} This version of \texttt{vsftpd} is extremely old (released in 2011) and contains a well-known, critical backdoor vulnerability (CVE-2011-2523). An attacker can gain a command shell on the server by sending a specific sequence of characters as a username.
            \item \textbf{Configuration Issue:} The scan confirmed that \textbf{Anonymous FTP login is allowed}. This permits any unauthenticated user on the internet to connect to the server, browse files, and potentially upload or download data. This configuration drastically increases the server's exposure and risk.
        \end{itemize}
\end{description}

% ===================================================================
\section{Consolidated Risk Assessment}
% ===================================================================

The following table synthesizes findings from the technical scan, control review, and pre-existing risk data into a prioritized list.

\begin{table}[h!]
\centering
\caption{Summary of Identified Risks}
\label{tab:risks}
\begin{tabular}{@{}p{0.5\linewidth} l l@{}}
\toprule
\textbf{Risk Description} & \textbf{Severity} & \textbf{Source} \\
\midrule
Vulnerable FTP Server (\texttt{vsftpd 2.3.4}) with Anonymous Login & \textbf{Critical} & Technical Scan \\
Lack of MFA for Sensitive Data Systems & \textbf{Critical} & Questionnaire \\
Lack of MFA for Computer Logins & High & Questionnaire \\
Missing Employee Acceptable Use Policy & High & Questionnaire \\
Missing Annual Security Awareness Training & High & Questionnaire \\
Outdated Windows 7 Operating Systems & Medium & Pre-existing Risk \\
\bottomrule
\end{tabular}
\end{table}

% ===================================================================
\section{Recommendations}
% ===================================================================

The following actions are recommended to mitigate the identified risks and improve the overall security posture of \textbf{[Organization Name]}.

\subsection{Immediate Actions (Remediate within 72 hours)}
\begin{enumerate}
    \item \textbf{Remediate Vulnerable FTP Server:}
        \begin{itemize}
            \item If the FTP service is not essential, \textbf{disable it immediately} and block port 21 at the firewall.
            \item If the service is essential, upgrade the \texttt{vsftpd} software to the latest stable version and \textbf{disable anonymous login} immediately. Access should be restricted to authenticated users only, and ideally, connections should be limited to trusted IP addresses. Consider using a more secure file transfer protocol like SFTP (SSH File Transfer Protocol).
        \end{itemize}
    \item \textbf{Implement MFA on Sensitive Systems:} Prioritize the deployment of MFA on all systems identified as containing sensitive data. This is a critical compensating control for potential credential compromise.
\end{enumerate}

\subsection{High-Priority Actions (Remediate within 30-60 days)}
\begin{enumerate}
    \item \textbf{Enforce MFA for All Logins:} Develop and execute a plan to roll out MFA for all employee computer logins.
    \item \textbf{Develop and Implement an Acceptable Use Policy (AUP):} Create a formal AUP that defines the rules for using company IT assets. Ensure all employees read and acknowledge the policy.
    \item \textbf{Establish Annual Security Training:} Institute a mandatory security awareness training program for all employees to be completed annually. This program should cover topics like phishing, password security, and data handling.
\end{enumerate}

\subsection{Ongoing Improvements}
\begin{enumerate}
    \item \textbf{Operating System Upgrades:} Continue with the project to upgrade or replace all workstations running Windows 7, as this operating system is no longer supported and does not receive security updates.
    \item \textbf{Regular Vulnerability Scanning:} Implement a regular, automated vulnerability scanning program for all external and internal assets to proactively identify and remediate new security issues.
\end{enumerate}

\end{document}
```