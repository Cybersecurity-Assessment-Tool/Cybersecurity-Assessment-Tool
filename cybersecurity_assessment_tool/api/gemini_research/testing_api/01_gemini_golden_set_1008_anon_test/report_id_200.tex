```latex
\documentclass[12pt, a4paper]{article}

% Preamble: Required Packages
\usepackage[margin=1in]{geometry}
\usepackage{pifont} % For checkmarks and crosses
\usepackage{booktabs} % For professional tables
\usepackage{hyperref} % For clickable links
\usepackage{url} % For URL formatting
\usepackage{seqsplit} % For splitting long strings without breaking
\usepackage{graphicx}
\usepackage{xcolor}

% Document Information
\title{Cybersecurity Posture Assessment Report}
\author{Cybersecurity Analyst}
\date{\today}

% Hyperref Setup
\hypersetup{
    colorlinks=true,
    linkcolor=blue,
    filecolor=magenta,      
    urlcolor=cyan,
    pdftitle={Cybersecurity Posture Assessment Report},
    pdfpagemode=FullScreen,
}

\begin{document}

\maketitle
\thispagestyle{empty}
\newpage

\tableofcontents
\newpage

% --- 1. Executive Summary ---
\section{Executive Summary}

This report provides a comprehensive cybersecurity assessment for \textbf{[Organization Name]}, based on an analysis of network scan data, organizational security controls, and known risks. The assessment was conducted on \today.

The analysis identified several critical and high-risk security gaps that require immediate attention. Key findings include:
\begin{itemize}
    \item \textbf{Critical Control Gap:} Multi-Factor Authentication (MFA) is not enforced for accessing sensitive data systems, leaving critical assets vulnerable to unauthorized access via compromised credentials.
    \item \textbf{High-Risk Programmatic Gap:} The organization lacks a formal security awareness training program for both new and existing employees. This significantly increases susceptibility to phishing, social engineering, and other human-centric attacks.
    \item \textbf{High-Risk Technical Finding:} An external-facing web server is operating over unencrypted HTTP (Port 80). This exposes all transmitted data, including potential credentials or sensitive information, to interception.
\end{itemize}

These findings, when correlated, indicate a security posture with significant exposure to common cyber threats. This report outlines these risks in detail and provides actionable recommendations to mitigate them and improve the overall security posture of the organization.

% --- 2. Organizational Information ---
\section{Organizational Information}

The following details were used as the basis for this assessment. Due to the anonymized nature of the provided data, placeholders have been used where necessary.

\begin{tabular}{@{}ll}
    \toprule
    \textbf{Attribute} & \textbf{Value} \\
    \midrule
    Organization Name & \textbf{[Organization Name]} \\
    Primary Email Domain & \texttt{[Domain]} \\
    External IP Address Scanned & \texttt{[Client IP]} \\
    \bottomrule
\end{tabular}

% --- 3. Security Control Review ---
\section{Security Control Review}

A review of the organization's self-reported security controls was conducted via a questionnaire. The results highlight significant gaps in foundational security practices. A green checkmark (\ding{51}) indicates a positive control is in place, while a red cross (\ding{55}) indicates a control gap.

\begin{table}[h!]
\centering
\caption{Organizational Security Controls Questionnaire Results}
\begin{tabular}{@{}p{0.8\linewidth}c@{}}
    \toprule
    \textbf{Control Question} & \textbf{Status} \\
    \midrule
    Do you require MFA to access email? & \textcolor{green}{\ding{51}} \\
    Do you require MFA to log into computers? & \textcolor{green}{\ding{51}} \\
    Do you require MFA to access sensitive data systems? & \textcolor{red}{\ding{55}} \\
    Does your organization have an employee acceptable use policy? & \textcolor{green}{\ding{51}} \\
    Does your organization do security awareness training for new employees? & \textcolor{red}{\ding{55}} \\
    Does your organization do security awareness training for all employees at least once per year? & \textcolor{red}{\ding{55}} \\
    \bottomrule
\end{tabular}
\end{table}

The primary areas of concern identified from this review are the lack of MFA for sensitive systems and the complete absence of a security awareness training program.

% --- 4. Technical Scan Results ---
\section{Technical Scan Results}

An external network scan was performed against the target IP address to identify open ports and exposed services.

\begin{itemize}
    \item \textbf{Target IP Address:} \texttt{[Target IP]}
    \item \textbf{Scan Date:} \today
    \item \textbf{Scan Tool:} Nmap
\end{itemize}

The scan revealed the following open port:

\begin{table}[h!]
\centering
\caption{Open Ports Detected on \texttt{[Target IP]}}
\begin{tabular}{@{}llll@{}}
    \toprule
    \textbf{Port} & \textbf{Protocol} & \textbf{State} & \textbf{Service/Inference} \\
    \midrule
    80 & TCP & open & HTTP (Unencrypted Web Traffic) \\
    \bottomrule
\end{tabular}
\end{table}

\subsection{Analysis of Technical Findings}
The presence of an open Port 80 indicates that a web server is hosting content using the Hypertext Transfer Protocol (HTTP). HTTP is an unencrypted protocol, meaning that any data exchanged between a user's browser and the server is sent in cleartext. This poses a significant security risk, as a malicious actor in a position to intercept traffic (e.g., on a public Wi-Fi network) could easily capture login credentials, session cookies, or other sensitive information. Modern security standards mandate the use of HTTPS (HTTP over TLS/SSL on Port 443) to encrypt this communication.

% --- 5. Risk Assessment Summary ---
\section{Risk Assessment Summary}

This section synthesizes the findings from the security control review and technical scan into a prioritized list of identified risks.
\begin{table}[h!]
\centering
\caption{Summary of Identified Risks}
\begin{tabular}{@{}p{0.1\linewidth}p{0.25\linewidth}p{0.15\linewidth}p{0.4\linewidth}@{}}
    \toprule
    \textbf{ID} & \textbf{Finding} & \textbf{Risk Level} & \textbf{Description} \\
    \midrule
    R-01 & Incomplete MFA Implementation & \textbf{Critical} & Sensitive data systems are not protected by MFA. A single compromised password could lead to a major data breach. \\
    \addlinespace
    R-02 & Lack of Security Awareness Program & \textbf{High} & Employees are not trained to recognize or respond to security threats like phishing, making them easy targets for initial access attacks. \\
    \addlinespace
    R-03 & Unencrypted Web Traffic (HTTP) & \textbf{High} & The use of HTTP for a web service allows for the interception of sensitive data, such as user credentials, in transit. \\
    \bottomrule
\end{tabular}
\end{table}

\textit{Note: The provided list of pre-existing risks contained a non-actionable, malicious entry and was therefore excluded from this assessment.}

% --- 6. Recommendations ---
\section{Recommendations}

The following actions are recommended to mitigate the identified risks and strengthen the organization's security posture.

\subsection{R-01: Remediate Incomplete MFA Implementation (Critical)}
\begin{itemize}
    \item \textbf{Immediate Action:} Enforce mandatory Multi-Factor Authentication (MFA) for all user accounts, especially privileged accounts, that have access to systems storing or processing sensitive data.
    \item \textbf{Long-Term Strategy:} Develop a policy that mandates MFA for all new systems and services by default.
\end{itemize}

\subsection{R-02: Establish a Security Awareness Program (High)}
\begin{itemize}
    \item \textbf{Immediate Action:} Implement a foundational security awareness training module for all new employees as part of their onboarding process.
    \item \textbf{Long-Term Strategy:} Procure and deploy a comprehensive security awareness training platform. Conduct annual training for all staff and supplement it with regular phishing simulations to measure and improve employee resilience to social engineering attacks.
\end{itemize}

\subsection{R-03: Remediate Unencrypted Web Traffic (High)}
\begin{itemize}
    \item \textbf{Immediate Action:} Acquire and install a valid TLS/SSL certificate on the web server at \texttt{[Target IP]}.
    \item \textbf{Long-Term Strategy:} Reconfigure the web server to redirect all HTTP traffic (Port 80) to HTTPS (Port 443). Implement HTTP Strict Transport Security (HSTS) to ensure that browsers only connect to the server over a secure connection.
\end{itemize}

\end{document}
```