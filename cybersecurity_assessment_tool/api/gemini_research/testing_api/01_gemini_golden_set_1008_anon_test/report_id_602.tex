```latex
\documentclass[12pt]{article}

% ----------------------------------------------------------------------
% PREAMBLE
% ----------------------------------------------------------------------
\usepackage[margin=1in]{geometry} % Set page margins
\usepackage{pifont}                 % For checkmarks and crosses (\ding)
\usepackage{booktabs}               % For professional-looking tables
\usepackage[hidelinks]{hyperref}    % For clickable links without boxes
\usepackage{url}                    % For typesetting URLs
\usepackage{seqsplit}               % For splitting long strings in texttt

% Custom commands for consistency
\newcommand{\cmark}{\ding{51}} % Checkmark
\newcommand{\xmark}{\ding{55}} % Cross

% ----------------------------------------------------------------------
% DOCUMENT METADATA
% ----------------------------------------------------------------------
\title{Cybersecurity Assessment Report}
\author{Cybersecurity Analysis Division}
\date{\today}

% ----------------------------------------------------------------------
% BEGIN DOCUMENT
% ----------------------------------------------------------------------
\begin{document}

\maketitle

\section*{Executive Summary}

This report details the findings of a cybersecurity assessment for \textbf{[Organization Name]}, conducted on \today. The assessment combined an external network scan, a review of internal security controls via a questionnaire, and an analysis of pre-existing risks.

The analysis revealed several high-priority risks requiring immediate attention. A critical vulnerability was identified on an external-facing system: an outdated and misconfigured FTP server (\texttt{vsftpd 2.3.4}) that allows anonymous access. This version is known to contain a critical backdoor vulnerability (CVE-2011-2523).

Furthermore, significant gaps were identified in the organization's access control policies. The lack of Multi-Factor Authentication (MFA) for computer and sensitive data system logins constitutes a high risk, substantially increasing the likelihood of a successful breach should user credentials be compromised. These findings, coupled with the known risk of outdated Windows 7 workstations, create a high-risk environment that could be exploited by threat actors.

This report provides a detailed breakdown of these findings and offers actionable recommendations to mitigate the identified risks and improve the overall security posture of the organization.

\section*{Organizational Information}

The following details were used as the basis for this assessment. As per the provided data, placeholder values are used where specific information was not available.

\begin{description}
    \item[Organization Name:] \textbf{[Organization Name]}
    \item[Primary Domain:] \texttt{[Domain]}
    \item[External IP Scanned:] \texttt{[Client IP]}
\end{description}

\section*{Security Control Review}

The following table summarizes the organization's current security controls based on the provided questionnaire. Items marked with \xmark\ represent significant gaps in the security framework and are discussed in the Risk Assessment section.

\begin{center}
\begin{tabular}{p{0.7\linewidth} c}
\toprule
\textbf{Control Question} & \textbf{Status} \\
\midrule
Do you require MFA to access email? & \cmark \\
Do you require MFA to log into computers? & \xmark \\
Do you require MFA to access sensitive data systems? & \xmark \\
Does your organization have an employee acceptable use policy? & \cmark \\
Does your organization do security awareness training for new employees? & \cmark \\
Does your organization do security awareness training for all employees at least once per year? & \cmark \\
\bottomrule
\end{tabular}
\end{center}

\section*{Technical Scan Results}

An external network scan was performed on the target IP address \texttt{[Target IP]}. The scan identified the following open ports and services.

\begin{description}
    \item[Port 21/TCP (FTP):] Open
    \begin{itemize}
        \item \textbf{Service:} \seqsplit{\texttt{vsftpd 2.3.4}}
        \item \textbf{Finding 1 (Critical):} The identified version of vsftpd (2.3.4) is dangerously outdated (circa 2011) and contains a well-known critical backdoor vulnerability (\textbf{CVE-2011-2523}). An attacker can exploit this vulnerability to gain a command shell on the underlying server.
        \item \textbf{Finding 2 (Critical):} The FTP service is configured to allow anonymous login (\texttt{Anonymous FTP login allowed}). This permits unauthenticated users to access, download, or potentially upload files, leading to data breaches or the introduction of malware. FTP also transmits credentials and data in cleartext, making it susceptible to eavesdropping.
    \end{itemize}
\end{description}

\section*{Risk Assessment and Findings}

The following table synthesizes findings from the technical scan, control review, and pre-existing risk data. Risks are prioritized based on their potential impact and exploitability.

\begin{center}
\begin{tabular}{p{0.1\linewidth} p{0.4\linewidth} p{0.15\linewidth} p{0.25\linewidth}}
\toprule
\textbf{ID} & \textbf{Risk Description} & \textbf{Severity} & \textbf{Affected Systems} \\
\midrule
\textbf{R-01} & A publicly accessible FTP server is running a vulnerable version (\texttt{vsftpd 2.3.4}) with a known backdoor and allows anonymous login. & \textbf{Critical} & External Server at \texttt{[Target IP]} \\
\addlinespace
\textbf{R-02} & Multi-Factor Authentication (MFA) is not enforced for workstation logins or access to sensitive data systems, severely weakening access controls. & \textbf{High} & All Workstations, Sensitive Data Systems \\
\addlinespace
\textbf{R-03} & Workstations are running the unsupported Windows 7 operating system, which no longer receives security updates, making them highly susceptible to known exploits. & \textbf{Medium} & Employee Workstations \\
\bottomrule
\end{tabular}
\end{center}

\section*{Recommendations}

The following actions are recommended to mitigate the identified risks.

\subsection*{R-01: Remediate Vulnerable FTP Server (Immediate Priority)}
\begin{enumerate}
    \item \textbf{Immediate Action:} Take the FTP service offline immediately to prevent exploitation. If this is not possible, firewall the service to allow access only from trusted IP addresses.
    \item \textbf{Short-Term:} If FTP is a business requirement, upgrade the server to the latest stable version and disable anonymous access.
    \item \textbf{Long-Term:} Decommission the FTP service entirely. Replace it with a secure file transfer protocol such as SFTP (SSH File Transfer Protocol) or a managed file transfer solution that enforces encryption and strong authentication.
\end{enumerate}

\subsection*{R-02: Implement Comprehensive MFA (High Priority)}
\begin{enumerate}
    \item \textbf{Immediate Action:} Develop a project plan for a phased rollout of MFA for all computer and sensitive system logins.
    \item \textbf{Short-Term:} Begin a pilot program with IT staff and key business units to test and deploy MFA on workstations and critical applications.
    \item \textbf{Long-Term:} Enforce MFA for all employees and contractors as a mandatory security control for accessing any corporate resource.
\end{enumerate}

\subsection*{R-03: Mitigate Outdated Operating Systems (Medium Priority)}
\begin{enumerate}
    \item \textbf{Immediate Action:} Isolate Windows 7 machines from sensitive network segments to limit their potential as a pivot point for attackers.
    \item \textbf{Short-Term:} Accelerate the existing plan to upgrade all remaining Windows 7 workstations to a modern, supported operating system (e.g., Windows 10/11).
    \item \textbf{Long-Term:} Implement a formal asset and patch management policy to ensure all systems are kept up-to-date and retired before they reach end-of-life.
\end{enumerate}

\end{document}
```