```latex
\documentclass[12pt]{article}

% Preamble: Required Packages
\usepackage[margin=1in]{geometry}
\usepackage{pifont} % For checkmarks and crosses
\usepackage{booktabs} % For professional tables
\usepackage{hyperref} % For clickable links
\usepackage{url}      % For URL formatting
\usepackage{seqsplit} % For splitting long strings
\usepackage{xcolor}   % For colors in tables

% Document Metadata
\title{Cybersecurity Posture Assessment Report}
\author{Cybersecurity Analysis Division}
\date{\today}

\hypersetup{
    colorlinks=true,
    linkcolor=blue,
    filecolor=magenta,      
    urlcolor=cyan,
    pdftitle={Cybersecurity Posture Assessment Report},
    pdfpagemode=FullScreen,
}

\begin{document}

\maketitle
\thispagestyle{empty}
\newpage

\tableofcontents
\newpage

% --- Section 1: Executive Summary ---
\section{Executive Summary}

This report provides a comprehensive analysis of the cybersecurity posture for \textbf{[Organization Name]}. The assessment is based on a synthesis of network scan data, a security controls questionnaire, and a review of pre-existing risks.

The analysis revealed several critical and high-risk gaps that require immediate attention. The most critical finding is the lack of Multi-Factor Authentication (MFA) for email access, which exposes the organization to significant risks of account compromise and subsequent data breaches.

Furthermore, significant procedural gaps were identified, including the absence of an employee Acceptable Use Policy and a lack of recurring, annual security awareness training for all staff. These deficiencies create an environment where human error is more likely to lead to a security incident.

A technical network scan identified an exposed Secure Shell (SSH) service on port 22. While necessary for remote administration, its exposure to the public internet without proper controls presents a tangible threat vector.

Overall, the organization's current security posture requires significant improvement. The recommendations provided in this report are designed to address these identified weaknesses in a prioritized manner, starting with the most critical vulnerabilities.

% --- Section 2: Organizational Information ---
\section{Organizational Information}

This section outlines the basic information for the organization under review. As some data was not provided, placeholders have been used.

\begin{itemize}
    \item \textbf{Organization Name:} \textbf{[Organization Name]}
    \item \textbf{Primary Email Domain:} \texttt{[Domain]}
    \item \textbf{External IP Address Scanned:} \texttt{[Client IP]}
\end{itemize}

% --- Section 3: Security Control Review ---
\section{Security Control Review}

The following table summarizes the organization's responses to a security controls questionnaire. A green checkmark (\ding{51}) indicates a positive control is in place, while a red cross (\ding{55}) indicates a control gap. Gaps are considered significant risks.

\begin{table}[h!]
\centering
\caption{Security Controls Questionnaire Results}
\begin{tabular}{p{0.7\linewidth} c}
\toprule
\textbf{Control Question} & \textbf{Response} \\
\midrule
Do you require MFA to access email? & \textcolor{red}{\ding{55}} \\
Do you require MFA to log into computers? & \textcolor{green}{\ding{51}} \\
Do you require MFA to access sensitive data systems? & \textcolor{green}{\ding{51}} \\
Does your organization have an employee acceptable use policy? & \textcolor{red}{\ding{55}} \\
Does your organization do security awareness training for new employees? & \textcolor{green}{\ding{51}} \\
Does your organization do security awareness training for all employees at least once per year? & \textcolor{red}{\ding{55}} \\
\bottomrule
\end{tabular}
\end{table}

\subsection*{Analysis of Control Gaps}
\begin{itemize}
    \item \textbf{MFA for Email (Critical Gap):} The absence of MFA on email is a critical vulnerability. Email accounts are primary targets for phishing and credential theft, often serving as the gateway to an organization's entire digital footprint.
    \item \textbf{Acceptable Use Policy (High Risk):} Without a formal AUP, there are no clear guidelines for employees regarding the safe and appropriate use of company assets, leading to inconsistent security practices.
    \item \textbf{Annual Security Training (High Risk):} Security threats evolve constantly. Failing to provide annual refresher training means that employees' security awareness will degrade over time, making them more susceptible to social engineering and other attacks.
\end{itemize}

% --- Section 4: Technical Scan Results ---
\section{Technical Scan Results}

A network scan was performed against the organization's external infrastructure to identify open ports and exposed services.

\begin{itemize}
    \item \textbf{Target IP Address:} \texttt{[Target IP]}
    \item \textbf{Scan Date:} Data not available in scan results.
\end{itemize}

\begin{table}[h!]
\centering
\caption{Open Ports Detected via Nmap Scan}
\begin{tabular}{c c c c c}
\toprule
\textbf{Port} & \textbf{State} & \textbf{Service} & \textbf{Product} & \textbf{Version} \\
\midrule
22 & open & ssh & \textit{Not Detected} & \textit{Not Detected} \\
\bottomrule
\end{tabular}
\end{table}

\subsection*{Analysis of Technical Findings}
The scan identified that port 22, the standard port for the Secure Shell (SSH) protocol, is open to the public internet. SSH is a common tool for remote system administration. However, a publicly exposed SSH service is a frequent target for brute-force password attacks. Without information on its configuration (e.g., password policy, use of key-based authentication), this represents a medium-to-high risk.

% --- Section 5: Consolidated Risk Assessment ---
\section{Consolidated Risk Assessment}

This section consolidates all findings from the security control review, technical scan, and pre-existing risk data into a single, prioritized list. No pre-existing vulnerabilities were reported.

\begin{table}[h!]
\centering
\caption{Summary of Identified Risks}
\begin{tabular}{p{0.1\linewidth} p{0.3\linewidth} p{0.15\linewidth} p{0.35\linewidth}}
\toprule
\textbf{ID} & \textbf{Finding / Vulnerability} & \textbf{Severity} & \textbf{Description} \\
\midrule
R-01 & MFA Not Enforced on Email & \textbf{Critical} & Lack of MFA on email accounts allows for trivial account takeovers via credential theft, leading to data breaches and phishing. \\
\addlinespace
R-02 & Lack of Annual Security Awareness Training & High & Without recurring training, employee knowledge of current threats becomes outdated, increasing susceptibility to social engineering. \\
\addlinespace
R-03 & No Employee Acceptable Use Policy (AUP) & High & The absence of a formal AUP creates ambiguity and a lack of accountability for employee actions on corporate systems. \\
\addlinespace
R-04 & Exposed SSH Service (Port 22) & Medium & The SSH management port is open to the internet, making it a target for automated brute-force attacks if not properly secured. \\
\bottomrule
\end{tabular}
\end{table}

% --- Section 6: Recommendations ---
\section{Recommendations}

The following actionable recommendations are provided to mitigate the risks identified in this report. They are prioritized based on severity.

\subsection*{R-01: Implement MFA for Email (Critical)}
\begin{itemize}
    \item \textbf{Action:} Immediately procure and enforce an MFA solution for all email accounts (e.g., Microsoft 365, Google Workspace).
    \item \textbf{Justification:} This is the single most effective control to prevent unauthorized access to email, which is the primary vector for more advanced attacks.
\end{itemize}

\subsection*{R-02: Establish Annual Security Training (High)}
\begin{itemize}
    \item \textbf{Action:} Develop or subscribe to a security awareness training program and make it mandatory for all employees to complete it on an annual basis.
    \item \textbf{Justification:} Regular training ensures that staff remain vigilant and are equipped to recognize and report modern cyber threats like phishing and ransomware.
\end{itemize}

\subsection*{R-03: Develop an Acceptable Use Policy (High)}
\begin{itemize}
    \item \textbf{Action:} Draft, approve, and disseminate a formal Acceptable Use Policy (AUP). Require all employees to read and acknowledge the policy.
    \item \textbf{Justification:} An AUP establishes clear rules of behavior for the use of IT assets, reduces organizational risk, and provides a basis for disciplinary action in case of violations.
\end{itemize}

\subsection*{R-04: Secure the Exposed SSH Service (Medium)}
\begin{itemize}
    \item \textbf{Action:}
        \begin{enumerate}
            \item If SSH access is not required from the public internet, block port 22 at the firewall.
            \item If external access is required, restrict access to a whitelist of trusted source IP addresses.
            \item Enforce strong, key-based authentication and disable direct root login and password-based authentication.
        \end{enumerate}
    \item \textbf{Justification:} These steps significantly reduce the attack surface of the SSH service, protecting a critical administrative entry point from automated and targeted attacks.
\end{itemize}

\end{document}
```