```latex
\documentclass[12pt]{article}

% ----------------------------------------------------------------------
% PREAMBLE
% ----------------------------------------------------------------------
\usepackage[margin=1in]{geometry}
\usepackage{pifont} % For \ding{51} (checkmark) and \ding{55} (x-mark)
\usepackage{booktabs} % For professional-looking tables (\toprule, \midrule, \bottomrule)
\usepackage{graphicx}
\usepackage[table]{xcolor}
\usepackage{hyperref} % For clickable links, references, etc.
\usepackage{url}      % For properly formatting URLs
\usepackage{seqsplit} % For splitting long monospaced text strings

% --- Document and Hyperlink Setup ---
\hypersetup{
    colorlinks=true,
    linkcolor=blue,
    filecolor=magenta,      
    urlcolor=cyan,
    pdftitle={Cybersecurity Assessment Report},
    pdfauthor={Cybersecurity Analysis Division},
    pdfsubject={Security Assessment},
    pdfkeywords={Security, Report, Analysis},
    hidelinks % Hides the boxes around links but keeps them clickable
}

% --- Custom Colors & Commands ---
\definecolor{sev_critical}{HTML}{990000}
\definecolor{sev_high}{HTML}{D14302}
\definecolor{sev_medium}{HTML}{F5A623}
\definecolor{sev_low}{HTML}{4A90E2}
\definecolor{tablehead}{gray}{0.9}

% --- Document Start ---
\begin{document}

% ----------------------------------------------------------------------
% TITLE PAGE
% ----------------------------------------------------------------------
\title{Cybersecurity Assessment Report \\ \large For \textbf{[Organization Name]}}
\author{Cybersecurity Analysis Division}
\date{\today}
\maketitle

% ----------------------------------------------------------------------
% ABSTRACT
% ----------------------------------------------------------------------
\begin{abstract}
This report details the findings of a cybersecurity assessment conducted for \textbf{[Organization Name]}. The analysis is based on network scan data, a security controls questionnaire, and a review of pre-existing risks. The assessment identified several critical and high-risk vulnerabilities, primarily related to the lack of fundamental security controls such as Multi-Factor Authentication (MFA) and formalized employee policies. An externally accessible SSH port and a pre-existing critical vulnerability compound these risks. Immediate remediation is strongly advised to mitigate significant threats to the organization's data and infrastructure.
\end{abstract}

\tableofcontents
\newpage

% ----------------------------------------------------------------------
% SECTION 1: EXECUTIVE SUMMARY
% ----------------------------------------------------------------------
\section{Executive Summary}
The overall security posture of \textbf{[Organization Name]} is rated as \textbf{Poor}. This assessment is based on the identification of multiple critical and high-severity risks that lack basic mitigating controls.

Key findings include:
\begin{itemize}
    \item \textbf{Critical Control Gaps:} The organization does not enforce Multi-Factor Authentication (MFA) for email or computer logins. This represents a critical vulnerability, as a single compromised password could lead to a widespread breach.
    \item \textbf{Policy Deficiencies:} The absence of an employee acceptable use policy and a mandatory annual security awareness training program indicates a low level of security maturity. This increases the susceptibility to social engineering and insider threats.
    \item \textbf{Exposed Services:} A network scan revealed an open SSH port (22) on the external network. Combined with the lack of MFA, this service presents a high-risk entry point for attackers.
    \item \textbf{Pre-existing Critical Risk:} A known vulnerability, "Localhost Exposed," with a CVSS score of 10.0, requires immediate attention and remediation.
\end{itemize}

This report provides a detailed analysis of these findings and outlines a prioritized list of actionable recommendations to improve the organization's defensive capabilities.

% ----------------------------------------------------------------------
% SECTION 2: ORGANIZATIONAL INFORMATION
% ----------------------------------------------------------------------
\section{Organizational Information}
This section contains the high-level information used as the basis for this assessment. Due to the anonymized nature of the provided data, placeholders have been used.

\begin{table}[h!]
\centering
\begin{tabular}{@{}ll@{}}
\toprule
\textbf{Attribute} & \textbf{Value} \\ \midrule
Organization Name  & \textbf{[Organization Name]} \\
Email Domain       & \texttt{[Domain]} \\
External IP Address & \texttt{[Client IP]} \\ \bottomrule
\end{tabular}
\caption{Client Organizational Data.}
\end{table}

% ----------------------------------------------------------------------
% SECTION 3: SECURITY CONTROL REVIEW
% ----------------------------------------------------------------------
\section{Security Control Review}
The following table summarizes the organization's responses to a security controls questionnaire. Each "No" response (\ding{55}) indicates a significant gap in the security framework and has been flagged as a risk.

\begin{table}[h!]
\centering
\rowcolors{2}{gray!10}{white}
\begin{tabular}{@{}p{0.5\linewidth}ccc@{}}
\toprule
\rowcolor{tablehead}
\textbf{Security Control Question} & \textbf{Response} & \textbf{Assessment} \\ \midrule
Do you require MFA to access email? & \ding{55} No & \textcolor{sev_critical}{\textbf{Critical Gap}} \\
Do you require MFA to log into computers? & \ding{55} No & \textcolor{sev_critical}{\textbf{Critical Gap}} \\
Do you require MFA to access sensitive data systems? & \ding{51} Yes & Meets Best Practice \\
Does your organization have an employee acceptable use policy? & \ding{55} No & \textcolor{sev_high}{\textbf{High Risk}} \\
Does your organization do security awareness training for new employees? & \ding{51} Yes & Meets Best Practice \\
Does your organization do security awareness training for all employees at least once per year? & \ding{55} No & \textcolor{sev_high}{\textbf{High Risk}} \\ \bottomrule
\end{tabular}
\caption{Security Controls Questionnaire Analysis.}
\end{table}

% ----------------------------------------------------------------------
% SECTION 4: TECHNICAL SCAN RESULTS
% ----------------------------------------------------------------------
\section{Technical Scan Results}
A network scan was performed to identify open ports and exposed services on the organization's external infrastructure.

\begin{itemize}
    \item \textbf{Scan Date:} \today
    \item \textbf{Target IP:} \texttt{[Target IP]}
\end{itemize}

The scan identified the following open port:

\begin{table}[h!]
\centering
\begin{tabular}{@{}llll@{}}
\toprule
\textbf{Port} & \textbf{State} & \textbf{Service (Assumed)} & \textbf{Notes} \\ \midrule
22/tcp & Open & SSH (Secure Shell) & Exposing SSH to the internet is a high risk, \\
& & & especially without MFA. This is a primary \\
& & & target for brute-force and credential stuffing attacks. \\ \bottomrule
\end{tabular}
\caption{Open Port Findings.}
\end{table}

\textit{Note: The scan did not return detailed service, product, or version information. A more in-depth authenticated scan is recommended to identify potential software vulnerabilities.}

% ----------------------------------------------------------------------
% SECTION 5: CONSOLIDATED RISK ASSESSMENT
% ----------------------------------------------------------------------
\section{Consolidated Risk Assessment}
The following table synthesizes findings from the questionnaire, technical scan, and pre-existing risk data into a consolidated list of identified risks.

\begin{table}[h!]
\centering
\begin{tabular}{@{}p{0.25\linewidth}p{0.5\linewidth}l@{}}
\toprule
\rowcolor{tablehead}
\textbf{Risk Name} & \textbf{Description} & \textbf{Severity} \\ \midrule
\textbf{Localhost Exposed} & Pre-existing critical vulnerability (CVSS 10.0) affecting host \texttt{[Target IP]}. Details on the exact nature of the exposure were not provided, but a 10.0 score implies a complete system compromise is possible without authentication. & \textcolor{sev_critical}{\textbf{Critical}} \\
\addlinespace
\textbf{Lack of Multi-Factor Authentication (MFA)} & MFA is not enforced for email or computer logins. A compromised password is all an attacker needs to gain access to sensitive communications and internal systems. & \textcolor{sev_critical}{\textbf{Critical}} \\
\addlinespace
\textbf{Exposed SSH Service without MFA} & The SSH management port is open to the internet. Correlated with the lack of MFA, this creates a direct path for attackers to gain shell access to the network via password-based attacks. & \textcolor{sev_high}{\textbf{High}} \\
\addlinespace
\textbf{Inadequate Security Policies and Training} & The lack of an Acceptable Use Policy and mandatory annual training leaves the organization vulnerable to insider threats and social engineering, as staff are not formally guided on security best practices. & \textcolor{sev_high}{\textbf{High}} \\ \bottomrule
\end{tabular}
\caption{Summary of Identified Risks.}
\end{table}

% ----------------------------------------------------------------------
% SECTION 6: RECOMMENDATIONS
% ----------------------------------------------------------------------
\section{Recommendations}
The following actions are recommended to mitigate the identified risks. They are prioritized based on severity and potential impact.

\subsection{Immediate Priority (Remediate within 72 hours)}
\begin{enumerate}
    \item \textbf{Address Critical "Localhost Exposed" Vulnerability:} Immediately investigate and remediate the CVSS 10.0 vulnerability. This should be the highest priority action.
    \item \textbf{Enforce MFA on Critical Systems:} Immediately enable and enforce MFA for all users on the following systems:
    \begin{itemize}
        \item Remote access systems (VPN, SSH).
        \item Email (e.g., Office 365, Google Workspace).
        \item All administrator and privileged accounts.
    \end{itemize}
    \item \textbf{Restrict SSH Access:} If SSH access from the public internet is not essential, block Port 22 at the firewall. If it is required, implement a firewall rule to restrict access to only known, trusted IP addresses.
\end{enumerate}

\subsection{High Priority (Remediate within 30 days)}
\begin{enumerate}
    \setcounter{enumi}{3} % Continue numbering
    \item \textbf{Develop and Implement an Acceptable Use Policy (AUP):} Create a formal AUP that all employees must read and sign. This policy should clearly define the rules for using company IT assets, data handling, and internet usage.
    \item \textbf{Establish Annual Security Awareness Training:} Implement a mandatory security awareness training program for all employees, to be completed annually. The training should cover topics such as phishing, password security, and social engineering.
    \item \textbf{Enforce MFA on All Logins:} Expand the MFA requirement to include all employee computer logins to strengthen internal security.
\end{enumerate}

\end{document}
```