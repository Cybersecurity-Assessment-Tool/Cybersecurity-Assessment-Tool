```latex
\documentclass[12pt]{article}

% --- PACKAGES ---
\usepackage[margin=1in]{geometry}
\usepackage{pifont} % For checkmarks and crosses
\usepackage{booktabs} % For professional tables
\usepackage{hyperref} % For clickable links
\usepackage{url} % For URL formatting
\usepackage{seqsplit} % To split long strings in texttt
\usepackage{graphicx}
\usepackage{xcolor}

% --- DOCUMENT SETUP ---
\hypersetup{
    colorlinks=true,
    linkcolor=blue,
    filecolor=magenta,      
    urlcolor=cyan,
    pdftitle={Cybersecurity Posture Assessment Report},
    pdfpagemode=FullScreen,
}

\newcommand{\yes}{\ding{51}}
\newcommand{\no}{\ding{55}}

% --- TITLE ---
\title{Cybersecurity Posture Assessment Report \\ \large For \textbf{[Organization Name]}}
\author{Cybersecurity Analyst}
\date{\today}

% --- DOCUMENT START ---
\begin{document}

\maketitle
\thispagestyle{empty}
\newpage

\tableofcontents
\newpage

% ===================================================================
% 1. EXECUTIVE SUMMARY
% ===================================================================
\section{Executive Summary}

This report details a cybersecurity posture assessment for \textbf{[Organization Name]}, conducted by correlating data from a network scan, an organizational security questionnaire, and a list of pre-existing risks.

The assessment reveals critical gaps in identity and access management controls. The lack of Multi-Factor Authentication (MFA) for email and computer access represents an immediate and significant threat, exposing the organization to risks of account takeover, data breaches, and ransomware. Furthermore, the absence of a formal Acceptable Use Policy indicates a foundational gap in security governance.

On the technical front, an external network scan of the target IP address \texttt{[Target IP]} did not identify any open ports. This is a positive finding; however, it directly contradicts a pre-existing risk report indicating an open and unencrypted web server on Port 80. This suggests the risk may have been remediated or was a false positive.

Immediate remediation should focus on implementing MFA across all critical systems, starting with email. Concurrently, developing and implementing core security policies is essential to establishing a baseline for secure operations.

% ===================================================================
% 2. ORGANIZATIONAL INFORMATION
% ===================================================================
\section{Organizational Information}

The following information was used as the basis for this assessment. Due to the anonymized nature of the provided data, placeholders have been used where necessary.

\begin{tabular}{@{}ll}
    \toprule
    \textbf{Attribute} & \textbf{Value} \\
    \midrule
    Organization Name & \textbf{[Organization Name]} \\
    Email Domain & \texttt{[Domain]} \\
    External IP Scanned & \texttt{[Client IP]} \\
    Target of Technical Scan & \texttt{[Target IP]} \\
    \bottomrule
\end{tabular}

% ===================================================================
% 3. SECURITY CONTROL REVIEW (QUESTIONNAIRE)
% ===================================================================
\section{Security Control Review}

A review of the organization's security controls was conducted via a questionnaire. The responses highlight significant areas for improvement, particularly in access control and policy enforcement. "No" answers indicate a failure to meet baseline security best practices.

\begin{table}[h!]
\centering
\caption{Security Questionnaire Analysis}
\begin{tabular}{@{}p{0.7\linewidth}cc@{}}
    \toprule
    \textbf{Control Question} & \textbf{Response} & \textbf{Status} \\
    \midrule
    Do you require MFA to access email? & No & \no \\
    Do you require MFA to log into computers? & No & \no \\
    Do you require MFA to access sensitive data systems? & Yes & \yes \\
    Does your organization have an employee acceptable use policy? & No & \no \\
    Does your organization do security awareness training for new employees? & Yes & \yes \\
    Does your organization do security awareness training for all employees at least once per year? & Yes & \yes \\
    \bottomrule
\end{tabular}
\end{table}

\paragraph{Key Findings:} The lack of MFA for email and computer logins, combined with the absence of an Acceptable Use Policy, constitutes three high-impact security gaps that significantly increase organizational risk.

% ===================================================================
% 4. TECHNICAL SCAN RESULTS
% ===================================================================
\section{Technical Scan Results}

An external network vulnerability scan was performed on the target system to identify open ports and exposed services.

\begin{itemize}
    \item \textbf{Target IP:} \texttt{[Target IP]}
    \item \textbf{Scan Date:} Not specified in scan data.
    \item \textbf{Scanner:} Nmap
\end{itemize}

\subsection{Scan Summary}
The scan revealed that the target host is online but did not identify any open TCP ports. The status of a common port is detailed below.

\begin{table}[h!]
\centering
\caption{Nmap Port Scan Details}
\begin{tabular}{@{}llll@{}}
    \toprule
    \textbf{Port} & \textbf{State} & \textbf{Service} & \textbf{Version/Product} \\
    \midrule
    80/tcp & closed & http & N/A \\
    \bottomrule
\end{tabular}
\end{table}

\paragraph{Analysis:} The scan indicates a strong external network posture for the target system, with no services exposed to the public internet. This finding is critical as it contradicts the pre-existing risk documented in the following section. It is possible the risk has been remediated since it was last identified.

% ===================================================================
% 5. CORRELATED RISK ASSESSMENT
% ===================================================================
\section{Correlated Risk Assessment}

This section synthesizes findings from the security control review, the technical scan, and pre-existing risk data to provide a holistic view of the organization's current risk profile.

\begin{table}[h!]
\centering
\caption{Summary of Identified Risks}
\begin{tabular}{@{}p{0.3\linewidth}p{0.5\linewidth}l@{}}
    \toprule
    \textbf{Risk Name} & \textbf{Overview} & \textbf{Severity} \\
    \midrule
    \textbf{No MFA on Email} & The absence of MFA on email accounts allows for account takeover with only a compromised password, leading to data exfiltration, phishing, and business email compromise. & \textbf{Critical} \\
    \addlinespace
    \textbf{No MFA on Endpoints} & Lack of MFA for computer logins means a stolen password provides direct access to an employee's machine and any connected network resources. & High \\
    \addlinespace
    \textbf{No Acceptable Use Policy} & Without a formal policy, there is no enforceable standard for employee behavior regarding company assets, data handling, and internet usage, increasing insider risk. & High \\
    \addlinespace
    \textbf{Unencrypted Web Server (Unverified)} & A pre-existing risk stated Port 80 was open. Our scan found this port to be \textbf{closed}. This risk is likely remediated but requires internal verification. & Medium \\
    \bottomrule
\end{tabular}
\end{table}

% ===================================================================
% 6. RECOMMENDATIONS
% ===================================================================
\section{Recommendations}

Based on the risk assessment, the following actions are recommended to improve the security posture of \textbf{[Organization Name]}.

\subsection{Immediate Priority (0-30 Days)}
\begin{enumerate}
    \item \textbf{Implement MFA for Email:} Enforce MFA for all user accounts accessing the email system (\texttt{[Domain]}). This is the single most effective control to prevent unauthorized access and business email compromise.
\end{enumerate}

\subsection{High Priority (30-90 Days)}
\begin{enumerate}
    \item \textbf{Implement MFA for Endpoint Login:} Deploy MFA for all employees logging into company computers. This adds a critical layer of defense against credential theft and unauthorized access to the internal network.
    \item \textbf{Develop and Implement an Acceptable Use Policy (AUP):} Create a formal AUP that defines the rules for using company IT assets. Ensure all employees read and acknowledge the policy as part of their ongoing employment.
\end{enumerate}

\subsection{Verification and Housekeeping}
\begin{enumerate}
    \item \textbf{Verify Port 80 Status:} The team responsible for the asset at \texttt{[Target IP]} should internally confirm that no web service is running on Port 80. If confirmed, the "Unencrypted Web Server" risk should be formally closed in the risk register.
\end{enumerate}

% ===================================================================
% 7. CONCLUSION
% ===================================================================
\section{Conclusion}

While the external technical scan of \texttt{[Target IP]} showed a secure configuration, the organizational review revealed critical deficiencies in fundamental security controls. The lack of MFA and foundational policies exposes \textbf{[Organization Name]} to severe and immediate threats.

We strongly urge the organization to prioritize the implementation of the recommendations outlined in this report to mitigate these risks and establish a more resilient security posture.

\end{document}
```