```latex
\documentclass[12pt]{article}

% ----------------------------------------------------------------------
% PREAMBLE
% ----------------------------------------------------------------------

% Package for page layout and margins
\usepackage[a4paper, margin=1in]{geometry}

% Package for symbols like checkmarks and crosses
\usepackage{pifont}

% Package for professional-looking tables
\usepackage{booktabs}

% Package for hyperlinks and PDF metadata
\usepackage[hidelinks]{hyperref}
\hypersetup{
    pdftitle={Cybersecurity Assessment Report},
    pdfauthor={Cybersecurity Analyst},
    pdfsubject={Security Assessment},
    pdfkeywords={Security, Risk, Assessment},
    colorlinks=true,
    linkcolor=black,
    citecolor=black,
    filecolor=black,
    urlcolor=blue
}

% Package for handling URLs
\usepackage{url}

% Package for breaking long strings in texttt
\usepackage{seqsplit}

% Package for colors
\usepackage[table]{xcolor}

% Define colors for severity levels
\definecolor{sev_critical}{HTML}{990000}
\definecolor{sev_high}{HTML}{D14302}
\definecolor{sev_medium}{HTML}{F5A623}
\definecolor{sev_low}{HTML}{006400}

% Define commands for Yes/No symbols
\newcommand{\yes}{\ding{51}}
\newcommand{\no}{\ding{55}}

% ----------------------------------------------------------------------
% DOCUMENT START
% ----------------------------------------------------------------------

\begin{document}

% ----------------------------------------------------------------------
% TITLE PAGE
% ----------------------------------------------------------------------
\begin{titlepage}
    \centering
    \vspace*{1cm}
    \Huge \textbf{Cybersecurity Assessment Report}
    \vspace{1.5cm}
    \Large \textbf{Prepared for:} \\
    \vspace{0.5cm}
    \huge \textbf{[Organization Name]}
    \vspace{2cm}
    \large \textbf{Date of Report:} \\
    \vspace{0.5cm}
    \Large \today
    \vfill
    \large \textbf{Confidentiality Notice:} \\
    \vspace{0.2cm}
    \parbox{0.8\textwidth}{\small This document contains confidential and sensitive information intended only for the use of the individual or entity named above. If you are not the intended recipient, you are hereby notified that any dissemination, distribution, or copying of this communication is strictly prohibited.}
\end{titlepage}

% ----------------------------------------------------------------------
% TABLE OF CONTENTS
% ----------------------------------------------------------------------
\tableofcontents
\newpage

% ----------------------------------------------------------------------
% SECTION 1: EXECUTIVE SUMMARY
% ----------------------------------------------------------------------
\section{Executive Summary}
This report details the findings of a cybersecurity assessment conducted for \textbf{[Organization Name]}. The assessment combined a review of organizational security controls, an external network scan, and an analysis of pre-existing risks to provide a comprehensive overview of the current security posture.

The analysis revealed several critical and high-risk gaps that require immediate attention. A pre-existing vulnerability, "Localhost Exposed," is rated at the highest severity (CVSS 10.0) and presents a significant threat. Furthermore, critical procedural gaps were identified, including the lack of multi-factor authentication (MFA) for computer logins and the absence of an employee acceptable use policy. Security awareness training for new hires is also missing, creating a window of vulnerability during the crucial onboarding period.

Technically, the external network scan identified an exposed Secure Shell (SSH) service on port 22. While necessary for remote administration, this service is a common target for attackers and must be properly hardened.

Recommendations focus on addressing these key areas by implementing foundational security controls, hardening exposed services, and formalizing security policies and training programs. Prioritizing these actions will substantially improve the organization's resilience against common cyber threats.

% ----------------------------------------------------------------------
% SECTION 2: ORGANIZATIONAL INFORMATION
% ----------------------------------------------------------------------
\section{Organizational Information}
The following details were used as the basis for this assessment.
\begin{itemize}
    \item \textbf{Organization Name:} \textbf{[Organization Name]}
    \item \textbf{Primary Email Domain:} \texttt{[Domain]}
    \item \textbf{Assessed External IP:} \texttt{[Client IP]}
\end{itemize}

% ----------------------------------------------------------------------
% SECTION 3: SECURITY CONTROL REVIEW
% ----------------------------------------------------------------------
\section{Security Control Review}
A questionnaire was conducted to evaluate the implementation of key administrative and technical security controls. The table below summarizes the organization's responses. A red cross (\no) indicates a potential security gap that increases risk.

\begin{table}[h!]
\centering
\caption{Security Controls Questionnaire Results}
\label{tab:controls}
\begin{tabular}{@{}lc@{}}
\toprule
\textbf{Control Question} & \textbf{Response} \\ \midrule
Do you require MFA to access email? & \yes \\
Do you require MFA to log into computers? & \textcolor{red}{\no} \\
Do you require MFA to access sensitive data systems? & \yes \\
Does your organization have an employee acceptable use policy? & \textcolor{red}{\no} \\
Does your organization do security awareness training for new employees? & \textcolor{red}{\no} \\
Does your organization do security awareness training for all employees at least once per year? & \yes \\ \bottomrule
\end{tabular}
\end{table}

\subsection*{Analysis of Gaps}
The responses highlight three significant gaps in the organization's security program:
\begin{enumerate}
    \item \textbf{No MFA for Computer Logins:} The absence of MFA on endpoints is a critical weakness. If an employee's credentials are stolen, an attacker can gain direct access to their computer and potentially the internal network without needing a second authentication factor.
    \item \textbf{No Acceptable Use Policy (AUP):} An AUP is a foundational document that sets clear expectations for employees on how to use company resources securely. Its absence can lead to inconsistent security practices and a lack of accountability.
    \item \textbf{No Security Training for New Hires:} New employees are often targeted by attackers. Failing to provide immediate security awareness training during onboarding leaves a critical window of vulnerability.
\end{enumerate}

% ----------------------------------------------------------------------
% SECTION 4: TECHNICAL SCAN RESULTS
% ----------------------------------------------------------------------
\section{Technical Scan Results}
An external network scan was performed against the target IP address to identify open ports and exposed services.

\subsection*{Network Port Scan}
\begin{itemize}
    \item \textbf{Target IP:} \texttt{[Target IP]}
    \item \textbf{Scan Date:} \today
\end{itemize}

The scan identified the following open port:

\begin{table}[h!]
\centering
\caption{Open Ports Detected on \texttt{[Target IP]}}
\label{tab:ports}
\begin{tabular}{@{}lllll@{}}
\toprule
\textbf{Port} & \textbf{State} & \textbf{Service} & \textbf{Product} & \textbf{Version} \\ \midrule
22/tcp        & open           & ssh              & Unknown          & Unknown          \\ \bottomrule
\end{tabular}
\end{table}

\subsection*{Analysis of Findings}
The scan confirmed that port 22, used for the Secure Shell (SSH) protocol, is open to the internet. SSH is a common protocol for remote server administration. However, an exposed SSH service is a prime target for automated brute-force attacks, where attackers attempt to guess usernames and passwords. Without proper hardening, this service can provide a direct entry point into the organization's network.

% ----------------------------------------------------------------------
% SECTION 5: CONSOLIDATED RISK ASSESSMENT
% ----------------------------------------------------------------------
\section{Consolidated Risk Assessment}
The following table synthesizes findings from the pre-existing risk register, the security control review, and the technical scan into a prioritized list of risks.

\begin{table}[h!]
\centering
\caption{Summary of Identified Risks}
\label{tab:risks}
\begin{tabular}{@{}p{0.1\linewidth}p{0.3\linewidth}p{0.15\linewidth}p{0.35\linewidth}@{}}
\toprule
\textbf{Risk ID} & \textbf{Risk Title} & \textbf{Severity} & \textbf{Description} \\ \midrule
RISK-001 & Localhost Exposed (Pre-existing) & \textcolor{sev_critical}{\textbf{Critical}} & A pre-existing vulnerability with a CVSS score of 10.0 indicates a service intended for internal use only is exposed to the external network. \\
\addlinespace
RISK-002 & Lack of Endpoint Multi-Factor Authentication & \textcolor{sev_critical}{\textbf{Critical}} & The absence of MFA on computer logins allows an attacker with stolen credentials to gain unauthorized access to an endpoint and the internal network. \\
\addlinespace
RISK-003 & Missing Foundational Security Policies \& Training & \textcolor{sev_high}{\textbf{High}} & The lack of an Acceptable Use Policy and security training for new hires creates a weak security culture and increases susceptibility to social engineering. \\
\addlinespace
RISK-004 & Exposed SSH Management Service & \textcolor{sev_medium}{\textbf{Medium}} & The SSH service on port 22 is exposed to the internet, making it a target for brute-force and credential stuffing attacks. \\ \bottomrule
\end{tabular}
\end{table}

% ----------------------------------------------------------------------
% SECTION 6: RECOMMENDATIONS
% ----------------------------------------------------------------------
\section{Recommendations}
The following actionable recommendations are provided to mitigate the identified risks.

\subsection*{RISK-001: Localhost Exposed (Critical)}
\begin{itemize}
    \item \textbf{Immediate Action:} Identify the service associated with this finding and immediately reconfigure it to bind only to the localhost interface (\texttt{127.0.0.1}). If external access is required, it must be placed behind a firewall and a secure authentication mechanism like a VPN.
\end{itemize}

\subsection*{RISK-002: Lack of Endpoint MFA (Critical)}
\begin{itemize}
    \item \textbf{High Priority:} Procure and deploy a robust MFA solution for all employee and administrator computer logins. This is one of the most effective controls to prevent unauthorized access from compromised credentials.
\end{itemize}

\subsection*{RISK-003: Missing Policies \& Training (High)}
\begin{itemize}
    \item \textbf{Policy Development:} Draft and ratify a formal Employee Acceptable Use Policy (AUP). Ensure all current employees read and acknowledge the policy.
    \item \textbf{Onboarding Process:} Integrate mandatory security awareness training into the new-hire onboarding process. This training should occur within the first week of employment.
\end{itemize}

\subsection*{RISK-004: Exposed SSH Service (Medium)}
\begin{itemize}
    \item \textbf{Hardening:} Implement the following hardening measures for the SSH service on \texttt{[Target IP]}:
    \begin{itemize}
        \item Disable password-based authentication and enforce the use of public key cryptography.
        \item Disable direct root login. Administrators should log in with a non-privileged account and elevate privileges as needed.
        \item Implement an intrusion prevention tool like \texttt{fail2ban} to automatically block IPs that exhibit malicious behavior (e.g., repeated failed login attempts).
        \item If feasible, restrict access to the SSH port to a list of known, trusted IP addresses using firewall rules.
    \end{itemize}
\end{itemize}

% ----------------------------------------------------------------------
% DOCUMENT END
% ----------------------------------------------------------------------
\end{document}
```