```latex
\documentclass[12pt, a4paper]{article}

% Preamble: Required Packages
\usepackage[margin=1in]{geometry}
\usepackage{pifont} % For checkmarks and crosses
\usepackage{booktabs} % For professional-looking tables
\usepackage{hyperref} % For clickable links and better PDF metadata
\usepackage{url} % For formatting URLs
\usepackage{seqsplit} % For splitting long, unbreakable strings like hashes or URLs
\usepackage{graphicx}
\usepackage{xcolor}

% --- Document Setup ---
\hypersetup{
    colorlinks=true,
    linkcolor=blue,
    filecolor=magenta,      
    urlcolor=cyan,
    pdftitle={Cybersecurity Assessment Report},
    pdfauthor={Automated Security Report Generator},
    pdfsubject={Security Analysis},
    pdfkeywords={Cybersecurity, Assessment, Report},
    bookmarks=true
}

% --- Title ---
\title{Cybersecurity Assessment Report \\ \large For: \textbf{[Organization Name]}}
\author{Cybersecurity Analysis Division}
\date{\today}

% --- Document Body ---
\begin{document}

\maketitle
\thispagestyle{empty}
\newpage

\tableofcontents
\newpage

\section{Executive Summary}

This report details the findings of a cybersecurity assessment conducted for \textbf{[Organization Name]}. The assessment incorporated an analysis of organizational security controls, a technical network scan of the external perimeter, and a review of previously identified risks.

The analysis reveals a mixed security posture. The organization has implemented foundational controls such as Multi-Factor Authentication (MFA) for email and computer access. However, several critical and high-risk gaps were identified in administrative and policy-based controls. Specifically, the lack of MFA for sensitive data systems, the absence of an employee Acceptable Use Policy, and incomplete annual security awareness training present significant risks.

On the technical front, the external network scan of \texttt{[Client IP]} indicates a minimal attack surface. A previously identified risk concerning an open unencrypted web server port (Port 80) appears to have been remediated, as the port was found to be closed during the current scan.

This report provides a detailed breakdown of these findings and offers actionable recommendations to mitigate the identified risks and strengthen the overall security posture of \textbf{[Organization Name]}.

\section{Organizational Information}

The following information was used as the basis for this assessment. Placeholder values are used where data was not provided.

\begin{itemize}
    \item \textbf{Organization Name:} \textbf{[Organization Name]}
    \item \textbf{Primary Domain:} \texttt{[Domain]}
    \item \textbf{External IP Scanned:} \texttt{[Client IP]}
\end{itemize}

\section{Security Control Review}

An internal review of security controls was conducted based on a standardized questionnaire. The responses highlight key areas of strength and weakness in the organization's security policies and procedures. "No" answers indicate significant gaps that require immediate attention.

\begin{table}[h!]
\centering
\caption{Security Controls Questionnaire Analysis}
\label{tab:controls}
\begin{tabular}{p{0.6\linewidth} c p{0.2\linewidth}}
\toprule
\textbf{Control Question} & \textbf{Response} & \textbf{Assessment} \\
\midrule
Do you require MFA to access email? & \ding{51} & Best Practice Met \\
Do you require MFA to log into computers? & \ding{51} & Best Practice Met \\
Do you require MFA to access sensitive data systems? & \textcolor{red}{\ding{55}} & \textcolor{red}{Critical Gap} \\
Does your organization have an employee acceptable use policy? & \textcolor{red}{\ding{55}} & \textcolor{red}{High Risk} \\
Does your organization do security awareness training for new employees? & \ding{51} & Best Practice Met \\
Does your organization do security awareness training for all employees at least once per year? & \textcolor{red}{\ding{55}} & \textcolor{red}{High Risk} \\
\bottomrule
\end{tabular}
\end{table}

\subsection{Analysis of Control Gaps}
The primary concerns identified are the lack of mandatory MFA for sensitive systems, the absence of a formal Acceptable Use Policy (AUP), and the failure to conduct annual security training for all staff. These gaps significantly increase the risk of unauthorized access, insider threat, and susceptibility to social engineering attacks.

\section{Technical Scan Results}

A network scan was performed on the designated target IP address to identify exposed services and potential vulnerabilities.

\begin{itemize}
    \item \textbf{Target IP:} \texttt{[Target IP]}
    \item \textbf{Scan Utility:} Nmap
    \item \textbf{Host Status:} Up
\end{itemize}

\subsection{Port Scan Findings}
The scan revealed a very limited external attack surface. The following is a summary of the port status:
\begin{itemize}
    \item \textbf{Port 80/tcp (HTTP):} State: \textbf{Closed}. No service was detected.
\end{itemize}
No other open or filtered ports were identified during the scan. This configuration is positive, as it minimizes the exposure of services to the public internet.

\subsection{Correlation with Existing Risks}
The finding that port 80 is closed directly contradicts a previously documented risk, "Unencrypted Web Server," which was predicated on this port being open. This indicates that the risk has been successfully remediated or was a false positive in a prior assessment.

\section{Risk Assessment and Findings}

The following table summarizes the key risks identified through the correlation of the security control review and the technical scan. Risks are prioritized by severity to guide remediation efforts.

\begin{table}[h!]
\centering
\caption{Summary of Identified Risks}
\label{tab:risks}
\begin{tabular}{p{0.1\linewidth} p{0.25\linewidth} p{0.15\linewidth} p{0.4\linewidth}}
\toprule
\textbf{ID} & \textbf{Finding} & \textbf{Severity} & \textbf{Description} \\
\midrule
\textbf{R-01} & No MFA for Sensitive Data Systems & \textbf{Critical} & Lack of MFA on systems storing or processing sensitive data exposes the organization to significant risk of data breach from compromised credentials. \\
\addlinespace
\textbf{R-02} & No Acceptable Use Policy (AUP) & \textbf{High} & Without a formal AUP, there are no clear guidelines for employees on the proper use of company assets, increasing the risk of misuse and insider threats. \\
\addlinespace
\textbf{R-03} & Inadequate Annual Security Training & \textbf{High} & Failing to train all employees annually leaves the organization vulnerable to phishing, social engineering, and other human-centric attacks. \\
\addlinespace
\textbf{I-01} & Previous Risk Remediated & Informational & The risk of an "Unencrypted Web Server" on port 80 is no longer valid, as the current scan confirms the port is closed. This is a positive security development. \\
\bottomrule
\end{tabular}
\end{table}

\section{Recommendations}

To address the identified risks and improve the overall security posture, the following actions are recommended with urgency:

\begin{itemize}
    \item \textbf{[Critical] R-01: Implement Comprehensive MFA} \\
    Immediately prioritize the deployment of Multi-Factor Authentication across all systems that access, store, or process sensitive organizational or customer data. This is the single most effective control to prevent unauthorized access.

    \item \textbf{[High] R-02: Develop and Enforce an Acceptable Use Policy} \\
    Draft, approve, and disseminate a formal AUP that clearly defines the rules and responsibilities for all employees when using company networks, systems, and data. Require all employees to read and acknowledge the policy.

    \item \textbf{[High] R-03: Establish Mandatory Annual Security Training} \\
    Implement a mandatory, recurring security awareness training program for all employees. The training should be conducted at least annually and cover topics such as phishing, password security, and data handling best practices.
\end{itemize}

\end{document}
```