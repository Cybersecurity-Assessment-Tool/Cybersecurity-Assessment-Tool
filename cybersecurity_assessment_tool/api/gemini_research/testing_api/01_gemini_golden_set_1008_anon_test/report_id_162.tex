```latex
\documentclass[12pt]{article}

% --- PACKAGES ---
\usepackage[margin=1in]{geometry}
\usepackage{pifont} % For dingbats (checkmarks and crosses)
\usepackage{booktabs} % For professional-looking tables
\usepackage{hyperref} % For hyperlinks
\usepackage{url} % For URL formatting
\usepackage{seqsplit} % To split long strings in tt font
\usepackage[utf8]{inputenc}

% --- DOCUMENT METADATA ---
\title{Cybersecurity Posture Assessment Report}
\author{Cybersecurity Analysis Division}
\date{\today}

% --- HYPERREF SETUP ---
\hypersetup{
    colorlinks=true,
    linkcolor=black,
    urlcolor=blue,
    pdftitle={Cybersecurity Posture Assessment Report},
    pdfauthor={Cybersecurity Analysis Division},
}

% --- DOCUMENT START ---
\begin{document}

\maketitle
\thispagestyle{empty}
\newpage
\tableofcontents
\thispagestyle{empty}
\newpage
\setcounter{page}{1}

% ==============================================================================
% 1. EXECUTIVE SUMMARY
% ==============================================================================
\section{Executive Summary}

This report provides a comprehensive cybersecurity assessment for \textbf{[Organization Name]}, based on an analysis of organizational security controls, an external network scan, and a review of pre-existing risk documentation. The assessment was conducted on \today.

The analysis reveals several critical and high-risk security gaps that require immediate attention. The most significant finding is the complete absence of Multi-Factor Authentication (MFA) across key services, including email, computer logins, and access to sensitive data systems. This represents a critical vulnerability that substantially increases the risk of unauthorized access and account compromise.

Furthermore, the organization's security awareness program is insufficient, lacking mandatory annual training for all employees. This gap leaves the organization more susceptible to social engineering and phishing attacks over time.

On a positive note, a technical network scan of the external IP address \texttt{[Client IP]} revealed that port 80 (HTTP) is closed. This finding contradicts a previously documented risk, "Unencrypted Web Server," suggesting that the risk may have been remediated. This discrepancy highlights the need for regular risk register validation.

Immediate remediation should focus on deploying a robust MFA solution across all critical platforms. Concurrently, an annual security awareness training program should be established to cultivate a stronger security culture.

% ==============================================================================
% 2. ORGANIZATIONAL INFORMATION
% ==============================================================================
\section{Organizational Information}

The following information was used as the basis for this assessment. Due to the anonymized nature of the provided data, placeholders have been used where necessary.

\begin{tabular}{@{}ll}
    \toprule
    \textbf{Attribute} & \textbf{Value} \\
    \midrule
    Organization Name & \textbf{[Organization Name]} \\
    Primary Domain & \texttt{[Domain]} \\
    External IP Address Scanned & \texttt{[Client IP]} \\
    \bottomrule
\end{tabular}

% ==============================================================================
% 3. SECURITY CONTROL REVIEW (QUESTIONNAIRE)
% ==============================================================================
\section{Security Control Review}

An internal security questionnaire was reviewed to assess the current state of administrative and technical controls. The responses indicate significant gaps in access control and employee training policies. A "No" response indicates a deviation from security best practices and a potential risk.

\begin{tabular}{@{}p{0.6\linewidth} c p{0.25\linewidth}@{}}
    \toprule
    \textbf{Control Question} & \textbf{Response} & \textbf{Assessment} \\
    \midrule
    Do you require MFA to access email? & \ding{55} & \textbf{Critical Gap} \\
    Do you require MFA to log into computers? & \ding{55} & \textbf{Critical Gap} \\
    Do you require MFA to access sensitive data systems? & \ding{55} & \textbf{Critical Gap} \\
    Does your organization do security awareness training for all employees at least once per year? & \ding{55} & \textbf{High Risk} \\
    \addlinespace
    Does your organization have an employee acceptable use policy? & \ding{51} & Good Practice \\
    Does your organization do security awareness training for new employees? & \ding{51} & Good Practice \\
    \bottomrule
\end{tabular}

\vspace{1em}
\noindent
\textbf{Note:} \ding{51} indicates a "Yes" response, and \ding{55} indicates a "No" response.

% ==============================================================================
% 4. TECHNICAL SCAN RESULTS
% ==============================================================================
\section{Technical Scan Results}

An external network scan was performed using Nmap to identify open ports and exposed services on the organization's public-facing infrastructure.

\begin{itemize}
    \item \textbf{Target IP Address:} \texttt{[Target IP]}
    \item \textbf{Scan Date:} Scan data provided on \today.
    \item \textbf{Target Status:} Host was detected as \textbf{Up}.
\end{itemize}

\subsection{Port Scan Findings}
The scan revealed no open ports on the target system. The status of a notable port is detailed below.

\begin{tabular}{@{}lllll@{}}
    \toprule
    \textbf{Port} & \textbf{State} & \textbf{Service} & \textbf{Version} & \textbf{Notes} \\
    \midrule
    80/tcp & Closed & http & N/A & The port for unencrypted web traffic is closed. \\
    & & & & This is a positive security finding. \\
    \bottomrule
\end{tabular}

\subsection{Analysis of Technical Findings}
The technical scan did not identify any immediate external vulnerabilities. The fact that port 80 is closed is a significant finding, as it directly contradicts a known risk in the organization's risk register (see Section 5). This suggests that the previously identified risk of an "Unencrypted Web Server" may be outdated or has been successfully remediated.

% ==============================================================================
% 5. CORRELATED RISK ASSESSMENT
% ==============================================================================
\section{Correlated Risk Assessment}

This section synthesizes findings from the security control review, the technical scan, and the pre-existing risk documentation into a prioritized list of risks.

\begin{tabular}{@{}p{0.05\linewidth} p{0.45\linewidth} p{0.15\linewidth} p{0.25\linewidth}@{}}
    \toprule
    \textbf{ID} & \textbf{Risk Description} & \textbf{Severity} & \textbf{Source / Correlation} \\
    \midrule
    \textbf{R-01} & \textbf{Lack of Multi-Factor Authentication (MFA):} No MFA is enforced for email, computer logins, or sensitive systems, severely increasing the risk of account compromise via stolen credentials. & \textbf{Critical} & Security Questionnaire \\
    \addlinespace
    \textbf{R-02} & \textbf{Inadequate Security Awareness Training:} Failure to conduct annual security training for all staff diminishes vigilance against phishing and social engineering attacks. & \textbf{High} & Security Questionnaire \\
    \addlinespace
    \textbf{R-03} & \textbf{Outdated Risk Register Information:} A documented risk for an "Unencrypted Web Server" on Port 80 exists, but the technical scan confirms this port is closed. The risk register is not aligned with the current technical reality. & \textbf{Informational} & Correlation of Nmap Scan and Current Risks JSON \\
    \bottomrule
\end{tabular}

% ==============================================================================
% 6. RECOMMENDATIONS
% ==============================================================================
\section{Recommendations}

Based on the correlated risk assessment, the following prioritized actions are recommended to improve the organization's security posture.

\subsection{Priority 1: Implement Multi-Factor Authentication (Critical)}
\begin{itemize}
    \item \textbf{Action:} Deploy a robust MFA solution across the entire organization.
    \item \textbf{Details:} Prioritize the rollout of MFA on the following systems in order:
        \begin{enumerate}
            \item Email systems (e.g., Office 365, Google Workspace).
            \item Access to sensitive data systems (e.g., financial software, databases, CRM).
            \item All employee computer and remote access logins (e.g., VPN, RDP).
        \end{enumerate}
    \item \textbf{Justification:} This is the single most effective control to prevent unauthorized access resulting from compromised credentials.
\end{itemize}

\subsection{Priority 2: Establish Annual Security Training Program (High)}
\begin{itemize}
    \item \textbf{Action:} Implement a mandatory, annual security awareness training program for all employees, including management.
    \item \textbf{Details:} The training should cover modern threats such as phishing, ransomware, social engineering, and proper data handling. The program should include a mechanism to track completion.
    \item \textbf{Justification:} Regular training reinforces security concepts and keeps employees vigilant against evolving cyber threats, reducing the likelihood of human error leading to a breach.
\end{itemize}

\subsection{Priority 3: Validate and Update Risk Register (Informational)}
\begin{itemize}
    \item \textbf{Action:} Conduct a formal review of the finding regarding the "Unencrypted Web Server" risk.
    \item \textbf{Details:} Verify internally that the closure of port 80 was an intentional and permanent change. If confirmed, update the risk register to formally close the risk item. Implement a quarterly process to review and validate all documented risks.
    \item \textbf{Justification:} An accurate and up-to-date risk register is essential for effective security management, resource allocation, and strategic decision-making.
\end{itemize}

% ==============================================================================
% 7. CONCLUSION
% ==============================================================================
\section{Conclusion}

The assessment for \textbf{[Organization Name]} reveals a mixed security posture. While the external network perimeter appears secure with no open ports detected, critical deficiencies in internal access controls and employee security training present a significant threat to the organization.

The lack of MFA is a severe vulnerability that must be addressed as the highest priority. By implementing the recommendations outlined in this report, particularly the deployment of MFA and the establishment of a recurring security awareness program, the organization can substantially reduce its risk exposure and build a more resilient security foundation.

\end{document}
```