```latex
\documentclass[12pt]{article}

% Preamble: Required Packages
\usepackage[margin=1in]{geometry}
\usepackage{pifont} % Required for checkmarks and crosses (\ding)
\usepackage{booktabs} % For professional-looking tables
\usepackage{hyperref} % For clickable links and references
\usepackage{url}      % For properly formatting URLs
\usepackage{seqsplit} % For splitting long, unbreakable strings like hashes or tokens

% Hyperref Setup for better visuals
\hypersetup{
    colorlinks=true,
    linkcolor=blue,
    filecolor=magenta,
    urlcolor=cyan,
    pdftitle={Cybersecurity Posture Assessment Report},
    pdfauthor={Cybersecurity Analysis Division},
}

% Custom commands for Yes/No in tables
\newcommand{\yes}{\ding{51}} % Checkmark
\newcommand{\no}{\ding{55}}  % X mark

\title{Cybersecurity Posture Assessment Report}
\author{Cybersecurity Analysis Division}
\date{\today}

\begin{document}

\maketitle
\thispagestyle{empty}
\newpage

\tableofcontents
\newpage

\section{Executive Summary}

This report provides a comprehensive analysis of the cybersecurity posture of \textbf{[Organization Name]}. The assessment synthesizes data from a network vulnerability scan, a security controls questionnaire, and a review of pre-existing risk documentation.

The most critical finding is the direct public exposure of the Remote Desktop Protocol (RDP) on port 3389. This vulnerability, confirmed by both the live network scan and existing risk data, presents a direct and high-impact vector for unauthorized access and ransomware deployment.

This critical technical flaw is severely exacerbated by systemic organizational weaknesses. The complete absence of Multi-Factor Authentication (MFA) for email, computer logins, and sensitive data systems means that a single compromised password could lead to a full network breach. Furthermore, foundational security policies, including an Acceptable Use Policy and mandatory annual security training, are not in place, indicating a reactive and immature security culture.

Immediate remediation of the exposed RDP service is required, followed by the rapid implementation of MFA and the development of core security policies to mitigate the extreme level of risk identified.

\section{Organizational Information}

The following details have been recorded for the organization under review. Placeholders are used where data was not provided.

\begin{itemize}
    \item \textbf{Organization Name:} \textbf{[Organization Name]}
    \item \textbf{Primary Email Domain:} \texttt{[Domain]}
    \item \textbf{Assessed External IP:} \texttt{[Client IP]}
\end{itemize}

\section{Security Control Review}

An internal review of security controls was conducted via a questionnaire. The results highlight significant gaps in fundamental security practices, particularly in Identity and Access Management (IAM). Gaps requiring immediate attention are marked with \no.

\begin{table}[h!]
\centering
\begin{tabular}{p{0.7\textwidth}c}
\toprule
\textbf{Control Question} & \textbf{Status} \\
\midrule
Do you require MFA to access email? & \no \\
Do you require MFA to log into computers? & \no \\
Do you require MFA to access sensitive data systems? & \no \\
Does your organization have an employee acceptable use policy? & \no \\
Does your organization do security awareness training for new employees? & \yes \\
Does your organization do security awareness training for all employees at least once per year? & \no \\
\bottomrule
\end{tabular}
\caption{Security Controls Questionnaire Results}
\label{tab:controls}
\end{table}

\paragraph{Analysis} The review reveals a critical failure to implement MFA, which is a baseline security control for protecting against credential theft. Additionally, the lack of an Acceptable Use Policy and recurring security training for all staff increases the organization's susceptibility to human-centric attacks like phishing.

\section{Technical Scan Results}

An external network scan was performed to identify exposed services on the organization's perimeter. The scan data did not include a specific timestamp; this report was generated on the date listed on the title page.

\begin{itemize}
    \item \textbf{Target IP Address:} \texttt{[Target IP]}
    \item \textbf{Host Status:} Up
\end{itemize}

The scan identified the following open port, which represents a significant security risk:

\begin{table}[h!]
\centering
\begin{tabular}{llll}
\toprule
\textbf{Port} & \textbf{State} & \textbf{Service Name} & \textbf{Description} \\
\midrule
3389/tcp & open & ms-wbt-server & Microsoft Remote Desktop Protocol (RDP) \\
\bottomrule
\end{tabular}
\caption{Open Ports Detected on \texttt{[Target IP]}}
\label{tab:scanresults}
\end{table}

\paragraph{Analysis} The confirmation of an open RDP port is a critical finding. RDP is a primary target for threat actors, who actively scan the internet for exposed servers. Attackers use this protocol for initial access, lateral movement, and the deployment of ransomware. This finding directly corroborates the pre-existing risk documented in Section 5.

\section{Correlated Risk Assessment}

This section correlates the findings from the security control review (Section 3), the technical scan (Section 4), and pre-existing risk data. The combination of these findings provides a holistic view of the organization's risk profile.

\begin{table}[h!]
\centering
\begin{tabular}{p{0.25\textwidth}p{0.55\textwidth}l}
\toprule
\textbf{Risk Name} & \textbf{Overview} & \textbf{Severity} \\
\midrule
\textbf{Critical RDP Exposure} & The network scan confirms that RDP (port 3389) is exposed on \texttt{[Target IP]}. This aligns with pre-existing risk data (CVSS 9.0) and presents a direct path for attackers into the network. & \textbf{Critical} \\
\addlinespace
\textbf{Systemic Lack of MFA} & The organization does not enforce MFA for any critical systems. This drastically lowers the barrier for an attacker to compromise an account and exploit the exposed RDP service or access sensitive email data. & \textbf{Critical} \\
\addlinespace
\textbf{Deficient Security Policies \& Training} & The absence of an Acceptable Use Policy and mandatory annual security training indicates a weak security culture. This increases the likelihood of a successful phishing attack, which could provide the credentials needed to exploit other weaknesses. & \textbf{High} \\
\bottomrule
\end{tabular}
\caption{Summary of Identified and Correlated Risks}
\label{tab:risks}
\end{table}

\section{Recommendations}

Based on the correlated findings, we recommend the following actions, prioritized by urgency, to remediate the identified risks and improve the overall security posture.

\subsection{Immediate Actions (To Be Completed Within 24 Hours)}
\begin{itemize}
    \item \textbf{Block RDP Access:} Immediately configure the perimeter firewall to \textbf{deny all inbound traffic} to TCP port 3389 on host \texttt{[Target IP]}. This is the single most important step to mitigate the threat of a ransomware attack or network breach.
\end{itemize}

\subsection{Short-Term Actions (To Be Completed Within 30 Days)}
\begin{itemize}
    \item \textbf{Implement MFA:} Aggressively deploy Multi-Factor Authentication for all users across all critical systems, prioritizing the following:
    \begin{itemize}
        \item Email (e.g., Office 365, Google Workspace)
        \item Remote access solutions (VPNs)
        \item All administrative and privileged accounts
        \item Logins to company computers and servers
    \end{itemize}
\end{itemize}

\subsection{Mid-Term Actions (To Be Completed Within 90 Days)}
\begin{itemize}
    \item \textbf{Develop an Acceptable Use Policy (AUP):} Create and distribute a formal AUP that defines the rules for using company IT assets, data handling, and internet usage. Require all employees to read and acknowledge the policy.
    \item \textbf{Establish Annual Security Training:} Implement a mandatory security awareness training program for all employees to be completed annually. This program must cover phishing, password security, social engineering, and the new AUP.
    \item \textbf{Deploy a Secure Remote Access Solution:} If remote administration is required, replace direct RDP access with a secure, modern solution such as a Virtual Private Network (VPN) with MFA or a Zero Trust Network Access (ZTNA) platform.
\end{itemize}

\end{document}
```