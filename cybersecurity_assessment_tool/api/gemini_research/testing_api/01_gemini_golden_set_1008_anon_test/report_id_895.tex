```latex
\documentclass[12pt]{article}

% Preamble: Required Packages
\usepackage[margin=1in]{geometry}
\usepackage{pifont} % For checkmarks and crosses
\usepackage{booktabs} % For professional tables
\usepackage{hyperref} % For clickable links
\usepackage{url} % For URL formatting
\usepackage{seqsplit} % For splitting long strings in texttt
\usepackage[T1]{fontenc}
\usepackage{array}
\usepackage{graphicx}
\usepackage{xcolor}

% Document Metadata
\title{Cybersecurity Posture Assessment Report}
\author{Cybersecurity Analysis Division}
\date{\today}

% Hyperref Setup
\hypersetup{
    colorlinks=true,
    linkcolor=blue,
    filecolor=magenta,      
    urlcolor=cyan,
    pdftitle={Cybersecurity Posture Assessment Report},
    pdfpagemode=FullScreen,
}

% Custom Commands for Table Formatting
\newcolumntype{L}[1]{>{\raggedright\let\newline\\\arraybackslash\hspace{0pt}}m{#1}}
\newcolumntype{C}[1]{>{\centering\let\newline\\\arraybackslash\hspace{0pt}}m{#1}}

\begin{document}

\maketitle
\tableofcontents
\newpage

% --- 1. Executive Summary ---
\section{Executive Summary}

This report provides a comprehensive analysis of the cybersecurity posture for \textbf{[Organization Name]}. The assessment is based on a correlation of organizational data, a review of existing security controls, and an external network scan.

The analysis identified several critical and high-risk gaps in the organization's security framework. Key findings include:
\begin{itemize}
    \item \textbf{Critical Control Gap:} Multi-Factor Authentication (MFA) is not enforced for accessing sensitive data systems. This significantly increases the risk of unauthorized access and potential data breach from compromised credentials.
    \item \textbf{High-Risk Policy Gaps:} The organization lacks a formal Acceptable Use Policy (AUP) and does not provide security awareness training for new employees. These foundational elements are essential for establishing a security-conscious culture and mitigating insider threats.
    \item \textbf{High-Risk Technical Finding:} The external network scan revealed an open Secure Shell (SSH) port (22/TCP). Exposed management services like SSH are primary targets for brute-force attacks and exploitation if not properly secured.
\end{itemize}

The overall security posture is considered to be at a \textbf{High Risk} level due to these fundamental deficiencies in both policy and technical controls. Immediate remediation is recommended to address the identified vulnerabilities and reduce the organization's attack surface.

% --- 2. Organizational Information ---
\section{Organizational Information}

The following information was used as the basis for this assessment. Due to the anonymized nature of the provided data, placeholders have been used where necessary.

\begin{table}[h!]
\centering
\begin{tabular}{@{}ll@{}}
\toprule
\textbf{Attribute} & \textbf{Value} \\ \midrule
Organization Name & \textbf{[Organization Name]} \\
Primary Email Domain & \texttt{[Domain]} \\
External IP Address Scanned & \texttt{[Client IP]} \\ \bottomrule
\end{tabular}
\caption{Client Organizational Data}
\label{tab:org_data}
\end{table}

% --- 3. Security Control Review ---
\section{Security Control Review}

A review of the organization's security controls was conducted via a standardized questionnaire. The responses highlight significant gaps in access control and employee security awareness programs. A "No" response indicates a deviation from security best practices and a potential area of high risk.

\begin{table}[h!]
\centering
\begin{tabular}{@{}L{12cm}C{2cm}@{}}
\toprule
\textbf{Control Question} & \textbf{Response} \\ \midrule
Do you require MFA to access email? & \textcolor{green}{\ding{51}} \\
Do you require MFA to log into computers? & \textcolor{green}{\ding{51}} \\
\textbf{Do you require MFA to access sensitive data systems?} & \textcolor{red}{\ding{55}} \\
\textbf{Does your organization have an employee acceptable use policy?} & \textcolor{red}{\ding{55}} \\
\textbf{Does your organization do security awareness training for new employees?} & \textcolor{red}{\ding{55}} \\
Does your organization do security awareness training for all employees at least once per year? & \textcolor{green}{\ding{51}} \\ \bottomrule
\end{tabular}
\caption{Security Controls Questionnaire Results}
\label{tab:controls}
\end{table}

% --- 4. Technical Scan Results ---
\section{Technical Scan Results}

An external network scan was performed against the target IP address \texttt{[Target IP]} to identify open ports and exposed services.

\subsection{Open Ports}
The following ports were found to be open and accessible from the public internet.

\begin{table}[h!]
\centering
\begin{tabular}{@{}llll@{}}
\toprule
\textbf{Port} & \textbf{State} & \textbf{Service} & \textbf{Product / Version} \\ \midrule
22/tcp & open & ssh & N/A \\ \bottomrule
\end{tabular}
\caption{Open Port Findings}
\label{tab:nmap_results}
\end{table}

\subsection{Technical Analysis}
The presence of an open SSH port (22) represents a significant security risk. SSH is a common protocol for remote server administration. If exposed to the internet, it becomes a primary target for automated brute-force attacks, where attackers attempt to guess usernames and passwords. Without information on the specific version or configuration, it is assumed to be a high-risk finding that requires immediate attention and hardening.

% --- 5. Consolidated Risk Assessment ---
\section{Consolidated Risk Assessment}
The following table synthesizes findings from the security control review, technical scan, and pre-existing risk data. Each risk is assigned a severity level based on its potential impact and likelihood of exploitation. As no pre-existing risks were provided, this assessment is based solely on new findings.

\begin{table}[h!]
\centering
\begin{tabular}{@{}p{2cm}p{9cm}p{2.5cm}@{}}
\toprule
\textbf{Risk ID} & \textbf{Description} & \textbf{Severity} \\ \midrule
RISK-001 & \textbf{Lack of MFA on Sensitive Data Systems:} User accounts with access to critical data are protected only by a password. A single compromised credential could lead to a major data breach. & \textbf{Critical} \\
\addlinespace
RISK-002 & \textbf{Exposed SSH Management Service:} Port 22 (SSH) is open to the public internet, making it a target for brute-force attacks and exploitation of potential vulnerabilities in the SSH server software. & \textbf{High} \\
\addlinespace
RISK-003 & \textbf{Inadequate Employee Security Policies \& Training:} The absence of an Acceptable Use Policy and security training for new hires creates an environment where employees are more likely to engage in risky behavior, fall for phishing attacks, or cause accidental data exposure. & \textbf{High} \\ \bottomrule
\end{tabular}
\caption{Summary of Identified Risks}
\label{tab:risks}
\end{table}

% --- 6. Recommendations ---
\section{Recommendations}
The following prioritized recommendations are provided to mitigate the identified risks and improve the overall security posture of \textbf{[Organization Name]}.

\subsection{Priority 1: Remediate Critical Risks}
\begin{itemize}
    \item \textbf{Implement MFA for Sensitive Systems (RISK-001):} Immediately enforce Multi-Factor Authentication for all access to systems containing sensitive or critical data. This is the single most effective control to prevent unauthorized access from compromised credentials.
\end{itemize}

\subsection{Priority 2: Remediate High Risks}
\begin{itemize}
    \item \textbf{Secure Exposed SSH Service (RISK-002):}
        \begin{itemize}
            \item If remote access is not required, block port 22 at the network firewall.
            \item If access is required, restrict access to a list of trusted IP addresses (IP allowlist).
            \item Disable password-based authentication and enforce the use of public-key cryptography.
            \item Ensure the SSH service is patched and running an up-to-date version.
        \end{itemize}
    \item \textbf{Develop and Implement an Acceptable Use Policy (RISK-003):} Create a formal AUP that clearly defines the rules and responsibilities for all employees when using company assets and data. Require all employees to read and acknowledge the policy.
    \item \textbf{Establish New Hire Security Training (RISK-003):} Integrate mandatory cybersecurity awareness training into the employee onboarding process. This training should cover key topics such as phishing, password hygiene, and the AUP.
\end{itemize}

\end{document}
```