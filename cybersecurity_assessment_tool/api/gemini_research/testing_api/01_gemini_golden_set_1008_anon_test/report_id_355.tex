```latex
\documentclass[12pt]{article}

% Required Packages
\usepackage[margin=1in]{geometry}
\usepackage{pifont} % For checkmarks and crosses
\usepackage{booktabs} % For professional tables
\usepackage{hyperref} % For clickable links
\usepackage{url} % For URL formatting
\usepackage{seqsplit} % For splitting long strings without breaking
\usepackage{graphicx}
\usepackage{xcolor}

% Hyperref Setup
\hypersetup{
    colorlinks=true,
    linkcolor=blue,
    filecolor=magenta,      
    urlcolor=cyan,
    pdftitle={Cybersecurity Assessment Report},
    pdfpagemode=FullScreen,
}

% Document Information
\title{Cybersecurity Assessment Report \\ \large For \textbf{[Organization Name]}}
\author{Cybersecurity Analyst}
\date{\today}

\begin{document}

\maketitle
\thispagestyle{empty}
\newpage

\tableofcontents
\newpage

% --- 1. Executive Summary ---
\section{Executive Summary}

This report provides a comprehensive cybersecurity assessment for \textbf{[Organization Name]}, based on an analysis of network scan data, organizational security controls, and pre-existing risk information. The assessment was conducted on \today.

The analysis reveals several critical gaps that expose the organization to significant risk. Key findings include an externally accessible Secure Shell (SSH) service, a lack of mandatory Multi-Factor Authentication (MFA) for computer logins, and the absence of annual security awareness training for all employees.

While the organization has implemented some positive security controls, such as MFA for email and sensitive systems, the identified vulnerabilities create pathways for potential unauthorized access, data breaches, and other malicious activities. Immediate remediation of the critical and high-severity risks outlined in this report is strongly recommended to improve the organization's overall security posture.

% --- 2. Organizational Information ---
\section{Organizational Information}

This section details the organizational data used as the basis for this assessment. Due to the anonymized nature of the provided data, placeholders have been used where specific information was not available.

\begin{table}[h!]
\centering
\caption{Client Details}
\begin{tabular}{@{}ll@{}}
\toprule
\textbf{Attribute} & \textbf{Value} \\ \midrule
Organization Name & \textbf{[Organization Name]} \\
Primary Domain & \texttt{[Domain]} \\
External IP Scanned & \texttt{[Client IP]} \\ \bottomrule
\end{tabular}
\end{table}

% --- 3. Security Control Review ---
\section{Security Control Review}

The following table summarizes the organization's responses to a security controls questionnaire. A green checkmark (\textcolor{green}{\ding{51}}) indicates a positive control is in place, while a red cross (\textcolor{red}{\ding{55}}) indicates a control gap that presents a security risk.

\begin{table}[h!]
\centering
\caption{Security Controls Questionnaire Analysis}
\begin{tabular}{@{}p{0.8\textwidth}c@{}}
\toprule
\textbf{Control Question} & \textbf{Response} \\ \midrule
Do you require MFA to access email? & \textcolor{green}{\ding{51}} \\
\textbf{Do you require MFA to log into computers?} & \textcolor{red}{\ding{55}} \\
Do you require MFA to access sensitive data systems? & \textcolor{green}{\ding{51}} \\
Does your organization have an employee acceptable use policy? & \textcolor{green}{\ding{51}} \\
Does your organization do security awareness training for new employees? & \textcolor{green}{\ding{51}} \\
\textbf{Does your organization do security awareness training for all employees at least once per year?} & \textcolor{red}{\ding{55}} \\ \bottomrule
\end{tabular}
\end{table}

\subsection*{Analysis of Control Gaps}
\begin{itemize}
    \item \textbf{MFA for Computer Logins:} The absence of MFA on endpoint devices is a critical vulnerability. If an employee's credentials are stolen (e.g., through phishing), an attacker could gain direct access to their computer and the corporate network without needing a second authentication factor.
    \item \textbf{Annual Security Awareness Training:} Security is an ongoing process. The lack of annual refresher training for all employees increases the likelihood that they will fall victim to evolving threats like sophisticated phishing and social engineering attacks. This undermines the security posture of the entire organization.
\end{itemize}

% --- 4. Technical Scan Results ---
\section{Technical Scan Results}

An external network scan was performed to identify open ports and exposed services.

\begin{itemize}
    \item \textbf{Target IP Address:} \texttt{[Target IP]}
    \item \textbf{Scan Date:} Not provided in scan data. Report generated on \today.
\end{itemize}

\begin{table}[h!]
\centering
\caption{Open Ports Detected}
\begin{tabular}{@{}llll@{}}
\toprule
\textbf{Port} & \textbf{State} & \textbf{Service} & \textbf{Product / Version} \\ \midrule
22/tcp & open & ssh & Not Fingerprinted \\ \bottomrule
\end{tabular}
\end{table}

\subsection*{Analysis of Technical Findings}
The scan identified that port \textbf{22 (SSH)} is open to the public internet. SSH is a common protocol for remote administration. When exposed externally, it becomes a primary target for automated brute-force attacks, where attackers attempt to guess credentials to gain unauthorized access. Without detailed version information, it is also not possible to determine if the running SSH service is vulnerable to known exploits.

% --- 5. Correlated Risk Assessment ---
\section{Correlated Risk Assessment}

This section synthesizes the findings from the security control review and the technical scan to provide a consolidated list of identified risks. No pre-existing vulnerabilities were reported.

\begin{table}[h!]
\centering
\caption{Summary of Identified Risks}
\begin{tabular}{@{}p{0.1\textwidth}p{0.6\textwidth}l@{}}
\toprule
\textbf{Risk ID} & \textbf{Description} & \textbf{Severity} \\ \midrule
RISK-001 & Lack of MFA on endpoint computer logins allows for trivial account takeover if credentials are compromised. & \textbf{Critical} \\
\addlinespace
RISK-002 & An externally exposed SSH service on port 22 is a target for brute-force attacks and potential unauthorized remote access. & \textbf{High} \\
\addlinespace
RISK-003 & Lack of mandatory annual security awareness training for all staff increases susceptibility to phishing and social engineering attacks. & \textbf{High} \\ \bottomrule
\end{tabular}
\end{table}

% --- 6. Recommendations ---
\section{Recommendations}

The following actions are recommended to mitigate the identified risks and strengthen the overall security posture of \textbf{[Organization Name]}.

\subsection*{RISK-001: Lack of MFA on Endpoints (Critical)}
\begin{itemize}
    \item \textbf{Immediate Action:} Implement and enforce Multi-Factor Authentication (MFA) for all user logins to workstations, laptops, and servers. This is the single most effective control to prevent unauthorized access from compromised credentials.
    \item \textbf{Supporting Action:} Prioritize deployment for privileged users (administrators) and executives.
\end{itemize}

\subsection*{RISK-002: Exposed SSH Service (High)}
\begin{itemize}
    \item \textbf{Immediate Action:} Restrict access to port 22 using a firewall. Access should be limited to specific, trusted IP addresses (e.g., office or administrator IPs) via an allow-list. If remote access is required from dynamic locations, consider implementing a Virtual Private Network (VPN) as the sole entry point.
    \item \textbf{Supporting Actions:}
        \begin{itemize}
            \item Ensure the SSH service is fully patched and updated to the latest stable version.
            \item Enforce the use of strong, complex passwords or, preferably, public key authentication.
            \item Implement an intrusion detection/prevention tool like \texttt{fail2ban} to automatically block IPs that exhibit brute-force behavior.
        \end{itemize}
\end{itemize}

\subsection*{RISK-003: Lack of Annual Security Training (High)}
\begin{itemize}
    \item \textbf{Immediate Action:} Develop and deploy a mandatory annual security awareness training program for all employees, including management.
    \item \textbf{Supporting Actions:}
        \begin{itemize}
            \item The training should cover current threats such as phishing, ransomware, social engineering, and proper data handling.
            \item Conduct periodic phishing simulation campaigns to test and reinforce the training's effectiveness.
        \end{itemize}
\end{itemize}

\end{document}
```