```latex
\documentclass[12pt]{article}

% --- PACKAGES ---
\usepackage[margin=1in]{geometry}
\usepackage{pifont}          % For checkmarks and crosses (\ding{51}, \ding{55})
\usepackage{booktabs}        % For professional-looking tables
\usepackage{hyperref}        % For hyperlinks and document metadata
\usepackage{url}             % For formatting URLs
\usepackage{seqsplit}        % To split long strings in text
\usepackage{graphicx}
\usepackage{xcolor}
\usepackage{array}           % For table column formatting

% --- DOCUMENT METADATA ---
\hypersetup{
    colorlinks=true,
    linkcolor=blue,
    urlcolor=cyan,
    pdftitle={Cybersecurity Posture Assessment Report},
    pdfauthor={Cybersecurity Analysis Division},
    pdfsubject={Security Assessment}
}

% --- CUSTOM COMMANDS ---
\newcommand{\yes}{\ding{51}}
\newcommand{\no}{\textcolor{red}{\ding{55}}}
\newcommand{\orgname}{\textbf{[Organization Name]}}
\newcommand{\clientip}{\texttt{[Client IP]}}
\newcommand{\targetip}{\texttt{[Target IP]}}
\newcommand{\domain}{\texttt{[Domain]}}

% --- DOCUMENT START ---
\begin{document}

% --- TITLE PAGE ---
\begin{titlepage}
    \centering
    \vspace*{1cm}
    \Huge\textbf{Cybersecurity Posture Assessment Report}
    \vspace{1.5cm}
    \Large
    \textbf{Prepared for:}\\
    \vspace{0.5cm}
    \orgname
    \vspace{2cm}
    \textbf{Date of Report:}\\
    \vspace{0.5cm}
    \today
    \vfill
    \large
    \textbf{Analysis Division}\\
    \textit{CONFIDENTIAL}
\end{titlepage}

\tableofcontents
\newpage

% --- 1. EXECUTIVE SUMMARY ---
\section{Executive Summary}

This report provides a comprehensive cybersecurity assessment for \orgname, based on an analysis of organizational security controls, an external network scan, and a review of previously identified risks. The assessment synthesizes these data points to provide a holistic view of the current security posture and offers actionable recommendations for risk mitigation.

\paragraph{Key Findings:} The assessment identified two significant gaps in administrative security controls that present a high risk to the organization:
\begin{itemize}
    \item \textbf{Critical Risk:} Multi-Factor Authentication (MFA) is not enforced for accessing sensitive data systems. This leaves critical assets vulnerable to unauthorized access via compromised credentials.
    \item \textbf{High Risk:} New employees do not receive mandatory security awareness training during onboarding. This increases the likelihood of human error leading to security incidents.
\end{itemize}

\paragraph{Technical Posture:} The external network scan of the designated target IP address (\targetip) revealed a positive security posture. No open ports were discovered, including Port 80, which was previously listed as a vulnerability. This suggests that the "Unencrypted Web Server" risk may have been remediated.

\paragraph{Recommendations:} We strongly advise prioritizing the implementation of MFA on all sensitive systems and integrating security awareness training into the new employee onboarding process. Detailed recommendations are provided in Section \ref{sec:recommendations}.

% --- 2. ORGANIZATIONAL INFORMATION ---
\section{Organizational Information}
This section details the information provided for the assessment.
\begin{center}
\begin{tabular}{ll}
\toprule
\textbf{Attribute} & \textbf{Value} \\
\midrule
Organization Name & \orgname \\
Primary Domain & \domain \\
External IP Scanned & \clientip \\
\bottomrule
\end{tabular}
\end{center}

% --- 3. SECURITY CONTROL REVIEW ---
\section{Security Control Review}
The following table summarizes the organization's responses to a security controls questionnaire. Items marked with a red 'X' (\no) indicate a deviation from security best practices and are addressed in the Risk Assessment section.

\begin{center}
\begin{tabular}{>{\raggedright\arraybackslash}p{10cm} c}
\toprule
\textbf{Control Question} & \textbf{Response} \\
\midrule
Do you require MFA to access email? & \yes \\
Do you require MFA to log into computers? & \yes \\
Do you require MFA to access sensitive data systems? & \no \\
Does your organization have an employee acceptable use policy? & \yes \\
Does your organization do security awareness training for new employees? & \no \\
Does your organization do security awareness training for all employees at least once per year? & \yes \\
\bottomrule
\end{tabular}
\end{center}

% --- 4. TECHNICAL SCAN RESULTS ---
\section{Technical Scan Results}
An external network vulnerability scan was conducted using Nmap to identify accessible services on the organization's public-facing infrastructure.

\begin{itemize}
    \item \textbf{Target IP Address:} \targetip
    \item \textbf{Scan Date:} Scan conducted prior to \today
    \item \textbf{Summary:} The target host was responsive, but all scanned ports were found to be in a 'closed' state. This indicates a strong network perimeter configuration, as no services were exposed to the public internet during the scan.
\end{itemize}

\begin{center}
\begin{tabular}{lllll}
\toprule
\textbf{Port} & \textbf{State} & \textbf{Service} & \textbf{Product} & \textbf{Version} \\
\midrule
80/tcp & closed & http & - & - \\
\bottomrule
\end{tabular}
\end{center}

% --- 5. RISK ASSESSMENT ---
\section{Risk Assessment}
This section correlates findings from the security control review, technical scan, and pre-existing risk data. Each identified risk is assigned a severity level to aid in prioritization.

\begin{center}
\begin{tabular}{lp{6cm}l}
\toprule
\textbf{Risk ID} & \textbf{Risk Name \& Overview} & \textbf{Severity} \\
\midrule
\textbf{RISK-001} & \textbf{Lack of MFA on Sensitive Systems} \newline \textit{Failure to enforce MFA on systems containing sensitive data exposes the organization to significant risk of data breach from credential theft.} & \textbf{Critical} \\
\addlinespace
\textbf{RISK-002} & \textbf{Inadequate New Hire Security Training} \newline \textit{Without initial security training, new employees are more susceptible to social engineering attacks and may mishandle sensitive data, creating an ongoing risk.} & \textbf{High} \\
\addlinespace
\textbf{RISK-003} & \textbf{Unencrypted Web Server (Legacy Risk)} \newline \textit{A pre-existing risk noted an open Port 80. Our current scan found this port to be closed, suggesting remediation. This risk should be formally closed if confirmed.} & \textbf{Medium} \\
\bottomrule
\end{tabular}
\end{center}

% --- 6. RECOMMENDATIONS ---
\section{Recommendations}
\label{sec:recommendations}
The following actions are recommended to mitigate the identified risks and improve the overall security posture of \orgname.

\subsection{RISK-001: Implement MFA on Sensitive Systems (Critical)}
\begin{itemize}
    \item \textbf{Immediate Action:} Inventory all systems, applications, and databases that store or process sensitive information. Prioritize them based on data criticality.
    \item \textbf{Long-Term Strategy:} Procure and deploy an MFA solution that integrates with the identified sensitive systems. Enforce its use for all privileged and non-privileged accounts as a matter of policy.
\end{itemize}

\subsection{RISK-002: Establish New Hire Security Training (High)}
\begin{itemize}
    \item \textbf{Immediate Action:} Develop or acquire a foundational security awareness training module. This module should cover phishing, acceptable use, password hygiene, and data handling policies.
    \item \textbf{Long-Term Strategy:} Integrate the mandatory completion of this training module into the formal HR onboarding process for all new employees and contractors before they are granted system access.
\end{itemize}

\subsection{RISK-003: Validate and Close Legacy Risk (Medium)}
\begin{itemize}
    \item \textbf{Immediate Action:} Confirm with the network and system administration teams that the service previously running on Port 80 has been intentionally decommissioned or migrated to a secure protocol (HTTPS on Port 443).
    \item \textbf{Long-Term Strategy:} Once confirmed, formally update the organization's risk register to mark this vulnerability as "Remediated" and document the date and method of resolution.
\end{itemize}

% --- DOCUMENT END ---
\end{document}
```