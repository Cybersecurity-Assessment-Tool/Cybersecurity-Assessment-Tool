```latex
\documentclass[12pt]{article}

% Preamble: Required Packages
\usepackage[margin=1in]{geometry}
\usepackage{pifont} % For checkmarks and crosses
\usepackage{booktabs} % For professional tables
\usepackage{hyperref} % For clickable links and table of contents
\usepackage{url} % For formatting URLs
\usepackage{seqsplit} % To split long strings without breaking
\usepackage{xcolor} % For colors in tables
\usepackage{graphicx} % For logo if needed

% Define colors for risk levels
\definecolor{critical}{HTML}{93000a}
\definecolor{high}{HTML}{d35400}
\definecolor{medium}{HTML}{f1c40f}
\definecolor{low}{HTML}{27ae60}

% Hyperref Setup
\hypersetup{
    colorlinks=true,
    linkcolor=blue,
    filecolor=magenta,      
    urlcolor=cyan,
    pdftitle={Cybersecurity Assessment Report},
    pdfpagemode=FullScreen,
}

% --- Document Start ---
\begin{document}

% --- Title Page ---
\begin{titlepage}
    \centering
    \vspace*{1cm}
    \Huge\textbf{Cybersecurity Assessment Report}
    \vspace{1.5cm}
    \Large
    Prepared for: \\
    \vspace{0.5cm}
    \textbf{[Organization Name]}
    \vspace{2cm}
    \large
    \textbf{Date:} \today \\
    \vspace{0.5cm}
    \textbf{Author:} Cybersecurity Analyst
    \vfill
    \textit{This report contains sensitive information and is intended solely for the use of the recipient organization. Distribution is strictly prohibited.}
\end{titlepage}

\tableofcontents
\newpage

% --- Section 1: Executive Summary ---
\section{Executive Summary}
This report provides a comprehensive cybersecurity assessment for \textbf{[Organization Name]}, based on an analysis of network scan data, organizational security controls, and pre-existing risk documentation. The assessment was conducted to identify key vulnerabilities and provide actionable recommendations to enhance the organization's security posture.

The analysis revealed several critical and high-risk gaps in the current security framework. Key findings include:
\begin{itemize}
    \item \textbf{Critical Gap in Access Control:} Multi-Factor Authentication (MFA) is not enforced for accessing sensitive data systems. This represents a significant vulnerability, as a single compromised credential could lead to a major data breach.
    \item \textbf{High-Risk Policy Deficiencies:} The organization lacks a formal Employee Acceptable Use Policy and does not provide mandatory annual security awareness training for all staff. These gaps increase the likelihood of human error leading to security incidents.
    \item \textbf{Exposed Management Service:} The external network scan identified an open SSH port (22), which is a common target for automated brute-force attacks.
\end{itemize}

The overall security posture requires immediate attention. The recommendations outlined in this report are designed to address these findings systematically, starting with the most critical vulnerabilities. We strongly advise prioritizing the implementation of MFA on all sensitive systems and developing foundational security policies and training programs.

% --- Section 2: Organizational Information ---
\section{Organizational Information}
The following details were used as the basis for this assessment. Due to the anonymized nature of the provided data, placeholders have been used where necessary.

\begin{table}[h!]
\centering
\begin{tabular}{@{}ll@{}}
\toprule
\textbf{Attribute} & \textbf{Value} \\ \midrule
Organization Name & \textbf{[Organization Name]} \\
Primary Domain & \texttt{[Domain]} \\
External IP Address & \texttt{[Client IP]} \\ \bottomrule
\end{tabular}
\caption{Client Organizational Details}
\end{table}

% --- Section 3: Security Control Review ---
\section{Security Control Review (Questionnaire Analysis)}
A review of the organization's security controls was conducted via a questionnaire. The responses highlight significant gaps in policy and access control measures. A "No" response indicates a deviation from security best practices and a potential risk.

\begin{table}[h!]
\centering
\begin{tabular}{@{}p{0.6\textwidth}cc@{}}
\toprule
\textbf{Control Question} & \textbf{Response} & \textbf{Status} \\ \midrule
Do you require MFA to access email? & Yes & \ding{51} \\
Do you require MFA to log into computers? & Yes & \ding{51} \\
\textbf{Do you require MFA to access sensitive data systems?} & \textbf{No} & \textbf{\color{critical}\ding{55}} \\
\textbf{Does your organization have an employee acceptable use policy?} & \textbf{No} & \textbf{\color{high}\ding{55}} \\
Does your organization do security awareness training for new employees? & Yes & \ding{51} \\
\textbf{Does your organization do security awareness training for all employees at least once per year?} & \textbf{No} & \textbf{\color{high}\ding{55}} \\
\bottomrule
\end{tabular}
\caption{Security Control Questionnaire Results}
\end{table}

\subsection{Analysis of Control Gaps}
\begin{itemize}
    \item \textbf{MFA on Sensitive Systems:} The absence of MFA on systems containing sensitive data is a critical vulnerability. It significantly lowers the barrier for an attacker to access crown jewel assets after compromising a user's credentials.
    \item \textbf{Acceptable Use Policy (AUP):} Without a formal AUP, there are no established rules for employees regarding the use of company assets. This can lead to unintentional misuse, data leakage, and a weakened legal position in the event of an insider incident.
    \item \textbf{Annual Security Training:} While training new hires is a good start, the threat landscape evolves continuously. A lack of annual refresher training for all employees means that their ability to recognize and respond to modern threats, such as sophisticated phishing attacks, diminishes over time.
\end{itemize}

% --- Section 4: Technical Scan Results ---
\section{Technical Scan Results}
An external network scan was performed to identify open ports and exposed services on the organization's perimeter.

\begin{itemize}
    \item \textbf{Target IP Address:} \texttt{[Target IP]}
    \item \textbf{Scan Date:} Scan date not provided in source data.
\end{itemize}

The scan identified the following open port:
\begin{table}[h!]
\centering
\begin{tabular}{@{}llll@{}}
\toprule
\textbf{Port} & \textbf{State} & \textbf{Service (Inferred)} & \textbf{Product / Version} \\ \midrule
22/tcp & open & SSH (Secure Shell) & Not Available \\ \bottomrule
\end{tabular}
\caption{Open Ports Detected on \texttt{[Target IP]}}
\end{table}

\subsection{Analysis of Technical Findings}
The presence of an open SSH port (22) indicates that a remote management service is exposed to the public internet. While necessary for administration, SSH is a primary target for automated brute-force and credential-stuffing attacks. Without information on the version or configuration, it is not possible to determine if it is vulnerable to specific exploits, but its exposure alone constitutes a security risk. Best practices dictate that access to such management ports should be strictly controlled.

% --- Section 5: Consolidated Risk Assessment ---
\section{Consolidated Risk Assessment}
The following table synthesizes findings from the security control review and the technical scan. No pre-existing vulnerabilities were documented in the provided data. Each risk has been assigned a severity level to aid in prioritization.

\begin{table}[h!]
\centering
\resizebox{\textwidth}{!}{%
\begin{tabular}{@{}llll@{}}
\toprule
\textbf{ID} & \textbf{Risk Name} & \textbf{Description} & \textbf{Severity} \\ \midrule
RISK-001 & Lack of MFA on Sensitive Systems & The absence of MFA on critical systems allows an attacker with valid credentials to gain unauthorized access to sensitive data. & \colorbox{critical}{\color{white}\textbf{ CRITICAL }} \\
\addlinespace
RISK-002 & Inadequate Security Policies & The lack of a formal Acceptable Use Policy (AUP) creates ambiguity and increases the risk of insider threat and accidental data exposure. & \colorbox{high}{\color{white}\textbf{ HIGH }} \\
\addlinespace
RISK-003 & Insufficient Security Training & The absence of mandatory, annual security awareness training for all employees weakens the human firewall against phishing and social engineering. & \colorbox{high}{\color{white}\textbf{ HIGH }} \\
\addlinespace
RISK-004 & Exposed SSH Management Port & Port 22 (SSH) is open to the internet, making it a target for brute-force attacks that could lead to unauthorized server access. & \colorbox{medium}{\color{black}\textbf{ MEDIUM }} \\
\bottomrule
\end{tabular}%
}
\caption{Summary of Identified Risks}
\end{table}

% --- Section 6: Recommendations ---
\section{Recommendations}
The following actions are recommended to mitigate the identified risks and improve the overall security posture of \textbf{[Organization Name]}.

\subsection{Immediate Actions (1-30 Days)}
\begin{description}
    \item[For RISK-001 (Critical):] \textbf{Implement MFA on Sensitive Systems.}
    \begin{itemize}
        \item Immediately deploy and enforce MFA for all user accounts (including administrative and service accounts) that have access to systems storing or processing sensitive data.
        \item Prioritize systems containing financial records, customer PII, or intellectual property.
    \end{itemize}

    \item[For RISK-004 (Medium):] \textbf{Restrict SSH Access.}
    \begin{itemize}
        \item Implement a firewall rule to restrict access to Port 22 to only known, trusted IP addresses (e.g., corporate offices, administrator locations).
        \item If remote access from dynamic IPs is required, implement a Virtual Private Network (VPN) and place the SSH service behind it.
        \item Ensure SSH is configured to disallow root login and password-based authentication, enforcing the use of strong cryptographic keys instead.
    \end{itemize}
\end{description}

\subsection{Strategic Actions (30-90 Days)}
\begin{description}
    \item[For RISK-002 (High):] \textbf{Develop and Implement an Acceptable Use Policy (AUP).}
    \begin{itemize}
        \item Draft a formal AUP that clearly defines the rules for using company networks, devices, and data.
        \item Communicate the policy to all employees and require them to formally acknowledge their understanding and agreement.
    \end{itemize}

    \item[For RISK-003 (High):] \textbf{Establish a Security Awareness Training Program.}
    \begin{itemize}
        \item Procure or develop a security awareness training module that covers key topics like phishing, password hygiene, and data handling.
        \item Make this training mandatory for all employees on an annual basis, and track completion.
        \item Supplement annual training with periodic phishing simulations to measure effectiveness.
    \end{itemize}
\end{description}

\end{document}
```