```latex
\documentclass[12pt]{article}

% --- PACKAGES ---
\usepackage[margin=1in]{geometry}
\usepackage{pifont} % For checkmarks and crosses
\usepackage{booktabs} % For professional tables
\usepackage{hyperref} % For clickable links
\usepackage{url}      % For URL formatting
\usepackage{seqsplit} % For splitting long strings
\usepackage{graphicx} % For logo (placeholder)
\usepackage{xcolor}   % For colors

% --- DOCUMENT METADATA ---
\title{Cybersecurity Posture Assessment Report}
\author{Cybersecurity Analysis Division}
\date{\today}

% --- HYPERREF SETUP ---
\hypersetup{
    colorlinks=true,
    linkcolor=blue,
    filecolor=magenta,      
    urlcolor=cyan,
    pdftitle={Cybersecurity Posture Assessment Report},
    pdfpagemode=FullScreen,
}

% --- CUSTOM COMMANDS ---
\newcommand{\yes}{\ding{51}}
\newcommand{\no}{\ding{55}}

\begin{document}

\maketitle
\thispagestyle{empty}
\newpage

\tableofcontents
\newpage

% ===================================================================
% 1. EXECUTIVE SUMMARY
% ===================================================================
\section{Executive Summary}

This report details the findings of a cybersecurity posture assessment for \textbf{[Organization Name]}. The analysis is based on a combination of a technical network scan, a review of existing risk documentation, and an organizational security controls questionnaire.

The assessment identified several critical and high-risk security gaps. The most significant vulnerabilities stem from a lack of foundational security controls, specifically the absence of Multi-Factor Authentication (MFA) across all key systems (email, computers, and sensitive data). This is compounded by a lack of security awareness training for both new and existing employees. These procedural gaps create a high risk of account compromise through phishing and other social engineering attacks.

On a technical level, the external network scan of the target IP address (\texttt{[Target IP]}) revealed a minimal attack surface, with no open ports detected. This indicates that a previously identified risk concerning an unencrypted web server on Port 80 has been mitigated, as our scan found this port to be closed.

Immediate action is required to address the identified policy and procedure gaps. Prioritized recommendations are provided to systematically reduce the organization's risk exposure.

% ===================================================================
% 2. ORGANIZATIONAL INFORMATION
% ===================================================================
\section{Organizational Information}

This assessment pertains to the following entity and associated assets. The information below has been used as the basis for this report.

\begin{tabular}{@{}ll}
\toprule
\textbf{Attribute} & \textbf{Value} \\
\midrule
Organization Name & \textbf{[Organization Name]} \\
Primary Domain & \texttt{[Domain]} \\
Scanned External IP & \texttt{[Client IP]} \\
\bottomrule
\end{tabular}

% ===================================================================
% 3. SECURITY CONTROL REVIEW
% ===================================================================
\section{Security Control Review}

A security questionnaire was completed to evaluate the organization's current policies and procedures. The responses are summarized below. Items marked with a red cross (\no) indicate significant control gaps that increase organizational risk.

\begin{table}[h!]
\centering
\begin{tabular}{@{}lc}
\toprule
\textbf{Control Question} & \textbf{Response} \\
\midrule
Do you require MFA to access email? & \textcolor{red}{\no} \\
Do you require MFA to log into computers? & \textcolor{red}{\no} \\
Do you require MFA to access sensitive data systems? & \textcolor{red}{\no} \\
Does your organization have an employee acceptable use policy? & \textcolor{green}{\yes} \\
Does your organization do security awareness training for new employees? & \textcolor{red}{\no} \\
Does your organization do security awareness training for all employees annually? & \textcolor{red}{\no} \\
\bottomrule
\end{tabular}
\caption{Organizational Security Controls Questionnaire Results}
\end{label{tab:controls}
\end{table}

\subsection*{Analysis of Control Gaps}
The questionnaire reveals critical deficiencies in identity and access management and employee security training.
\begin{itemize}
    \item \textbf{Lack of Multi-Factor Authentication (MFA):} The absence of MFA for email, computer logins, and sensitive data access is a critical vulnerability. It means that a compromised password is all an attacker needs to gain significant access to the organization's digital assets.
    \item \textbf{Lack of Security Awareness Training:} Without initial and ongoing training, employees are significantly more likely to fall victim to phishing attacks, inadvertently install malware, or mishandle sensitive data. This represents a high-risk gap in the organization's human firewall.
\end{itemize}

% ===================================================================
% 4. TECHNICAL SCAN RESULTS
% ===================================================================
\section{Technical Scan Results}

A network scan was performed using Nmap on \today\ to identify the external attack surface of the designated target system.

\begin{itemize}
    \item \textbf{Target IP Address:} \texttt{[Target IP]}
    \item \textbf{Scan Status:} Host was detected as "up".
\end{itemize}

The scan results indicate a very limited external exposure, which is a positive security finding. No open ports were discovered.

\begin{table}[h!]
\centering
\begin{tabular}{@{}llll}
\toprule
\textbf{Port} & \textbf{State} & \textbf{Service} & \textbf{Product / Version} \\
\midrule
80/tcp & closed & http & N/A \\
\bottomrule
\end{tabular}
\caption{Port Scan Results for \texttt{[Target IP]}}
\label{tab:scanresults}
\end{table}

\subsection*{Analysis of Technical Findings}
The scan confirms that Port 80 is closed. This directly contradicts a pre-existing risk entry ("Unencrypted Web Server") which stated the port was open. This suggests that the risk has been successfully remediated or was a false positive. The lack of any open ports on the scanned IP address is an excellent security posture from a network perimeter perspective.

% ===================================================================
% 5. RISK ASSESSMENT SUMMARY
% ===================================================================
\section{Risk Assessment Summary}

By correlating the security control gaps, technical findings, and pre-existing risk data, we have identified the following key risks to the organization.

\begin{table}[h!]
\centering
\begin{tabular}{@{}p{0.3\linewidth}p{0.5\linewidth}l}
\toprule
\textbf{Risk Name} & \textbf{Description} & \textbf{Severity} \\
\midrule
\textbf{Absence of Multi-Factor Authentication (MFA)} & The lack of MFA on email, endpoints, and sensitive systems makes the organization highly vulnerable to account takeover via credential theft or phishing. & \textbf{Critical} \\
\addlinespace
\textbf{Inadequate Security Awareness Program} & Employees are not trained to recognize or respond to security threats. This elevates the likelihood of successful phishing, malware infection, and social engineering attacks. & \textbf{High} \\
\addlinespace
\textbf{Previously Identified Web Server Risk Mitigated} & A pre-existing risk noted an open Port 80. Our scan confirmed this port is now closed, mitigating the immediate threat of unencrypted web traffic from this host. & Informational \\
\bottomrule
\end{tabular}
\caption{Synthesized Risk Register}
\label{tab:risks}
\end{table}

% ===================================================================
% 6. RECOMMENDATIONS
% ===================================================================
\section{Recommendations}

The following actions are recommended to mitigate the identified risks. They are prioritized based on severity and potential impact.

\subsection*{Priority 1: Implement Multi-Factor Authentication (Critical)}
\begin{itemize}
    \item \textbf{Action:} Deploy a robust MFA solution across the organization.
    \item \textbf{Details:}
    \begin{enumerate}
        \item Prioritize enabling MFA for all email accounts (e.g., via Office 365 or Google Workspace security settings).
        \item Enable MFA for access to all systems containing sensitive or critical data.
        \item Phase in MFA for all standard employee computer logins.
    \end{enumerate}
    \item \textbf{Impact:} Drastically reduces the risk of unauthorized access from compromised credentials.
\end{itemize}

\subsection*{Priority 2: Establish a Security Awareness Program (High)}
\begin{itemize}
    \item \textbf{Action:} Develop and implement a mandatory security awareness training program.
    \item \textbf{Details:}
    \begin{enumerate}
        \item Institute a training module for all new employees as part of the onboarding process.
        \item Conduct annual refresher training for all staff covering topics like phishing, password security, and acceptable use.
        \item Consider periodic phishing simulation campaigns to test and reinforce employee learning.
    \end{enumerate}
    \item \textbf{Impact:} Strengthens the "human firewall" and reduces the likelihood of human error leading to a security breach.
\end{itemize}

\subsection*{Priority 3: Maintain Network Security Posture (Good Practice)}
\begin{itemize}
    \item \textbf{Action:} Continue to monitor and validate the external network perimeter.
    \item \textbf{Details:}
    \begin{enumerate}
        \item Formally document the closure of Port 80 and ensure it was an intentional change.
        \item Implement a regular schedule for external vulnerability scanning to ensure no new, unauthorized services are exposed to the internet.
    \end{enumerate}
    \item \textbf{Impact:} Ensures the current positive network security posture is maintained and prevents future misconfigurations.
\end{itemize}

\end{document}
```