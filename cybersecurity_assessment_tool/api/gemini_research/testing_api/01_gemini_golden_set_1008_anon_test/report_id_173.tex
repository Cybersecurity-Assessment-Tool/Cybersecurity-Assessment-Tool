```latex
\documentclass[12pt]{article}

% --- PACKAGES ---
\usepackage[margin=1in]{geometry}
\usepackage{pifont} % For checkmarks and crosses
\usepackage{booktabs} % For professional tables
\usepackage{hyperref} % For clickable links
\usepackage{url} % For URL formatting
\usepackage{seqsplit} % To split long strings in texttt
\usepackage[table]{xcolor} % For coloring table cells

% --- DOCUMENT SETUP ---
\hypersetup{
    colorlinks=true,
    linkcolor=blue,
    filecolor=magenta,      
    urlcolor=cyan,
    pdftitle={Cybersecurity Assessment Report},
    pdfpagemode=FullScreen,
}

% Define severity colors
\definecolor{sev_critical}{HTML}{940000}
\definecolor{sev_high}{HTML}{D14000}
\definecolor{sev_medium}{HTML}{E8A100}

% --- DOCUMENT START ---
\begin{document}

% --- TITLE PAGE ---
\begin{titlepage}
    \centering
    \vspace*{\fill}
    \huge\textbf{Cybersecurity Assessment Report}
    \vspace{1.5cm}
    \Large
    \textbf{Prepared for:} \\
    \vspace{0.5cm}
    \textbf{[Organization Name]}
    \vspace{2cm}
    \large
    \textbf{Date of Report:} \today \\
    \vspace{0.5cm}
    \textbf{Author:} Cybersecurity Analyst
    \vfill
\end{titlepage}

\tableofcontents
\newpage

% --- EXECUTIVE SUMMARY ---
\section{Executive Summary}
This report details the findings of a cybersecurity assessment conducted for \textbf{[Organization Name]}. The assessment combined a review of organizational security controls, an external network scan, and an analysis of pre-existing risk data.

The analysis revealed several critical-priority risks that require immediate attention. The most significant findings are:
\begin{itemize}
    \item \textbf{Publicly Exposed Remote Desktop Protocol (RDP):} The external network scan identified a server with RDP (Port 3389) open to the public internet. This is a highly common attack vector used by threat actors, particularly for deploying ransomware.
    \item \textbf{Systemic Lack of Multi-Factor Authentication (MFA):} The organization does not enforce MFA for accessing email, logging into computers, or accessing sensitive data systems. This significantly increases the risk of unauthorized access through credential compromise (e.g., phishing or password spraying).
    \item \textbf{Absence of an Acceptable Use Policy:} The lack of a formal policy creates ambiguity for employees regarding the secure use of company assets, increasing the likelihood of unintentional security incidents.
\end{itemize}

The combination of an exposed RDP service and the absence of MFA creates a severe and immediate risk of a full network compromise. This report provides a detailed breakdown of these findings and offers actionable recommendations to mitigate the identified risks and improve the overall security posture.

% --- ORGANIZATIONAL INFORMATION ---
\section{Organizational Information}
The following information was used as the basis for this assessment.
\begin{table}[h!]
\centering
\begin{tabular}{@{}ll@{}}
\toprule
\textbf{Identifier} & \textbf{Value} \\ \midrule
Organization Name   & \textbf{[Organization Name]} \\
Primary Domain      & \seqsplit{\texttt{[Domain]}} \\
External IP Scanned & \seqsplit{\texttt{[Client IP]}} \\ \bottomrule
\end{tabular}
\caption{Client Organizational Details.}
\end{table}

% --- SECURITY CONTROL REVIEW ---
\section{Security Control Review (Questionnaire Analysis)}
A review of the organization's security controls was conducted via a questionnaire. The responses indicate significant gaps in foundational security practices. A "No" response highlights a missing control that increases organizational risk.

\begin{table}[h!]
\centering
\begin{tabular}{@{}p{0.5\textwidth}cp{0.3\textwidth}@{}}
\toprule
\textbf{Control Question} & \textbf{Response} & \textbf{Analyst Assessment} \\ \midrule
Do you require MFA to access email? & \ding{55} & \textbf{Critical Gap.} Email is a primary target for phishing and account takeover. \\
\addlinespace
Do you require MFA to log into computers? & \ding{55} & \textbf{Critical Gap.} Lack of MFA allows lateral movement if credentials are stolen. \\
\addlinespace
Do you require MFA to access sensitive data systems? & \ding{55} & \textbf{Critical Gap.} The organization's most valuable data is not adequately protected. \\
\addlinespace
Does your organization have an employee acceptable use policy? & \ding{55} & \textbf{High Risk.} No clear guidelines for employees on security responsibilities. \\
\addlinespace
Does your organization do security awareness training for new employees? & \ding{51} & Good Practice. Establishes a security baseline for new hires. \\
\addlinespace
Does your organization do security awareness training for all employees at least once per year? & \ding{51} & Good Practice. Reinforces security concepts and addresses new threats. \\ \bottomrule
\end{tabular}
\caption{Security Control Questionnaire Results.}
\end{table}

% --- TECHNICAL SCAN RESULTS ---
\section{Technical Scan Results}
An external network scan was performed on the target IP address to identify open ports and exposed services.

\begin{itemize}
    \item \textbf{Target IP Address:} \seqsplit{\texttt{[Target IP]}}
    \item \textbf{Scan Date:} Scan data provided on \today
\end{itemize}

\begin{table}[h!]
\centering
\begin{tabular}{@{}llll@{}}
\toprule
\textbf{Port} & \textbf{State} & \textbf{Service} & \textbf{Product / Version} \\ \midrule
3389/tcp & open & ms-wbt-server & Not Determined \\ \bottomrule
\end{tabular}
\caption{Open Ports Identified on \seqsplit{\texttt{[Target IP]}}.}
\end{table}

\subsection{Analysis of Findings}
The scan identified that TCP port 3389 is open. This port is used for Microsoft's Remote Desktop Protocol (RDP). Exposing RDP directly to the public internet is extremely dangerous and is a leading cause of security breaches. It allows attackers to attempt brute-force or password-spraying attacks to gain direct control over the server. This finding confirms the pre-existing risk identified in \texttt{Input\_3\_Current\_Risks\_JSON}.

% --- CORRELATED RISK ASSESSMENT ---
\section{Correlated Risk Assessment}
This section synthesizes the findings from the security control review, the technical scan, and pre-existing risk data into a consolidated list of key risks facing the organization.

\begin{table}[h!]
\centering
\rowcolors{2}{gray!10}{white}
\begin{tabular}{@{}p{0.1\textwidth}p{0.2\textwidth}p{0.45\textwidth}p{0.15\textwidth}@{}}
\toprule
\textbf{Risk ID} & \textbf{Risk Name} & \textbf{Description} & \textbf{Severity} \\ \midrule
RISK-001 & Publicly Exposed RDP Service & The RDP service on port 3389 is exposed on \seqsplit{\texttt{[Target IP]}}. This is a primary vector for ransomware attacks and unauthorized remote access. This risk is severely amplified by the lack of MFA. & \cellcolor{sev_critical}\color{white} \textbf{Critical (9.0)} \\
\addlinespace
RISK-002 & Lack of Multi-Factor Authentication (MFA) & MFA is not enforced for email, computer logins, or sensitive systems. This makes user accounts highly vulnerable to compromise via stolen credentials. & \cellcolor{sev_critical}\color{white} \textbf{Critical} \\
\addlinespace
RISK-003 & Missing Acceptable Use Policy (AUP) & The organization lacks a formal policy to govern the use of IT assets. This leads to inconsistent security practices and a lack of accountability for employee actions. & \cellcolor{sev_high}\color{white} \textbf{High} \\ \bottomrule
\end{tabular}
\caption{Summary of Identified Risks.}
\end{table}

% --- RECOMMENDATIONS ---
\section{Recommendations}
The following actions are recommended to mitigate the identified risks. Recommendations are prioritized based on severity and potential impact.

\subsection{RISK-001: Publicly Exposed RDP Service (Critical)}
\begin{itemize}
    \item \textbf{Immediate Action (Remediation):} Immediately close port 3389 on the external firewall for the host at \seqsplit{\texttt{[Target IP]}}. If remote access is business-critical, restrict access to a whitelist of known, trusted IP addresses as a temporary measure.
    \item \textbf{Long-Term Solution (Strategy):} Implement a Virtual Private Network (VPN) solution for all remote access. A VPN provides a secure, encrypted tunnel and ensures that services like RDP are not exposed to the public internet.
\end{itemize}

\subsection{RISK-002: Lack of Multi-Factor Authentication (Critical)}
\begin{itemize}
    \item \textbf{Immediate Action (Remediation):} Prioritize and enable MFA on all email accounts (e.g., via Microsoft 365 or Google Workspace admin settings). This is the single most effective control to prevent email account takeovers.
    \item \textbf{Phased Rollout (Strategy):} Develop a project plan to roll out MFA for all computer logins and access to sensitive data systems. Provide clear communication and training to employees during the rollout process.
\end{itemize}

\subsection{RISK-003: Missing Acceptable Use Policy (High)}
\begin{itemize}
    \item \textbf{Action (Governance):} Develop, approve, and implement a formal Acceptable Use Policy (AUP). This policy should clearly define rules for password security, data handling, internet usage, and the consequences of non-compliance. All employees should be required to read and acknowledge the policy.
\end{itemize}

\end{document}
```