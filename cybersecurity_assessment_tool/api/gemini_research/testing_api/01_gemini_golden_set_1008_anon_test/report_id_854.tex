```latex
\documentclass[12pt]{article}

% ----------------------------------------------------------------------
% PREAMBLE
% ----------------------------------------------------------------------
\usepackage[margin=1in]{geometry}
\usepackage{pifont} % For checkmarks and crosses
\usepackage{booktabs} % For professional tables
\usepackage[hidelinks]{hyperref} % For clickable links
\usepackage{url} % For formatting URLs
\usepackage{seqsplit} % To split long strings in tt font
\usepackage{xcolor} % For colors
\usepackage{graphicx} % For potential logos
\usepackage{array} % For better column definitions

% Define custom colors for risk levels
\definecolor{critical}{HTML}{D7263D}
\definecolor{high}{HTML}{F49D0C}
\definecolor{medium}{HTML}{F4D03F}
\definecolor{low}{HTML}{4CAF50}
\definecolor{info}{HTML}{2196F3}

% Define commands for checkmark and cross
\newcommand{\cmark}{\ding{51}}
\newcommand{\xmark}{\ding{55}}

% Define a new column type for tables to allow wrapping
\newcolumntype{L}[1]{>{\raggedright\let\newline\\\arraybackslash\hspace{0pt}}m{#1}}
\newcolumntype{C}[1]{>{\centering\let\newline\\\arraybackslash\hspace{0pt}}m{#1}}

% ----------------------------------------------------------------------
% DOCUMENT START
% ----------------------------------------------------------------------
\begin{document}

% ----------------------------------------------------------------------
% TITLE PAGE
% ----------------------------------------------------------------------
\begin{titlepage}
    \centering
    \vspace*{2cm}
    
    \Huge \textbf{Cybersecurity Posture Assessment Report}
    
    \vspace{1.5cm}
    
    \Large Prepared for: \\
    \vspace{0.5cm}
    \textbf{[Organization Name]}
    
    \vspace{2cm}
    
    \large \textbf{Date of Report:} \today
    
    \vfill
    
    \large \textit{This report contains sensitive information and should be handled with care. Distribution is restricted to authorized personnel only.}
    
\end{titlepage}

\tableofcontents
\newpage

% ----------------------------------------------------------------------
% 1. EXECUTIVE SUMMARY
% ----------------------------------------------------------------------
\section{Executive Summary}

This report provides a comprehensive assessment of the cybersecurity posture for \textbf{[Organization Name]}, based on a synthesis of network scan data, a security controls questionnaire, and a review of pre-existing risks.

The analysis has uncovered several \textbf{critical-risk} findings that require immediate attention. A public-facing FTP server was identified running a dangerously outdated version of \texttt{vsftpd} (2.3.4), which is known to contain a critical backdoor vulnerability (CVE-2011-2523). This server also permits anonymous logins, exposing the organization to significant risk of data breach and system compromise.

Furthermore, critical gaps were identified in the organization's access control policies. The lack of Multi-Factor Authentication (MFA) for computer logins and access to sensitive data systems dramatically increases the risk of unauthorized access should an employee's credentials be compromised.

These technical vulnerabilities are compounded by procedural weaknesses, including the absence of an employee Acceptable Use Policy and a lack of recurring annual security awareness training. The combination of these issues creates a high-risk environment. We strongly urge the immediate remediation of the critical vulnerabilities outlined in the Recommendations section to mitigate the imminent threats to the organization's data and operations.

% ----------------------------------------------------------------------
% 2. ORGANIZATIONAL INFORMATION
% ----------------------------------------------------------------------
\section{Organizational Information}

This section details the information provided for the assessment. Placeholders are used where data was not available.

\begin{itemize}
    \item \textbf{Organization Name:} \textbf{[Organization Name]}
    \item \textbf{Primary Email Domain:} \texttt{[Domain]}
    \item \textbf{External IP Scanned:} \texttt{[Client IP]}
    \item \textbf{Target IP from Scan:} \texttt{[Target IP]}
\end{itemize}

% ----------------------------------------------------------------------
% 3. SECURITY CONTROL REVIEW
% ----------------------------------------------------------------------
\section{Security Control Review}

The following table summarizes the organization's responses to a security controls questionnaire. "No" answers indicate significant gaps in the security framework and are flagged as risks.

\begin{table}[h!]
\centering
\caption{Security Controls Questionnaire Analysis}
\begin{tabular}{L{7.5cm} C{1.5cm} L{5cm}}
\toprule
\textbf{Control Question} & \textbf{Response} & \textbf{Analyst Notes} \\
\midrule
Do you require MFA to access email? & \cmark & Good practice. Reduces risk of email account takeover. \\
\addlinespace
Do you require MFA to log into computers? & \xmark & \textbf{Critical Gap.} Lack of MFA on endpoints allows for easy lateral movement if credentials are stolen. \\
\addlinespace
Do you require MFA to access sensitive data systems? & \xmark & \textbf{Critical Gap.} This exposes the organization's most valuable data to unauthorized access. \\
\addlinespace
Does your organization have an employee acceptable use policy? & \xmark & \textbf{High Risk.} Without an AUP, employees lack clear guidance on safe technology use, increasing insider threat risk. \\
\addlinespace
Does your organization do security awareness training for new employees? & \cmark & Good baseline for onboarding new staff. \\
\addlinespace
Does your organization do security awareness training for all employees at least once per year? & \xmark & \textbf{High Risk.} Security knowledge degrades over time. Lack of annual training leaves the organization vulnerable to evolving threats like phishing. \\
\bottomrule
\end{tabular}
\end{table}

% ----------------------------------------------------------------------
% 4. TECHNICAL SCAN RESULTS
% ----------------------------------------------------------------------
\section{Technical Scan Results}

An external network scan was performed on the target IP address \texttt{[Target IP]}. The scan identified one host with a critical vulnerability.

\begin{table}[h!]
\centering
\caption{Open Ports and Services on \texttt{[Target IP]}}
\begin{tabular}{@{}lllll@{}}
\toprule
\textbf{Port} & \textbf{State} & \textbf{Service} & \textbf{Product / Version} & \textbf{Finding} \\
\midrule
21/tcp & open & ftp & vsftpd 2.3.4 & \parbox[t]{5cm}{\textbf{CRITICAL:} Anonymous FTP login is allowed. The software version is vulnerable to a known backdoor (CVE-2011-2523), allowing remote command execution.} \\
\bottomrule
\end{tabular}
\end{table}

\subsection{Analysis of Findings}
The presence of an FTP server running \texttt{vsftpd 2.3.4} is a severe and immediate threat. This version, released in 2011, contains a well-documented backdoor that was inserted into the source code. An attacker can gain a command shell on the server by sending a specific string as the username. Combined with the allowance of anonymous logins, this server is trivial to compromise and poses a direct threat to the integrity and confidentiality of the host system and the internal network.

% ----------------------------------------------------------------------
% 5. CORRELATED RISK ASSESSMENT
% ----------------------------------------------------------------------
\section{Correlated Risk Assessment}

This section synthesizes all findings from the questionnaire, technical scan, and pre-existing risk data into a unified risk register.

\begin{table}[h!]
\centering
\caption{Summary of Identified Risks}
\begin{tabular}{@{}lL{5.5cm}l@{}}
\toprule
\textbf{Risk ID} & \textbf{Risk Name \& Description} & \textbf{Severity} \\
\midrule
R-01 & \textbf{Vulnerable Public FTP Server:} A server running vsftpd 2.3.4 with a known backdoor (CVE-2011-2523) is exposed to the internet with anonymous login enabled. & \textcolor{critical}{\textbf{Critical}} \\
\addlinespace
R-02 & \textbf{Lack of MFA on Critical Systems:} No MFA is required for workstation logins or access to sensitive data systems, exposing the organization to credential theft and unauthorized access. & \textcolor{critical}{\textbf{Critical}} \\
\addlinespace
R-03 & \textbf{Insufficient Security Training:} Security awareness training is not performed annually, leading to a decline in employee vigilance against evolving threats like phishing. & \textcolor{high}{\textbf{High}} \\
\addlinespace
R-04 & \textbf{No Acceptable Use Policy (AUP):} Lack of a formal AUP creates ambiguity and increases the risk of unintentional or malicious insider actions. & \textcolor{high}{\textbf{High}} \\
\addlinespace
R-05 & \textbf{Outdated Windows Policy:} Workstations are running Windows 7, which is an end-of-life OS that no longer receives security updates, making them easy targets for exploitation. & \textcolor{medium}{\textbf{Medium}} \\
\bottomrule
\end{tabular}
\end{table}

% ----------------------------------------------------------------------
% 6. RECOMMENDATIONS
% ----------------------------------------------------------------------
\section{Recommendations}

The following actions are recommended to mitigate the identified risks. They are prioritized based on severity and potential impact.

\subsection{Immediate Priority (Remediate within 72 hours)}
\begin{itemize}
    \item \textbf{Remediate Vulnerable FTP Server (R-01):}
    \begin{itemize}
        \item Immediately take the FTP server offline.
        \item Investigate the server for signs of compromise.
        \item If a file transfer service is required, deploy a secure alternative such as SFTP or FTPS, configured with strong authentication (no anonymous access) and running on up-to-date software.
    \end{itemize}
    \item \textbf{Implement Multi-Factor Authentication (R-02):}
    \begin{itemize}
        \item Enforce MFA for all user accounts for logging into company workstations and laptops.
        \item Enforce MFA for access to all systems classified as containing sensitive or critical data.
    \end{itemize}
\end{itemize}

\subsection{High Priority (Remediate within 30 days)}
\begin{itemize}
    \item \textbf{Establish Annual Security Training (R-03):}
    \begin{itemize}
        \item Procure and implement a security awareness training program for all employees.
        \item Ensure the training is completed by all staff annually and tracked for compliance.
    \end{itemize}
    \item \textbf{Develop and Implement an AUP (R-04):}
    \begin{itemize}
        \item Draft a formal Acceptable Use Policy that clearly defines rules for using company assets, data handling, and internet usage.
        \item Require all employees to read and acknowledge the policy.
    \end{itemize}
\end{itemize}

\subsection{Medium Priority (Plan for remediation within 90 days)}
\begin{itemize}
    \item \textbf{Upgrade Outdated Operating Systems (R-05):}
    \begin{itemize}
        \item Develop a project plan to upgrade or replace all workstations running Windows 7.
        \item Prioritize systems used by employees with access to sensitive information.
    \end{itemize}
\end{itemize}

% ----------------------------------------------------------------------
% DOCUMENT END
% ----------------------------------------------------------------------
\end{document}
```