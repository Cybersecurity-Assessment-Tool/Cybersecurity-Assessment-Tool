\documentclass[12pt]{article}

% Preamble: Required Packages
\usepackage[margin=1in]{geometry}
\usepackage{pifont} % For checkmarks and crosses
\usepackage{booktabs} % For professional tables
\usepackage[hidelinks]{hyperref} % For clickable links without boxes
\usepackage{url} % For URL formatting
\usepackage{seqsplit} % For splitting long strings to prevent overflow
\usepackage{graphicx}
\usepackage{xcolor}

% Define colors for severity
\definecolor{critical}{HTML}{990000}
\definecolor{high}{HTML}{D14302}
\definecolor{medium}{HTML}{E5A50A}
\definecolor{low}{HTML}{3E8E41}

% Document Information
\title{Cybersecurity Posture Assessment Report}
\author{Cybersecurity Analysis Division}
\date{\today}

\begin{document}

\maketitle
\thispagestyle{empty}
\newpage

\tableofcontents
\thispagestyle{empty}
\newpage

\setcounter{page}{1}

% ==============================================================================
\section{Executive Summary}
% ==============================================================================

This report provides a comprehensive analysis of the cybersecurity posture for \textbf{[Organization Name]}. The assessment is based on a correlation of network scan data, a security controls questionnaire, and a review of pre-existing risks.

The analysis identified several critical and high-risk gaps that require immediate attention. The most significant findings include a lack of Multi-Factor Authentication (MFA) for email and sensitive data systems, a complete absence of a security awareness training program, and an externally exposed Secure Shell (SSH) service.

Collectively, these vulnerabilities create a high-risk environment susceptible to credential theft, unauthorized access, and social engineering attacks. This report outlines the detailed findings and provides actionable recommendations to mitigate the identified risks and strengthen the organization's overall security posture.

% ==============================================================================
\section{Organizational Information}
% ==============================================================================

The following information was used as the basis for this assessment. As per the provided data, placeholder values are used where specific details were not available.

\begin{itemize}
    \item \textbf{Organization Name:} \textbf{[Organization Name]}
    \item \textbf{Primary Domain:} \texttt{[Domain]}
    \item \textbf{External IP Address Scanned:} \texttt{[Client IP]}
\end{itemize}

% ==============================================================================
\section{Security Control Review}
% ==============================================================================

A review of the organization's security controls was conducted via a questionnaire. The responses indicate significant gaps in foundational security practices, particularly concerning user authentication and security awareness. The table below summarizes the findings.

\begin{table}[h!]
\centering
\caption{Security Controls Questionnaire Analysis}
\label{tab:controls}
\begin{tabular}{@{}p{8cm}cc@{}}
\toprule
\textbf{Control Question} & \textbf{Response} & \textbf{Assessment} \\
\midrule
Do you require MFA to access email? & \ding{55} No & \textcolor{critical}{\textbf{Critical Gap}} \\
Do you require MFA to log into computers? & \ding{51} Yes & In Place \\
Do you require MFA to access sensitive data systems? & \ding{55} No & \textcolor{critical}{\textbf{Critical Gap}} \\
Does your organization have an employee acceptable use policy? & \ding{51} Yes & In Place \\
Does your organization do security awareness training for new employees? & \ding{55} No & \textcolor{high}{\textbf{High Risk}} \\
Does your organization do security awareness training for all employees at least once per year? & \ding{55} No & \textcolor{high}{\textbf{High Risk}} \\
\bottomrule
\end{tabular}
\end{table}

% ==============================================================================
\section{Technical Scan Results}
% ==============================================================================

An external network scan was performed on the target IP address. The scan data was limited but revealed one open port.

\begin{itemize}
    \item \textbf{Target IP Address:} \texttt{[Target IP]}
    \item \textbf{Scan Date:} Not provided in scan data.
    \item \textbf{Host Status:} Up
\end{itemize}

\subsection{Open Ports}
The following table details the open port discovered on the target system.

\begin{table}[h!]
\centering
\caption{Discovered Open Ports}
\label{tab:ports}
\begin{tabular}{@{}llll@{}}
\toprule
\textbf{Port} & \textbf{State} & \textbf{Service (Inferred)} & \textbf{Version} \\
\midrule
22/tcp & open & SSH (Secure Shell) & Not Available \\
\bottomrule
\end{tabular}
\end{table}

\subsection{Analysis}
The presence of an open SSH port (22) indicates that remote administrative access is enabled and exposed to the public internet. While SSH is a secure, encrypted protocol, its exposure is a significant risk. When combined with the identified lack of MFA on sensitive systems and the absence of security training, this exposed service becomes a prime target for brute-force attacks or attacks using stolen credentials. Without detailed version information, it is also not possible to rule out vulnerabilities in the SSH software itself.

% ==============================================================================
\section{Consolidated Risk Assessment}
% ==============================================================================

The following table synthesizes findings from the security control review and the technical scan into a prioritized list of risks. No pre-existing vulnerabilities were reported in the input data.

\begin{table}[h!]
\centering
\caption{Summary of Identified Risks}
\label{tab:risks}
\begin{tabular}{@{}p{0.5cm}p{5.5cm}p{2cm}p{3cm}@{}}
\toprule
\textbf{ID} & \textbf{Risk Description} & \textbf{Severity} & \textbf{Finding Source} \\
\midrule
R-01 & Lack of MFA on email exposes the organization to business email compromise (BEC) and phishing. & \textcolor{critical}{\textbf{Critical}} & Questionnaire \\
\addlinespace
R-02 & Lack of MFA on sensitive data systems allows for single-factor authentication to critical assets. & \textcolor{critical}{\textbf{Critical}} & Questionnaire \\
\addlinespace
R-03 & Exposed SSH service provides a direct vector for external attackers to attempt unauthorized access. & \textcolor{high}{\textbf{High}} & Technical Scan \\
\addlinespace
R-04 & Absence of security awareness training leaves employees vulnerable to social engineering and phishing attacks. & \textcolor{high}{\textbf{High}} & Questionnaire \\
\bottomrule
\end{tabular}
\end{table}

% ==============================================================================
\section{Recommendations}
% ==============================================================================

To address the identified risks, the following actions are recommended with high priority.

\begin{enumerate}
    \item \textbf{Implement Comprehensive MFA (Risk R-01, R-02):}
    \begin{itemize}
        \item Immediately enforce MFA for all user accounts on the email platform (e.g., Office 365, Google Workspace).
        \item Prioritize and deploy MFA for all systems identified as containing sensitive data, including databases, file servers, and critical applications.
    \end{itemize}
    
    \item \textbf{Secure the Exposed SSH Service (Risk R-03):}
    \begin{itemize}
        \item If remote access is not required, disable the SSH service and block port 22 at the firewall.
        \item If access is required, restrict access to a whitelist of trusted IP addresses.
        \item Enforce strong authentication by disabling password-based logins and requiring public key authentication for all SSH access.
    \end{itemize}
    
    \item \textbf{Establish a Security Awareness Training Program (Risk R-04):}
    \begin{itemize}
        \item Develop and implement a mandatory security awareness training module for all new employees as part of the onboarding process.
        \item Conduct annual, mandatory security awareness training for all staff, covering topics such as phishing, password hygiene, and acceptable use.
        \item Consider periodic phishing simulations to test and reinforce employee awareness.
    \end{itemize}
\end{enumerate}

\end{document}