```latex
\documentclass[12pt]{article}

% Preamble: Required Packages
\usepackage[margin=1in]{geometry}
\usepackage{pifont} % Required for \ding{51} (checkmark) and \ding{55} (cross)
\usepackage{booktabs} % For professional-looking tables (\toprule, \midrule, \bottomrule)
\usepackage{hyperref} % For clickable links and references
\usepackage{url}      % For formatting URLs
\usepackage{seqsplit} % For splitting long strings in \texttt
\usepackage{xcolor}   % For custom colors
\usepackage{fancyhdr} % For custom headers and footers
\usepackage{lastpage} % To reference the last page number

% --- Document Setup ---
\hypersetup{
    colorlinks=true,
    linkcolor=blue,
    filecolor=magenta,
    urlcolor=cyan,
    pdftitle={Cybersecurity Posture Assessment Report},
    pdfauthor={Cybersecurity Analysis Division},
}

% --- Header and Footer Configuration ---
\pagestyle{fancy}
\fancyhf{} % Clear all header and footer fields
\lhead{Cybersecurity Assessment Report}
\rhead{\textbf{[Organization Name]}}
\cfoot{Page \thepage\ of \pageref{LastPage}}
\renewcommand{\headrulewidth}{0.4pt}
\renewcommand{\footrulewidth}{0.4pt}

% --- Custom Colors ---
\definecolor{severitycritical}{HTML}{990000}
\definecolor{severityhigh}{HTML}{D14302}
\definecolor{severitymedium}{HTML}{F5A623}
\definecolor{severitylow}{HTML}{7ED321}

\begin{document}

% --- Title Page ---
\begin{titlepage}
    \centering
    \vspace*{\fill}
    \Huge \textbf{Cybersecurity Posture Assessment Report}
    \vspace{1.5cm}
    \Large \textbf{For: [Organization Name]}
    \vspace{2cm}
    \normalsize
    \begin{tabular}{ll}
        \textbf{Date of Report:} & \today \\
        \textbf{Prepared by:} & Cybersecurity Analysis Division \\
    \end{tabular}
    \vspace*{\fill}
    \textit{This document contains sensitive information and is intended for the exclusive use of the recipient organization.}
\end{titlepage}

\tableofcontents
\newpage

% --- Section 1: Executive Summary ---
\section{Executive Summary}
This report provides a comprehensive analysis of the cybersecurity posture for \textbf{[Organization Name]}. The assessment is based on a synthesis of an external network scan, a review of internal security controls via a questionnaire, and an evaluation of pre-existing documented risks.

\paragraph{Key Findings:} The organization demonstrates a strong external network security posture, as the network scan of the target asset \texttt{[Client IP]} revealed no open ports. This indicates a well-configured firewall and a minimal attack surface from an external perspective.

However, the security control review identified several critical and high-risk gaps in internal policies and procedures. The most significant concerns are the lack of Multi-Factor Authentication (MFA) for computer and sensitive data access, the absence of a formal Employee Acceptable Use Policy, and incomplete annual security awareness training. These deficiencies expose the organization to significant risks, including unauthorized access, data breaches, and insider threats, despite the strong perimeter security.

\paragraph{Overall Assessment:} While the external defenses are robust, the internal security controls require immediate attention. The recommendations outlined in this report focus on mitigating the identified policy and access control weaknesses to build a more resilient, defense-in-depth security strategy.

% --- Section 2: Organizational and Scan Information ---
\section{Organizational and Scan Information}
This section details the information provided for the assessment. Placeholders are used where data was not available.

\begin{tabular}{@{}ll}
    \toprule
    \textbf{Detail} & \textbf{Value} \\
    \midrule
    Organization Name & \textbf{[Organization Name]} \\
    Email Domain & \texttt{[Domain]} \\
    External IP Scanned & \texttt{[Client IP]} \\
    Target IP from Scan Data & \texttt{[Target IP]} \\
    \bottomrule
\end{tabular}

% --- Section 3: Security Control Review ---
\section{Security Control Review (Questionnaire Analysis)}
The following table summarizes the organization's responses to the security controls questionnaire. A checkmark (\ding{51}) indicates a positive control is in place, while a cross (\ding{55}) highlights a potential gap that increases risk.

\begin{table}[h!]
\centering
\begin{tabular}{@{}lc@{}}
    \toprule
    \textbf{Control Question} & \textbf{Response} \\
    \midrule
    Do you require MFA to access email? & \ding{51} \\
    Do you require MFA to log into computers? & \textcolor{red}{\ding{55}} \\
    Do you require MFA to access sensitive data systems? & \textcolor{red}{\ding{55}} \\
    Does your organization have an employee acceptable use policy? & \textcolor{red}{\ding{55}} \\
    Does your organization do security awareness training for new employees? & \ding{51} \\
    Does your organization do security awareness training for all employees at least once per year? & \textcolor{red}{\ding{55}} \\
    \bottomrule
\end{tabular}
\caption{Analysis of Security Control Questionnaire.}
\end{table}

The analysis reveals critical gaps in access control (MFA) and foundational cybersecurity policies (Acceptable Use Policy, Annual Training), which are detailed in the Risk Assessment section.

% --- Section 4: Technical Scan Results ---
\section{Technical Scan Results}
An external network scan was performed using Nmap on the target IP address. The results are summarized below.

\paragraph{Target:} \texttt{[Target IP]}
\paragraph{Status:} Host is Up.
\paragraph{Port Scan Results:} The scan determined that all scanned ports were in a \textbf{closed} state. No open ports or active services were detected.

\paragraph{Analysis:} This is a positive security finding. A host with no externally open ports presents a minimal attack surface to external threats, suggesting a properly configured firewall is in place that denies all unsolicited inbound traffic. This significantly reduces the risk of external network-based attacks.

% --- Section 5: Risk Assessment ---
\section{Risk Assessment}
This section synthesizes findings from the questionnaire and technical scan to identify and prioritize current risks. The pre-existing risk register was empty. The following new risks have been identified.

\begin{table}[h!]
\centering
\begin{tabular}{@{}p{0.1\linewidth} p{0.3\linewidth} p{0.15\linewidth} p{0.35\linewidth}@{}}
    \toprule
    \textbf{Risk ID} & \textbf{Risk Name} & \textbf{Severity} & \textbf{Overview} \\
    \midrule
    RISK-001 & Lack of MFA on Sensitive Systems & \textcolor{severitycritical}{\textbf{Critical}} & The absence of MFA on systems holding sensitive data creates a high risk of a data breach. Compromised credentials alone could grant an attacker direct access to critical assets. \\
    \addlinespace
    RISK-002 & Missing Acceptable Use Policy (AUP) & \textcolor{severityhigh}{\textbf{High}} & Without a formal AUP, there are no clear rules for employees regarding the use of company assets. This increases the risk of insider threats, data misuse, and legal liabilities. \\
    \addlinespace
    RISK-003 & Incomplete Annual Security Training & \textcolor{severityhigh}{\textbf{High}} & Security threats evolve constantly. Without annual training for all staff, the organization is more susceptible to phishing, social engineering, and other human-centric attacks. \\
    \addlinespace
    RISK-004 & Lack of MFA on Workstations & \textcolor{severityhigh}{\textbf{High}} & Compromised employee credentials could allow an attacker to gain full access to a company workstation, which can then be used as a pivot point to move laterally within the network. \\
    \bottomrule
\end{tabular}
\caption{Summary of Identified Risks.}
\end{table}

% --- Section 6: Recommendations ---
\section{Recommendations}
Based on the risk assessment, the following actionable recommendations are provided to enhance the organization's cybersecurity posture.

\begin{enumerate}
    \item \textbf{[Critical] Implement MFA for All Sensitive Systems (RISK-001):}
    \begin{itemize}
        \item \textbf{Action:} Immediately prioritize and enforce MFA on all applications, databases, and systems classified as containing sensitive or critical data.
        \item \textbf{Impact:} Drastically reduces the risk of unauthorized access and data breaches resulting from stolen credentials.
    \end{itemize}

    \item \textbf{[High] Develop and Enforce an Acceptable Use Policy (RISK-002):}
    \begin{itemize}
        \item \textbf{Action:} Create a formal AUP document that clearly defines the rules for using company networks, devices, and data. Ensure all employees read and acknowledge the policy.
        \item \textbf{Impact:} Establishes a baseline for secure employee behavior, reduces insider risk, and provides a basis for disciplinary action in case of violations.
    \end{itemize}

    \item \textbf{[High] Mandate Annual Security Awareness Training (RISK-003):}
    \begin{itemize}
        \item \textbf{Action:} Implement a mandatory annual security awareness training program for all employees. The training should cover current threats such as phishing, ransomware, and social engineering.
        \item \textbf{Impact:} Creates a more security-conscious culture and reduces the likelihood of human error leading to a security incident.
    \end{itemize}

    \item \textbf{[High] Enforce MFA for Workstation Logins (RISK-004):}
    \begin{itemize}
        \item \textbf{Action:} Deploy MFA for all employee computer logins (both local and remote). This can be achieved through solutions like Windows Hello for Business, Duo, or other identity providers.
        \item \textbf{Impact:} Protects against unauthorized workstation access and limits an attacker's ability to move laterally within the network after a credential compromise.
    \end{itemize}
\end{enumerate}

\end{document}
```