```latex
\documentclass[12pt]{article}

% 1. DOCUMENT SETUP & PACKAGES
\usepackage[a4paper, margin=1in]{geometry}
\usepackage{pifont} % For checkmarks and crosses
\usepackage{booktabs} % For professional tables
\usepackage{hyperref} % For hyperlinks
\usepackage{url} % For URL formatting
\usepackage{seqsplit} % For splitting long strings
\usepackage{graphicx} % For logos, etc.
\usepackage[table]{xcolor} % For coloring table cells

% Define custom colors for severity
\definecolor{criticalred}{RGB}{217, 83, 79}
\definecolor{highorange}{RGB}{240, 173, 78}
\definecolor{mediumyellow}{RGB}{255, 217, 102}
\definecolor{lowblue}{RGB}{91, 192, 222}
\definecolor{infogray}{RGB}{245, 245, 245}

% Hyperref setup
\hypersetup{
    colorlinks=true,
    linkcolor=blue,
    filecolor=magenta,      
    urlcolor=cyan,
    pdftitle={Cybersecurity Posture Report},
    pdfpagemode=FullScreen,
}

% Checkmark and Cross definitions
\newcommand{\cmark}{\ding{51}}%
\newcommand{\xmark}{\ding{55}}%

% 2. DOCUMENT START
\begin{document}

% --- TITLE PAGE ---
\begin{titlepage}
    \centering
    \vspace*{2cm}
    
    {\Huge \textbf{Cybersecurity Posture Report}}
    
    \vspace{1.5cm}
    
    {\Large \textbf{Prepared for:}}
    
    \vspace{0.5cm}
    
    {\Huge \textbf{[Organization Name]}}
    
    \vfill
    
    {\large \today}
    
\end{titlepage}

\tableofcontents
\newpage

% --- EXECUTIVE SUMMARY ---
\section{Executive Summary}

This report provides a comprehensive analysis of the current cybersecurity posture for \textbf{[Organization Name]}. The assessment is based on a correlation of external network scan data, a review of internal security controls via a questionnaire, and an analysis of pre-existing risk documentation.

The overall security posture is assessed as \textbf{CRITICAL}. A confluence of high-risk factors was identified:
\begin{itemize}
    \item \textbf{Direct Critical Service Exposure:} The external network scan confirmed that Remote Desktop Protocol (RDP) on port 3389 is directly exposed to the internet at \texttt{[Client IP]}. This is a primary attack vector for ransomware and unauthorized access.
    \item \textbf{Complete Lack of Multi-Factor Authentication (MFA):} The organization does not enforce MFA for any system, including email, computer logins, and sensitive data access. This drastically lowers the barrier for attackers to compromise accounts.
    \item \textbf{Absence of Foundational Security Policies and Training:} There is no Acceptable Use Policy or security awareness training program in place. This indicates a low level of security maturity and increases the likelihood of security incidents caused by human error.
\end{itemize}

The combination of an exposed RDP service with the absence of MFA and user training creates an immediate and severe risk of a significant security breach. Immediate remediation is strongly recommended to mitigate these threats.

% --- ORGANIZATIONAL INFORMATION ---
\section{Organizational Information}

The following details were used as the basis for this assessment.
\begin{itemize}
    \item \textbf{Organization Name:} \textbf{[Organization Name]}
    \item \textbf{Primary Domain:} \texttt{[Domain]}
    \item \textbf{External IP Address Scanned:} \texttt{[Client IP]}
\end{itemize}

% --- SECURITY CONTROL REVIEW ---
\section{Security Control Review}
The following table summarizes the responses to the security controls questionnaire. Each "No" response represents a significant gap in the organization's defensive posture.

\begin{table}[h!]
\centering
\caption{Security Controls Questionnaire Analysis}
\begin{tabular}{p{8cm} c c}
\toprule
\textbf{Control Question} & \textbf{Response} & \textbf{Assessment} \\
\midrule
Do you require MFA to access email? & \xmark & \cellcolor{criticalred!25}Critical Gap \\
Do you require MFA to log into computers? & \xmark & \cellcolor{criticalred!25}Critical Gap \\
Do you require MFA to access sensitive data systems? & \xmark & \cellcolor{criticalred!25}Critical Gap \\
Does your organization have an employee acceptable use policy? & \xmark & \cellcolor{highorange!25}High Risk \\
Does your organization do security awareness training for new employees? & \xmark & \cellcolor{highorange!25}High Risk \\
Does your organization do security awareness training for all employees at least once per year? & \xmark & \cellcolor{highorange!25}High Risk \\
\bottomrule
\end{tabular}
\end{table}

The complete absence of these fundamental controls indicates a reactive and immature security program. The lack of MFA is particularly alarming, as it is one of the most effective controls for preventing account takeovers.

% --- TECHNICAL SCAN RESULTS ---
\section{Technical Scan Results}
An external network scan was performed on the target IP address \texttt{[Target IP]}. The scan identified the following open port, which presents a significant security risk.

\begin{table}[h!]
\centering
\caption{Open Port Analysis for Target: \texttt{[Target IP]}}
\begin{tabular}{c c l p{6cm}}
\toprule
\textbf{Port} & \textbf{State} & \textbf{Service} & \textbf{Analysis} \\
\midrule
3389/tcp & Open & ms-wbt-server & This port is used for Microsoft Remote Desktop Protocol (RDP). Direct exposure to the internet is a critical vulnerability, frequently exploited by threat actors for initial access and ransomware deployment. \\
\bottomrule
\end{tabular}
\end{table}

This technical finding directly corroborates the pre-existing risk documented in the organization's risk register, confirming that the "RDP Exposure" risk is active and requires immediate attention.

% --- RISK ASSESSMENT SUMMARY ---
\section{Risk Assessment}
The following table synthesizes findings from the security control review, technical scan, and existing risk data into a prioritized list of identified risks.

\begin{table}[h!]
\centering
\caption{Synthesized Risk Summary}
\begin{tabular}{p{4cm} p{7cm} l}
\toprule
\textbf{Risk Name} & \textbf{Description} & \textbf{Severity} \\
\midrule
\rowcolor{criticalred!25}
\textbf{Critical RDP Exposure} & The RDP service on port 3389 is exposed on \texttt{[Target IP]}. This is amplified by the lack of MFA, making the system highly vulnerable to brute-force password attacks and ransomware. & Critical (9.0) \\
\rowcolor{criticalred!25}
\textbf{Lack of Multi-Factor Authentication (MFA)} & MFA is not enforced on any system. This allows an attacker with valid credentials (e.g., from a phishing attack or password spray) to gain unauthorized access without any additional challenge. & Critical \\
\rowcolor{highorange!25}
\textbf{Absence of Security Policies \& Training} & The lack of an Acceptable Use Policy and security awareness training means employees are likely unaware of security best practices, making them more susceptible to social engineering and accidental data exposure. & High \\
\bottomrule
\end{tabular}
\end{table}

% --- RECOMMENDATIONS ---
\section{Recommendations}
Based on the critical risks identified, the following prioritized actions are recommended to improve the security posture of \textbf{[Organization Name]}.

\begin{enumerate}
    \item \textbf{Immediate Priority: Remediate RDP Exposure}
    \begin{itemize}
        \item \textbf{Action:} Immediately implement a firewall rule to block all inbound traffic to TCP port 3389 on \texttt{[Client IP]} from the internet.
        \item \textbf{Long-Term Solution:} If remote access is required, deploy a secure Virtual Private Network (VPN) solution. Access to the VPN must be protected with Multi-Factor Authentication.
    \end{itemize}
    
    \item \textbf{High Priority: Implement Multi-Factor Authentication}
    \begin{itemize}
        \item \textbf{Action:} Deploy MFA across all critical systems and applications.
        \item \textbf{Prioritization:}
        \begin{enumerate}
            \item All external-facing services (e.g., email, VPN).
            \item All administrator and privileged accounts.
            \item All employee accounts for computer and sensitive data system access.
        \end{enumerate}
    \end{itemize}
    
    \item \textbf{Medium Priority: Establish Foundational Security Programs}
    \begin{itemize}
        \item \textbf{Action:} Develop and implement a formal Employee Acceptable Use Policy that outlines the rules for using company IT assets.
        \item \textbf{Action:} Institute a mandatory security awareness training program for all new and existing employees. This program should be conducted at least annually and cover topics such as phishing, password security, and data handling.
    \end{itemize}
\end{enumerate}

% --- CONCLUSION ---
\section{Conclusion}
The current security posture of \textbf{[Organization Name]} contains critical, high-impact vulnerabilities that require immediate action. The direct exposure of RDP, combined with a systemic lack of basic security controls like MFA and user training, places the organization at an exceptionally high risk of a damaging cyberattack.

We strongly urge the implementation of the recommendations outlined in this report, beginning with the immediate remediation of the exposed RDP service. Proactive measures taken now will significantly reduce the organization's attack surface and build a more resilient security foundation for the future.

% 3. DOCUMENT END
\end{document}
```