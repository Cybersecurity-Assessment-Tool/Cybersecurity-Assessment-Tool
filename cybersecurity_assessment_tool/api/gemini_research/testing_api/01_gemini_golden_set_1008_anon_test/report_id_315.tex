```latex
\documentclass[12pt]{article}

% Preamble: Required Packages
\usepackage[margin=1in]{geometry}
\usepackage{pifont} % For checkmarks and crosses
\usepackage{booktabs} % For professional tables
\usepackage{hyperref} % For hyperlinks and document metadata
\usepackage{url}      % For formatting URLs
\usepackage{seqsplit} % For splitting long text strings like hashes or IPs

% Document Metadata
\hypersetup{
    colorlinks=true,
    linkcolor=black,
    filecolor=magenta,      
    urlcolor=blue,
    pdftitle={Cybersecurity Assessment Report},
    pdfauthor={Cybersecurity Analyst},
    pdfsubject={Security Assessment},
    pdfkeywords={Security, Analysis, Report},
    bookmarks=true
}

% Document Start
\begin{document}

% Title Block
\title{Cybersecurity Assessment Report \\ \large For \textbf{[Organization Name]}}
\author{Cybersecurity Analyst}
\date{\today}
\maketitle

\hrule
\vspace{1em}

% --- 1. Executive Overview ---
\section*{1. Executive Overview}
This report details the findings of a cybersecurity assessment conducted for \textbf{[Organization Name]}. The analysis combines a review of organizational security controls, an external network scan, and a summary of pre-existing risks.

The assessment identified several high-risk and critical vulnerabilities that require immediate attention. The most critical finding is a publicly exposed MySQL database service on port 3306. This service is running an End-of-Life (EOL) version of MySQL (5.7.33), which is no longer receiving security updates from the vendor. This exposes the organization to numerous known exploits.

Furthermore, significant gaps were identified in the organization's security policies. The lack of Multi-Factor Authentication (MFA) on employee computers and the absence of security awareness training for new hires create substantial risks of unauthorized access and social engineering attacks. These procedural gaps, combined with the technical vulnerabilities, create a high-risk environment that could lead to a significant data breach.

Immediate remediation of the exposed database and implementation of stronger access controls are strongly recommended.

% --- 2. Organizational Information ---
\section*{2. Organizational Information}
The following details were used as the basis for this assessment. Due to the anonymized nature of the provided data, placeholders have been used where necessary.

\begin{itemize}
    \item \textbf{Organization Name:} \textbf{[Organization Name]}
    \item \textbf{Primary Domain:} \texttt{[Domain]}
    \item \textbf{External IP Scanned:} \texttt{[Client IP]}
    \item \textbf{Target IP Scanned:} \texttt{[Target IP]}
\end{itemize}

% --- 3. Security Control Review ---
\section*{3. Security Control Review}
A review of the organization's security controls was conducted based on a standard questionnaire. The responses indicate key areas where security posture can be improved. "No" answers represent significant gaps in the defense-in-depth strategy.

\begin{table}[h!]
\centering
\caption{Security Controls Questionnaire Results}
\begin{tabular}{p{0.8\linewidth} c}
\toprule
\textbf{Control Question} & \textbf{Response} \\
\midrule
Do you require MFA to access email? & \ding{51} \\
\textbf{Do you require MFA to log into computers?} & \textbf{\ding{55}} \\
Do you require MFA to access sensitive data systems? & \ding{51} \\
Does your organization have an employee acceptable use policy? & \ding{51} \\
\textbf{Does your organization do security awareness training for new employees?} & \textbf{\ding{55}} \\
Does your organization do security awareness training for all employees at least once per year? & \ding{51} \\
\bottomrule
\end{tabular}
\end{table}

% --- 4. Technical Scan Results ---
\section*{4. Technical Scan Results}
An external network scan was performed on the target IP address \texttt{[Target IP]}. The scan identified the following open ports and services accessible from the public internet.

\begin{table}[h!]
\centering
\caption{Open Port Scan Findings}
\begin{tabular}{l l l l l}
\toprule
\textbf{Port} & \textbf{State} & \textbf{Service} & \textbf{Product} & \textbf{Version} \\
\midrule
3306/tcp & open & mysql & MySQL & 5.7.33 \\
\bottomrule
\end{tabular}
\end{table}

\paragraph{Analysis:} The presence of an open MySQL port (3306) is a critical security risk. This allows attackers to directly interact with the database from anywhere on the internet, exposing it to brute-force attacks, credential stuffing, and exploitation of vulnerabilities.

Critically, the detected version, \textbf{MySQL 5.7.33}, reached its official End of Life (EOL) in October 2023. This means it no longer receives security patches, and any vulnerabilities discovered since that date remain unpatched. This elevates the risk of compromise significantly.

% --- 5. Risk Assessment Summary ---
\section*{5. Risk Assessment Summary}
The following table synthesizes findings from the technical scan, the security control review, and pre-existing risk data. Risks are categorized by severity to guide prioritization efforts.

\begin{table}[h!]
\centering
\caption{Consolidated Risk Register}
\begin{tabular}{p{0.25\linewidth} p{0.15\linewidth} p{0.5\linewidth}}
\toprule
\textbf{Risk Name} & \textbf{Severity} & \textbf{Description} \\
\midrule
\textbf{End-of-Life Software} & \textbf{Critical (9.8)} & The publicly exposed MySQL 5.7.33 database is no longer supported and is vulnerable to known exploits. \\
\addlinespace
\textbf{Database Exposure} & \textbf{High (7.5)} & MySQL port 3306 is open to the public internet, allowing direct attack attempts against the database. \\
\addlinespace
\textbf{Lack of Endpoint MFA} & \textbf{High (7.2)} & The absence of MFA on computer logins significantly increases the risk of unauthorized access via compromised credentials. \\
\addlinespace
\textbf{Inadequate Employee Onboarding} & \textbf{Medium (5.4)} & New employees do not receive security awareness training, making them prime targets for phishing and social engineering attacks. \\
\bottomrule
\end{tabular}
\end{table}

% --- 6. Recommendations ---
\section*{6. Recommendations}
Based on the analysis, the following actions are recommended to mitigate the identified risks and improve the overall security posture of \textbf{[Organization Name]}.

\subsection*{Immediate Actions (Critical Priority)}
\begin{enumerate}
    \item \textbf{Restrict Access to Port 3306:} Immediately implement firewall rules to block all public access to TCP port 3306 on \texttt{[Target IP]}. Access should be restricted to a whitelist of trusted IP addresses only.
    \item \textbf{Plan Database Upgrade:} Develop an urgent plan to migrate the MySQL 5.7.33 database to a currently supported version (e.g., MySQL 8.x). This is essential to receive security patches for future vulnerabilities.
\end{enumerate}

\subsection*{High Priority Actions}
\begin{enumerate}
    \setcounter{enumi}{2} % Continue numbering
    \item \textbf{Enforce Endpoint MFA:} Deploy and enforce a Multi-Factor Authentication solution for all employee computer and laptop logins. This adds a critical layer of security against credential theft.
    \item \textbf{Implement New Hire Training:} Create and mandate a security awareness training module as part of the onboarding process for all new employees. This should cover topics such as phishing, password hygiene, and acceptable use policies.
\end{enumerate}

\subsection*{Strategic Recommendations}
\begin{enumerate}
    \setcounter{enumi}{4} % Continue numbering
    \item \textbf{Adopt a Defense-in-Depth Network Architecture:} For long-term security, critical services like databases should not be directly exposed to the internet. Plan to move the database into a private network segment, accessible only via a secure Virtual Private Network (VPN).
\end{enumerate}

\end{document}
```