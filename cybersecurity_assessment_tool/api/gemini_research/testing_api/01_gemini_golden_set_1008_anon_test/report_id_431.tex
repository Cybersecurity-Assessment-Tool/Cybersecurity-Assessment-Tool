```latex
\documentclass[12pt]{article}

% --- PACKAGES ---
\usepackage[margin=1in]{geometry}
\usepackage{pifont} % For checkmarks and crosses
\usepackage{booktabs} % For professional tables
\usepackage{hyperref} % For hyperlinks
\usepackage{url} % For URL formatting
\usepackage{seqsplit} % For splitting long text sequences
\usepackage{graphicx}
\usepackage{xcolor}

% --- DOCUMENT SETUP ---
\hypersetup{
    colorlinks=true,
    linkcolor=blue,
    filecolor=magenta,      
    urlcolor=cyan,
    pdftitle={Cybersecurity Posture Report},
    pdfpagemode=FullScreen,
}

\newcommand{\yes}{\ding{51}} % Green checkmark
\newcommand{\no}{\ding{55}}  % Red cross

\begin{document}

% --- TITLE PAGE ---
\begin{titlepage}
    \centering
    \vspace*{1cm}
    \Huge\textbf{Cybersecurity Posture Report}
    \vspace{1.5cm}
    \Large
    \textbf{Prepared for:} \\
    \vspace{0.5cm}
    \textbf{[Organization Name]}
    \vspace{2cm}
    \large
    \textbf{Date of Report:} \today \\
    \vspace{1cm}
    \textbf{Analysis Period:} October 2023
    \vfill
    \textit{This report contains sensitive information and should be handled with care. Access is restricted to authorized personnel only.}
\end{titlepage}

\tableofcontents
\newpage

% --- EXECUTIVE SUMMARY ---
\section*{Executive Summary}

This report provides a comprehensive analysis of the cybersecurity posture for \textbf{[Organization Name]}, based on a review of organizational security controls, an external network scan, and pre-existing risk data. The assessment reveals a mixed security landscape with several effective controls in place, but also identifies critical vulnerabilities that require immediate attention.

Key findings indicate that while the organization mandates multi-factor authentication (MFA) for computer and sensitive system access, a critical gap exists in its application to email services. This, combined with the lack of mandatory annual security awareness training for all staff, significantly elevates the risk of phishing attacks and subsequent account compromise.

Furthermore, technical scanning identified an exposed SSH service on the perimeter, presenting a direct vector for unauthorized access if not properly secured. A pre-existing critical risk, "Localhost Exposed," with a CVSS score of 10.0, remains an urgent issue to be addressed.

This report outlines these findings in detail and provides actionable recommendations to mitigate the identified risks and strengthen the overall security posture of the organization.

% --- ORGANIZATIONAL INFORMATION ---
\section*{Organizational Information}

The following details were used as the basis for this assessment. As per the provided data, some identifying information has been replaced with placeholders.

\begin{itemize}
    \item \textbf{Organization Name:} \textbf{[Organization Name]}
    \item \textbf{Primary Email Domain:} \texttt{[Domain]}
    \item \textbf{External IP Scanned:} \texttt{[Client IP]}
\end{itemize}

% --- SECURITY CONTROL REVIEW ---
\section*{Security Control Review}

A review of the organization's security controls was conducted via a questionnaire. The responses highlight areas of both strength and weakness in the current security policies and their implementation. Gaps identified here often represent systemic risks that can be exploited by threat actors.

\begin{table}[h!]
\centering
\caption{Security Control Questionnaire Analysis}
\begin{tabular}{p{0.7\linewidth} c}
\toprule
\textbf{Control Question} & \textbf{Response} \\
\midrule
Do you require MFA to access email? & \no \\
Do you require MFA to log into computers? & \yes \\
Do you require MFA to access sensitive data systems? & \yes \\
Does your organization have an employee acceptable use policy? & \yes \\
Does your organization do security awareness training for new employees? & \yes \\
Does your organization do security awareness training for all employees at least once per year? & \no \\
\bottomrule
\end{tabular}
\end{table}

\paragraph{Analysis:} The lack of MFA on email is a \textbf{critical} vulnerability. Email is the primary target for phishing attacks, and a compromised email account can lead to widespread system compromise. The absence of annual security training for all employees is a \textbf{high} risk, as it reduces the organization's resilience against social engineering attacks.

% --- TECHNICAL SCAN RESULTS ---
\section*{Technical Scan Results}

An external network scan was performed to identify open ports and exposed services on the organization's perimeter.

\begin{itemize}
    \item \textbf{Target IP Address:} \texttt{[Target IP]}
    \item \textbf{Scan Date:} Data not available in scan metadata.
\end{itemize}

The following table details the open ports discovered during the scan.

\begin{table}[h!]
\centering
\caption{Open Ports Detected on \texttt{[Target IP]}}
\begin{tabular}{c c c l}
\toprule
\textbf{Port} & \textbf{Protocol} & \textbf{State} & \textbf{Inferred Service} \\
\midrule
22 & TCP & Open & SSH (Secure Shell) \\
\bottomrule
\end{tabular}
\end{table}

\paragraph{Analysis:} The presence of an open SSH port (22) on an external-facing asset is a significant finding. SSH is a common target for brute-force attacks. If configured with weak credentials or outdated software, it can provide a direct entry point for an attacker into the internal network.

% --- RISK ASSESSMENT SUMMARY ---
\section*{Risk Assessment Summary}

The following table synthesizes findings from the security control review, technical scan, and pre-existing risk data into a prioritized list of risks.

\begin{table}[h!]
\centering
\caption{Consolidated Risk Register}
\begin{tabular}{p{0.4\linewidth} p{0.4\linewidth} l}
\toprule
\textbf{Risk Name} & \textbf{Description} & \textbf{Severity} \\
\midrule
\textbf{Localhost Exposed} & A pre-existing critical vulnerability (CVSS 10.0) was identified. Details on the nature of this exposure were not provided, but it represents the highest possible risk. & \textbf{Critical} \\
\addlinespace
\textbf{No MFA for Email Access} & The lack of multi-factor authentication on email accounts exposes the organization to a high likelihood of account compromise via phishing or credential stuffing. & \textbf{Critical} \\
\addlinespace
\textbf{Exposed SSH Service} & Port 22 (SSH) is open to the public internet, creating a direct attack vector for brute-force attacks and exploitation of potential vulnerabilities. & \textbf{High} \\
\addlinespace
\textbf{Lack of Annual Security Training} & Without regular, recurring security training, employees are more likely to fall victim to social engineering attacks, undermining other security controls. & \textbf{High} \\
\bottomrule
\end{tabular}
\end{table}

% --- RECOMMENDATIONS ---
\section*{Recommendations}

The following actionable recommendations are provided to address the identified risks. They are prioritized based on severity.

\subsection*{Immediate Actions (Critical Risks)}

\begin{enumerate}
    \item \textbf{Investigate and Remediate "Localhost Exposed" Vulnerability:}
    \begin{itemize}
        \item \textbf{Action:} Immediately assemble a technical team to investigate the root cause of the CVSS 10.0 "Localhost Exposed" finding. The term implies a service intended for internal use only is accessible externally.
        \item \textbf{Impact:} Failure to address this could lead to a complete system compromise.
    \end{itemize}

    \item \textbf{Enforce MFA on All Email Accounts:}
    \begin{itemize}
        \item \textbf{Action:} Implement and enforce MFA for all user access to the email system (\texttt{[Domain]}) within the next 30 days. Prioritize phishing-resistant methods like FIDO2 security keys or authenticator apps over SMS.
        \item \textbf{Impact:} This is the single most effective control to prevent email account takeovers.
    \end{itemize}
\end{enumerate}

\subsection*{Near-Term Actions (High Risks)}

\begin{enumerate}
    \setcounter{enumi}{2} % Continue numbering
    \item \textbf{Secure or Disable Exposed SSH Service:}
    \begin{itemize}
        \item \textbf{Action:} Review the business justification for the exposed SSH service on \texttt{[Target IP]}. If it is not essential, disable the service immediately. If it is required, implement strict access controls, including firewall rules to allow access only from trusted IP addresses, and enforce key-based authentication (disabling password authentication).
        \item \textbf{Impact:} Reduces the external attack surface and protects against brute-force intrusion attempts.
    \end{itemize}

    \item \textbf{Implement Annual Security Awareness Training:}
    \begin{itemize}
        \item \textbf{Action:} Procure and roll out a mandatory security awareness training program for all employees, to be completed annually. The training should cover phishing, password hygiene, and acceptable use policies.
        \item \textbf{Impact:} Creates a more security-conscious culture and turns employees into a human firewall, reducing the likelihood of successful social engineering attacks.
    \end{itemize}
\end{enumerate}

\end{document}
```