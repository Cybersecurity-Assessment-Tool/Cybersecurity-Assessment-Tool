```latex
\documentclass[12pt]{article}

% Preamble: Required Packages
\usepackage[margin=1in]{geometry}
\usepackage{pifont} % For checkmarks and crosses
\usepackage{booktabs} % For professional tables
\usepackage{hyperref} % For hyperlinks
\usepackage{url} % For URL formatting
\usepackage{seqsplit} % To split long strings in texttt
\usepackage{xcolor} % For colors in tables

% Document Metadata
\title{Cybersecurity Posture Assessment Report \\ \large For \textbf{[Organization Name]}}
\author{Cybersecurity Analyst}
\date{\today}

% Hyperref Setup
\hypersetup{
    colorlinks=true,
    linkcolor=blue,
    filecolor=magenta,      
    urlcolor=cyan,
    pdftitle={Cybersecurity Posture Assessment Report},
    pdfpagemode=FullScreen,
}

\begin{document}

\maketitle
\thispagestyle{empty}
\newpage

\tableofcontents
\newpage

\section{Executive Summary}

This report details the findings of a cybersecurity posture assessment conducted for \textbf{[Organization Name]}. The assessment combined an automated network scan, a review of existing risks, and an analysis of organizational security controls based on a questionnaire.

The overall security posture is a mix of implemented controls and significant, high-impact gaps. While the organization has successfully implemented Multi-Factor Authentication (MFA) for computer and sensitive system access, along with a security awareness training program, several critical deficiencies were identified.

Key findings include:
\begin{itemize}
    \item \textbf{Critical Risk - Lack of MFA on Email:} The absence of MFA on email exposes the organization to a high risk of business email compromise, phishing attacks, and subsequent account takeovers.
    \item \textbf{High Risk - Exposed Management Service:} The external network scan identified an open Secure Shell (SSH) port (22/TCP) on the client's public-facing IP address. This service is a common target for brute-force attacks and exploitation.
    \item \textbf{High Risk - Foundational Policy Gap:} The organization lacks a formal employee Acceptable Use Policy (AUP). This governance gap can lead to inconsistent security practices and a lack of enforceable standards for employee behavior.
\end{itemize}

No pre-existing vulnerabilities were documented, making this assessment a new baseline for risk management. Immediate remediation of the identified risks is strongly recommended to reduce the organization's attack surface and improve its overall defensive posture.

\section{Organizational Information}

The following information was used as the basis for this assessment. As per the template mode for this report, placeholders are used where data was not provided.

\begin{itemize}
    \item \textbf{Organization Name:} \textbf{[Organization Name]}
    \item \textbf{Primary Domain:} \seqsplit{\texttt{[Domain]}}
    \item \textbf{External IP Scanned:} \seqsplit{\texttt{[Client IP]}}
\end{itemize}

\section{Security Control Review}

The following table summarizes the organization's responses to the security controls questionnaire. "No" responses indicate significant gaps in the security program and are highlighted for immediate attention.

\begin{table}[h!]
\centering
\caption{Security Controls Questionnaire Analysis}
\label{tab:controls}
\begin{tabular}{p{0.6\linewidth} c p{0.25\linewidth}}
\toprule
\textbf{Control Question} & \textbf{Response} & \textbf{Analyst Notes} \\
\midrule
Do you require MFA to access email? & \textcolor{red}{\ding{55}} & \textbf{Critical Gap.} Email is a primary vector for account takeover. \\
\addlinespace
Do you require MFA to log into computers? & \textcolor{green}{\ding{51}} & Best practice implemented. \\
\addlinespace
Do you require MFA to access sensitive data systems? & \textcolor{green}{\ding{51}} & Best practice implemented. \\
\addlinespace
Does your organization have an employee acceptable use policy? & \textcolor{red}{\ding{55}} & \textbf{High Risk.} Foundational governance control is missing. \\
\addlinespace
Does your organization do security awareness training for new employees? & \textcolor{green}{\ding{51}} & Good practice for onboarding. \\
\addlinespace
Does your organization do security awareness training for all employees at least once per year? & \textcolor{green}{\ding{51}} & Meets compliance standards. \\
\bottomrule
\end{tabular}
\end{table}

\section{Technical Scan Results}

An external network scan was performed against the organization's public IP address. The target for this scan was specified as \seqsplit{\texttt{[Target IP]}}. The scan revealed the following open port, indicating a service exposed to the public internet.

\begin{table}[h!]
\centering
\caption{Open Ports Detected on \seqsplit{\texttt{[Target IP]}}}
\label{tab:nmap}
\begin{tabular}{l l l p{0.5\linewidth}}
\toprule
\textbf{Port} & \textbf{Protocol} & \textbf{Service} & \textbf{Notes} \\
\midrule
22 & TCP & SSH & Secure Shell is a remote management protocol. Exposing this service to the internet creates a significant risk of brute-force attacks and exploitation of potential vulnerabilities. No version information was available from the scan. \\
\bottomrule
\end{tabular}
\end{table}

\section{Consolidated Risk Assessment}

The following table synthesizes findings from the security control review and the technical scan into a prioritized list of risks.

\begin{table}[h!]
\centering
\caption{Summary of Identified Risks}
\label{tab:risks}
\begin{tabular}{p{0.1\linewidth} p{0.25\linewidth} p{0.4\linewidth} l}
\toprule
\textbf{Risk ID} & \textbf{Risk Title} & \textbf{Description} & \textbf{Severity} \\
\midrule
RISK-001 & Lack of MFA on Email & The absence of Multi-Factor Authentication on email accounts makes them highly susceptible to compromise via phishing or credential stuffing attacks. & \textbf{Critical} \\
\addlinespace
RISK-002 & Exposed SSH Management Port & The SSH service on port 22 is open to the public internet, exposing it to automated brute-force attacks and potential remote code execution if a vulnerability exists. & \textbf{High} \\
\addlinespace
RISK-003 & No Acceptable Use Policy (AUP) & The lack of a formal AUP means there are no clear, enforceable rules for employees regarding the use of company assets, data handling, and security responsibilities. & \textbf{High} \\
\bottomrule
\end{tabular}
\end{table}

\section{Recommendations}

Based on the risks identified in this assessment, the following remediation actions are recommended. These are prioritized to address the most critical vulnerabilities first.

\begin{enumerate}
    \item \textbf{Remediate RISK-001 (MFA on Email):}
    \begin{itemize}
        \item \textbf{Immediately} enable and enforce Multi-Factor Authentication for all user mailboxes.
        \item Prioritize implementation for privileged accounts (administrators, executives) before rolling out to the entire organization.
        \item Consider using phishing-resistant MFA methods like FIDO2 security keys over SMS-based codes.
    \end{itemize}
    
    \item \textbf{Remediate RISK-002 (Exposed SSH):}
    \begin{itemize}
        \item \textbf{Immediately} restrict access to the SSH service (port 22) using a firewall. Access should be limited to specific, trusted IP addresses (e.g., corporate VPN, administrator workstations).
        \item If remote access is required from untrusted networks, implement a Virtual Private Network (VPN) and require users to connect to the VPN before accessing internal resources like SSH.
        \item Ensure the SSH service is configured securely by disabling root login and enforcing key-based authentication instead of passwords.
    \end{itemize}
    
    \item \textbf{Remediate RISK-003 (No AUP):}
    \begin{itemize}
        \item Develop a formal Acceptable Use Policy that clearly defines the rules for using company technology, data, and internet services.
        \item Communicate the policy to all employees and require them to formally acknowledge their understanding and agreement.
        \item Integrate the AUP into the new employee onboarding process and the annual security awareness training program.
    \end{itemize}
\end{enumerate}

\end{document}
```