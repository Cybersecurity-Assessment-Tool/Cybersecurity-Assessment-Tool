```latex
\documentclass[12pt]{article}

% Preamble: Required Packages
\usepackage[margin=1in]{geometry}
\usepackage{pifont} % For checkmarks and crosses
\usepackage{booktabs} % For professional tables
\usepackage{hyperref} % For clickable links and references
\usepackage{url} % For formatting URLs
\usepackage{seqsplit} % For splitting long strings to prevent overflow
\usepackage{graphicx}
\usepackage{xcolor}
\usepackage{fancyhdr}
\usepackage{lastpage}

% --- Document Setup ---
\hypersetup{
    colorlinks=true,
    linkcolor=blue,
    filecolor=magenta,      
    urlcolor=cyan,
    pdftitle={Cybersecurity Posture Assessment Report},
    pdfpagemode=FullScreen,
}

% --- Custom Colors ---
\definecolor{darkred}{rgb}{0.55, 0.0, 0.0}
\definecolor{darkgreen}{rgb}{0.0, 0.39, 0.0}

% --- Header and Footer ---
\pagestyle{fancy}
\fancyhf{} % Clear all header and footer fields
\fancyhead[L]{Cybersecurity Posture Assessment Report}
\fancyhead[R]{\textbf{[Organization Name]}}
\fancyfoot[C]{Page \thepage\ of \pageref{LastPage}}
\renewcommand{\headrulewidth}{0.4pt}
\renewcommand{\footrulewidth}{0.4pt}

% --- Document Start ---
\begin{document}

% --- Title Page ---
\begin{titlepage}
    \centering
    \vspace*{1cm}
    
    \includegraphics[width=0.4\textwidth]{example-image-a} % Placeholder logo
    
    \vspace{1.5cm}
    
    \Huge
    \textbf{Cybersecurity Posture Assessment Report}
    
    \vspace{1.5cm}
    
    \Large
    Prepared for: \textbf{[Organization Name]}
    
    \vspace{1cm}
    
    \large
    Date of Report: \today
    
    \vfill
    
    \normalsize
    \textit{This report contains sensitive information regarding the security posture of the organization. Distribution should be limited to authorized personnel only.}
    
\end{titlepage}

\tableofcontents
\newpage

% --- Section 1: Executive Summary ---
\section{Executive Summary}
This report provides a comprehensive assessment of the cybersecurity posture for \textbf{[Organization Name]}, based on an analysis of organizational security controls, a network scan of external infrastructure, and a review of pre-existing risks. The assessment was conducted on \today.

The analysis reveals a mixed security posture. The organization has demonstrated a strong commitment to identity and access management by implementing Multi-Factor Authentication (MFA) across email, computer logins, and sensitive data systems. This is a critical and commendable control that significantly reduces the risk of unauthorized access.

However, several critical gaps were identified in foundational security policies and employee training. The absence of an Acceptable Use Policy (AUP) and a formal security awareness training program for employees represents a significant risk. These human-layer vulnerabilities leave the organization highly susceptible to social engineering, phishing attacks, and insider threats.

Furthermore, technical scanning identified an open port for unencrypted web traffic (HTTP Port 80). This exposes any data transmitted to or from the web service to interception, posing a direct risk to data confidentiality and integrity.

In summary, while strong authentication controls are in place, the lack of fundamental security policies and training, combined with a key technical vulnerability, creates an environment where a security incident is likely. This report provides actionable recommendations to address these high-priority risks and strengthen the overall security posture.

% --- Section 2: Organizational Information ---
\section{Organizational Information}
The following details were used as the basis for this assessment. Due to the anonymized nature of the provided data, placeholders have been used.

\begin{table}[h!]
\centering
\begin{tabular}{@{}ll@{}}
\toprule
\textbf{Attribute} & \textbf{Value} \\ \midrule
Organization Name & \textbf{[Organization Name]} \\
Primary Email Domain & \texttt{[Domain]} \\
External IP Address (Source) & \texttt{[Client IP]} \\
Target IP Address (Scanned) & \texttt{[Target IP]} \\ \bottomrule
\end{tabular}
\caption{Client and Target Information}
\label{tab:org_info}
\end{table}

% --- Section 3: Security Control Review ---
\section{Security Control Review}
A review of the organization's self-reported security controls was conducted via a questionnaire. The results highlight both strengths and critical weaknesses in the current security program.

\begin{table}[h!]
\centering
\begin{tabular}{@{}p{0.7\linewidth}c@{}}
\toprule
\textbf{Security Control Question} & \textbf{Status} \\ \midrule
Do you require MFA to access email? & \textcolor{darkgreen}{\ding{51}} \\
Do you require MFA to log into computers? & \textcolor{darkgreen}{\ding{51}} \\
Do you require MFA to access sensitive data systems? & \textcolor{darkgreen}{\ding{51}} \\
Does your organization have an employee acceptable use policy? & \textcolor{darkred}{\ding{55}} \\
Does your organization do security awareness training for new employees? & \textcolor{darkred}{\ding{55}} \\
Does your organization do security awareness training for all employees at least once per year? & \textcolor{darkred}{\ding{55}} \\ \bottomrule
\end{tabular}
\caption{Organizational Security Controls Questionnaire}
\label{tab:controls}
\end{table}

\subsection{Analysis}
\textbf{Strengths:} The mandatory implementation of MFA for email, computer access, and sensitive systems is an excellent security practice. This greatly mitigates risks associated with compromised credentials.

\textbf{Critical Gaps:}
\begin{itemize}
    \item \textbf{No Acceptable Use Policy (AUP):} The absence of an AUP means there are no formally documented rules for how employees should use company technology and data. This can lead to unintentional data exposure, misuse of assets, and a lack of legal recourse in the event of an insider incident.
    \item \textbf{No Security Awareness Training:} The complete lack of security training for both new and existing employees is a severe vulnerability. Employees are the first line of defense against phishing and social engineering. Without training, they are significantly more likely to fall victim to attacks, potentially compromising the strong MFA controls already in place.
\end{itemize}

% --- Section 4: Technical Scan Results ---
\section{Technical Scan Results}
An external network scan was performed on the target IP address \texttt{[Target IP]} to identify open ports and exposed services.

\subsection{Nmap Scan Findings}
The scan revealed the following open port:

\begin{table}[h!]
\centering
\begin{tabular}{@{}llll@{}}
\toprule
\textbf{Port} & \textbf{State} & \textbf{Service} & \textbf{Description} \\ \midrule
80/tcp & Open & HTTP & Hypertext Transfer Protocol (Unencrypted) \\ \bottomrule
\end{tabular}
\caption{Open Ports on \texttt{[Target IP]}}
\label{tab:nmap}
\end{table}

\subsection{Analysis}
The presence of an open Port 80 indicates that a web server is operating and accepting connections over HTTP. HTTP is an unencrypted protocol, meaning that all data exchanged between a user's browser and the server, including login credentials, session cookies, or sensitive information, is transmitted in cleartext. This makes the communication vulnerable to eavesdropping and Man-in-the-Middle (MitM) attacks. The industry standard is to use HTTPS (Port 443), which encrypts this traffic.

% --- Section 5: Consolidated Risk Assessment ---
\section{Consolidated Risk Assessment}
The following table synthesizes findings from the security control review and technical scan into a prioritized list of risks.

\begin{table}[h!]
\centering
\begin{tabular}{@{}p{0.3\linewidth}p{0.5\linewidth}l@{}}
\toprule
\textbf{Risk Title} & \textbf{Description} & \textbf{Severity} \\ \midrule
\textbf{Lack of Security Awareness Training} & Employees are not trained to identify or respond to phishing, social engineering, or other common cyber threats. This makes them a primary target for attackers. & \textbf{High} \\
\addlinespace
\textbf{Unencrypted Web Traffic (HTTP)} & The use of HTTP on an external-facing web server exposes all transmitted data to interception, compromising confidentiality and integrity. & \textbf{High} \\
\addlinespace
\textbf{Lack of Security Policies (AUP)} & Without a formal AUP, there is no enforceable standard for employee behavior regarding IT assets, increasing the risk of misuse and insider threat. & \textbf{High} \\
\bottomrule
\end{tabular}
\caption{Summary of Identified Risks}
\label{tab:risks}
\end{table}

% --- Section 6: Recommendations ---
\section{Recommendations}
The following actionable recommendations are provided to mitigate the identified risks and improve the overall security posture of \textbf{[Organization Name]}.

\subsection{High Priority / Immediate Actions}
\begin{enumerate}
    \item \textbf{Remediate Unencrypted Web Traffic:}
    \begin{itemize}
        \item Immediately configure the web server at \texttt{[Target IP]} to redirect all HTTP traffic on Port 80 to HTTPS on Port 443.
        \item Ensure a valid TLS certificate is installed and properly configured.
        \item Implement HTTP Strict Transport Security (HSTS) to enforce encrypted connections.
    \end{itemize}
    
    \item \textbf{Implement Foundational Security Training:}
    \begin{itemize}
        \item Procure and enroll all current employees in a foundational security awareness training program.
        \item This training must cover, at a minimum: phishing identification, password hygiene, and safe browsing habits.
    \end{itemize}
    
    \item \textbf{Develop and Implement an Acceptable Use Policy (AUP):}
    \begin{itemize}
        \item Draft a formal AUP that clearly defines the rules for using company networks, devices, and data.
        \item Require all employees to read and formally acknowledge the policy.
    \end{itemize}
\end{enumerate}

\subsection{Medium Priority / Long-Term Actions}
\begin{enumerate}
    \item \textbf{Establish an Ongoing Training Program:}
    \begin{itemize}
        \item Integrate security awareness training into the onboarding process for all new hires.
        \item Implement a mandatory annual security refresher course and regular phishing simulation exercises for all staff to ensure continuous vigilance.
    \end{itemize}

    \item \textbf{Initiate a Vulnerability Management Program:}
    \begin{itemize}
        \item Schedule regular, authenticated vulnerability scans of all external and internal network assets.
        \item Establish a process for triaging and remediating identified vulnerabilities based on their severity.
    \end{itemize}
\end{enumerate}

\end{document}
```