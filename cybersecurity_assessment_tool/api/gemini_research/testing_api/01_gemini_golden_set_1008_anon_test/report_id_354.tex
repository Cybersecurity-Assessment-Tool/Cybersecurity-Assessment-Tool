```latex
\documentclass[12pt]{article}

% Preamble: Required Packages
\usepackage[margin=1in]{geometry}
\usepackage{pifont} % For checkmarks and crosses
\usepackage{booktabs} % For professional tables
\usepackage{hyperref} % For clickable links
\usepackage{url} % For URL formatting
\usepackage{seqsplit} % To split long strings in texttt
\usepackage{graphicx}
\usepackage[table]{xcolor} % For table cell colors

% Document Metadata
\hypersetup{
    colorlinks=true,
    linkcolor=blue,
    filecolor=magenta,      
    urlcolor=cyan,
    pdftitle={Cybersecurity Posture Assessment Report},
    pdfauthor={Cybersecurity Analyst},
    pdfsubject={Security Analysis},
    pdfkeywords={Security, Report, Analysis},
}

\newcommand{\yes}{\ding{51}}
\newcommand{\no}{\ding{55}}

\begin{document}

% --- Title Page ---
\begin{titlepage}
    \centering
    \vspace*{1cm}
    \Huge \textbf{Cybersecurity Posture Assessment Report}
    \vspace{1.5cm}
    \large
    \begin{tabular}{ll}
        \textbf{Prepared For:} & \textbf{[Organization Name]} \\
        \textbf{Date of Report:} & \today \\
        \textbf{Author:} & Cybersecurity Analyst \\
    \end{tabular}
    \vfill
    \small
    \textit{This report is confidential and intended solely for the use of \textbf{[Organization Name]}. It contains a detailed analysis of security controls, technical scan results, and identified risks. Unauthorized distribution is prohibited.}
\end{titlepage}

\tableofcontents
\newpage

% --- 1. Executive Summary ---
\section{Executive Summary}
This report provides a comprehensive cybersecurity assessment for \textbf{[Organization Name]}, based on an analysis of organizational security controls, an external network scan, and a review of pre-existing risks. The assessment was conducted to identify vulnerabilities, evaluate the effectiveness of current security measures, and provide actionable recommendations to enhance the organization's security posture.

The analysis revealed a mixed security posture. While the organization has implemented foundational controls such as Multi-Factor Authentication (MFA) for email and computer access, critical gaps were identified. The most significant risks stem from policy and process deficiencies, including:
\begin{itemize}
    \item \textbf{Critical Risk:} The absence of MFA for accessing sensitive data systems.
    \item \textbf{High Risk:} The lack of a formal security awareness training program for both new and existing employees.
\end{itemize}

On a technical level, the external network scan of the target IP address (\texttt{[Target IP]}) did not identify any open ports, indicating a positive security configuration for the scanned asset. Notably, this scan's finding that port 80 is closed contradicts a pre-existing documented risk. This suggests that either the risk has been remediated or the risk register is outdated.

Recommendations prioritize addressing the identified critical and high-risk gaps to significantly reduce the organization's exposure to common cyber threats such as data breaches and social engineering attacks.

% --- 2. Organizational Information ---
\section{Organizational Information}
The following details were used as the basis for this assessment. Due to the anonymized nature of the provided data, placeholders have been used where necessary.

\begin{table}[h!]
\centering
\caption{Client Information}
\begin{tabular}{@{}ll@{}}
\toprule
\textbf{Attribute} & \textbf{Value} \\ \midrule
Organization Name & \textbf{[Organization Name]} \\
Primary Domain & \texttt{[Domain]} \\
External IP Scanned & \texttt{[Client IP]} \\
Target of Network Scan & \texttt{[Target IP]} \\ \bottomrule
\end{tabular}
\end{table}

% --- 3. Security Control Review ---
\section{Security Control Review}
A review of the organization's security controls was conducted via a questionnaire. The responses highlight areas of strength and significant weakness in the current security framework. "No" answers indicate control gaps that directly translate to increased risk.

\begin{table}[h!]
\centering
\caption{Security Questionnaire Analysis}
\label{tab:questionnaire}
\begin{tabular}{@{}p{0.7\linewidth}cc@{}}
\toprule
\textbf{Control Question} & \textbf{Response} & \textbf{Status} \\ \midrule
Do you require MFA to access email? & Yes & \yes \\
Do you require MFA to log into computers? & Yes & \yes \\
\rowcolor{red!20}
Do you require MFA to access sensitive data systems? & No & \no \\
Does your organization have an employee acceptable use policy? & Yes & \yes \\
\rowcolor{red!20}
Does your organization do security awareness training for new employees? & No & \no \\
\rowcolor{red!20}
Does your organization do security awareness training for all employees at least once per year? & No & \no \\ \bottomrule
\end{tabular}
\end{table}

\subsection*{Analysis of Gaps}
\begin{itemize}
    \item \textbf{MFA on Sensitive Systems:} The lack of MFA on systems containing sensitive data is a critical vulnerability. Should an employee's credentials be compromised, an attacker could gain direct access to the organization's most valuable information.
    \item \textbf{Security Awareness Training:} The complete absence of a security awareness training program leaves the organization highly susceptible to phishing, social engineering, and insider threats due to unintentional employee error. This is a foundational security control that is currently missing.
\end{itemize}

% --- 4. Technical Scan Results ---
\section{Technical Scan Results}
An external network scan was performed on the designated target to identify open ports and exposed services.

\begin{itemize}
    \item \textbf{Target IP Address:} \texttt{[Target IP]}
    \item \textbf{Scanner Used:} Nmap
    \item \textbf{Scan Date:} \today
\end{itemize}

The scan revealed that the target host is online, but no open ports were discovered. The status of all scanned ports was `closed`.

\begin{table}[h!]
\centering
\caption{Nmap Scan Port Summary for \texttt{[Target IP]}}
\label{tab:nmap}
\begin{tabular}{@{}ccc@{}}
\toprule
\textbf{Port} & \textbf{State} & \textbf{Service} \\ \midrule
80 & closed & http \\ \bottomrule
\end{tabular}
\end{table}

\subsection*{Technical Findings}
The results indicate a strong perimeter security posture for the scanned asset, as no services are exposed to the public internet. This finding is positive; however, it contradicts the pre-existing risk documented in the organization's risk register which states "Port 80 is open." This discrepancy suggests the risk register may be outdated or the risk pertains to a different asset.

% --- 5. Consolidated Risk Assessment ---
\section{Consolidated Risk Assessment}
The following table synthesizes findings from the security control review, technical scan, and pre-existing risk data. Risks are prioritized based on their potential impact on the organization.

\begin{table}[h!]
\centering
\caption{Summary of Identified Risks}
\label{tab:risks}
\begin{tabular}{@{}p{0.3\linewidth}p{0.5\linewidth}l@{}}
\toprule
\textbf{Risk Name} & \textbf{Overview} & \textbf{Severity} \\ \midrule
\rowcolor{red!40}
No MFA for Sensitive Data Systems & The absence of a second authentication factor allows for unauthorized access to critical data if credentials are stolen. & Critical \\
\rowcolor{orange!40}
Lack of Security Awareness Training & Employees are not trained to recognize or respond to cyber threats (e.g., phishing), making them a primary target for attackers. & High \\
\rowcolor{yellow!40}
Outdated Risk Register & A pre-existing risk ("Unencrypted Web Server") was not validated by the current scan, indicating that risk documentation may not be current. & Medium \\
\bottomrule
\end{tabular}
\end{table}

% --- 6. Recommendations ---
\section{Recommendations}
Based on the consolidated risk assessment, the following actions are recommended to mitigate the identified vulnerabilities and strengthen the overall security posture of \textbf{[Organization Name]}.

\subsection{Immediate Priority (Critical Risk)}
\begin{enumerate}
    \item \textbf{Implement MFA on All Sensitive Systems:}
    \begin{itemize}
        \item \textbf{Action:} Enforce mandatory Multi-Factor Authentication for all user accounts (including administrative and service accounts) that have access to databases, file shares, or applications containing sensitive or regulated data.
        \item \textbf{Impact:} Drastically reduces the risk of a data breach resulting from compromised credentials.
    \end{itemize}
\end{enumerate}

\subsection{High Priority}
\begin{enumerate}
    \setcounter{enumi}{1}
    \item \textbf{Establish a Security Awareness Training Program:}
    \begin{itemize}
        \item \textbf{Action:} Develop and implement a formal security awareness program. This must include mandatory training for all new hires during onboarding and annual refresher training for all current employees.
        \item \textbf{Content:} Training should cover phishing identification, password security, acceptable use of company assets, and incident reporting procedures.
        \item \textbf{Impact:} Creates a security-conscious culture and reduces the likelihood of human error leading to a security incident.
    \end{itemize}
\end{enumerate}

\subsection{Medium Priority}
\begin{enumerate}
    \setcounter{enumi}{2}
    \item \textbf{Review and Update the Risk Register:}
    \begin{itemize}
        \item \textbf{Action:} Conduct a full review of the existing risk register. Validate all documented risks to confirm they are still active and accurately assessed.
        \item \textbf{Impact:} Ensures that security efforts and resources are focused on current, relevant threats to the organization.
    \end{itemize}
\end{enumerate}

\end{document}
```