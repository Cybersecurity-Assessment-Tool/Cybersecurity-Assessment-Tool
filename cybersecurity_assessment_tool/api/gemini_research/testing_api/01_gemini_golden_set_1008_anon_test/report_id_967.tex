```latex
\documentclass[12pt]{article}

% --- PACKAGES ---
\usepackage[margin=1in]{geometry}
\usepackage{pifont} % For \ding
\usepackage{booktabs} % For professional tables
\usepackage{hyperref} % For clickable links
\usepackage{url} % For URL formatting
\usepackage{seqsplit} % To split long strings in texttt
\usepackage{graphicx}
\usepackage{xcolor}
\usepackage{fancyhdr}
\usepackage{lastpage}

% --- DOCUMENT SETUP ---
\definecolor{darkblue}{rgb}{0.0, 0.0, 0.55}
\definecolor{darkred}{rgb}{0.55, 0.0, 0.0}

\hypersetup{
    colorlinks=true,
    linkcolor=darkblue,
    filecolor=darkblue,      
    urlcolor=darkblue,
    citecolor=darkblue,
}

% --- HEADER & FOOTER ---
\pagestyle{fancy}
\fancyhf{} % Clear all header and footer fields
\fancyhead[L]{Cybersecurity Posture Assessment}
\fancyhead[R]{\textbf{[Organization Name]}}
\fancyfoot[C]{\thepage\ of \pageref{LastPage}}
\renewcommand{\headrulewidth}{0.4pt}
\renewcommand{\footrulewidth}{0.4pt}

% --- TITLE ---
\title{
    \vspace{2cm}
    \textbf{\Huge Cybersecurity Posture Assessment Report}
    \vspace{1cm}
    \hrule
    \vspace{0.5cm}
    {\Large Prepared for: \textbf{[Organization Name]}}
    \vspace{2cm}
}
\author{Cybersecurity Analysis Division}
\date{\today}

% --- DOCUMENT START ---
\begin{document}

\maketitle
\thispagestyle{empty}
\newpage

\tableofcontents
\newpage

% --- EXECUTIVE SUMMARY ---
\section*{Executive Summary}

This report provides a comprehensive analysis of the cybersecurity posture for \textbf{[Organization Name]}, based on a combination of organizational data, security control questionnaires, and external network scanning. The assessment was conducted to identify key vulnerabilities, evaluate existing controls, and provide actionable recommendations to mitigate identified risks.

The analysis reveals several \textbf{critical} and \textbf{high-risk} security deficiencies that require immediate attention. The most significant findings include:

\begin{itemize}
    \item \textbf{Critical Lack of Multi-Factor Authentication (MFA):} The organization does not enforce MFA for accessing email or for logging into employee computers. This represents a critical vulnerability, as it leaves the organization highly susceptible to account takeover attacks resulting from phishing or credential compromise.
    
    \item \textbf{Absence of Foundational Security Policies and Training:} There is no formal employee acceptable use policy, nor is there a security awareness training program for new or existing employees. This indicates a significant gap in security governance and culture, substantially increasing the risk of human error leading to a security incident.
    
    \item \textbf{Exposed Administrative Services:} An external network scan identified an open SSH port (22/TCP) on the public-facing IP address \texttt{[Client IP]}. While SSH is a standard administrative protocol, its exposure is a notable risk, especially when combined with the lack of MFA and security awareness.
\end{itemize}

These findings collectively point to a reactive and underdeveloped security posture. This report outlines specific, prioritized recommendations to address these gaps, strengthen defenses, and build a more resilient security foundation. We strongly advise that the recommendations in the "Critical" and "High" risk categories be actioned as a matter of urgency.

\newpage

% --- ORGANIZATIONAL INFORMATION ---
\section*{1. Organizational Information}

This section details the organizational data provided for the assessment. As the data was anonymized, placeholders are used where necessary.

\begin{table}[h!]
\centering
\begin{tabular}{@{}ll@{}}
\toprule
\textbf{Attribute} & \textbf{Value} \\ \midrule
Client Organization & \textbf{[Organization Name]} \\
Primary Email Domain & \texttt{[Domain]} \\
External IP Address Scanned & \texttt{[Client IP]} \\ \bottomrule
\end{tabular}
\caption{Client Organizational Details}
\end{table}

% --- SECURITY CONTROL REVIEW ---
\section*{2. Security Control Review}

The following table summarizes the organization's responses to a security controls questionnaire. A green checkmark (\textcolor{green}{\ding{51}}) indicates a positive control is in place, while a red 'X' (\textcolor{darkred}{\ding{55}}) indicates a control gap that introduces risk.

\begin{table}[h!]
\centering
\begin{tabular}{@{}p{0.75\linewidth}c@{}}
\toprule
\textbf{Control Question} & \textbf{Response} \\ \midrule
Do you require MFA to access email? & \textcolor{darkred}{\ding{55}} \\
Do you require MFA to log into computers? & \textcolor{darkred}{\ding{55}} \\
Do you require MFA to access sensitive data systems? & \textcolor{green}{\ding{51}} \\
Does your organization have an employee acceptable use policy? & \textcolor{darkred}{\ding{55}} \\
Does your organization do security awareness training for new employees? & \textcolor{darkred}{\ding{55}} \\
Does your organization do security awareness training for all employees at least once per year? & \textcolor{darkred}{\ding{55}} \\ \bottomrule
\end{tabular}
\caption{Security Controls Questionnaire Results}
\end{table}

\subsection*{Analysis of Control Gaps}
The questionnaire reveals critical gaps in fundamental security controls. The absence of MFA for primary access points like email and workstations, coupled with a complete lack of security policies and training, creates an environment where both technical and human-related security incidents are highly probable. While MFA on sensitive systems is a positive step, it is undermined by the weak security of the primary accounts used to access them.

\newpage

% --- TECHNICAL SCAN RESULTS ---
\section*{3. Technical Scan Results}

An external network vulnerability scan was performed against the organization's public-facing IP address. The scan was conducted to identify open ports and exposed services that could be leveraged by an attacker.

\begin{itemize}
    \item \textbf{Target IP Address:} \texttt{[Target IP]}
    \item \textbf{Scan Date:} Scan conducted prior to this report date.
    \item \textbf{Status:} Host is UP.
\end{itemize}

\begin{table}[h!]
\centering
\begin{tabular}{@{}lllll@{}}
\toprule
\textbf{Port} & \textbf{Protocol} & \textbf{State} & \textbf{Service} & \textbf{Notes} \\ \midrule
22 & TCP & \textbf{OPEN} & ssh & Secure Shell (SSH) is a common protocol for remote \\
 & & & (assumed) & administration. No version information was available. \\
\bottomrule
\end{tabular}
\caption{Open Ports Detected on \texttt{[Client IP]}}
\end{table}

\subsection*{Analysis of Technical Findings}
The scan identified that port 22 (SSH) is open to the internet. This service is a primary target for brute-force and credential-stuffing attacks. Given the lack of enforced MFA on employee computers, a compromised user credential could potentially be used to gain unauthorized remote access to a critical system via this exposed service. The risk is further amplified by the absence of security awareness training, making employees more likely to fall for phishing attacks that could harvest their credentials.

% --- PRE-EXISTING RISKS ---
\section*{4. Pre-Existing Risk Register}
An analysis of the provided risk data was conducted. The data indicated that there were \textbf{no pre-existing vulnerabilities or risks} documented.

\newpage

% --- CONSOLIDATED RISK ASSESSMENT ---
\section*{5. Consolidated Risk Assessment}

This section correlates the findings from the security control review and the technical scan to provide a consolidated view of the primary risks facing the organization.

\begin{table}[h!]
\centering
\begin{tabular}{@{}p{0.2\linewidth}p{0.15\linewidth}p{0.55\linewidth}@{}}
\toprule
\textbf{Risk Title} & \textbf{Severity} & \textbf{Overview} \\ \midrule
\textbf{Lack of Essential MFA} & \textbf{\textcolor{darkred}{CRITICAL}} & The absence of MFA on email and computer logins exposes the organization to a high likelihood of account compromise via phishing, password spraying, or credential stuffing. A single compromised password could grant an attacker widespread access. \\
\addlinespace
\textbf{No Security Policies or Training} & \textbf{\textcolor{orange}{HIGH}} & Without an Acceptable Use Policy or security awareness training, employees are unaware of their security responsibilities. This creates a significant "human firewall" weakness, making the organization highly vulnerable to social engineering and accidental data breaches. \\
\addlinespace
\textbf{Exposed SSH Administrative Service} & \textbf{\textcolor{yellow!80!black}{MEDIUM}} & An open SSH port provides a direct vector for attackers to attempt unauthorized access. While a standard service, its risk is elevated to Medium due to the lack of compensating controls like MFA and strong password policies, which are not currently enforced. \\
\bottomrule
\end{tabular}
\caption{Summary of Identified Risks}
\end{table}

% --- RECOMMENDATIONS ---
\section*{6. Recommendations}

The following recommendations are prioritized based on the risk assessment. It is strongly advised to address the Critical and High severity items immediately.

\subsection*{Critical Priority}
\begin{enumerate}
    \item \textbf{Implement Multi-Factor Authentication (MFA) Immediately:}
    \begin{itemize}
        \item \textbf{Action:} Enforce MFA for all user accounts across all critical systems, prioritizing email (e.g., Office 365, Google Workspace) and VPN access.
        \item \textbf{Action:} Enforce MFA for all computer logins (both local and remote) for all employees, especially those with administrative privileges.
        \item \textbf{Justification:} This is the single most effective control to prevent account takeover attacks and will immediately mitigate the highest-priority risk.
    \end{itemize}
\end{enumerate}

\subsection*{High Priority}
\begin{enumerate}
    \setcounter{enumi}{1}
    \item \textbf{Develop and Implement Foundational Security Policies:}
    \begin{itemize}
        \item \textbf{Action:} Create and distribute a formal Acceptable Use Policy (AUP) that all employees must read and acknowledge. This policy should define the rules for using company assets, data, and networks.
        \item \textbf{Justification:} An AUP establishes a baseline for secure behavior and provides a formal framework for enforcing security standards.
    \end{itemize}

    \item \textbf{Establish a Security Awareness Training Program:}
    \begin{itemize}
        \item \textbf{Action:} Implement a mandatory security awareness training module for all new hires during their onboarding process.
        \item \textbf{Action:} Roll out an annual security awareness training and phishing simulation program for all existing employees.
        \item \textbf{Justification:} Training empowers employees to recognize and report security threats like phishing, reducing the likelihood of human-error-related incidents.
    \end{itemize}
\end{enumerate}

\subsection*{Medium Priority}
\begin{enumerate}
    \setcounter{enumi}{3}
    \item \textbf{Secure the Exposed SSH Service:}
    \begin{itemize}
        \item \textbf{Action:} If SSH access is required from the internet, restrict access to a whitelist of trusted IP addresses (e.g., office or administrator home IPs).
        \item \textbf{Action:} If external access is not required, block port 22 at the network firewall.
        \item \textbf{Action:} Enforce the use of key-based authentication instead of passwords for all SSH access and disable password-based logins.
        \item \textbf{Justification:} These actions reduce the attack surface by preventing unauthorized or automated login attempts against this sensitive administrative service.
    \end{itemize}
\end{enumerate}

\end{document}
```