```latex
\documentclass[12pt]{article}

% Preamble: Required Packages
\usepackage[margin=1in]{geometry}
\usepackage{pifont} % For checkmarks and crosses
\usepackage{booktabs} % For professional tables
\usepackage{hyperref} % For clickable links
\usepackage{url}      % For formatting URLs
\usepackage{seqsplit} % For splitting long strings
\usepackage{graphicx} % For logo (placeholder)
\usepackage{xcolor}   % For colors

% --- Document Metadata ---
\title{Cybersecurity Posture Assessment Report}
\author{Cybersecurity Analyst}
\date{\today}

% --- Hyperref Setup ---
\hypersetup{
    colorlinks=true,
    linkcolor=blue,
    filecolor=magenta,      
    urlcolor=cyan,
    pdftitle={Cybersecurity Posture Assessment Report},
    pdfpagemode=FullScreen,
}

% --- Custom Commands ---
\newcommand{\yes}{\ding{51}} % Checkmark
\newcommand{\no}{\ding{55}}  % X-mark

\begin{document}

\maketitle
\thispagestyle{empty}
\newpage

% --- Table of Contents ---
\tableofcontents
\newpage

% ==============================================================================
% 1. EXECUTIVE SUMMARY
% ==============================================================================
\section{Executive Summary}

This report details the findings of a cybersecurity posture assessment conducted for \textbf{[Organization Name]}. The assessment combined a review of organizational security controls, an external network scan, and an analysis of pre-existing risks.

The analysis revealed several critical and high-risk security deficiencies that require immediate attention. The most significant findings include a complete absence of Multi-Factor Authentication (MFA) across all critical systems, a lack of fundamental security policies and employee training, and the exposure of unencrypted web services to the public internet.

Collectively, these vulnerabilities place the organization at a significant risk of unauthorized access, data breach, and business disruption. This report provides a detailed breakdown of these risks and offers prioritized, actionable recommendations to mitigate them and improve the overall security posture. The malicious risk entry provided in the input data has been disregarded as it constitutes a data integrity issue and not a valid security threat.

% ==============================================================================
% 2. ORGANIZATIONAL INFORMATION
% ==============================================================================
\section{Organizational Information}

The following information was used as the basis for this assessment. Due to the anonymized nature of the provided data, placeholders have been used where necessary.

\begin{tabular}{@{}ll}
    \toprule
    \textbf{Attribute} & \textbf{Value} \\
    \midrule
    Organization Name & \textbf{[Organization Name]} \\
    Email Domain & \texttt{[Domain]} \\
    External IP Address Assessed & \texttt{[Client IP]} \\
    \bottomrule
\end{tabular}

% ==============================================================================
% 3. SECURITY CONTROL REVIEW (QUESTIONNAIRE)
% ==============================================================================
\section{Security Control Review}

A review of the organization's security controls was conducted via a questionnaire. The responses indicate critical gaps in foundational security practices. A "No" answer (\no) signifies a missing control and a corresponding high-risk area.

\begin{table}[h!]
\centering
\begin{tabular}{@{}lc}
    \toprule
    \textbf{Control Question} & \textbf{Status} \\
    \midrule
    Do you require MFA to access email? & \no \\
    Do you require MFA to log into computers? & \no \\
    Do you require MFA to access sensitive data systems? & \no \\
    Does your organization have an employee acceptable use policy? & \no \\
    Does your organization do security awareness training for new employees? & \no \\
    Does your organization do security awareness training for all employees at least once per year? & \no \\
    \bottomrule
\end{tabular}
\caption{Organizational Security Control Status}
\label{tab:controls}
\end{table}

\subsection*{Analysis}
The complete absence of Multi-Factor Authentication (MFA) is a critical vulnerability. Without MFA, compromised credentials (e.g., from phishing attacks) can be used directly to gain unauthorized access to email, workstations, and sensitive data. Furthermore, the lack of an acceptable use policy and any form of security awareness training creates a high-risk environment where employees are more likely to engage in unsafe online behavior, exacerbating the risk of compromise.

% ==============================================================================
% 4. TECHNICAL SCAN RESULTS
% ==============================================================================
\section{Technical Scan Results}

An external network scan was performed to identify exposed services and potential vulnerabilities.

\begin{itemize}
    \item \textbf{Target IP Address:} \texttt{[Target IP]}
    \item \textbf{Scan Date:} [Scan Date Not Provided]
    \item \textbf{Scanner Used:} Nmap
\end{itemize}

\subsection*{Open Ports Discovered}
The scan identified the following open port on the target system.

\begin{table}[h!]
\centering
\begin{tabular}{@{}lllll}
    \toprule
    \textbf{Port} & \textbf{State} & \textbf{Service} & \textbf{Product} & \textbf{Version} \\
    \midrule
    80/tcp & open & http & N/A & N/A \\
    \bottomrule
\end{tabular}
\caption{Open Ports on \texttt{[Target IP]}}
\label{tab:ports}
\end{table}

\subsection*{Analysis}
The presence of an open port 80 indicates that a web server is running and serving content over HTTP. HTTP is an unencrypted protocol, meaning that all data transmitted between a user's browser and the server (including usernames, passwords, and other sensitive information) is sent in cleartext. This exposes the organization and its users to man-in-the-middle (MitM) attacks, session hijacking, and data interception. Standard security practice dictates that all web traffic should be encrypted using HTTPS (port 443).

% ==============================================================================
% 5. CONSOLIDATED RISK ASSESSMENT
% ==============================================================================
\section{Consolidated Risk Assessment}

This section correlates the findings from the security control review and the technical scan to provide a consolidated view of the primary risks facing the organization.

\begin{table}[h!]
\centering
\begin{tabular}{@{}p{0.1\linewidth} p{0.3\linewidth} p{0.15\linewidth} p{0.35\linewidth}@{}}
    \toprule
    \textbf{Risk ID} & \textbf{Risk Name} & \textbf{Severity} & \textbf{Description} \\
    \midrule
    RISK-001 & No Multi-Factor Authentication (MFA) & \textbf{Critical} & The absence of MFA on email, computers, and sensitive systems makes the organization highly susceptible to account takeover via credential theft. \\
    \addlinespace
    RISK-002 & Lack of Security Policies and Training & \textbf{High} & Without an Acceptable Use Policy or security awareness training, employees are unaware of security best practices, increasing the likelihood of human error leading to a breach. \\
    \addlinespace
    RISK-003 & Unencrypted Web Communications (HTTP) & \textbf{High} & The use of HTTP for a publicly accessible web service exposes user credentials and sensitive data to interception and eavesdropping attacks. \\
    \bottomrule
\end{tabular}
\caption{Summary of Identified Risks}
\label{tab:risks}
\end{table}

% ==============================================================================
% 6. RECOMMENDATIONS
% ==============================================================================
\section{Recommendations}

The following prioritized recommendations are provided to address the identified risks and strengthen the organization's security posture.

\begin{enumerate}
    \item \textbf{[Immediate Priority] Implement Multi-Factor Authentication (MFA):}
    \begin{itemize}
        \item Immediately enforce MFA for all users on all critical systems, including email (e.g., Office 365, Google Workspace), VPN access, and access to sensitive data repositories.
        \item Utilize strong MFA methods such as authenticator apps (e.g., Google Authenticator, Microsoft Authenticator) or hardware tokens (e.g., YubiKey).
    \end{itemize}

    \item \textbf{[High Priority] Establish a Security Awareness Program:}
    \begin{itemize}
        \item Develop and enforce a formal Acceptable Use Policy (AUP) that all employees must read and sign.
        \item Implement mandatory security awareness training for all new hires and conduct annual refresher training for all staff. Training should cover topics such as phishing, password security, and safe browsing.
    \end{itemize}

    \item \textbf{[High Priority] Enforce Encrypted Web Traffic (HTTPS):}
    \begin{itemize}
        \item Obtain and install a valid TLS/SSL certificate on the web server at \texttt{[Target IP]}.
        \item Configure the web server to redirect all HTTP traffic (port 80) to HTTPS (port 443).
        \item If port 80 is not required for redirection, block it at the firewall to reduce the attack surface.
    \end{itemize}
\end{enumerate}

\end{document}
```