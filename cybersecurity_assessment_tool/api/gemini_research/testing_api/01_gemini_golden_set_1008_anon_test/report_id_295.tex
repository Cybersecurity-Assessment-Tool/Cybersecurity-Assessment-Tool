```latex
\documentclass[12pt]{article}

% --- PACKAGES ---
\usepackage[margin=1in]{geometry}
\usepackage{pifont} % For checkmarks and crosses
\usepackage{booktabs} % For professional tables
\usepackage{hyperref} % For clickable links and references
\usepackage{url} % For formatting URLs
\usepackage{seqsplit} % To split long strings without breaking
\usepackage[utf8]{inputenc}

% --- DOCUMENT METADATA ---
\hypersetup{
    colorlinks=true,
    linkcolor=black,
    filecolor=magenta,      
    urlcolor=blue,
    pdftitle={Cybersecurity Posture Assessment Report},
    pdfpagemode=FullScreen,
}

\title{Cybersecurity Posture Assessment Report \\ \large for \textbf{[Organization Name]}}
\author{Cybersecurity Analysis Division}
\date{\today}

% --- DOCUMENT START ---
\begin{document}

\maketitle
\hrule
\vspace{1em}

\begin{abstract}
This report details the findings of a cybersecurity posture assessment conducted for \textbf{[Organization Name]}. The analysis is based on a combination of network scanning, a review of existing risk documentation, and an organizational security controls questionnaire. The assessment identified several critical and high-risk gaps in the current security posture, primarily related to the lack of Multi-Factor Authentication (MFA) and the absence of a formal security awareness training program. Additionally, an externally exposed administrative service was discovered. This report provides a detailed breakdown of these findings and offers actionable recommendations to mitigate the identified risks.
\end{abstract}

\tableofcontents
\newpage

% --- SECTION 1: OVERVIEW ---
\section{Executive Summary}
The primary objective of this assessment was to evaluate the overall security posture of \textbf{[Organization Name]} by correlating technical vulnerabilities with organizational security practices.

\paragraph{Key Findings:}
The assessment revealed significant areas for improvement. While foundational controls such as an Acceptable Use Policy are in place, critical security measures are absent.
\begin{itemize}
    \item \textbf{Critical Gaps in Access Control:} Multi-Factor Authentication (MFA) is not enforced for logging into computers or accessing sensitive data systems. This exposes the organization to a high risk of unauthorized access should user credentials be compromised.
    \item \textbf{Lack of Security Awareness:} The organization does not provide security awareness training to new or existing employees. This increases the susceptibility to social engineering and phishing attacks, which are common initial access vectors.
    \item \textbf{Exposed Network Services:} An external network scan identified an open Secure Shell (SSH) port. Exposed administrative services are prime targets for automated attacks and brute-force attempts.
\end{itemize}

\paragraph{Overall Posture:}
The current security posture is considered weak due to the identified gaps. The combination of inadequate access controls, a lack of employee security training, and an exposed administrative service creates a high-risk environment. The recommendations in this report are prioritized to address the most critical vulnerabilities first.

% --- SECTION 2: ORGANIZATIONAL INFORMATION ---
\section{Organizational Information}
This section provides the context for the assessment based on the information provided.

\begin{tabular}{@{}ll}
\toprule
\textbf{Attribute} & \textbf{Value} \\
\midrule
Organization Name & \textbf{[Organization Name]} \\
Email Domain & \texttt{[Domain]} \\
External IP Scanned & \texttt{[Client IP]} \\
\bottomrule
\end{tabular}

% --- SECTION 3: SECURITY CONTROL REVIEW ---
\section{Security Control Review}
The following table summarizes the organization's responses to the security controls questionnaire. A "No" response indicates a potential gap in the security framework that requires attention.

\begin{table}[h!]
\centering
\caption{Security Controls Questionnaire Analysis}
\begin{tabular}{@{}p{0.6\linewidth} c l@{}}
\toprule
\textbf{Control Question} & \textbf{Response} & \textbf{Assessment} \\
\midrule
Do you require MFA to access email? & \ding{51} (Yes) & Control Met \\
Do you require MFA to log into computers? & \ding{55} (No) & \textbf{Critical Gap} \\
Do you require MFA to access sensitive data systems? & \ding{55} (No) & \textbf{Critical Gap} \\
Does your organization have an employee acceptable use policy? & \ding{51} (Yes) & Control Met \\
Does your organization do security awareness training for new employees? & \ding{55} (No) & \textbf{High Risk} \\
Does your organization do security awareness training for all employees at least once per year? & \ding{55} (No) & \textbf{High Risk} \\
\bottomrule
\end{tabular}
\end{table}

% --- SECTION 4: TECHNICAL SCAN RESULTS ---
\section{Technical Scan Results}
An external, non-intrusive network scan was performed on the provided target IP address to identify open ports and exposed services.

\subsection{Port Scan Findings}
\begin{itemize}
    \item \textbf{Target IP:} \texttt{[Target IP]}
    \item \textbf{Scan Date:} [Scan Date]
    \item \textbf{Summary:} The scan revealed one open port, which corresponds to a common administrative service.
\end{itemize}

\begin{table}[h!]
\centering
\caption{Open Ports Detected on \texttt{[Target IP]}}
\begin{tabular}{@{}llll@{}}
\toprule
\textbf{Port} & \textbf{State} & \textbf{Service} & \textbf{Product / Version} \\
\midrule
22/tcp & open & ssh & Not Determined \\
\bottomrule
\end{tabular}
\end{table}

\paragraph{Analysis:} The presence of an open Secure Shell (SSH) port (22/tcp) represents a direct vector for external attack. Without version information, it is not possible to determine if it is vulnerable to known exploits. However, any exposed administrative service is a target for brute-force password attacks and credential stuffing. This finding, when correlated with the lack of MFA, presents a significant risk.

% --- SECTION 5: RISK ASSESSMENT ---
\section{Risk Assessment}
This section synthesizes the findings from the security control review and technical scan into a consolidated list of identified risks. The pre-existing risk register was empty.

\begin{table}[h!]
\centering
\caption{Summary of Identified Risks}
\begin{tabular}{@{}p{0.1\linewidth} p{0.25\linewidth} p{0.45\linewidth} l@{}}
\toprule
\textbf{ID} & \textbf{Risk Name} & \textbf{Description} & \textbf{Severity} \\
\midrule
RISK-001 & Lack of MFA on Endpoints & The absence of MFA on computer logins allows an attacker with valid credentials to gain unauthorized access to an employee's workstation and potentially the internal network. & \textbf{Critical} \\
\addlinespace
RISK-002 & Lack of MFA on Sensitive Systems & Sensitive data systems are not protected by MFA, making them highly vulnerable to compromise if user credentials are stolen. & \textbf{Critical} \\
\addlinespace
RISK-003 & Inadequate Security Awareness Training & Without training, employees are more likely to fall victim to phishing and other social engineering attacks, leading to credential theft or malware infection. & \textbf{High} \\
\addlinespace
RISK-004 & Exposed SSH Service & The SSH service is exposed to the public internet, making it a target for brute-force attacks and exploitation of potential vulnerabilities. & \textbf{High} \\
\bottomrule
\end{tabular}
\end{table}

% --- SECTION 6: RECOMMENDATIONS ---
\section{Recommendations}
The following actionable recommendations are provided to address the identified risks and improve the overall security posture of \textbf{[Organization Name]}.

\subsection{Immediate Actions (1-30 Days)}
\begin{itemize}
    \item \textbf{Recommendation for RISK-004 (Exposed SSH Service):}
    \begin{enumerate}
        \item Immediately review the business necessity of the publicly accessible SSH service on \texttt{[Target IP]}.
        \item If it is not required, disable the service or block it at the firewall.
        \item If it is required, restrict access to a whitelist of trusted IP addresses.
    \end{enumerate}
\end{itemize}

\subsection{Short-Term Actions (30-90 Days)}
\begin{itemize}
    \item \textbf{Recommendation for RISK-001 \& RISK-002 (Lack of MFA):}
    \begin{enumerate}
        \item Procure and deploy an MFA solution for all employees.
        \item Prioritize the rollout of MFA for all sensitive data systems, administrative accounts, and remote access solutions (e.g., VPN, SSH).
        \item Enforce MFA for all computer logins.
    \end{enumerate}
    \item \textbf{Recommendation for RISK-004 (Exposed SSH Service):}
    \begin{enumerate}
        \item Enforce public key authentication and disable password-based authentication for SSH.
        \item Implement an intrusion detection/prevention tool like Fail2Ban to automatically block IPs that exhibit malicious behavior.
        \item Ensure the SSH service is patched and running the latest stable version.
    \end{enumerate}
\end{itemize}

\subsection{Mid-Term Actions (90-180 Days)}
\begin{itemize}
    \item \textbf{Recommendation for RISK-003 (Security Awareness Training):}
    \begin{enumerate}
        \item Develop or procure a security awareness training program.
        \item Implement mandatory training for all new employees as part of their onboarding process.
        \item Schedule and conduct mandatory annual security awareness training for all staff, covering topics like phishing, password hygiene, and acceptable use.
    \end{enumerate}
\end{itemize}

% --- DOCUMENT END ---
\end{document}
```