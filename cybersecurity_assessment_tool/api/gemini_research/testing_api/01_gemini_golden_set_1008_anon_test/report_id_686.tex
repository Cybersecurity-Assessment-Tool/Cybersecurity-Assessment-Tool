```latex
\documentclass[12pt]{article}

% --- PACKAGES ---
\usepackage[margin=1in]{geometry}
\usepackage{pifont} % Required for \ding
\usepackage{booktabs} % For professional-looking tables
\usepackage{xcolor}   % For custom colors
\usepackage{graphicx}
\usepackage{hyperref}
\usepackage{url}
\usepackage{seqsplit} % To break long strings in \texttt

% --- DOCUMENT CONFIGURATION ---
\definecolor{darkblue}{rgb}{0.0, 0.0, 0.55}
\hypersetup{
    colorlinks=true,
    linkcolor=darkblue,
    filecolor=darkblue,      
    urlcolor=darkblue,
    citecolor=darkblue,
}

% --- CUSTOM COMMANDS ---
\newcommand{\yes}{\ding{51}} % Checkmark
\newcommand{\no}{\ding{55}}  % X mark

% --- DOCUMENT START ---
\begin{document}

\title{Cybersecurity Posture Assessment Report}
\author{Cybersecurity Analysis Division}
\date{\today}
\maketitle

\begin{abstract}
This report provides a comprehensive analysis of the cybersecurity posture for \textbf{[Organization Name]}. The assessment is based on a synthesis of external network scan data, a review of internal security controls via a questionnaire, and an analysis of pre-existing risk documentation. The findings indicate several critical and high-risk vulnerabilities that require immediate attention to mitigate potential threats to the organization's data and operations. Key issues identified include the absence of mandatory Multi-Factor Authentication (MFA), the presence of a critically vulnerable public-facing service, and gaps in the employee security training program.
\end{abstract}

\tableofcontents
\newpage

% ===================================================================
\section{Overview and Executive Summary}
% ===================================================================

This assessment synthesizes data from three sources to provide a holistic view of the organization's security posture. The analysis reveals significant weaknesses that expose the organization to a high likelihood of security incidents, including unauthorized access, data breaches, and ransomware attacks.

\paragraph{Key Findings:}
\begin{itemize}
    \item \textbf{Critical Service Vulnerability:} An external network scan identified a public-facing FTP server running a dangerously outdated and vulnerable version of \texttt{vsftpd} (2.3.4). This service is also configured to allow anonymous logins, presenting a direct and immediate path for an attacker.
    \item \textbf{Insufficient Access Controls:} The organization does not enforce Multi-Factor Authentication (MFA) for email or computer logins. This is a critical security gap that significantly increases the risk of account compromise.
    \item \textbf{Inadequate Security Training:} While new employees receive security training, there is no recurring annual training for all staff. This leads to a gradual decay in security awareness, making employees more susceptible to social engineering and phishing attacks.
    \item \textbf{Pre-existing OS Risk:} The organization is aware of an existing risk related to outdated Windows 7 workstations, which are no longer supported and do not receive security updates.
\end{itemize}

\paragraph{Overall Posture:} The combination of these findings places the organization in a \textbf{High-Risk} category. Immediate and decisive remediation is required to reduce the attack surface and strengthen defenses.

% ===================================================================
\section{Organizational Information}
% ===================================================================

The following information was used as the basis for this assessment. Due to the anonymized nature of the provided data, placeholders have been used where necessary.

\begin{tabular}{@{}ll}
    \toprule
    \textbf{Attribute} & \textbf{Value} \\
    \midrule
    Organization Name & \textbf{[Organization Name]} \\
    Primary Email Domain & \texttt{[Domain]} \\
    External IP (Client) & \texttt{[Client IP]} \\
    Scan Target IP & \texttt{[Target IP]} \\
    \bottomrule
\end{tabular}

% ===================================================================
\section{Security Control Review}
% ===================================================================

A review of the organization's security controls was conducted based on a standardized questionnaire. The responses highlight significant gaps in fundamental security practices, particularly concerning identity and access management.

\begin{table}[h!]
\centering
\caption{Security Controls Questionnaire Results}
\begin{tabular}{@{}p{0.8\linewidth}c@{}}
    \toprule
    \textbf{Control Question} & \textbf{Response} \\
    \midrule
    Do you require MFA to access email? & \textcolor{red}{\no} \\
    Do you require MFA to log into computers? & \textcolor{red}{\no} \\
    Do you require MFA to access sensitive data systems? & \textcolor{green}{\yes} \\
    Does your organization have an employee acceptable use policy? & \textcolor{green}{\yes} \\
    Does your organization do security awareness training for new employees? & \textcolor{green}{\yes} \\
    Does your organization do security awareness training for all employees at least once per year? & \textcolor{red}{\no} \\
    \bottomrule
\end{tabular}
\end{table}

The lack of MFA for email and general computer access is a critical weakness. Email is a primary target for phishing attacks, and a compromised account can serve as a pivot point into the rest of the network. Similarly, the absence of annual security training for all employees fails to reinforce security best practices.

% ===================================================================
\section{Technical Scan Results}
% ===================================================================

An external network scan was performed against the target IP address \texttt{[Target IP]}. The scan revealed one open port with a critically vulnerable service.

\begin{table}[h!]
\centering
\caption{Open Ports and Services Detected}
\begin{tabular}{@{}llll@{}}
    \toprule
    \textbf{Port} & \textbf{Service} & \textbf{Product / Version} & \textbf{Notes} \\
    \midrule
    21/tcp & ftp & vsftpd 2.3.4 & \textbf{CRITICAL.} Anonymous FTP login allowed. \\
    & & & This version is known to be vulnerable \\
    & & & to a backdoor (CVE-2011-2523). \\
    \bottomrule
\end{tabular}
\end{table}

\paragraph{Analysis:} The presence of \texttt{vsftpd 2.3.4} is a severe security risk. This specific version contained a backdoor that was introduced into the source code, allowing an attacker to gain a command shell on the server. Compounded by the allowance of anonymous FTP logins, this vulnerability is trivial to exploit and could lead to a complete system compromise.

% ===================================================================
\section{Consolidated Risk Assessment}
% ===================================================================

The following table correlates findings from the security questionnaire, the technical scan, and pre-existing risk documentation to provide a unified view of the top security risks.

\begin{table}[h!]
\centering
\caption{Summary of Identified Risks}
\begin{tabular}{@{}p{0.25\linewidth}p{0.45\linewidth}l@{}}
    \toprule
    \textbf{Risk Name} & \textbf{Description} & \textbf{Severity} \\
    \midrule
    \textbf{Vulnerable Public FTP Service} & A public-facing FTP server is running a version with a known remote code execution backdoor (CVE-2011-2523) and allows anonymous access. & \textbf{Critical} \\
    \addlinespace
    \textbf{Lack of MFA Enforcement} & MFA is not required for email or computer logins, making credential-based attacks (e.g., phishing, password spraying) highly effective. & \textbf{Critical} \\
    \addlinespace
    \textbf{Inadequate Security Training Program} & The absence of annual security awareness training for all staff increases susceptibility to social engineering and human error. & \textbf{High} \\
    \addlinespace
    \textbf{Outdated Operating Systems} & Workstations are running Windows 7, which is End-of-Life and no longer receives security updates from Microsoft. & \textbf{Medium} \\
    \bottomrule
\end{tabular}
\end{table}

% ===================================================================
\section{Recommendations}
% ===================================================================

The following actionable recommendations are provided to address the identified risks. They are prioritized based on severity and ease of implementation.

\subsection{Priority 1: Immediate Actions (Remediate within 72 hours)}
\begin{enumerate}
    \item \textbf{Decommission or Isolate Vulnerable FTP Server:}
    \begin{itemize}
        \item Immediately take the server at \texttt{[Target IP]} offline.
        \item Conduct a forensic analysis to determine if it has already been compromised.
        \item If FTP is a business requirement, replace it with a secure alternative like SFTP (SSH File Transfer Protocol). If FTP must be used, upgrade \texttt{vsftpd} to the latest stable version and disable anonymous access.
    \end{itemize}
\end{enumerate}

\subsection{Priority 2: High-Priority Actions (Remediate within 30 days)}
\begin{enumerate}
    \item \textbf{Enforce Universal MFA:}
    \begin{itemize}
        \item Procure and deploy an MFA solution for all employees.
        \item Prioritize enforcement on all external-facing services, including email (Office 365 / G Suite), VPN, and other cloud services.
        \item Plan the rollout of MFA for internal computer logins.
    \end{itemize}
    \item \textbf{Establish Annual Security Awareness Training:}
    \begin{itemize}
        \item Implement a mandatory security awareness training program for all employees, to be completed annually.
        \item Include modules on phishing identification, password hygiene, and acceptable use policies.
    \end{itemize}
\end{enumerate}

\subsection{Priority 3: Medium-Priority Actions (Remediate within 90 days)}
\begin{enumerate}
    \item \textbf{Accelerate Operating System Upgrades:}
    \begin{itemize}
        \item Execute the existing plan to upgrade all Windows 7 workstations to a modern, supported operating system (e.g., Windows 10/11).
        \item Ensure that any new hardware procured meets the requirements for modern operating systems.
    \end{itemize}
\end{enumerate}

\end{document}
```