```latex
\documentclass[12pt]{article}

% Preamble: Required Packages
\usepackage[margin=1in]{geometry}
\usepackage{pifont} % For check and cross marks
\usepackage{booktabs} % For professional tables
\usepackage[hidelinks]{hyperref} % For clickable links without boxes
\usepackage{url} % For URL formatting
\usepackage{seqsplit} % For splitting long strings in texttt
\usepackage{graphicx} % For logo (placeholder)

% Document Information
\title{Cybersecurity Posture Assessment Report}
\author{Cybersecurity Analysis Division}
\date{\today}

\begin{document}

\maketitle
\thispagestyle{empty}
\newpage

\tableofcontents
\newpage

% --- 1. Executive Summary ---
\section*{1. Executive Summary}

This report provides a cybersecurity assessment for \textbf{[Organization Name]}, based on an analysis of network scan data, organizational security controls, and pre-existing risk information. The assessment was conducted on \today.

The overall security posture of \textbf{[Organization Name]} is considered weak due to several critical control gaps. The most significant findings include a complete lack of Multi-Factor Authentication (MFA) for email, computer logins, and sensitive data systems. This is compounded by an exposed Secure Shell (SSH) service on the external network, creating a high-risk vector for unauthorized access. Furthermore, the absence of mandatory annual security awareness training for all employees increases susceptibility to social engineering and phishing attacks.

Immediate remediation is required to address these critical vulnerabilities. Key recommendations focus on the rapid deployment of MFA, securing externally facing services, and implementing a comprehensive security training program.

% --- 2. Organizational Information ---
\section*{2. Organizational Information}

This section details the information provided by the client. Due to the anonymized nature of the data provided, placeholders are used where necessary.

\begin{table}[h!]
\centering
\begin{tabular}{@{}ll@{}}
\toprule
\textbf{Attribute} & \textbf{Value} \\ \midrule
Organization Name & \textbf{[Organization Name]} \\
Primary Domain & \texttt{[Domain]} \\
External IP Address Scanned & \texttt{[Client IP]} \\ \bottomrule
\end{tabular}
\caption{Client Organizational Data}
\label{tab:org_info}
\end{table}

% --- 3. Security Control Review (Questionnaire Analysis) ---
\section*{3. Security Control Review}

The following table summarizes the organization's responses to a security controls questionnaire. Answers marked with a cross (\ding{55}) indicate a deviation from security best practices and represent a significant gap in the organization's defenses.

\begin{table}[h!]
\centering
\begin{tabular}{@{}lc@{}}
\toprule
\textbf{Security Control Question} & \textbf{Status} \\ \midrule
Do you require MFA to access email? & \ding{55} \\
Do you require MFA to log into computers? & \ding{55} \\
Do you require MFA to access sensitive data systems? & \ding{55} \\
Does your organization have an employee acceptable use policy? & \ding{51} \\
Does your organization do security awareness training for new employees? & \ding{51} \\
Does your organization do security awareness training for all employees at least once per year? & \ding{55} \\ \bottomrule
\end{tabular}
\caption{Security Controls Questionnaire Results (\ding{51}=Yes, \ding{55}=No)}
\label{tab:controls}
\end{table}

\subsection*{Analysis of Control Gaps}
\begin{itemize}
    \item \textbf{Lack of Multi-Factor Authentication (MFA):} The absence of MFA across all critical access points (email, endpoints, sensitive systems) is a critical vulnerability. It means that a single compromised password could lead to a full-scale breach.
    \item \textbf{Inadequate Security Awareness Training:} While new employees receive training, the lack of an annual refresher for all staff is a high-risk gap. The threat landscape evolves rapidly, and without continuous education, employees are more likely to fall victim to modern phishing and social engineering tactics.
\end{itemize}

% --- 4. Technical Scan Results ---
\section*{4. Technical Scan Results}

An external network scan was performed using Nmap to identify open ports and services on the client's public-facing infrastructure.

\begin{itemize}
    \item \textbf{Target IP Address:} \texttt{[Target IP]}
    \item \textbf{Scan Date:} Information not provided in scan metadata.
\end{itemize}

\begin{table}[h!]
\centering
\begin{tabular}{@{}llll@{}}
\toprule
\textbf{Port/Proto} & \textbf{State} & \textbf{Service} & \textbf{Analysis} \\ \midrule
22/tcp & open & ssh & Secure Shell (SSH) is a common protocol for remote \\
& & & administration. Exposing SSH directly to the internet \\
& & & is a significant risk, as it is a constant target for \\
& & & automated brute-force and credential stuffing attacks. \\
& & & No version information was obtained. \\ \bottomrule
\end{tabular}
\caption{Open Ports Detected on \texttt{[Target IP]}}
\label{tab:scan_results}
\end{table}

% --- 5. Correlated Risk Assessment ---
\section*{5. Correlated Risk Assessment}

This section synthesizes findings from the security control review and the technical scan to provide a consolidated view of the primary risks facing the organization. No pre-existing vulnerabilities were reported.

\begin{table}[h!]
\centering
\begin{tabular}{@{}p{0.2\linewidth}p{0.15\linewidth}p{0.55\linewidth}@{}}
\toprule
\textbf{Risk Title} & \textbf{Severity} & \textbf{Description} \\ \midrule
\textbf{Lack of Multi-Factor Authentication} & \textbf{Critical} & The absence of MFA for email, computer, and sensitive data access exposes the organization to a high likelihood of account compromise via phishing or credential theft. This is the most severe risk identified. \\
\addlinespace
\textbf{Exposed SSH Service without MFA} & \textbf{High} & The open SSH port on \texttt{[Target IP]} provides a direct entry point for attackers. When combined with the lack of MFA, a single compromised password could grant an attacker administrative access to a critical system. \\
\addlinespace
\textbf{Inadequate Security Awareness Program} & \textbf{High} & Without mandatory annual training, employees are less likely to recognize and report sophisticated phishing and social engineering attempts, making them the weakest link in the organization's defense. \\ \bottomrule
\end{tabular}
\caption{Summary of Identified Risks}
\label{tab:risk_summary}
\end{table}

% --- 6. Recommendations ---
\section*{6. Recommendations}

The following actionable recommendations are prioritized based on the risk severity to help \textbf{[Organization Name]} improve its security posture.

\subsection*{Priority 1: Critical Risk Mitigation}
\begin{enumerate}
    \item \textbf{Implement Multi-Factor Authentication (MFA):}
    \begin{itemize}
        \item Immediately enable MFA for all user accounts on all systems, especially email (e.g., Office 365, Google Workspace), VPNs, and administrative interfaces.
        \item Develop a plan to roll out MFA for all computer logins and access to systems containing sensitive data within the next 30 days.
    \end{itemize}
\end{enumerate}

\subsection*{Priority 2: High Risk Mitigation}
\begin{enumerate}
    \setcounter{enumi}{1}
    \item \textbf{Secure the Exposed SSH Service:}
    \begin{itemize}
        \item \textbf{Review Business Need:} Determine if exposing SSH to the public internet is necessary. If not, disable access immediately.
        \item \textbf{Restrict Access:} If required, restrict SSH access to a whitelist of trusted source IP addresses.
        \item \textbf{Strengthen Authentication:} Disable password-based authentication for SSH and enforce the use of public key cryptography.
    \end{itemize}
    \item \textbf{Enhance Security Awareness Training:}
    \begin{itemize}
        \item Institute a mandatory annual security awareness training program for all employees, contractors, and executives.
        \item Conduct periodic, simulated phishing campaigns to test and reinforce employee knowledge.
    \end{itemize}
\end{enumerate}

\end{document}
```