```latex
\documentclass[12pt]{article}

% Preamble: Required Packages
\usepackage[margin=1in]{geometry}
\usepackage{pifont} % For checkmarks and crosses
\usepackage{booktabs} % For professional tables
\usepackage{hyperref} % For clickable links
\usepackage{url} % For formatting URLs
\usepackage{seqsplit} % For splitting long strings without breaking
\usepackage{graphicx}
\usepackage{xcolor}

% Document Metadata
\title{Cybersecurity Posture Assessment Report}
\author{Cybersecurity Analysis Division}
\date{\today}

% Hyperref Setup
\hypersetup{
    colorlinks=true,
    linkcolor=blue,
    filecolor=magenta,      
    urlcolor=cyan,
    pdftitle={Cybersecurity Posture Assessment Report},
    pdfpagemode=FullScreen,
}

\begin{document}

\maketitle
\thispagestyle{empty}
\newpage

\tableofcontents
\newpage

% --- 1. Executive Summary ---
\section{Executive Summary}
This report provides a cybersecurity posture assessment for \textbf{[Organization Name]}, conducted on \today. The analysis is based on a combination of an external network scan, a review of existing risks, and a security controls questionnaire.

The assessment identified several critical and high-risk findings that require immediate attention. Key areas of concern include significant gaps in the implementation of Multi-Factor Authentication (MFA) for critical services like email and sensitive data systems. Additionally, the external network scan revealed an exposed administrative service (SSH on port 22), which increases the organization's attack surface and exposure to brute-force or credential-based attacks.

While the organization has foundational security practices in place, such as security awareness training and an acceptable use policy, the identified control gaps significantly weaken its defense against common cyber threats. Prioritized remediation of the findings detailed in this report is strongly recommended to reduce the risk of unauthorized access and potential data compromise.

% --- 2. Organizational Information ---
\section{Organizational Information}
The following information was used as the basis for this assessment. Due to the anonymized nature of the provided data, placeholders have been used where necessary.

\begin{table}[h!]
\centering
\begin{tabular}{@{}ll@{}}
\toprule
\textbf{Attribute} & \textbf{Value} \\ \midrule
Organization Name & \textbf{[Organization Name]} \\
Primary Domain & \texttt{[Domain]} \\
External IP Scanned & \texttt{[Client IP]} \\
Target of Scan & \texttt{[Target IP]} \\ \bottomrule
\end{tabular}
\caption{Client and Assessment Scope}
\label{tab:org_info}
\end{table}

% --- 3. Security Control Review ---
\section{Security Control Review}
A review of the organization's security controls was conducted via a questionnaire. The responses indicate a solid foundation in policy and training but reveal critical deficiencies in technical access controls. The "No" responses represent significant security gaps that should be addressed promptly.

\begin{table}[h!]
\centering
\begin{tabular}{@{}p{0.6\linewidth}cc@{}}
\toprule
\textbf{Control Question} & \textbf{Response} & \textbf{Status} \\ \midrule
Do you require MFA to access email? & No & \ding{55} \\
Do you require MFA to log into computers? & Yes & \ding{51} \\
Do you require MFA to access sensitive data systems? & No & \ding{55} \\
Does your organization have an employee acceptable use policy? & Yes & \ding{51} \\
Does your organization do security awareness training for new employees? & Yes & \ding{51} \\
Does your organization do security awareness training for all employees at least once per year? & Yes & \ding{51} \\ \bottomrule
\end{tabular}
\caption{Security Controls Questionnaire Results}
\label{tab:controls_review}
\end{table}

% --- 4. Technical Scan Results ---
\section{Technical Scan Results}
An external network vulnerability scan was performed against the target IP address \texttt{[Target IP]}. The scan identified the following open port.

\subsection{Open Ports}
The presence of open ports, especially for administrative services, on an external interface can provide an entry point for attackers.

\begin{table}[h!]
\centering
\begin{tabular}{@{}llll@{}}
\toprule
\textbf{Port} & \textbf{State} & \textbf{Service} & \textbf{Notes} \\ \midrule
22/tcp & open & ssh & Secure Shell (SSH) is a common administrative protocol. \\
& & & No version information was available in the scan data. \\
& & & Exposing SSH to the public internet is a high risk. \\ \bottomrule
\end{tabular}
\caption{External Port Scan Findings}
\label{tab:port_scan}
\end{table}

\subsection{Pre-existing Vulnerabilities}
The provided data on current risks indicated that there were no previously documented vulnerabilities.

% --- 5. Risk Assessment ---
\section{Risk Assessment}
The following table synthesizes findings from the security control review and the technical scan into a prioritized list of risks.

\begin{table}[h!]
\centering
\begin{tabular}{@{}p{0.1\linewidth}p{0.5\linewidth}ll@{}}
\toprule
\textbf{Risk ID} & \textbf{Description} & \textbf{Severity} & \textbf{Source} \\ \midrule
RISK-001 & \textbf{Lack of MFA on Email:} Corporate email is a primary target for phishing and account takeover. Without MFA, a compromised password directly leads to a breach. & \textbf{Critical} & Questionnaire \\
\addlinespace
RISK-002 & \textbf{Lack of MFA on Sensitive Systems:} Failure to protect sensitive data systems with MFA exposes critical assets to unauthorized access if credentials are stolen. & \textbf{Critical} & Questionnaire \\
\addlinespace
RISK-003 & \textbf{Exposed SSH Administrative Port:} The SSH service on port 22 is open to the internet, making it a target for brute-force attacks, credential stuffing, and exploitation of potential software vulnerabilities. & \textbf{High} & Network Scan \\ \bottomrule
\end{tabular}
\caption{Summary of Identified Risks}
\label{tab:risk_summary}
\end{table}

% --- 6. Recommendations ---
\section{Recommendations}
Based on the analysis, the following actions are recommended to mitigate the identified risks and improve the overall security posture of \textbf{[Organization Name]}.

\begin{enumerate}
    \item \textbf{Implement Comprehensive MFA (RISK-001, RISK-002):}
    \begin{itemize}
        \item \textbf{Immediate Priority:} Enforce mandatory MFA for all user access to the email system (e.g., Office 365, Google Workspace).
        \item \textbf{High Priority:} Deploy MFA for all systems containing sensitive or critical data, including remote access solutions (VPNs) and administrative interfaces.
    \end{itemize}

    \item \textbf{Restrict and Harden SSH Access (RISK-003):}
    \begin{itemize}
        \item \textbf{Restrict Access:} If SSH access from the public internet is not essential, block it at the firewall. If it is required, restrict access to a whitelist of trusted IP addresses (e.g., office or administrator home IPs). For broader access, place the service behind a VPN.
        \item \textbf{Harden Configuration:} Strengthen the SSH server configuration by:
        \begin{itemize}
            \item Disabling password-based authentication and requiring public key authentication.
            \item Disabling direct root login.
            \item Using a non-standard port (security by obscurity, but can reduce automated scans).
        \end{itemize}
    \end{itemize}
    
    \item \textbf{Conduct Regular Vulnerability Scanning:}
    \begin{itemize}
        \item Implement a recurring, authenticated vulnerability scanning program for both external and internal assets to proactively identify and remediate technical vulnerabilities.
    \end{itemize}
\end{enumerate}

\end{document}
```