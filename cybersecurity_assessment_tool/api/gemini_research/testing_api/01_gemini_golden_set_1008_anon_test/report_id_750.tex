```latex
\documentclass[12pt]{article}

% Preamble: Required Packages
\usepackage[margin=1in]{geometry}
\usepackage{pifont} % For checkmarks and crosses
\usepackage{booktabs} % For professional tables
\usepackage{hyperref} % For clickable links
\usepackage{url} % For URL formatting
\usepackage{seqsplit} % To split long strings in tt font
\usepackage{graphicx}
\usepackage{xcolor}
\usepackage{lastpage}
\usepackage{fancyhdr}

% Document Metadata and Hyperref Setup
\hypersetup{
    colorlinks=true,
    linkcolor=blue,
    filecolor=magenta,      
    urlcolor=cyan,
    pdftitle={Cybersecurity Posture Assessment Report},
    pdfauthor={Cybersecurity Analyst},
    pdfsubject={Security Assessment},
    pdfkeywords={Security, Analysis, Report},
    bookmarks=true
}

% Define colors for severity
\definecolor{critseverity}{HTML}{6B0504}
\definecolor{highseverity}{HTML}{A31621}
\definecolor{medseverity}{HTML}{DB7C26}
\definecolor{lowseverity}{HTML}{3A7D44}
\definecolor{infoseverity}{HTML}{3B738F}

% Header and Footer
\pagestyle{fancy}
\fancyhf{} % clear all header and footer fields
\fancyhead[L]{Cybersecurity Posture Assessment}
\fancyhead[R]{\textbf{[Organization Name]}}
\fancyfoot[C]{\thepage\ of \pageref{LastPage}}
\renewcommand{\headrulewidth}{0.4pt}
\renewcommand{\footrulewidth}{0.4pt}

% Document Title
\title{Cybersecurity Posture Assessment Report \\ \large For \textbf{[Organization Name]}}
\author{Cybersecurity Analyst}
\date{\today}

\begin{document}

\maketitle
\thispagestyle{fancy}

\begin{abstract}
This report provides a comprehensive cybersecurity assessment for \textbf{[Organization Name]}, synthesizing data from a technical network scan, an organizational security questionnaire, and a review of pre-existing risks. The analysis identifies critical security gaps, evaluates current controls, and offers actionable recommendations to enhance the organization's security posture. Key findings include significant deficiencies in Multi-Factor Authentication (MFA) for critical systems and gaps in the security awareness training program. A positive finding indicates that a previously identified vulnerability related to an unencrypted web server may have been remediated.
\end{abstract}

\tableofcontents
\newpage

\section{Executive Summary}
The primary objective of this assessment was to evaluate the current security posture of \textbf{[Organization Name]} by correlating self-reported security controls with technical scan data and known risks.

\subsection{Key Findings}
\begin{itemize}
    \item \textbf{Critical Risk - Inadequate MFA Controls:} Multi-Factor Authentication is not enforced for accessing email or sensitive data systems. This represents a critical vulnerability, as compromised credentials could lead to unauthorized access to confidential information and system control.
    \item \textbf{High Risk - Incomplete Security Training:} While annual security training is in place, new employees do not receive mandatory training upon being hired. This gap leaves the organization vulnerable, as new staff are often prime targets for social engineering attacks.
    \item \textbf{Informational - Potential Risk Remediation:} A pre-existing risk identified an open port for an unencrypted web server (Port 80). However, our recent technical scan found this port to be \textbf{closed}. This suggests the risk may have been resolved, but requires official verification and closure.
\end{itemize}

\subsection{Overall Posture}
The organization has foundational security elements in place, such as an acceptable use policy and MFA for computer logins. However, the identified gaps in MFA for critical assets and in the onboarding process for new employees significantly elevate the risk of a security incident. The recommendations in this report are prioritized to address these critical weaknesses first.

\section{Organizational Information}
The following details were used as the basis for this assessment. As per the provided data, placeholder values are used where specific information was not available.

\begin{table}[h!]
\centering
\caption{Client Organizational Details}
\begin{tabular}{@{}ll@{}}
\toprule
\textbf{Attribute} & \textbf{Value} \\ \midrule
Organization Name & \textbf{[Organization Name]} \\
Primary Domain & \texttt{[Domain]} \\
External IP Address (Source) & \texttt{[Client IP]} \\ \bottomrule
\end{tabular}
\end{table}

\section{Security Control Review}
The following table summarizes the responses from the organizational security questionnaire. Each response is assessed against cybersecurity best practices. A checkmark (\ding{51}) indicates alignment with best practices, while a cross (\ding{55}) indicates a significant gap.

\begin{table}[h!]
\centering
\caption{Security Questionnaire Analysis}
\label{tab:questionnaire}
\begin{tabular}{@{}p{0.6\linewidth}cc@{}}
\toprule
\textbf{Control Question} & \textbf{Response} & \textbf{Assessment} \\ \midrule
Do you require MFA to access email? & No & \textcolor{red}{\ding{55}} Critical Gap \\
Do you require MFA to log into computers? & Yes & \textcolor{green}{\ding{51}} Strength \\
Do you require MFA to access sensitive data systems? & No & \textcolor{red}{\ding{55}} Critical Gap \\
Does your organization have an employee acceptable use policy? & Yes & \textcolor{green}{\ding{51}} Strength \\
Does your organization do security awareness training for new employees? & No & \textcolor{red}{\ding{55}} High Risk Gap \\
Does your organization do security awareness training for all employees at least once per year? & Yes & \textcolor{green}{\ding{51}} Strength \\ \bottomrule
\end{tabular}
\end{table}

\section{Technical Scan Results}
A network scan was performed to identify externally accessible services and potential vulnerabilities.

\begin{itemize}
    \item \textbf{Target IP Address:} \texttt{[Target IP]}
    \item \textbf{Scan Date:} \today
    \item \textbf{Scanner Used:} Nmap
\end{itemize}

The scan revealed the following port status. It is notable that Port 80, which was listed as open in a pre-existing risk document, was found to be closed during this scan. This is a positive development that suggests a potential remediation has occurred.

\begin{table}[h!]
\centering
\caption{Nmap Port Scan Results for \texttt{[Target IP]}}
\label{tab:nmap}
\begin{tabular}{@{}llll@{}}
\toprule
\textbf{Port} & \textbf{State} & \textbf{Service} & \textbf{Product / Version} \\ \midrule
80/tcp & Closed & http & N/A \\ \bottomrule
\end{tabular}
\end{table}

\section{Consolidated Risk Assessment}
This section synthesizes findings from the security control review, technical scan, and pre-existing risk data into a prioritized list of risks.

\begin{table}[h!]
\centering
\caption{Summary of Identified Risks}
\label{tab:risks}
\begin{tabular}{@{}p{0.1\linewidth}p{0.25\linewidth}p{0.15\linewidth}p{0.4\linewidth}@{}}
\toprule
\textbf{Risk ID} & \textbf{Risk Name} & \textbf{Severity} & \textbf{Description} \\ \midrule
\textbf{RISK-001} & Inadequate MFA Controls & \colorbox{critseverity}{\color{white}\textbf{\phantom{i}Critical\phantom{i}}} & Lack of MFA on email and sensitive data systems exposes the organization to account takeover, data breaches, and phishing attacks. \\
\textbf{RISK-002} & Incomplete Security Training Program & \colorbox{highseverity}{\color{white}\textbf{\phantom{ii}High\phantom{ii}}} & New employees are not trained on security policies, making them highly susceptible to social engineering and accidental policy violations from day one. \\
\textbf{RISK-003} & Unverified Risk Remediation & \colorbox{infoseverity}{\color{white}\textbf{Informational}} & Pre-existing risk "Unencrypted Web Server" (Port 80) appears to be resolved based on the technical scan, but this has not been formally verified or documented. \\
\bottomrule
\end{tabular}
\end{table}

\section{Recommendations}
The following actionable recommendations are provided to address the identified risks and strengthen the overall security posture of \textbf{[Organization Name]}.

\subsection{RISK-001: Inadequate MFA Controls (Critical)}
\begin{itemize}
    \item \textbf{Immediate Action:} Prioritize and enforce MFA for all user accounts on the primary email system (e.g., Microsoft 365, Google Workspace). This is the single most effective control to prevent unauthorized email access.
    \item \textbf{Short-Term Action:} Conduct an inventory of all systems containing sensitive data. Develop and execute a phased plan to enable MFA on all identified systems within the next 90 days.
\end{itemize}

\subsection{RISK-002: Incomplete Security Training Program (High)}
\begin{itemize}
    \item \textbf{Immediate Action:} Integrate a mandatory security awareness training module into the new employee onboarding process. This training should cover, at a minimum, the acceptable use policy, phishing identification, and password security.
    \item \textbf{Long-Term Action:} Develop a continuous training program that includes periodic phishing simulations to test and reinforce employee awareness.
\end{itemize}

\subsection{RISK-003: Unverified Risk Remediation (Informational)}
\begin{itemize}
    \item \textbf{Immediate Action:} Confirm with the network or systems administration team that the closure of Port 80 on \texttt{[Target IP]} was an intentional act to remediate the "Unencrypted Web Server" risk.
    \item \textbf{Short-Term Action:} If remediation is confirmed, formally update the organization's risk register to mark the vulnerability as "Remediated" and document the date and actions taken.
\end{itemize}

\end{document}
```