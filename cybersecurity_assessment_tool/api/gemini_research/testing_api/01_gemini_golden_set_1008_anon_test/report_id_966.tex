```latex
\documentclass[12pt]{article}

% Preamble: Required Packages
\usepackage[margin=1in]{geometry}
\usepackage{pifont} % For checkmarks and crosses
\usepackage{booktabs} % For professional tables
\usepackage{hyperref} % For clickable links
\usepackage{url}      % For URL formatting
\usepackage{seqsplit} % For splitting long strings
\usepackage{graphicx}
\usepackage{fancyhdr}
\usepackage{xcolor}

% Document Metadata and Styling
\hypersetup{
    colorlinks=true,
    linkcolor=blue,
    filecolor=magenta,      
    urlcolor=cyan,
    pdftitle={Cybersecurity Posture Assessment Report},
    pdfpagemode=FullScreen,
}
\pagestyle{fancy}
\fancyhf{}
\fancyhead[L]{\textbf{Cybersecurity Posture Assessment}}
\fancyfoot[C]{\thepage}
\renewcommand{\headrulewidth}{0.4pt}
\renewcommand{\footrulewidth}{0.4pt}

% Define checkmark and cross symbols
\newcommand{\cmark}{\ding{51}}
\newcommand{\xmark}{\ding{55}}

\begin{document}

% --- Title Page ---
\begin{titlepage}
    \centering
    \vspace*{1cm}
    \Huge\textbf{Cybersecurity Posture Assessment Report}
    \vspace{1.5cm}
    \Large
    \textbf{Prepared for:}\\
    \vspace{0.5cm}
    \textbf{[Organization Name]}
    \vspace{2cm}
    \large
    \textbf{Date of Report:}\\
    \today
    \vfill
    \textit{This report contains sensitive information and should be handled with care. Access is restricted to authorized personnel only.}
\end{titlepage}

\tableofcontents
\newpage

% --- Section 1: Executive Summary ---
\section*{1. Executive Summary}

This report provides a comprehensive analysis of the cybersecurity posture for \textbf{[Organization Name]}, based on a review of organizational security controls, a technical network scan, and pre-existing risk data. The assessment identified a mixed security landscape with several positive controls offset by critical gaps in policy and technical safeguards.

While the organization has successfully implemented Multi-Factor Authentication (MFA) for email and computer access, three significant risks were identified that require immediate attention:
\begin{itemize}
    \item \textbf{Critical Risk:} The absence of MFA for accessing sensitive data systems exposes critical assets to unauthorized access.
    \item \textbf{High Risk:} The lack of a formal Acceptable Use Policy (AUP) creates ambiguity regarding employee responsibilities and secure behavior.
    \item \textbf{High Risk:} The failure to conduct annual security awareness training for all employees leaves the organization vulnerable to social engineering and human error.
\end{itemize}

On the technical front, the network scan of the target system \texttt{[Target IP]} revealed no open ports, which is a positive security finding. This result contradicts a pre-existing risk entry regarding an unencrypted web server on Port 80, suggesting that this specific vulnerability has been successfully remediated.

Recommendations in this report are prioritized to address the most critical vulnerabilities first, focusing on strengthening access controls, establishing foundational security policies, and enhancing employee security awareness.

% --- Section 2: Organizational Information ---
\section*{2. Organizational Information}

This assessment was conducted for the following organization and associated assets. The information provided was anonymized for the purpose of this report.

\begin{tabular}{@{}ll}
    \toprule
    \textbf{Attribute} & \textbf{Value} \\
    \midrule
    Organization Name & \textbf{[Organization Name]} \\
    Email Domain & \texttt{[Domain]} \\
    External IP Scanned & \texttt{[Client IP]} \\
    \bottomrule
\end{tabular}

% --- Section 3: Security Control Review ---
\section*{3. Security Control Review}

A review of organizational security controls was performed based on a standardized questionnaire. The responses indicate foundational gaps in policy and access control enforcement that increase organizational risk.

\begin{table}[h!]
\centering
\caption{Organizational Security Control Questionnaire Results}
\begin{tabular}{@{}p{0.6\linewidth} c p{0.2\linewidth}@{}}
    \toprule
    \textbf{Control Question} & \textbf{Response} & \textbf{Assessment} \\
    \midrule
    Do you require MFA to access email? & \cmark & Best Practice Met \\
    Do you require MFA to log into computers? & \cmark & Best Practice Met \\
    Do you require MFA to access sensitive data systems? & \xmark & \textbf{Critical Gap} \\
    Does your organization have an employee acceptable use policy? & \xmark & \textbf{High Risk Gap} \\
    Does your organization do security awareness training for new employees? & \cmark & Good Practice \\
    Does your organization do security awareness training for all employees at least once per year? & \xmark & \textbf{High Risk Gap} \\
    \bottomrule
\end{tabular}
\end{table}

% --- Section 4: Technical Scan Results ---
\section*{4. Technical Scan Results}

A network port scan was conducted to identify externally exposed services. The scan provides a snapshot of the target's network perimeter.

\begin{itemize}
    \item \textbf{Target IP Address:} \texttt{[Target IP]}
    \item \textbf{Scan Date:} Scan data processed on \today
    \item \textbf{Synopsis:} The scan revealed no open ports on the target system. This indicates a strong network perimeter configuration for this specific host, minimizing its attack surface. The finding that port 80 is closed suggests that a previously identified risk may have been remediated.
\end{itemize}

\begin{table}[h!]
\centering
\caption{Port Scan Details for Target: \texttt{[Target IP]}}
\begin{tabular}{@{}llll@{}}
    \toprule
    \textbf{Port} & \textbf{State} & \textbf{Service} & \textbf{Product / Version} \\
    \midrule
    80 & closed & http & N/A \\
    \multicolumn{4}{l}{\textit{Note: Only noteworthy ports are listed. No open ports were discovered.}} \\
    \bottomrule
\end{tabular}
\end{table}

% --- Section 5: Consolidated Risk Assessment ---
\section*{5. Consolidated Risk Assessment}

This section synthesizes findings from the security control review, technical scan, and pre-existing risk data into a consolidated list of current risks.

\begin{table}[h!]
\centering
\caption{Summary of Identified Risks}
\begin{tabular}{@{}p{0.3\linewidth} p{0.15\linewidth} p{0.45\linewidth}@{}}
    \toprule
    \textbf{Risk Name} & \textbf{Severity} & \textbf{Overview} \\
    \midrule
    \textbf{Lack of MFA on Sensitive Systems} & \textbf{Critical} & The absence of MFA on systems holding sensitive data creates a single point of failure (passwords) for protecting the organization's most valuable assets. \\
    \addlinespace
    \textbf{Missing Acceptable Use Policy} & \textbf{High} & Without a formal AUP, employees lack clear guidance on the proper use of company assets, data handling, and security responsibilities, increasing the risk of insider threat. \\
    \addlinespace
    \textbf{Inadequate Security Awareness Training} & \textbf{High} & Failing to provide annual training leaves employees unprepared to identify and respond to modern threats like phishing and social engineering. \\
    \addlinespace
    Unencrypted Web Server & Medium (Remediated) & A pre-existing risk noted that Port 80 (HTTP) was open. Our technical scan confirms this port is now \textbf{closed}, indicating the risk has likely been remediated. \\
    \bottomrule
\end{tabular}
\end{table}

% --- Section 6: Recommendations ---
\section*{6. Recommendations}

The following prioritized recommendations are provided to mitigate the identified risks and improve the overall security posture of \textbf{[Organization Name]}.

\subsection*{Priority 1: Critical}
\begin{enumerate}
    \item \textbf{Implement MFA for Sensitive Data Systems:}
    \begin{itemize}
        \item \textbf{Action:} Immediately deploy mandatory Multi-Factor Authentication (MFA) on all applications, databases, and administrative interfaces that process or store sensitive data.
        \item \textbf{Impact:} Drastically reduces the risk of unauthorized access due to compromised credentials.
    \end{itemize}
\end{enumerate}

\subsection*{Priority 2: High}
\begin{enumerate}
    \setcounter{enumi}{1}
    \item \textbf{Develop and Implement an Acceptable Use Policy (AUP):}
    \begin{itemize}
        \item \textbf{Action:} Draft a formal AUP that clearly defines rules for the use of company technology, data handling, and internet access. Require all employees to read and acknowledge the policy.
        \item \textbf{Impact:} Establishes a baseline for secure employee behavior and provides a framework for enforcing security standards.
    \end{itemize}
    \item \textbf{Establish an Annual Security Awareness Training Program:}
    \begin{itemize}
        \item \textbf{Action:} Procure and mandate an annual security awareness training program for all employees. The training should cover phishing, password security, social engineering, and the new AUP.
        \item \textbf{Impact:} Strengthens the "human firewall" and reduces the likelihood of security incidents caused by human error.
    \end{itemize}
\end{enumerate}

\subsection*{Priority 3: Informational}
\begin{enumerate}
    \setcounter{enumi}{3}
    \item \textbf{Validate and Close Remediated Risk:}
    \begin{itemize}
        \item \textbf{Action:} Formally confirm that the closure of Port 80 on \texttt{[Target IP]} was an intentional remediation action. Update the internal risk register to reflect that the "Unencrypted Web Server" risk is closed.
        \item \textbf{Impact:} Ensures the risk register is accurate and reflects the current security posture.
    \end{itemize}
\end{enumerate}

\end{document}
```