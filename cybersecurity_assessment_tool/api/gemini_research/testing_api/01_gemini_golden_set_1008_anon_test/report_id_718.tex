```latex
\documentclass[12pt]{article}

% ----------------------------------------------------------------------
% PREAMBLE
% ----------------------------------------------------------------------
\usepackage[margin=1in]{geometry} % Set page margins
\usepackage{pifont}               % For \ding symbols (checkmark, x-mark)
\usepackage{booktabs}             % For professional-looking tables
\usepackage{graphicx}             % For including images (e.g., logos)
\usepackage{xcolor}               % For color definitions
\usepackage{hyperref}             % For hyperlinks and document metadata
\usepackage{url}                  % For formatting URLs
\usepackage{seqsplit}             % To split long strings in \texttt
\usepackage{datetime}             % To use \today

% --- Hyperref Setup ---
\hypersetup{
    colorlinks=true,
    linkcolor=black,
    citecolor=black,
    urlcolor=blue,
    pdftitle={Cybersecurity Posture Assessment Report},
    pdfauthor={Cybersecurity Analysis Division},
    pdfsubject={Security Assessment},
    pdfkeywords={Cybersecurity, Risk, Assessment, Scan}
}

% --- Custom Commands ---
\newcommand{\yes}{\ding{51}} % Checkmark
\newcommand{\no}{\ding{55}}  % X-mark

% ----------------------------------------------------------------------
% DOCUMENT START
% ----------------------------------------------------------------------
\begin{document}

% ----------------------------------------------------------------------
% TITLE PAGE
% ----------------------------------------------------------------------
\begin{titlepage}
    \centering
    \vspace*{2cm}
    
    \Huge
    \textbf{Cybersecurity Posture Assessment Report}
    
    \vspace{1.5cm}
    
    \Large
    Prepared for: \\
    \vspace{0.5cm}
    \textbf{[Organization Name]}
    
    \vspace{2cm}
    
    \large
    Report Date: \today
    
    \vfill
    
    \large
    \textbf{CONFIDENTIAL}
    
    \vspace{1cm}
    
    \normalsize
    This document contains sensitive information. Distribution is restricted to authorized personnel only.
    
\end{titlepage}

% ----------------------------------------------------------------------
% TABLE OF CONTENTS
% ----------------------------------------------------------------------
\tableofcontents
\newpage

% ----------------------------------------------------------------------
% EXECUTIVE SUMMARY
% ----------------------------------------------------------------------
\section*{Executive Summary}

This report details the findings of a cybersecurity posture assessment conducted for \textbf{[Organization Name]}. The assessment incorporated an analysis of organizational security controls via a questionnaire, a technical network scan of the designated external asset, and a review of pre-existing risks.

The primary finding of this assessment is a \textbf{critical gap in the organization's security awareness training program}. The questionnaire revealed that security training is not provided to new employees during onboarding, nor is it conducted annually for all staff. This absence represents a high risk, as the human element is a primary target for cyber-attacks such as phishing and social engineering.

The technical network scan performed on the target IP address did not identify any open ports or services. While this may indicate a strong firewall configuration, it also limits the visibility into potential vulnerabilities on the external perimeter from this assessment's perspective. No pre-existing vulnerabilities were provided for review.

Recommendations focus on the immediate implementation of a comprehensive security awareness training program to mitigate the significant human-factor risks identified.

\newpage

% ----------------------------------------------------------------------
% ORGANIZATIONAL INFORMATION
% ----------------------------------------------------------------------
\section{Organizational Information}

The following details were used as the basis for this assessment. The information has been anonymized as per the engagement requirements.

\begin{itemize}
    \item \textbf{Organization Name:} \textbf{[Organization Name]}
    \item \textbf{Primary Domain:} \texttt{[Domain]}
    \item \textbf{Assessed External IP:} \texttt{[Client IP]}
\end{itemize}

% ----------------------------------------------------------------------
% SECURITY CONTROL REVIEW
% ----------------------------------------------------------------------
\section{Security Control Review}

A review of administrative and technical security controls was conducted based on the provided questionnaire. The results are summarized below. "Yes" answers, indicating a control is in place, are marked with a \yes. "No" answers, indicating a potential control gap, are marked with a \no.

\begin{table}[h!]
\centering
\caption{Security Controls Questionnaire Results}
\begin{tabular}{p{0.8\textwidth}c}
\toprule
\textbf{Control Question} & \textbf{Status} \\
\midrule
Do you require MFA to access email? & \yes \\
Do you require MFA to log into computers? & \yes \\
Do you require MFA to access sensitive data systems? & \yes \\
Does your organization have an employee acceptable use policy? & \yes \\
Does your organization do security awareness training for new employees? & \textcolor{red}{\no} \\
Does your organization do security awareness training for all employees at least once per year? & \textcolor{red}{\no} \\
\bottomrule
\end{tabular}
\end{table}

\subsection*{Analysis}
The organization has implemented strong identity and access management controls, with Multi-Factor Authentication (MFA) required for email, computer logins, and sensitive systems. An acceptable use policy is also in place. However, the lack of a formal security awareness training program for new and existing employees is a critical deficiency that undermines these technical controls.

% ----------------------------------------------------------------------
% TECHNICAL SCAN RESULTS
% ----------------------------------------------------------------------
\section{Technical Scan Results}

An external network scan was conducted to identify open ports, running services, and potential vulnerabilities on the public-facing infrastructure.

\begin{itemize}
    \item \textbf{Target IP Address:} \texttt{[Target IP]}
    \item \textbf{Scan Date:} Not specified in scan data.
\end{itemize}

\subsection*{Findings}
The network scan completed successfully, but \textbf{no open ports or services were identified} on the target system. This outcome typically indicates one of the following:
\begin{itemize}
    \item The host was not online or responsive at the time of the scan.
    \item A robust firewall or Intrusion Prevention System (IPS) is in place, effectively blocking all incoming probes and hiding services from external discovery.
\end{itemize}
No vulnerabilities can be reported from this scan due to the lack of discovered services.

% ----------------------------------------------------------------------
% RISK ASSESSMENT
% ----------------------------------------------------------------------
\section{Risk Assessment}

This section synthesizes findings from the security control review, technical scan, and pre-existing risk data. The primary risks identified are related to procedural and human-factor gaps.

\begin{table}[h!]
\centering
\caption{Summary of Identified Risks}
\begin{tabular}{p{0.25\linewidth}p{0.5\linewidth}p{0.15\linewidth}}
\toprule
\textbf{Risk Name} & \textbf{Overview} & \textbf{Severity} \\
\midrule
\textbf{Lack of Security Awareness Training} & The organization does not provide security awareness training to new or existing employees. This significantly increases susceptibility to phishing, social engineering, and malware infection, as employees are not equipped to identify or respond to common threats. & \textbf{High} \\
\hline
\multicolumn{3}{c}{\textit{No technical risks were identified from the network scan.}} \\
\multicolumn{3}{c}{\textit{No pre-existing risks were provided for review.}} \\
\bottomrule
\end{tabular}
\end{table}

% ----------------------------------------------------------------------
% RECOMMENDATIONS
% ----------------------------------------------------------------------
\section{Recommendations}

Based on the findings of this assessment, the following actions are recommended to improve the organization's cybersecurity posture.

\begin{enumerate}
    \item \textbf{Implement Mandatory Onboarding Security Training:}
    \begin{itemize}
        \item \textbf{Action:} Develop and implement a mandatory security awareness training module as part of the new employee onboarding process.
        \item \textbf{Details:} This training should cover essential topics, including the acceptable use policy, phishing identification, password hygiene, and procedures for reporting security incidents.
        \item \textbf{Priority:} \textbf{Critical}
    \end{itemize}
    \vspace{0.5cm}
    
    \item \textbf{Establish an Annual Security Training and Awareness Program:}
    \begin{itemize}
        \item \textbf{Action:} Launch a mandatory, annual security awareness training program for all employees.
        \item \textbf{Details:} The program should include refresher courses on key security topics and be supplemented with periodic phishing simulation campaigns to test and reinforce employee knowledge.
        \item \textbf{Priority:} \textbf{High}
    \end{itemize}
\end{enumerate}

\end{document}
```