```latex
\documentclass[12pt]{article}

% ----------------------------------------------------------------------
% METADATA & PACKAGES
% ----------------------------------------------------------------------
\usepackage[margin=1in]{geometry}
\usepackage{pifont} % For checkmarks and crosses
\usepackage{booktabs} % For professional tables
\usepackage{hyperref} % For hyperlinks
\usepackage{url} % For URL formatting
\usepackage{seqsplit} % To split long strings in tt font
\usepackage[table]{xcolor} % For coloring table cells
\usepackage{graphicx} % For logo (placeholder)

% Define severity colors
\definecolor{sev_critical}{HTML}{990000}
\definecolor{sev_high}{HTML}{D15100}
\definecolor{sev_medium}{HTML}{F0C200}
\definecolor{sev_low}{HTML}{008000}

% Hyperref setup
\hypersetup{
    colorlinks=true,
    linkcolor=blue,
    filecolor=magenta,      
    urlcolor=cyan,
    pdftitle={Cybersecurity Posture Assessment Report},
    pdfauthor={Cybersecurity Analyst},
    pdfsubject={Security Analysis},
    pdfkeywords={Security, Report, Analysis},
    bookmarks=true
}

% Define checkmark and cross symbols for clarity
\newcommand{\cmark}{\ding{51}}%
\newcommand{\xmark}{\ding{55}}%

% ----------------------------------------------------------------------
% DOCUMENT START
% ----------------------------------------------------------------------
\begin{document}

% ----------------------------------------------------------------------
% TITLE PAGE
% ----------------------------------------------------------------------
\begin{titlepage}
    \centering
    \vspace*{1cm}
    
    \rule{\textwidth}{1.5pt}\vspace*{0.5cm}
    \Huge\textbf{Cybersecurity Posture Assessment Report}
    \vspace*{0.5cm}\rule{\textwidth}{1.5pt}
    
    \vspace{2cm}
    
    \Large Prepared for: \\
    \vspace{0.5cm}
    \Huge\textbf{[Organization Name]}
    
    \vspace{3cm}
    
    \Large Prepared by: \\
    \vspace{0.5cm}
    \Large Expert Cybersecurity Analyst
    
    \vfill
    
    \Large\today
    
\end{titlepage}

\tableofcontents
\newpage

% ----------------------------------------------------------------------
% 1. EXECUTIVE SUMMARY
% ----------------------------------------------------------------------
\section{Executive Summary}

This report provides a comprehensive analysis of the cybersecurity posture of \textbf{[Organization Name]}, based on technical network scans, a review of existing risks, and a security controls questionnaire. The assessment was conducted to identify vulnerabilities, misconfigurations, and policy gaps that could expose the organization to cyber threats.

The analysis revealed two primary areas of significant concern requiring immediate attention:

\begin{enumerate}
    \item \textbf{Critical Risk - Exposed and Outdated Database:} A network scan confirmed that a MySQL database server is publicly accessible from the internet on port 3306. Further analysis indicates the running version, MySQL 5.7.33, is outdated, no longer supported by the vendor (End-of-Life), and lacks critical security patches, making it an attractive target for attackers.
    
    \item \textbf{High Risk - Inadequate Access Controls:} The organization does not enforce Multi-Factor Authentication (MFA) for accessing email. As email accounts are a primary target for phishing and account takeover attacks, this represents a critical gap in the organization's identity and access management controls. A compromised email account can lead to further system compromise, data breaches, and financial fraud.
\end{enumerate}

While the organization has implemented several positive security controls, such as MFA for computer logins and security awareness training, the identified critical and high-risk vulnerabilities substantially elevate the overall risk profile. This report provides detailed findings and actionable recommendations to mitigate these threats and improve the overall security posture.

\newpage

% ----------------------------------------------------------------------
% 2. ORGANIZATIONAL INFORMATION
% ----------------------------------------------------------------------
\section{Organizational Information}
This section contains the high-level information used as the basis for this assessment. The data provided was anonymized for the purpose of this report template.

\begin{table}[h!]
\centering
\begin{tabular}{@{}ll@{}}
\toprule
\textbf{Attribute} & \textbf{Value} \\ \midrule
Organization Name & \textbf{[Organization Name]} \\
Primary Email Domain & \texttt{[Domain]} \\
External IP Address Scanned & \texttt{[Client IP]} \\
Target IP in Scan Data & \texttt{[Target IP]} \\ \bottomrule
\end{tabular}
\caption{Client Organizational Details.}
\label{tab:org_info}
\end{table}

% ----------------------------------------------------------------------
% 3. SECURITY CONTROL REVIEW
% ----------------------------------------------------------------------
\section{Security Control Review}
The following table summarizes the organization's responses to a security controls questionnaire. A green checkmark (\textcolor{green}{\cmark}) indicates a positive control is in place, while a red cross (\textcolor{red}{\xmark}) indicates a potential security gap.

\begin{table}[h!]
\centering
\begin{tabular}{@{}p{0.75\textwidth}c@{}}
\toprule
\textbf{Control Question} & \textbf{Response} \\ \midrule
Do you require MFA to access email? & \textcolor{red}{\xmark} \\
Do you require MFA to log into computers? & \textcolor{green}{\cmark} \\
Do you require MFA to access sensitive data systems? & \textcolor{green}{\cmark} \\
Does your organization have an employee acceptable use policy? & \textcolor{green}{\cmark} \\
Does your organization do security awareness training for new employees? & \textcolor{green}{\cmark} \\
Does your organization do security awareness training for all employees at least once per year? & \textcolor{green}{\cmark} \\ \bottomrule
\end{tabular}
\caption{Security Controls Questionnaire Results.}
\label{tab:controls}
\end{table}

\subsection*{Analysis}
The primary gap identified is the \textbf{lack of MFA for email access}. Email is a foundational service and often holds the keys to resetting passwords for other critical systems. Without MFA, a single compromised password can lead to a full account takeover. This is considered a high-risk finding.

% ----------------------------------------------------------------------
% 4. TECHNICAL SCAN RESULTS
% ----------------------------------------------------------------------
\section{Technical Scan Results}
An external network scan was performed to identify open ports and exposed services on the target system.

\subsection{Open Ports and Services}
The following services were found to be accessible from the public internet.

\begin{table}[h!]
\centering
\begin{tabular}{@{}llll@{}}
\toprule
\textbf{Port} & \textbf{State} & \textbf{Service} & \textbf{Product \& Version} \\ \midrule
3306/tcp & open & mysql & MySQL 5.7.33 \\ \bottomrule
\end{tabular}
\caption{Network Scan Findings for target \texttt{[Target IP]}.}
\label{tab:scan_results}
\end{table}

\subsection{Analysis of Findings}
The scan identified an open MySQL port (3306), which directly confirms the pre-existing risk of "Database Exposure". Exposing a database directly to the internet is a highly discouraged practice as it significantly increases the attack surface.

More critically, the detected version, \textbf{MySQL 5.7.33}, is outdated. The MySQL 5.7 series reached its official End-of-Life (EOL) in October 2023. This means it no longer receives security updates, patches, or bug fixes from the vendor. Running EOL software, especially on an internet-facing system, introduces a severe and unpatchable risk. Numerous vulnerabilities have been discovered since the release of version 5.7.33.

% ----------------------------------------------------------------------
% 5. CONSOLIDATED RISK ASSESSMENT
% ----------------------------------------------------------------------
\section{Consolidated Risk Assessment}
This section synthesizes findings from all data sources into a consolidated list of identified risks, prioritized by severity.

\begin{table}[h!]
\centering
\renewcommand{\arraystretch}{1.5}
\begin{tabular}{@{}p{0.25\linewidth}p{0.55\linewidth}l@{}}
\toprule
\textbf{Risk Name} & \textbf{Description} & \textbf{Severity} \\ \midrule
\rowcolor{sev_critical!25}
Unsupported Database Software & The internet-facing MySQL server is running version 5.7.33, which is End-of-Life and no longer receives security patches. & \textcolor{sev_critical}{\textbf{Critical}} \\
\rowcolor{sev_high!25}
Lack of Email MFA & Email accounts, a primary target for attackers, are not protected by Multi-Factor Authentication, relying only on passwords. & \textcolor{sev_high}{\textbf{High}} \\
\rowcolor{sev_high!25}
Public Database Exposure & The MySQL database port (3306) is open to the entire internet, inviting brute-force attacks and exploitation attempts. & \textcolor{sev_high}{\textbf{High}} \\ \bottomrule
\end{tabular}
\caption{Summary of Identified Risks.}
\label{tab:risk_summary}
\end{table}

% ----------------------------------------------------------------------
% 6. RECOMMENDATIONS
% ----------------------------------------------------------------------
\section{Recommendations}
The following actionable recommendations are provided to mitigate the identified risks. They are prioritized to address the most severe threats first.

\subsection{Critical Priority}
\subsubsection{Remediate Unsupported Database Software}
\begin{itemize}
    \item \textbf{Immediate Action:} Develop an urgent migration plan to upgrade the MySQL 5.7 instance to a currently supported version, such as MySQL 8.0 or later.
    \item \textbf{Long-Term Strategy:} Implement a comprehensive patch and vulnerability management program to ensure all software and systems are kept up-to-date and retired before their End-of-Life date.
\end{itemize}

\subsection{High Priority}
\subsubsection{Implement MFA on Email}
\begin{itemize}
    \item \textbf{Immediate Action:} Procure and enforce an MFA solution for all user email accounts without delay. Communicate the change to all employees and provide clear instructions for enrollment.
    \item \textbf{Long-Term Strategy:} Establish a policy that mandates MFA for access to all critical systems and services, not just email.
\end{itemize}

\subsubsection{Restrict Database Access}
\begin{itemize}
    \item \textbf{Immediate Action:} Implement strict firewall rules to block all public access to TCP port 3306. Access should be restricted to a whitelist of specific, trusted IP addresses, such as application servers or administrator workstations.
    \item \textbf{Long-Term Strategy:} Re-architect the network so that the database server is located in a private network segment, completely inaccessible from the public internet. Access should be managed through a secure bastion host or VPN.
\end{itemize}

% ----------------------------------------------------------------------
% DOCUMENT END
% ----------------------------------------------------------------------
\end{document}
```