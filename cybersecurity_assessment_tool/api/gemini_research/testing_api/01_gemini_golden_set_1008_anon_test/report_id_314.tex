```latex
\documentclass[12pt]{article}

% Preamble: Required Packages
\usepackage[margin=1in]{geometry}
\usepackage{pifont} % For checkmarks and crosses
\usepackage{booktabs} % For professional tables
\usepackage{hyperref} % For clickable links and references
\usepackage{url} % For formatting URLs
\usepackage{seqsplit} % For splitting long strings in texttt
\usepackage[T1]{fontenc}

% Document Metadata
\hypersetup{
    colorlinks=true,
    linkcolor=blue,
    filecolor=magenta,      
    urlcolor=cyan,
    pdftitle={Cybersecurity Assessment Report},
    pdfauthor={Cybersecurity Analyst},
    pdfsubject={Security Posture Analysis},
    pdfkeywords={Cybersecurity, Risk, Assessment},
    bookmarks=true
}

\begin{document}

% --- Title Page ---
\begin{titlepage}
    \centering
    \vspace*{\fill}
    \Huge\textbf{Cybersecurity Assessment Report}
    \vfill
    \Large Prepared for: \\
    \textbf{[Organization Name]}
    \vfill
    \large Report Date: \today \\
    \large Author: Cybersecurity Analyst
    \vspace*{\fill}
\end{titlepage}

\tableofcontents
\newpage

% --- Section 1: Executive Summary ---
\section{Executive Summary}
This report provides a comprehensive analysis of the cybersecurity posture of \textbf{[Organization Name]}, based on a review of organizational security controls, an external network scan, and pre-existing risk documentation. The assessment identified several critical and high-risk vulnerabilities that require immediate attention to mitigate the risk of unauthorized access, data breach, and operational disruption.

Key findings include a publicly accessible FTP server with a dangerously outdated version and anonymous login enabled, a systemic lack of Multi-Factor Authentication (MFA) for critical access points like email and computer logins, and the absence of a formal Acceptable Use Policy for employees. These issues, combined with the known risk of outdated Windows 7 workstations, create a significant attack surface.

Immediate remediation should focus on securing the external FTP service and implementing MFA. Strategic initiatives must include developing foundational security policies and upgrading end-of-life operating systems.

% --- Section 2: Organizational Information ---
\section{Organizational Information}
This section details the organizational data available for this assessment. The information has been anonymized as per the engagement protocol.

\begin{itemize}
    \item \textbf{Organization Name:} \textbf{[Organization Name]}
    \item \textbf{Primary Email Domain:} \texttt{[Domain]}
    \item \textbf{External IP Scanned:} \texttt{[Client IP]}
\end{itemize}

% --- Section 3: Security Control Review ---
\section{Security Control Review}
A review of organizational security controls was conducted via a questionnaire. The responses indicate significant gaps in fundamental security practices, particularly concerning access control and governance. The table below summarizes the findings.

\begin{table}[h!]
\centering
\caption{Organizational Security Controls Questionnaire}
\label{tab:controls}
\begin{tabular}{@{}p{0.7\linewidth}c@{}}
\toprule
\textbf{Control Question} & \textbf{Response} \\
\midrule
Do you require MFA to access email? & \ding{55} \\
Do you require MFA to log into computers? & \ding{55} \\
Do you require MFA to access sensitive data systems? & \ding{51} \\
Does your organization have an employee acceptable use policy? & \ding{55} \\
Does your organization do security awareness training for new employees? & \ding{51} \\
Does your organization do security awareness training for all employees at least once per year? & \ding{51} \\
\bottomrule
\end{tabular}
\end{table}

\subsection*{Analysis}
The lack of MFA for email and computer logins represents a \textbf{critical risk}. Email is a primary target for phishing and account takeover attacks, while unprotected computer logins expose the internal network to significant threats. The absence of an Acceptable Use Policy is a \textbf{high-risk} governance gap, as it fails to establish clear rules for employees regarding the protection of company assets and data.

% --- Section 4: Technical Scan Results ---
\section{Technical Scan Results}
An external network vulnerability scan was performed to identify exposed services and potential weaknesses.

\begin{itemize}
    \item \textbf{Scan Target:} \texttt{[Target IP]}
    \item \textbf{Scan Date:} Scan date not provided in source data.
\end{itemize}

The following open ports and services were discovered:

\begin{table}[h!]
\centering
\caption{Open Port Scan Results}
\label{tab:scan}
\begin{tabular}{@{}llllll@{}}
\toprule
\textbf{Port} & \textbf{State} & \textbf{Service} & \textbf{Product} & \textbf{Version} & \textbf{Notes} \\
\midrule
21/tcp & open & ftp & vsftpd & 2.3.4 & \textbf{Anonymous FTP login allowed} \\
\bottomrule
\end{tabular}
\end{table}

\subsection*{Analysis}
The scan identified a critical vulnerability. The FTP server is running \textbf{vsftpd version 2.3.4}, a version released in 2011 that is known to contain a critical backdoor vulnerability (CVE-2011-2523). This vulnerability allows an attacker to execute arbitrary commands on the server.

Furthermore, the server is configured to allow \textbf{anonymous FTP login}. This configuration permits any external user to connect to the server and potentially upload, download, or delete files, posing an extreme risk of data breach, malware injection, or system compromise. This finding requires immediate remediation.

% --- Section 5: Consolidated Risk Assessment ---
\section{Consolidated Risk Assessment}
This section synthesizes findings from the security control review, technical scan, and pre-existing risk data into a prioritized list of identified risks.

\begin{table}[h!]
\centering
\caption{Summary of Identified Risks}
\label{tab:risks}
\begin{tabular}{@{}p{0.1\linewidth}p{0.5\linewidth}ll@{}}
\toprule
\textbf{Risk ID} & \textbf{Description} & \textbf{Severity} & \textbf{Source} \\
\midrule
RISK-001 & \textbf{Insecure Anonymous FTP Server:} A publicly accessible FTP server is running a vulnerable version of vsftpd (2.3.4) with anonymous login enabled. & \textbf{Critical} & Network Scan \\
\addlinespace
RISK-002 & \textbf{Lack of MFA for Critical Systems:} Multi-Factor Authentication is not enforced for email or computer logins, leaving them vulnerable to credential theft. & \textbf{Critical} & Questionnaire \\
\addlinespace
RISK-003 & \textbf{Missing Acceptable Use Policy:} The organization lacks a formal policy defining rules for technology and data use, leading to inconsistent security practices. & \textbf{High} & Questionnaire \\
\addlinespace
RISK-004 & \textbf{Outdated Windows 7 Workstations:} End-of-life operating systems are in use, which no longer receive security updates and are highly susceptible to exploitation. & Medium & Existing Risks \\
\bottomrule
\end{tabular}
\end{table}

% --- Section 6: Recommendations ---
\section{Recommendations}
Based on the consolidated risk assessment, the following prioritized actions are recommended to improve the organization's security posture.

\subsection*{Immediate Priority (Critical Risks)}
\begin{enumerate}
    \item \textbf{Remediate Insecure FTP Server (RISK-001):}
    \begin{itemize}
        \item Immediately take the FTP server offline at \texttt{[Target IP]}.
        \item If the service is not required for business, decommission it permanently.
        \item If FTP is required, upgrade \texttt{vsftpd} to the latest stable version, disable anonymous access, and enforce strong, unique user credentials.
    \end{itemize}
    
    \item \textbf{Deploy Multi-Factor Authentication (RISK-002):}
    \begin{itemize}
        \item Prioritize the rollout of MFA for the email system (\texttt{[Domain]}) for all users.
        \item Begin planning the implementation of MFA for all employee computer logins.
    \end{itemize}
\end{enumerate}

\subsection*{High Priority}
\begin{enumerate}
    \setcounter{enumi}{2}
    \item \textbf{Develop and Implement an Acceptable Use Policy (RISK-003):}
    \begin{itemize}
        \item Draft a formal policy that governs the use of company computers, networks, and data.
        \item Communicate the policy to all employees and require acknowledgment of its terms.
    \end{itemize}
\end{enumerate}

\subsection*{Medium Priority}
\begin{enumerate}
    \setcounter{enumi}{3}
    \item \textbf{Upgrade End-of-Life Systems (RISK-004):}
    \begin{itemize}
        \item Execute the plan to upgrade all Windows 7 workstations to a currently supported operating system, such as Windows 10 or 11.
    \end{itemize}
\end{enumerate}

\end{document}
```