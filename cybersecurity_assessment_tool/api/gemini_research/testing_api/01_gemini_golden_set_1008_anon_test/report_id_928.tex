```latex
\documentclass[12pt]{article}

% Preamble: Required Packages
\usepackage[margin=1in]{geometry}
\usepackage{pifont} % For checkmarks and crosses
\usepackage{booktabs} % For professional tables
\usepackage{hyperref} % For hyperlinks
\usepackage{url}      % For URL formatting
\usepackage{seqsplit} % For splitting long text sequences
\usepackage{xcolor}   % For colors

% Document Information
\title{Cybersecurity Posture Assessment Report}
\author{Cybersecurity Analysis Division}
\date{\today}

\begin{document}

\maketitle
\thispagestyle{empty}
\newpage

\tableofcontents
\newpage

% ==============================================================================
% 1. Executive Summary
% ==============================================================================
\section{Executive Summary}

This report provides a cybersecurity posture assessment for \textbf{[Organization Name]}, conducted on \today. The analysis synthesizes data from an external network scan, a security controls questionnaire, and a review of pre-existing risks.

The assessment reveals a mixed security posture. On a positive note, the organization demonstrates a solid foundation in security policy and training, with an established employee acceptable use policy and a consistent security awareness training program. Furthermore, the external network scan of the target IP address \texttt{[Target IP]} indicated no open ports, suggesting a well-configured perimeter firewall that effectively limits external exposure.

However, a critical vulnerability exists in the organization's identity and access management strategy. The questionnaire responses confirm a complete lack of Multi-Factor Authentication (MFA) for email, computer logins, and access to sensitive data systems. This absence of MFA represents a significant and immediate risk. A single compromised password could grant an attacker widespread access to critical assets, potentially leading to a severe data breach, financial loss, or operational disruption.

Immediate remediation efforts must focus on the rapid deployment of MFA across all critical platforms. Prioritizing this single control will drastically reduce the risk of unauthorized access and significantly strengthen the organization's overall security resilience.

% ==============================================================================
% 2. Organizational Information
% ==============================================================================
\section{Organizational Information}

This section details the information provided about the organization. Missing data is indicated by placeholders.

\begin{itemize}
    \item \textbf{Organization Name:} \textbf{[Organization Name]}
    \item \textbf{Primary Email Domain:} \texttt{[Domain]}
    \item \textbf{Known External IP:} \texttt{[Client IP]}
\end{itemize}

% ==============================================================================
% 3. Security Control Review (Questionnaire Analysis)
% ==============================================================================
\section{Security Control Review}

The following table summarizes the organization's responses to a security controls questionnaire. Items marked with a red \ding{55} represent significant gaps in the security framework and are correlated with high-priority risks in Section 5.

\begin{table}[h!]
\centering
\begin{tabular}{p{0.75\linewidth} c c}
\toprule
\textbf{Control Question} & \textbf{Response} & \textbf{Status} \\
\midrule
Does your organization have an employee acceptable use policy? & Yes & \ding{51} \\
Does your organization do security awareness training for new employees? & Yes & \ding{51} \\
Does your organization do security awareness training for all employees at least once per year? & Yes & \ding{51} \\
\midrule
\textcolor{red}{Do you require MFA to access email?} & \textcolor{red}{No} & \textcolor{red}{\ding{55}} \\
\textcolor{red}{Do you require MFA to log into computers?} & \textcolor{red}{No} & \textcolor{red}{\ding{55}} \\
\textcolor{red}{Do you require MFA to access sensitive data systems?} & \textcolor{red}{No} & \textcolor{red}{\ding{55}} \\
\bottomrule
\end{tabular}
\caption{Security Controls Questionnaire Results}
\label{tab:controls}
\end{table}

\subsection*{Analysis}
The "No" responses highlight a critical deficiency in access control measures. The lack of MFA for email, endpoints, and sensitive systems exposes the organization to severe risks from credential theft attacks, such as phishing and password spraying. While foundational policies are in place, they are insufficient to protect against modern identity-based threats without the enforcement of MFA.

% ==============================================================================
% 4. Technical Scan Results
% ==============================================================================
\section{Technical Scan Results}

An external network vulnerability scan was performed to identify exposed services and potential vulnerabilities accessible from the public internet.

\begin{itemize}
    \item \textbf{Target IP Address:} \texttt{[Target IP]}
    \item \textbf{Scan Date:} [Scan Date Not Provided]
    \item \textbf{Scan Summary:} The scan reported the host as "up" but found no open TCP or UDP ports. All scanned ports were in a "closed" state.
\end{itemize}

\subsection*{Findings}
The scan results are positive. The absence of open ports on the scanned IP address indicates that the perimeter firewall is correctly configured to deny unsolicited inbound traffic. This "default deny" posture is a security best practice and significantly reduces the external attack surface. No vulnerabilities were identified through this scan.

% ==============================================================================
% 5. Risk Assessment
% ==============================================================================
\section{Risk Assessment}

This section correlates findings from the security control review, technical scan, and pre-existing risk data. The primary risks identified stem from the policy and procedure gaps noted in Section 3. No pre-existing vulnerabilities were reported.

\begin{table}[h!]
\centering
\begin{tabular}{p{0.2\linewidth} p{0.6\linewidth} p{0.15\linewidth}}
\toprule
\textbf{Risk Name} & \textbf{Overview} & \textbf{Severity} \\
\midrule
\textbf{Lack of MFA for Email} & Without MFA, a single compromised password allows an attacker to access company email. This can lead to business email compromise (BEC), data exfiltration, and a launchpad for internal phishing campaigns. & \textbf{Critical} \\
\addlinespace
\textbf{Lack of MFA for Sensitive Systems} & The organization's most valuable data is protected only by passwords. An attacker with valid credentials can directly access and exfiltrate sensitive information, leading to severe regulatory, financial, and reputational damage. & \textbf{Critical} \\
\addlinespace
\textbf{Lack of MFA for Endpoints} & Compromised credentials could allow an attacker to log directly into employee computers or servers. This provides a strong foothold within the network to escalate privileges, move laterally, and deploy ransomware. & \textbf{High} \\
\bottomrule
\end{tabular}
\caption{Summary of Identified Risks}
\label{tab:risks}
\end{table}

% ==============================================================================
% 6. Recommendations
% ==============================================================================
\section{Recommendations}

Based on the analysis, the following actions are recommended to mitigate the identified risks and improve the overall security posture of \textbf{[Organization Name]}.

\subsection*{Immediate Priority (0-30 Days)}
\begin{itemize}
    \item \textbf{Implement MFA for Email:} Immediately enable and enforce MFA for all user access to the corporate email system (e.g., Office 365, Google Workspace). This is the single most effective control to prevent account takeovers.
\end{itemize}

\subsection*{High Priority (30-90 Days)}
\begin{itemize}
    \item \textbf{Enforce MFA for Sensitive Systems:} Deploy MFA for all applications and systems that store, process, or transmit sensitive or regulated data. This includes financial systems, customer databases, and administrative portals.
    \item \textbf{Enforce MFA for Remote Access and Endpoints:} Require MFA for all remote access solutions (e.g., VPN) and for logging into company-managed computers and servers. This prevents attackers from using stolen credentials to gain an internal network foothold.
\end{itemize}

\subsection*{Sustaining Actions (Ongoing)}
\begin{itemize}
    \item \textbf{Continue Security Awareness Training:} Continue the existing security awareness program. Incorporate new training modules that educate employees on the importance of MFA and how to respond to MFA fatigue attacks.
    \item \textbf{Perform Regular Network Scanning:} Continue to conduct regular, authenticated and unauthenticated, internal and external vulnerability scans to ensure new services are not inadvertently exposed and to identify vulnerabilities in existing systems.
\end{itemize}

\end{document}
```