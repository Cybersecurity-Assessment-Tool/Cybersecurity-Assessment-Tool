```latex
\documentclass[12pt]{article}

% Required Packages
\usepackage[margin=1in]{geometry}
\usepackage{pifont} % For checkmarks and crosses
\usepackage{booktabs} % For professional tables
\usepackage{url}
\usepackage{seqsplit} % To split long strings in tt font
\usepackage{graphicx}
\usepackage[table]{xcolor} % For colors in tables
\usepackage[hidelinks]{hyperref} % For clickable links without boxes

% --- Document Setup ---
\definecolor{darkblue}{rgb}{0.0, 0.0, 0.55}
\definecolor{darkred}{rgb}{0.55, 0.0, 0.0}
\definecolor{darkgreen}{rgb}{0.0, 0.35, 0.0}
\definecolor{severitycritical}{rgb}{0.6, 0.0, 0.0}
\definecolor{severityhigh}{rgb}{0.9, 0.4, 0.0}
\definecolor{severitymedium}{rgb}{1.0, 0.8, 0.0}
\definecolor{severitylow}{rgb}{0.2, 0.6, 0.2}

\hypersetup{
    colorlinks=true,
    linkcolor=darkblue,
    filecolor=darkblue,      
    urlcolor=darkblue,
    citecolor=darkblue,
}

% Custom commands for status indicators
\newcommand{\yes}{\textcolor{darkgreen}{\ding{51}}}
\newcommand{\no}{\textcolor{darkred}{\ding{55}}}

% --- Title Page ---
\title{Cybersecurity Assessment Report \\ \large for \\ \textbf{[Organization Name]}}
\author{Cybersecurity Analyst}
\date{\today}

\begin{document}

\maketitle
\thispagestyle{empty}
\newpage

\tableofcontents
\thispagestyle{empty}
\newpage

\setcounter{page}{1}

% --- Executive Summary ---
\section{Executive Summary}

This report details the findings of a cybersecurity assessment conducted for \textbf{[Organization Name]}. The evaluation combined a review of organizational security controls via a questionnaire, an external network vulnerability scan, and an analysis of pre-existing risks.

The assessment revealed a mixed security posture. The organization has implemented several positive security controls, including mandatory Multi-Factor Authentication (MFA) for computer and sensitive system access, as well as a consistent security awareness training program for all employees. The external network scan of the target IP address revealed no open ports, which indicates a strong perimeter defense for that specific asset.

However, two critical gaps were identified through the security questionnaire:
\begin{itemize}
    \item \textbf{Lack of MFA for Email Access:} This is a critical vulnerability that exposes the organization to significant risks, including business email compromise (BEC), phishing attacks, and unauthorized data access.
    \item \textbf{Absence of an Acceptable Use Policy (AUP):} This policy gap creates ambiguity regarding the secure use of company assets, increasing the risk of insider threats and non-compliance.
\end{itemize}

This report provides a detailed analysis of these findings and concludes with actionable recommendations to mitigate the identified risks and strengthen the overall security posture of \textbf{[Organization Name]}.

% --- Organizational Information ---
\section{Organizational Information}

The following information was used as the basis for this assessment. As per the provided data, placeholders are used where specific details were not available.

\begin{center}
\begin{tabular}{@{}ll@{}}
\toprule
\textbf{Attribute} & \textbf{Value} \\ \midrule
Organization Name & \textbf{[Organization Name]} \\
Primary Email Domain & \texttt{[Domain]} \\
Client External IP & \texttt{[Client IP]} \\
Target IP Scanned & \texttt{[Target IP]} \\ \bottomrule
\end{tabular}
\end{center}

% --- Security Control Review ---
\section{Security Control Review (Questionnaire)}

A security questionnaire was completed to evaluate the implementation of key administrative and technical controls. The table below summarizes the responses. A green checkmark (\yes) indicates a positive control, while a red 'X' (\no) highlights a potential security gap that requires attention.

\begin{center}
\rowcolors{2}{gray!10}{white}
\begin{tabular}{p{0.7\textwidth}lc}
\toprule
\textbf{Control Question} & \textbf{Response} & \textbf{Status} \\ \midrule
Do you require MFA to access email? & No & \no \\
Do you require MFA to log into computers? & Yes & \yes \\
Do you require MFA to access sensitive data systems? & Yes & \yes \\
Does your organization have an employee acceptable use policy? & No & \no \\
Does your organization do security awareness training for new employees? & Yes & \yes \\
Does your organization do security awareness training for all employees at least once per year? & Yes & \yes \\
\bottomrule
\end{tabular}
\end{center}

The primary areas of concern identified from this review are the lack of MFA for email and the absence of a formal Acceptable Use Policy.

% --- Technical Scan Results ---
\section{Technical Scan Results}

An external network scan was performed on the designated target IP address to identify open ports and exposed services.

\begin{itemize}
    \item \textbf{Target IP Address:} \texttt{[Target IP]}
    \item \textbf{Scan Date:} Scan data was processed on \today.
    \item \textbf{Findings:} The scan completed successfully and \textbf{found no open ports or services}.
\end{itemize}

\paragraph{Analyst Notes:}
Discovering no open ports on an external-facing asset is a positive security finding. It suggests a well-configured firewall that adheres to the principle of least privilege, significantly reducing the external attack surface of the target system. This result indicates that either the system is properly secured or was not online during the scan.

% --- Risk Assessment ---
\section{Risk Assessment}

This section synthesizes findings from the security control review and technical scan. No pre-existing vulnerabilities were provided for analysis (\textit{Input\_3\_Current\_Risks\_JSON} was empty). The risks below are derived directly from this assessment.

\begin{center}
\begin{tabular}{p{0.1\textwidth}p{0.25\textwidth}p{0.45\textwidth}p{0.1\textwidth}}
\toprule
\textbf{ID} & \textbf{Risk Name} & \textbf{Overview} & \textbf{Severity} \\ \midrule
R-01 & Lack of MFA for Email Access & Without MFA, email accounts are vulnerable to takeover via stolen or weak passwords. This can lead to business email compromise, data breaches, and further network intrusion. & \cellcolor{severitycritical!25}Critical \\
\addlinespace
R-02 & Missing Acceptable Use Policy (AUP) & The absence of a formal AUP creates ambiguity for employees on the proper use of IT assets. This increases the likelihood of unintentional data exposure, malware infections, and insider threats. & \cellcolor{severityhigh!25}High \\
\bottomrule
\end{tabular}
\end{center}

% --- Recommendations ---
\section{Recommendations}

The following recommendations are provided to address the risks identified in the previous section. These actions are prioritized based on severity and potential impact.

\subsection{R-01: Enforce MFA for Email Access (Critical)}
\begin{itemize}
    \item \textbf{Action:} Immediately enable and enforce Multi-Factor Authentication for all user accounts, including administrative and service accounts, across the email platform (e.g., Microsoft 365, Google Workspace).
    \item \textbf{Justification:} MFA is one of the most effective controls to prevent unauthorized account access. It provides a critical layer of defense against phishing and credential stuffing attacks, protecting sensitive communications and data.
\end{itemize}

\subsection{R-02: Develop and Implement an AUP (High)}
\begin{itemize}
    \item \textbf{Action:} Develop a comprehensive Acceptable Use Policy that clearly defines the rules and responsibilities for all employees when using company-owned IT assets, networks, and data. This policy should be formally communicated, and all employees should be required to read and acknowledge it.
    \item \textbf{Justification:} An AUP establishes a baseline for secure behavior, reduces organizational liability, and empowers the organization to take corrective action in case of policy violations. It is a foundational component of any cybersecurity program.
\end{itemize}

\end{document}
```