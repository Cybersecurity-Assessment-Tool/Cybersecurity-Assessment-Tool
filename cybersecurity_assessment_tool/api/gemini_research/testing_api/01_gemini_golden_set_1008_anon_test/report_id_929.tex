```latex
\documentclass[12pt]{article}

% Preamble: Required Packages
\usepackage[margin=1in]{geometry}
\usepackage{pifont} % For checkmarks and crosses (\ding)
\usepackage{booktabs} % For professional-looking tables
\usepackage{hyperref} % For clickable links and references
\usepackage{url}
\usepackage{seqsplit} % To break long strings in texttt
\usepackage{xcolor}
\usepackage{graphicx}

% Hyperlink Setup
\hypersetup{
    colorlinks=true,
    linkcolor=blue,
    filecolor=magenta,
    urlcolor=cyan,
    pdftitle={Cybersecurity Posture Assessment Report},
    pdfauthor={Cybersecurity Analysis Division},
}

% Custom Commands for Readability
\newcommand{\yes}{\textcolor{green}{\ding{51}}}
\newcommand{\no}{\textcolor{red}{\ding{55}}}
\newcommand{\riskcritical}{\textcolor{red}{\textbf{Critical}}}
\newcommand{\riskhigh}{\textcolor{orange}{\textbf{High}}}

% Document Information
\title{Cybersecurity Posture Assessment Report}
\author{Cybersecurity Analysis Division}
\date{\today}

\begin{document}

\maketitle
\thispagestyle{empty}
\newpage

\tableofcontents
\newpage

\section*{Executive Summary}

This report provides a cybersecurity posture assessment for \textbf{[Organization Name]}. The analysis is based on a combination of network scanning, a review of existing risk documentation, and an organizational security controls questionnaire.

The assessment reveals a \riskcritical{} risk posture, primarily driven by the direct exposure of a Remote Desktop Protocol (RDP) service on the public internet at \texttt{[Target IP]}. This finding is corroborated by pre-existing risk documentation and represents a severe vulnerability that could be exploited by attackers to gain unauthorized access to the internal network.

This critical technical vulnerability is dangerously compounded by significant gaps in administrative security controls. The organization does not enforce Multi-Factor Authentication (MFA) for email or access to sensitive data systems. This lack of MFA dramatically increases the likelihood that a compromised user credential could lead to a full system breach via the exposed RDP service.

Furthermore, foundational governance controls are missing, including an employee acceptable use policy and mandatory annual security awareness training. These gaps indicate a need to strengthen the overall security culture and reduce human-factor risks.

Immediate remediation is required. Recommendations are prioritized to first eliminate the direct external threat, then implement critical access controls, and finally, establish foundational security policies and training programs.

\section{Organizational Information}

This section details the information provided about the organization. Due to the anonymized nature of the data provided, placeholders are used where specific information was not available.

\begin{itemize}
    \item \textbf{Organization Name:} \textbf{[Organization Name]}
    \item \textbf{Primary Email Domain:} \texttt{[Domain]}
    \item \textbf{Assessed External IP:} \texttt{[Client IP]}
\end{itemize}

\section{Security Control Review}

The following table summarizes the organization's responses to a security controls questionnaire. "No" answers indicate significant gaps in the security framework and are flagged as areas of high concern.

\begin{table}[h!]
\centering
\caption{Organizational Security Controls Questionnaire}
\label{tab:controls}
\begin{tabular}{p{0.6\linewidth} c l}
\toprule
\textbf{Control Question} & \textbf{Response} & \textbf{Assessment} \\
\midrule
Do you require MFA to access email? & \no & \riskcritical{} Gap \\
Do you require MFA to log into computers? & \yes & Best Practice Met \\
Do you require MFA to access sensitive data systems? & \no & \riskcritical{} Gap \\
Does your organization have an employee acceptable use policy? & \no & \riskhigh{} Risk \\
Does your organization do security awareness training for new employees? & \yes & Best Practice Met \\
Does your organization do security awareness training for all employees at least once per year? & \no & \riskhigh{} Risk \\
\bottomrule
\end{tabular}
\end{table}

\section{Technical Scan Results}

An external network scan was performed against the target IP address. The scan identified one open port, which presents a significant security risk.

\begin{table}[h!]
\centering
\caption{Nmap Scan Findings for Target: \texttt{[Target IP]}}
\label{tab:nmap}
\begin{tabular}{l l l l l}
\toprule
\textbf{Host IP} & \textbf{Port/Proto} & \textbf{State} & \textbf{Service} & \textbf{Product/Version} \\
\midrule
\texttt{[Target IP]} & 3389/tcp & open & ms-wbt-server & (Not specified) \\
\bottomrule
\end{tabular}
\end{table}

\subsection*{Analysis of Technical Findings}
The scan confirms that port \textbf{3389/tcp}, the standard port for Microsoft Remote Desktop Protocol (RDP), is open to the public internet. RDP is a frequent target for brute-force password attacks, credential stuffing, and exploitation of known vulnerabilities (e.g., BlueKeep). Exposing this service without mitigating controls like a VPN or IP whitelisting is a critical security flaw.

\section{Risk Assessment Summary}

This section synthesizes findings from the security questionnaire, technical scan, and pre-existing risk data into a consolidated list of identified risks.

\begin{table}[h!]
\centering
\caption{Consolidated Risk Register}
\label{tab:risks}
\begin{tabular}{p{0.2\linewidth} p{0.2\linewidth} p{0.5\linewidth}}
\toprule
\textbf{Risk ID} & \textbf{Severity} & \textbf{Description} \\
\midrule
\textbf{RISK-001:} RDP Exposure & \riskcritical{} (9.0) & The RDP service on \texttt{[Target IP]} is directly accessible from the internet. This allows attackers to attempt to gain control of the system through brute-force attacks or by exploiting vulnerabilities in the RDP service itself. \\
\addlinespace
\textbf{RISK-002:} Lack of MFA on Critical Systems & \riskcritical{} & The absence of MFA for email and sensitive data systems means that a single compromised password is sufficient for an attacker to gain access. This dramatically elevates the threat posed by phishing and credential theft, and directly compounds the risk of the exposed RDP service. \\
\addlinespace
\textbf{RISK-003:} Foundational Policy \& Training Gaps & \riskhigh{} & The lack of an Acceptable Use Policy and mandatory annual security training indicates a reactive security posture. This increases the likelihood of human error leading to security incidents, such as falling for phishing attacks that could compromise credentials. \\
\bottomrule
\end{tabular}
\end{table}

\section{Recommendations}

The following actionable recommendations are provided to mitigate the identified risks. They are prioritized based on severity and potential impact.

\subsection*{Priority 1: Immediate Remediation (Address RISK-001)}
\begin{enumerate}
    \item \textbf{Immediately Block Port 3389:} Configure the external firewall to deny all inbound traffic to port 3389/tcp on \texttt{[Target IP]}. This is the most critical first step to remove the immediate threat.
    \item \textbf{Implement a Secure Remote Access Solution:} For long-term needs, replace direct RDP access with a secure alternative. Options include:
    \begin{itemize}
        \item A Virtual Private Network (VPN) with MFA.
        \item A modern Remote Desktop Gateway.
        \item A Zero Trust Network Access (ZTNA) solution.
    \end{itemize}
\end{enumerate}

\subsection*{Priority 2: High-Impact Controls (Address RISK-002)}
\begin{enumerate}
    \setcounter{enumi}{2}
    \item \textbf{Enforce MFA Everywhere:} Deploy and mandate the use of MFA for all users and services, prioritizing the following:
    \begin{itemize}
        \item All email accounts (cloud-based and on-premise).
        \item All remote access solutions (VPN, etc.).
        \item All systems containing sensitive or critical organizational data.
    \end{itemize}
\end{enumerate}

\subsection*{Priority 3: Foundational Improvements (Address RISK-003)}
\begin{enumerate}
    \setcounter{enumi}{3}
    \item \textbf{Develop and Implement an Acceptable Use Policy (AUP):} Create a formal AUP that clearly defines the rules for using company IT assets, data, and internet access. Ensure all employees read and acknowledge the policy.
    \item \textbf{Establish Annual Security Awareness Training:} Implement a mandatory, annual security awareness training program for all employees. The training should cover topics such as phishing, password security, and social engineering.
\end{enumerate}

\section*{Conclusion}
The current security posture of \textbf{[Organization Name]} contains critical, high-impact risks that require immediate attention. The combination of an exposed RDP service and a lack of Multi-Factor Authentication creates a direct path for a potential network compromise. By following the prioritized recommendations outlined in this report, the organization can significantly reduce its attack surface, strengthen its defenses, and build a more resilient security foundation.

\end{document}
```