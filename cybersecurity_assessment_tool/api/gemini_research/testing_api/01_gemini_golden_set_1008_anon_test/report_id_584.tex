```latex
\documentclass[12pt]{article}

% Preamble: Required Packages
\usepackage[margin=1in]{geometry}
\usepackage{pifont} % For checkmarks and crosses
\usepackage{booktabs} % For professional tables
\usepackage{hyperref} % For clickable links
\usepackage{url} % For formatting URLs
\usepackage{seqsplit} % For splitting long strings in tt font
\usepackage{graphicx}
\usepackage{xcolor}
\usepackage{fancyhdr}

% --- Document Setup ---
\hypersetup{
    colorlinks=true,
    linkcolor=blue,
    filecolor=magenta,      
    urlcolor=cyan,
    pdftitle={Cybersecurity Posture Report},
    pdfpagemode=FullScreen,
}

\pagestyle{fancy}
\fancyhf{}
\lhead{Cybersecurity Posture Report}
\rhead{\textbf{[Organization Name]}}
\cfoot{\thepage}

% --- Document Start ---
\begin{document}

% --- Title Page ---
\begin{titlepage}
    \centering
    \vspace*{1cm}
    \Huge\textbf{Cybersecurity Posture Report}
    \vspace{1.5cm}
    \Large
    Prepared for: \\
    \vspace{0.5cm}
    \textbf{[Organization Name]}
    \vspace{2cm}
    \large
    Report Date: \today \\
    \vspace{1.5cm}
    Generated by: \\
    \textbf{Cybersecurity Analysis Division}
    \vfill
    \textit{This report contains sensitive information and should be handled with care. Distribution is restricted to authorized personnel only.}
\end{titlepage}

\tableofcontents
\newpage

% --- Section 1: Executive Summary ---
\section{Executive Summary}
This report provides a comprehensive analysis of the cybersecurity posture for \textbf{[Organization Name]}, based on a combination of network scanning, a security controls questionnaire, and a review of pre-existing risks. The assessment reveals several critical and high-risk vulnerabilities that require immediate attention.

Key findings indicate significant gaps in fundamental security controls. The lack of Multi-Factor Authentication (MFA) on employee computers, coupled with the absence of a formal acceptable use policy and security awareness training, exposes the organization to substantial risk from credential theft and insider threats.

Furthermore, technical scanning identified an open port serving unencrypted HTTP traffic, which presents a clear and immediate risk of data interception. These findings, when correlated, paint a picture of an environment vulnerable to common attack vectors. We strongly recommend prioritizing the implementation of MFA, decommissioning insecure services, and establishing a robust security training and policy framework as outlined in the Recommendations section.

% --- Section 2: Organizational Information ---
\section{Organizational Information}
This assessment was conducted based on the following organizational details. As some data was not provided, placeholders have been used.

\begin{itemize}
    \item \textbf{Organization Name:} \textbf{[Organization Name]}
    \item \textbf{Primary Domain:} \texttt{[Domain]}
    \item \textbf{External IP Scanned:} \texttt{[Client IP]}
\end{itemize}

% --- Section 3: Security Control Review ---
\section{Security Control Review}
The following table summarizes the organization's responses to a security controls questionnaire. Items marked with \ding{55} (No) represent significant gaps in the security framework and are considered high-risk findings.

\begin{table}[h!]
\centering
\caption{Security Controls Questionnaire Results}
\label{tab:controls}
\begin{tabular}{p{0.7\linewidth} c}
\toprule
\textbf{Control Question} & \textbf{Response} \\
\midrule
Do you require MFA to access email? & \ding{51} \\
Do you require MFA to log into computers? & \textbf{\color{red}\ding{55}} \\
Do you require MFA to access sensitive data systems? & \ding{51} \\
Does your organization have an employee acceptable use policy? & \textbf{\color{red}\ding{55}} \\
Does your organization do security awareness training for new employees? & \textbf{\color{red}\ding{55}} \\
Does your organization do security awareness training for all employees at least once per year? & \textbf{\color{red}\ding{55}} \\
\bottomrule
\end{tabular}
\end{table}

\subsection*{Analysis of Control Gaps}
The questionnaire reveals critical deficiencies in foundational security practices:
\begin{itemize}
    \item \textbf{No MFA for Computer Logins:} This is a critical vulnerability. If an employee's credentials are stolen (e.g., via phishing), an attacker can gain direct access to their computer and potentially the internal network without any secondary authentication barrier.
    \item \textbf{Lack of Security Policy and Training:} The absence of an Acceptable Use Policy (AUP) and any form of security awareness training creates a high-risk environment. Employees are not formally educated on security best practices or their responsibilities, making them significantly more susceptible to social engineering and phishing attacks. This represents a major gap in the human element of the defense-in-depth strategy.
\end{itemize}

% --- Section 4: Technical Scan Results ---
\section{Technical Scan Results}
An external network scan was performed on the target IP address to identify accessible services. The scan was conducted using Nmap.

\begin{itemize}
    \item \textbf{Target IP:} \texttt{[Target IP]}
    \item \textbf{Scan Date:} Scan date not provided in source data.
\end{itemize}

The following table details the open ports discovered during the scan.

\begin{table}[h!]
\centering
\caption{Open Port Analysis}
\label{tab:ports}
\begin{tabular}{l l l p{0.5\linewidth}}
\toprule
\textbf{Port} & \textbf{State} & \textbf{Service (Inferred)} & \textbf{Analysis \& Notes} \\
\midrule
80/tcp & Open & HTTP & \textbf{High Risk.} This port is used for unencrypted web traffic. Any data, including credentials or sensitive information, transmitted over this port can be easily intercepted. Standard practice is to redirect all HTTP traffic to HTTPS (Port 443). \\
\bottomrule
\end{tabular}
\end{table}

% --- Section 5: Synthesized Risk Assessment ---
\section{Synthesized Risk Assessment}
This section correlates findings from the security control review and the technical scan to provide a holistic view of the primary risks facing the organization. The malicious/test data entry from Input 3 has been disregarded as invalid.

\begin{table}[h!]
\centering
\caption{Summary of Identified Risks}
\label{tab:risks}
\begin{tabular}{p{0.5\linewidth} l l}
\toprule
\textbf{Risk Description} & \textbf{Source} & \textbf{Severity} \\
\midrule
\textbf{Lack of MFA on Endpoints:} Stolen credentials can lead to direct endpoint and network compromise. & Questionnaire & \textbf{High} \\
\addlinespace
\textbf{Inadequate Security Policies \& Training:} Employees are unprepared to identify and resist social engineering attacks, increasing the likelihood of a breach. & Questionnaire & \textbf{High} \\
\addlinespace
\textbf{Unencrypted Web Traffic (HTTP):} Sensitive data transmitted to and from the web server is vulnerable to eavesdropping and man-in-the-middle attacks. & Network Scan & \textbf{High} \\
\bottomrule
\end{tabular}
\end{table}

% --- Section 6: Recommendations ---
\section{Recommendations}
Based on the synthesized risk assessment, we recommend the following actions, prioritized by urgency and impact.

\subsection{Priority 1: Critical}
\begin{enumerate}
    \item \textbf{Enforce MFA on All Endpoints:} Immediately deploy a solution to require Multi-Factor Authentication for all computer and remote access logins. This single control drastically reduces the risk of a breach from compromised credentials.
    \item \textbf{Remediate Insecure Web Service:}
        \begin{itemize}
            \item Immediately deploy a valid TLS/SSL certificate on the web server.
            \item Configure the web server to enforce HTTPS (Port 443) for all connections.
            \item Implement a permanent redirect (HTTP 301) for all traffic from HTTP (Port 80) to HTTPS.
        \end{itemize}
\end{enumerate}

\subsection{Priority 2: High}
\begin{enumerate}
    \setcounter{enumi}{2} % Continue numbering
    \item \textbf{Develop and Implement Security Training:}
        \begin{itemize}
            \item Institute a mandatory security awareness training program for all new employees as part of the onboarding process.
            \item Conduct annual, mandatory security awareness training for all staff, covering topics like phishing, password hygiene, and social engineering.
        \end{itemize}
    \item \textbf{Establish an Acceptable Use Policy (AUP):} Draft and implement a formal AUP that clearly defines the rules and responsibilities for employees when using company IT assets. This policy should be reviewed and signed by all employees.
\end{enumerate}

% --- Section 7: Conclusion ---
\section{Conclusion}
The assessment of \textbf{[Organization Name]} has identified several high-risk vulnerabilities stemming from both policy and technical control deficiencies. The combination of missing endpoint MFA, lack of employee security training, and the exposure of unencrypted services creates a significant risk of a security breach.

By implementing the prioritized recommendations in this report, the organization can substantially improve its security posture, reduce its attack surface, and build a more resilient defense against common cyber threats.

\end{document}
```