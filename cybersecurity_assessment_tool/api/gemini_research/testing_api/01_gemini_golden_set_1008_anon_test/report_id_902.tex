```latex
\documentclass[12pt]{article}

% --- PACKAGES ---
\usepackage[margin=1in]{geometry}
\usepackage{pifont} % For checkmarks and crosses
\usepackage{booktabs} % For professional tables
\usepackage{hyperref} % For clickable links
\usepackage{url} % For URL formatting
\usepackage{seqsplit} % For splitting long strings
\usepackage{graphicx}
\usepackage{xcolor}

% --- DOCUMENT METADATA ---
\title{Cybersecurity Risk Assessment Report}
\author{Cybersecurity Analyst}
\date{\today}

% --- HYPERREF SETUP ---
\hypersetup{
    colorlinks=true,
    linkcolor=black,
    urlcolor=blue,
    pdftitle={Cybersecurity Risk Assessment Report},
    pdfauthor={Cybersecurity Analyst},
}

% --- DOCUMENT START ---
\begin{document}

\maketitle
\thispagestyle{empty}
\newpage

\tableofcontents
\newpage

% ===================================================================
% SECTION 1: EXECUTIVE SUMMARY
% ===================================================================
\section{Executive Summary}

This report provides a comprehensive cybersecurity risk assessment for \textbf{[Organization Name]}. The analysis is based on a synthesis of network scan data, a review of organizational security controls, and a list of pre-existing risks.

The assessment has identified several critical and high-severity vulnerabilities that require immediate attention. The most severe findings include:

\begin{itemize}
    \item \textbf{Critical External Vulnerability:} A publicly accessible FTP server was discovered at \texttt{[Client IP]} running a dangerously outdated version of \texttt{vsftpd} (2.3.4). This version contains a well-known backdoor vulnerability (CVE-2011-2523) that allows for remote code execution. The server is also configured to allow anonymous logins, significantly increasing the risk of a breach.
    \item \textbf{Critical Identity and Access Management Gaps:} Multi-Factor Authentication (MFA) is not enforced for accessing email or other sensitive data systems. This exposes the organization to significant risk from phishing attacks and credential theft.
    \item \textbf{High-Risk Policy Gap:} The organization does not conduct mandatory annual security awareness training for all employees, leading to a degradation of security consciousness and increasing susceptibility to social engineering attacks.
\end{itemize}

These findings, combined with the pre-existing risk of outdated Windows 7 workstations, create a vulnerable environment. This report details these risks and provides prioritized, actionable recommendations to mitigate them and improve the overall security posture of the organization.

% ===================================================================
% SECTION 2: ORGANIZATIONAL INFORMATION
% ===================================================================
\section{Organizational Information}

This assessment pertains to the following entity and associated assets. The information provided was either supplied directly or derived during the assessment process.

\begin{tabular}{@{}ll}
    \toprule
    \textbf{Attribute} & \textbf{Value} \\
    \midrule
    Organization Name & \textbf{[Organization Name]} \\
    Primary Domain & \texttt{[Domain]} \\
    External IP Address Scanned & \texttt{[Client IP]} \\
    \bottomrule
\end{tabular}

% ===================================================================
% SECTION 3: SECURITY CONTROL REVIEW
% ===================================================================
\section{Security Control Review}

A review of internal security policies and controls was conducted based on a standardized questionnaire. The responses highlight significant gaps in the organization's access control and security training programs.

\begin{table}[h!]
\centering
\caption{Security Controls Questionnaire Analysis}
\begin{tabular}{@{}p{0.6\linewidth} c p{0.2\linewidth}@{}}
    \toprule
    \textbf{Control Question} & \textbf{Response} & \textbf{Assessment} \\
    \midrule
    Do you require MFA to access email? & \ding{55} & \textcolor{red}{\textbf{Critical Gap}} \\
    Do you require MFA to log into computers? & \ding{51} & Implemented \\
    Do you require MFA to access sensitive data systems? & \ding{55} & \textcolor{red}{\textbf{Critical Gap}} \\
    Does your organization have an employee acceptable use policy? & \ding{51} & Implemented \\
    Does your organization do security awareness training for new employees? & \ding{51} & Implemented \\
    Does your organization do security awareness training for all employees at least once per year? & \ding{55} & \textcolor{orange}{\textbf{High Risk}} \\
    \bottomrule
\end{tabular}
\end{table}

% ===================================================================
% SECTION 4: TECHNICAL SCAN RESULTS
% ===================================================================
\section{Technical Scan Results}

An external network scan was performed on the public-facing IP address provided. The scan identified one host with a critical vulnerability.

\begin{itemize}
    \item \textbf{Target IP Address:} \texttt{[Target IP]}
    \item \textbf{Host Status:} Up
\end{itemize}

\begin{table}[h!]
\centering
\caption{Open Ports and Services Detected on \texttt{[Target IP]}}
\begin{tabular}{@{}lllll@{}}
    \toprule
    \textbf{Port} & \textbf{State} & \textbf{Service} & \textbf{Product / Version} & \textbf{Notes} \\
    \midrule
    21/tcp & Open & ftp & vsftpd 2.3.4 & \begin{tabular}[t]{@{}l@{}}\textcolor{red}{\textbf{CRITICAL: Known RCE vulnerability}}\\ \textcolor{red}{(CVE-2011-2523).}\\ Anonymous FTP login is allowed.\end{tabular} \\
    \bottomrule
\end{tabular}
\end{table}

\subsection{Analysis of Technical Findings}
The presence of an open FTP port is highly discouraged, as the protocol transmits credentials and data in cleartext. The specific version of the FTP daemon, \texttt{vsftpd 2.3.4}, is over a decade old and contains a critical backdoor vulnerability that was intentionally added to the source code. This vulnerability allows an unauthenticated attacker to gain a command shell on the server, leading to a complete system compromise. The configuration allowing anonymous login exacerbates this risk by removing the need for credentials entirely.

% ===================================================================
% SECTION 5: CORRELATED RISK ASSESSMENT
% ===================================================================
\section{Correlated Risk Assessment}

This section synthesizes findings from the security control review, technical scan, and pre-existing risk data into a prioritized list of security risks.

\begin{table}[h!]
\centering
\caption{Summary of Identified Risks}
\begin{tabular}{@{}p{0.1\linewidth} p{0.4\linewidth} p{0.2\linewidth} p{0.15\linewidth}@{}}
    \toprule
    \textbf{Risk ID} & \textbf{Description} & \textbf{Source} & \textbf{Severity} \\
    \midrule
    RISK-001 & A publicly accessible server is running a vulnerable version of vsftpd (2.3.4) with a known RCE backdoor and allows anonymous login. & Technical Scan & \textcolor{red}{\textbf{Critical}} \\
    \addlinespace
    RISK-002 & Lack of MFA on email and sensitive data systems exposes the organization to account takeover and data breaches via credential theft. & Questionnaire & \textcolor{red}{\textbf{Critical}} \\
    \addlinespace
    RISK-003 & The absence of annual security awareness training increases the likelihood of employees falling victim to social engineering and phishing attacks. & Questionnaire & \textcolor{orange}{\textbf{High}} \\
    \addlinespace
    RISK-004 & Workstations are running the unsupported Windows 7 operating system, which no longer receives security updates. & Existing Risks & \textcolor{yellow!80!black}{\textbf{Medium}} \\
    \bottomrule
\end{tabular}
\end{table}

% ===================================================================
% SECTION 6: RECOMMENDATIONS
% ===================================================================
\section{Recommendations}

The following actions are recommended to mitigate the identified risks. They are prioritized based on severity and potential impact.

\subsection{Immediate Priority (Critical Risks)}
\begin{enumerate}
    \item \textbf{Remediate Vulnerable FTP Server (RISK-001):}
    \begin{itemize}
        \item Immediately take the FTP service on \texttt{[Target IP]} offline.
        \item Conduct a forensic analysis of the server to determine if it has already been compromised.
        \item If the FTP service is business-critical, replace it with a secure alternative such as SFTP (SSH File Transfer Protocol) and ensure it is fully patched and configured securely with mandatory authentication.
        \item If the service is not required, it should be permanently disabled and firewalled.
    \end{itemize}

    \item \textbf{Implement Multi-Factor Authentication (RISK-002):}
    \begin{itemize}
        \item Prioritize the deployment of MFA across all email accounts (e.g., via Microsoft 365 or Google Workspace controls).
        \item Identify all systems classified as holding sensitive data and enforce MFA for all user access.
    \end{itemize}
\end{enumerate}

\subsection{High Priority}
\begin{enumerate}
    \setcounter{enumi}{2}
    \item \textbf{Establish Annual Security Training (RISK-003):}
    \begin{itemize}
        \item Procure and implement a security awareness training program for all employees.
        \item Make the training mandatory, with completion tracked annually. Training should cover phishing, password security, and acceptable use policies.
    \end{itemize}
\end{enumerate}

\subsection{Medium Priority}
\begin{enumerate}
    \setcounter{enumi}{3}
    \item \textbf{Upgrade Outdated Workstations (RISK-004):}
    \begin{itemize}
        \item Continue with the existing plan to upgrade all Windows 7 workstations to a modern, supported operating system like Windows 10 or 11.
        \item Ensure that any new hardware is procured with a supported OS pre-installed.
    \end{itemize}
\end{enumerate}

\end{document}
```