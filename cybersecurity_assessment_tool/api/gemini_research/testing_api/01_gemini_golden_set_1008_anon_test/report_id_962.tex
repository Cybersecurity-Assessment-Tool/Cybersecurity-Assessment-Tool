```latex
\documentclass[12pt]{article}

% Preamble: Required Packages
\usepackage[margin=1in]{geometry}
\usepackage{pifont} % For symbols like checkmarks (\ding{51}) and crosses (\ding{55})
\usepackage{booktabs} % For professional-looking tables
\usepackage{hyperref}
\usepackage{url}
\usepackage{seqsplit} % To break long strings without spaces
\usepackage{xcolor}   % For color-coding severity levels
\usepackage{graphicx} % For potential logos or diagrams
\usepackage{fancyhdr} % For headers and footers

% --- Document Setup ---
\hypersetup{
    colorlinks=true,
    linkcolor=blue,
    filecolor=magenta,
    urlcolor=cyan,
    pdftitle={Cybersecurity Posture Assessment Report},
    pdfauthor={Cybersecurity Analysis Division},
}

\pagestyle{fancy}
\fancyhf{} % Clear all header and footer fields
\fancyhead[L]{Cybersecurity Assessment Report}
\fancyhead[R]{\textbf{[Organization Name]}}
\fancyfoot[C]{\thepage}

% --- Custom Commands for Severity ---
\newcommand{\sevCRITICAL}{\textcolor{red}{\textbf{Critical}}}
\newcommand{\sevHIGH}{\textcolor{orange}{\textbf{High}}}
\newcommand{\sevMEDIUM}{\textcolor{yellow!80!black}{\textbf{Medium}}}
\newcommand{\sevLOW}{\textcolor{green!70!black}{\textbf{Low}}}

% --- Document Start ---
\begin{document}

% --- Title Page ---
\begin{titlepage}
    \centering
    \vspace*{1cm}
    \Huge{\textbf{Cybersecurity Posture Assessment Report}}
    \vspace{1.5cm}
    \Large{Prepared for:}
    \vspace{0.5cm}
    \huge{\textbf{[Organization Name]}}
    \vspace{2cm}
    \large{Date of Report: \today}
    \vfill
    \large{Generated by: Cybersecurity Analysis Division}
\end{titlepage}

\tableofcontents
\newpage

% --- Executive Summary ---
\section*{Executive Summary}
This report details the findings of a cybersecurity assessment for \textbf{[Organization Name]}. The assessment combined an external network scan, a review of internal security controls via questionnaire, and an analysis of pre-existing risks.

A \sevCRITICAL{} vulnerability was identified on an external-facing server at \texttt{[Client IP]}. An outdated and highly vulnerable FTP service (\texttt{vsftpd 2.3.4}) is exposed, permitting anonymous access. This poses an immediate and severe risk of data breach, malware distribution, and complete system compromise. The specific version of this software is known to contain a public backdoor exploit.

Furthermore, a \sevHIGH{} risk was identified in the organization's security procedures: the lack of mandatory annual security awareness training for all employees. This procedural gap significantly increases susceptibility to social engineering attacks, such as phishing, which could lead to credential theft or a ransomware incident.

These findings, combined with the known \sevMEDIUM{} risk of outdated Windows 7 workstations, indicate a need for immediate remediation to secure the organization's perimeter and improve its internal security culture. Prioritized, actionable recommendations are provided in the final section of this report.

\newpage

% --- Organizational Information ---
\section{Organizational Information}
This section provides the context for the assessment based on the provided data.
\begin{center}
\begin{tabular}{@{}ll@{}}
\toprule
\textbf{Attribute} & \textbf{Value} \\
\midrule
Organization Name & \textbf{[Organization Name]} \\
Email Domain & \texttt{[Domain]} \\
External IP Scanned & \texttt{[Client IP]} \\
Target IP Scanned & \texttt{[Target IP]} \\
\bottomrule
\end{tabular}
\end{center}

% --- Security Control Review ---
\section{Security Control Review}
The following table summarizes the organization's responses to a security controls questionnaire. Gaps in security practices are highlighted and analyzed.

\begin{center}
\begin{tabular}{@{}p{0.8\textwidth}c@{}}
\toprule
\textbf{Control Question} & \textbf{Status} \\
\midrule
Do you require MFA to access email? & \ding{51} \\
Do you require MFA to log into computers? & \ding{51} \\
Do you require MFA to access sensitive data systems? & \ding{51} \\
Does your organization have an employee acceptable use policy? & \ding{51} \\
Does your organization do security awareness training for new employees? & \ding{51} \\
Does your organization do security awareness training for all employees at least once per year? & \textcolor{red}{\ding{55}} \\
\bottomrule
\end{tabular}
\end{center}
\vspace{1em}

\textbf{Analysis:} The organization demonstrates a strong commitment to identity and access management through the widespread implementation of Multi-Factor Authentication (MFA). However, the lack of recurring, annual security awareness training for all staff represents a significant procedural gap. Threat landscapes evolve, and without continuous education, employees are more likely to fall victim to modern social engineering tactics, negating other technical controls.

% --- Technical Scan Results ---
\section{Technical Scan Results}
An external network scan was performed against the target IP \texttt{[Target IP]}. The results reveal a critical misconfiguration on a public-facing service.

\subsection{Open Ports and Services}
\begin{center}
\begin{tabular}{@{}lllll@{}}
\toprule
\textbf{Port} & \textbf{State} & \textbf{Service} & \textbf{Product} & \textbf{Version} \\
\midrule
21/tcp & Open & ftp & vsftpd & 2.3.4 \\
\bottomrule
\end{tabular}
\end{center}

\subsection{Critical Findings from Technical Scan}
The scan identified two major issues with the open FTP port:
\begin{itemize}
    \item \textbf{Vulnerable Service Version:} The FTP service is running \texttt{vsftpd version 2.3.4}. This version, released in 2011, contains a critical backdoor vulnerability (\textbf{CVE-2011-2523}) that allows for remote command execution by a malicious actor. This effectively gives an attacker full control over the server.
    \item \textbf{Insecure Configuration:} The scan confirmed that \textbf{Anonymous FTP login is allowed}. This permits any unauthenticated user on the internet to connect to the server, potentially accessing, uploading, or deleting files. The FTP protocol also transmits all data, including any potential credentials, in cleartext, making it susceptible to eavesdropping.
\end{itemize}

% --- Consolidated Risk Assessment ---
\section{Consolidated Risk Assessment}
The following table consolidates and prioritizes risks identified from the technical scan, control review, and pre-existing data.

\begin{center}
\begin{tabular}{@{}p{0.3\textwidth}p{0.2\textwidth}p{0.4\textwidth}@{}}
\toprule
\textbf{Risk / Vulnerability} & \textbf{Severity} & \textbf{Impact Summary} \\
\midrule
Exposed Vulnerable FTP Server (\texttt{vsftpd 2.3.4}) with Anonymous Login & \sevCRITICAL & Immediate risk of unauthorized access, data exfiltration, malware hosting, or full system compromise via a known public exploit. \\
\addlinespace
Lack of Annual Security Awareness Training & \sevHIGH & Increased susceptibility to phishing, social engineering, and malware infection, potentially leading to credential theft, financial loss, or ransomware. \\
\addlinespace
Outdated Windows 7 Workstations & \sevMEDIUM & End-of-life operating system no longer receives security updates, leaving workstations vulnerable to a wide range of known exploits and compromising internal network security. \\
\bottomrule
\end{tabular}
\end{center}

% --- Recommendations ---
\section{Recommendations}
The following actions are recommended to mitigate the identified risks, prioritized by urgency.

\subsection{Immediate Priority (Critical Risk)}
\begin{enumerate}
    \item \textbf{Disable or Isolate the FTP Server:} Immediately take the FTP server at \texttt{[Target IP]} offline by shutting down the service or applying a firewall rule to block all access to port 21.
    \item \textbf{Investigate for Compromise:} Due to the age and severity of the vulnerability, assume the server has been compromised. Conduct a forensic analysis to search for indicators of compromise (IoCs), unauthorized accounts, or malicious files.
    \item \textbf{Migrate to a Secure Protocol:} If file transfer functionality is a business requirement, decommission the FTP service permanently and replace it with a secure alternative such as SFTP (SSH File Transfer Protocol) or FTPS (FTP over SSL/TLS).
\end{enumerate}

\subsection{High Priority}
\begin{enumerate}
    \item \textbf{Implement Annual Security Training:} Establish a mandatory security awareness training program for all employees, to be completed annually. This program should cover, at a minimum, phishing identification, password security, and the organization's acceptable use policy.
\end{enumerate}

\subsection{Medium Priority}
\begin{enumerate}
    \item \textbf{Accelerate OS Upgrade Plan:} Continue and prioritize the project to upgrade or replace all workstations running Windows 7. Establish a firm deadline for decommissioning all end-of-life systems to reduce the internal attack surface.
\end{enumerate}

\end{document}
```