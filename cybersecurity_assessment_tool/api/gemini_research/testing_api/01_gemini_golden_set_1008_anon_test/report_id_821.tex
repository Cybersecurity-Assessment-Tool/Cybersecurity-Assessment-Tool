```latex
\documentclass[12pt]{article}

% Preamble: Required Packages
\usepackage[margin=1in]{geometry}
\usepackage{pifont} % For checkmarks and crosses
\usepackage{booktabs} % For professional tables
\usepackage{hyperref} % For hyperlinks
\usepackage{url} % For URL formatting
\usepackage{seqsplit} % For splitting long strings
\usepackage{graphicx}
\usepackage{xcolor}

% --- Document Metadata ---
\title{Cybersecurity Posture Assessment Report}
\author{Cybersecurity Analysis Division}
\date{\today}

% --- Hyperref Setup ---
\hypersetup{
    colorlinks=true,
    linkcolor=blue,
    filecolor=magenta,      
    urlcolor=cyan,
    pdftitle={Cybersecurity Posture Assessment Report},
    pdfpagemode=FullScreen,
}

\begin{document}

\maketitle
\thispagestyle{empty}
\newpage

\tableofcontents
\newpage

% ==============================================================================
% Section 1: Executive Summary
% ==============================================================================
\section{Executive Summary}

This report provides a cybersecurity posture assessment for \textbf{[Organization Name]}. The analysis is based on a synthesis of external network scan data, a security controls questionnaire, and a review of pre-existing risk documentation.

The most critical finding is the direct exposure of the Remote Desktop Protocol (RDP) on port 3389 for the asset at \texttt{[Target IP]}. This vulnerability, rated with a CVSS score of 9.0 (Critical), is actively targeted by threat actors for initial access, often leading to ransomware deployment and data exfiltration.

This critical technical vulnerability is significantly exacerbated by two major gaps in organizational security controls identified from the questionnaire:
\begin{itemize}
    \item \textbf{Lack of Multi-Factor Authentication (MFA) for computer logins:} This allows an attacker with compromised credentials to gain direct access to internal systems without a secondary verification step.
    \item \textbf{Absence of a formal Security Awareness Training program:} Employees are not equipped to recognize or report phishing attacks, which is a primary vector for credential theft.
\end{itemize}

The combination of an exposed, high-value service (RDP) with weak credential protection and a lack of user security awareness creates a high-likelihood path to a significant security breach. Immediate remediation is required to mitigate this imminent threat.

% ==============================================================================
% Section 2: Organizational Information
% ==============================================================================
\section{Organizational Information}

This section details the information provided about the organization. The placeholders indicate that this data was not available at the time of the report generation and should be populated internally.

\begin{tabular}{@{}ll}
    \toprule
    \textbf{Attribute} & \textbf{Value} \\
    \midrule
    Organization Name & \textbf{[Organization Name]} \\
    Primary Domain & \texttt{[Domain]} \\
    External IP Address (Scanned) & \texttt{[Client IP]} \\
    \bottomrule
\end{tabular}

% ==============================================================================
% Section 3: Security Control Review
% ==============================================================================
\section{Security Control Review}

A review of the organization's security controls was conducted via a questionnaire. The results below highlight key strengths and critical weaknesses. A red cross (\ding{55}) indicates a control gap that increases organizational risk.

\begin{table}[h!]
\centering
\begin{tabular}{@{}p{0.7\textwidth}c}
    \toprule
    \textbf{Control Question} & \textbf{Status} \\
    \midrule
    Do you require MFA to access email? & \ding{51} \\
    Do you require MFA to log into computers? & \textcolor{red}{\ding{55}} \\
    Do you require MFA to access sensitive data systems? & \ding{51} \\
    Does your organization have an employee acceptable use policy? & \ding{51} \\
    Does your organization do security awareness training for new employees? & \textcolor{red}{\ding{55}} \\
    Does your organization do security awareness training for all employees at least once per year? & \textcolor{red}{\ding{55}} \\
    \bottomrule
\end{tabular}
\caption{Security Controls Questionnaire Results}
\label{tab:controls}
\end{table}

\subsection*{Analysis of Control Gaps}
\begin{itemize}
    \item \textbf{No MFA on Computer Logins:} This is a high-risk gap. If an employee's credentials are stolen (e.g., via phishing), an attacker can use them to log into a company computer remotely or locally, providing a strong foothold within the network.
    \item \textbf{No Security Awareness Training:} The lack of both initial and recurring training leaves the organization highly vulnerable to social engineering attacks. Employees are the first line of defense, and without training, they cannot be expected to identify and avoid sophisticated phishing emails or other lures.
\end{itemize}

% ==============================================================================
% Section 4: Technical Scan Results
% ==============================================================================
\section{Technical Scan Results}

An external network scan was performed to identify exposed services. The scan confirmed the findings from the pre-existing risk documentation.

\begin{itemize}
    \item \textbf{Target IP Address:} \texttt{[Target IP]}
    \item \textbf{Status:} Host is Up
\end{itemize}

\begin{table}[h!]
\centering
\begin{tabular}{@{}llll@{}}
    \toprule
    \textbf{Port} & \textbf{State} & \textbf{Service Name} & \textbf{Analysis} \\
    \midrule
    3389/tcp & Open & ms-wbt-server & \begin{tabular}[t]{@{}l@{}}Critical Risk. This port is used for \\ Microsoft Remote Desktop Protocol (RDP). \\ Exposing RDP directly to the internet is a \\ major security risk and a primary target \\ for ransomware groups.\end{tabular} \\
    \bottomrule
\end{tabular}
\caption{Open Ports Detected on \texttt{[Target IP]}}
\label{tab:scanresults}
\end{table}

% ==============================================================================
% Section 5: Correlated Risk Assessment
% ==============================================================================
\section{Correlated Risk Assessment}

This section synthesizes the findings from the security control review, technical scan, and pre-existing risk data into a consolidated list of identified risks.

\begin{table}[h!]
\centering
\begin{tabular}{@{}p{0.25\textwidth}p{0.5\textwidth}l@{}}
    \toprule
    \textbf{Risk Name} & \textbf{Description} & \textbf{Severity} \\
    \midrule
    \textbf{Public RDP Exposure} & Port 3389 (RDP) is open to the internet on \texttt{[Target IP]}, allowing direct remote access attempts. This finding is confirmed by both the technical scan and existing risk documentation. & \textbf{Critical} \\
    \addlinespace
    \textbf{No MFA on Endpoints} & The absence of MFA on computer logins drastically lowers the effort required for an attacker with stolen credentials to compromise an endpoint and gain internal network access. & \textbf{High} \\
    \addlinespace
    \textbf{Inadequate Security Awareness} & The lack of a formal training program makes employees highly susceptible to phishing and other social engineering attacks, which are the primary methods for stealing credentials. & \textbf{High} \\
    \bottomrule
\end{tabular}
\caption{Summary of Identified Risks}
\label{tab:risks}
\end{table}

% ==============================================================================
% Section 6: Recommendations
% ==============================================================================
\section{Recommendations}

Based on the correlated risk assessment, the following actions are recommended to improve the security posture of \textbf{[Organization Name]}. Recommendations are prioritized by urgency.

\subsection*{Immediate Actions (Within 24 Hours)}
\begin{enumerate}
    \item \textbf{Block RDP Access:} Immediately configure the perimeter firewall to block all inbound traffic to TCP port 3389 for the asset at \texttt{[Target IP]}. If remote access is essential, restrict access to a list of known, trusted IP addresses as a temporary measure.
\end{enumerate}

\subsection*{Short-Term Actions (1-4 Weeks)}
\begin{enumerate}
    \item \textbf{Implement MFA for Endpoints:} Enforce MFA for all computer logins, prioritizing accounts with administrative privileges and remote access capabilities. Solutions like Windows Hello for Business, Duo, or other third-party tools can be used.
    \item \textbf{Launch Security Awareness Training:} Procure and deploy a security awareness training platform. Enroll all employees in an initial training module covering phishing, password hygiene, and acceptable use.
\end{enumerate}

\subsection*{Long-Term Strategic Actions (1-6 Months)}
\begin{enumerate}
    \item \textbf{Deploy a VPN Solution:} For all remote access requirements, implement a modern Virtual Private Network (VPN) solution. The VPN should be configured to require MFA for user authentication, providing a secure and encrypted tunnel into the network and eliminating the need for any direct RDP exposure.
    \item \textbf{Establish a Continuous Training Program:} Formalize the security awareness training into an ongoing program. This should include mandatory annual training for all staff and regular phishing simulations to test and reinforce learning.
\end{enumerate}

\end{document}
```