```latex
\documentclass[12pt]{article}

% Preamble: Required Packages
\usepackage[margin=1in]{geometry}
\usepackage{pifont} % For using dingbats like checkmarks and crosses
\usepackage{booktabs} % For professional-looking tables
\usepackage{hyperref} % For creating hyperlinks in the document
\usepackage{url}      % For formatting URLs
\usepackage{seqsplit} % To split long strings without breaking words
\usepackage{xcolor}   % For custom colors

% Hyperref and Color Setup
\hypersetup{
    colorlinks=true,
    linkcolor=blue,
    filecolor=magenta,
    urlcolor=cyan,
    pdftitle={Cybersecurity Posture Assessment Report},
    pdfauthor={Cybersecurity Analyst},
}

% Define custom colors for severity
\definecolor{criticalred}{HTML}{D7263D}
\definecolor{highorange}{HTML}{F49D40}
\definecolor{mediumyellow}{HTML}{F4D440}
\definecolor{lowgreen}{HTML}{5CBA47}
\definecolor{infoblue}{HTML}{408AF4}

% Custom commands for checkmarks and crosses
\newcommand{\yes}{\ding{51}}
\newcommand{\no}{\ding{55}}

% Document Start
\begin{document}

% --- TITLE PAGE ---
\begin{titlepage}
    \centering
    \vspace*{\stretch{1.0}}
    \Huge\textbf{Cybersecurity Posture Assessment Report}
    \vspace{1.5cm}
    \Large
    Prepared for: \textbf{[Organization Name]} \\
    \vspace{0.5cm}
    Date: \today \\
    \vspace{1.5cm}
    \normalsize
    This report provides an analysis of the organization's security posture based on a network scan, a security controls questionnaire, and a review of pre-existing risks.
    \vspace*{\stretch{2.0}}
    \rule{\textwidth}{0.4pt}
    \textit{Confidential}
\end{titlepage}

\tableofcontents
\newpage

% --- 1. EXECUTIVE SUMMARY ---
\section{Executive Summary}

This assessment provides a point-in-time analysis of the cybersecurity posture for \textbf{[Organization Name]}. The evaluation is based on a combination of technical scanning, a review of administrative security controls, and correlation with existing risk data.

The key findings indicate significant gaps in identity and access management, specifically the lack of Multi-Factor Authentication (MFA) for email and computer access. This represents a \textbf{critical risk} of unauthorized access and potential account compromise. Furthermore, the absence of annual security awareness training for all employees presents a \textbf{high risk}, as it weakens the human firewall against phishing and social engineering attacks.

On a positive note, the external network scan of the target IP address \texttt{[Client IP]} revealed a hardened perimeter. Notably, Port 80 (HTTP), which was listed as an existing risk, was found to be closed. This suggests that remediation has occurred or the previous finding was a false positive.

Immediate priorities should be the enterprise-wide implementation of MFA and the establishment of a recurring security awareness training program.

% --- 2. ORGANIZATIONAL INFORMATION ---
\section{Organizational Information}

This section outlines the basic information for the organization under review. As the provided data was anonymized, placeholders are used.

\begin{itemize}
    \item \textbf{Organization Name:} \textbf{[Organization Name]}
    \item \textbf{Primary Email Domain:} \texttt{[Domain]}
    \item \textbf{External IP Address Scanned:} \texttt{[Client IP]}
\end{itemize}

% --- 3. SECURITY CONTROL REVIEW ---
\section{Security Control Review}

The following table summarizes the organization's responses to a security controls questionnaire. Items marked with a red cross (\no) indicate a deviation from security best practices and represent a gap in the defensive posture.

\begin{table}[h!]
\centering
\caption{Security Controls Questionnaire Results}
\begin{tabular}{p{0.7\textwidth} c c}
\toprule
\textbf{Control Question} & \textbf{Response} & \textbf{Status} \\
\midrule
Does your organization have an employee acceptable use policy? & Yes & \yes \\
Does your organization do security awareness training for new employees? & Yes & \yes \\
Do you require MFA to access sensitive data systems? & Yes & \yes \\
\midrule
\textcolor{criticalred}{Do you require MFA to access email?} & \textcolor{criticalred}{No} & \textcolor{criticalred}{\no} \\
\textcolor{criticalred}{Do you require MFA to log into computers?} & \textcolor{criticalred}{No} & \textcolor{criticalred}{\no} \\
\textcolor{highorange}{Does your organization do security awareness training for all employees at least once per year?} & \textcolor{highorange}{No} & \textcolor{highorange}{\no} \\
\bottomrule
\end{tabular}
\end{table}

The most critical gaps identified are the lack of MFA for email and endpoint logins. These are primary targets for attackers seeking to gain an initial foothold in a network. The absence of annual security training for all staff is also a high-risk finding.

% --- 4. TECHNICAL SCAN RESULTS ---
\section{Technical Scan Results}

An external network scan was performed to identify open ports and exposed services on the organization's public-facing infrastructure.

\begin{itemize}
    \item \textbf{Target IP Address:} \texttt{[Target IP]}
    \item \textbf{Scan Date:} Data not provided in scan metadata.
    \item \textbf{Host Status:} Up
\end{itemize}

\subsection{Port Scan Analysis}
The scan revealed a minimal attack surface, with no common vulnerable ports found open.

\begin{table}[h!]
\centering
\caption{Port Scan Details}
\begin{tabular}{l l l}
\toprule
\textbf{Port} & \textbf{State} & \textbf{Analysis} \\
\midrule
80/tcp & Closed & The port for unencrypted web traffic (HTTP) is closed. \\
\bottomrule
\end{tabular}
\end{table}

\textbf{Analyst Note:} The current risk register (Input 3) lists a vulnerability named "Unencrypted Web Server" related to an open Port 80. This technical scan \textbf{contradicts} that finding, as Port 80 was found to be closed. This indicates the pre-existing risk has likely been remediated.

% --- 5. CORRELATED RISK ASSESSMENT ---
\section{Correlated Risk Assessment}

This section synthesizes findings from the security control review, technical scan, and pre-existing risk data into a prioritized list of current risks.

\begin{table}[h!]
\centering
\caption{Summary of Identified Risks}
\begin{tabular}{p{0.2\textwidth} p{0.15\textwidth} p{0.55\textwidth}}
\toprule
\textbf{Risk Name} & \textbf{Severity} & \textbf{Description and Business Impact} \\
\midrule
\textbf{Lack of MFA for Critical Systems} & \textcolor{criticalred}{\textbf{Critical}} & The absence of MFA on email and computer logins exposes the organization to a high likelihood of account takeover, data breach, and ransomware deployment via compromised credentials. \\
\hline
\textbf{Inadequate Security Training Cadence} & \textcolor{highorange}{\textbf{High}} & Without mandatory annual refresher training, employees are more susceptible to phishing and social engineering attacks, which are the leading causes of security incidents. \\
\hline
\textbf{Outdated Risk Register} & \textcolor{infoblue}{\textbf{Informational}} & The pre-existing risk "Unencrypted Web Server" appears to be resolved, as Port 80 is closed. An outdated risk register can lead to misallocation of security resources. \\
\bottomrule
\end{tabular}
\end{table}

% --- 6. RECOMMENDATIONS ---
\section{Recommendations}

The following actions are recommended to mitigate the identified risks and improve the overall security posture of \textbf{[Organization Name]}.

\subsection{Immediate Actions (0-30 Days)}
\begin{enumerate}
    \item \textbf{Implement MFA for Email:} Prioritize the deployment of MFA for all user email accounts (e.g., via Microsoft 365 or Google Workspace conditional access policies). This is the single most effective control to prevent business email compromise.
    \item \textbf{Plan Enterprise-Wide MFA Rollout:} Develop a project plan to enforce MFA for all computer logins and other critical systems. Solutions like Windows Hello for Business, Duo, or Okta should be evaluated.
\end{enumerate}

\subsection{Short-Term Actions (30-90 Days)}
\begin{enumerate}
    \item \textbf{Establish Annual Security Training:} Procure and schedule mandatory annual security awareness training for all employees. This program should cover key topics such as phishing, password hygiene, and acceptable use.
    \item \textbf{Update Risk Register:} Formally validate that Port 80 is closed and update the internal risk register to reflect that the "Unencrypted Web Server" risk has been mitigated.
\end{enumerate}

\subsection{Long-Term Actions (90+ Days)}
\begin{enumerate}
    \item \textbf{Conduct Regular Posture Reviews:} Implement a process for quarterly or semi-annual reviews of security controls and technical posture to ensure continuous improvement and adaptation to new threats.
\end{enumerate}

\end{document}
```