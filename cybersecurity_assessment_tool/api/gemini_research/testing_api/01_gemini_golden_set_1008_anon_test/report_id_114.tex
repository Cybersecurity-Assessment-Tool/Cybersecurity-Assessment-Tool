```latex
\documentclass[12pt]{article}

% Preamble: Required Packages
\usepackage[margin=1in]{geometry}
\usepackage{pifont} % For checkmarks and crosses
\usepackage{booktabs} % For professional tables
\usepackage{hyperref} % For clickable links
\usepackage{url} % For URL formatting
\usepackage{seqsplit} % For splitting long strings
\usepackage{xcolor} % For colors

% Document Information
\title{Cybersecurity Posture Assessment Report}
\author{Cybersecurity Analysis Division}
\date{\today}

% Hyperref Setup
\hypersetup{
    colorlinks=true,
    linkcolor=blue,
    filecolor=magenta,      
    urlcolor=cyan,
    pdftitle={Cybersecurity Posture Assessment Report},
    pdfpagemode=FullScreen,
}

\begin{document}

\maketitle
\thispagestyle{empty}
\newpage
\tableofcontents
\newpage

% --- Section 1: Executive Summary ---
\section{Executive Summary}

This report provides a comprehensive cybersecurity assessment for \textbf{[Organization Name]}, synthesizing data from a technical network scan, a security controls questionnaire, and a review of pre-existing risks. The analysis aims to identify key vulnerabilities, control gaps, and provide actionable recommendations to enhance the organization's security posture.

\paragraph{Key Findings:} The organization demonstrates a solid foundation in some areas, such as enforcing Multi-Factor Authentication (MFA) for email and computer access, and maintaining a consistent security awareness training program. However, two critical gaps were identified through the security controls review:
\begin{itemize}
    \item \textbf{Lack of MFA for Sensitive Data Systems:} This represents a high-risk exposure, as it leaves critical assets vulnerable to unauthorized access through compromised credentials.
    \item \textbf{Absence of an Employee Acceptable Use Policy (AUP):} This creates ambiguity regarding the proper use of corporate assets and increases the risk of insider threats and policy violations.
\end{itemize}

\paragraph{Technical Assessment:} The external network scan of the target IP address \texttt{[Target IP]} revealed that port 80 (HTTP) is closed. This is a positive security control. This finding contradicts a pre-existing documented risk, "Unencrypted Web Server," suggesting that the risk has been successfully mitigated or was a false positive.

\paragraph{Overall Posture:} While foundational controls are in place, the identified critical gaps significantly elevate the organization's risk profile. Immediate action is required to address the MFA and policy-related deficiencies to protect sensitive data and establish clear operational guidelines for employees.

% --- Section 2: Organizational Information ---
\section{Organizational Information}

This section details the organizational context for this assessment. The information provided was used to correlate technical findings with the organization's operational environment.

\begin{itemize}
    \item \textbf{Organization Name:} \textbf{[Organization Name]}
    \item \textbf{Primary Email Domain:} \texttt{[Domain]}
    \item \textbf{Scanned External IP:} \texttt{[Client IP]}
\end{itemize}

% --- Section 3: Security Control Review ---
\section{Security Control Review}

The following table summarizes the organization's responses to a security controls questionnaire. The status column indicates alignment with cybersecurity best practices, where a checkmark (\ding{51}) signifies a positive control and a cross (\ding{55}) indicates a significant gap requiring attention.

\begin{table}[h!]
\centering
\caption{Security Controls Questionnaire Results}
\begin{tabular}{p{0.7\linewidth} c c}
\toprule
\textbf{Control Question} & \textbf{Response} & \textbf{Status} \\
\midrule
Do you require MFA to access email? & Yes & \ding{51} \\
Do you require MFA to log into computers? & Yes & \ding{51} \\
Do you require MFA to access sensitive data systems? & No & \textcolor{red}{\ding{55}} \\
Does your organization have an employee acceptable use policy? & No & \textcolor{red}{\ding{55}} \\
Does your organization do security awareness training for new employees? & Yes & \ding{51} \\
Does your organization do security awareness training for all employees at least once per year? & Yes & \ding{51} \\
\bottomrule
\end{tabular}
\end{table}

% --- Section 4: Technical Scan Results ---
\section{Technical Scan Results}

An external network scan was performed on the designated target IP address to identify open ports and exposed services.

\subsection{Target: \texttt{[Target IP]}}
The scan revealed that the host was online, but all commonly scanned ports were closed, which is a strong security posture from an external perspective. The status of the most relevant port is detailed below.

\begin{table}[h!]
\centering
\caption{Port Scan Details for \texttt{[Target IP]}}
\begin{tabular}{l l l l}
\toprule
\textbf{Port} & \textbf{State} & \textbf{Service} & \textbf{Notes} \\
\midrule
80 & Closed & http & Port 80 being closed prevents unencrypted web traffic. \\
\bottomrule
\end{tabular}
\end{table}

% --- Section 5: Risk Assessment ---
\section{Risk Assessment}

This section correlates findings from the questionnaire, the technical scan, and pre-existing risk data to provide a consolidated view of the current risk landscape.

\begin{table}[h!]
\centering
\caption{Consolidated Risk Summary}
\begin{tabular}{p{0.3\linewidth} p{0.15\linewidth} p{0.45\linewidth}}
\toprule
\textbf{Risk Name} & \textbf{Severity} & \textbf{Description \& Status} \\
\midrule
\textbf{Lack of MFA for Sensitive Systems} & \textbf{High} & Failure to enforce MFA on systems containing sensitive data exposes the organization to a significant risk of unauthorized access and data breach. This is a critical control gap. \\
\noalign{\vspace{2mm}}
\textbf{Missing Acceptable Use Policy (AUP)} & \textbf{High} & Without a formal AUP, there is no clear guidance for employees on the proper use of company assets, increasing the risk of insider threat, data leakage, and misuse of resources. \\
\noalign{\vspace{2mm}}
\textbf{Unencrypted Web Server} & \textbf{Mitigated} & A pre-existing risk stated that Port 80 was open. Our technical scan confirmed that Port 80 is \textbf{closed} on the target system. This risk appears to be resolved. \\
\bottomrule
\end{tabular}
\end{table}

% --- Section 6: Recommendations ---
\section{Recommendations}

Based on the analysis, the following actions are recommended to mitigate the identified risks and improve the overall security posture of \textbf{[Organization Name]}.

\begin{enumerate}
    \item \textbf{Implement MFA for Sensitive Systems (Priority: Critical):}
    \begin{itemize}
        \item \textbf{Action:} Immediately prioritize the deployment of Multi-Factor Authentication across all applications, databases, and administrative interfaces that store or process sensitive corporate or customer data.
        \item \textbf{Impact:} This is the single most effective control to prevent account takeover and unauthorized access to critical assets.
    \end{itemize}
    \vspace{5mm}
    \item \textbf{Develop and Implement an Acceptable Use Policy (Priority: Critical):}
    \begin{itemize}
        \item \textbf{Action:} Draft a formal AUP that clearly defines the rules and expectations for using company technology, networks, and data. This policy should be reviewed by legal counsel and communicated to all employees, who must formally acknowledge it.
        \item \textbf{Impact:} Establishes a clear security baseline for employee behavior, reduces legal liability, and mitigates the risk of insider threats.
    \end{itemize}
    \vspace{5mm}
    \item \textbf{Validate and Update Risk Register (Priority: Medium):}
    \begin{itemize}
        \item \textbf{Action:} Formally review the "Unencrypted Web Server" risk in the organization's risk register. Based on the technical scan results, this risk should be marked as "Mitigated" or "Closed."
        \item \textbf{Impact:} Ensures the risk register accurately reflects the current security posture, allowing resources to be focused on active threats.
    \end{itemize}
\end{enumerate}

\end{document}
```