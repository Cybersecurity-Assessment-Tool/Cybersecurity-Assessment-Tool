```latex
\documentclass[12pt]{article}

% --- PACKAGES ---
\usepackage[margin=1in]{geometry}
\usepackage{pifont} % For checkmarks and crosses
\usepackage{booktabs} % For professional tables
\usepackage{hyperref} % For clickable links and ToC
\usepackage{url} % For URL formatting
\usepackage{seqsplit} % For splitting long strings in tt font
\usepackage{xcolor} % For colors
\usepackage{graphicx} % For images (e.g., logo)
\usepackage{fancyhdr} % For headers and footers
\usepackage{tocbibind} % Add ToC, LoF, LoT to ToC

% --- DOCUMENT METADATA & SETUP ---
\hypersetup{
    colorlinks=true,
    linkcolor=blue,
    filecolor=magenta,      
    urlcolor=cyan,
    pdftitle={Cybersecurity Posture Assessment Report},
    pdfauthor={Cybersecurity Analysis Group},
    pdfsubject={Security Assessment},
    pdfkeywords={Cybersecurity, Nmap, Risk, Analysis},
}

% --- HEADER & FOOTER ---
\pagestyle{fancy}
\fancyhf{} % Clear all header and footer fields
\fancyhead[L]{\textbf{Cybersecurity Posture Assessment}}
\fancyhead[R]{\textbf{[Organization Name]}}
\fancyfoot[C]{\thepage}
\renewcommand{\headrulewidth}{0.4pt}
\renewcommand{\footrulewidth}{0.4pt}

% --- TITLE ---
\title{Cybersecurity Posture Assessment Report \\ \large For \textbf{[Organization Name]}}
\author{Cybersecurity Analysis Group}
\date{\today}

% --- BEGIN DOCUMENT ---
\begin{document}

\maketitle
\thispagestyle{empty}
\newpage

\tableofcontents
\newpage

% ==============================================================================
\section{Executive Summary}
% ==============================================================================

This report details the findings of a cybersecurity posture assessment conducted for \textbf{[Organization Name]}. The assessment combined a technical network scan, a review of existing risks, and an analysis of organizational security controls based on a questionnaire.

The overall security posture is determined to be at a \textbf{CRITICAL} risk level. This is primarily due to the discovery of a publicly exposed, end-of-life database service, coupled with significant gaps in foundational security controls. The absence of Multi-Factor Authentication (MFA) on email and the lack of a security awareness training program present immediate and severe threats to the organization.

\subsection*{Key Findings}
\begin{itemize}
    \item \textbf{Critical - Exposed and Outdated Database:} A MySQL database (version 5.7.33) is publicly accessible from the internet on port 3306. This version reached its official End-of-Life (EOL) in October 2023 and no longer receives security updates, making it an easy target for exploitation.
    \item \textbf{Critical - No MFA for Email:} The lack of mandatory MFA for email access exposes the organization to a high risk of Business Email Compromise (BEC), phishing, and account takeovers.
    \item \textbf{High - Foundational Policy Gaps:} The organization lacks a formal Acceptable Use Policy and does not conduct security awareness training. This significantly increases the human-related risk factor, making employees more susceptible to social engineering attacks.
\end{itemize}

Immediate remediation of these issues is strongly recommended to reduce the risk of a significant security breach. Detailed recommendations are provided in Section \ref{sec:recommendations}.

% ==============================================================================
\section{Organizational Information}
% ==============================================================================

The following placeholder information was used for this assessment, as specific organizational data was not provided.

\begin{itemize}
    \item \textbf{Organization Name:} \textbf{[Organization Name]}
    \item \textbf{Primary Domain:} \texttt{[Domain]}
    \item \textbf{Assessed External IP:} \texttt{[Client IP]}
\end{itemize}

% ==============================================================================
\section{Security Control Review}
% ==============================================================================

The following table summarizes the organization's current security controls based on the provided questionnaire. "No" answers indicate significant gaps in the security framework.

\begin{table}[h!]
\centering
\caption{Security Controls Questionnaire Analysis}
\label{tab:controls}
\begin{tabular}{p{0.6\linewidth} c l}
\toprule
\textbf{Control Question} & \textbf{Response} & \textbf{Assessment} \\
\midrule
Do you require MFA to access email? & \ding{55} & \textcolor{red}{\textbf{Critical Gap}} \\
Do you require MFA to log into computers? & \ding{51} & Best Practice Met \\
Do you require MFA to access sensitive data systems? & \ding{51} & Best Practice Met \\
Does your organization have an employee acceptable use policy? & \ding{55} & \textcolor{orange}{High Risk} \\
Does your organization do security awareness training for new employees? & \ding{55} & \textcolor{orange}{High Risk} \\
Does your organization do security awareness training for all employees at least once per year? & \ding{55} & \textcolor{orange}{High Risk} \\
\bottomrule
\end{tabular}
\end{table}

The lack of MFA for email is the most critical finding in this section. Email is a primary target for attackers, and its compromise often serves as a gateway to further infiltration of an organization's systems. The absence of an Acceptable Use Policy and security training programs creates a culture where employees are unaware of security best practices, making them vulnerable to phishing and other social engineering tactics.

% ==============================================================================
\section{Technical Scan Results}
% ==============================================================================

A network scan was performed on the target host to identify open ports and running services.

\subsection*{Scan Details}
\begin{itemize}
    \item \textbf{Target IP Address:} \texttt{[Target IP]}
\end{itemize}

\subsection*{Open Ports and Services}
The following table details the services discovered to be accessible on the target system.

\begin{table}[h!]
\centering
\caption{Discovered Open Ports}
\label{tab:nmap}
\begin{tabular}{l l l l l}
\toprule
\textbf{Port} & \textbf{State} & \textbf{Service} & \textbf{Product / Version} & \textbf{Finding} \\
\midrule
3306/tcp & Open & mysql & MySQL 5.7.33 & \textcolor{red}{\textbf{Critical Risk}} \\
\bottomrule
\end{tabular}
\end{table}

\paragraph{Analysis:} The scan confirms that a MySQL database server is directly exposed to the public internet. The running version, \textbf{MySQL 5.7.33}, is particularly alarming as the entire 5.7 branch reached its \textbf{End-of-Life (EOL) in October 2023}. EOL software no longer receives security patches from the vendor, meaning any newly discovered vulnerabilities will remain unpatched. This presents a severe and unmitigated risk of data breach.

% ==============================================================================
\section{Consolidated Risk Assessment}
% ==============================================================================

The following table synthesizes findings from the technical scan, control review, and pre-existing risk data into a consolidated list of key risks facing the organization.

\begin{table}[h!]
\centering
\caption{Summary of Identified Risks}
\label{tab:risks}
\begin{tabular}{p{0.25\linewidth} p{0.5\linewidth} l}
\toprule
\textbf{Risk Name} & \textbf{Description} & \textbf{Severity} \\
\midrule
\textbf{Exposed \& Outdated Database} & The MySQL database on port 3306 is publicly accessible and is an End-of-Life version (5.7.33), which no longer receives security updates. & \textcolor{red}{\textbf{Critical}} \\
\addlinespace
\textbf{Lack of MFA for Email} & Email accounts are protected only by passwords, making them highly susceptible to compromise via phishing, credential stuffing, or password spraying attacks. & \textcolor{red}{\textbf{Critical}} \\
\addlinespace
\textbf{Absence of Security Training} & Employees are not trained on security best practices, making them the weakest link and highly vulnerable to social engineering attacks. & \textcolor{orange}{\textbf{High}} \\
\addlinespace
\textbf{No Acceptable Use Policy} & The lack of a formal policy creates ambiguity regarding the secure and acceptable use of company assets, increasing insider threat and compliance risks. & \textcolor{orange}{\textbf{High}} \\
\bottomrule
\end{tabular}
\end{table}

% ==============================================================================
\section{Recommendations}
\label{sec:recommendations}
% ==============================================================================

The following actionable recommendations are provided to mitigate the identified risks. They are prioritized based on severity and potential impact.

\subsection{Risk: Exposed \& Outdated Database (Critical)}
\begin{itemize}
    \item \textbf{Immediate Action:} Implement strict firewall rules to block all public access to TCP port 3306. Access should be restricted to only trusted internal IP addresses.
    \item \textbf{Short-Term Project:} Plan and execute an urgent migration from MySQL 5.7 to a currently supported version (e.g., MySQL 8.x). This is essential to ensure the database receives ongoing security patches.
    \item \textbf{Long-Term Strategy:} For any required remote administrative access, implement a Virtual Private Network (VPN). This provides a secure, encrypted tunnel for administrators and eliminates the need to expose critical services directly to the internet.
\end{itemize}

\subsection{Risk: Lack of MFA for Email (Critical)}
\begin{itemize}
    \item \textbf{Immediate Action:} Enforce mandatory Multi-Factor Authentication (MFA) for all user email accounts. This is the single most effective control to prevent unauthorized account access and Business Email Compromise (BEC).
\end{itemize}

\subsection{Risk: Foundational Policy Gaps (High)}
\begin{itemize}
    \item \textbf{Short-Term Project:}
        \begin{enumerate}
            \item Develop and ratify a formal Acceptable Use Policy (AUP) that clearly defines rules for employee use of company technology and data.
            \item Procure and implement a security awareness training program. Enroll all employees immediately, with a focus on identifying phishing emails.
            \item Establish a formal policy requiring all new hires to complete security training upon onboarding and all staff to complete refresher training at least annually.
        \end{enumerate}
\end{itemize}

\end{document}
```