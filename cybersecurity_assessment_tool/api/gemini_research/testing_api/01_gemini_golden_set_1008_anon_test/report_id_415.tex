```latex
\documentclass[12pt]{article}

% Preamble: Required Packages
\usepackage[margin=1in]{geometry}
\usepackage{pifont} % For checkmarks and crosses
\usepackage{booktabs} % For professional tables
\usepackage{hyperref} % For clickable links and ToC
\usepackage{url} % For URL formatting
\usepackage{seqsplit} % For splitting long strings in texttt
\usepackage{graphicx}
\usepackage[table]{xcolor} % For coloring table rows

% --- Document Setup ---
\hypersetup{
    colorlinks=true,
    linkcolor=blue,
    filecolor=magenta,      
    urlcolor=cyan,
    pdftitle={Cybersecurity Posture Report},
    pdfpagemode=FullScreen,
}

% Define colors for risk levels
\definecolor{criticalred}{HTML}{D10000}
\definecolor{highorange}{HTML}{E57300}
\definecolor{mediumyellow}{HTML}{FFBF00}
\definecolor{lowblue}{HTML}{0073E6}
\definecolor{infogray}{HTML}{808080}

% --- Document Start ---
\begin{document}

% --- Title Page ---
\begin{titlepage}
    \centering
    \vspace*{1cm}
    \Huge \textbf{Cybersecurity Posture Report}
    \vspace{1.5cm}
    \large
    \begin{center}
        \includegraphics[width=0.4\textwidth]{example-image-a} % Placeholder logo
    \end{center}
    \vspace{1.5cm}
    \textbf{Prepared for:}\\
    \vspace{0.5cm}
    \huge \textbf{[Organization Name]}
    \vfill
    \large
    \textbf{Date of Report:}\\
    \today
    \vspace{1cm}
    \textbf{Generated by:}\\
    Expert Cybersecurity Analyst
\end{titlepage}

\tableofcontents
\newpage

% --- Section 1: Executive Summary ---
\section{Executive Summary}
This report provides a comprehensive analysis of the cybersecurity posture for \textbf{[Organization Name]}, based on a combination of organizational data, network scan results, and a review of existing risks. The assessment was conducted on \today.

The analysis identified several key areas of concern that require immediate attention. The most significant findings include:
\begin{itemize}
    \item \textbf{Critical Risk:} The absence of Multi-Factor Authentication (MFA) on systems containing sensitive data. This represents a severe security gap, as it leaves the organization's most valuable assets vulnerable to compromise via stolen credentials.
    \item \textbf{High Risk:} The lack of mandatory, annual security awareness training for all employees. This oversight significantly increases the organization's susceptibility to social engineering and phishing attacks, which are primary vectors for security breaches.
    \item \textbf{Medium Risk:} The external exposure of a Secure Shell (SSH) service on port 22. While necessary for remote administration, public exposure invites automated brute-force attacks and requires stringent security controls to mitigate risk.
\end{itemize}

This report details these findings and provides actionable recommendations to strengthen the organization's defenses, reduce the attack surface, and improve the overall security posture.

% --- Section 2: Organizational Information ---
\section{Organizational Information}
The following details were used as the basis for this assessment. Due to the anonymized nature of the input data, placeholders have been used where specific information was not provided.

\begin{table}[h!]
\centering
\begin{tabular}{@{}ll@{}}
\toprule
\textbf{Attribute} & \textbf{Value} \\ \midrule
Organization Name & \textbf{[Organization Name]} \\
Primary Domain & \texttt{[Domain]} \\
External IP Scanned & \texttt{[Client IP]} \\
Target IP Scanned & \texttt{[Target IP]} \\ \bottomrule
\end{tabular}
\caption{Client Organizational Details}
\end{table}

% --- Section 3: Security Control Review ---
\section{Security Control Review}
A review of the organization's security controls was conducted based on a questionnaire. The responses highlight both strengths and weaknesses in the current security policies and their implementation. Gaps identified with a 'No' response are directly correlated with increased risk.

\begin{table}[h!]
\centering
\begin{tabular}{@{}p{0.8\textwidth}c@{}}
\toprule
\textbf{Control Question} & \textbf{Response} \\ \midrule
Do you require MFA to access email? & \textcolor{green}{\ding{51}} \\
Do you require MFA to log into computers? & \textcolor{green}{\ding{51}} \\
\rowcolor{red!15} Does your organization require MFA to access sensitive data systems? & \textcolor{red}{\ding{55}} \\
Does your organization have an employee acceptable use policy? & \textcolor{green}{\ding{51}} \\
Does your organization do security awareness training for new employees? & \textcolor{green}{\ding{51}} \\
\rowcolor{orange!20} Does your organization do security awareness training for all employees at least once per year? & \textcolor{red}{\ding{55}} \\ \bottomrule
\end{tabular}
\caption{Security Control Questionnaire Results}
\end{table}

% --- Section 4: Technical Scan Results ---
\section{Technical Scan Results}
An external network scan was performed on the target IP address \texttt{[Target IP]}. The scan identified the following open ports and services accessible from the public internet.

\begin{table}[h!]
\centering
\begin{tabular}{@{}lllll@{}}
\toprule
\textbf{Port} & \textbf{State} & \textbf{Service} & \textbf{Version} & \textbf{Notes} \\ \midrule
22/tcp & open & ssh & Not Identified & Exposed to public internet. \\ \bottomrule
\end{tabular}
\caption{Open Port Scan Findings}
\end{table}

\subsection{Analysis of Technical Findings}
The scan identified that port 22 (SSH) is open. Exposing SSH directly to the internet is a significant security risk, as it becomes a target for automated brute-force and credential-stuffing attacks. Without knowledge of the service version, it is not possible to determine if it is vulnerable to known exploits, but the exposure itself constitutes a notable risk.

% --- Section 5: Risk Assessment ---
\section{Risk Assessment}
This section synthesizes the findings from the security control review and the technical scan into a prioritized list of risks. No pre-existing vulnerabilities were documented in the provided data.

\begin{table}[h!]
\centering
\begin{tabular}{@{}p{0.05\textwidth}p{0.3\textwidth}p{0.15\textwidth}p{0.4\textwidth}@{}}
\toprule
\textbf{ID} & \textbf{Finding} & \textbf{Severity} & \textbf{Description} \\ \midrule
\rowcolor{criticalred!25}
R-01 & No MFA on Sensitive Data Systems & \textbf{Critical} & The absence of MFA on critical systems means that a single compromised password could lead to a catastrophic data breach. This is the most severe security gap identified. \\
\addlinespace
\rowcolor{highorange!25}
R-02 & No Annual Security Awareness Training & \textbf{High} & Without regular training, employees are more likely to fall victim to phishing, malware, and other social engineering tactics. This elevates the risk of initial compromise across the entire organization. \\
\addlinespace
\rowcolor{mediumyellow!30}
R-03 & Exposed SSH Service & \textbf{Medium} & The public-facing SSH service is a constant target for automated attacks. If configured with weak credentials or left unpatched, it could serve as a direct entry point for an attacker. \\ \bottomrule
\end{tabular}
\caption{Summary of Identified Risks}
\end{table}

% --- Section 6: Recommendations ---
\section{Recommendations}
The following actions are recommended to mitigate the identified risks and improve the overall security posture of \textbf{[Organization Name]}.

\subsection{For Risk R-01: No MFA on Sensitive Data Systems (Critical)}
\begin{enumerate}
    \item \textbf{Immediate Action:} Prioritize and deploy a robust MFA solution for all systems, applications, and databases that contain or provide access to sensitive organizational or customer data.
    \item \textbf{Policy Enforcement:} Update security policies to mandate the use of MFA for all privileged access and access to sensitive data, without exception.
\end{enumerate}

\subsection{For Risk R-02: No Annual Security Training (High)}
\begin{enumerate}
    \item \textbf{Program Implementation:} Establish a mandatory, annual security awareness training program for all employees, including management. The training should cover phishing, password security, acceptable use, and incident reporting.
    \item \textbf{Continuous Reinforcement:} Supplement annual training with periodic phishing simulations and security newsletters to keep security top-of-mind for all staff.
\end{enumerate}

\subsection{For Risk R-03: Exposed SSH Service (Medium)}
\begin{enumerate}
    \item \textbf{Access Control:} If possible, restrict access to the SSH service to specific, trusted IP addresses using a firewall. For broader remote access, implement a Virtual Private Network (VPN) and require users to connect to the VPN before accessing SSH.
    \item \textbf{Hardening:} Ensure the SSH service is configured according to security best practices:
    \begin{itemize}
        \item Disable password-based authentication and enforce the use of public key cryptography.
        \item Disable root login over SSH.
        \item Keep the SSH server software patched and up-to-date.
    \end{itemize}
\end{enumerate}

\end{document}
```