```latex
\documentclass[12pt]{article}

% Required Packages
\usepackage[margin=1in]{geometry}
\usepackage{pifont} % For checkmarks and crosses
\usepackage{booktabs} % For professional tables
\usepackage{hyperref} % For hyperlinks
\usepackage{url} % For URL formatting
\usepackage{seqsplit} % For splitting long strings
\usepackage{xcolor} % For colors

% Document Metadata
\title{Cybersecurity Posture Assessment Report}
\author{Cybersecurity Analysis Division}
\date{November 22, 2025}

% Hyperref Setup
\hypersetup{
    colorlinks=true,
    linkcolor=blue,
    filecolor=magenta,      
    urlcolor=cyan,
    pdftitle={Cybersecurity Posture Assessment Report},
    pdfpagemode=FullScreen,
}

\begin{document}

\maketitle
\tableofcontents
\newpage

\section{Executive Summary}

This report provides a comprehensive cybersecurity assessment for \textbf{[Organization Name]}, conducted on November 22, 2025. The analysis is based on a network vulnerability scan, a review of organizational security controls, and an evaluation of pre-existing risks.

The assessment reveals several critical and high-risk security gaps that require immediate attention. Key findings include the absence of Multi-Factor Authentication (MFA) on critical systems, the use of outdated and potentially vulnerable web server software, and a lack of security awareness training for new employees. While some foundational controls are in place, such as an acceptable use policy and annual training for existing staff, the identified weaknesses expose the organization to significant threats, including unauthorized access, data breaches, and ransomware attacks.

This document details these findings and provides actionable recommendations to mitigate the identified risks and strengthen the organization's overall security posture.

\section{Organizational Information}

The following information was used as the basis for this assessment. Due to the anonymized nature of the provided data, placeholders have been used where necessary.

\begin{table}[h!]
\centering
\begin{tabular}{@{}ll@{}}
\toprule
\textbf{Attribute} & \textbf{Value} \\ \midrule
Organization Name & \textbf{[Organization Name]} \\
Primary Email Domain & \texttt{[Domain]} \\
External IP Scanned & \texttt{[Client IP]} \\
Assessment Date & November 22, 2025 \\ \bottomrule
\end{tabular}
\caption{Client and Assessment Details}
\end{table}

\section{Security Control Review}

A review of the organization's security controls was conducted via a standardized questionnaire. The responses indicate significant gaps in access control and employee security training. A "No" response (\ding{55}) highlights a deviation from security best practices.

\begin{table}[h!]
\centering
\begin{tabular}{@{}p{0.7\textwidth}c@{}}
\toprule
\textbf{Control Question} & \textbf{Response} \\ \midrule
Do you require MFA to access email? & \textcolor{green}{\ding{51}} \\
Do you require MFA to log into computers? & \textcolor{red}{\ding{55}} \\
Do you require MFA to access sensitive data systems? & \textcolor{red}{\ding{55}} \\
Does your organization have an employee acceptable use policy? & \textcolor{green}{\ding{51}} \\
Does your organization do security awareness training for new employees? & \textcolor{red}{\ding{55}} \\
Does your organization do security awareness training for all employees at least once per year? & \textcolor{green}{\ding{51}} \\ \bottomrule
\end{tabular}
\caption{Security Controls Questionnaire Results}
\end{table}

\section{Technical Scan Results}

An external network scan was performed to identify open ports and exposed services. The scan targeted the primary external IP address provided.

\begin{itemize}
    \item \textbf{Target IP:} \texttt{[Target IP]}
    \item \textbf{Scan Date:} 2025-11-22T10:00:00Z
\end{itemize}

The following table details the services discovered.

\begin{table}[h!]
\centering
\begin{tabular}{@{}ccccc@{}}
\toprule
\textbf{Port} & \textbf{State} & \textbf{Service} & \textbf{Product} & \textbf{Version} \\ \midrule
443/tcp & open & https & nginx & 1.18.0 \\ \bottomrule
\end{tabular}
\caption{Open Ports and Services}
\end{table}

\subsection*{Analysis of Technical Findings}
The scan identified an Nginx web server, version \textbf{1.18.0}, exposed on port 443 (HTTPS). This version was released in April 2020 and is now considered outdated. End-of-life for this branch was reached in May 2022. Running outdated software exposes the organization to numerous publicly known vulnerabilities that have been patched in newer versions.

\section{Synthesized Risk Assessment}

By correlating the security control gaps with the technical findings, we have identified the following risks. The pre-existing risk list was empty, so all risks below are newly identified during this assessment.

\begin{table}[h!]
\centering
\begin{tabular}{@{}p{0.25\textwidth}p{0.5\textwidth}l@{}}
\toprule
\textbf{Risk Name} & \textbf{Overview} & \textbf{Severity} \\ \midrule
\textbf{Insufficient MFA Implementation} & MFA is not enforced for computer logins or access to sensitive data systems. This significantly increases the risk of unauthorized access via compromised credentials. & \textbf{Critical} \\
\addlinespace
\textbf{Outdated Web Server Software} & The public-facing Nginx server is running version 1.18.0, which is end-of-life and has multiple known vulnerabilities. This could lead to a server compromise. & \textbf{High} \\
\addlinespace
\textbf{Inadequate Employee Onboarding Security} & New employees do not receive security awareness training upon hiring. This makes them highly susceptible to phishing and social engineering attacks before they are familiar with company policy. & \textbf{High} \\
\bottomrule
\end{tabular}
\caption{Summary of Identified Risks}
\end{table}

\section{Recommendations}

To address the identified risks and improve the overall security posture, we recommend the following actions be taken with urgency.

\begin{enumerate}
    \item \textbf{Enforce Comprehensive MFA (Critical):}
    \begin{itemize}
        \item \textbf{Action:} Immediately deploy and mandate the use of MFA for all employee computer logins (endpoints) and for all systems containing sensitive or critical data.
        \item \textbf{Justification:} This is the single most effective control to prevent unauthorized access resulting from stolen or weak passwords. It directly mitigates the highest-priority risk identified.
    \end{itemize}
    \vspace{1em}
    \item \textbf{Upgrade Nginx Web Server (High):}
    \begin{itemize}
        \item \textbf{Action:} Plan and execute an upgrade of the Nginx server from version 1.18.0 to the latest stable version. Establish a regular patch management cycle for all internet-facing systems.
        \item \textbf{Justification:} Upgrading the software will patch known vulnerabilities, reducing the server's attack surface and protecting against automated exploits.
    \end{itemize}
    \vspace{1em}
    \item \textbf{Implement Onboarding Security Training (High):}
    \begin{itemize}
        \item \textbf{Action:} Develop and integrate a mandatory security awareness training module into the new employee onboarding process. This training should occur within the first week of employment.
        \item \textbf{Justification:} New hires are a common target for attackers. Immediate training reduces the window of vulnerability and establishes a security-conscious mindset from day one.
    \end{itemize}
\end{enumerate}

\end{document}
```