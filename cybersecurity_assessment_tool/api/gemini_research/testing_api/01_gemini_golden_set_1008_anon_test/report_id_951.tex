```latex
\documentclass[12pt]{article}

% --- PACKAGES ---
\usepackage[margin=1in]{geometry} % Set page margins
\usepackage{pifont}               % For symbols like checkmarks and crosses
\usepackage{booktabs}             % For professional-looking tables
\usepackage{xcolor}               % For custom colors
\usepackage{graphicx}             % For including images
\usepackage{seqsplit}             % For splitting long strings in \texttt
\usepackage{hyperref}             % For creating hyperlinks
\usepackage{url}                  % For formatting URLs

% --- DOCUMENT CONFIGURATION ---
\hypersetup{
    colorlinks=true,
    linkcolor=blue,
    filecolor=magenta,
    urlcolor=cyan,
    pdftitle={Cybersecurity Posture Assessment Report},
    pdfauthor={Cybersecurity Analyst},
}

% --- CUSTOM COMMANDS ---
\newcommand{\yes}{\ding{51}} % Green checkmark
\newcommand{\no}{\textcolor{red}{\ding{55}}}  % Red cross

% --- DOCUMENT START ---
\begin{document}

% --- TITLE PAGE ---
\begin{titlepage}
    \centering
    \vspace*{1cm}
    \Huge\textbf{Cybersecurity Posture Assessment Report}
    \vspace{1.5cm}
    \Large
    \textbf{Prepared for:} \textbf{[Organization Name]} \\
    \vspace{3cm}
    \normalsize
    \textbf{Author:} Cybersecurity Analyst \\
    \textbf{Date:} \today
    \vfill
    \textit{This report is confidential and intended solely for the use of \textbf{[Organization Name]}.}
\end{titlepage}

\tableofcontents
\newpage

% --- EXECUTIVE SUMMARY ---
\section{Executive Summary}

This report details the findings of a cybersecurity assessment for \textbf{[Organization Name]}, synthesizing data from network scans, a security controls questionnaire, and a review of existing risks. The assessment identified a \textbf{critical risk}: a publicly exposed MySQL database on host \texttt{[Client IP]}. The detected version, MySQL 5.7.33, has reached its End-of-Life (EOL) and no longer receives security updates, significantly increasing the risk of a data breach.

Furthermore, significant gaps in administrative controls were identified that exacerbate this technical vulnerability. These include the lack of Multi-Factor Authentication (MFA) for computer and sensitive system access, and the absence of a formal Acceptable Use Policy (AUP). While the organization has implemented security awareness training, the lack of foundational access controls and policies leaves it vulnerable to attack.

Immediate remediation is required to restrict access to the exposed database. Strategic improvements to identity management and governance controls are also strongly recommended to build a more resilient and defensible security posture.

% --- ORGANIZATIONAL INFORMATION ---
\section{Organizational Information}

The following details were used as the basis for this assessment. As per the provided data, placeholder values are used where specific information was not available.

\begin{center}
\begin{tabular}{@{}ll}
\toprule
\textbf{Attribute} & \textbf{Value} \\
\midrule
Organization Name & \textbf{[Organization Name]} \\
Primary Domain & \texttt{[Domain]} \\
External IP Assessed & \texttt{[Client IP]} \\
\bottomrule
\end{tabular}
\end{center}

% --- SECURITY CONTROL REVIEW ---
\section{Security Control Review}

The following table summarizes the organization's responses to a security controls questionnaire. Items marked with a \no\ represent significant gaps in the current security posture and are primary areas for improvement.

\begin{center}
\begin{tabular}{p{0.7\textwidth}c}
\toprule
\textbf{Control Question} & \textbf{Response} \\
\midrule
Do you require MFA to access email? & \yes \\
Do you require MFA to log into computers? & \no \\
Do you require MFA to access sensitive data systems? & \no \\
Does your organization have an employee acceptable use policy? & \no \\
Does your organization do security awareness training for new employees? & \yes \\
Does your organization do security awareness training for all employees at least once per year? & \yes \\
\bottomrule
\end{tabular}
\end{center}

% --- TECHNICAL SCAN RESULTS ---
\section{Technical Scan Results}

An external network scan was performed on the target IP address to identify open ports and exposed services.

\subsection{Host Status}
\textbf{Target IP:} \texttt{[Target IP]} \\
\textbf{Status:} Up

\subsection{Open Ports Discovered}
The scan identified the following open port:

\begin{center}
\begin{tabular}{lllll}
\toprule
\textbf{Port} & \textbf{State} & \textbf{Service} & \textbf{Product} & \textbf{Version} \\
\midrule
3306/tcp & open & mysql & MySQL & 5.7.33 \\
\bottomrule
\end{tabular}
\end{center}

\subsection{Analysis}
The scan revealed that port 3306 is open, exposing a MySQL database service directly to the internet. The detected version, \textbf{MySQL 5.7.33}, reached its official End-of-Life (EOL) in October 2023. EOL software no longer receives security patches from the vendor, making it an easy target for attackers exploiting known vulnerabilities. This finding confirms the pre-existing risk "Database Exposure" and elevates its urgency due to the outdated software version. The lack of MFA for sensitive systems, as noted in Section 3, makes this exposed database particularly vulnerable to credential-based attacks.

% --- RISK ASSESSMENT ---
\section{Risk Assessment}

The following table synthesizes findings from the security questionnaire, technical scans, and pre-existing risk data. Risks are categorized by severity to guide prioritization.

\begin{center}
\begin{tabular}{p{0.25\textwidth}p{0.1\textwidth}p{0.6\textwidth}}
\toprule
\textbf{Risk Title} & \textbf{Severity} & \textbf{Description \& Business Impact} \\
\midrule
\textbf{Exposed End-of-Life Database} & \textbf{Critical} & A MySQL 5.7.33 database (CVSS 7.5) is publicly accessible. This version is past its EOL and is unpatched. This could lead to a complete database compromise, data breach, or ransomware attack. Correlates with the technical scan and the "Database Exposure" risk. \\
\addlinespace
\textbf{No MFA for Sensitive Systems} & \textbf{Critical} & The lack of a second authentication factor for sensitive systems means a single compromised password could grant an attacker access to critical company data, including the exposed database. This is a severe administrative control gap. \\
\addlinespace
\textbf{No MFA for Computer Logins} & High & Without MFA on workstations, an attacker with valid credentials can gain a foothold on the internal network, facilitating lateral movement and deeper compromise. \\
\addlinespace
\textbf{Missing Acceptable Use Policy} & Medium & The absence of a formal AUP creates ambiguity for employees regarding security responsibilities and acceptable behavior, increasing the risk of insider threat and non-compliance. \\
\bottomrule
\end{tabular}
\end{center}

% --- RECOMMENDATIONS ---
\section{Recommendations}

The following prioritized recommendations are provided to mitigate the identified risks and strengthen the overall security posture.

\subsection{Immediate Actions (Urgency: Critical)}
\begin{itemize}
    \item \textbf{Restrict Database Access:} Immediately implement firewall rules to block all public access to port 3306 on \texttt{[Target IP]}. Access should be restricted to trusted internal IP addresses only. This directly addresses the "Database Exposure" risk as per the "easy fix" recommendation.
\end{itemize}

\subsection{High-Priority Actions (Urgency: High)}
\begin{itemize}
    \item \textbf{Implement MFA:} Deploy Multi-Factor Authentication for all employees for both computer logins and access to any systems containing sensitive data. This is the single most effective control to prevent unauthorized access from compromised credentials.
    \item \textbf{Plan Database Upgrade:} Begin planning the migration from MySQL 5.7.33 to a fully supported version (e.g., MySQL 8.x). This is essential for long-term security and stability.
    \item \textbf{Develop Acceptable Use Policy (AUP):} Create and enforce an AUP that clearly defines rules for computer, network, and data usage for all employees.
\end{itemize}

\subsection{Strategic Recommendations}
\begin{itemize}
    \item \textbf{Implement a VPN:} For long-term secure remote access to internal resources like the database, deploy a corporate Virtual Private Network (VPN). This aligns with the "long-term fix" for the "Database Exposure" risk.
    \item \textbf{Establish Vulnerability Management:} Implement a program for regular, automated vulnerability scanning of all external and internal assets to proactively identify and remediate security weaknesses before they can be exploited.
\end{itemize}

\end{document}
```