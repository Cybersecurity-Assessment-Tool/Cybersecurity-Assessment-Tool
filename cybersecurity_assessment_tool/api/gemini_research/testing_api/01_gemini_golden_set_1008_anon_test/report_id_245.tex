```latex
\documentclass[12pt]{article}

% === PACKAGES ===
\usepackage[margin=1in]{geometry}
\usepackage{pifont} % For checkmarks and crosses (\ding{51} and \ding{55})
\usepackage{booktabs} % For professional tables
\usepackage{hyperref} % For clickable links and table of contents
\usepackage{url}
\usepackage{seqsplit} % For breaking long strings in \texttt
\usepackage{graphicx}
\usepackage{fancyhdr}
\usepackage{xcolor}
\usepackage{array}

% === DOCUMENT AND HYPERLINK SETUP ===
\definecolor{darkblue}{rgb}{0.0, 0.0, 0.55}
\hypersetup{
    colorlinks=true,
    linkcolor=darkblue,
    filecolor=darkblue,
    urlcolor=darkblue,
    citecolor=darkblue,
}

% === HEADER AND FOOTER ===
\pagestyle{fancy}
\fancyhf{}
\lhead{Cybersecurity Assessment Report}
\rhead{\textbf{[Organization Name]}}
\cfoot{Page \thepage}
\renewcommand{\headrulewidth}{0.4pt}
\renewcommand{\footrulewidth}{0.4pt}

% === COMMANDS ===
\newcommand{\yes}{\ding{51}}
\newcommand{\no}{\ding{55}}
\newcommand{\riskcritical}[1]{\textcolor{red}{\textbf{#1}}}
\newcommand{\riskhigh}[1]{\textcolor{orange}{\textbf{#1}}}
\newcommand{\riskmedium}[1]{\textcolor{yellow!80!black}{\textbf{#1}}}
\newcommand{\riskinformational}[1]{\textcolor{blue}{\textbf{#1}}}

% === DOCUMENT START ===
\begin{document}

\title{
    \vspace{2cm}
    \textbf{Cybersecurity Posture Assessment Report} \\
    \large For: \textbf{[Organization Name]}
    \vspace{1cm}
}
\author{Cybersecurity Analysis Division}
\date{\today}
\maketitle
\thispagestyle{empty}

\newpage

\tableofcontents

\newpage

% ===================================================================
\section{Executive Overview}
% ===================================================================

This report details the findings of a cybersecurity assessment conducted for \textbf{[Organization Name]}. The analysis is based on a combination of network scanning, a review of existing risk documentation, and an evaluation of self-reported security controls.

The assessment reveals a mixed security posture. The organization has implemented several strong foundational controls, most notably the consistent enforcement of Multi-Factor Authentication (MFA) across email, computer logins, and sensitive data systems. This significantly reduces the risk of account compromise through stolen credentials.

However, two critical areas of concern were identified that substantially elevate the organization's overall risk profile:

\begin{enumerate}
    \item \textbf{Critical External Exposure:} A network scan identified a publicly exposed Remote Desktop Protocol (RDP) service on port 3389. This is a high-value target for attackers and a common entry point for ransomware and other malicious intrusions. This technical finding directly validates a pre-existing high-severity risk.

    \item \textbf{Systemic Lack of Security Training:} The organization does not provide security awareness training to new or existing employees. This represents a significant gap in the human firewall, leaving the organization highly vulnerable to phishing, social engineering, and other attacks that prey on user behavior.
\end{enumerate}

Due to the severity of the exposed RDP service, the overall risk posture for \textbf{[Organization Name]} is assessed as \riskcritical{CRITICAL}. Immediate remediation of the identified vulnerabilities is strongly recommended to reduce the likelihood of a significant security incident.

% ===================================================================
\section{Organizational Information}
% ===================================================================

The following information was used as the basis for this assessment. As the provided data was anonymized, placeholders have been used.

\begin{itemize}
    \item \textbf{Organization Name:} \textbf{[Organization Name]}
    \item \textbf{Primary Domain:} \texttt{[Domain]}
    \item \textbf{External IP Address Assessed:} \texttt{[Client IP]}
\end{itemize}


% ===================================================================
\section{Security Control Review}
% ===================================================================

The following table summarizes the organization's self-reported security controls based on the provided questionnaire data. While several controls are in place, the gaps identified are significant.

\begin{table}[h!]
\centering
\caption{Security Controls Questionnaire Results}
\begin{tabular}{p{0.7\textwidth} >{\centering\arraybackslash}p{0.1\textwidth} >{\centering\arraybackslash}p{0.1\textwidth}}
\toprule
\textbf{Control Question} & \textbf{Response} & \textbf{Status} \\
\midrule
Do you require MFA to access email? & Yes & \yes \\
Do you require MFA to log into computers? & Yes & \yes \\
Do you require MFA to access sensitive data systems? & Yes & \yes \\
Does your organization have an employee acceptable use policy? & Yes & \yes \\
\midrule
\rowcolor{red!15}
Does your organization do security awareness training for new employees? & No & \no \\
\rowcolor{red!15}
Does your organization do security awareness training for all employees at least once per year? & No & \no \\
\bottomrule
\end{tabular}
\end{table}

\subsection*{Analysis}
The "No" responses indicate a complete absence of a formal security awareness training program. This is a critical deficiency. Without training, employees are significantly more likely to fall victim to phishing attacks, mishandle sensitive data, or unintentionally violate security policies, thereby undermining other technical controls.

% ===================================================================
\section{Technical Scan Results}
% ===================================================================

An external network scan was performed against the target IP address to identify open ports and exposed services.

\begin{itemize}
    \item \textbf{Target IP Address:} \texttt{[Target IP]}
\end{itemize}

\begin{table}[h!]
\centering
\caption{Open Ports Identified on \texttt{[Target IP]}}
\begin{tabular}{llll}
\toprule
\textbf{Port} & \textbf{State} & \textbf{Service Name} & \textbf{Analyst Notes} \\
\midrule
\rowcolor{red!15}
3389/tcp & open & \texttt{ms-wbt-server} & \riskcritical{Critical Risk}. This is the Remote Desktop Protocol (RDP). \\
\bottomrule
\end{tabular}
\end{table}

\subsection*{Analysis}
The scan confirms that the Remote Desktop Protocol (RDP) service is directly exposed to the public internet. RDP is a primary target for attackers who use brute-force password attacks, credential stuffing, and exploits for known vulnerabilities (e.g., BlueKeep) to gain unauthorized access to internal networks. Exposing RDP is extremely dangerous and is a leading cause of ransomware infections.

% ===================================================================
\section{Correlated Risk Assessment}
% ===================================================================

This section synthesizes the findings from the security questionnaire, technical scan, and pre-existing risk data into a consolidated view of the top risks facing the organization.

\begin{table}[h!]
\centering
\caption{Summary of Key Risks}
\begin{tabular}{p{0.2\textwidth} p{0.6\textwidth} p{0.15\textwidth}}
\toprule
\textbf{Risk Name} & \textbf{Description} & \textbf{Severity} \\
\midrule
\textbf{RDP Exposure} & The technical scan confirmed that RDP (port 3389) is publicly accessible on \texttt{[Target IP]}. This aligns with the pre-existing risk data and presents an immediate and severe threat of unauthorized access and ransomware. & \riskcritical{Critical (9.0)} \\
\addlinespace
\textbf{Lack of Security Awareness Training} & Questionnaire results show no security training for new or current employees. This weakness in the "human firewall" magnifies the threat of all other risks, especially phishing and social engineering attacks that could lead to credential theft. & \riskhigh{High} \\
\bottomrule
\end{tabular}
\end{table}

% ===================================================================
\section{Recommendations}
% ===================================================================

The following actionable recommendations are provided to address the identified risks. They are prioritized based on severity and potential impact.

\subsection{Remediation for: RDP Exposure (\riskcritical{Critical})}
This vulnerability requires immediate attention to prevent a potential network compromise.

\begin{itemize}
    \item \textbf{Immediate Action (Containment):}
    Implement a firewall rule to \textbf{block all inbound traffic to TCP port 3389} on the external interface of the network firewall protecting \texttt{[Target IP]}. This will immediately remove the public exposure.

    \item \textbf{Long-Term Solution (Remediation):}
    Deploy a secure remote access solution, such as a \textbf{Virtual Private Network (VPN)} with MFA. All remote administration and user access should occur exclusively through the VPN tunnel. Direct RDP access from the internet should be permanently disabled.
\end{itemize}

\subsection{Remediation for: Lack of Security Awareness Training (\riskhigh{High})}
Establishing a security-conscious culture is essential for long-term defense.

\begin{itemize}
    \item \textbf{Short-Term Action (Foundation):}
    Procure and enroll all employees in a reputable security awareness training program. The initial training should be mandatory and cover, at a minimum: phishing and spear-phishing identification, password security best practices, and the organization's acceptable use policy.

    \item \textbf{Long-Term Solution (Program Development):}
    Establish a formal, ongoing security awareness program. This program should include:
    \begin{itemize}
        \item Mandatory training for all new hires as part of their onboarding process.
        \item Annual refresher training for all staff.
        \item Regular simulated phishing campaigns to test and reinforce learning.
    \end{itemize}
\end{itemize}

\end{document}
```