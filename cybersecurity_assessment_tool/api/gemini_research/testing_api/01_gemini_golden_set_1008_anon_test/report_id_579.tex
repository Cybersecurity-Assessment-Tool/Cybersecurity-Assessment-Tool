```latex
\documentclass[12pt]{article}

% === PACKAGES ===
\usepackage[margin=1in]{geometry}
\usepackage{pifont} % For checkmarks and crosses (\ding)
\usepackage{booktabs} % For professional tables
\usepackage{hyperref} % For clickable links
\usepackage{url} % For formatting URLs
\usepackage{seqsplit} % To split long strings without breaking
\usepackage{xcolor} % For custom colors
\usepackage{graphicx} % To include images (e.g., logos)
\usepackage{fancyhdr} % For custom headers and footers
\usepackage{lastpage} % To get the total page count

% === DOCUMENT SETUP ===
% Hyperlink setup
\hypersetup{
    colorlinks=true,
    linkcolor=black,
    urlcolor=blue,
    pdftitle={Cybersecurity Posture Assessment Report},
    pdfauthor={Cybersecurity Analysis Division}
}

% Header and Footer setup
\pagestyle{fancy}
\fancyhf{} % Clear all header and footer fields
\fancyhead[L]{Cybersecurity Assessment Report}
\fancyhead[R]{\textbf{[Organization Name]}}
\fancyfoot[C]{Page \thepage\ of \pageref{LastPage}}
\renewcommand{\headrulewidth}{0.4pt}
\renewcommand{\footrulewidth}{0.4pt}

% Define a color for high-risk items
\definecolor{riskred}{RGB}{192,0,0}

% === DOCUMENT START ===
\begin{document}

% === TITLE PAGE ===
\begin{titlepage}
    \centering
    \vspace*{2cm}
    
    {\Huge \textbf{Cybersecurity Posture Assessment Report}\par}
    \vspace{1.5cm}
    
    {\Large Prepared For:\par}
    \vspace{0.5cm}
    {\Huge \textbf{[Organization Name]}\par}
    
    \vspace{3cm}
    
    {\large \today\par}
    
    \vfill
    
    {\large \textbf{Cybersecurity Analysis Division}\par}
    \vspace{0.5cm}
    {\small This report contains sensitive information and should be handled with care.\par}
    
\end{titlepage}

\newpage
\tableofcontents
\newpage

% === EXECUTIVE SUMMARY ===
\section{Executive Summary}
This report provides a comprehensive assessment of the cybersecurity posture for \textbf{[Organization Name]}, based on an analysis of network scan data, organizational security controls, and known risks. The assessment identified several critical and high-risk vulnerabilities that require immediate attention.

The most critical finding is a publicly exposed MySQL database service running on an End-of-Life (EOL) software version (\texttt{5.7.33}). This exposes the organization to significant data breach risks, as EOL software no longer receives security patches.

This technical vulnerability is severely compounded by significant gaps in administrative security controls. The lack of Multi-Factor Authentication (MFA) for email and sensitive data systems, combined with the absence of a formal security awareness training program, creates a high-risk environment. An attacker could leverage a single compromised credential to gain direct access to the exposed database.

Immediate remediation should focus on restricting network access to the database. Subsequent actions must address the MFA and security training deficiencies to build a more resilient security foundation.

% === ORGANIZATIONAL INFORMATION ===
\section{Organizational Information}
This section outlines the high-level information used as the basis for this assessment. Due to the anonymized nature of the provided data, placeholders are used where necessary.

\begin{itemize}
    \item \textbf{Organization Name:} \textbf{[Organization Name]}
    \item \textbf{Primary Domain:} \texttt{[Domain]}
    \item \textbf{Scanned External IP:} \texttt{[Target IP]}
    \item \textbf{Assessment Date:} 2024-05-21
\end{itemize}

% === SECURITY CONTROL REVIEW ===
\section{Security Control Review}
A review of the organization's administrative security controls was conducted via a questionnaire. The responses highlight critical gaps in identity and access management and employee security awareness. Answers marked with a red 'X' indicate a deviation from security best practices and represent a significant risk.

\begin{table}[h!]
\centering
\caption{Security Controls Questionnaire Results}
\begin{tabular}{@{}lc@{}}
\toprule
\textbf{Control Question} & \textbf{Response} \\
\midrule
Do you require MFA to access email? & \textcolor{riskred}{\ding{55}} \\
Do you require MFA to log into computers? & \ding{51} \\
Do you require MFA to access sensitive data systems? & \textcolor{riskred}{\ding{55}} \\
Does your organization have an employee acceptable use policy? & \ding{51} \\
Does your organization do security awareness training for new employees? & \textcolor{riskred}{\ding{55}} \\
Does your organization do security awareness training for all employees annually? & \textcolor{riskred}{\ding{55}} \\
\bottomrule
\end{tabular}
\end{table}

\subsection*{Analysis}
The absence of MFA for email and sensitive data systems is a critical weakness. Email is a primary target for phishing attacks aimed at credential theft. Without MFA, a single compromised password could grant an attacker access to sensitive communications and a foothold into the network. The lack of security awareness training for employees further increases the likelihood of such a compromise succeeding.

% === TECHNICAL SCAN RESULTS ===
\section{Technical Scan Results}
An external network scan was performed on the target IP address \texttt{[Target IP]} to identify open ports and exposed services.

\subsection*{Open Ports Discovered}
The scan revealed one open port, which hosts a critical database service.

\begin{table}[h!]
\centering
\caption{Nmap Scan Results for \texttt{[Target IP]}}
\begin{tabular}{@{}lllll@{}}
\toprule
\textbf{Port} & \textbf{State} & \textbf{Service} & \textbf{Product} & \textbf{Version} \\
\midrule
3306/tcp & open & mysql & MySQL & 5.7.33 \\
\bottomrule
\end{tabular}
\end{table}

\subsection*{Analysis}
The discovery of TCP port \texttt{3306} open to the public internet is a critical security risk. This port is the default for the MySQL database service. Exposing a database directly to the internet makes it a prime target for brute-force attacks, credential stuffing, and exploitation of known vulnerabilities.

Furthermore, the identified version, \textbf{MySQL 5.7.33}, is part of a product line that reached its official \textbf{End of Life (EOL) in October 2023}. EOL software no longer receives security updates from the vendor, meaning any newly discovered vulnerabilities will remain unpatched. This elevates the risk of compromise significantly.

% === SYNTHESIZED RISK ASSESSMENT ===
\section{Synthesized Risk Assessment}
The following table correlates the findings from the security control review, technical scan, and pre-existing risk data into a prioritized list of security risks.

\begin{table}[h!]
\centering
\caption{Summary of Identified Risks}
\begin{tabular}{@{}p{0.05\textwidth} p{0.3\textwidth} p{0.45\textwidth} p{0.1\textwidth}@{}}
\toprule
\textbf{ID} & \textbf{Risk Title} & \textbf{Description} & \textbf{Severity} \\
\midrule
\textbf{R-01} & \textbf{Exposed End-of-Life Database} & The MySQL database on port 3306 is publicly accessible and runs on an unsupported, EOL version (5.7.33). This combines the pre-existing "Database Exposure" risk with the new finding of EOL software. & \textbf{Critical} \\
\addlinespace
\textbf{R-02} & \textbf{Insufficient MFA Implementation} & MFA is not enforced on email or sensitive data systems. A compromised password could lead to a full breach of these systems, including the exposed database. & \textbf{High} \\
\addlinespace
\textbf{R-03} & \textbf{Lack of Security Awareness Program} & Employees are not trained to recognize or report security threats like phishing. This increases the likelihood of credential compromise, which is the primary attack vector against systems without MFA. & \textbf{High} \\
\bottomrule
\end{tabular}
\end{table}

% === RECOMMENDATIONS ===
\section{Recommendations}
The following actions are recommended to mitigate the identified risks. They are prioritized to address the most critical threats first.

\subsection{Immediate Actions (Priority 1)}
\begin{itemize}
    \item \textbf{Restrict Database Access:} Immediately implement a firewall rule to block all public inbound traffic to TCP port \texttt{3306} on \texttt{[Target IP]}. Access should only be permitted from trusted internal IP addresses or through a secure VPN connection.
\end{itemize}

\subsection{Short-Term Actions (Priority 2)}
\begin{itemize}
    \item \textbf{Upgrade Database Software:} Plan and execute the migration of the MySQL 5.7 database to a currently supported version (e.g., MySQL 8.x or a managed cloud equivalent). This is essential to ensure the system receives future security patches.
    \item \textbf{Enforce Multi-Factor Authentication:} Procure and deploy an MFA solution. Prioritize its enforcement on all email accounts and any systems identified as containing sensitive data.
\end{itemize}

\subsection{Long-Term Strategic Actions (Priority 3)}
\begin{itemize}
    \item \textbf{Implement Security Awareness Training:} Develop and launch a mandatory security awareness training program. This should include initial training for new hires and at least one annual refresher course for all employees, focusing on topics like phishing, password hygiene, and acceptable use.
    \item \textbf{Conduct a Comprehensive Vulnerability Assessment:} Perform a full internal and external vulnerability assessment to identify any other potential weaknesses in the IT environment.
\end{itemize}

\end{document}
```