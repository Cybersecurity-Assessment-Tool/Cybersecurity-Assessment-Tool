```latex
\documentclass[12pt]{article}

% Preamble: Required Packages
\usepackage[T1]{fontenc}
\usepackage[margin=1in]{geometry}
\usepackage{pifont} % For \ding
\usepackage{booktabs} % For professional tables (\toprule, \midrule, \bottomrule)
\usepackage{graphicx}
\usepackage{xcolor}
\usepackage{seqsplit} % For splitting long strings in \texttt
\usepackage[hidelinks]{hyperref} % For clickable links without boxes
\usepackage{url}

% Define custom colors
\definecolor{darkblue}{rgb}{0.0, 0.0, 0.55}
\definecolor{darkred}{rgb}{0.55, 0.0, 0.0}

% Document Information
\title{Cybersecurity Posture Assessment Report}
\author{Cybersecurity Analysis Division}
\date{\today}

\begin{document}

\maketitle
\thispagestyle{empty}
\newpage

\tableofcontents
\newpage

% --- 1. Executive Summary ---
\section*{1. Executive Summary}

This report provides a comprehensive cybersecurity assessment for \textbf{[Organization Name]}, conducted on \today. The analysis synthesizes data from an external network penetration test, a review of internal security controls via a questionnaire, and a list of pre-existing risks.

The external network scan revealed a strong perimeter security posture, with no open ports detected on the target system \seqsplit{\texttt{[Target IP]}}. This indicates effective firewall configuration and a minimized external attack surface, which is a significant strength.

However, the internal security control review identified several critical gaps in organizational policy and user access controls. The most pressing issues include the absence of Multi-Factor Authentication (MFA) for computer logins, the lack of a formal Acceptable Use Policy (AUP), and no mandatory security awareness training for new employees. These deficiencies expose the organization to significant risks, including unauthorized access via compromised credentials, insider threats, and social engineering attacks.

This report details these findings and provides actionable recommendations to mitigate the identified risks and strengthen the overall security posture of \textbf{[Organization Name]}.

% --- 2. Organizational Information ---
\section*{2. Organizational Information}

This section contains the high-level details of the organization under review. The information provided is based on the data supplied for this assessment.

\begin{itemize}
    \item \textbf{Organization Name:} \textbf{[Organization Name]}
    \item \textbf{Primary Domain:} \texttt{[Domain]}
    \item \textbf{Scanned External IP:} \seqsplit{\texttt{[Client IP]}}
\end{itemize}

% --- 3. Security Control Review ---
\section*{3. Security Control Review}

The following table summarizes the organization's responses to a security controls questionnaire. The status indicates whether the control is in place ("Yes") or not ("No"). "No" answers represent significant gaps in the security framework.

\begin{table}[h!]
\centering
\caption{Security Controls Questionnaire Results}
\begin{tabular}{p{0.7\linewidth} c}
\toprule
\textbf{Control Question} & \textbf{Status} \\
\midrule
Do you require MFA to access email? & \ding{51} \\
Do you require MFA to log into computers? & \textcolor{darkred}{\ding{55}} \\
Do you require MFA to access sensitive data systems? & \ding{51} \\
Does your organization have an employee acceptable use policy? & \textcolor{darkred}{\ding{55}} \\
Does your organization do security awareness training for new employees? & \textcolor{darkred}{\ding{55}} \\
Does your organization do security awareness training for all employees at least once per year? & \ding{51} \\
\bottomrule
\end{tabular}
\end{table}

\subsection*{Analysis of Control Gaps}
While the organization has implemented important controls such as MFA for email and sensitive systems, three critical gaps were identified:
\begin{itemize}
    \item \textbf{No MFA for Computer Logins:} This is a high-risk vulnerability. If an employee's password is stolen (e.g., via phishing), an attacker could gain direct access to their computer and the internal network.
    \item \textbf{No Acceptable Use Policy (AUP):} The absence of an AUP means there are no formally documented rules for employees regarding the use of company assets, data handling, and security responsibilities. This increases the risk of both accidental and malicious insider threats.
    \item \textbf{No Security Training for New Hires:} New employees are often prime targets for social engineering. Without immediate training upon being hired, they represent a significant weak point in the organization's human firewall.
\end{itemize}

% --- 4. Technical Scan Results ---
\section*{4. Technical Scan Results}

An external network scan was performed using Nmap against the designated target IP address.

\begin{itemize}
    \item \textbf{Target IP Address:} \seqsplit{\texttt{[Target IP]}}
    \item \textbf{Scan Status:} Completed
\end{itemize}

\subsection*{Findings}
The scan concluded that the host was online, but reported that all scanned ports were in a \textbf{closed} state. No open or filtered ports were discovered.

\textbf{Conclusion:} This is a positive security finding. It indicates that the external firewall is properly configured to deny unsolicited inbound traffic, effectively minimizing the network's attack surface from the public internet. No vulnerabilities related to exposed services were identified.

% --- 5. Overall Risk Assessment ---
\section*{5. Overall Risk Assessment}

This section correlates findings from the security control review and the technical scan. While no pre-existing risks were provided and no technical vulnerabilities were found on the perimeter, the policy and procedural gaps identified in the questionnaire present a high level of risk to the organization.

\begin{table}[h!]
\centering
\caption{Summary of Identified Risks}
\begin{tabular}{p{0.2\linewidth} p{0.6\linewidth} c}
\toprule
\textbf{Risk Name} & \textbf{Overview} & \textbf{Severity} \\
\midrule
Lack of MFA on Workstations & Compromised credentials could grant an attacker direct access to an employee's computer, the internal network, and sensitive data. & \textbf{High} \\
\addlinespace
Missing Acceptable Use Policy & Without a formal policy, the risk of data mishandling, unauthorized software installation, and insider threat is significantly increased. & \textbf{High} \\
\addlinespace
No Onboarding Security Training & New employees are highly susceptible to social engineering. Lack of immediate training makes them an easy target for phishing and other attacks. & \textbf{High} \\
\bottomrule
\end{tabular}
\end{table}

% --- 6. Recommendations ---
\section*{6. Recommendations}

Based on the analysis, we provide the following actionable recommendations to mitigate the identified risks and improve the overall security posture of \textbf{[Organization Name]}.

\begin{enumerate}
    \item \textbf{Implement MFA for All Computer Logins:}
    \begin{itemize}
        \item \textbf{Action:} Deploy a robust MFA solution (e.g., authenticator app, hardware token, or biometrics) for all user endpoints, including laptops and desktops.
        \item \textbf{Priority:} Critical. Begin with privileged accounts (administrators) and then roll out to all users.
        \item \textbf{Impact:} Drastically reduces the risk of unauthorized access from compromised credentials.
    \end{itemize}
    \vspace{1em}
    \item \textbf{Develop and Enforce an Acceptable Use Policy (AUP):}
    \begin{itemize}
        \item \textbf{Action:} Create a comprehensive AUP that clearly defines the rules for using company technology, accessing data, and employee security responsibilities. This policy should be reviewed by legal counsel.
        \item \textbf{Priority:} High. All current employees must read and formally acknowledge the policy. It should be a mandatory part of the new hire onboarding process.
        \item \textbf{Impact:} Establishes a clear security baseline for all employees and provides a framework for enforcement.
    \end{itemize}
    \vspace{1em}
    \item \textbf{Integrate Security Training into New Hire Onboarding:}
    \begin{itemize}
        \item \textbf{Action:} Develop a mandatory security awareness training module for all new employees to be completed within their first week of employment. This should cover topics like phishing, password security, and the new AUP.
        \item \textbf{Priority:} High. This closes a critical window of vulnerability when new hires are most susceptible.
        \item \textbf{Impact:} Reduces the likelihood of new employees falling victim to social engineering attacks and strengthens the organization's human firewall from day one.
    \end{itemize}
\end{enumerate}

\end{document}
```