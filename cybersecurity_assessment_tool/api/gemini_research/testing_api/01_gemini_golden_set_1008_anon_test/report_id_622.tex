```latex
\documentclass[12pt]{article}

% 1. DOCUMENT SETUP & PACKAGES
\usepackage[margin=1in]{geometry}
\usepackage{pifont} % For checkmarks and crosses
\usepackage{booktabs} % For professional tables
\usepackage{hyperref} % For clickable links and references
\usepackage{url} % For URL formatting
\usepackage{seqsplit} % For splitting long strings in tt font
\usepackage{graphicx} % For potential logos
\usepackage[utf8]{inputenc}

% Hyperref Setup
\hypersetup{
    colorlinks=true,
    linkcolor=black,
    filecolor=magenta,      
    urlcolor=blue,
    pdftitle={Cybersecurity Assessment Report},
    pdfpagemode=FullScreen,
}

% 2. DOCUMENT METADATA
\title{Cybersecurity Assessment Report \\ \large For: \textbf{[Organization Name]}}
\author{Cybersecurity Analysis Division}
\date{\today}

% 3. BEGIN DOCUMENT
\begin{document}

\maketitle
\thispagestyle{empty}
\newpage

\tableofcontents
\newpage

% 4. EXECUTIVE SUMMARY
\section*{Executive Summary}

This report details the findings of a cybersecurity assessment conducted for \textbf{[Organization Name]}. The evaluation was based on a combination of a security controls questionnaire, a network vulnerability scan of the external asset at \texttt{[Client IP]}, and a review of pre-existing risks.

The assessment reveals a mixed security posture. The organization demonstrates strong foundational controls in identity and access management, with consistent enforcement of Multi-Factor Authentication (MFA) across email, workstations, and sensitive data systems. An acceptable use policy is also in place.

However, a critical gap was identified in the area of security awareness and training. The organization currently does not provide security training for new employees during onboarding, nor does it conduct annual security awareness training for all staff. This deficiency exposes the organization to a significantly higher risk of human-error-related security incidents, such as phishing attacks, malware infections, and data breaches.

The external network scan of the target IP address \texttt{[Target IP]} did not detect any open ports or services, which is a positive security finding. No pre-existing vulnerabilities were provided for review.

Our primary recommendations focus on the immediate implementation of a comprehensive security awareness training program to mitigate the identified human-factor risks.

% 5. ORGANIZATIONAL INFORMATION
\section*{Organizational Information}

The following details were used as the basis for this assessment. Due to the anonymized nature of the provided data, placeholders have been used where necessary.

\begin{tabular}{@{}ll}
\toprule
\textbf{Attribute} & \textbf{Value} \\
\midrule
Organization Name & \textbf{[Organization Name]} \\
Primary Domain & \texttt{[Domain]} \\
External IP Address (Client) & \texttt{[Client IP]} \\
\bottomrule
\end{tabular}

% 6. SECURITY CONTROL REVIEW (QUESTIONNAIRE)
\section*{Security Control Review}

The following table summarizes the organization's responses to the security controls questionnaire. A checkmark (\ding{51}) indicates a positive control is in place, while a cross (\ding{55}) indicates a control gap.

\begin{table}[h!]
\centering
\begin{tabular}{@{}p{0.7\linewidth}c@{}}
\toprule
\textbf{Control Question} & \textbf{Response} \\
\midrule
Do you require MFA to access email? & \ding{51} \\
Do you require MFA to log into computers? & \ding{51} \\
Do you require MFA to access sensitive data systems? & \ding{51} \\
Does your organization have an employee acceptable use policy? & \ding{51} \\
\addlinespace
\textbf{Does your organization do security awareness training for new employees?} & \textbf{\ding{55}} \\
\addlinespace
\textbf{Does your organization do security awareness training for all employees at least once per year?} & \textbf{\ding{55}} \\
\bottomrule
\end{tabular}
\caption{Security Controls Questionnaire Results}
\end{table}

The review highlights critical deficiencies in the security training and awareness domain, which are detailed in the Risk Assessment section.

% 7. TECHNICAL SCAN RESULTS
\section*{Technical Scan Results}

An external network scan was performed to identify potential vulnerabilities on the organization's perimeter.

\begin{itemize}
    \item \textbf{Target IP Address:} \texttt{[Target IP]}
    \item \textbf{Scan Date:} [Scan Date]
    \item \textbf{Findings:} The scan did not identify any open TCP or UDP ports on the target host. This indicates that the system is either well-firewalled, offline, or configured not to respond to network probes, which is a strong defensive posture from an external perspective. No vulnerabilities were detected.
\end{itemize}

% 8. RISK ASSESSMENT
\section*{Risk Assessment}

This section correlates findings from the security control review and technical scans. The primary risks identified stem from gaps in organizational policy and training rather than technical vulnerabilities.

\begin{table}[h!]
\centering
\begin{tabular}{@{}p{0.1\linewidth}p{0.25\linewidth}p{0.45\linewidth}p{0.1\linewidth}@{}}
\toprule
\textbf{ID} & \textbf{Risk Name} & \textbf{Overview} & \textbf{Severity} \\
\midrule
RISK-001 & Lack of Onboarding Security Training & New employees are not provided with security awareness training upon joining. This leaves them uninformed about policies and highly susceptible to social engineering attacks like phishing from their first day. & \textbf{High} \\
\addlinespace
RISK-002 & Lack of Annual Security Training & The absence of a recurring annual security training program means that employees' security knowledge is not refreshed. This increases the long-term risk of security incidents as threat landscapes evolve and vigilance decreases. & \textbf{High} \\
\bottomrule
\end{tabular}
\caption{Identified Risks and Severity}
\end{table}

% 9. RECOMMENDATIONS
\section*{Recommendations}

The following actionable recommendations are provided to address the identified risks and improve the overall security posture of \textbf{[Organization Name]}.

\subsection*{Immediate Actions (Next 30 Days)}
\begin{itemize}
    \item \textbf{Address RISK-001:} Implement a mandatory security awareness training module as part of the new employee onboarding process. This training must cover, at a minimum:
    \begin{itemize}
        \item The organization's acceptable use policy.
        \item How to identify and report phishing emails.
        \item Best practices for password security and MFA.
        \item Procedures for reporting a security incident.
    \end{itemize}
\end{itemize}

\subsection*{Strategic Actions (Next 90 Days)}
\begin{itemize}
    \item \textbf{Address RISK-002:} Establish a formal, annual security awareness training program for all employees. This program should be tracked for completion and supplemented with periodic phishing simulations to test and reinforce learning. The content should be updated yearly to reflect current threats.
\end{itemize}

\subsection*{Continuous Improvement}
\begin{itemize}
    \item \textbf{Maintain Strong Technical Controls:} Continue to enforce MFA across all critical systems and maintain a restrictive firewall policy for external-facing assets.
\end{itemize}

% 10. END DOCUMENT
\end{document}
```