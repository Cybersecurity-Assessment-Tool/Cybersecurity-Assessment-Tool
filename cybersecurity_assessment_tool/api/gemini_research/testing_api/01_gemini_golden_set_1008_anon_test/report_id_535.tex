```latex
\documentclass[12pt]{article}

% --- PACKAGES ---
\usepackage[margin=1in]{geometry}
\usepackage{pifont} % For checkmarks and crosses
\usepackage{booktabs} % For professional tables
\usepackage{hyperref} % For clickable links
\usepackage{url} % For URL formatting
\usepackage{seqsplit} % For splitting long strings in texttt
\usepackage[T1]{fontenc}

% --- DOCUMENT METADATA ---
\title{Cybersecurity Posture Assessment Report}
\author{Cybersecurity Analyst}
\date{\today}

% --- HYPERREF SETUP ---
\hypersetup{
    colorlinks=true,
    linkcolor=black,
    urlcolor=blue,
    pdftitle={Cybersecurity Posture Assessment Report},
    pdfauthor={Cybersecurity Analyst},
    pdfsubject={Security Analysis},
    pdfkeywords={Cybersecurity, Nmap, Risk Assessment}
}

% --- BEGIN DOCUMENT ---
\begin{document}

\maketitle
\thispagestyle{empty}
\newpage

\tableofcontents
\newpage

% ==============================================================================
\section{Executive Summary}
% ==============================================================================

This report details the findings of a cybersecurity posture assessment conducted for \textbf{[Organization Name]}. The assessment combined an analysis of organizational security controls via a questionnaire, an external network scan, and a review of pre-existing risks.

The analysis revealed several critical and high-risk security gaps that require immediate attention. The most significant findings include a complete lack of Multi-Factor Authentication (MFA) across all key systems (email, computers, and sensitive data), and the absence of a security awareness training program for employees. These administrative control failures create a high susceptibility to account compromise and social engineering attacks.

Technically, an exposed Secure Shell (SSH) service was identified on the external network perimeter. When combined with the lack of MFA, this exposed service presents a significant risk of unauthorized access to internal systems.

Immediate remediation should focus on deploying MFA across all critical assets, establishing a comprehensive security awareness training program, and hardening the exposed SSH service. Addressing these core issues will substantially improve the organization's overall security posture and resilience against common cyber threats.

% ==============================================================================
\section{Organizational Information}
% ==============================================================================

The following information was used as the basis for this assessment. Due to missing data in the provided inputs, placeholders have been used.

\begin{table}[h!]
\centering
\begin{tabular}{@{}ll@{}}
\toprule
\textbf{Attribute} & \textbf{Value} \\ \midrule
Organization Name & \textbf{[Organization Name]} \\
Primary Email Domain & \texttt{[Domain]} \\
Assessed External IP & \texttt{[Client IP]} \\ \bottomrule
\end{tabular}
\caption{Client Organizational Details}
\end{table}

% ==============================================================================
\section{Security Control Review}
% ==============================================================================

An administrative security control review was conducted based on a questionnaire. The responses indicate significant gaps in foundational security practices. A summary of the findings is presented below.

\begin{table}[h!]
\centering
\begin{tabular}{@{}p{0.5\textwidth}cp{0.25\textwidth}@{}}
\toprule
\textbf{Control Question} & \textbf{Response} & \textbf{Analyst Notes} \\ \midrule
Do you require MFA to access email? & \ding{55} & \textbf{Critical Gap.} Email is a primary target for account takeover. \\
Do you require MFA to log into computers? & \ding{55} & \textbf{Critical Gap.} Lack of MFA allows lateral movement post-compromise. \\
Do you require MFA to access sensitive data systems? & \ding{55} & \textbf{Critical Gap.} Direct risk to sensitive corporate and client data. \\
Does your organization have an employee acceptable use policy? & \ding{51} & Foundational policy is in place. \\
Does your organization do security awareness training for new employees? & \ding{55} & \textbf{High Risk.} New staff are unaware of policies and threats. \\
Does your organization do security awareness training for all employees at least once per year? & \ding{55} & \textbf{High Risk.} Security skills degrade; staff are not updated on new threats. \\ \bottomrule
\end{tabular}
\caption{Security Controls Questionnaire Analysis}
\end{table}

% ==============================================================================
\section{Technical Scan Results}
% ==============================================================================

An external network scan was performed against the target IP address \texttt{[Target IP]}. The scan identified the following open ports and services.

\begin{table}[h!]
\centering
\begin{tabular}{@{}llll@{}}
\toprule
\textbf{Port} & \textbf{State} & \textbf{Service} & \textbf{Notes} \\ \midrule
22/tcp & open & SSH & Secure Shell is a common remote management protocol. \\
& & & Exposing SSH to the public internet is a significant risk \\
& & & and a common target for brute-force attacks. \\
& & & The specific version was not identified in this scan. \\ \bottomrule
\end{tabular}
\caption{Open Ports Detected on \texttt{[Target IP]}}
\end{table}

% ==============================================================================
\section{Risk Assessment}
% ==============================================================================

The following risks have been identified by correlating the findings from the security control review and the technical scan. No pre-existing vulnerabilities were reported.

\begin{table}[h!]
\centering
\begin{tabular}{@{}p{0.1\textwidth}p{0.25\textwidth}p{0.1\textwidth}p{0.45\textwidth}@{}}
\toprule
\textbf{Risk ID} & \textbf{Risk Name} & \textbf{Severity} & \textbf{Description} \\ \midrule
RISK-001 & Lack of Multi-Factor Authentication (MFA) & \textbf{Critical} & The absence of MFA on email, endpoints, and sensitive systems means that a single compromised password could lead to a full account takeover and unauthorized data access. \\
\addlinespace
RISK-002 & Inadequate Security Awareness Program & \textbf{High} & Without initial or ongoing training, employees are significantly more likely to fall victim to phishing, social engineering, and malware, making them the weakest link in the security chain. \\
\addlinespace
RISK-003 & Exposed SSH Management Port & \textbf{Medium} & Port 22 (SSH) is open to the internet, creating a direct vector for brute-force and credential-stuffing attacks. This risk is amplified to High/Critical by the lack of MFA (RISK-001). \\ \bottomrule
\end{tabular}
\caption{Summary of Identified Risks}
\end{table}

% ==============================================================================
\section{Recommendations}
% ==============================================================================

Based on the identified risks, the following remediation actions are recommended to improve the organization's security posture.

\begin{enumerate}
    \item \textbf{[Critical] Implement Multi-Factor Authentication:}
    \begin{itemize}
        \item \textbf{Action:} Immediately deploy a robust MFA solution for all users.
        \item \textbf{Priority:} 1) Email accounts, 2) Remote access systems (including SSH), 3) Sensitive data repositories, and 4) Workstation logins.
        \item \textbf{Justification:} This is the single most effective control to prevent account takeovers, directly mitigating RISK-001.
    \end{itemize}
    \vspace{1em}
    \item \textbf{[High] Establish a Security Awareness Program:}
    \begin{itemize}
        \item \textbf{Action:} Develop and implement a mandatory security awareness training program.
        \item \textbf{Details:} The program should include onboarding training for new hires and at least one annual refresher for all staff. Topics must include phishing identification, password security, and acceptable use policies.
        \item \textbf{Justification:} This directly addresses RISK-002 by empowering employees to act as a human firewall, reducing the likelihood of successful social engineering attacks.
    \end{itemize}
    \vspace{1em}
    \item \textbf{[Medium] Harden External Services:}
    \begin{itemize}
        \item \textbf{Action:} Review and secure the exposed SSH service on \texttt{[Target IP]}.
        \item \textbf{Recommendations:}
            \begin{enumerate}
                \item If possible, restrict SSH access to a whitelist of trusted IP addresses.
                \item Disable password-based authentication and enforce the use of public key cryptography.
                \item Ensure the SSH server software is fully patched and running the latest stable version.
            \end{enumerate}
        \item \textbf{Justification:} These steps will harden the service against automated attacks and reduce the attack surface, mitigating RISK-003.
    \end{itemize}
\end{enumerate}

\end{document}
```