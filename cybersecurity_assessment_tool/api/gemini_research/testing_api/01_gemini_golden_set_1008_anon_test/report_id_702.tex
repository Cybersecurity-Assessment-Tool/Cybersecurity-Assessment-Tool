```latex
\documentclass[12pt]{article}

% --- PACKAGES ---
\usepackage[margin=1in]{geometry}
\usepackage{pifont} % For check and cross marks
\usepackage{booktabs} % For professional tables
\usepackage{hyperref} % For clickable links
\usepackage{url} % For URL formatting
\usepackage{seqsplit} % To split long strings in tt font
\usepackage{graphicx} % For logos, etc.
\usepackage{xcolor} % For colors

% --- DOCUMENT SETUP ---
\hypersetup{
    colorlinks=true,
    linkcolor=blue,
    filecolor=magenta,      
    urlcolor=cyan,
    pdftitle={Cybersecurity Posture Report},
    pdfpagemode=FullScreen,
}

% --- TITLE ---
\title{
    \vspace{-1.5cm}
    \includegraphics[width=0.4\textwidth]{example-image-a} \\ % Placeholder for a logo
    \vspace{1cm}
    \textbf{Cybersecurity Posture Report} \\
    \large For \textbf{[Organization Name]}
}
\author{Cybersecurity Analysis Team}
\date{\today}

% --- BEGIN DOCUMENT ---
\begin{document}

\maketitle
\thispagestyle{empty}
\newpage

\tableofcontents
\newpage

% ==============================================================================
% 1. EXECUTIVE SUMMARY
% ==============================================================================
\section{Executive Summary}

This report provides a comprehensive cybersecurity assessment for \textbf{[Organization Name]}, synthesizing data from an external network scan, a security controls questionnaire, and a review of pre-existing risks. The analysis was conducted to identify vulnerabilities, policy gaps, and areas for security posture improvement.

\paragraph{Key Findings:} The primary areas of concern identified are not technical vulnerabilities on the external perimeter, but significant gaps in internal security policies and procedures. The external network scan of \texttt{[Client IP]} revealed a secure posture regarding the tested ports, indicating that a previously identified risk related to an unencrypted web server has been successfully remediated.

However, the security control review highlights three critical areas requiring immediate attention:
\begin{itemize}
    \item \textbf{Lack of Endpoint Multi-Factor Authentication (MFA):} The absence of MFA for computer logins presents a critical risk, as compromised credentials could grant an attacker direct access to workstations and internal network resources.
    \item \textbf{Absence of an Acceptable Use Policy (AUP):} Without a formal AUP, there is no clear guidance for employees on the secure and acceptable use of company assets, increasing the risk of insider threats and unintentional data breaches.
    \item \textbf{Inadequate Security Awareness Training:} While new hires receive training, the lack of a mandatory, annual training program for all staff leaves the organization vulnerable to evolving threats like phishing and social engineering.
\end{itemize}

This report outlines these findings in detail and provides a prioritized list of actionable recommendations to mitigate the identified risks and strengthen the overall security posture of \textbf{[Organization Name]}.

% ==============================================================================
% 2. ORGANIZATIONAL INFORMATION
% ==============================================================================
\section{Organizational Information}

This section details the information provided for the assessment. The data has been anonymized as requested.

\begin{table}[h!]
\centering
\begin{tabular}{@{}ll@{}}
\toprule
\textbf{Attribute} & \textbf{Value} \\ \midrule
Organization Name  & \textbf{[Organization Name]} \\
Primary Domain     & \texttt{[Domain]} \\
External IP Address & \texttt{[Client IP]} \\ \bottomrule
\end{tabular}
\caption{Client Information}
\end{table}

% ==============================================================================
% 3. SECURITY CONTROL REVIEW
% ==============================================================================
\section{Security Control Review}

A review of the organization's security controls was conducted via a questionnaire. The responses indicate a mix of strong controls in some areas and significant gaps in others. The table below summarizes the findings. A green checkmark (\ding{51}) indicates a positive control is in place, while a red cross (\ding{55}) indicates a gap.

\begin{table}[h!]
\centering
\begin{tabular}{@{}p{0.8\linewidth}c@{}}
\toprule
\textbf{Control Question} & \textbf{Status} \\ \midrule
Do you require MFA to access email? & \textcolor{green}{\ding{51}} \\
Do you require MFA to log into computers? & \textcolor{red}{\ding{55}} \\
Do you require MFA to access sensitive data systems? & \textcolor{green}{\ding{51}} \\
Does your organization have an employee acceptable use policy? & \textcolor{red}{\ding{55}} \\
Does your organization do security awareness training for new employees? & \textcolor{green}{\ding{51}} \\
Does your organization do security awareness training for all employees at least once per year? & \textcolor{red}{\ding{55}} \\ \bottomrule
\end{tabular}
\caption{Security Controls Questionnaire Results}
\end{table}

\paragraph{Analysis of Gaps:} The identified gaps represent a significant risk to the organization.
\begin{itemize}
    \item \textbf{No MFA for Computer Logins:} This is a critical vulnerability. If an employee's password is stolen (e.g., through phishing or a third-party breach), an attacker could gain direct access to their workstation, potentially leading to lateral movement, data exfiltration, or ransomware deployment.
    \item \textbf{No Acceptable Use Policy (AUP):} An AUP is a foundational policy that sets expectations for employee behavior when using company technology. Its absence creates ambiguity and removes a key tool for enforcing security standards.
    \item \textbf{No Annual Security Training:} The threat landscape is constantly changing. Training only new hires is insufficient. A recurring, annual training program is essential to keep all staff vigilant against the latest phishing tactics and social engineering schemes.
\end{itemize}

% ==============================================================================
% 4. TECHNICAL SCAN RESULTS
% ==============================================================================
\section{Technical Scan Results}

An external network scan was performed to identify open ports and services exposed to the internet.

\begin{itemize}
    \item \textbf{Target IP Address:} \texttt{[Target IP]}
    \item \textbf{Scan Date:} [Scan Date]
    \item \textbf{Scanner Used:} Nmap
\end{itemize}

The scan confirmed the target host was online but found no open ports. The status of common ports is detailed below.

\begin{table}[h!]
\centering
\begin{tabular}{@{}llll@{}}
\toprule
\textbf{Port} & \textbf{Protocol} & \textbf{State}  & \textbf{Service} \\ \midrule
80            & TCP               & \textbf{closed} & http             \\ \bottomrule
\end{tabular}
\caption{Nmap Scan Results for \texttt{[Target IP]}}
\end{table}

\paragraph{Analysis:} The scan results are positive. The fact that port 80 (HTTP) is closed indicates that there is no unencrypted web service exposed to the internet from this IP address. This finding directly contradicts a pre-existing risk documented in the following section and suggests that successful remediation has already occurred. This is a commendable security improvement.

% ==============================================================================
% 5. RISK ASSESSMENT SUMMARY
% ==============================================================================
\section{Risk Assessment Summary}

This section synthesizes findings from the security control review, the technical scan, and pre-existing risk data into a consolidated list.

\begin{table}[h!]
\centering
\begin{tabular}{@{}p{0.3\linewidth}p{0.4\linewidth}ll@{}}
\toprule
\textbf{Risk Name} & \textbf{Overview} & \textbf{Severity} & \textbf{Status} \\ \midrule
\textbf{Lack of Endpoint MFA} & No MFA is required for computer logins, exposing the network to takeover via compromised credentials. & \textbf{Critical} & \textbf{Active} \\
\addlinespace
\textbf{Inadequate Security Training} & Lack of annual security training for all staff increases susceptibility to phishing and social engineering. & High & \textbf{Active} \\
\addlinespace
\textbf{Missing Acceptable Use Policy} & No formal policy exists to govern the use of company IT assets, leading to inconsistent security practices. & High & \textbf{Active} \\
\addlinespace
Unencrypted Web Server & Port 80 was believed to be open, exposing unencrypted traffic. & Medium & \textcolor{green}{\textbf{Remediated}} \\ \bottomrule
\end{tabular}
\caption{Consolidated Risk Register}
\end{table}

% ==============================================================================
% 6. RECOMMENDATIONS
% ==============================================================================
\section{Recommendations}

Based on the analysis, we provide the following prioritized recommendations to mitigate the identified risks and enhance the organization's security posture.

\subsection{Priority 1: Critical}
\begin{description}
    \item[Implement Endpoint MFA:] Deploy Multi-Factor Authentication for all employee computer and remote access logins. This is the single most effective control to prevent unauthorized access from compromised credentials.
    \begin{itemize}
        \item \textbf{Easy Fix:} Enable Windows Hello for Business or a similar OS-level MFA.
        \item \textbf{Long-Term Fix:} Integrate a centralized identity and access management (IAM) solution (e.g., Okta, Duo, Microsoft Entra ID) that enforces MFA across all endpoints and applications.
    \end{itemize}
\end{description}

\subsection{Priority 2: High}
\begin{description}
    \item[Develop and Implement an Acceptable Use Policy (AUP):] Create a formal AUP that clearly defines the rules and responsibilities for all employees when using company networks, devices, and data. The policy should be reviewed by legal/HR, communicated to all staff, and require a signed acknowledgment of receipt.
    
    \item[Establish Annual Security Awareness Training:] Implement a mandatory, annual security awareness training program for all employees. The training should cover current threats, including phishing, ransomware, password hygiene, and social engineering. Track completion to ensure 100\% compliance.
\end{description}

\subsection{Priority 3: Informational}
\begin{description}
    \item[Acknowledge and Maintain Positive Controls:] The closure of port 80 is a significant security improvement. We recommend establishing a program of regular, automated vulnerability scanning to ensure that this secure configuration is maintained and that new vulnerabilities are identified promptly.
\end{description}

\end{document}
```