```latex
\documentclass[12pt]{article}

% 1. DOCUMENT SETUP & PACKAGES
\usepackage[margin=1in]{geometry}
\usepackage{pifont} % For checkmarks and crosses
\usepackage{booktabs} % For professional tables
\usepackage{hyperref} % For clickable links
\usepackage{url}      % For URL formatting
\usepackage{seqsplit} % For splitting long strings to prevent overflow
\usepackage{xcolor}   % For colors

% Hyperref setup for a professional look
\hypersetup{
    colorlinks=true,
    linkcolor=black,
    filecolor=magenta,      
    urlcolor=blue,
    pdftitle={Cybersecurity Assessment Report},
    pdfpagemode=FullScreen,
}

% Define checkmark and cross symbols for clarity
\newcommand{\cmark}{\ding{51}}%
\newcommand{\xmark}{\ding{55}}%

% 2. DOCUMENT HEADER
\title{Cybersecurity Assessment Report \\ \large For \textbf{[Organization Name]}}
\author{Cybersecurity Analyst}
\date{November 22, 2025}

\begin{document}
\maketitle
\hrule
\vspace{1em}

% 3. EXECUTIVE SUMMARY
\section*{Executive Summary}
This report details the findings of a cybersecurity assessment conducted on November 22, 2025. The assessment combined a technical network scan, a review of organizational security controls, and an analysis of pre-existing risks.

The overall security posture of \textbf{[Organization Name]} is mixed. The organization has implemented some positive security controls, such as requiring multi-factor authentication (MFA) for email access and conducting regular security awareness training. However, several critical and high-risk vulnerabilities were identified that require immediate attention.

Key findings include:
\begin{itemize}
    \item \textbf{Critical Gaps in Access Control:} Multi-factor authentication is not enforced for computer logins or access to sensitive data systems. This represents a significant risk of unauthorized access and potential data breach.
    \item \textbf{Outdated Public-Facing Software:} The external web server is running an outdated version of Nginx (1.18.0), which is no longer supported and likely contains publicly known vulnerabilities.
    \item \textbf{Policy Gaps:} The organization lacks a formal Employee Acceptable Use Policy (AUP), which is a foundational component of a mature security program.
\end{itemize}
This report provides a detailed breakdown of these findings and offers actionable recommendations to mitigate the identified risks and improve the organization's overall security posture.

% 4. ORGANIZATIONAL INFORMATION
\section{Organizational Information}
The following information was used as the basis for this assessment. As per the provided data, placeholders have been used where specific details were not available.

\begin{tabular}{@{}ll}
    \toprule
    \textbf{Attribute} & \textbf{Value} \\
    \midrule
    Organization Name & \textbf{[Organization Name]} \\
    Assumed Email Domain & \texttt{[Domain]} \\
    Client External IP & \texttt{[Client IP]} \\
    \bottomrule
\end{tabular}

% 5. SECURITY CONTROL REVIEW (QUESTIONNAIRE)
\section{Security Control Review}
A review of organizational security controls was conducted based on a standardized questionnaire. The responses indicate critical gaps in administrative and technical controls. "No" answers highlight areas that deviate from security best practices and require remediation.

\begin{table}[h!]
\centering
\caption{Organizational Security Control Questionnaire Results}
\begin{tabular}{@{}p{0.7\linewidth} c c@{}}
    \toprule
    \textbf{Control Question} & \textbf{Response} & \textbf{Status} \\
    \midrule
    Do you require MFA to access email? & Yes & \cmark \\
    Do you require MFA to log into computers? & No & \textcolor{red}{\xmark} \\
    Do you require MFA to access sensitive data systems? & No & \textcolor{red}{\xmark} \\
    Does your organization have an employee acceptable use policy? & No & \textcolor{red}{\xmark} \\
    Does your organization do security awareness training for new employees? & Yes & \cmark \\
    Does your organization do security awareness training for all employees at least once per year? & Yes & \cmark \\
    \bottomrule
\end{tabular}
\end{table}

% 6. TECHNICAL SCAN RESULTS
\section{Technical Scan Results}
An external network scan was performed to identify open ports and exposed services. The scan was executed on \textbf{November 22, 2025}, against the target IP address \texttt{[Target IP]}.

\subsection{Open Ports and Services}
One open port was discovered, exposing a web server to the public internet.

\begin{table}[h!]
\centering
\caption{Network Scan Findings for Target: \texttt{[Target IP]}}
\begin{tabular}{@{}l l l l l@{}}
    \toprule
    \textbf{Port} & \textbf{State} & \textbf{Service} & \textbf{Product} & \textbf{Version} \\
    \midrule
    443/tcp & open & https & nginx & 1.18.0 \\
    \bottomrule
\end{tabular}
\end{table}

\subsection{Analyst Notes}
The scan identified an Nginx web server, version \textbf{1.18.0}. This version was released in April 2020 and its mainline support ended in May 2021. Running outdated software on internet-facing systems is a high-risk practice, as it is likely susceptible to numerous publicly disclosed vulnerabilities that have been patched in newer versions. Attackers frequently scan for and exploit such outdated services.

% 7. RISK ASSESSMENT SUMMARY
\section{Risk Assessment Summary}
The following table synthesizes findings from the security control review, the technical scan, and pre-existing risk data. Each identified risk has been assigned a severity level based on its potential impact and likelihood of exploitation.

\begin{table}[h!]
\centering
\caption{Summary of Identified Risks}
\begin{tabular}{@{}p{0.15\linewidth} p{0.55\linewidth} l@{}}
    \toprule
    \textbf{Risk ID} & \textbf{Description} & \textbf{Severity} \\
    \midrule
    RISK-001 & \textbf{Lack of MFA on Workstations:} User accounts are protected only by passwords for computer logins, making them vulnerable to credential stuffing, phishing, and brute-force attacks. & \textbf{Critical} \\
    \addlinespace
    RISK-002 & \textbf{Lack of MFA on Sensitive Systems:} Access to systems containing sensitive data is not protected by MFA, creating a high risk of a data breach if credentials are compromised. & \textbf{Critical} \\
    \addlinespace
    RISK-003 & \textbf{Outdated Nginx Web Server:} The public-facing web server at \texttt{[Target IP]} is running an unsupported version of Nginx (1.18.0), exposing the organization to known vulnerabilities. & \textbf{High} \\
    \addlinespace
    RISK-004 & \textbf{Missing Acceptable Use Policy (AUP):} The absence of a formal AUP means there are no clear, enforceable rules for employees regarding the use of company assets, which can lead to insider threats and misuse. & \textbf{High} \\
    \bottomrule
\end{tabular}
\end{table}

% 8. RECOMMENDATIONS
\section{Recommendations}
To address the identified risks, the following actions are recommended. These recommendations are prioritized based on the severity of the corresponding risk.

\begin{enumerate}
    \item \textbf{Implement Comprehensive MFA (RISK-001 \& RISK-002):}
    \begin{itemize}
        \item Immediately deploy a mandatory Multi-Factor Authentication (MFA) solution for all employee and contractor computer logins.
        \item Enforce MFA for access to all systems classified as containing sensitive, confidential, or proprietary data.
        \item This action is the single most effective control to mitigate the risk of unauthorized access from compromised credentials.
    \end{itemize}
    \vspace{1em}
    \item \textbf{Remediate Outdated Web Server (RISK-003):}
    \begin{itemize}
        \item Plan and execute an upgrade of the Nginx server at \texttt{[Target IP]} from version 1.18.0 to the latest stable version recommended by the vendor.
        \item Implement a formal patch and vulnerability management process to ensure all internet-facing systems are updated on a regular, timely basis.
    \end{itemize}
    \vspace{1em}
    \item \textbf{Develop and Implement an AUP (RISK-004):}
    \begin{itemize}
        \item Draft a comprehensive Employee Acceptable Use Policy (AUP) that clearly defines the rules for using company networks, devices, and data.
        \item Integrate the AUP into the employee onboarding process and require all current employees to read and formally acknowledge the policy.
        \item Conduct annual reviews and acknowledgements of the AUP as part of the security awareness training program.
    \end{itemize}
\end{enumerate}

\end{document}
```