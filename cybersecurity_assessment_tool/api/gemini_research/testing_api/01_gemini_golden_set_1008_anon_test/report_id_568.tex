```latex
\documentclass[12pt]{article}

% Preamble: Required Packages
\usepackage[margin=1in]{geometry}
\usepackage{pifont} % For checkmarks and crosses (\ding{51} and \ding{55})
\usepackage{booktabs} % For professional-looking tables
\usepackage{hyperref} % For clickable links and table of contents
\usepackage{url}      % For formatting URLs
\usepackage{seqsplit} % For splitting long strings to prevent overflow
\usepackage[utf8]{inputenc}

% Document Metadata
\title{Cybersecurity Posture Assessment Report for \textbf{[Organization Name]}}
\author{Cybersecurity Analyst}
\date{\today}

\begin{document}

\maketitle
\tableofcontents
\newpage

% --- Section 1: Executive Overview ---
\section{Executive Overview}

This report details the findings of a cybersecurity posture assessment conducted for \textbf{[Organization Name]}. The assessment incorporated a review of organizational security controls via a questionnaire, an external network scan of the designated perimeter, and an analysis of pre-existing risk data.

The overall security posture presents a mixed landscape. On a positive note, the external network scan of the target IP address \texttt{[Client IP]} revealed no open ports, suggesting a well-hardened network perimeter at the point of testing. This significantly reduces the external attack surface.

However, the security control review identified several critical and high-risk gaps in internal policies and technical controls. The most severe findings include the lack of multi-factor authentication (MFA) for accessing sensitive data systems and employee computers. Furthermore, the absence of a formal Acceptable Use Policy (AUP) and a mandatory annual security awareness training program for all employees creates significant risk. These policy and procedural deficiencies could expose the organization to insider threats, phishing attacks, and unauthorized data access, despite the strong external defenses.

Immediate action is recommended to address the identified control gaps, focusing on the implementation of MFA and the development of foundational security policies.

% --- Section 2: Organizational Information ---
\section{Organizational Information}

The following information was used as the basis for this assessment. As per the provided data, placeholder values are used where specific details were not available.

\begin{table}[h!]
\centering
\begin{tabular}{@{}ll@{}}
\toprule
\textbf{Attribute} & \textbf{Value} \\ \midrule
Organization Name & \textbf{[Organization Name]} \\
Email Domain & \texttt{[Domain]} \\
External IP Scanned & \texttt{[Client IP]} \\ \bottomrule
\end{tabular}
\caption{Client Organizational Data}
\label{tab:org_data}
\end{table}

% --- Section 3: Security Control Review ---
\section{Security Control Review}

A review of administrative and technical security controls was conducted based on a standardized questionnaire. The responses indicate critical gaps in the organization's security framework. A "No" response highlights a missing control and a potential area of high risk.

\begin{table}[h!]
\centering
\begin{tabular}{@{}p{0.6\textwidth}cc@{}}
\toprule
\textbf{Control Question} & \textbf{Response} & \textbf{Status} \\ \midrule
Do you require MFA to access email? & Yes & \ding{51} \\
Do you require MFA to log into computers? & No & \textbf{\color{red}\ding{55}} \\
Do you require MFA to access sensitive data systems? & No & \textbf{\color{red}\ding{55}} \\
Does your organization have an employee acceptable use policy? & No & \textbf{\color{red}\ding{55}} \\
Does your organization do security awareness training for new employees? & Yes & \ding{51} \\
Does your organization do security awareness training for all employees at least once per year? & No & \textbf{\color{red}\ding{55}} \\ \bottomrule
\end{tabular}
\caption{Security Control Questionnaire Results}
\label{tab:controls}
\end{table}

% --- Section 4: Technical Scan Results ---
\section{Technical Scan Results}

An external network vulnerability scan was performed using Nmap to identify open ports and services exposed to the internet.

\begin{itemize}
    \item \textbf{Target IP Address:} \texttt{[Target IP]}
    \item \textbf{Scan Date:} Not specified in scan data.
    \item \textbf{Host Status:} Up
\end{itemize}

\subsection{Scan Summary}
The scan confirmed that the target host is online and responsive. However, the scan did not identify any open TCP or UDP ports within the scanned range. The state of all extra ports was reported as "closed".

\textbf{Conclusion:} This result indicates a hardened network perimeter for the scanned IP address. The absence of open ports significantly reduces the risk of external network-based attacks against this specific target.

% --- Section 5: Risk Assessment ---
\section{Risk Assessment}

This section synthesizes findings from the security control review, technical scans, and pre-existing risk data. The primary risks identified are related to internal policy and identity management weaknesses. No pre-existing vulnerabilities were reported in the input data.

\begin{table}[h!]
\centering
\begin{tabular}{@{}p{0.15\textwidth}p{0.25\textwidth}p{0.4\textwidth}l@{}}
\toprule
\textbf{Risk ID} & \textbf{Risk Name} & \textbf{Description} & \textbf{Severity} \\ \midrule
RISK-001 & No MFA on Sensitive Systems & The lack of MFA for systems containing sensitive data exposes critical assets to unauthorized access via compromised credentials. & \textbf{Critical} \\
\addlinespace
RISK-002 & No MFA on Workstations & User workstations are not protected by MFA, making them vulnerable to takeover if credentials are stolen. This can lead to lateral movement within the network. & \textbf{High} \\
\addlinespace
RISK-003 & Lack of Annual Security Training & Without mandatory, recurring security awareness training, employees are more susceptible to social engineering and phishing attacks. & \textbf{High} \\
\addlinespace
RISK-004 & No Acceptable Use Policy (AUP) & The absence of a formal AUP means there are no clear, enforceable rules for employees regarding the use of company assets, data handling, and internet access. & \textbf{High} \\ \bottomrule
\end{tabular}
\caption{Summary of Identified Risks}
\label{tab:risks}
\end{table}

% --- Section 6: Recommendations ---
\section{Recommendations}

The following actionable recommendations are provided to mitigate the risks identified in Section 5. These are prioritized based on severity.

\begin{enumerate}
    \item \textbf{Implement MFA on Sensitive Systems (RISK-001 - Critical):}
    \begin{itemize}
        \item \textbf{Action:} Immediately deploy a robust multi-factor authentication solution (e.g., TOTP, FIDO2/WebAuthn, or push notifications) for all access to systems classified as containing sensitive or critical data.
        \item \textbf{Justification:} This is the single most effective control to prevent unauthorized access to critical data resulting from credential compromise.
    \end{itemize}
    \item \textbf{Deploy MFA for Workstation Logins (RISK-002 - High):}
    \begin{itemize}
        \item \textbf{Action:} Roll out MFA for all employee computer logins (Windows, macOS, etc.). This should apply to both local and remote access.
        \item \textbf{Justification:} Protects endpoint devices, which are a primary target for attackers seeking to establish a foothold in the network.
    \end{itemize}
    \item \textbf{Establish an Annual Security Awareness Program (RISK-003 - High):}
    \begin{itemize}
        \item \textbf{Action:} Develop and mandate an annual security awareness training program for all employees. The program should cover topics such as phishing, password hygiene, data handling, and social engineering.
        \item \textbf{Justification:} A well-trained workforce is a critical layer of defense against common cyber threats, turning a potential weakness into a strength.
    \end{itemize}
    \item \textbf{Develop and Implement an Acceptable Use Policy (RISK-004 - High):}
    \begin{itemize}
        \item \textbf{Action:} Draft a comprehensive AUP that clearly defines the rules for using company IT assets, including internet usage, data protection responsibilities, and software installation. All employees must read and acknowledge this policy.
        \item \textbf{Justification:} An AUP establishes a baseline for secure behavior and provides a legal and administrative framework for enforcing security standards.
    \end{itemize}
\end{enumerate}

\end{document}
```