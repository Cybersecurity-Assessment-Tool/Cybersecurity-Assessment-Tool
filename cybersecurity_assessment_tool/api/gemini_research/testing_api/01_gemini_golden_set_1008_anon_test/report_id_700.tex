```latex
\documentclass[12pt]{article}

% Preamble: Required Packages
\usepackage[margin=1in]{geometry}
\usepackage{pifont} % For checkmarks and crosses
\usepackage{booktabs} % For professional tables
\usepackage{hyperref} % For clickable links
\usepackage{url}      % For URL formatting
\usepackage{seqsplit} % For splitting long strings in texttt
\usepackage{graphicx}
\usepackage{xcolor}

% Document Metadata
\title{Cybersecurity Assessment Report}
\author{Cybersecurity Analysis Division}
\date{\today}

% Hyperref Setup
\hypersetup{
    colorlinks=true,
    linkcolor=blue,
    filecolor=magenta,      
    urlcolor=cyan,
    pdftitle={Cybersecurity Assessment Report},
    pdfpagemode=FullScreen,
}

% Custom Commands
\newcommand{\yes}{\ding{51}}
\newcommand{\no}{\ding{55}}

\begin{document}

\maketitle
\thispagestyle{empty}
\newpage

\tableofcontents
\newpage

% --- 1. Executive Summary ---
\section{Executive Summary}

This report details the findings of a cybersecurity assessment conducted for \textbf{[Organization Name]}. The assessment combined a review of organizational security controls, an external network vulnerability scan, and an analysis of pre-existing risks.

The key finding is a significant disparity between the organization's external network security posture and its internal procedural controls. The external network scan of the target IP address (\texttt{[Target IP]}) revealed a strong perimeter defense, with no open ports detected. This indicates a well-configured firewall and a minimized external attack surface.

However, the security control review identified several critical and high-risk gaps in internal policies and procedures. The most severe findings include:
\begin{itemize}
    \item \textbf{Critical Risk:} Lack of Multi-Factor Authentication (MFA) for accessing sensitive data systems.
    \item \textbf{High Risk:} A complete absence of a security awareness training program for employees.
    \item \textbf{High Risk:} The lack of a formal employee Acceptable Use Policy (AUP).
\end{itemize}

These gaps expose the organization to significant threats, particularly from phishing, social engineering, and insider threats, which could bypass the strong network perimeter. This report provides a detailed breakdown of these risks and offers prioritized, actionable recommendations to mitigate them and improve the overall security posture.

% --- 2. Organizational Information ---
\section{Organizational Information}

This section contains the high-level information for the organization under review. The data provided for this assessment was anonymized.

\begin{table}[h!]
\centering
\begin{tabular}{@{}ll@{}}
\toprule
\textbf{Attribute} & \textbf{Value} \\ \midrule
Organization Name & \textbf{[Organization Name]} \\
Primary Email Domain & \texttt{[Domain]} \\
External IP Address Assessed & \texttt{[Client IP]} \\ \bottomrule
\end{tabular}
\caption{Client Organizational Details.}
\end{table}

% --- 3. Security Control Review ---
\section{Security Control Review}

A security questionnaire was completed to evaluate the status of key administrative and technical controls. The responses indicate several foundational security measures are not in place. "No" answers represent significant gaps that increase organizational risk.

\begin{table}[h!]
\centering
\begin{tabular}{@{}p{0.6\linewidth}cc@{}}
\toprule
\textbf{Control Question} & \textbf{Response} & \textbf{Status} \\ \midrule
Do you require MFA to access email? & Yes & \yes \\
Do you require MFA to log into computers? & Yes & \yes \\
\textbf{Do you require MFA to access sensitive data systems?} & \textbf{No} & \textcolor{red}{\no} \\
\textbf{Does your organization have an employee acceptable use policy?} & \textbf{No} & \textcolor{red}{\no} \\
\textbf{Does your organization do security awareness training for new employees?} & \textbf{No} & \textcolor{red}{\no} \\
\textbf{Does your organization do security awareness training for all employees at least once per year?} & \textbf{No} & \textcolor{red}{\no} \\ \bottomrule
\end{tabular}
\caption{Security Controls Questionnaire Results.}
\end{table}

% --- 4. Technical Scan Results ---
\section{Technical Scan Results}

An external network scan was performed to identify open ports, running services, and potential vulnerabilities on the public-facing infrastructure.

\subsection{Scan Summary}
\begin{itemize}
    \item \textbf{Target IP Address:} \texttt{[Target IP]}
    \item \textbf{Scan Date:} \today
    \item \textbf{Scanner Used:} Nmap (Network Mapper)
\end{itemize}

\subsection{Findings}
The scan of the target host indicated that the host was online and responsive. However, \textbf{no open TCP ports were discovered}. All 1000 scanned ports were reported to be in a 'closed' state.

\textbf{Conclusion:} This is a positive security finding. It suggests that a firewall is properly configured to deny unsolicited inbound traffic, effectively minimizing the external network attack surface for the scanned IP address.

% --- 5. Risk Assessment ---
\section{Risk Assessment}

This section synthesizes the findings from the security control review and technical scan. While no pre-existing vulnerabilities were reported and the network scan was clean, the procedural gaps identified in the questionnaire constitute significant risks to the organization.

\begin{table}[h!]
\centering
\begin{tabular}{@{}p{0.1\linewidth}p{0.3\linewidth}p{0.4\linewidth}l@{}}
\toprule
\textbf{ID} & \textbf{Risk Name} & \textbf{Overview} & \textbf{Severity} \\ \midrule
RISK-001 & No MFA on Sensitive Systems & The absence of MFA on systems containing sensitive data exposes critical assets to unauthorized access via compromised credentials. & \textcolor{red}{\textbf{Critical}} \\
\addlinespace
RISK-002 & Lack of Security Awareness Training & Without formal training, employees are more susceptible to phishing, social engineering, and malware, making them the weakest link in the security chain. & \textcolor{orange}{High} \\
\addlinespace
RISK-003 & No Acceptable Use Policy (AUP) & Without a formal AUP, there are no clear guidelines for employees on the proper use of company assets, increasing the risk of insider threat and misuse. & \textcolor{orange}{High} \\
\bottomrule
\end{tabular}
\caption{Identified Risks and Severity.}
\end{table}

% --- 6. Recommendations ---
\section{Recommendations}

Based on the risk assessment, the following prioritized actions are recommended to strengthen the organization's cybersecurity posture.

\subsection{Priority 1: Critical}
\begin{description}
    \item[RISK-001: Implement MFA on Sensitive Systems]
    \textbf{Action:} Immediately deploy and enforce Multi-Factor Authentication (MFA) for all user accounts (including administrative and service accounts) that have access to systems storing or processing sensitive data.
    \textbf{Justification:} This is the single most effective control to prevent unauthorized access resulting from stolen or weak credentials.
\end{description}

\subsection{Priority 2: High}
\begin{description}
    \item[RISK-002: Establish a Security Awareness Program]
    \textbf{Action:} Develop and implement a mandatory security awareness training program. This program should include initial training for all new hires and annual refresher training for all staff. Topics should cover phishing identification, password hygiene, and incident reporting.
    \textbf{Justification:} A well-trained workforce serves as a human firewall, providing a critical layer of defense against common cyberattacks.

    \item[RISK-003: Develop and Enforce an Acceptable Use Policy (AUP)]
    \textbf{Action:} Draft a comprehensive AUP that clearly defines the rules and responsibilities for all users of the organization's IT resources. This policy should be formally communicated to all employees, and acknowledgement should be tracked.
    \textbf{Justification:} An AUP establishes a baseline for secure behavior, reduces ambiguity, and provides a framework for enforcing security standards.
\end{description}

\end{document}
```