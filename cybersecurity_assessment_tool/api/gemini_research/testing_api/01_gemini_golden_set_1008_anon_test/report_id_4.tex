```latex
\documentclass[12pt]{article}

% --- PACKAGES ---
\usepackage[margin=1in]{geometry}
\usepackage{pifont} % For checkmarks and crosses
\usepackage{booktabs} % For professional tables
\usepackage{hyperref} % For hyperlinks
\usepackage{url} % For URL formatting
\usepackage{seqsplit} % To split long strings without breaking
\usepackage{graphicx}
\usepackage{xcolor}

% --- DOCUMENT SETUP ---
\hypersetup{
    colorlinks=true,
    linkcolor=blue,
    filecolor=magenta,      
    urlcolor=cyan,
    pdftitle={Cybersecurity Assessment Report},
    pdfpagemode=FullScreen,
}

\newcommand{\yes}{\ding{51}} % Checkmark
\newcommand{\no}{\ding{55}}  % Cross

% --- DOCUMENT START ---
\begin{document}

% --- TITLE PAGE ---
\begin{titlepage}
    \centering
    \vspace*{1cm}
    \Huge\textbf{Cybersecurity Assessment Report}
    \vspace{0.5cm}
    \Large For
    \vspace{1.5cm}
    \Huge \textbf{[Organization Name]}
    \vspace{2cm}
    \includegraphics[width=0.4\textwidth]{example-image-a} % Placeholder for a logo
    \vfill
    \Large Prepared by: Cybersecurity Assessment Team \\
    \large \today
\end{titlepage}

\tableofcontents
\newpage

% --- 1. EXECUTIVE SUMMARY ---
\section{Executive Summary}
This report details the findings of a cybersecurity assessment conducted for \textbf{[Organization Name]}. The evaluation combined a review of organizational security controls, an external network scan, and an analysis of pre-existing risks.

The assessment identified a mixed security posture. The organization has implemented foundational controls, such as requiring Multi-Factor Authentication (MFA) for email and computer access. The external network scan of the target host \texttt{[Target IP]} revealed no open ports, indicating a strong perimeter defense for that specific asset.

However, several critical and high-risk gaps were discovered in the organization's policies and procedures. The most significant finding is the absence of MFA for accessing sensitive data systems. Additionally, the lack of a formal Acceptable Use Policy (AUP) and mandatory annual security awareness training for all staff exposes the organization to significant risks from insider threats and human error.

Immediate remediation should focus on implementing MFA for all sensitive systems, developing and enforcing an AUP, and establishing a comprehensive, recurring security training program.

% --- 2. ORGANIZATIONAL INFORMATION ---
\section{Organizational Information}
This section provides the key identification details for the organization under review. As this report was generated in template mode, placeholder values are used.

\begin{itemize}
    \item \textbf{Organization Name:} \textbf{[Organization Name]}
    \item \textbf{Primary Domain:} \texttt{[Domain]}
    \item \textbf{Scanned External IP:} \texttt{[Client IP]}
\end{itemize}

% --- 3. SECURITY CONTROL REVIEW ---
\section{Security Control Review}
A review of administrative and procedural security controls was conducted via a questionnaire. The results highlight both strengths and areas requiring immediate attention. "Yes" indicates a control is in place, while "No" signifies a gap.

\begin{table}[h!]
\centering
\caption{Organizational Security Controls Questionnaire}
\begin{tabular}{p{0.7\linewidth} c}
\toprule
\textbf{Control Question} & \textbf{Status} \\
\midrule
Do you require MFA to access email? & \yes \\
Do you require MFA to log into computers? & \yes \\
Do you require MFA to access sensitive data systems? & \textcolor{red}{\no} \\
Does your organization have an employee acceptable use policy? & \textcolor{red}{\no} \\
Does your organization do security awareness training for new employees? & \yes \\
Does your organization do security awareness training for all employees at least once per year? & \textcolor{red}{\no} \\
\bottomrule
\end{tabular}
\end{table}

The findings indicate critical gaps in protecting sensitive data and in establishing clear policies and ongoing training, which are essential components of a defense-in-depth strategy.

% --- 4. TECHNICAL SCAN RESULTS ---
\section{Technical Scan Results}
An external network vulnerability scan was performed to identify potential weaknesses in the organization's internet-facing infrastructure.

\begin{itemize}
    \item \textbf{Target IP Address:} \texttt{[Target IP]}
    \item \textbf{Scan Date:} Not specified in scan data.
    \item \textbf{Scan Summary:} The Nmap scan reported the host status as "up".
    \item \textbf{Findings:} No open TCP or UDP ports were discovered on the target host. All scanned ports were reported as "closed". This is a positive finding, suggesting that the firewall configuration for this specific host is effectively limiting its external attack surface. No services were exposed to the public internet on this IP address at the time of the scan.
\end{itemize}

% --- 5. RISK ASSESSMENT ---
\section{Risk Assessment}
This section synthesizes findings from the security control review and technical scan to provide a consolidated list of identified risks. No pre-existing vulnerabilities were reported. The primary risks identified are procedural and policy-based.

\begin{table}[h!]
\centering
\caption{Identified Risks and Severity}
\begin{tabular}{p{0.3\linewidth} p{0.5\linewidth} l}
\toprule
\textbf{Risk Name} & \textbf{Overview} & \textbf{Severity} \\
\midrule
\textbf{No MFA on Sensitive Data} & Sensitive data systems lack Multi-Factor Authentication, creating a single point of failure (passwords) for protecting critical assets. A compromised password could lead to a major data breach. & \textbf{Critical} \\
\addlinespace
\textbf{Missing Acceptable Use Policy (AUP)} & The absence of a formal AUP means employees lack clear guidelines on the acceptable use of company systems, data, and network resources. This increases the risk of misuse and insider threats. & \textbf{High} \\
\addlinespace
\textbf{Inadequate Security Awareness Training} & While new employees receive training, the lack of an annual refresher for all staff allows security knowledge to become outdated, making the organization more susceptible to phishing and social engineering attacks. & \textbf{High} \\
\bottomrule
\end{tabular}
\end{table}

% --- 6. RECOMMENDATIONS ---
\section{Recommendations}
Based on the risk assessment, the following actions are recommended to improve the security posture of \textbf{[Organization Name]}.

\begin{enumerate}
    \item \textbf{Implement MFA for Sensitive Systems (Critical):}
    \begin{itemize}
        \item \textbf{Action:} Prioritize and enforce the use of MFA on all applications, databases, and administrative interfaces that contain or provide access to sensitive or regulated data.
        \item \textbf{Impact:} Drastically reduces the risk of unauthorized access and data breaches resulting from compromised credentials.
    \end{itemize}
    \vspace{0.5cm}
    \item \textbf{Develop and Implement an Acceptable Use Policy (High):}
    \begin{itemize}
        \item \textbf{Action:} Create a comprehensive AUP that clearly defines the rules for using company technology and data. This policy should be formally acknowledged by all employees upon hiring and annually thereafter.
        \item \textbf{Impact:} Establishes a clear security baseline for employee behavior, reduces legal liability, and mitigates the risk of insider threats.
    \end{itemize}
    \vspace{0.5cm}
    \item \textbf{Establish an Annual Security Training Program (High):}
    \begin{itemize}
        \item \textbf{Action:} Institute a mandatory security awareness training program for all employees, to be completed at least once per year. The training should cover current threats such as phishing, ransomware, and social engineering.
        \item \textbf{Impact:} Reinforces a culture of security, keeps defenses against evolving threats current, and reduces the likelihood of successful social engineering attacks.
    \end{itemize}
\end{enumerate}

% --- 7. CONCLUSION ---
\section{Conclusion}
The assessment of \textbf{[Organization Name]} reveals a security posture with solid foundational elements but significant, high-impact gaps in policy and advanced controls. The lack of open ports on the scanned external host is a positive indicator of good network hygiene for that asset. However, the procedural weaknesses, particularly the absence of MFA for sensitive data, represent a clear and present danger to the organization.

By implementing the recommendations outlined in this report, \textbf{[Organization Name]} can substantially mitigate these risks, strengthen its defenses against modern cyber threats, and build a more resilient security framework.

\end{document}
```