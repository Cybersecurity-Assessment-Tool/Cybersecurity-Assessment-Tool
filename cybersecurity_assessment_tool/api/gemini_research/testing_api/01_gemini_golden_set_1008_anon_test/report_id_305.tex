```latex
\documentclass[12pt, a4paper]{article}

% Preamble: Required Packages
\usepackage[margin=1in]{geometry}
\usepackage{pifont} % For checkmarks and crosses
\usepackage{booktabs} % For professional tables
\usepackage{hyperref} % For hyperlinks and PDF metadata
\usepackage{url} % For formatting URLs
\usepackage{seqsplit} % For splitting long strings to prevent overflow
\usepackage{xcolor} % For colors
\usepackage{graphicx} % For potential logos or images
\usepackage{fancyhdr} % For headers and footers

% --- Document Setup ---
\hypersetup{
    colorlinks=true,
    linkcolor=blue,
    filecolor=magenta,      
    urlcolor=cyan,
    pdftitle={Cybersecurity Assessment Report},
    pdfauthor={Cybersecurity Analyst},
    pdfsubject={Security Analysis},
    pdfkeywords={Cybersecurity, Risk, Assessment},
    bookmarks=true
}

% Define colors for severity
\definecolor{sev_critical}{HTML}{990000}
\definecolor{sev_high}{HTML}{D14302}
\definecolor{sev_medium}{HTML}{E5A50A}
\definecolor{sev_low}{HTML}{3A854F}

% --- Header and Footer ---
\pagestyle{fancy}
\fancyhf{} % clear all header and footer fields
\fancyhead[L]{Cybersecurity Assessment Report}
\fancyhead[R]{\textbf{[Organization Name]}}
\fancyfoot[C]{\thepage}
\renewcommand{\headrulewidth}{0.4pt}
\renewcommand{\footrulewidth}{0.4pt}

% --- Document Start ---
\begin{document}

% --- Title Page ---
\begin{titlepage}
    \centering
    \vspace*{1cm}
    \Huge{\textbf{Cybersecurity Assessment Report}}
    \vspace{1.5cm}
    \large
    \begin{tabular}{ll}
        \textbf{Prepared For:} & \textbf{[Organization Name]} \\
        \textbf{Primary Domain:} & \texttt{[Domain]} \\
        \textbf{Assessed IP:} & \texttt{[Client IP]} \\
    \end{tabular}
    \vspace{2cm}
    
    \includegraphics[width=0.4\textwidth]{example-image-a} % Placeholder for a logo
    
    \vfill
    
    \large
    \textbf{Date of Report:} \today \\
    \textbf{Author:} Cybersecurity Analyst
    
\end{titlepage}

\tableofcontents
\newpage

% --- Section 1: Executive Summary ---
\section{Executive Summary}
This report details the findings of a cybersecurity assessment conducted for \textbf{[Organization Name]}. The analysis correlates data from an external network scan, a security controls questionnaire, and a review of pre-existing documented risks.

The assessment identified a \textbf{critical risk}: the direct exposure of a Remote Desktop Protocol (RDP) service on port 3389 to the public internet. This finding, observed on the target IP \texttt{[Target IP]}, is a common vector for ransomware attacks and unauthorized access.

This technical vulnerability is severely compounded by significant gaps in the organization's access control policies. Specifically, Multi-Factor Authentication (MFA) is not enforced for computer logins or for access to sensitive data systems. The combination of an exposed remote access service and insufficient authentication controls creates a high-likelihood path for a malicious actor to compromise the network.

Immediate remediation is required to address the exposed RDP service. Strategic initiatives must be undertaken to implement comprehensive MFA across the organization to mitigate the risk of credential-based attacks.

% --- Section 2: Organizational Information ---
\section{Organizational Information}
This section provides the baseline information used for this assessment. The data may be anonymized.

\begin{itemize}
    \item \textbf{Organization Name:} \textbf{[Organization Name]}
    \item \textbf{Primary Email Domain:} \texttt{[Domain]}
    \item \textbf{External IP Address Scanned:} \texttt{[Client IP]}
\end{itemize}

% --- Section 3: Security Control Review ---
\section{Security Control Review}
The following table summarizes the organization's responses to a security controls questionnaire. "No" answers indicate significant gaps in the security posture and are highlighted as areas of concern.

\begin{table}[h!]
\centering
\caption{Security Controls Questionnaire Analysis}
\begin{tabular}{p{0.6\linewidth} c l}
\toprule
\textbf{Control Question} & \textbf{Response} & \textbf{Assessment} \\
\midrule
Do you require MFA to access email? & \ding{51} & Best Practice Met \\
\addlinespace
Do you require MFA to log into computers? & \textbf{\color{red}\ding{55}} & \textbf{High Risk Gap} \\
\addlinespace
Do you require MFA to access sensitive data systems? & \textbf{\color{red}\ding{55}} & \textbf{Critical Risk Gap} \\
\addlinespace
Does your organization have an employee acceptable use policy? & \ding{51} & Best Practice Met \\
\addlinespace
Does your organization do security awareness training for new employees? & \ding{51} & Best Practice Met \\
\addlinespace
Does your organization do security awareness training for all employees at least once per year? & \ding{51} & Best Practice Met \\
\bottomrule
\end{tabular}
\end{table}

The failure to enforce MFA on computer and sensitive data system logins is a critical weakness. A single compromised password could grant an attacker significant access to internal resources.

% --- Section 4: Technical Scan Results ---
\section{Technical Scan Results}
An external network scan was performed to identify open ports and exposed services.

\begin{itemize}
    \item \textbf{Target IP Address:} \texttt{[Target IP]}
    \item \textbf{Scan Date:} Data Not Provided in Scan
\end{itemize}

The scan revealed the following open port:

\begin{table}[h!]
\centering
\caption{Open Port Analysis}
\begin{tabular}{l l l p{0.4\linewidth}}
\toprule
\textbf{Port} & \textbf{State} & \textbf{Service Name} & \textbf{Analyst Notes} \\
\midrule
3389/tcp & Open & \texttt{ms-wbt-server} & This is the standard port for Microsoft Remote Desktop Protocol (RDP). Exposing RDP directly to the internet is extremely dangerous and a primary target for ransomware gangs. \\
\bottomrule
\end{tabular}
\end{table}

% --- Section 5: Correlated Risk Assessment ---
\section{Correlated Risk Assessment}
This section synthesizes the findings from all data sources into a prioritized list of risks. The technical scan directly validates the pre-existing documented risk concerning RDP exposure.

\begin{table}[h!]
\centering
\caption{Summary of Key Risks}
\begin{tabular}{p{0.25\linewidth} p{0.5\linewidth} l}
\toprule
\textbf{Risk Name} & \textbf{Description} & \textbf{Severity} \\
\midrule
\textbf{Publicly Exposed RDP} & The network scan confirms that RDP is open on \texttt{[Target IP]}. This allows attackers to attempt brute-force password attacks or exploit RDP vulnerabilities. This risk is amplified by the lack of MFA on computer logins. & \textbf{\color{sev_critical}Critical (9.0)} \\
\addlinespace
\textbf{Insufficient MFA Coverage} & The organization does not require MFA for computer logins or access to sensitive data systems. This significantly increases the risk of a successful breach via stolen or weak credentials. & \textbf{\color{sev_high}High} \\
\bottomrule
\end{tabular}
\end{table}

% --- Section 6: Recommendations ---
\section{Recommendations}
The following actions are recommended to mitigate the identified risks. Recommendations are prioritized based on severity and impact.

\subsection{Immediate Priority (Critical)}
\begin{enumerate}
    \item \textbf{Disable Public RDP Access Immediately:} The most critical action is to block all inbound traffic to TCP port 3389 on the external firewall for IP address \texttt{[Target IP]}.
    \begin{itemize}
        \item \textbf{Short-Term Fix:} Implement a firewall rule to deny all incoming connections to this port from the internet.
        \item \textbf{Long-Term Solution:} If remote access is required, implement a Virtual Private Network (VPN) with mandatory MFA. All RDP access should occur exclusively through the secure VPN tunnel.
    \end{itemize}
\end{enumerate}

\subsection{High Priority}
\begin{enumerate}
    \setcounter{enumi}{1} % Continue numbering
    \item \textbf{Deploy Comprehensive MFA:}
    \begin{itemize}
        \item Enforce MFA for all user logins to company computers (desktops and laptops).
        \item Enforce MFA for all access to applications and repositories containing sensitive company or customer data.
    \end{itemize}
    \item \textbf{Conduct a Credential Audit:} Review all user accounts, especially those with administrative privileges, to ensure they are using strong, unique passwords. Enforce a password policy that aligns with NIST guidelines.
\end{enumerate}

\subsection{Medium Priority}
\begin{enumerate}
    \setcounter{enumi}{3} % Continue numbering
    \item \textbf{Review Network Segmentation:} Ensure that critical servers and systems are isolated on separate network segments from general user workstations to limit lateral movement in the event of a compromise.
    \item \textbf{Enhance Logging and Monitoring:} Implement robust logging for all remote access attempts (both successful and failed) and ensure logs are monitored for suspicious activity.
\end{enumerate}

\end{document}
```