```latex
\documentclass[12pt]{article}

% Preamble: Required Packages
\usepackage[margin=1in]{geometry}
\usepackage{pifont} % For checkmarks and crosses
\usepackage{booktabs} % For professional tables
\usepackage{hyperref} % For hyperlinks
\usepackage{url} % For URL formatting
\usepackage{seqsplit} % To split long monospaced strings
\usepackage{graphicx}
\usepackage{xcolor}

% Document Metadata and Hyperref Setup
\hypersetup{
    colorlinks=true,
    linkcolor=blue,
    filecolor=magenta,      
    urlcolor=cyan,
    pdftitle={Cybersecurity Posture Assessment Report},
    pdfauthor={Cybersecurity Analyst},
    pdfsubject={Security Analysis},
    pdfkeywords={Security, Report, Analysis},
}

% Define colors for severity
\definecolor{critseverity}{HTML}{6B0504}
\definecolor{highseverity}{HTML}{A40606}
\definecolor{medseverity}{HTML}{D76A03}
\definecolor{lowseverity}{HTML}{F6C800}
\definecolor{infoseverity}{HTML}{005F99}

% Helper command for severity text
\newcommand{\severity}[2]{\colorbox{#1}{\textcolor{white}{\textbf{\phantom{M}#2\phantom{M}}}}}

\begin{document}

% --- Title Page ---
\begin{titlepage}
    \centering
    \vspace*{1cm}
    \Huge \textbf{Cybersecurity Posture Assessment Report}
    \vspace{1.5cm}
    \
    \large
    Prepared for: \\
    \vspace{0.5cm}
    \textbf{[Organization Name]}
    
    \vspace{2cm}
    
    \large
    \textbf{Date of Report:} \today
    
    \vfill
    
    \large
    \textbf{CONFIDENTIAL} \\
    \textit{This document contains sensitive information. Distribution is restricted.}
\end{titlepage}

\tableofcontents
\newpage

% --- Section 1: Executive Overview ---
\section*{1. Executive Overview}

This report provides a comprehensive analysis of the cybersecurity posture for \textbf{[Organization Name]}. The assessment is based on a correlation of network scan data, a security controls questionnaire, and a review of pre-existing risk documentation.

The analysis revealed critical gaps in organizational security controls, primarily related to user authentication and security awareness. Specifically, the absence of Multi-Factor Authentication (MFA) on employee computers and the lack of a formal security awareness training program present a high level of risk. These gaps significantly increase the organization's susceptibility to credential theft, phishing attacks, and subsequent unauthorized access to internal systems.

On a positive note, the technical network scan of the target asset did not confirm a previously identified risk. The scan found that Port 80 (HTTP) was closed, indicating that the "Unencrypted Web Server" risk has likely been remediated for this specific asset.

Immediate action should be focused on implementing the high-priority recommendations outlined in Section 6 to mitigate the identified control gaps and strengthen the overall security posture.

% --- Section 2: Organizational Information ---
\section*{2. Organizational Information}

This section details the organizational context used for this assessment. The data has been anonymized as per client request.

\begin{itemize}
    \item \textbf{Organization Name:} \textbf{[Organization Name]}
    \item \textbf{Primary Domain:} \texttt{[Domain]}
    \item \textbf{External IP Scanned:} \texttt{[Client IP]}
\end{itemize}

% --- Section 3: Security Control Review ---
\section*{3. Security Control Review}

The following table summarizes the organization's responses to a security controls questionnaire. The assessment column highlights areas that deviate from established security best practices and represent significant control gaps.

\begin{table}[h!]
\centering
\caption{Security Controls Questionnaire Analysis}
\begin{tabular}{p{8cm} c p{4cm}}
\toprule
\textbf{Control Question} & \textbf{Response} & \textbf{Assessment} \\
\midrule
Do you require MFA to access email? & \ding{51} & Control Met \\
\addlinespace
Do you require MFA to log into computers? & \textbf{\color{red}\ding{55}} & \textbf{Critical Gap}. Lack of endpoint MFA is a high-risk vulnerability. \\
\addlinespace
Do you require MFA to access sensitive data systems? & \ding{51} & Control Met \\
\addlinespace
Does your organization have an employee acceptable use policy? & \ding{51} & Control Met \\
\addlinespace
Does your organization do security awareness training for new employees? & \textbf{\color{red}\ding{55}} & \textbf{High Risk}. New hires are a primary target for social engineering. \\
\addlinespace
Does your organization do security awareness training for all employees at least once per year? & \textbf{\color{red}\ding{55}} & \textbf{High Risk}. Lack of ongoing training increases susceptibility to phishing. \\
\bottomrule
\end{tabular}
\end{table}

\subsection*{Analysis of Control Gaps}
The most critical findings from the review are:
\begin{itemize}
    \item \textbf{Lack of Endpoint MFA:} A compromised password could grant an attacker direct access to an employee's computer, facilitating lateral movement and deeper network compromise.
    \item \textbf{No Security Awareness Training:} Employees are the first line of defense. Without initial and recurring training, the organization is highly vulnerable to phishing, malware, and other social engineering tactics.
\end{itemize}

% --- Section 4: Technical Scan Results ---
\section*{4. Technical Scan Results}

A network scan was performed to identify open ports and exposed services on the target system.

\begin{itemize}
    \item \textbf{Target IP Address:} \texttt{[Target IP]}
    \item \textbf{Scanner Used:} Nmap
\end{itemize}

\subsection*{Port Scan Findings}
The scan results for the target host are summarized below.

\begin{table}[h!]
\centering
\caption{Nmap Port Scan Results for \texttt{[Target IP]}}
\begin{tabular}{l l l}
\toprule
\textbf{Port} & \textbf{State} & \textbf{Service} \\
\midrule
80/tcp & Closed & http \\
\bottomrule
\end{tabular}
\end{table}

\subsection*{Technical Analysis}
The scan revealed that the target host is online, but Port 80 (HTTP) is \textbf{closed}. This is a positive security finding. It directly contradicts the pre-existing risk titled "Unencrypted Web Server," which stated that Port 80 was open. This suggests that the previously identified risk has been successfully remediated or was a false positive for this specific asset. No other open ports or vulnerable services were identified on this target during the scan.

% --- Section 5: Consolidated Risk Assessment ---
\section*{5. Consolidated Risk Assessment}

This section synthesizes findings from the security control review, technical scan, and pre-existing risk data into a consolidated list.

\begin{table}[h!]
\centering
\caption{Summary of Identified Risks}
\begin{tabular}{p{4.5cm} p{7cm} l}
\toprule
\textbf{Risk Name} & \textbf{Description} & \textbf{Severity} \\
\midrule
\textbf{Lack of Endpoint MFA} & User computers do not require Multi-Factor Authentication for login. A compromised password provides direct endpoint access. & \severity{highseverity}{High} \\
\addlinespace
\textbf{Inadequate Security Awareness Program} & The organization does not provide security training to new or existing employees, increasing susceptibility to phishing and social engineering. & \severity{highseverity}{High} \\
\addlinespace
\textbf{Unencrypted Web Server} & \textit{(From Input 3)} Port 80 was believed to be open, exposing unencrypted traffic. \textbf{Status: Not Confirmed.} The scan shows this port is closed. & \severity{infoseverity}{Info} \\
\bottomrule
\end{tabular}
\end{table}

% --- Section 6: Recommendations ---
\section*{6. Recommendations}

The following actions are recommended to address the identified risks and improve the overall security posture of \textbf{[Organization Name]}. Recommendations are prioritized based on risk severity.

\subsection*{Priority 1: High-Risk Items}

\subsubsection*{Implement MFA for Endpoint Logon}
\begin{itemize}
    \item \textbf{Action:} Deploy and enforce a Multi-Factor Authentication solution for all employee computer logins (Windows, macOS, etc.).
    \item \textbf{Rationale:} This creates a critical defense layer against credential theft. Even if an attacker obtains a user's password, they cannot log in to the machine without the second factor.
    \item \textbf{Suggested Tools:} Windows Hello for Business, Duo Security, Okta.
\end{itemize}

\subsubsection*{Establish a Security Awareness Training Program}
\begin{itemize}
    \item \textbf{Action:} Develop and implement a mandatory security awareness training program. This program must include:
    \begin{enumerate}
        \item Onboarding training for all new employees before they are granted system access.
        \item Annual refresher training for all staff covering current threats like phishing, ransomware, and proper data handling.
        \item Periodic phishing simulations to test and reinforce training.
    \end{enumerate}
    \item \textbf{Rationale:} A well-trained workforce is significantly less likely to fall victim to common cyberattacks, reducing the organization's overall risk profile.
\end{itemize}

\subsection*{Priority 2: Risk Management}

\subsubsection*{Review and Close Remediated Risk}
\begin{itemize}
    \item \textbf{Action:} Formally review the "Unencrypted Web Server" risk in your risk register. Based on the technical scan results, this risk appears to be remediated for the target \texttt{[Target IP]}.
    \item \textbf{Rationale:} Maintaining an accurate risk register is crucial for focusing resources on current, relevant threats. Closing remediated risks ensures that security efforts are not wasted on non-existent issues.
\end{itemize}

\end{document}
```