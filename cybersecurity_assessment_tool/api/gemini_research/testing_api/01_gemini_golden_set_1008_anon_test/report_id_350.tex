```latex
\documentclass[12pt]{article}

% Preamble: Required Packages
\usepackage[margin=1in]{geometry}
\usepackage{pifont} % For checkmarks and crosses
\usepackage{booktabs} % For professional tables
\usepackage{hyperref} % For clickable links and references
\usepackage{url} % For formatting URLs
\usepackage{seqsplit} % For splitting long strings in texttt
\usepackage{graphicx}
\usepackage{xcolor}

% Hyperref Setup
\hypersetup{
    colorlinks=true,
    linkcolor=blue,
    filecolor=magenta,      
    urlcolor=cyan,
    pdftitle={Cybersecurity Assessment Report},
    pdfpagemode=FullScreen,
}

% Document Title and Author
\title{Cybersecurity Assessment Report \\ \large For \textbf{[Organization Name]}}
\author{Cybersecurity Analyst}
\date{\today}

\begin{document}

\maketitle
\tableofcontents
\newpage

% --- 1. Executive Summary ---
\section{Executive Summary}

This report details the findings of a cybersecurity assessment conducted for \textbf{[Organization Name]}. The assessment combined an analysis of organizational security controls, a technical network scan of external infrastructure, and a review of known existing risks.

The assessment identified several critical and high-severity risks that expose the organization to significant threats, including unauthorized access, data breach, and system compromise.

\textbf{Key Findings Include:}
\begin{itemize}
    \item \textbf{Critical External Vulnerability:} A publicly accessible FTP server was found running a dangerously outdated and misconfigured version of \texttt{vsftpd}. This vulnerability (CVE-2011-2523) allows for remote code execution by an unauthenticated attacker. The server also permits anonymous login, posing a severe data leak risk.
    \item \textbf{Critical Internal Control Gaps:} The organization does not enforce Multi-Factor Authentication (MFA) for email or computer logins. This significantly increases the risk of account compromise through phishing or password theft.
    \item \textbf{High-Risk Policy Gap:} The absence of an employee Acceptable Use Policy (AUP) creates ambiguity regarding security responsibilities and acceptable system usage, weakening the overall security posture.
    \item \textbf{Pre-existing Medium Risk:} The continued use of the End-of-Life Windows 7 operating system on workstations remains a known risk, as these systems no longer receive security updates.
\end{itemize}

Immediate remediation of the identified critical vulnerabilities is strongly recommended to reduce the likelihood of a security incident.

% --- 2. Organizational Information ---
\section{Organizational Information}

This section contains the high-level information used as the basis for this assessment. Due to the anonymized nature of the provided data, placeholders have been used where necessary.

\begin{tabular}{@{}ll}
    \toprule
    \textbf{Attribute} & \textbf{Value} \\
    \midrule
    Organization Name & \textbf{[Organization Name]} \\
    Primary Email Domain & \texttt{[Domain]} \\
    Primary External IP & \texttt{[Client IP]} \\
    Scanned Target IP & \texttt{[Target IP]} \\
    \bottomrule
\end{tabular}

% --- 3. Security Control Review ---
\section{Security Control Review}

The following table summarizes the organization's self-reported security controls based on the provided questionnaire. Items marked with \ding{55} represent significant gaps in the security framework and are considered high-impact findings.

\begin{table}[h!]
\centering
\begin{tabular}{@{}p{0.6\linewidth} c l@{}}
    \toprule
    \textbf{Control Question} & \textbf{Response} & \textbf{Assessment} \\
    \midrule
    Do you require MFA to access email? & \ding{55} & \textcolor{red}{\textbf{Critical Gap}} \\
    Do you require MFA to log into computers? & \ding{55} & \textcolor{red}{\textbf{Critical Gap}} \\
    Do you require MFA to access sensitive data systems? & \ding{51} & Meets Best Practice \\
    Does your organization have an employee acceptable use policy? & \ding{55} & \textcolor{orange}{High Risk} \\
    Does your organization do security awareness training for new employees? & \ding{51} & Meets Best Practice \\
    Does your organization do security awareness training for all employees at least once per year? & \ding{51} & Meets Best Practice \\
    \bottomrule
\end{tabular}
\caption{Organizational Security Controls Questionnaire Results.}
\label{tab:controls}
\end{table}

The lack of MFA for primary access vectors like email and computer logins is a critical weakness. These are common entry points for attackers, and their compromise can lead to widespread system access and data exfiltration. The absence of an Acceptable Use Policy is also a notable administrative control failure.

% --- 4. Technical Scan Results ---
\section{Technical Scan Results}

An external network scan was performed against the target IP address \texttt{[Target IP]}. The scan identified one host with a critical vulnerability.

\subsection{Host: \texttt{[Target IP]}}
\textbf{Status:} Up

\begin{table}[h!]
\centering
\begin{tabular}{@{}lllll@{}}
    \toprule
    \textbf{Port} & \textbf{State} & \textbf{Service} & \textbf{Product \& Version} & \textbf{Notes} \\
    \midrule
    21/tcp & Open & ftp & \seqsplit{\texttt{vsftpd 2.3.4}} & Anonymous FTP login allowed. \\
    \bottomrule
\end{tabular}
\caption{Open Ports and Services on \texttt{[Target IP]}.}
\label{tab:scanresults}
\end{table}

\subsubsection{Analysis of Findings}
The FTP service is running \textbf{vsftpd version 2.3.4}. This specific version is widely known to contain a critical backdoor vulnerability, cataloged as \textbf{CVE-2011-2523}. This flaw was intentionally added to the source code and allows an unauthenticated remote attacker to execute arbitrary commands with root privileges.

Furthermore, the server is configured to allow \textbf{anonymous FTP login}. This is a dangerous misconfiguration that can be exploited for data exfiltration or to stage malicious files. The combination of these two findings presents an immediate and severe threat to the organization's network.

% --- 5. Consolidated Risk Assessment ---
\section{Consolidated Risk Assessment}

This section synthesizes findings from the security control review, technical scan, and pre-existing risk data into a consolidated list.

\begin{table}[h!]
\centering
\begin{tabular}{@{}p{0.25\linewidth} p{0.55\linewidth} l@{}}
    \toprule
    \textbf{Risk Name} & \textbf{Description} & \textbf{Severity} \\
    \midrule
    Vulnerable External FTP Server & A public-facing FTP server is running vsftpd 2.3.4, which is vulnerable to remote code execution (CVE-2011-2523) and allows anonymous login. & \textcolor{red}{\textbf{Critical}} \\
    \addlinespace
    Lack of Multi-Factor Authentication & No MFA is enforced for email or computer logins, exposing the organization to account takeover and lateral movement attacks. & \textcolor{red}{\textbf{Critical}} \\
    \addlinespace
    Missing Acceptable Use Policy & The absence of a formal AUP creates compliance and operational risks by not defining rules for employee use of IT assets. & \textcolor{orange}{\textbf{High}} \\
    \addlinespace
    End-of-Life Operating System & Workstations are running Windows 7, an unsupported OS that no longer receives security updates, leaving them vulnerable to known exploits. & \textbf{Medium} \\
    \bottomrule
\end{tabular}
\caption{Summary of Identified Risks.}
\label{tab:risks}
\end{table}

% --- 6. Recommendations ---
\section{Recommendations}

The following actions are recommended to mitigate the identified risks. Recommendations are prioritized based on severity.

\subsection{Remediation for Vulnerable External FTP Server (Critical)}
\begin{itemize}
    \item \textbf{Immediate:} Take the FTP server at \texttt{[Target IP]} offline immediately to prevent exploitation.
    \item \textbf{Short-Term:} If FTP is a business necessity, migrate the service to a new server with a modern, patched version of an FTP server (e.g., vsftpd 3.0.5+). Disable anonymous access and enforce strong authentication.
    \item \textbf{Long-Term:} Evaluate the need for FTP. Migrate to a more secure file transfer protocol such as SFTP (SSH File Transfer Protocol) or a managed cloud-based file sharing solution.
\end{itemize}

\subsection{Remediation for Lack of MFA (Critical)}
\begin{itemize}
    \item \textbf{Immediate:} Enforce MFA for all accounts with administrative privileges across all systems.
    \item \textbf{Short-Term:} Enable and enforce MFA for all user email accounts. Most modern email providers (e.g., Microsoft 365, Google Workspace) have robust MFA capabilities.
    \item \textbf{Long-Term:} Implement an MFA solution for all computer and remote access logins (e.g., via Windows Hello for Business, Duo, or a similar identity provider).
\end{itemize}

\subsection{Remediation for Missing Acceptable Use Policy (High)}
\begin{itemize}
    \item \textbf{Short-Term:} Draft and formally approve an Acceptable Use Policy (AUP). The policy should clearly define rules for using company networks, devices, and data.
    \item \textbf{Ongoing:} Require all new and existing employees to read and formally acknowledge the AUP. Integrate the policy into the security awareness training program.
\end{itemize}

\subsection{Remediation for End-of-Life OS (Medium)}
\begin{itemize}
    \item \textbf{Immediate:} If immediate upgrades are not possible, ensure all Windows 7 machines are isolated on the network and restricted from accessing sensitive data or the public internet.
    \item \textbf{Short-Term:} Execute the plan to upgrade all remaining Windows 7 workstations to a supported operating system, such as Windows 10 or Windows 11.
\end{itemize}

\end{document}
```