```latex
\documentclass[12pt]{article}

% ----------------------------------------------------------------------
% PREAMBLE
% ----------------------------------------------------------------------
\usepackage[margin=1in]{geometry}
\usepackage{pifont} % Required for \ding
\usepackage{booktabs} % For professional tables (\toprule, \midrule, \bottomrule)
\usepackage{hyperref} % For clickable links and references
\usepackage{url}      % For formatting URLs
\usepackage{seqsplit} % For splitting long strings in \texttt
\usepackage{xcolor}   % For custom colors
\usepackage{fancyhdr} % For headers and footers
\usepackage{lastpage} % To get the total number of pages

% --- Color Definitions ---
\definecolor{tablehead}{gray}{0.9}
\definecolor{severitycritical}{HTML}{990000}
\definecolor{severityhigh}{HTML}{D14302}
\definecolor{severitymedium}{HTML}{EFAF00}

% --- Hyperref Setup ---
\hypersetup{
    colorlinks=true,
    linkcolor=blue,
    filecolor=magenta,      
    urlcolor=cyan,
    pdftitle={Cybersecurity Posture Assessment Report},
    pdfauthor={Cybersecurity Analysis Division},
}

% --- Header and Footer Setup ---
\pagestyle{fancy}
\fancyhf{} % Clear all header and footer fields
\lhead{Cybersecurity Assessment Report}
\rhead{\textbf{[Organization Name]}}
\cfoot{Page \thepage\ of \pageref{LastPage}}
\renewcommand{\headrulewidth}{0.4pt}
\renewcommand{\footrulewidth}{0.4pt}

% ----------------------------------------------------------------------
% DOCUMENT START
% ----------------------------------------------------------------------
\begin{document}

\title{Cybersecurity Posture Assessment Report}
\author{Cybersecurity Analysis Division}
\date{\today}
\maketitle

\begin{abstract}
This report provides a comprehensive analysis of the cybersecurity posture for \textbf{[Organization Name]}. The assessment is based on a synthesis of external network scan data, a security controls questionnaire, and a review of previously identified risks. The analysis reveals several critical and high-severity risks that require immediate attention. Key findings include a publicly exposed and highly vulnerable FTP server, a systemic lack of Multi-Factor Authentication (MFA), and the absence of foundational security policies and training. These issues collectively create a significant risk of unauthorized access, data breach, and system compromise.
\end{abstract}

\tableofcontents
\newpage

% ----------------------------------------------------------------------
% SECTION 1: OVERVIEW
% ----------------------------------------------------------------------
\section{Overview and Scope}
This assessment was conducted to evaluate the external security posture and internal security controls of the organization. The scope included:
\begin{itemize}
    \item \textbf{Organizational Information Review:} Analysis of client-provided data.
    \item \textbf{Security Control Review:} Evaluation of responses to a security questionnaire.
    \item \textbf{Technical Network Scan:} An external vulnerability scan of the provided IP address.
    \item \textbf{Existing Risk Correlation:} Integration of pre-existing risk data.
\end{itemize}

The overall security posture is assessed as \textbf{CRITICAL} due to the presence of an exploitable vulnerability on a public-facing service combined with weak access control policies.

% ----------------------------------------------------------------------
% SECTION 2: ORGANIZATIONAL INFORMATION
% ----------------------------------------------------------------------
\section{Organizational Information}
The following details were used as the basis for this assessment. As per the template mode, placeholders are used where data was not provided.

\begin{tabular}{@{}ll}
    \toprule
    \textbf{Attribute} & \textbf{Value} \\
    \midrule
    Organization Name & \textbf{[Organization Name]} \\
    Primary Domain & \texttt{[Domain]} \\
    External IP Address Scanned & \texttt{[Client IP]} \\
    \bottomrule
\end{tabular}

% ----------------------------------------------------------------------
% SECTION 3: SECURITY CONTROL REVIEW
% ----------------------------------------------------------------------
\section{Security Control Review}
The following table summarizes the organization's responses to a security controls questionnaire. A checkmark (\ding{51}) indicates a positive control is in place, while a cross (\ding{55}) indicates a control gap. The numerous gaps identified below represent significant policy and procedural weaknesses.

\begin{table}[h!]
\centering
\begin{tabular}{@{}p{0.7\textwidth}c@{}}
    \toprule
    \rowcolor{tablehead}
    \textbf{Control Question} & \textbf{Response} \\
    \midrule
    Do you require MFA to access email? & \textcolor{red}{\ding{55}} \\
    Do you require MFA to log into computers? & \textcolor{red}{\ding{55}} \\
    Do you require MFA to access sensitive data systems? & \textcolor{green}{\ding{51}} \\
    Does your organization have an employee acceptable use policy? & \textcolor{red}{\ding{55}} \\
    Does your organization do security awareness training for new employees? & \textcolor{red}{\ding{55}} \\
    Does your organization do security awareness training for all employees at least once per year? & \textcolor{red}{\ding{55}} \\
    \bottomrule
\end{tabular}
\caption{Security Controls Questionnaire Results.}
\end{table}

\textbf{Analysis:} The lack of MFA for email and computer logins is a critical weakness, exposing the organization to credential theft and unauthorized access. Furthermore, the absence of an acceptable use policy and any form of security awareness training indicates a low level of security maturity and leaves the organization vulnerable to insider threats and social engineering attacks.

% ----------------------------------------------------------------------
% SECTION 4: TECHNICAL SCAN RESULTS
% ----------------------------------------------------------------------
\section{Technical Scan Results}
An Nmap scan was performed on the target IP address. The target was identified as \texttt{[Target IP]} based on the scan data. The results reveal a critical vulnerability.

\begin{table}[h!]
\centering
\begin{tabular}{@{}lllll@{}}
    \toprule
    \rowcolor{tablehead}
    \textbf{Port} & \textbf{State} & \textbf{Service} & \textbf{Version} & \textbf{Notes} \\
    \midrule
    21/tcp & Open & ftp & vsftpd 2.3.4 & \begin{tabular}[t]{@{}l@{}}\textbf{CRITICAL FINDING:}\\- Anonymous FTP login allowed.\\- This version is vulnerable to a\\ well-known backdoor command\\ execution exploit (CVE-2011-2523).\end{tabular} \\
    \bottomrule
\end{tabular}
\caption{Nmap Scan Findings for Target \texttt{[Target IP]}.}
\end{table}

\textbf{Analysis:} The presence of an open FTP port running \textbf{vsftpd version 2.3.4} is a severe and immediate threat. This specific version contains a critical backdoor vulnerability that was introduced into the source code, allowing an unauthenticated attacker to execute arbitrary commands with root privileges on the server. The allowance of anonymous FTP login further lowers the barrier to exploitation and poses a significant data leakage risk.

% ----------------------------------------------------------------------
% SECTION 5: RISK ASSESSMENT SUMMARY
% ----------------------------------------------------------------------
\section{Risk Assessment Summary}
The following table correlates findings from the questionnaire, technical scan, and pre-existing risk data into a prioritized list.

\begin{table}[h!]
\centering
\begin{tabular}{@{}p{0.45\textwidth}p{0.2\textwidth}p{0.25\textwidth}@{}}
    \toprule
    \rowcolor{tablehead}
    \textbf{Risk Description} & \textbf{Severity} & \textbf{Source of Finding} \\
    \midrule
    \textbf{Vulnerable FTP Service (vsftpd 2.3.4):} An internet-facing service is vulnerable to remote code execution (CVE-2011-2523). & \textcolor{severitycritical}{\textbf{Critical}} & Network Scan \\
    \addlinespace
    \textbf{Anonymous FTP Access Enabled:} Unauthenticated users can access the FTP server, posing a high risk of data exfiltration or malware upload. & \textcolor{severitycritical}{\textbf{Critical}} & Network Scan \\
    \addlinespace
    \textbf{Lack of Multi-Factor Authentication:} Email and computer logins are protected only by passwords, making them highly susceptible to phishing and credential stuffing attacks. & \textcolor{severityhigh}{\textbf{High}} & Questionnaire \\
    \addlinespace
    \textbf{Lack of Security Policies \& Training:} The absence of an Acceptable Use Policy and security training program indicates a weak security culture and high susceptibility to human error. & \textcolor{severityhigh}{\textbf{High}} & Questionnaire \\
    \addlinespace
    \textbf{Outdated Windows Policy:} Workstations are running Windows 7, which is an unsupported End-of-Life (EOL) operating system. & \textcolor{severitymedium}{\textbf{Medium}} & Provided Data \\
    \bottomrule
\end{tabular}
\caption{Consolidated Risk Register.}
\end{table}

% ----------------------------------------------------------------------
% SECTION 6: RECOMMENDATIONS
% ----------------------------------------------------------------------
\section{Recommendations}
The following actions are recommended to mitigate the identified risks. They are prioritized based on severity.

\subsection{Priority 1: Critical Risks (Immediate Action Required)}
\begin{enumerate}
    \item \textbf{Remediate Vulnerable FTP Server:} Immediately take the server at \texttt{[Target IP]} offline. Upgrade the \texttt{vsftpd} software to the latest stable version (e.g., 3.0.5 or newer) or replace it with a secure file transfer alternative like SFTP (SSH File Transfer Protocol).
    \item \textbf{Disable Anonymous FTP:} If FTP must be used, anonymous access must be disabled immediately. All access should require strong, unique user credentials.
\end{enumerate}

\subsection{Priority 2: High Risks (Action within 30-60 Days)}
\begin{enumerate}
    \item \textbf{Implement MFA:} Deploy Multi-Factor Authentication across all critical systems, prioritizing email (e.g., Office 365, G Suite) and remote access solutions (VPNs). Extend MFA to all computer logins.
    \item \textbf{Develop Foundational Policies:} Create and enforce an Employee Acceptable Use Policy that governs the use of company assets, data handling, and internet usage.
    \item \textbf{Establish Security Awareness Training:} Implement a mandatory security awareness training program for all new and existing employees. Training should be conducted upon hiring and at least annually thereafter, covering topics like phishing, password security, and data protection.
\end{enumerate}

\subsection{Priority 3: Medium Risks (Action within 90-180 Days)}
\begin{enumerate}
    \item \textbf{Upgrade End-of-Life Systems:} Continue with the plan to upgrade all Windows 7 workstations to a supported operating system, such as Windows 10 or 11, to ensure security patches are received.
\end{enumerate}

% ----------------------------------------------------------------------
% DOCUMENT END
% ----------------------------------------------------------------------
\end{document}
```