```latex
\documentclass[12pt]{article}

% Preamble: Required Packages
\usepackage[margin=1in]{geometry}
\usepackage{pifont}          % For checkmarks and crosses (\ding)
\usepackage{booktabs}        % For professional-looking tables
\usepackage{hyperref}        % For clickable links and references
\usepackage{url}             % For formatting URLs
\usepackage{seqsplit}        % To split long strings in \texttt
\usepackage{xcolor}          % For colored text
\usepackage{graphicx}

% --- Document Setup ---
\hypersetup{
    colorlinks=true,
    linkcolor=blue,
    filecolor=magenta,      
    urlcolor=cyan,
}

% Custom commands for clarity in tables
\newcommand{\yes}{\ding{51}}
\newcommand{\no}{\ding{55}}
\newcommand{\riskcritical}{\textcolor{red}{Critical}}
\newcommand{\riskhigh}{\textcolor{orange}{High}}
\newcommand{\risklow}{Low}

% --- Document Start ---
\begin{document}

% --- Title Page ---
\title{
    \vspace{-1.5cm}
    \includegraphics[width=0.4\textwidth]{example-image-a} \\ % Placeholder for a logo
    \vspace{1cm}
    \textbf{Cybersecurity Posture Assessment Report}
}
\author{Cybersecurity Analysis Division}
\date{\today}
\maketitle
\thispagestyle{empty}
\newpage

% --- Table of Contents ---
\tableofcontents
\newpage

% --- Executive Summary ---
\section{Executive Summary}

This report provides a comprehensive analysis of the cybersecurity posture for \textbf{[Organization Name]}. The assessment synthesizes data from an external network scan, a security controls questionnaire, and a review of pre-existing risk documentation.

The analysis reveals a \textbf{critical-risk security posture}. A primary internet-facing service, Remote Desktop Protocol (RDP), is directly exposed on host \texttt{[Target IP]}. This presents an immediate and severe threat of unauthorized access, data breach, and ransomware.

This critical technical vulnerability is dangerously compounded by significant gaps in foundational security controls. Key findings include:
\begin{itemize}
    \item \textbf{Lack of Multi-Factor Authentication (MFA) on Email:} The absence of MFA on the primary communication and identity platform makes user accounts highly susceptible to compromise.
    \item \textbf{Absence of Security Awareness Training:} The organization does not conduct security awareness training for new or existing employees, leaving it vulnerable to phishing and social engineering attacks.
\end{itemize}

The combination of an exposed, high-value target (RDP) and an untrained workforce with easily compromised email accounts creates a high likelihood of a security incident with significant business impact. Immediate remediation is required to address these findings.

% --- Organizational Information ---
\section{Organizational Information}

This section details the organizational context for this assessment. The data provided was anonymized.

\begin{tabular}{@{}ll}
\toprule
\textbf{Attribute} & \textbf{Value} \\
\midrule
Organization Name & \textbf{[Organization Name]} \\
Primary Email Domain & \texttt{[Domain]} \\
External IP Scanned & \texttt{[Target IP]} \\
Assessment Date & \today \\
\bottomrule
\end{tabular}

% --- Security Control Review ---
\section{Security Control Review}

This section reviews the organization's security policies and procedures based on the provided questionnaire. Gaps identified here often represent significant organizational risk by increasing the attack surface or reducing defensive capabilities.

\begin{tabular}{@{}p{0.6\textwidth}cc@{}}
\toprule
\textbf{Control Question} & \textbf{In Place?} & \textbf{Associated Risk} \\
\midrule
Do you require MFA to access email? & \no & \riskcritical \\
Do you require MFA to log into computers? & \yes & \risklow \\
Do you require MFA to access sensitive data systems? & \yes & \risklow \\
Does your organization have an employee acceptable use policy? & \yes & \risklow \\
Does your organization do security awareness training for new employees? & \no & \riskhigh \\
Does your organization do security awareness training for all employees at least once per year? & \no & \riskhigh \\
\bottomrule
\end{tabular}

\subsection*{Analysis of Gaps}
The "No" responses highlight critical deficiencies. The lack of MFA on email is the most severe gap, as email accounts are a primary target for attackers seeking to gain an initial foothold. The complete absence of a security awareness training program indicates a low level of security maturity and leaves the human element as the weakest link in the defense chain.

% --- Technical Scan Results ---
\section{Technical Scan Results}

An external network scan was performed to identify exposed services on the organization's perimeter.

\subsection*{Host: \texttt{[Target IP]}}
The host was found to be online and responsive during the scan. The following open port was identified.

\begin{tabular}{@{}llll@{}}
\toprule
\textbf{Port} & \textbf{State} & \textbf{Service} & \textbf{Analysis} \\
\midrule
3389/tcp & open & ms-wbt-server & \parbox[t]{0.6\textwidth}{This port is used for Microsoft Remote Desktop Protocol (RDP). Exposing RDP directly to the public internet is a \textbf{critical security risk}. It is a primary target for brute-force credential attacks, credential stuffing, and exploitation of known vulnerabilities (e.g., BlueKeep, DejaBlue). This finding directly confirms the pre-existing risk documented in Input 3.} \\
\bottomrule
\end{tabular}

% --- Overall Risk Assessment ---
\section{Overall Risk Assessment}

The following table synthesizes findings from the security control review, technical scan, and pre-existing risk data to provide a holistic view of the organization's risk profile.

\begin{tabular}{@{}p{0.25\textwidth}p{0.55\textwidth}l@{}}
\toprule
\textbf{Identified Risk} & \textbf{Description & Correlation} & \textbf{Severity} \\
\midrule
\textbf{Public RDP Exposure} & The technical scan confirmed that RDP (port 3389) is open on host \texttt{[Target IP]}. This aligns with the pre-existing known risk ("RDP Exposure") and exposes the organization to severe threats like ransomware and unauthorized remote access. & \riskcritical \\
\addlinespace
\textbf{No MFA on Email} & The lack of MFA on the \texttt{[Domain]} email system makes user accounts highly susceptible to compromise via phishing or credential stuffing. A compromised email account can lead to further internal compromise and be used to attack the exposed RDP service. & \riskcritical \\
\addlinespace
\textbf{No Security Awareness Training} & Employees are not trained to identify or report security threats. This significantly increases the likelihood of a successful phishing attack, which is the primary delivery mechanism for malware and credential theft that could be used against exposed services. & \riskhigh \\
\bottomrule
\end{tabular}

% --- Recommendations ---
\section{Recommendations}

Based on the analysis, the following actions are recommended to mitigate the identified risks. Recommendations are prioritized by urgency to address the most critical threats first.

\subsection*{Immediate (0-24 Hours)}
\begin{itemize}
    \item \textbf{Block RDP Access:} Immediately configure the perimeter firewall to \textbf{deny all inbound traffic} to TCP port 3389 on host \texttt{[Target IP]}. This is the most critical step to prevent an imminent compromise.
\end{itemize}

\subsection*{High Priority (1-2 Weeks)}
\begin{itemize}
    \item \textbf{Enforce MFA for Email:} Enable and enforce Multi-Factor Authentication (MFA) for all user email accounts associated with \texttt{[Domain]}. This drastically reduces the risk of account takeovers.
    \item \textbf{Implement Security Awareness Training:} Procure and implement a security awareness training program. All current employees should complete initial training, and it must be integrated into the onboarding process for all new hires.
\end{itemize}

\subsection*{Strategic (1-3 Months)}
\begin{itemize}
    \item \textbf{Deploy a Secure Remote Access Solution:} For legitimate remote access needs, deploy a Virtual Private Network (VPN) or Zero Trust Network Access (ZTNA) solution. All remote administrative access should occur exclusively through this secure, MFA-protected channel.
    \item \textbf{Establish an Annual Training Policy:} Formalize a policy requiring all employees to complete security awareness training at least once per year to ensure security knowledge remains current and effective.
\end{itemize}

\end{document}
```