```latex
\documentclass[12pt]{article}

% --- PACKAGES ---
\usepackage[margin=1in]{geometry}
\usepackage{pifont} % For checkmarks and crosses
\usepackage{booktabs} % For professional tables
\usepackage{hyperref} % For hyperlinks
\usepackage{url}      % For URL formatting
\usepackage{seqsplit} % For splitting long strings in texttt

% --- DOCUMENT METADATA ---
\title{Cybersecurity Posture Assessment Report}
\author{Cybersecurity Analyst}
\date{\today}

% --- DOCUMENT START ---
\begin{document}

\maketitle
\thispagestyle{empty}
\newpage
\tableofcontents
\thispagestyle{empty}
\newpage
\setcounter{page}{1}

% ==============================================================================
\section{Executive Summary}
% ==============================================================================
This report provides a cybersecurity assessment for \textbf{[Organization Name]}, based on an analysis of network scan data, a security controls questionnaire, and a review of pre-existing risks.

The assessment reveals a mixed security posture. While foundational controls like Multi-Factor Authentication (MFA) for email and computer access are in place, several critical and high-risk gaps were identified. The most significant concerns are the lack of MFA for sensitive data systems and the absence of a formal security awareness training program for employees.

Furthermore, a technical scan identified an exposed administrative service (SSH on port 22) on the external network perimeter. When combined with the identified policy gaps, this technical finding elevates the risk of unauthorized access and potential data compromise.

Immediate remediation is recommended to address these vulnerabilities, focusing on implementing comprehensive MFA, establishing a security training program, and securing the exposed network service.

% ==============================================================================
\section{Organizational Information}
% ==============================================================================
The following information was used as the basis for this assessment. Placeholders are used where data was not provided.

\begin{itemize}
    \item \textbf{Organization Name:} \textbf{[Organization Name]}
    \item \textbf{Primary Domain:} \texttt{[Domain]}
    \item \textbf{External IP Scanned:} \texttt{[Client IP]}
\end{itemize}

% ==============================================================================
\section{Security Control Review}
% ==============================================================================
The following table summarizes the organization's responses to a security controls questionnaire. A green checkmark (\ding{51}) indicates a positive control is in place, while a red cross (\ding{55}) indicates a control gap.

\begin{table}[h!]
\centering
\caption{Security Controls Questionnaire Results}
\begin{tabular}{p{0.75\linewidth} c}
\toprule
\textbf{Control Question} & \textbf{Response} \\
\midrule
Do you require MFA to access email? & \ding{51} \\
Do you require MFA to log into computers? & \ding{51} \\
Do you require MFA to access sensitive data systems? & \textcolor{red}{\ding{55}} \\
Does your organization have an employee acceptable use policy? & \ding{51} \\
Does your organization do security awareness training for new employees? & \textcolor{red}{\ding{55}} \\
Does your organization do security awareness training for all employees at least once per year? & \textcolor{red}{\ding{55}} \\
\bottomrule
\end{tabular}
\end{table}

\paragraph{Analysis:} The lack of MFA on sensitive data systems represents a critical security gap. This control is essential for protecting the organization's most valuable assets. Additionally, the complete absence of a security awareness training program leaves the organization highly vulnerable to social engineering and phishing attacks, which are primary vectors for initial compromise.

% ==============================================================================
\section{Technical Scan Results}
% ==============================================================================
An external network scan was performed on the target IP address to identify open ports and exposed services.

\begin{itemize}
    \item \textbf{Target IP:} \texttt{[Target IP]}
    \item \textbf{Scan Date:} Not provided in scan data.
\end{itemize}

\begin{table}[h!]
\centering
\caption{Open Ports Detected}
\begin{tabular}{l l l l}
\toprule
\textbf{Port} & \textbf{State} & \textbf{Service} & \textbf{Product / Version} \\
\midrule
22/tcp & open & ssh (inferred) & Not Detected \\
\bottomrule
\end{tabular}
\end{table}

\paragraph{Analysis:} The scan identified that port 22 (SSH - Secure Shell) is open to the public internet. SSH is a common protocol for remote server administration. While necessary for management, its public exposure makes it a prime target for automated brute-force attacks and exploitation if vulnerabilities exist. The scan did not retrieve version information, which prevents a specific check for known vulnerabilities (CVEs). However, any publicly exposed administrative interface should be considered a significant risk.

% ==============================================================================
\section{Risk Assessment Summary}
% ==============================================================================
The following risks were identified by correlating the security control gaps with the technical findings. The pre-existing risk list was empty, indicating these are newly identified findings from this assessment.

\begin{table}[h!]
\centering
\caption{Identified Risks}
\begin{tabular}{p{0.1\linewidth} p{0.6\linewidth} l}
\toprule
\textbf{Risk ID} & \textbf{Description} & \textbf{Severity} \\
\midrule
RISK-001 & Lack of MFA on sensitive data systems allows for unauthorized access with compromised credentials alone. & \textbf{Critical} \\
\addlinespace
RISK-002 & Inadequate security awareness training increases the likelihood of successful phishing and social engineering attacks, leading to credential theft. & \textbf{High} \\
\addlinespace
RISK-003 & The SSH administrative port is exposed to the public internet, increasing the risk of brute-force attacks and unauthorized system access. & \textbf{High} \\
\bottomrule
\end{tabular}
\end{table}

% ==============================================================================
\section{Recommendations}
% ==============================================================================
The following actions are recommended to mitigate the identified risks and improve the overall security posture of the organization.

\begin{enumerate}
    \item \textbf{[Critical] Implement MFA on Sensitive Systems:} Immediately prioritize the deployment of Multi-Factor Authentication on all systems classified as containing sensitive data. This is the single most effective control to prevent unauthorized access resulting from compromised credentials.

    \item \textbf{[High] Establish a Security Awareness Training Program:} Develop and implement a mandatory security awareness training program. This program must include:
    \begin{itemize}
        \item Onboarding training for all new employees before they are granted system access.
        \item Annual refresher training for all staff to keep them informed of current threats.
        \item Regular phishing simulations to test and reinforce employee vigilance.
    \end{itemize}

    \item \textbf{[High] Secure the Exposed SSH Port:} Restrict access to the public-facing SSH service. The preferred methods are:
    \begin{itemize}
        \item Use a firewall to limit access to a whitelist of trusted IP addresses (e.g., corporate office or VPN).
        \item Enforce strong, key-based authentication and completely disable password-based authentication for SSH.
    \end{itemize}

    \item \textbf{[Informational] Conduct In-Depth Vulnerability Scanning:} Perform an authenticated and unauthenticated vulnerability scan across the external perimeter and internal network. This will provide detailed service version information and identify specific Common Vulnerabilities and Exposures (CVEs) that require patching.
\end{enumerate}

% ==============================================================================
\section{Conclusion}
% ==============================================================================
This assessment has identified significant and actionable risks to \textbf{[Organization Name]}. While some security controls are in place, the gaps related to MFA and security training, combined with the exposed administrative service, create a high-risk environment. We strongly advise that the organization's leadership review these findings and allocate the necessary resources to implement the provided recommendations in a timely manner.

\end{document}
```