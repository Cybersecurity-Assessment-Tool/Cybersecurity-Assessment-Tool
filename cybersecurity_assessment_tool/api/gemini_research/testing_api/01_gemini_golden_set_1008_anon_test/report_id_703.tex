```latex
\documentclass[12pt]{article}

% Required Packages
\usepackage[margin=1in]{geometry}
\usepackage{pifont} % For checkmarks and crosses
\usepackage{booktabs} % For professional tables
\usepackage{hyperref} % For hyperlinks
\usepackage{url}      % For URL formatting
\usepackage{seqsplit} % To split long strings without breaking
\usepackage{graphicx} % For potential logos
\usepackage{xcolor}   % For colors in text

% --- Document Metadata ---
\title{Cybersecurity Posture Assessment Report}
\author{Cybersecurity Analysis Division}
\date{\today}

% --- Hyperref Setup ---
\hypersetup{
    colorlinks=true,
    linkcolor=blue,
    filecolor=magenta,      
    urlcolor=cyan,
    pdftitle={Cybersecurity Posture Assessment Report},
    pdfpagemode=FullScreen,
}

\begin{document}

\maketitle
\thispagestyle{empty}
\newpage

\tableofcontents
\newpage

% --- Section 1: Executive Summary ---
\section{Executive Summary}

This report details the findings of a cybersecurity posture assessment conducted for \textbf{[Organization Name]}. The assessment incorporated a review of organizational security controls, an external network scan, and an analysis of pre-existing risks.

The overall security posture reveals a mix of strong preventative controls and significant foundational gaps. The organization has commendably implemented Multi-Factor Authentication (MFA) across key areas, including email, computer logins, and access to sensitive systems. Furthermore, a robust security awareness training program is in place for all employees.

However, two key areas of concern were identified. Firstly, a critical governance gap exists due to the absence of an Employee Acceptable Use Policy (AUP). This policy is fundamental for setting clear expectations for employee behavior and mitigating insider threats. Secondly, the external network scan identified an exposed Secure Shell (SSH) service (port 22) on the network perimeter. While necessary for remote administration, its public exposure presents a potential attack vector if not properly secured.

This report provides a detailed breakdown of these findings and offers actionable recommendations to mitigate the identified risks and strengthen the organization's overall security posture.

% --- Section 2: Organizational Information ---
\section{Organizational Information}

The following information was used as the basis for this assessment. Due to the anonymized nature of the provided data, placeholders have been used where necessary.

\begin{table}[h!]
\centering
\begin{tabular}{@{}ll@{}}
\toprule
\textbf{Attribute} & \textbf{Value} \\ \midrule
Organization Name & \textbf{[Organization Name]} \\
Primary Domain & \texttt{[Domain]} \\
External IP Address & \texttt{[Client IP]} \\ \bottomrule
\end{tabular}
\caption{Client Organizational Details.}
\label{tab:org_info}
\end{table}

% --- Section 3: Security Control Review ---
\section{Security Control Review}

A review of administrative and organizational security controls was conducted based on a supplied questionnaire. The responses indicate a strong focus on identity and access management but highlight a critical policy gap.

\begin{table}[h!]
\centering
\begin{tabular}{@{}p{0.6\linewidth}cp{0.2\linewidth}@{}}
\toprule
\textbf{Control Question} & \textbf{Response} & \textbf{Assessment} \\ \midrule
Do you require MFA to access email? & \ding{51} & Compliant \\
Do you require MFA to log into computers? & \ding{51} & Compliant \\
Do you require MFA to access sensitive data systems? & \ding{51} & Compliant \\
Does your organization have an employee acceptable use policy? & \textcolor{red}{\ding{55}} & \textbf{High Risk Gap} \\
Does your organization do security awareness training for new employees? & \ding{51} & Compliant \\
Does your organization do security awareness training for all employees at least once per year? & \ding{51} & Compliant \\ \bottomrule
\end{tabular}
\caption{Organizational Security Control Questionnaire Results.}
\label{tab:controls}
\end{table}

The primary finding from this review is the lack of an Employee Acceptable Use Policy. This policy is a cornerstone of a corporate security program, defining the rules and expected behaviors for all users of company IT assets. Its absence can lead to inconsistent practices, misuse of resources, and a weakened legal standing in the event of an insider-related incident.

% --- Section 4: Technical Scan Results ---
\section{Technical Scan Results}

An external network scan was performed to identify open ports and exposed services on the client's perimeter.

\begin{itemize}
    \item \textbf{Scan Target:} \texttt{[Target IP]}
    \item \textbf{Scan Date:} \textbf{[Scan Date]}
\end{itemize}

The scan revealed one open port, detailed in Table \ref{tab:scan_results}.

\begin{table}[h!]
\centering
\begin{tabular}{@{}llll@{}}
\toprule
\textbf{Port} & \textbf{State} & \textbf{Service} & \textbf{Product / Version} \\ \midrule
22/tcp & open & ssh (inferred) & Not Determined \\ \bottomrule
\end{tabular}
\caption{Open Ports Detected on \texttt{[Target IP]}.}
\label{tab:scan_results}
\end{table}

\subsection{Analysis of Findings}
The scan identified that port 22, commonly used for the Secure Shell (SSH) protocol, is open to the public internet. SSH is a critical tool for remote system administration. However, a publicly exposed SSH service is a common target for brute-force attacks and exploitation of potential vulnerabilities. The scan was unable to fingerprint the specific version of the SSH server, which prevents a direct check for known vulnerabilities. Nevertheless, its exposure constitutes a security risk that must be managed.

% --- Section 5: Risk Assessment ---
\section{Risk Assessment}

This section correlates the findings from the security control review and the technical scan. No pre-existing vulnerabilities were reported for consideration. The identified risks are summarized in Table \ref{tab:risk_summary}.

\begin{table}[h!]
\centering
\begin{tabular}{@{}lp{0.5\linewidth}ll@{}}
\toprule
\textbf{ID} & \textbf{Risk Description} & \textbf{Source} & \textbf{Severity} \\ \midrule
RISK-001 & Lack of an Employee Acceptable Use Policy weakens governance and increases insider risk. & Questionnaire & \textbf{High} \\
RISK-002 & Exposed SSH service on the external perimeter provides a potential vector for unauthorized access. & Network Scan & \textbf{Medium} \\ \bottomrule
\end{tabular}
\caption{Summary of Identified Risks.}
\label{tab:risk_summary}
\end{table}

% --- Section 6: Recommendations ---
\section{Recommendations}

The following actionable recommendations are provided to address the risks identified in this assessment.

\subsection{RISK-001: Develop and Implement an Acceptable Use Policy}
\begin{itemize}
    \item \textbf{Action:} Draft, approve, and implement a comprehensive Employee Acceptable Use Policy (AUP). This policy should clearly define the acceptable and unacceptable uses of all company IT assets, including computers, networks, email, and internet access.
    \item \textbf{Details:} The policy should be distributed to all current employees for acknowledgement and integrated into the onboarding process for new hires. Regular reviews and updates to the AUP should be scheduled.
    \item \textbf{Impact:} Reduces ambiguity regarding employee responsibilities, mitigates insider threats, and provides a clear framework for disciplinary action in case of policy violations.
\end{itemize}

\subsection{RISK-002: Secure the Exposed SSH Service}
\begin{itemize}
    \item \textbf{Action:} Review the business necessity for the publicly exposed SSH service. If it is not essential, the firewall rule allowing access should be removed.
    \item \textbf{Details:} If remote access via SSH is required, implement the following hardening measures:
    \begin{enumerate}
        \item \textbf{Restrict Access:} Limit SSH access to a whitelist of trusted IP addresses (e.g., corporate offices, administrator VPNs).
        \item \textbf{Disable Password Authentication:} Enforce the use of public key authentication exclusively, which is significantly more resistant to brute-force attacks.
        \item \textbf{Disable Root Login:} Prohibit direct login for the `root` user over SSH. Administrators should log in with a standard user account and elevate privileges as needed.
        \item \textbf{Implement Intrusion Prevention:} Deploy a tool such as \texttt{fail2ban} to automatically block IP addresses that exhibit malicious behavior, such as repeated failed login attempts.
    \end{enumerate}
    \item \textbf{Impact:} Significantly reduces the attack surface of the network perimeter and protects a critical administrative access point from common automated attacks.
\end{itemize}

\end{document}
```