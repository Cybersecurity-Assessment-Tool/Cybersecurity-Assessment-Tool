```latex
\documentclass[12pt]{article}

% Preamble: Required Packages
\usepackage[margin=1in]{geometry}
\usepackage{pifont} % For checkmarks (\ding{51}) and crosses (\ding{55})
\usepackage{booktabs} % For professional-looking tables
\usepackage{hyperref} % For creating hyperlinks in the document
\usepackage{url}      % For formatting URLs
\usepackage{seqsplit} % For splitting long strings in \texttt
\usepackage{xcolor}   % For colors

% Hyperref Setup
\hypersetup{
    colorlinks=true,
    linkcolor=blue,
    filecolor=magenta,
    urlcolor=cyan,
    pdftitle={Cybersecurity Posture Assessment Report},
    pdfauthor={Cybersecurity Analyst},
}

% Document Start
\begin{document}

% Title Page
\title{Cybersecurity Posture Assessment Report \\ \large For \textbf{[Organization Name]}}
\author{Cybersecurity Analyst}
\date{\today}
\maketitle

\tableofcontents
\newpage

% --- Executive Summary ---
\section{Executive Summary}

This report details the findings of a cybersecurity posture assessment for \textbf{[Organization Name]}. The assessment incorporated a review of organizational security controls, an external network vulnerability scan, and an analysis of pre-existing risks.

The key findings indicate a mixed security posture. While the organization has implemented some positive security controls, such as requiring Multi-Factor Authentication (MFA) for computer and sensitive system access, critical gaps were identified that expose the organization to significant risk.

The most severe risks stem from procedural and policy-based deficiencies:
\begin{itemize}
    \item \textbf{Critical Risk:} The absence of mandatory MFA for email access. This represents a primary vulnerability, as email accounts are high-value targets for attackers aiming to conduct phishing, business email compromise (BEC), and account takeover attacks.
    \item \textbf{High Risk:} A complete lack of a security awareness training program for both new and existing employees. This significantly increases the organization's susceptibility to social engineering attacks, which are a leading cause of security breaches.
\end{itemize}

On a technical level, the external network scan of the target IP address (\texttt{[Target IP]}) did not reveal any open ports or exposed services. This suggests a strong network perimeter configuration for the asset under test. However, this positive finding does not mitigate the severe organizational risks identified.

This report concludes with actionable recommendations to address each identified risk, prioritizing the immediate implementation of MFA for email and the development of a comprehensive security awareness training program.

% --- Organizational Information ---
\section{Organizational Information}
This section provides the details of the organization as understood for this assessment. The data has been anonymized as per the engagement requirements.

\begin{itemize}
    \item \textbf{Organization Name:} \textbf{[Organization Name]}
    \item \textbf{Primary Email Domain:} \texttt{[Domain]}
    \item \textbf{Known External IP Address:} \texttt{[Client IP]}
\end{itemize}

% --- Security Control Review ---
\section{Security Control Review}
The following table summarizes the organization's responses to a security controls questionnaire. A checkmark (\ding{51}) indicates a positive control is in place, while a cross (\ding{55}) indicates a control gap that introduces risk.

\begin{table}[h!]
\centering
\caption{Organizational Security Controls Questionnaire}
\label{tab:controls}
\begin{tabular}{p{0.7\textwidth} c}
\toprule
\textbf{Control Question} & \textbf{Response} \\
\midrule
Do you require MFA to access email? & \textcolor{red}{\ding{55}} \\
Do you require MFA to log into computers? & \textcolor{green}{\ding{51}} \\
Do you require MFA to access sensitive data systems? & \textcolor{green}{\ding{51}} \\
Does your organization have an employee acceptable use policy? & \textcolor{green}{\ding{51}} \\
Does your organization do security awareness training for new employees? & \textcolor{red}{\ding{55}} \\
Does your organization do security awareness training for all employees at least once per year? & \textcolor{red}{\ding{55}} \\
\bottomrule
\end{tabular}
\end{table}

The identified gaps in email security and employee training are the primary sources of risk discovered during this assessment.

% --- Technical Scan Results ---
\section{Technical Scan Results}
An external network vulnerability scan was conducted against the designated target system to identify potential weaknesses visible from the public internet.

\begin{itemize}
    \item \textbf{Target IP Address:} \texttt{[Target IP]}
    \item \textbf{Scan Type:} External Port Scan (TCP/UDP)
\end{itemize}

\textbf{Findings:} The scan completed successfully and \textbf{did not identify any open ports or exposed services} on the target system. This is a positive security finding, indicating that the network firewall and perimeter security controls are effectively configured to prevent unauthorized external access to this specific asset.

% --- Identified Risks and Vulnerabilities ---
\section{Risk Assessment}
This section synthesizes findings from the security control review and technical scan. As no pre-existing vulnerabilities were reported and no technical vulnerabilities were discovered, the following risks are derived directly from the identified organizational control gaps.

\begin{table}[h!]
\centering
\caption{Summary of Identified Risks}
\label{tab:risks}
\begin{tabular}{p{0.1\textwidth} p{0.25\textwidth} p{0.45\textwidth} c}
\toprule
\textbf{Risk ID} & \textbf{Risk Name} & \textbf{Description} & \textbf{Severity} \\
\midrule
RISK-001 & Lack of MFA for Email Access & The absence of MFA on email accounts allows an attacker with compromised credentials (e.g., from a phishing attack or password reuse) to gain full access to an employee's mailbox, leading to data breaches, internal phishing, and BEC. & \textbf{Critical} \\
\addlinespace
RISK-002 & Inadequate Security Awareness Training & The organization does not provide security training to new or existing employees. This results in a workforce that is ill-equipped to identify and report security threats like phishing, malware, and social engineering. & \textbf{High} \\
\bottomrule
\end{tabular}
\end{table}

% --- Recommendations ---
\section{Recommendations}
The following actions are recommended to mitigate the identified risks and improve the overall security posture of \textbf{[Organization Name]}.

\subsection{RISK-001: Implement MFA for Email (Critical)}
\textbf{Action:} Immediately enforce mandatory Multi-Factor Authentication (MFA) for all user accounts, including administrative, service, and standard user accounts, across the organization's email platform (e.g., Microsoft 365, Google Workspace).
\begin{itemize}
    \item \textbf{Justification:} MFA is one of the most effective controls to prevent account takeovers. Even if a user's password is stolen, MFA prevents the attacker from accessing the account without the second factor (e.g., an authenticator app, hardware token, or SMS code).
    \item \textbf{Priority:} Immediate.
\end{itemize}

\subsection{RISK-002: Establish Security Awareness Training Program (High)}
\textbf{Action:} Develop and implement a formal security awareness training program.
\begin{itemize}
    \item \textbf{Onboarding:} All new employees must complete a foundational security awareness training module as part of their onboarding process.
    \item \textbf{Annual Training:} All employees must complete a refresher training course at least once per year to stay informed about evolving threats.
    \item \textbf{Phishing Simulations:} Conduct regular, simulated phishing campaigns to test employee awareness and provide targeted, just-in-time training to those who click on malicious links.
    \item \textbf{Justification:} A well-trained workforce is a critical layer of defense. Training reduces the likelihood of successful social engineering attacks, turning employees from potential victims into an active part of the organization's security apparatus.
    \item \textbf{Priority:} High.
\end{itemize}

\end{document}
```