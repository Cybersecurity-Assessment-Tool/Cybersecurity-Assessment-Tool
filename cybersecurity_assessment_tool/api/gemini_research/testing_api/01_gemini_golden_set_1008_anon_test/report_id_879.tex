```latex
\documentclass[12pt]{article}

% Preamble: Required Packages
\usepackage[utf8]{inputenc}
\usepackage[a4paper, margin=1in]{geometry}
\usepackage{pifont} % For checkmarks and crosses (\ding{51}, \ding{55})
\usepackage{booktabs} % For professional-looking tables
\usepackage{hyperref} % For hyperlinks and document metadata
\usepackage{url}      % For typesetting URLs
\usepackage{seqsplit} % To split long strings in texttt
\usepackage{xcolor}   % For colored text
\usepackage{graphicx} % For potential logos (though none used here)

% --- Custom Commands ---
% For questionnaire responses
\newcommand{\yes}{\textcolor{green}{\ding{51}}}
\newcommand{\no}{\textcolor{red}{\ding{55}}}

% For risk severity levels
\newcommand{\riskcritical}{\textcolor{red!80!black}{\textbf{Critical}}}
\newcommand{\riskhigh}{\textcolor{orange!90!black}{\textbf{High}}}
\newcommand{\riskmedium}{\textcolor{yellow!90!black}{\textbf{Medium}}}
\newcommand{\risklow}{\textcolor{green!70!black}{\textbf{Low}}}

% --- Hyperref Setup ---
\hypersetup{
    colorlinks=true,
    linkcolor=blue!50!black,
    citecolor=green!50!black,
    urlcolor=cyan!70!black,
    pdftitle={Cybersecurity Assessment Report},
    pdfauthor={Cybersecurity Analyst},
    pdfsubject={Security Assessment for [Organization Name]},
    pdfkeywords={security, assessment, report, nmap, risk}
}

% --- Document Title ---
\title{
    \textbf{Cybersecurity Assessment Report} \\
    \vspace{1cm}
    \large For: \textbf{[Organization Name]}
}
\author{Cybersecurity Analyst}
\date{\today}

% --- Document Body ---
\begin{document}

\maketitle
\thispagestyle{empty}
\newpage

\tableofcontents
\newpage

% ==============================================================================
\section{Executive Summary}
% ==============================================================================

This report details the findings of a cybersecurity assessment conducted for \textbf{[Organization Name]}. The assessment combined a review of organizational security controls, an external network scan, and an analysis of pre-existing risk data.

A \riskcritical{} risk was identified: an externally exposed MySQL database (\texttt{5.7.33}) on port 3306. This version is past its End-of-Life (EOL) as of October 2023, meaning it no longer receives security updates and is highly susceptible to known exploits. Direct exposure of a database to the public internet is a severe security risk that can lead to data breaches, ransomware, or complete system compromise.

Additionally, a \riskhigh{} administrative gap was noted: the absence of a formal employee acceptable use policy. This exposes the organization to insider threats and inconsistent security practices.

On a positive note, the organization has demonstrated a strong commitment to identity security by implementing Multi-Factor Authentication (MFA) across email, computer logins, and sensitive systems. A robust security awareness training program is also in place.

Immediate remediation should focus on isolating the exposed database from the internet and developing a plan to upgrade the outdated software.

% ==============================================================================
\section{Organizational Information}
% ==============================================================================

The following information was used as the basis for this assessment. Anonymized data is indicated by placeholders.

\begin{itemize}
    \item \textbf{Organization Name:} \textbf{[Organization Name]}
    \item \textbf{Primary Domain:} \seqsplit{\texttt{[Domain]}}
    \item \textbf{Target IP Address:} \seqsplit{\texttt{[Client IP]}}
\end{itemize}

% ==============================================================================
\section{Security Control Review}
% ==============================================================================

A review of administrative and technical security controls was conducted via a questionnaire. The results are summarized in Table \ref{tab:controls}.

\begin{table}[h!]
    \centering
    \caption{Organizational Security Control Questionnaire}
    \label{tab:controls}
    \begin{tabular}{p{0.75\linewidth} c}
        \toprule
        \textbf{Control Question} & \textbf{Response} \\
        \midrule
        Do you require MFA to access email? & \yes \\
        Do you require MFA to log into computers? & \yes \\
        Do you require MFA to access sensitive data systems? & \yes \\
        Does your organization have an employee acceptable use policy? & \no \\
        Does your organization do security awareness training for new employees? & \yes \\
        Does your organization do security awareness training for all employees at least once per year? & \yes \\
        \bottomrule
    \end{tabular}
\end{table}

\subsection*{Analysis}
The organization has successfully implemented critical MFA controls across key access points, significantly reducing the risk of unauthorized access via compromised credentials. The security awareness training program is comprehensive.

However, the lack of an employee acceptable use policy is a significant gap. This policy is a foundational administrative control that sets clear expectations for employee behavior, data handling, and use of company assets, thereby reducing insider risk and providing a basis for enforcement.

% ==============================================================================
\section{Technical Scan Results}
% ==============================================================================

An external network scan was performed to identify exposed services and potential vulnerabilities.

\begin{itemize}
    \item \textbf{Target IP Scanned:} \seqsplit{\texttt{[Target IP]}}
    \item \textbf{Scan Date:} \today
\end{itemize}

\subsection*{Open Ports Discovered}
The scan identified one open port, detailed in Table \ref{tab:ports}.

\begin{table}[h!]
    \centering
    \caption{Open Port Findings}
    \label{tab:ports}
    \begin{tabular}{l l l l}
        \toprule
        \textbf{Port} & \textbf{State} & \textbf{Service} & \textbf{Product / Version} \\
        \midrule
        3306/tcp & open & mysql & MySQL 5.7.33 \\
        \bottomrule
    \end{tabular}
\end{table}

\subsection*{Analysis and Observations}
The technical scan confirms the pre-existing risk "Database Exposure". The finding is more severe than initially documented due to two key factors:

\begin{enumerate}
    \item \textbf{Direct Public Exposure:} Port 3306 (MySQL) is open to the public internet. This allows any attacker on the internet to attempt to connect, brute-force credentials, or exploit vulnerabilities in the database service.
    \item \textbf{End-of-Life Software:} The detected version, MySQL 5.7.33, reached its official End-of-Life (EOL) in October 2023. EOL software no longer receives security patches from the vendor, making it a prime target for attackers who can leverage publicly known vulnerabilities without fear of them being patched.
\end{enumerate}

This combination represents a critical and immediate threat to the confidentiality, integrity, and availability of the data stored within this database.

% ==============================================================================
\section{Synthesized Risk Assessment}
% ==============================================================================

The following table synthesizes findings from the security control review, technical scan, and pre-existing risk data into a prioritized list of current risks.

\begin{table}[h!]
    \centering
    \caption{Summary of Identified Risks}
    \label{tab:risks}
    \begin{tabular}{p{0.1\linewidth} p{0.25\linewidth} p{0.35\linewidth} p{0.15\linewidth}}
        \toprule
        \textbf{ID} & \textbf{Risk Name} & \textbf{Description} & \textbf{Severity} \\
        \midrule
        RISK-001 & Exposed \& Outdated Database Service & A MySQL 5.7.33 database is publicly exposed on port 3306. This version is End-of-Life and vulnerable to known exploits. & \riskcritical{} \\
        \addlinespace
        RISK-002 & Lack of Acceptable Use Policy & The organization lacks a formal policy governing the acceptable use of IT assets, increasing insider risk and legal liability. & \riskhigh{} \\
        \bottomrule
    \end{tabular}
\end{table}

% ==============================================================================
\section{Recommendations}
% ==============================================================================

The following actions are recommended to mitigate the identified risks.

\subsection*{RISK-001: Exposed \& Outdated Database Service (\riskcritical{})}
\begin{itemize}
    \item \textbf{Immediate (Containment):}
    \begin{itemize}
        \item Apply firewall rules to immediately block all public inbound access to TCP port 3306 on \seqsplit{\texttt{[Target IP]}}. Access should only be permitted from trusted internal IP addresses.
    \end{itemize}
    \item \textbf{Short-Term (Remediation):}
    \begin{itemize}
        \item Develop and execute a plan to upgrade the MySQL 5.7 database to a currently supported version (e.g., MySQL 8.x).
        \item Perform a vulnerability scan on the database server after the upgrade to ensure no other critical issues exist.
    \end{itemize}
    \item \textbf{Long-Term (Strategic):}
    \begin{itemize}
        \item Implement a secure access solution, such as a Virtual Private Network (VPN) or a bastion host, for all future database administration and access. Avoid exposing database ports directly to the internet.
    \end{itemize}
\end{itemize}

\subsection*{RISK-002: Lack of Acceptable Use Policy (\riskhigh{})}
\begin{itemize}
    \item \textbf{Short-Term (Remediation):}
    \begin{itemize}
        \item Draft, approve, and implement a comprehensive Employee Acceptable Use Policy (AUP).
        \item Communicate the new policy to all existing employees and incorporate it into the onboarding process for new hires.
        \item Require all employees to formally acknowledge they have read and understood the policy.
    \end{itemize}
\end{itemize}

\end{document}
```