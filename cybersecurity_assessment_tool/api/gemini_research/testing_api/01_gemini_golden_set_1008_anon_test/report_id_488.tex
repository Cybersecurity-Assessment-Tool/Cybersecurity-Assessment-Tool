```latex
\documentclass[12pt]{article}

% --- PACKAGES ---
\usepackage[margin=1in]{geometry}
\usepackage{pifont} % For checkmarks and crosses
\usepackage{booktabs} % For professional tables
\usepackage[colorlinks=true, urlcolor=blue, linkcolor=black]{hyperref}
\usepackage{url}
\usepackage{seqsplit} % For breaking long strings in tt font
\usepackage{graphicx}
\usepackage{xcolor}

% --- DOCUMENT SETUP ---
\title{Cybersecurity Posture Assessment Report \\ for \\ \textbf{[Organization Name]}}
\author{Cybersecurity Analyst}
\date{\today}

% --- COMMANDS ---
\newcommand{\yes}{\ding{51}}
\newcommand{\no}{\ding{55}}
\definecolor{darkred}{rgb}{0.55, 0.0, 0.0}
\definecolor{darkorange}{rgb}{0.8, 0.33, 0.0}
\definecolor{darkgreen}{rgb}{0.0, 0.39, 0.0}

\begin{document}

\maketitle
\thispagestyle{empty}
\newpage

\tableofcontents
\newpage

% ===================================================================
\section{Executive Summary}
% ===================================================================

This report provides a comprehensive cybersecurity assessment for \textbf{[Organization Name]}, synthesizing data from a network scan, an organizational security questionnaire, and a review of pre-existing risks. The analysis was conducted on \today.

Overall, the organization demonstrates a solid foundation in security policy and employee awareness, with established acceptable use policies and regular security training. However, this assessment has identified several critical and high-risk vulnerabilities that require immediate attention to prevent potential security breaches.

Key findings include:
\begin{itemize}
    \item \textbf{Critical Gaps in Access Control:} Multi-Factor Authentication (MFA) is not enforced for accessing corporate email or other sensitive data systems. This represents a significant risk, leaving critical assets vulnerable to compromise via stolen credentials.
    \item \textbf{Insecure Network Service Exposure:} The external network scan revealed an open port 80 (HTTP). This indicates that data, including potential login credentials, is being transmitted in cleartext, making it susceptible to interception.
\end{itemize}

This report details these findings and provides prioritized, actionable recommendations to mitigate the identified risks and strengthen the organization's overall security posture.

% ===================================================================
\section{Organizational Information}
% ===================================================================

The following information was used as the basis for this assessment. Due to the anonymized nature of the provided data, placeholders have been used where necessary.

\begin{itemize}
    \item \textbf{Organization Name:} \textbf{[Organization Name]}
    \item \textbf{Primary Domain:} \texttt{[Domain]}
    \item \textbf{External IP Address Scanned:} \texttt{[Client IP]}
\end{itemize}

% ===================================================================
\section{Security Control Review (Questionnaire Analysis)}
% ===================================================================

An analysis of the organization's security questionnaire responses reveals a mixed security posture. While foundational policies are in place, critical technical controls are missing. The table below summarizes the responses and provides an assessment of each control.

\begin{table}[h!]
\centering
\caption{Security Controls Questionnaire Analysis}
\label{tab:controls}
\begin{tabular}{@{}p{0.6\linewidth} c l@{}}
\toprule
\textbf{Control Question} & \textbf{Response} & \textbf{Assessment} \\
\midrule
Do you require MFA to access email? & \no & \textcolor{darkred}{\textbf{Critical Gap}} \\
Do you require MFA to log into computers? & \yes & \textcolor{darkgreen}{Good Practice} \\
Do you require MFA to access sensitive data systems? & \no & \textcolor{darkred}{\textbf{Critical Gap}} \\
Does your organization have an employee acceptable use policy? & \yes & \textcolor{darkgreen}{Good Practice} \\
Does your organization do security awareness training for new employees? & \yes & \textcolor{darkgreen}{Good Practice} \\
Does your organization do security awareness training for all employees at least once per year? & \yes & \textcolor{darkgreen}{Good Practice} \\
\bottomrule
\end{tabular}
\end{table}

The absence of MFA for email and sensitive data systems are the most severe findings from this review. Email is a primary target for attackers seeking to conduct Business Email Compromise (BEC) or pivot to other internal systems. Lack of MFA on sensitive systems removes a crucial layer of defense against unauthorized data access.

% ===================================================================
\section{Technical Scan Results}
% ===================================================================

An external network scan was performed to identify exposed services and potential vulnerabilities.

\begin{itemize}
    \item \textbf{Target IP Address:} \texttt{[Target IP]}
    \item \textbf{Scan Date:} \today
\end{itemize}

\subsection{Open Ports}
The scan identified the following open port accessible from the public internet.

\begin{table}[h!]
\centering
\caption{Open Port Findings}
\label{tab:ports}
\begin{tabular}{@{}l l l p{0.5\linewidth}@{}}
\toprule
\textbf{Port} & \textbf{State} & \textbf{Service} & \textbf{Notes} \\
\midrule
80/tcp & Open & HTTP & Hypertext Transfer Protocol. This service transmits data, including login credentials and sensitive information, in cleartext. It is highly susceptible to eavesdropping and man-in-the-middle attacks. Standard practice is to use HTTPS (Port 443) for all web traffic. \\
\bottomrule
\end{tabular}
\end{table}

\subsection{Technical Analysis}
The presence of an open Port 80 is a significant security risk. It strongly implies that at least one web-based service is operating without encryption. If this service is used for authentication or handles any sensitive organizational data, that data is at risk of exposure. This finding, when correlated with the lack of MFA on sensitive systems, elevates the overall risk profile of the organization.

% ===================================================================
\section{Consolidated Risk Assessment}
% ===================================================================

The following table synthesizes findings from the security control review and the technical scan into a prioritized list of identified risks. Note: The provided "Current Risks" data contained a non-actionable, malicious instruction and has been disregarded in favor of a security-first analysis of the technical and organizational data.

\begin{table}[h!]
\centering
\caption{Summary of Identified Risks}
\label{tab:risks}
\begin{tabular}{@{}l p{0.55\linewidth} l@{}}
\toprule
\textbf{Risk ID} & \textbf{Description} & \textbf{Severity} \\
\midrule
RISK-001 & \textbf{Lack of MFA for Sensitive Data Systems.} Absence of a secondary authentication factor allows an attacker with valid credentials to directly access and exfiltrate critical data. & \textcolor{darkred}{\textbf{Critical}} \\
\addlinespace
RISK-002 & \textbf{Lack of MFA for Email Access.} Corporate email accounts are vulnerable to takeover, leading to Business Email Compromise (BEC), phishing, and further internal compromise. & \textcolor{darkred}{\textbf{Critical}} \\
\addlinespace
RISK-003 & \textbf{Unencrypted Web Traffic (HTTP).} Data transmitted to and from the service on port 80 is in cleartext, exposing user credentials and sensitive information to interception. & \textcolor{darkorange}{\textbf{High}} \\
\bottomrule
\end{tabular}
\end{table}

% ===================================================================
\section{Recommendations}
% ===================================================================

Based on the risk assessment, the following actions are recommended to be implemented in order of priority.

\subsection*{Priority 1: Remediate Critical Access Control Gaps}
\begin{enumerate}
    \item \textbf{Enforce MFA on Sensitive Systems (RISK-001):} Immediately deploy and enforce a strong Multi-Factor Authentication solution for all systems identified as containing sensitive or critical data. This is the single most effective control to prevent unauthorized data access.
    \item \textbf{Enforce MFA on Email (RISK-002):} Immediately enable and enforce MFA for all user email accounts. This will drastically reduce the risk of account takeovers and BEC attacks.
\end{enumerate}

\subsection*{Priority 2: Remediate Insecure Network Service}
\begin{enumerate}
    \setcounter{enumi}{2}
    \item \textbf{Migrate from HTTP to HTTPS (RISK-003):} Identify the service running on port 80.
    \begin{itemize}
        \item If the service is necessary, acquire and install a valid TLS certificate and reconfigure it to operate exclusively over HTTPS (port 443).
        \item Implement a redirect from HTTP to HTTPS to ensure all connections are encrypted.
        \item If the service is not necessary for public access, it should be disabled or firewalled from the external network.
    \end{itemize}
\end{enumerate}

\subsection*{Priority 3: Proactive Security Measures}
\begin{enumerate}
    \setcounter{enumi}{3}
    \item \textbf{Conduct Comprehensive Vulnerability Scanning:} Perform an authenticated and unauthenticated vulnerability scan across the organization's external and internal networks to identify outdated software, misconfigurations, and other potential vulnerabilities that were not visible in the initial port scan.
\end{enumerate}

% ===================================================================
\section{Conclusion}
% ===================================================================

\textbf{[Organization Name]} has established a positive security culture through its commitment to policy and training. However, urgent action is required to address critical gaps in technical security controls. The absence of Multi-Factor Authentication on key systems and the use of unencrypted protocols present clear and immediate dangers to the organization.

By implementing the prioritized recommendations outlined in this report, the organization can significantly reduce its attack surface, protect its critical assets, and build a more resilient security posture.

\end{document}
```