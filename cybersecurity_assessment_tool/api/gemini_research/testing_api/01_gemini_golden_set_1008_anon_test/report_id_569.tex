```latex
\documentclass[12pt, a4paper]{article}

% Preamble: Required Packages
\usepackage[margin=1in]{geometry}
\usepackage{pifont} % For checkmarks and crosses
\usepackage{booktabs} % For professional tables
\usepackage{hyperref} % For hyperlinks
\usepackage{url} % For URL formatting
\usepackage{seqsplit} % For splitting long strings
\usepackage{xcolor} % For colors
\usepackage{graphicx} % For logo (placeholder)

% Document Information
\title{Cybersecurity Posture Assessment Report}
\author{Cybersecurity Analysis Division}
\date{\today}

% Hyperref Setup
\hypersetup{
    colorlinks=true,
    linkcolor=blue,
    filecolor=magenta,      
    urlcolor=cyan,
    pdftitle={Cybersecurity Posture Assessment Report},
    pdfpagemode=FullScreen,
}

\begin{document}

\begin{titlepage}
    \centering
    \vspace*{1cm}
    \Huge\textbf{Cybersecurity Posture Assessment Report}
    \vspace{1.5cm}
    \large
    \textbf{Prepared for:}\\
    \vspace{0.5cm}
    \textbf{[Organization Name]}
    \vspace{2.5cm}
    \textbf{Report Date:}\\
    \today
    \vfill
    \large
    \textit{This document contains sensitive information and is intended for the exclusive use of the recipient organization. Distribution is strictly prohibited.}
\end{titlepage}

\tableofcontents
\newpage

% --- 1. Executive Summary ---
\section{Executive Summary}
This report provides a comprehensive analysis of the cybersecurity posture for \textbf{[Organization Name]}, based on network scans, a security controls questionnaire, and a review of known risks.

The assessment identified a \textbf{critical risk}: the direct exposure of Remote Desktop Protocol (RDP) on port 3389 to the public internet at IP address \texttt{[Target IP]}. This vulnerability, with a CVSS score of 9.0, presents a direct and immediate path for attackers to gain unauthorized access to the internal network.

This technical vulnerability is severely compounded by significant gaps in organizational security controls. Key findings include:
\begin{itemize}
    \item \textbf{Lack of Multi-Factor Authentication (MFA):} MFA is not enforced for accessing email or sensitive data systems, drastically increasing the risk of account compromise.
    - \textbf{Absence of Foundational Policies:} The organization lacks a formal employee acceptable use policy.
    - \textbf{No Security Awareness Training:} There is no security training program for new or existing employees, leaving the organization highly susceptible to phishing and social engineering attacks.
\end{itemize}

Immediate remediation is required to address the exposed RDP service. Strategic initiatives must be launched to implement MFA and develop a foundational cybersecurity program including policies and training.

% --- 2. Organizational Information ---
\section{Organizational Information}
This section details the information provided by the client organization. The placeholders indicate that this data was not available at the time of the assessment.

\begin{tabular}{@{}ll}
    \toprule
    \textbf{Attribute} & \textbf{Value} \\
    \midrule
    Organization Name & \textbf{[Organization Name]} \\
    Primary Domain & \texttt{[Domain]} \\
    External IP Address & \texttt{[Client IP]} \\
    \bottomrule
\end{tabular}

% --- 3. Security Control Review ---
\section{Security Control Review (Questionnaire Analysis)}
The following table summarizes the organization's self-reported security controls. Answers marked with a red cross (\textcolor{red}{\ding{55}}) indicate a deviation from security best practices and represent a significant gap in the defensive posture.

\begin{table}[h!]
\centering
\begin{tabular}{p{0.7\linewidth} c}
    \toprule
    \textbf{Control Question} & \textbf{Status} \\
    \midrule
    Do you require MFA to access email? & \textcolor{red}{\ding{55}} \\
    Do you require MFA to log into computers? & \textcolor{green}{\ding{51}} \\
    Do you require MFA to access sensitive data systems? & \textcolor{red}{\ding{55}} \\
    Does your organization have an employee acceptable use policy? & \textcolor{red}{\ding{55}} \\
    Does your organization do security awareness training for new employees? & \textcolor{red}{\ding{55}} \\
    Does your organization do security awareness training for all employees at least once per year? & \textcolor{red}{\ding{55}} \\
    \bottomrule
\end{tabular}
\caption{Security Controls Questionnaire Results}
\end{table}

\paragraph{Analysis:}
The lack of MFA for email and sensitive systems is a critical weakness. Email is a primary target for attackers, and its compromise can lead to widespread data breaches. The complete absence of an acceptable use policy and security awareness training creates a high-risk environment where employees are more likely to engage in unsafe practices, fall victim to phishing, or mishandle sensitive data.

% --- 4. Technical Scan Results ---
\section{Technical Scan Results}
An external network scan was performed to identify exposed services. The scan revealed the following open port on the target system.

\begin{table}[h!]
\centering
\begin{tabular}{l l l l}
    \toprule
    \textbf{Target IP} & \textbf{Port/Protocol} & \textbf{State} & \textbf{Service} \\
    \midrule
    \texttt{[Target IP]} & 3389/tcp & Open & ms-wbt-server (RDP) \\
    \bottomrule
\end{tabular}
\caption{External Port Scan Findings}
\end{table}

\paragraph{Analysis:}
The scan confirms that Microsoft Remote Desktop Protocol (RDP) is open to the public internet. RDP is a primary target for ransomware gangs and other malicious actors who use brute-force attacks, credential stuffing, and exploits against unpatched RDP services to gain initial access to a network. This finding aligns with the pre-existing risk documented in the organization's risk register.

% --- 5. Consolidated Risk Assessment ---
\section{Consolidated Risk Assessment}
The following table synthesizes findings from the technical scan, control review, and existing risk data into a prioritized list of security risks.

\begin{table}[h!]
\centering
\begin{tabular}{p{0.25\linewidth} p{0.5\linewidth} l}
    \toprule
    \textbf{Risk / Finding} & \textbf{Description} & \textbf{Severity} \\
    \midrule
    \textbf{Public RDP Exposure} & Port 3389 is open to the internet, allowing attackers to attempt brute-force logins or exploit RDP vulnerabilities. & \textbf{Critical} \\
    \addlinespace
    \textbf{Lack of MFA on Critical Systems} & No MFA on email or sensitive data systems. A single stolen password could lead to a major breach. & \textbf{Critical} \\
    \addlinespace
    \textbf{No Security Policy or Training Program} & The absence of an AUP and security training leaves the organization vulnerable to human error, phishing, and insider threats. & \textbf{High} \\
    \bottomrule
\end{tabular}
\caption{Summary of Identified Risks}
\end{table}

% --- 6. Recommendations ---
\section{Recommendations}
The following actions are recommended to mitigate the identified risks. Recommendations are prioritized based on severity and potential impact.

\subsection{Immediate Actions (To Be Completed in < 24 Hours)}
\begin{enumerate}
    \item \textbf{Isolate RDP Service:} Immediately use a firewall to block all inbound traffic to TCP port 3389 on \texttt{[Target IP]} from the public internet. Access should only be permitted from trusted, internal IP addresses.
\end{enumerate}

\subsection{Short-Term Actions (To Be Completed in < 30 Days)}
\begin{enumerate}
    \item \textbf{Implement a VPN for Remote Access:} Replace direct RDP access with a secure Virtual Private Network (VPN) solution. All remote administrative access must be routed through the VPN.
    \item \textbf{Enforce MFA Everywhere:} Enable and enforce MFA for all users on all critical systems, especially:
    \begin{itemize}
        \item Email (e.g., Office 365, Google Workspace)
        \item VPN access
        \item Access to any system storing sensitive data
    \end{itemize}
\end{enumerate}

\subsection{Long-Term Strategic Actions (To Be Completed in < 90 Days)}
\begin{enumerate}
    \item \textbf{Develop an Acceptable Use Policy (AUP):} Create and implement a formal AUP that defines the rules for using company IT assets, data, and the network. All employees must read and acknowledge this policy.
    \item \textbf{Establish a Security Awareness Training Program:}
    \begin{itemize}
        \item Implement mandatory security training for all new hires during their onboarding process.
        \item Conduct annual security awareness training for all employees, covering topics such as phishing, password security, and data handling.
    \end{itemize}
\end{enumerate}

\end{document}
```