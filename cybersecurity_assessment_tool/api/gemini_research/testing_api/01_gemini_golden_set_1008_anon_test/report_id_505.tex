```latex
\documentclass[12pt]{article}

% Preamble: Required Packages
\usepackage[margin=1in]{geometry}
\usepackage{pifont} % For checkmarks and crosses
\usepackage{booktabs} % For professional tables
\usepackage[colorlinks=true, urlcolor=blue, linkcolor=black]{hyperref}
\usepackage{url}
\usepackage{seqsplit} % For splitting long strings in tt font
\usepackage{graphicx}
\usepackage{fancyhdr}

% --- Document Setup ---
\pagestyle{fancy}
\fancyhf{} % clear all header and footers
\renewcommand{\headrulewidth}{0.4pt}
\renewcommand{\footrulewidth}{0.4pt}
\fancyhead[L]{Cybersecurity Posture Assessment}
\fancyhead[R]{\textbf{[Organization Name]}}
\fancyfoot[C]{\thepage}

% --- Hyperref Setup ---
\hypersetup{
    pdftitle={Cybersecurity Posture Assessment Report},
    pdfauthor={Cybersecurity Analyst},
    pdfsubject={Security Analysis},
    pdfkeywords={Cybersecurity, Nmap, Risk Assessment},
    bookmarks=true
}

% --- Document Start ---
\begin{document}

% --- Title Page ---
\begin{titlepage}
    \centering
    \vspace*{1cm}
    \includegraphics[width=0.4\textwidth]{example-image-a} % Placeholder for company logo
    
    \vspace{1.5cm}
    
    \Huge
    \textbf{Cybersecurity Posture Assessment Report}
    
    \vspace{1.5cm}
    
    \Large
    For: \textbf{[Organization Name]}
    
    \vspace{2cm}
    
    \normalsize
    \begin{tabular}{ll}
        \textbf{Date of Report:} & \today \\
        \textbf{Date of Assessment:} & November 22, 2025 \\
        \textbf{Report ID:} & SEC-2025-11-22-001 \\
    \end{tabular}
    
    \vfill
    
    \normalsize
    \textit{This document contains sensitive information and is intended for the exclusive use of \textbf{[Organization Name]}. Distribution without prior written consent is prohibited.}
    
\end{titlepage}

\tableofcontents
\newpage

% --- Section 1: Executive Summary ---
\section{Executive Summary}

This report details the findings of a cybersecurity posture assessment conducted for \textbf{[Organization Name]} on November 22, 2025. The assessment combined a review of organizational security controls, an external network vulnerability scan, and an analysis of pre-existing risks.

The assessment identified several \textbf{critical and high-risk vulnerabilities} that require immediate attention. The overall security posture is considered weak due to significant gaps in fundamental security controls.

Key findings include:
\begin{itemize}
    \item \textbf{Critical Control Gaps:} The organization lacks Multi-Factor Authentication (MFA) for email access, a formal Acceptable Use Policy (AUP), and a security awareness training program for its employees. These omissions create a high susceptibility to phishing, business email compromise, and insider threats.
    \item \textbf{Vulnerable External Infrastructure:} The external network scan identified a web server running Nginx version 1.18.0. This version is outdated, no longer supported, and has multiple publicly disclosed vulnerabilities, exposing the organization to potential compromise.
    \item \textbf{Compounded Risk:} The combination of untrained users and a vulnerable external-facing server creates a significantly elevated risk profile. An attacker could easily exploit the human factor via a phishing attack to gain a foothold and subsequently attack the vulnerable infrastructure.
\end{itemize}

This report provides a detailed analysis of these findings and outlines actionable recommendations to mitigate the identified risks and strengthen the organization's overall security posture. We urge management to prioritize the remediation steps outlined in Section 6.

% --- Section 2: Organizational Information ---
\section{Organizational Information}

This section provides the organizational details used as the basis for this assessment. Due to the anonymized nature of the provided data, placeholders have been used where necessary.

\begin{tabular}{@{}ll}
    \toprule
    \textbf{Attribute} & \textbf{Value} \\
    \midrule
    Organization Name & \textbf{[Organization Name]} \\
    Primary Email Domain & \texttt{[Domain]} \\
    External IP Address Scanned & \texttt{[Client IP]} \\
    Target IP Address Analyzed & \texttt{[Target IP]} \\
    \bottomrule
\end{tabular}

% --- Section 3: Security Control Review ---
\section{Security Control Review}

A security questionnaire was completed to evaluate the implementation of essential administrative and technical controls. The responses are summarized below. A checkmark (\ding{51}) indicates a positive control, while a cross (\ding{55}) indicates a control gap.

\begin{table}[h!]
\centering
\caption{Security Controls Questionnaire Results}
\begin{tabular}{p{0.8\textwidth}c}
    \toprule
    \textbf{Control Question} & \textbf{Response} \\
    \midrule
    Do you require MFA to access email? & \ding{55} \\
    Do you require MFA to log into computers? & \ding{51} \\
    Do you require MFA to access sensitive data systems? & \ding{51} \\
    Does your organization have an employee acceptable use policy? & \ding{55} \\
    Does your organization do security awareness training for new employees? & \ding{55} \\
    Does your organization do security awareness training for all employees at least once per year? & \ding{55} \\
    \bottomrule
\end{tabular}
\end{table}

\subsection{Analysis of Control Gaps}
The review reveals critical deficiencies in foundational security practices:
\begin{itemize}
    \item \textbf{No MFA for Email:} Email is the primary target for phishing and account takeover attacks. The absence of MFA on this critical service represents a \textbf{critical risk}. A compromised email account can lead to data breaches, financial fraud, and further infiltration of the network.
    \item \textbf{No Acceptable Use Policy (AUP):} Without a formal AUP, there are no clear guidelines for employees on the secure and acceptable use of company assets. This ambiguity increases the risk of unintentional data exposure and misuse of resources.
    \item \textbf{No Security Awareness Training:} The complete lack of a security awareness training program for both new and existing employees is a major vulnerability. Employees are the first line of defense; without training, they are significantly more likely to fall victim to social engineering attacks like phishing, which could compromise the entire organization.
\end{itemize}

% --- Section 4: Technical Scan Results ---
\section{Technical Scan Results}

An external network scan was performed using Nmap on November 22, 2025, against the target IP address \texttt{[Target IP]}. The scan identified the following open ports and services.

\begin{table}[h!]
\centering
\caption{Open Port Scan Results for \texttt{[Target IP]}}
\begin{tabular}{lllll}
    \toprule
    \textbf{Port} & \textbf{State} & \textbf{Service} & \textbf{Product} & \textbf{Version} \\
    \midrule
    443/tcp & open & https & nginx & 1.18.0 \\
    \bottomrule
\end{tabular}
\end{table}

\subsection{Analysis of Technical Findings}
The scan revealed one open port, 443 (HTTPS), running the Nginx web server.

\textbf{High Risk Finding:} The detected version, \textbf{Nginx 1.18.0}, was released in April 2020 and reached its End of Life (EOL) in May 2022. Running EOL software is extremely dangerous as it no longer receives security patches from the vendor. This specific version is known to be vulnerable to several security issues, including but not limited to:
\begin{itemize}
    \item \textbf{HTTP Request Smuggling (e.g., CVE-2021-43556):} Vulnerabilities that can allow an attacker to bypass security controls, poison web caches, and compromise user sessions.
    \item Other vulnerabilities discovered in subsequent versions that are not patched in 1.18.0.
\end{itemize}
An attacker could exploit these known vulnerabilities to gain unauthorized access to the server, deface the website, or exfiltrate sensitive data.

% --- Section 5: Risk Assessment ---
\section{Risk Assessment}

This section synthesizes the findings from the security control review and the technical scan. No pre-existing risks were provided, so the following risks have been newly identified during this assessment.

\begin{table}[h!]
\centering
\caption{Summary of Identified Risks}
\begin{tabular}{p{0.1\linewidth} p{0.3\linewidth} p{0.4\linewidth} p{0.1\linewidth}}
    \toprule
    \textbf{Risk ID} & \textbf{Risk Title} & \textbf{Description} & \textbf{Severity} \\
    \midrule
    \textbf{RISK-001} & Lack of MFA on Email & The absence of MFA on the primary communication platform exposes the organization to a high likelihood of account takeover, leading to data breaches and financial loss. & \textbf{Critical} \\
    \noalign{\vspace{2mm}}
    \textbf{RISK-002} & Outdated and Vulnerable Web Server & The public-facing web server is running an End-of-Life version of Nginx with known, exploitable vulnerabilities. This could lead to a full system compromise. & \textbf{High} \\
    \noalign{\vspace{2mm}}
    \textbf{RISK-003} & Absence of Security Awareness Program & With no AUP or security training, employees are unprepared to identify and respond to threats like phishing, making them the weakest link in the security chain. This directly elevates the threat level of RISK-001 and RISK-002. & \textbf{High} \\
    \bottomrule
\end{tabular}
\end{table}

% --- Section 6: Recommendations ---
\section{Recommendations}

Based on the identified risks, we recommend the following remediation actions, prioritized by severity.

\subsection{Immediate Priority (Remediate within 7 days)}
\begin{enumerate}
    \item \textbf{Implement MFA for Email (Mitigates RISK-001):} Immediately enforce mandatory MFA for all user access to the email system (\texttt{[Domain]}). This is the single most effective control to prevent email account takeovers.
\end{enumerate}

\subsection{High Priority (Remediate within 30 days)}
\begin{enumerate}
    \setcounter{enumi}{1}
    \item \textbf{Patch Web Server Vulnerability (Mitigates RISK-002):} Develop and execute a patch management plan to upgrade the Nginx server from version 1.18.0 to a current, stable, and supported version. This plan should include testing to ensure application compatibility before deploying to production.
    \item \textbf{Establish a Security Awareness Program (Mitigates RISK-003):}
        \begin{itemize}
            \item Draft and ratify a formal Acceptable Use Policy (AUP) and ensure all employees read and acknowledge it.
            \item Procure and deploy a security awareness training module for all employees, with a specialized onboarding module for new hires. Conduct regular phishing simulations to reinforce training.
        \end{itemize}
\end{enumerate}

\subsection{Ongoing Recommendations}
\begin{enumerate}
    \setcounter{enumi}{3}
    \item \textbf{Implement a Vulnerability Management Program:} Establish a formal process for regularly scanning internal and external assets for vulnerabilities. This program should include policies for timely patch deployment based on vulnerability severity.
    \item \textbf{Conduct Regular Security Assessments:} Perform periodic security assessments (at least annually) to identify new risks and validate the effectiveness of existing security controls.
\end{enumerate}

\end{document}
```