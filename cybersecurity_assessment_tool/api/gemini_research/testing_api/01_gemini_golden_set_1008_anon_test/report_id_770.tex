```latex
\documentclass[12pt]{article}

% Preamble: Required Packages
\usepackage[margin=1in]{geometry} % For setting page margins
\usepackage{pifont}               % For dingbats like checkmarks and crosses
\usepackage{booktabs}             % For professional-looking tables
\usepackage{hyperref}             % For hyperlinks and document metadata
\usepackage{url}                  % For formatting URLs
\usepackage{seqsplit}             % For splitting long strings in texttt
\usepackage{graphicx}             % For including logos or images
\usepackage{xcolor}               % For custom colors

% Document Metadata
\hypersetup{
    colorlinks=true,
    linkcolor=blue,
    filecolor=magenta,
    urlcolor=cyan,
    pdftitle={Cybersecurity Posture Assessment Report},
    pdfauthor={Cybersecurity Analysis Division},
    pdfsubject={Security Assessment},
    pdfkeywords={Cybersecurity, Nmap, Risk, Assessment}
}

% Custom Commands for Readability
\newcommand{\yes}{\textcolor{green!70!black}{\ding{51}}} % Green checkmark
\newcommand{\no}{\textcolor{red!90!black}{\ding{55}}}    % Red cross

% --- DOCUMENT START ---
\begin{document}

% --- TITLE PAGE ---
\begin{titlepage}
    \centering
    \vspace*{1cm}
    \Huge\textbf{Cybersecurity Posture Assessment Report}
    \vspace{1.5cm}
    \Large
    \textbf{Prepared for:}\\
    \vspace{0.5cm}
    \textbf{[Organization Name]}
    \vspace{2cm}
    \textbf{Date of Report:}\\
    \vspace{0.5cm}
    \today
    \vfill
    \large
    \textbf{Generated by:}\\
    Cybersecurity Analysis Division
    \vspace{0.5cm}
    \textit{This report contains sensitive information and should be handled with care.}
\end{titlepage}

\tableofcontents
\newpage

% --- EXECUTIVE SUMMARY ---
\section{Executive Summary}
This report provides a comprehensive cybersecurity assessment for \textbf{[Organization Name]}, synthesizing findings from an external network scan, a review of internal security controls, and an analysis of pre-existing risks.

The assessment identified several critical and high-severity risks that require immediate attention. The most critical finding is a publicly exposed FTP server running a dangerously outdated and vulnerable version of \texttt{vsftpd} (2.3.4), which is known to contain a backdoor (CVE-2011-2523). This server also permits anonymous login, significantly increasing the risk of unauthorized access and data compromise.

Furthermore, organizational policies exhibit critical gaps. The lack of multi-factor authentication (MFA) on sensitive data systems and the absence of a formal security awareness training program for employees create a permissive environment for attackers. These policy gaps, combined with the technical vulnerabilities, result in a significantly elevated risk profile.

This report outlines detailed findings and provides actionable recommendations to mitigate these risks and improve the overall security posture of the organization.

% --- ORGANIZATIONAL INFORMATION ---
\section{Organizational Information}
This section details the information provided by the client for this assessment.
\begin{itemize}
    \item \textbf{Organization Name:} \textbf{[Organization Name]}
    \item \textbf{Primary Domain:} \texttt{[Domain]}
    \item \textbf{External IP Scanned:} \texttt{[Client IP]}
\end{itemize}

% --- SECURITY CONTROL REVIEW ---
\section{Security Control Review}
A review of the organization's security controls was conducted via a questionnaire. The results highlight key areas of strength and weakness in the current security policy framework. "No" answers indicate significant gaps that increase organizational risk.

\begin{center}
\begin{tabular}{p{0.8\textwidth}c}
    \toprule
    \textbf{Control Question} & \textbf{Status} \\
    \midrule
    Do you require MFA to access email? & \yes \\
    Do you require MFA to log into computers? & \yes \\
    Do you require MFA to access sensitive data systems? & \no \\
    \addlinespace
    Does your organization have an employee acceptable use policy? & \yes \\
    \addlinespace
    Does your organization do security awareness training for new employees? & \no \\
    Does your organization do security awareness training for all employees at least once per year? & \no \\
    \bottomrule
\end{tabular}
\end{center}

\subsection{Analysis of Control Gaps}
\begin{itemize}
    \item \textbf{MFA for Sensitive Systems:} The absence of MFA on sensitive data systems is a critical vulnerability. Should an attacker compromise a user's credentials, they would have direct access to the organization's most valuable data.
    \item \textbf{Security Awareness Training:} The complete lack of a security awareness training program for both new and existing employees is a high-risk gap. Employees are the first line of defense, and without training, they are significantly more susceptible to phishing, social engineering, and other common attack vectors.
\end{itemize}

% --- TECHNICAL SCAN RESULTS ---
\section{Technical Scan Results}
An external network scan was performed on the target IP address to identify open ports and exposed services.

\subsection{Scan Details}
\begin{itemize}
    \item \textbf{Target IP:} \texttt{[Target IP]}
    \item \textbf{Scan Date:} Not specified in scan data.
\end{itemize}

\subsection{Open Ports and Services}
The scan identified one open port, which presents a significant security risk.

\begin{center}
\begin{tabular}{llll}
    \toprule
    \textbf{Port} & \textbf{State} & \textbf{Service} & \textbf{Version} \\
    \midrule
    21/tcp & Open & FTP & vsftpd 2.3.4 \\
    \bottomrule
\end{tabular}
\end{center}

\subsection{Technical Findings and Analysis}
\begin{itemize}
    \item \textbf{Critical Vulnerability - Outdated FTP Server:} The identified service, \textbf{\texttt{vsftpd 2.3.4}}, is a dangerously outdated version released in 2011. This specific version is widely known to be vulnerable to \textbf{CVE-2011-2523}, a critical backdoor that allows an unauthenticated attacker to execute arbitrary commands on the server with root privileges.
    \item \textbf{High Risk - Anonymous FTP Login:} The scan confirmed that \textbf{anonymous FTP login is allowed}. This configuration permits any user on the internet to connect to the server and potentially access, upload, or download files without authentication. This could lead to data leakage, malware distribution, or be used as a staging point for further attacks into the network.
\end{itemize}

% --- RISK ASSESSMENT SUMMARY ---
\section{Risk Assessment Summary}
The following table correlates and summarizes the identified risks from all data sources, ranked by severity.

\begin{center}
\begin{tabular}{p{0.3\textwidth} p{0.5\textwidth} l}
    \toprule
    \textbf{Risk Name} & \textbf{Description} & \textbf{Severity} \\
    \midrule
    \textbf{Vulnerable FTP Service (CVE-2011-2523)} & An outdated FTP server is exposed, containing a known backdoor that allows for remote code execution. & \textbf{Critical} \\
    \addlinespace
    \textbf{Lack of MFA on Sensitive Systems} & The absence of a second authentication factor on critical systems exposes sensitive data to compromise via stolen credentials. & \textbf{Critical} \\
    \addlinespace
    \textbf{Anonymous FTP Access} & The FTP server allows unauthenticated access, risking data exfiltration, malware uploads, and unauthorized system use. & \textbf{High} \\
    \addlinespace
    \textbf{Insufficient Security Training} & Employees are not trained to recognize or respond to security threats, increasing susceptibility to phishing and social engineering. & \textbf{High} \\
    \addlinespace
    \textbf{Outdated Windows Policy} & Pre-existing risk: Workstations are running Windows 7, an unsupported OS lacking modern security patches. (CVSS 5.0) & Medium \\
    \bottomrule
\end{tabular}
\end{center}

% --- RECOMMENDATIONS ---
\section{Recommendations}
The following actions are recommended to mitigate the identified risks and improve the organization's security posture. Recommendations are prioritized based on risk severity.

\begin{enumerate}
    \item \textbf{Remediate Exposed FTP Server (Immediate):}
    \begin{itemize}
        \item Immediately take the FTP server at \texttt{[Target IP]} offline or place it behind a firewall that blocks all external access.
        \item If the FTP service is business-critical, it must be upgraded to the latest stable version immediately.
        \item Disable anonymous FTP login. Access should be restricted to authenticated users via strong, unique passwords.
        \item Consider replacing FTP with a more secure file transfer protocol like SFTP (SSH File Transfer Protocol).
    \end{itemize}

    \item \textbf{Implement MFA on All Sensitive Systems (Immediate):}
    \begin{itemize}
        \item Prioritize the deployment of a robust MFA solution for all systems containing sensitive or critical data.
        \item Develop a policy that mandates the use of MFA for all privileged accounts and remote access sessions.
    \end{itemize}

    \item \textbf{Establish a Security Awareness Training Program (High Priority):}
    \begin{itemize}
        \item Procure or develop a security awareness training program that is mandatory for all new hires during their onboarding process.
        \item Implement a mandatory annual refresher training for all employees to keep them updated on current threats.
        \item Training should cover topics such as phishing identification, password hygiene, and acceptable use policies.
    \end{itemize}

    \item \textbf{Expedite Operating System Upgrades (Medium Priority):}
    \begin{itemize}
        \item Continue and prioritize the existing plan to upgrade all workstations from Windows 7 to a modern, supported operating system like Windows 10 or 11 to ensure they receive critical security updates.
    \end{itemize}
\end{enumerate}

% --- DOCUMENT END ---
\end{document}
```