```latex
\documentclass[12pt]{article}

% --- PACKAGES ---
\usepackage[margin=1in]{geometry} % Set page margins
\usepackage{pifont}               % For checkmarks and crosses (\ding)
\usepackage{booktabs}             % For professional-looking tables
\usepackage{hyperref}             % For clickable links and a modern feel
\usepackage{url}                  % For formatting URLs
\usepackage{seqsplit}             % To split long text strings without breaking
\usepackage{graphicx}             % For logos, etc.
\usepackage{fancyhdr}             % For headers and footers
\usepackage{xcolor}               % For custom colors

% --- DOCUMENT METADATA ---
\hypersetup{
    colorlinks=true,
    linkcolor=blue,
    filecolor=magenta,      
    urlcolor=cyan,
    pdftitle={Cybersecurity Posture Assessment Report},
    pdfauthor={Automated Security Analysis System},
    pdfsubject={Security Report},
    pdfkeywords={Cybersecurity, Nmap, Risk Assessment},
}

% --- HEADER & FOOTER ---
\pagestyle{fancy}
\fancyhf{} % Clear all header and footer fields
\fancyhead[L]{Cybersecurity Posture Assessment}
\fancyhead[R]{\textbf{[Organization Name]}}
\fancyfoot[C]{\thepage}
\renewcommand{\headrulewidth}{0.4pt}
\renewcommand{\footrulewidth}{0.4pt}

% --- TITLE ---
\title{
    \vspace{2cm}
    \textbf{Cybersecurity Posture Assessment Report}\\
    \large For \textbf{[Organization Name]}
    \vspace{1cm}
}
\author{Cybersecurity Analysis Division}
\date{\today}

% --- DOCUMENT START ---
\begin{document}

\maketitle
\thispagestyle{empty}
\newpage

\tableofcontents
\newpage

% ==============================================================================
\section{Executive Summary}
% ==============================================================================

This report provides a comprehensive cybersecurity posture assessment for \textbf{[Organization Name]}, synthesizing data from a network vulnerability scan, a security controls questionnaire, and a review of pre-existing risks.

The assessment identified several critical gaps in organizational security policies and user-level controls. The most significant risks stem from the lack of Multi-Factor Authentication (MFA) for email access and the absence of a formal security awareness training program. These deficiencies create a high susceptibility to phishing, business email compromise, and other social engineering attacks. Furthermore, the lack of an Acceptable Use Policy (AUP) introduces ambiguity regarding secure employee behavior.

On a positive note, the technical network scan of the target host \texttt{[Target IP]} revealed a secure perimeter, with no open ports detected. This finding contradicts a pre-existing risk concerning an unencrypted web server on Port 80, suggesting that the vulnerability has been successfully remediated or was a false positive.

Immediate action is recommended to address the identified policy and training gaps to significantly strengthen the organization's overall defense against common cyber threats.

% ==============================================================================
\section{Organizational Information}
% ==============================================================================

The following information was used as the basis for this assessment. In cases where data was not provided, placeholders have been used.

\begin{table}[h!]
\centering
\caption{Client Organizational Data}
\begin{tabular}{@{}ll@{}}
\toprule
\textbf{Attribute} & \textbf{Value} \\ \midrule
Organization Name  & \textbf{[Organization Name]} \\
Email Domain       & \texttt{[Domain]} \\
External IP Address & \texttt{[Client IP]} \\ \bottomrule
\end{tabular}
\end{table}

% ==============================================================================
\section{Security Control Review}
% ==============================================================================

A review of the organization's security controls was conducted via a questionnaire. The results below highlight critical areas requiring immediate attention. Answers marked with \ding{55} (No) represent significant gaps in the current security posture.

\begin{table}[h!]
\centering
\caption{Security Controls Questionnaire Results}
\begin{tabular}{@{}lc@{}}
\toprule
\textbf{Control Question} & \textbf{Status} \\ \midrule
Do you require MFA to access email? & \ding{55} \\
Do you require MFA to log into computers? & \ding{51} \\
Do you require MFA to access sensitive data systems? & \ding{51} \\
Does your organization have an employee acceptable use policy? & \ding{55} \\
Does your organization do security awareness training for new employees? & \ding{55} \\
Does your organization do security awareness training for all employees annually? & \ding{55} \\ \bottomrule
\end{tabular}
\label{tab:controls}
\end{table}

\subsection*{Analysis of Control Gaps}
\begin{itemize}
    \item \textbf{MFA on Email (Critical Gap):} The absence of MFA on email is a critical vulnerability. Email accounts are primary targets for attackers seeking to conduct phishing campaigns, gain initial access to the network, or perform business email compromise (BEC) fraud.
    \item \textbf{Acceptable Use Policy (High Risk):} Without a formal AUP, employees may be unaware of their responsibilities regarding the secure use of company assets, data handling, and internet usage, increasing the risk of insider threats and accidental data breaches.
    \item \textbf{Security Awareness Training (Critical Gap):} The complete lack of a security awareness training program leaves the organization's primary line of defense—its employees—unprepared to identify and report security threats like phishing, malware, and social engineering.
\end{itemize}

% ==============================================================================
\section{Technical Scan Results}
% ==============================================================================

An external network scan was performed using Nmap to identify accessible services and potential vulnerabilities on the perimeter.

\begin{itemize}
    \item \textbf{Target Host:} \texttt{[Target IP]}
    \item \textbf{Scan Date:} \today
    \item \textbf{Host Status:} Up
\end{itemize}

The scan results indicate a well-configured firewall, as no open ports were discovered. This is a strong security posture for the scanned asset.

\begin{table}[h!]
\centering
\caption{Nmap Port Scan Findings for \texttt{[Target IP]}}
\begin{tabular}{@{}llll@{}}
\toprule
\textbf{Port} & \textbf{State} & \textbf{Service} & \textbf{Version} \\ \midrule
80/tcp        & closed         & http             & N/A              \\ \bottomrule
\end{tabular}
\label{tab:nmap}
\end{table}

\subsection*{Correlation with Existing Risks}
A pre-existing risk, "Unencrypted Web Server," was noted, indicating that Port 80 was believed to be open. The current scan results \textbf{do not validate} this risk. The finding that Port 80 is \texttt{closed} suggests this vulnerability has been successfully remediated.

% ==============================================================================
\section{Consolidated Risk Assessment}
% ==============================================================================

The following table summarizes the identified and reviewed risks, combining findings from the security questionnaire, technical scans, and pre-existing data.

\begin{table}[h!]
\centering
\caption{Summary of Identified Risks}
\resizebox{\textwidth}{!}{%
\begin{tabular}{@{}llll@{}}
\toprule
\textbf{Risk Name} & \textbf{Description} & \textbf{Severity} & \textbf{Source} \\ \midrule
\textcolor{red}{Lack of MFA on Email} & Email accounts are vulnerable to takeover via credential theft. & \textbf{Critical} & Questionnaire \\
\textcolor{red}{No Security Awareness Training} & Employees are unable to identify or respond to common threats. & \textbf{High} & Questionnaire \\
\textcolor{orange}{Missing Acceptable Use Policy} & No formal policy governing the secure use of IT assets. & Medium & Questionnaire \\
Unencrypted Web Server & Port 80 was reported as open, allowing unencrypted traffic. & \textit{Remediated} & Existing Data \\
\bottomrule
\end{tabular}%
}
\label{tab:risks}
\end{table}

% ==============================================================================
\section{Recommendations}
% ==============================================================================

Based on the analysis, the following actions are recommended to mitigate the identified risks and improve the overall security posture of \textbf{[Organization Name]}.

\subsection*{Priority 1: Critical Risk Mitigation}
\begin{enumerate}
    \item \textbf{Implement MFA for Email Access:}
    \begin{itemize}
        \item \textbf{Action:} Immediately enforce Multi-Factor Authentication (MFA) for all user access to the email system (\texttt{[Domain]}).
        \item \textbf{Impact:} This is the single most effective control to prevent account takeovers resulting from phishing or credential stuffing attacks.
    \end{itemize}
\end{enumerate}

\subsection*{Priority 2: High Risk Mitigation}
\begin{enumerate}
    \setcounter{enumi}{1}
    \item \textbf{Establish a Security Awareness Training Program:}
    \begin{itemize}
        \item \textbf{Action:} Procure and implement a security awareness training solution. All new hires must complete initial training, and all staff must complete annual refresher training and simulated phishing exercises.
        \item \textbf{Impact:} Empowers employees to become a proactive defense layer, reducing the likelihood of successful social engineering attacks.
    \end{itemize}
\end{enumerate}

\subsection*{Priority 3: Medium Risk Mitigation}
\begin{enumerate}
    \setcounter{enumi}{2}
    \item \textbf{Develop and Enforce an Acceptable Use Policy (AUP):}
    \begin{itemize}
        \item \textbf{Action:} Draft a clear AUP that outlines the rules for using company networks, devices, and data. Require all employees to read and acknowledge the policy.
        \item \textbf{Impact:} Establishes a baseline for secure employee conduct and provides a basis for enforcement actions if violations occur.
    \end{itemize}
\end{enumerate}

\subsection*{Observational Notes}
\begin{itemize}
    \item \textbf{Validate Risk Remediation:} The technical scan confirmed that the risk of an "Unencrypted Web Server" on Port 80 is no longer present. It is recommended to formally update internal risk registers to reflect this remediation.
    \item \textbf{Continuous Monitoring:} Maintain regular vulnerability scanning schedules to ensure the external perimeter remains secure against emerging threats.
\end{itemize}

\end{document}
```