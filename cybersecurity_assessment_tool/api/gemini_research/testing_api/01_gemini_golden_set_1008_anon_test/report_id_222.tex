```latex
\documentclass[12pt]{article}

% Preamble: Required Packages
\usepackage[margin=1in]{geometry}
\usepackage{pifont} % For checkmarks and crosses
\usepackage{booktabs} % For professional tables
\usepackage{hyperref} % For hyperlinks
\usepackage{url}      % For URL formatting
\usepackage{seqsplit} % For splitting long text strings

% Document Metadata
\title{Cybersecurity Posture Assessment Report}
\author{Cybersecurity Analysis Division}
\date{\today}

\begin{document}

\maketitle
\tableofcontents
\newpage

% --- 1. Executive Summary ---
\section{Executive Summary}
This report provides a comprehensive cybersecurity assessment for \textbf{[Organization Name]}, synthesizing information from organizational questionnaires, technical network scans, and a review of pre-existing risks.

The analysis reveals a mixed security posture. The organization demonstrates a strong foundation in security policy and awareness training, with established acceptable use policies and regular training schedules for all employees. However, this is critically undermined by significant gaps in access control. The absence of Multi-Factor Authentication (MFA) for computer logins and, most importantly, for access to sensitive data systems, presents a critical risk of unauthorized access and potential data breach.

A technical scan of the designated target IP address, \texttt{[Target IP]}, did not identify any open ports or exposed services. This finding conflicts with a pre-existing documented risk concerning an unencrypted web server on port 80. This discrepancy suggests the pre-existing risk may be outdated or related to a different asset.

Immediate remediation should focus on the deployment of MFA across all critical systems, followed by a re-validation of the existing risk register to ensure its accuracy.

% --- 2. Organizational Information ---
\section{Organizational Information}
This section details the organizational data used for this assessment. As the provided data was anonymized, placeholders have been used where necessary.

\begin{itemize}
    \item \textbf{Organization Name:} \textbf{[Organization Name]}
    \item \textbf{Primary Domain:} \texttt{[Domain]}
    \item \textbf{Scanned External IP:} \texttt{[Client IP]}
\end{itemize}

% --- 3. Security Control Review (Questionnaire Analysis) ---
\section{Security Control Review}
The following table summarizes the organization's self-reported security controls based on the provided questionnaire. Answers marked with \ding{55} (No) indicate significant gaps in the security framework and are discussed in the Risk Assessment section.

\begin{table}[h!]
\centering
\caption{Security Controls Questionnaire Results}
\begin{tabular}{p{0.8\linewidth} c}
\toprule
\textbf{Control Question} & \textbf{Status} \\
\midrule
Do you require MFA to access email? & \ding{51} \\
Do you require MFA to log into computers? & \textbf{\color{red}\ding{55}} \\
Do you require MFA to access sensitive data systems? & \textbf{\color{red}\ding{55}} \\
Does your organization have an employee acceptable use policy? & \ding{51} \\
Does your organization do security awareness training for new employees? & \ding{51} \\
Does your organization do security awareness training for all employees at least once per year? & \ding{51} \\
\bottomrule
\end{tabular}
\end{table}

% --- 4. Technical Scan Results ---
\section{Technical Scan Results}
A network scan was performed on the target system to identify exposed services and potential vulnerabilities.

\begin{itemize}
    \item \textbf{Target IP Address:} \texttt{[Target IP]}
    \item \textbf{Scan Tool:} Nmap
    \item \textbf{Scan Summary:} The scan reported the host as "up". No open ports were discovered. Port 80 (HTTP) was explicitly checked and found to be \textbf{closed}. This indicates that, at the time of the scan, the target system was not exposing any common network services to the internet.
\end{itemize}

\textbf{Note:} This technical finding directly contradicts a pre-existing risk documented in the organization's risk register, which states that Port 80 is open. This discrepancy requires investigation.

% --- 5. Risk Assessment ---
\section{Risk Assessment}
This section correlates the findings from the security control review, technical scans, and pre-existing risk data to provide a unified view of the current risk landscape.

\begin{table}[h!]
\centering
\caption{Synthesized Risk Summary}
\begin{tabular}{p{0.3\linewidth} p{0.5\linewidth} l}
\toprule
\textbf{Risk Title} & \textbf{Description} & \textbf{Severity} \\
\midrule
\textbf{No MFA on Sensitive Systems} & The lack of MFA for systems containing sensitive data creates a critical vulnerability. A single compromised password could lead to a significant data breach. & \textbf{Critical} \\
\addlinespace
\textbf{No MFA on Endpoints} & The absence of MFA for computer logins allows an attacker with valid credentials (e.g., from a phishing attack) to gain initial access to the internal network, escalate privileges, and move laterally. & \textbf{High} \\
\addlinespace
\textbf{Unencrypted Web Server (Discrepancy)} & A pre-existing risk notes that Port 80 is open, exposing an unencrypted web service. However, our current scan shows this port is closed. This risk may be outdated or apply to a different asset. & Medium \\
\bottomrule
\end{tabular}
\end{table}

% --- 6. Recommendations ---
\section{Recommendations}
Based on the analysis, the following actions are recommended to mitigate the identified risks and improve the overall security posture of \textbf{[Organization Name]}.

\begin{enumerate}
    \item \textbf{[Critical Priority] Implement MFA for Sensitive Data Access:} Immediately deploy a mandatory MFA solution for all user accounts (including administrative and service accounts) that have access to sensitive data systems. This is the single most effective control to prevent unauthorized access to critical assets.

    \item \textbf{[High Priority] Enforce MFA for Endpoint Logins:} Roll out MFA for all employee and contractor computer logins. This strengthens initial access controls and contains the impact of compromised credentials.

    \item \textbf{[Medium Priority] Validate and Update Risk Register:} Conduct an internal review to resolve the discrepancy regarding the "Unencrypted Web Server" risk.
    \begin{itemize}
        \item Verify if the risk applies to a different IP address or asset not included in this assessment's scope.
        \item If the service has been decommissioned, formally close the risk in the register.
        \item If the service still exists elsewhere, ensure it is secured via TLS/SSL (HTTPS) and that the risk register accurately reflects its status.
    \end{itemize}
\end{enumerate}

\end{document}
```