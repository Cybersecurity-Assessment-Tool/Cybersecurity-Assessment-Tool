```latex
\documentclass[12pt]{article}

% Preamble: Required Packages
\usepackage[margin=1in]{geometry}
\usepackage{pifont} % For checkmarks and crosses
\usepackage{booktabs} % For professional tables
\usepackage{hyperref} % For hyperlinks
\usepackage{url} % For URL formatting
\usepackage{seqsplit} % For splitting long strings in texttt
\usepackage[utf8]{inputenc}

% Document Metadata
\title{Cybersecurity Posture Assessment Report\\ \large For \textbf{[Organization Name]}}
\author{Cybersecurity Analysis Division}
\date{\today}

\begin{document}

\maketitle
\thispagestyle{empty}
\newpage
\tableofcontents
\newpage

% --- 1. Executive Summary ---
\section*{1. Executive Summary}

This report provides a cybersecurity posture assessment for \textbf{[Organization Name]}, based on an analysis of network scan data, a security controls questionnaire, and a review of pre-existing risks. The assessment was conducted to identify vulnerabilities, security gaps, and areas for improvement.

The overall security posture is determined to be at a \textbf{high-risk level}. This conclusion is based on several critical findings:
\begin{itemize}
    \item \textbf{Critical Control Gaps:} The organization does not enforce Multi-Factor Authentication (MFA) for accessing sensitive data systems, which exposes critical assets to unauthorized access.
    \item \textbf{Inadequate Policies and Training:} There is no formal employee Acceptable Use Policy (AUP), and security awareness training is not integrated into the new employee onboarding process. These gaps significantly increase the risk of human error leading to a security incident.
    \item \textbf{Insecure Network Services:} The external network scan identified an open port running an unencrypted HTTP service (Port 80). This exposes any data transmitted to or from the service to interception and manipulation.
\end{itemize}

Immediate and strategic remediation is required to mitigate these risks. Recommendations are provided in Section 6 to address these findings and strengthen the organization's overall security posture.

% --- 2. Organizational Information ---
\section*{2. Organizational Information}

The following information was used as the basis for this assessment. Due to the anonymized nature of the provided data, placeholders have been used where necessary.

\begin{table}[h!]
\centering
\begin{tabular}{@{}ll@{}}
\toprule
\textbf{Attribute} & \textbf{Value} \\ \midrule
Organization Name & \textbf{[Organization Name]} \\
Primary Domain & \texttt{[Domain]} \\
External IP Address Scanned & \texttt{[Client IP]} \\ \bottomrule
\end{tabular}
\caption{Client Organizational Details}
\end{table}

% --- 3. Security Control Review ---
\section*{3. Security Control Review}

A review of the organization's security controls was conducted via a questionnaire. The responses indicate several significant gaps in foundational security practices. A "No" response highlights a missing control that should be addressed with high priority.

\begin{table}[h!]
\centering
\begin{tabular}{@{}p{0.7\textwidth}c@{}}
\toprule
\textbf{Control Question} & \textbf{Response} \\ \midrule
Do you require MFA to access email? & \ding{51} Yes \\
Do you require MFA to log into computers? & \ding{51} Yes \\
\textbf{Do you require MFA to access sensitive data systems?} & \textbf{\ding{55} No} \\
\textbf{Does your organization have an employee acceptable use policy?} & \textbf{\ding{55} No} \\
\textbf{Does your organization do security awareness training for new employees?} & \textbf{\ding{55} No} \\
Does your organization do security awareness training for all employees at least once per year? & \ding{51} Yes \\ \bottomrule
\end{tabular}
\caption{Security Controls Questionnaire Results}
\end{table}

\subsection*{Analysis of Control Gaps}
\begin{itemize}
    \item \textbf{MFA on Sensitive Systems:} The absence of MFA for sensitive data is a critical vulnerability. Should an attacker compromise a user's credentials, they would have direct access to the organization's most valuable information.
    \item \textbf{Acceptable Use Policy (AUP):} Without a formal AUP, employees may be unaware of their responsibilities regarding the secure use of company assets, leading to unintentional policy violations and security incidents.
    \item \textbf{New Employee Training:} Failing to train new employees on security best practices from day one leaves a window of high vulnerability. New hires are often prime targets for social engineering attacks.
\end{itemize}

% --- 4. Technical Scan Results ---
\section*{4. Technical Scan Results}

An external network scan was performed on the target IP address to identify open ports and exposed services.

\subsection*{Scan Details}
\begin{itemize}
    \item \textbf{Target IP Address:} \texttt{[Target IP]}
    \item \textbf{Scan Date:} Not Specified
\end{itemize}

\subsection*{Open Ports and Services}
The scan revealed the following open port accessible from the public internet.

\begin{table}[h!]
\centering
\begin{tabular}{@{}llll@{}}
\toprule
\textbf{Port} & \textbf{Protocol} & \textbf{State} & \textbf{Inferred Service} \\ \midrule
80 & TCP & open & HTTP (Web Server) \\ \bottomrule
\end{tabular}
\caption{Network Scan Findings}
\end{table}

\subsection*{Technical Analysis}
The presence of an open Port 80 indicates that a web server is running and is configured to use the Hypertext Transfer Protocol (HTTP). HTTP is an unencrypted protocol. Any data, including usernames, passwords, or other sensitive information transmitted between a user and this server, can be easily intercepted and read by an attacker on the network. This is a significant security risk and is considered a deprecated practice for modern web applications. All web traffic should be encrypted using HTTPS (Port 443).

% --- 5. Consolidated Risk Assessment ---
\section*{5. Risk Assessment}

This section synthesizes findings from the security control review and the technical scan to provide a consolidated view of the most significant risks facing the organization. The pre-existing risk data provided was determined to be invalid and has been excluded from this analysis.

\begin{table}[h!]
\centering
\begin{tabular}{@{}p{0.3\textwidth}p{0.5\textwidth}l@{}}
\toprule
\textbf{Risk Name} & \textbf{Description} & \textbf{Severity} \\ \midrule
\textbf{Lack of MFA on Sensitive Systems} & User accounts with access to critical data are protected only by a password, making them highly susceptible to credential theft and compromise. & \textbf{Critical} \\
\addlinespace
\textbf{Unencrypted Web Service (HTTP)} & The web service on port 80 transmits data in cleartext, exposing it to eavesdropping and man-in-the-middle attacks. & \textbf{High} \\
\addlinespace
\textbf{Missing Acceptable Use Policy} & Lack of a formal policy creates ambiguity for employees on secure practices, increasing the likelihood of insider threats and data mishandling. & \textbf{High} \\
\addlinespace
\textbf{Inadequate New Hire Security Training} & New employees are not trained on security policies and threats upon joining, making them vulnerable targets for social engineering. & \textbf{High} \\ \bottomrule
\end{tabular}
\caption{Summary of Identified Risks}
\end{table}

% --- 6. Recommendations ---
\section*{6. Recommendations}

The following actions are recommended to mitigate the identified risks and improve the overall security posture of \textbf{[Organization Name]}.

\subsection*{Immediate Actions (1-30 Days)}
\begin{enumerate}
    \item \textbf{Enable MFA on All Sensitive Systems:} Prioritize the implementation of mandatory MFA on all applications and systems that store, process, or transmit sensitive data. This is the single most effective control to mitigate the risk of unauthorized access.
    \item \textbf{Migrate HTTP to HTTPS:} Immediately configure the web server listening on port 80 to redirect all traffic to HTTPS (Port 443). Obtain and install a valid TLS/SSL certificate to encrypt all web communications. If the HTTP service is not required, it should be disabled at the firewall.
\end{enumerate}

\subsection*{Strategic Actions (30-90 Days)}
\begin{enumerate}
    \item \textbf{Develop and Implement an Acceptable Use Policy (AUP):} Draft a formal AUP that clearly defines the rules and responsibilities for all employees when using company IT assets. This policy should be communicated to all staff and acknowledged in writing.
    \item \textbf{Integrate Security Training into Onboarding:} Develop a mandatory security awareness training module for all new employees. This training should be a required part of the onboarding process and should cover key topics such as phishing, password security, MFA, and the new AUP.
\end{enumerate}

\end{document}
```