```latex
\documentclass[12pt]{article}

% Preamble: Required Packages
\usepackage[margin=1in]{geometry}
\usepackage{pifont} % For checkmarks and crosses
\usepackage{booktabs} % For professional tables
\usepackage{hyperref} % For hyperlinks
\usepackage{url} % For URL formatting
\usepackage{seqsplit} % For splitting long strings in texttt
\usepackage{graphicx}
\usepackage{xcolor}

% Document Metadata
\title{Cybersecurity Posture Assessment Report}
\author{Cybersecurity Analysis Division}
\date{\today}

% Hyperref Setup
\hypersetup{
    colorlinks=true,
    linkcolor=blue,
    filecolor=magenta,      
    urlcolor=cyan,
    pdftitle={Cybersecurity Posture Assessment Report},
    pdfpagemode=FullScreen,
}

\begin{document}

\maketitle
\hrule
\vspace{1em}
\begin{center}
    \textbf{Prepared for: \textbf{[Organization Name]}}
\end{center}
\vspace{1em}
\hrule
\vspace{2em}

\tableofcontents
\newpage

% ===================================================================
% Section 1: Executive Overview
% ===================================================================
\section{Executive Overview}

This report provides a comprehensive analysis of the cybersecurity posture for \textbf{[Organization Name]}, based on data gathered from a network scan, a security controls questionnaire, and a review of pre-existing risks. The assessment aims to identify vulnerabilities, procedural gaps, and misconfigurations that could expose the organization to cyber threats.

\paragraph{Key Findings:}
The assessment reveals a mixed security posture. On a positive note, the external network scan of the target IP address \texttt{[Target IP]} did not identify any open ports. This indicates that a previously identified risk, an "Unencrypted Web Server" on Port 80, appears to have been successfully remediated. This is a significant improvement in the external security perimeter.

However, a critical security gap was identified through the organizational questionnaire. The lack of mandatory Multi-Factor Authentication (MFA) for logging into company computers represents a high-severity risk. This weakness could allow an attacker with compromised credentials to gain initial access to the internal network, bypass other security controls, and potentially escalate privileges to access sensitive data.

\paragraph{Summary:}
While the organization demonstrates a strong external posture and good adherence to security policies like employee training, the absence of endpoint MFA is a critical vulnerability. Recommendations in this report are prioritized to address this gap immediately to prevent unauthorized access and protect critical assets.

% ===================================================================
% Section 2: Organizational Information
% ===================================================================
\section{Organizational Information}

This section details the organizational context for this assessment. The information has been anonymized as per the reporting protocol.

\begin{itemize}
    \item \textbf{Organization Name:} \textbf{[Organization Name]}
    \item \textbf{Primary Domain:} \texttt{[Domain]}
    \item \textbf{External IP Scanned:} \texttt{[Client IP]}
\end{itemize}

% ===================================================================
% Section 3: Security Control Review (Questionnaire)
% ===================================================================
\section{Security Control Review (Questionnaire)}

The following table summarizes the organization's responses to a security controls questionnaire. These answers provide insight into the internal security policies and procedures currently in place. A green checkmark (\textcolor{green}{\ding{51}}) indicates a positive control, while a red cross (\textcolor{red}{\ding{55}}) indicates a potential security gap.

\begin{table}[h!]
\centering
\caption{Security Controls Questionnaire Results}
\begin{tabular}{p{0.7\linewidth} c}
\toprule
\textbf{Control Question} & \textbf{Response} \\
\midrule
Do you require MFA to access email? & \textcolor{green}{\ding{51}} \\
\textbf{Do you require MFA to log into computers?} & \textcolor{red}{\ding{55}} \\
Do you require MFA to access sensitive data systems? & \textcolor{green}{\ding{51}} \\
Does your organization have an employee acceptable use policy? & \textcolor{green}{\ding{51}} \\
Does your organization do security awareness training for new employees? & \textcolor{green}{\ding{51}} \\
Does your organization do security awareness training for all employees at least once per year? & \textcolor{green}{\ding{51}} \\
\bottomrule
\end{tabular}
\end{table}

\paragraph{Analysis:}
The questionnaire reveals a significant gap in endpoint security. The absence of MFA for computer logins is a critical weakness. If an employee's password is stolen, lost, or cracked, there is no secondary authentication factor to prevent an attacker from logging directly into their workstation. This provides a direct entry point into the corporate network, from which an attacker can perform reconnaissance, move laterally, and escalate privileges. All other responses indicate that foundational security policies and training are in place, which is commendable.

% ===================================================================
% Section 4: Technical Scan Results
% ===================================================================
\section{Technical Scan Results}

An external network scan was performed to identify open ports and exposed services.

\begin{itemize}
    \item \textbf{Target IP Address:} \texttt{[Target IP]}
    \item \textbf{Scan Date:} Scan data processed on \today
    \item \textbf{Scanner Used:} Nmap
\end{itemize}

\subsection{Scan Findings}
The scan results were positive from a security perspective. No open ports were detected on the target host. The host was responsive, but all common ports, including those for web (80, 443), email (25, 110), and remote access (22, 3389), were found to be in a \texttt{closed} or \texttt{filtered} state.

\paragraph{Notable Finding: Port 80 (HTTP)}
Specifically, Port 80, which was listed as a concern in the pre-existing risk register, was confirmed to be \textbf{closed}. This technical finding suggests that the risk of an unencrypted web server has been successfully mitigated.

% ===================================================================
% Section 5: Consolidated Risk Assessment
% ===================================================================
\section{Consolidated Risk Assessment}

This section synthesizes findings from the security questionnaire, the technical scan, and the pre-existing risk data into a consolidated list.

\begin{table}[h!]
\centering
\caption{Summary of Identified Risks}
\begin{tabular}{p{0.25\linewidth} p{0.4\linewidth} l l}
\toprule
\textbf{Risk Name} & \textbf{Description} & \textbf{Severity} & \textbf{Status} \\
\midrule
\textbf{Lack of MFA on Workstations} & The absence of a secondary authentication factor for computer logins exposes the organization to unauthorized access via compromised credentials. & \textbf{High} & \textbf{Active} \\
\addlinespace
Unencrypted Web Server & A previously identified risk where Port 80 was open, potentially exposing an unencrypted web service. & Medium & \textbf{Remediated} \\
\bottomrule
\end{tabular}
\end{table}

\paragraph{Risk Correlation Analysis:}
The primary active risk is the lack of MFA on workstations, identified from the questionnaire. The technical scan data directly contradicts the pre-existing risk of an "Unencrypted Web Server," indicating that this risk is no longer active and can be closed. This is a positive outcome, demonstrating progress in securing the network perimeter. The focus must now shift to strengthening internal and endpoint security controls.

% ===================================================================
% Section 6: Recommendations
% ===================================================================
\section{Recommendations}

Based on the analysis, the following actions are recommended to improve the organization's cybersecurity posture. Recommendations are prioritized by severity.

\subsection{Priority 1: Implement MFA for Endpoint Logins (Critical)}
\begin{itemize}
    \item \textbf{Action:} Deploy a mandatory Multi-Factor Authentication (MFA) solution for all employee logins to workstations, laptops, and servers. This should apply to both on-premise and remote access.
    \item \textbf{Justification:} This is the most effective single control to mitigate the risk of account takeover. Even if an attacker obtains a user's password, they will be unable to access the system without the second factor (e.g., a code from a mobile app, a physical security key).
    \item \textbf{Suggested Solutions:} Solutions like Windows Hello for Business, Duo Security, or Okta can be integrated with Active Directory or other identity providers to enforce this control.
\end{itemize}

\subsection{Priority 2: Update Internal Risk Register (Administrative)}
\begin{itemize}
    \item \textbf{Action:} Formally mark the risk titled "Unencrypted Web Server" as "Remediated" or "Closed" in the organization's internal risk tracking system.
    \item \textbf{Justification:} The technical scan confirmed that Port 80 is closed, mitigating the risk. Maintaining an accurate and up-to-date risk register is crucial for focusing resources on current and relevant threats.
\end{itemize}

\end{document}
```