```latex
\documentclass[12pt]{article}

% --- PACKAGES ---
\usepackage[margin=1in]{geometry}
\usepackage{pifont} % For checkmarks and crosses
\usepackage{booktabs} % For professional tables
\usepackage{hyperref} % For clickable links
\usepackage{url} % For URL formatting
\usepackage{seqsplit} % To split long strings without breaking
\usepackage{graphicx}
\usepackage{xcolor}

% --- DOCUMENT SETUP ---
\hypersetup{
    colorlinks=true,
    linkcolor=blue,
    filecolor=magenta,      
    urlcolor=cyan,
    pdftitle={Cybersecurity Posture Report},
    pdfpagemode=FullScreen,
}

\newcommand{\yes}{\ding{51}}
\newcommand{\no}{\ding{55}}

% --- TITLE ---
\title{Cybersecurity Posture Report \\ \large For \textbf{[Organization Name]}}
\author{Cybersecurity Analysis Division}
\date{November 22, 2025}

\begin{document}

\maketitle
\thispagestyle{empty}
\newpage

\tableofcontents
\newpage

% --- 1. EXECUTIVE SUMMARY ---
\section{Executive Summary}

This report provides a comprehensive analysis of the cybersecurity posture for \textbf{[Organization Name]}, based on data collected on November 22, 2025. The assessment combined a review of organizational security controls, an external network scan, and an evaluation of pre-existing risks.

The analysis revealed several \textbf{critical and high-risk deficiencies} that require immediate attention. The most significant findings include a complete lack of Multi-Factor Authentication (MFA) for email and computer access, the absence of a formal security awareness program, and the use of an outdated and vulnerable web server on an external-facing system.

These gaps collectively expose the organization to a high likelihood of security incidents, including account compromise, data breaches, and service disruption. This report outlines these risks in detail and provides prioritized, actionable recommendations to mitigate them and strengthen the organization's overall security posture.

% --- 2. ORGANIZATIONAL INFORMATION ---
\section{Organizational Information}

This section details the information provided by the client for the scope of this assessment. Due to the anonymized nature of the data provided, placeholders are used where necessary.

\begin{itemize}
    \item \textbf{Organization Name:} \textbf{[Organization Name]}
    \item \textbf{Primary Domain:} \texttt{[Domain]}
    \item \textbf{External IP Address Scanned:} \texttt{[Client IP]}
\end{itemize}

% --- 3. SECURITY CONTROL REVIEW ---
\section{Security Control Review}

A review of the organization's self-reported security controls was conducted via a questionnaire. The responses indicate significant gaps in foundational security practices. A "Yes" (\yes) indicates a control is in place, while a "No" (\no) indicates a control is absent and represents a potential risk.

\subsection{Questionnaire Results}

\begin{table}[h!]
\centering
\begin{tabular}{p{0.75\linewidth} c}
\toprule
\textbf{Control Question} & \textbf{Response} \\
\midrule
Do you require MFA to access email? & \no \\
Do you require MFA to log into computers? & \no \\
Do you require MFA to access sensitive data systems? & \yes \\
Does your organization have an employee acceptable use policy? & \no \\
Does your organization do security awareness training for new employees? & \no \\
Does your organization do security awareness training for all employees at least once per year? & \no \\
\bottomrule
\end{tabular}
\caption{Organizational Security Control Status}
\label{tab:controls}
\end{table}

\subsection{Analysis of Controls}
The responses highlight critical weaknesses in user access and security governance:
\begin{itemize}
    \item \textbf{Lack of MFA:} The absence of MFA on email and computer logins is a critical vulnerability. Email is a primary target for phishing attacks, and compromised credentials could lead to widespread system access without this crucial control.
    \item \textbf{Lack of Security Governance:} The organization lacks a formal Acceptable Use Policy and does not conduct any security awareness training. This creates a high-risk environment where employees are unaware of security best practices and threats, making them susceptible to social engineering and accidental data exposure.
\end{itemize}

% --- 4. TECHNICAL SCAN RESULTS ---
\section{Technical Scan Results}

An external network scan was performed to identify open ports and exposed services.

\begin{itemize}
    \item \textbf{Target IP Address:} \texttt{[Target IP]}
    \item \textbf{Scan Date:} November 22, 2025
\end{itemize}

\subsection{Open Ports and Services}

\begin{table}[h!]
\centering
\begin{tabular}{l l l l l}
\toprule
\textbf{Port} & \textbf{State} & \textbf{Service} & \textbf{Product} & \textbf{Version} \\
\midrule
443/tcp & open & https & nginx & 1.18.0 \\
\bottomrule
\end{tabular}
\caption{Open Ports Detected on \texttt{[Target IP]}}
\label{tab:ports}
\end{table}

\subsection{Technical Analysis}
The scan identified an Nginx web server running on port 443 (HTTPS). The detected version, \textbf{1.18.0}, was released in April 2020 and is now significantly outdated. This version is known to be vulnerable to multiple security issues, including a high-severity vulnerability (CVE-2021-23017) that can lead to request smuggling and security bypass. Exposing outdated software to the internet presents a direct and exploitable attack vector.

% --- 5. RISK ASSESSMENT SUMMARY ---
\section{Risk Assessment Summary}

This section correlates the findings from the security control review and the technical scan. As no pre-existing vulnerabilities were reported, all risks listed below are new findings from this assessment.

\begin{table}[h!]
\centering
\begin{tabular}{p{0.1\linewidth} p{0.25\linewidth} p{0.45\linewidth} p{0.1\linewidth}}
\toprule
\textbf{ID} & \textbf{Risk Name} & \textbf{Description} & \textbf{Severity} \\
\midrule
RISK-001 & \textbf{Lack of Multi-Factor Authentication (MFA)} & Email and computer logins are protected only by passwords, making them highly susceptible to compromise via phishing or credential stuffing attacks. & \textbf{Critical} \\
\addlinespace
RISK-002 & \textbf{Inadequate Security Awareness Program} & The absence of an Acceptable Use Policy and security training leaves employees unprepared to identify and respond to cyber threats, increasing human-related risk. & \textbf{High} \\
\addlinespace
RISK-003 & \textbf{Outdated Web Server Software} & The public-facing Nginx server (v1.18.0) is outdated and has known vulnerabilities, creating an exploitable entry point for attackers. & \textbf{High} \\
\bottomrule
\end{tabular}
\caption{Identified Cybersecurity Risks}
\label{tab:risks}
\end{table}

% --- 6. RECOMMENDATIONS ---
\section{Recommendations}

The following prioritized recommendations are provided to address the identified risks and improve the overall security posture of \textbf{[Organization Name]}.

\subsection{Immediate Actions (Priority 1)}

\begin{enumerate}
    \item \textbf{Implement MFA (RISK-001):}
    \begin{itemize}
        \item \textbf{Action:} Enforce MFA for all users on all critical systems, with the highest priority on email (e.g., Office 365, Google Workspace) and remote access VPNs.
        \item \textbf{Justification:} This is the single most effective control to prevent unauthorized access from compromised credentials.
    \end{itemize}
    
    \item \textbf{Patch Web Server (RISK-003):}
    \begin{itemize}
        \item \textbf{Action:} Immediately upgrade the Nginx server running on \texttt{[Target IP]} from version 1.18.0 to the latest stable release.
        \item \textbf{Justification:} Patching this known vulnerability closes a direct, public-facing attack vector. A formal patch management policy should be established for all internet-facing systems.
    \end{itemize}
\end{enumerate}

\subsection{Strategic Actions (Priority 2)}

\begin{enumerate}
    \setcounter{enumi}{2} % Continue numbering from previous list
    \item \textbf{Establish Security Awareness Program (RISK-002):}
    \begin{itemize}
        \item \textbf{Action:} Develop and implement a formal security awareness program. This must include:
        \begin{itemize}
            \item An \textbf{Acceptable Use Policy (AUP)} that all employees must read and sign.
            \item Mandatory security training for all new hires.
            \item Annual refresher training for all employees covering topics like phishing, password security, and data handling.
        \end{itemize}
        \item \textbf{Justification:} A well-trained workforce is a critical layer of defense, turning a potential weakness into a security asset.
    \end{itemize}
\end{enumerate}

\end{document}
```