```latex
\documentclass[12pt]{article}

% --- PACKAGES ---
\usepackage[margin=1in]{geometry}
\usepackage{pifont} % For checkmarks and crosses
\usepackage{booktabs} % For professional tables
\usepackage{hyperref} % For clickable links
\usepackage{url} % For URL formatting
\usepackage{seqsplit} % For splitting long strings
\usepackage{graphicx} % For logo
\usepackage{xcolor} % For colors

% --- DOCUMENT METADATA ---
\title{Cybersecurity Posture Assessment Report}
\author{Cybersecurity Analysis Division}
\date{\today}

% --- HYPERREF SETUP ---
\hypersetup{
    colorlinks=true,
    linkcolor=blue,
    filecolor=magenta,      
    urlcolor=cyan,
    pdftitle={Cybersecurity Posture Assessment Report},
    pdfpagemode=FullScreen,
}

% --- COMMANDS ---
\newcommand{\yes}{\ding{51}}
\newcommand{\no}{\ding{55}}

\begin{document}

\maketitle
\thispagestyle{empty}
\newpage

\tableofcontents
\newpage

% ==============================================================================
% SECTION 1: EXECUTIVE SUMMARY
% ==============================================================================
\section{Executive Summary}

This report provides a comprehensive cybersecurity assessment for \textbf{[Organization Name]}, based on an analysis of organizational security controls, an external network scan, and a review of existing risks.

The assessment reveals a mixed security posture. On one hand, the external network scan of the target host \texttt{[Target IP]} showed a strong defensive configuration, with no open ports detected. This indicates a well-hardened perimeter for the scanned asset. The organization also demonstrates a commitment to security awareness training and has implemented Multi-Factor Authentication (MFA) for email access, which is a commendable control.

However, significant and critical gaps were identified in core access control and governance policies. The absence of MFA on employee computers and, most critically, on sensitive data systems, exposes the organization to a high risk of unauthorized access and data breach through credential compromise. Furthermore, the lack of a formal Acceptable Use Policy (AUP) represents a foundational governance weakness, leaving the organization without clear guidelines for employee technology use and lacking a key administrative control.

Immediate remediation should focus on implementing MFA across all critical systems and developing a formal AUP to establish a stronger security baseline.

% ==============================================================================
% SECTION 2: ORGANIZATIONAL INFORMATION
% ==============================================================================
\section{Organizational Information}

This section details the information provided about the organization. As this assessment was conducted in a template mode due to missing identity data, placeholders have been used.

\begin{table}[h!]
\centering
\begin{tabular}{@{}ll@{}}
\toprule
\textbf{Attribute}        & \textbf{Value}                 \\ \midrule
Organization Name         & \textbf{[Organization Name]}   \\
Primary Email Domain      & \texttt{[Domain]}              \\
Scanned External IP       & \texttt{[Client IP]}           \\ \bottomrule
\end{tabular}
\caption{Client Organizational Details.}
\label{tab:org_info}
\end{table}

% ==============================================================================
% SECTION 3: SECURITY CONTROL REVIEW
% ==============================================================================
\section{Security Control Review}

The following table summarizes the organization's responses to a security controls questionnaire. "No" answers indicate potential gaps in the security framework that require attention.

\begin{table}[h!]
\centering
\begin{tabular}{@{}p{0.8\linewidth}c@{}}
\toprule
\textbf{Control Question} & \textbf{Response} \\ \midrule
Do you require MFA to access email? & \yes \\
Do you require MFA to log into computers? & \no \\
Do you require MFA to access sensitive data systems? & \no \\
Does your organization have an employee acceptable use policy? & \no \\
Does your organization do security awareness training for new employees? & \yes \\
Does your organization do security awareness training for all employees at least once per year? & \yes \\ \bottomrule
\end{tabular}
\caption{Security Controls Questionnaire Results.}
\label{tab:controls}
\end{table}

\subsection{Analysis of Control Gaps}
The review identified three significant control gaps:
\begin{itemize}
    \item \textbf{No MFA for Computer Logins:} The lack of MFA on workstations means that a single compromised password could grant an attacker full access to an employee's machine and a foothold on the internal network.
    \item \textbf{No MFA for Sensitive Systems:} This is a critical vulnerability. Sensitive data repositories are primary targets for attackers. Without MFA, they are protected only by a password, making them highly susceptible to credential stuffing, phishing, and brute-force attacks.
    \item \textbf{No Acceptable Use Policy (AUP):} An AUP is a fundamental governance document that sets expectations for employee use of company assets. Its absence can lead to inconsistent security practices, misuse of resources, and weakened legal standing in the event of an insider incident.
\end{itemize}

% ==============================================================================
% SECTION 4: TECHNICAL SCAN RESULTS
% ==============================================================================
\section{Technical Scan Results}

An external network scan was performed to identify open ports and exposed services on the public-facing infrastructure.

\begin{itemize}
    \item \textbf{Target IP Address:} \texttt{[Target IP]}
    \item \textbf{Scan Date:} Data not provided in scan results.
\end{itemize}

\subsection{Findings}
The scan completed successfully against the target host. \textbf{No open TCP or UDP ports were discovered.}

This result indicates a strong network security posture for this specific host. A "default-deny" firewall configuration, which blocks all incoming traffic unless explicitly allowed, is a security best practice. This significantly reduces the external attack surface and limits the ability of unauthorized actors to probe for vulnerabilities.

% ==============================================================================
% SECTION 5: RISK ASSESSMENT
% ==============================================================================
\section{Risk Assessment}

This section synthesizes findings from the security control review and technical scan. No pre-existing vulnerabilities were provided for this assessment. The following risks have been identified based on the control gaps discovered.

\begin{table}[h!]
\centering
\begin{tabular}{@{}p{0.25\linewidth}p{0.5\linewidth}p{0.15\linewidth}@{}}
\toprule
\textbf{Risk Name} & \textbf{Overview} & \textbf{Severity} \\ \midrule
\textbf{Lack of MFA on Sensitive Systems} & Sensitive data systems are protected only by username and password. A compromised credential would lead directly to a data breach. This is a primary target for attackers. & \textbf{Critical} \\
\addlinespace
\textbf{Lack of MFA on Workstations} & A compromised password is sufficient for an attacker to gain control of an employee's computer, providing a launchpad for lateral movement within the internal network. & \textbf{High} \\
\addlinespace
\textbf{Missing Acceptable Use Policy (AUP)} & The absence of a formal AUP creates ambiguity regarding security responsibilities for employees. It weakens the organization's ability to enforce security standards and respond to insider policy violations. & \textbf{High} \\ \bottomrule
\end{tabular}
\caption{Identified Risks and Severity.}
\label{tab:risks}
\end{table}

% ==============================================================================
% SECTION 6: RECOMMENDATIONS
% ==============================================================================
\section{Recommendations}

Based on the risks identified in Section 5, the following actions are recommended to improve the organization's cybersecurity posture. Recommendations are prioritized by severity.

\begin{enumerate}
    \item \textbf{Implement MFA on Sensitive Data Systems (Critical):}
    \begin{itemize}
        \item \textbf{Action:} Immediately deploy a robust MFA solution for all systems classified as containing sensitive or critical data. This includes databases, financial applications, and administrative portals.
        \item \textbf{Justification:} This is the most effective single control to prevent unauthorized access to the organization's most valuable data assets. It mitigates the risk of a data breach resulting from stolen credentials.
    \end{itemize}
    \vspace{1em}
    \item \textbf{Enforce MFA for all Workstation and VPN Logins (High):}
    \begin{itemize}
        \item \textbf{Action:} Roll out MFA for all employee computer logins (Windows, macOS) and for all remote access solutions (VPN).
        \item \textbf{Justification:} This action protects the initial entry points to the corporate network, preventing attackers from gaining a foothold even if they acquire a user's password.
    \end{itemize}
    \vspace{1em}
    \item \textbf{Develop and Implement an Acceptable Use Policy (High):}
    \begin{itemize}
        \item \textbf{Action:} Draft, approve, and disseminate a formal AUP that clearly defines the rules and expectations for the use of company technology, data, and network resources. Require all employees to read and acknowledge the policy.
        \item \textbf{Justification:} An AUP is a foundational administrative control that establishes a baseline for secure user behavior, supports security awareness training, and provides a legal framework for enforcing security policies.
    \end{itemize}
\end{enumerate}

\end{document}
```