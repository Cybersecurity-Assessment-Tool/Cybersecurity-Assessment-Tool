```latex
\documentclass[12pt]{article}

% Preamble: Required Packages
\usepackage[margin=1in]{geometry} % Set page margins
\usepackage{pifont}               % For checkmarks and crosses (\ding)
\usepackage{booktabs}             % For professional-looking tables
\usepackage{hyperref}             % For clickable links and references
\usepackage{url}                  % For formatting URLs
\usepackage{seqsplit}             % For splitting long strings in texttt
\usepackage{xcolor}               % For colors

% Document Information
\title{Cybersecurity Posture Assessment Report}
\author{Cybersecurity Analysis Division}
\date{\today}

% Define colors for severity
\definecolor{criticalred}{HTML}{D10000}
\definecolor{highorange}{HTML}{E57200}

\begin{document}

\maketitle

% --- Executive Summary ---
\section*{Executive Summary}
This report provides a cybersecurity posture assessment for \textbf{[Organization Name]}, based on an analysis of network scan data, a security controls questionnaire, and a review of pre-existing risks. The assessment reveals a mixed security posture. While the external network scan of the target system showed no exposed services---a positive finding---significant gaps were identified in internal security controls.

Critical weaknesses include the absence of Multi-Factor Authentication (MFA) for computer logins and access to sensitive data systems. Furthermore, the lack of a mandatory security awareness training program for new employees presents a high risk. These gaps expose the organization to significant threats, including unauthorized access, data breaches, and social engineering attacks. This report details these findings and provides actionable recommendations to mitigate the identified risks and strengthen the overall security posture.

\newpage

% --- Organizational Information ---
\section{Organizational Information}
This section provides the high-level details of the organization under review. The information is based on the data provided for this assessment.

\begin{tabular}{@{}ll}
\toprule
\textbf{Attribute} & \textbf{Value} \\
\midrule
Organization Name & \textbf{[Organization Name]} \\
Primary Domain & \texttt{[Domain]} \\
External IP Address & \texttt{[Client IP]} \\
\bottomrule
\end{tabular}

% --- Security Control Review ---
\section{Security Control Review}
The following table summarizes the organization's responses to a security controls questionnaire. The status indicates whether the control is in place ("Yes") or not ("No"). "No" answers represent significant gaps in the security framework and are highlighted for immediate attention.

\begin{table}[h!]
\centering
\begin{tabular}{@{}p{0.6\linewidth} c p{0.2\linewidth}@{}}
\toprule
\textbf{Control Question} & \textbf{Status} & \textbf{Analysis} \\
\midrule
Do you require MFA to access email? & \ding{51} & Implemented \\
\addlinespace
Do you require MFA to log into computers? & \textbf{\color{criticalred}\ding{55}} & \textbf{Critical Gap} \\
\addlinespace
Do you require MFA to access sensitive data systems? & \textbf{\color{criticalred}\ding{55}} & \textbf{Critical Gap} \\
\addlinespace
Does your organization have an employee acceptable use policy? & \ding{51} & Implemented \\
\addlinespace
Does your organization do security awareness training for new employees? & \textbf{\color{highorange}\ding{55}} & \textbf{High Risk} \\
\addlinespace
Does your organization do security awareness training for all employees at least once per year? & \ding{51} & Implemented \\
\bottomrule
\end{tabular}
\caption{Security Controls Questionnaire Analysis}
\end{table}

% --- Technical Scan Results ---
\section{Technical Scan Results}
A network port scan was conducted to identify exposed services on the organization's external infrastructure.

\begin{itemize}
    \item \textbf{Target IP Address:} \texttt{[Target IP]}
    \item \textbf{Scan Status:} The target host was responsive (status: up).
    \item \textbf{Findings:} The scan revealed \textbf{no open ports}. All 65,535 TCP ports were found to be in a 'closed' state. This is a positive security finding, as it indicates a properly configured firewall or a host with no network services exposed to the internet, significantly reducing the external attack surface.
\end{itemize}

% --- Risk Assessment ---
\section{Risk Assessment}
This section synthesizes findings from the security control review, technical scans, and pre-existing risk data. Based on the analysis, the following new risks have been identified. No pre-existing vulnerabilities were reported in the input data.

\begin{table}[h!]
\centering
\begin{tabular}{@{}p{0.25\linewidth} p{0.55\linewidth} l@{}}
\toprule
\textbf{Risk Name} & \textbf{Overview} & \textbf{Severity} \\
\midrule
\addlinespace
Lack of Endpoint and System MFA & The absence of MFA on computer logins and sensitive data systems allows an attacker with stolen credentials to gain direct access to endpoints and critical data, bypassing a fundamental security layer. & \textbf{\color{criticalred}Critical} \\
\addlinespace
Inadequate New Hire Training & New employees are not receiving security awareness training upon joining. This makes them highly susceptible to phishing, social engineering, and unintentional policy violations from day one. & \textbf{\color{highorange}High} \\
\addlinespace
\bottomrule
\end{tabular}
\caption{Summary of Identified Risks}
\end{table}

% --- Recommendations ---
\section{Recommendations}
The following actions are recommended to mitigate the identified risks and improve the organization's security posture.

\subsection{Critical Priority}
\begin{itemize}
    \item \textbf{Implement Comprehensive MFA:}
    \begin{itemize}
        \item \textbf{Action:} Deploy a robust Multi-Factor Authentication (MFA) solution for all employee computer logins (e.g., Windows Hello for Business, Duo, Okta).
        \item \textbf{Action:} Enforce MFA for access to all systems classified as containing sensitive, confidential, or critical data, including databases, financial applications, and administrative portals.
        \item \textbf{Justification:} This is the single most effective control to prevent unauthorized access resulting from compromised credentials.
    \end{itemize}
\end{itemize}

\subsection{High Priority}
\begin{itemize}
    \item \textbf{Establish Onboarding Security Training:}
    \begin{itemize}
        \item \textbf{Action:} Integrate a mandatory security awareness training module into the new employee onboarding process. This training should be completed before access to sensitive systems is granted.
        \item \textbf{Action:} The training curriculum should cover key topics such as phishing identification, password hygiene, acceptable use of company assets, and incident reporting procedures.
        \item \textbf{Justification:} Equipping new hires with security knowledge from the start reduces the likelihood of early, preventable security incidents and fosters a culture of security.
    \end{itemize}
\end{itemize}

\end{document}
```