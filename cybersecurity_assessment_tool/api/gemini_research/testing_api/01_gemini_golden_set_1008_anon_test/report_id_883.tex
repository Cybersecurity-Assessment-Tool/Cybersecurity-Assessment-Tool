```latex
\documentclass[12pt]{article}

% Preamble: Required Packages
\usepackage[margin=1in]{geometry}
\usepackage{pifont} % For checkmarks and crosses
\usepackage{booktabs} % For professional tables
\usepackage{hyperref} % For clickable links
\usepackage{url} % For URL formatting
\usepackage{seqsplit} % For splitting long strings in tt font
\usepackage{graphicx}
\usepackage{fancyhdr}
\usepackage{xcolor}

% Document Metadata and Styling
\hypersetup{
    colorlinks=true,
    linkcolor=blue,
    filecolor=magenta,      
    urlcolor=cyan,
    pdftitle={Cybersecurity Posture Report},
    pdfpagemode=FullScreen,
}

\definecolor{darkblue}{rgb}{0.0, 0.0, 0.55}
\definecolor{darkred}{rgb}{0.55, 0.0, 0.0}

\pagestyle{fancy}
\fancyhf{}
\lhead{\textbf{Cybersecurity Posture Report}}
\rhead{\textbf{[Organization Name]}}
\cfoot{\thepage}

% --- DOCUMENT START ---
\begin{document}

\begin{titlepage}
    \centering
    \vspace*{1cm}
    
    \Huge
    \textbf{Cybersecurity Posture Report}
    
    \vspace{1.5cm}
    
    \Large
    Prepared for:
    
    \vspace{0.5cm}
    
    \textbf{\huge [Organization Name]}
    
    \vfill
    
    \large
    \today
    
\end{titlepage}

\tableofcontents
\newpage

% --- EXECUTIVE SUMMARY ---
\section{Executive Summary}

This report provides a comprehensive analysis of the cybersecurity posture for \textbf{[Organization Name]}, based on a network scan, a security controls questionnaire, and a review of pre-existing risks. The assessment reveals several critical and high-risk gaps that require immediate attention.

Key findings indicate significant deficiencies in foundational security controls. The complete absence of a security awareness training program and an employee acceptable use policy creates a high-risk environment susceptible to human error and insider threats. Furthermore, the lack of Multi-Factor Authentication (MFA) for computer and sensitive data system access is a critical vulnerability that significantly increases the risk of unauthorized access and data breaches.

The technical scan identified an exposed Secure Shell (SSH) service (Port 22) on the external network. When combined with the identified policy and access control weaknesses, this presents a tangible and attractive target for external attackers.

Immediate remediation is recommended, focusing on implementing MFA, securing the exposed SSH service, and establishing a formal security awareness program and acceptable use policy. Addressing these core issues will substantially improve the organization's defensive capabilities and overall security resilience.

% --- ORGANIZATIONAL INFORMATION ---
\section{Organizational Information}
This section details the information provided for the assessment.
\begin{itemize}
    \item \textbf{Organization Name:} \textbf{[Organization Name]}
    \item \textbf{Primary Domain:} \seqsplit{\texttt{[Domain]}}
    \item \textbf{External IP Address Assessed:} \seqsplit{\texttt{[Client IP]}}
\end{itemize}

% --- SECURITY CONTROL REVIEW ---
\section{Security Control Review}
The following table summarizes the organization's responses to a security controls questionnaire. A green checkmark (\textcolor{green}{\ding{51}}) indicates a positive control is in place, while a red cross (\textcolor{red}{\ding{55}}) indicates a control gap.

\begin{table}[h!]
\centering
\caption{Security Controls Questionnaire Results}
\begin{tabular}{p{0.7\linewidth} c}
\toprule
\textbf{Control Question} & \textbf{Response} \\
\midrule
Do you require MFA to access email? & \textcolor{green}{\ding{51}} \\
Do you require MFA to log into computers? & \textcolor{red}{\ding{55}} \\
Do you require MFA to access sensitive data systems? & \textcolor{red}{\ding{55}} \\
Does your organization have an employee acceptable use policy? & \textcolor{red}{\ding{55}} \\
Does your organization do security awareness training for new employees? & \textcolor{red}{\ding{55}} \\
Does your organization do security awareness training for all employees at least once per year? & \textcolor{red}{\ding{55}} \\
\bottomrule
\end{tabular}
\end{table}

\subsection*{Analysis}
The review reveals critical gaps in administrative and technical controls. The lack of MFA for computer and sensitive system access, coupled with the complete absence of an acceptable use policy and any form of security awareness training, represents a significant risk to the organization. These gaps make the organization highly vulnerable to phishing, credential theft, and insider threats.

% --- TECHNICAL SCAN RESULTS ---
\section{Technical Scan Results}
An external network scan was performed on the target IP address to identify open ports and exposed services.
\begin{itemize}
    \item \textbf{Target IP:} \seqsplit{\texttt{[Target IP]}}
\end{itemize}

\begin{table}[h!]
\centering
\caption{Open Port Scan Findings}
\begin{tabular}{l l l l}
\toprule
\textbf{Port} & \textbf{State} & \textbf{Service} & \textbf{Product / Version} \\
\midrule
22 & open & ssh (presumed) & Not available from scan \\
\bottomrule
\end{tabular}
\end{table}

\subsection*{Analysis}
The scan identified that port 22 is open, which is standard for the Secure Shell (SSH) protocol used for remote administration. While necessary for remote management, an SSH port exposed to the public internet is a common target for brute-force and credential-stuffing attacks. Without detailed version information, it is not possible to determine if the service is vulnerable to specific exploits, but its exposure alone constitutes a notable risk.

% --- RISK ASSESSMENT ---
\section{Risk Assessment}
This section synthesizes findings from the security control review and technical scan. No pre-existing vulnerabilities were provided for this assessment. The following table details the newly identified risks.

\begin{table}[h!]
\centering
\caption{Summary of Identified Risks}
\begin{tabular}{p{0.25\linewidth} p{0.5\linewidth} l}
\toprule
\textbf{Risk Name} & \textbf{Overview} & \textbf{Severity} \\
\midrule
\textbf{Lack of MFA} & The absence of MFA on computer logins and sensitive data systems allows an attacker with valid credentials to gain unauthorized access without a second authentication factor. & \textcolor{darkred}{\textbf{Critical}} \\
\addlinespace
\textbf{Absence of Security Awareness Program} & Without training, employees are more likely to fall victim to social engineering attacks like phishing, leading to credential compromise or malware infection. & \textcolor{red}{\textbf{High}} \\
\addlinespace
\textbf{Exposed SSH Service} & The publicly accessible SSH port is a direct vector for remote access attacks. This risk is amplified by the lack of MFA and potential for weak user passwords. & \textcolor{red}{\textbf{High}} \\
\addlinespace
\textbf{Missing Acceptable Use Policy (AUP)} & The lack of a formal AUP means there are no clear rules for employees regarding the use of company assets, which can lead to unintentional data exposure or system misuse. & \textbf{Medium} \\
\bottomrule
\end{tabular}
\end{table}

% --- RECOMMENDATIONS ---
\section{Recommendations}
Based on the identified risks, the following prioritized actions are recommended to improve the security posture of \textbf{[Organization Name]}.

\subsection*{Immediate Actions (Critical Priority)}
\begin{enumerate}
    \item \textbf{Implement MFA on Sensitive Systems:} Immediately enforce MFA for all access to systems containing sensitive data. This is the single most effective control to prevent unauthorized access.
    \item \textbf{Secure the Exposed SSH Service:} Restrict access to port 22 using a firewall, allowing connections only from trusted IP addresses (e.g., corporate VPN, specific administrator locations). Furthermore, disable password-based authentication in favor of public-key cryptography for SSH access.
\end{enumerate}

\subsection*{High Priority Actions}
\begin{enumerate}
    \setcounter{enumi}{2} % Continue numbering
    \item \textbf{Establish a Security Awareness Program:} Develop and implement a mandatory security awareness training program. All new hires must complete the training, and all employees must undergo annual refresher training.
    \item \textbf{Develop and Enforce an Acceptable Use Policy (AUP):} Create a formal AUP that all employees must read and acknowledge. This policy should clearly define the rules for using company networks, devices, and data.
    \item \textbf{Deploy MFA for Computer Logins:} Roll out MFA for all employee computer logins to protect endpoints from unauthorized use, especially in the event of credential theft.
\end{enumerate}

\subsection*{General Recommendations}
\begin{enumerate}
    \setcounter{enumi}{5} % Continue numbering
    \item \textbf{Conduct Authenticated Vulnerability Scanning:} Perform regular, authenticated vulnerability scans on all external and internal systems to identify outdated software and specific vulnerabilities that need to be patched.
\end{enumerate}

% --- DOCUMENT END ---
\end{document}
```