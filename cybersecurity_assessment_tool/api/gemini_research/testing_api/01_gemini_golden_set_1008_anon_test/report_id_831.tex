```latex
\documentclass[12pt]{article}

% ----------------------------------------------------------------------
% PREAMBLE
% ----------------------------------------------------------------------
\usepackage[margin=1in]{geometry}
\usepackage{pifont} % For checkmarks and crosses
\usepackage{booktabs} % For professional tables
\usepackage{hyperref} % For hyperlinks
\usepackage{url}      % For URL formatting
\usepackage{seqsplit} % For splitting long strings without spaces
\usepackage{graphicx}
\usepackage{xcolor}

% --- Hyperref Setup ---
\hypersetup{
    colorlinks=true,
    linkcolor=blue,
    filecolor=magenta,      
    urlcolor=cyan,
    pdftitle={Cybersecurity Assessment Report},
    pdfpagemode=FullScreen,
}

% --- Custom Commands ---
\newcommand{\yes}{\ding{51}} % Green checkmark
\newcommand{\no}{\ding{55}}  % Red cross

% --- Document Metadata ---
\title{Cybersecurity Assessment Report}
\author{Cybersecurity Analysis Division}
\date{\today}

% ----------------------------------------------------------------------
% DOCUMENT BODY
% ----------------------------------------------------------------------
\begin{document}

\maketitle
\thispagestyle{empty}
\newpage

\tableofcontents
\thispagestyle{empty}
\newpage

% ----------------------------------------------------------------------
\section{Executive Overview}
% ----------------------------------------------------------------------
This report details the findings of a cybersecurity assessment conducted for \textbf{[Organization Name]}. The assessment combined a review of organizational security controls, an external network scan, and an analysis of pre-existing risk data.

The analysis revealed several high-priority risks that require immediate attention. The most critical finding is the direct network exposure of a MySQL database service. This service is running on a version (\textbf{MySQL 5.7.33}) that has reached its End-of-Life (EOL) as of October 2023, meaning it no longer receives security updates and is highly susceptible to exploitation.

Furthermore, significant gaps were identified in the organization's security policies. The lack of mandatory Multi-Factor Authentication (MFA) for email access presents a critical vulnerability to phishing and account takeover attacks. This is compounded by the absence of a recurring, annual security awareness training program for all employees, which diminishes the organization's resilience against social engineering threats.

Collectively, these findings indicate a high-risk security posture. The recommendations provided in this report are designed to mitigate the most severe threats and establish a stronger security foundation.

% ----------------------------------------------------------------------
\section{Organizational Information}
% ----------------------------------------------------------------------
The following information was used as the basis for this assessment. Due to the anonymized nature of the provided data, placeholders have been used where necessary.

\begin{itemize}
    \item \textbf{Organization Name:} \textbf{[Organization Name]}
    \item \textbf{Primary Domain:} \texttt{[Domain]}
    \item \textbf{Assessed External IP:} \texttt{[Client IP]}
\end{itemize}

% ----------------------------------------------------------------------
\section{Security Control Review}
% ----------------------------------------------------------------------
A review of the organization's security controls was conducted via a questionnaire. The responses indicate a solid foundation in some areas, such as MFA for computer and sensitive system access. However, two critical gaps were identified that significantly increase organizational risk.

\begin{table}[h!]
\centering
\caption{Security Controls Questionnaire Results}
\begin{tabular}{p{0.75\linewidth} c}
\toprule
\textbf{Control Question} & \textbf{Response} \\
\midrule
Do you require MFA to access email? & \no \\
Do you require MFA to log into computers? & \yes \\
Do you require MFA to access sensitive data systems? & \yes \\
Does your organization have an employee acceptable use policy? & \yes \\
Does your organization do security awareness training for new employees? & \yes \\
Does your organization do security awareness training for all employees at least once per year? & \no \\
\bottomrule
\end{tabular}
\end{table}

\paragraph{Analysis:} The two "No" responses are major causes for concern.
\begin{itemize}
    \item \textbf{No MFA for Email:} Email is the primary vector for phishing attacks. Without MFA, a single compromised password can lead to a full email account takeover, data breaches, and further internal network compromise.
    \item \textbf{No Annual Security Training:} The threat landscape evolves constantly. Without annual training, employees are less likely to recognize and appropriately respond to modern phishing, ransomware, and social engineering tactics.
\end{itemize}

% ----------------------------------------------------------------------
\section{Technical Scan Results}
% ----------------------------------------------------------------------
An external network scan was performed on the target IP address to identify open ports and exposed services.

\begin{itemize}
    \item \textbf{Target IP:} \texttt{[Target IP]}
    \item \textbf{Scan Date:} Data from the most recent scan was used for this report.
\end{itemize}

\begin{table}[h!]
\centering
\caption{Open Ports Detected}
\begin{tabular}{l l l l}
\toprule
\textbf{Port} & \textbf{State} & \textbf{Service} & \textbf{Product \& Version} \\
\midrule
3306/tcp & open & mysql & MySQL 5.7.33 \\
\bottomrule
\end{tabular}
\end{table}

\paragraph{Analysis:} The scan identified that port \textbf{3306}, the default port for MySQL, is open to the public internet. This is a highly dangerous configuration as it exposes the database directly to attackers. Attackers can perform brute-force password attacks, exploit known vulnerabilities, or attempt to inject malicious code.

Crucially, the detected version, \textbf{MySQL 5.7.33}, is \textbf{End-of-Life (EOL)} as of October 2023. EOL software no longer receives security patches from the vendor, making it a prime target for exploitation of newly discovered vulnerabilities. This elevates the risk of this finding from High to \textbf{Critical}.

% ----------------------------------------------------------------------
\section{Consolidated Risk Assessment}
% ----------------------------------------------------------------------
The following table synthesizes findings from the security control review, technical scan, and pre-existing risk data into a consolidated list of key risks facing the organization.

\begin{table}[h!]
\centering
\caption{Summary of Identified Risks}
\begin{tabular}{p{0.25\linewidth} p{0.1\linewidth} p{0.35\linewidth} p{0.2\linewidth}}
\toprule
\textbf{Risk Name} & \textbf{Severity} & \textbf{Description} & \textbf{Affected Elements} \\
\midrule
\textbf{Exposed End-of-Life Database} & \textbf{Critical} & A MySQL database running an unsupported, EOL version is directly exposed to the internet, making it highly vulnerable to compromise. & \texttt{[Target IP]}:3306 \\
\addlinespace
\textbf{Lack of MFA for Email} & \textbf{Critical} & The absence of MFA on email accounts creates a high risk of account takeover via phishing or credential stuffing, leading to data breaches. & All user email accounts \\
\addlinespace
\textbf{Inadequate Security Training Program} & \textbf{High} & Without mandatory annual training, employees are more susceptible to social engineering attacks, increasing the likelihood of a security incident. & All employees \\
\bottomrule
\end{tabular}
\end{table}

% ----------------------------------------------------------------------
\section{Recommendations}
% ----------------------------------------------------------------------
The following actions are recommended to mitigate the identified risks. They are prioritized based on severity and potential impact.

\subsection{Immediate Priority: Secure the Database Server}
The exposed, end-of-life database represents an active and severe threat.
\begin{enumerate}
    \item \textbf{Immediate Containment:} Implement strict firewall rules to block all public access to TCP port 3306. Access should be restricted to only trusted IP addresses. This is the most critical first step.
    \item \textbf{Plan Upgrade:} Develop an immediate plan to upgrade the MySQL 5.7 instance to a currently supported version (e.g., MySQL 8.x) to ensure it receives security patches.
    \item \textbf{Long-Term Architecture:} Relocate the database to a private network segment. Future access should only be permitted through a secure channel, such as a Virtual Private Network (VPN) or a bastion host.
\end{enumerate}

\subsection{High Priority: Implement MFA for Email}
Protecting the primary communication channel is paramount.
\begin{enumerate}
    \item \textbf{Mandatory MFA Enforcement:} Enforce MFA for all user email accounts without exception. This single control dramatically reduces the risk of account compromise.
\end{enumerate}

\subsection{High Priority: Establish a Security Training Program}
A well-informed workforce is a critical layer of defense.
\begin{enumerate}
    \item \textbf{Implement Annual Training:} Procure and roll out a mandatory security awareness training program for all employees, to be completed annually.
    \item \textbf{Conduct Phishing Simulations:} Supplement annual training with periodic phishing simulation campaigns to test and reinforce employee knowledge in a practical setting.
\end{enumerate}

\end{document}
```