```latex
\documentclass[12pt]{article}

% Preamble: Required Packages
\usepackage[margin=1in]{geometry}
\usepackage{pifont} % For checkmarks and crosses
\usepackage{booktabs} % For professional tables
\usepackage{hyperref} % For clickable links
\usepackage{url}      % For URL formatting
\usepackage{seqsplit} % For splitting long strings
\usepackage{xcolor}   % For colors

% Document Information
\title{Cybersecurity Posture and Risk Assessment Report}
\author{Cybersecurity Analyst}
\date{\today}

% Hyperref Setup
\hypersetup{
    colorlinks=true,
    linkcolor=blue,
    filecolor=magenta,      
    urlcolor=cyan,
    pdftitle={Cybersecurity Posture and Risk Assessment Report},
    pdfpagemode=FullScreen,
}

\begin{document}

\maketitle
\thispagestyle{empty}
\newpage
\tableofcontents
\newpage

% --- 1. Executive Summary ---
\section{Executive Summary}

This report provides a comprehensive analysis of the cybersecurity posture for \textbf{[Organization Name]}. The assessment is based on a correlation of network scan data, a security controls questionnaire, and a review of pre-existing risks.

The overall security posture is assessed as \textbf{CRITICAL RISK}. This assessment is driven by two primary findings:
\begin{enumerate}
    \item \textbf{Exposed Remote Desktop Protocol (RDP):} The external network scan identified that RDP (port 3389) is open to the public internet on the primary IP address \texttt{[Client IP]}. This is a severe vulnerability, frequently exploited by threat actors for ransomware deployment and unauthorized network access.
    \item \textbf{Systemic Lack of Multi-Factor Authentication (MFA):} The organization does not enforce MFA for email, computer logins, or access to sensitive data systems. This critical control gap significantly increases the risk of account compromise and subsequent data breaches.
\end{enumerate}

The combination of an exposed, high-value service (RDP) with weak authentication controls (no MFA) creates an environment highly susceptible to compromise. Immediate remediation is required to mitigate these risks. Further recommendations are detailed in Section 6.

% --- 2. Organizational Information ---
\section{Organizational Information}

This report was prepared for the following entity. As identity data was not provided, placeholders are used.

\begin{itemize}
    \item \textbf{Organization Name:} \textbf{[Organization Name]}
    \item \textbf{Email Domain:} \texttt{[Domain]}
    \item \textbf{Primary External IP:} \texttt{[Client IP]}
\end{itemize}

% --- 3. Security Control Review ---
\section{Security Control Review}

The following table summarizes the organization's responses to a security controls questionnaire. "No" answers indicate significant gaps in the security program and are flagged for review.

\begin{table}[h!]
\centering
\caption{Security Controls Questionnaire Analysis}
\begin{tabular}{p{8cm} c l}
\toprule
\textbf{Control Question} & \textbf{Response} & \textbf{Analyst Note} \\
\midrule
Do you require MFA to access email? & \ding{55} & \textcolor{red}{\textbf{Critical Gap}} \\
Do you require MFA to log into computers? & \ding{55} & \textcolor{red}{\textbf{Critical Gap}} \\
Do you require MFA to access sensitive data systems? & \ding{55} & \textcolor{red}{\textbf{Critical Gap}} \\
Does your organization do security awareness training for all employees at least once per year? & \ding{55} & \textcolor{orange}{\textbf{High Risk}} \\
\midrule
Does your organization have an employee acceptable use policy? & \ding{51} & Good Practice \\
Does your organization do security awareness training for new employees? & \ding{51} & Good Practice \\
\bottomrule
\end{tabular}
\end{table}

The complete absence of Multi-Factor Authentication (MFA) is the most critical finding from this review. MFA is a foundational security control that prevents the vast majority of account compromise attacks. The lack of annual security awareness training for all staff increases susceptibility to phishing and social engineering attacks, which are primary methods for stealing credentials.

% --- 4. Technical Scan Results ---
\section{Technical Scan Results}

An external network scan was performed to identify exposed services. The target IP address was not specified in the scan data, so the placeholder \texttt{[Target IP]} is used. The scan revealed the following:

\begin{itemize}
    \item \textbf{Target IP Address:} \texttt{[Target IP]}
    \item \textbf{Host Status:} Up
\end{itemize}

\begin{table}[h!]
\centering
\caption{Open Ports Detected on \texttt{[Client IP]}}
\begin{tabular}{l l l l}
\toprule
\textbf{Port} & \textbf{State} & \textbf{Service Name} & \textbf{Description} \\
\midrule
3389/tcp & Open & \texttt{ms-wbt-server} & Microsoft Remote Desktop Protocol (RDP) \\
\bottomrule
\end{tabular}
\end{table}

\subsection{Analysis of Technical Findings}
The scan confirms that port 3389 (RDP) is open and accessible from the public internet. Exposing RDP directly is a well-documented and highly dangerous practice. Attackers continuously scan the internet for open RDP ports to exploit via brute-force attacks, credential stuffing, or by leveraging known vulnerabilities. This finding directly correlates with the pre-existing risk documented in the following section.

% --- 5. Correlated Risk Assessment ---
\section{Correlated Risk Assessment}

This section synthesizes findings from the security control review, technical scan, and pre-existing risk data into a unified risk register.

\begin{table}[h!]
\centering
\caption{Summary of Identified Risks}
\begin{tabular}{p{4cm} p{6cm} l}
\toprule
\textbf{Risk Name} & \textbf{Overview} & \textbf{Severity} \\
\midrule
\textbf{RDP Exposure} & Port 3389 (RDP) is exposed to the internet on \texttt{[Client IP]}, allowing direct connection attempts from any source. This is confirmed by both the network scan and pre-existing risk data. & \textcolor{red}{\textbf{Critical (9.0)}} \\
\addlinespace
\textbf{No Multi-Factor Authentication (MFA)} & The lack of MFA for email, computer, and sensitive data access means a compromised password provides an attacker with full access. This dramatically increases the risk of the exposed RDP service. & \textcolor{red}{\textbf{Critical}} \\
\addlinespace
\textbf{Insufficient Security Training} & The absence of annual security awareness training for all employees increases the likelihood of successful phishing attacks, leading to credential compromise. & \textcolor{orange}{\textbf{High}} \\
\bottomrule
\end{tabular}
\end{table}

% --- 6. Recommendations ---
\section{Recommendations}

The following actionable recommendations are provided to mitigate the identified risks. They are prioritized based on severity and ease of implementation.

\subsection{Immediate Actions (To Be Completed Within 72 Hours)}
\begin{enumerate}
    \item \textbf{Close Port 3389:} Immediately configure the perimeter firewall to block all inbound traffic to port 3389 on \texttt{[Client IP]}. This is the most critical step to prevent an imminent breach.
    \item \textbf{Implement a VPN for Remote Access:} If remote access is required, deploy a Virtual Private Network (VPN) solution. Access to the network via RDP should only be possible after a user has successfully connected to the VPN.
    \item \textbf{Enable MFA on VPN and Email:} Immediately begin the process of enabling MFA on the new VPN solution and, critically, on all email accounts (\texttt{[Domain]}). This provides a crucial layer of defense against credential theft.
\end{enumerate}

\subsection{Short-Term Actions (To Be Completed Within 30 Days)}
\begin{enumerate}
    \item \textbf{Expand MFA Deployment:} Complete the rollout of MFA to all systems, including computer logins and access to any applications storing sensitive data.
    \item \textbf{Implement Annual Security Training:} Procure and schedule mandatory annual security awareness training for all employees. This training should cover phishing, password hygiene, and acceptable use policies.
\end{enumerate}

\subsection{Long-Term Strategic Actions}
\begin{enumerate}
    \item \textbf{Establish a Vulnerability Management Program:} Implement a formal program to regularly scan internal and external assets for vulnerabilities and track them to remediation.
    \item \textbf{Conduct a Comprehensive Penetration Test:} Engage a third party to conduct a full penetration test to identify any other weaknesses that were not discovered during this high-level assessment.
\end{enumerate}

\end{document}
```