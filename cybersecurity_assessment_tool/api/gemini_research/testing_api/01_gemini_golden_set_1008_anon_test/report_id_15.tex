```latex
\documentclass[12pt]{article}

% Preamble: Required Packages
\usepackage[margin=1in]{geometry}
\usepackage{pifont} % For checkmarks and crosses
\usepackage{booktabs} % For professional tables
\usepackage{hyperref} % For hyperlinks
\usepackage{url}      % For URL formatting
\usepackage{seqsplit} % For splitting long strings in tt font

% Document Metadata
\title{Cybersecurity Posture Assessment Report}
\author{Cybersecurity Analysis Division}
\date{\today}

% Hyperref Setup
\hypersetup{
    colorlinks=true,
    linkcolor=black,
    urlcolor=blue,
    pdftitle={Cybersecurity Posture Assessment Report},
    pdfauthor={Cybersecurity Analysis Division},
}

\begin{document}

\maketitle

\section*{Executive Overview}
This report details the findings of a cybersecurity posture assessment for \textbf{[Organization Name]}. The analysis correlates results from an external network scan, a security controls questionnaire, and a review of pre-existing risks.

The assessment reveals several critical and high-risk vulnerabilities that require immediate attention. The most significant finding is the direct exposure of a MySQL database to the public internet. This database is running on an End-of-Life (EOL) version of MySQL (5.7.33), which no longer receives security patches, exponentially increasing its risk profile.

This technical vulnerability is compounded by critical gaps in organizational security controls, most notably the lack of Multi-Factor Authentication (MFA) for email and sensitive data systems. Furthermore, the absence of a formal Acceptable Use Policy indicates a need for stronger foundational security governance. The combination of these factors creates a high-risk environment susceptible to data breaches, unauthorized access, and other malicious activities.

\section{Organizational Information}
The following details were used as the basis for this assessment.
\begin{itemize}
    \item \textbf{Organization Name:} \textbf{[Organization Name]}
    \item \textbf{Primary Domain:} \texttt{[Domain]}
    \item \textbf{Target IP Address:} \texttt{[Client IP]}
\end{itemize}

\section{Security Control Review}
The following table summarizes the organization's responses to a security controls questionnaire. Responses marked with \ding{55} (No) indicate significant gaps in the current security framework and are considered high-risk findings.

\begin{table}[h!]
\centering
\caption{Security Controls Questionnaire Results}
\begin{tabular}{@{}lc@{}}
\toprule
\textbf{Control Question} & \textbf{Response} \\
\midrule
Do you require MFA to access email? & \ding{55} \\
Do you require MFA to log into computers? & \ding{51} \\
Do you require MFA to access sensitive data systems? & \ding{55} \\
Does your organization have an employee acceptable use policy? & \ding{55} \\
Does your organization do security awareness training for new employees? & \ding{51} \\
Does your organization do security awareness training for all employees at least once per year? & \ding{51} \\
\bottomrule
\end{tabular}
\end{table}

\subsection*{Analysis}
The lack of MFA for email and sensitive data systems represents a critical vulnerability. Email is a primary vector for phishing and account takeover attacks. Without MFA, compromised credentials could grant an attacker direct access to sensitive communications and systems, including the exposed database. The absence of an Acceptable Use Policy points to a maturity gap in security governance.

\section{Technical Scan Results}
An external network scan was performed against the target IP address to identify open ports and exposed services.

\begin{itemize}
    \item \textbf{Target IP Scanned:} \texttt{[Target IP]}
\end{itemize}

\begin{table}[h!]
\centering
\caption{Open Ports and Services Detected}
\begin{tabular}{@{}llll@{}}
\toprule
\textbf{Port} & \textbf{State} & \textbf{Service} & \textbf{Product \& Version} \\
\midrule
3306/tcp & open & mysql & MySQL 5.7.33 \\
\bottomrule
\end{tabular}
\end{table}

\subsection*{Analysis}
The scan confirms that port 3306 is open, exposing a MySQL database service directly to the internet. This configuration is highly discouraged as it makes the database a prime target for brute-force attacks, credential stuffing, and exploitation of known vulnerabilities.

Crucially, the detected version, \textbf{MySQL 5.7.33}, reached its official End-of-Life (EOL) in October 2023. This means it no longer receives security updates from the vendor, and any vulnerabilities discovered since that date will remain unpatched. Running EOL software, especially on an internet-facing service, constitutes a critical risk.

\section{Correlated Risk Assessment}
The following table synthesizes findings from the technical scan, control review, and pre-existing risk data into a prioritized list of correlated risks.

\begin{table}[h!]
\centering
\caption{Summary of Correlated Risks}
\begin{tabular}{@{}p{0.2\linewidth}p{0.55\linewidth}p{0.15\linewidth}@{}}
\toprule
\textbf{Risk} & \textbf{Description} & \textbf{Severity} \\
\midrule
\textbf{Exposed End-of-Life Database} & A MySQL 5.7 database is publicly accessible on port 3306. This version is EOL and unpatched. This issue is exacerbated by the lack of MFA on sensitive systems, creating a direct path for a data breach. & \textbf{Critical} \\
\addlinespace
\textbf{Inadequate Access Controls} & The absence of MFA for email and sensitive data systems drastically increases the risk of account compromise via phishing or credential theft, leading to unauthorized access. & \textbf{Critical} \\
\addlinespace
\textbf{Missing Foundational Policies} & The lack of an Acceptable Use Policy indicates a gap in security governance. This can lead to inconsistent security practices and increased insider risk. & \textbf{High} \\
\bottomrule
\end{tabular}
\end{table}

\section{Recommendations}
Based on the correlated findings, the following actions are recommended to mitigate the identified risks. They are prioritized by urgency and impact.

\subsection*{Priority 1: Immediate Actions (Within 72 Hours)}
\begin{enumerate}
    \item \textbf{Restrict Access to Database Port:} Immediately implement firewall rules to restrict all access to port 3306. Access should only be permitted from specific, trusted IP addresses. Public access must be disabled.
\end{enumerate}

\subsection*{Priority 2: High-Priority Actions (Within 30 Days)}
\begin{enumerate}
    \setcounter{enumi}{1} % Continue numbering
    \item \textbf{Implement Multi-Factor Authentication (MFA):} Deploy MFA across all critical platforms. Prioritize email (e.g., Office 365, Google Workspace) and all systems classified as containing sensitive data.
    \item \textbf{Develop and Implement an Acceptable Use Policy (AUP):} Create a formal AUP that defines the rules for using company IT assets. Ensure all employees review and formally acknowledge the policy.
\end{enumerate}

\subsection*{Priority 3: Strategic Actions (Within 90-180 Days)}
\begin{enumerate}
    \setcounter{enumi}{3} % Continue numbering
    \item \textbf{Upgrade End-of-Life Database:} Plan and execute the migration of the MySQL 5.7 database to a currently supported version (e.g., MySQL 8.x) or a managed cloud database service (e.g., AWS RDS, Azure Database for MySQL). This is the only way to ensure the service receives future security patches.
    \item \textbf{Implement Secure Remote Access:} For long-term secure database administration, deploy a Virtual Private Network (VPN). This will ensure that all remote connections to the database are encrypted and authenticated, eliminating the need for any direct public exposure.
\end{enumerate}

\end{document}
```