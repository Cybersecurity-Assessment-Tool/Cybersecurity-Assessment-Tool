```latex
\documentclass[12pt, a4paper]{article}

% Preamble: Required Packages
\usepackage[margin=1in]{geometry}
\usepackage{pifont} % For checkmarks and crosses
\usepackage{booktabs} % For professional tables
\usepackage{hyperref} % For links and document metadata
\usepackage{url} % For formatting URLs
\usepackage{seqsplit} % For splitting long strings without spaces
\usepackage{xcolor} % For colors

% Document Metadata
\hypersetup{
    colorlinks=true,
    linkcolor=blue,
    filecolor=magenta,      
    urlcolor=cyan,
    pdftitle={Cybersecurity Posture Assessment Report},
    pdfauthor={Cybersecurity Analysis Service},
    pdfsubject={Security Report},
    pdfkeywords={Security, Analysis, Risk},
}

% Define custom colors for severity
\definecolor{criticalred}{HTML}{D7263D}
\definecolor{highorange}{HTML}{F49D40}
\definecolor{mediumyellow}{HTML}{F4D440}

% Title Page Information
\title{Cybersecurity Posture Assessment Report \\ \large For \textbf{[Organization Name]}}
\author{Cybersecurity Analysis Service}
\date{\today}

\begin{document}

\maketitle
\thispagestyle{empty}
\newpage

\tableofcontents
\newpage

% --- 1. EXECUTIVE SUMMARY ---
\section*{1. Executive Summary}

This report provides a comprehensive cybersecurity assessment for \textbf{[Organization Name]}, synthesizing data from technical network scans, an organizational security controls questionnaire, and a review of pre-existing risks. The analysis reveals a high-risk security posture characterized by critical deficiencies in both technical and administrative controls.

Key findings include a publicly exposed and severely outdated FTP server allowing anonymous access, which presents an immediate and critical threat of system compromise. Furthermore, significant administrative gaps were identified, including the lack of multi-factor authentication (MFA) for sensitive data systems and the complete absence of an employee acceptable use policy and security awareness training. These weaknesses, compounded by existing risks such as outdated operating systems, create a significant and attractive attack surface for malicious actors.

Immediate remediation is required to address the critical-risk findings. Recommendations are provided to guide the mitigation of these vulnerabilities and improve the organization's overall security resilience.

% --- 2. ORGANIZATIONAL INFORMATION ---
\section*{2. Organizational Information}

This section details the information provided by the client organization. Placeholders are used where data was not available.

\begin{tabular}{@{}ll}
    \toprule
    \textbf{Attribute} & \textbf{Value} \\
    \midrule
    Organization Name & \textbf{[Organization Name]} \\
    Primary Domain & \texttt{[Domain]} \\
    External IP Address & \texttt{[Client IP]} \\
    \bottomrule
\end{tabular}

% --- 3. SECURITY CONTROL REVIEW (QUESTIONNAIRE ANALYSIS) ---
\section*{3. Security Control Review (Questionnaire Analysis)}

The following table summarizes the organization's responses to a security controls questionnaire. "No" answers indicate significant gaps in administrative controls and are flagged as high-impact risks.

\begin{tabular}{@{}p{0.6\linewidth} c p{0.25\linewidth}@{}}
    \toprule
    \textbf{Control Question} & \textbf{Response} & \textbf{Assessment} \\
    \midrule
    Do you require MFA to access email? & \ding{51} & Compliant with best practices. \\
    \addlinespace
    Do you require MFA to log into computers? & \ding{51} & Compliant with best practices. \\
    \addlinespace
    Do you require MFA to access sensitive data systems? & \textcolor{criticalred}{\ding{55}} & \textbf{Critical Gap.} Lack of MFA on sensitive systems drastically increases the risk of data breach from compromised credentials. \\
    \addlinespace
    Does your organization have an employee acceptable use policy? & \textcolor{criticalred}{\ding{55}} & \textbf{High Risk.} Without a formal policy, there are no enforceable rules for employee use of IT assets, increasing insider threat risk. \\
    \addlinespace
    Does your organization do security awareness training for new employees? & \textcolor{criticalred}{\ding{55}} & \textbf{High Risk.} New employees are not equipped to identify or respond to common threats like phishing, making them prime targets. \\
    \addlinespace
    Does your organization do security awareness training for all employees at least once per year? & \textcolor{criticalred}{\ding{55}} & \textbf{High Risk.} The lack of ongoing training ensures that security is not a top-of-mind concern, increasing susceptibility to social engineering. \\
    \bottomrule
\end{tabular}

% --- 4. TECHNICAL SCAN RESULTS ---
\section*{4. Technical Scan Results}

A network scan was performed on the target IP address \texttt{[Target IP]}. The scan identified one open port with a service containing a critical vulnerability.

\begin{tabular}{@{}llll@{}}
    \toprule
    \textbf{Port} & \textbf{Service} & \textbf{Version} & \textbf{Notes} \\
    \midrule
    21/tcp & FTP & vsftpd 2.3.4 & \textbf{CRITICAL VULNERABILITY.} This version is over a decade old and contains a known backdoor vulnerability (CVE-2011-2523). \\
    & & & \textbf{INSECURE CONFIGURATION.} The service allows anonymous FTP login, permitting unauthenticated users to access the server. \\
    \bottomrule
\end{tabular}

% --- 5. CONSOLIDATED RISK ASSESSMENT ---
\section*{5. Consolidated Risk Assessment}

This table consolidates findings from all data sources into a prioritized list of identified risks.

\begin{tabular}{@{}p{0.1\linewidth} p{0.6\linewidth} l@{}}
    \toprule
    \textbf{Risk ID} & \textbf{Description} & \textbf{Severity} \\
    \midrule
    \textbf{RISK-001} & A publicly accessible FTP server is running vsftpd 2.3.4, a version with a known critical backdoor vulnerability (CVE-2011-2523). Anonymous login is enabled. & \textcolor{criticalred}{\textbf{Critical}} \\
    \addlinespace
    \textbf{RISK-002} & Multi-factor authentication is not enforced for access to sensitive data systems, exposing critical assets to credential-based attacks. & \textcolor{criticalred}{\textbf{Critical}} \\
    \addlinespace
    \textbf{RISK-003} & The organization lacks fundamental security policies, including an Acceptable Use Policy and a security awareness training program for employees. & \textcolor{highorange}{\textbf{High}} \\
    \addlinespace
    \textbf{RISK-004} & (Pre-existing) Workstations are running Windows 7, an unsupported operating system that no longer receives security updates from Microsoft. & \textcolor{mediumyellow}{\textbf{Medium}} \\
    \bottomrule
\end{tabular}

% --- 6. RECOMMENDATIONS ---
\section*{6. Recommendations}

The following actions are recommended to mitigate the identified risks. They are prioritized based on severity and potential impact.

\subsection*{Immediate Actions (To Be Completed Within 24 Hours)}
\begin{enumerate}
    \item \textbf{Mitigate Vulnerable FTP Server (RISK-001):} Take the FTP server at \texttt{[Target IP]} offline immediately. Conduct a forensic analysis to determine if it has already been compromised. If FTP is a business requirement, deploy a new, fully patched server and ensure anonymous access is disabled.
\end{enumerate}

\subsection*{High-Priority Actions (To Be Completed Within 30 Days)}
\begin{enumerate}
    \setcounter{enumi}{1}
    \item \textbf{Implement MFA for Sensitive Systems (RISK-002):} Procure and enforce an MFA solution for all access to applications and repositories containing sensitive or critical data.
    \item \textbf{Establish Foundational Policies (RISK-003):}
        \begin{itemize}
            \item Develop and ratify an official Acceptable Use Policy (AUP) that all employees must read and sign.
            \item Procure and deploy a security awareness training module. Mandate completion for all new hires and annually for all existing staff.
        \end{itemize}
\end{enumerate}

\subsection*{Medium-Priority Actions (Ongoing)}
\begin{enumerate}
    \setcounter{enumi}{3}
    \item \textbf{Address Outdated Operating Systems (RISK-004):} Continue with the planned project to upgrade or replace all workstations running Windows 7. Isolate any remaining Windows 7 machines from the main network if they cannot be immediately upgraded.
\end{enumerate}

\end{document}
```