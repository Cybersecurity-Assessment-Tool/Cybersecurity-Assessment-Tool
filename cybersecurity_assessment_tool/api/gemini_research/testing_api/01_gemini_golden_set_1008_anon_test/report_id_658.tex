```latex
\documentclass[12pt]{article}

% Preamble: Required Packages
\usepackage[margin=1in]{geometry}
\usepackage{pifont} % For checkmarks and crosses
\usepackage{booktabs} % For professional tables
\usepackage{hyperref} % For clickable links
\usepackage{url} % For URL formatting
\usepackage{seqsplit} % To split long strings without breaking
\usepackage{graphicx}
\usepackage[table]{xcolor}
\usepackage{fancyhdr}
\usepackage{lastpage}

% --- Document Setup ---
\hypersetup{
    colorlinks=true,
    linkcolor=blue,
    filecolor=magenta,      
    urlcolor=cyan,
    pdftitle={Cybersecurity Assessment Report},
    pdfauthor={Cybersecurity Analyst},
    pdfsubject={Security Analysis},
    pdfkeywords={Cybersecurity, Report, Analysis},
}

% --- Custom Colors ---
\definecolor{tablehead}{gray}{0.9}
\definecolor{critical}{RGB}{217, 83, 79}
\definecolor{high}{RGB}{240, 173, 78}
\definecolor{medium}{RGB}{91, 192, 222}
\definecolor{low}{RGB}{92, 184, 92}

% --- Header and Footer ---
\pagestyle{fancy}
\fancyhf{} % Clear all header and footer fields
\fancyhead[L]{Cybersecurity Assessment Report}
\fancyhead[R]{\textbf{[Organization Name]}}
\fancyfoot[C]{\thepage\ of \pageref{LastPage}}
\renewcommand{\headrulewidth}{0.4pt}
\renewcommand{\footrulewidth}{0.4pt}

\begin{document}

% --- Title Page ---
\begin{titlepage}
    \centering
    \vspace*{1cm}
    \includegraphics[width=0.3\textwidth]{example-image-a} % Placeholder logo
    
    \vspace{1.5cm}
    
    \Huge
    \textbf{Cybersecurity Assessment Report}
    
    \vspace{1.5cm}
    
    \Large
    Prepared for: \textbf{[Organization Name]}
    
    \vspace{2cm}
    
    \large
    Date of Report: \today
    
    \vfill
    
    \normalsize
    \textit{This report contains sensitive information and should be handled with the utmost confidentiality. Distribution is restricted to authorized personnel only.}
    
\end{titlepage}

\tableofcontents
\newpage

% --- Executive Summary ---
\section*{Executive Summary}

This report details the findings of a cybersecurity assessment conducted for \textbf{[Organization Name]}. The analysis is based on a combination of an external network scan, a review of organizational security controls via a questionnaire, and an evaluation of pre-existing risk data.

The assessment identified several critical and high-risk gaps in the organization's security posture. Most notably, Multi-Factor Authentication (MFA) is not enforced for accessing email or other sensitive data systems. This exposes the organization to significant risks of unauthorized access and data breaches. Furthermore, the absence of a formal Acceptable Use Policy and a lack of mandatory, annual security awareness training for all employees weaken the overall security culture and leave the organization vulnerable to human error.

On a positive note, the external network vulnerability scan of the provided target IP address did not identify any open ports or exposed services, suggesting a strong network perimeter configuration for the asset scanned.

Immediate action is recommended to address the identified control gaps, focusing on the rapid implementation of MFA across all critical platforms and the development of foundational security policies and training programs.

\newpage

% --- Organizational Information ---
\section{Organizational Information}

This section provides the key identification details for the organization under review. As the provided data was anonymized, placeholders have been used where necessary.

\begin{tabular}{@{}ll}
\toprule
\textbf{Attribute} & \textbf{Value} \\
\midrule
Organization Name & \textbf{[Organization Name]} \\
Primary Email Domain & \texttt{[Domain]} \\
External IP Address & \texttt{[Client IP]} \\
\bottomrule
\end{tabular}

% --- Security Control Review ---
\section{Security Control Review}

The following table summarizes the organization's responses to a security controls questionnaire. A green checkmark (\textcolor{green}{\ding{51}}) indicates a positive control is in place, while a red cross (\textcolor{red}{\ding{55}}) indicates a control gap that introduces risk.

\begin{table}[h!]
\centering
\rowcolors{2}{gray!10}{white}
\begin{tabular}{p{0.8\linewidth} c}
\toprule
\rowcolor{tablehead}
\textbf{Control Question} & \textbf{Status} \\
\midrule
Do you require MFA to log into computers? & \textcolor{green}{\ding{51}} \\
Does your organization do security awareness training for new employees? & \textcolor{green}{\ding{51}} \\
\addlinespace[3pt]
\rowcolor{red!10}
Do you require MFA to access email? & \textcolor{red}{\ding{55}} \\
\addlinespace[3pt]
\rowcolor{red!10}
Do you require MFA to access sensitive data systems? & \textcolor{red}{\ding{55}} \\
\addlinespace[3pt]
\rowcolor{red!10}
Does your organization have an employee acceptable use policy? & \textcolor{red}{\ding{55}} \\
\addlinespace[3pt]
\rowcolor{red!10}
Does your organization do security awareness training for all employees at least once per year? & \textcolor{red}{\ding{55}} \\
\bottomrule
\end{tabular}
\caption{Organizational Security Controls Questionnaire Results.}
\end{table}

The identified gaps, marked with \textcolor{red}{\ding{55}}, are significant and are detailed further in the Risk Assessment section of this report.

% --- Technical Scan Results ---
\section{Technical Scan Results}

An external network scan was performed to identify potential vulnerabilities on the public-facing infrastructure.

\begin{itemize}
    \item \textbf{Target IP Address:} \texttt{[Target IP]}
    \item \textbf{Scan Date:} Not specified in scan data.
    \item \textbf{Summary:} The scan completed successfully and \textbf{found no open TCP or UDP ports}. This is a positive security finding, indicating that the network firewall or security group for this specific IP address is configured to deny unsolicited inbound traffic, effectively minimizing its external attack surface.
\end{itemize}

\textbf{Note:} While the scanned asset appears secure, this result does not provide visibility into other external assets or internal network vulnerabilities.

% --- Risk Assessment ---
\section{Risk Assessment}

This section synthesizes the findings from the security control review and technical scan. No pre-existing vulnerabilities were provided for this assessment. The primary risks identified stem from gaps in administrative and technical controls.

\begin{table}[h!]
\centering
\begin{tabular}{p{0.25\linewidth} p{0.55\linewidth} p{0.1\linewidth}}
\toprule
\rowcolor{tablehead}
\textbf{Risk Name} & \textbf{Overview} & \textbf{Severity} \\
\midrule
\rowcolor{critical!15}
No MFA for Email Access & Lack of MFA on email accounts (e.g., Office 365, Google Workspace) makes them highly susceptible to compromise via phishing or password spraying. A compromised email account is a primary vector for further network intrusion and data exfiltration. & \textbf{Critical} \\
\addlinespace[5pt]
\rowcolor{critical!15}
No MFA for Sensitive Data Systems & Failure to protect systems containing sensitive corporate or customer data with MFA allows an attacker with stolen credentials to gain direct access, leading to a high-impact data breach. & \textbf{Critical} \\
\addlinespace[5pt]
\rowcolor{high!15}
Lack of Acceptable Use Policy (AUP) & Without a formal AUP, there are no clear, enforceable rules for employees regarding the use of company assets. This increases the risk of insider threat, misuse of resources, and legal liability. & \textbf{High} \\
\addlinespace[5pt]
\rowcolor{high!15}
Insufficient Security Awareness Training & Only training new hires and failing to provide annual refresher training for all employees allows security knowledge to become outdated. Staff are less likely to recognize and report modern phishing and social engineering attacks. & \textbf{High} \\
\bottomrule
\end{tabular}
\caption{Summary of Identified Risks.}
\end{table}

% --- Recommendations ---
\section{Recommendations}

Based on the risk assessment, the following actions are recommended to improve the cybersecurity posture of \textbf{[Organization Name]}. Recommendations are prioritized based on the severity of the associated risk.

\begin{enumerate}
    \item \textbf{[Critical] Implement MFA for Email and Sensitive Systems:}
    \begin{itemize}
        \item Immediately enable and enforce MFA for all user accounts with access to the primary email system.
        \item Identify all systems containing sensitive data (e.g., financial, customer, HR) and enforce MFA for all access, including administrative and user-level.
    \end{itemize}

    \item \textbf{[High] Develop and Implement an Acceptable Use Policy (AUP):}
    \begin{itemize}
        \item Draft a comprehensive AUP that clearly defines the rules for using company networks, computers, email, and internet access.
        \item Require all current employees to read and formally acknowledge the policy. Incorporate this step into the onboarding process for new hires.
    \end{itemize}
    
    \item \textbf{[High] Establish a Mandatory Annual Security Training Program:}
    \begin{itemize}
        \item Procure or develop a security awareness training program that covers key topics such as phishing, password security, and data handling.
        \item Mandate that all employees complete this training annually and track completion to ensure compliance.
    \end{itemize}
\end{enumerate}

\end{document}
```