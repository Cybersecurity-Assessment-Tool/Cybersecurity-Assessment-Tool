```latex
\documentclass[12pt]{article}

% --- PACKAGES ---
\usepackage[margin=1in]{geometry}
\usepackage{pifont} % For checkmarks and crosses
\usepackage{booktabs} % For professional tables
\usepackage{url} % For URL formatting
\usepackage{seqsplit} % To split long strings in tt font
\usepackage[
    colorlinks=true,
    linkcolor=black,
    urlcolor=blue,
    pdftitle={Cybersecurity Posture Assessment},
    pdfauthor={Cybersecurity Analyst},
]{hyperref}

% --- DOCUMENT ---
\begin{document}

% --- TITLE PAGE ---
\begin{titlepage}
    \centering
    \vspace*{\fill}
    \Huge{\textbf{Cybersecurity Posture Assessment}}
    \vspace{1.5cm}
    \Large{\textbf{For: \textbf{[Organization Name]}}}
    \vspace{2cm}
    \large{Report Date: \today}
    \vspace*{\fill}
    \small{This report contains sensitive information and should be handled with care. Distribution is restricted to authorized personnel only.}
\end{titlepage}

\tableofcontents
\newpage

% --- EXECUTIVE SUMMARY ---
\section*{1. Executive Summary}

This report provides a cybersecurity assessment for \textbf{[Organization Name]}, based on an analysis of network scan data, a security controls questionnaire, and a review of pre-existing risks. The assessment reveals several critical and high-risk issues that require immediate attention.

The most critical finding is a publicly exposed MySQL database service on port 3306 at \texttt{[Target IP]}. This service is running MySQL version 5.7.33, which is an End-of-Life (EOL) product and no longer receives security updates. This exposure creates a direct and severe risk of data breach, compromise, and service disruption.

Furthermore, significant organizational policy gaps were identified. The absence of a formal Acceptable Use Policy (AUP) and the lack of mandatory, annual security awareness training for all employees weaken the organization's human firewall. These procedural deficiencies exacerbate technical risks by increasing the likelihood of human error leading to a security incident.

Immediate remediation should focus on restricting access to the exposed database. Concurrently, efforts must be made to upgrade the EOL software and address the identified policy and training gaps to build a more resilient security posture.

% --- ORGANIZATIONAL INFORMATION ---
\section*{2. Organizational Information}

This section details the information provided for the assessment. Placeholders are used where data was not available.

\begin{tabular}{@{}ll}
    \toprule
    \textbf{Attribute} & \textbf{Value} \\
    \midrule
    Organization Name & \textbf{[Organization Name]} \\
    Primary Domain & \texttt{[Domain]} \\
    External IP Scanned & \texttt{[Client IP]} \\
    \bottomrule
\end{tabular}

% --- SECURITY CONTROL REVIEW ---
\section*{3. Security Control Review}

The following table summarizes the organization's responses to the security controls questionnaire. A green checkmark (\ding{51}) indicates a positive control is in place, while a red cross (\ding{55}) indicates a gap.

\begin{tabular}{@{}p{0.8\linewidth}c@{}}
    \toprule
    \textbf{Control Question} & \textbf{Status} \\
    \midrule
    Do you require MFA to access email? & \ding{51} \\
    Do you require MFA to log into computers? & \ding{51} \\
    Do you require MFA to access sensitive data systems? & \ding{51} \\
    Does your organization have an employee acceptable use policy? & \textcolor{red}{\ding{55}} \\
    Does your organization do security awareness training for new employees? & \ding{51} \\
    Does your organization do security awareness training for all employees at least once per year? & \textcolor{red}{\ding{55}} \\
    \bottomrule
\end{tabular}

\subsection*{Analysis of Gaps}
The organization has implemented strong Multi-Factor Authentication (MFA) controls across key systems, which is commendable. However, two significant procedural gaps were identified:
\begin{itemize}
    \item \textbf{No Acceptable Use Policy (AUP):} An AUP is a foundational document that sets clear expectations for employee behavior when using company resources. Its absence creates ambiguity and legal/compliance risks.
    \item \textbf{No Annual Security Awareness Training:} While new hires are trained, the lack of an annual refresher program for all staff is a major weakness. The threat landscape evolves constantly, and so must employee awareness. This gap increases susceptibility to phishing, social engineering, and other human-targeted attacks.
\end{itemize}

% --- TECHNICAL SCAN RESULTS ---
\section*{4. Technical Scan Results}

An external network scan was performed on the target IP address. The results indicate one open port, which presents a critical risk.

\subsection*{Target Host}
\textbf{IP Address:} \texttt{[Target IP]}

\subsection*{Open Ports Discovered}
\begin{tabular}{@{}lllll@{}}
    \toprule
    \textbf{Port} & \textbf{State} & \textbf{Service} & \textbf{Product} & \textbf{Version} \\
    \midrule
    3306/tcp & open & mysql & MySQL & 5.7.33 \\
    \bottomrule
\end{tabular}

\subsection*{Technical Analysis}
The scan confirms that port 3306 is open to the public internet, exposing a MySQL database service. This finding correlates directly with the pre-existing risk "Database Exposure".

\textbf{Critical Finding:} The detected version, MySQL 5.7.33, reached its official End-of-Life (EOL) in October 2023. EOL software no longer receives security patches from the vendor, meaning any vulnerabilities discovered after this date will remain unpatched. Running EOL software, especially on a publicly accessible service, constitutes a critical and unacceptable risk. Attackers actively scan for such systems and can exploit known vulnerabilities to gain unauthorized access, exfiltrate data, or deploy ransomware.

% --- RISK ASSESSMENT SUMMARY ---
\section*{5. Risk Assessment Summary}

The following table synthesizes findings from the questionnaire, technical scan, and pre-existing risk data into a prioritized list of current security risks.

\begin{tabular}{@{}p{0.25\linewidth}p{0.55\linewidth}l@{}}
    \toprule
    \textbf{Risk Name} & \textbf{Description} & \textbf{Severity} \\
    \midrule
    \textbf{Exposed End-of-Life Database Service} & Port 3306 is open to the internet, running an unsupported version of MySQL (5.7.33). This allows attackers to directly target the database with known exploits. & \textbf{Critical} \\
    \addlinespace
    \textbf{Inadequate Security Awareness Program} & The lack of mandatory annual training for all employees increases the risk of successful phishing and social engineering attacks, which are primary initial access vectors. & \textbf{High} \\
    \addlinespace
    \textbf{Lack of Acceptable Use Policy} & Without a formal AUP, there are no defined rules for the use of company IT assets, creating insider threat risks and potential non-compliance with regulations. & \textbf{High} \\
    \bottomrule
\end{tabular}

% --- RECOMMENDATIONS ---
\section*{6. Recommendations}

The following actions are recommended to mitigate the identified risks. They are prioritized based on severity and potential impact.

\subsection*{Immediate Priority (To Be Completed Within 72 Hours)}
\begin{enumerate}
    \item \textbf{Restrict Access to Port 3306:} Immediately apply firewall rules to block all public access to port 3306 on \texttt{[Target IP]}. Access should be restricted to a whitelist of trusted IP addresses or, preferably, be made available only through a secure Virtual Private Network (VPN).
\end{enumerate}

\subsection*{High Priority (To Be Completed Within 30-60 Days)}
\begin{enumerate}
    \item \textbf{Upgrade End-of-Life Software:} Plan and execute the migration of the MySQL 5.7.33 database to a currently supported version (e.g., MySQL 8.x). This is essential for receiving future security patches.
    \item \textbf{Implement Annual Security Training:} Procure and roll out a mandatory security awareness training program for all employees, to be completed annually. This program should cover topics like phishing, password hygiene, and incident reporting.
    \item \textbf{Develop and Enforce an AUP:} Draft an Acceptable Use Policy in consultation with HR and legal counsel. Require all employees to read and formally acknowledge the policy.
\end{enumerate}

\end{document}
```