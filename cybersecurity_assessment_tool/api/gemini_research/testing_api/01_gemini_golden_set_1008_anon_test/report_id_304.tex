```latex
\documentclass[12pt]{article}

% Preamble: Required Packages
\usepackage[margin=1in]{geometry}
\usepackage{pifont} % For checkmarks and crosses
\usepackage{booktabs} % For professional tables
\usepackage{hyperref} % For clickable links
\usepackage{url} % For URL formatting
\usepackage{seqsplit} % For splitting long strings
\usepackage{xcolor} % For colors

% Document Metadata
\title{Cybersecurity Posture Assessment Report}
\author{Cybersecurity Analysis Division}
\date{\today}

% Hyperref Setup
\hypersetup{
    colorlinks=true,
    linkcolor=blue,
    filecolor=magenta,      
    urlcolor=cyan,
    pdftitle={Cybersecurity Posture Assessment Report},
    pdfpagemode=FullScreen,
}

\begin{document}

\maketitle
\tableofcontents
\newpage

% --- 1. EXECUTIVE SUMMARY ---
\section{Executive Summary}

This report provides a comprehensive analysis of the cybersecurity posture for \textbf{[Organization Name]}. The assessment combines a review of organizational security controls, an external network vulnerability scan, and an evaluation of pre-existing risks.

The key finding of this assessment is a significant disparity between the organization's external and internal security postures. The external network perimeter appears strong, with no open ports detected on the scanned public-facing IP address. This suggests a well-configured firewall and a commendable "default deny" stance.

However, the review of internal security controls reveals critical gaps that expose the organization to substantial risk. The absence of Multi-Factor Authentication (MFA) for computer logins and access to sensitive data systems is a primary concern. Furthermore, the lack of mandatory security awareness training for new employees creates a significant vulnerability to social engineering and phishing attacks. These internal weaknesses could allow an attacker who gains a foothold (e.g., via a compromised credential) to move laterally and access critical assets with minimal resistance.

Immediate remediation should focus on implementing a comprehensive MFA strategy and integrating security training into the employee onboarding process.

% --- 2. ORGANIZATIONAL INFORMATION ---
\section{Organizational Information}

This section details the organizational data used as the basis for this assessment. Due to the nature of the data provided, placeholders have been used where specific information was not available.

\begin{itemize}
    \item \textbf{Organization Name:} \textbf{[Organization Name]}
    \item \textbf{Primary Domain:} \texttt{[Domain]}
    \item \textbf{Scanned External IP:} \texttt{[Client IP]}
\end{itemize}

% --- 3. SECURITY CONTROL REVIEW ---
\section{Security Control Review}

A review of the organization's security policies and controls was conducted via a questionnaire. The responses are summarized below. A green checkmark (\textcolor{green}{\ding{51}}) indicates a positive control is in place, while a red cross (\textcolor{red}{\ding{55}}) indicates a control gap that introduces risk.

\begin{table}[h!]
\centering
\caption{Security Controls Questionnaire Results}
\begin{tabular}{p{0.75\linewidth} c}
\toprule
\textbf{Control Question} & \textbf{Response} \\
\midrule
Do you require MFA to access email? & \textcolor{green}{\ding{51}} \\
Do you require MFA to log into computers? & \textcolor{red}{\ding{55}} \\
Do you require MFA to access sensitive data systems? & \textcolor{red}{\ding{55}} \\
Does your organization have an employee acceptable use policy? & \textcolor{green}{\ding{51}} \\
Does your organization do security awareness training for new employees? & \textcolor{red}{\ding{55}} \\
Does your organization do security awareness training for all employees at least once per year? & \textcolor{green}{\ding{51}} \\
\bottomrule
\end{tabular}
\end{table}

\subsection*{Analysis of Control Gaps}
The questionnaire reveals three significant control deficiencies:
\begin{itemize}
    \item \textbf{Lack of MFA for Computer Logins:} This is a critical vulnerability. If an employee's credentials are stolen, an attacker can directly access their workstation and any connected network resources without a secondary authentication challenge.
    \item \textbf{Lack of MFA for Sensitive Systems:} Failure to protect sensitive data systems with MFA means that a single compromised password could lead directly to a major data breach.
    \item \textbf{No Security Training for New Hires:} New employees are often prime targets for phishing and social engineering attacks. Without immediate training, they are unaware of organizational policies and common threats, making them a high-risk user group.
\end{itemize}

% --- 4. TECHNICAL SCAN RESULTS ---
\section{Technical Scan Results}

An external network scan was performed to identify open ports and exposed services on the organization's public-facing infrastructure.

\begin{itemize}
    \item \textbf{Target IP Address:} \texttt{[Target IP]}
    \item \textbf{Scan Tool:} Nmap
    \item \textbf{Scan Date:} \today
\end{itemize}

\subsection*{Findings}
The scan results were positive, indicating a strong perimeter security posture.
\begin{itemize}
    \item \textbf{Host Status:} Up
    \item \textbf{Open Ports:} None detected.
    \item \textbf{Port State:} All 1000 commonly scanned TCP ports were found to be in a \textbf{`closed`} state. This means the firewall is correctly configured to reject unsolicited incoming traffic, significantly reducing the external attack surface.
\end{itemize}

% --- 5. RISK ASSESSMENT SUMMARY ---
\section{Risk Assessment Summary}

This section synthesizes findings from the security control review and technical scan to present a consolidated list of identified risks. No pre-existing vulnerabilities were reported.

\begin{table}[h!]
\centering
\caption{Identified Cybersecurity Risks}
\begin{tabular}{p{0.2\linewidth} p{0.55\linewidth} p{0.15\linewidth}}
\toprule
\textbf{Risk Name} & \textbf{Overview} & \textbf{Severity} \\
\midrule
\textbf{Inadequate MFA for Endpoint Access} & The absence of MFA on computer logins exposes the organization to unauthorized access via stolen credentials, enabling lateral movement and endpoint compromise. & \textbf{Critical} \\
\addlinespace
\textbf{Inadequate MFA for Data Systems} & Sensitive data systems are not protected by MFA. A single compromised password could grant an attacker direct access, likely resulting in a significant data breach. & \textbf{Critical} \\
\addlinespace
\textbf{Missing New-Hire Security Training} & New employees are not receiving security awareness training upon being hired. This makes them highly susceptible to social engineering attacks and unintentional policy violations. & \textbf{High} \\
\bottomrule
\end{tabular}
\end{table}

% --- 6. RECOMMENDATIONS ---
\section{Recommendations}

Based on the analysis, the following actions are recommended to mitigate the identified risks and improve the overall security posture. Recommendations are prioritized based on severity.

\begin{enumerate}
    \item \textbf{Deploy MFA for All Endpoint Logins (Critical):}
    \begin{itemize}
        \item \textbf{Action:} Mandate and deploy a robust MFA solution for all employee and contractor computer logins.
        \item \textbf{Justification:} This action directly mitigates the risk of unauthorized access from compromised credentials, which is a primary vector in ransomware and data breach incidents. Solutions can include Windows Hello for Business, Duo Security, or other identity provider integrations.
    \end{itemize}
    \vspace{0.5cm}
    \item \textbf{Enforce MFA on Sensitive Data Systems (Critical):}
    \begin{itemize}
        \item \textbf{Action:} Conduct an inventory of all systems containing sensitive, proprietary, or regulated data. Prioritize and enforce MFA for all user access to these systems.
        \item \textbf{Justification:} This creates a critical defense-in-depth layer, ensuring that even if an attacker compromises a user's account, they cannot access the organization's most valuable data without a second factor.
    \end{itemize}
    \vspace{0.5cm}
    \item \textbf{Integrate Security Training into Employee Onboarding (High):}
    \begin{itemize}
        \item \textbf{Action:} Develop a mandatory security awareness training module that is a required component of the new-hire onboarding process. This training should be completed before a new employee is granted full network access.
        \item \textbf{Justification:} This ensures that all employees, from day one, understand their security responsibilities, company policies, and how to identify and report common threats like phishing.
    \end{itemize}
\end{enumerate}

\end{document}
```