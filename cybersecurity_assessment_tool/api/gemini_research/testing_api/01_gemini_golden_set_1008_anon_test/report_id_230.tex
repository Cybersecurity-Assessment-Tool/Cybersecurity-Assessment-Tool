```latex
\documentclass[12pt]{article}

% Preamble: Required Packages
\usepackage[a4paper, margin=1in]{geometry}
\usepackage{pifont} % For checkmarks and crosses
\usepackage{booktabs} % For professional tables
\usepackage{hyperref} % For clickable links
\usepackage{url} % For formatting URLs
\usepackage{seqsplit} % For splitting long strings
\usepackage{graphicx}
\usepackage{amsmath}

% Document Metadata
\title{Cybersecurity Risk Assessment Report \\ \large for \textbf{[Organization Name]}}
\author{Cybersecurity Analysis Division}
\date{\today}

\hypersetup{
    colorlinks=true,
    linkcolor=black,
    urlcolor=blue,
    pdftitle={Cybersecurity Risk Assessment Report},
    pdfauthor={Cybersecurity Analysis Division},
}

\begin{document}

\maketitle
\thispagestyle{empty}
\newpage

\tableofcontents
\newpage

% --- 1. Executive Summary ---
\section*{1. Executive Summary}

This report provides a comprehensive analysis of the cybersecurity posture of \textbf{[Organization Name]}, based on network scans, a review of organizational security controls, and pre-existing risk data.

The assessment has identified a \textbf{CRITICAL} risk that requires immediate attention: a publicly accessible FTP server on host \texttt{[Target IP]} is running a dangerously outdated version of \texttt{vsftpd} (2.3.4). This specific version contains a well-known backdoor vulnerability (CVE-2011-2523) which could allow an attacker to gain complete control of the server. This risk is amplified by the fact that anonymous FTP login is enabled, allowing unauthenticated access.

Furthermore, significant gaps were identified in internal security controls. The lack of Multi-Factor Authentication (MFA) on employee computers, the absence of a formal Acceptable Use Policy, and a complete lack of security awareness training represent \textbf{HIGH} risks to the organization. These procedural and policy-based failings make the organization highly vulnerable to credential theft, insider threats, and social engineering attacks like phishing.

The pre-existing risk of outdated Windows 7 workstations remains a medium-level concern that should be addressed as part of a broader security hardening initiative.

Immediate remediation of the vulnerable FTP server is paramount. Following this, we strongly recommend a focused effort to implement foundational security controls, including MFA, policy development, and employee training, to significantly improve the organization's overall resilience against cyber threats.

% --- 2. Organizational Information ---
\section*{2. Organizational Information}

This section details the information provided for the scope of this assessment. Due to the anonymized nature of the input data, placeholders have been used where necessary.

\begin{itemize}
    \item \textbf{Organization Name:} \textbf{[Organization Name]}
    \item \textbf{Primary Domain:} \texttt{[Domain]}
    \item \textbf{External IP Scanned:} \texttt{[Client IP]}
\end{itemize}

% --- 3. Security Control Review ---
\section*{3. Security Control Review}

A review of the organization's security controls was conducted via a questionnaire. The responses indicate several significant gaps in foundational security practices. A "No" response highlights a missing control and a potential area of high risk.

\begin{table}[h!]
\centering
\caption{Security Controls Questionnaire Results}
\begin{tabular}{p{0.7\linewidth} c}
\toprule
\textbf{Control Question} & \textbf{Status} \\
\midrule
Do you require MFA to access email? & \ding{51} \\
Do you require MFA to log into computers? & \textbf{\color{red}\ding{55}} \\
Do you require MFA to access sensitive data systems? & \ding{51} \\
Does your organization have an employee acceptable use policy? & \textbf{\color{red}\ding{55}} \\
Does your organization do security awareness training for new employees? & \textbf{\color{red}\ding{55}} \\
Does your organization do security awareness training for all employees at least once per year? & \textbf{\color{red}\ding{55}} \\
\bottomrule
\end{tabular}
\end{table}

% --- 4. Technical Scan Results ---
\section*{4. Technical Scan Results}

An external network scan was performed on the target IP address \texttt{[Target IP]}. The scan identified one open port with a critically vulnerable service.

\begin{table}[h!]
\centering
\caption{Nmap Scan Findings for \texttt{[Target IP]}}
\begin{tabular}{l l l l p{0.4\linewidth}}
\toprule
\textbf{Port} & \textbf{State} & \textbf{Service} & \textbf{Version} & \textbf{Details} \\
\midrule
21/tcp & Open & ftp & vsftpd 2.3.4 & \textbf{Critical Vulnerability.} Anonymous login is allowed. This version is vulnerable to a known backdoor (CVE-2011-2523) that allows for remote code execution. \\
\bottomrule
\end{tabular}
\end{table}

% --- 5. Synthesized Risk Assessment ---
\section*{5. Synthesized Risk Assessment}

The following table correlates findings from the technical scan, the controls review, and pre-existing data into a prioritized list of identified risks.

\begin{table}[h!]
\centering
\caption{Summary of Identified Risks}
\begin{tabular}{p{0.3\linewidth} l p{0.5\linewidth}}
\toprule
\textbf{Risk Title} & \textbf{Severity} & \textbf{Description} \\
\midrule
\textbf{Exposed Vulnerable FTP Server} & \textbf{Critical} & A public-facing FTP server is running \texttt{vsftpd 2.3.4}, which contains a backdoor vulnerability (CVE-2011-2523). Anonymous login is also enabled, removing the first line of defense. \\
\addlinespace
\textbf{Inadequate Endpoint Access Control} & \textbf{High} & The absence of Multi-Factor Authentication (MFA) on employee computers exposes the organization to unauthorized access via stolen or weak credentials. \\
\addlinespace
\textbf{Insufficient Security Awareness Program} & \textbf{High} & With no security training, employees are significantly more likely to fall victim to phishing, malware, and social engineering attacks, making them the weakest link in the defense chain. \\
\addlinespace
\textbf{Lack of Security Governance Policy} & \textbf{High} & The absence of a formal Acceptable Use Policy creates ambiguity and a lack of accountability for employee actions on corporate IT systems. \\
\addlinespace
\textbf{Outdated Operating Systems} & \textbf{Medium} & Workstations are running Windows 7, an end-of-life OS that no longer receives security updates, exposing them to a wide range of unpatchable vulnerabilities. \\
\bottomrule
\end{tabular}
\end{table}

% --- 6. Recommendations ---
\section*{6. Recommendations}

Based on the risk assessment, the following actions are recommended to mitigate the identified vulnerabilities and improve the overall security posture of \textbf{[Organization Name]}. Recommendations are prioritized by severity.

\subsection*{Immediate Actions (Critical Priority)}
\begin{itemize}
    \item \textbf{Remediate Vulnerable FTP Server:}
    \begin{enumerate}
        \item Immediately take the FTP server at \texttt{[Target IP]} offline or place it behind a firewall to block all external access.
        \item If the service is business-critical, upgrade the \texttt{vsftpd} software to the latest stable version and reconfigure it to disable anonymous access.
        \item If the service is not business-critical, decommission the server permanently.
    \end{enumerate}
\end{itemize}

\subsection*{High Priority Actions}
\begin{itemize}
    \item \textbf{Deploy Endpoint MFA:} Implement and enforce a mandatory MFA policy for all employee computer logins. This is one of the most effective controls for preventing unauthorized access.
    \item \textbf{Develop and Implement Security Policies:} Draft and formally adopt a comprehensive Acceptable Use Policy (AUP). Ensure all employees read and acknowledge the policy as a condition of using company IT resources.
    \item \textbf{Establish a Security Awareness Program:}
    \begin{enumerate}
        \item Procure and implement a security awareness training solution.
        \item Enroll all current employees in foundational training immediately.
        \item Integrate security training into the onboarding process for all new hires.
        \item Schedule mandatory annual refresher training for all staff.
    \end{enumerate}
\end{itemize}

\subsection*{Medium Priority Actions}
\begin{itemize}
    \item \textbf{Upgrade Outdated Workstations:} Prioritize and execute the plan to upgrade or replace all workstations running Windows 7 with a modern, supported operating system (e.g., Windows 10/11) to ensure they receive critical security patches.
\end{itemize}

\end{document}
```