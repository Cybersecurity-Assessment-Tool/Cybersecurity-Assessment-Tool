```latex
\documentclass[12pt, a4paper]{article}

% Preamble: Required Packages
\usepackage[margin=1in]{geometry}
\usepackage{pifont} % For checkmarks and crosses
\usepackage{booktabs} % For professional tables
\usepackage{hyperref} % For clickable links
\usepackage{url} % For formatting URLs
\usepackage{seqsplit} % For splitting long strings in texttt
\usepackage{graphicx}
\usepackage{xcolor}
\usepackage{fancyhdr}

% Document Metadata
\title{Cybersecurity Posture Assessment Report \\ \large For \textbf{[Organization Name]}}
\author{Cybersecurity Analyst}
\date{\today}

% Hyperref Setup
\hypersetup{
    colorlinks=true,
    linkcolor=blue,
    filecolor=magenta,      
    urlcolor=cyan,
    pdftitle={Cybersecurity Posture Assessment Report},
    pdfpagemode=FullScreen,
}

% Header and Footer
\pagestyle{fancy}
\fancyhf{}
\lhead{Confidential Report}
\rhead{\textbf{[Organization Name]}}
\cfoot{\thepage}

\begin{document}

\maketitle
\thispagestyle{empty}
\newpage

\tableofcontents
\newpage

\section{Executive Overview}

This report provides a cybersecurity posture assessment for \textbf{[Organization Name]}, based on an analysis of network scan data, organizational security controls, and known risks. The assessment was conducted on \today.

The analysis reveals several critical and high-risk security gaps that require immediate attention. The overall security posture is considered high-risk due to significant deficiencies in fundamental security controls.

Key findings include:
\begin{itemize}
    \item \textbf{Critical Gaps in Access Control:} Multi-Factor Authentication (MFA) is not enforced for computer logins or access to sensitive data systems. This dramatically increases the risk of unauthorized access from compromised credentials.
    \item \textbf{Inadequate Security Policies and Training:} The organization lacks a formal employee acceptable use policy and does not provide security awareness training. This leaves the organization highly vulnerable to social engineering, phishing attacks, and insider threats.
    \item \textbf{Exposed Management Services:} A network scan identified an exposed Secure Shell (SSH) service (port 22) on an external-facing system. If not properly secured, this service can be a primary target for brute-force attacks and unauthorized intrusion attempts.
\end{itemize}

Immediate remediation of these issues is strongly recommended to reduce the organization's attack surface and mitigate the risk of a significant security breach. Detailed findings and actionable recommendations are provided in the subsequent sections of this report.

\section{Organizational Information}

The following information was used as the basis for this assessment. Due to the anonymized nature of the provided data, placeholders have been used where necessary.

\begin{table}[h!]
\centering
\begin{tabular}{@{}ll@{}}
\toprule
\textbf{Attribute} & \textbf{Value} \\ \midrule
Organization Name & \textbf{[Organization Name]} \\
Primary Domain & \seqsplit{\texttt{[Domain]}} \\
External IP Address (Target) & \seqsplit{\texttt{[Client IP]}} \\ \bottomrule
\end{tabular}
\caption{Client Organizational Data}
\label{tab:org_data}
\end{table}

\section{Security Control Review}

A review of the organization's self-reported security controls was conducted via a questionnaire. The responses highlight significant gaps in foundational security practices. A "No" response indicates a missing control and a potential area of high risk.

\begin{table}[h!]
\centering
\begin{tabular}{@{}p{0.5\linewidth}cp{0.3\linewidth}@{}}
\toprule
\textbf{Control Question} & \textbf{Response} & \textbf{Analyst Note} \\ \midrule
Do you require MFA to access email? & \ding{51} & Good practice. Helps protect a primary communication channel. \\
\addlinespace
Do you require MFA to log into computers? & \textbf{\color{red}\ding{55}} & \textbf{Critical Gap.} Lack of MFA on endpoints allows an attacker with valid credentials to gain direct network access. \\
\addlinespace
Do you require MFA to access sensitive data systems? & \textbf{\color{red}\ding{55}} & \textbf{Critical Gap.} The organization's most valuable data is not protected by a crucial security layer. \\
\addlinespace
Does your organization have an employee acceptable use policy? & \textbf{\color{red}\ding{55}} & \textbf{High Risk.} Without a policy, there are no clear rules for employees regarding system usage, leading to inconsistent and risky behavior. \\
\addlinesspace
Does your organization do security awareness training for new employees? & \textbf{\color{red}\ding{55}} & \textbf{High Risk.} New hires are not equipped with the knowledge to identify and avoid common threats like phishing. \\
\addlinespace
Does your organization do security awareness training for all employees at least once per year? & \textbf{\color{red}\ding{55}} & \textbf{High Risk.} Security is a perishable skill. Lack of ongoing training increases susceptibility to evolving threats. \\ \bottomrule
\end{tabular}
\caption{Security Controls Questionnaire Analysis}
\label{tab:controls}
\end{table}

\section{Technical Scan Results}

An external network scan was performed to identify open ports and exposed services. The scan provided limited details but revealed a significant finding.

\begin{itemize}
    \item \textbf{Target IP Address:} \seqsplit{\texttt{[Target IP]}}
    \item \textbf{Scan Date:} Not provided in scan data. Report generated on \today.
    \item \textbf{Status:} Host is online and responsive.
\end{itemize}

\begin{table}[h!]
\centering
\begin{tabular}{@{}lllll@{}}
\toprule
\textbf{Port} & \textbf{State} & \textbf{Service} & \textbf{Product/Version} & \textbf{Finding} \\ \midrule
22/tcp & OPEN & ssh (inferred) & Not available & \begin{tabular}[c]{@{}l@{}}The SSH port is exposed to the public\\ internet. This is a common vector for\\ brute-force and credential stuffing attacks.\end{tabular} \\ \bottomrule
\end{tabular}
\caption{Open Port Analysis}
\label{tab:scan_results}
\end{table}

\subsection{Analysis of Technical Findings}
The presence of an open SSH port (22/tcp) is a primary concern. This port is used for remote system administration. When exposed to the internet, it becomes a constant target for automated attacks seeking to guess or steal credentials. Correlated with the lack of MFA for computer and system access, a single compromised password could grant an attacker direct administrative access to a critical server.

\section{Risk Assessment Summary}

The following table summarizes the key risks identified by correlating the security control gaps, technical findings, and pre-existing risk data (which was empty for this assessment).

\begin{table}[h!]
\centering
\begin{tabular}{@{}p{0.1\linewidth}p{0.25\linewidth}p{0.45\linewidth}l@{}}
\toprule
\textbf{Risk ID} & \textbf{Risk Name} & \textbf{Description} & \textbf{Severity} \\ \midrule
RISK-001 & Lack of MFA on Endpoints and Systems & The absence of MFA for computer and sensitive system access allows for trivial account takeover if credentials are stolen, phished, or leaked. & \textbf{Critical} \\
\addlinespace
RISK-002 & Inadequate Security Awareness Program & Employees are not trained to recognize or respond to cyber threats (e.g., phishing), making them the weakest link in the organization's defense. & \textbf{High} \\
\addlinespace
RISK-003 & Exposed SSH Management Port & The SSH administrative service is open to the internet, inviting automated brute-force attacks and increasing the risk of unauthorized server access. & \textbf{High} \\
\addlinespace
RISK-004 & Missing Foundational Security Policies & The lack of an Acceptable Use Policy results in an undefined security culture and no enforceable rules for employee behavior, increasing insider risk. & \textbf{High} \\ \bottomrule
\end{tabular}
\caption{Summary of Identified Risks}
\label{tab:risks}
\end{table}

\section{Recommendations}

To address the identified risks and improve the overall security posture of \textbf{[Organization Name]}, the following actions are recommended with urgency.

\begin{enumerate}
    \item \textbf{Implement Multi-Factor Authentication (MFA):}
    \begin{itemize}
        \item \textbf{Priority 1 (Critical):} Immediately deploy and enforce MFA for all user accounts when logging into company computers and accessing any systems containing sensitive data.
        \item This is the single most effective control to mitigate the risk of credential compromise.
    \end{itemize}

    \item \textbf{Secure Remote Access Ports:}
    \begin{itemize}
        \item \textbf{Priority 1 (Critical):} Restrict access to the SSH port (22) on \seqsplit{\texttt{[Target IP]}}. Access should be limited via a firewall to only trusted, known IP addresses (e.g., office or administrator IPs).
        \item If public access is unavoidable, enforce the use of public-key authentication and disable password-based logins for SSH.
    \end{itemize}
    
    \item \textbf{Develop a Security Awareness Program:}
    \begin{itemize}
        \item \textbf{Priority 2 (High):} Implement a mandatory security awareness training program for all employees. This must be included in the onboarding process for new hires and conducted at least annually for all staff.
        \item Training should cover phishing, password hygiene, and safe internet use.
    \end{itemize}

    \item \textbf{Establish Foundational Security Policies:}
    \begin{itemize}
        \item \textbf{Priority 2 (High):} Develop and implement a formal Acceptable Use Policy (AUP). This policy should be read and acknowledged by all employees to establish a baseline for secure behavior and provide grounds for enforcement.
    \end{itemize}

    \item \textbf{Conduct Comprehensive Vulnerability Scanning:}
    \begin{itemize}
        \item \textbf{Priority 3 (Medium):} The initial scan was basic. A comprehensive, authenticated vulnerability scan should be conducted on external and internal systems to identify outdated software and specific configuration weaknesses that could be exploited.
    \end{itemize}
\end{enumerate}

\end{document}
```