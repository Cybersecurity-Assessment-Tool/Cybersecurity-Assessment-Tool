```latex
\documentclass[12pt]{article}

% Preamble: Required Packages
\usepackage[margin=1in]{geometry}
\usepackage{pifont} % For \ding
\usepackage{booktabs} % For professional tables (\toprule, \midrule, \bottomrule)
\usepackage{hyperref}
\usepackage{url}
\usepackage{seqsplit} % To break long strings in \texttt
\usepackage{xcolor}
\usepackage{graphicx}

% Hyperlink Setup
\hypersetup{
    colorlinks=true,
    linkcolor=blue,
    filecolor=magenta,
    urlcolor=cyan,
}

% Custom commands for checkmarks and crosses
\newcommand{\yes}{\textcolor{green}{\ding{51}}}
\newcommand{\no}{\textcolor{red}{\ding{55}}}

% --- Document Start ---
\begin{document}

\title{Cybersecurity Posture Assessment Report}
\author{Cybersecurity Analysis Division}
\date{\today}
\maketitle

\begin{abstract}
    This report provides a comprehensive cybersecurity assessment for \textbf{[Organization Name]}. The analysis is based on a synthesis of external network scan data, a review of organizational security controls, and an evaluation of pre-existing risk documentation. The assessment identified a critical risk of external exposure of the Remote Desktop Protocol (RDP), which is strongly correlated with a known high-severity vulnerability. This technical finding is exacerbated by significant gaps in access control policies, specifically the lack of multi-factor authentication (MFA) for email and sensitive data systems. Immediate remediation is required to mitigate the high likelihood of a security breach.
\end{abstract}

\tableofcontents
\newpage

% --- Section 1: Overview ---
\section{Overview and Executive Summary}
The primary objective of this assessment was to evaluate the current cybersecurity posture of \textbf{[Organization Name]}. Our analysis correlated three key data sources: a technical network scan, a security controls questionnaire, and a list of current known risks.

\subsection{Key Findings}
\begin{itemize}
    \item \textbf{Critical Risk - RDP Exposure:} The external network scan confirmed that port 3389 (Remote Desktop Protocol) is open on the external IP address \texttt{[Client IP]}. This finding directly validates the pre-existing high-severity risk, "RDP Exposure," and presents a significant and immediate threat to the organization. RDP is a primary target for ransomware and unauthorized access attacks.
    \item \textbf{Critical Gap - Insufficient Access Control:} The organization does not enforce Multi-Factor Authentication (MFA) for accessing email or other sensitive data systems. This dramatically increases the risk of account compromise via phishing or credential stuffing attacks.
    \item \textbf{High Risk - Inadequate Security Training:} While new employees receive security training, there is no mandatory annual training for all staff. This gap allows for security knowledge to become outdated, increasing susceptibility to social engineering attacks.
\end{itemize}

\subsection{Overall Posture}
The combination of a publicly exposed, high-value service (RDP) and weak access controls places the organization at a \textbf{CRITICAL} risk level. An attacker could potentially exploit these weaknesses to gain unauthorized access to the internal network, deploy ransomware, or exfiltrate sensitive data.

% --- Section 2: Organizational Information ---
\section{Organizational Information}
This section details the information provided about the organization.
\begin{itemize}
    \item \textbf{Organization Name:} \textbf{[Organization Name]}
    \item \textbf{Primary Domain:} \texttt{[Domain]}
    \item \textbf{External IP Scanned:} \texttt{[Client IP]}
\end{itemize}

% --- Section 3: Security Control Review ---
\section{Security Control Review}
The following table summarizes the organization's responses to the security controls questionnaire. Items marked with \no\ represent significant gaps in the security framework and require attention.

\begin{table}[h!]
\centering
\caption{Security Controls Questionnaire Analysis}
\label{tab:controls}
\begin{tabular}{@{}lcc@{}}
\toprule
\textbf{Control Question} & \textbf{Response} & \textbf{Status} \\
\midrule
Do you require MFA to access email? & No & \no \\
Do you require MFA to log into computers? & Yes & \yes \\
Do you require MFA to access sensitive data systems? & No & \no \\
Does your organization have an employee acceptable use policy? & Yes & \yes \\
Does your organization do security awareness training for new employees? & Yes & \yes \\
Does your organization do security training for all employees annually? & No & \no \\
\bottomrule
\end{tabular}
\end{table}

% --- Section 4: Technical Scan Results ---
\section{Technical Scan Results}
An external network scan was performed on the target IP address. The scan identified the following open ports and services.

\subsection{Scan Target}
\begin{itemize}
    \item \textbf{Target IP:} \texttt{[Target IP]}
\end{itemize}

\subsection{Open Ports Discovered}
The scan revealed one open port, which presents a critical attack surface.

\begin{table}[h!]
\centering
\caption{Nmap Scan Findings}
\label{tab:nmap}
\begin{tabular}{@{}llll@{}}
\toprule
\textbf{Port} & \textbf{State} & \textbf{Service Name} & \textbf{Analysis} \\
\midrule
3389/tcp & open & \texttt{ms-wbt-server} & Critical Risk \\
\bottomrule
\end{tabular}
\end{table}

\paragraph{Finding Detail:} The service \texttt{ms-wbt-server} is the Microsoft Remote Desktop Protocol (RDP). Exposing RDP directly to the public internet is extremely dangerous and is a leading cause of ransomware infections and network intrusions. This technical finding confirms the high-severity risk documented in Input 3.

% --- Section 5: Correlated Risk Assessment ---
\section{Correlated Risk Assessment}
This section synthesizes findings from all data sources into a prioritized list of risks.

\begin{table}[h!]
\centering
\caption{Summary of Identified Risks}
\label{tab:risks}
\begin{tabular}{@{}p{0.1\linewidth} p{0.25\linewidth} p{0.45\linewidth} p{0.1\linewidth}@{}}
\toprule
\textbf{Risk ID} & \textbf{Risk Title} & \textbf{Description} & \textbf{Severity} \\
\midrule
\textbf{RISK-001} & External RDP Exposure & The network scan confirmed that TCP port 3389 is open on \texttt{[Client IP]}, exposing RDP services. This is a common vector for brute-force attacks and exploitation (e.g., BlueKeep). This validates a known risk with a CVSS score of 9.0. & \textbf{Critical} \\
\addlinespace
\textbf{RISK-002} & Inadequate Identity and Access Management & The lack of MFA on email and sensitive data systems means a compromised password is all an attacker needs for access. This weakness compounds the threat of RISK-001. & \textbf{Critical} \\
\addlinespace
\textbf{RISK-003} & Insufficient Security Awareness Program & The absence of annual security training for all employees increases the likelihood of successful phishing attacks, which are the primary source of credential theft. & \textbf{High} \\
\bottomrule
\end{tabular}
\end{table}

% --- Section 6: Recommendations ---
\section{Recommendations}
The following actions are recommended to mitigate the identified risks. Recommendations are prioritized based on severity.

\subsection{Remediation for RISK-001: External RDP Exposure}
\begin{itemize}
    \item \textbf{Immediate Action (Containment):} Block all inbound traffic to TCP port 3389 on the network firewall immediately. This will remove the direct threat from the public internet.
    \item \textbf{Long-Term Solution:} Implement a secure remote access solution, such as a Virtual Private Network (VPN) or a Zero Trust Network Access (ZTNA) gateway. All remote administration must occur through this secure channel.
\end{itemize}

\subsection{Remediation for RISK-002: Inadequate IAM}
\begin{itemize}
    \item \textbf{Immediate Action (Mitigation):} Enforce mandatory Multi-Factor Authentication (MFA) for all users on all external-facing services, especially email and any systems containing sensitive data. This should be treated as an emergency change.
    \item \textbf{Long-Term Solution:} Develop a comprehensive Identity and Access Management (IAM) policy that mandates MFA for all services, implements the principle of least privilege, and includes regular access reviews.
\end{itemize}

\subsection{Remediation for RISK-003: Insufficient Security Awareness}
\begin{itemize}
    \item \textbf{Immediate Action (Improvement):} Schedule and deliver mandatory security awareness training for all current employees within the next 60 days.
    \item \textbf{Long-Term Solution:} Establish a formal security awareness program that includes, at a minimum, mandatory annual training and periodic phishing simulations to measure and improve employee resilience.
\end{itemize}

% --- Section 7: Conclusion ---
\section{Conclusion}
The assessment reveals a critical security posture for \textbf{[Organization Name]}, primarily driven by the exposure of RDP services and inadequate access controls. The identified risks, if left unaddressed, present a high probability of a significant security incident. We strongly urge the organization to implement the "Immediate Action" recommendations outlined in this report without delay to reduce the attack surface and protect critical assets.

% --- Document End ---
\end{document}
```