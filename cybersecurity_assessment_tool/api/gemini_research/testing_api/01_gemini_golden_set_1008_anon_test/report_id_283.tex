```latex
\documentclass[12pt]{article}

% Preamble: Required Packages
\usepackage[margin=1in]{geometry}
\usepackage{pifont} % For checkmarks and crosses
\usepackage{booktabs} % For professional tables
\usepackage{hyperref} % For clickable links and better PDF navigation
\usepackage{url} % For formatting URLs
\usepackage{seqsplit} % For splitting long strings in tt font
\usepackage{graphicx}
\usepackage{xcolor}

% --- Document Metadata ---
\title{Cybersecurity Posture Assessment Report}
\author{Cybersecurity Analysis Division}
\date{\today}

% --- Hyperref Setup ---
\hypersetup{
    colorlinks=true,
    linkcolor=blue,
    filecolor=magenta,      
    urlcolor=cyan,
    pdftitle={Cybersecurity Posture Assessment Report},
    pdfpagemode=FullScreen,
}

% --- Document Start ---
\begin{document}

\maketitle
\thispagestyle{empty}
\newpage

\tableofcontents
\newpage

% ===================================================================
% SECTION 1: EXECUTIVE SUMMARY
% ===================================================================
\section{Executive Summary}

This report provides a cybersecurity posture assessment for \textbf{[Organization Name]}, based on a combination of network scanning, a security controls questionnaire, and a review of known risks. The analysis was conducted on \today.

Overall, the organization demonstrates a foundational level of security maturity, particularly with the consistent implementation of Multi-Factor Authentication (MFA) across email, computers, and sensitive data systems. This is a commendable strength.

However, two significant risks were identified that require immediate attention:
\begin{itemize}
    \item \textbf{Critical Control Gap:} The lack of mandatory, annual security awareness training for all employees presents a high risk. This gap leaves the organization vulnerable to phishing, social engineering, and other human-centric attacks, which are among the most common initial access vectors for threat actors.
    \item \textbf{High-Risk Technical Finding:} An externally accessible Secure Shell (SSH) port (22/TCP) was discovered on the network perimeter. Exposed management services like SSH are prime targets for automated brute-force attacks and exploitation of potential vulnerabilities, posing a direct threat to network integrity.
\end{itemize}

While there were no pre-existing vulnerabilities documented, these new findings indicate critical areas for improvement. This report details these findings and provides actionable recommendations to mitigate the identified risks and enhance the organization's overall security posture.

% ===================================================================
% SECTION 2: ORGANIZATIONAL INFORMATION
% ===================================================================
\section{Organizational Information}

The following information was used as the basis for this assessment. Due to the anonymized nature of the provided data, placeholders have been used where necessary.

\begin{table}[h!]
\centering
\begin{tabular}{@{}ll@{}}
\toprule
\textbf{Attribute} & \textbf{Value} \\ \midrule
Organization Name & \textbf{[Organization Name]} \\
Primary Email Domain & \texttt{[Domain]} \\
External IP Address Scanned & \texttt{[Client IP]} \\ \bottomrule
\end{tabular}
\caption{Client Organizational Details}
\end{table}

% ===================================================================
% SECTION 3: SECURITY CONTROL REVIEW
% ===================================================================
\section{Security Control Review}

A security controls questionnaire was completed to evaluate existing administrative and procedural safeguards. The results are summarized below. A checkmark (\ding{51}) indicates a positive response (control in place), while a cross (\ding{55}) indicates a negative response, highlighting a potential gap.

\begin{table}[h!]
\centering
\begin{tabular}{@{}p{0.75\textwidth}c@{}}
\toprule
\textbf{Control Question} & \textbf{Response} \\ \midrule
Do you require MFA to access email? & \textcolor{green}{\ding{51}} \\
Do you require MFA to log into computers? & \textcolor{green}{\ding{51}} \\
Do you require MFA to access sensitive data systems? & \textcolor{green}{\ding{51}} \\
Does your organization have an employee acceptable use policy? & \textcolor{green}{\ding{51}} \\
Does your organization do security awareness training for new employees? & \textcolor{green}{\ding{51}} \\
\textbf{Does your organization do security awareness training for all employees at least once per year?} & \textcolor{red}{\ding{55}} \\ \bottomrule
\end{tabular}
\caption{Security Controls Questionnaire Results}
\end{table}

\subsection*{Analysis of Control Gaps}
The primary gap identified is the lack of ongoing, annual security awareness training for all staff. While training new hires is a good first step, the threat landscape evolves continuously. Regular training ensures that all employees remain vigilant and are kept up-to-date on the latest phishing techniques and social engineering tactics. This gap is classified as a \textbf{High} risk.

% ===================================================================
% SECTION 4: TECHNICAL SCAN RESULTS
% ===================================================================
\section{Technical Scan Results}

An external network scan was performed against the target IP address \texttt{[Target IP]} to identify open ports and exposed services.

\subsection*{Open Ports Discovered}
The following table details the open ports discovered during the reconnaissance phase.

\begin{table}[h!]
\centering
\begin{tabular}{@{}llll@{}}
\toprule
\textbf{Port} & \textbf{State} & \textbf{Service} & \textbf{Product / Version} \\ \midrule
22/TCP & open & ssh & \textit{Not Detected} \\ \bottomrule
\end{tabular}
\caption{Open Port Findings on Target: \texttt{[Target IP]}}
\end{table}

\subsection*{Analysis of Technical Findings}
The scan identified that port 22 (SSH) is open to the public internet. SSH is a common protocol for remote server administration. However, exposing it directly to the internet creates a significant attack surface. It is a frequent target for:
\begin{itemize}
    \item \textbf{Brute-Force Attacks:} Automated tools constantly scan the internet for open SSH ports and attempt to guess credentials.
    \item \textbf{Credential Stuffing:} Using credentials stolen from other data breaches to attempt logins.
    \item \textbf{Vulnerability Exploitation:} If the SSH server software is outdated or misconfigured, it may be susceptible to known exploits.
\end{itemize}
This finding is classified as a \textbf{High} risk due to the potential for unauthorized access to a critical management interface.

% ===================================================================
% SECTION 5: RISK ASSESSMENT SUMMARY
% ===================================================================
\section{Risk Assessment Summary}

This section correlates the findings from the security control review and the technical scan. No pre-existing risks were provided for this assessment.

\begin{table}[h!]
\centering
\begin{tabular}{@{}lp{0.5\textwidth}ll@{}}
\toprule
\textbf{ID} & \textbf{Risk Description} & \textbf{Source} & \textbf{Severity} \\ \midrule
R-01 & Lack of mandatory annual security awareness training for all employees increases susceptibility to phishing and social engineering. & Questionnaire & \textbf{High} \\
\addlinespace
R-02 & Exposed SSH management port (22/TCP) on the external perimeter, creating a target for brute-force attacks and exploitation. & Network Scan & \textbf{High} \\ \bottomrule
\end{tabular}
\caption{Summary of Identified Risks}
\end{table}

% ===================================================================
% SECTION 6: RECOMMENDATIONS
% ===================================================================
\section{Recommendations}

The following actionable recommendations are provided to address the identified risks and improve the overall security posture.

\subsection*{R-01: Implement Annual Security Awareness Training (High)}
\begin{itemize}
    \item \textbf{Action:} Procure and implement a security awareness training platform or program. Ensure that training is mandatory for all employees, including management, on an annual basis.
    \item \textbf{Details:} The training should cover modern threats such as phishing, business email compromise (BEC), ransomware, and proper data handling. Incorporate periodic phishing simulations to test and reinforce the training.
    \item \textbf{Impact:} Significantly reduces the risk of security incidents caused by human error and strengthens the "human firewall."
\end{itemize}

\subsection*{R-02: Secure the Exposed SSH Port (High)}
\begin{itemize}
    \item \textbf{Action:} Immediately review the business requirement for the publicly accessible SSH port on target \texttt{[Target IP]}.
    \item \textbf{Details:}
        \begin{enumerate}
            \item \textbf{If not required:} Block port 22 at the network firewall immediately.
            \item \textbf{If required:} Implement a defense-in-depth strategy:
            \begin{itemize}
                \item Restrict access to a whitelist of trusted IP addresses.
                \item If whitelisting is not feasible, deploy a Virtual Private Network (VPN) and require users to connect to the VPN before accessing internal resources like SSH.
                \item Enforce public key authentication and completely disable password-based authentication.
                \item Implement an intrusion prevention tool like \texttt{fail2ban} to automatically block IPs that exhibit malicious behavior (e.g., multiple failed login attempts).
            \end{itemize}
        \end{enumerate}
    \item \textbf{Impact:} Drastically reduces the attack surface and mitigates the risk of a network breach via the exposed management interface.
\end{itemize}

\end{document}
```