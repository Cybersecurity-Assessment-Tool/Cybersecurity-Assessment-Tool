```latex
\documentclass[12pt]{article}

% === PACKAGES ===
\usepackage[margin=1in]{geometry}
\usepackage{pifont}                 % For checkmarks and crosses (\ding)
\usepackage{booktabs}               % For professional-looking tables
\usepackage{hyperref}               % For clickable links and references
\usepackage{url}                    % For URL formatting
\usepackage{seqsplit}               % For splitting long strings without spaces
\usepackage{xcolor}                 % For coloring text
\usepackage{graphicx}

% === DOCUMENT SETUP ===
\hypersetup{
    colorlinks=true,
    linkcolor=blue,
    filecolor=magenta,
    urlcolor=cyan,
    pdftitle={Cybersecurity Posture Assessment Report},
    pdfauthor={Cybersecurity Analysis Division},
}

% === CUSTOM COMMANDS ===
\newcommand{\yes}{\ding{51}}
\newcommand{\no}{\textcolor{red}{\ding{55}}} % Make "No" answers stand out in red

% === DOCUMENT START ===
\begin{document}

% === TITLE PAGE ===
\title{
    \vspace{2cm}
    \textbf{Cybersecurity Posture Assessment Report} \\
    \large For \\
    \textbf{[Organization Name]}
    \vspace{2cm}
}
\author{Cybersecurity Analysis Division}
\date{\today}
\maketitle
\thispagestyle{empty}
\newpage

% === TABLE OF CONTENTS ===
\tableofcontents
\newpage

% === EXECUTIVE SUMMARY ===
\section*{Executive Summary}
This report details the findings of a cybersecurity posture assessment for \textbf{[Organization Name]}, conducted on \today. The assessment combined a review of organizational security controls via questionnaire, an external network scan, and an analysis of pre-existing risk data.

Several critical and high-risk issues were identified that require immediate attention. The synthesis of the provided data points to a security posture with foundational strengths but significant, exploitable gaps.

\textbf{Key Findings:}
\begin{itemize}
    \item \textbf{Critical Risk - Pre-existing Vulnerability:} A known critical vulnerability, "Localhost Exposed," with a CVSS score of 10.0, remains unmitigated on a key asset.
    \item \textbf{Critical Risk - Lack of MFA on Sensitive Systems:} Sensitive data systems are not protected by Multi-Factor Authentication (MFA), exposing critical assets to significant risk from compromised credentials.
    \item \textbf{High Risk - Exposed Management Service:} The Secure Shell (SSH) service on port 22 is exposed to the public internet on \texttt{[Target IP]}, creating a target for brute-force attacks and unauthorized access attempts.
    \item \textbf{High Risk - Inadequate Employee Onboarding:} New employees do not receive mandatory security awareness training, leaving the organization vulnerable to social engineering and policy violations from their first day.
\end{itemize}

This report provides a detailed analysis of these findings and offers actionable recommendations to mitigate the identified risks and improve the overall security posture of \textbf{[Organization Name]}.

% === ASSESSMENT SCOPE ===
\section*{Assessment Scope}
The following placeholder information, derived from the provided data, defines the scope of this assessment.
\begin{itemize}
    \item \textbf{Organization Name:} \textbf{[Organization Name]}
    \item \textbf{Primary Domain:} \texttt{[Domain]}
    \item \textbf{Assumed Client IP:} \texttt{[Client IP]}
    \item \textbf{Network Scan Target:} \texttt{[Target IP]}
\end{itemize}

% === SECURITY CONTROL REVIEW ===
\section*{Security Control Review (Questionnaire Analysis)}
A review of organizational security controls was conducted via a standardized questionnaire. The responses revealed significant gaps in the security framework, particularly concerning access control for sensitive assets and employee training.

\begin{table}[h!]
\centering
\caption{Security Controls Questionnaire Results}
\label{tab:controls}
\begin{tabular}{p{8cm} c l}
\toprule
\textbf{Control Question} & \textbf{Response} & \textbf{Finding} \\
\midrule
Do you require MFA to access email? & \yes & Compliant \\
Do you require MFA to log into computers? & \yes & Compliant \\
Do you require MFA to access sensitive data systems? & \no & \textbf{Critical Gap} \\
Does your organization have an employee acceptable use policy? & \yes & Compliant \\
Does your organization do security awareness training for new employees? & \no & \textbf{High Risk} \\
Does your organization do security awareness training for all employees at least once per year? & \yes & Compliant \\
\bottomrule
\end{tabular}
\end{table}

The "No" responses indicate areas of immediate concern. The absence of MFA on sensitive systems means a single compromised password could lead to a significant data breach. The lack of security training during employee onboarding creates a recurring vulnerability, as new staff are often prime targets for phishing and social engineering attacks.

% === TECHNICAL SCAN RESULTS ===
\section*{Technical Scan Results}
An external network scan was performed against the target IP address \texttt{[Target IP]}. The scan was a basic port discovery and did not include service version detection or vulnerability analysis.

\subsection*{Open Ports}
The following service was found to be accessible from the public internet:
\begin{table}[h!]
\centering
\caption{Open Ports on \texttt{[Target IP]}}
\label{tab:ports}
\begin{tabular}{c c p{8cm}}
\toprule
\textbf{Port} & \textbf{State} & \textbf{Inferred Service \& Risk} \\
\midrule
22/tcp & Open & \textbf{SSH (Secure Shell):} This service is used for remote system administration. Exposing SSH directly to the internet is a high-risk configuration. It becomes a primary target for automated brute-force attacks attempting to guess user credentials. Without robust controls like IP whitelisting, strong password policies, and intrusion detection, this port presents a significant vector for unauthorized access. \\
\bottomrule
\end{tabular}
\end{table}

% === CONSOLIDATED RISK ASSESSMENT ===
\section*{Consolidated Risk Assessment}
The following table synthesizes findings from the security control review, technical scan, and pre-existing risk data into a consolidated list of identified risks.

\begin{table}[h!]
\centering
\caption{Summary of Identified Risks}
\label{tab:risks}
\begin{tabular}{p{4.5cm} l p{6.5cm}}
\toprule
\textbf{Risk Title} & \textbf{Severity} & \textbf{Description} \\
\midrule
Localhost Exposed & \textbf{Critical (10.0)} & A pre-existing, unmitigated vulnerability with the highest possible CVSS score was noted. The nature of this risk implies a severe misconfiguration allowing unauthorized access to the affected element: \texttt{[Target IP]}. \\
\addlinespace
No MFA on Sensitive Systems & \textbf{Critical} & Lack of a second authentication factor for high-value data systems. A single compromised password could lead to a major data breach. \\
\addlinespace
Exposed SSH Service & \textbf{High} & The remote administration port (22/tcp) is open to the internet on \texttt{[Target IP]}, inviting automated attacks and increasing the risk of unauthorized server access. \\
\addlinespace
No Security Training for New Hires & \textbf{High} & New employees are not trained on security policies and threats, making them highly susceptible to phishing and social engineering attacks from their first day. \\
\bottomrule
\end{tabular}
\end{table}

% === RECOMMENDATIONS ===
\section*{Recommendations}
Based on the consolidated risk assessment, the following actions are recommended to improve the security posture of \textbf{[Organization Name]}. Recommendations are prioritized by severity.

\subsection*{Priority 1: Critical Risks (Address Immediately)}
\begin{enumerate}
    \item \textbf{Remediate "Localhost Exposed" Vulnerability:} \textbf{Immediately} investigate and remediate the critical risk identified as "Localhost Exposed" on host \texttt{[Target IP]}. Given its 10.0 CVSS score, this should be the highest priority for the security and infrastructure teams.
    \item \textbf{Implement MFA for All Sensitive Systems:} Deploy mandatory Multi-Factor Authentication (MFA) on all systems classified as containing sensitive data. This is the single most effective control to prevent unauthorized access resulting from credential theft.
\end{enumerate}

\subsection*{Priority 2: High Risks (Address within 30 days)}
\begin{enumerate}
    \item \textbf{Secure the Exposed SSH Service:}
        \begin{itemize}
            \item \textbf{If remote access is not required,} block port 22 at the firewall.
            \item \textbf{If required,} restrict access to a whitelist of trusted IP addresses (e.g., a corporate VPN or specific administrator IPs).
            \item Enforce the use of public key authentication and disable password-based logins.
            \item Implement an intrusion prevention system (IPS) or a tool like Fail2Ban to block brute-force attempts.
        \end{itemize}
    \item \textbf{Integrate Security Training into Onboarding:} Develop and mandate a security awareness training module for all new employees. This training must be completed as part of the formal onboarding process before system access is granted. The training should cover acceptable use, phishing identification, and incident reporting procedures.
\end{enumerate}

\subsection*{Priority 3: Continuous Improvement}
\begin{enumerate}
    \item \textbf{Conduct Authenticated Vulnerability Scanning:} The initial scan was a non-intrusive port scan. We recommend implementing a program of regular, authenticated vulnerability scans on all external and internal systems to identify outdated software, missing patches, and detailed misconfigurations that this scan could not detect.
\end{enumerate}

\end{document}
```