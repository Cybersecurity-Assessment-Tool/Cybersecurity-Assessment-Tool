```latex
\documentclass[12pt]{article}

% Preamble: Required Packages
\usepackage[margin=1in]{geometry}
\usepackage{pifont} % For checkmarks and crosses
\usepackage{booktabs} % For professional tables
\usepackage{hyperref} % For hyperlinks
\usepackage{url}      % For URL formatting
\usepackage{seqsplit} % For splitting long strings without spaces

% Document Metadata
\hypersetup{
    colorlinks=true,
    linkcolor=blue,
    filecolor=magenta,      
    urlcolor=cyan,
    pdftitle={Cybersecurity Posture Report},
    pdfauthor={Cybersecurity Analyst},
    pdfsubject={Security Assessment},
    pdfkeywords={Cybersecurity, Risk, Analysis},
}

\begin{document}

% --- Title Page ---
\begin{titlepage}
    \centering
    \vspace*{\fill}
    \Huge\textbf{Cybersecurity Posture Report}
    \vspace{1.5cm}
    \Large For
    \vspace{0.5cm}
    \huge\textbf{[Organization Name]}
    \vspace{2cm}
    \rule{0.8\textwidth}{0.4pt}
    \vspace{1cm}
    \large \textbf{Date:} \today \\
    \textbf{Report ID:} CSR-2023-08-15
    \vspace*{\fill}
\end{titlepage}

\tableofcontents
\newpage

% --- Section 1: Executive Summary ---
\section{Executive Summary}
This report provides a comprehensive analysis of the cybersecurity posture for \textbf{[Organization Name]}, based on a review of organizational security controls, an external network vulnerability scan, and pre-existing risk data.

The assessment revealed a mixed security posture. While foundational controls like Multi-Factor Authentication (MFA) are implemented for email and computer access, critical gaps exist in protecting sensitive data systems. Furthermore, the absence of key administrative controls, such as an Acceptable Use Policy and mandatory security training for new hires, presents a significant risk to the organization.

The external network scan of the target IP address \texttt{[Client IP]} did not identify any open ports or running services. This suggests a well-configured perimeter firewall, which is a positive security finding.

Key recommendations focus on immediately addressing the identified control gaps. This includes enforcing MFA on all sensitive systems, developing and implementing core security policies, and establishing a mandatory onboarding security training program. Prioritizing these actions will substantially improve the organization's resilience against common cyber threats.

% --- Section 2: Organizational Information ---
\section{Organizational Information}
This section details the information provided by the client for this assessment.
\begin{itemize}
    \item \textbf{Organization Name:} \textbf{[Organization Name]}
    \item \textbf{Primary Domain:} \texttt{[Domain]}
    \item \textbf{External IP Scanned:} \texttt{[Client IP]}
\end{itemize}

% --- Section 3: Security Control Review ---
\section{Security Control Review}
The following table summarizes the organization's responses to a security controls questionnaire. Each response has been assessed against industry best practices. A green checkmark (\ding{51}) indicates an aligned practice, while a red cross (\ding{55}) highlights a significant control gap.

\begin{table}[h!]
\centering
\caption{Security Controls Questionnaire Analysis}
\begin{tabular}{p{0.6\linewidth} c p{0.2\linewidth}}
\toprule
\textbf{Control Question} & \textbf{Response} & \textbf{Assessment} \\
\midrule
Do you require MFA to access email? & \ding{51} (Yes) & Aligned \\
\addlinespace
Do you require MFA to log into computers? & \ding{51} (Yes) & Aligned \\
\addlinespace
Do you require MFA to access sensitive data systems? & \ding{55} (No) & \textbf{Critical Gap} \\
\addlinespace
Does your organization have an employee acceptable use policy? & \ding{55} (No) & \textbf{High Risk} \\
\addlinespace
Does your organization do security awareness training for new employees? & \ding{55} (No) & \textbf{High Risk} \\
\addlinespace
Does your organization do security awareness training for all employees at least once per year? & \ding{51} (Yes) & Aligned \\
\bottomrule
\end{tabular}
\end{table}

% --- Section 4: Technical Scan Results ---
\section{Technical Scan Results}
An external network scan was conducted to identify listening services and potential vulnerabilities on the organization's public-facing infrastructure.

\begin{itemize}
    \item \textbf{Target IP Address:} \texttt{[Target IP]}
    \item \textbf{Scan Date:} [Scan Date]
    \item \textbf{Summary:} The scan completed successfully and \textbf{no open ports or services were detected}. This indicates that the perimeter firewall is likely configured to deny all unsolicited inbound traffic, which is a strong security practice.
\end{itemize}

\subsection{Detailed Findings}
The scan results were empty, meaning no services were exposed to the public internet on the scanned IP address.

% --- Section 5: Risk Assessment ---
\section{Risk Assessment}
This section synthesizes findings from the security control review, technical scan, and pre-existing risk data. The primary risks identified are related to administrative and access control policies rather than technical vulnerabilities on the external perimeter.

\begin{table}[h!]
\centering
\caption{Summary of Identified Risks}
\begin{tabular}{p{0.1\linewidth} p{0.6\linewidth} l}
\toprule
\textbf{Risk ID} & \textbf{Description} & \textbf{Severity} \\
\midrule
RISK-001 & \textbf{Lack of MFA on Sensitive Data Systems:} The absence of MFA on systems storing or processing sensitive information exposes the organization to significant risk of unauthorized access and data breach, should user credentials be compromised. & \textbf{Critical} \\
\addlinespace
RISK-002 & \textbf{No Employee Acceptable Use Policy (AUP):} Without a formal AUP, there are no clear guidelines for employees on the acceptable use of company assets. This increases the risk of insider threat, data misuse, and legal liability. & \textbf{High} \\
\addlinespace
RISK-003 & \textbf{No Security Training for New Hires:} New employees are not provided with security awareness training during onboarding. This makes them highly susceptible to phishing and social engineering attacks from their first day, creating an immediate vulnerability. & \textbf{High} \\
\bottomrule
\end{tabular}
\end{table}

% --- Section 6: Recommendations ---
\section{Recommendations}
The following actions are recommended to mitigate the identified risks and strengthen the overall security posture of \textbf{[Organization Name]}.

\begin{enumerate}
    \item \textbf{Enforce MFA on Sensitive Systems (RISK-001):}
    \begin{itemize}
        \item \textbf{Action:} Immediately prioritize the implementation and enforcement of Multi-Factor Authentication (MFA) on all applications, databases, and administrative interfaces that contain sensitive, confidential, or regulated data.
        \item \textbf{Impact:} Drastically reduces the risk of unauthorized access via compromised credentials.
        \item \textbf{Priority:} \textbf{Critical}
    \end{itemize}
    \vspace{0.5cm}
    \item \textbf{Develop and Implement an Acceptable Use Policy (RISK-002):}
    \begin{itemize}
        \item \textbf{Action:} Draft a comprehensive AUP that clearly defines the rules for using company networks, devices, and data. This policy should be formally communicated to all employees, and acknowledgement should be tracked.
        \item \textbf{Impact:} Establishes a baseline for secure employee behavior and provides a framework for disciplinary action in case of violation.
        \item \textbf{Priority:} \textbf{High}
    \end{itemize}
    \vspace{0.5cm}
    \item \textbf{Establish a New Hire Security Training Program (RISK-003):}
    \begin{itemize}
        \item \textbf{Action:} Create a mandatory security awareness training module as part of the employee onboarding process. This training should cover key topics such as phishing, password security, data handling, and the new AUP.
        \item \textbf{Impact:} Equips new employees with essential security knowledge from day one, reducing their susceptibility to common cyber attacks.
        \item \textbf{Priority:} \textbf{High}
    \end{itemize}
\end{enumerate}

\end{document}
```