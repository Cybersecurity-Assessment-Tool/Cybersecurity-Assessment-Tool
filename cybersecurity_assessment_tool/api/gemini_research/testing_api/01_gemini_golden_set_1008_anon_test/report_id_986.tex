```latex
\documentclass[12pt]{article}

% Preamble: Required Packages
\usepackage[margin=1in]{geometry}
\usepackage{pifont} % For checkmarks and crosses
\usepackage{booktabs} % For professional tables
\usepackage{hyperref} % For hyperlinks
\usepackage{url} % For URL formatting
\usepackage{seqsplit} % To split long strings in tt font
\usepackage{graphicx}
\usepackage{xcolor}

% Document Information
\title{Cybersecurity Posture Assessment Report}
\author{Cybersecurity Analyst}
\date{\today}

% Hyperref Setup
\hypersetup{
    colorlinks=true,
    linkcolor=blue,
    filecolor=magenta,      
    urlcolor=cyan,
    pdftitle={Cybersecurity Posture Assessment Report},
    pdfpagemode=FullScreen,
}

% Custom Commands
\newcommand{\yes}{\ding{51}}
\newcommand{\no}{\ding{55}}

\begin{document}

\maketitle
\thispagestyle{empty}
\newpage

\tableofcontents
\newpage

% --- 1. Executive Summary ---
\section*{1. Executive Summary}

This report provides a comprehensive analysis of the cybersecurity posture for \textbf{[Organization Name]}. The assessment is based on a combination of technical network scanning, a review of organizational security controls, and an analysis of pre-existing risks.

The overall security posture is determined to be critically weak. Several high-impact vulnerabilities and control gaps were identified that expose the organization to significant risk of data breach, unauthorized access, and service disruption.

Key findings include:
\begin{itemize}
    \item \textbf{Critical Network Vulnerability:} An externally facing FTP server was found running a dangerously outdated version of \texttt{vsftpd} (2.3.4), which is known to contain a critical backdoor vulnerability (CVE-2011-2523). The service is further misconfigured to allow anonymous logins, presenting an immediate and severe threat.
    \item \textbf{Lack of Multi-Factor Authentication (MFA):} MFA is not enforced for accessing email or other sensitive data systems. This represents a critical gap in identity and access management, leaving key assets vulnerable to credential theft and phishing attacks.
    \item \textbf{Inadequate Employee Training:} Security awareness training is not provided to new employees during their onboarding process. This failure to instill security best practices from day one creates a persistent human-layer vulnerability.
    \item \textbf{Unsupported Operating Systems:} The organization continues to use Windows 7, an end-of-life operating system that no longer receives security updates, for its workstations.
\end{itemize}

Immediate remediation of the identified critical risks is strongly recommended to mitigate the potential for a significant security incident.

% --- 2. Organizational Information ---
\section*{2. Organizational Information}
This section details the high-level information for the organization under review. The data has been anonymized as per the engagement requirements.

\begin{tabular}{@{}ll}
    \toprule
    \textbf{Attribute} & \textbf{Value} \\
    \midrule
    Organization Name & \textbf{[Organization Name]} \\
    Primary Email Domain & \texttt{[Domain]} \\
    External IP Address & \texttt{[Client IP]} \\
    \bottomrule
\end{tabular}

% --- 3. Security Control Review ---
\section*{3. Security Control Review}
The following table summarizes the organization's responses to a security controls questionnaire. "No" answers indicate significant gaps in the security framework and are flagged as risks.

\begin{table}[h!]
\centering
\begin{tabular}{@{}lcc@{}}
    \toprule
    \textbf{Control Question} & \textbf{Response} & \textbf{Assessment} \\
    \midrule
    Do you require MFA to access email? & \no & \textcolor{red}{\textbf{Critical Gap}} \\
    Do you require MFA to log into computers? & \yes & Met \\
    Do you require MFA to access sensitive data systems? & \no & \textcolor{red}{\textbf{Critical Gap}} \\
    Does your organization have an employee acceptable use policy? & \yes & Met \\
    Does your organization do security awareness training for new employees? & \no & \textcolor{orange}{\textbf{High Risk}} \\
    Does your organization do security awareness for all employees annually? & \yes & Met \\
    \bottomrule
\end{tabular}
\caption{Organizational Security Controls Questionnaire Results.}
\end{table}

The lack of MFA on email and sensitive systems, combined with the absence of security training during employee onboarding, constitutes a major deficiency in the organization's defense-in-depth strategy.

% --- 4. Technical Scan Results ---
\section*{4. Technical Scan Results}
A network scan was performed to identify open ports, running services, and potential vulnerabilities on the organization's external infrastructure.

\begin{itemize}
    \item \textbf{Target IP:} \texttt{[Target IP]}
    \item \textbf{Scan Date:} \today
\end{itemize}

\begin{table}[h!]
\centering
\begin{tabular}{@{}lllll@{}}
    \toprule
    \textbf{Port} & \textbf{Service} & \textbf{Product} & \textbf{Version} & \textbf{Finding} \\
    \midrule
    21/tcp & ftp & vsftpd & 2.3.4 & \begin{tabular}[t]{@{}l@{}}\textcolor{red}{\textbf{Critical Vulnerability:}} Outdated version \\ with known backdoor (CVE-2011-2523). \\ \textcolor{orange}{\textbf{High Risk:}} Anonymous FTP login allowed.\end{tabular} \\
    \bottomrule
\end{tabular}
\caption{Open Ports and Services Detected on \texttt{[Target IP]}.}
\end{table}

\subsection*{Analysis of Technical Findings}
The presence of an open FTP port is highly discouraged, as the protocol transmits credentials in cleartext. The specific version of \texttt{vsftpd} (2.3.4) detected is extremely old (circa 2011) and contains a well-documented backdoor vulnerability that allows an attacker to gain a remote command shell. The additional finding that anonymous login is permitted exacerbates this risk, as it allows unauthenticated users to interact with this highly vulnerable service. This configuration presents an immediate and critical threat to the network.

% --- 5. Consolidated Risk Assessment ---
\section*{5. Consolidated Risk Assessment}
This section synthesizes findings from the security control review, technical scan, and pre-existing risk register into a consolidated list of prioritized risks.

\begin{table}[h!]
\centering
\begin{tabular}{@{}lp{6.5cm}l@{}}
    \toprule
    \textbf{Risk Title} & \textbf{Description} & \textbf{Severity} \\
    \midrule
    Vulnerable FTP Server & An outdated FTP service (\texttt{vsftpd 2.3.4}) with a known remote code execution backdoor is exposed to the internet. & \textcolor{red}{\textbf{Critical}} \\
    \addlinespace
    Lack of MFA on Critical Systems & Email and sensitive data systems do not require Multi-Factor Authentication, making them susceptible to account takeover via compromised credentials. & \textcolor{red}{\textbf{Critical}} \\
    \addlinespace
    Anonymous FTP Access & The vulnerable FTP server is configured to allow anonymous (unauthenticated) user logins, lowering the barrier for exploitation. & \textcolor{orange}{\textbf{High}} \\
    \addlinespace
    Inadequate Onboarding Training & New employees do not receive security awareness training, leaving them unprepared to identify and respond to threats like phishing. & \textcolor{orange}{\textbf{High}} \\
    \addlinespace
    Outdated Windows Policy & Workstations are running Windows 7, an unsupported OS that no longer receives security patches from Microsoft, making them easy targets for malware. & \textcolor{yellow!80!black}{\textbf{Medium}} \\
    \bottomrule
\end{tabular}
\caption{Summary of Identified Risks.}
\end{table}

% --- 6. Recommendations ---
\section*{6. Recommendations}
The following prioritized actions are recommended to mitigate the identified risks and improve the overall security posture of \textbf{[Organization Name]}.

\subsection*{Immediate Priority (Remediate within 72 hours)}
\begin{enumerate}
    \item \textbf{Decommission Vulnerable FTP Server:} Immediately take the FTP service on \texttt{[Target IP]} offline. If file transfer functionality is required, deploy a secure alternative such as SFTP (SSH File Transfer Protocol) or FTPS (FTP over SSL/TLS) with strong authentication and up-to-date software.
    \item \textbf{Enforce MFA on Email and Sensitive Systems:} Procure and implement an MFA solution for all user accounts, prioritizing email (e.g., Office 365, Google Workspace) and any systems identified as containing sensitive data.
\end{enumerate}

\subsection*{High Priority (Remediate within 30 days)}
\begin{enumerate}
    \item \textbf{Disable Anonymous FTP Access:} As part of decommissioning or securing the FTP service, ensure that anonymous access is disabled across all file-sharing systems. Access should be granted on a need-to-know basis only.
    \item \textbf{Implement Onboarding Security Training:} Develop and integrate a mandatory security awareness training module into the new employee onboarding process. This training should cover acceptable use, phishing identification, password hygiene, and incident reporting.
\end{enumerate}

\subsection*{Medium Priority (Remediate within 6 months)}
\begin{enumerate}
    \item \textbf{Upgrade End-of-Life Operating Systems:} Initiate and execute a plan to upgrade all workstations running Windows 7 to a modern, supported operating system such as Windows 10 or Windows 11. This may require hardware refreshes for incompatible machines.
\end{enumerate}

% --- 7. Conclusion ---
\section*{7. Conclusion}
The assessment for \textbf{[Organization Name]} reveals a fragile security posture characterized by critical, systemic weaknesses. The combination of an easily exploitable external service, weak access controls on core systems, and gaps in employee security education creates a high probability of a successful cyberattack.

The recommendations provided in this report are designed to address the most severe risks first. By taking swift and decisive action, the organization can significantly reduce its attack surface and build a more resilient security foundation. Continuous monitoring and regular security assessments are advised to maintain and improve this posture over time.

\end{document}
```