```latex
\documentclass[12pt]{article}

% Preamble: Required Packages
\usepackage[margin=1in]{geometry} % Set page margins
\usepackage{pifont}               % For checkmarks and crosses (dingbats)
\usepackage{booktabs}             % For professional-looking tables
\usepackage{hyperref}             % For hyperlinks and document metadata
\usepackage{url}                  % For formatting URLs
\usepackage{seqsplit}             % For splitting long strings in \texttt
\usepackage{graphicx}             % For logos, etc.
\usepackage[table]{xcolor}        % For coloring table cells
\usepackage{longtable}            % For tables that span multiple pages

% --- Document Metadata ---
\hypersetup{
    colorlinks=true,
    linkcolor=blue,
    filecolor=magenta,      
    urlcolor=cyan,
    pdftitle={Cybersecurity Posture Assessment Report},
    pdfauthor={Cybersecurity Analyst},
    pdfsubject={Security Analysis},
    pdfkeywords={Cybersecurity, Nmap, Risk Assessment},
}

% --- Custom Commands & Colors ---
\newcommand{\yes}{\ding{51}} % Green checkmark
\newcommand{\no}{\ding{55}}  % Red X
\definecolor{highrisk}{HTML}{D9534F}
\definecolor{mediumrisk}{HTML}{F0AD4E}
\definecolor{lowrisk}{HTML}{5CB85C}

% --- Document Start ---
\begin{document}

% --- Title Page ---
\begin{titlepage}
    \centering
    \vspace*{1cm}
    \Huge\textbf{Cybersecurity Posture Assessment Report}
    \vspace{1.5cm}
    \Large
    \textbf{Prepared for:} \\
    \vspace{0.5cm}
    \Huge\textbf{[Organization Name]}
    \vfill
    \large
    \textbf{Date of Report:} \today \\
    \vspace{0.5cm}
    \textbf{Report ID:} CYBER-2023-001
\end{titlepage}

\tableofcontents
\newpage

% --- Section 1: Executive Summary ---
\section{Executive Summary}
This report provides a comprehensive assessment of the cybersecurity posture for \textbf{[Organization Name]}, based on an analysis of organizational security controls, a network vulnerability scan, and a review of pre-existing risk documentation.

The assessment reveals a mixed security posture. The organization has implemented critical controls such as Multi-Factor Authentication (MFA) for email and sensitive data access. However, significant gaps exist in foundational security practices that present a high level of risk.

\textbf{Key Findings:}
\begin{itemize}
    \item \textbf{Critical Control Gaps:} The lack of mandatory MFA for computer logins, the absence of an employee Acceptable Use Policy (AUP), and the failure to conduct annual security awareness training for all employees represent critical vulnerabilities. These gaps significantly increase the risk of unauthorized access, insider threats, and successful social engineering attacks.
    \item \textbf{Technical Scan Results:} An external network scan of the designated target IP address showed no open ports, with port 80 (HTTP) confirmed as closed. This is a positive finding, indicating that unencrypted web services are not exposed on this specific asset.
    \item \textbf{Data Discrepancy:} A notable discrepancy was found between the technical scan results and the current risk register. The register lists a medium-severity risk for an "Unencrypted Web Server" on port 80, while the scan shows this port is closed. This suggests that either the risk has been remediated without updating documentation, or it pertains to a different asset not included in the scan scope.
\end{itemize}

Immediate action is recommended to address the identified policy and endpoint security gaps to mitigate the most severe risks to the organization.

% --- Section 2: Organizational Information ---
\section{Organizational Information}
The following details were used as the basis for this assessment. Due to the anonymized nature of the provided data, placeholders have been used where necessary.

\begin{tabular}{@{}ll}
\toprule
\textbf{Attribute} & \textbf{Value} \\
\midrule
Organization Name & \textbf{[Organization Name]} \\
Primary Email Domain & \texttt{[Domain]} \\
External IP Scanned & \texttt{[Client IP]} \\
\bottomrule
\end{tabular}

% --- Section 3: Security Control Review ---
\section{Security Control Review}
A review of the organization's security controls was conducted via a questionnaire. The results highlight areas of both strength and weakness in the current security framework. "No" answers indicate significant gaps that require immediate attention.

\begin{longtable}{p{0.5\textwidth} c p{0.3\textwidth}}
\toprule
\textbf{Control Question} & \textbf{Status} & \textbf{Analyst Notes} \\
\midrule
\endfirsthead
\toprule
\textbf{Control Question} & \textbf{Status} & \textbf{Analyst Notes} \\
\midrule
\endhead
\bottomrule
\endfoot
Do you require MFA to access email? & \yes & Strong control. Protects against phishing and credential theft for a primary communication vector. \\
\addlinespace
Do you require MFA to log into computers? & \no & \textbf{High Risk.} Lack of endpoint MFA means a compromised password could grant an attacker direct access to a workstation and the internal network. \\
\addlinespace
Do you require MFA to access sensitive data systems? & \yes & Excellent practice. Protects the organization's most critical data assets. \\
\addlinespace
Does your organization have an employee acceptable use policy? & \no & \textbf{Critical Gap.} Without an AUP, there is no formal guidance for employees on the proper use of company assets, increasing legal and security risks. \\
\addlinespace
Does your organization do security awareness training for new employees? & \yes & Good practice for onboarding. Ensures a baseline of security knowledge from day one. \\
\addlinespace
Does your organization do security awareness training for all employees at least once per year? & \no & \textbf{High Risk.} The threat landscape evolves constantly. Lack of annual training leaves employees vulnerable to new phishing and social engineering tactics. \\
\end{longtable}

% --- Section 4: Technical Scan Results ---
\section{Technical Scan Results}
A network scan was performed to identify exposed services on the organization's external infrastructure.

\begin{itemize}
    \item \textbf{Target IP Address:} \texttt{[Target IP]}
    \item \textbf{Scan Date:} Scan data provided on \today
    \item \textbf{Status:} Host is up.
\end{itemize}

\subsection{Port Scan Analysis}
The scan revealed no open ports on the target system. Specifically, port 80 (HTTP) was found to be in a \textbf{closed} state.

\begin{tabular}{@{}llll}
\toprule
\textbf{Port} & \textbf{State} & \textbf{Service} & \textbf{Notes} \\
\midrule
80/tcp & closed & http & No unencrypted web service is exposed on this host. \\
\bottomrule
\end{tabular}

\subsection{Discrepancy with Existing Risk Data}
The current risk register (see Section 5) contains a finding for an "Unencrypted Web Server" on port 80. This scan's finding that port 80 is closed directly contradicts that risk. This indicates a potential gap in risk management processes, where risks are not being properly verified or closed out after remediation.

% --- Section 5: Consolidated Risk Assessment ---
\section{Consolidated Risk Assessment}
This table synthesizes findings from the security control review, technical scan, and pre-existing risk data. New risks identified during this assessment are included.

\begin{longtable}{p{0.25\textwidth} p{0.1\textwidth} p{0.6\textwidth}}
\toprule
\textbf{Risk Name} & \textbf{Severity} & \textbf{Description \& Affected Elements} \\
\midrule
\endfirsthead
\toprule
\textbf{Risk Name} & \textbf{Severity} & \textbf{Description \& Affected Elements} \\
\midrule
\endhead
\bottomrule
\endfoot
Lack of Endpoint MFA & \cellcolor{highrisk}High & The absence of MFA on computer logins exposes the organization to significant risk. A single compromised password could lead to full workstation and potential network compromise. \newline \textit{Affected: All employee workstations.} \\
\addlinespace
Inadequate Security Awareness Training & \cellcolor{highrisk}High & Without mandatory annual training, employees are not kept up-to-date on evolving threats. This makes the organization highly susceptible to phishing, ransomware, and other social engineering attacks. \newline \textit{Affected: All employees.} \\
\addlinespace
Absence of Acceptable Use Policy (AUP) & \cellcolor{mediumrisk}Medium & This is a critical governance gap. Without a formal AUP, there is no enforceable policy defining appropriate use of company systems, data, and internet access, creating legal and operational risks. \newline \textit{Affected: Entire organization (policy).} \\
\addlinespace
Unencrypted Web Server (Existing Risk) & \cellcolor{mediumrisk}Medium & \textbf{Verification Required.} The existing risk register states that port 80 is open. However, our scan of \texttt{[Target IP]} found this port to be closed. This risk may be outdated or may apply to a different asset. \newline \textit{Affected: Port 80 (as documented).} \\
\end{longtable}

% --- Section 6: Recommendations ---
\section{Recommendations}
The following actions are recommended to address the identified risks and improve the overall security posture of \textbf{[Organization Name]}.

\subsection{High Priority (Implement Immediately)}
\begin{enumerate}
    \item \textbf{Deploy Endpoint MFA:} Mandate and enforce the use of Multi-Factor Authentication for all employee computer and laptop logins. This is the single most effective control to mitigate the risk of compromised credentials.
    \item \textbf{Establish and Enforce an AUP:} Develop a comprehensive Acceptable Use Policy that clearly defines rules for all employees regarding the use of company technology, data, and network resources. All employees must read and acknowledge the policy.
    \item \textbf{Implement Annual Security Training:} Schedule and mandate annual security awareness training for all employees. The training should cover current threats such as phishing, ransomware, and secure data handling.
\end{enumerate}

\subsection{Medium Priority (Address within 90 Days)}
\begin{enumerate}
    \item \textbf{Re-evaluate "Unencrypted Web Server" Risk:} Conduct a thorough review of all external-facing assets to determine if port 80 is open anywhere.
    \begin{itemize}
        \item If the port is confirmed to be closed everywhere, formally close the risk in the risk register.
        \item If the port is found open on another asset, update the risk with the correct information and prioritize remediation (e.g., implementing TLS/SSL encryption).
    \end{itemize}
\end{enumerate}

\end{document}
```