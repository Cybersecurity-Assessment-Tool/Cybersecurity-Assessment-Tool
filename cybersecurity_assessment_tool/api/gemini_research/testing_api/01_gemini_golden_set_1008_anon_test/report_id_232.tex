```latex
\documentclass[12pt]{article}

% Preamble: Required Packages
\usepackage[margin=1in]{geometry}
\usepackage{pifont} % For checkmarks and crosses
\usepackage{booktabs} % For professional tables
\usepackage{hyperref} % For clickable links
\usepackage{url} % For URL formatting
\usepackage{seqsplit} % To split long strings without breaking
\usepackage{graphicx}
\usepackage{xcolor}
\usepackage{fancyhdr}

% --- Document Setup ---
\hypersetup{
    colorlinks=true,
    linkcolor=blue,
    filecolor=magenta,      
    urlcolor=cyan,
    pdftitle={Cybersecurity Posture Report},
    pdfpagemode=FullScreen,
}

\pagestyle{fancy}
\fancyhf{}
\lhead{Cybersecurity Posture Report}
\rhead{\textbf{[Organization Name]}}
\cfoot{\thepage}

% --- Custom Commands ---
\newcommand{\yes}{\ding{51}}
\newcommand{\no}{\ding{55}}
\newcommand{\riskcritical}{\textcolor{red}{\textbf{Critical}}}
\newcommand{\riskhigh}{\textcolor{orange}{\textbf{High}}}
\newcommand{\riskmedium}{\textcolor{yellow}{\textbf{Medium}}}
\newcommand{\risklow}{\textcolor{green}{\textbf{Low}}}
\newcommand{\riskinformational}{\textcolor{blue}{\textbf{Informational}}}

\begin{document}

% --- Title Page ---
\begin{titlepage}
    \centering
    \vspace*{1cm}
    \Huge
    \textbf{Cybersecurity Posture Report}
    \vspace{1.5cm}
    \Large
    Prepared for: \\
    \vspace{0.5cm}
    \textbf{[Organization Name]}
    \vspace{2cm}
    \includegraphics[width=0.4\textwidth]{example-image-a} % Placeholder for a logo
    \vfill
    \Large
    Report Date: \today
\end{titlepage}

\tableofcontents
\newpage

% --- Section 1: Executive Summary ---
\section{Executive Summary}
This report provides a comprehensive analysis of the cybersecurity posture for \textbf{[Organization Name]}, based on a review of organizational security controls, an external network scan, and pre-existing risk data.

The assessment reveals a mixed security posture. On one hand, the external network scan of the target IP address indicates a strong perimeter defense, with no open ports detected. This significantly reduces the external attack surface and is a commendable security practice.

However, the review of organizational security controls has identified several critical and high-risk gaps. The absence of Multi-Factor Authentication (MFA) on sensitive data systems represents a \riskcritical{} vulnerability. Furthermore, the lack of a formal security awareness training program for both new and existing employees constitutes a \riskhigh{} risk, leaving the organization highly susceptible to social engineering and phishing attacks.

While the external defenses are robust, these internal control deficiencies could allow an attacker who compromises a single user account to gain direct access to sensitive information. Immediate remediation of these identified gaps is strongly recommended to mitigate the risk of a significant security breach.

% --- Section 2: Organizational Information ---
\section{Organizational Information}
This section details the organizational data provided for this assessment. The information has been anonymized as per the engagement requirements.

\begin{tabular}{@{}ll}
\toprule
\textbf{Attribute} & \textbf{Value} \\
\midrule
Organization Name & \textbf{[Organization Name]} \\
Email Domain & \texttt{[Domain]} \\
External IP Scanned & \texttt{[Target IP]} \\
\bottomrule
\end{tabular}

% --- Section 3: Security Control Review ---
\section{Security Control Review}
The following table summarizes the responses from the organizational security questionnaire. "No" answers indicate potential gaps in the security framework and are flagged for review.

\begin{tabular}{@{}p{0.6\linewidth} c p{0.2\linewidth}@{}}
\toprule
\textbf{Control Question} & \textbf{Response} & \textbf{Assessment} \\
\midrule
Do you require MFA to access email? & \yes & Compliant \\
Do you require MFA to log into computers? & \yes & Compliant \\
Do you require MFA to access sensitive data systems? & \no & \riskcritical{} Gap \\
Does your organization have an employee acceptable use policy? & \yes & Compliant \\
Does your organization do security awareness training for new employees? & \no & \riskhigh{} Risk \\
Does your organization do security awareness training for all employees at least once per year? & \no & \riskhigh{} Risk \\
\bottomrule
\end{tabular}

\subsection*{Analysis of Gaps}
\begin{itemize}
    \item \textbf{MFA on Sensitive Systems:} The lack of MFA on systems containing sensitive data is a critical oversight. Should an employee's credentials be compromised, there is no secondary control to prevent an attacker from accessing the organization's most valuable data.
    \item \textbf{Security Awareness Training:} Without a formal training program, employees are not equipped to identify and respond to common cyber threats like phishing, malware, and social engineering. This turns the workforce into a potential vulnerability rather than a line of defense.
\end{itemize}

% --- Section 4: Technical Scan Results ---
\section{Technical Scan Results}
An external network vulnerability scan was conducted to identify weaknesses visible from the public internet.

\subsection*{Scan Details}
\begin{itemize}
    \item \textbf{Target IP Address:} \texttt{[Target IP]}
    \item \textbf{Scan Type:} Nmap TCP Port Scan
    \item \textbf{Scan Date:} Data provided on \today
\end{itemize}

\subsection*{Summary of Findings}
The scan results were positive, indicating a strong network perimeter. The target host was responsive, but all scanned ports were found to be in a \texttt{closed} state. This suggests that a firewall is properly configured to deny unsolicited inbound traffic, effectively minimizing the external attack surface.

\begin{verbatim}
Scan Target: [Target IP]
Host Status: up
Open Ports: None
Extra Ports State: 1000 closed ports
\end{verbatim}

\textbf{Conclusion:} No externally accessible vulnerabilities were identified during this scan. This is an excellent security posture from a network perspective.

% --- Section 5: Overall Risk Assessment ---
\section{Overall Risk Assessment}
This section synthesizes findings from the security control review, technical scan, and pre-existing risk data. The following new risks have been identified based on this assessment. No pre-existing vulnerabilities were reported.

\begin{tabular}{@{}p{0.1\linewidth} p{0.25\linewidth} p{0.45\linewidth} p{0.1\linewidth}@{}}
\toprule
\textbf{Risk ID} & \textbf{Risk Name} & \textbf{Description} & \textbf{Severity} \\
\midrule
RISK-001 & Lack of MFA on Sensitive Systems & The absence of a secondary authentication factor on critical data repositories allows for unauthorized access if primary credentials are stolen. & \riskcritical{} \\
\addlinespace
RISK-002 & Inadequate Security Awareness Training & Employees are not trained to recognize or report security threats like phishing, increasing the likelihood of a successful social engineering attack. & \riskhigh{} \\
\bottomrule
\end{tabular}

% --- Section 6: Recommendations ---
\section{Recommendations}
The following actionable recommendations are provided to address the identified risks and improve the overall security posture of \textbf{[Organization Name]}.

\subsection*{RISK-001: Lack of MFA on Sensitive Systems (Critical)}
\begin{itemize}
    \item \textbf{Immediate Action:} Prioritize the implementation of a robust MFA solution for all systems, applications, and databases that store, process, or transmit sensitive organizational or customer data.
    \item \textbf{Policy Enforcement:} Update access control policies to mandate the use of MFA for all privileged users and any user accessing sensitive information, regardless of their location (internal or remote).
    \item \textbf{Technology:} Select an MFA solution that supports a range of methods (e.g., authenticator apps, hardware tokens, biometrics) to ensure both security and user convenience.
\end{itemize}

\subsection*{RISK-002: Inadequate Security Awareness Training (High)}
\begin{itemize}
    \item \textbf{Immediate Action:} Procure and deploy a security awareness training platform. Enroll all current employees in a baseline training module covering phishing, password security, and acceptable use.
    \item \textbf{Onboarding Process:} Integrate mandatory security awareness training into the onboarding process for all new hires. Access to systems should not be granted until this training is complete.
    \item \textbf{Continuous Training:} Establish a recurring training schedule, requiring all employees to complete a refresher course at least annually. Supplement this with regular phishing simulation campaigns to test and reinforce learning.
\end{itemize}

\end{document}
```