```latex
\documentclass[12pt, a4paper]{article}

% Preamble: Required Packages
\usepackage[margin=1in]{geometry}
\usepackage{pifont} % For checkmarks and crosses
\usepackage{booktabs} % For professional tables
\usepackage{hyperref} % For clickable links
\usepackage{url} % For URL formatting
\usepackage{seqsplit} % For splitting long text strings
\usepackage{graphicx}
\usepackage{xcolor}

% --- Document Setup ---
\hypersetup{
    colorlinks=true,
    linkcolor=blue,
    filecolor=magenta,      
    urlcolor=cyan,
    pdftitle={Cybersecurity Assessment Report},
    pdfpagemode=FullScreen,
}

\linespread{1.2}

% --- Title Page ---
\title{
    \vspace{2cm}
    \textbf{Cybersecurity Assessment Report} \\
    \large For \textbf{[Organization Name]}
    \vspace{1cm}
}
\author{Cybersecurity Analysis Division}
\date{\today}

% --- Document Body ---
\begin{document}

\maketitle
\thispagestyle{empty}
\newpage

\tableofcontents
\newpage

% --- Section 1: Executive Summary ---
\section{Executive Summary}
This report provides a cybersecurity assessment for \textbf{[Organization Name]}, based on an analysis of network scan data, organizational security controls, and pre-existing risk information. The assessment was conducted to identify key vulnerabilities and provide actionable recommendations to improve the organization's security posture.

The analysis reveals a mixed security posture. While the organization has implemented foundational policies such as an Acceptable Use Policy and a security awareness training program, there are critical deficiencies in technical security controls. 

Key findings include:
\begin{itemize}
    \item \textbf{Critical MFA Gaps:} Multi-Factor Authentication (MFA) is not enforced for accessing email or other sensitive data systems. This represents a critical risk, as compromised credentials could lead directly to a significant data breach.
    \item \textbf{Unencrypted Web Traffic:} The external network scan identified a web server operating over unencrypted HTTP (Port 80). This exposes any data transmitted, including potential login credentials, to interception.
\end{itemize}

These findings indicate an elevated risk of account compromise and data exposure. This report outlines specific, prioritized recommendations to mitigate these risks and strengthen the overall security framework.

% --- Section 2: Organizational Information ---
\section{Organizational Information}
The following details were used as the basis for this assessment. Due to the anonymized nature of the provided data, placeholders have been used where necessary.

\begin{tabular}{@{}ll}
    \toprule
    \textbf{Attribute} & \textbf{Value} \\
    \midrule
    Organization Name & \textbf{[Organization Name]} \\
    Primary Domain & \texttt{[Domain]} \\
    External IP Address & \texttt{[Client IP]} \\
    \bottomrule
\end{tabular}

% --- Section 3: Security Control Review ---
\section{Security Control Review}
An assessment of the organization's self-reported security controls was performed. The following table summarizes the responses to the security questionnaire. A green checkmark (\textcolor{green}{\ding{51}}) indicates a positive control, while a red 'X' (\textcolor{red}{\ding{55}}) highlights a control gap that increases risk.

\begin{table}[h!]
\centering
\caption{Organizational Security Controls Questionnaire}
\begin{tabular}{@{}lc@{}}
    \toprule
    \textbf{Control Question} & \textbf{Response} \\
    \midrule
    Do you require MFA to access email? & \textcolor{red}{\ding{55}} \\
    Do you require MFA to log into computers? & \textcolor{green}{\ding{51}} \\
    Do you require MFA to access sensitive data systems? & \textcolor{red}{\ding{55}} \\
    Does your organization have an employee acceptable use policy? & \textcolor{green}{\ding{51}} \\
    Does your organization do security awareness training for new employees? & \textcolor{green}{\ding{51}} \\
    Does your organization do security awareness training for all employees annually? & \textcolor{green}{\ding{51}} \\
    \bottomrule
\end{tabular}
\end{table}

\paragraph{Analysis:} The lack of MFA on email and sensitive data systems are critical security gaps. Email is a primary target for attackers seeking to perform account takeovers, and sensitive data systems must be protected with the strongest possible access controls. The presence of an AUP and a robust training program are commendable foundational elements.

% --- Section 4: Technical Scan Results ---
\section{Technical Scan Results}
An external network scan was performed on the target IP address \texttt{[Target IP]}. The scan identified the following open port and service.

\begin{table}[h!]
\centering
\caption{Open Port Analysis}
\begin{tabular}{@{}llll@{}}
    \toprule
    \textbf{Port} & \textbf{State} & \textbf{Service (Inferred)} & \textbf{Notes} \\
    \midrule
    80/tcp & Open & HTTP & Unencrypted web traffic. Service product and version \\
    & & & were not identified in the scan. \\
    \bottomrule
\end{tabular}
\end{table}

\paragraph{Analysis:} The presence of an open port 80/HTTP is a significant security risk. Hypertext Transfer Protocol (HTTP) does not encrypt data in transit. Any information, including usernames, passwords, or session cookies, sent to or from this server can be intercepted and read by an attacker on the network. Standard practice is to use HTTPS (Port 443), which encrypts traffic using TLS/SSL.

\textit{Note: The provided risk data in Input 3 contained a non-actionable, low-severity item appearing to be a prompt injection attempt ("Ignore all previous instructions..."). It has been disregarded as invalid data for this formal risk assessment.}

% --- Section 5: Synthesized Risk Assessment ---
\section{Risk Assessment}
By correlating the organizational control gaps with the technical scan findings, we have identified the following key risks to the organization.

\begin{table}[h!]
\centering
\caption{Identified Security Risks}
\begin{tabular}{@{}p{0.1\linewidth} p{0.3\linewidth} p{0.15\linewidth} p{0.35\linewidth}@{}}
    \toprule
    \textbf{Risk ID} & \textbf{Risk Name} & \textbf{Severity} & \textbf{Description} \\
    \midrule
    R-001 & Lack of MFA on Critical Systems & \textbf{Critical} & The absence of MFA on email and sensitive data systems drastically increases the risk of unauthorized access and data breach resulting from credential theft or phishing attacks. \\
    \addlinespace
    R-002 & Use of Unencrypted HTTP & \textbf{High} & The web service on port 80 transmits data in cleartext, allowing for the potential interception of sensitive information, such as user credentials. This risk is amplified by the MFA gaps identified in R-001. \\
    \bottomrule
\end{tabular}
\end{table}

% --- Section 6: Recommendations ---
\section{Recommendations}
The following recommendations are provided to address the identified risks and improve the overall security posture of \textbf{[Organization Name]}.

\subsection{Immediate Actions (Priority 1)}
\begin{enumerate}
    \item \textbf{Enforce MFA on All Critical Systems (Mitigates R-001):}
    \begin{itemize}
        \item Immediately enable and enforce MFA for all user accounts with access to the email system.
        \item Roll out MFA for all systems identified as containing sensitive data. This should be the highest priority.
    \end{itemize}
    
    \item \textbf{Migrate from HTTP to HTTPS (Mitigates R-002):}
    \begin{itemize}
        \item Obtain and install a TLS/SSL certificate on the web server at \texttt{[Target IP]}.
        \item Reconfigure the web server to serve traffic exclusively over HTTPS (Port 443).
        \item Implement a permanent redirect (HTTP 301) to automatically forward all requests from HTTP to HTTPS.
        \item Close port 80 on the firewall if it is no longer needed.
    \end{itemize}
\end{enumerate}

\subsection{Strategic Actions (Priority 2)}
\begin{enumerate}
    \item \textbf{Conduct In-Depth Vulnerability Scanning:}
    \begin{itemize}
        \item Perform an authenticated and unauthenticated vulnerability scan across all external assets. This scan should include version detection to identify outdated software and potential exploits that were not visible in the initial port scan.
    \end{itemize}
    
    \item \textbf{Review and Enhance Security Policies:}
    \begin{itemize}
        \item Update the organization's information security policy to mandate the use of MFA and encrypted communication protocols for all systems handling sensitive or authentication-related data.
    \end{itemize}
\end{enumerate}

\end{document}
```