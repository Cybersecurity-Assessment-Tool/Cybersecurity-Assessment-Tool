```latex
\documentclass[12pt, a4paper]{article}

% Preamble: Required Packages
\usepackage[margin=1in]{geometry}
\usepackage{pifont} % For checkmarks and crosses (\ding{51}, \ding{55})
\usepackage{booktabs} % For professional tables (\toprule, \midrule, \bottomrule)
\usepackage[hidelinks]{hyperref} % For clickable links without boxes
\usepackage{url} % For formatting URLs
\usepackage{seqsplit} % For splitting long strings in \texttt
\usepackage{graphicx}
\usepackage{xcolor}
\usepackage{fancyhdr}
\usepackage{lastpage}

% --- Document Setup ---
\definecolor{darkblue}{rgb}{0.0, 0.0, 0.55}
\definecolor{darkred}{rgb}{0.55, 0.0, 0.0}

\hypersetup{
    colorlinks=true,
    linkcolor=darkblue,
    filecolor=darkblue,      
    urlcolor=darkblue,
    citecolor=darkblue,
}

% --- Header and Footer ---
\pagestyle{fancy}
\fancyhf{} % Clear all header and footer fields
\fancyhead[L]{Cybersecurity Assessment Report}
\fancyhead[R]{\textbf{[Organization Name]}}
\fancyfoot[C]{\thepage\ of \pageref{LastPage}}
\renewcommand{\headrulewidth}{0.4pt}
\renewcommand{\footrulewidth}{0.4pt}

% --- Title ---
\title{
    \vspace{2cm}
    \textbf{Cybersecurity Posture Assessment Report}\\
    \large For: \textbf{[Organization Name]}
    \vspace{1cm}
}
\author{Cybersecurity Analysis Division}
\date{\today}

\begin{document}

\maketitle
\thispagestyle{empty}
\newpage

\tableofcontents
\newpage

% --- Section 1: Executive Summary ---
\section{Executive Summary}

This report provides a comprehensive analysis of the cybersecurity posture for \textbf{[Organization Name]}, based on data collected from network scans, organizational questionnaires, and a review of pre-existing risks. The assessment was conducted on \today.

The analysis reveals several critical and high-risk security gaps that require immediate attention. Key findings include the absence of Multi-Factor Authentication (MFA) for computer logins, the lack of a formal Acceptable Use Policy (AUP), and incomplete security awareness training programs. These policy and procedural weaknesses create a significant risk of unauthorized access and internal threats.

From a technical perspective, an external scan identified an open Secure Shell (SSH) port (22/TCP) on the network perimeter. While necessary for remote administration, its exposure without proper controls like IP whitelisting and robust authentication presents a tangible attack vector.

This report synthesizes these findings into a prioritized list of risks and provides actionable recommendations to mitigate them. We urge management to review these findings and allocate resources to implement the proposed security enhancements promptly.

% --- Section 2: Organizational Information ---
\section{Organizational Information}

The following information was used as the basis for this assessment. Due to the anonymized nature of the provided data, placeholders have been used where necessary.

\begin{table}[h!]
\centering
\caption{Client Details}
\begin{tabular}{@{}ll@{}}
\toprule
\textbf{Attribute} & \textbf{Value} \\ \midrule
Organization Name    & \textbf{[Organization Name]} \\
Primary Domain       & \texttt{[Domain]} \\
External IP Scanned  & \texttt{[Client IP]} \\
Scan Target IP       & \texttt{[Target IP]} \\
Assessment Date      & \today \\ \bottomrule
\end{tabular}
\end{table}

% --- Section 3: Security Control Review ---
\section{Security Control Review}

The following table summarizes the organization's responses to a security controls questionnaire. Answers marked with a red 'X' (\textcolor{darkred}{\ding{55}}) indicate a deviation from security best practices and represent a significant gap in the organization's defensive posture.

\begin{table}[h!]
\centering
\caption{Security Questionnaire Analysis}
\begin{tabular}{@{}p{0.6\linewidth}cc@{}}
\toprule
\textbf{Question} & \textbf{Response} & \textbf{Status} \\ \midrule
Do you require MFA to access email? & \textcolor{green}{\ding{51}} & Aligned \\
Do you require MFA to log into computers? & \textcolor{darkred}{\ding{55}} & \textbf{Critical Gap} \\
Do you require MFA to access sensitive data systems? & \textcolor{green}{\ding{51}} & Aligned \\
Does your organization have an employee acceptable use policy? & \textcolor{darkred}{\ding{55}} & \textbf{Critical Gap} \\
Does your organization do security awareness training for new employees? & \textcolor{green}{\ding{51}} & Aligned \\
Does your organization do security awareness training for all employees at least once per year? & \textcolor{darkred}{\ding{55}} & \textbf{High Risk} \\ \bottomrule
\end{tabular}
\end{table}

% --- Section 4: Technical Scan Results ---
\section{Technical Scan Results}

An external network scan was performed against the target IP address \texttt{[Target IP]} to identify open ports and exposed services.

\subsection{Nmap Scan Findings}
The scan revealed one open port on the host, as detailed below.

\begin{table}[h!]
\centering
\caption{Open Ports on \texttt{[Target IP]}}
\begin{tabular}{@{}lllll@{}}
\toprule
\textbf{Port} & \textbf{Protocol} & \textbf{State} & \textbf{Service} & \textbf{Notes} \\ \midrule
22 & TCP & open & ssh & Secure Shell (SSH) is exposed. Version information was not available from the scan. \\ \bottomrule
\end{tabular}
\end{table}

\paragraph{Analysis:} The presence of an open SSH port is common for remote system administration. However, exposing this service directly to the internet increases the risk of brute-force attacks and exploitation of potential vulnerabilities. This risk is compounded by the lack of MFA for computer logins, as compromised credentials could be used to gain direct server access.

% --- Section 5: Consolidated Risk Assessment ---
\section{Consolidated Risk Assessment}

This section correlates the findings from the security control review and the technical scan. The pre-existing risk register was empty, so all identified risks are new findings from this assessment.

\begin{table}[h!]
\centering
\caption{Identified Risks}
\begin{tabular}{@{}p{0.1\linewidth}p{0.25\linewidth}p{0.4\linewidth}l@{}}
\toprule
\textbf{Risk ID} & \textbf{Risk Name} & \textbf{Description} & \textbf{Severity} \\ \midrule
RISK-001 & Inadequate Access Control & The absence of MFA on computer logins allows an attacker with stolen credentials to easily gain access and move laterally within the network. & \textcolor{darkred}{\textbf{Critical}} \\
\addlinespace
RISK-002 & Lack of Formal Security Policies & Without a documented Acceptable Use Policy (AUP), there are no clear guidelines for employees, increasing the risk of insider threats and non-compliance. & \textcolor{orange}{\textbf{High}} \\
\addlinespace
RISK-003 & Insufficient Security Training & Failing to provide annual security training for all employees leads to a decline in security awareness, making them more susceptible to phishing and social engineering. & \textcolor{orange}{\textbf{High}} \\
\addlinespace
RISK-004 & Exposed Management Service & The SSH port is open to the internet, creating a direct vector for attackers to attempt unauthorized access. & \textcolor{yellow!80!black}{\textbf{Medium}} \\ \bottomrule
\end{tabular}
\end{table}

% --- Section 6: Recommendations ---
\section{Recommendations}

Based on the risk assessment, the following actions are recommended to improve the security posture of \textbf{[Organization Name]}.

\begin{enumerate}
    \item \textbf{Implement MFA for Endpoint Logins (RISK-001):}
    \begin{itemize}
        \item \textbf{Action:} Deploy a Multi-Factor Authentication solution for all workstation and server logins (e.g., Windows Hello for Business, Duo, or similar).
        \item \textbf{Priority:} \textbf{Critical}. This is the most effective single control to prevent unauthorized access from compromised credentials.
    \end{itemize}
    \vspace{0.5cm}

    \item \textbf{Develop and Enforce an Acceptable Use Policy (RISK-002):}
    \begin{itemize}
        \item \textbf{Action:} Create a formal AUP that defines rules for the use of company assets, data handling, and internet usage. Ensure all employees read and acknowledge the policy.
        \item \textbf{Priority:} \textbf{High}. A foundational policy is essential for governance and compliance.
    \end{itemize}
    \vspace{0.5cm}

    \item \textbf{Establish Annual Security Awareness Training (RISK-003):}
    \begin{itemize}
        \item \textbf{Action:} Institute a mandatory security awareness training program for all employees, to be completed annually. The training should cover phishing, password security, and social engineering.
        \item \textbf{Priority:} \textbf{High}. A well-informed workforce is a critical layer of defense.
    \end{itemize}
    \vspace{0.5cm}

    \item \textbf{Secure the SSH Service (RISK-004):}
    \begin{itemize}
        \item \textbf{Action:} Restrict access to the SSH port (22/TCP) using a firewall. Only allow connections from trusted, whitelisted IP addresses. Additionally, configure the SSH service to disallow password-based authentication and require public key authentication instead.
        \item \textbf{Priority:} \textbf{Medium}. This is a straightforward technical control that significantly reduces external attack surface.
    \end{itemize}
\end{enumerate}

\end{document}
```