```latex
\documentclass[12pt]{article}

% --- PACKAGES ---
\usepackage[margin=1in]{geometry}
\usepackage{pifont}                 % For symbols like checkmarks
\usepackage{booktabs}               % For professional-looking tables
\usepackage{hyperref}               % For clickable links and references
\usepackage{url}                    % For formatting URLs
\usepackage{seqsplit}               % For splitting long strings without spaces
\usepackage{graphicx}
\usepackage{xcolor}

% --- DOCUMENT CONFIGURATION ---
\hypersetup{
    colorlinks=true,
    linkcolor=blue,
    filecolor=magenta,
    urlcolor=cyan,
}

% --- CUSTOM COMMANDS ---
\newcommand{\yes}{\ding{51}} % Checkmark
\newcommand{\no}{\ding{55}}  % X-mark

% --- DOCUMENT START ---
\begin{document}

% --- TITLE PAGE ---
\title{Cybersecurity Posture Assessment Report}
\author{Cybersecurity Analysis Division}
\date{\today}
\maketitle
\thispagestyle{empty}
\newpage

\tableofcontents
\newpage

% --- EXECUTIVE SUMMARY ---
\section{Executive Summary}

This report provides a cybersecurity posture assessment for \textbf{[Organization Name]}, based on an analysis of organizational security controls, external network scan results, and a review of pre-existing risks.

The assessment has identified several critical and high-severity risks that require immediate attention. The most critical finding is the direct exposure of a Remote Desktop Protocol (RDP) service to the public internet. This configuration is a primary vector for ransomware attacks and unauthorized access.

Furthermore, significant gaps in fundamental security controls were identified, including the lack of Multi-Factor Authentication (MFA) for email and computer access. The absence of a recurring, annual security awareness training program for all staff exacerbates these risks, leaving the organization vulnerable to credential theft and social engineering attacks.

Immediate remediation of the exposed RDP service and the enforcement of MFA are paramount to reducing the organization's risk of a significant security breach.

% --- ORGANIZATIONAL INFORMATION ---
\section{Organizational Information}

This section outlines the high-level information for the organization under review. The data provided was anonymized for this report template.

\begin{center}
\begin{tabular}{@{}ll}
\toprule
\textbf{Attribute} & \textbf{Value} \\
\midrule
Organization Name & \textbf{[Organization Name]} \\
Primary Email Domain & \texttt{[Domain]} \\
External IP Address Scanned & \texttt{[Client IP]} \\
\bottomrule
\end{tabular}
\end{center}

% --- SECURITY CONTROL REVIEW ---
\section{Security Control Review}

The following table summarizes the organization's self-reported security controls based on a standard questionnaire. Gaps in these controls often indicate systemic weaknesses in the overall security posture. A red \no\ indicates a significant gap.

\begin{center}
\begin{tabular}{p{0.7\textwidth} c}
\toprule
\textbf{Control Question} & \textbf{Status} \\
\midrule
Do you require MFA to access email? & \textcolor{red}{\no} \\
Do you require MFA to log into computers? & \textcolor{red}{\no} \\
Do you require MFA to access sensitive data systems? & \textcolor{green}{\yes} \\
\addlinespace
Does your organization have an employee acceptable use policy? & \textcolor{green}{\yes} \\
Does your organization do security awareness training for new employees? & \textcolor{green}{\yes} \\
Does your organization do security awareness training for all employees at least once per year? & \textcolor{red}{\no} \\
\bottomrule
\end{tabular}
\end{center}

% --- TECHNICAL SCAN RESULTS ---
\section{Technical Scan Results}

An external network vulnerability scan was conducted to identify publicly exposed services. The scan confirmed the findings from the pre-existing risk report.

\subsection*{Host: \texttt{[Target IP]}}
The following services were found to be open and accessible from the public internet.

\begin{center}
\begin{tabular}{l l l l}
\toprule
\textbf{Port} & \textbf{State} & \textbf{Service} & \textbf{Notes} \\
\midrule
3389/tcp & Open & ms-wbt-server & Remote Desktop Protocol (RDP) \\
\bottomrule
\end{tabular}
\end{center}

% --- RISK ASSESSMENT ---
\section{Risk Assessment}

This section synthesizes the findings from the security control review, technical scan, and pre-existing risk data into a prioritized list of security risks.

\begin{center}
\begin{tabular}{p{0.3\textwidth} p{0.15\textwidth} p{0.5\textwidth}}
\toprule
\textbf{Risk} & \textbf{Severity} & \textbf{Description and Business Impact} \\
\midrule
\textbf{Publicly Exposed RDP} & \textbf{\textcolor{red}{Critical}} & The scan confirmed that RDP (Port 3389) is open on host \texttt{[Target IP]}. This aligns with the pre-existing risk data and is a common target for brute-force attacks and exploitation, often leading to ransomware deployment and full network compromise. \\
\addlinespace
\textbf{No MFA for Email Access} & \textbf{\textcolor{red}{Critical}} & The lack of MFA on email accounts makes them highly susceptible to compromise via phishing or credential stuffing. A compromised email account is a gateway to business email compromise (BEC), data exfiltration, and further attacks against employees and clients. \\
\addlinespace
\textbf{No MFA for Computer Logins} & \textbf{\textcolor{red}{Critical}} & The absence of MFA on computer logins removes a critical layer of defense. If an attacker obtains valid credentials, they can gain direct access to endpoints and move laterally within the network, escalating privileges and accessing sensitive data. \\
\addlinespace
\textbf{Lack of Annual Security Training} & \textbf{\textcolor{orange}{High}} & Without regular, recurring security awareness training, employees are more likely to fall victim to social engineering and phishing attacks. This directly increases the likelihood of credential compromise, which would enable the exploitation of the other critical risks listed above. \\
\bottomrule
\end{tabular}
\end{center}

% --- RECOMMENDATIONS ---
\section{Recommendations}

The following actions are recommended to mitigate the identified risks. Recommendations are prioritized based on severity and the potential for impact reduction.

\subsection*{Immediate Priority (Remediate within 24-48 Hours)}
\begin{itemize}
    \item \textbf{Mitigate RDP Exposure:} Immediately close port 3389 on the external firewall for host \texttt{[Target IP]}. If remote access is essential, it \textbf{must} be placed behind a Virtual Private Network (VPN) that requires Multi-Factor Authentication.
    \item \textbf{Enforce MFA for Email:} Immediately enable and enforce MFA for all user accounts on the \texttt{[Domain]} email system. This is the single most effective control to prevent business email compromise.
\end{itemize}

\subsection*{High Priority (Remediate within 1-2 Weeks)}
\begin{itemize}
    \item \textbf{Enforce MFA for Endpoints:} Deploy and require MFA for all computer and server logins, prioritizing administrative accounts and users with access to sensitive data.
    \item \textbf{Establish Security Training Program:} Develop and schedule a mandatory annual security awareness training program for all employees. This program should cover, at a minimum, phishing identification, password security, and the organization's acceptable use policy.
\end{itemize}

\subsection*{Ongoing Recommendations}
\begin{itemize}
    \item \textbf{Implement Regular Vulnerability Scanning:} Establish a recurring schedule for both external and internal vulnerability scans to proactively identify and remediate new security weaknesses before they can be exploited.
    \item \textbf{Conduct Access Control Review:} Perform a comprehensive review of user access rights across all systems to ensure the principle of least privilege is being followed.
\end{itemize}

% --- DOCUMENT END ---
\end{document}
```