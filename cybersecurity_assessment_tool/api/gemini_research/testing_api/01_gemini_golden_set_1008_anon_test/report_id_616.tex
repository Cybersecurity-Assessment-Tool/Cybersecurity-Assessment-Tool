```latex
\documentclass[12pt]{article}

% Preamble: Required Packages
\usepackage[margin=1in]{geometry}
\usepackage{pifont} % For checkmarks and crosses
\usepackage{booktabs} % For professional tables
\usepackage{hyperref} % For hyperlinks
\usepackage{url} % For URL formatting
\usepackage{seqsplit} % For splitting long strings without breaking
\usepackage{graphicx}
\usepackage{xcolor}

% Document Metadata
\title{Cybersecurity Posture Assessment Report}
\author{Cybersecurity Analysis Division}
\date{\today}

% Hyperref Setup
\hypersetup{
    colorlinks=true,
    linkcolor=blue,
    filecolor=magenta,      
    urlcolor=cyan,
    pdftitle={Cybersecurity Posture Assessment Report},
    pdfpagemode=FullScreen,
}

\begin{document}

\maketitle
\thispagestyle{empty}
\newpage

\tableofcontents
\newpage

% --- 1. Executive Summary ---
\section{Executive Summary}

This report details the findings of a cybersecurity assessment for \textbf{[Organization Name]}. The evaluation combined an external network scan, a review of organizational security controls via a questionnaire, and an analysis of pre-existing risks.

The assessment revealed a mixed security posture. On one hand, the external network perimeter appears robust. The network scan of the target IP address, \texttt{[Target IP]}, found no open ports, indicating a well-configured firewall that effectively limits external exposure. This is a significant strength.

On the other hand, critical gaps were identified in internal security controls. The two most significant findings are:
\begin{itemize}
    \item \textbf{Lack of Multi-Factor Authentication (MFA) for computer logins:} This is a critical vulnerability that significantly increases the risk of unauthorized access and lateral movement within the network should an employee's credentials be compromised.
    \item \textbf{Absence of security awareness training for new employees:} New hires are a primary target for social engineering and phishing attacks. Failing to provide immediate training leaves the organization vulnerable to initial-access threats.
\end{itemize}

While no pre-existing vulnerabilities were reported and the external defenses are strong, the identified internal control weaknesses present a clear and present danger. Recommendations in this report focus on rectifying these procedural and policy-based gaps to build a more resilient, defense-in-depth security strategy.

% --- 2. Organizational Information ---
\section{Organizational Information}

The following details were used as the basis for this assessment. Due to the anonymized nature of the provided data, placeholders have been used where necessary.

\begin{itemize}
    \item \textbf{Organization Name:} \textbf{[Organization Name]}
    \item \textbf{Primary Domain:} \texttt{[Domain]}
    \item \textbf{External IP Scanned:} \texttt{[Client IP]}
\end{itemize}

% --- 3. Security Control Review (Questionnaire Analysis) ---
\section{Security Control Review (Questionnaire Analysis)}

An analysis of the security questionnaire reveals the organization's current adherence to essential security controls. The following table summarizes the responses and provides a high-level assessment of each. Answers marked with \ding{55} represent significant gaps in the security framework.

\begin{table}[h!]
\centering
\caption{Security Controls Questionnaire Results}
\begin{tabular}{p{0.6\textwidth} c p{0.2\textwidth}}
\toprule
\textbf{Control Question} & \textbf{Response} & \textbf{Assessment} \\
\midrule
Do you require MFA to access email? & \ding{51} & Compliant \\
\addlinespace
Do you require MFA to log into computers? & \textcolor{red}{\ding{55}} & \textbf{Critical Gap} \\
\addlinespace
Do you require MFA to access sensitive data systems? & \ding{51} & Compliant \\
\addlinespace
Does your organization have an employee acceptable use policy? & \ding{51} & Compliant \\
\addlinespace
Does your organization do security awareness training for new employees? & \textcolor{red}{\ding{55}} & \textbf{High Risk} \\
\addlinespace
Does your organization do security awareness training for all employees at least once per year? & \ding{51} & Compliant \\
\bottomrule
\end{tabular}
\end{table}

% --- 4. Technical Scan Results ---
\section{Technical Scan Results}

An external network scan was performed to identify exposed services and potential vulnerabilities on the public-facing infrastructure.

\begin{itemize}
    \item \textbf{Target IP Address:} \texttt{[Target IP]}
    \item \textbf{Scan Date:} Not provided in scan metadata.
\end{itemize}

\subsection{Key Findings}
The scan results were positive, indicating a strong network perimeter defense.
\begin{itemize}
    \item \textbf{Open Ports:} No open TCP or UDP ports were discovered on the target system.
    \item \textbf{Port State:} All scanned ports were reported as `closed`. This means the firewall is actively rejecting connection attempts, which is a more secure configuration than `filtered` (where packets are dropped silently).
\end{itemize}

\subsection{Analysis}
The absence of open ports significantly reduces the external attack surface of the organization. This finding suggests that a well-configured firewall is in place, properly implementing a policy of "default deny." This is an excellent security practice and a major mitigating factor against external threats.

% --- 5. Consolidated Risk Assessment ---
\section{Consolidated Risk Assessment}

This section synthesizes findings from the security control review, technical scan, and pre-existing risk data. The primary risks identified are procedural and policy-related, originating from the questionnaire responses. No pre-existing vulnerabilities were reported in the input data.

\begin{table}[h!]
\centering
\caption{Identified Risks and Severity}
\begin{tabular}{p{0.1\textwidth} p{0.25\textwidth} p{0.4\textwidth} p{0.1\textwidth}}
\toprule
\textbf{Risk ID} & \textbf{Risk Name} & \textbf{Overview} & \textbf{Severity} \\
\midrule
RISK-001 & Lack of Workstation MFA & The absence of MFA on computer logins allows an attacker with stolen credentials to gain direct access to an endpoint, facilitating data theft and lateral movement. & \textbf{Critical} \\
\addlinespace
RISK-002 & No Security Training for New Hires & New employees are not trained on security best practices, making them highly susceptible to phishing and social engineering attacks, which are common initial access vectors. & \textbf{High} \\
\bottomrule
\end{tabular}
\end{table}

% --- 6. Recommendations ---
\section{Recommendations}

Based on the consolidated risk assessment, the following actions are recommended to mitigate the identified vulnerabilities and improve the overall security posture of \textbf{[Organization Name]}.

\subsection{RISK-001: Implement Workstation MFA (Critical)}
\begin{itemize}
    \item \textbf{Action:} Deploy a Multi-Factor Authentication solution for all employee computer logins (both laptops and desktops). This should apply to both on-premise and remote access.
    \item \textbf{Justification:} This is the single most effective control to prevent unauthorized access resulting from compromised credentials. Even if an attacker steals a user's password, they will be unable to log in without the second factor (e.g., a push notification, authenticator code, or hardware key).
\end{itemize}

\subsection{RISK-002: Mandate Onboarding Security Training (High)}
\begin{itemize}
    \item \textbf{Action:} Integrate a mandatory security awareness training module into the new employee onboarding process. This training should be completed before the employee is granted full access to corporate systems.
    \item \textbf{Justification:} Educating employees from day one about threats like phishing, malware, and proper data handling creates a security-conscious culture. It significantly reduces the likelihood of a successful social engineering attack and strengthens the human element of the organization's defenses.
\end{itemize}

\end{document}
```