```latex
\documentclass[12pt]{article}

% --- PACKAGES ---
\usepackage[margin=1in]{geometry}
\usepackage{pifont} % For checkmarks and crosses
\usepackage{booktabs} % For professional tables
\usepackage{hyperref} % For hyperlinks
\usepackage{url} % For URL formatting
\usepackage{seqsplit} % For splitting long strings
\usepackage{graphicx}
\usepackage{xcolor}

% --- HYPERREF SETUP ---
\hypersetup{
    colorlinks=true,
    linkcolor=blue,
    filecolor=magenta,      
    urlcolor=cyan,
    pdftitle={Cybersecurity Assessment Report},
    pdfpagemode=FullScreen,
}

% --- DOCUMENT START ---
\begin{document}

% --- TITLE PAGE ---
\begin{titlepage}
    \centering
    \vspace*{1cm}
    \Huge\textbf{Cybersecurity Assessment Report}
    \vspace{1.5cm}
    \Large
    Prepared for: \\
    \vspace{0.5cm}
    \textbf{[Organization Name]}
    \vspace{2cm}
    \large
    \textbf{Date of Report:} \today \\
    \textbf{Assessment Date:} November 22, 2025
    \vfill
    \large
    \textbf{Generated by:} Expert Cybersecurity Analyst
\end{titlepage}

\tableofcontents
\newpage

% --- EXECUTIVE SUMMARY ---
\section*{Executive Summary}

This report details the findings of a cybersecurity assessment conducted for \textbf{[Organization Name]} on November 22, 2025. The assessment combined a review of organizational security controls via a questionnaire, an external network scan of a key asset, and an analysis of pre-existing risks.

The assessment identified several critical and high-impact risks that require immediate attention. Key findings include:
\begin{itemize}
    \item \textbf{Critical Control Gap:} Multi-Factor Authentication (MFA) is not enforced for email access. This exposes the organization to a significant risk of business email compromise (BEC), data breaches, and phishing attacks.
    \item \textbf{High-Risk Technical Vulnerability:} The public-facing web server at \texttt{[Target IP]} is running an outdated version of nginx (1.18.0). This version is no longer supported and has multiple publicly known vulnerabilities, making it a prime target for exploitation.
    \item \textbf{High-Risk Policy Gap:} The organization lacks a formal Acceptable Use Policy (AUP) for employees. This creates ambiguity regarding the secure use of company assets and increases the potential for insider threats and unintentional security incidents.
\end{itemize}

While the organization has implemented some positive security controls, such as MFA for computer and sensitive system access, the identified gaps substantially elevate the overall risk profile. This report provides specific, actionable recommendations to mitigate these risks and improve the organization's security posture.

% --- ORGANIZATIONAL INFORMATION ---
\section*{Organizational Information}
This section provides the key identification details for the organization under review.
\begin{itemize}
    \item \textbf{Organization Name:} \textbf{[Organization Name]}
    \item \textbf{Primary Email Domain:} \texttt{[Domain]}
    \item \textbf{External IP Address Scanned:} \texttt{[Client IP]}
\end{itemize}

% --- SECURITY CONTROL REVIEW ---
\section*{Security Control Review (Questionnaire Analysis)}
The following table summarizes the organization's responses to a security controls questionnaire. Answers marked with \ding{55} (No) represent significant gaps in the security framework and are discussed in the Risk Assessment section.

\begin{table}[h!]
\centering
\caption{Security Controls Questionnaire Results}
\label{tab:controls}
\begin{tabular}{p{0.7\textwidth} c}
\toprule
\textbf{Control Question} & \textbf{Response} \\
\midrule
Do you require MFA to access email? & \textcolor{red}{\ding{55}} \\
Do you require MFA to log into computers? & \textcolor{green}{\ding{51}} \\
Do you require MFA to access sensitive data systems? & \textcolor{green}{\ding{51}} \\
Does your organization have an employee acceptable use policy? & \textcolor{red}{\ding{55}} \\
Does your organization do security awareness training for new employees? & \textcolor{green}{\ding{51}} \\
Does your organization do security awareness training for all employees at least once per year? & \textcolor{green}{\ding{51}} \\
\bottomrule
\end{tabular}
\end{table}

\subsection*{Analysis of Gaps}
\begin{itemize}
    \item \textbf{MFA for Email:} The lack of MFA on email is a critical weakness. Email accounts are high-value targets for attackers seeking to conduct phishing campaigns, gain access to sensitive data, or pivot to other internal systems.
    \item \textbf{Acceptable Use Policy (AUP):} The absence of an AUP means there are no formally documented rules for employees regarding the use of company technology and data. This can lead to inconsistent security practices and makes it difficult to enforce security standards.
\end{itemize}

% --- TECHNICAL SCAN RESULTS ---
\section*{Technical Scan Results}
An external network scan was performed against the target IP address \texttt{[Target IP]} on November 22, 2025. The scan identified the following open ports and services.

\begin{table}[h!]
\centering
\caption{Open Ports and Services on \texttt{[Target IP]}}
\label{tab:nmap}
\begin{tabular}{l l l l}
\toprule
\textbf{Port} & \textbf{State} & \textbf{Service} & \textbf{Product / Version} \\
\midrule
443/tcp & Open & https & nginx 1.18.0 \\
\bottomrule
\end{tabular}
\end{table}

\subsection*{Analysis of Technical Findings}
The scan revealed that a web server running \textbf{nginx version 1.18.0} is exposed to the internet. This version was released in April 2020 and reached its End of Life (EOL) in May 2022. It is no longer receiving security updates and is known to be vulnerable to several Common Vulnerabilities and Exposures (CVEs). Exposing outdated software to the internet presents a high risk of compromise, as automated attack tools constantly scan for and exploit such vulnerabilities.

% --- RISK ASSESSMENT ---
\section*{Risk Assessment Summary}
This section correlates the findings from the security control review and the technical scan. No pre-existing vulnerabilities were reported. The following new risks have been identified and prioritized based on their potential impact.

\begin{table}[h!]
\centering
\caption{Identified Risks and Severity}
\label{tab:risks}
\begin{tabular}{p{0.1\textwidth} p{0.25\textwidth} p{0.4\textwidth} p{0.1\textwidth}}
\toprule
\textbf{Risk ID} & \textbf{Risk Name} & \textbf{Description} & \textbf{Severity} \\
\midrule
RISK-001 & Lack of MFA on Email & The absence of a second authentication factor for email allows an attacker with stolen credentials to gain full access to an employee's mailbox, leading to data theft and further attacks. & \textbf{Critical} \\
\addlinespace
RISK-002 & Outdated Web Server Software & The public-facing nginx 1.18.0 server is vulnerable to known exploits, which could lead to server compromise, website defacement, or a breach of data hosted on the server. & \textbf{High} \\
\addlinespace
RISK-003 & Missing Acceptable Use Policy & Without a formal AUP, there is an increased risk of misuse of company assets and data by employees, whether intentional or unintentional. This is a foundational governance gap. & \textbf{High} \\
\bottomrule
\end{tabular}
\end{table}

% --- RECOMMENDATIONS ---
\section*{Recommendations}
The following actions are recommended to mitigate the identified risks and strengthen the overall security posture of \textbf{[Organization Name]}.

\subsection*{RISK-001: Lack of MFA on Email (Critical)}
\begin{itemize}
    \item \textbf{Immediate Action:} Implement and enforce Multi-Factor Authentication (MFA) across all email accounts associated with the \texttt{[Domain]} domain.
    \item \textbf{Prioritization:} Begin with administrator, executive, and finance-related accounts, followed by a mandatory rollout for all remaining users.
    \item \textbf{Justification:} This is the single most effective control to prevent unauthorized access to email accounts and mitigate the risk of Business Email Compromise (BEC).
\end{itemize}

\subsection*{RISK-002: Outdated Web Server Software (High)}
\begin{itemize}
    \item \textbf{Immediate Action:} Develop a patch management plan to upgrade the nginx server at \texttt{[Target IP]} from version 1.18.0 to the latest stable version.
    \item \textbf{Procedure:} Before deploying to production, the upgrade should be tested in a staging environment to ensure full compatibility with the web application it hosts.
    \item \textbf{Justification:} Patching this vulnerability removes a direct and easily exploitable attack vector from the organization's external perimeter.
\end{itemize}

\subsection*{RISK-003: Missing Acceptable Use Policy (High)}
\begin{itemize}
    \item \textbf{Immediate Action:} Draft, approve, and implement a formal Acceptable Use Policy (AUP).
    \item \textbf{Content:} The policy should clearly define the rules for using company networks, devices, email, and internet access. It should also outline consequences for non-compliance.
    \item \textbf{Implementation:} The AUP should be distributed to all current employees and integrated into the onboarding process for new hires. Require all employees to sign an acknowledgment form.
    \item \textbf{Justification:} An AUP establishes a baseline for secure behavior, reduces insider risk, and provides a legal and administrative framework for enforcing security standards.
\end{itemize}

\end{document}
```