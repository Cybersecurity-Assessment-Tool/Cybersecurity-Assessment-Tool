```latex
\documentclass[12pt]{article}

% Preamble: Required Packages
\usepackage[a4paper, margin=1in]{geometry}
\usepackage{pifont} % For checkmarks and crosses
\usepackage{booktabs} % For professional tables
\usepackage{hyperref} % For clickable links
\usepackage{url} % For URL formatting
\usepackage{seqsplit} % To split long strings in tt font
\usepackage{graphicx}
\usepackage{xcolor}

% Document Metadata
\title{Cybersecurity Assessment Report}
\author{Cybersecurity Analysis Division}
\date{\today}

% Hyperref Setup
\hypersetup{
    colorlinks=true,
    linkcolor=blue,
    filecolor=magenta,      
    urlcolor=cyan,
    pdftitle={Cybersecurity Assessment Report},
    pdfpagemode=FullScreen,
}

% Define a command for the checkmark and cross for consistency
\newcommand{\yes}{\ding{51}}
\newcommand{\no}{\ding{55}}

\begin{document}

\maketitle
\thispagestyle{empty}
\newpage

\tableofcontents
\newpage

% --- 1. Executive Summary ---
\section{Executive Summary}

This report details the findings of a cybersecurity assessment conducted for \textbf{[Organization Name]}. The assessment combined a review of organizational security controls, an external network scan, and an analysis of pre-existing risk data.

The overall security posture is assessed as \textbf{High-Risk}. Several critical vulnerabilities and security gaps were identified that require immediate attention.

Key findings include:
\begin{itemize}
    \item \textbf{Critical Database Exposure:} An outdated and unsupported version of MySQL (5.7.33) is publicly exposed on the internet. This version is past its end-of-life and contains known vulnerabilities, presenting a direct and severe risk of data breach.
    \item \textbf{Insufficient Multi-Factor Authentication (MFA):} MFA is not enforced for accessing email or for logging into company computers. This significantly increases the risk of unauthorized access through compromised credentials.
    \item \textbf{Inadequate Security Training:} While new employees receive security training, there is no mandatory annual training program for all staff. This leads to a gradual decline in security awareness and makes the organization more susceptible to social engineering attacks like phishing.
\end{itemize}

Immediate remediation of these issues is strongly recommended to reduce the organization's attack surface and protect sensitive data. Detailed recommendations are provided in Section \ref{sec:recommendations}.

% --- 2. Organizational Information ---
\section{Organizational Information}

The following information was used as the basis for this assessment. Due to the anonymized nature of the provided data, placeholders have been used where necessary.

\begin{table}[h!]
\centering
\begin{tabular}{@{}ll@{}}
\toprule
\textbf{Attribute} & \textbf{Value} \\ \midrule
Organization Name & \textbf{[Organization Name]} \\
Primary Domain & \texttt{[Domain]} \\
External IP Address Scanned & \texttt{[Client IP]} \\ \bottomrule
\end{tabular}
\caption{Client Organizational Details.}
\end{table}

% --- 3. Security Control Review ---
\section{Security Control Review (Questionnaire Analysis)}

A review of the organization's security controls was conducted via a questionnaire. The responses indicate significant gaps in fundamental security practices, particularly concerning identity and access management and ongoing employee education.

\begin{table}[h!]
\centering
\begin{tabular}{@{}p{0.7\linewidth}cc@{}}
\toprule
\textbf{Control Question} & \textbf{Response} & \textbf{Status} \\ \midrule
Do you require MFA to access email? & No & \no \\
Do you require MFA to log into computers? & No & \no \\
Do you require MFA to access sensitive data systems? & Yes & \yes \\
Does your organization have an employee acceptable use policy? & Yes & \yes \\
Does your organization do security awareness training for new employees? & Yes & \yes \\
Does your organization do security awareness training for all employees at least once per year? & No & \no \\ \bottomrule
\end{tabular}
\caption{Security Controls Questionnaire Results.}
\end{table}

\paragraph{Analysis:} The lack of MFA for email and computer logins (\textbf{No} responses) represents a critical weakness. Email is a primary target for account takeover, and unprotected endpoints provide an entry point for attackers to move laterally within the network. The absence of annual security awareness training for all employees is a high-risk gap, as the threat landscape constantly evolves and employee vigilance is a key defense against phishing and other social engineering attacks.

% --- 4. Technical Scan Results ---
\section{Technical Scan Results}

An external network scan was performed on the target IP address to identify open ports and exposed services.

\begin{itemize}
    \item \textbf{Target IP Address:} \texttt{[Target IP]}
\end{itemize}

\begin{table}[h!]
\centering
\begin{tabular}{@{}llll@{}}
\toprule
\textbf{Port} & \textbf{State} & \textbf{Service} & \textbf{Product \& Version} \\ \midrule
3306/tcp & open & mysql & \seqsplit{\texttt{MySQL 5.7.33}} \\ \bottomrule
\end{tabular}
\caption{Open Ports Detected on Target Host.}
\end{table}

\paragraph{Analysis:} The scan confirms that port 3306 is open, exposing a MySQL database server directly to the internet. The detected version, \textbf{MySQL 5.7.33}, is particularly concerning for two reasons:
\begin{enumerate}
    \item \textbf{End-of-Life (EOL) Software:} MySQL version 5.7 reached its official end of life in October 2023. This means it no longer receives security patches from the vendor, and any newly discovered vulnerabilities will remain unpatched.
    \item \textbf{Known Vulnerabilities:} This specific version and older versions in the 5.7 branch are associated with multiple publicly disclosed vulnerabilities (CVEs) that could be exploited by attackers to gain unauthorized access, execute arbitrary code, or cause a denial of service.
\end{enumerate}
This finding directly corroborates the pre-existing "Database Exposure" risk and elevates its severity to \textbf{Critical}.

% --- 5. Consolidated Risk Assessment ---
\section{Consolidated Risk Assessment}

The following table synthesizes findings from the security questionnaire, technical scan, and pre-existing risk data into a prioritized list of security risks.

\begin{table}[h!]
\centering
\resizebox{\textwidth}{!}{%
\begin{tabular}{@{}llll@{}}
\toprule
\textbf{Risk Title} & \textbf{Severity} & \textbf{Description} & \textbf{Affected Systems} \\ \midrule
\begin{tabular}[c]{@{}l@{}}Publicly Exposed \& \\ Outdated Database\end{tabular} & \textbf{Critical} & \begin{tabular}[c]{@{}l@{}}An unsupported, EOL version of MySQL is \\ directly accessible from the internet, posing a \\ severe risk of data compromise.\end{tabular} & \begin{tabular}[c]{@{}l@{}}Database Server at \\ \texttt{[Target IP]}:3306\end{tabular} \\ \\
\begin{tabular}[c]{@{}l@{}}Insufficient Multi-Factor \\ Authentication (MFA)\end{tabular} & \textbf{Critical} & \begin{tabular}[c]{@{}l@{}}Lack of MFA on email and computer logins \\ makes user accounts highly vulnerable to \\ credential stuffing and phishing attacks.\end{tabular} & \begin{tabular}[c]{@{}l@{}}All user accounts, \\ email system, endpoints\end{tabular} \\ \\
\begin{tabular}[c]{@{}l@{}}Inadequate Security \\ Awareness Training\end{tabular} & \textbf{High} & \begin{tabular}[c]{@{}l@{}}Without mandatory annual training, employees \\ are less prepared to identify and respond to \\ modern cyber threats like phishing.\end{tabular} & All employees \\ \bottomrule
\end{tabular}%
}
\caption{Summary of Identified Security Risks.}
\end{table}

% --- 6. Recommendations ---
\section{Recommendations}
\label{sec:recommendations}

The following actions are recommended to mitigate the identified risks. They are prioritized based on severity and potential impact.

\subsection{Immediate Actions (To Be Completed within 72 Hours)}
\begin{enumerate}
    \item \textbf{Restrict Database Access:} Implement strict firewall rules to block all public access to TCP port 3306 on \texttt{[Target IP]}. Access should only be permitted from trusted internal IP addresses or via a secure VPN connection.
\end{enumerate}

\subsection{High-Priority Actions (To Be Completed within 30 Days)}
\begin{enumerate}
    \item \textbf{Enforce MFA Everywhere:}
        \begin{itemize}
            \item Immediately enable and enforce MFA for all user accounts on the primary email system.
            \item Develop and execute a plan to roll out MFA for all computer and remote access logins.
        \end{itemize}
    \item \textbf{Plan Database Upgrade:} Initiate a project to upgrade the MySQL 5.7.33 database to a currently supported version (e.g., MySQL 8.x). This is critical for receiving future security patches.
\end{enumerate}

\subsection{Strategic Actions (To Be Completed within 90 Days)}
\begin{enumerate}
    \item \textbf{Implement Annual Security Training:} Procure and implement a mandatory security awareness training program for all employees, to be completed annually. This program should cover topics such as phishing, password security, and acceptable use policies.
    \item \textbf{Establish Secure Remote Access:} If not already in place, deploy a Virtual Private Network (VPN) solution for all remote administrative access to internal systems, including the database server. This provides a secure, encrypted tunnel and avoids direct service exposure.
\end{enumerate}

\end{document}
```