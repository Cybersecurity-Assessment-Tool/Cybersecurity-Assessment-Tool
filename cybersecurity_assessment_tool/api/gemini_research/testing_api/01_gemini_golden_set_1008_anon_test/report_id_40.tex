```latex
\documentclass[12pt, a4paper]{article}

% Preamble: Required Packages
\usepackage[margin=1in]{geometry}
\usepackage{pifont} % For checkmarks and crosses
\usepackage{booktabs} % For professional tables
\usepackage[hidelinks]{hyperref} % For clickable links
\usepackage{url} % For URL formatting
\usepackage{seqsplit} % To split long strings without breaking
\usepackage{graphicx} % For potential logos
\usepackage{xcolor} % For colors

% Define colors for severity
\definecolor{criticalred}{HTML}{D10000}
\definecolor{highorange}{HTML}{E97400}
\definecolor{mediumyellow}{HTML}{F5C300}

% Document Information
\title{Cybersecurity Posture Assessment Report}
\author{Cybersecurity Analyst}
\date{\today}

\begin{document}

\maketitle
\thispagestyle{empty}
\newpage

\tableofcontents
\newpage

% --- 1. Executive Summary ---
\section{Executive Summary}

This report provides a cybersecurity assessment for \textbf{[Organization Name]}, conducted on \today. The analysis is based on a combination of an external network scan, a review of existing risks, and an organizational security controls questionnaire.

The assessment revealed a mixed security posture. On the positive side, the external network scan of the target IP address \texttt{[Client IP]} showed a strong perimeter defense, with no open ports detected. This indicates a well-configured firewall, significantly reducing the external attack surface.

However, significant internal security gaps were identified through the organizational questionnaire. The most critical findings are the lack of Multi-Factor Authentication (MFA) for accessing sensitive data systems and the absence of a formal security awareness training program for employees. These policy and procedure-based weaknesses expose the organization to substantial risks, particularly from phishing, credential theft, and insider threats.

Immediate remediation is required to address these critical gaps to prevent potential data breaches and mitigate financial and reputational damage. This report details the findings and provides actionable recommendations to enhance the organization's overall security posture.

% --- 2. Organizational Information ---
\section{Organizational Information}

The following information was used as the basis for this assessment. Due to the anonymized nature of the provided data, placeholders have been used where necessary.

\begin{table}[h!]
\centering
\begin{tabular}{@{}ll@{}}
\toprule
\textbf{Attribute} & \textbf{Value} \\ \midrule
Organization Name & \textbf{[Organization Name]} \\
Primary Domain & \texttt{[Domain]} \\
External IP Scanned & \texttt{[Client IP]} \\ \bottomrule
\end{tabular}
\caption{Client Organizational Details}
\label{tab:org_info}
\end{table}

% --- 3. Security Control Review ---
\section{Security Control Review}

A review of internal security controls was conducted via a questionnaire. The responses highlight key areas of strength and weakness in the organization's policies and procedures. "No" answers indicate significant gaps that increase organizational risk.

\begin{table}[h!]
\centering
\begin{tabular}{@{}p{0.75\linewidth}c@{}}
\toprule
\textbf{Control Question} & \textbf{Response} \\ \midrule
Do you require MFA to access email? & \ding{51} \\
Do you require MFA to log into computers? & \ding{51} \\
Does your organization have an employee acceptable use policy? & \ding{51} \\
\midrule
\textcolor{criticalred}{Do you require MFA to access sensitive data systems?} & \textcolor{criticalred}{\ding{55}} \\
\textcolor{highorange}{Does your organization do security awareness training for new employees?} & \textcolor{highorange}{\ding{55}} \\
\textcolor{highorange}{Does your organization do security awareness training for all employees at least once per year?} & \textcolor{highorange}{\ding{55}} \\ \bottomrule
\end{tabular}
\caption{Organizational Security Controls Questionnaire Results (\ding{51}=Yes, \ding{55}=No)}
\label{tab:controls}
\end{table}

% --- 4. Technical Scan Results ---
\section{Technical Scan Results}

An external network vulnerability scan was performed using Nmap to identify open ports and exposed services on the public-facing IP address.

\begin{itemize}
    \item \textbf{Target IP:} \texttt{[Target IP]}
    \item \textbf{Scan Date:} \today
    \item \textbf{Summary:} The scan concluded successfully. No open ports were detected on the target host. All 1000 scanned ports were reported as being in a \texttt{closed} state.
\end{itemize}

\subsection{Findings}
\textbf{Positive Finding:} The absence of open ports indicates a strong firewall configuration and adherence to the principle of least privilege for network access. This significantly minimizes the external attack surface available to malicious actors. No vulnerabilities were identified from this external perspective.

% --- 5. Risk Assessment ---
\section{Risk Assessment}

This section synthesizes the findings from the security control review, technical scan, and pre-existing risk data. The primary risks identified are procedural and policy-related. No pre-existing vulnerabilities were reported in the input data.

\begin{table}[h!]
\centering
\begin{tabular}{@{}p{0.1\linewidth}p{0.3\linewidth}p{0.15\linewidth}p{0.35\linewidth}@{}}
\toprule
\textbf{Risk ID} & \textbf{Finding} & \textbf{Severity} & \textbf{Description} \\ \midrule
RISK-001 & No MFA on Sensitive Data Systems & \textcolor{criticalred}{\textbf{Critical}} & The absence of MFA on systems holding sensitive data means that a single compromised password could lead to a major data breach. Attackers can leverage stolen credentials from phishing attacks or other breaches to gain direct access. \\
\addlinespace
RISK-002 & Lack of Security Awareness Training & \textcolor{highorange}{\textbf{High}} & Without training at onboarding or annually, employees are significantly more likely to fall victim to phishing, social engineering, and malware attacks. This turns the workforce into an unintentional primary attack vector. \\
\bottomrule
\end{tabular}
\caption{Summary of Identified Risks}
\label{tab:risks}
\end{table}

% --- 6. Recommendations ---
\section{Recommendations}

Based on the risk assessment, the following actions are recommended to strengthen the organization's security posture. Recommendations are prioritized by severity.

\subsection{Critical Priority}
\begin{enumerate}
    \item \textbf{Implement MFA for Sensitive Systems (RISK-001):}
    \begin{itemize}
        \item \textbf{Action:} Immediately deploy a mandatory Multi-Factor Authentication (MFA) solution for all user accounts (including administrative and service accounts) that have access to systems containing sensitive, confidential, or proprietary data.
        \item \textbf{Impact:} This is the single most effective control to prevent unauthorized access resulting from compromised credentials.
    \end{itemize}
\end{enumerate}

\subsection{High Priority}
\begin{enumerate}
    \setcounter{enumi}{1} % Continue numbering
    \item \textbf{Establish a Security Awareness Training Program (RISK-002):}
    \begin{itemize}
        \item \textbf{Action:} Develop and implement a comprehensive security awareness training program. This program must be mandatory for all new employees during onboarding and required for all staff on an annual basis.
        \item \textbf{Content:} Training should cover key topics such as phishing identification, password security, acceptable use of company assets, and incident reporting procedures.
        \item \textbf{Impact:} A well-trained workforce serves as a human firewall, drastically reducing the success rate of common cyberattacks.
    \end{itemize}
\end{enumerate}

% --- 7. Conclusion ---
\section{Conclusion}

The assessment of \textbf{[Organization Name]} highlights a security posture with commendable external network defenses but critical internal vulnerabilities. The lack of MFA on sensitive systems and the absence of employee security training represent significant, unmitigated risks. While the technical perimeter is secure, the organization is highly exposed to attacks that target its employees.

We strongly urge the implementation of the recommendations outlined in this report, starting with the critical priority items. Addressing these gaps will build a more resilient, defense-in-depth security strategy and significantly reduce the likelihood of a successful cyberattack.

\end{document}
```