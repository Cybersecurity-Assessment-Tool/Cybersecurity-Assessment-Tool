```latex
\documentclass[12pt]{article}

% Preamble: Required Packages
\usepackage[margin=1in]{geometry} % For setting page margins
\usepackage{pifont}                % For checkmarks and crosses (dingbats)
\usepackage{booktabs}              % For professional-looking tables
\usepackage{hyperref}              % For hyperlinks and document metadata
\usepackage{url}                   % For formatting URLs
\usepackage{seqsplit}              % For splitting long strings without spaces
\usepackage{graphicx}              % For including logos (optional)
\usepackage{xcolor}                % For custom colors

% Document Metadata
\hypersetup{
    colorlinks=true,
    linkcolor=blue,
    filecolor=magenta,      
    urlcolor=cyan,
    pdftitle={Cybersecurity Assessment Report},
    pdfauthor={Cybersecurity Analyst},
    pdfsubject={Security Assessment},
    pdfkeywords={Cybersecurity, Nmap, Risk Assessment},
}

% Define custom colors for severity
\definecolor{critical}{HTML}{990000}
\definecolor{high}{HTML}{D14124}
\definecolor{medium}{HTML}{E8A317}
\definecolor{low}{HTML}{3A7D44}

% Title Information
\title{Cybersecurity Assessment Report \\ \large For \textbf{[Organization Name]}}
\author{Cybersecurity Analyst}
\date{November 22, 2025}

\begin{document}

\maketitle
\thispagestyle{empty}
\newpage

\tableofcontents
\thispagestyle{empty}
\newpage

\setcounter{page}{1}

% ==============================================================================
% SECTION 1: EXECUTIVE SUMMARY
% ==============================================================================
\section{Executive Summary}

This report details the findings of a cybersecurity assessment conducted on November 22, 2025. The assessment combined a technical network scan, a review of existing risks, and an analysis of organizational security controls based on a questionnaire.

The overall security posture of \textbf{[Organization Name]} requires immediate attention. Several critical and high-risk vulnerabilities were identified that expose the organization to significant threats, including data breaches, account compromise, and service disruption.

Key findings include:
\begin{itemize}
    \item \textbf{Critical Control Gaps:} The lack of mandatory Multi-Factor Authentication (MFA) for accessing email and sensitive data systems represents a critical vulnerability. These gaps severely increase the risk of unauthorized access and data exfiltration through compromised credentials.
    \item \textbf{Outdated Public-Facing Software:} The external web server is running Nginx version 1.18.0, which has been End-of-Life (EOL) since April 2022. This unsupported software is exposed to numerous publicly known vulnerabilities that could be exploited by attackers.
    \item \textbf{Policy and Training Deficiencies:} The absence of a formal Employee Acceptable Use Policy and a lack of security awareness training for new hires create a high-risk environment. Employees are a primary target for social engineering attacks, and these gaps leave the organization vulnerable to human error.
\end{itemize}

This report provides a detailed breakdown of these findings and offers actionable recommendations to mitigate the identified risks and strengthen the organization's overall security posture. We urge management to prioritize the remediation of the critical and high-risk items outlined herein.

% ==============================================================================
% SECTION 2: ORGANIZATIONAL INFORMATION
% ==============================================================================
\section{Organizational Information}

This section provides the high-level organizational details used as a baseline for this assessment.

\begin{table}[h!]
\centering
\begin{tabular}{@{}ll@{}}
\toprule
\textbf{Attribute} & \textbf{Value} \\ \midrule
Organization Name  & \textbf{[Organization Name]} \\
Primary Domain     & \texttt{[Domain]} \\
External IP Address & \texttt{[Client IP]} \\ \bottomrule
\end{tabular}
\caption{Client Organizational Details.}
\end{table}

% ==============================================================================
% SECTION 3: SECURITY CONTROL REVIEW
% ==============================================================================
\section{Security Control Review}

The following table summarizes the organization's security posture based on the provided questionnaire. Responses marked with a red cross (\textcolor{red}{\ding{55}}) indicate a deviation from security best practices and represent a significant gap in the control environment.

\begin{table}[h!]
\centering
\begin{tabular}{@{}p{0.5\textwidth}cp{0.3\textwidth}@{}}
\toprule
\textbf{Control Question} & \textbf{Response} & \textbf{Assessment} \\ \midrule
Do you require MFA to access email? & \textcolor{red}{\ding{55}} & \textbf{Critical Gap.} Increases risk of business email compromise. \\
\addlinespace
Do you require MFA to log into computers? & \textcolor{green}{\ding{51}} & Good Practice. Reduces risk of unauthorized local access. \\
\addlinespace
Do you require MFA to access sensitive data systems? & \textcolor{red}{\ding{55}} & \textbf{Critical Gap.} Exposes sensitive data to high risk of breach. \\
\addlinespace
Does your organization have an employee acceptable use policy? & \textcolor{red}{\ding{55}} & \textbf{High Risk.} Lack of clear guidelines for employees. \\
\addlinespace
Does your organization do security awareness training for new employees? & \textcolor{red}{\ding{55}} & \textbf{High Risk.} New hires are a prime target for attackers. \\
\addlinespace
Does your organization do security awareness training for all employees at least once per year? & \textcolor{green}{\ding{51}} & Good Practice. Maintains security awareness. \\ \bottomrule
\end{tabular}
\caption{Security Controls Questionnaire Analysis.}
\end{table}

% ==============================================================================
% SECTION 4: TECHNICAL SCAN RESULTS
% ==============================================================================
\section{Technical Scan Results}

An external network scan was performed to identify open ports and exposed services.

\subsection{Scan Details}
\begin{itemize}
    \item \textbf{Target IP Address:} \texttt{[Target IP]}
    \item \textbf{Scan Date:} November 22, 2025
\end{itemize}

\subsection{Open Ports and Services}
The following table details the services discovered on the target system.

\begin{table}[h!]
\centering
\begin{tabular}{@{}llllll@{}}
\toprule
\textbf{Port} & \textbf{State} & \textbf{Service} & \textbf{Product} & \textbf{Version} & \textbf{Finding} \\ \midrule
443/tcp & Open & https & nginx & 1.18.0 & \textbf{Outdated \& Unsupported.} \\ \bottomrule
\end{tabular}
\caption{Discovered Network Services.}
\end{table}

\paragraph{Analysis of Nginx 1.18.0:} The web server is running Nginx version 1.18.0. This version reached its official End-of-Life (EOL) in April 2022. Running unsupported software on a public-facing server is a significant security risk, as it no longer receives security patches for newly discovered vulnerabilities. This exposes the system to potential remote code execution, denial-of-service, or information disclosure attacks.

% ==============================================================================
% SECTION 5: CONSOLIDATED RISK ASSESSMENT
% ==============================================================================
\section{Consolidated Risk Assessment}

This section synthesizes findings from the security control review and technical scan into a consolidated list of identified risks. No pre-existing vulnerabilities were reported.

\begin{table}[h!]
\centering
\begin{tabular}{@{}lp{0.3\textwidth}p{0.4\textwidth}l@{}}
\toprule
\textbf{ID} & \textbf{Risk Name} & \textbf{Description} & \textbf{Severity} \\ \midrule
RISK-001 & Lack of MFA on Email & The absence of MFA on email accounts allows for account takeover with only a compromised password, enabling phishing and data theft. & \textcolor{critical}{\textbf{Critical}} \\
\addlinespace
RISK-002 & Lack of MFA on Sensitive Data Systems & Critical business data is accessible without a second factor of authentication, posing a direct threat to data confidentiality and integrity. & \textcolor{critical}{\textbf{Critical}} \\
\addlinespace
RISK-003 & Outdated Nginx Web Server & The public-facing web server is running an unsupported, End-of-Life version of Nginx, exposing it to numerous known vulnerabilities. & \textcolor{high}{\textbf{High}} \\
\addlinespace
RISK-004 & Missing Acceptable Use Policy (AUP) & Without a formal AUP, there are no enforceable rules governing the use of company IT assets, increasing the risk of insider threat and misuse. & \textcolor{high}{\textbf{High}} \\
\addlinespace
RISK-005 & Inadequate New Hire Security Training & New employees are not formally trained on security best practices, making them more susceptible to social engineering and phishing attacks. & \textcolor{high}{\textbf{High}} \\ \bottomrule
\end{tabular}
\caption{Summary of Identified Risks.}
\end{table}

% ==============================================================================
% SECTION 6: RECOMMENDATIONS
% ==============================================================================
\section{Recommendations}

The following actions are recommended to mitigate the identified risks and improve the overall security posture of \textbf{[Organization Name]}. Recommendations are prioritized based on risk severity.

\begin{enumerate}
    \item \textbf{[Critical] Implement MFA for Email and Sensitive Systems (RISK-001, RISK-002):} Immediately deploy and enforce a mandatory Multi-Factor Authentication solution for all user access to email services (e.g., Office 365, Google Workspace) and all systems identified as containing sensitive data.
    
    \item \textbf{[High] Upgrade Public-Facing Web Server (RISK-003):} Plan and execute the upgrade of the Nginx server at \texttt{[Target IP]} from version 1.18.0 to a current, stable, and fully supported version. A patch management policy should be implemented to prevent future occurrences of running EOL software.
    
    \item \textbf{[High] Develop and Implement an Acceptable Use Policy (RISK-004):} Create a formal AUP that clearly defines the rules and responsibilities for all employees when using company information systems. This policy should be distributed to all staff and acknowledged via signature.
    
    \item \textbf{[High] Integrate Security Training into Onboarding (RISK-005):} Develop a security awareness training module and make it a mandatory part of the onboarding process for all new employees. This training should cover key topics such as phishing, password hygiene, and the new AUP.
\end{enumerate}

\end{document}
```