```latex
\documentclass[12pt]{article}

% Preamble: Required Packages
\usepackage[margin=1in]{geometry}
\usepackage{pifont} % For checkmarks and crosses
\usepackage{booktabs} % For professional tables
\usepackage{hyperref} % For hyperlinks
\usepackage{url}      % For URL formatting
\usepackage{seqsplit} % For splitting long strings to prevent overflow

% Document Metadata
\title{Cybersecurity Posture Assessment Report}
\author{Cybersecurity Analysis Division}
\date{\today}

\begin{document}

\maketitle
\thispagestyle{empty}
\newpage
\tableofcontents
\newpage

% --- 1. Executive Summary ---
\section*{1. Executive Summary}

This report provides a cybersecurity posture assessment for \textbf{[Organization Name]}, conducted on \today. The analysis is based on a network scan, a review of organizational security controls, and a list of pre-existing risks.

The assessment reveals a mixed security posture. The organization demonstrates strong identity and access management controls, with consistent enforcement of Multi-Factor Authentication (MFA) across key systems. This significantly reduces the risk of account compromise.

However, critical gaps were identified in foundational security policies and employee training. The absence of an Acceptable Use Policy and the lack of security awareness training for new hires represent high-risk vulnerabilities. These administrative control failures expose the organization to significant human-factor risks, such as social engineering and insider threats.

From a technical perspective, the external network scan identified an exposed Secure Shell (SSH) service. While not an immediate vulnerability, its public accessibility increases the attack surface and warrants immediate review and mitigation.

Recommendations focus on closing these administrative gaps by developing and implementing core security policies and training programs, and on hardening the network perimeter by restricting access to unnecessary services.

% --- 2. Organizational Information ---
\section*{2. Organizational Information}

This section details the information provided about the organization.
\begin{itemize}
    \item \textbf{Organization Name:} \textbf{[Organization Name]}
    \item \textbf{Primary Email Domain:} \texttt{[Domain]}
    \item \textbf{External IP Address Scanned:} \texttt{[Client IP]}
\end{itemize}

% --- 3. Security Control Review ---
\section*{3. Security Control Review}

The following table summarizes the organization's responses to a security controls questionnaire. These answers provide insight into the current administrative and policy-based security measures.

\begin{table}[h!]
\centering
\caption{Organizational Security Controls Questionnaire}
\begin{tabular}{p{0.7\linewidth} c}
\toprule
\textbf{Control Question} & \textbf{Status} \\
\midrule
Do you require MFA to access email? & \ding{51} \\
Do you require MFA to log into computers? & \ding{51} \\
Do you require MFA to access sensitive data systems? & \ding{51} \\
Does your organization have an employee acceptable use policy? & \ding{55} \\
Does your organization do security awareness training for new employees? & \ding{55} \\
Does your organization do security awareness training for all employees at least once per year? & \ding{51} \\
\bottomrule
\end{tabular}
\end{table}

\subsection*{Analysis}
\textbf{Strengths:} The organization has effectively implemented Multi-Factor Authentication (MFA) across email, computer logins, and sensitive data systems. This is a commendable and critical control for preventing unauthorized access.

\textbf{Weaknesses:} Two significant gaps were identified:
\begin{enumerate}
    \item \textbf{No Acceptable Use Policy (AUP):} The absence of an AUP creates ambiguity regarding the proper use of company assets, data handling, and employee responsibilities. This is a foundational policy failure.
    \item \textbf{No Security Training for New Employees:} New hires are not receiving security awareness training upon joining the organization. This leaves a critical window of vulnerability where new staff are more susceptible to social engineering attacks before they are integrated into the annual training cycle.
\end{enumerate}

% --- 4. Technical Scan Results ---
\section*{4. Technical Scan Results}

An external network scan was performed to identify open ports and services visible on the public internet.

\begin{itemize}
    \item \textbf{Target IP Address:} \texttt{[Target IP]}
    \item \textbf{Scan Date:} [Scan Date]
\end{itemize}

The following table details the open ports discovered during the scan.

\begin{table}[h!]
\centering
\caption{Open Ports Detected on \texttt{[Target IP]}}
\begin{tabular}{l l l l}
\toprule
\textbf{Port} & \textbf{State} & \textbf{Service} & \textbf{Notes} \\
\midrule
22/tcp & open & ssh & Secure Shell (SSH) access is exposed to the public internet. \\
\bottomrule
\end{tabular}
\end{table}

\subsection*{Analysis}
The scan identified that port 22 (SSH) is open. Publicly exposing SSH increases the risk of brute-force password guessing attacks and potential exploitation if the SSH server software has any unpatched vulnerabilities. Access to administrative services like SSH should be restricted to trusted sources.

% --- 5. Consolidated Risk Assessment ---
\section*{5. Consolidated Risk Assessment}

This section correlates findings from the security control review, technical scan, and pre-existing risk data. No pre-existing vulnerabilities were reported.

\begin{table}[h!]
\centering
\caption{Summary of Identified Risks}
\begin{tabular}{p{0.3\linewidth} p{0.5\linewidth} l}
\toprule
\textbf{Risk Name} & \textbf{Overview} & \textbf{Severity} \\
\midrule
\textbf{Lack of Acceptable Use Policy} & The absence of a formal policy defining the acceptable use of corporate assets exposes the organization to insider threats, data misuse, and legal liabilities. & High \\
\addlinespace
\textbf{No Onboarding Security Training} & New employees are not trained on security best practices upon being hired, making them prime targets for phishing and other social engineering attacks. & High \\
\addlinespace
\textbf{Exposed SSH Service} & The SSH management port is open to the entire internet, increasing the attack surface and exposing the service to automated brute-force attacks and potential exploits. & Informational \\
\bottomrule
\end{tabular}
\end{table}

% --- 6. Recommendations ---
\section*{6. Recommendations}

Based on the findings of this assessment, we provide the following actionable recommendations to improve the organization's security posture.

\begin{enumerate}
    \item \textbf{[High] Develop and Implement an Acceptable Use Policy (AUP):}
    \begin{itemize}
        \item Draft a formal AUP that clearly outlines the rules and responsibilities for all employees regarding the use of company networks, systems, and data.
        \item Require all current and new employees to read and acknowledge the policy.
        \item This policy is a foundational component of a mature security program and is often required for regulatory compliance.
    \end{itemize}

    \item \textbf{[High] Institute Mandatory Security Training for New Hires:}
    \begin{itemize}
        \item Integrate a security awareness training module into the standard employee onboarding process.
        \item This training should cover key topics such as phishing identification, password hygiene, and the new Acceptable Use Policy.
        \item This ensures a baseline level of security awareness from day one of employment.
    \end{itemize}
    
    \item \textbf{[Informational] Restrict Access to SSH Service:}
    \begin{itemize}
        \item Review the business justification for exposing the SSH service on \texttt{[Target IP]} to the public internet.
        \item If remote access is required, implement a firewall rule to restrict access to a whitelist of trusted IP addresses (e.g., office or administrator IPs).
        \item If public access is not required, disable the service or block the port at the network firewall.
    \end{itemize}
\end{enumerate}

% --- 7. Conclusion ---
\section*{7. Conclusion}
\textbf{[Organization Name]} has established strong authentication controls but must address critical deficiencies in its administrative security framework. By implementing the recommendations outlined in this report, the organization can significantly reduce its risk exposure from both internal and external threats and build a more resilient security posture.

\end{document}
```