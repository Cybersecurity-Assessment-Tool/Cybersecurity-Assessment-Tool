```latex
\documentclass[12pt]{article}

% -----------------------------------------------------------------------------
% PREAMBLE
% -----------------------------------------------------------------------------
\usepackage[margin=1in]{geometry}
\usepackage{pifont} % For checkmarks and crosses
\usepackage{booktabs} % For professional tables
\usepackage{hyperref} % For clickable links
\usepackage{url}      % For URL formatting
\usepackage{seqsplit} % For splitting long strings
\usepackage{graphicx}
\usepackage{xcolor}

% --- Document Metadata ---
\title{Cybersecurity Posture Assessment Report}
\author{Cybersecurity Analysis Division}
\date{\today}

% --- Hyperref Setup ---
\hypersetup{
    colorlinks=true,
    linkcolor=blue,
    filecolor=magenta,      
    urlcolor=cyan,
    pdftitle={Cybersecurity Posture Assessment Report},
    pdfpagemode=FullScreen,
}

% -----------------------------------------------------------------------------
% DOCUMENT START
% -----------------------------------------------------------------------------
\begin{document}

\maketitle
\thispagestyle{empty}
\newpage

\tableofcontents
\newpage

% -----------------------------------------------------------------------------
% 1. EXECUTIVE SUMMARY
% -----------------------------------------------------------------------------
\section{Executive Summary}

This report provides a cybersecurity posture assessment for \textbf{[Organization Name]}, based on a combination of organizational questionnaires, network scanning, and a review of known risks. The analysis was conducted on \today.

The assessment identified several key areas of risk that require immediate attention. The most critical finding is the absence of Multi-Factor Authentication (MFA) for email access. This represents a significant security gap, as email is a primary vector for account compromise and subsequent attacks like Business Email Compromise (BEC).

Additionally, technical scanning revealed an exposed Secure Shell (SSH) service (port 22) on the external network perimeter. While necessary for remote administration, direct exposure to the public internet increases the risk of brute-force attacks and exploitation of potential vulnerabilities.

The organization demonstrates a solid foundation in policy and employee training. However, the identified technical and access control weaknesses must be remediated to elevate the overall security posture from its current state. This report details these findings and provides prioritized, actionable recommendations.

% -----------------------------------------------------------------------------
% 2. ORGANIZATIONAL INFORMATION
% -----------------------------------------------------------------------------
\section{Organizational Information}

The following details were used as the basis for this assessment. Due to the anonymized nature of the provided data, placeholders have been used where necessary.

\begin{table}[h!]
\centering
\begin{tabular}{@{}ll@{}}
\toprule
\textbf{Attribute} & \textbf{Value} \\ \midrule
Organization Name & \textbf{[Organization Name]} \\
Primary Email Domain & \texttt{[Domain]} \\
External IP Address Assessed & \texttt{[Client IP]} \\ \bottomrule
\end{tabular}
\caption{Client Organizational Details.}
\label{tab:org_info}
\end{table}

% -----------------------------------------------------------------------------
% 3. SECURITY CONTROL REVIEW (QUESTIONNAIRE)
% -----------------------------------------------------------------------------
\section{Security Control Review (Questionnaire)}

A review of the organization's security controls was conducted via a standardized questionnaire. The responses indicate a good baseline for policy and training, but highlight a critical gap in access control for a key system. A summary of the responses is provided in Table \ref{tab:controls}.

\begin{table}[h!]
\centering
\begin{tabular}{@{}lc@{}}
\toprule
\textbf{Control Question} & \textbf{Response} \\ \midrule
Do you require MFA to access email? & \ding{55} \\
Do you require MFA to log into computers? & \ding{51} \\
Do you require MFA to access sensitive data systems? & \ding{51} \\
Does your organization have an employee acceptable use policy? & \ding{51} \\
Does your organization do security awareness training for new employees? & \ding{51} \\
Does your organization do security awareness training for all employees at least once per year? & \ding{51} \\ \bottomrule
\end{tabular}
\caption{Security Control Questionnaire Responses (\ding{51}=Yes, \ding{55}=No).}
\label{tab:controls}
\end{table}

\subsection*{Analysis}
The single "No" response regarding MFA for email access is a major security concern. Email accounts are high-value targets for attackers. Without MFA, a compromised password (obtained via phishing, credential stuffing, or other means) is all an attacker needs to gain full access to an employee's mailbox. This can lead to data exfiltration, internal phishing campaigns, and financial fraud.

% -----------------------------------------------------------------------------
% 4. TECHNICAL SCAN RESULTS
% -----------------------------------------------------------------------------
\section{Technical Scan Results}

An external network scan was performed to identify open ports and exposed services on the organization's perimeter.

\begin{itemize}
    \item \textbf{Target IP Address:} \texttt{[Target IP]}
    \item \textbf{Scan Date:} Scan date not provided in source data.
\end{itemize}

The scan identified the following open port(s):

\begin{table}[h!]
\centering
\begin{tabular}{@{}llll@{}}
\toprule
\textbf{Port} & \textbf{State} & \textbf{Service} & \textbf{Product / Version} \\ \midrule
22/tcp & open & ssh & Not Identified \\ \bottomrule
\end{tabular}
\caption{Open Ports Identified on the External Perimeter.}
\label{tab:scan_results}
\end{table}

\subsection*{Analysis}
The presence of an open SSH port (22) indicates that a system is configured for remote administration directly from the internet. This configuration is a common target for automated brute-force login attempts by malicious actors. Furthermore, without detailed version information, it is not possible to determine if the running SSH service is vulnerable to known exploits. Best practice dictates that administrative services like SSH should not be directly exposed and should instead be protected behind a Virtual Private Network (VPN) or a bastion host, with access restricted to authorized IP addresses.

% -----------------------------------------------------------------------------
% 5. CONSOLIDATED RISK ASSESSMENT
% -----------------------------------------------------------------------------
\section{Consolidated Risk Assessment}

This section correlates the findings from the security control review and the technical scan. The pre-existing risk register reported no active vulnerabilities. The following new risks have been identified.

\begin{table}[h!]
\centering
\begin{tabular}{@{}p{0.1\linewidth}p{0.25\linewidth}p{0.4\linewidth}p{0.15\linewidth}@{}}
\toprule
\textbf{Risk ID} & \textbf{Risk Name} & \textbf{Description} & \textbf{Severity} \\ \midrule
\textbf{RISK-001} & Lack of MFA on Email & The absence of MFA on email accounts allows for account takeover with only a compromised password. This exposes the organization to phishing, data breaches, and financial fraud. & \textbf{Critical} \\
\addlinespace
\textbf{RISK-002} & Exposed SSH Management Port & The SSH service on port 22 is open to the public internet, making it a target for brute-force attacks and potential exploitation if unpatched. & \textbf{High} \\ \bottomrule
\end{tabular}
\caption{Summary of Identified Risks.}
\label{tab:risks}
\end{table}

% -----------------------------------------------------------------------------
% 6. RECOMMENDATIONS
% -----------------------------------------------------------------------------
\section{Recommendations}

Based on the consolidated risk assessment, the following prioritized actions are recommended to mitigate the identified vulnerabilities and improve the overall security posture of \textbf{[Organization Name]}.

\subsection*{Priority 1: Remediate Critical Risks}
\begin{enumerate}
    \item \textbf{Implement MFA for All Email Accounts (RISK-001):}
    \begin{itemize}
        \item \textbf{Action:} Enforce mandatory MFA for all user accounts accessing the email system (e.g., Microsoft 365, Google Workspace).
        \item \textbf{Justification:} This is the single most effective control to prevent email account takeovers. It mitigates the risk of password compromise and significantly reduces the attack surface for phishing and BEC attacks.
        \item \textbf{Timeline:} Immediate (within 7 days).
    \end{itemize}
\end{enumerate}

\subsection*{Priority 2: Remediate High Risks}
\begin{enumerate}
    \item \textbf{Secure Exposed SSH Service (RISK-002):}
    \begin{itemize}
        \item \textbf{Action:} Remove direct public access to the SSH service. The preferred method is to place the service behind a VPN, requiring users to connect to the VPN before accessing the SSH server.
        \item \textbf{Alternative Action:} If direct access is a business requirement, implement strict firewall rules to allow connections only from a limited set of trusted IP addresses (IP whitelisting).
        \item \textbf{Additional Hardening:} Regardless of the access method, ensure the SSH service is fully patched, configured to disallow password-based authentication (enforce key-based authentication), and uses modern, strong cryptographic ciphers.
        \item \textbf{Timeline:} High (within 30 days).
    \end{itemize}
\end{enumerate}

% -----------------------------------------------------------------------------
% DOCUMENT END
% -----------------------------------------------------------------------------
\end{document}
```