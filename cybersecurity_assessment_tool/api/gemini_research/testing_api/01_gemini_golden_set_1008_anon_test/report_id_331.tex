```latex
\documentclass[12pt]{article}

% Preamble: Required Packages
\usepackage[margin=1in]{geometry}
\usepackage{pifont} % For checkmarks and crosses
\usepackage{booktabs} % For professional tables
\usepackage{hyperref} % For hyperlinks and metadata
\usepackage{url}      % For properly formatting URLs
\usepackage{seqsplit} % For splitting long strings without spaces
\usepackage{xcolor}   % For colors
\usepackage{graphicx} % For logo (placeholder)

% Document Metadata
\hypersetup{
    colorlinks=true,
    linkcolor=blue,
    filecolor=magenta,      
    urlcolor=cyan,
    pdftitle={Cybersecurity Assessment Report},
    pdfauthor={Cybersecurity Analyst},
    pdfsubject={Security Posture Analysis},
    pdfkeywords={Cybersecurity, Risk Assessment, Nmap, Security Controls},
}

% Custom Commands
\newcommand{\yes}{\ding{51}}
\newcommand{\no}{\ding{55}}
\newcommand{\riskcritical}[1]{\textcolor{red}{\textbf{#1}}}
\newcommand{\riskhigh}[1]{\textcolor{orange}{\textbf{#1}}}
\newcommand{\riskmedium}[1]{\textcolor{yellow!80!black}{\textbf{#1}}}

\begin{document}

% --- Title Page ---
\begin{titlepage}
    \centering
    \vspace*{1cm}
    
    \Huge
    \textbf{Cybersecurity Assessment Report}
    
    \vspace{1.5cm}
    
    \Large
    Prepared for: \\
    \vspace{0.5cm}
    \textbf{[Organization Name]}
    
    \vspace{2cm}
    
    \Large
    \textbf{Date of Report:} \today
    
    \vfill
    
    \large
    \textbf{Author:} Cybersecurity Analyst \\
    \textbf{Classification:} Confidential
    
\end{titlepage}

\tableofcontents
\newpage

% --- Section 1: Executive Summary ---
\section{Executive Summary}
This report details the findings of a cybersecurity assessment conducted for \textbf{[Organization Name]}. The assessment combined a review of organizational security controls, an external network scan, and an analysis of known risks to evaluate the organization's overall security posture.

The analysis revealed \riskcritical{critical} deficiencies in fundamental security controls. The complete absence of Multi-Factor Authentication (MFA) across all key systems—including email, computer logins, and access to sensitive data—represents a severe and immediate risk of account compromise. This gap is further exacerbated by a lack of a formal security awareness training program, leaving the organization highly susceptible to social engineering and phishing attacks.

The external technical scan identified an exposed administrative service (SSH on port 22) on the network perimeter. While common, this service becomes a high-risk entry point for attackers when combined with weak identity and access management controls.

Immediate and decisive action is required to remediate these findings. The highest priority should be the rapid implementation of MFA, followed by the establishment of a comprehensive security awareness program and the hardening of all internet-facing services.

% --- Section 2: Organizational Information ---
\section{Organizational Information}
This section provides the key identification details for the organization under review. As this data was not provided, placeholders are used.

\begin{itemize}
    \item \textbf{Organization Name:} \textbf{[Organization Name]}
    \item \textbf{Primary Email Domain:} \texttt{[Domain]}
    \item \textbf{Primary External IP Address:} \texttt{[Client IP]}
\end{itemize}

% --- Section 3: Security Control Review ---
\section{Security Control Review}
The following table summarizes the organization's responses to a security controls questionnaire. "No" answers indicate significant gaps in the security framework and are flagged as risks.

\begin{table}[h!]
\centering
\caption{Security Controls Questionnaire Analysis}
\label{tab:controls}
\begin{tabular}{p{0.6\linewidth} c p{0.25\linewidth}}
\toprule
\textbf{Control Question} & \textbf{Response} & \textbf{Assessment} \\
\midrule
Do you require MFA to access email? & \no & \riskcritical{Critical Gap}. Email is a primary target for account takeover. \\
\addlinespace
Do you require MFA to log into computers? & \no & \riskcritical{Critical Gap}. Allows for lateral movement after initial compromise. \\
\addlinespace
Do you require MFA to access sensitive data systems? & \no & \riskcritical{Critical Gap}. Exposes critical data to unauthorized access. \\
\addlinespace
Does your organization have an employee acceptable use policy? & \yes & Foundational policy is in place. \\
\addlinespace
Does your organization do security awareness training for new employees? & \no & \riskhigh{High Risk}. New staff are unaware of security policies and threats. \\
\addlinespace
Does your organization do security awareness training for all employees at least once per year? & \no & \riskhigh{High Risk}. Lack of ongoing training increases vulnerability to phishing. \\
\bottomrule
\end{tabular}
\end{table}

% --- Section 4: Technical Scan Results ---
\section{Technical Scan Results}
An external network scan was performed to identify open ports and services exposed to the internet.

\begin{itemize}
    \item \textbf{Scan Target:} \texttt{[Target IP]}
    \item \textbf{Scan Date:} Not provided in scan data.
\end{itemize}

The following table details the open ports discovered on the target system.

\begin{table}[h!]
\centering
\caption{Open Port Analysis}
\label{tab:nmap}
\begin{tabular}{l l l p{0.5\linewidth}}
\toprule
\textbf{Port} & \textbf{State} & \textbf{Service (Inferred)} & \textbf{Notes} \\
\midrule
22/tcp & open & SSH (Secure Shell) & Administrative service exposed to the public internet. This is a common target for automated brute-force and credential stuffing attacks. Version information was not available from the scan. \\
\bottomrule
\end{tabular}
\end{table}

% --- Section 5: Consolidated Risk Assessment ---
\section{Consolidated Risk Assessment}
This section synthesizes findings from the security control review and technical scan into a prioritized list of risks. No pre-existing vulnerabilities were reported.

\begin{table}[h!]
\centering
\caption{Summary of Identified Risks}
\label{tab:risks}
\begin{tabular}{p{0.1\linewidth} p{0.25\linewidth} l p{0.45\linewidth}}
\toprule
\textbf{Risk ID} & \textbf{Risk Title} & \textbf{Severity} & \textbf{Description} \\
\midrule
RISK-001 & Lack of Multi-Factor Authentication (MFA) & \riskcritical{Critical} & The absence of MFA for email, computers, and sensitive systems allows an attacker with a single stolen password to gain complete access, leading to potential data breach and system compromise. \\
\addlinespace
RISK-002 & Inadequate Security Awareness Program & \riskhigh{High} & Without training, employees are significantly more likely to fall victim to phishing and other social engineering attacks, which are the primary vectors for credential theft and malware delivery. \\
\addlinespace
RISK-003 & Exposed Administrative Service (SSH) & \riskhigh{High} & The SSH service on \texttt{[Target IP]} is exposed to the internet. This risk is elevated from Medium to High due to the lack of MFA (RISK-001), making it a prime target for credential-based attacks. \\
\bottomrule
\end{tabular}
\end{table}

% --- Section 6: Recommendations ---
\section{Recommendations}
The following actionable recommendations are provided to mitigate the identified risks and improve the overall security posture of \textbf{[Organization Name]}.

\subsection{Recommendation 1: Implement MFA (Mitigates RISK-001)}
\begin{itemize}
    \item \textbf{Priority:} \riskcritical{Critical}
    \item \textbf{Action:} Immediately deploy and enforce MFA across all critical systems and applications. Prioritize the following:
    \begin{enumerate}
        \item Email (e.g., Office 365, Google Workspace).
        \item Remote access solutions (VPNs, SSH).
        \item Access to all systems containing sensitive or regulated data.
    \end{enumerate}
    \item \textbf{Justification:} MFA is the single most effective control to prevent unauthorized access resulting from stolen credentials. It provides a crucial second layer of defense that stops the vast majority of account takeover attacks.
\end{itemize}

\subsection{Recommendation 2: Establish Security Awareness Training (Mitigates RISK-002)}
\begin{itemize}
    \item \textbf{Priority:} \riskhigh{High}
    \item \textbf{Action:} Develop and implement a mandatory security awareness training program.
    \begin{enumerate}
        \item Conduct foundational training for all new employees during onboarding.
        \item Require annual refresher training for all staff.
        \item Training should cover phishing identification, password security, acceptable use, and incident reporting.
    \end{enumerate}
    \item \textbf{Justification:} A well-trained workforce is a critical component of a defense-in-depth strategy. Training reduces the likelihood of human error, which is a factor in over 80\% of data breaches.
\end{itemize}

\subsection{Recommendation 3: Secure External SSH Service (Mitigates RISK-003)}
\begin{itemize}
    \item \textbf{Priority:} \riskhigh{High}
    \item \textbf{Action:} Reduce the attack surface of the exposed SSH service.
    \begin{enumerate}
        \item If external access is not required, block port 22 at the firewall.
        \item If external access is required, restrict access to a whitelist of trusted IP addresses.
        \item Enforce public key authentication and disable password-based authentication entirely.
    \end{enumerate}
    \item \textbf{Justification:} These measures drastically reduce the risk of brute-force attacks and prevent attackers from using stolen passwords to gain administrative access to the network.
\end{itemize}

\end{document}
```