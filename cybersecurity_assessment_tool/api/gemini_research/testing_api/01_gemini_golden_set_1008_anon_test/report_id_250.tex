```latex
\documentclass[12pt]{article}

% --- PACKAGES ---
\usepackage[a4paper, margin=1in]{geometry}
\usepackage{pifont} % For checkmark and X symbols
\usepackage{booktabs} % For professional tables
\usepackage{hyperref} % For clickable links and metadata
\usepackage{url}      % For formatting URLs
\usepackage{seqsplit} % For splitting long strings to prevent overflow
\usepackage{graphicx} % For logo
\usepackage{xcolor}   % For colors

% --- DOCUMENT METADATA ---
\hypersetup{
    colorlinks=true,
    linkcolor=blue,
    filecolor=magenta,      
    urlcolor=cyan,
    pdftitle={Cybersecurity Posture Assessment Report},
    pdfauthor={Cybersecurity Analyst},
    pdfsubject={Security Analysis},
    pdfkeywords={Cybersecurity, Pentest, Nmap, Risk Assessment},
    pdffitwindow=true,
    pdfstartview={FitH}
}

% --- TITLE ---
\title{
    \vspace{-1.5cm}
    \includegraphics[width=0.4\textwidth]{example-image-a} \\ % Placeholder for a logo
    \vspace{1cm}
    \textbf{Cybersecurity Posture Assessment Report} \\
    \large For: \textbf{[Organization Name]}
}
\author{Cybersecurity Analysis Division}
\date{\today}

% --- BEGIN DOCUMENT ---
\begin{document}

\maketitle
\thispagestyle{empty}
\newpage

\tableofcontents
\newpage

% ===================================================================
% 1. EXECUTIVE SUMMARY
% ===================================================================
\section{Executive Summary}

This report details the findings of a cybersecurity posture assessment conducted for \textbf{[Organization Name]}. The assessment combined a review of organizational security controls, an external network vulnerability scan, and an analysis of pre-existing risks.

\paragraph{Key Findings:} The external network scan of the target IP address revealed a strong perimeter security posture, with no open ports or services detected. This is a positive indicator of a well-configured firewall that effectively blocks unsolicited inbound traffic.

However, the primary areas of concern originate from internal security controls. Two significant gaps were identified:
\begin{itemize}
    \item \textbf{Critical Risk:} The absence of Multi-Factor Authentication (MFA) for logging into employee computers. This gap exposes the organization to significant risk of unauthorized access should an employee's password be compromised.
    \item \textbf{High Risk:} The lack of mandatory, annual security awareness training for all employees. This increases the organization's susceptibility to social engineering and phishing attacks, which are common initial access vectors for threat actors.
\end{itemize}

\paragraph{Conclusion:} While the organization's external network perimeter is well-hardened, critical improvements are required for internal endpoint security and employee security awareness. Prioritizing the implementation of MFA on all endpoints and establishing a recurring security training program are essential next steps to mitigate these identified risks.

% ===================================================================
% 2. ORGANIZATIONAL INFORMATION
% ===================================================================
\section{Organizational Information}

The following details were used as the basis for this assessment.
\begin{itemize}
    \item \textbf{Organization Name:} \textbf{[Organization Name]}
    \item \textbf{Primary Email Domain:} \texttt{[Domain]}
    \item \textbf{External IP Scanned:} \texttt{[Client IP]}
\end{itemize}

% ===================================================================
% 3. SECURITY CONTROL REVIEW
% ===================================================================
\section{Security Control Review}

A review of organizational security controls was conducted via a standardized questionnaire. The responses are summarized below. "No" answers indicate potential gaps in the security framework and are highlighted for remediation.

\begin{table}[h!]
\centering
\caption{Organizational Security Control Questionnaire}
\begin{tabular}{p{0.6\linewidth} c l}
\toprule
\textbf{Control Question} & \textbf{Response} & \textbf{Assessment} \\
\midrule
Do you require MFA to access email? & \ding{51} & Best Practice Met \\
Do you require MFA to log into computers? & \textbf{\color{red}\ding{55}} & \textbf{Critical Gap} \\
Do you require MFA to access sensitive data systems? & \ding{51} & Best Practice Met \\
Does your organization have an employee acceptable use policy? & \ding{51} & Best Practice Met \\
Does your organization do security awareness training for new employees? & \ding{51} & Best Practice Met \\
Does your organization do security awareness training for all employees at least once per year? & \textbf{\color{orange}\ding{55}} & \textbf{High Risk} \\
\bottomrule
\end{tabular}
\end{table}

% ===================================================================
% 4. TECHNICAL SCAN RESULTS
% ===================================================================
\section{Technical Scan Results}

An external network scan was performed to identify open ports and exposed services on the organization's public-facing infrastructure.

\begin{itemize}
    \item \textbf{Target IP Address:} \texttt{[Target IP]}
    \item \textbf{Scan Date:} \today
\end{itemize}

\subsection{Scan Summary}
The scan did not identify any open TCP or UDP ports on the target host. All probes sent to the host were either dropped or rejected, indicating that no services are exposed to the public internet on this IP address.

\subsection{Analysis}
This is a positive security finding. It suggests a well-configured perimeter firewall is in place, operating on a principle of "default deny." This configuration significantly reduces the external attack surface and is a foundational element of a strong network security posture. No vulnerabilities were discovered during this phase of the assessment.

% ===================================================================
% 5. OVERALL RISK ASSESSMENT
% ===================================================================
\section{Overall Risk Assessment}

This section synthesizes findings from the security control review, technical scan, and pre-existing risk data. The following table summarizes the key identified risks.

\begin{table}[h!]
\centering
\caption{Synthesized Risk Summary}
\begin{tabular}{p{0.1\linewidth} p{0.3\linewidth} p{0.4\linewidth} l}
\toprule
\textbf{ID} & \textbf{Risk Name} & \textbf{Description} & \textbf{Severity} \\
\midrule
ORG-001 & Lack of MFA for Endpoint Login & User computers do not require MFA for login. A compromised password could grant an attacker direct access to an endpoint and the internal network. & \textbf{Critical} \\
\addlinespace
ORG-002 & Inadequate Security Awareness Training & Security training is not conducted annually for all staff. This increases the organization's susceptibility to phishing, social engineering, and other human-centric attacks. & \textbf{High} \\
\addlinespace
NET-001 & Strong External Firewall Configuration & The external network scan of the target IP revealed no open ports, indicating a hardened perimeter that denies unsolicited traffic. & \textit{Informational} \\
\bottomrule
\end{tabular}
\end{table}

\paragraph{Note on Pre-existing Risks:} The provided data on current risks contained no entries. Therefore, all findings in this report are newly identified during this assessment cycle.

% ===================================================================
% 6. RECOMMENDATIONS
% ===================================================================
\section{Recommendations}

Based on the findings of this assessment, the following actions are recommended to enhance the cybersecurity posture of \textbf{[Organization Name]}. Recommendations are prioritized by severity.

\begin{enumerate}
    \item \textbf{Implement MFA for All Endpoint Logins (Critical):}
    \begin{itemize}
        \item \textbf{Action:} Deploy and enforce Multi-Factor Authentication (MFA) for all employee computer and laptop logins (Windows, macOS, etc.).
        \item \textbf{Justification:} This is the single most effective control to prevent unauthorized access resulting from stolen or weak credentials. It creates a critical barrier for attackers attempting to move laterally after an initial compromise.
        \item \textbf{Tools:} Consider solutions compatible with your existing identity provider, such as Microsoft Azure AD Conditional Access with Windows Hello for Business, Duo Security, or Okta.
    \end{itemize}
    \vspace{0.5cm}
    
    \item \textbf{Establish an Annual Security Awareness Program (High):}
    \begin{itemize}
        \item \textbf{Action:} Develop and implement a mandatory annual security awareness training program for all employees, including management.
        \item \textbf{Justification:} A well-trained workforce is a critical layer of defense. Regular training ensures that employees can recognize and appropriately respond to modern threats like phishing, business email compromise, and social engineering.
        \item \textbf{Content:} The program should cover key topics such as phishing identification, password hygiene, safe browsing habits, and the organization's acceptable use policy.
    \end{itemize}
\end{enumerate}

\end{document}
```