```latex
\documentclass[12pt]{article}

% Preamble: Required Packages
\usepackage[margin=1in]{geometry}
\usepackage{pifont} % For \ding symbols (checkmarks/crosses)
\usepackage{booktabs} % For professional-looking tables
\usepackage{hyperref} % For creating hyperlinks
\usepackage{url}      % For typesetting URLs
\usepackage{seqsplit} % For splitting long strings without spaces
\usepackage{graphicx}
\usepackage{xcolor}
\usepackage{tocloft}

% --- Custom Commands & Settings ---
\newcommand{\yes}{\textcolor{green}{\ding{51}}}
\newcommand{\no}{\textcolor{red}{\ding{55}}}

\hypersetup{
    colorlinks=true,
    linkcolor=blue,
    filecolor=magenta,
    urlcolor=cyan,
    pdftitle={Cybersecurity Posture Assessment Report},
    pdfauthor={Cybersecurity Analysis Division},
}

\begin{document}

% --- Title Page ---
\begin{titlepage}
    \centering
    \vspace*{1cm}
    \Huge\textbf{Cybersecurity Posture Assessment Report}
    \vspace{1.5cm}
    \Large
    \textbf{Prepared for:} \\
    \vspace{0.5cm}
    \Huge\textbf{[Organization Name]}
    \vfill
    \Large
    \textbf{Date of Report:} \today \\
    \textbf{Analysis Period:} \today
    \vspace{1.5cm}
    \hrule
    \vspace{0.5cm}
    \large \textit{This document contains sensitive information and is intended for the exclusive use of the recipient.}
    \vspace{0.5cm}
    \hrule
\end{titlepage}

% --- Table of Contents ---
\tableofcontents
\newpage

% --- Section 1: Executive Summary ---
\section{Executive Summary}
This report provides a comprehensive assessment of the cybersecurity posture for \textbf{[Organization Name]}, synthesizing information from technical network scans, a security controls questionnaire, and a review of pre-existing risks. The analysis reveals several areas of concern requiring immediate attention to mitigate potential threats.

A critical vulnerability was identified on an external-facing service at \texttt{[Target IP]}. An FTP server is running a version of \texttt{vsftpd} (2.3.4) known to contain a backdoor vulnerability (\textbf{CVE-2011-2523}), which could allow an attacker to gain complete control of the system. This risk is compounded by the server's configuration, which permits anonymous user logins.

Furthermore, a significant gap in internal security controls was noted: Multi-Factor Authentication (MFA) is not required for accessing sensitive data systems. This policy oversight creates a substantial risk of unauthorized access to critical information.

These new findings, combined with the pre-existing risk of outdated Windows 7 workstations, indicate a need for a focused effort to remediate vulnerabilities and strengthen security controls across the organization. Recommendations for mitigation are detailed in Section \ref{sec:recommendations}.

% --- Section 2: Organizational Information ---
\section{Organizational Information}
The following details were used as the basis for this assessment. Due to the anonymized nature of the input data, placeholders have been used where necessary.

\begin{itemize}
    \item \textbf{Organization Name:} \textbf{[Organization Name]}
    \item \textbf{Primary Email Domain:} \texttt{[Domain]}
    \item \textbf{External IP Scanned:} \texttt{[Client IP]}
    \item \textbf{Target of Network Scan:} \texttt{[Target IP]}
\end{itemize}

% --- Section 3: Security Control Review ---
\section{Security Control Review}
A review of the organization's security controls was conducted via a questionnaire. The responses indicate a solid foundation in employee awareness and endpoint security but reveal a critical gap in protecting sensitive data.

\begin{table}[h!]
\centering
\caption{Security Controls Questionnaire Results}
\begin{tabular}{p{0.7\linewidth} c}
\toprule
\textbf{Control Question} & \textbf{Status} \\
\midrule
Do you require MFA to access email? & \yes \\
Do you require MFA to log into computers? & \yes \\
\textbf{Do you require MFA to access sensitive data systems?} & \no \\
Does your organization have an employee acceptable use policy? & \yes \\
Does your organization do security awareness training for new employees? & \yes \\
Does your organization do security awareness training for all employees at least once per year? & \yes \\
\bottomrule
\end{tabular}
\end{table}

The lack of MFA for sensitive data systems is a high-risk finding. MFA is a critical defense against credential theft and unauthorized access, and its absence on high-value systems exposes the organization to significant risk.

% --- Section 4: Technical Scan Results ---
\section{Technical Scan Results}
An external network scan was performed on the target IP address to identify open ports and exposed services.

\subsection{Host: \texttt{[Target IP]}}
The host was found to be online and responsive. The following open port was discovered:

\begin{table}[h!]
\centering
\caption{Open Ports and Services on \texttt{[Target IP]}}
\begin{tabular}{l l l l p{0.4\linewidth}}
\toprule
\textbf{Port} & \textbf{State} & \textbf{Service} & \textbf{Version} & \textbf{Notes} \\
\midrule
21/tcp & Open & ftp & vsftpd 2.3.4 & Anonymous FTP login allowed. \textbf{This version is critically vulnerable to a backdoor (CVE-2011-2523).} \\
\bottomrule
\end{tabular}
\end{table}

The presence of \texttt{vsftpd 2.3.4} is a \textbf{critical risk}. This specific version was compromised in 2011, and a malicious backdoor was inserted into the source code. An attacker can exploit this by sending a specific sequence of characters to the server, which opens a command shell on port 6200, granting them remote control. The allowance of anonymous logins further lowers the barrier to entry for potential attackers.

% --- Section 5: Consolidated Risk Assessment ---
\section{Consolidated Risk Assessment}
The following table consolidates findings from the technical scan, the controls review, and pre-existing risk data to provide a unified view of the organization's current risk profile.

\begin{table}[h!]
\centering
\caption{Summary of Identified Risks}
\label{tab:risksummary}
\begin{tabular}{p{0.25\linewidth} p{0.45\linewidth} l p{0.15\linewidth}}
\toprule
\textbf{Risk Name} & \textbf{Overview} & \textbf{Severity} & \textbf{Affected Systems} \\
\midrule
\textbf{Vulnerable FTP Server (CVE-2011-2523)} & An external-facing FTP server is running a version with a known remote command execution backdoor. Anonymous login is enabled. & \textbf{Critical} & External Server at \texttt{[Target IP]} \\
\addlinespace
\textbf{No MFA on Sensitive Data Systems} & Lack of multi-factor authentication for systems storing or processing sensitive data exposes them to unauthorized access via compromised credentials. & \textbf{High} & Internal Data Systems \\
\addlinespace
\textbf{Outdated Windows Policy} & Workstations are running Windows 7, which is an end-of-life operating system no longer receiving security updates. & \textbf{Medium} & Workstations \\
\bottomrule
\end{tabular}
\end{table}

% --- Section 6: Recommendations ---
\section{Recommendations}
\label{sec:recommendations}
Based on the findings in this report, the following actions are recommended to reduce the identified risks.

\subsection{Critical Risk: Vulnerable FTP Server}
\begin{itemize}
    \item \textbf{Immediate Action (Containment):} Immediately disable the FTP service on the server at \texttt{[Target IP]} or use a firewall to block all access to port 21 from the internet. This is the most effective way to prevent exploitation.
    \item \textbf{Short-Term Action (Remediation):} If the FTP service is required, upgrade \texttt{vsftpd} to the latest stable version. If it is not required, uninstall the software completely.
    \item \textbf{Configuration Hardening:} Disable anonymous FTP logins. Access should be restricted to authenticated users only.
\end{itemize}

\subsection{High Risk: No MFA on Sensitive Data Systems}
\begin{itemize}
    \item \textbf{Short-Term Action (Scoping):} Identify all systems, applications, and databases that store or process sensitive organizational data.
    \item \textbf{Long-Term Action (Implementation):} Develop and execute a project to implement and enforce MFA for all user access to the identified sensitive systems.
\end{itemize}

\subsection{Medium Risk: Outdated Windows Policy}
\begin{itemize}
    \item \textbf{Action:} Continue with the existing recommendation to mitigate this risk. Prioritize the upgrade of all Windows 7 workstations to a currently supported operating system, such as Windows 10 or 11. Plan for hardware replacements where necessary to support the new OS.
\end{itemize}

\end{document}
```