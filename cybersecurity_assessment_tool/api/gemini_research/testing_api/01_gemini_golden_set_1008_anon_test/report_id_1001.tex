```latex
\documentclass[12pt]{article}

% Preamble: Required Packages
\usepackage[margin=1in]{geometry}
\usepackage{pifont} % For checkmarks and crosses (\ding)
\usepackage{booktabs} % For professional tables (\toprule, \midrule, \bottomrule)
\usepackage{hyperref}
\usepackage{url}
\usepackage{seqsplit} % For breaking long strings in \texttt
\usepackage{xcolor}

% Hyperlink and PDF metadata setup
\hypersetup{
    colorlinks=true,
    linkcolor=blue,
    filecolor=magenta,
    urlcolor=cyan,
    pdftitle={Cybersecurity Assessment Report},
    pdfauthor={Cybersecurity Analyst},
    pdfsubject={Security Analysis},
    pdfkeywords={security, report, analysis},
}

% Custom commands for Yes/No indicators
\newcommand{\yes}{\textcolor{green}{\ding{51}}}
\newcommand{\no}{\textcolor{red}{\ding{55}}}

% --- Document Start ---
\begin{document}

\title{Cybersecurity Assessment Report \\ \large For \textbf{[Organization Name]}}
\author{Cybersecurity Analyst}
\date{\today}
\maketitle

\begin{abstract}
This report provides a comprehensive cybersecurity assessment for \textbf{[Organization Name]}. The analysis is based on a synthesis of network scan data, a review of organizational security controls, and an evaluation of pre-existing risks. The assessment reveals a critical risk posture, primarily driven by the direct exposure of Remote Desktop Protocol (RDP) to the public internet. This technical vulnerability is significantly amplified by critical gaps in organizational security policies, including a lack of mandatory Multi-Factor Authentication (MFA) for computer access and the absence of a formal Acceptable Use Policy. Immediate remediation is required to mitigate the high probability of a security breach.
\end{abstract}

\section{Organizational Information}
This section details the information provided about the organization.
\begin{itemize}
    \item \textbf{Organization Name:} \textbf{[Organization Name]}
    \item \textbf{Primary Domain:} \texttt{[Domain]}
    \item \textbf{External IP Address Scanned:} \texttt{[Client IP]}
\end{itemize}

\section{Security Control Review}
The following table summarizes the organization's responses to a security controls questionnaire. Each response is assessed against industry best practices. "No" answers indicate significant gaps in the security framework.

\begin{table}[h!]
\centering
\caption{Security Controls Questionnaire Analysis}
\label{tab:controls}
\begin{tabular}{@{}p{0.6\linewidth} c p{0.2\linewidth}@{}}
\toprule
\textbf{Control Question} & \textbf{Response} & \textbf{Assessment} \\
\midrule
Do you require MFA to access email? & \yes & Best Practice Met \\
Do you require MFA to log into computers? & \no & \textbf{Critical Gap} \\
Do you require MFA to access sensitive data systems? & \yes & Best Practice Met \\
Does your organization have an employee acceptable use policy? & \no & \textbf{High Risk} \\
Does your organization do security awareness training for new employees? & \no & \textbf{High Risk} \\
Does your organization do security awareness training for all employees at least once per year? & \yes & Best Practice Met \\
\bottomrule
\end{tabular}
\end{table}

\section{Technical Scan Results}
An external network scan was performed to identify exposed services. The target IP address was identified as \texttt{[Target IP]}.

\subsection{Open Ports}
The scan identified the following open port, which presents a significant security risk.

\begin{table}[h!]
\centering
\caption{Nmap Scan Results for \texttt{[Target IP]}}
\label{tab:nmap}
\begin{tabular}{@{}llll@{}}
\toprule
\textbf{Port} & \textbf{State} & \textbf{Service} & \textbf{Analysis} \\
\midrule
3389/tcp & open & ms-wbt-server & Remote Desktop Protocol (RDP) \\
\bottomrule
\end{tabular}
\end{table}

\subsection{Analysis of Findings}
The presence of an open port \textbf{3389 (RDP)} is a critical vulnerability. Exposing RDP directly to the internet makes the network a prime target for automated brute-force attacks and exploitation by ransomware gangs. This finding directly confirms the pre-existing risk documented in the organization's risk register.

\section{Consolidated Risk Assessment}
This section synthesizes findings from the security control review, technical scan, and pre-existing risk data into a consolidated list of key risks.

\begin{table}[h!]
\centering
\caption{Summary of Identified Risks}
\label{tab:risks}
\begin{tabular}{@{}p{0.25\linewidth} p{0.15\linewidth} p{0.5\linewidth}@{}}
\toprule
\textbf{Risk Name} & \textbf{Severity} & \textbf{Description} \\
\midrule
\textbf{RDP Exposure} & \textbf{Critical} & The RDP service on \texttt{[Target IP]} is exposed to the public internet, inviting brute-force attacks and exploitation. This is the most urgent threat. \\
\addlinespace
\textbf{Lack of Endpoint MFA} & \textbf{High} & The absence of MFA for computer logins means that a single compromised password could grant an attacker full access to a user's workstation, facilitating lateral movement. \\
\addlinespace
\textbf{Missing Acceptable Use Policy (AUP)} & \textbf{High} & Without a formal AUP, employees lack clear guidelines on security responsibilities, data handling, and acceptable online behavior, increasing the likelihood of insider threats and accidental breaches. \\
\addlinespace
\textbf{Inadequate Security Onboarding} & \textbf{High} & Failing to train new employees on security best practices from day one leaves a critical window of vulnerability where they are more susceptible to phishing and social engineering attacks. \\
\bottomrule
\end{tabular}
\end{table}

\section{Recommendations}
The following actions are recommended to mitigate the identified risks. They are prioritized based on severity and potential impact.

\begin{enumerate}
    \item \textbf{[Immediate Priority] Remediate RDP Exposure:} The exposed RDP service on \texttt{[Target IP]} must be secured immediately.
    \begin{itemize}
        \item \textbf{Short-Term Fix:} Implement a firewall rule to block all inbound traffic to TCP port 3389 from the internet. Access should only be permitted from trusted IP addresses, if absolutely necessary.
        \item \textbf{Long-Term Solution:} Implement a Virtual Private Network (VPN) solution. All remote access to internal resources, including RDP, must be routed through the VPN, which provides an encrypted and authenticated tunnel.
    \end{itemize}
    \item \textbf{[High Priority] Implement MFA for All Computer Logins:} Deploy a mandatory MFA solution for all employee and administrative computer logins. This drastically reduces the risk of unauthorized access from compromised credentials.
    \item \textbf{[High Priority] Develop and Enforce an Acceptable Use Policy (AUP):} Create a formal AUP that clearly defines rules for computer and network usage, data handling, password security, and the consequences of non-compliance. All employees must read and acknowledge this policy.
    \item \textbf{[Medium Priority] Integrate Security Training into Onboarding:} Develop a security awareness training module specifically for new hires. This training should be a mandatory part of the onboarding process and cover key topics like phishing, password hygiene, and the new AUP.
\end{enumerate}

\end{document}
```