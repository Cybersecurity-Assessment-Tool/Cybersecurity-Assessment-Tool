```latex
\documentclass[12pt]{article}

% -----------------------------------------------------------------------------
% PREAMBLE
% -----------------------------------------------------------------------------
\usepackage[margin=1in]{geometry}
\usepackage{pifont} % For checkmarks and crosses
\usepackage{booktabs} % For professional tables
\usepackage{hyperref} % For clickable links
\usepackage{url}      % For URL formatting
\usepackage{seqsplit} % To split long strings in tt font
\usepackage{graphicx}
\usepackage{xcolor}

% --- Hyperref Setup ---
\hypersetup{
    colorlinks=true,
    linkcolor=black,
    filecolor=magenta,      
    urlcolor=blue,
    pdftitle={Cybersecurity Assessment Report},
    pdfpagemode=FullScreen,
}

% --- Title Information ---
\title{Cybersecurity Assessment Report \\ \large For \textbf{[Organization Name]}}
\author{Cybersecurity Analysis Division}
\date{\today}

% -----------------------------------------------------------------------------
% DOCUMENT START
% -----------------------------------------------------------------------------
\begin{document}

\maketitle
\thispagestyle{empty}
\newpage
\tableofcontents
\newpage

% -----------------------------------------------------------------------------
% SECTION 1: EXECUTIVE OVERVIEW
% -----------------------------------------------------------------------------
\section{Executive Overview}

This report details the findings of a cybersecurity assessment conducted for \textbf{[Organization Name]}. The evaluation combined a review of organizational security controls via a questionnaire, an external network vulnerability scan, and an analysis of pre-existing risks.

The organization's security posture presents a significant contrast. From an external network perspective, the scanned asset at \texttt{[Client IP]} demonstrates a strong defensive configuration, with no open ports or exposed services detected. This indicates effective firewall and network hardening practices.

However, the internal security posture reveals several critical and high-risk gaps in administrative and policy-based controls. The most severe findings are the absence of Multi-Factor Authentication (MFA) for computer logins and access to sensitive data systems. This exposes the organization to significant risk from credential theft and unauthorized access. Furthermore, the lack of an employee Acceptable Use Policy and mandatory security training for new hires creates a high-risk environment susceptible to insider threats and social engineering.

Immediate remediation should focus on implementing MFA across all critical assets and developing foundational security policies and training programs to address these identified deficiencies.

% -----------------------------------------------------------------------------
% SECTION 2: ORGANIZATIONAL INFORMATION
% -----------------------------------------------------------------------------
\section{Organizational Information}

The following details were used as the basis for this assessment. Placeholders are used where information was not provided.

\begin{itemize}
    \item \textbf{Organization Name:} \textbf{[Organization Name]}
    \item \textbf{Primary Domain:} \texttt{[Domain]}
    \item \textbf{External IP Scanned:} \texttt{[Client IP]}
\end{itemize}

% -----------------------------------------------------------------------------
% SECTION 3: SECURITY CONTROL REVIEW
% -----------------------------------------------------------------------------
\section{Security Control Review}

A review of internal security controls was conducted based on a standardized questionnaire. The responses indicate the current state of implemented policies and procedures. A (\ding{51}) indicates a positive control is in place, while a (\ding{55}) indicates a control gap that introduces risk.

\begin{table}[h!]
\centering
\caption{Organizational Security Controls Questionnaire}
\begin{tabular}{p{0.8\linewidth} c}
\toprule
\textbf{Control Question} & \textbf{Response} \\
\midrule
Do you require MFA to access email? & \ding{51} \\
Do you require MFA to log into computers? & \textcolor{red}{\ding{55}} \\
Do you require MFA to access sensitive data systems? & \textcolor{red}{\ding{55}} \\
Does your organization have an employee acceptable use policy? & \textcolor{red}{\ding{55}} \\
Does your organization do security awareness training for new employees? & \textcolor{red}{\ding{55}} \\
Does your organization do security awareness training for all employees at least once per year? & \ding{51} \\
\bottomrule
\end{tabular}
\end{table}

% -----------------------------------------------------------------------------
% SECTION 4: TECHNICAL SCAN RESULTS
% -----------------------------------------------------------------------------
\section{Technical Scan Results}

An external network scan was performed on the provided target IP address to identify exposed services and potential vulnerabilities.

\begin{itemize}
    \item \textbf{Target IP:} \texttt{[Target IP]}
    \item \textbf{Scan Date:} Not provided in scan data.
\end{itemize}

\subsection{Summary of Findings}
The scan results were positive, indicating a very strong external security posture for the assessed host.
\begin{itemize}
    \item \textbf{Open Ports Found:} 0
    \item \textbf{Port Status:} All 1000 scanned TCP ports were found to be in a \texttt{closed} state. This means that while the host is reachable, there are no listening services available for external interaction, effectively minimizing the external attack surface.
\end{itemize}

No vulnerabilities were identified from the external network scan.

% -----------------------------------------------------------------------------
% SECTION 5: RISK ASSESSMENT
% -----------------------------------------------------------------------------
\section{Risk Assessment}

This section synthesizes findings from the security control review, technical scan, and any pre-existing risk data. The primary risks identified are related to internal policy and identity management. No pre-existing vulnerabilities were provided for this assessment.

\begin{table}[h!]
\centering
\caption{Identified Risks}
\begin{tabular}{p{0.1\linewidth} p{0.5\linewidth} p{0.15\linewidth} p{0.15\linewidth}}
\toprule
\textbf{Risk ID} & \textbf{Description} & \textbf{Severity} & \textbf{Source} \\
\midrule
RISK-001 & Lack of MFA on endpoint computers allows a compromised password to grant an attacker full access to a user's workstation and connected network resources. & \textbf{Critical} & Questionnaire \\
\addlinespace
RISK-002 & Lack of MFA on sensitive data systems. A single stolen credential could lead to a significant data breach of confidential or regulated information. & \textbf{Critical} & Questionnaire \\
\addlinespace
RISK-003 & Absence of a formal Acceptable Use Policy (AUP) leads to inconsistent user behavior and a lack of enforceable standards for data handling and system usage. & High & Questionnaire \\
\addlinespace
RISK-004 & No security awareness training for new employees. New hires are a primary target for phishing and are unaware of organizational policies, creating a significant vulnerability window. & High & Questionnaire \\
\bottomrule
\end{tabular}
\end{table}

% -----------------------------------------------------------------------------
% SECTION 6: RECOMMENDATIONS
% -----------------------------------------------------------------------------
\section{Recommendations}

The following actions are recommended to mitigate the identified risks and improve the overall security posture of \textbf{[Organization Name]}.

\subsection{Recommendation 1 (Critical): Implement MFA}
\begin{itemize}
    \item \textbf{Action:} Deploy a robust Multi-Factor Authentication (MFA) solution for all employees.
    \item \textbf{Priority 1:} Require MFA for all user logins to company computers (endpoints). This is a critical defense against credential theft and lateral movement.
    \item \textbf{Priority 2:} Require MFA for access to all systems containing sensitive, confidential, or proprietary data, including databases, file shares, and critical applications.
    \item \textbf{Justification:} This action directly mitigates the two highest-rated risks (RISK-001, RISK-002) and is one of the most effective controls for preventing unauthorized access.
\end{itemize}

\subsection{Recommendation 2 (High): Develop Foundational Policies}
\begin{itemize}
    \item \textbf{Action:} Create, approve, and disseminate a formal Employee Acceptable Use Policy (AUP).
    \item \textbf{Details:} This policy should clearly define the rules for using company assets, handling data, accessing the internet, and the consequences of non-compliance.
    \item \textbf{Implementation:} All current employees must read and formally acknowledge the policy. This process should be integrated into the onboarding checklist for all new hires.
    \item \textbf{Justification:} Mitigates RISK-003 by establishing a clear security baseline for all personnel and provides a framework for enforcing security standards.
\end{itemize}

\subsection{Recommendation 3 (High): Enhance Security Training Program}
\begin{itemize}
    \item \textbf{Action:} Institute a mandatory security awareness training module for all new employees as part of the onboarding process.
    \item \textbf{Details:} This training should occur before a new hire is granted significant access to network resources. It should cover key topics such as phishing identification, password hygiene, data handling, and the new AUP.
    \item \textbf{Justification:} Closes the critical vulnerability window identified in RISK-004, ensuring that new staff are equipped with basic security knowledge from day one.
\end{itemize}

% -----------------------------------------------------------------------------
% DOCUMENT END
% -----------------------------------------------------------------------------
\end{document}
```