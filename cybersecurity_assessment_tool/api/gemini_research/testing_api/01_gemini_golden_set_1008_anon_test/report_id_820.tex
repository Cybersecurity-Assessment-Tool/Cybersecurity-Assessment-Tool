```latex
\documentclass[12pt, a4paper]{article}

% Required Packages
\usepackage[margin=1in]{geometry}
\usepackage{pifont} % For checkmarks and crosses
\usepackage{booktabs} % For professional tables
\usepackage{hyperref} % For hyperlinks
\usepackage{url} % For URL formatting
\usepackage{seqsplit} % To split long strings without breaking
\usepackage{graphicx}
\usepackage{xcolor}
\usepackage{array}

% Define colors
\definecolor{darkblue}{rgb}{0.0, 0.0, 0.55}
\definecolor{darkred}{rgb}{0.55, 0.0, 0.0}

% Hyperref Setup
\hypersetup{
    colorlinks=true,
    linkcolor=darkblue,
    filecolor=darkblue,      
    urlcolor=darkblue,
    citecolor=darkblue,
}

% Document Information
\title{Cybersecurity Assessment Report}
\author{Cybersecurity Analysis Division}
\date{\today}

\begin{document}

\maketitle
\thispagestyle{empty}
\newpage

\tableofcontents
\thispagestyle{empty}
\newpage

\setcounter{page}{1}

% --- Executive Summary ---
\section{Executive Summary}

This report provides a comprehensive cybersecurity assessment for \textbf{[Organization Name]}. The analysis is based on a correlation of organizational data, an external network scan, and a review of pre-existing risks.

The assessment reveals a mixed security posture. On a positive note, the external network scan of the target IP address (\texttt{[Target IP]}) indicates a strong perimeter defense, with no open ports discovered. This significantly reduces the external attack surface.

However, critical gaps were identified in internal security controls and policies. The most severe findings are the absence of Multi-Factor Authentication (MFA) for email access and the lack of a formal security awareness training program for employees. These deficiencies expose the organization to a high risk of social engineering attacks, such as phishing, which could lead to account compromise, data breaches, and financial loss.

Immediate remediation should focus on implementing MFA for email and establishing a comprehensive security awareness training schedule. Continued vigilance in maintaining the secure network perimeter is also recommended.

% --- Organizational Information ---
\section{Organizational Information}

This section details the information provided by the organization. The data is used as a baseline for the assessment.

\begin{tabular}{@{}ll}
\toprule
\textbf{Attribute} & \textbf{Value} \\
\midrule
Organization Name & \textbf{[Organization Name]} \\
Primary Email Domain & \texttt{[Domain]} \\
Primary External IP & \texttt{[Client IP]} \\
\bottomrule
\end{tabular}

% --- Security Control Review ---
\section{Security Control Review}

The following table summarizes the organization's responses to a security controls questionnaire. These answers highlight the current state of implemented policies and procedures. Gaps identified here are a primary source of organizational risk.

\begin{tabular}{p{0.7\linewidth} >{\centering\arraybackslash}p{0.2\linewidth}}
\toprule
\textbf{Control Question} & \textbf{Status} \\
\midrule
Do you require MFA to access email? & \textcolor{darkred}{\ding{55}} \\
Do you require MFA to log into computers? & \textcolor{green}{\ding{51}} \\
Do you require MFA to access sensitive data systems? & \textcolor{green}{\ding{51}} \\
Does your organization have an employee acceptable use policy? & \textcolor{green}{\ding{51}} \\
Does your organization do security awareness training for new employees? & \textcolor{darkred}{\ding{55}} \\
Does your organization do security awareness training for all employees at least once per year? & \textcolor{darkred}{\ding{55}} \\
\bottomrule
\end{tabular}

\subsection*{Analysis of Controls}
The review indicates a critical weakness in email security. The lack of MFA on email is a significant vulnerability, as email accounts are a primary target for attackers seeking to gain an initial foothold in an organization. Furthermore, the complete absence of a security awareness training program leaves employees, both new and existing, unprepared to identify and respond to common threats like phishing and social engineering.

% --- Technical Scan Results ---
\section{Technical Scan Results}

An external network vulnerability scan was conducted to identify potential weaknesses in the organization's internet-facing infrastructure.

\begin{itemize}
    \item \textbf{Target IP Address:} \texttt{[Target IP]}
    \item \textbf{Scan Date:} Not provided in scan data.
    \item \textbf{Host Status:} UP
\end{itemize}

\subsection*{Findings}
The scan confirmed that the target host is online and responsive. However, \textbf{no open TCP ports were discovered}. All scanned ports were found to be in a `closed` state.

\paragraph{Conclusion:} This is a positive security finding. A minimal external footprint, with no exposed services, drastically reduces the risk of automated attacks and opportunistic exploitation from external threat actors. It indicates effective firewall configuration and network hardening.

% --- Consolidated Risk Assessment ---
\section{Consolidated Risk Assessment}

This section synthesizes findings from the security control review, technical scan, and pre-existing risk data into a prioritized list.

\begin{tabular}{@{}lp{0.5\linewidth}ll}
\toprule
\textbf{Risk ID} & \textbf{Description} & \textbf{Source} & \textbf{Severity} \\
\midrule
RISK-001 & \textbf{Lack of MFA on Email:} User email accounts can be compromised with only a password, making them highly susceptible to phishing and credential stuffing attacks. & Questionnaire & \textcolor{darkred}{\textbf{Critical}} \\
\addlinespace
RISK-002 & \textbf{Inadequate Security Awareness Training:} Employees are not trained to recognize or report security threats, increasing the likelihood of a successful social engineering attack. & Questionnaire & \textcolor{darkred}{\textbf{High}} \\
\addlinespace
\multicolumn{4}{l}{\textit{No pre-existing vulnerabilities were reported in the input data.}} \\
\bottomrule
\end{tabular}

% --- Recommendations ---
\section{Recommendations}

Based on the consolidated risk assessment, the following actions are recommended to improve the organization's security posture. Recommendations are prioritized by severity.

\subsection*{Priority 1: Remediate Critical Risks}
\begin{enumerate}
    \item \textbf{Enforce MFA on All Email Accounts:}
    \begin{itemize}
        \item \textbf{Action:} Immediately enable and enforce MFA for all user mailboxes. This is the single most effective control to prevent unauthorized access to email.
        \item \textbf{Justification:} Mitigates RISK-001. Even if an attacker obtains a user's password, they will be unable to access the account without the second authentication factor.
    \end{itemize}
\end{enumerate}

\subsection*{Priority 2: Remediate High Risks}
\begin{enumerate}
    \setcounter{enumi}{1} % Continue numbering from previous list
    \item \textbf{Implement a Security Awareness Training Program:}
    \begin{itemize}
        \item \textbf{Action:} Procure or develop a security awareness training program. The program must be mandatory for all new hires upon onboarding and for all existing employees at least annually. Conduct periodic phishing simulations to test and reinforce the training.
        \item \textbf{Justification:} Mitigates RISK-002. A well-trained workforce acts as a human firewall, providing a critical layer of defense against attacks that bypass technical controls.
    \end{itemize}
\end{enumerate}

\subsection*{Priority 3: Maintain and Monitor}
\begin{enumerate}
    \setcounter{enumi}{2} % Continue numbering
    \item \textbf{Continue Regular Network Scanning:}
    \begin{itemize}
        \item \textbf{Action:} Maintain the current strong perimeter security. Schedule regular, automated external network scans (at least quarterly) to ensure no new services are inadvertently exposed.
        \item \textbf{Justification:} Proactively identifies changes in the external attack surface, ensuring the strong technical posture is maintained over time.
    \end{itemize}
\end{enumerate}

\end{document}
```