```latex
\documentclass[12pt]{article}

% ----------------------------------------------------------------------
% PREAMBLE
% ----------------------------------------------------------------------
\usepackage[margin=1in]{geometry}
\usepackage{pifont} % For checkmarks and crosses
\usepackage{booktabs} % For professional tables
\usepackage[hidelinks]{hyperref} % For clickable links without boxes
\usepackage{url} % For URL formatting
\usepackage{seqsplit} % For splitting long strings like hashes or IPs
\usepackage{graphicx}
\usepackage{xcolor}

% Define colors for severity
\definecolor{sevhigh}{RGB}{217, 83, 79}
\definecolor{sevmedium}{RGB}{240, 173, 78}
\definecolor{sevlow}{RGB}{92, 184, 92}

% Hyperref setup
\hypersetup{
    colorlinks=true,
    linkcolor=blue,
    filecolor=magenta,      
    urlcolor=cyan,
    pdftitle={Cybersecurity Assessment Report},
    pdfauthor={Cybersecurity Analyst},
    pdfsubject={Security Assessment},
    pdfkeywords={Security, Assessment, Report},
    bookmarks=true
}

% ----------------------------------------------------------------------
% DOCUMENT START
% ----------------------------------------------------------------------
\begin{document}

% ----------------------------------------------------------------------
% TITLE PAGE
% ----------------------------------------------------------------------
\begin{titlepage}
    \centering
    \vspace*{1cm}
    
    \Huge
    \textbf{Cybersecurity Assessment Report}
    
    \vspace{1.5cm}
    
    \Large
    Prepared for: \\
    \vspace{0.5cm}
    \textbf{[Organization Name]}
    
    \vspace{2cm}
    
    \Large
    \textbf{Date of Report:} \today
    
    \vfill
    
    \large
    \textbf{CONFIDENTIAL}
    \vspace{0.8cm}
    
    \textit{This document contains sensitive information. Access is restricted to authorized personnel only. Do not distribute without explicit permission.}
    
\end{titlepage}

\tableofcontents
\newpage

% ----------------------------------------------------------------------
% 1. EXECUTIVE SUMMARY
% ----------------------------------------------------------------------
\section{Executive Summary}

This report details the findings of a cybersecurity assessment conducted for \textbf{[Organization Name]}. The assessment combined a review of organizational security controls via a questionnaire, an external network scan, and an analysis of pre-existing risks.

The overall security posture requires immediate attention. Several critical gaps were identified in foundational security policies and technical controls. Key findings include:

\begin{itemize}
    \item \textbf{Critical Policy Gaps:} The organization lacks a formal Acceptable Use Policy (AUP) and does not provide security awareness training to new employees during onboarding. These omissions create significant risk by leaving employees unaware of their security responsibilities.
    \item \textbf{Weak Endpoint Security:} Multi-Factor Authentication (MFA) is not required for computer logins. This exposes the organization to significant risk from compromised credentials, potentially leading to unauthorized access and lateral movement within the network.
    \item \textbf{Exposed Network Service:} An external network scan identified an open Secure Shell (SSH) port (TCP/22). Publicly exposed management services like SSH are high-value targets for attackers and must be secured with extreme prejudice.
\end{itemize}

This report provides a detailed analysis of these findings and offers actionable recommendations to mitigate the identified risks and strengthen the organization's overall security posture. We urge management to prioritize the implementation of the proposed remediation steps.

% ----------------------------------------------------------------------
% 2. ORGANIZATIONAL INFORMATION
% ----------------------------------------------------------------------
\section{Organizational Information}

This section contains the high-level information provided for the assessment. The data has been anonymized as requested.

\begin{itemize}
    \item \textbf{Organization Name:} \textbf{[Organization Name]}
    \item \textbf{Primary Email Domain:} \texttt{[Domain]}
    \item \textbf{Scanned External IP:} \seqsplit{\texttt{[Client IP]}}
\end{itemize}

% ----------------------------------------------------------------------
% 3. SECURITY CONTROL REVIEW
% ----------------------------------------------------------------------
\section{Security Control Review}

The following table summarizes the organization's responses to a security controls questionnaire. A "No" response indicates a potential gap in the security framework that may require remediation.

\begin{table}[h!]
\centering
\caption{Security Controls Questionnaire Analysis}
\label{tab:controls}
\begin{tabular}{p{0.6\textwidth} c p{0.2\textwidth}}
\toprule
\textbf{Control Question} & \textbf{Response} & \textbf{Assessment} \\
\midrule
Do you require MFA to access email? & \ding{51} & Best Practice Met \\
\addlinespace
Do you require MFA to log into computers? & \textbf{\color{red}\ding{55}} & \textbf{Critical Gap} \\
\addlinespace
Do you require MFA to access sensitive data systems? & \ding{51} & Best Practice Met \\
\addlinespace
Does your organization have an employee acceptable use policy? & \textbf{\color{red}\ding{55}} & \textbf{High Risk} \\
\addlinespace
Does your organization do security awareness training for new employees? & \textbf{\color{red}\ding{55}} & \textbf{High Risk} \\
\addlinespace
Does your organization do security awareness training for all employees at least once per year? & \ding{51} & Best Practice Met \\
\bottomrule
\end{tabular}
\end{table}

The analysis highlights three significant control gaps: a lack of MFA for computer access, the absence of an Acceptable Use Policy, and no security training for new hires. These represent fundamental weaknesses in the organization's defense-in-depth strategy.

% ----------------------------------------------------------------------
% 4. TECHNICAL SCAN RESULTS
% ----------------------------------------------------------------------
\section{Technical Scan Results}

An external network vulnerability scan was performed against the target IP address provided. The scan was limited in scope and did not include service version detection or vulnerability analysis.

\begin{itemize}
    \item \textbf{Target IP Address:} \seqsplit{\texttt{[Target IP]}}
    \item \textbf{Scan Date:} Not provided in scan data. Report generated on \today.
\end{itemize}

\subsection{Open Ports}
The scan identified one open port accessible from the public internet.

\begin{table}[h!]
\centering
\caption{Open Port Findings}
\label{tab:ports}
\begin{tabular}{c c l l}
\toprule
\textbf{Port} & \textbf{State} & \textbf{Service (Inferred)} & \textbf{Notes} \\
\midrule
22/TCP & Open & SSH (Secure Shell) & Remote administrative access. \\
\bottomrule
\end{tabular}
\end{table}

\subsection{Analysis of Findings}
The presence of an open SSH port (22/TCP) on an external-facing IP address is a significant finding. SSH is a common protocol for remote server administration. If not properly configured, it can be a primary vector for attack. Common weaknesses include weak or default passwords, lack of brute-force protection, and outdated software versions with known vulnerabilities. The scan data did not provide version information, so the service could not be checked against known exploits.

% ----------------------------------------------------------------------
% 5. RISK ASSESSMENT
% ----------------------------------------------------------------------
\section{Risk Assessment}

This section synthesizes the findings from the security control review and the technical scan. No pre-existing vulnerabilities were provided for this assessment. The following table summarizes the newly identified risks.

\begin{table}[h!]
\centering
\caption{Summary of Identified Risks}
\label{tab:risks}
\begin{tabular}{p{0.1\textwidth} p{0.25\textwidth} p{0.45\textwidth} c}
\toprule
\textbf{ID} & \textbf{Risk Name} & \textbf{Description} & \textbf{Severity} \\
\midrule
RISK-001 & Lack of Endpoint MFA & No MFA on computer logins allows an attacker with valid credentials to gain initial access to the internal network, bypassing a critical security layer. & \colorbox{sevhigh}{\color{white} High} \\
\addlinespace
RISK-002 & Foundational Policy Gaps & The absence of an AUP and new-hire security training leaves the organization vulnerable to insider threats (both malicious and unintentional) and social engineering attacks. & \colorbox{sevhigh}{\color{white} High} \\
\addlinespace
RISK-003 & Exposed SSH Service & The SSH management port is open to the public internet, making it a target for brute-force attacks and exploitation if any vulnerabilities exist. & \colorbox{sevmedium}{\color{black} Medium} \\
\bottomrule
\end{tabular}
\end{table}

% ----------------------------------------------------------------------
% 6. RECOMMENDATIONS
% ----------------------------------------------------------------------
\section{Recommendations}

The following actions are recommended to mitigate the identified risks and improve the overall security posture of \textbf{[Organization Name]}.

\subsection{Remediation for RISK-001: Lack of Endpoint MFA}
\begin{itemize}
    \item \textbf{Immediate Action:} Prioritize the deployment of a Multi-Factor Authentication solution for all employee computer and laptop logins (e.g., Windows Hello for Business, Duo, Okta).
    \item \textbf{Strategic Goal:} Enforce MFA for all authentication events across the enterprise, especially for privileged access.
\end{itemize}

\subsection{Remediation for RISK-002: Foundational Policy Gaps}
\begin{itemize}
    \item \textbf{Immediate Action:} Develop and implement a formal Acceptable Use Policy (AUP). This policy should be reviewed and signed by all employees.
    \item \textbf{Immediate Action:} Create a mandatory security awareness training module as part of the new employee onboarding process. This training should cover the AUP, phishing identification, password hygiene, and incident reporting.
\end{itemize}

\subsection{Remediation for RISK-003: Exposed SSH Service}
\begin{itemize}
    \item \textbf{Immediate Action:} Conduct an internal review to determine the business justification for the publicly exposed SSH service on \seqsplit{\texttt{[Target IP]}}.
    \item \textbf{If Not Required:} Close port 22/TCP at the network firewall immediately.
    \item \textbf{If Required:}
        \begin{enumerate}
            \item Restrict access to the port at the firewall to only trusted, whitelisted source IP addresses.
            \item Disable password-based authentication and enforce the use of strong SSH key pairs for all logins.
            \item Ensure the SSH service is running the latest patched version and is continuously monitored for anomalous login activity.
        \end{enumerate}
\end{itemize}

% ----------------------------------------------------------------------
% DOCUMENT END
% ----------------------------------------------------------------------
\end{document}
```