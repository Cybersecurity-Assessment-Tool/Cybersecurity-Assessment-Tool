```latex
\documentclass[12pt]{article}

% ----------------------------------------------------------------------
% PREAMBLE
% ----------------------------------------------------------------------
\usepackage[margin=1in]{geometry}
\usepackage{pifont} % For checkmark and x-mark symbols (\ding)
\usepackage{booktabs} % For professional-looking tables
\usepackage{hyperref} % For clickable links and metadata
\usepackage{url}      % For properly formatting URLs
\usepackage{seqsplit} % To split long strings without breaking
\usepackage{graphicx} % For potential logo inclusion
\usepackage{xcolor}   % For color definitions

% Define colors for severity
\definecolor{sev_critical}{HTML}{940000}
\definecolor{sev_high}{HTML}{D14000}
\definecolor{sev_medium}{HTML}{E8A000}

% Hyperref setup for PDF metadata
\hypersetup{
    colorlinks=true,
    linkcolor=blue,
    filecolor=magenta,      
    urlcolor=cyan,
    pdftitle={Cybersecurity Posture Assessment Report},
    pdfauthor={Cybersecurity Analysis Division},
    pdfsubject={Security Assessment},
    pdfkeywords={Cybersecurity, Risk, Assessment, Scan},
    bookmarks=true
}

% Checkmark and X-mark definitions
\newcommand{\cmark}{\ding{51}}
\newcommand{\xmark}{\ding{55}}

% ----------------------------------------------------------------------
% DOCUMENT START
% ----------------------------------------------------------------------
\begin{document}

% ----------------------------------------------------------------------
% TITLE PAGE
% ----------------------------------------------------------------------
\begin{titlepage}
    \centering
    \vspace*{1cm}
    \Huge\textbf{Cybersecurity Posture Assessment Report}
    \vspace{1.5cm}
    \Large
    \textbf{Prepared for:}\\
    \vspace{0.5cm}
    \textbf{[Organization Name]}
    \vspace{2cm}
    \large
    \textbf{Date of Report:}\\
    \vspace{0.5cm}
    \today
    \vfill
    \large
    \textbf{Generated by:}\\
    \vspace{0.5cm}
    Cybersecurity Analysis Division
\end{titlepage}

\tableofcontents
\newpage

% ----------------------------------------------------------------------
% EXECUTIVE SUMMARY
% ----------------------------------------------------------------------
\section*{Executive Summary}

This report details the findings of a cybersecurity posture assessment conducted for \textbf{[Organization Name]}. The assessment combined a review of organizational security controls via a questionnaire, an external network vulnerability scan, and an analysis of pre-existing risks.

The external network scan of the designated target IP address (\texttt{[Target IP]}) revealed a strong perimeter security posture, with no open ports discovered. This is a positive finding and significantly reduces the external attack surface.

However, the review of organizational security controls identified several \textbf{critical} and \textbf{high-risk} deficiencies. The most severe gaps are the complete absence of Multi-Factor Authentication (MFA) for email, computer, and sensitive data access. Additionally, the lack of foundational security policies, such as an Employee Acceptable Use Policy and mandatory security training for new hires, exposes the organization to significant internal and external threats, including credential theft, ransomware, and insider threats.

There were no pre-existing vulnerabilities documented in the provided data, meaning all risks identified in this report are newly discovered. Immediate remediation of the identified policy and access control gaps is strongly recommended to reduce the likelihood of a security incident.

% ----------------------------------------------------------------------
% ORGANIZATIONAL INFORMATION
% ----------------------------------------------------------------------
\section{Organizational Information}

The following information was used as the basis for this assessment. Where data was not provided, placeholders have been used.

\begin{tabular}{@{}ll}
    \toprule
    \textbf{Attribute} & \textbf{Value} \\
    \midrule
    Organization Name & \textbf{[Organization Name]} \\
    Email Domain & \texttt{[Domain]} \\
    Scanned External IP & \texttt{[Client IP]} \\
    \bottomrule
\end{tabular}

% ----------------------------------------------------------------------
% SECURITY CONTROL REVIEW
% ----------------------------------------------------------------------
\section{Security Control Review}

A security questionnaire was completed to evaluate the maturity of existing administrative and technical controls. The responses indicate significant gaps in foundational security practices. A "No" response highlights a missing control that should be addressed.

\begin{tabular}{@{}p{9cm}cc}
    \toprule
    \textbf{Control Question} & \textbf{Response} & \textbf{Assessment} \\
    \midrule
    Do you require MFA to access email? & \xmark & \textcolor{sev_critical}{\textbf{Critical Gap}} \\
    Do you require MFA to log into computers? & \xmark & \textcolor{sev_critical}{\textbf{Critical Gap}} \\
    Do you require MFA to access sensitive data systems? & \xmark & \textcolor{sev_critical}{\textbf{Critical Gap}} \\
    Does your organization have an employee acceptable use policy? & \xmark & \textcolor{sev_high}{\textbf{High Risk}} \\
    Does your organization do security awareness training for new employees? & \xmark & \textcolor{sev_high}{\textbf{High Risk}} \\
    Does your organization do security awareness training for all employees at least once per year? & \cmark & Met \\
    \bottomrule
\end{tabular}

% ----------------------------------------------------------------------
% TECHNICAL SCAN RESULTS
% ----------------------------------------------------------------------
\section{Technical Scan Results}

An external network scan was performed to identify open ports and exposed services.

\subsection{Nmap Scan of \texttt{[Target IP]}}

The scan against the target IP address revealed no open TCP or UDP ports. All 1000 of the most common ports were in a 'closed' state.

\begin{itemize}
    \item \textbf{Target IP:} \texttt{[Target IP]}
    \item \textbf{Scan Date:} Not Provided in Scan Data
    \item \textbf{Status:} Host is Up
    \item \textbf{Key Finding:} No open ports were discovered. This indicates a strong network perimeter security posture for the scanned host, likely due to a well-configured firewall or network access control list (ACL) that effectively limits external exposure.
\end{itemize}

% ----------------------------------------------------------------------
% CONSOLIDATED RISK ASSESSMENT
% ----------------------------------------------------------------------
\section{Consolidated Risk Assessment}

This section correlates the findings from the security control review and technical scan to provide a consolidated list of identified risks. Since no pre-existing vulnerabilities were reported, all risks listed below are new findings.

\begin{tabular}{@{}lp{5.5cm}p{5.5cm}l@{}}
    \toprule
    \textbf{ID} & \textbf{Risk Name} & \textbf{Overview} & \textbf{Severity} \\
    \midrule
    RISK-001 & Lack of Multi-Factor Authentication (MFA) & The absence of MFA for email, computers, and sensitive systems allows an attacker with stolen credentials (e.g., from a phishing attack) to gain immediate, unrestricted access to critical company resources. & \textcolor{sev_critical}{\textbf{Critical}} \\
    \addlinespace
    RISK-002 & Missing Employee Acceptable Use Policy (AUP) & Without a formal AUP, employees lack clear guidance on the secure and acceptable use of company assets. This increases the risk of unintentional data exposure, malware infections, and insider threats. & \textcolor{sev_high}{\textbf{High}} \\
    \addlinespace
    RISK-003 & Inadequate Security Awareness Training Program & Failing to train new employees upon hiring leaves a critical window of vulnerability. New staff are often prime targets for social engineering and may be unaware of security policies and procedures. & \textcolor{sev_high}{\textbf{High}} \\
    \bottomrule
\end{tabular}

% ----------------------------------------------------------------------
% RECOMMENDATIONS
% ----------------------------------------------------------------------
\section{Recommendations}

The following actionable recommendations are provided to mitigate the identified risks. They are prioritized by severity.

\subsection{RISK-001: Implement Multi-Factor Authentication (Critical)}
\begin{itemize}
    \item \textbf{Immediate Action:} Enable and enforce MFA across all email accounts (e.g., Microsoft 365, Google Workspace). This is the single most effective control to prevent account takeovers.
    \item \textbf{Next Steps:} Deploy MFA for all remote access solutions (VPNs) and for logging into company computers (endpoints).
    \item \textbf{Long-Term Goal:} Extend MFA enforcement to all applications and systems that process or store sensitive company data.
\end{itemize}

\subsection{RISK-002: Develop and Implement an Acceptable Use Policy (High)}
\begin{itemize}
    \item \textbf{Immediate Action:} Draft a formal AUP that clearly defines the rules for using company networks, computers, and data.
    \item \textbf{Next Steps:} Have the AUP reviewed by legal and HR departments. Require all current employees to read and formally acknowledge the policy.
    \item \textbf{Long-Term Goal:} Integrate the AUP acknowledgement into the new-hire onboarding process.
\end{itemize}

\subsection{RISK-003: Enhance Security Awareness Training (High)}
\begin{itemize}
    \item \textbf{Immediate Action:} Develop a security awareness training module specifically for new employees and make it a mandatory part of the onboarding process.
    \item \textbf{Next Steps:} Ensure this onboarding training covers key topics such as phishing identification, password hygiene, and the new AUP.
    \item \textbf{Long-Term Goal:} Continue the annual refresher training for all staff to reinforce security concepts and address emerging threats.
\end{itemize}

% ----------------------------------------------------------------------
% CONCLUSION
% ----------------------------------------------------------------------
\section{Conclusion}

The security posture of \textbf{[Organization Name]} presents a sharp contrast between its external network security and its internal administrative controls. While the network perimeter is commendably secure, the lack of fundamental controls like MFA and security policies creates a significant risk of a breach through non-technical means, such as phishing or credential theft.

By implementing the recommendations outlined in this report, particularly the immediate rollout of MFA, the organization can substantially improve its resilience against common cyberattacks and build a stronger, more defensible security foundation.

% ----------------------------------------------------------------------
% DOCUMENT END
% ----------------------------------------------------------------------
\end{document}
```