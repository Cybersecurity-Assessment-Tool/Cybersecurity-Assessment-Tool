```latex
\documentclass[12pt]{article}

% ----------------------------------------------------------------------
% PREAMBLE
% ----------------------------------------------------------------------
\usepackage[margin=1in]{geometry}
\usepackage{pifont} % For checkmarks and crosses
\usepackage{booktabs} % For professional tables
\usepackage{hyperref} % For clickable links
\usepackage{url}      % For URL formatting
\usepackage{seqsplit} % For splitting long strings in tt font
\usepackage[T1]{fontenc}

\hypersetup{
    colorlinks=true,
    linkcolor=black,
    filecolor=magenta,      
    urlcolor=blue,
    pdftitle={Cybersecurity Assessment Report},
    pdfpagemode=FullScreen,
}

\newcommand{\yes}{\ding{51}}
\newcommand{\no}{\ding{55}}

% ----------------------------------------------------------------------
% DOCUMENT START
% ----------------------------------------------------------------------
\begin{document}

% ----------------------------------------------------------------------
% TITLE PAGE
% ----------------------------------------------------------------------
\begin{titlepage}
    \centering
    \vspace*{\fill}
    \huge \textbf{Cybersecurity Assessment Report}
    \vspace{1.5cm}
    \Large \textbf{For: \textbf{[Organization Name]}}
    \vspace{2cm}
    \normalsize
    \begin{tabular}{ll}
        \textbf{Report Date:} & \today \\
        \textbf{Author:} & Cybersecurity Analyst \\
    \end{tabular}
    \vspace*{\fill}
    \textit{This document is confidential and intended for the exclusive use of the recipient.}
\end{titlepage}

\tableofcontents
\newpage

% ----------------------------------------------------------------------
% EXECUTIVE SUMMARY
% ----------------------------------------------------------------------
\section*{1.0 Executive Summary}

This report details the findings of a cybersecurity assessment conducted for \textbf{[Organization Name]}. The evaluation combined a review of organizational security controls, an external network scan, and an analysis of pre-existing risks.

The assessment revealed a mixed security posture. On the technical front, the external network scan of the target system showed a strong perimeter defense, with no open ports detected. This is a positive indicator of a well-configured firewall.

However, significant gaps were identified in administrative and procedural controls. The organization currently lacks a formal Employee Acceptable Use Policy and does not provide security awareness training during the critical new employee onboarding phase. These policy and training deficiencies introduce a high level of risk related to insider threats and human error, which could undermine the effectiveness of existing technical controls.

This report provides a detailed breakdown of these findings and offers actionable recommendations to mitigate the identified risks and strengthen the organization's overall security posture.

% ----------------------------------------------------------------------
% ORGANIZATIONAL INFORMATION
% ----------------------------------------------------------------------
\section*{2.0 Organizational Information}

The following details were used as the basis for this assessment. Due to the anonymized nature of the provided data, placeholders have been used where necessary.

\begin{itemize}
    \item \textbf{Organization Name:} \textbf{[Organization Name]}
    \item \textbf{Primary Email Domain:} \texttt{[Domain]}
    \item \textbf{External IP Address Scanned:} \texttt{[Client IP]}
\end{itemize}

% ----------------------------------------------------------------------
% SECURITY CONTROL REVIEW
% ----------------------------------------------------------------------
\section*{3.0 Security Control Review}

A review of the organization's security controls was conducted via a questionnaire. The responses indicate a strong implementation of Multi-Factor Authentication (MFA) but reveal critical gaps in policy and employee training.

\begin{table}[h!]
\centering
\caption{Security Control Questionnaire Responses}
\begin{tabular}{p{0.75\linewidth} c}
\toprule
\textbf{Control Question} & \textbf{Response} \\
\midrule
Do you require MFA to access email? & \yes \\
Do you require MFA to log into computers? & \yes \\
Do you require MFA to access sensitive data systems? & \yes \\
Does your organization have an employee acceptable use policy? & \no \\
Does your organization do security awareness training for new employees? & \no \\
Does your organization do security awareness training for all employees at least once per year? & \yes \\
\bottomrule
\end{tabular}
\end{table}

The absence of an Acceptable Use Policy and security training for new hires are significant findings that are addressed in the Risk Assessment section of this report.

% ----------------------------------------------------------------------
% TECHNICAL SCAN RESULTS
% ----------------------------------------------------------------------
\section*{4.0 Technical Scan Results}

An external network vulnerability scan was performed to identify potential weaknesses in the organization's internet-facing infrastructure.

\subsection*{4.1 Scan Details}
\begin{itemize}
    \item \textbf{Target IP Address:} \texttt{[Target IP]}
    \item \textbf{Scan Date:} 2023-10-27 (Placeholder Date)
\end{itemize}

\subsection*{4.2 Findings}
The scan completed successfully and found \textbf{no open ports} on the target system. This result strongly suggests that a well-configured firewall or other network security appliance is in place, effectively blocking unsolicited inbound traffic. This is a commendable security practice that significantly reduces the external attack surface.

% ----------------------------------------------------------------------
% RISK ASSESSMENT
% ----------------------------------------------------------------------
\section*{5.0 Risk Assessment}

This section synthesizes findings from the security control review, technical scan, and pre-existing risk data. While no technical vulnerabilities were discovered during the scan and no pre-existing vulnerabilities were reported, the procedural gaps identified present a tangible threat to the organization.

\begin{table}[h!]
\centering
\caption{Identified Risks}
\begin{tabular}{p{0.15\linewidth} p{0.55\linewidth} l}
\toprule
\textbf{Risk ID} & \textbf{Description} & \textbf{Severity} \\
\midrule
\textbf{RISK-001} & \textbf{Lack of Employee Acceptable Use Policy (AUP):} Without a formal AUP, employees lack clear guidelines on the acceptable use of company assets, data handling, and security responsibilities. This ambiguity can lead to unintentional data breaches and non-compliance. & \textbf{Critical} \\
\addlinespace
\textbf{RISK-002} & \textbf{No Security Awareness Training for New Employees:} New hires are a common target for social engineering attacks. The absence of mandatory security training during onboarding leaves them unprepared to identify and respond to threats, creating a significant window of vulnerability. & \textbf{High} \\
\bottomrule
\end{tabular}
\end{table}

% ----------------------------------------------------------------------
% RECOMMENDATIONS
% ----------------------------------------------------------------------
\section*{6.0 Recommendations}

The following actions are recommended to mitigate the identified risks and improve the overall security posture of \textbf{[Organization Name]}.

\subsection*{6.1 For RISK-001: Lack of AUP}
\begin{itemize}
    \item \textbf{Action:} Develop and implement a comprehensive Acceptable Use Policy (AUP). This policy should clearly define rules for computer, network, email, and internet usage, as well as guidelines for data protection and incident reporting.
    \item \textbf{Implementation:} The policy should be reviewed by legal counsel, communicated to all current employees, and integrated into the new employee onboarding process. All employees should be required to formally acknowledge that they have read and understood the policy.
\end{itemize}

\subsection*{6.2 For RISK-002: No New Employee Training}
\begin{itemize}
    \item \textbf{Action:} Establish a mandatory security awareness training module as a required component of the new employee onboarding process. This training should occur before or within the first week of employment.
    \item \textbf{Implementation:} The training should cover key topics such as phishing identification, password security, safe browsing habits, and the new Acceptable Use Policy. This initial training will complement the existing annual refresher course and ensure a consistent security baseline for all staff from day one.
\end{itemize}

% ----------------------------------------------------------------------
% CONCLUSION
% ----------------------------------------------------------------------
\section*{7.0 Conclusion}

The security assessment of \textbf{[Organization Name]} highlights a robust technical perimeter but reveals critical weaknesses in administrative security controls. The absence of an Acceptable Use Policy and the lack of security training for new hires represent the most significant risks.

By implementing the recommendations outlined in this report—namely, creating a formal AUP and integrating security training into the employee onboarding process—the organization can substantially reduce its exposure to human-factor risks and build a more resilient and security-conscious culture.

% ----------------------------------------------------------------------
% DOCUMENT END
% ----------------------------------------------------------------------
\end{document}
```