```latex
\documentclass[12pt]{article}

% Preamble: Required Packages
\usepackage[margin=1in]{geometry}
\usepackage{pifont} % For checkmarks and crosses (\ding{51}, \ding{55})
\usepackage{booktabs} % For professional-looking tables
\usepackage{hyperref} % For clickable links and ToC
\usepackage{url} % For formatting URLs
\usepackage{seqsplit} % For splitting long strings without spaces
\usepackage{graphicx}
\usepackage{xcolor}

% Hyperref Setup
\hypersetup{
    colorlinks=true,
    linkcolor=blue,
    filecolor=magenta,      
    urlcolor=cyan,
    pdftitle={Cybersecurity Posture Report},
    pdfpagemode=FullScreen,
}

% Document Metadata
\title{Cybersecurity Posture Report}
\author{Cybersecurity Analysis Division}
\date{November 22, 2025}

\begin{document}

\maketitle
\thispagestyle{empty}
\newpage

\tableofcontents
\thispagestyle{empty}
\newpage

\setcounter{page}{1}

% --- Section 1: Executive Summary ---
\section{Executive Summary}

This report provides a comprehensive analysis of the cybersecurity posture for \textbf{[Organization Name]}. The assessment was conducted on November 22, 2025, and synthesizes data from an external network scan, a security controls questionnaire, and a review of pre-existing risks.

\paragraph{Key Findings:}
The organization demonstrates a strong foundation in administrative and user-level security controls. The mandatory use of Multi-Factor Authentication (MFA) for email, computer logins, and sensitive systems, combined with a robust security awareness training program, significantly reduces the risk of account compromise and social engineering attacks.

However, a critical technical vulnerability was identified during the external network scan. The public-facing web server at \texttt{[Target IP]} is running an outdated version of Nginx (1.18.0). This software version has multiple publicly disclosed vulnerabilities, exposing the organization to potential remote exploitation, service disruption, or data breach.

\paragraph{Overall Posture:}
While the organization's internal policies and user controls are commendable, the external-facing technical vulnerability presents a \textbf{High} level of risk that requires immediate attention. Prioritizing the remediation of this finding is crucial to maintaining a secure operational environment.

% --- Section 2: Organizational Information ---
\section{Organizational Information}

This section details the information provided by the client organization for this assessment.

\begin{table}[h!]
\centering
\begin{tabular}{@{}ll@{}}
\toprule
\textbf{Attribute} & \textbf{Value} \\ \midrule
Organization Name    & \textbf{[Organization Name]} \\
Primary Email Domain & \texttt{[Domain]} \\
External IP Address  & \texttt{[Client IP]} \\ \bottomrule
\end{tabular}
\caption{Client Organizational Details.}
\label{tab:org_info}
\end{table}

% --- Section 3: Security Control Review ---
\section{Security Control Review (Questionnaire)}

The following table summarizes the organization's self-reported security controls. The responses indicate a mature approach to endpoint and identity security. A green checkmark (\ding{51}) signifies a positive control is in place, while a red cross (\ding{55}) would indicate a gap.

\begin{table}[h!]
\centering
\begin{tabular}{@{}lc@{}}
\toprule
\textbf{Security Control Question} & \textbf{Status} \\ \midrule
Do you require MFA to access email? & \textcolor{green}{\ding{51}} \\
Do you require MFA to log into computers? & \textcolor{green}{\ding{51}} \\
Do you require MFA to access sensitive data systems? & \textcolor{green}{\ding{51}} \\
Does your organization have an employee acceptable use policy? & \textcolor{green}{\ding{51}} \\
Does your organization do security awareness training for new employees? & \textcolor{green}{\ding{51}} \\
Does your organization do security awareness training annually? & \textcolor{green}{\ding{51}} \\ \bottomrule
\end{tabular}
\caption{Self-Assessed Security Controls.}
\label{tab:controls}
\end{table}

\paragraph{Analysis:} All questionnaire responses were affirmative, indicating that strong foundational security controls are established and enforced. This is a significant strength in the organization's overall security posture.

% --- Section 4: Technical Scan Results ---
\section{Technical Scan Results}

An external network scan was performed to identify open ports and services visible on the public internet.

\subsection{Scan Metadata}
\begin{itemize}
    \item \textbf{Target IP Address:} \texttt{[Target IP]}
    \item \textbf{Scan Date:} 2025-11-22T10:00:00Z
\end{itemize}

\subsection{Open Ports and Services}
A single open port was discovered on the target system. The details are provided in Table \ref{tab:scan_results}.

\begin{table}[h!]
\centering
\begin{tabular}{@{}lllll@{}}
\toprule
\textbf{Port} & \textbf{State} & \textbf{Service} & \textbf{Product} & \textbf{Version} \\ \midrule
443/tcp & open & https & nginx & 1.18.0 \\ \bottomrule
\end{tabular}
\caption{Network Scan Findings.}
\label{tab:scan_results}
\end{table}

\paragraph{Analysis:} The scan identified an Nginx web server on port 443 (HTTPS). The detected version, \textbf{1.18.0}, was released in April 2020 and is now considered outdated. It is known to be vulnerable to several security issues, including CVE-2021-23017, which could allow an attacker to bypass security restrictions. This represents a significant external attack vector.

% --- Section 5: Risk Assessment ---
\section{Risk Assessment}

This section correlates the findings from the security control review and the technical scan to provide a consolidated list of identified risks. The pre-existing risk register was empty, so all risks listed below are new findings from this assessment.

\begin{table}[h!]
\centering
\begin{tabular}{@{}p{0.15\linewidth}p{0.25\linewidth}p{0.4\linewidth}p{0.1\linewidth}@{}}
\toprule
\textbf{Risk ID} & \textbf{Risk Name} & \textbf{Overview} & \textbf{Severity} \\ \midrule
CYB-RISK-001 & Outdated External Web Server & The public-facing web server at \texttt{[Target IP]} is running Nginx 1.18.0, an outdated version with multiple publicly disclosed vulnerabilities. This could allow an attacker to compromise the server. & \textbf{High} \\ \bottomrule
\end{tabular}
\caption{Consolidated Risk Register.}
\label{tab:risks}
\end{table}

% --- Section 6: Recommendations ---
\section{Recommendations}

Based on the analysis, the following prioritized recommendations are provided to enhance the organization's security posture.

\subsection{High Priority}
\begin{itemize}
    \item \textbf{Upgrade Nginx Server:} The Nginx instance on host \texttt{[Target IP]} must be upgraded from version 1.18.0 to the latest stable version immediately. This action will mitigate known vulnerabilities and is the most critical step to reduce the organization's external attack surface.
\end{itemize}

\subsection{Medium Priority}
\begin{itemize}
    \item \textbf{Implement Vulnerability Management:} Establish a formal vulnerability management program. This should include regular, automated scans of all external-facing assets to proactively identify and remediate outdated software, misconfigurations, and other security weaknesses before they can be exploited.
\end{itemize}

\subsection{Low Priority}
\begin{itemize}
    \item \textbf{Maintain and Review Controls:} Continue to enforce and periodically review the excellent administrative controls already in place, including MFA policies and the security awareness training program. Ensure these policies are updated to reflect evolving threats.
\end{itemize}

\end{document}
```