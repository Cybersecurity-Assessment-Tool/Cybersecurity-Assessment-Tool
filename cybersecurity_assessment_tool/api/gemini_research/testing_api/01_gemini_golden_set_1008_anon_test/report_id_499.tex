```latex
\documentclass[12pt]{article}

% Preamble: Required Packages
\usepackage[margin=1in]{geometry}
\usepackage{pifont} % For \ding
\usepackage{booktabs} % For professional tables
\usepackage{hyperref} % For clickable links
\usepackage{url} % For URL formatting
\usepackage{seqsplit} % For splitting long strings
\usepackage[utf8]{inputenc}

% Document Metadata
\title{Cybersecurity Posture Assessment Report}
\author{Cybersecurity Analyst}
\date{\today}

\hypersetup{
    colorlinks=true,
    linkcolor=black,
    urlcolor=blue,
    pdftitle={Cybersecurity Posture Assessment Report},
    pdfauthor={Cybersecurity Analyst},
}

\begin{document}

\maketitle
\thispagestyle{empty}
\newpage
\tableofcontents
\newpage

\section{Executive Summary}

This report provides a comprehensive analysis of the cybersecurity posture for \textbf{[Organization Name]}. The assessment is based on a synthesis of network scan data, a review of organizational security controls, and an evaluation of pre-existing risks.

The analysis identified several critical and high-risk findings that require immediate attention. Key areas of concern include significant gaps in the implementation of Multi-Factor Authentication (MFA) for sensitive systems and a complete lack of a formal security awareness training program for employees.

Furthermore, technical scanning revealed a publicly exposed Secure Shell (SSH) service. While necessary for remote administration, its exposure to the public internet without proper hardening presents a significant attack vector.

This report outlines these findings in detail and provides a series of actionable recommendations designed to mitigate the identified risks and strengthen the organization's overall security posture. Addressing these issues proactively is crucial to protecting sensitive data and maintaining operational integrity.

\section{Organizational Information}

This section details the information provided by the client for this assessment.
\begin{itemize}
    \item \textbf{Organization Name:} \textbf{[Organization Name]}
    \item \textbf{Primary Domain:} \texttt{[Domain]}
    \item \textbf{External IP Scanned:} \texttt{[Client IP]}
\end{itemize}

\section{Security Control Review}

The following table summarizes the organization's responses to a security controls questionnaire. The responses were evaluated against industry best practices. Gaps, indicated by a \ding{55}, often correspond to significant security risks.

\begin{table}[h!]
\centering
\caption{Organizational Security Controls Questionnaire}
\label{tab:controls}
\begin{tabular}{p{11cm}c}
\toprule
\textbf{Control Question} & \textbf{Response} \\
\midrule
Do you require MFA to access email? & \ding{51} \\
Do you require MFA to log into computers? & \ding{51} \\
\textbf{Do you require MFA to access sensitive data systems?} & \textbf{\ding{55}} \\
Does your organization have an employee acceptable use policy? & \ding{51} \\
\textbf{Does your organization do security awareness training for new employees?} & \textbf{\ding{55}} \\
\textbf{Does your organization do security awareness training for all employees at least once per year?} & \textbf{\ding{55}} \\
\bottomrule
\end{tabular}
\end{table}

\subsection*{Analysis of Control Gaps}
\begin{itemize}
    \item \textbf{MFA for Sensitive Systems:} The absence of MFA on sensitive data systems is a critical vulnerability. Should an attacker compromise a user's credentials, they would have direct access to the organization's most valuable data.
    \item \textbf{Security Awareness Training:} The lack of any security awareness training program, both at onboarding and annually, is a high-risk gap. Employees are the first line of defense, and without training, they are significantly more susceptible to phishing, social engineering, and other common attack vectors.
\end{itemize}

\section{Technical Scan Results}

A network scan was performed on the organization's external infrastructure to identify open ports and exposed services.

\begin{itemize}
    \item \textbf{Target IP Address:} \texttt{[Target IP]}
    \item \textbf{Scan Date:} \today
\end{itemize}

The following table details the services found to be accessible from the public internet.

\begin{table}[h!]
\centering
\caption{Open Ports and Services}
\label{tab:ports}
\begin{tabular}{l l l l}
\toprule
\textbf{Port} & \textbf{State} & \textbf{Service} & \textbf{Product / Version} \\
\midrule
22/tcp & open & ssh & Not Disclosed \\
\bottomrule
\end{tabular}
\end{table}

\subsection*{Analysis of Technical Findings}
The scan identified that port 22 (SSH) is open to the internet. Exposing SSH directly is a high-risk configuration, as it is a primary target for automated brute-force attacks. Without version information, it is not possible to determine if the service is vulnerable to known exploits, but the exposure itself constitutes a significant risk.

\section{Risk Assessment}

This section correlates the findings from the security control review and the technical scan into a prioritized list of risks. No pre-existing vulnerabilities were reported.

\begin{table}[h!]
\centering
\caption{Summary of Identified Risks}
\label{tab:risks}
\begin{tabular}{l p{8cm} l}
\toprule
\textbf{Risk ID} & \textbf{Risk Description} & \textbf{Severity} \\
\midrule
RISK-001 & Lack of MFA on sensitive data systems allows for single-factor credential compromise. & Critical \\
\addlinespace
RISK-002 & Exposed SSH service on the external perimeter is a target for brute-force attacks. & High \\
\addlinespace
RISK-003 & No security awareness training program, increasing susceptibility to phishing and social engineering. & High \\
\bottomrule
\end{tabular}
\end{table}

\section{Recommendations}

The following actions are recommended to mitigate the identified risks and improve the overall security posture of \textbf{[Organization Name]}.

\begin{enumerate}
    \item \textbf{Implement MFA on Sensitive Systems (RISK-001):}
    \begin{itemize}
        \item \textbf{Immediate Action:} Prioritize the deployment of a robust Multi-Factor Authentication solution for all access to systems identified as containing sensitive or critical data.
        \item \textbf{Strategic Goal:} Enforce MFA as a default security policy for all new systems and applications.
    \end{itemize}
    \vspace{1em}
    \item \textbf{Secure the Exposed SSH Service (RISK-002):}
    \begin{itemize}
        \item \textbf{Immediate Action:} If public access is not required, restrict SSH access via firewall rules to only trusted IP addresses (e.g., corporate VPN, administrator locations).
        \item \textbf{Best Practice:} If public access is necessary, enforce the use of public key authentication and disable password-based logins entirely. Additionally, implement an intrusion prevention tool like Fail2Ban to automatically block IPs that exhibit malicious behavior.
    \end{itemize}
    \vspace{1em}
    \item \textbf{Establish a Security Awareness Training Program (RISK-003):}
    \begin{itemize}
        \item \textbf{Immediate Action:} Procure and implement a security awareness training solution. Enroll all current employees in foundational training covering topics like phishing, password hygiene, and acceptable use.
        \item \textbf{Strategic Goal:} Integrate security awareness training into the new employee onboarding process. Conduct mandatory, annual refresher training and periodic phishing simulations for all staff to ensure continuous education.
    \end{itemize}
\end{enumerate}

\end{document}
```