```latex
\documentclass[12pt, a4paper]{article}

% Preamble: Required Packages
\usepackage[margin=1in]{geometry}
\usepackage{pifont} % For checkmarks and crosses
\usepackage{booktabs} % For professional tables
\usepackage{hyperref} % For clickable links
\usepackage{url} % For formatting URLs
\usepackage{seqsplit} % For splitting long strings to prevent overflow
\usepackage{graphicx}
\usepackage{xcolor}
\usepackage{fancyhdr}
\usepackage{lastpage}

% --- Document Setup ---

% Define colors for severity
\definecolor{criticalred}{HTML}{D73B3E}
\definecolor{highorange}{HTML}{F5A623}
\definecolor{mediumyellow}{HTML}{F8E71C}
\definecolor{lowblue}{HTML}{4A90E2}
\definecolor{infogray}{HTML}{9B9B9B}

% Hyperlink setup
\hypersetup{
    colorlinks=true,
    linkcolor=blue,
    filecolor=magenta,      
    urlcolor=cyan,
    pdftitle={Cybersecurity Posture Report},
    pdfpagemode=FullScreen,
}

% Header and Footer
\pagestyle{fancy}
\fancyhf{} % clear all header and footer fields
\fancyhead[L]{Cybersecurity Posture Report}
\fancyhead[R]{\textbf{[Organization Name]}}
\fancyfoot[C]{Page \thepage\ of \pageref{LastPage}}
\renewcommand{\headrulewidth}{0.4pt}
\renewcommand{\footrulewidth}{0.4pt}

% --- Document Start ---
\begin{document}

% --- Title Page ---
\begin{titlepage}
    \centering
    \vspace*{1cm}
    \includegraphics[width=0.4\textwidth]{example-image-a} % Placeholder logo
    \vfill
    \huge\textbf{Cybersecurity Posture Report}
    \vspace{1.5cm}
    \Large
    \textbf{Prepared for:}\\
    \vspace{0.5cm}
    \textbf{[Organization Name]}\\
    \vspace{2cm}
    \textbf{Date of Report:}\\
    \vspace{0.5cm}
    \today
    \vfill
    \small
    This report is confidential and intended solely for the use of \textbf{[Organization Name]}.
\end{titlepage}

\tableofcontents
\newpage

% --- Section 1: Executive Summary ---
\section{Executive Summary}
This report provides a comprehensive analysis of the cybersecurity posture for \textbf{[Organization Name]}, synthesizing data from a technical network scan, a security controls questionnaire, and a review of pre-existing risks. The assessment reveals several critical and high-risk gaps in the organization's security framework that require immediate attention.

The most significant findings are severe deficiencies in identity and access management. The complete absence of Multi-Factor Authentication (MFA) for email, computer logins, and sensitive data systems exposes the organization to a high risk of account compromise and unauthorized access.

Furthermore, foundational governance controls are lacking. The absence of an employee acceptable use policy and a mandatory annual security awareness training program significantly weakens the organization's "human firewall," making it more susceptible to social engineering and phishing attacks.

On a technical level, the network scan of the target host \texttt{[Target IP]} did not identify any open ports or active services, which is a positive finding. However, this result directly contradicts a pre-existing risk entry that indicated an "Unencrypted Web Server" on Port 80 was active. This discrepancy must be investigated to ensure the accuracy of the organization's risk register.

Recommendations are prioritized to address the most critical risks first, focusing on the immediate implementation of MFA and the development of essential security policies and training programs.

% --- Section 2: Organizational Information ---
\section{Organizational Information}
This section outlines the basic information provided for the assessment. The data has been anonymized as per the engagement requirements.

\begin{table}[h!]
\centering
\begin{tabular}{@{}ll@{}}
\toprule
\textbf{Attribute} & \textbf{Value} \\ \midrule
Organization Name & \textbf{[Organization Name]} \\
Primary Email Domain & \texttt{[Domain]} \\
Client Primary IP & \texttt{[Client IP]} \\
Target of Network Scan & \texttt{[Target IP]} \\ \bottomrule
\end{tabular}
\caption{Client and Assessment Scope Information.}
\label{tab:org_info}
\end{table}

% --- Section 3: Security Control Review ---
\section{Security Control Review}
The following table summarizes the organization's responses to a security controls questionnaire. A \textcolor{green}{\ding{51}} indicates a positive control is in place, while a \textcolor{red}{\ding{55}} indicates a control gap.

\begin{table}[h!]
\centering
\begin{tabular}{@{}lc@{}}
\toprule
\textbf{Control Question} & \textbf{Status} \\ \midrule
Do you require MFA to access email? & \textcolor{red}{\ding{55}} \\
Do you require MFA to log into computers? & \textcolor{red}{\ding{55}} \\
Do you require MFA to access sensitive data systems? & \textcolor{red}{\ding{55}} \\
Does your organization have an employee acceptable use policy? & \textcolor{red}{\ding{55}} \\
Does your organization do security awareness training for new employees? & \textcolor{green}{\ding{51}} \\
Does your organization do security awareness training for all employees at least once per year? & \textcolor{red}{\ding{55}} \\ \bottomrule
\end{tabular}
\caption{Security Controls Questionnaire Results.}
\label{tab:controls}
\end{table}

\subsection*{Analysis of Control Gaps}
The questionnaire reveals critical deficiencies in both technical and administrative controls:
\begin{itemize}
    \item \textbf{Multi-Factor Authentication (MFA):} The lack of MFA across all key areas (email, endpoints, data systems) is a critical vulnerability. Stolen credentials alone would be sufficient for an attacker to gain widespread access.
    \item \textbf{Governance and Policy:} The absence of an Acceptable Use Policy means there are no formal rules governing how employees use company assets, increasing the risk of misuse.
    \item \textbf{Security Training:} While new-hire training is a good first step, the lack of annual refresher training allows security knowledge to become stale, reducing its effectiveness over time.
\end{itemize}

% --- Section 4: Technical Scan Results ---
\section{Technical Scan Results}
A network scan was performed on the target host \texttt{[Target IP]} to identify exposed services.

\subsection*{Scan Summary}
The scan revealed that the target host is online, but no open ports were discovered. This indicates a well-configured firewall at the perimeter of this specific host, which is a positive security finding.

\begin{table}[h!]
\centering
\begin{tabular}{@{}llll@{}}
\toprule
\textbf{Port} & \textbf{State} & \textbf{Service} & \textbf{Version} \\ \midrule
80/tcp & closed & http & N/A \\ \bottomrule
\end{tabular}
\caption{Nmap Scan Results for Target: \texttt{[Target IP]}.}
\label{tab:nmap_results}
\end{table}

\subsection*{Discrepancy with Existing Risk Data}
It is crucial to note that this scan result conflicts with the pre-existing risk data provided in \texttt{Input\_3\_Current\_Risks\_JSON}, which listed a vulnerability related to Port 80 being open. The current scan shows this port is \textbf{closed}. This could be due to:
\begin{enumerate}
    \item The vulnerability has been successfully remediated since the last assessment.
    \item The scan was performed on a different or incorrect target IP address.
    \item The initial risk data was inaccurate.
\end{enumerate}
This discrepancy requires further investigation to validate the current state and ensure the risk register is accurate.

% --- Section 5: Risk Assessment ---
\section{Risk Assessment}
This section consolidates findings from all data sources into a prioritized list of risks.

\begin{table}[h!]
\centering
\begin{tabular}{@{}p{0.1\linewidth} p{0.3\linewidth} p{0.45\linewidth}@{}}
\toprule
\textbf{Severity} & \textbf{Risk Title} & \textbf{Description} \\ \midrule
\colorbox{criticalred}{\color{white}\textbf{CRITICAL}} & \textbf{Lack of Multi-Factor Authentication (MFA)} & The absence of MFA for email, computer, and sensitive data access creates a single point of failure (passwords) for authentication, exposing the organization to severe risk from credential theft and account takeovers. \\
\addlinespace
\colorbox{highorange}{\color{white}\textbf{HIGH}} & \textbf{Inadequate Security Policies \& Training} & The lack of an Acceptable Use Policy and annual security awareness training results in a workforce that is less prepared to identify and resist social engineering, phishing, and insider threats. \\
\addlinespace
\colorbox{infogray}{\color{white}\textbf{INFO}} & \textbf{Unconfirmed Risk: Unencrypted Web Server} & A pre-existing risk indicated Port 80 was open. Our scan found it closed. This risk is marked as unconfirmed pending investigation to validate the asset's current state and update the risk register. \\
\bottomrule
\end{tabular}
\caption{Consolidated Risk Summary.}
\label{tab:risk_summary}
\end{table}

% --- Section 6: Recommendations ---
\section{Recommendations}
The following actionable recommendations are provided to mitigate the identified risks. They are prioritized based on severity and potential impact.

\subsection*{Immediate Priority (Critical Risks)}
\begin{description}
    \item[R-01: Implement Multi-Factor Authentication (MFA)] \hfill \\
    \textbf{Action:} Deploy and mandate the use of MFA across all critical platforms.
    \begin{itemize}
        \item \textbf{Phase 1 (Immediate):} Enforce MFA on all email accounts (e.g., Office 365, Google Workspace) and any external-facing services like VPNs.
        \item \textbf{Phase 2 (Next 30 days):} Extend MFA requirements to all employee computer logins and access to systems containing sensitive or regulated data.
    \end{itemize}
\end{description}

\subsection*{High Priority (High Risks)}
\begin{description}
    \item[R-02: Develop and Enforce an Acceptable Use Policy (AUP)] \hfill \\
    \textbf{Action:} Create a formal AUP that clearly defines the rules for using company networks, devices, and data. Require all current employees to read and acknowledge the policy, and integrate this step into the new-hire onboarding process.

    \item[R-03: Establish Annual Security Awareness Training] \hfill \\
    \textbf{Action:} Implement a mandatory, annual security awareness training program for all employees. The training should cover modern threats such as phishing, ransomware, and social engineering. Track completion to ensure 100\% compliance.
\end{description}

\subsection*{Administrative Priority}
\begin{description}
    \item[R-04: Validate and Update Risk Register] \hfill \\
    \textbf{Action:} Investigate the discrepancy regarding the "Unencrypted Web Server" risk. Confirm whether the vulnerability on Port 80 has been remediated or if the asset was misidentified. Update the organization's central risk register with the correct information to ensure an accurate security posture overview.
\end{description}

\end{document}
```