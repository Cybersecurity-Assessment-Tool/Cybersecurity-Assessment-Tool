```latex
\documentclass[12pt]{article}

% Preamble: Required Packages
\usepackage[margin=1in]{geometry}
\usepackage{pifont} % For checkmarks and crosses
\usepackage{booktabs} % For professional tables
\usepackage{hyperref} % For clickable links
\usepackage{url} % For URL formatting
\usepackage{seqsplit} % For splitting long strings
\usepackage{graphicx}
\usepackage{xcolor}

% Hyperref Setup
\hypersetup{
    colorlinks=true,
    linkcolor=blue,
    filecolor=magenta,      
    urlcolor=cyan,
}

% Define checkmark and crossmark for convenience
\newcommand{\cmark}{\ding{51}}
\newcommand{\xmark}{\ding{55}}

\begin{document}

% --- Title Section ---
\begin{center}
    \vspace*{1cm}
    \includegraphics[width=4cm]{example-image-a} % Placeholder for a logo
    \vspace*{1cm}
    
    \huge{\textbf{Cybersecurity Posture Assessment Report}}
    \vspace{1.5cm}
    
    \large
    \begin{tabular}{ll}
        \textbf{Client:} & \textbf{[Organization Name]} \\
        \textbf{Date of Report:} & \today \\
        \textbf{Date of Scan:} & Not Specified \\ % Placeholder as scan_date was not in the provided JSON
    \end{tabular}
    \vspace{2cm}
    
    \hrule
    \vspace{0.5cm}
    \textbf{CONFIDENTIAL}
    \vspace{0.5cm}
    \hrule
\end{center}

\newpage

% --- Table of Contents ---
\tableofcontents
\newpage

% --- Executive Summary ---
\section{Executive Summary}
This report details the findings of a cybersecurity assessment conducted for \textbf{[Organization Name]}. The analysis combines a review of organizational security controls, an external network scan, and an evaluation of pre-existing risk data.

The assessment identified several critical and high-risk vulnerabilities that require immediate attention. The most significant findings include a systemic lack of Multi-Factor Authentication (MFA) across key systems including email, workstations, and sensitive data repositories. This gap dramatically increases the risk of unauthorized access and account compromise.

Furthermore, technical scanning revealed an open HTTP port (\texttt{80/tcp}) on the target system \texttt{[Target IP]}, indicating that web traffic is being transmitted in cleartext. This exposes the organization to data interception and man-in-the-middle attacks.

While the organization demonstrates a solid foundation in security policy and awareness training, the identified technical and access control weaknesses present a significant and immediate threat. We strongly recommend prioritizing the remediation steps outlined in Section \ref{sec:recommendations} to mitigate these risks and improve the overall security posture.

% --- Organizational Information ---
\section{Organizational Information}
This section provides the context for the assessment based on the information provided.
\begin{itemize}
    \item \textbf{Organization Name:} \textbf{[Organization Name]}
    \item \textbf{Primary Domain:} \texttt{[Domain]}
    \item \textbf{Scanned External IP:} \texttt{[Client IP]}
\end{itemize}

% --- Security Control Review ---
\section{Security Control Review}
The following table summarizes the organization's responses to a security controls questionnaire. "No" responses indicate potential gaps in the security framework and are highlighted for review.

\begin{table}[h!]
\centering
\caption{Security Controls Questionnaire Results}
\begin{tabular}{p{0.75\linewidth} c}
\toprule
\textbf{Control Question} & \textbf{Response} \\
\midrule
Do you require MFA to access email? & \textcolor{red}{\xmark} \\
Do you require MFA to log into computers? & \textcolor{red}{\xmark} \\
Do you require MFA to access sensitive data systems? & \textcolor{red}{\xmark} \\
Does your organization have an employee acceptable use policy? & \textcolor{green}{\cmark} \\
Does your organization do security awareness training for new employees? & \textcolor{green}{\cmark} \\
Does your organization do security awareness training for all employees at least once per year? & \textcolor{green}{\cmark} \\
\bottomrule
\end{tabular}
\end{table}

\subsection*{Analysis}
The questionnaire reveals a critical weakness in access control management. The absence of MFA for email, computer logins, and sensitive systems is a major security risk. A threat actor who obtains a user's credentials (e.g., through phishing) would be able to gain direct access to these critical assets without needing a second authentication factor. While the organization's commitment to policy and training is commendable, it is not a sufficient control to prevent credential-based attacks.

% --- Technical Scan Results ---
\section{Technical Scan Results}
An external network scan was performed on the target IP address to identify open ports and exposed services.

\begin{table}[h!]
\centering
\caption{Open Port Analysis for Target: \texttt{[Target IP]}}
\begin{tabular}{llll}
\toprule
\textbf{Port} & \textbf{State} & \textbf{Service} & \textbf{Product / Version} \\
\midrule
80/tcp & open & http & Not Identified \\
\bottomrule
\end{tabular}
\end{table}

\subsection*{Analysis}
The scan identified that port \textbf{80 (HTTP)} is open to the internet. The HTTP protocol transmits data in cleartext, meaning that any information, including usernames, passwords, or session cookies, sent between a user and the server can be intercepted and read by an attacker on the same network. This is a significant security risk that undermines data confidentiality and integrity. The standard best practice is to use HTTPS (port 443), which encrypts traffic using TLS/SSL.

% --- Consolidated Risk Assessment ---
\section{Consolidated Risk Assessment}
The following table synthesizes findings from the security control review, technical scan, and pre-existing risk data into a consolidated list of identified risks.

\begin{table}[h!]
\centering
\caption{Summary of Identified Risks}
\begin{tabular}{p{0.25\linewidth} p{0.5\linewidth} l}
\toprule
\textbf{Risk Name} & \textbf{Description} & \textbf{Severity} \\
\midrule
\textbf{Lack of Multi-Factor Authentication (MFA)} & The absence of MFA for email, computer, and sensitive system access allows for single-factor authentication, making account compromise trivial if credentials are stolen. & \textbf{Critical} \\
\addlinespace
\textbf{Unencrypted Web Traffic (HTTP)} & The active HTTP service on port 80 transmits data in cleartext, exposing it to eavesdropping and man-in-the-middle attacks. & \textbf{High} \\
\addlinespace
\textbf{System Overriden} & An anomalous entry was found in the risk register with a CVSS score of 0.0. This may indicate a data integrity issue within the risk tracking system itself. & \textbf{Informational} \\
\bottomrule
\end{tabular}
\end{table}

% --- Recommendations ---
\section{Recommendations}
\label{sec:recommendations}
Based on the consolidated risk assessment, we provide the following prioritized recommendations to enhance the organization's security posture.

\subsection{Priority 1 (Critical): Implement MFA}
\begin{itemize}
    \item \textbf{Action:} Enforce MFA across all user accounts.
    \item \textbf{Details:} Prioritize the rollout of MFA for all remote access solutions (VPN), email platforms (e.g., Office 365, Google Workspace), and any systems containing sensitive or regulated data. This is the single most effective control to prevent unauthorized access from compromised credentials.
\end{itemize}

\subsection{Priority 2 (High): Remediate Unencrypted Services}
\begin{itemize}
    \item \textbf{Action:} Decommission HTTP in favor of HTTPS.
    \item \textbf{Details:} The web service running on port 80 should be reconfigured to use HTTPS on port 443 exclusively. A valid TLS certificate must be installed. Configure the server to automatically redirect all HTTP requests to their HTTPS equivalent to ensure all traffic is encrypted.
\end{itemize}

\subsection{Priority 3 (Informational): Review Risk Register Integrity}
\begin{itemize}
    \item \textbf{Action:} Investigate the anomalous risk entry.
    \item \textbf{Details:} Review the source of the "System Overriden" risk entry. While rated as informational due to its 0.0 CVSS score, its content is unusual and warrants an investigation to ensure the risk management process and its data sources are functioning correctly and have not been tampered with.
\end{itemize}

\end{document}
```