```latex
\documentclass[12pt]{article}

% Preamble: Required Packages
\usepackage[margin=1in]{geometry} % Set page margins
\usepackage{pifont}               % For checkmarks and crosses (\ding)
\usepackage{booktabs}             % For professional-looking tables
\usepackage[hidelinks]{hyperref}  % For clickable links without boxes
\usepackage{url}                  % For formatting URLs
\usepackage{seqsplit}             % For splitting long strings in texttt
\usepackage{graphicx}             % For logos (placeholder)
\usepackage{fancyhdr}             % For headers and footers

% --- Document Metadata ---
\title{Cybersecurity Posture Assessment Report}
\author{Cybersecurity Analysis Division}
\date{\today}

% --- Header and Footer Configuration ---
\pagestyle{fancy}
\fancyhf{} % Clear all header and footer fields
\fancyhead[L]{Cybersecurity Assessment for \textbf{[Organization Name]}}
\fancyfoot[C]{\thepage}
\renewcommand{\headrulewidth}{0.4pt}
\renewcommand{\footrulewidth}{0.4pt}

\begin{document}

\maketitle
\thispagestyle{empty}
\newpage

\tableofcontents
\newpage

% --- Section 1: Executive Overview ---
\section{Executive Overview}

This report details the findings of a cybersecurity posture assessment conducted for \textbf{[Organization Name]}. The assessment combined a review of organizational security controls via a questionnaire, an external network vulnerability scan, and an analysis of pre-existing risks.

The overall security posture presents a mixed landscape. The organization has implemented some foundational controls, such as requiring Multi-Factor Authentication (MFA) for computer logins and conducting annual security awareness training. However, several critical and high-risk gaps were identified that significantly increase the organization's exposure to common cyber threats.

Key findings include:
\begin{itemize}
    \item \textbf{Critical MFA Gaps:} Multi-Factor Authentication is not enforced for accessing email or other sensitive data systems. This exposes the organization to significant risk from credential compromise and business email compromise (BEC) attacks.
    \item \textbf{Inadequate Employee Onboarding:} New employees do not receive security awareness training, leaving a critical window of vulnerability before they are included in the annual training cycle.
    \item \textbf{Exposed Network Services:} The external network scan identified an open port for SSH (Port 22), a common target for brute-force and credential-based attacks.
\end{itemize}

This report provides a detailed breakdown of these findings and offers actionable recommendations to mitigate the identified risks and strengthen the overall security posture of \textbf{[Organization Name]}.

% --- Section 2: Organizational Information ---
\section{Organizational Information}

This section contains the high-level information used as the basis for this assessment. As the provided data was anonymized, placeholders have been used.

\begin{table}[h!]
\centering
\begin{tabular}{@{}ll@{}}
\toprule
\textbf{Attribute} & \textbf{Value} \\
\midrule
Organization Name & \textbf{[Organization Name]} \\
Primary Domain & \texttt{[Domain]} \\
External IP Scanned & \texttt{[Client IP]} \\
\bottomrule
\end{tabular}
\caption{Client Organizational Data}
\label{tab:org_data}
\end{table}

% --- Section 3: Security Control Review ---
\section{Security Control Review}

The following table summarizes the organization's responses to a security controls questionnaire. Items marked with a red cross (\ding{55}) indicate a deviation from security best practices and represent a gap in the organization's defenses.

\begin{table}[h!]
\centering
\begin{tabular}{@{}p{0.75\textwidth}c@{}}
\toprule
\textbf{Control Question} & \textbf{Status} \\
\midrule
Do you require MFA to log into computers? & \ding{51} \\
Does your organization have an employee acceptable use policy? & \ding{51} \\
Does your organization do security awareness training for all employees at least once per year? & \ding{51} \\
\midrule
\textcolor{red}{Do you require MFA to access email?} & \textcolor{red}{\ding{55}} \\
\textcolor{red}{Do you require MFA to access sensitive data systems?} & \textcolor{red}{\ding{55}} \\
\textcolor{red}{Does your organization do security awareness training for new employees?} & \textcolor{red}{\ding{55}} \\
\bottomrule
\end{tabular}
\caption{Security Controls Questionnaire Analysis}
\label{tab:controls_review}
\end{table}

The identified gaps in MFA for email and sensitive systems are considered critical vulnerabilities. The lack of security training during employee onboarding is a high-risk deficiency that could lead to early and preventable security incidents.

% --- Section 4: Technical Scan Results ---
\section{Technical Scan Results}

An external network scan was performed against the target IP address to identify open ports and exposed services.

\begin{itemize}
    \item \textbf{Target IP Address:} \texttt{[Target IP]}
    \item \textbf{Scan Date:} Not specified in scan data.
\end{itemize}

\subsection{Open Ports}
The scan revealed the following open port(s) accessible from the public internet:

\begin{table}[h!]
\centering
\begin{tabular}{@{}llll@{}}
\toprule
\textbf{Port} & \textbf{State} & \textbf{Common Service} & \textbf{Notes} \\
\midrule
22/TCP & open & SSH (Secure Shell) & Exposing SSH to the internet is a high risk. \\
\bottomrule
\end{tabular}
\caption{Open Ports Detected on \texttt{[Target IP]}}
\label{tab:nmap_results}
\end{table}

\subsection{Analysis}
The presence of an open SSH port (22) is a significant finding. This service is a primary target for automated brute-force attacks, where attackers attempt to guess usernames and passwords to gain unauthorized remote access to the server. Without proper controls, such as IP whitelisting, strong password policies, and key-based authentication, this exposed service poses a direct threat to the integrity and confidentiality of the underlying system.

% --- Section 5: Consolidated Risk Assessment ---
\section{Consolidated Risk Assessment}

This section synthesizes the findings from the security control review, technical scan, and any pre-existing vulnerabilities. Each identified risk is assigned a severity level based on its potential impact and likelihood of exploitation.

\begin{table}[h!]
\centering
\begin{tabular}{@{}p{0.3\textwidth}p{0.5\textwidth}l@{}}
\toprule
\textbf{Risk Name} & \textbf{Overview} & \textbf{Severity} \\
\midrule
\textbf{No MFA on Email Access} & Lack of MFA on email accounts allows for account takeover if credentials are stolen, leading to data breaches and Business Email Compromise (BEC). & \textbf{Critical} \\
\addlinespace
\textbf{No MFA on Sensitive Data Systems} & Sensitive corporate and customer data is protected only by a password, making it highly vulnerable to unauthorized access via credential theft. & \textbf{Critical} \\
\addlinespace
\textbf{Exposed SSH Service} & Port 22 (SSH) is open to the public internet, making the system a target for brute-force attacks and unauthorized access attempts. & \textbf{High} \\
\addlinespace
\textbf{Lack of Onboarding Security Training} & New employees are not trained on security policies and threat identification upon hiring, creating a high-risk period for social engineering and policy violations. & \textbf{High} \\
\bottomrule
\end{tabular}
\caption{Summary of Identified Risks}
\label{tab:risk_summary}
\end{table}

\textit{Note: The pre-existing risk input contained no vulnerabilities.}

% --- Section 6: Recommendations ---
\section{Recommendations}

The following actionable recommendations are provided to address the identified risks. They are prioritized based on severity.

\subsection{Critical Priority}
\begin{enumerate}
    \item \textbf{Implement MFA for Email and Sensitive Systems:}
    \begin{itemize}
        \item \textbf{Action:} Immediately enable and enforce MFA for all user accounts on the primary email platform (e.g., Microsoft 365, Google Workspace) and all systems identified as containing sensitive data.
        \item \textbf{Justification:} This is the single most effective control to prevent account takeovers resulting from stolen credentials, directly mitigating the risk of data breaches and BEC.
    \end{itemize}
\end{enumerate}

\subsection{High Priority}
\begin{enumerate}
    \item \textbf{Secure the Exposed SSH Service:}
    \begin{itemize}
        \item \textbf{Action:} If remote access via SSH is required, implement a firewall rule to restrict access to only known, trusted IP addresses (IP whitelisting). If it is not required, disable the service entirely.
        \item \textbf{Justification:} This action removes the service from public exposure, eliminating the threat of internet-based brute-force attacks. Further hardening, such as enforcing public key authentication and disabling password-based logins, is also strongly advised.
    \end{itemize}
    \item \textbf{Establish a Security Training Program for New Hires:}
    \begin{itemize}
        \item \textbf{Action:} Develop a mandatory security awareness training module that is integrated into the employee onboarding process. This training should cover the acceptable use policy, phishing identification, password security, and incident reporting procedures.
        \item \textbf{Justification:} Training employees from day one establishes a security-conscious culture and reduces the likelihood of human error leading to a security incident.
    \end{itemize}
\end{enumerate}

\end{document}
```