```latex
\documentclass[12pt]{article}

% --- PACKAGE IMPORTS ---
\usepackage[margin=1in]{geometry}
\usepackage{pifont} % For checkmarks and crosses
\usepackage{booktabs} % For professional tables
\usepackage{hyperref} % For clickable links
\usepackage{url} % For URL formatting
\usepackage{seqsplit} % For splitting long strings
\usepackage{graphicx} % For logo (placeholder)
\usepackage{xcolor} % For colors

% --- DOCUMENT METADATA ---
\title{Cybersecurity Posture Assessment Report}
\author{Cybersecurity Analysis Division}
\date{\today}

% --- HYPERREF SETUP ---
\hypersetup{
    colorlinks=true,
    linkcolor=blue,
    filecolor=magenta,      
    urlcolor=cyan,
    pdftitle={Cybersecurity Posture Assessment Report},
    pdfpagemode=FullScreen,
}

% --- CUSTOM COMMANDS ---
\newcommand{\yes}{\ding{51}}
\newcommand{\no}{\ding{55}}

\begin{document}

\maketitle

\begin{abstract}
    This report provides a comprehensive analysis of the cybersecurity posture for \textbf{[Organization Name]}. The assessment is based on a synthesis of network scan data, a security controls questionnaire, and a review of pre-existing risks. The findings indicate several critical vulnerabilities that require immediate attention. Key issues include a complete lack of Multi-Factor Authentication (MFA) on critical systems, the exposure of a highly vulnerable public-facing FTP service, and the continued use of an unsupported operating system (Windows 7) on internal workstations. These factors combine to create a high-risk environment susceptible to unauthorized access, data breach, and malware infection. Prioritized, actionable recommendations are provided to mitigate these risks and improve the overall security posture.
\end{abstract}

\newpage

% ===================================================================
% SECTION 1: ORGANIZATIONAL INFORMATION
% ===================================================================
\section*{1. Organizational Information}

This assessment was conducted for the following entity. The information provided has been used to contextualize the technical findings.

\begin{itemize}
    \item \textbf{Organization Name:} \textbf{[Organization Name]}
    \item \textbf{Primary Email Domain:} \texttt{[Domain]}
    \item \textbf{External IP Scanned:} \texttt{[Client IP]}
\end{itemize}


% ===================================================================
% SECTION 2: SECURITY CONTROL REVIEW (QUESTIONNAIRE)
% ===================================================================
\section*{2. Security Control Review}

A review of the organization's security controls was conducted via a questionnaire. The responses highlight significant gaps in access control policies, which are a foundational component of a robust security program. The absence of Multi-Factor Authentication (MFA) is a critical weakness.

\begin{table}[h!]
\centering
\caption{Security Controls Questionnaire Results}
\begin{tabular}{p{0.7\linewidth} c c}
\toprule
\textbf{Control Question} & \textbf{Status} & \textbf{Analyst Note} \\
\midrule
Do you require MFA to access email? & \textcolor{red}{\no} & Critical Gap \\
Do you require MFA to log into computers? & \textcolor{red}{\no} & High Risk \\
Do you require MFA to access sensitive data systems? & \textcolor{red}{\no} & Critical Gap \\
\addlinespace
Does your organization have an employee acceptable use policy? & \textcolor{green}{\yes} & Good Practice \\
Does your organization do security awareness training for new employees? & \textcolor{green}{\yes} & Good Practice \\
Does your organization do security awareness training for all employees at least once per year? & \textcolor{green}{\yes} & Good Practice \\
\bottomrule
\end{tabular}
\end{table}


% ===================================================================
% SECTION 3: TECHNICAL SCAN RESULTS
% ===================================================================
\section*{3. Technical Scan Results}

An external network scan was performed against the target IP address \texttt{[Target IP]}. The scan identified an open port running a dangerously outdated and vulnerable service.

\begin{table}[h!]
\centering
\caption{Open Port Analysis}
\begin{tabular}{l l l l p{0.4\linewidth}}
\toprule
\textbf{Port} & \textbf{State} & \textbf{Service} & \textbf{Version} & \textbf{Notes} \\
\midrule
21/tcp & Open & ftp & vsftpd 2.3.4 & \textbf{Critical Vulnerability.} This version is from 2011 and contains a known backdoor (CVE-2011-2523). The scan also confirmed that anonymous FTP login is allowed, permitting unauthenticated access. \\
\bottomrule
\end{tabular}
\end{table}


% ===================================================================
% SECTION 4: CONSOLIDATED RISK ASSESSMENT
% ===================================================================
\section*{4. Consolidated Risk Assessment}

The following table summarizes the key risks identified through the correlation of the security questionnaire, technical scan, and pre-existing risk data. Risks are rated based on their potential impact and likelihood of exploitation.

\begin{table}[h!]
\centering
\caption{Summary of Identified Risks}
\begin{tabular}{l p{0.5\linewidth} l}
\toprule
\textbf{Risk ID} & \textbf{Risk Name \& Description} & \textbf{Severity} \\
\midrule
\textbf{RISK-001} & \textbf{Exposed Vulnerable FTP Service:} An outdated version of vsftpd (2.3.4) is publicly accessible. This version has a known backdoor vulnerability (CVE-2011-2523) and is configured to allow anonymous logins. This could lead to a full system compromise. & \textbf{Critical} \\
\addlinespace
\textbf{RISK-002} & \textbf{Lack of Multi-Factor Authentication:} MFA is not enforced for email, computer logins, or access to sensitive data. This significantly increases the risk of account compromise via phishing or credential stuffing, as a password is the only barrier to entry. & \textbf{Critical} \\
\addlinespace
\textbf{RISK-003} & \textbf{Outdated Windows Operating System:} Workstations are running Windows 7, which is an end-of-life OS and no longer receives security updates from Microsoft. This makes the internal network highly susceptible to malware and lateral movement by an attacker. & \textbf{High} \\
\bottomrule
\end{tabular}
\end{table}


% ===================================================================
% SECTION 5: RECOMMENDATIONS
% ===================================================================
\section*{5. Recommendations}

Based on the analysis, we provide the following prioritized recommendations to mitigate the identified risks and strengthen the overall security posture.

\subsection*{Immediate Actions (Mitigate Critical Risks)}
\begin{enumerate}
    \item \textbf{Decommission the Vulnerable FTP Server:} The FTP service on \texttt{[Target IP]} must be taken offline immediately.
    \begin{itemize}
        \item If file transfer functionality is essential, replace it with a modern, secure protocol such as SFTP (SSH File Transfer Protocol).
        \item Ensure any new solution is configured securely, disabling anonymous access and enforcing strong authentication.
    \end{itemize}
    \item \textbf{Implement Multi-Factor Authentication (MFA):} Begin a phased rollout of MFA across the organization, prioritizing the most critical systems.
    \begin{itemize}
        \item \textbf{Phase 1 (Immediate):} Enforce MFA for email access (e.g., Office 365, Google Workspace) and all administrative accounts.
        \item \textbf{Phase 2:} Enforce MFA for access to all systems containing sensitive data.
    \end{itemize}
\end{enumerate}

\subsection*{High-Priority Actions}
\begin{enumerate}
    \setcounter{enumi}{2} % Continue numbering from previous list
    \item \textbf{Upgrade End-of-Life Systems:} Develop and execute a plan to upgrade all Windows 7 workstations to a supported operating system (e.g., Windows 10/11). This is crucial for protecting the internal network from known exploits.
    \item \textbf{Conduct a Comprehensive Vulnerability Scan:} The identified FTP server is likely not the only vulnerability. Perform a comprehensive, authenticated vulnerability scan of all external and internal assets to identify and remediate other weaknesses.
\end{enumerate}

\end{document}
```