```latex
\documentclass[12pt]{article}

% --- PACKAGES ---
\usepackage[margin=1in]{geometry} % Set page margins
\usepackage{pifont}               % For checkmark and X symbols (\ding)
\usepackage{booktabs}             % For professional-looking tables
\usepackage{hyperref}             % For clickable links and metadata
\usepackage{url}                  % For formatting URLs
\usepackage{seqsplit}             % For splitting long strings without spaces
\usepackage{graphicx}             % For logos (optional, but good practice)
\usepackage{datetime}             % To automatically generate the report date

% --- DOCUMENT METADATA ---
\hypersetup{
    colorlinks=true,
    linkcolor=black,
    urlcolor=blue,
    pdftitle={Cybersecurity Posture Assessment Report},
    pdfauthor={Cybersecurity Analyst},
    pdfsubject={Security Analysis},
    pdfkeywords={Security, Risk, Assessment, Nmap, Vulnerability}
}

% --- DOCUMENT START ---
\begin{document}

% --- TITLE PAGE ---
\begin{titlepage}
    \centering
    \vspace*{1cm}
    \Huge \textbf{Cybersecurity Posture Assessment Report}
    \vspace{1.5cm}
    \Large \textbf{Prepared for:} \\
    \vspace{0.5cm}
    \huge \textbf{[Organization Name]}
    \vfill
    \large \textbf{Date of Report:} \today \\
    \vspace{0.5cm}
    \large \textbf{Report ID:} CYBER-2023-001
\end{titlepage}

\tableofcontents
\newpage

% --- EXECUTIVE SUMMARY ---
\section*{1. Executive Summary}
This report details the findings of a cybersecurity posture assessment conducted for \textbf{[Organization Name]}. The analysis combines a review of organizational security controls, an external network scan, and pre-existing risk data to provide a consolidated view of the current security landscape.

The assessment identified two \textbf{Critical} risks that require immediate attention. A public-facing FTP server was discovered running a dangerously outdated and vulnerable version of \texttt{vsftpd} (2.3.4), which is known to contain a critical backdoor vulnerability (CVE-2011-2523). This service also permits anonymous login, significantly increasing the risk of unauthorized access and data compromise. Furthermore, a critical gap in access control was identified: multi-factor authentication (MFA) is not enforced for access to sensitive data systems.

Additional findings include a pre-existing medium-risk issue related to outdated Windows 7 workstations. While the organization demonstrates a solid foundation in other areas, such as employee security training and MFA for email, the identified critical risks present a clear and present danger to the organization's data and operational integrity. Immediate remediation of the FTP server and implementation of MFA on sensitive systems are strongly recommended.

% --- ORGANIZATIONAL INFORMATION ---
\section*{2. Organizational Information}
This section provides the baseline information used for this assessment. Due to the anonymized nature of the input data, placeholders have been used.

\begin{itemize}
    \item \textbf{Organization Name:} \textbf{[Organization Name]}
    \item \textbf{Primary Domain:} \texttt{[Domain]}
    \item \textbf{External IP Scanned:} \texttt{[Client IP]}
\end{itemize}

% --- SECURITY CONTROL REVIEW ---
\section*{3. Security Control Review (Questionnaire Analysis)}
A review of the organization's security controls was conducted based on a standardized questionnaire. The results indicate a strong foundation in employee awareness and endpoint security but reveal a critical gap in protecting sensitive data. The checkmark (\ding{51}) indicates a positive control is in place, while the X (\ding{55}) indicates a control gap.

\begin{table}[h!]
\centering
\caption{Security Controls Questionnaire Results}
\begin{tabular}{p{0.7\linewidth} c}
\toprule
\textbf{Control Question} & \textbf{Status} \\
\midrule
Do you require MFA to access email? & \ding{51} \\
Do you require MFA to log into computers? & \ding{51} \\
\textbf{Do you require MFA to access sensitive data systems?} & \textbf{\ding{55}} \\
Does your organization have an employee acceptable use policy? & \ding{51} \\
Does your organization do security awareness training for new employees? & \ding{51} \\
Does your organization do security awareness training for all employees at least once per year? & \ding{51} \\
\bottomrule
\end{tabular}
\end{table}

\paragraph{Analysis:} The failure to enforce MFA on sensitive data systems represents a \textbf{Critical Risk}. Should an attacker compromise a user's credentials, they would have direct access to the organization's most valuable data without needing a second authentication factor.

% --- TECHNICAL SCAN RESULTS ---
\section*{4. Technical Scan Results (External Network Scan)}
An external network scan was performed against the target IP address \texttt{[Target IP]} to identify open ports and exposed services. The scan revealed a critically vulnerable File Transfer Protocol (FTP) service.

\begin{table}[h!]
\centering
\caption{Open Ports and Services on \texttt{[Target IP]}}
\begin{tabular}{l l l p{0.4\linewidth}}
\toprule
\textbf{Port/Proto} & \textbf{Service} & \textbf{Product / Version} & \textbf{Details} \\
\midrule
21/tcp & ftp & vsftpd 2.3.4 & \textbf{Critical Vulnerability:} This version is susceptible to CVE-2011-2523, a backdoor allowing remote command execution. \\
& & & \textbf{High Risk Configuration:} Anonymous FTP login is allowed, permitting unauthenticated access to files. \\
\bottomrule
\end{tabular}
\end{table}

\paragraph{Analysis:} The presence of \texttt{vsftpd 2.3.4} is a severe security threat. This specific version was compromised by an attacker who inserted a backdoor into the source code, which was then distributed. An attacker can gain a root shell on the server by simply sending a specific string as the username. The allowance of anonymous login further lowers the barrier to exploitation. This service should be taken offline or patched immediately.

% --- CONSOLIDATED RISK ASSESSMENT ---
\section*{5. Consolidated Risk Assessment}
This section synthesizes findings from the security control review, technical scan, and pre-existing risk data into a prioritized list.

\begin{table}[h!]
\centering
\caption{Summary of Identified Risks}
\begin{tabular}{p{0.45\linewidth} p{0.2\linewidth} p{0.15\linewidth}}
\toprule
\textbf{Risk Description} & \textbf{Source} & \textbf{Severity} \\
\midrule
\textbf{Vulnerable FTP Server (vsftpd 2.3.4)} & Technical Scan & \textbf{Critical} \\
An exposed FTP server is running a version with a known remote command execution backdoor (CVE-2011-2523). & & \\
\addlinespace
\textbf{Lack of MFA on Sensitive Systems} & Questionnaire & \textbf{Critical} \\
Failure to enforce MFA for sensitive data access exposes critical assets to compromise via stolen credentials. & & \\
\addlinespace
\textbf{Anonymous FTP Access Enabled} & Technical Scan & \textbf{High} \\
The FTP server allows unauthenticated users to log in, posing a risk of data leakage or unauthorized uploads. & & \\
\addlinespace
\textbf{Outdated Windows 7 Workstations} & Pre-existing Risk & \textbf{Medium} \\
Workstations are running an end-of-life operating system that no longer receives security updates. & & \\
\bottomrule
\end{tabular}
\end{table}

% --- RECOMMENDATIONS ---
\section*{6. Recommendations}
Based on the consolidated risk assessment, the following actions are recommended to improve the security posture of \textbf{[Organization Name]}. Recommendations are prioritized by severity.

\subsection*{Immediate Actions (Critical Risks)}
\begin{enumerate}
    \item \textbf{Remediate Vulnerable FTP Server:}
    \begin{itemize}
        \item Immediately take the FTP service on port 21 at \texttt{[Target IP]} offline.
        \item If FTP is a business necessity, migrate to a secure file transfer protocol like SFTP (SSH File Transfer Protocol).
        \item If \texttt{vsftpd} must be used, upgrade it to the latest stable version and reconfigure it from scratch, explicitly disabling anonymous access.
    \end{itemize}
    \item \textbf{Enforce MFA on Sensitive Systems:}
    \begin{itemize}
        \item Immediately develop and execute a plan to enforce MFA on all systems identified as containing or providing access to sensitive data.
        \item Prioritize systems accessible from the internet and systems holding financial, personal (PII), or proprietary information.
    \end{itemize}
\end{enumerate}

\subsection*{Near-Term Actions (High/Medium Risks)}
\begin{enumerate}
    \setcounter{enumi}{2} % Continue numbering from previous list
    \item \textbf{Disable Anonymous FTP Access:}
    \begin{itemize}
        \item As part of the FTP server remediation (Item 1), ensure that anonymous login is permanently disabled. Access should be restricted to authenticated, authorized users only.
    \end{itemize}
    \item \textbf{Accelerate Operating System Upgrades:}
    \begin{itemize}
        \item Continue and accelerate the project to upgrade all remaining Windows 7 workstations to a modern, supported operating system (e.g., Windows 10/11).
        \item Ensure that these legacy systems are isolated from critical network segments until they can be fully decommissioned or upgraded.
    \end{itemize}
\end{enumerate}

% --- DOCUMENT END ---
\end{document}
```