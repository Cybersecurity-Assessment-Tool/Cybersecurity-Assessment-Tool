```latex
\documentclass[12pt]{article}

% --- PACKAGES ---
\usepackage[margin=1in]{geometry}
\usepackage{pifont} % For checkmarks and crosses
\usepackage{booktabs} % For professional tables
\usepackage{hyperref} % For clickable links
\usepackage{url} % For URL formatting
\usepackage{seqsplit} % For splitting long strings
\usepackage{xcolor} % For color-coded severity
\usepackage{graphicx}
\usepackage{fancyhdr}

% --- DOCUMENT SETUP ---
\hypersetup{
    colorlinks=true,
    linkcolor=blue,
    urlcolor=blue,
    pdftitle={Cybersecurity Posture Assessment Report},
    pdfauthor={Cybersecurity Analysis Division}
}

% --- CUSTOM COMMANDS ---
\newcommand{\sevCritical}[1]{\textcolor{red!80!black}{\textbf{#1}}}
\newcommand{\sevHigh}[1]{\textcolor{red!90!white}{\textbf{#1}}}
\newcommand{\sevMedium}[1]{\textcolor{orange}{\textbf{#1}}}
\newcommand{\sevLow}[1]{\textcolor{yellow!80!black}{\textbf{#1}}}
\newcommand{\yes}{\ding{51}}
\newcommand{\no}{\ding{55}}

% --- HEADER & FOOTER ---
\pagestyle{fancy}
\fancyhf{}
\fancyhead[L]{\textbf{Cybersecurity Posture Assessment}}
\fancyhead[R]{\textbf{[Organization Name]}}
\fancyfoot[C]{\thepage}

% --- DOCUMENT START ---
\begin{document}

\title{
    \vspace{2cm}
    \textbf{Cybersecurity Posture Assessment Report} \\
    \large For \\
    \vspace{0.5cm}
    \textbf{[Organization Name]}
    \vspace{2cm}
}

\author{Cybersecurity Analysis Division}
\date{\today}

\maketitle
\thispagestyle{empty}
\newpage

\tableofcontents
\newpage

% ==============================================================================
% 1. EXECUTIVE SUMMARY
% ==============================================================================
\section{Executive Summary}

This report details the findings of a cybersecurity posture assessment conducted for \textbf{[Organization Name]}. The assessment combined an automated network scan, a review of existing risk documentation, and an analysis of organizational security controls via a questionnaire.

The overall security posture is determined to be \sevCritical{Critical}. This rating is based on the discovery of multiple, severe vulnerabilities and control gaps that expose the organization to a high likelihood of compromise.

Key findings include:
\begin{itemize}
    \item \textbf{Critical Lack of Multi-Factor Authentication (MFA):} MFA is not enforced for email, computer logins, or access to sensitive data systems. This represents a severe weakness, leaving accounts vulnerable to takeover via credential theft.
    \item \textbf{Direct Public Exposure of a Critical Database:} A MySQL database server was found to be directly accessible from the public internet. The service is running an outdated version (MySQL 5.7.33), which is past its End of Life (EOL) and no longer receives security updates.
    \item \textbf{Significant Gaps in Security Governance:} The organization lacks a formal Acceptable Use Policy and does not provide mandatory annual security awareness training for all employees. These gaps contribute to a weakened security culture and increased risk from insider threats.
\end{itemize}

Immediate remediation of the identified risks is strongly recommended. Actionable recommendations are provided in Section \ref{sec:recommendations} to guide mitigation efforts.

% ==============================================================================
% 2. ORGANIZATIONAL INFORMATION
% ==============================================================================
\section{Organizational and Assessment Scope}

This section outlines the basic information used as the basis for this assessment.

\begin{tabular}{@{}ll}
    \toprule
    \textbf{Item} & \textbf{Detail} \\
    \midrule
    Organization Name & \textbf{[Organization Name]} \\
    Primary Email Domain & \texttt{[Domain]} \\
    Client External IP & \texttt{[Client IP]} \\
    Assessed Target IP & \texttt{[Target IP]} \\
    Report Generation Date & \today \\
    \bottomrule
\end{tabular}

% ==============================================================================
% 3. SECURITY CONTROL REVIEW
% ==============================================================================
\section{Security Control Review}

The following table summarizes the organization's responses to a security controls questionnaire. "No" answers indicate significant gaps in the security framework and are correlated with identified risks.

\begin{table}[h!]
\centering
\begin{tabular}{p{0.6\linewidth} c p{0.25\linewidth}}
    \toprule
    \textbf{Control Question} & \textbf{Response} & \textbf{Assessment} \\
    \midrule
    Do you require MFA to access email? & \no & \sevCritical{Critical Gap}. High risk of business email compromise. \\
    \addlinespace
    Do you require MFA to log into computers? & \no & \sevCritical{Critical Gap}. Increases risk of lateral movement after a breach. \\
    \addlinespace
    Do you require MFA to access sensitive data systems? & \no & \sevCritical{Critical Gap}. Leaves critical data vulnerable to credential theft. \\
    \addlinespace
    Does your organization have an employee acceptable use policy? & \no & \sevHigh{High Risk}. Lack of formal policy creates ambiguity and legal risk. \\
    \addlinespace
    Does your organization do security awareness training for new employees? & \yes & Foundational control is in place. \\
    \addlinespace
    Does your organization do security awareness training for all employees at least once per year? & \no & \sevHigh{High Risk}. Security skills degrade over time; increases susceptibility to phishing. \\
    \bottomrule
\end{tabular}
\caption{Organizational Security Controls Questionnaire Analysis.}
\label{tab:controls}
\end{table}

% ==============================================================================
% 4. TECHNICAL SCAN RESULTS
% ==============================================================================
\section{Technical Scan Results}

An external network scan was performed against the target IP address. The scan identified the following open ports and services.

\subsection{Scan of Target: \texttt{[Target IP]}}

\begin{table}[h!]
\centering
\begin{tabular}{l l l l l p{0.3\linewidth}}
    \toprule
    \textbf{Port} & \textbf{State} & \textbf{Service} & \textbf{Product} & \textbf{Version} & \textbf{Analyst Notes} \\
    \midrule
    3306/tcp & open & mysql & MySQL & 5.7.33 & \sevHigh{High Risk}. Publicly exposed database. Version 5.7 reached End of Life (EOL) in October 2023 and is no longer supported with security patches. \\
    \bottomrule
\end{tabular}
\caption{Open Ports and Services Detected on \texttt{[Target IP]}.}
\label{tab:scanresults}
\end{table}

The presence of an open MySQL port is a severe finding. It allows attackers to directly attempt to brute-force credentials, exploit known vulnerabilities in the outdated version, or potentially cause a denial-of-service condition. This finding confirms the pre-existing risk documented in the organization's risk register.

% ==============================================================================
% 5. RISK ASSESSMENT SUMMARY
% ==============================================================================
\section{Risk Assessment Summary}

The following table consolidates and prioritizes the risks identified through the analysis of all data sources.

\begin{table}[h!]
\centering
\begin{tabular}{p{0.1\linewidth} p{0.25\linewidth} p{0.1\linewidth} p{0.45\linewidth}}
    \toprule
    \textbf{Risk ID} & \textbf{Risk Title} & \textbf{Severity} & \textbf{Description} \\
    \midrule
    \textbf{RISK-001} & Lack of Multi-Factor Authentication & \sevCritical{Critical} & The absence of MFA across email, endpoints, and sensitive systems exposes the organization to a high risk of account takeover and subsequent data breach. \\
    \addlinespace
    \textbf{RISK-002} & Public Database Exposure & \sevHigh{High} & A MySQL database on an EOL version is directly accessible from the internet, inviting brute-force attacks and exploitation of unpatched vulnerabilities. (CVSS 7.5) \\
    \addlinespace
    \textbf{RISK-003} & Insufficient Security Training & \sevHigh{High} & Lack of mandatory annual security training for all staff increases the organization's susceptibility to social engineering attacks like phishing. \\
    \addlinespace
    \textbf{RISK-004} & Inadequate Security Policies & \sevMedium{Medium} & The absence of a formal Acceptable Use Policy creates ambiguity regarding secure employee behavior and weakens the organization's overall governance posture. \\
    \bottomrule
\end{tabular}
\caption{Consolidated Risk Register.}
\label{tab:risks}
\end{table}

% ==============================================================================
% 6. RECOMMENDATIONS
% ==============================================================================
\section{Recommendations}
\label{sec:recommendations}

The following actionable recommendations are provided to mitigate the identified risks. They are prioritized to address the most critical vulnerabilities first.

\subsection{Immediate Actions (0-30 Days)}
\begin{enumerate}
    \item \textbf{Firewall the Exposed Database (RISK-002):} Immediately configure network firewall rules to \textbf{deny all public access} to TCP port 3306 on target \texttt{[Target IP]}. Access should be restricted to a whitelist of trusted internal IP addresses only.
    \item \textbf{Enable MFA for Email (RISK-001):} Begin the immediate rollout of mandatory MFA for all user accounts on the primary email system (\texttt{[Domain]}). This is the single most effective control to prevent business email compromise.
\end{enumerate}

\subsection{Medium-Term Actions (30-90 Days)}
\begin{enumerate}
    \item \textbf{Expand MFA Enforcement (RISK-001):} Enforce MFA for all remote access systems (e.g., VPN), computer logins (if supported), and access to all systems identified as storing sensitive data.
    \item \textbf{Develop an Acceptable Use Policy (RISK-004):} Draft, approve, and communicate a formal Employee Acceptable Use Policy (AUP). This policy should clearly define the rules for using company assets, data, and network resources.
    \item \textbf{Plan Database Upgrade (RISK-002):} Develop a project plan to upgrade the MySQL 5.7.33 instance to a currently supported version to ensure it receives ongoing security patches.
\end{enumerate}

\subsection{Long-Term Strategic Actions (90+ Days)}
\begin{enumerate}
    \item \textbf{Implement Annual Security Training (RISK-003):} Procure and schedule a mandatory annual security awareness training program for all employees. Track completion to ensure 100\% compliance.
    \item \textbf{Architect Secure Remote Access (RISK-002):} For long-term secure database administration, implement a Virtual Private Network (VPN) or Zero Trust Network Access (ZTNA) solution. This eliminates the need for any direct public exposure of administrative ports.
\end{enumerate}

\vfill
\begin{center}
    \textbf{--- End of Report ---}
\end{center}

\end{document}
```