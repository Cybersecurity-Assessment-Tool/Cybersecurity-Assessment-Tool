```latex
\documentclass[12pt]{article}

% Preamble: Required Packages
\usepackage[margin=1in]{geometry}
\usepackage{pifont} % For checkmarks and crosses
\usepackage{booktabs} % For professional tables
\usepackage{hyperref} % For hyperlinks and metadata
\usepackage{url} % For URL formatting
\usepackage{seqsplit} % For splitting long strings
\usepackage{graphicx}
\usepackage{xcolor}

% Hyperref Setup
\hypersetup{
    colorlinks=true,
    linkcolor=blue,
    filecolor=magenta,      
    urlcolor=cyan,
    pdftitle={Cybersecurity Posture Report},
    pdfauthor={Cybersecurity Analyst},
    pdfsubject={Security Assessment},
    pdfkeywords={Cybersecurity, Risk, Assessment},
    bookmarks=true
}

% Document Title and Header
\title{Cybersecurity Posture Report \\ \large For \textbf{[Organization Name]}}
\author{Cybersecurity Analyst}
\date{November 22, 2025}

\begin{document}

\maketitle
\thispagestyle{empty}
\newpage

\tableofcontents
\newpage

% --- 1. Executive Summary ---
\section{Executive Summary}
This report details the findings of a cybersecurity assessment conducted for \textbf{[Organization Name]}. The assessment combined a review of organizational security controls, an external network scan, and an analysis of pre-existing risks.

The overall security posture presents several areas of significant concern that require immediate attention. Key findings include:
\begin{itemize}
    \item \textbf{Critical Gaps in Access Control:} Multi-Factor Authentication (MFA) is not enforced for employee email or computer logins. This exposes the organization to a high risk of account compromise and subsequent data breaches.
    \item \textbf{Inadequate Employee Onboarding:} New employees do not receive mandatory security awareness training, leaving them highly susceptible to phishing and social engineering attacks from day one.
    \item \textbf{Vulnerable External Services:} The external network scan identified a web server running an outdated and vulnerable version of Nginx (\texttt{1.18.0}). This service is a prime target for automated attacks and could lead to a system compromise.
\end{itemize}

Immediate remediation of these issues is strongly recommended to reduce the organization's risk profile and protect its digital assets. Detailed recommendations are provided in Section \ref{sec:recommendations}.

% --- 2. Organizational Information ---
\section{Organizational Information}
This section provides the organizational details used as the basis for this assessment. Due to the anonymized nature of the provided data, placeholders are used where necessary.

\begin{tabular}{@{}ll}
    \toprule
    \textbf{Attribute} & \textbf{Value} \\
    \midrule
    Organization Name & \textbf{[Organization Name]} \\
    Primary Email Domain & \texttt{[Domain]} \\
    Assessed External IP & \texttt{[Client IP]} \\
    \bottomrule
\end{tabular}

% --- 3. Security Control Review ---
\section{Security Control Review}
The following table summarizes the organization's responses to a security controls questionnaire. "No" answers indicate significant gaps in the security framework and are highlighted for review.

\begin{table}[h!]
\centering
\begin{tabular}{@{}p{0.6\linewidth} c l@{}}
    \toprule
    \textbf{Control Question} & \textbf{Response} & \textbf{Assessment} \\
    \midrule
    Do you require MFA to access email? & \ding{55} & \textcolor{red}{\textbf{Critical Gap}} \\
    Do you require MFA to log into computers? & \ding{55} & \textcolor{red}{\textbf{Critical Gap}} \\
    Do you require MFA to access sensitive data systems? & \ding{51} & Meets Best Practice \\
    Does your organization have an employee acceptable use policy? & \ding{51} & Meets Best Practice \\
    Does your organization do security awareness training for new employees? & \ding{55} & \textcolor{orange}{\textbf{High Risk}} \\
    Does your organization do security awareness training for all employees at least once per year? & \ding{51} & Meets Best Practice \\
    \bottomrule
\end{tabular}
\caption{Security Controls Questionnaire Analysis}
\label{tab:controls}
\end{table}

% --- 4. Technical Scan Results ---
\section{Technical Scan Results}
An external network scan was performed to identify open ports and exposed services.

\begin{itemize}
    \item \textbf{Target IP Address:} \texttt{[Target IP]}
    \item \textbf{Scan Date:} November 22, 2025
\end{itemize}

The scan revealed the following open port:

\begin{table}[h!]
\centering
\begin{tabular}{@{}l l l l@{}}
    \toprule
    \textbf{Port} & \textbf{State} & \textbf{Service} & \textbf{Product \& Version} \\
    \midrule
    443/tcp & Open & HTTPS & Nginx \texttt{1.18.0} \\
    \bottomrule
\end{tabular}
\caption{Open Ports and Services}
\label{tab:scanresults}
\end{table}

\subsection*{Analysis of Findings}
The scan identified a web server running \textbf{Nginx version 1.18.0}. This version was released in April 2020 and is now significantly outdated. It is known to be affected by multiple security vulnerabilities, including but not limited to CVE-2021-23017. Exposing outdated software to the internet creates a substantial risk of system compromise.

% --- 5. Risk Assessment ---
\section{Risk Assessment}
This section synthesizes the findings from the security control review and the technical scan into a prioritized list of risks. No pre-existing vulnerabilities were reported.

\begin{table}[h!]
\centering
\begin{tabular}{@{}p{0.2\linewidth} p{0.55\linewidth} p{0.15\linewidth}@{}}
    \toprule
    \textbf{Risk Name} & \textbf{Overview} & \textbf{Severity} \\
    \midrule
    \textbf{Lack of MFA} & The absence of MFA on primary access vectors like email and workstations makes the organization highly vulnerable to credential theft and Business Email Compromise (BEC). A single compromised password could grant an attacker significant access. & \textcolor{red}{\textbf{Critical}} \\
    \addlinespace
    \textbf{Vulnerable Web Server} & The public-facing web server at \texttt{[Target IP]} runs an outdated Nginx version with known vulnerabilities. This could be exploited by attackers to gain unauthorized access to the server and the internal network. & \textcolor{orange}{\textbf{High}} \\
    \addlinespace
    \textbf{Inadequate Security Training} & New employees are not provided with security awareness training. This group is often targeted by phishing attacks, and this gap increases the likelihood of a successful social engineering incident. & \textcolor{orange}{\textbf{High}} \\
    \bottomrule
\end{tabular}
\caption{Synthesized Risk Summary}
\label{tab:risks}
\end{table}

% --- 6. Recommendations ---
\section{Recommendations}
\label{sec:recommendations}
Based on the identified risks, the following prioritized actions are recommended to improve the security posture of \textbf{[Organization Name]}.

\begin{enumerate}
    \item \textbf{Implement Multi-Factor Authentication (Critical):}
    \begin{itemize}
        \item Immediately enable and enforce MFA for all user accounts on the primary email system (e.g., Microsoft 365, Google Workspace).
        \item Develop a plan to deploy MFA for all workstation and laptop logins within the next quarter.
    \end{itemize}

    \item \textbf{Upgrade Web Server Software (High):}
    \begin{itemize}
        \item The Nginx server running on \texttt{[Target IP]}:443 must be upgraded to the latest stable version.
        \item Before upgrading, perform a full backup of the server's configuration and data.
        \item After upgrading, establish a regular patch management cycle to ensure all public-facing software remains up-to-date.
    \end{itemize}

    \item \textbf{Mandate New Hire Security Training (High):}
    \begin{itemize}
        \item Develop or procure a security awareness training module specifically for new employees.
        \item This training should be a mandatory part of the onboarding process for all new hires, to be completed within their first week of employment.
        \item Training should cover, at a minimum: phishing identification, password security, and the company's acceptable use policy.
    \end{itemize}
\end{enumerate}

\end{document}
```