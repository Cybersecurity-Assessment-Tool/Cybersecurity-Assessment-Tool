```latex
\documentclass[12pt]{article}

% Preamble: Required Packages
\usepackage[margin=1in]{geometry}
\usepackage{pifont} % For checkmarks and crosses
\usepackage{booktabs} % For professional tables
\usepackage{hyperref} % For hyperlinks
\usepackage{url} % For URL formatting
\usepackage{seqsplit} % For splitting long strings
\usepackage{graphicx}
\usepackage{xcolor}
\usepackage{fancyhdr}

% Document Information
\title{Cybersecurity Posture Assessment Report}
\author{Cybersecurity Analyst}
\date{\today}

% Hyperref Setup
\hypersetup{
    colorlinks=true,
    linkcolor=blue,
    filecolor=magenta,      
    urlcolor=cyan,
    pdftitle={Cybersecurity Posture Assessment Report},
    pdfpagemode=FullScreen,
}

% Header and Footer
\pagestyle{fancy}
\fancyhf{}
\lhead{\textbf{Confidential Report}}
\rhead{\textbf{[Organization Name]}}
\cfoot{\thepage}

\begin{document}

\maketitle
\thispagestyle{empty}
\newpage

\tableofcontents
\newpage

% --- 1. Executive Summary ---
\section*{1. Executive Summary}

This report provides a comprehensive analysis of the cybersecurity posture for \textbf{[Organization Name]}. The assessment is based on a correlation of data from an external network scan, a review of internal security controls via a questionnaire, and an analysis of pre-existing risks.

The external network scan of the target host \texttt{[Target IP]} revealed a strong security posture. No open ports were detected, and all other scanned ports were found to be in a 'closed' state. This indicates effective firewall configuration and perimeter defense against unsolicited network traffic.

However, the security controls review identified a significant policy gap. While the organization has implemented strong controls for Multi-Factor Authentication (MFA) and has a baseline for new employee training, there is a lack of mandatory, annual security awareness training for all employees. This oversight is classified as a \textbf{High} risk, as it increases the organization's susceptibility to human-centric threats such as phishing and social engineering over time.

This report outlines the detailed findings and provides actionable recommendations to mitigate the identified risk and further strengthen the organization's overall security posture.

% --- 2. Organizational Information ---
\section*{2. Organizational Information}

This assessment was conducted for the following organization. The information provided is based on the data supplied for this analysis.

\begin{itemize}
    \item \textbf{Organization Name:} \textbf{[Organization Name]}
    \item \textbf{Primary Domain:} \texttt{[Domain]}
    \item \textbf{Assessed External IP:} \texttt{[Client IP]}
\end{itemize}

% --- 3. Security Control Review ---
\section*{3. Security Control Review}

A review of organizational security controls was conducted based on a standardized questionnaire. The results highlight the current state of implemented policies and procedures. A "No" response indicates a potential control gap that requires attention.

\begin{table}[h!]
\centering
\caption{Security Controls Questionnaire Results}
\begin{tabular}{p{0.8\linewidth} c}
\toprule
\textbf{Control Question} & \textbf{Status} \\
\midrule
Do you require MFA to access email? & \ding{51} \\
Do you require MFA to log into computers? & \ding{51} \\
Do you require MFA to access sensitive data systems? & \ding{51} \\
Does your organization have an employee acceptable use policy? & \ding{51} \\
Does your organization do security awareness training for new employees? & \ding{51} \\
\midrule
\textbf{\textcolor{red}{Does your organization do security awareness training for all employees at least once per year?}} & \textbf{\textcolor{red}{\ding{55}}} \\
\bottomrule
\end{tabular}
\end{table}

\subsection*{Analysis}
The organization demonstrates a strong commitment to identity and access management through the comprehensive enforcement of Multi-Factor Authentication (MFA). Foundational policies like an Acceptable Use Policy and new hire security training are also in place.

The critical finding is the absence of annual security awareness training for all staff. Human behavior is a key factor in security. Without regular reinforcement, employees are more likely to forget best practices, fall victim to evolving phishing tactics, and inadvertently expose the organization to risk. This gap undermines the effectiveness of technical controls.

% --- 4. Technical Scan Results ---
\section*{4. Technical Scan Results}

An external network scan was performed to identify accessible services and potential vulnerabilities on the organization's perimeter.

\begin{itemize}
    \item \textbf{Target IP Address:} \texttt{[Target IP]}
    \item \textbf{Scan Date:} \today
    \item \textbf{Scanner Used:} Nmap
\end{itemize}

\subsection*{Findings}
The scan results were positive, indicating a very secure external network posture for the assessed host.
\begin{itemize}
    \item \textbf{Host Status:} Up
    \item \textbf{Open Ports:} 0 (None Detected)
    \item \textbf{Filtered/Closed Ports:} All 1000 scanned ports were reported as 'closed'.
\end{itemize}

\subsection*{Analysis}
The absence of any open ports suggests that the external firewall is correctly configured to deny all unsolicited inbound traffic by default (principle of least privilege). This significantly reduces the attack surface available to external adversaries and is considered a best practice for network security.

% --- 5. Risk Assessment Summary ---
\section*{5. Risk Assessment Summary}

This section consolidates risks identified from the security control review, technical scans, and pre-existing vulnerability data. In this assessment, no pre-existing vulnerabilities were reported, and no technical vulnerabilities were discovered. The primary risk stems from the identified policy gap.

\begin{table}[h!]
\centering
\caption{Identified Risks}
\begin{tabular}{p{0.25\linewidth} p{0.55\linewidth} p{0.1\linewidth}}
\toprule
\textbf{Risk Name} & \textbf{Overview} & \textbf{Severity} \\
\midrule
Lack of Annual Security Awareness Training & The absence of a mandatory, recurring security training program for all employees significantly increases the likelihood of successful phishing, social engineering, and malware attacks due to human error. & \textbf{High} \\
\bottomrule
\end{tabular}
\end{table}

% --- 6. Recommendations ---
\section*{6. Recommendations}

Based on the findings of this assessment, the following actions are recommended to mitigate identified risks and enhance the overall security posture of \textbf{[Organization Name]}.

\subsection*{High Priority}
\begin{itemize}
    \item \textbf{Implement Annual Security Awareness Training:}
    \begin{itemize}
        \item \textbf{Action:} Develop and deploy a mandatory annual security awareness training program for all employees, including management.
        \item \textbf{Details:} The training should cover current threats such as phishing, ransomware, proper data handling, and password security. The program should include a mechanism to track completion for all staff.
        \item \textbf{Impact:} Reduces the risk of security incidents caused by human error and fosters a stronger security culture.
    \end{itemize}
\end{itemize}

\subsection*{General Recommendations}
\begin{itemize}
    \item \textbf{Maintain Strong Network Perimeter:}
    \begin{itemize}
        \item \textbf{Action:} Continue the excellent practice of maintaining a restrictive firewall policy.
        \item \textbf{Details:} Regularly review firewall rules to ensure they align with business needs and the principle of least privilege. Periodically perform external scans to validate that no unintended ports have been opened.
        \item \textbf{Impact:} Ensures the external attack surface remains minimized.
    \end{itemize}
\end{itemize}

\end{document}
```