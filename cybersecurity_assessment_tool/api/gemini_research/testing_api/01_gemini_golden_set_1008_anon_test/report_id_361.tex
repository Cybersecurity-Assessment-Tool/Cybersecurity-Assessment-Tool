```latex
\documentclass[12pt]{article}

% ----------------------------------------------------------------------
% PREAMBLE
% ----------------------------------------------------------------------
\usepackage[margin=1in]{geometry}
\usepackage{pifont} % For checkmarks and crosses
\usepackage{booktabs} % For professional tables
\usepackage[colorlinks=true, urlcolor=blue]{hyperref}
\usepackage{url}
\usepackage{seqsplit} % For splitting long strings in tt font
\usepackage{graphicx}

% Define check and cross symbols for clarity
\newcommand{\cmark}{\ding{51}}
\newcommand{\xmark}{\ding{55}}

% Document Metadata
\title{Cybersecurity Posture Assessment Report}
\author{Cybersecurity Analysis Division}
\date{November 22, 2025}

% ----------------------------------------------------------------------
% DOCUMENT START
% ----------------------------------------------------------------------
\begin{document}

\maketitle
\hrule
\vspace{1cm}

\begin{center}
    \textbf{CONFIDENTIAL} \\
    \vspace{0.5cm}
    Prepared for: \textbf{[Organization Name]}
\end{center}

\newpage

\tableofcontents

\newpage

% ----------------------------------------------------------------------
% 1. EXECUTIVE SUMMARY
% ----------------------------------------------------------------------
\section{Executive Summary}

This report details the findings of a cybersecurity posture assessment conducted on November 22, 2025. The assessment combined an external network scan, a review of internal security controls via a questionnaire, and an analysis of pre-existing risks.

The overall security posture of \textbf{[Organization Name]} requires immediate attention. The assessment identified several critical and high-risk vulnerabilities that expose the organization to significant threats, including unauthorized data access, system compromise, and business disruption.

\textbf{Key Findings Include:}
\begin{itemize}
    \item \textbf{Critical Control Gap:} Multi-Factor Authentication (MFA) is not enforced for accessing sensitive data systems. This represents a severe weakness in the organization's defense-in-depth strategy.
    \item \textbf{Significant External Exposure:} The public-facing web server is running an outdated version of Nginx (1.18.0), which has multiple known vulnerabilities. This poses a high risk of external compromise.
    \item \textbf{Foundational Policy Gaps:} The organization lacks a formal Employee Acceptable Use Policy and does not conduct mandatory annual security awareness training for all staff. These gaps increase the risk of insider threats and successful social engineering attacks.
\end{itemize}

This report provides a detailed breakdown of these findings and offers prioritized, actionable recommendations to mitigate the identified risks and strengthen the organization's overall security posture. We strongly advise that the recommendations in Section 6 be addressed with urgency.

% ----------------------------------------------------------------------
% 2. ORGANIZATIONAL INFORMATION
% ----------------------------------------------------------------------
\section{Organizational Information}

This section contains the high-level information used as the basis for this assessment. Due to the anonymized nature of the input data, placeholders are used where necessary.

\begin{itemize}
    \item \textbf{Organization Name:} \textbf{[Organization Name]}
    \item \textbf{Primary Email Domain:} \texttt{[Domain]}
    \item \textbf{External IP Address Scanned:} \texttt{[Client IP]}
\end{itemize}

% ----------------------------------------------------------------------
% 3. SECURITY CONTROL REVIEW
% ----------------------------------------------------------------------
\section{Security Control Review}

The following table summarizes the organization's responses to a security controls questionnaire. A green checkmark (\cmark) indicates a positive control is in place, while a red cross (\xmark) indicates a control gap that introduces risk.

\begin{table}[h!]
\centering
\caption{Security Controls Questionnaire Results}
\begin{tabular}{p{0.75\linewidth} c}
\toprule
\textbf{Control Question} & \textbf{Response} \\
\midrule
Do you require MFA to access email? & \cmark \\
Do you require MFA to log into computers? & \cmark \\
Do you require MFA to access sensitive data systems? & \textcolor{red}{\xmark} \\
Does your organization have an employee acceptable use policy? & \textcolor{red}{\xmark} \\
Does your organization do security awareness training for new employees? & \cmark \\
Does your organization do security awareness training for all employees at least once per year? & \textcolor{red}{\xmark} \\
\bottomrule
\end{tabular}
\end{table}

\subsection{Analysis of Control Gaps}
The questionnaire revealed three significant control gaps:
\begin{itemize}
    \item \textbf{No MFA for Sensitive Systems:} The absence of MFA on systems containing sensitive data is a critical vulnerability. Should an attacker compromise a user's credentials, they would have direct access to the organization's most valuable information.
    \item \textbf{No Acceptable Use Policy (AUP):} Without a formal AUP, there are no clear guidelines for employees regarding the acceptable use of company assets. This can lead to unintentional security incidents and creates ambiguity in enforcing security standards.
    \item \textbf{No Annual Security Training:} While new employees receive training, the lack of an annual refresher for all staff means that security knowledge can become outdated. This increases susceptibility to evolving threats like phishing and other social engineering tactics.
\end{itemize}

% ----------------------------------------------------------------------
% 4. TECHNICAL SCAN RESULTS
% ----------------------------------------------------------------------
\section{Technical Scan Results}

An external network scan was performed on \textbf{November 22, 2025}, against the target IP address \texttt{[Target IP]}. The scan identified the following open ports and services.

\begin{table}[h!]
\centering
\caption{Open Port Scan Results}
\begin{tabular}{l l l l}
\toprule
\textbf{Port} & \textbf{State} & \textbf{Service} & \textbf{Product / Version} \\
\midrule
443/tcp & open & https & nginx 1.18.0 \\
\bottomrule
\end{tabular}
\end{table}

\subsection{Analysis of Technical Findings}
The scan identified a single service, HTTPS, running on port 443. This service is provided by an \textbf{Nginx web server, version 1.18.0}.

This version was released in April 2020 and is now considered outdated. It is known to be affected by several publicly disclosed vulnerabilities, including but not limited to CVE-2021-23017. Running outdated software on a public-facing server presents a high risk of compromise, as attackers can exploit these known flaws to gain unauthorized access to the server and potentially pivot to the internal network.

% ----------------------------------------------------------------------
% 5. CONSOLIDATED RISK ASSESSMENT
% ----------------------------------------------------------------------
\section{Consolidated Risk Assessment}

This section synthesizes the findings from the security control review and the technical scan into a consolidated list of identified risks. No pre-existing vulnerabilities were reported in the provided data.

\begin{table}[h!]
\centering
\caption{Summary of Identified Risks}
\begin{tabular}{p{0.1\linewidth} p{0.3\linewidth} p{0.4\linewidth} p{0.1\linewidth}}
\toprule
\textbf{ID} & \textbf{Risk Title} & \textbf{Description} & \textbf{Severity} \\
\midrule
\textbf{RISK-001} & Lack of MFA on Sensitive Systems & User credentials are the only barrier to sensitive data, creating a single point of failure against credential theft. & \textbf{Critical} \\
\addlinespace
\textbf{RISK-002} & Outdated Web Server Software & The public-facing Nginx server (v1.18.0) has known vulnerabilities, making it a prime target for automated attacks and external compromise. & \textbf{High} \\
\addlinespace
\textbf{RISK-003} & Missing Acceptable Use Policy & Lack of a formal policy creates ambiguity for employees and management, increasing the risk of insider threats and misuse of assets. & \textbf{High} \\
\addlinespace
\textbf{RISK-004} & Inadequate Security Awareness Training & Without annual training, employees are less prepared to identify and respond to modern cyber threats like phishing, increasing human-factor risks. & \textbf{High} \\
\bottomrule
\end{tabular}
\end{table}

% ----------------------------------------------------------------------
% 6. RECOMMENDATIONS
% ----------------------------------------------------------------------
\section{Recommendations}

Based on the consolidated risk assessment, we provide the following prioritized recommendations to mitigate the identified vulnerabilities and improve the overall security posture of \textbf{[Organization Name]}.

\begin{enumerate}
    \item \textbf{[Critical] Implement MFA on Sensitive Systems:}
    Immediately deploy a robust Multi-Factor Authentication solution for all access to systems classified as containing sensitive or critical data. This is the highest priority action to prevent unauthorized data access.

    \item \textbf{[High] Upgrade Public-Facing Web Server:}
    Plan and execute an upgrade of the Nginx server from version 1.18.0 to the latest stable version recommended by the vendor. This will patch known vulnerabilities and reduce the external attack surface. A continuous patch management program should be implemented.

    \item \textbf{[High] Develop and Enforce an Acceptable Use Policy (AUP):}
    Draft a comprehensive AUP that clearly defines the rules and responsibilities for all employees when using company technology and data. This policy should be formally communicated to all staff and acknowledged in writing.

    \item \textbf{[High] Establish an Annual Security Awareness Program:}
    Implement a mandatory security awareness training program for all employees to be completed at least once per year. The training should cover current threats such as phishing, malware, and proper data handling procedures.
\end{enumerate}

% ----------------------------------------------------------------------
% DOCUMENT END
% ----------------------------------------------------------------------
\end{document}
```