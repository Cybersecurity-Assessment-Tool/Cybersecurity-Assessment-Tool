```latex
\documentclass[12pt]{article}

% Preamble: Required Packages
\usepackage[margin=1in]{geometry}
\usepackage{pifont} % For checkmarks and crosses (\ding{51} and \ding{55})
\usepackage{booktabs} % For professional-looking tables
\usepackage{hyperref} % For clickable links and references
\usepackage{url}      % For formatting URLs
\usepackage{seqsplit} % For splitting long strings to prevent overflow
\usepackage{xcolor}   % For custom colors
\usepackage{graphicx} % For potential logos or diagrams

% --- Document Setup ---
\hypersetup{
    colorlinks=true,
    linkcolor=blue,
    filecolor=magenta,
    urlcolor=cyan,
    pdftitle={Cybersecurity Posture Assessment Report},
    pdfauthor={Cybersecurity Analysis Division},
}

% --- Custom Commands ---
\newcommand{\yes}{\textcolor{green!70!black}{\ding{51}}}
\newcommand{\no}{\textcolor{red!80!black}{\ding{55}}}
\newcommand{\riskcritical}{\textcolor{red!80!black}{\textbf{Critical}}}
\newcommand{\riskhigh}{\textcolor{orange!90!black}{\textbf{High}}}

% --- Document Start ---
\begin{document}

% --- Title Page ---
\begin{titlepage}
    \centering
    \vspace*{1cm}
    \Huge\textbf{Cybersecurity Posture Assessment Report}
    \vspace{1.5cm}
    \Large
    \textbf{Prepared for:}\\
    \vspace{0.5cm}
    \textbf{[Organization Name]}
    \vspace{2.5cm}
    \large
    \textbf{Date of Assessment:}\\
    \vspace{0.5cm}
    November 22, 2025
    \vfill
    \large
    \textbf{Generated By:}\\
    Cybersecurity Analysis Division
\end{titlepage}

\tableofcontents
\newpage

% --- Section 1: Executive Summary ---
\section{Executive Summary}
This report details the findings of a cybersecurity posture assessment conducted on November 22, 2025. The assessment combined a review of organizational security controls, an external network scan, and an analysis of pre-existing risks to provide a holistic view of the security posture for \textbf{[Organization Name]}.

The organization demonstrates a strong commitment to identity and access management, with Multi-Factor Authentication (MFA) consistently enforced across email, computer logins, and sensitive data systems. This is a commendable and effective foundational control.

However, the assessment identified several areas of significant concern that require immediate attention. Critical gaps were found in administrative controls, specifically the absence of an employee Acceptable Use Policy (AUP) and a formal security awareness training program. These deficiencies create a high-risk environment where employees may be more susceptible to social engineering attacks.

Furthermore, the external network scan revealed a public-facing web server running an outdated and vulnerable version of Nginx (1.18.0). This technical vulnerability, coupled with the lack of employee security awareness, presents a clear and present danger to the organization's data and reputation.

Overall, while the organization has a solid authentication framework, its security posture is undermined by critical policy gaps and a high-risk technical vulnerability. The recommendations in this report are designed to address these specific weaknesses and significantly improve the organization's resilience against cyber threats.

% --- Section 2: Organizational Information ---
\section{Organizational Information}
The following details were used as the basis for this assessment. Due to the anonymized nature of the provided data, placeholders have been used where necessary.

\begin{table}[h!]
\centering
\begin{tabular}{@{}ll@{}}
\toprule
\textbf{Attribute} & \textbf{Value} \\
\midrule
Organization Name & \textbf{[Organization Name]} \\
Primary Email Domain & \texttt{[Domain]} \\
External IP Address Scanned & \texttt{[Client IP]} \\
\bottomrule
\end{tabular}
\caption{Client Organizational Details}
\end{table}

% --- Section 3: Security Control Review ---
\section{Security Control Review}
A review of administrative and policy-based security controls was conducted via a standardized questionnaire. The results are summarized below. "Yes" answers indicate a control is in place, while "No" answers represent a gap in the security program.

\begin{table}[h!]
\centering
\begin{tabular}{@{}lc@{}}
\toprule
\textbf{Control Question} & \textbf{Status} \\
\midrule
Do you require MFA to access email? & \yes \\
Do you require MFA to log into computers? & \yes \\
Do you require MFA to access sensitive data systems? & \yes \\
Does your organization have an employee acceptable use policy? & \no \\
Does your organization do security awareness training for new employees? & \no \\
Does your organization do security awareness training for all employees at least once per year? & \no \\
\bottomrule
\end{tabular}
\caption{Security Controls Questionnaire Results}
\end{table}

\subsection*{Analysis of Controls}
The consistent implementation of MFA is an excellent security practice. However, the three "No" responses highlight critical deficiencies in the organization's governance and risk management framework. The lack of an Acceptable Use Policy and any form of security awareness training significantly increases the risk of human error leading to a security incident, such as falling victim to phishing or mishandling sensitive data.

% --- Section 4: Technical Scan Results ---
\section{Technical Scan Results}
An external network vulnerability scan was performed against the target IP address \texttt{[Target IP]} on November 22, 2025. The scan identified the following open ports and services.

\begin{table}[h!]
\centering
\begin{tabular}{@{}lllll@{}}
\toprule
\textbf{Port} & \textbf{State} & \textbf{Service} & \textbf{Product} & \textbf{Version} \\
\midrule
443/tcp & open & https & nginx & 1.18.0 \\
\bottomrule
\end{tabular}
\caption{Nmap Scan Findings for Target \texttt{[Target IP]}}
\end{table}

\subsection*{Analysis of Technical Findings}
The scan identified a single open port, 443/tcp, which is standard for secure web traffic (HTTPS). The service running on this port is Nginx version 1.18.0.

\textbf{This version of Nginx is significantly outdated and no longer supported.} Nginx 1.18.0 was released in April 2020. Since its release, numerous security vulnerabilities have been discovered and patched in subsequent versions. Running this outdated software exposes the organization to a wide range of publicly known exploits, including but not limited to request smuggling, denial-of-service, and potential remote code execution vulnerabilities. This finding represents a high-severity technical risk.

% --- Section 5: Consolidated Risk Assessment ---
\section{Consolidated Risk Assessment}
This section synthesizes the findings from the security control review, technical scan, and any pre-existing known risks. As no pre-existing risks were provided, all items listed below are new findings from this assessment.

\begin{table}[h!]
\centering
\begin{tabular}{@{}p{0.1\linewidth} p{0.3\linewidth} p{0.4\linewidth} p{0.1\linewidth}@{}}
\toprule
\textbf{ID} & \textbf{Risk Title} & \textbf{Description} & \textbf{Severity} \\
\midrule
RISK-001 & Deficient Security Awareness Program & The organization does not conduct security training for new or existing employees. This elevates the likelihood of successful phishing, social engineering, and malware incidents. & \riskcritical \\
\addlinespace
RISK-002 & Outdated Web Server Software & The public-facing web server at \texttt{[Target IP]} is running Nginx 1.18.0, a version with multiple known, high-impact vulnerabilities. & \riskhigh \\
\addlinespace
RISK-003 & Lack of Acceptable Use Policy (AUP) & There is no formal policy governing the acceptable use of company assets. This leads to inconsistent behavior and difficulty in enforcing security standards. & \riskhigh \\
\bottomrule
\end{tabular}
\caption{Summary of Identified Risks}
\end{table}

% --- Section 6: Recommendations ---
\section{Recommendations}
Based on the identified risks, the following prioritized actions are recommended to improve the organization's security posture.

\begin{enumerate}
    \item \textbf{Establish a Security Awareness Training Program (Addresses RISK-001):}
    \begin{itemize}
        \item Immediately implement a mandatory security awareness training module for all new hires as part of their onboarding process.
        \item Procure and deploy a training solution to provide all employees with annual training on topics such as phishing, password security, and data handling.
        \item Conduct periodic phishing simulation campaigns to measure the effectiveness of the training and identify areas for improvement.
    \end{itemize}
    
    \item \textbf{Upgrade Nginx Web Server (Addresses RISK-002):}
    \begin{itemize}
        \item Plan and execute an upgrade of the Nginx server at \texttt{[Target IP]} to the latest stable version.
        \item Prior to the upgrade, perform testing in a staging environment to ensure application compatibility.
        \item Implement a patch management process to ensure the server and its components are kept up-to-date going forward.
    \end{itemize}
    
    \item \textbf{Develop and Implement an Acceptable Use Policy (Addresses RISK-003):}
    \begin{itemize}
        \item Draft a formal AUP that clearly defines the rules for using company networks, devices, and data.
        \item Have the policy reviewed by legal and HR departments.
        \item Communicate the policy to all employees and require them to formally acknowledge their understanding and agreement.
    \end{itemize}
\end{enumerate}

% --- Section 7: Conclusion ---
\section{Conclusion}
The security posture of \textbf{[Organization Name]} is a mix of strong foundational controls and critical, unaddressed risks. The robust MFA implementation provides a solid defense against credential-based attacks. However, this strength is severely undermined by the lack of employee security training, the absence of a governing AUP, and a vulnerable public-facing web server.

By implementing the recommendations outlined in this report—prioritizing the security training program, upgrading the vulnerable Nginx server, and formalizing an AUP—the organization can effectively mitigate these risks and build a more resilient and mature security program.

\end{document}
```