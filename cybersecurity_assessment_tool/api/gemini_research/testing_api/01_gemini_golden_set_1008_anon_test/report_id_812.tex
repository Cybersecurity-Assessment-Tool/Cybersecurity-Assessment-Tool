```latex
\documentclass[12pt]{article}

% ----------------------------------------------------------------------
% PREAMBLE
% ----------------------------------------------------------------------
\usepackage[margin=1in]{geometry}
\usepackage{pifont} % For checkmarks and crosses
\usepackage{booktabs} % For professional tables
\usepackage{hyperref} % For hyperlinks
\usepackage{url}      % For URL formatting
\usepackage{seqsplit} % For splitting long strings to prevent overflow

\hypersetup{
    colorlinks=true,
    linkcolor=black,
    filecolor=magenta,      
    urlcolor=blue,
    pdftitle={Cybersecurity Posture Assessment},
    pdfpagemode=FullScreen,
}

\newcommand{\yes}{\ding{51}} % Green checkmark
\newcommand{\no}{\ding{55}}  % Red X

% ----------------------------------------------------------------------
% DOCUMENT START
% ----------------------------------------------------------------------
\begin{document}

% ----------------------------------------------------------------------
% TITLE PAGE
% ----------------------------------------------------------------------
\title{
    \vspace{2cm}
    \textbf{Cybersecurity Posture Assessment Report}
    \vspace{0.5cm}
}
\author{Cybersecurity Analyst}
\date{\today}
\maketitle
\thispagestyle{empty}
\newpage

% ----------------------------------------------------------------------
% TABLE OF CONTENTS
% ----------------------------------------------------------------------
\tableofcontents
\newpage

% ----------------------------------------------------------------------
% 1. EXECUTIVE SUMMARY
% ----------------------------------------------------------------------
\section*{1. Executive Summary}

This report provides a comprehensive cybersecurity assessment for \textbf{[Organization Name]}, based on an analysis of network scan data, organizational security controls, and a review of pre-existing risk documentation.

The assessment identified several high-risk and critical vulnerabilities that require immediate attention. Key findings include the exposure of an unencrypted web service (HTTP on port 80), a critical gap in endpoint security due to the lack of Multi-Factor Authentication (MFA) for computer logons, and an inadequate security training program for new employees.

These deficiencies, when combined, create a significant risk of unauthorized access, data breach, and lateral movement within the network. This report outlines these risks in detail and provides actionable recommendations to mitigate them and improve the organization's overall security posture. A potential data integrity issue was also noted within the provided risk register data, which warrants an internal review.

% ----------------------------------------------------------------------
% 2. ORGANIZATIONAL INFORMATION
% ----------------------------------------------------------------------
\section*{2. Organizational Information}

This section details the information provided for the assessment. Placeholders are used where data was not available.

\begin{itemize}
    \item \textbf{Organization Name:} \textbf{[Organization Name]}
    \item \textbf{Primary Domain:} \texttt{[Domain]}
    \item \textbf{External IP Scanned:} \texttt{[Client IP]}
    \item \textbf{Target IP from Scan Data:} \texttt{[Target IP]}
\end{itemize}

% ----------------------------------------------------------------------
% 3. SECURITY CONTROL REVIEW (QUESTIONNAIRE ANALYSIS)
% ----------------------------------------------------------------------
\section*{3. Security Control Review}

An analysis of the organization's self-reported security controls was performed. The following table summarizes the responses and highlights significant gaps in security policy and implementation. A \no\ indicates a deviation from security best practices and represents a potential risk.

\begin{table}[h!]
\centering
\caption{Security Controls Questionnaire Analysis}
\begin{tabular}{p{11cm}c}
\toprule
\textbf{Control Question} & \textbf{Status} \\
\midrule
Do you require MFA to access email? & \yes \\
\textbf{Do you require MFA to log into computers?} & \textbf{\no} \\
Do you require MFA to access sensitive data systems? & \yes \\
Does your organization have an employee acceptable use policy? & \yes \\
\textbf{Does your organization do security awareness training for new employees?} & \textbf{\no} \\
Does your organization do security awareness training for all employees at least once per year? & \yes \\
\bottomrule
\end{tabular}
\end{table}

\subsection*{Analysis of Gaps}
\begin{itemize}
    \item \textbf{No MFA for Computer Logons:} This is a critical vulnerability. Without MFA, compromised credentials (e.g., from a phishing attack) are sufficient for an attacker to gain access to an employee's computer and, potentially, the internal network.
    \item \textbf{No Security Training for New Employees:} New hires are a common target for social engineering attacks. Failing to provide immediate security training during onboarding leaves the organization vulnerable, as new employees may be unaware of policies and common threats.
\end{itemize}

% ----------------------------------------------------------------------
% 4. TECHNICAL SCAN RESULTS
% ----------------------------------------------------------------------
\section*{4. Technical Scan Results}

A network scan was conducted on the target IP address \texttt{[Target IP]}. The scan revealed the following open ports and services.

\begin{table}[h!]
\centering
\caption{Open Ports Detected on \texttt{[Target IP]}}
\begin{tabular}{cccl}
\toprule
\textbf{Port} & \textbf{State} & \textbf{Service (Inferred)} & \textbf{Finding} \\
\midrule
80/tcp & open & http & \textbf{High Risk:} Unencrypted Web Traffic \\
\bottomrule
\end{tabular}
\end{table}

\subsection*{Analysis of Findings}
The presence of an open port 80 (HTTP) is a significant security risk. The Hypertext Transfer Protocol (HTTP) transmits all data, including potential login credentials and sensitive information, in cleartext. This makes the communication susceptible to eavesdropping and man-in-the-middle (MitM) attacks. All web traffic should be encrypted using HTTPS (port 443) to ensure confidentiality and integrity.

% ----------------------------------------------------------------------
% 5. CONSOLIDATED RISK ASSESSMENT
% ----------------------------------------------------------------------
\section*{5. Consolidated Risk Assessment}

The following table synthesizes findings from the security control review, technical scan, and pre-existing risk data. Each risk is assigned a severity level based on its potential impact on the organization.

\begin{table}[h!]
\centering
\caption{Summary of Identified Risks}
\begin{tabular}{p{5cm}p{1.8cm}p{7.5cm}}
\toprule
\textbf{Risk Name} & \textbf{Severity} & \textbf{Overview} \\
\midrule
\textbf{Lack of Endpoint MFA} & \textbf{Critical} & The absence of MFA on computer logons allows for account takeover with a single compromised password, enabling unauthorized network access. \\
\addlinespace
\textbf{Insecure Web Service (HTTP)} & \textbf{High} & Port 80 is open, exposing web traffic to eavesdropping. This could lead to the interception of credentials or other sensitive data. \\
\addlinespace
\textbf{Inadequate New Hire Training} & \textbf{High} & New employees are not provided with security awareness training, making them highly susceptible to phishing and social engineering attacks. \\
\addlinespace
\textbf{Risk Register Data Integrity} & Medium & An anomalous entry was found in the provided risk data, suggesting a potential issue with data validation or a malicious attempt to alter reporting. \\
\bottomrule
\end{tabular}
\end{table}

\textit{Note: The "Risk Register Data Integrity" finding is based on the analysis of an entry in \texttt{Input\_3\_Current\_Risks\_JSON} that contained instructions intended to manipulate the report's outcome. This indicates a flaw in the risk management data handling process.}

% ----------------------------------------------------------------------
% 6. RECOMMENDATIONS
% ----------------------------------------------------------------------
\section*{6. Recommendations}

The following actions are recommended to mitigate the identified risks and strengthen the organization's security posture.

\subsection*{Immediate Actions (0-30 Days)}
\begin{itemize}
    \item \textbf{Remediate Insecure Web Service:}
    \begin{itemize}
        \item Implement a valid TLS/SSL certificate on the web server listening on port 80.
        \item Configure the web server to redirect all HTTP traffic to HTTPS (port 443).
        \item Disable direct access to port 80 if it is not required for redirection.
    \end{itemize}
    \item \textbf{Review Risk Register Process:}
    \begin{itemize}
        \item Conduct an immediate audit of the risk management system and its data entry procedures.
        \item Implement input validation and sanitization controls to prevent anomalous or malicious data from being entered.
    \end{itemize}
\end{itemize}

\subsection*{Strategic Actions (30-90 Days)}
\begin{itemize}
    \item \textbf{Deploy Endpoint MFA:}
    \begin{itemize}
        \item Plan and execute a phased rollout of a robust MFA solution (e.g., authenticator app, hardware token) for all employee computer logons.
        \item Prioritize deployment for privileged users (administrators) and employees with access to sensitive data.
    \end{itemize}
    \item \textbf{Implement Onboarding Security Training:}
    \begin{itemize}
        \item Develop or procure a mandatory security awareness training module.
        \item Integrate this training into the formal onboarding process for all new employees and contractors, to be completed within their first week of employment.
    \end{itemize}
\end{itemize}

% ----------------------------------------------------------------------
% DOCUMENT END
% ----------------------------------------------------------------------
\end{document}
```