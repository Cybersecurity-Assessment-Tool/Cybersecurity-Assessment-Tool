```latex
\documentclass[12pt]{article}

% Preamble: Required Packages
\usepackage[margin=1in]{geometry}
\usepackage{pifont} % For checkmarks and crosses
\usepackage{booktabs} % For professional tables
\usepackage{hyperref} % For clickable links
\usepackage{url} % For formatting URLs
\usepackage{seqsplit} % For splitting long strings in tt font
\usepackage{graphicx}
\usepackage[utf8]{inputenc}

% Document Metadata
\title{Cybersecurity Posture Assessment Report}
\author{Cybersecurity Analyst}
\date{\today}

\begin{document}

\maketitle
\thispagestyle{empty}
\newpage
\tableofcontents
\newpage

% --- 1. Executive Summary ---
\section{Executive Summary}

This report provides a comprehensive analysis of the cybersecurity posture for \textbf{[Organization Name]}. The assessment is based on a correlation of network scan data, a security controls questionnaire, and a review of pre-existing risks.

The overall security posture is determined to be critically weak, with several high-impact vulnerabilities requiring immediate attention. Key findings include:

\begin{itemize}
    \item \textbf{Critical Network Vulnerability:} An externally facing FTP server was identified running a severely outdated version of \texttt{vsftpd} (2.3.4), which is known to contain a backdoor vulnerability (CVE-2011-2523). The server is dangerously misconfigured to allow anonymous public access.
    \item \textbf{Critical Access Control Gap:} Multi-Factor Authentication (MFA) is not enforced for accessing sensitive data systems. This significantly increases the risk of a data breach from compromised credentials.
    \item \textbf{High-Risk Policy Gap:} The organization lacks a formal Acceptable Use Policy (AUP), creating ambiguity regarding the responsibilities of employees when using company assets.
    \item \textbf{Systemic Weakness:} The presence of both an outdated network service and outdated operating systems (Windows 7) indicates a systemic weakness in patch and vulnerability management.
\end{itemize}

Immediate remediation of the FTP server and the implementation of MFA on sensitive systems are paramount to reducing the organization's risk of a significant security incident.

% --- 2. Organizational Information ---
\section{Organizational Information}

This section details the information provided by the client. The data has been anonymized as per the engagement's requirements.

\begin{description}
    \item[Organization Name:] \textbf{[Organization Name]}
    \item[Primary Domain:] \texttt{[Domain]}
    \item[External IP Address Scanned:] \texttt{[Client IP]}
\end{description}

% --- 3. Security Control Review ---
\section{Security Control Review}

The following table summarizes the organization's responses to a security controls questionnaire. Responses marked with \ding{55} (No) indicate significant gaps in the security framework.

\begin{table}[h!]
\centering
\caption{Security Controls Questionnaire Results}
\begin{tabular}{p{0.7\linewidth} c}
\toprule
\textbf{Control Question} & \textbf{Response} \\
\midrule
Do you require MFA to access email? & \ding{51} \\
Do you require MFA to log into computers? & \ding{51} \\
\textbf{Do you require MFA to access sensitive data systems?} & \textbf{\ding{55}} \\
\textbf{Does your organization have an employee acceptable use policy?} & \textbf{\ding{55}} \\
Does your organization do security awareness training for new employees? & \ding{51} \\
Does your organization do security awareness training for all employees at least once per year? & \ding{51} \\
\bottomrule
\end{tabular}
\end{table}

\subsection*{Analysis of Gaps}
\begin{itemize}
    \item \textbf{No MFA for Sensitive Systems:} This is a critical deficiency. Should an attacker compromise an employee's credentials, they would have direct access to the organization's most valuable data assets without a secondary authentication challenge.
    \item \textbf{No Acceptable Use Policy (AUP):} The absence of an AUP is a high-risk governance gap. It fails to establish clear rules for employees regarding data handling, internet usage, and the use of company IT resources, which can lead to unintentional insider threats or policy violations.
\end{itemize}

% --- 4. Technical Scan Results ---
\section{Technical Scan Results}

An external network scan was performed to identify open ports and exposed services.

\begin{description}
    \item[Scan Target:] \texttt{[Target IP]}
    \item[Scan Date:] [Scan Date]
\end{description}

The following table details the findings from the scan.

\begin{table}[h!]
\centering
\caption{Open Port Analysis}
\begin{tabular}{l l l l p{0.3\linewidth}}
\toprule
\textbf{Port} & \textbf{State} & \textbf{Service} & \textbf{Product / Version} & \textbf{Notes} \\
\midrule
21/tcp & Open & ftp & vsftpd 2.3.4 & \textbf{Critical.} Anonymous FTP login is allowed. This version is vulnerable to CVE-2011-2523 (Backdoor Command Execution). \\
\bottomrule
\end{tabular}
\end{table}

% --- 5. Consolidated Risk Assessment ---
\section{Consolidated Risk Assessment}

This section synthesizes findings from the security questionnaire, technical scan, and pre-existing risk register into a prioritized list.

\begin{table}[h!]
\centering
\caption{Summary of Identified Risks}
\begin{tabular}{p{0.25\linewidth} p{0.4\linewidth} l l}
\toprule
\textbf{Risk Name} & \textbf{Overview} & \textbf{Severity} & \textbf{Source} \\
\midrule
\textbf{Vulnerable FTP Server with Anonymous Access} & An outdated and vulnerable FTP server (vsftpd 2.3.4) is publicly accessible and allows anonymous login, enabling potential remote code execution and unauthorized file access. & \textbf{Critical} & Technical Scan \\
\addlinespace
\textbf{Lack of MFA for Sensitive Data} & The absence of Multi-Factor Authentication on critical systems allows for single-factor authentication, making data breaches more likely if credentials are compromised. & \textbf{Critical} & Questionnaire \\
\addlinespace
\textbf{Missing Acceptable Use Policy} & The organization has no formal policy governing the use of IT assets, leading to a high risk of misuse and insider threat. & High & Questionnaire \\
\addlinespace
\textbf{Outdated Windows Policy} & Workstations are running Windows 7, an unsupported operating system that no longer receives security updates, leaving them vulnerable to exploitation. & Medium & Pre-existing Risk \\
\bottomrule
\end{tabular}
\end{table}

% --- 6. Recommendations ---
\section{Recommendations}

The following actionable recommendations are prioritized based on risk severity to help \textbf{[Organization Name]} improve its security posture.

\subsection{Immediate Actions (Critical Priority)}
\begin{enumerate}
    \item \textbf{Remediate FTP Server Vulnerability:} Immediately take the FTP server at \texttt{[Target IP]} offline.
    \begin{itemize}
        \item If the FTP service is not essential, it should be decommissioned permanently.
        \item If it is business-critical, it must be upgraded to the latest stable version of \texttt{vsftpd} or another secure file transfer solution (e.g., SFTP). Anonymous access must be disabled.
    \end{itemize}
    \item \textbf{Implement MFA for Sensitive Systems:} Enforce MFA across all systems identified as containing sensitive data. This should be treated as an emergency change to mitigate the risk of unauthorized access.
\end{enumerate}

\subsection{High Priority Actions}
\begin{enumerate}
    \setcounter{enumi}{2} % Continue numbering
    \item \textbf{Develop and Implement an Acceptable Use Policy (AUP):} Draft a formal AUP that clearly defines the rules and responsibilities for all employees when using company technology and data. Require all current and new employees to read and acknowledge the policy.
\end{enumerate}

\subsection{Strategic Recommendations}
\begin{enumerate}
    \setcounter{enumi}{3} % Continue numbering
    \item \textbf{Establish a Patch and Vulnerability Management Program:} Create a formal program to ensure that all software, operating systems, and network services are regularly scanned for vulnerabilities and patched in a timely manner. This will prevent systemic issues like outdated Windows 7 and \texttt{vsftpd} instances.
    \item \textbf{Upgrade End-of-Life Systems:} Plan and execute the upgrade of all Windows 7 workstations to a modern, supported operating system like Windows 11 to ensure they receive critical security updates.
\end{enumerate}

\end{document}
```