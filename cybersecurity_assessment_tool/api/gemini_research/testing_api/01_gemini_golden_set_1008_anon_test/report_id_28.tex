```latex
\documentclass[12pt]{article}

% --- PACKAGE IMPORTS ---
\usepackage[margin=1in]{geometry}
\usepackage{pifont} % For checkmarks and crosses
\usepackage{booktabs} % For professional tables
\usepackage{hyperref} % For hyperlinks
\usepackage{url}      % For URL formatting
\usepackage{seqsplit} % For splitting long strings
\usepackage{graphicx} % For logo (placeholder)
\usepackage{fancyhdr} % For header/footer

% --- DOCUMENT METADATA & STYLING ---
\hypersetup{
    colorlinks=true,
    linkcolor=blue,
    filecolor=magenta,      
    urlcolor=cyan,
    pdftitle={Cybersecurity Assessment Report},
    pdfauthor={Cybersecurity Analyst},
    pdfsubject={Security Assessment},
    pdfkeywords={Security, Risk, Analysis},
    bookmarks=true
}

\pagestyle{fancy}
\fancyhf{} % clear all header and footer fields
\fancyhead[L]{\textbf{Cybersecurity Assessment Report}}
\fancyhead[R]{\textbf{[Organization Name]}}
\fancyfoot[C]{\thepage}

% --- DOCUMENT START ---
\begin{document}

% --- TITLE PAGE ---
\begin{titlepage}
    \centering
    \vspace*{1cm}
    
    \Huge
    \textbf{Cybersecurity Assessment Report}
    
    \vspace{1.5cm}
    
    \Large
    Prepared for: \\
    \vspace{0.5cm}
    \textbf{[Organization Name]}
    
    \vfill
    
    \Large
    \textbf{Date of Report:} \today \\
    \textbf{Date of Scan:} \textbf{[Scan Date]}
    
    \vspace{1.5cm}
    
    \normalsize
    This document is confidential and intended solely for the use of \textbf{[Organization Name]}.
    
\end{titlepage}

\tableofcontents
\newpage

% --- EXECUTIVE OVERVIEW ---
\section{Executive Overview}
This report details the findings of a cybersecurity assessment conducted for \textbf{[Organization Name]}. The assessment combined a technical network scan of the external perimeter, a review of organizational security controls via a questionnaire, and an analysis of pre-existing risks.

The primary finding is a significant disparity between the organization's external network security and its internal security policies. The external network scan of the target IP address (\texttt{[Client IP]}) revealed a strong security posture, with \textbf{no open ports detected}. This indicates a well-configured firewall that effectively limits the external attack surface.

However, the security control review identified several \textbf{critical gaps} in internal policies and access controls. The most severe findings are the lack of multi-factor authentication (MFA) for accessing email and sensitive data systems. Additionally, the absence of a formal Employee Acceptable Use Policy represents a high-impact organizational risk.

While the network perimeter is secure, these internal control weaknesses expose the organization to significant threats, including account compromise, phishing attacks, unauthorized data access, and insider threats. This report provides a detailed breakdown of these risks and offers actionable recommendations to mitigate them.

% --- ORGANIZATIONAL INFORMATION ---
\section{Organizational Information}
The following details were used as the basis for this assessment.
\begin{itemize}
    \item \textbf{Organization Name:} \textbf{[Organization Name]}
    \item \textbf{Primary Domain:} \texttt{[Domain]}
    \item \textbf{External IP Assessed:} \texttt{[Client IP]}
\end{itemize}

% --- SECURITY CONTROL REVIEW ---
\section{Security Control Review}
The following table summarizes the organization's responses to a security controls questionnaire. "No" answers indicate potential gaps in the security framework and are highlighted for immediate attention.

\begin{table}[h!]
\centering
\caption{Security Controls Questionnaire Analysis}
\label{tab:controls}
\begin{tabular}{p{8cm} c l}
\toprule
\textbf{Control Question} & \textbf{Response} & \textbf{Assessment} \\
\midrule
Do you require MFA to access email? & \ding{55} & \textbf{Critical Gap} \\
Do you require MFA to log into computers? & \ding{51} & Best Practice \\
Do you require MFA to access sensitive data systems? & \ding{55} & \textbf{Critical Gap} \\
Does your organization have an employee acceptable use policy? & \ding{55} & \textbf{High Risk} \\
Does your organization do security awareness training for new employees? & \ding{51} & Best Practice \\
Does your organization do security awareness training for all employees at least once per year? & \ding{51} & Best Practice \\
\bottomrule
\end{tabular}
\end{table}

% --- TECHNICAL SCAN RESULTS ---
\section{Technical Scan Results}
An external network scan was performed on the provided IP address to identify accessible services and potential vulnerabilities.

\begin{itemize}
    \item \textbf{Target IP:} \texttt{[Target IP]}
    \item \textbf{Scan Summary:} The scan completed successfully and found the host to be online. However, no open TCP or UDP ports were discovered within the top 1000 scanned ports. All ports were reported as being in a \textbf{`closed`} state.
\end{itemize}

\subsection{Analysis}
The absence of open ports is a positive security finding. It indicates that the perimeter firewall is correctly configured to deny unsolicited inbound traffic, significantly reducing the external attack surface. This configuration effectively prevents attackers from directly accessing services running on the network.

% --- RISK ASSESSMENT ---
\section{Risk Assessment}
This section synthesizes findings from the security control review and technical scan. No pre-existing vulnerabilities were provided for this assessment. The following new risks have been identified based on the questionnaire results.

\begin{table}[h!]
\centering
\caption{Identified Risks}
\label{tab:risks}
\begin{tabular}{p{4cm} p{6.5cm} l}
\toprule
\textbf{Risk Name} & \textbf{Overview} & \textbf{Severity} \\
\midrule
\textbf{Email Account Compromise} & The lack of MFA on email accounts makes them highly susceptible to takeover via phishing or credential stuffing. A compromised email account is a primary vector for further attacks. & \textbf{Critical} \\
\addlinespace
\textbf{Unauthorized Access to Sensitive Data} & Sensitive data systems are protected only by a username and password. This exposes critical business data to unauthorized access, modification, or exfiltration. & \textbf{Critical} \\
\addlinespace
\textbf{Lack of Formal Acceptable Use Policy (AUP)} & Without a formal AUP, there are no clear guidelines for employees on the proper use of company assets. This increases the risk of insider threat, data leakage, and legal liability. & \textbf{High} \\
\bottomrule
\end{tabular}
\end{table}

% --- RECOMMENDATIONS ---
\section{Recommendations}
Based on the identified risks, the following prioritized recommendations are provided to enhance the security posture of \textbf{[Organization Name]}.

\begin{itemize}
    \item[\textbf{1.}] \textbf{(Critical) Implement MFA for Email:} Immediately deploy a mandatory MFA solution for all email accounts. This is the single most effective control to prevent email account takeovers.
    
    \item[\textbf{2.}] \textbf{(Critical) Enforce MFA on Sensitive Systems:} Prioritize the implementation of MFA for all applications and systems that store, process, or transmit sensitive organizational data. This is crucial for protecting core assets.
    
    \item[\textbf{3.}] \textbf{(High) Develop and Implement an Acceptable Use Policy:} Draft a formal AUP that clearly defines the rules and responsibilities for employees when using company technology and data. Require all employees to read and acknowledge the policy as a condition of use.
    
    \item[\textbf{4.}] \textbf{(Informational) Maintain Strong Perimeter Security:} Continue the current practice of maintaining a restrictive firewall policy. Conduct periodic external vulnerability scans to ensure that the secure perimeter is maintained and no unauthorized services become exposed.
\end{itemize}

\end{document}
```