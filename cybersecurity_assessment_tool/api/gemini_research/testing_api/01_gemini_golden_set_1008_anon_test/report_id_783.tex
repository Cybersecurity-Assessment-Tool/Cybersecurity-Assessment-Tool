```latex
\documentclass[12pt]{article}

% Preamble: Required Packages
\usepackage[margin=1in]{geometry}
\usepackage{pifont} % For using dingbats like checkmarks and crosses
\usepackage{booktabs} % For professional-looking tables
\usepackage{hyperref} % For creating hyperlinks within the document
\usepackage{url}      % For formatting URLs
\usepackage{seqsplit} % For splitting long strings without breaking words
\usepackage{graphicx}
\usepackage{xcolor}
\usepackage{lastpage}
\usepackage{fancyhdr}

% --- Document Setup ---
\hypersetup{
    colorlinks=true,
    linkcolor=blue,
    filecolor=magenta,
    urlcolor=cyan,
    pdftitle={Cybersecurity Posture Assessment Report},
    pdfauthor={Cybersecurity Analysis Division},
}

% --- Custom Commands ---
\newcommand{\yes}{\textcolor{green}{\ding{51}}} % Green checkmark for "Yes"
\newcommand{\no}{\textcolor{red}{\ding{55}}}   % Red X for "No"

% --- Header and Footer ---
\pagestyle{fancy}
\fancyhf{} % clear all header and footer fields
\fancyhead[L]{Cybersecurity Posture Assessment}
\fancyhead[R]{\textbf{[Organization Name]}}
\fancyfoot[C]{Page \thepage\ of \pageref{LastPage}}
\renewcommand{\headrulewidth}{0.4pt}
\renewcommand{\footrulewidth}{0.4pt}

% --- Document Start ---
\begin{document}

\title{Cybersecurity Posture Assessment Report}
\author{Cybersecurity Analysis Division}
\date{\today}
\maketitle
\thispagestyle{empty}
\tableofcontents
\newpage

\section{Executive Summary}

This report details the findings of a cybersecurity posture assessment for \textbf{[Organization Name]}. The assessment incorporated a review of organizational security controls, an external network scan of asset \texttt{[Target IP]}, and an analysis of pre-existing risk data.

The overall security posture is assessed as \textbf{HIGH RISK}.

This assessment is based on several critical findings that require immediate attention:
\begin{itemize}
    \item \textbf{Critical Service Exposure:} A MySQL database server is directly exposed to the public internet. Furthermore, the running version (\textbf{MySQL 5.7.33}) is past its \textbf{End-of-Life (EOL)} as of October 2023 and no longer receives security updates, posing a severe risk of exploitation.
    \item \textbf{Insufficient Access Controls:} Multi-Factor Authentication (MFA) is not enforced for accessing email or sensitive data systems. This gap significantly increases the likelihood of an account compromise leading to a major data breach.
    \item \textbf{Inadequate Employee Onboarding:} The lack of security awareness training for new employees creates a significant vulnerability to social engineering and phishing attacks, which are common initial access vectors for threat actors.
\end{itemize}

Immediate remediation of the exposed database and MFA gaps is strongly recommended to reduce the risk of a significant security incident.

\section{Organizational Information}

\begin{itemize}
    \item \textbf{Organization Name:} \textbf{[Organization Name]}
    \item \textbf{Primary Domain:} \texttt{[Domain]}
    \item \textbf{Asset Scanned:} \texttt{[Target IP]}
\end{itemize}

\section{Security Control Review (Questionnaire)}

This section reviews the organization's self-reported security controls. Gaps identified here often indicate systemic policy or implementation weaknesses that can be exploited by threat actors. The results are summarized in Table \ref{tab:controls}.

\begin{table}[h!]
\centering
\begin{tabular}{p{0.7\textwidth} c c}
\toprule
\textbf{Control Question} & \textbf{Response} & \textbf{Status} \\
\midrule
Do you require MFA to access email? & No & \no \\
Do you require MFA to log into computers? & Yes & \yes \\
Do you require MFA to access sensitive data systems? & No & \no \\
Does your organization have an employee acceptable use policy? & Yes & \yes \\
Does your organization do security awareness training for new employees? & No & \no \\
Does your organization do security awareness training for all employees at least once per year? & Yes & \yes \\
\bottomrule
\end{tabular}
\caption{Security Controls Questionnaire Results.}
\label{tab:controls}
\end{table}

The review identifies \textbf{critical gaps} in Multi-Factor Authentication for email and sensitive systems, and a high-risk gap in security training during employee onboarding.

\section{Technical Scan Results}

An external network scan was performed on the target asset \texttt{[Target IP]}. The following key findings were identified.

\subsection{Open Ports}
A single open port was discovered, exposing a critical database service directly to the network. This configuration is highly discouraged and presents a significant attack surface.

\begin{table}[h!]
\centering
\begin{tabular}{l l l l}
\toprule
\textbf{Port} & \textbf{Service} & \textbf{Product} & \textbf{Version} \\
\midrule
3306/tcp & mysql & MySQL & 5.7.33 \\
\bottomrule
\end{tabular}
\caption{Open Ports Detected on \texttt{[Target IP]}.}
\label{tab:ports}
\end{table}

\subsection{Version Analysis}
The detected MySQL version, \textbf{5.7.33}, reached its official \textbf{End-of-Life (EOL) in October 2023}. EOL software no longer receives security patches from the vendor, making it a prime target for exploitation using publicly known vulnerabilities. Running EOL software on an internet-facing system is a critical security risk.

\section{Overall Risk Assessment}

The following table summarizes the correlated risks based on the technical scan, control review, and pre-existing risk data. Risks are prioritized by their potential impact on the organization.

\begin{table}[h!]
\centering
\begin{tabular}{p{0.25\textwidth} p{0.55\textwidth} l}
\toprule
\textbf{Risk Name} & \textbf{Overview} & \textbf{Severity} \\
\midrule
Exposed End-of-Life Database & The MySQL database (v5.7.33) is publicly accessible on port 3306. This version is past its End-of-Life and no longer receives security updates, exposing it to known exploits. This confirms and elevates the pre-existing "Database Exposure" risk. & \textbf{\textcolor{red}{Critical}} \\
\addlinespace
Insufficient MFA Controls & Multi-Factor Authentication is not enforced for accessing email or sensitive data systems. This significantly increases the risk of account compromise and unauthorized data access, especially when combined with the exposed database. & \textbf{\textcolor{red}{Critical}} \\
\addlinespace
Inadequate Onboarding Security & New employees do not receive security awareness training. This makes them more susceptible to phishing and social engineering attacks, which could lead to initial network access for an attacker. & \textbf{\textcolor{orange}{High}} \\
\bottomrule
\end{tabular}
\caption{Summary of Identified Risks.}
\label{tab:risks}
\end{table}

\section{Recommendations}

Based on the findings, the following prioritized actions are recommended to mitigate the identified risks and improve the overall security posture.

\subsection{Immediate Priority (Critical Risks)}
\begin{enumerate}
    \item \textbf{Restrict Database Access:} Immediately implement firewall rules to \textbf{block all public access} to TCP port 3306 on \texttt{[Target IP]}. Access should be restricted to a whitelist of trusted internal IP addresses only.
    \item \textbf{Enforce MFA on Critical Systems:} Immediately enable and enforce MFA for all user accounts with access to email and any systems classified as containing sensitive data.
\end{enumerate}

\subsection{High Priority}
\begin{enumerate}
    \item \textbf{Upgrade End-of-Life Database:} Develop and execute a plan to migrate the MySQL 5.7.33 database to a currently supported version (e.g., MySQL 8.x or a managed cloud equivalent). This is essential for receiving future security patches.
    \item \textbf{Implement Onboarding Security Training:} Integrate mandatory security awareness training into the new employee onboarding process. This training should cover phishing, acceptable use, and password hygiene.
\end{enumerate}

\subsection{Long-Term Strategy}
\begin{enumerate}
    \item \textbf{Adopt a Zero-Trust Architecture:} Move away from perimeter-based security by implementing a Zero-Trust model where access to resources like databases is never trusted by default and is always verified, ideally through a secure gateway or VPN rather than direct exposure.
\end{enumerate}

\end{document}
```