```latex
\documentclass[12pt]{article}

% Preamble: Required Packages
\usepackage[margin=1in]{geometry}
\usepackage{pifont} % For checkmarks and crosses
\usepackage{booktabs} % For professional tables
\usepackage{hyperref} % For hyperlinks
\usepackage{url} % For URL formatting
\usepackage{seqsplit} % To split long strings in texttt
\usepackage{graphicx}
\usepackage{xcolor}
\usepackage{fancyhdr}

% --- Document Setup ---

% Define colors for severity levels
\definecolor{sev_critical}{HTML}{940000}
\definecolor{sev_high}{HTML}{D14000}
\definecolor{sev_medium}{HTML}{E8A100}

% Hyperlink setup
\hypersetup{
    colorlinks=true,
    linkcolor=blue,
    filecolor=magenta,      
    urlcolor=cyan,
    pdftitle={Cybersecurity Assessment Report},
    pdfpagemode=FullScreen,
}

% Header and Footer
\pagestyle{fancy}
\fancyhf{}
\lhead{Cybersecurity Assessment Report}
\rhead{\textbf{[Organization Name]}}
\cfoot{\thepage}
\renewcommand{\headrulewidth}{0.4pt}
\renewcommand{\footrulewidth}{0.4pt}

% --- Document Body ---

\begin{document}

% --- Title Page ---
\begin{titlepage}
    \centering
    \vspace*{1cm}
    
    \includegraphics[width=0.3\textwidth]{example-image-a} % Placeholder logo
    
    \vspace{1.5cm}
    
    {\Huge \textbf{Cybersecurity Assessment Report}\par}
    
    \vspace{1cm}
    
    {\Large \textbf{Prepared for:}} \\
    \vspace{0.2cm}
    {\Large \textbf{[Organization Name]}}
    
    \vspace{2cm}
    
    {\Large \textbf{Date of Report:}} \\
    \vspace{0.2cm}
    {\Large \today}
    
    \vspace{2cm}
    
    {\Large \textbf{Scan Date:}} \\
    \vspace{0.2cm}
    {\Large November 22, 2025}
    
    \vfill
    
    {\large \textit{This report contains sensitive information and should be handled with care. Access is restricted to authorized personnel only.}}
    
\end{titlepage}

\newpage

% --- Table of Contents ---
\tableofcontents
\newpage

% --- Executive Summary ---
\section*{1. Executive Summary}

This report details the findings of a cybersecurity assessment conducted for \textbf{[Organization Name]}. The assessment combined an external network scan, a review of existing security risks, and an analysis of organizational security controls via a questionnaire.

The assessment identified several critical and high-risk vulnerabilities that expose the organization to significant threats, including unauthorized access, data breaches, and social engineering attacks. The most pressing issues are the complete lack of a security awareness training program and the absence of Multi-Factor Authentication (MFA) on critical systems, including computer logins and sensitive data repositories.

Furthermore, a key external-facing web server was found to be running outdated and vulnerable software (Nginx 1.18.0). This technical vulnerability, coupled with the identified organizational control gaps, creates a high-likelihood path for threat actors to compromise the network.

Immediate remediation of the identified risks is strongly recommended to improve the organization's security posture and mitigate potential damages. A prioritized list of actionable recommendations is provided in Section 6.

% --- Organizational Information ---
\section*{2. Organizational Information}

This section provides the organizational details used as the basis for this assessment. As some data was not provided, placeholders have been used.

\begin{itemize}
    \item \textbf{Organization Name:} \textbf{[Organization Name]}
    \item \textbf{Primary Domain:} \texttt{[Domain]}
    \item \textbf{Client External IP:} \texttt{[Client IP]}
\end{itemize}

% --- Security Control Review ---
\section*{3. Security Control Review (Questionnaire Analysis)}

The following table summarizes the organization's responses to a security controls questionnaire. "No" answers indicate significant gaps in the security framework and are flagged as risks.

\begin{table}[h!]
\centering
\caption{Security Controls Questionnaire Results}
\begin{tabular}{p{0.6\linewidth} c p{0.25\linewidth}}
\toprule
\textbf{Control Question} & \textbf{Response} & \textbf{Assessment} \\
\midrule
Do you require MFA to access email? & \ding{51} Yes & Good Practice. Reduces risk of email compromise. \\
\addlinespace
Do you require MFA to log into computers? & \ding{55} No & \textbf{Critical Gap.} Increases risk of unauthorized endpoint access and lateral movement. \\
\addlinespace
Do you require MFA to access sensitive data systems? & \ding{55} No & \textbf{Critical Gap.} Exposes crown jewel data assets to credential theft. \\
\addlinespace
Does your organization have an employee acceptable use policy? & \ding{55} No & \textbf{High Risk.} Lack of clear guidelines for employees increases insider threat risk. \\
\addlinespace
Does your organization do security awareness training for new employees? & \ding{55} No & \textbf{Critical Gap.} New hires are a primary target for social engineering. \\
\addlinespace
Does your organization do security awareness training for all employees at least once per year? & \ding{55} No & \textbf{Critical Gap.} The organization is highly vulnerable to phishing and other social engineering attacks. \\
\bottomrule
\end{tabular}
\end{table}

% --- Technical Scan Results ---
\section*{4. Technical Scan Results}

An external network scan was performed to identify open ports and exposed services.

\subsection*{Target: \texttt{[Target IP]}}
The scan revealed the following open ports on the target system.

\begin{table}[h!]
\centering
\caption{Open Ports and Services on \texttt{[Target IP]}}
\begin{tabular}{c c l l l p{0.3\linewidth}}
\toprule
\textbf{Port} & \textbf{State} & \textbf{Service} & \textbf{Product} & \textbf{Version} & \textbf{Notes} \\
\midrule
443/tcp & Open & https & nginx & 1.18.0 & \textbf{High Risk.} This version is outdated (released April 2020) and vulnerable to multiple known exploits, such as CVE-2021-23017. \\
\bottomrule
\end{tabular}
\end{table}

% --- Risk Assessment ---
\section*{5. Risk Assessment Summary}

This section synthesizes findings from the questionnaire, technical scan, and pre-existing risk data. The following table prioritizes the most significant risks identified during the assessment.

\begin{table}[h!]
\centering
\caption{Summary of Identified Risks}
\begin{tabular}{p{0.05\linewidth} p{0.3\linewidth} p{0.15\linewidth} p{0.4\linewidth}}
\toprule
\textbf{ID} & \textbf{Risk / Vulnerability} & \textbf{Severity} & \textbf{Description} \\
\midrule
R-01 & Lack of Multi-Factor Authentication (MFA) & \textcolor{sev_critical}{\textbf{Critical}} & The absence of MFA on computer logins and sensitive data systems makes the organization highly susceptible to compromise via stolen credentials. \\
\addlinespace
R-02 & Lack of Security Awareness Program & \textcolor{sev_critical}{\textbf{Critical}} & Without training, employees are significantly more likely to fall victim to phishing and other social engineering attacks, providing an entry point for attackers. \\
\addlinespace
R-03 & Outdated Web Server Software (Nginx 1.18.0) & \textcolor{sev_high}{\textbf{High}} & The public-facing web server is running software with known vulnerabilities, which could be exploited to gain unauthorized access to the server and internal network. \\
\addlinespace
R-04 & Missing Acceptable Use Policy (AUP) & \textcolor{sev_high}{\textbf{High}} & The lack of a formal AUP creates ambiguity regarding safe computing practices, increasing the risk of both malicious and unintentional insider threats. \\
\bottomrule
\end{tabular}
\end{table}

% --- Recommendations ---
\section*{6. Recommendations}

The following actions are recommended to mitigate the identified risks. They are prioritized based on severity and potential impact.

\subsection*{Priority 1: Critical Risks}
\begin{enumerate}
    \item \textbf{Implement Comprehensive MFA:}
    \begin{itemize}
        \item \textbf{Action:} Deploy a robust MFA solution across all endpoints (computer logins) and for all access to systems containing sensitive data.
        \item \textbf{Justification:} This is the single most effective control to prevent unauthorized access resulting from compromised credentials. Mitigates \textbf{R-01}.
    \end{itemize}
    \vspace{0.5cm}
    \item \textbf{Establish a Security Awareness Program:}
    \begin{itemize}
        \item \textbf{Action:} Immediately develop and deploy a mandatory security awareness training program. This must include an onboarding module for new hires and annual refresher training for all staff. Training should focus on identifying phishing, password hygiene, and safe browsing.
        \item \textbf{Justification:} A well-trained workforce is the first line of defense against social engineering, which is the most common attack vector. Mitigates \textbf{R-02}.
    \end{itemize}
\end{enumerate}

\subsection*{Priority 2: High Risks}
\begin{enumerate}
    \setcounter{enumi}{2} % Continue numbering from previous list
    \item \textbf{Patch Vulnerable Web Server:}
    \begin{itemize}
        \item \textbf{Action:} Plan and execute an upgrade of the Nginx server at \texttt{[Target IP]} to the latest stable version. Follow a proper patch management and testing process.
        \item \textbf{Justification:} Patching this vulnerability closes a direct, public-facing attack vector into the organization's network. Mitigates \textbf{R-03}.
    \end{itemize}
    \vspace{0.5cm}
    \item \textbf{Develop and Implement an Acceptable Use Policy (AUP):}
    \begin{itemize}
        \item \textbf{Action:} Draft a formal AUP that clearly defines the rules and responsibilities for all employees when using company IT assets. Require all employees to read and acknowledge the policy.
        \item \textbf{Justification:} An AUP establishes a baseline for secure behavior, reduces ambiguity, and provides a basis for enforcement actions if necessary. Mitigates \textbf{R-04}.
    \end{itemize}
\end{enumerate}

\end{document}
```