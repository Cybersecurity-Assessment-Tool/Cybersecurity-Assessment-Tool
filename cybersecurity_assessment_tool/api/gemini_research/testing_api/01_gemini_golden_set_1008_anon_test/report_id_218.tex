```latex
\documentclass[12pt]{article}

% Preamble: Required Packages
\usepackage[margin=1in]{geometry}
\usepackage{pifont} % For \ding symbols (checkmarks/crosses)
\usepackage{booktabs} % For professional-looking tables
\usepackage{hyperref} % For clickable links and table of contents
\usepackage{url}      % For formatting URLs
\usepackage{seqsplit} % To split long strings without breaking
\usepackage{xcolor}   % For custom colors
\usepackage{graphicx}

% --- Document Setup ---
\hypersetup{
    colorlinks=true,
    linkcolor=blue,
    filecolor=magenta,
    urlcolor=cyan,
    pdftitle={Cybersecurity Posture Assessment Report},
    pdfauthor={Cybersecurity Analysis Division},
}

% --- Custom Commands ---
\newcommand{\yes}{\textcolor{green}{\ding{51}}} % Green checkmark
\newcommand{\no}{\textcolor{red}{\ding{55}}}    % Red cross

\begin{document}

% --- Title Page ---
\title{Cybersecurity Posture Assessment Report}
\author{Cybersecurity Analysis Division}
\date{\today}
\maketitle
\thispagestyle{empty}
\newpage

% --- Table of Contents ---
\tableofcontents
\newpage

% --- Section 1: Executive Summary ---
\section{Executive Summary}
This report provides a comprehensive cybersecurity assessment for \textbf{[Organization Name]}, based on an analysis of network scan data, organizational security controls, and pre-existing risk information. The assessment was conducted to identify vulnerabilities, security gaps, and areas of non-compliance with cybersecurity best practices.

Key findings indicate a critical external exposure and significant internal security weaknesses. An externally facing FTP server was found to be running a dangerously outdated version of \texttt{vsftpd} (2.3.4), which is known to contain a critical backdoor vulnerability (CVE-2011-2523). This service also permits anonymous access, providing a direct and trivial entry point for an attacker.

Internally, the lack of Multi-Factor Authentication (MFA) for computer logins represents a high-risk gap. This weakness, combined with the pre-existing risk of outdated Windows 7 workstations, significantly increases the potential for lateral movement and privilege escalation should an attacker gain an initial foothold.

Immediate remediation is required for the external FTP service. A strategic initiative to implement MFA on all workstations and accelerate the operating system upgrade plan is strongly recommended to mitigate substantial internal risks.

% --- Section 2: Organizational Information ---
\section{Organizational Information}
This section details the organizational context for this assessment. The data provided was anonymized for the purpose of this report generation.

\begin{itemize}
    \item \textbf{Organization Name:} \textbf{[Organization Name]}
    \item \textbf{Primary Domain:} \texttt{[Domain]}
    \item \textbf{External IP Scanned:} \texttt{[Client IP]}
\end{itemize}

% --- Section 3: Security Control Review ---
\section{Security Control Review (Questionnaire Analysis)}
The following table summarizes the organization's self-reported security controls. Responses marked with a \no\ represent significant gaps in the security posture and are discussed in the risk assessment section.

\begin{table}[h!]
\centering
\caption{Security Controls Questionnaire Results}
\begin{tabular}{p{0.6\textwidth} c p{0.2\textwidth}}
\toprule
\textbf{Control Question} & \textbf{Response} & \textbf{Assessment} \\
\midrule
Do you require MFA to access email? & \yes & Good Practice \\
Do you require MFA to log into computers? & \no & \textbf{Critical Gap} \\
Do you require MFA to access sensitive data systems? & \yes & Good Practice \\
Does your organization have an employee acceptable use policy? & \yes & Good Practice \\
Does your organization do security awareness training for new employees? & \yes & Good Practice \\
Does your organization do security awareness training for all employees at least once per year? & \yes & Good Practice \\
\bottomrule
\end{tabular}
\end{table}

The most critical finding from this review is the absence of MFA for workstation logins. This allows an attacker with compromised credentials to gain direct access to an endpoint, bypassing a fundamental security layer.

% --- Section 4: Technical Scan Results ---
\section{Technical Scan Results}
An external network scan was performed on the target IP address. The findings are detailed below.

\subsection{Scan Target}
\begin{itemize}
    \item \textbf{Target IP:} \texttt{[Target IP]}
    \item \textbf{Scan Date:} Not specified in scan metadata.
\end{itemize}

\subsection{Open Ports and Services}
A single open port was identified, exposing a critical service to the public internet.

\begin{table}[h!]
\centering
\caption{Nmap Scan Findings}
\begin{tabular}{l l l l p{0.3\textwidth}}
\toprule
\textbf{Port} & \textbf{State} & \textbf{Service} & \textbf{Version} & \textbf{Details} \\
\midrule
21/tcp & Open & ftp & vsftpd 2.3.4 & Anonymous FTP login is allowed. This version is critically vulnerable (CVE-2011-2523). \\
\bottomrule
\end{tabular}
\end{table}

\subsection{Vulnerability Analysis}
The identified service, \textbf{vsftpd version 2.3.4}, is associated with a well-known and severe vulnerability, \textbf{CVE-2011-2523}. This specific version contains a backdoor that was intentionally added to the source code, which allows an unauthenticated remote attacker to execute arbitrary commands with root privileges. The presence of this vulnerability on an internet-facing server constitutes a critical risk.

% --- Section 5: Consolidated Risk Assessment ---
\section{Consolidated Risk Assessment}
The following table synthesizes findings from the security control review, technical scan, and pre-existing risk data into a prioritized list of risks.

\begin{table}[h!]
\centering
\caption{Summary of Identified Risks}
\begin{tabular}{p{0.2\textwidth} p{0.3\textwidth} l p{0.3\textwidth}}
\toprule
\textbf{Risk Name} & \textbf{Description} & \textbf{Severity} & \textbf{Source} \\
\midrule
\textbf{Vulnerable FTP Service (CVE-2011-2523)} & An internet-facing FTP server is running a version with a known remote code execution backdoor. & \textbf{Critical} & Network Scan \\
\addlinespace
\textbf{Lack of Workstation MFA} & User computers do not require Multi-Factor Authentication for login, exposing them to credential theft. & \textbf{High} & Questionnaire \\
\addlinespace
\textbf{Anonymous FTP Access} & The FTP server allows anonymous login, which could be used for reconnaissance or to stage malicious files. & \textbf{High} & Network Scan \\
\addlinespace
\textbf{Outdated Windows Policy} & Workstations are running the unsupported Windows 7 OS, which no longer receives security updates. & Medium & Pre-existing Risk \\
\bottomrule
\end{tabular}
\end{table}

% --- Section 6: Recommendations ---
\section{Recommendations}
The following actions are recommended to mitigate the identified risks. Recommendations are prioritized based on severity.

\subsection{Remediate Vulnerable FTP Service (Critical)}
\begin{itemize}
    \item \textbf{Immediate Action:} If the FTP service is not essential for business operations, disable and uninstall it immediately from the server at \texttt{[Target IP]}.
    \item \textbf{Long-Term Action:} If the service is required, it must be upgraded to the latest stable version of vsftpd or replaced with a secure file transfer alternative (e.g., SFTP). Access should be restricted via firewall rules to only trusted IP addresses. Anonymous access must be disabled.
\end{itemize}

\subsection{Implement Workstation MFA (High)}
\begin{itemize}
    \item \textbf{Immediate Action:} Begin planning the deployment of an MFA solution for all workstation and laptop logins. Solutions like Windows Hello for Business, Duo, or other third-party tools should be evaluated.
    \item \textbf{Long-Term Action:} Enforce MFA for all employees, with special priority for privileged users (administrators, executives). This single control dramatically reduces the risk of account compromise.
\end{itemize}

\subsection{Upgrade Outdated Windows Systems (Medium)}
\begin{itemize}
    \item \textbf{Immediate Action:} Accelerate the existing project to upgrade all Windows 7 machines to a supported operating system (Windows 10/11). Isolate any remaining Windows 7 machines from the main network if they cannot be immediately upgraded.
    \item \textbf{Long-Term Action:} Implement a formal asset and patch management policy to ensure all operating systems and software are kept up-to-date and retired before they reach end-of-life.
\end{itemize}

\end{document}
```