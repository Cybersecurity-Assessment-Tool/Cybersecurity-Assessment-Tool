```latex
\documentclass[12pt]{article}

% Preamble: Required Packages
\usepackage[margin=1in]{geometry} % Set page margins
\usepackage{pifont}                % For checkmarks and crosses (\ding)
\usepackage{booktabs}              % For professional-looking tables
\usepackage{hyperref}              % For clickable links
\usepackage{url}                   % For formatting URLs
\usepackage{seqsplit}              % To split long strings without breaking
\usepackage{graphicx}              % For potential logos
\usepackage{xcolor}                % For custom colors

% Define custom colors for severity
\definecolor{criticalred}{HTML}{D73027}
\definecolor{highorange}{HTML}{F46D43}
\definecolor{mediumyellow}{HTML}{FEE08B}
\definecolor{lowblue}{HTML}{4575B4}
\definecolor{infogray}{HTML}{999999}

% Hyperref setup
\hypersetup{
    colorlinks=true,
    linkcolor=blue,
    filecolor=magenta,      
    urlcolor=cyan,
    pdftitle={Cybersecurity Posture Report},
    pdfpagemode=FullScreen,
}

% --- Document Start ---
\begin{document}

% --- Title Page ---
\begin{titlepage}
    \centering
    \vspace*{1cm}
    \Huge\textbf{Cybersecurity Posture Report}
    \vspace{1.5cm}
    \Large
    \textbf{Prepared for:} \\
    \vspace{0.5cm}
    \Huge\textbf{[Organization Name]}
    \vspace{2.5cm}
    \large
    \textbf{Date of Report:} \today \\
    \vspace{0.5cm}
    \textbf{Report ID:} CYBER-SEC-REP-2023-001
    \vfill
    \large
    \textbf{Confidential} \\
    \textit{This document contains sensitive information and is intended solely for the use of the designated recipient.}
\end{titlepage}

\tableofcontents
\newpage

% --- Executive Summary ---
\section*{Executive Summary}

This report provides a comprehensive analysis of the cybersecurity posture for \textbf{[Organization Name]}, based on a synthesis of network scan data, organizational security controls, and a review of pre-existing risks.

The assessment reveals critical gaps in foundational security controls, primarily concerning the lack of Multi-Factor Authentication (MFA) for computer and sensitive data access. Furthermore, the absence of an employee acceptable use policy and security training for new hires presents a significant risk to the organization's operational security. These procedural and policy-based weaknesses are the most pressing concerns identified during this assessment.

On a positive note, the technical network scan of the external asset at \texttt{[Client IP]} shows a secure perimeter. A previously identified risk, "Unencrypted Web Server," appears to have been successfully remediated, as the associated port (80/tcp) was found to be closed.

Immediate focus should be placed on implementing robust identity and access management controls (MFA) and developing core security policies and training programs to mitigate the most severe risks.

% --- Organizational Information ---
\section*{Organizational Information}
The following details were used as the basis for this assessment. As per the provided data, placeholder values are used where specific information was not available.

\begin{table}[h!]
\centering
\begin{tabular}{@{}ll@{}}
\toprule
\textbf{Attribute} & \textbf{Value} \\ \midrule
Organization Name & \textbf{[Organization Name]} \\
Primary Domain & \texttt{[Domain]} \\
External IP Address & \texttt{[Client IP]} \\ \bottomrule
\end{tabular}
\caption{Client Organizational Details}
\end{table}

% --- Security Control Review ---
\section*{Security Control Review}
A review of the organization's security controls was conducted via a questionnaire. The results highlight several areas requiring immediate attention. "No" answers indicate a gap in security best practices and are correlated with identified risks in Section \ref{sec:risk_assessment}.

\begin{table}[h!]
\centering
\begin{tabular}{@{}lc@{}}
\toprule
\textbf{Security Control Question} & \textbf{Status} \\ \midrule
Do you require MFA to access email? & \ding{51} \\
Do you require MFA to log into computers? & \textcolor{criticalred}{\ding{55}} \\
Do you require MFA to access sensitive data systems? & \textcolor{criticalred}{\ding{55}} \\
Does your organization have an employee acceptable use policy? & \textcolor{highorange}{\ding{55}} \\
Does your organization do security awareness training for new employees? & \textcolor{highorange}{\ding{55}} \\
Does your organization do security awareness training annually? & \ding{51} \\ \bottomrule
\end{tabular}
\caption{Security Controls Questionnaire Results}
\end{table}

\textbf{Analysis:} The lack of MFA on workstations and sensitive systems represents a critical vulnerability. An attacker with compromised credentials could gain significant access to internal resources. The absence of an acceptable use policy and onboarding security training creates an environment where employees may unintentionally engage in risky behavior, increasing the likelihood of a security incident.

% --- Technical Scan Results ---
\section*{Technical Scan Results}
An external network scan was performed to identify open ports and exposed services on the client's perimeter.

\begin{itemize}
    \item \textbf{Target IP Address:} \texttt{[Target IP]}
    \item \textbf{Scan Date:} Not provided in scan data; report generated on \today.
\end{itemize}

\begin{table}[h!]
\centering
\begin{tabular}{@{}llll@{}}
\toprule
\textbf{Port} & \textbf{State} & \textbf{Service} & \textbf{Analysis} \\ \midrule
80/tcp & closed & http & The port is closed. No service is exposed. \\ \bottomrule
\end{tabular}
\caption{Nmap Scan Results for \texttt{[Target IP]}}
\end{table}

\textbf{Analysis:} The scan shows a minimal attack surface, which is a strong security posture. Notably, port 80 (HTTP) is closed. This finding contradicts a pre-existing risk from \texttt{Input\_3\_Current\_Risks\_JSON} titled "Unencrypted Web Server." This indicates that the previously identified risk has been successfully remediated. It is recommended to update the internal risk register to reflect this positive change.

% --- Consolidated Risk Assessment ---
\section*{Consolidated Risk Assessment}
\label{sec:risk_assessment}
The following table synthesizes findings from the security control review and technical scans. Risks are prioritized based on their potential impact on the organization.

\begin{table}[h!]
\centering
\begin{tabular}{@{}p{0.4\linewidth}p{0.4\linewidth}p{0.15\linewidth}@{}}
\toprule
\textbf{Risk Title} & \textbf{Description} & \textbf{Severity} \\ \midrule
\textbf{Lack of MFA on Sensitive Systems} & The absence of a second authentication factor for sensitive data systems allows an attacker with stolen credentials to access critical data directly. & \textcolor{criticalred}{\textbf{Critical}} \\
\addlinespace
\textbf{Lack of MFA on Workstations} & User computers are protected only by a password. This increases the risk of unauthorized access, lateral movement, and ransomware deployment. & \textcolor{highorange}{\textbf{High}} \\
\addlinespace
\textbf{Missing Acceptable Use Policy (AUP)} & Without a formal AUP, employees are unaware of their responsibilities regarding company assets and data, increasing the risk of insider threat and data leakage. & \textcolor{highorange}{\textbf{High}} \\
\addlinespace
\textbf{No Security Training for New Hires} & New employees are not equipped with the knowledge to identify and avoid common threats like phishing, creating an immediate vulnerability upon their joining. & \textcolor{highorange}{\textbf{High}} \\
\bottomrule
\end{tabular}
\caption{Summary of Identified Risks}
\end{table}

% --- Recommendations ---
\section*{Recommendations}
Based on the consolidated risk assessment, the following actions are recommended to improve the security posture of \textbf{[Organization Name]}. Recommendations are prioritized to address the most critical risks first.

\begin{enumerate}
    \item \textbf{Implement MFA for Sensitive Systems (Critical):} Immediately deploy mandatory MFA for all access to systems identified as containing sensitive or critical data. This is the single most effective control to mitigate the risk of a major data breach.
    
    \item \textbf{Enforce MFA for All Workstation Logins (High):} Roll out MFA for all employee and privileged user computer logins. This will significantly reduce the risk of unauthorized access and lateral movement within the network.
    
    \item \textbf{Develop and Implement an Acceptable Use Policy (High):} Create a formal AUP that clearly defines the rules for using company technology, data, and network resources. Ensure all employees read and acknowledge the policy.
    
    \item \textbf{Integrate Security Training into Onboarding (High):} Develop a mandatory security awareness training module for all new employees. This module should be completed as part of the onboarding process before system access is granted.
    
    \item \textbf{Update Risk Register (Informational):} Formally review and close the "Unencrypted Web Server" risk in the organization's risk register. The technical scan confirms that Port 80 is closed, and this risk is considered remediated.
\end{enumerate}

% --- Document End ---
\end{document}
```