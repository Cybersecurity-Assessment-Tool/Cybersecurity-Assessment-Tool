```latex
\documentclass[12pt]{article}

% ----------------------------------------------------------------------
% PREAMBLE
% ----------------------------------------------------------------------
\usepackage[margin=1in]{geometry}
\usepackage{pifont} % For checkmarks and crosses
\usepackage{booktabs} % For professional tables
\usepackage{hyperref} % For hyperlinks
\usepackage{url}      % For URL formatting
\usepackage{seqsplit} % For splitting long strings in \texttt
\usepackage{xcolor}   % For colors

% Hyperref setup
\hypersetup{
    colorlinks=true,
    linkcolor=blue,
    filecolor=magenta,      
    urlcolor=cyan,
    pdftitle={Cybersecurity Assessment Report},
    pdfpagemode=FullScreen,
}

% Define custom colors for severity
\definecolor{criticalred}{HTML}{990000}
\definecolor{highorange}{HTML}{E69138}
\definecolor{mediumyellow}{HTML}{F1C232}

% ----------------------------------------------------------------------
% DOCUMENT START
% ----------------------------------------------------------------------
\begin{document}

% --- Title Page ---
\begin{titlepage}
    \centering
    \vspace*{1cm}
    \Huge\textbf{Cybersecurity Assessment Report}
    \vspace{1.5cm}
    \Large
    Prepared for: \\
    \vspace{0.5cm}
    \textbf{[Organization Name]}
    \vfill
    \large
    Date of Report: \today \\
    \vspace{0.2cm}
    Report ID: CSR-2024-001
\end{titlepage}

\tableofcontents
\newpage

% ----------------------------------------------------------------------
% SECTION 1: EXECUTIVE SUMMARY
% ----------------------------------------------------------------------
\section{Executive Summary}

This report details the findings of a cybersecurity assessment conducted for \textbf{[Organization Name]}. The assessment combined an external network scan, a review of existing risk documentation, and an analysis of organizational security controls via a questionnaire.

The assessment identified several high-priority risks that require immediate attention. The most critical finding is a publicly exposed MySQL database service on port 3306. This confirms a previously documented risk and elevates its severity, as the service is running \textbf{MySQL version 5.7.33}, which is now \textbf{End-of-Life (EOL)} and no longer receives security updates from the vendor. This exposes the organization to numerous known and future vulnerabilities.

Furthermore, significant gaps were identified in fundamental administrative and policy controls. Key weaknesses include:
\begin{itemize}
    \item The absence of mandatory Multi-Factor Authentication (MFA) for computer logins.
    \item A lack of a formal employee Acceptable Use Policy (AUP).
    \item No security awareness training integrated into the new employee onboarding process.
\end{itemize}

These combined findings indicate a heightened risk of unauthorized access, data breach, and system compromise. This report provides a detailed breakdown of these risks and offers actionable recommendations to mitigate them and improve the overall security posture of the organization.

% ----------------------------------------------------------------------
% SECTION 2: ORGANIZATIONAL INFORMATION
% ----------------------------------------------------------------------
\section{Organizational Information}

The following details were used as the basis for this assessment. As per our template mode for anonymized data, placeholders have been used where information was not provided.

\begin{itemize}
    \item \textbf{Organization Name:} \textbf{[Organization Name]}
    \item \textbf{Primary Email Domain:} \texttt{[Domain]}
    \item \textbf{Scanned Target IP Address:} \texttt{[Target IP]}
    \item \textbf{Associated Client IP Block:} \texttt{[Client IP]}
\end{itemize}

% ----------------------------------------------------------------------
% SECTION 3: SECURITY CONTROL REVIEW
% ----------------------------------------------------------------------
\section{Security Control Review (Questionnaire Analysis)}

An internal security questionnaire was reviewed to evaluate the current state of administrative and policy-based controls. The results are summarized in the table below. Answers marked with \ding{55} represent significant gaps in the security framework.

\begin{table}[h!]
\centering
\caption{Security Controls Questionnaire Results}
\begin{tabular}{p{0.75\linewidth} c}
\toprule
\textbf{Control Question} & \textbf{Response} \\
\midrule
Do you require MFA to access email? & \textcolor{green}{\ding{51}} \\
Do you require MFA to log into computers? & \textcolor{red}{\ding{55}} \\
Do you require MFA to access sensitive data systems? & \textcolor{green}{\ding{51}} \\
Does your organization have an employee acceptable use policy? & \textcolor{red}{\ding{55}} \\
Does your organization do security awareness training for new employees? & \textcolor{red}{\ding{55}} \\
Does your organization do security awareness training for all employees at least once per year? & \textcolor{green}{\ding{51}} \\
\bottomrule
\end{tabular}
\end{table}

\paragraph{Analysis:} The lack of MFA for computer logins, the absence of an Acceptable Use Policy, and the failure to train new hires on security best practices are critical deficiencies. These policy gaps significantly increase the risk of insider threats (both malicious and accidental) and successful phishing or social engineering attacks.

% ----------------------------------------------------------------------
% SECTION 4: TECHNICAL SCAN RESULTS
% ----------------------------------------------------------------------
\section{Technical Scan Results}

An external network scan was performed on the target IP address \texttt{[Target IP]}. The scan identified the following open port and service.

\begin{table}[h!]
\centering
\caption{Open Ports Detected on \texttt{[Target IP]}}
\begin{tabular}{l l l l}
\toprule
\textbf{Port} & \textbf{State} & \textbf{Service} & \textbf{Product \& Version} \\
\midrule
3306/tcp & open & mysql & MySQL 5.7.33 \\
\bottomrule
\end{tabular}
\end{table}

\paragraph{Analysis:} The scan confirms that a MySQL database server is directly exposed to the public internet on port 3306. This is a highly dangerous configuration, as it allows attackers worldwide to attempt to brute-force credentials, exploit vulnerabilities, or launch denial-of-service attacks against the database.

\textbf{Crucially, MySQL version 5.7 reached its official End-of-Life in October 2023.} This means it no longer receives security patches for newly discovered vulnerabilities, making this exposed server an extremely high-risk target for exploitation.

% ----------------------------------------------------------------------
% SECTION 5: CONSOLIDATED RISK ASSESSMENT
% ----------------------------------------------------------------------
\section{Consolidated Risk Assessment}

The following table synthesizes findings from the technical scan, the controls review, and pre-existing risk documentation into a consolidated list of prioritized risks.

\begin{table}[h!]
\centering
\caption{Summary of Identified Risks}
\begin{tabular}{p{0.1\linewidth} p{0.3\linewidth} p{0.15\linewidth} p{0.35\linewidth}}
\toprule
\textbf{Risk ID} & \textbf{Risk Name} & \textbf{Severity} & \textbf{Description} \\
\midrule
RISK-001 & Publicly Exposed End-of-Life Database & \textbf{\textcolor{criticalred}{Critical}} & A MySQL 5.7.33 database is exposed to the internet. This version is EOL and unpatched, creating a severe risk of data breach. This finding confirms and elevates the pre-existing risk "Database Exposure". \\
\addlinespace
RISK-002 & Lack of Endpoint Multi-Factor Authentication & \textbf{\textcolor{highorange}{High}} & No MFA is required for computer logins. A single compromised password could lead to widespread system access and lateral movement within the network. \\
\addlinespace
RISK-003 & Inadequate New Hire Security Training & \textbf{\textcolor{highorange}{High}} & New employees are not trained on security policies, making them highly vulnerable to social engineering and phishing attacks from their first day of employment. \\
\addlinespace
RISK-004 & Missing Acceptable Use Policy (AUP) & \textbf{\textcolor{mediumyellow}{Medium}} & The absence of a formal AUP creates ambiguity regarding the secure use of company assets and leaves the organization with limited recourse for policy violations. \\
\bottomrule
\end{tabular}
\end{table}

% ----------------------------------------------------------------------
% SECTION 6: RECOMMENDATIONS
% ----------------------------------------------------------------------
\section{Recommendations}

The following actions are recommended to mitigate the identified risks. Recommendations are prioritized based on severity and potential impact.

\subsection{RISK-001: Publicly Exposed End-of-Life Database (Critical)}
\begin{itemize}
    \item \textbf{Immediate Containment:} Implement firewall rules on the network perimeter to \textbf{block all inbound traffic to TCP port 3306} on \texttt{[Target IP]} from the public internet. Access should only be allowed from trusted, internal IP addresses.
    \item \textbf{Urgent Remediation:} Initiate a project to \textbf{upgrade the MySQL 5.7.33 instance} to a fully supported version (e.g., MySQL 8.x). This is essential to resume receiving security patches.
    \item \textbf{Long-Term Strategy:} For any required remote administration, implement a \textbf{Virtual Private Network (VPN)} solution. Administrators should connect to the VPN first before accessing internal database resources.
\end{itemize}

\subsection{RISK-002: Lack of Endpoint MFA (High)}
\begin{itemize}
    \item \textbf{Immediate Action:} Procure and deploy an MFA solution for all employee computer and laptop logins (both on-premises and remote).
    \item \textbf{Policy Update:} Update IT and security policies to mandate the use of MFA for all endpoint authentication.
\end{itemize}

\subsection{RISK-003: Inadequate New Hire Security Training (High)}
\begin{itemize}
    \item \textbf{Immediate Action:} Develop or procure a foundational security awareness training module and \textbf{integrate it into the mandatory new employee onboarding process}. This should cover phishing, password hygiene, and acceptable use.
\end{itemize}

\subsection{RISK-004: Missing Acceptable Use Policy (Medium)}
\begin{itemize}
    \item \textbf{Immediate Action:} Draft a formal Acceptable Use Policy (AUP) that clearly defines the rules for using company networks, devices, and data.
    \item \textbf{Implementation:} Require all current and new employees to read and formally acknowledge the AUP.
\end{itemize}

% ----------------------------------------------------------------------
% DOCUMENT END
% ----------------------------------------------------------------------
\end{document}
```