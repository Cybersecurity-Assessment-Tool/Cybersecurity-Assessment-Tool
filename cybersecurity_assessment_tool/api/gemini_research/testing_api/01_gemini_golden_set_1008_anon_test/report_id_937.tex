```latex
\documentclass[12pt]{article}

% Preamble: Required Packages
\usepackage[margin=1in]{geometry}
\usepackage{pifont} % For checkmarks and crosses
\usepackage{booktabs} % For professional tables
\usepackage{hyperref} % For hyperlinks
\usepackage{url} % For URL formatting
\usepackage{seqsplit} % For splitting long strings
\usepackage{graphicx} % For logo (optional)
\usepackage{xcolor} % For colors

% Document Information
\title{Cybersecurity Posture Assessment Report}
\author{Cybersecurity Analysis Division}
\date{November 22, 2025}

% Hyperref Setup
\hypersetup{
    colorlinks=true,
    linkcolor=blue,
    filecolor=magenta,      
    urlcolor=cyan,
    pdftitle={Cybersecurity Posture Assessment Report},
    pdfpagemode=FullScreen,
}

\begin{document}

\maketitle

\begin{abstract}
\noindent This report provides a comprehensive cybersecurity assessment for \textbf{[Organization Name]}. The analysis is based on a synthesis of network scan data, a review of organizational security controls via a questionnaire, and an evaluation of pre-existing risks. The findings indicate critical gaps in administrative controls and a significant technical vulnerability, elevating the organization's overall risk profile. This document details these findings and provides actionable recommendations to mitigate identified risks and improve the overall security posture.
\end{abstract}

\tableofcontents
\newpage

% --- Section 1: Executive Overview ---
\section{Executive Overview}
This assessment was conducted to evaluate the current cybersecurity posture of \textbf{[Organization Name]}. The analysis combined technical scanning of external infrastructure with a review of internal security policies and procedures.

The results reveal several areas of high concern. The most critical findings are related to fundamental administrative controls. The organization lacks Multi-Factor Authentication (MFA) for email and sensitive data systems, has no formal Acceptable Use Policy (AUP), and does not conduct security awareness training. These gaps create a significant risk of account compromise, data breaches, and insider threats.

From a technical perspective, the external network scan identified a public-facing web server running an outdated version of Nginx (1.18.0). This version is known to have multiple security vulnerabilities, presenting a direct and exploitable attack vector for external threats.

The combination of weak administrative controls and a vulnerable external perimeter places the organization at a \textbf{High Risk} of a significant cybersecurity incident. Immediate remediation is strongly recommended, focusing on implementing MFA, patching the vulnerable server, and establishing foundational security policies and training programs.

% --- Section 2: Organizational Information ---
\section{Organizational Information}
The following details were used as the basis for this assessment. Where information was not provided, placeholders have been used.

\begin{tabular}{@{}ll}
\toprule
\textbf{Attribute} & \textbf{Value} \\
\midrule
Organization Name & \textbf{[Organization Name]} \\
Primary Email Domain & \texttt{[Domain]} \\
Primary External IP & \texttt{[Client IP]} \\
Assessment Target IP & \texttt{[Target IP]} \\
Assessment Date & November 22, 2025 \\
\bottomrule
\end{tabular}

% --- Section 3: Security Control Review ---
\section{Security Control Review}
A questionnaire was used to assess the implementation of key administrative and procedural security controls. The responses indicate major deficiencies in foundational security practices. A summary of the findings is presented in Table \ref{tab:controls}. The symbol \ding{51} indicates a positive control is in place, while \ding{55} indicates a control gap.

\begin{table}[h!]
\centering
\caption{Organizational Security Controls Questionnaire}
\label{tab:controls}
\begin{tabular}{@{}lc}
\toprule
\textbf{Control Question} & \textbf{Response} \\
\midrule
Do you require MFA to access email? & \ding{55} \\
Do you require MFA to log into computers? & \ding{51} \\
Do you require MFA to access sensitive data systems? & \ding{55} \\
Does your organization have an employee acceptable use policy? & \ding{55} \\
Does your organization do security awareness training for new employees? & \ding{55} \\
Does your organization do security awareness training for all employees annually? & \ding{55} \\
\bottomrule
\end{tabular}
\end{table}

\subsection*{Analysis of Control Gaps}
\begin{itemize}
    \item \textbf{MFA Deficiencies:} The lack of MFA on email and sensitive data systems is a critical vulnerability. Email is a primary target for phishing and Business Email Compromise (BEC) attacks. Without MFA, a single compromised password could lead to a full breach of email and sensitive corporate data.
    \item \textbf{Policy and Training Gaps:} The absence of an Acceptable Use Policy and any form of security awareness training leaves employees without guidance on secure behavior and makes them highly susceptible to social engineering attacks. This fundamentally undermines the organization's human firewall.
\end{itemize}

% --- Section 4: Technical Scan Results ---
\section{Technical Scan Results}
An external network scan was performed against the target IP address \texttt{[Target IP]} on November 22, 2025. The scan identified the following open ports and services.

\begin{table}[h!]
\centering
\caption{Open Port Scan Results}
\label{tab:scan_results}
\begin{tabular}{@{}lllll}
\toprule
\textbf{Port} & \textbf{State} & \textbf{Service} & \textbf{Product} & \textbf{Version} \\
\midrule
443/tcp & open & https & nginx & 1.18.0 \\
\bottomrule
\end{tabular}
\end{table}

\subsection*{Analysis of Technical Findings}
The scan identified one primary service exposed to the internet:
\begin{itemize}
    \item \textbf{Nginx 1.18.0 on Port 443:} The web server is running Nginx version 1.18.0, which was released in April 2020. This version is significantly outdated and is no longer supported. It is known to be vulnerable to multiple Common Vulnerabilities and Exposures (CVEs), including but not limited to issues that could lead to information disclosure or request smuggling attacks. Running outdated software on a public-facing server presents a high-impact risk that could be exploited by automated scanners and malicious actors.
\end{itemize}

% --- Section 5: Consolidated Risk Assessment ---
\section{Consolidated Risk Assessment}
The following table synthesizes findings from the security control review, technical scan, and pre-existing risk data. Each identified risk has been assigned a severity level based on its potential impact and likelihood of exploitation. As no pre-existing risks were provided, all items listed below are newly identified during this assessment.

\begin{table}[h!]
\centering
\caption{Summary of Identified Risks}
\label{tab:risks}
\begin{tabular}{@{}lp{4cm}p{6cm}l}
\toprule
\textbf{ID} & \textbf{Risk Name} & \textbf{Description} & \textbf{Severity} \\
\midrule
RISK-001 & Critical MFA Gaps & Lack of MFA on email and sensitive data systems exposes the organization to account takeovers and data breaches from simple password compromise. & \textbf{Critical} \\
\addlinespace
RISK-002 & Outdated Public-Facing Web Server & The Nginx server (v1.18.0) is outdated and has known vulnerabilities, creating a direct vector for external attackers to compromise the system. & \textbf{High} \\
\addlinespace
RISK-003 & Lack of Security Policy and Training & The absence of an AUP and security awareness training program significantly increases the likelihood of human error leading to a security incident (e.g., phishing). & \textbf{High} \\
\bottomrule
\end{tabular}
\end{table}

% --- Section 6: Recommendations ---
\section{Recommendations}
Based on the consolidated risk assessment, the following prioritized recommendations are provided to mitigate the identified risks and strengthen the organization's cybersecurity posture.

\subsection{Immediate Priority (Remediate within 30 days)}
\begin{enumerate}
    \item \textbf{Implement Comprehensive MFA (RISK-001):} Enforce MFA across all user accounts for email access (e.g., Office 365, Google Workspace) and any systems containing sensitive data. This is the single most effective control to prevent unauthorized access.
    \item \textbf{Upgrade Nginx Web Server (RISK-002):} Immediately plan and execute an upgrade of the Nginx server from version 1.18.0 to the latest stable release. Ensure a process is in place for ongoing patching and security updates for all internet-facing systems.
\end{enumerate}

\subsection{High Priority (Remediate within 90 days)}
\begin{enumerate}
    \setcounter{enumi}{2}
    \item \textbf{Develop and Implement an Acceptable Use Policy (RISK-003):} Create a formal AUP that clearly defines the rules and responsibilities for all employees when using company technology and data. This policy should be signed by all employees.
    \item \textbf{Establish a Security Awareness Training Program (RISK-003):} Launch a mandatory security awareness training program for all employees. The program should include initial training for new hires and at least one annual refresher for all staff, covering topics like phishing, password security, and data handling.
\end{enumerate}

\subsection{General Recommendations}
\begin{enumerate}
    \setcounter{enumi}{4}
    \item \textbf{Implement a Vulnerability Management Program:} Establish a formal process to regularly scan internal and external assets for vulnerabilities, track findings, and remediate them based on risk.
\end{enumerate}

\end{document}
```