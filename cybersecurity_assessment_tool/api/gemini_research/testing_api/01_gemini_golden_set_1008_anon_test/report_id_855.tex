```latex
\documentclass[12pt]{article}

% --- PACKAGES ---
\usepackage[margin=1in]{geometry}
\usepackage{pifont} % For checkmarks and crosses
\usepackage{booktabs} % For professional tables
\usepackage{hyperref} % For hyperlinks
\usepackage{url} % For URL formatting
\usepackage{seqsplit} % For splitting long strings in tt font

% --- DOCUMENT METADATA ---
\hypersetup{
    colorlinks=true,
    linkcolor=black,
    urlcolor=blue,
    pdftitle={Cybersecurity Posture Assessment Report},
    pdfauthor={Cybersecurity Analyst},
    pdfsubject={Security Analysis},
    pdfkeywords={Cybersecurity, Risk Assessment, Network Scan}
}

\begin{document}

% --- TITLE PAGE ---
\begin{titlepage}
    \centering
    \vspace*{2cm}
    
    \Huge
    \textbf{Cybersecurity Posture Assessment Report}
    
    \vspace{1.5cm}
    
    \Large
    Prepared for: \\
    \vspace{0.5cm}
    \textbf{[Organization Name]}
    
    \vfill
    
    \Large
    \textbf{Date:} \today \\
    \textbf{Report ID:} CSA-2024-001
    
    \vspace{1cm}
    \rule{\textwidth}{0.4pt}
    \par
    \small
    This document contains sensitive information. Access is restricted to authorized personnel only. Do not distribute without explicit permission.
    
\end{titlepage}

\tableofcontents
\newpage

% --- EXECUTIVE SUMMARY ---
\section*{Executive Summary}

This report provides a comprehensive assessment of the cybersecurity posture for \textbf{[Organization Name]}, based on a technical network scan, a review of existing risks, and an analysis of organizational security controls.

The assessment reveals a mixed security posture. The organization has implemented security awareness training and requires multi-factor authentication (MFA) for sensitive data systems, which are positive controls. However, several critical-risk findings require immediate attention.

Key findings include an externally exposed, end-of-life database service (MySQL 5.7), which presents a significant and immediate threat of data breach. This technical vulnerability is compounded by critical gaps in administrative and access controls, specifically the lack of mandatory MFA for email and computer access, and the absence of an employee Acceptable Use Policy (AUP).

This report outlines the identified risks and provides prioritized, actionable recommendations to mitigate these threats and strengthen the organization's overall security resilience.

% --- ORGANIZATIONAL INFORMATION ---
\section*{1.0 Organizational Information}

The following information was used as the basis for this assessment. Anonymized placeholders are used where data was not provided.

\begin{itemize}
    \item \textbf{Organization Name:} \textbf{[Organization Name]}
    \item \textbf{Primary Domain:} \texttt{[Domain]}
    \item \textbf{Scanned IP Address:} \texttt{[Target IP]}
\end{itemize}

% --- SECURITY CONTROL REVIEW ---
\section*{2.0 Security Control Review}

An administrative review of security controls was conducted based on a standardized questionnaire. The results indicate foundational gaps in identity management and corporate policy. "No" answers represent significant weaknesses that increase organizational risk.

\begin{table}[h!]
\centering
\caption{Security Controls Questionnaire Results}
\label{tab:controls}
\begin{tabular}{p{0.8\textwidth} c}
\toprule
\textbf{Control Question} & \textbf{Status} \\
\midrule
Do you require MFA to access email? & \ding{55} \\
Do you require MFA to log into computers? & \ding{55} \\
Do you require MFA to access sensitive data systems? & \ding{51} \\
Does your organization have an employee acceptable use policy? & \ding{55} \\
Does your organization do security awareness training for new employees? & \ding{51} \\
Does your organization do security awareness training for all employees at least once per year? & \ding{51} \\
\bottomrule
\end{tabular}
\end{table}

\subsection*{Analysis of Control Gaps}
\begin{itemize}
    \item \textbf{Lack of MFA for Email and Computers:} The absence of MFA on primary communication (email) and endpoint (computer) systems is a critical vulnerability. These are common entry points for attackers. A compromised email account or computer can serve as a launchpad for further attacks within the network.
    \item \textbf{Missing Acceptable Use Policy (AUP):} An AUP is a foundational administrative control that sets clear expectations for employee behavior and use of company assets. Its absence creates legal and security ambiguities, making it difficult to enforce security standards.
\end{itemize}

% --- TECHNICAL SCAN RESULTS ---
\section*{3.0 Technical Scan Results}

A network scan was performed against the target IP address \texttt{[Target IP]} to identify open ports and exposed services.

\begin{table}[h!]
\centering
\caption{Open Port Scan Findings}
\label{tab:nmap}
\begin{tabular}{l l l l}
\toprule
\textbf{Port} & \textbf{State} & \textbf{Service} & \textbf{Version} \\
\midrule
3306/tcp & open & mysql & MySQL 5.7.33 \\
\bottomrule
\end{tabular}
\end{table}

\subsection*{Analysis of Technical Findings}
The scan identified a critical exposure:
\begin{itemize}
    \item \textbf{Exposed MySQL Database (Port 3306):} The MySQL database port is open to the network. Publicly exposing a database is a severe security risk, inviting brute-force attacks, credential stuffing, and exploitation of vulnerabilities.
    \item \textbf{End-of-Life (EOL) Software:} The detected version, \textbf{MySQL 5.7}, reached its official End of Life in \textbf{October 2023}. EOL software no longer receives security patches from the vendor, meaning any newly discovered vulnerabilities will remain unpatched. This makes the service a prime target for attackers.
\end{itemize}
This technical finding directly confirms and elevates the severity of the pre-existing risk noted as "Database Exposure."

% --- CONSOLIDATED RISK ASSESSMENT ---
\section*{4.0 Consolidated Risk Assessment}

The following table synthesizes findings from the security control review, technical scan, and pre-existing risk data into a prioritized list.

\begin{table}[h!]
\centering
\caption{Summary of Identified Risks}
\label{tab:risks}
\begin{tabular}{p{0.3\textwidth} p{0.5\textwidth} l}
\toprule
\textbf{Risk Name} & \textbf{Description} & \textbf{Severity} \\
\midrule
\textbf{Exposed End-of-Life Database Service} & Port 3306 (MySQL) is publicly accessible, and the running version (5.7.33) is no longer supported with security updates. This poses an immediate risk of data breach. & \textbf{Critical} \\
\addlinespace
\textbf{Insufficient Identity and Access Management} & Lack of MFA on email and computer logins significantly weakens defenses against credential-based attacks, such as phishing and password spraying. & \textbf{Critical} \\
\addlinespace
\textbf{Missing Foundational Security Policy} & The absence of an Acceptable Use Policy (AUP) represents a high-risk administrative gap, leading to inconsistent security practices and lack of employee accountability. & \textbf{High} \\
\bottomrule
\end{tabular}
\end{table}

% --- RECOMMENDATIONS ---
\section*{5.0 Recommendations}

The following actions are recommended to mitigate the identified risks. They are prioritized based on severity and potential impact.

\subsection*{Priority 1: Immediate Actions (Within 72 Hours)}
\begin{enumerate}
    \item \textbf{Restrict Access to MySQL Database:} Immediately implement firewall rules to block all public access to TCP port 3306 on \texttt{[Target IP]}. Access should only be permitted from trusted, internal IP addresses.
\end{enumerate}

\subsection*{Priority 2: High-Impact Mitigations (Within 30 Days)}
\begin{enumerate}
    \item \textbf{Enforce Multi-Factor Authentication (MFA):} Procure and deploy an MFA solution for all employees. Prioritize enforcement on the following systems:
        \begin{itemize}
            \item All email accounts (e.g., Office 365, Google Workspace).
            \item All computer/endpoint logins (e.g., Windows Hello, Duo).
        \end{itemize}
    \item \textbf{Develop and Implement an Acceptable Use Policy (AUP):} Draft a formal AUP that defines the rules for using company IT assets, data, and internet access. Require all employees to read and formally acknowledge the policy.
\end{enumerate}

\subsection*{Priority 3: Strategic Improvements (Within 90 Days)}
\begin{enumerate}
    \item \textbf{Upgrade End-of-Life Database:} Plan and execute the migration of the MySQL 5.7 database to a currently supported version (e.g., MySQL 8.0 or later). This is critical for long-term security and stability.
    \item \textbf{Implement Secure Remote Access:} For long-term database administration, replace direct access with a secure solution like a Virtual Private Network (VPN) or a bastion host that requires MFA.
\end{enumerate}

\end{document}
```