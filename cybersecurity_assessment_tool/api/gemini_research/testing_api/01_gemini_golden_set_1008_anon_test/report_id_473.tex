```latex
\documentclass[12pt]{article}

% --- PACKAGES ---
\usepackage[margin=1in]{geometry} % Set page margins
\usepackage{pifont}               % For checkmarks and crosses (\ding)
\usepackage{booktabs}             % For professional-looking tables
\usepackage{hyperref}             % For hyperlinks
\usepackage{url}                  % For URL formatting
\usepackage{seqsplit}             % To split long strings in \texttt
\usepackage{graphicx}             % For potential logos (not used here, but good practice)
\usepackage{xcolor}               % For colors

% --- DOCUMENT INFORMATION ---
\title{Cybersecurity Posture Assessment Report}
\author{Cybersecurity Analyst}
\date{\today}

% --- HYPERREF SETUP ---
\hypersetup{
    colorlinks=true,
    linkcolor=blue,
    filecolor=magenta,      
    urlcolor=cyan,
    pdftitle={Cybersecurity Posture Assessment Report},
    pdfpagemode=FullScreen,
}

% --- BEGIN DOCUMENT ---
\begin{document}

\maketitle
\hrule

\section*{Executive Summary}
This report details the findings of a cybersecurity assessment conducted for \textbf{[Organization Name]}. The analysis correlates data from an external network scan, a security controls questionnaire, and a review of pre-existing risks.

The assessment identified a \textbf{critical-risk vulnerability}: the direct exposure of Remote Desktop Protocol (RDP) on port 3389 to the public internet. This technical finding is severely exacerbated by critical gaps in organizational security controls, most notably the lack of Multi-Factor Authentication (MFA) for email access and the absence of a formal Acceptable Use Policy.

This combination of technical and procedural weaknesses places the organization at an immediate and high risk of unauthorized access, data breach, and ransomware attacks. Immediate remediation is required to mitigate these threats. This report provides a detailed breakdown of the risks and a prioritized list of actionable recommendations.

\section{Organizational Information}
The following information was used as the basis for this assessment.
\begin{itemize}
    \item \textbf{Organization Name:} \textbf{[Organization Name]}
    \item \textbf{Primary Domain:} \texttt{[Domain]}
    \item \textbf{External IP Address Scanned:} \seqsplit{\texttt{[Client IP]}}
\end{itemize}

\section{Security Control Review}
A review of the organization's security controls was conducted via a questionnaire. The responses reveal significant gaps in foundational security practices. "No" answers indicate a lack of a critical control and represent a high or critical risk.

\begin{table}[h!]
\centering
\caption{Security Controls Questionnaire Results}
\begin{tabular}{p{0.7\linewidth} c}
\toprule
\textbf{Control Question} & \textbf{Implemented} \\
\midrule
Do you require MFA to log into computers? & \ding{51} \\ % Yes
Do you require MFA to access sensitive data systems? & \ding{51} \\ % Yes
Does your organization do security awareness training for new employees? & \ding{51} \\ % Yes
\midrule
\textcolor{red}{Do you require MFA to access email?} & \textcolor{red}{\ding{55}} \\ % No
\textcolor{red}{Does your organization have an employee acceptable use policy?} & \textcolor{red}{\ding{55}} \\ % No
\textcolor{red}{Does your organization do security awareness training for all employees at least once per year?} & \textcolor{red}{\ding{55}} \\ % No
\bottomrule
\end{tabular}
\end{table}

\section{Technical Scan Results}
An external network scan was performed against the target IP address \seqsplit{\texttt{[Target IP]}}. The scan identified one open port, which presents a significant security risk.

\begin{table}[h!]
\centering
\caption{Open Port Analysis}
\begin{tabular}{llll}
\toprule
\textbf{Port} & \textbf{State} & \textbf{Service} & \textbf{Analysis} \\
\midrule
3389/tcp & Open & ms-wbt-server & This port is used for Remote Desktop Protocol (RDP). \\
& & (RDP) & Exposing RDP directly to the internet is a critical risk. \\
& & & It is a primary target for brute-force attacks and \\
& & & exploitation (e.g., BlueKeep). \\
\bottomrule
\end{tabular}
\end{table}

\section{Risk Assessment and Correlation}
The following table synthesizes findings from the security control review, the technical scan, and pre-existing risk data. The correlation between these items demonstrates a compounded risk profile.

\begin{table}[h!]
\centering
\caption{Synthesized Risk Summary}
\begin{tabular}{p{0.2\linewidth} p{0.6\linewidth} l}
\toprule
\textbf{Risk Name} & \textbf{Description & Correlation} & \textbf{Severity} \\
\midrule
\textbf{RDP Exposure} & The technical scan confirmed that RDP (port 3389) is open on \seqsplit{\texttt{[Target IP]}}. This aligns with the pre-existing known risk and presents a direct path for attackers into the internal network. & \textbf{Critical} \\
\addlinespace
\textbf{Lack of MFA on Email} & Email accounts are not protected by MFA. A compromised email account can lead to credential harvesting for other services (like RDP) or be used in sophisticated phishing attacks. This significantly increases the likelihood of a successful attack against the exposed RDP service. & \textbf{Critical} \\
\addlinespace
\textbf{Inadequate Security Policies \& Training} & The absence of an Acceptable Use Policy and annual security training for all staff creates a weak human firewall. Employees are more likely to use weak passwords or fall for phishing attacks, providing attackers with the credentials needed to exploit the open RDP port. & \textbf{High} \\
\bottomrule
\end{tabular}
\end{table}

\section{Recommendations}
To mitigate the identified risks, the following actions are recommended, prioritized by urgency.

\begin{enumerate}
    \item \textbf{Immediate (Urgency: Critical): Close Port 3389}
    \begin{itemize}
        \item \textbf{Action:} Immediately configure the perimeter firewall to block all inbound traffic to TCP port 3389 on \seqsplit{\texttt{[Target IP]}}.
        \item \textbf{Justification:} This is the single most effective step to remove the immediate threat of an external compromise via RDP.
    \end{itemize}
    
    \item \textbf{Short-Term (Urgency: Critical): Enforce MFA for Email}
    \begin{itemize}
        \item \textbf{Action:} Procure and enforce an MFA solution for all email accounts.
        \item \textbf{Justification:} Secures the organization's primary communication channel against account takeovers, which are often a precursor to larger breaches.
    \end{itemize}

    \item \textbf{Mid-Term (Urgency: High): Establish Secure Remote Access}
    \begin{itemize}
        \item \textbf{Action:} Implement a Virtual Private Network (VPN) solution with mandatory MFA for all remote access. Users should connect to the VPN first, then access internal resources like RDP.
        \item \textbf{Justification:} This provides a secure, encrypted tunnel for remote work, eliminating the need for direct exposure of services like RDP.
    \end{itemize}

    \item \textbf{Mid-Term (Urgency: High): Develop and Implement Security Policies}
    \begin{itemize}
        \item \textbf{Action:} Draft, approve, and communicate an official Employee Acceptable Use Policy.
        \item \textbf{Justification:} This policy establishes clear rules for technology use, defines security responsibilities, and forms the basis for enforcing security standards.
    \end{itemize}

    \item \textbf{Long-Term (Urgency: Medium): Implement Annual Security Training}
    \begin{itemize}
        \item \textbf{Action:} Institute a mandatory security awareness training program for all employees, to be completed annually.
        \item \textbf{Justification:} Ongoing training reinforces best practices, reduces the risk of human error, and helps staff identify and report security threats like phishing.
    \end{itemize}
\end{enumerate}

\section{Conclusion}
The current security posture of \textbf{[Organization Name]} is precarious due to a confluence of a critical technical vulnerability and significant procedural gaps. While the individual findings are serious, their combination creates a scenario where a security breach is highly probable. The recommendations outlined in this report provide a clear roadmap for mitigating these risks. Prioritizing the immediate closure of the exposed RDP port and the implementation of MFA is paramount to securing the organization.

\end{document}
```