```latex
\documentclass[12pt]{article}

% Required Packages
\usepackage[margin=1in]{geometry}
\usepackage{pifont} % For checkmarks and crosses
\usepackage{booktabs} % For professional tables
\usepackage{hyperref} % For hyperlinks
\usepackage{url} % For URL formatting
\usepackage{seqsplit} % To split long strings in texttt
\usepackage{graphicx}
\usepackage{xcolor}
\usepackage{fancyhdr}

% --- Document Setup ---
\definecolor{darkblue}{rgb}{0.0, 0.0, 0.33}
\definecolor{darkred}{rgb}{0.55, 0.0, 0.0}

\hypersetup{
    colorlinks=true,
    linkcolor=darkblue,
    filecolor=darkblue,      
    urlcolor=darkblue,
    citecolor=darkblue,
}

\pagestyle{fancy}
\fancyhf{}
\lhead{Cybersecurity Assessment Report}
\rhead{\textbf{[Organization Name]}}
\cfoot{\thepage}
\renewcommand{\headrulewidth}{0.4pt}
\renewcommand{\footrulewidth}{0.4pt}

% --- Helper Commands ---
\newcommand{\yes}{\ding{51}}
\newcommand{\no}{\ding{55}}

% --- Document Start ---
\begin{document}

% --- Title Page ---
\begin{titlepage}
    \centering
    \vspace*{1cm}
    
    \Huge
    \textbf{Cybersecurity Posture Assessment Report}
    
    \vspace{1.5cm}
    
    \Large
    Prepared for:
    
    \vspace{0.5cm}
    
    \textbf{[Organization Name]}
    
    \vspace{2cm}
    
    \includegraphics[width=0.4\textwidth]{example-image-a} % Placeholder for a logo
    
    \vfill
    
    \Large
    \today
    
\end{titlepage}

\tableofcontents
\newpage

% --- Executive Summary ---
\section*{Executive Summary}

This report provides a comprehensive analysis of the cybersecurity posture of \textbf{[Organization Name]}, based on technical network scans, a review of organizational security controls, and an assessment of pre-existing risks. The assessment identified several critical and high-risk vulnerabilities that require immediate attention to mitigate the threat of unauthorized access, data breach, and operational disruption.

The most critical findings include:
\begin{itemize}
    \item \textbf{Exposed and Outdated Database:} An external scan identified a MySQL database (version 5.7.33) on port 3306 directly accessible from the internet. This version reached its End-of-Life (EOL) in October 2023 and no longer receives security updates, posing a severe risk of exploitation.
    \item \textbf{Lack of Multi-Factor Authentication (MFA):} MFA is not enforced for employee email or computer logins. This significant control gap dramatically increases the risk of account compromise through phishing or credential theft.
    \item \textbf{Deficient Security Awareness Program:} The organization does not provide security awareness training for new or existing employees, nor does it have an Acceptable Use Policy. This leaves the organization highly vulnerable to social engineering attacks.
\end{itemize}

Immediate remediation is recommended, focusing on securing the exposed database, implementing MFA across all critical systems, and establishing a foundational security awareness program.

% --- Organizational Information ---
\section*{Organizational Information}

This section details the organizational data used for this assessment. As the provided information was anonymized, placeholders have been used.

\begin{table}[h!]
\centering
\begin{tabular}{@{}ll@{}}
\toprule
\textbf{Attribute} & \textbf{Value} \\ \midrule
Organization Name & \textbf{[Organization Name]} \\
Primary Domain & \texttt{[Domain]} \\
External IP Scanned & \texttt{[Client IP]} \\
Target IP Scanned & \texttt{[Target IP]} \\ \bottomrule
\end{tabular}
\caption{Client Organizational Details.}
\label{tab:org_info}
\end{table}

% --- Security Control Review ---
\section*{Security Control Review}

A review of administrative and organizational security controls was conducted via a questionnaire. The results, detailed in Table \ref{tab:controls}, reveal critical gaps in fundamental security practices. The absence of MFA and a formal security awareness program are of primary concern.

\begin{table}[h!]
\centering
\begin{tabular}{@{}lc@{}}
\toprule
\textbf{Control Question} & \textbf{Response} \\ \midrule
Do you require MFA to access email? & \textcolor{darkred}{\no} \\
Do you require MFA to log into computers? & \textcolor{darkred}{\no} \\
Do you require MFA to access sensitive data systems? & \textcolor{green}{\yes} \\
Does your organization have an employee acceptable use policy? & \textcolor{darkred}{\no} \\
Does your organization do security awareness training for new employees? & \textcolor{darkred}{\no} \\
Does your organization do security awareness training for all employees annually? & \textcolor{darkred}{\no} \\ \bottomrule
\end{tabular}
\caption{Organizational Security Control Questionnaire Results.}
\label{tab:controls}
\end{table}

\subsection*{Analysis of Control Gaps}
The "No" responses highlight a lack of foundational security hygiene. Without MFA on email and endpoints, the organization is highly susceptible to credential-based attacks. The absence of security training and policies means employees are likely unaware of common threats and their role in preventing them, making them easy targets for phishing and other social engineering tactics.

% --- Technical Scan Results ---
\section*{Technical Scan Results}

An external network scan was performed against the target IP address \texttt{[Target IP]}. The scan identified one open port, which presents a significant security risk.

\begin{table}[h!]
\centering
\begin{tabular}{@{}lllll@{}}
\toprule
\textbf{Port} & \textbf{State} & \textbf{Service} & \textbf{Product} & \textbf{Version} \\ \midrule
3306/tcp & Open & mysql & MySQL & 5.7.33 \\ \bottomrule
\end{tabular}
\caption{Open Ports Detected on \texttt{[Target IP]}.}
\label{tab:scan_results}
\end{table}

\subsection*{Finding 1: Publicly Exposed Database (Critical)}
The scan confirms that a MySQL database server is directly exposed to the public internet on port 3306. Database services should never be publicly accessible unless absolutely necessary and should be protected by stringent access controls. This configuration exposes the database to brute-force attacks, credential stuffing, and exploitation of known vulnerabilities.

\subsection*{Finding 2: End-of-Life (EOL) Software (Critical)}
The detected MySQL version, 5.7.33, reached its official End-of-Life in October 2023. EOL software no longer receives security patches or updates from the vendor. Any vulnerabilities discovered after this date will remain unpatched, making the system an easy target for attackers. Running EOL software, especially on a publicly exposed service, is a critical security risk.

% --- Risk Assessment Summary ---
\section*{Risk Assessment Summary}

This section synthesizes findings from the security control review, technical scan, and pre-existing risk data into a prioritized list of security risks.

\begin{table}[h!]
\centering
\begin{tabular}{@{}p{0.3\linewidth}p{0.5\linewidth}l@{}}
\toprule
\textbf{Risk Name} & \textbf{Overview} & \textbf{Severity} \\ \midrule
\textbf{Exposed and Outdated Database} & A MySQL 5.7 (EOL) database is publicly accessible on port 3306, exposing it to unpatched vulnerabilities and brute-force attacks. & \textbf{Critical} \\
\addlinespace
\textbf{Lack of Multi-Factor Authentication} & MFA is not enforced on email or computer logins, making user accounts highly vulnerable to compromise via phishing or stolen credentials. & \textbf{Critical} \\
\addlinespace
\textbf{Inadequate Security Awareness Program} & The absence of an acceptable use policy and security training leaves employees unprepared to identify and respond to social engineering attacks. & \textbf{High} \\
\addlinespace
\textbf{Database Exposure} \textit{(Existing Risk)} & Pre-existing risk confirmed by technical scan. The MySQL port 3306 is open to the network, increasing the attack surface. & \textbf{High} \\
\bottomrule
\end{tabular}
\caption{Aggregated Risk Summary.}
\label{tab:risk_summary}
\end{table}

% --- Recommendations ---
\section*{Recommendations}

The following prioritized recommendations are provided to address the identified risks and improve the overall security posture of \textbf{[Organization Name]}.

\subsection*{Priority 1: Immediate Actions (Within 72 Hours)}
\begin{enumerate}
    \item \textbf{Restrict Access to Database:} Immediately implement firewall rules to block all public access to TCP port 3306 on \texttt{[Target IP]}. Access should be restricted to trusted internal IP addresses or managed exclusively through a secure VPN connection.
    \item \textbf{Plan Database Upgrade:} Develop an urgent plan to migrate the MySQL 5.7.33 database to a currently supported version (e.g., MySQL 8.x). This is essential to ensure the system receives critical security patches.
\end{enumerate}

\subsection*{Priority 2: High-Impact Actions (Within 30 Days)}
\begin{enumerate}
    \item \textbf{Deploy Multi-Factor Authentication (MFA):} Enforce MFA for all users across all critical systems. Prioritize the following:
    \begin{itemize}
        \item Email (e.g., Office 365, Google Workspace)
        \item VPN and remote access solutions
        \item Endpoint logins (computer access)
    \end{itemize}
    \item \textbf{Implement Security Awareness Training:} Procure and roll out a security awareness training program for all employees. New hires must complete this training as part of their onboarding process, and all staff must complete it at least annually.
\end{enumerate}

\subsection*{Priority 3: Foundational Improvements (Within 90 Days)}
\begin{enumerate}
    \item \textbf{Develop and Enforce an Acceptable Use Policy (AUP):} Create a formal AUP that clearly defines the rules for using company assets, data, and networks. This policy should be reviewed and signed by all employees.
\end{enumerate}

% --- Conclusion ---
\section*{Conclusion}
The current cybersecurity posture of \textbf{[Organization Name]} contains critical deficiencies that expose the organization to significant risk. The combination of a publicly accessible, outdated database and weak access controls creates a high-likelihood path for a security breach. 

By implementing the prioritized recommendations outlined in this report, the organization can substantially reduce its attack surface, mitigate the most severe threats, and build a stronger, more resilient security foundation.

\end{document}
```