```latex
\documentclass[12pt]{article}

% Preamble: Required Packages
\usepackage[margin=1in]{geometry}
\usepackage{pifont} % For checkmarks and crosses
\usepackage{booktabs} % For professional-looking tables
\usepackage{hyperref} % For clickable links
\usepackage{url} % For formatting URLs
\usepackage{seqsplit} % To split long strings without breaking
\usepackage{xcolor} % For colors in text

% Document Metadata
\title{Cybersecurity Posture Assessment Report}
\author{Cybersecurity Analysis Division}
\date{\today}

% Hyperref Setup
\hypersetup{
    colorlinks=true,
    linkcolor=blue,
    filecolor=magenta,      
    urlcolor=cyan,
    pdftitle={Cybersecurity Posture Assessment Report},
    pdfpagemode=FullScreen,
}

\begin{document}

\maketitle
\thispagestyle{empty}
\newpage

\tableofcontents
\newpage

\section*{Executive Summary}

This report provides a comprehensive cybersecurity assessment for \textbf{[Organization Name]}, based on an analysis of network scan data, organizational security controls, and pre-existing risk information. The assessment reveals a \textbf{Critical} overall risk posture, primarily driven by the direct exposure of a Remote Desktop Protocol (RDP) service to the public internet.

The technical network scan confirmed that port 3389 (RDP) is open on the external IP address \texttt{[Client IP]}. This finding directly correlates with a known high-severity vulnerability and presents a significant and immediate threat. Attackers actively scan for and exploit exposed RDP services to gain initial access for ransomware deployment, data theft, and further network intrusion.

This critical technical vulnerability is severely compounded by significant gaps in foundational security controls identified through the organizational questionnaire. Specifically, the lack of Multi-Factor Authentication (MFA) for email and computer logins, combined with an absence of a formal security awareness training program, creates a high-probability attack path. An adversary could leverage a single compromised password—often obtained through phishing attacks which thrive in untrained environments—to gain direct, unimpeded access to the internal network.

Immediate remediation is required to address the exposed RDP service. Strategic initiatives must then be undertaken to implement comprehensive MFA and establish a robust security awareness program to mitigate future risks.

\section{Organizational Information}

The following details have been used for this assessment. Due to the anonymized nature of the provided data, placeholders have been used where necessary.

\begin{itemize}
    \item \textbf{Organization Name:} \textbf{[Organization Name]}
    \item \textbf{Primary Email Domain:} \texttt{[Domain]}
    \item \textbf{External IP Address Scanned:} \texttt{[Client IP]}
\end{itemize}

\section{Security Control Review}

A review of the organization's security controls was conducted via a questionnaire. The responses indicate several critical and high-risk gaps in the current security posture. "No" answers highlight areas that do not meet baseline security best practices and require immediate attention.

\begin{table}[h!]
\centering
\caption{Organizational Security Control Questionnaire Results}
\begin{tabular}{p{0.6\linewidth} c c}
\toprule
\textbf{Control Question} & \textbf{Response} & \textbf{Status} \\
\midrule
Do you require MFA to access email? & No & \textcolor{red}{\ding{55}} \\
Do you require MFA to log into computers? & No & \textcolor{red}{\ding{55}} \\
Do you require MFA to access sensitive data systems? & Yes & \textcolor{green}{\ding{51}} \\
Does your organization have an employee acceptable use policy? & Yes & \textcolor{green}{\ding{51}} \\
Does your organization do security awareness training for new employees? & No & \textcolor{red}{\ding{55}} \\
Does your organization do security awareness training for all employees at least once per year? & No & \textcolor{red}{\ding{55}} \\
\bottomrule
\end{tabular}
\end{table}

\subsection*{Analysis of Control Gaps}
\begin{itemize}
    \item \textbf{Lack of MFA:} The absence of MFA on email and computer logins is a critical weakness. Email is a primary target for account takeover, and compromised computers provide a direct foothold into the network.
    \item \textbf{Insufficient Security Awareness Training:} Without regular training, employees are significantly more susceptible to phishing and social engineering attacks, which are the leading causes of credential compromise.
\end{itemize}

\section{Technical Scan Results}

An external network scan was performed on the target IP address to identify accessible services.

\begin{itemize}
    \item \textbf{Target IP Address:} \texttt{[Target IP]}
    \item \textbf{Host Status:} UP
\end{itemize}

The scan identified the following open port, which represents a direct exposure of an internal service to the public internet.

\begin{table}[h!]
\centering
\caption{Open Ports and Services Detected}
\begin{tabular}{l l l l}
\toprule
\textbf{Port} & \textbf{Protocol} & \textbf{State} & \textbf{Service Name} \\
\midrule
3389 & TCP & open & ms-wbt-server (Microsoft RDP) \\
\bottomrule
\end{tabular}
\end{table}

\subsection*{Analysis of Technical Findings}
The presence of an open RDP port (3389) is a \textbf{Critical} finding. This service is a primary target for brute-force password attacks and exploitation of known vulnerabilities (e.g., BlueKeep). This directly confirms the pre-existing risk identified in \texttt{Input\_3\_Current\_Risks\_JSON} and elevates its urgency.

\section{Consolidated Risk Assessment}

By correlating the security control gaps with the technical findings and known risks, we have identified the following key risks to the organization.

\begin{table}[h!]
\centering
\caption{Summary of Identified Risks}
\begin{tabular}{p{0.2\linewidth} p{0.15\linewidth} p{0.55\linewidth}}
\toprule
\textbf{Risk Title} & \textbf{Severity} & \textbf{Description} \\
\midrule
\textbf{Publicly Exposed RDP Service} & \textbf{Critical} & Port 3389 is open to the internet, allowing attackers to attempt direct remote access. This is a common vector for ransomware attacks. This risk is confirmed by both the network scan and pre-existing intelligence. \\
\addlinespace
\textbf{Inadequate Identity and Access Management} & \textbf{Critical} & The lack of MFA for email and computer logins means a single compromised password provides an attacker with significant access. This dramatically lowers the difficulty of exploiting the exposed RDP service. \\
\addlinespace
\textbf{Deficient Security Awareness Program} & \textbf{High} & The absence of employee security training increases the likelihood of successful phishing attacks, leading to credential theft. These stolen credentials can then be used to access exposed services like RDP. \\
\bottomrule
\end{tabular}
\end{table}

\section{Recommendations}

The following prioritized recommendations are provided to mitigate the identified risks and improve the overall security posture of \textbf{[Organization Name]}.

\subsection*{Immediate Actions (0-7 Days)}
\begin{enumerate}
    \item \textbf{Close Port 3389:} Immediately implement a firewall rule to block all inbound traffic to TCP port 3389 on \texttt{[Client IP]}. This is the single most important action to take to remove the immediate threat.
\end{enumerate}

\subsection*{Short-Term Actions (1-3 Months)}
\begin{enumerate}
    \item \textbf{Implement MFA Everywhere:} Enforce MFA for all employees on all critical systems, prioritizing:
    \begin{itemize}
        \item Email (e.g., Office 365, Google Workspace).
        \item All remote access solutions (VPNs, etc.).
        \item Computer logins (Windows Hello, Duo, etc.).
    \end{itemize}
    \item \textbf{Establish Secure Remote Access:} If remote access is required, replace direct RDP exposure with a secure solution such as a Virtual Private Network (VPN) or a Zero Trust Network Access (ZTNA) gateway. All access through this solution must require MFA.
\end{enumerate}

\subsection*{Long-Term Strategy (3-12 Months)}
\begin{enumerate}
    \item \textbf{Develop a Security Awareness Program:} Implement a mandatory security awareness training program for all employees. This should include:
    \begin{itemize}
        \item Onboarding training for all new hires.
        \item At least one annual refresher training for all staff.
        \item Regular simulated phishing campaigns to test and reinforce learning.
    \end{itemize}
    \item \textbf{Conduct Regular Vulnerability Scanning:} Establish a recurring schedule for both internal and external network vulnerability scans to proactively identify and remediate security weaknesses before they can be exploited.
\end{enumerate}

\end{document}
```