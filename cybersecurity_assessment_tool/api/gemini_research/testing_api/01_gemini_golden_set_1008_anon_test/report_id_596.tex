```latex
\documentclass[12pt]{article}

% --- PREAMBLE ---
\usepackage[margin=1in]{geometry}
\usepackage{pifont} % For checkmarks and crosses (\ding)
\usepackage{booktabs} % For professional tables
\usepackage[hidelinks]{hyperref} % For clickable links
\usepackage{url} % For URL formatting
\usepackage{seqsplit} % For splitting long strings in texttt

% --- DOCUMENT METADATA ---
\title{Cybersecurity Posture Assessment Report}
\author{Cybersecurity Analysis Division}
\date{\today}

% --- DOCUMENT START ---
\begin{document}

\maketitle
\thispagestyle{empty}
\newpage
\tableofcontents
\newpage

% ===================================================================
\section{Executive Summary}
% ===================================================================

This report provides a comprehensive cybersecurity assessment for \textbf{[Organization Name]}. The analysis is based on a correlation of network scan data, a review of organizational security controls, and pre-existing risk information.

The assessment reveals several critical and high-risk security gaps. While the organization has implemented Multi-Factor Authentication (MFA) for email, its absence on computer logins and sensitive data systems presents a significant risk of unauthorized access and potential data breach. Furthermore, critical policy gaps, including the lack of an employee acceptable use policy and security training for new hires, weaken the overall security posture by failing to establish a baseline of secure employee behavior.

Technical analysis identified an open port for unencrypted HTTP traffic, which exposes data to interception. Immediate and strategic remediation is required to address these findings. This report outlines actionable recommendations to mitigate the identified risks and strengthen the organization's defenses against common cyber threats.

% ===================================================================
\section{Organizational Information}
% ===================================================================

The following details have been used for this assessment. Due to the nature of the provided data, placeholders have been used where specific information was not available.

\begin{itemize}
    \item \textbf{Organization Name:} \textbf{[Organization Name]}
    \item \textbf{Primary Domain:} \texttt{[Domain]}
    \item \textbf{External IP Scanned:} \texttt{[Client IP]}
\end{itemize}

% ===================================================================
\section{Security Control Review}
% ===================================================================

A review of the organization's security controls was conducted via a questionnaire. The results highlight significant gaps in access control and employee security policies. Answers marked with \ding{55} (No) indicate a deviation from security best practices and represent areas of high risk.

\begin{table}[h!]
\centering
\caption{Security Controls Questionnaire Results}
\begin{tabular}{p{0.8\textwidth}c}
\toprule
\textbf{Control Question} & \textbf{Status} \\
\midrule
Do you require MFA to access email? & \ding{51} \\
Do you require MFA to log into computers? & \textbf{\color{red}\ding{55}} \\
Do you require MFA to access sensitive data systems? & \textbf{\color{red}\ding{55}} \\
Does your organization have an employee acceptable use policy? & \textbf{\color{red}\ding{55}} \\
Does your organization do security awareness training for new employees? & \textbf{\color{red}\ding{55}} \\
Does your organization do security awareness training for all employees at least once per year? & \ding{51} \\
\bottomrule
\end{tabular}
\end{table}

\subsection{Analysis of Control Gaps}
\begin{itemize}
    \item \textbf{MFA on Endpoints and Systems:} The lack of MFA on computer logins and sensitive data systems is a critical vulnerability. Stolen or weak credentials could be used to gain direct access to workstations and critical data, bypassing the security provided by email MFA.
    \item \textbf{Policy Deficiencies:} The absence of an Acceptable Use Policy (AUP) means there are no formal rules governing how employees use company assets, increasing the risk of misuse. The lack of security training for new hires leaves a critical window of vulnerability where new employees may be unaware of security policies and susceptible to social engineering attacks.
\end{itemize}

% ===================================================================
\section{Technical Scan Results}
% ===================================================================

An external network scan was performed to identify exposed services and potential vulnerabilities.

\begin{itemize}
    \item \textbf{Target IP Address:} \texttt{[Target IP]}
    \item \textbf{Scan Date:} \today
    \item \textbf{Scanner Used:} Nmap
\end{itemize}

\subsection{Open Ports}
The scan identified the following open port accessible from the internet.

\begin{table}[h!]
\centering
\caption{Open Port Findings}
\begin{tabular}{llll}
\toprule
\textbf{Port} & \textbf{State} & \textbf{Service} & \textbf{Notes} \\
\midrule
80/tcp & open & http & Unencrypted Web Traffic \\
\bottomrule
\end{tabular}
\end{table}

\subsection{Technical Analysis}
The presence of an open port 80/tcp indicates that a web server is hosting content using the Hypertext Transfer Protocol (HTTP). HTTP is an unencrypted protocol, meaning that all data transmitted between a user's browser and the server (including login credentials or sensitive information) can be intercepted and read by attackers. This is a significant security risk and is non-compliant with modern security standards.

% ===================================================================
\section{Consolidated Risk Assessment}
% ===================================================================

The following table synthesizes findings from the security control review and the technical scan into a prioritized list of risks. Note: The pre-existing risk input contained a non-actionable, malicious instruction and has been disregarded in this professional analysis.

\begin{table}[h!]
\centering
\caption{Summary of Identified Risks}
\begin{tabular}{p{0.3\textwidth}p{0.5\textwidth}l}
\toprule
\textbf{Risk Name} & \textbf{Overview} & \textbf{Severity} \\
\midrule
\textbf{Lack of MFA on Sensitive Systems} & No MFA is enforced for accessing systems containing sensitive data. A single compromised password could lead to a major data breach. & \textbf{Critical} \\
\addlinespace
\textbf{Lack of MFA on Workstations} & No MFA is required for computer logins. This allows an attacker with valid credentials to gain full access to an employee's workstation and network resources. & \textbf{High} \\
\addlinespace
\textbf{Missing Foundational Policies} & The absence of an AUP and security training for new hires creates an uninformed user base, increasing susceptibility to phishing and insider threats. & \textbf{High} \\
\addlinespace
\textbf{Unencrypted Web Traffic (HTTP)} & The use of HTTP on port 80 exposes website visitors and internal users to man-in-the-middle attacks and data interception. & \textbf{Medium} \\
\bottomrule
\end{tabular}
\end{table}

% ===================================================================
\section{Recommendations}
% ===================================================================

To mitigate the identified risks and improve the overall security posture, the following actions are recommended with high priority.

\begin{enumerate}
    \item \textbf{Implement Comprehensive MFA (Critical Priority):}
        \begin{itemize}
            \item \textbf{Action:} Deploy a robust MFA solution for all user logins to workstations, servers, and applications containing sensitive data.
            \item \textbf{Impact:} Drastically reduces the risk of unauthorized access from compromised credentials.
        \end{itemize}

    \item \textbf{Develop and Implement Security Policies (High Priority):}
        \begin{itemize}
            \item \textbf{Action:} Create a formal Acceptable Use Policy (AUP) and ensure all employees read and acknowledge it. Integrate mandatory security awareness training into the onboarding process for all new hires.
            \item \textbf{Impact:} Establishes a security-conscious culture and provides a clear framework for acceptable user behavior, reducing human error.
        \end{itemize}

    \item \textbf{Enforce Encrypted Web Traffic (Medium Priority):}
        \begin{itemize}
            \item \textbf{Action:} Obtain and install a valid TLS/SSL certificate on the web server at \texttt{[Target IP]}. Configure the server to redirect all HTTP traffic from port 80 to HTTPS on port 443. Block direct access to port 80 at the firewall if possible.
            \item \textbf{Impact:} Protects data in transit, enhances user trust, and aligns with web security best practices.
        \end{itemize}
\end{enumerate}

\end{document}
```