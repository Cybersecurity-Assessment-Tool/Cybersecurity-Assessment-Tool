```latex
\documentclass[12pt]{article}

% Preamble: Required Packages and Document Setup
\usepackage[margin=1in]{geometry}
\usepackage{pifont} % For checkmarks and crosses
\usepackage{booktabs} % For professional-looking tables
\usepackage{hyperref} % For clickable links and references
\usepackage{url} % For formatting URLs
\usepackage{seqsplit} % For splitting long strings in tt font
\usepackage[T1]{fontenc}

% Document Metadata
\title{Cybersecurity Assessment Report for \textbf{[Organization Name]}}
\author{Cybersecurity Analyst}
\date{November 22, 2025}

\begin{document}

\maketitle
\thispagestyle{empty}
\newpage

\tableofcontents
\newpage

% --- 1. Executive Summary ---
\section{Executive Summary}
This report details the findings of a cybersecurity assessment conducted for \textbf{[Organization Name]} on November 22, 2025. The assessment combined a review of organizational security controls, an external network scan, and an analysis of pre-existing risks.

The overall security posture of the organization is critically weak and requires immediate and decisive action. The analysis revealed significant, fundamental gaps in foundational security controls. Key findings include:

\begin{itemize}
    \item \textbf{Critical Lack of Multi-Factor Authentication (MFA):} The absence of MFA for email and computer access exposes the organization to a high risk of account compromise through common attacks like phishing and password spraying.
    \item \textbf{No Formal Security Awareness Program:} The lack of an Acceptable Use Policy and security training for employees means the organization's human firewall is non-existent. Employees are likely unprepared to identify or properly respond to security threats.
    \item \textbf{Exposure of Outdated Software:} The external network scan identified a public-facing web server running an outdated version of Nginx (1.18.0), which has multiple known vulnerabilities.
\end{itemize}

These deficiencies create a high likelihood of a significant security incident. This report provides a detailed breakdown of these risks and offers prioritized, actionable recommendations to mitigate them effectively. We urge management to allocate resources to address these findings without delay.

% --- 2. Organizational Information ---
\section{Organizational Information}
The following information was used as the basis for this assessment. Due to the anonymized nature of the provided data, placeholders have been used where necessary.

\begin{itemize}
    \item \textbf{Organization Name:} \textbf{[Organization Name]}
    \item \textbf{Primary Domain:} \texttt{[Domain]}
    \item \textbf{Scanned External IP:} \texttt{[Client IP]}
\end{itemize}

% --- 3. Security Control Review ---
\section{Security Control Review (Questionnaire Analysis)}
A review of the organization's security controls was conducted via a standardized questionnaire. The responses indicate major gaps in administrative and access control policies. A "No" response to any of these fundamental questions represents a significant security risk.

\begin{table}[h!]
\centering
\caption{Security Controls Questionnaire Results}
\label{tab:controls}
\begin{tabular}{p{0.6\linewidth} c p{0.2\linewidth}}
\toprule
\textbf{Control Question} & \textbf{Response} & \textbf{Assessment} \\
\midrule
Do you require MFA to access email? & \ding{55} & Critical Gap \\
Do you require MFA to log into computers? & \ding{55} & Critical Gap \\
Do you require MFA to access sensitive data systems? & \ding{51} & Good Practice \\
Does your organization have an employee acceptable use policy? & \ding{55} & High Risk \\
Does your organization do security awareness training for new employees? & \ding{55} & High Risk \\
Does your organization do security awareness training for all employees at least once per year? & \ding{55} & High Risk \\
\bottomrule
\end{tabular}
\end{table}

The single positive control (MFA for sensitive systems) is significantly undermined by the lack of MFA on email and workstations, which are the primary entry vectors attackers use to gain initial access to a network.

% --- 4. Technical Scan Results ---
\section{Technical Scan Results}
An external network scan was performed to identify open ports and exposed services on the organization's public-facing infrastructure.

\begin{itemize}
    \item \textbf{Scan Target:} \texttt{[Target IP]}
    \item \textbf{Scan Date:} 2025-11-22
\end{itemize}

\subsection{Open Ports and Services}
The scan identified the following open port:

\begin{table}[h!]
\centering
\caption{Discovered Open Ports}
\label{tab:ports}
\begin{tabular}{c c l l}
\toprule
\textbf{Port} & \textbf{State} & \textbf{Service} & \textbf{Product \& Version} \\
\midrule
443/tcp & Open & https & Nginx 1.18.0 \\
\bottomrule
\end{tabular}
\end{table}

\subsection{Technical Analysis}
The scan revealed a web server running \textbf{Nginx version 1.18.0}. This version was released in April 2020 and is now significantly outdated. It is known to be vulnerable to several security issues, including request smuggling and other denial-of-service or information disclosure vulnerabilities. Running outdated software on internet-facing systems presents a high risk of compromise, as attackers can exploit these known flaws to gain unauthorized access.

% --- 5. Risk Assessment Summary ---
\section{Risk Assessment Summary}
By correlating the findings from the security control review and the technical scan, we have identified the following key risks to the organization. Pre-existing risks were not provided for this assessment.

\begin{table}[h!]
\centering
\caption{Summary of Identified Risks}
\label{tab:risks}
\begin{tabular}{p{0.1\linewidth} p{0.3\linewidth} p{0.15\linewidth} p{0.35\linewidth}}
\toprule
\textbf{Risk ID} & \textbf{Risk Name} & \textbf{Severity} & \textbf{Description} \\
\midrule
RISK-001 & Widespread Lack of MFA & \textbf{Critical} & The absence of MFA on email and workstations makes user accounts highly susceptible to takeover via phishing, credential stuffing, or password spraying. \\
\addlinespace
RISK-002 & Lack of Security Awareness Program & \textbf{Critical} & Without a formal policy (AUP) or training, employees are unable to recognize and respond to threats, making them the weakest link in the security chain. \\
\addlinespace
RISK-003 & Outdated Web Server Software & \textbf{High} & The public-facing Nginx server is running a vulnerable, end-of-life version, exposing the organization to automated exploits and targeted attacks. \\
\bottomrule
\end{tabular}
\end{table}

% --- 6. Recommendations ---
\section{Recommendations}
The following actions are recommended to mitigate the identified risks. They are prioritized based on severity and potential impact.

\subsection{Immediate Priority (0-30 Days)}
\begin{itemize}
    \item \textbf{RISK-001 Mitigation (MFA Implementation):}
    \begin{enumerate}
        \item Immediately enforce mandatory MFA for all user accounts on the primary email system (e.g., Microsoft 365, Google Workspace).
        \item Begin a phased rollout of MFA for all workstation and remote access (VPN) logins, prioritizing administrative and executive accounts.
    \end{enumerate}
    \item \textbf{RISK-003 Mitigation (Patch Management):}
    \begin{enumerate}
        \item Develop an emergency patch plan for the public-facing Nginx server.
        \item Test and deploy the latest stable version of Nginx to a staging environment before promoting it to production to ensure compatibility.
        \item Isolate the server or apply virtual patching if an immediate upgrade is not possible.
    \end{enumerate}
\end{itemize}

\subsection{High Priority (30-90 Days)}
\begin{itemize}
    \item \textbf{RISK-002 Mitigation (Policy and Training):}
    \begin{enumerate}
        \item Draft and ratify a formal Employee Acceptable Use Policy (AUP) that clearly defines rules for using company assets and data.
        \item Procure and deploy a security awareness training platform.
        \item Mandate foundational training for all current employees and integrate it into the onboarding process for all new hires.
        \item Schedule and conduct quarterly phishing simulation campaigns to measure and improve employee resilience to social engineering.
    \end{enumerate}
\end{itemize}

\end{document}
```