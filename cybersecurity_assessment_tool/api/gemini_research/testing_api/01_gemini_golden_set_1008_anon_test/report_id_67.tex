```latex
\documentclass[12pt]{article}

% ----------------------------------------------------------------------
% PREAMBLE
% ----------------------------------------------------------------------

% Package for page geometry
\usepackage[a4paper, margin=1in]{geometry}

% Package for symbols like checkmarks and crosses
\usepackage{pifont}

% Package for professional tables
\usepackage{booktabs}

% Package for hyperlinks and URLs
\usepackage[hidelinks]{hyperref}
\usepackage{url}

% Package for breaking long strings in tt font
\usepackage{seqsplit}

% Package for colors
\usepackage{xcolor}

% Define colors for risk levels
\definecolor{riskcritical}{HTML}{990000}
\definecolor{riskhigh}{HTML}{DD4B39}
\definecolor{riskmedium}{HTML}{F4B400}
\definecolor{risklow}{HTML}{4CAF50}

% Document Metadata
\title{Cybersecurity Posture Assessment Report}
\author{Cybersecurity Analyst}
\date{\today}

% ----------------------------------------------------------------------
% DOCUMENT START
% ----------------------------------------------------------------------

\begin{document}

\maketitle

\begin{center}
    \textbf{Prepared for:} \textbf{[Organization Name]} \\
    \textbf{Domain:} \seqsplit{\texttt{[Domain]}} \\
    \textbf{Assessed IP:} \seqsplit{\texttt{[Client IP]}}
\end{center}

\hrule\vspace{1em}

\tableofcontents

\newpage

% ----------------------------------------------------------------------
% SECTION 1: EXECUTIVE SUMMARY
% ----------------------------------------------------------------------
\section{Executive Summary}

This report provides a cybersecurity posture assessment for \textbf{[Organization Name]}, based on an analysis of organizational security controls, a network scan of external infrastructure, and a review of pre-existing risks. The assessment was conducted on \today.

The analysis identified several key areas of concern requiring immediate attention. The most critical findings include a lack of mandatory Multi-Factor Authentication (MFA) for computer logins and the absence of security awareness training for new employees during their onboarding process. These administrative control gaps significantly increase the risk of credential compromise and unauthorized access.

From a technical perspective, an external scan of the target IP address \seqsplit{\texttt{[Target IP]}} revealed an open Secure Shell (SSH) port (22/TCP). While common for remote administration, its public exposure constitutes a notable attack vector that must be properly secured.

No pre-existing vulnerabilities were reported. The risks identified in this report are newly generated based on the current assessment. Recommendations are provided to address each finding, with the primary goal of strengthening the organization's defense against common cyber threats.

% ----------------------------------------------------------------------
% SECTION 2: ORGANIZATIONAL INFORMATION
% ----------------------------------------------------------------------
\section{Organizational Information}

This section details the information provided by the client organization, which forms the basis for the administrative and policy-level review.

\begin{itemize}
    \item \textbf{Organization Name:} \textbf{[Organization Name]}
    \item \textbf{Primary Domain:} \seqsplit{\texttt{[Domain]}}
    \item \textbf{External IP Provided:} \seqsplit{\texttt{[Client IP]}}
\end{itemize}

% ----------------------------------------------------------------------
% SECTION 3: SECURITY CONTROL REVIEW
% ----------------------------------------------------------------------
\section{Security Control Review}

The following table summarizes the organization's responses to a security controls questionnaire. A green checkmark (\textcolor{green}{\ding{51}}) indicates a positive control is in place, while a red cross (\textcolor{red}{\ding{55}}) indicates a control gap.

\begin{table}[h!]
\centering
\caption{Security Controls Questionnaire Results}
\begin{tabular}{p{0.8\linewidth} c}
\toprule
\textbf{Control Question} & \textbf{Response} \\
\midrule
Do you require MFA to access email? & \textcolor{green}{\ding{51}} \\
Do you require MFA to log into computers? & \textcolor{red}{\ding{55}} \\
Do you require MFA to access sensitive data systems? & \textcolor{green}{\ding{51}} \\
Does your organization have an employee acceptable use policy? & \textcolor{green}{\ding{51}} \\
Does your organization do security awareness training for new employees? & \textcolor{red}{\ding{55}} \\
Does your organization do security awareness training for all employees at least once per year? & \textcolor{green}{\ding{51}} \\
\bottomrule
\end{tabular}
\end{table}

\subsection*{Analysis of Control Gaps}
Two significant control gaps were identified:
\begin{itemize}
    \item \textbf{Lack of MFA on Computer Logins:} This is a critical weakness. If an employee's credentials are stolen (e.g., via phishing), an attacker can gain direct access to their workstation and the corporate network without needing a second authentication factor. This facilitates lateral movement and deepens the impact of a breach.
    \item \textbf{No Security Training for New Employees:} New hires are often prime targets for social engineering attacks. By not providing security training during onboarding, the organization leaves a critical window of vulnerability open until the annual training cycle.
\end{itemize}

% ----------------------------------------------------------------------
% SECTION 4: TECHNICAL SCAN RESULTS
% ----------------------------------------------------------------------
\section{Technical Scan Results}

An Nmap scan was performed on the target IP address to identify open ports and exposed services.

\begin{itemize}
    \item \textbf{Target IP Address:} \seqsplit{\texttt{[Target IP]}}
    \item \textbf{Scan Date:} \today
\end{itemize}

\begin{table}[h!]
\centering
\caption{Open Ports Detected on \seqsplit{\texttt{[Target IP]}}}
\begin{tabular}{l l l l}
\toprule
\textbf{Port} & \textbf{State} & \textbf{Service (Inferred)} & \textbf{Product / Version} \\
\midrule
22/tcp & open & SSH (Secure Shell) & Not Identified \\
\bottomrule
\end{tabular}
\end{table}

\subsection*{Analysis of Technical Findings}
The scan revealed that port 22 (SSH) is open to the public internet. SSH is a standard protocol for secure remote administration of servers. However, its public exposure presents a risk if not configured securely. Potential threats include:
\begin{itemize}
    \item \textbf{Brute-force attacks:} Automated attempts to guess usernames and passwords.
    \item \textbf{Credential stuffing:} Using credentials stolen from other breaches to attempt logins.
    \item \textbf{Exploitation of vulnerabilities:} If the SSH server software is outdated, it may be vulnerable to known exploits. The scan was unable to determine the specific version in use.
\end{itemize}
This finding, when correlated with the lack of MFA on computer logins, elevates the overall risk profile.

% ----------------------------------------------------------------------
% SECTION 5: CONSOLIDATED RISK ASSESSMENT
% ----------------------------------------------------------------------
\section{Consolidated Risk Assessment}

This section synthesizes the findings from the security control review and the technical scan into a prioritized list of risks.

\begin{table}[h!]
\centering
\caption{Identified Risks and Severity}
\begin{tabular}{p{0.1\linewidth} p{0.3\linewidth} p{0.15\linewidth} p{0.35\linewidth}}
\toprule
\textbf{ID} & \textbf{Risk Title} & \textbf{Severity} & \textbf{Description} \\
\midrule
RISK-001 & Lack of MFA on Workstation Logins & \textbf{\textcolor{riskcritical}{Critical}} & The absence of a second authentication factor for computer access allows a compromised password to grant an attacker full access to an employee's workstation and potentially the internal network. \\
\addlinespace
RISK-002 & Inadequate New Employee Security Training & \textbf{\textcolor{riskhigh}{High}} & New hires are not trained on security best practices, making them highly susceptible to phishing and social engineering attacks from their first day of employment. \\
\addlinespace
RISK-003 & Exposed SSH Management Port & \textbf{\textcolor{riskmedium}{Medium}} & The SSH port is publicly accessible, increasing the attack surface. This exposes the organization to brute-force attacks and potential exploitation if the service is not hardened and monitored. \\
\bottomrule
\end{tabular}
\end{table}

% ----------------------------------------------------------------------
% SECTION 6: RECOMMENDATIONS
% ----------------------------------------------------------------------
\section{Recommendations}

The following actions are recommended to mitigate the identified risks and improve the overall security posture of \textbf{[Organization Name]}.

\subsection*{For RISK-001: Lack of MFA on Workstation Logins (Critical)}
\begin{enumerate}
    \item \textbf{Immediate Action:} Procure and implement an MFA solution for all employee computer logins (both on-site and remote).
    \item \textbf{Policy Update:} Update the organization's access control policy to mandate the use of MFA for all workstation access.
\end{enumerate}

\subsection*{For RISK-002: Inadequate New Employee Security Training (High)}
\begin{enumerate}
    \item \textbf{Immediate Action:} Develop a security awareness training module specifically for new hires. This should be a mandatory part of the employee onboarding process.
    \item \textbf{Content:} The training should cover, at a minimum: phishing identification, password security, acceptable use of company assets, and how to report a security incident.
\end{enumerate}

\subsection*{For RISK-003: Exposed SSH Management Port (Medium)}
\begin{enumerate}
    \item \textbf{Access Control:} If remote access via SSH is required, restrict access at the firewall to only trusted IP addresses (e.g., office IP, administrator home IPs via a VPN).
    \item \textbf{Configuration Hardening:}
    \begin{itemize}
        \item Disable password-based authentication and enforce the use of cryptographic keys (SSH keys).
        \item Disable root user login over SSH.
        \item Implement an intrusion prevention tool like \texttt{fail2ban} to automatically block IPs that exhibit malicious behavior.
    \end{itemize}
    \item \textbf{Vulnerability Management:} Ensure the SSH server software is kept up-to-date with the latest security patches.
\end{enumerate}

% ----------------------------------------------------------------------
% DOCUMENT END
% ----------------------------------------------------------------------
\end{document}
```