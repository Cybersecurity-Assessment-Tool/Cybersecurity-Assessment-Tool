```latex
\documentclass[12pt]{article}

% --- PACKAGES ---
\usepackage[a4paper, margin=1in]{geometry}
\usepackage{pifont} % For \ding
\usepackage{booktabs} % For professional tables
\usepackage[hidelinks]{hyperref} % For clickable links
\usepackage{url} % For URL formatting
\usepackage{seqsplit} % To split long strings in tt font
\usepackage{graphicx}
\usepackage{xcolor}

% --- DOCUMENT SETUP ---
\title{Cybersecurity Assessment Report}
\author{Cybersecurity Analysis Division}
\date{\today}

% --- COMMANDS ---
\newcommand{\yes}{\ding{51}}
\newcommand{\no}{\ding{55}}
\definecolor{darkred}{rgb}{0.55, 0.0, 0.0}
\definecolor{darkorange}{rgb}{1.0, 0.55, 0.0}
\definecolor{darkgreen}{rgb}{0.0, 0.39, 0.0}

\begin{document}

\maketitle
\thispagestyle{empty}
\newpage

\tableofcontents
\newpage

% ==============================================================================
\section*{1. Executive Summary}
% ==============================================================================

This report details the findings of a cybersecurity assessment conducted for \textbf{[Organization Name]}. The assessment combined an external network scan, a review of internal security controls via a questionnaire, and an analysis of pre-existing risks.

The analysis revealed several critical and high-risk vulnerabilities that require immediate attention. A key finding is an externally exposed FTP server running a dangerously outdated and vulnerable version of \texttt{vsftpd} (2.3.4), which is configured to allow anonymous logins. This presents a direct and trivial entry point for an attacker.

Furthermore, significant gaps were identified in organizational security controls. The lack of mandatory Multi-Factor Authentication (MFA) for email access and the absence of a formal security awareness training program for employees create a high susceptibility to phishing and social engineering attacks. When combined with the technical vulnerability, these issues create a compound risk that could lead to a full network compromise, data breach, and significant operational disruption.

Immediate remediation of the vulnerable FTP server and implementation of MFA on email are the highest priorities. Strategic improvements in employee security training and patching policies are also strongly recommended to build a more resilient security posture.

% ==============================================================================
\section*{2. Organizational Information}
% ==============================================================================

The following information was used as the basis for this assessment. Due to the anonymized nature of the provided data, placeholders have been used where necessary.

\begin{itemize}
    \item \textbf{Organization Name:} \textbf{[Organization Name]}
    \item \textbf{Primary Email Domain:} \texttt{[Domain]}
    \item \textbf{External IP Scanned:} \texttt{[Client IP]}
\end{itemize}

% ==============================================================================
\section*{3. Security Control Review (Questionnaire Analysis)}
% ==============================================================================

An assessment of internal security controls was conducted based on the provided questionnaire. The results highlight critical gaps in user access controls and security awareness. Answers marked with \no{} indicate a deviation from security best practices and represent a significant risk.

\begin{table}[h!]
\centering
\caption{Security Controls Questionnaire Results}
\begin{tabular}{p{0.7\linewidth} c}
\toprule
\textbf{Control Question} & \textbf{Response} \\
\midrule
Do you require MFA to access email? & \no \\
Do you require MFA to log into computers? & \yes \\
Do you require MFA to access sensitive data systems? & \yes \\
Does your organization have an employee acceptable use policy? & \yes \\
Does your organization do security awareness training for new employees? & \no \\
Does your organization do security awareness training for all employees at least once per year? & \no \\
\bottomrule
\end{tabular}
\end{table}

\subsection*{Analysis of Gaps}
\begin{itemize}
    \item \textbf{No MFA for Email (Critical Risk):} Email is the primary vector for phishing attacks. Without MFA, a single compromised password gives an attacker full access to an employee's mailbox, which can be used for further attacks, data exfiltration, and business email compromise (BEC).
    \item \textbf{No Security Awareness Training (High Risk):} The lack of training for both new and existing employees means staff are likely unable to recognize and report phishing attempts or other social engineering tactics. This makes the "human firewall" ineffective and significantly increases the likelihood of a successful attack.
\end{itemize}

% ==============================================================================
\section*{4. Technical Scan Results}
% ==============================================================================

An external network scan was performed on the target IP address \texttt{[Target IP]}. The scan identified one open port with a critically vulnerable service.

\subsection*{Open Ports and Services}
\begin{table}[h!]
\centering
\caption{Nmap Scan Findings for \texttt{[Target IP]}}
\begin{tabular}{c c c c}
\toprule
\textbf{Port} & \textbf{State} & \textbf{Service} & \textbf{Version} \\
\midrule
21/tcp & open & ftp & vsftpd 2.3.4 \\
\bottomrule
\end{tabular}
\end{table}

\subsection*{Vulnerability Analysis}
The scan revealed the following critical issues associated with port 21:

\begin{itemize}
    \item \textbf{Vulnerable Software Version (Critical):} The server is running \texttt{vsftpd 2.3.4}. This specific version is widely known to contain a critical backdoor vulnerability (\textbf{CVE-2011-2523}). An attacker can exploit this vulnerability to gain a command shell on the underlying server with minimal effort.
    \item \textbf{Anonymous FTP Login Allowed (Critical):} The scan confirmed that anonymous FTP login is permitted. This allows any unauthenticated user on the internet to connect to the server and potentially access, upload, or download files. This configuration is extremely dangerous and is often exploited to host malicious content or exfiltrate data.
\end{itemize}

The combination of a backdoored software version and anonymous access presents an immediate and severe threat to the security of the server and the entire network it is connected to.

% ==============================================================================
\section*{5. Consolidated Risk Assessment}
% ==============================================================================

The following table synthesizes findings from the technical scan, the controls review, and pre-existing risk data into a consolidated list of identified risks.

\begin{table}[h!]
\centering
\caption{Summary of Identified Risks}
\begin{tabular}{p{0.3\linewidth} p{0.5\linewidth} c}
\toprule
\textbf{Risk Name} & \textbf{Overview} & \textbf{Severity} \\
\midrule
\textbf{Exposed Vulnerable FTP Server} & An internet-facing server is running vsftpd 2.3.4 (CVE-2011-2523) with anonymous login enabled, allowing for remote code execution. & \textcolor{darkred}{\textbf{Critical}} \\
\addlinespace
\textbf{Lack of MFA on Email} & Employee email accounts are secured only by passwords, making them highly vulnerable to phishing and credential stuffing attacks. & \textcolor{darkred}{\textbf{Critical}} \\
\addlinespace
\textbf{Inadequate Security Training} & The absence of a security awareness program leaves the organization vulnerable to social engineering, as employees cannot identify or report threats. & \textcolor{darkorange}{\textbf{High}} \\
\addlinespace
\textbf{Outdated Windows Policy} & Workstations are running Windows 7, an unsupported OS that no longer receives security updates, making them easy to exploit once a network is breached. & \textbf{Medium} \\
\bottomrule
\end{tabular}
\end{table}

% ==============================================================================
\section*{6. Recommendations}
% ==============================================================================

Based on the findings, we recommend the following actions, prioritized by severity.

\subsection*{Immediate Actions (To be completed within 72 hours)}
\begin{enumerate}
    \item \textbf{Remediate Vulnerable FTP Server:}
    \begin{itemize}
        \item Immediately take the FTP server offline.
        \item If FTP is not a business requirement, decommission the server permanently.
        \item If a file transfer service is required, replace it with a secure alternative like SFTP (SSH File Transfer Protocol) or FTPS (FTP over SSL/TLS).
        \item Ensure any new service is fully patched and configured securely, with anonymous access disabled.
    \end{itemize}
    \item \textbf{Enforce MFA for Email:}
    \begin{itemize}
        \item Immediately enable and enforce MFA for all user accounts on the email system.
        \item Prioritize implementation for administrative and executive accounts.
    \end{itemize}
\end{enumerate}

\subsection*{Strategic Actions (To be completed within 3-6 months)}
\begin{enumerate}
    \setcounter{enumi}{2} % Continue numbering
    \item \textbf{Implement Security Awareness Training:}
    \begin{itemize}
        \item Procure and implement a security awareness training program for all employees.
        \item Ensure the program is mandatory for all new hires during their onboarding process.
        \item Conduct annual refresher training and periodic phishing simulations to maintain a high level of awareness.
    \end{itemize}
    \item \textbf{Upgrade Outdated Operating Systems:}
    \begin{itemize}
        \item Continue and prioritize the project to upgrade all workstations from Windows 7 to a modern, supported operating system like Windows 10 or 11.
        \item Develop a patch management policy to ensure all systems receive security updates in a timely manner.
    \end{itemize}
\end{enumerate}

\end{document}
```