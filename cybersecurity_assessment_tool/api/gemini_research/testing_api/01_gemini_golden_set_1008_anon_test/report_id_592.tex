```latex
\documentclass[12pt]{article}

% Preamble: Required Packages
\usepackage[margin=1in]{geometry}
\usepackage{pifont} % For checkmarks and crosses
\usepackage{booktabs} % For professional tables
\usepackage{hyperref} % For hyperlinks
\usepackage{url} % For URL formatting
\usepackage{seqsplit} % For splitting long strings in texttt
\usepackage{graphicx} % For logo (placeholder)
\usepackage{fancyhdr} % For header/footer
\usepackage{lastpage} % For page numbering

% --- Document Setup ---
\hypersetup{
    colorlinks=true,
    linkcolor=black,
    urlcolor=blue,
    pdftitle={Cybersecurity Assessment Report},
    pdfauthor={Cybersecurity Analyst},
    pdfsubject={Security Posture Analysis}
}

% --- Header and Footer ---
\pagestyle{fancy}
\fancyhf{} % Clear all header and footer fields
\fancyhead[L]{Cybersecurity Assessment Report}
\fancyhead[R]{\textbf{[Organization Name]}}
\fancyfoot[C]{Page \thepage\ of \pageref{LastPage}}
\renewcommand{\headrulewidth}{0.4pt}
\renewcommand{\footrulewidth}{0.4pt}

% --- Start of Document ---
\begin{document}

% --- Title Page ---
\begin{titlepage}
    \centering
    \vspace*{2cm}
    
    \Huge{\textbf{Cybersecurity Assessment Report}}
    
    \vspace{1.5cm}
    
    \Large{\textbf{Prepared for:}}
    
    \vspace{0.5cm}
    
    \Large{\textbf{[Organization Name]}}
    
    \vspace{2cm}
    
    \large{\textbf{Date of Report:}}
    
    \vspace{0.5cm}
    
    \large{\today}
    
    \vfill
    
    \large{
        \textbf{Report Generated By:} \\
        Expert Cybersecurity Analyst
    }
    
\end{titlepage}

\newpage
\tableofcontents
\newpage

% --- Section 1: Executive Overview ---
\section{Executive Overview}
This report provides a comprehensive cybersecurity assessment for \textbf{[Organization Name]}, synthesizing data from an external network scan, a security controls questionnaire, and a review of pre-existing risks.

The assessment reveals a mixed security posture. The organization demonstrates a significant strength in its network perimeter security, as the external scan of \texttt{[Client IP]} revealed no open ports. This indicates a well-configured firewall and a reduced external attack surface. Additionally, positive security practices are in place, including mandatory Multi-Factor Authentication (MFA) for email and computer access, as well as a consistent security awareness training program for all employees.

However, two critical gaps were identified through the security controls review. The absence of MFA for accessing sensitive data systems represents a \textbf{Critical} risk, leaving high-value assets vulnerable to unauthorized access. Furthermore, the lack of a formal Employee Acceptable Use Policy is a \textbf{High} risk, creating ambiguity regarding security responsibilities and acceptable behavior.

Immediate remediation should focus on implementing MFA for all sensitive systems and developing a comprehensive Acceptable Use Policy to strengthen the organization's overall security framework.

% --- Section 2: Organizational Information ---
\section{Organizational Information}
This section details the information provided about the organization. The data has been used to contextualize the findings of this report.

\begin{tabular}{@{}ll}
    \toprule
    \textbf{Attribute} & \textbf{Value} \\
    \midrule
    Organization Name & \textbf{[Organization Name]} \\
    Primary Email Domain & \texttt{[Domain]} \\
    External IP Address Scanned & \texttt{[Client IP]} \\
    \bottomrule
\end{tabular}

% --- Section 3: Security Control Review ---
\section{Security Control Review}
The following table summarizes the organization's responses to a security controls questionnaire. This review helps identify gaps in administrative and policy-based security measures. "No" answers indicate areas requiring immediate attention.

\begin{tabular}{@{}p{0.6\linewidth}ccp{0.2\linewidth}@{}}
    \toprule
    \textbf{Control Question} & \textbf{Status} & \textbf{Comment} \\
    \midrule
    Do you require MFA to access email? & \ding{51} & Best Practice Met \\
    Do you require MFA to log into computers? & \ding{51} & Best Practice Met \\
    Do you require MFA to access sensitive data systems? & \textcolor{red}{\ding{55}} & \textbf{Critical Gap} \\
    Does your organization have an employee acceptable use policy? & \textcolor{red}{\ding{55}} & \textbf{High Risk} \\
    Does your organization do security awareness training for new employees? & \ding{51} & Best Practice Met \\
    Does your organization do security awareness training for all employees at least once per year? & \ding{51} & Best Practice Met \\
    \bottomrule
\end{tabular}

% --- Section 4: Technical Scan Results ---
\section{Technical Scan Results}
An external network vulnerability scan was conducted to identify open ports and services exposed to the internet.

\begin{itemize}
    \item \textbf{Target IP Address:} \texttt{[Target IP]}
    \item \textbf{Scan Date:} [Scan Date Not Provided]
    \item \textbf{Scanner Used:} Nmap
\end{itemize}

\subsection{Summary of Findings}
The scan results were positive, indicating a strong network perimeter.
\begin{itemize}
    \item \textbf{Host Status:} Up
    \item \textbf{Open Ports Found:} 0
    \item \textbf{Filtered/Closed Ports:} All other ports were found to be in a closed state.
\end{itemize}

\textbf{Analysis:} The absence of open ports on the scanned external IP address is an excellent security posture. This significantly reduces the attack surface available to external threats and suggests that the organization's firewall is effectively configured to deny unsolicited inbound traffic. No further technical vulnerabilities were identified from this scan.

% --- Section 5: Risk Assessment ---
\section{Risk Assessment}
This section synthesizes findings from the security control review, technical scan, and pre-existing risk data. The pre-existing risk log was empty. The primary risks identified in this assessment stem from policy and administrative control gaps.

\begin{tabular}{@{}p{0.25\linewidth}p{0.5\linewidth}p{0.15\linewidth}@{}}
    \toprule
    \textbf{Risk Name} & \textbf{Overview} & \textbf{Severity} \\
    \midrule
    \textbf{Lack of MFA for Sensitive Data} & The absence of mandatory MFA for systems storing or processing sensitive information (e.g., financial, customer, or proprietary data) exposes these critical assets to credential theft and unauthorized access. & \textbf{Critical} \\
    \addlinespace
    \textbf{No Employee Acceptable Use Policy (AUP)} & Without a formal AUP, employees may be unaware of their security responsibilities, leading to unintentional data exposure, misuse of company assets, and a weakened security culture. This also creates legal and compliance challenges. & \textbf{High} \\
    \bottomrule
\end{tabular}

% --- Section 6: Recommendations ---
\section{Recommendations}
Based on the risk assessment, the following actions are recommended to mitigate the identified vulnerabilities and improve the overall security posture of \textbf{[Organization Name]}.

\subsection{Critical Priority}
\begin{enumerate}
    \item \textbf{Implement MFA for All Sensitive Systems:}
    \begin{itemize}
        \item \textbf{Action:} Identify all applications, databases, and administrative interfaces that handle sensitive data. Procure and deploy a robust MFA solution for these systems.
        \item \textbf{Rationale:} This is the most effective single control to prevent unauthorized access resulting from compromised credentials. It directly mitigates the highest-priority risk identified in this report.
    \end{itemize}
\end{enumerate}

\subsection{High Priority}
\begin{enumerate}
    \setcounter{enumi}{1} % Continue numbering from previous list
    \item \textbf{Develop and Implement an Acceptable Use Policy (AUP):}
    \begin{itemize}
        \item \textbf{Action:} Draft a formal AUP that clearly defines the rules for using company networks, systems, and data. The policy should be reviewed by legal counsel, communicated to all employees, and integrated into the new-hire onboarding process. Require all employees to formally acknowledge they have read and understood the policy.
        \item \textbf{Rationale:} An AUP establishes a baseline for secure behavior, clarifies employee responsibilities, and provides a framework for enforcing security standards across the organization.
    \end{itemize}
\end{enumerate}

\end{document}
```