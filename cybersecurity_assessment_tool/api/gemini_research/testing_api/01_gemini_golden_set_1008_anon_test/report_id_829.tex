```latex
\documentclass[12pt, a4paper]{article}

% Preamble: Required Packages
\usepackage[margin=1in]{geometry}
\usepackage{pifont} % For checkmarks and crosses
\usepackage{booktabs} % For professional tables
\usepackage{hyperref} % For clickable links and metadata
\usepackage{url} % For URL formatting
\usepackage{seqsplit} % To split long strings without breaking
\usepackage{graphicx}
\usepackage[table]{xcolor}
\usepackage{fancyhdr}

% --- Document Setup ---
\hypersetup{
    colorlinks=true,
    linkcolor=blue,
    filecolor=magenta,      
    urlcolor=cyan,
    pdftitle={Cybersecurity Assessment Report},
    pdfauthor={Cybersecurity Analyst},
    pdfsubject={Security Assessment},
    pdfkeywords={Security, Analysis, Report},
    bookmarks=true
}

% Define colors for risk levels
\definecolor{riskcritical}{HTML}{940000}
\definecolor{riskhigh}{HTML}{D14124}
\definecolor{riskmedium}{HTML}{E8A317}
\definecolor{risklow}{HTML}{3A7D44}
\definecolor{tableheader}{gray}{0.9}

% --- Header and Footer ---
\pagestyle{fancy}
\fancyhf{} % clear all header and footer fields
\fancyhead[L]{Cybersecurity Assessment Report}
\fancyhead[R]{\textbf{[Organization Name]}}
\fancyfoot[C]{\thepage}
\renewcommand{\headrulewidth}{0.4pt}
\renewcommand{\footrulewidth}{0.4pt}

% --- Document Start ---
\begin{document}

% --- Title Page ---
\begin{titlepage}
    \centering
    \vspace*{1cm}
    \includegraphics[width=0.4\textwidth]{example-image-a} % Placeholder logo
    \vfill
    \huge\textbf{Cybersecurity Assessment Report}
    \vspace{1.5cm}
    \Large
    \textbf{Prepared for:}\\
    \vspace{0.5cm}
    \textbf{[Organization Name]}
    \vspace{2cm}
    \large
    \textbf{Date of Assessment:}\\
    \vspace{0.5cm}
    November 22, 2025
    \vfill
    \textit{This report contains sensitive information and should be handled with care.}
\end{titlepage}

\tableofcontents
\newpage

% --- Section 1: Executive Summary ---
\section{Executive Summary}
This report details the findings of a cybersecurity assessment conducted for \textbf{[Organization Name]}. The assessment combined a review of organizational security controls, an external network scan, and an analysis of pre-existing risks. The objective is to provide a clear overview of the current security posture and offer actionable recommendations to mitigate identified vulnerabilities.

The overall security posture is considered \textbf{poor} and requires immediate attention. Several critical-risk and high-risk vulnerabilities were identified that expose the organization to significant threats, including data breaches, ransomware attacks, and unauthorized access to sensitive systems.

\textbf{Key Findings Include:}
\begin{itemize}
    \item \textbf{Critical - Widespread Lack of Multi-Factor Authentication (MFA):} The organization does not enforce MFA for email, computer logins, or access to sensitive data. This represents a critical failure in access control and makes the organization highly susceptible to credential-based attacks.
    \item \textbf{High - Vulnerable Public-Facing Web Server:} The external scan identified an outdated Nginx web server (version 1.18.0) exposed to the internet. This version has multiple publicly known vulnerabilities that could be exploited by attackers to compromise the server.
    \item \textbf{High - Absence of Foundational Security Policies:} The organization lacks a formal Employee Acceptable Use Policy. This administrative gap creates ambiguity and increases the risk of insider threats, whether malicious or accidental.
\end{itemize}

On a positive note, the organization has implemented security awareness training for all employees. However, the effectiveness of this control is severely undermined by the critical technical and administrative gaps identified. We strongly recommend that the leadership team prioritizes the remediation steps outlined in Section 6 of this report.

% --- Section 2: Organizational Information ---
\section{Organizational Information}
The following details were used as the basis for this assessment. The data provided was anonymized.

\begin{tabular}{@{}ll}
    \toprule
    \textbf{Attribute} & \textbf{Value} \\
    \midrule
    Organization Name & \textbf{[Organization Name]} \\
    Primary Email Domain & \texttt{[Domain]} \\
    Assessed External IP & \texttt{[Client IP]} \\
    Assessment Date & November 22, 2025 \\
    \bottomrule
\end{tabular}

% --- Section 3: Security Control Review ---
\section{Security Control Review}
A security questionnaire was completed to evaluate the implementation of key administrative and technical controls. The responses indicate significant gaps in fundamental security practices. A "No" response highlights a missing control and a potential area of high risk.

\rowcolors{2}{gray!10}{white}
\begin{table}[h!]
\centering
\caption{Security Controls Questionnaire Results}
\begin{tabular}{p{0.8\linewidth} c}
    \toprule
    \rowcolor{tableheader}
    \textbf{Control Question} & \textbf{Response} \\
    \midrule
    Do you require MFA to access email? & \ding{55} \\
    Do you require MFA to log into computers? & \ding{55} \\
    Do you require MFA to access sensitive data systems? & \ding{55} \\
    Does your organization have an employee acceptable use policy? & \ding{55} \\
    Does your organization do security awareness training for new employees? & \ding{51} \\
    Does your organization do security awareness training for all employees at least once per year? & \ding{51} \\
    \bottomrule
\end{tabular}
\end{table}

% --- Section 4: Technical Scan Results ---
\section{Technical Scan Results}
An external network scan was performed against the target IP address \texttt{[Target IP]} on November 22, 2025. The scan identified the following open ports and services.

\begin{table}[h!]
\centering
\caption{Open Port Scan Findings}
\begin{tabular}{l l l l l p{0.3\linewidth}}
    \toprule
    \rowcolor{tableheader}
    \textbf{Port} & \textbf{State} & \textbf{Service} & \textbf{Product} & \textbf{Version} & \textbf{Notes} \\
    \midrule
    443/tcp & Open & https & nginx & 1.18.0 & \textbf{Outdated.} This version was released in 2020 and has multiple known vulnerabilities (e.g., CVE-2021-23017). \\
    \bottomrule
\end{tabular}
\end{table}

% --- Section 5: Risk Assessment ---
\section{Risk Assessment}
The following table synthesizes findings from the security control review, technical scan, and pre-existing risk data. Each risk is assigned a severity level based on its potential impact and likelihood of exploitation. The pre-existing risk list was empty, so all findings are new as of this assessment.

\begin{table}[h!]
\centering
\caption{Summary of Identified Risks}
\begin{tabular}{p{0.1\linewidth} p{0.3\linewidth} p{0.4\linewidth} p{0.1\linewidth}}
    \toprule
    \rowcolor{tableheader}
    \textbf{Risk ID} & \textbf{Risk Title} & \textbf{Description} & \textbf{Severity} \\
    \midrule
    RISK-001 & Widespread Lack of Multi-Factor Authentication & The absence of MFA for email, computers, and sensitive systems exposes the organization to account takeovers via credential theft or phishing. & \cellcolor{riskcritical!25}\textcolor{riskcritical}{\textbf{Critical}} \\
    \addlinespace
    RISK-002 & Outdated and Vulnerable Web Server & The public-facing Nginx server is running an outdated version (1.18.0) with known vulnerabilities, which could allow an attacker to compromise the server and gain a foothold in the network. & \cellcolor{riskhigh!25}\textcolor{riskhigh}{\textbf{High}} \\
    \addlinespace
    RISK-003 & Missing Acceptable Use Policy (AUP) & The lack of a formal AUP creates ambiguity for employees regarding security responsibilities and acceptable behavior, increasing the likelihood of insider threats and accidental data exposure. & \cellcolor{riskhigh!25}\textcolor{riskhigh}{\textbf{High}} \\
    \bottomrule
\end{tabular}
\end{table}

% --- Section 6: Recommendations ---
\section{Recommendations}
The following actions are recommended to mitigate the identified risks and improve the overall security posture of \textbf{[Organization Name]}.

\subsection{Remediation for RISK-001: Lack of MFA}
\begin{itemize}
    \item \textbf{Immediate Priority:} Enforce MFA on all externally accessible services, with the highest priority on email (\texttt{[Domain]}). This is the most critical step to prevent account takeovers.
    \item \textbf{Phased Rollout:} Develop a plan to deploy MFA for all employee computer logins and access to systems containing sensitive data. Start with administrators and privileged users.
    \item \textbf{Solution Selection:} Choose a user-friendly MFA solution, such as authenticator apps (e.g., Google Authenticator, Microsoft Authenticator) or hardware tokens for high-privilege users.
\end{itemize}

\subsection{Remediation for RISK-002: Vulnerable Web Server}
\begin{itemize}
    \item \textbf{Immediate Priority:} Plan and execute an upgrade of the Nginx server on \texttt{[Target IP]} to the latest stable version. This should be done in a staging environment first to ensure compatibility.
    \item \textbf{Establish Policy:} Implement a formal patch management policy and procedure. All internet-facing systems should be reviewed for updates at least monthly.
    \item \textbf{Verify:} After the upgrade, perform a new vulnerability scan against the server to confirm that the vulnerabilities have been successfully remediated.
\end{itemize}

\subsection{Remediation for RISK-003: Missing AUP}
\begin{itemize}
    \item \textbf{Develop Policy:} Draft a comprehensive Acceptable Use Policy that clearly defines rules for using company assets, data handling, internet usage, and security responsibilities.
    \item \textbf{Communicate and Train:} Distribute the new policy to all current employees and integrate it into the onboarding process for new hires. Require all employees to read and formally acknowledge the policy.
    \item \textbf{Integrate:} Incorporate key aspects of the AUP into the annual security awareness training to reinforce its importance.
\end{itemize}

\end{document}
```