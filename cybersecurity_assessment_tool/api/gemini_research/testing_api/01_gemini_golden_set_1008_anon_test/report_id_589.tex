```latex
\documentclass[12pt]{article}

% Preamble: Required Packages
\usepackage[margin=1in]{geometry}
\usepackage{pifont} % For checkmarks and crosses
\usepackage{booktabs} % For professional tables
\usepackage{hyperref} % For hyperlinks
\usepackage{url} % For URL formatting
\usepackage{seqsplit} % For splitting long strings in tt font
\usepackage{graphicx}
\usepackage{xcolor}

% Hyperref Setup
\hypersetup{
    colorlinks=true,
    linkcolor=blue,
    filecolor=magenta,      
    urlcolor=cyan,
    pdftitle={Cybersecurity Assessment Report},
    pdfpagemode=FullScreen,
}

% Define checkmark and cross symbols for clarity
\newcommand{\cmark}{\ding{51}}%
\newcommand{\xmark}{\ding{55}}%

% --- Document Start ---
\begin{document}

% --- Title Page ---
\begin{titlepage}
    \centering
    \vspace*{1cm}
    
    \Huge
    \textbf{Cybersecurity Assessment Report}
    
    \vspace{1.5cm}
    
    \Large
    Prepared for: \\
    \vspace{0.5cm}
    \textbf{[Organization Name]}
    
    \vspace{2cm}
    
    \Large
    \textbf{Date of Report:} \today \\
    \textbf{Date of Scan:} November 22, 2025
    
    \vfill
    
    \Large
    \textbf{Generated by:} \\
    Cybersecurity Analyst
    
\end{titlepage}

% --- Table of Contents ---
\tableofcontents
\newpage

% --- Section 1: Executive Summary ---
\section{Executive Summary}

This report provides a comprehensive cybersecurity assessment for \textbf{[Organization Name]}, based on the analysis of network scan data, an organizational security questionnaire, and a review of pre-existing risks. The assessment was conducted on November 22, 2025.

The analysis reveals several critical and high-risk security deficiencies that require immediate attention. Key findings include:

\begin{itemize}
    \item \textbf{Critical Control Gaps:} A significant lack of Multi-Factor Authentication (MFA) on email and sensitive data systems exposes the organization to account takeover and data breach risks.
    \item \textbf{Policy and Training Deficiencies:} The absence of a formal Acceptable Use Policy and a security awareness training program for employees represents a high risk, making the organization more susceptible to phishing and social engineering attacks.
    \item \textbf{Technical Vulnerabilities:} The external network scan identified an internet-facing web server running an outdated version of Nginx (1.18.0). This software version has known vulnerabilities and poses a direct threat to the organization's network perimeter.
\end{itemize}

This report details these findings and provides actionable recommendations to mitigate the identified risks and strengthen the overall security posture of \textbf{[Organization Name]}.

% --- Section 2: Organizational Information ---
\section{Organizational Information}

The following information was used as the basis for this assessment. Where data was not provided, placeholders have been used.

\begin{itemize}
    \item \textbf{Organization Name:} \textbf{[Organization Name]}
    \item \textbf{Primary Domain:} \texttt{[Domain]}
    \item \textbf{Target IP Address:} \texttt{[Client IP]}
\end{itemize}

% --- Section 3: Security Control Review ---
\section{Security Control Review}

A review of the organization's security controls was conducted via a questionnaire. The responses indicate significant gaps in foundational security practices. A summary of the findings is presented in Table \ref{tab:controls}.

\begin{table}[h!]
\centering
\caption{Security Control Questionnaire Results}
\label{tab:controls}
\begin{tabular}{p{0.8\linewidth} c}
\toprule
\textbf{Control Question} & \textbf{Response} \\
\midrule
Do you require MFA to access email? & \textcolor{red}{\xmark} \\
Do you require MFA to log into computers? & \textcolor{green}{\cmark} \\
Do you require MFA to access sensitive data systems? & \textcolor{red}{\xmark} \\
Does your organization have an employee acceptable use policy? & \textcolor{red}{\xmark} \\
Does your organization do security awareness training for new employees? & \textcolor{red}{\xmark} \\
Does your organization do security awareness training for all employees at least once per year? & \textcolor{red}{\xmark} \\
\bottomrule
\end{tabular}
\end{table}

\subsection*{Analysis of Control Gaps}
The "No" responses highlight critical weaknesses:
\begin{itemize}
    \item \textbf{Lack of MFA:} Failure to enforce MFA on email and sensitive data systems drastically increases the risk of unauthorized access from compromised credentials. Email is a primary target for attackers seeking to pivot into other systems.
    \item \textbf{Lack of Policy and Training:} Without an Acceptable Use Policy, employees may not understand their responsibilities regarding data protection and system usage. The complete absence of security awareness training leaves the organization highly vulnerable to human-targeted attacks like phishing, which are the root cause of most security breaches.
\end{itemize}

% --- Section 4: Technical Scan Results ---
\section{Technical Scan Results}

An external network scan was performed to identify exposed services and potential vulnerabilities on the perimeter.

\begin{itemize}
    \item \textbf{Target IP:} \texttt{[Target IP]}
    \item \textbf{Scan Date:} 2025-11-22T10:00:00Z
\end{itemize}

The scan identified the following open port and service:

\begin{table}[h!]
\centering
\caption{Open Ports and Services}
\label{tab:nmap}
\begin{tabular}{l l l l l}
\toprule
\textbf{Port} & \textbf{State} & \textbf{Service} & \textbf{Product} & \textbf{Version} \\
\midrule
443/tcp & open & https & nginx & 1.18.0 \\
\bottomrule
\end{tabular}
\end{table}

\subsection*{Analysis of Technical Findings}
The scan revealed that port 443 (HTTPS) is open and running \textbf{Nginx version 1.18.0}. This version was released in April 2020 and is now considered outdated. It is known to be affected by multiple security vulnerabilities, some of which can lead to information disclosure, request smuggling, or denial of service, depending on the specific configuration. Exposing outdated software to the internet presents a significant and unnecessary risk.

% --- Section 5: Consolidated Risk Assessment ---
\section{Consolidated Risk Assessment}

This section correlates findings from the control review and technical scan to present a consolidated list of identified risks. No pre-existing risks were provided for this assessment.

\begin{table}[h!]
\centering
\caption{Summary of Identified Risks}
\label{tab:risks}
\begin{tabular}{p{0.15\linewidth} p{0.6\linewidth} l}
\toprule
\textbf{Risk ID} & \textbf{Description} & \textbf{Severity} \\
\midrule
RISK-001 & \textbf{Lack of Multi-Factor Authentication:} Critical systems, including email and sensitive data repositories, are protected only by passwords, making them highly susceptible to account takeover. & \textbf{Critical} \\
\addlinespace
RISK-002 & \textbf{Outdated Web Server Software:} An externally facing web server is running Nginx 1.18.0, a version with publicly known vulnerabilities that could be exploited by attackers. & \textbf{High} \\
\addlinespace
RISK-003 & \textbf{Inadequate Security Policy and Training:} The absence of an Acceptable Use Policy and any form of security awareness training leaves the organization vulnerable to insider threats and external social engineering attacks. & \textbf{High} \\
\bottomrule
\end{tabular}
\end{table}

% --- Section 6: Recommendations ---
\section{Recommendations}

The following actions are recommended to mitigate the identified risks and improve the security posture of \textbf{[Organization Name]}. Recommendations are prioritized by severity.

\begin{enumerate}
    \item \textbf{[Critical] Implement Comprehensive MFA:}
    \begin{itemize}
        \item Immediately enforce MFA for all user access to email services (e.g., Microsoft 365, Google Workspace).
        \item Deploy MFA for all systems containing sensitive or critical business data, including remote access solutions (VPN) and cloud applications.
    \end{itemize}

    \item \textbf{[High] Remediate Vulnerable Web Server:}
    \begin{itemize}
        \item Plan and execute an upgrade of the Nginx server from version 1.18.0 to the latest stable version.
        \item Implement a patch management process to ensure all internet-facing systems are kept up-to-date with security patches on a regular basis.
    \end{itemize}

    \item \textbf{[High] Establish a Security Awareness and Policy Program:}
    \begin{itemize}
        \item Develop and implement a formal Employee Acceptable Use Policy that all staff must read and acknowledge.
        \item Institute a mandatory security awareness training program. This program should be required for all new hires during onboarding and for all employees on an annual basis. Training should cover topics such as phishing, password security, and data handling.
    \end{itemize}
\end{enumerate}

% --- Section 7: Conclusion ---
\section{Conclusion}

The assessment for \textbf{[Organization Name]} has identified significant risks stemming from both policy-level and technical control deficiencies. While the discovery of these issues is concerning, they are addressable. By implementing the recommendations outlined in this report—prioritizing MFA enforcement, upgrading vulnerable software, and establishing a robust security training program—the organization can substantially reduce its attack surface and build a more resilient security foundation.

\end{document}
```