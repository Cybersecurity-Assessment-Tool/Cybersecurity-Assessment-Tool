```latex
\documentclass[12pt]{article}

% Required Packages
\usepackage[margin=1in]{geometry}
\usepackage{pifont} % For checkmarks and crosses
\usepackage{booktabs} % For professional tables
\usepackage{hyperref} % For clickable links
\usepackage{url} % For URL formatting
\usepackage{seqsplit} % To split long strings in tt font
\usepackage{xcolor} % For colors

% Hyperref Setup
\hypersetup{
    colorlinks=true,
    linkcolor=blue,
    filecolor=magenta,      
    urlcolor=cyan,
    pdftitle={Cybersecurity Posture Assessment Report},
    pdfpagemode=FullScreen,
}

% Define checkmark and cross symbols for convenience
\newcommand{\cmark}{\ding{51}}%
\newcommand{\xmark}{\ding{55}}%

\begin{document}

% --- Title Page ---
\begin{titlepage}
    \centering
    \vspace*{\stretch{1.0}}
    \Huge\textbf{Cybersecurity Posture Assessment Report}
    \vspace{0.5cm}
    \LARGE For
    \vspace{0.5cm}
    \LARGE\textbf{[Organization Name]}
    \vspace{\stretch{2.0}}
    \large
    \textbf{Report Date:} November 22, 2025 \\
    \textbf{Prepared By:} Cybersecurity Analyst
    \vspace*{\stretch{1.0}}
\end{titlepage}

\tableofcontents
\newpage

% --- 1. Executive Summary ---
\section{Executive Summary}

This report details the findings of a cybersecurity posture assessment conducted for \textbf{[Organization Name]} on November 22, 2025. The assessment combined a review of organizational security controls via a questionnaire, an external network scan, and an analysis of pre-existing risks.

The overall security posture reveals a mix of implemented controls and critical deficiencies. While the organization has successfully implemented Multi-Factor Authentication (MFA) for computer and sensitive data system access, there are significant gaps in foundational security practices. 

Key findings include:
\begin{itemize}
    \item \textbf{Critical Risk:} The absence of MFA on email exposes the organization to a high likelihood of account compromise and subsequent business email compromise (BEC) attacks.
    \item \textbf{High Risk:} A complete lack of a formal security awareness training program and an employee acceptable use policy creates a significant human-based risk factor. Employees are likely unaware of current threats and their security responsibilities.
    \item \textbf{Medium Risk:} The external-facing web server is running an outdated version of nginx (1.18.0), which may contain unpatched vulnerabilities.
\end{itemize}

Immediate remediation should focus on implementing MFA for email and establishing a baseline security awareness program. Addressing these foundational issues will substantially improve the organization's resilience against common cyber threats.

% --- 2. Organizational Information ---
\section{Organizational Information}

This section contains the high-level information used as the basis for this assessment. Due to the anonymized nature of the provided data, placeholders have been used.

\begin{tabular}{@{}ll}
    \toprule
    \textbf{Attribute} & \textbf{Value} \\
    \midrule
    Organization Name & \textbf{[Organization Name]} \\
    Primary Email Domain & \texttt{[Domain]} \\
    External IP Address Scanned & \texttt{[Client IP]} \\
    \bottomrule
\end{tabular}

% --- 3. Security Control Review ---
\section{Security Control Review}

The following table summarizes the organization's responses to a security controls questionnaire. A green checkmark (\cmark) indicates a positive control is in place, while a red cross (\xmark) indicates a control gap that introduces risk.

\begin{table}[h!]
\centering
\begin{tabular}{p{0.7\linewidth} c}
    \toprule
    \textbf{Control Question} & \textbf{Response} \\
    \midrule
    Do you require MFA to access email? & \textcolor{red}{\xmark} \\
    Do you require MFA to log into computers? & \textcolor{green}{\cmark} \\
    Do you require MFA to access sensitive data systems? & \textcolor{green}{\cmark} \\
    Does your organization have an employee acceptable use policy? & \textcolor{red}{\xmark} \\
    Does your organization do security awareness training for new employees? & \textcolor{red}{\xmark} \\
    Does your organization do security awareness training for all employees at least once per year? & \textcolor{red}{\xmark} \\
    \bottomrule
\end{tabular}
\caption{Organizational Security Controls Questionnaire Results.}
\end{table}

The identified gaps in email security, acceptable use policy, and security awareness training are significant and are detailed in the Risk Assessment section of this report.

% --- 4. Technical Scan Results ---
\section{Technical Scan Results}

An external network scan was performed to identify open ports and exposed services.

\begin{itemize}
    \item \textbf{Scan Target:} \texttt{[Target IP]}
    \item \textbf{Scan Date:} 2025-11-22
\end{itemize}

The following table details the services discovered on the target system.

\begin{table}[h!]
\centering
\begin{tabular}{lllll}
    \toprule
    \textbf{Port} & \textbf{State} & \textbf{Service} & \textbf{Product} & \textbf{Version} \\
    \midrule
    443/tcp & open & https & nginx & 1.18.0 \\
    \bottomrule
\end{tabular}
\caption{Open Ports and Services Detected.}
\end{table}

\subsection{Analysis of Technical Findings}
The scan identified a single open port, 443 (HTTPS), which is standard for secure web traffic. The service is identified as nginx version 1.18.0. This version was released in April 2020. While it may receive security patches from an underlying OS distribution (e.g., Ubuntu LTS), it is considered outdated compared to current stable (1.26.x) or mainline branches. Running older software versions increases the risk of exposure to known and unpatched vulnerabilities.

% --- 5. Risk Assessment ---
\section{Risk Assessment}

This section synthesizes the findings from the security control review and the technical scan into a prioritized list of identified risks. No pre-existing vulnerabilities were reported.

\begin{table}[h!]
\centering
\begin{tabular}{p{0.1\linewidth} p{0.25\linewidth} p{0.4\linewidth} p{0.1\linewidth}}
    \toprule
    \textbf{Risk ID} & \textbf{Risk Name} & \textbf{Overview} & \textbf{Severity} \\
    \midrule
    RISK-001 & No MFA on Email & The lack of Multi-Factor Authentication on email accounts makes them highly susceptible to takeover via phishing or credential stuffing. This is a primary vector for Business Email Compromise (BEC). & \textbf{Critical} \\
    \addlinespace
    RISK-002 & Lack of Security Awareness Program & Without training, employees are unable to recognize and appropriately respond to phishing, social engineering, and other common threats, making them the weakest link in the security chain. & \textbf{High} \\
    \addlinespace
    RISK-003 & No Acceptable Use Policy (AUP) & The absence of a formal AUP means there are no clear, enforceable rules for employees regarding the use of company assets. This can lead to unsafe behavior and insider threats. & \textbf{High} \\
    \addlinespace
    RISK-004 & Outdated Web Server Software & The public-facing web server is running nginx 1.18.0, an older version that may lack security patches for recently discovered vulnerabilities, increasing the risk of a server compromise. & \textbf{Medium} \\
    \bottomrule
\end{tabular}
\caption{Summary of Identified Risks.}
\end{table}

% --- 6. Recommendations ---
\section{Recommendations}

Based on the identified risks, the following prioritized actions are recommended to improve the security posture of \textbf{[Organization Name]}.

\begin{description}
    \item[\textbf{RISK-001 (Critical):}] \textbf{Implement MFA for Email Immediately.}
    \begin{itemize}
        \item Procure and deploy an MFA solution compatible with your email provider (e.g., Microsoft 365, Google Workspace).
        \item Enforce MFA for all user accounts, including administrative and service accounts, without exception.
        \item Provide clear instructions to all employees on how to enroll in and use MFA.
    \end{itemize}
    
    \item[\textbf{RISK-002 (High):}] \textbf{Establish a Security Awareness Program.}
    \begin{itemize}
        \item Implement mandatory security awareness training for all new hires as part of their onboarding process.
        \item Conduct annual, mandatory security awareness training for all existing employees.
        \item Perform regular phishing simulation campaigns to test and reinforce employee knowledge.
    \end{itemize}
    
    \item[\textbf{RISK-003 (High):}] \textbf{Develop and Enforce an Acceptable Use Policy.}
    \begin{itemize}
        \item Draft a clear AUP that defines the rules for using company networks, devices, and data.
        \item Have the policy reviewed by legal and HR departments.
        \item Require all employees to read and formally acknowledge the policy.
    \end{itemize}
    
    \item[\textbf{RISK-004 (Medium):}] \textbf{Upgrade Web Server Software.}
    \begin{itemize}
        \item Plan and schedule a maintenance window to upgrade the nginx server on \texttt{[Target IP]} to a currently supported stable version.
        \item Before upgrading, test the new version in a staging environment to ensure application compatibility.
        \item Implement a patch management process to ensure the server software is kept up-to-date going forward.
    \end{itemize}
\end{description}

\end{document}
```