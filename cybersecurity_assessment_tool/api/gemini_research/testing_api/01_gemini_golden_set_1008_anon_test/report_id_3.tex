```latex
\documentclass[12pt]{article}

% ----------------------------------------------------------------------
% PREAMBLE
% ----------------------------------------------------------------------
\usepackage[margin=1in]{geometry}
\usepackage{pifont} % For checkmarks and crosses
\usepackage{booktabs} % For professional tables
\usepackage{hyperref} % For clickable links and references
\usepackage{url} % For formatting URLs
\usepackage{seqsplit} % For splitting long strings in tt font
\usepackage{xcolor} % For colors

% Hyperref setup
\hypersetup{
    colorlinks=true,
    linkcolor=blue,
    filecolor=magenta,      
    urlcolor=cyan,
    pdftitle={Cybersecurity Posture Report},
    pdfpagemode=FullScreen,
}

% Define custom colors for severity
\definecolor{criticalred}{HTML}{990000}
\definecolor{highorange}{HTML}{E69138}
\definecolor{mediumyellow}{HTML}{F1C232}

% Custom commands for severity
\newcommand{\sevCRITICAL}{\textcolor{criticalred}{\textbf{Critical}}}
\newcommand{\sevHIGH}{\textcolor{highorange}{\textbf{High}}}
\newcommand{\sevMEDIUM}{\textcolor{mediumyellow}{\textbf{Medium}}}

% Checkmark and Cross symbols
\newcommand{\cmark}{\ding{51}}
\newcommand{\xmark}{\ding{55}}

% ----------------------------------------------------------------------
% DOCUMENT START
% ----------------------------------------------------------------------
\begin{document}

% ----------------------------------------------------------------------
% TITLE PAGE
% ----------------------------------------------------------------------
\begin{titlepage}
    \centering
    \vspace*{1cm}
    \Huge
    \textbf{Cybersecurity Posture Report}
    \vspace{1.5cm}
    \Large
    Prepared for: \\
    \vspace{0.5cm}
    \textbf{[Organization Name]}
    \vfill
    \large
    \textbf{Analysis Date:} \today \\
    \textbf{Author:} Cybersecurity Analyst
\end{titlepage}

\tableofcontents
\newpage

% ----------------------------------------------------------------------
% SECTION 1: EXECUTIVE OVERVIEW
% ----------------------------------------------------------------------
\section{Executive Overview}
This report provides a comprehensive analysis of the cybersecurity posture of \textbf{[Organization Name]}. The assessment is based on a correlation of external network scan data, a review of internal security controls via a questionnaire, and an analysis of pre-existing risk documentation.

The overall security posture is assessed as \sevCRITICAL. This is due to the convergence of several high-risk factors:
\begin{itemize}
    \item \textbf{Critical Technical Vulnerabilities:} An externally-facing MySQL database was identified. This service is running an End-of-Life (EoL) version, which no longer receives security updates, exposing the organization to numerous known and future exploits.
    \item \textbf{Significant Control Gaps:} Foundational security controls are absent. The lack of Multi-Factor Authentication (MFA) for computer and sensitive data access creates a substantial risk of unauthorized access and credential compromise.
    \item \textbf{Policy and Training Deficiencies:} The absence of an employee Acceptable Use Policy and a recurring annual security awareness training program indicates a lack of security governance and a workforce that may be unprepared to identify and respond to modern cyber threats.
\end{itemize}

Immediate remediation is required to address the exposed database and implement critical access controls. Strategic initiatives must be undertaken to develop and enforce security policies and training programs to build a more resilient security foundation.

% ----------------------------------------------------------------------
% SECTION 2: ORGANIZATIONAL INFORMATION
% ----------------------------------------------------------------------
\section{Organizational Information}
The following information was used as the basis for this assessment. Due to the anonymized nature of the provided data, placeholders have been used where necessary.

\begin{tabular}{@{}ll}
    \toprule
    \textbf{Attribute} & \textbf{Value} \\
    \midrule
    Organization Name & \textbf{[Organization Name]} \\
    Primary Domain & \texttt{[Domain]} \\
    External IP Address Scanned & \texttt{[Client IP]} \\
    \bottomrule
\end{tabular}

% ----------------------------------------------------------------------
% SECTION 3: SECURITY CONTROL REVIEW
% ----------------------------------------------------------------------
\section{Security Control Review}
The following table summarizes the organization's responses to a security controls questionnaire. "No" answers represent significant gaps in the security framework and are highlighted as risks.

\begin{table}[h!]
\centering
\begin{tabular}{@{}p{8cm} c l@{}}
    \toprule
    \textbf{Control Question} & \textbf{Response} & \textbf{Assessment} \\
    \midrule
    Do you require MFA to access email? & \cmark & Good Practice \\
    Do you require MFA to log into computers? & \xmark & \sevHIGH\ Risk \\
    Do you require MFA to access sensitive data systems? & \xmark & \sevCRITICAL\ Gap \\
    Does your organization have an employee acceptable use policy? & \xmark & \sevHIGH\ Risk \\
    Does your organization do security awareness training for new employees? & \cmark & Good Practice \\
    Does your organization do security awareness training for all employees at least once per year? & \xmark & \sevHIGH\ Risk \\
    \bottomrule
\end{tabular}
\caption{Security Controls Questionnaire Analysis.}
\end{table}

The lack of MFA on computers and sensitive systems, combined with the absence of an Acceptable Use Policy and annual training, significantly increases the risk of a security incident caused by either malicious action or human error.

% ----------------------------------------------------------------------
% SECTION 4: TECHNICAL SCAN RESULTS
% ----------------------------------------------------------------------
\section{Technical Scan Results}
An external network scan was performed on the target IP address.

\subsection{Scan Details}
\begin{itemize}
    \item \textbf{Target IP:} \texttt{[Target IP]}
    \item \textbf{Scan Tool:} Nmap
\end{itemize}

\subsection{Open Ports Discovered}
The following table details the open ports and services discovered on the target system.

\begin{table}[h!]
\centering
\begin{tabular}{@{}l l l l@{}}
    \toprule
    \textbf{Port} & \textbf{State} & \textbf{Service} & \textbf{Product \& Version} \\
    \midrule
    3306/tcp & open & mysql & MySQL 5.7.33 \\
    \bottomrule
\end{tabular}
\caption{Discovered Open Ports and Services.}
\end{table}

\subsection{Analysis of Findings}
The scan identified one critical finding:
\begin{itemize}
    \item \textbf{Exposed Database Service:} Port 3306 is open to the public internet, which corresponds to the MySQL database service. Exposing a database directly to the internet is a highly discouraged practice as it dramatically increases the attack surface.
    \item \textbf{End-of-Life Software:} The detected version, \textbf{MySQL 5.7.33}, reached its official End of Life (EoL) in \textbf{October 2023}. EoL software no longer receives security patches from the vendor, making it an easy target for attackers who can exploit publicly known vulnerabilities.
\end{itemize}

% ----------------------------------------------------------------------
% SECTION 5: RISK ASSESSMENT SUMMARY
% ----------------------------------------------------------------------
\section{Risk Assessment Summary}
The following table synthesizes findings from the technical scan, control review, and existing risk data into a prioritized list of security risks.

\begin{table}[h!]
\centering
\begin{tabular}{@{}p{3.5cm} l p{5cm} p{3.5cm}@{}}
    \toprule
    \textbf{Risk Name} & \textbf{Severity} & \textbf{Description} & \textbf{Affected Elements} \\
    \midrule
    End-of-Life Software in Use & \sevCRITICAL & The public-facing MySQL database is running version 5.7.33, which is past its End-of-Life and is unpatched against known vulnerabilities. & \texttt{[Target IP]}:3306 \\
    \addlinespace
    Database Exposure & \sevHIGH & The MySQL database port (3306) is open to the public internet, inviting brute-force attacks and direct exploitation attempts. & \texttt{[Target IP]}:3306 \\
    \addlinespace
    Insufficient Access Control & \sevHIGH & Multi-Factor Authentication (MFA) is not enforced for computer logins or access to sensitive data systems, increasing the risk of credential theft and lateral movement. & User workstations, sensitive data repositories \\
    \addlinespace
    Inadequate Security Governance & \sevMEDIUM & The lack of an Acceptable Use Policy (AUP) means there are no formal rules for employees regarding the use of company assets, data handling, and security responsibilities. & All employees, company data \\
    \addlinespace
    Deficient Security Training & \sevMEDIUM & Without mandatory annual security training, employees' ability to recognize and respond to threats like phishing and social engineering will diminish over time. & All employees \\
    \bottomrule
\end{tabular}
\caption{Aggregated Security Risks.}
\end{table}

% ----------------------------------------------------------------------
% SECTION 6: RECOMMENDATIONS
% ----------------------------------------------------------------------
\section{Recommendations}
The following actions are recommended to mitigate the identified risks. They are prioritized into immediate and strategic categories.

\subsection{Immediate Actions (Remediate within 30 days)}
\begin{enumerate}
    \item \textbf{Isolate the Database:} Immediately implement firewall rules to restrict access to port 3306. Access should only be permitted from trusted internal IP addresses or via a secure VPN connection. This directly mitigates the \textit{Database Exposure} risk.
    \item \textbf{Plan Database Upgrade:} Initiate a project to upgrade the MySQL 5.7.33 instance to a currently supported version (e.g., MySQL 8.x). This is the only way to mitigate the \textit{End-of-Life Software} risk and receive critical security patches.
    \item \textbf{Deploy MFA:} Prioritize the deployment of MFA for all access to sensitive data systems and for all computer logins. This is the single most effective control to mitigate the \textit{Insufficient Access Control} risk.
\end{enumerate}

\subsection{Strategic Actions (Remediate within 90-180 days)}
\begin{enumerate}
    \item \textbf{Develop an Acceptable Use Policy (AUP):} Create and implement a formal AUP that all employees must read and acknowledge. This policy should define the rules for using company technology and data, mitigating the \textit{Inadequate Security Governance} risk.
    \item \textbf{Establish an Annual Security Training Program:} Implement a mandatory security awareness training program for all employees to be completed annually. This program should cover topics such as phishing, password security, and data handling to mitigate the \textit{Deficient Security Training} risk.
\end{enumerate}

% ----------------------------------------------------------------------
% DOCUMENT END
% ----------------------------------------------------------------------
\end{document}
```