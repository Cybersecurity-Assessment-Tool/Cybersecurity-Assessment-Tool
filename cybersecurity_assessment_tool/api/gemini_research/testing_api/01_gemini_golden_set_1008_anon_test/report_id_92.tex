```latex
\documentclass[12pt]{article}

% --- PACKAGES ---
\usepackage[margin=1in]{geometry}
\usepackage{pifont} % For checkmarks and crosses
\usepackage{booktabs} % For professional tables
\usepackage{hyperref} % For hyperlinks
\usepackage{url} % For URL formatting
\usepackage{seqsplit} % To split long strings in tt font

% --- DOCUMENT METADATA ---
\title{Cybersecurity Posture Assessment Report}
\author{Cybersecurity Analyst}
\date{\today}

% --- HYPERREF SETUP ---
\hypersetup{
    colorlinks=true,
    linkcolor=black,
    urlcolor=blue,
    pdftitle={Cybersecurity Posture Assessment Report},
    pdfauthor={Cybersecurity Analyst},
}

% --- DOCUMENT START ---
\begin{document}

\maketitle
\thispagestyle{empty}
\newpage

\tableofcontents
\newpage

% ==============================================================================
% 1. EXECUTIVE SUMMARY
% ==============================================================================
\section{Executive Summary}

This report details the findings of a cybersecurity posture assessment for \textbf{[Organization Name]}. The assessment incorporated a review of organizational security controls, an external network scan, and an analysis of pre-existing risk data.

The analysis reveals several critical and high-risk security deficiencies that require immediate attention. The most significant findings include:

\begin{itemize}
    \item \textbf{Critical - Widespread Lack of Multi-Factor Authentication (MFA):} The organization has not implemented MFA for email, computer logins, or access to sensitive data systems. This represents a critical vulnerability, as compromised credentials could lead to a significant data breach.
    \item \textbf{High - Unencrypted Web Traffic:} The external network scan identified a web server operating over unencrypted HTTP (Port 80). This exposes any data transmitted to or from the server, including potential credentials or sensitive information, to interception.
    \item \textbf{High - Foundational Policy Gaps:} The organization lacks a formal Acceptable Use Policy (AUP) and does not provide recurring annual security awareness training for all staff. These gaps indicate a low level of security maturity and increase the risk of insider threats and human error.
\end{itemize}

The overall security posture is considered weak. The combination of technical vulnerabilities and policy gaps creates a high-risk environment. We strongly recommend prioritizing the remediation steps outlined in Section \ref{sec:recommendations} to mitigate these risks effectively.

% ==============================================================================
% 2. ORGANIZATIONAL INFORMATION
% ==============================================================================
\section{Organizational Information}

This section provides the organizational details relevant to this assessment. The data has been anonymized as per the engagement requirements.

\begin{table}[h!]
\centering
\begin{tabular}{@{}ll@{}}
\toprule
\textbf{Attribute} & \textbf{Value} \\ \midrule
Organization Name & \textbf{[Organization Name]} \\
Primary Email Domain & \seqsplit{\texttt{[Domain]}} \\
External IP Address & \seqsplit{\texttt{[Client IP]}} \\ \bottomrule
\end{tabular}
\caption{Client Organizational Details.}
\end{table}

% ==============================================================================
% 3. SECURITY CONTROL REVIEW
% ==============================================================================
\section{Security Control Review}

A review of internal security controls was conducted via a questionnaire. The responses highlight significant gaps in fundamental security practices, particularly concerning identity and access management and security governance.

\begin{table}[h!]
\centering
\begin{tabular}{@{}p{0.75\textwidth}c@{}}
\toprule
\textbf{Control Question} & \textbf{Response} \\ \midrule
Do you require MFA to access email? & \ding{55} \\
Do you require MFA to log into computers? & \ding{55} \\
Do you require MFA to access sensitive data systems? & \ding{55} \\
Does your organization have an employee acceptable use policy? & \ding{55} \\
Does your organization do security awareness training for new employees? & \ding{51} \\
Does your organization do security awareness training for all employees at least once per year? & \ding{55} \\ \bottomrule
\end{tabular}
\caption{Organizational Security Controls Questionnaire. (\ding{51}=Yes, \ding{55}=No)}
\label{tab:controls}
\end{table}

\paragraph{Analysis:} The absence of MFA across all critical access points is a critical failure. The lack of an Acceptable Use Policy and annual security training further exacerbates the organization's risk profile by failing to establish clear security expectations and maintain employee vigilance.

% ==============================================================================
% 4. TECHNICAL SCAN RESULTS
% ==============================================================================
\section{Technical Scan Results}

An external network scan was performed to identify open ports and exposed services on the organization's public-facing infrastructure.

\begin{itemize}
    \item \textbf{Target IP Address:} \seqsplit{\texttt{[Target IP]}}
    \item \textbf{Host Status:} Up
\end{itemize}

The following open port was identified:

\begin{table}[h!]
\centering
\begin{tabular}{@{}llll@{}}
\toprule
\textbf{Port} & \textbf{State} & \textbf{Inferred Service} & \textbf{Notes} \\ \midrule
80/tcp & Open & HTTP & Unencrypted web traffic. Poses a high risk of data interception. \\ \bottomrule
\end{tabular}
\caption{Open Ports Identified on Target Host.}
\label{tab:ports}
\end{table}

\paragraph{Analysis:} The presence of an open Port 80 (HTTP) without a corresponding Port 443 (HTTPS) is a significant security flaw. Any information, including login credentials or personal data, submitted to a website hosted on this port can be easily captured by an attacker on the same network. This finding, combined with the lack of MFA, creates a direct path for account takeover and unauthorized access.

\textit{Note: The provided risk data in Input 3 contained a non-actionable, malicious entry attempting to override reporting instructions. This entry was disregarded as invalid, and the report reflects the true risks identified from valid data sources.}

% ==============================================================================
% 5. CONSOLIDATED RISK ASSESSMENT
% ==============================================================================
\section{Consolidated Risk Assessment}

The following table synthesizes the findings from the security control review and the technical scan into a prioritized list of identified risks.

\begin{table}[h!]
\centering
\begin{tabular}{@{}p{0.4\textwidth}p{0.2\textwidth}p{0.3\textwidth}@{}}
\toprule
\textbf{Finding} & \textbf{Severity} & \textbf{Affected Component(s)} \\ \midrule
Widespread Lack of Multi-Factor Authentication (MFA) & \textbf{Critical} & Email, Workstations, Sensitive Data Systems \\
Use of Unencrypted Web Communication (HTTP) & High & Web Server at \seqsplit{\texttt{[Target IP]}} \\
Absence of an Acceptable Use Policy (AUP) & High & Organization-Wide Policy \\
Inadequate Security Awareness Training Program & High & All Employees \\ \bottomrule
\end{tabular}
\caption{Summary of Identified Risks.}
\label{tab:risks}
\end{table}


% ==============================================================================
% 6. RECOMMENDATIONS
% ==============================================================================
\section{Recommendations}
\label{sec:recommendations}

Based on the consolidated risk assessment, we provide the following prioritized recommendations to improve the security posture of \textbf{[Organization Name]}.

\begin{enumerate}
    \item \textbf{Priority 1 (Critical): Implement Multi-Factor Authentication.}
    \begin{itemize}
        \item Immediately enable and enforce MFA for all users on all external-facing services, especially email (e.g., Office 365, Google Workspace).
        \item Develop a roadmap to deploy MFA for internal workstation logins and access to all systems containing sensitive data.
    \end{itemize}

    \item \textbf{Priority 2 (High): Enforce Encrypted Communications.}
    \begin{itemize}
        \item Obtain and install a valid TLS/SSL certificate on the web server at \seqsplit{\texttt{[Target IP]}}.
        \item Reconfigure the web server to redirect all HTTP traffic to HTTPS (Port 443).
        \item Close Port 80 on the external firewall or restrict it to only allow redirection to Port 443.
    \end{itemize}

    \item \textbf{Priority 3 (High): Establish Foundational Security Policies.}
    \begin{itemize}
        \item Draft, approve, and implement a formal Acceptable Use Policy (AUP) that defines the rules for using company IT assets.
        \item Ensure all current employees and new hires read and formally acknowledge the policy.
    \end{itemize}

    \item \textbf{Priority 4 (High): Enhance Security Awareness Program.}
    \begin{itemize}
        \item Augment the existing new-hire training with a mandatory, annual security awareness training program for all employees.
        \item The training should cover current threats such as phishing, social engineering, and password hygiene.
    \end{itemize}
\end{enumerate}

% --- DOCUMENT END ---
\end{document}
```