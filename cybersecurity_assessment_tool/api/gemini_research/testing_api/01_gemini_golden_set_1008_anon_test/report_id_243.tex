```latex
\documentclass[12pt, a4paper]{article}

% Preamble: Required Packages
\usepackage[margin=1in]{geometry}
\usepackage{pifont} % For checkmarks and crosses
\usepackage{booktabs} % For professional tables
\usepackage[hidelinks]{hyperref} % For clickable links
\usepackage{url} % For URL formatting
\usepackage{seqsplit} % To split long text sequences like versions
\usepackage{graphicx}
\usepackage{xcolor}

% Define colors for severity
\definecolor{criticalred}{HTML}{D7263D}
\definecolor{highorange}{HTML}{F49D42}
\definecolor{mediumyellow}{HTML}{F4D442}
\definecolor{lowblue}{HTML}{4287f5}
\definecolor{infogray}{HTML}{808080}

% Document Information
\title{Cybersecurity Posture Assessment Report}
\author{Cybersecurity Analyst}
\date{\today}

\begin{document}

\maketitle
\thispagestyle{empty}
\newpage

\tableofcontents
\newpage

% --- 1. Executive Summary ---
\section{Executive Summary}

This report provides a comprehensive cybersecurity assessment for \textbf{[Organization Name]}, based on an analysis of network scan data, organizational security controls, and pre-existing risk information. The assessment was conducted on \today.

The overall security posture is determined to be at a \textbf{High Risk} level. This conclusion is based on several critical findings that require immediate attention:

\begin{itemize}
    \item \textbf{Critical Database Exposure:} An externally facing MySQL database (version 5.7.33) was identified on port 3306. This version is past its official End-of-Life (EOL) as of October 2023 and no longer receives security patches, exposing it to numerous known vulnerabilities.
    \item \textbf{Critical Authentication Gap:} Multi-Factor Authentication (MFA) is not enforced for email access. Email is a primary target for phishing and account takeover attacks, and the lack of MFA significantly increases the risk of a breach.
    \item \textbf{High-Risk Policy Gap:} The organization does not conduct annual security awareness training for all employees. This lapse allows security knowledge to decay, making staff more susceptible to social engineering and other human-targeted attacks.
\end{itemize}

Immediate remediation of the exposed database and the implementation of MFA for email are paramount to reducing the organization's attack surface. Detailed findings and actionable recommendations are provided in the subsequent sections of this report.

% --- 2. Organizational Information ---
\section{Organizational Information}

The following information was used as the basis for this assessment. Where data was not provided, placeholders have been used.

\begin{table}[h!]
\centering
\begin{tabular}{@{}ll@{}}
\toprule
\textbf{Attribute} & \textbf{Value} \\ \midrule
Organization Name & \textbf{[Organization Name]} \\
Primary Domain & \texttt{[Domain]} \\
External IP Address Scanned & \texttt{[Client IP]} \\
Target IP from Scan & \texttt{[Target IP]} \\
\bottomrule
\end{tabular}
\caption{Client Organizational Details}
\label{tab:org_info}
\end{table}

% --- 3. Security Control Review ---
\section{Security Control Review}

A review of the organization's security controls was conducted via a questionnaire. The results highlight key areas of strength and weakness in the current security policies and procedures. "No" answers indicate significant gaps that elevate risk.

\begin{table}[h!]
\centering
\begin{tabular}{@{}p{0.8\linewidth}c@{}}
\toprule
\textbf{Control Question} & \textbf{Status} \\ \midrule
Do you require MFA to access email? & \textcolor{criticalred}{\ding{55}} \\
Do you require MFA to log into computers? & \textcolor{green}{\ding{51}} \\
Do you require MFA to access sensitive data systems? & \textcolor{green}{\ding{51}} \\
Does your organization have an employee acceptable use policy? & \textcolor{green}{\ding{51}} \\
Does your organization do security awareness training for new employees? & \textcolor{green}{\ding{51}} \\
Does your organization do security awareness training for all employees at least once per year? & \textcolor{highorange}{\ding{55}} \\
\bottomrule
\end{tabular}
\caption{Security Control Questionnaire Results}
\label{tab:controls}
\end{table}

% --- 4. Technical Scan Results ---
\section{Technical Scan Results}

An Nmap scan was performed on the target system to identify open ports and running services. The results are detailed below.

\subsection{Host: \texttt{[Target IP]}}
The scan identified one host as `up` and discovered the following open port:

\begin{table}[h!]
\centering
\begin{tabular}{@{}lllll@{}}
\toprule
\textbf{Port} & \textbf{State} & \textbf{Service} & \textbf{Product} & \textbf{Version} \\ \midrule
3306/tcp & open & mysql & MySQL & 5.7.33 \\
\bottomrule
\end{tabular}
\caption{Open Ports and Services Detected}
\label{tab:scan_results}
\end{table}

\subsection{Analysis of Technical Findings}
The primary finding is the exposure of a MySQL database service on port 3306. Direct public access to a database is highly discouraged as it provides a direct vector for attackers to attempt brute-force logins, exploit vulnerabilities, or perform denial-of-service attacks.

Furthermore, the detected version, \textbf{MySQL 5.7.33}, is a significant concern. The MySQL 5.7 series reached its \textbf{End-of-Life (EOL) in October 2023}. This means it no longer receives security updates from the vendor, and any vulnerabilities discovered after this date will remain unpatched. This elevates the risk of compromise substantially.

% --- 5. Consolidated Risk Assessment ---
\section{Consolidated Risk Assessment}

This section synthesizes findings from the security control review, technical scan, and pre-existing risk data into a consolidated list of identified risks.

\begin{table}[h!]
\centering
\begin{tabular}{@{}p{0.25\linewidth}p{0.15\linewidth}p{0.5\linewidth}@{}}
\toprule
\textbf{Risk Name} & \textbf{Severity} & \textbf{Overview} \\ \midrule
\textbf{End-of-Life Database Software} & \textcolor{criticalred}{\textbf{Critical (9.8)}} & The MySQL 5.7.33 service is past its EOL date and is no longer receiving security patches, exposing it to numerous unpatched vulnerabilities. \\
\addlinespace
\textbf{Lack of Email MFA} & \textcolor{criticalred}{\textbf{Critical (9.1)}} & MFA is not enforced for email access. This exposes the organization to a high risk of business email compromise (BEC), phishing, and account takeover. \\
\addlinespace
\textbf{Database Exposure} & \textcolor{highorange}{\textbf{High (7.5)}} & The MySQL database port (3306) is open to the network, allowing unauthorized users to directly connect and attempt to compromise the database. \\
\addlinespace
\textbf{Inadequate Security Training} & \textcolor{highorange}{\textbf{High (7.2)}} & Failure to conduct annual security awareness training for all employees increases the organizational risk of falling victim to social engineering attacks. \\
\bottomrule
\end{tabular}
\caption{Summary of Identified Risks}
\label{tab:risk_summary}
\end{table}

% --- 6. Recommendations ---
\section{Recommendations}

The following prioritized recommendations are provided to mitigate the identified risks and improve the overall security posture of \textbf{[Organization Name]}.

\subsection{Priority 1: Critical Risks (Immediate Action Required)}
\begin{enumerate}
    \item \textbf{Restrict Database Access:}
        \begin{itemize}
            \item \textit{Immediate Fix:} Implement strict firewall rules to deny all public access to TCP port 3306 on \texttt{[Target IP]}. Access should only be permitted from trusted internal IP addresses.
            \item \textit{Long-Term Fix:} Transition all database access to occur over a secure, authenticated channel such as a Virtual Private Network (VPN).
        \end{itemize}

    \item \textbf{Upgrade End-of-Life Software:}
        \begin{itemize}
            \item \textit{Action:} Develop and execute a plan to migrate the MySQL 5.7.33 database to a currently supported version (e.g., MySQL 8.0 or later). This is crucial for receiving ongoing security patches.
        \end{itemize}

    \item \textbf{Implement MFA for Email:}
        \begin{itemize}
            \item \textit{Action:} Immediately enable and enforce Multi-Factor Authentication for all user accounts across the email system. This is the single most effective control to prevent unauthorized account access.
        \end{itemize}
\end{enumerate}

\subsection{Priority 2: High Risks (Action within 3-6 Months)}
\begin{enumerate}
    \setcounter{enumi}{3} % Continue numbering from previous list
    \item \textbf{Establish Annual Security Training:}
        \begin{itemize}
            \item \textit{Action:} Procure and implement a security awareness training program for all employees. This training should be conducted upon hiring and repeated annually to ensure that security remains a top-of-mind concern for all staff.
        \end{itemize}
\end{enumerate}

\end{document}
```