```latex
\documentclass[12pt]{article}

% Required Packages
\usepackage[margin=1in]{geometry}
\usepackage{pifont} % For checkmarks and crosses
\usepackage{booktabs} % For professional tables
\usepackage{hyperref} % For hyperlinks and metadata
\usepackage{url} % For formatting URLs
\usepackage{seqsplit} % For splitting long strings in tt font
\usepackage{graphicx} % For logo (optional, placeholder)
\usepackage{xcolor} % For colors

% --- Document Metadata ---
\hypersetup{
    colorlinks=true,
    linkcolor=blue,
    filecolor=magenta,      
    urlcolor=cyan,
    pdftitle={Cybersecurity Posture Assessment Report},
    pdfauthor={Cybersecurity Analyst},
    pdfsubject={Security Analysis},
    pdfkeywords={Security, Report, Analysis},
}

% --- Custom Commands ---
\newcommand{\yes}{\ding{51}}
\newcommand{\no}{\ding{55}}

% --- Title Page ---
\title{Cybersecurity Posture Assessment Report}
\author{Cybersecurity Analyst}
\date{\today}

\begin{document}

\maketitle
\thispagestyle{empty}
\newpage

\tableofcontents
\newpage

% --- Section 1: Executive Summary ---
\section{Executive Summary}

This report provides a comprehensive cybersecurity posture assessment for \textbf{[Organization Name]}. The analysis is based on a synthesis of an external network scan, a review of organizational security controls via a questionnaire, and an evaluation of pre-existing documented risks.

The assessment reveals a mixed security posture. On a positive note, the external network scan of the target host \texttt{[Target IP]} did not identify any open ports, indicating a strong network perimeter for that specific asset. This significantly reduces the external attack surface.

However, the organizational security control review identified several critical gaps that present a high level of risk. The most significant findings include the lack of Multi-Factor Authentication (MFA) for accessing email and sensitive data systems. Furthermore, the absence of a formal employee Acceptable Use Policy (AUP) creates ambiguity and increases the risk of insider threats and misuse of company assets.

This report details these findings and provides prioritized, actionable recommendations to mitigate the identified risks and strengthen the overall security posture of the organization.

% --- Section 2: Organizational Information ---
\section{Organizational Information}

This section contains the high-level information provided by the client organization. The data is used to establish the context for the security assessment.

\begin{table}[h!]
\centering
\begin{tabular}{@{}ll@{}}
\toprule
\textbf{Attribute} & \textbf{Value} \\ \midrule
Organization Name & \textbf{[Organization Name]} \\
Primary Email Domain & \texttt{[Domain]} \\
External IP Scanned & \texttt{[Client IP]} \\
Assessment Date & \today \\ \bottomrule
\end{tabular}
\caption{Client Organizational Details.}
\label{tab:org_info}
\end{table}

% --- Section 3: Security Control Review ---
\section{Security Control Review (Questionnaire)}

The following table summarizes the organization's responses to a security controls questionnaire. This review is critical for understanding the administrative and policy-based defenses currently in place. Gaps identified here often represent significant organizational risks.

\begin{table}[h!]
\centering
\begin{tabular}{@{}p{0.6\linewidth} c p{0.2\linewidth}@{}}
\toprule
\textbf{Control Question} & \textbf{Response} & \textbf{Assessment} \\ \midrule
Do you require MFA to access email? & \no & \textcolor{red}{\textbf{Critical Gap}} \\
Do you require MFA to log into computers? & \yes & Best Practice Met \\
Do you require MFA to access sensitive data systems? & \no & \textcolor{red}{\textbf{Critical Gap}} \\
Does your organization have an employee acceptable use policy? & \no & \textcolor{orange}{High Risk} \\
Does your organization do security awareness training for new employees? & \yes & Best Practice Met \\
Does your organization do security awareness training for all employees at least once per year? & \yes & Best Practice Met \\ \bottomrule
\end{tabular}
\caption{Security Controls Questionnaire Analysis.}
\label{tab:controls}
\end{table}

% --- Section 4: Technical Scan Results ---
\section{Technical Scan Results}

An external network scan was performed using Nmap to identify open ports and services on the public-facing infrastructure.

\begin{itemize}
    \item \textbf{Target IP Address:} \texttt{[Target IP]}
    \item \textbf{Scan Summary:} The scan completed successfully and determined the host to be online.
    \item \textbf{Findings:}
        \begin{itemize}
            \item \textbf{Open Ports:} \textbf{None.} No open TCP or UDP ports were discovered during the scan.
            \item \textbf{Port State:} All scanned ports were found to be in a \texttt{closed} state. This is a strong security configuration, as it indicates that the firewall is correctly blocking unsolicited incoming traffic while still responding to probes. This makes it more difficult for an attacker to perform reconnaissance.
        \end{itemize}
\end{itemize}

\textbf{Conclusion:} The technical scan of the specified target indicates a hardened external perimeter with no exposed services, which is an excellent security practice.

% --- Section 5: Consolidated Risk Assessment ---
\section{Consolidated Risk Assessment}

This section synthesizes findings from the security control review, technical scans, and pre-existing risk documentation into a consolidated list of identified risks. The severity level is assigned based on the potential impact and likelihood of exploitation.

\begin{table}[h!]
\centering
\begin{tabular}{@{}p{0.25\linewidth} p{0.5\linewidth} l@{}}
\toprule
\textbf{Risk Name} & \textbf{Overview} & \textbf{Severity} \\ \midrule
\textbf{No MFA on Email} & The absence of MFA on email accounts makes them highly susceptible to compromise via phishing or password spraying. A compromised email account is a primary vector for Business Email Compromise (BEC), data exfiltration, and further internal network attacks. & \textcolor{red}{\textbf{Critical}} \\
\addlinespace
\textbf{No MFA on Sensitive Data Systems} & Lack of MFA on systems storing sensitive or critical data exposes the organization to a high risk of a data breach. An attacker with stolen credentials can gain direct access to confidential information, leading to severe financial, reputational, and regulatory damage. & \textcolor{red}{\textbf{Critical}} \\
\addlinespace
\textbf{Absence of Acceptable Use Policy (AUP)} & Without a formal AUP, employees may be unaware of their responsibilities regarding the secure use of company assets. This increases the risk of unintentional data leakage, misuse of resources, and insider threats. It also complicates disciplinary action in case of a violation. & \textcolor{orange}{\textbf{High}} \\
\addlinespace
\textit{No Pre-existing Risks} & \textit{The provided data contained no pre-existing vulnerabilities.} & \textit{N/A} \\
\addlinespace
\textit{No Technical Vulnerabilities} & \textit{The external network scan did not identify any technical vulnerabilities on the target host.} & \textit{N/A} \\
\bottomrule
\end{tabular}
\caption{Summary of Identified Risks.}
\label{tab:risks}
\end{table}

% --- Section 6: Recommendations ---
\section{Recommendations}

Based on the consolidated risk assessment, the following actions are recommended to mitigate the identified vulnerabilities and improve the organization's overall security posture. Recommendations are prioritized based on risk severity.

\begin{enumerate}
    \item \textbf{[Critical] Implement MFA for Email Access:}
    \begin{itemize}
        \item \textbf{Action:} Enforce MFA for all users accessing email, whether through a web client, desktop application, or mobile device. This is the single most effective control to prevent account takeovers and Business Email Compromise.
        \item \textbf{Tools:} Leverage built-in MFA capabilities in Microsoft 365 or Google Workspace, or use third-party solutions like Duo or Okta.
    \end{itemize}
    
    \item \textbf{[Critical] Enforce MFA on Sensitive Systems:}
    \begin{itemize}
        \item \textbf{Action:} Conduct an inventory of all systems that process, store, or transmit sensitive data (e.g., financial systems, HR databases, customer CRMs). Mandate the use of MFA for all access, especially privileged access, to these systems.
        \item \textbf{Justification:} This creates a critical defense-in-depth layer, ensuring that a single credential compromise does not lead to a catastrophic data breach.
    \end{itemize}
    
    \item \textbf{[High] Develop and Implement an Acceptable Use Policy (AUP):}
    \begin{itemize}
        \item \textbf{Action:} Draft a formal AUP that clearly defines the rules and expectations for employees when using company technology and data. The policy should cover topics such as data handling, internet usage, password security, and reporting security incidents.
        \item \textbf{Implementation:} Ensure all employees read and formally acknowledge the policy. Integrate the AUP into the new employee onboarding process and conduct annual refreshers.
    \end{itemize}
\end{enumerate}

\end{document}
```