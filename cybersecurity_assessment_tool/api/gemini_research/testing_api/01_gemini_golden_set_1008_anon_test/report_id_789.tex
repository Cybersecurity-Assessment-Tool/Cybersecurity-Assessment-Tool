```latex
\documentclass[12pt]{article}

% --- PACKAGES ---
\usepackage[margin=1in]{geometry} % Set page margins
\usepackage{pifont}                % For checkmark and cross symbols
\usepackage{booktabs}              % For professional-looking tables
\usepackage{hyperref}              % For hyperlinks
\usepackage{url}                   % For URL formatting
\usepackage{seqsplit}              % For splitting long strings in texttt
\usepackage[utf8]{inputenc}        % Input encoding

% --- DOCUMENT METADATA ---
\title{Cybersecurity Posture Assessment Report}
\author{Cybersecurity Analysis Division}
\date{\today}

% --- HYPERREF SETUP ---
\hypersetup{
    colorlinks=true,
    linkcolor=blue,
    filecolor=magenta,      
    urlcolor=cyan,
    pdftitle={Cybersecurity Posture Assessment Report},
    pdfpagemode=FullScreen,
}

% --- BEGIN DOCUMENT ---
\begin{document}

\maketitle
\hrule
\vspace{1cm}

\begin{center}
    \textbf{Prepared for: \textbf{[Organization Name]}}
\end{center}

\newpage
\tableofcontents
\newpage

% ===================================================================
% SECTION 1: EXECUTIVE SUMMARY
% ===================================================================
\section{Executive Summary}

This report details the findings of a cybersecurity posture assessment conducted for \textbf{[Organization Name]}. The assessment combined an external network scan, a review of existing risk documentation, and an analysis of organizational security controls based on a provided questionnaire.

The assessment identified several critical and high-severity risks that require immediate attention. The most significant finding is the discovery of a publicly accessible service on port 8080 of the external IP address \texttt{[Target IP]}. This service presents a title, \textbf{``TOP SECRET DB''}, which suggests a highly sensitive database is exposed. This finding directly contradicts the existing risk register, which incorrectly lists this port as a secure false positive.

Furthermore, critical gaps were identified in internal security controls. The lack of mandatory Multi-Factor Authentication (MFA) for computer logins significantly increases the risk of unauthorized access from a compromised employee account. This is compounded by the absence of a mandatory annual security awareness training program for all employees, which heightens the organization's susceptibility to phishing and social engineering attacks.

Immediate remediation is required to address the exposed service and enforce endpoint MFA. A comprehensive review of the risk management process is also strongly recommended to prevent such dangerous discrepancies in the future.

% ===================================================================
% SECTION 2: ORGANIZATIONAL INFORMATION
% ===================================================================
\section{Organizational Information}

This section provides the key identifying information for the organization under review. As per the provided data, the following placeholders are used.

\begin{itemize}
    \item \textbf{Organization Name:} \textbf{[Organization Name]}
    \item \textbf{Primary Domain:} \texttt{[Domain]}
    \item \textbf{External IP Scanned:} \texttt{[Client IP]}
\end{itemize}

% ===================================================================
% SECTION 3: SECURITY CONTROL REVIEW
% ===================================================================
\section{Security Control Review}

The following table summarizes the organization's security controls based on the provided questionnaire. Items marked with a cross (\ding{55}) represent significant gaps in the security posture and are discussed in the Risk Assessment section.

\begin{table}[h!]
\centering
\caption{Organizational Security Control Status}
\begin{tabular}{p{0.6\textwidth} c p{0.2\textwidth}}
\toprule
\textbf{Control Question} & \textbf{Status} & \textbf{Assessment} \\
\midrule
Do you require MFA to access email? & \ding{51} & Meets best practice. \\
\addlinespace
Do you require MFA to log into computers? & \textbf{\color{red}\ding{55}} & \textbf{Critical Gap.} \\
\addlinespace
Do you require MFA to access sensitive data systems? & \ding{51} & Meets best practice. \\
\addlinespace
Does your organization have an employee acceptable use policy? & \ding{51} & Good practice. \\
\addlinespace
Does your organization do security awareness training for new employees? & \ding{51} & Good practice. \\
\addlinespace
Does your organization do security awareness training for all employees at least once per year? & \textbf{\color{red}\ding{55}} & \textbf{High Risk.} \\
\bottomrule
\end{tabular}
\end{table}

% ===================================================================
% SECTION 4: TECHNICAL SCAN RESULTS
% ===================================================================
\section{Technical Scan Results}

An external network scan was performed on the target IP address. The target was identified as `up` and responsive. The following table details the open ports and services discovered.

\begin{itemize}
    \item \textbf{Target IP Address:} \texttt{[Target IP]}
    \item \textbf{Scan Tool:} Nmap
\end{itemize}

\begin{table}[h!]
\centering
\caption{Open Port Scan Findings}
\begin{tabular}{c c p{0.3\textwidth} p{0.4\textwidth}}
\toprule
\textbf{Port} & \textbf{State} & \textbf{Service/Product} & \textbf{Details \& Analysis} \\
\midrule
8080/tcp & OPEN & http & The HTTP service returned a title: \textbf{``TOP SECRET DB''}. This is a critical information disclosure and suggests a sensitive database is directly exposed. This finding contradicts the current risk register. \\
\bottomrule
\end{tabular}
\end{table}

% ===================================================================
% SECTION 5: RISK ASSESSMENT
% ===================================================================
\section{Risk Assessment}

The following risks have been identified and prioritized based on the correlation of technical findings, control gaps, and existing documentation.

\begin{table}[h!]
\centering
\caption{Synthesized Risk Summary}
\begin{tabular}{p{0.1\textwidth} p{0.4\textwidth} p{0.15\textwidth} p{0.2\textwidth}}
\toprule
\textbf{Risk ID} & \textbf{Description} & \textbf{Severity} & \textbf{Affected Systems} \\
\midrule
\textbf{RISK-001} & A potentially sensitive database or application titled ``TOP SECRET DB'' is exposed to the public internet on port 8080. & \textbf{Critical} & External Server: \texttt{[Target IP]} \\
\addlinespace
\textbf{RISK-002} & Lack of MFA for computer logins allows an attacker with a single compromised password to gain full access to an employee's endpoint and internal network resources. & \textbf{Critical} & All employee workstations and laptops. \\
\addlinespace
\textbf{RISK-003} & The risk register is outdated and inaccurate. It incorrectly states that port 8080 is secure, while active scanning proves it is open and exposing a sensitive service. This indicates a failure in the risk management process. & \textbf{High} & Security Governance \& Risk Management Program. \\
\addlinespace
\textbf{RISK-004} & Lack of mandatory annual security awareness training for all staff increases the likelihood of successful phishing and social engineering attacks, leading to credential compromise. & \textbf{High} & All employees. \\
\bottomrule
\end{tabular}
\end{table}

% ===================================================================
% SECTION 6: RECOMMENDATIONS
% ===================================================================
\section{Recommendations}

The following actions are recommended to mitigate the identified risks. Recommendations are prioritized based on severity.

\subsection*{Immediate Priority (Critical Risks)}

\begin{enumerate}
    \item \textbf{Remediate Exposed Service (RISK-001):}
    \begin{itemize}
        \item Immediately investigate the service running on port 8080 of \texttt{[Target IP]}.
        \item If the service is not intended for public access, place it behind a firewall and restrict access to authorized IP addresses only.
        \item If the service must be public, ensure it is fully patched, hardened, and requires strong authentication.
        \item Change the service title to remove sensitive information.
    \end{itemize}

    \item \textbf{Implement Endpoint MFA (RISK-002):}
    \begin{itemize}
        \item Procure and deploy an MFA solution for all employee computer and laptop logins (Windows, macOS, etc.).
        \item Enforce this policy for all users, including privileged administrators and executives.
    \end{itemize}
\end{enumerate}

\subsection*{High Priority}

\begin{enumerate}
    \setcounter{enumi}{2} % Continue numbering
    \item \textbf{Update Risk Management Processes (RISK-003):}
    \begin{itemize}
        \item Conduct a full review of the current risk register to identify and correct other inaccuracies.
        \item Implement a formal process for validating risk assessments with periodic technical scans.
    \end{itemize}
    
    \item \textbf{Establish Annual Security Training (RISK-004):}
    \begin{itemize}
        \item Implement a mandatory security awareness training program that all employees must complete at least once per year.
        \item Include modules on phishing, password security, and acceptable use.
    \end{itemize}
\end{enumerate}

\end{document}
```