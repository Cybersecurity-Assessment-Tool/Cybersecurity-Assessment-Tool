```latex
\documentclass[12pt]{article}

% Preamble: Required Packages
\usepackage[margin=1in]{geometry} % Set page margins
\usepackage{pifont}               % For checkmarks and crosses (\ding)
\usepackage{booktabs}             % For professional tables
\usepackage{graphicx}             % For logos (placeholder)
\usepackage{xcolor}               % For colors
\usepackage{hyperref}             % For hyperlinks
\usepackage{url}                  % For URL formatting
\usepackage{seqsplit}             % To split long strings in texttt

% --- Document Setup ---
\hypersetup{
    colorlinks=true,
    linkcolor=black,
    urlcolor=blue,
    pdftitle={Cybersecurity Assessment Report},
    pdfauthor={Cybersecurity Analyst},
    hidelinks % Hides the boxes around links for a cleaner look
}

% --- Custom Commands ---
\newcommand{\yes}{\ding{51}} % Green checkmark
\newcommand{\no}{\ding{55}}  % Red cross

\begin{document}

% --- Title Page ---
\begin{titlepage}
    \centering
    \vspace*{1cm}
    
    \Huge
    \textbf{Cybersecurity Assessment Report}
    
    \vspace{1.5cm}
    
    \Large
    Prepared for: \\
    \vspace{0.5cm}
    \textbf{[Organization Name]}
    
    \vspace{2cm}
    
    \large
    Report Date: \today \\
    Report ID: SEC-2023-042
    
    \vfill
    
    \large
    \textbf{CONFIDENTIAL} \\
    \textit{This document contains sensitive information. Distribution is restricted.}
    
\end{titlepage}

\tableofcontents
\newpage

% --- Section 1: Executive Overview ---
\section{Executive Overview}
This report provides a cybersecurity assessment for \textbf{[Organization Name]}, based on an analysis of organizational security controls, an external network scan, and a review of known risks. The assessment was conducted on \today.

The organization has implemented several positive security controls, including Multi-Factor Authentication (MFA) for email and sensitive data access. However, this assessment has identified two critical gaps and one significant technical exposure that require immediate attention:
\begin{itemize}
    \item \textbf{Critical - Lack of MFA on Workstations:} The absence of MFA for computer logins presents a significant risk, as a single compromised password could lead to full device and potential network access.
    \item \textbf{High - Inadequate Onboarding Security Training:} New employees are not receiving security awareness training, leaving them vulnerable to social engineering attacks from their first day.
    \item \textbf{Medium - Exposed SSH Service:} The external network scan identified an open SSH port (22/TCP). If not properly configured and monitored, this service is a primary target for brute-force attacks and unauthorized access attempts.
\end{itemize}

This report details these findings and provides actionable recommendations to mitigate the identified risks and improve the overall security posture of \textbf{[Organization Name]}.

% --- Section 2: Organizational Information ---
\section{Organizational Information}
This section contains the high-level information used for this assessment. As per the template mode for this report, placeholder values are used where specific data was not provided.

\begin{tabular}{@{}ll}
    \toprule
    \textbf{Attribute} & \textbf{Value} \\
    \midrule
    Organization Name & \textbf{[Organization Name]} \\
    Primary Domain & \texttt{[Domain]} \\
    External IP Address Scanned & \texttt{[Client IP]} \\
    \bottomrule
\end{tabular}

% --- Section 3: Security Control Review ---
\section{Security Control Review}
The following table summarizes the organization's responses to a security controls questionnaire. "No" answers indicate significant gaps in the current security posture.

\begin{tabular}{@{}p{0.6\linewidth}cc}
    \toprule
    \textbf{Control Question} & \textbf{Status} & \textbf{Analysis} \\
    \midrule
    Do you require MFA to access email? & \yes & Compliant. \\
    \addlinespace
    Do you require MFA to log into computers? & \no & \textbf{Critical Gap}. Lack of endpoint MFA is a major risk. \\
    \addlinespace
    Do you require MFA to access sensitive data systems? & \yes & Compliant. \\
    \addlinespace
    Does your organization have an employee acceptable use policy? & \yes & Compliant. \\
    \addlinespace
    Does your organization do security awareness training for new employees? & \no & \textbf{High Risk}. New hires are a primary target for attackers. \\
    \addlinespace
    Does your organization do security awareness training for all employees at least once per year? & \yes & Compliant. \\
    \bottomrule
\end{tabular}

\subsection*{Analysis of Gaps}
\begin{itemize}
    \item \textbf{MFA for Computers:} A compromised user password (e.g., from a phishing attack or credential stuffing) could grant an attacker direct access to an employee's computer. From there, an attacker can access local data, escalate privileges, and move laterally across the network. This is considered a foundational security control.
    \item \textbf{New Employee Security Training:} The onboarding process is a critical time to establish security-conscious habits. Without initial training, new employees may be unaware of company policies regarding phishing, data handling, and acceptable use, making them significantly more susceptible to social engineering attacks.
\end{itemize}

% --- Section 4: Technical Scan Results ---
\section{Technical Scan Results}
An external network scan was performed to identify open ports and services visible to the public internet.

\begin{itemize}
    \item \textbf{Target IP Address:} \texttt{[Target IP]}
    \item \textbf{Scan Date:} Scan data processed on \today
    \item \textbf{Host Status:} UP
\end{itemize}

\subsection*{Open Ports Discovered}
The following table details the ports found to be open on the target system.

\begin{tabular}{@{}lllll}
    \toprule
    \textbf{Port} & \textbf{Protocol} & \textbf{State} & \textbf{Service} & \textbf{Notes} \\
    \midrule
    22 & TCP & open & ssh & Secure Shell (SSH) access. \\
    \bottomrule
\end{tabular}

\subsection*{Technical Analysis}
The scan identified that port 22 (SSH) is open to the internet. SSH is a powerful administrative protocol that, if not securely configured, presents a significant risk.
\begin{itemize}
    \item \textbf{Attack Vector:} Exposed SSH services are constantly targeted by automated scanners and brute-force attacks attempting to guess usernames and passwords.
    \item \textbf{Configuration Concerns:} Without further details, it is unknown if the service is patched, allows password-based authentication (a weak practice), or is restricted to specific source IP addresses.
\end{itemize}

% --- Section 5: Consolidated Risk Assessment ---
\section{Consolidated Risk Assessment}
This section synthesizes findings from the security control review and the technical scan. The pre-existing risk list was empty, so all risks listed below are new findings from this assessment.

\begin{tabular}{@{}p{0.4\linewidth}p{0.4\linewidth}l}
    \toprule
    \textbf{Risk Name} & \textbf{Overview} & \textbf{Severity} \\
    \midrule
    \textbf{Lack of Endpoint MFA} & A compromised password allows direct login to workstations, enabling data theft and lateral movement. & \textbf{Critical} \\
    \addlinespace
    \textbf{No Security Training for New Hires} & New employees are not equipped to identify or report phishing and social engineering attempts, increasing organizational vulnerability. & \textbf{High} \\
    \addlinespace
    \textbf{Exposed SSH Service} & The publicly accessible SSH port is a target for brute-force attacks and could lead to unauthorized server access if misconfigured. & \textbf{Medium} \\
    \bottomrule
\end{tabular}

% --- Section 6: Recommendations ---
\section{Recommendations}
The following actions are recommended to mitigate the identified risks. Recommendations are prioritized by severity.

\subsection*{Critical Priority}
\begin{enumerate}
    \item \textbf{Implement MFA for All Workstation Logins:}
        \begin{itemize}
            \item \textbf{Action:} Deploy a mandatory MFA solution for all employee computer logins (e.g., Windows Hello for Business, Duo, YubiKey).
            \item \textbf{Justification:} This adds a critical layer of defense, ensuring that a stolen password alone is not enough to compromise an endpoint. Prioritize deployment for administrators and executives.
        \end{itemize}
\end{enumerate}

\subsection*{High Priority}
\begin{enumerate}
    \setcounter{enumi}{1} % Continue numbering
    \item \textbf{Integrate Security Training into Employee Onboarding:}
        \begin{itemize}
            \item \textbf{Action:} Develop or procure a security awareness training module and make it a mandatory part of the new-hire onboarding process, to be completed within the first week of employment.
            \item \textbf{Justification:} This establishes a baseline of security knowledge from day one, reducing the likelihood of early, preventable security incidents.
        \end{itemize}
\end{enumerate}

\subsection*{Medium Priority}
\begin{enumerate}
    \setcounter{enumi}{2} % Continue numbering
    \item \textbf{Secure the Exposed SSH Service:}
        \begin{itemize}
            \item \textbf{Action:} Review the business need for external SSH access to \texttt{[Target IP]}.
            \begin{itemize}
                \item If not needed, block port 22 at the firewall.
                \item If required, implement the following controls:
                    \begin{enumerate}
                        \item Disable password-based authentication and enforce public key authentication only.
                        \item Restrict access via firewall rules to only trusted source IP addresses.
                        \item Ensure the SSH server software is fully patched and updated.
                        \item Implement an intrusion detection tool like Fail2Ban to block malicious IPs.
                    \end{enumerate}
            \end{itemize}
            \item \textbf{Justification:} These steps significantly reduce the attack surface of the SSH service, protecting it from automated attacks and unauthorized access attempts.
        \end{itemize}
\end{enumerate}

\end{document}
```