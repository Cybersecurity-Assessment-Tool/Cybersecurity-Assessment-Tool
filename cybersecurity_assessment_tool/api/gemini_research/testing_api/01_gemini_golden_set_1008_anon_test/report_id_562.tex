```latex
\documentclass[12pt, a4paper]{article}

% Preamble: Required Packages and Document Setup
\usepackage[margin=1in]{geometry}
\usepackage{pifont} % For checkmarks and crosses (\ding)
\usepackage{booktabs} % For professional tables
\usepackage{graphicx}
\usepackage[table]{xcolor}
\usepackage{hyperref}
\usepackage{url}
\usepackage{seqsplit} % For splitting long strings in \texttt
\usepackage{longtable} % For tables that can span multiple pages

% --- Document Metadata and Hyperlink Setup ---
\hypersetup{
    colorlinks=true,
    linkcolor=blue,
    filecolor=magenta,      
    urlcolor=cyan,
    pdftitle={Cybersecurity Posture Assessment Report},
    pdfpagemode=FullScreen,
}

% --- Custom Commands & Colors ---
\definecolor{darkred}{rgb}{0.55, 0.0, 0.0}
\definecolor{darkgreen}{rgb}{0.0, 0.39, 0.0}
\newcommand{\yes}{\textcolor{darkgreen}{\ding{51}}} % Green checkmark
\newcommand{\no}{\textcolor{darkred}{\ding{55}}}   % Red X

% --- Document Start ---
\begin{document}

% --- Title Page ---
\begin{titlepage}
    \centering
    \vspace*{1cm}
    
    \includegraphics[width=0.4\textwidth]{example-image-a} % Placeholder for a logo
    
    \vspace{1.5cm}
    
    \Huge
    \textbf{Cybersecurity Posture Assessment Report}
    
    \vspace{1.5cm}
    
    \Large
    Prepared for: \textbf{[Organization Name]}
    
    \vspace{2cm}
    
    \large
    \begin{tabular}{ll}
        \textbf{Date of Report:} & \today \\
        \textbf{Author:} & Cybersecurity Analyst \\
    \end{tabular}
    
    \vfill
    
    \normalsize
    \textit{This report contains sensitive information and should be handled with care. Distribution is restricted to authorized personnel only.}
    
\end{titlepage}

\tableofcontents
\newpage

% --- Section 1: Executive Summary ---
\section{Executive Summary}

This report provides a comprehensive assessment of the cybersecurity posture for \textbf{[Organization Name]}. The analysis is based on a combination of a self-reported security controls questionnaire, a technical network scan of the external perimeter, and a review of pre-existing risks.

The assessment identified two significant areas of concern requiring immediate attention. The most critical finding is the absence of Multi-Factor Authentication (MFA) for accessing sensitive data systems. This gap exposes the organization's most valuable assets to significant risk from credential theft and unauthorized access.

Additionally, the lack of a formal Employee Acceptable Use Policy (AUP) represents a high-risk governance gap. Without a defined AUP, the organization cannot enforce secure employee behavior or establish a baseline for acceptable use of company resources, increasing the likelihood of insider threats and unintentional data breaches.

The technical network scan performed against the designated target IP address (\texttt{[Target IP]}) did not yield any open ports. While this can indicate a strong firewall configuration, it also limits our ability to assess vulnerabilities on externally-facing services.

This report details these findings and provides actionable recommendations to mitigate the identified risks and strengthen the overall security posture of \textbf{[Organization Name]}.

\newpage

% --- Section 2: Organizational Information ---
\section{Organizational Information}

The following information was used as the basis for this assessment. As per our analysis protocol, certain identifying details have been templated due to their absence in the provided data.

\begin{table}[h!]
\centering
\caption{Client and Scope Details}
\begin{tabular}{@{}ll@{}}
\toprule
\textbf{Attribute} & \textbf{Value} \\ \midrule
Organization Name & \textbf{[Organization Name]} \\
Primary Domain & \texttt{[Domain]} \\
External IP Scanned & \texttt{[Client IP]} \\ \bottomrule
\end{tabular}
\end{table}

% --- Section 3: Security Control Review ---
\section{Security Control Review (Questionnaire)}

The following table summarizes the organization's responses to the security controls questionnaire. A green checkmark (\yes) indicates a positive control is in place, while a red 'X' (\no) highlights a control gap that introduces risk.

\begin{longtable}{p{0.75\linewidth} c}
\caption{Security Controls Questionnaire Results} \\
\toprule
\textbf{Control Question} & \textbf{Response} \\ \midrule
\endfirsthead
\multicolumn{2}{c}%
{{\bfseries \tablename\ \thetable{} -- continued from previous page}} \\
\toprule
\textbf{Control Question} & \textbf{Response} \\ \midrule
\endhead
\bottomrule
\endfoot
Do you require MFA to access email? & \yes \\ \addlinespace
Do you require MFA to log into computers? & \yes \\ \addlinespace
\rowcolor{red!15} Do you require MFA to access sensitive data systems? & \no \\ \addlinespace
\rowcolor{red!15} Does your organization have an employee acceptable use policy? & \no \\ \addlinespace
Does your organization do security awareness training for new employees? & \yes \\ \addlinespace
Does your organization do security awareness training for all employees at least once per year? & \yes \\ \bottomrule
\end{longtable}

\textbf{Analysis:} The questionnaire reveals a strong implementation of MFA for email and computer access, which is commendable. However, the two identified gaps are significant and are detailed further in the Risk Assessment section of this report.

\newpage

% --- Section 4: Technical Scan Results ---
\section{Technical Scan Results}

A network port scan was conducted to identify externally accessible services and potential vulnerabilities.

\begin{table}[h!]
\centering
\caption{Network Scan Details}
\begin{tabular}{@{}ll@{}}
\toprule
\textbf{Parameter} & \textbf{Value} \\ \midrule
Target IP Address & \texttt{[Target IP]} \\
Scan Date & 2023-10-27 \\ % Placeholder date as none was provided
\bottomrule
\end{tabular}
\end{table}

\subsection{Scan Findings}
The network scan did not identify any open TCP or UDP ports on the target host. 

\textbf{Interpretation:}
\begin{itemize}
    \item \textbf{Stealth Configuration:} This result may indicate a robust firewall configuration that is correctly blocking all unsolicited inbound traffic, which is a positive security practice.
    \item \textbf{Host Unresponsive:} Alternatively, the target host may have been offline or unreachable at the time of the scan.
    \item \textbf{No Services:} It is also possible that no network services are hosted at this specific IP address.
\end{itemize}
Due to the lack of discovered services, no technical vulnerabilities could be identified from this scan. Further internal testing is recommended to assess the security of internal systems.

% --- Section 5: Risk Assessment ---
\section{Risk Assessment}

This section synthesizes findings from the security control review and technical scan. The following risks have been identified and prioritized based on their potential impact on the organization.

\begin{longtable}{p{0.1\linewidth} p{0.3\linewidth} p{0.45\linewidth} p{0.1\linewidth}}
\caption{Identified Risks} \\
\toprule
\textbf{ID} & \textbf{Risk Name} & \textbf{Description} & \textbf{Severity} \\ \midrule
\endfirsthead
\multicolumn{4}{c}%
{{\bfseries \tablename\ \thetable{} -- continued from previous page}} \\
\toprule
\textbf{ID} & \textbf{Risk Name} & \textbf{Description} & \textbf{Severity} \\ \midrule
\endhead
\bottomrule
\endfoot
RISK-001 & Lack of MFA for Sensitive Systems & The absence of Multi-Factor Authentication on systems containing sensitive or critical data. A compromised password would be sufficient for an attacker to gain direct access to high-value assets, potentially leading to a major data breach. & \textbf{Critical} \\ \addlinespace
RISK-002 & No Employee Acceptable Use Policy (AUP) & The organization lacks a formal policy defining the rules and guidelines for the use of its networks, systems, and data. This leads to inconsistent security practices and a lack of legal and administrative recourse in cases of misuse or negligence. & \textbf{High} \\ \bottomrule
\end{longtable}

% --- Section 6: Recommendations ---
\section{Recommendations}

The following actions are recommended to mitigate the risks identified in this report.

\begin{longtable}{p{0.1\linewidth} p{0.85\linewidth}}
\caption{Actionable Recommendations} \\
\toprule
\textbf{For Risk} & \textbf{Recommendation} \\ \midrule
\endfirsthead
\multicolumn{2}{c}%
{{\bfseries \tablename\ \thetable{} -- continued from previous page}} \\
\toprule
\textbf{For Risk} & \textbf{Recommendation} \\ \midrule
\endhead
\bottomrule
\endfoot
RISK-001 & \textbf{Implement MFA on All Sensitive Systems:}
\begin{itemize}
    \item Prioritize the immediate deployment of a robust MFA solution for all user accounts (including administrative and service accounts) that can access sensitive data repositories, databases, and critical applications.
    \item Ensure the policy is enforced and cannot be bypassed by users.
    \item This action is the single most effective control to prevent unauthorized access resulting from credential compromise.
\end{itemize} \\ \addlinespace
RISK-002 & \textbf{Develop and Implement an Acceptable Use Policy:}
\begin{itemize}
    \item Draft a comprehensive AUP that clearly outlines the rules for employee use of company technology, including email, internet, software, and data handling.
    \item Require all current and new employees to read and formally acknowledge the policy.
    \item Conduct regular training to ensure the policy is understood and followed. This policy will form the foundation of the organization's security culture.
\end{itemize} \\ \bottomrule
\end{longtable}

\end{document}
```