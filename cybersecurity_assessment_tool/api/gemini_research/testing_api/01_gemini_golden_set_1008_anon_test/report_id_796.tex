```latex
\documentclass[12pt]{article}

% --- PACKAGES ---
\usepackage[margin=1in]{geometry}
\usepackage{pifont} % For checkmarks and crosses
\usepackage{booktabs} % For professional tables
\usepackage{xcolor} % For colors
\usepackage{hyperref} % For clickable links
\usepackage{url} % For URL formatting
\usepackage{seqsplit} % For splitting long strings

% --- DOCUMENT SETUP ---
\hypersetup{
    colorlinks=true,
    linkcolor=blue,
    filecolor=magenta,      
    urlcolor=cyan,
    pdftitle={Cybersecurity Assessment Report},
    pdfpagemode=FullScreen,
}

\newcommand{\yes}{\ding{51}} % Green checkmark
\newcommand{\no}{\ding{55}}  % Red X

\definecolor{darkred}{rgb}{0.55, 0.0, 0.0}
\definecolor{darkorange}{rgb}{0.8, 0.33, 0.0}
\definecolor{darkgreen}{rgb}{0.0, 0.39, 0.0}

% --- DOCUMENT START ---
\begin{document}

% --- TITLE PAGE ---
\begin{titlepage}
    \centering
    \vspace*{\stretch{1.0}}
    \Huge{\textbf{Cybersecurity Assessment Report}}
    \vspace{0.5cm}
    \LARGE{\textbf{[Organization Name]}}
    \vspace{1.5cm}
    \large{Date: \today}
    \vspace{0.5cm}
    \large{Generated by: Cybersecurity Analyst}
    \vspace*{\stretch{2.0}}
\end{titlepage}

\tableofcontents
\newpage

% --- EXECUTIVE SUMMARY ---
\section{Executive Summary}
This report provides a cybersecurity assessment for \textbf{[Organization Name]}, based on an analysis of organizational security controls, an external network scan, and a review of pre-existing risks.

The assessment reveals a mixed security posture. On a technical level, the network perimeter appears secure, as the external scan of the target host \texttt{[Target IP]} found no open ports. This is a significant positive finding, suggesting a well-configured firewall or a host with no exposed services.

However, the review of organizational security controls identified two significant gaps that introduce substantial risk. The absence of a formal Acceptable Use Policy is a \textbf{Critical} governance failure. Additionally, the lack of mandatory annual security awareness training for all employees is a \textbf{High} risk, leaving the organization vulnerable to human-centric threats like phishing and social engineering.

Immediate action should be focused on addressing these policy and training deficiencies to bolster the human element of the organization's defense strategy.

% --- ORGANIZATIONAL INFORMATION ---
\section{Organizational Information}
The following information was used as the basis for this assessment.
\begin{center}
\begin{tabular}{ll}
\toprule
\textbf{Attribute} & \textbf{Value} \\
\midrule
Organization Name & \textbf{[Organization Name]} \\
Primary Domain & \texttt{[Domain]} \\
Target IP Address & \texttt{[Client IP]} \\
\bottomrule
\end{tabular}
\end{center}

% --- SECURITY CONTROL REVIEW ---
\section{Security Control Review}
A review of the organization's security controls was conducted via a questionnaire. The results below highlight current practices and identify key areas for improvement. Answers marked with a \no\ represent significant gaps in the security framework.

\begin{center}
\begin{tabular}{p{0.8\linewidth}c}
\toprule
\textbf{Control Question} & \textbf{Response} \\
\midrule
Do you require MFA to access email? & \yes \\
Do you require MFA to log into computers? & \yes \\
Do you require MFA to access sensitive data systems? & \yes \\
Does your organization have an employee acceptable use policy? & \textcolor{darkred}{\no} \\
Does your organization do security awareness training for new employees? & \yes \\
Does your organization do security awareness training for all employees at least once per year? & \textcolor{darkred}{\no} \\
\bottomrule
\end{tabular}
\end{center}

\subsection*{Analysis of Gaps}
\begin{itemize}
    \item \textbf{Acceptable Use Policy (AUP):} The lack of a formal AUP is a critical governance gap. An AUP sets clear expectations for employees on how to use company assets, protects the organization from legal liability, and is a foundational component of any cybersecurity program.
    \item \textbf{Annual Security Awareness Training:} While training for new hires is in place, the absence of ongoing, annual training for all staff is a high-risk oversight. The threat landscape evolves continuously, and without regular refreshers, employees are more likely to fall victim to modern phishing, ransomware, and social engineering attacks.
\end{itemize}

% --- TECHNICAL SCAN RESULTS ---
\section{Technical Scan Results}
An external network scan was performed to identify exposed services and potential vulnerabilities on the perimeter.

\begin{center}
\begin{tabular}{ll}
\toprule
\textbf{Scan Parameter} & \textbf{Value} \\
\midrule
Target IP & \texttt{[Target IP]} \\
Scan Date & \today \\
Host Status & Up \\
Open Ports Found & 0 \\
Filtered/Closed Ports & All other ports are closed \\
\bottomrule
\end{tabular}
\end{center}

\subsection*{Summary}
The scan of the target host \texttt{[Target IP]} found \textbf{no open ports}. This is a positive security finding, indicating that the host is either not running any publicly accessible services or is protected by a firewall that is correctly configured to block all inbound traffic. This significantly reduces the external attack surface of this specific asset.

% --- RISK ASSESSMENT ---
\section{Risk Assessment}
This section synthesizes findings from the security control review, technical scans, and pre-existing risk data. Based on the current inputs, the primary risks are related to organizational policy and procedure.

\begin{center}
\begin{tabular}{p{0.1\linewidth} p{0.5\linewidth} p{0.2\linewidth} l}
\toprule
\textbf{Risk ID} & \textbf{Risk Description} & \textbf{Source} & \textbf{Severity} \\
\midrule
R-001 & Lack of a formal Acceptable Use Policy (AUP) for all employees. & Org. Questionnaire & \textcolor{darkred}{Critical} \\
\addlinespace
R-002 & Lack of mandatory annual security awareness training for all employees. & Org. Questionnaire & \textcolor{darkorange}{High} \\
\bottomrule
\end{tabular}
\end{center}

% --- RECOMMENDATIONS ---
\section{Recommendations}
The following actions are recommended to mitigate the identified risks and improve the overall security posture of \textbf{[Organization Name]}.

\subsection*{R-001: Implement an Acceptable Use Policy (Critical)}
\begin{itemize}
    \item \textbf{Action:} Develop and implement a comprehensive Acceptable Use Policy (AUP) that clearly defines the rules and expectations for the use of company technology, network, and data resources.
    \item \textbf{Details:} The policy should cover topics such as data privacy, internet usage, email etiquette, and security responsibilities. All current and future employees must be required to read and formally acknowledge the policy.
    \item \textbf{Impact:} Establishes a clear security baseline for employee behavior, reduces insider threat risk, and strengthens the organization's legal and compliance posture.
\end{itemize}

\subsection*{R-002: Establish Annual Security Awareness Training (High)}
\begin{itemize}
    \item \textbf{Action:} Institute a mandatory security awareness training program that all employees must complete at least once per year.
    \item \textbf{Details:} The training should cover current threats, including phishing, ransomware, password security, and social engineering. Consider incorporating periodic phishing simulations to test and reinforce the training.
    \item \textbf{Impact:} Creates a security-conscious culture, reduces the likelihood of successful phishing attacks, and empowers employees to become an active part of the organization's defense.
\end{itemize}

\end{document}
```