```latex
\documentclass[12pt]{article}

% -------------------
% --- PREAMBLE ---
% -------------------
\usepackage[margin=1in]{geometry}
\usepackage{pifont} % For checkmarks and crosses
\usepackage{booktabs} % For professional tables
\usepackage{graphicx}
\usepackage{xcolor}
\usepackage{url}
\usepackage{seqsplit} % For breaking long strings in tt font
\usepackage[T1]{fontenc}
\usepackage{hyperref}

% --- Color and Hyperlink Setup ---
\definecolor{darkblue}{rgb}{0.0, 0.0, 0.55}
\definecolor{darkred}{rgb}{0.55, 0.0, 0.0}
\hypersetup{
    colorlinks=true,
    linkcolor=darkblue,
    filecolor=magenta,
    urlcolor=darkblue,
    pdftitle={Cybersecurity Posture Assessment Report},
    pdfauthor={Cybersecurity Analysis Division},
    pdfsubject={Security Assessment},
    pdfkeywords={Cybersecurity, Risk, Assessment},
}

% --- Custom Commands ---
\newcommand{\yes}{\textcolor{green}{\ding{51}}}
\newcommand{\no}{\textcolor{red}{\ding{55}}}

% -------------------
% --- DOCUMENT ---
% -------------------
\begin{document}

% -------------------
% --- TITLE PAGE ---
% -------------------
\begin{titlepage}
    \centering
    \vspace*{1cm}
    \includegraphics[width=0.3\textwidth]{example-image-a} % Placeholder for a logo
    \vfill
    \Huge\bfseries Cybersecurity Posture Assessment Report
    \vfill
    \large
    \textbf{Prepared for:}\\
    \vspace{0.2cm}
    \textbf{[Organization Name]}
    \vfill
    \large
    \textbf{Date of Assessment:}\\
    \vspace{0.2cm}
    November 22, 2025
    \vfill
    \large
    \textbf{Prepared by:}\\
    \vspace{0.2cm}
    Cybersecurity Analysis Division
    \vfill
\end{titlepage}

\tableofcontents
\newpage

% ---------------------------
% --- EXECUTIVE SUMMARY ---
% ---------------------------
\section{Executive Summary}

This report details the findings of a cybersecurity posture assessment conducted on \textbf{November 22, 2025}. The assessment combined a review of organizational security controls, an external network vulnerability scan, and an analysis of pre-existing risks to provide a comprehensive overview of the organization's security posture.

The organization demonstrates a solid foundation in security policy and awareness, with established acceptable use policies and regular employee training. Furthermore, Multi-Factor Authentication (MFA) is correctly enforced for email and access to sensitive data systems.

However, two significant risks were identified that require immediate attention:
\begin{itemize}
    \item \textbf{Lack of MFA on Endpoints:} Employee computers are not protected by MFA, leaving them vulnerable to unauthorized access if credentials are compromised. This is a critical gap in the organization's defense-in-depth strategy.
    \item \textbf{Outdated Web Server Software:} The external-facing web server at \texttt{[Target IP]} is running an outdated version of Nginx (1.18.0). This version is known to have multiple security vulnerabilities, which could be exploited by attackers to compromise the server and gain access to the internal network.
\end{itemize}

Due to these findings, the organization's overall security posture is assessed as \textbf{Moderate}. This report provides specific, actionable recommendations to mitigate these risks and improve the overall security posture.

% ---------------------------------
% --- ORGANIZATIONAL INFORMATION ---
% ---------------------------------
\section{Organizational Information}

This section provides the key identification details for the organization under review. As per the provided data, the following placeholders are used.

\begin{table}[h!]
\centering
\begin{tabular}{@{}ll@{}}
\toprule
\textbf{Attribute} & \textbf{Value} \\ \midrule
Organization Name & \textbf{[Organization Name]} \\
Primary Email Domain & \texttt{[Domain]} \\
External IP Address Scanned & \texttt{[Client IP]} \\ \bottomrule
\end{tabular}
\caption{Client Identification Details.}
\end{table}

% ---------------------------------
% --- SECURITY CONTROL REVIEW ---
% ---------------------------------
\section{Security Control Review}

A review of administrative and technical security controls was conducted via a questionnaire. The responses indicate the current state of implemented security policies. A "No" response highlights a potential control gap that may introduce significant risk.

\begin{table}[h!]
\centering
\begin{tabular}{@{}p{0.8\linewidth}c@{}}
\toprule
\textbf{Control Question} & \textbf{Response} \\ \midrule
Do you require MFA to access email? & \yes \\
Do you require MFA to log into computers? & \no \\
Do you require MFA to access sensitive data systems? & \yes \\
Does your organization have an employee acceptable use policy? & \yes \\
Does your organization do security awareness training for new employees? & \yes \\
Does your organization do security awareness training for all employees at least once per year? & \yes \\ \bottomrule
\end{tabular}
\caption{Security Controls Questionnaire Results.}
\end{table}

\subsection*{Analysis}
The organization has implemented several critical security controls effectively, particularly concerning MFA for email and sensitive systems, and a robust security awareness program. However, the lack of MFA for computer logins is a \textbf{critical control gap}. An attacker with stolen credentials (e.g., from a phishing attack) could gain direct access to an employee's workstation, bypassing other security measures.

% -------------------------------
% --- TECHNICAL SCAN RESULTS ---
% -------------------------------
\section{Technical Scan Results}

An external network scan was performed on \textbf{November 22, 2025}, to identify open ports and exposed services.

\begin{itemize}
    \item \textbf{Target IP Address:} \texttt{[Target IP]}
    \item \textbf{Scan Date:} 2025-11-22T10:00:00Z
\end{itemize}

The following table summarizes the findings for the host that was found to be online.

\begin{table}[h!]
\centering
\begin{tabular}{@{}lllll@{}}
\toprule
\textbf{Port} & \textbf{State} & \textbf{Service} & \textbf{Product} & \textbf{Version} \\ \midrule
443/tcp & open & https & nginx & 1.18.0 \\ \bottomrule
\end{tabular}
\caption{Open Ports and Services on \texttt{[Target IP]}.}
\end{table}

\subsection*{Analysis}
The scan identified a web server running \textbf{Nginx version 1.18.0}. This version was released in April 2020 and is now considered outdated. The Nginx project maintains a list of security advisories, and versions prior to the latest stable releases are susceptible to numerous vulnerabilities, including but not limited to request smuggling, denial-of-service, and information disclosure. Exposing an outdated web server to the internet presents a \textbf{High} risk to the organization.

% -----------------------
% --- RISK ASSESSMENT ---
% -----------------------
\section{Risk Assessment}

This section synthesizes the findings from the security control review and the technical scan. The provided list of current risks was empty; therefore, the risks below are newly identified during this assessment.

\begin{table}[h!]
\centering
\begin{tabular}{@{}lp{0.25\linewidth}p{0.5\linewidth}l@{}}
\toprule
\textbf{ID} & \textbf{Risk Title} & \textbf{Description} & \textbf{Severity} \\ \midrule
R-001 & Lack of MFA on Employee Computers & The absence of MFA on workstations allows an attacker with compromised credentials to gain unauthorized access to the endpoint and potentially the internal network. & \textbf{High} \\
\addlinespace
R-002 & Outdated Web Server Software & The public-facing web server runs Nginx 1.18.0, which has multiple known public vulnerabilities. This exposes the server to remote attacks, which could lead to a system compromise. & \textbf{High} \\ \bottomrule
\end{tabular}
\caption{Summary of Identified Risks.}
\end{table}

% ---------------------
% --- RECOMMENDATIONS ---
% ---------------------
\section{Recommendations}

The following actions are recommended to mitigate the identified risks and strengthen the organization's security posture. Recommendations are prioritized based on severity and potential impact.

\subsection{R-001: Implement MFA for Endpoint Authentication (Priority: Critical)}
\begin{itemize}
    \item \textbf{Action:} Deploy a mandatory Multi-Factor Authentication solution for all employee computer and laptop logins. This control should be enforced for both local and remote access.
    \item \textbf{Justification:} This will significantly reduce the risk of unauthorized access from compromised credentials, acting as a critical compensating control for phishing and password-based attacks.
    \item \textbf{Suggested Solutions:} Evaluate and implement solutions such as Windows Hello for Business, Duo Security, Okta, or other FIDO2-compliant hardware tokens.
\end{itemize}

\subsection{R-002: Upgrade Nginx Web Server (Priority: High)}
\begin{itemize}
    \item \textbf{Action:} Develop a patch management plan to upgrade the Nginx server from version 1.18.0 to the latest stable version recommended by the vendor.
    \item \textbf{Justification:} Upgrading the software will patch known vulnerabilities, protecting the server from publicly known exploits. A regular patch management cycle should be established to prevent this issue from recurring.
    \item \textbf{Suggested Solutions:} The upgrade should first be tested in a non-production (staging) environment to ensure no disruption to web services or applications. Follow official Nginx documentation for the upgrade process.
\end{itemize}

\end{document}
```