```latex
\documentclass[12pt]{article}

% Preamble: Required Packages
\usepackage[margin=1in]{geometry} % Set page margins
\usepackage{pifont}               % For symbols like checkmarks (\ding{51}) and crosses (\ding{55})
\usepackage{booktabs}             % For professional-looking tables (\toprule, \midrule, \bottomrule)
\usepackage[hidelinks]{hyperref}  % For clickable links without boxes
\usepackage{url}                  % For formatting URLs
\usepackage{seqsplit}             % To split long strings without breaking words
\usepackage{xcolor}               % For defining custom colors
\usepackage{colortbl}             % For coloring table cells and rows

% --- Custom Definitions ---
% Define colors for severity and responses
\definecolor{severitycritical}{HTML}{990000}
\definecolor{severityhigh}{HTML}{D14302}
\definecolor{severitymedium}{HTML}{EFAF00}
\definecolor{severitylow}{HTML}{3A7A2A}
\definecolor{responseyes}{HTML}{2E8B57} % SeaGreen
\definecolor{responseno}{HTML}{B22222}  % FireBrick
\definecolor{tableheader}{gray}{0.90}

% Define commands for checkmarks and crosses with color
\newcommand{\yes}{\textcolor{responseyes}{\ding{51}}}
\newcommand{\no}{\textcolor{responseno}{\ding{55}}}

% --- Document Metadata ---
\title{Cybersecurity Posture Assessment Report \\ \large For: \textbf{[Organization Name]}}
\author{Cybersecurity Analysis Division}
\date{\today}

\begin{document}

\maketitle
\thispagestyle{empty} % No page number on the title page

\newpage
\tableofcontents
\thispagestyle{empty} % No page number on the ToC page

\newpage
\setcounter{page}{1} % Start page numbering at 1

%======================================================================
\section{Executive Summary}
%======================================================================

This report details the findings of a cybersecurity posture assessment conducted for \textbf{[Organization Name]}. The assessment combined a review of organizational security controls, an external network scan, and an analysis of pre-existing risk data.

The most critical finding is the direct exposure of Remote Desktop Protocol (RDP) on port 3389 to the public internet at the client's external IP address, \texttt{[Client IP]}. This configuration represents a \textbf{Critical Risk} (CVSS 9.0) and makes the organization a prime target for brute-force attacks, credential theft, and ransomware deployment.

This technical vulnerability is dangerously compounded by a significant gap in organizational policy: the lack of multi-factor authentication (MFA) required to access sensitive data systems. An attacker who successfully compromises credentials via the exposed RDP service could potentially gain direct access to critical assets without facing a secondary authentication challenge.

Immediate remediation is required to address the RDP exposure. Further recommendations are provided to strengthen the organization's overall security posture and mitigate the identified risks.

%======================================================================
\section{Organizational Information}
%======================================================================

The following information was used as the basis for this assessment. As per the provided data, placeholder values are used where specific details were not available.

\begin{itemize}
    \item \textbf{Organization Name:} \textbf{[Organization Name]}
    \item \textbf{Primary Email Domain:} \texttt{[Domain]}
    \item \textbf{External IP Assessed:} \texttt{[Client IP]}
\end{itemize}

%======================================================================
\section{Security Control Review}
%======================================================================

A review of the organization's security controls was conducted via a questionnaire. The responses highlight a strong foundation in general security practices, such as MFA for email and regular security training. However, a critical control gap was identified regarding access to sensitive systems.

\begin{table}[h!]
\centering
\caption{Security Controls Questionnaire Analysis}
\begin{tabular}{p{0.6\linewidth} c p{0.2\linewidth}}
\toprule
\rowcolor{tableheader}
\textbf{Control Question} & \textbf{Response} & \textbf{Assessment} \\
\midrule
Do you require MFA to access email? & \yes & Best Practice Met \\
Do you require MFA to log into computers? & \yes & Best Practice Met \\
Do you require MFA to access sensitive data systems? & \no & \textbf{Critical Gap} \\
Does your organization have an employee acceptable use policy? & \yes & Best Practice Met \\
Does your organization do security awareness training for new employees? & \yes & Best Practice Met \\
Does your organization do security awareness training for all employees at least once per year? & \yes & Best Practice Met \\
\bottomrule
\end{tabular}
\end{table}

The absence of MFA for sensitive data systems significantly increases the potential impact of a credential compromise, as it removes a crucial layer of defense for the organization's most valuable assets.

%======================================================================
\section{Technical Scan Results}
%======================================================================

An external network scan was performed against the target IP address \texttt{[Target IP]}. The scan identified one open port, which presents a significant security risk.

\subsection{Open Ports Discovered}

\begin{table}[h!]
\centering
\caption{Open Port Findings for Target: \texttt{[Target IP]}}
\begin{tabular}{l l l l}
\toprule
\rowcolor{tableheader}
\textbf{Port} & \textbf{State} & \textbf{Service} & \textbf{Analysis} \\
\midrule
3389/tcp & Open & ms-wbt-server & Remote Desktop Protocol (RDP) \\
\bottomrule
\end{tabular}
\end{table}

\subsection{Analysis of Findings}

The scan confirms that port \textbf{3389}, the standard port for Microsoft's Remote Desktop Protocol (RDP), is open to the public internet. RDP is a common vector for cyberattacks. Exposing this service directly allows attackers to:
\begin{itemize}
    \item Perform brute-force or password-spraying attacks to guess user credentials.
    \item Exploit unpatched vulnerabilities in the RDP service itself (e.g., BlueKeep).
    \item Gain an initial foothold into the internal network, which is often a precursor to a ransomware attack.
\end{itemize}
This finding validates the pre-existing risk identified as "RDP Exposure" and elevates its urgency.

%======================================================================
\section{Consolidated Risk Assessment}
%======================================================================

The following table synthesizes findings from the security control review, the technical scan, and pre-existing risk data to provide a consolidated view of the organization's current risk profile.

\begin{table}[h!]
\centering
\caption{Summary of Identified Risks}
\begin{tabular}{p{0.25\linewidth} p{0.5\linewidth} l}
\toprule
\rowcolor{tableheader}
\textbf{Risk Name} & \textbf{Description} & \textbf{Severity} \\
\midrule
\textbf{Public RDP Exposure} & The RDP service on port 3389 is exposed on \texttt{[Client IP]}, allowing unauthorized connection attempts from the internet. This is confirmed by both the network scan and existing risk data. & \textcolor{severitycritical}{\textbf{Critical (9.0)}} \\
\addlinespace
\textbf{Lack of MFA on Sensitive Systems} & Sensitive data systems do not require multi-factor authentication for access. This control gap, identified in the questionnaire, drastically increases the impact of a successful credential compromise. & \textcolor{severityhigh}{\textbf{High}} \\
\bottomrule
\end{tabular}
\end{table}

%======================================================================
\section{Recommendations}
%======================================================================

The following prioritized recommendations are provided to mitigate the identified risks and improve the overall security posture of \textbf{[Organization Name]}.

\subsection{Immediate Priority (Remediate within 24 hours)}
\begin{enumerate}
    \item \textbf{Block Port 3389 at the Network Firewall:} Immediately configure the firewall protecting \texttt{[Client IP]} to deny all inbound traffic to TCP port 3389. This will instantly remove the public RDP exposure and is the single most important action to take.
\end{enumerate}

\subsection{High Priority (Remediate within 30 days)}
\begin{enumerate}
    \setcounter{enumi}{1} % Continue numbering
    \item \textbf{Implement a Secure Remote Access Solution:} Replace direct RDP access with a Virtual Private Network (VPN) solution. The VPN must be configured to require multi-factor authentication for all users, providing a secure and authenticated tunnel for remote administration.
    \item \textbf{Enforce MFA on All Sensitive Systems:} Address the identified control gap by deploying and mandating the use of MFA for all systems classified as containing sensitive data. This ensures that even if credentials are stolen, an additional factor is required for access.
\end{enumerate}

\subsection{Medium Priority (Remediate within 90 days)}
\begin{enumerate}
    \setcounter{enumi}{3} % Continue numbering
    \item \textbf{Conduct a Comprehensive Vulnerability Scan:} Perform an authenticated and unauthenticated vulnerability scan across the organization's external and internal networks to identify any other exposed services, misconfigurations, or missing security patches.
    \item \textbf{Review and Strengthen Password Policies:} Ensure that strong password complexity and length requirements are enforced for all accounts, especially those with administrative privileges.
\end{enumerate}

\end{document}
```