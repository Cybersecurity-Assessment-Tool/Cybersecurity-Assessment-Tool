```latex
\documentclass[12pt]{article}

% Preamble: Required Packages
\usepackage[margin=1in]{geometry}
\usepackage{pifont} % For checkmarks and crosses
\usepackage{booktabs} % For professional-looking tables
\usepackage{hyperref} % For hyperlinks and metadata
\usepackage{url} % For formatting URLs
\usepackage{seqsplit} % To split long monospaced strings
\usepackage{graphicx} % For potential logos
\usepackage{fancyhdr} % For headers and footers

% --- Document Metadata ---
\hypersetup{
    colorlinks=true,
    linkcolor=black,
    urlcolor=blue,
    pdftitle={Cybersecurity Posture Assessment Report},
    pdfauthor={Cybersecurity Analysis Cell},
    pdfsubject={Security Assessment},
    pdfkeywords={Cybersecurity, Risk, Assessment, Nmap, Policy}
}

% --- Header and Footer Configuration ---
\pagestyle{fancy}
\fancyhf{} % Clear all header and footer fields
\fancyhead[L]{Cybersecurity Posture Assessment}
\fancyhead[R]{\textbf{[Organization Name]}}
\fancyfoot[C]{\thepage}
\renewcommand{\headrulewidth}{0.4pt}
\renewcommand{\footrulewidth}{0.4pt}

% --- Helper Command for Checkmarks/Crosses ---
\newcommand{\response}[1]{\ifstrequal{#1}{Yes}{\textcolor[rgb]{0.1,0.6,0.1}{\ding{51}}}{\textcolor[rgb]{0.8,0.1,0.1}{\ding{55}}}}

\begin{document}

% --- Title Page ---
\begin{titlepage}
    \centering
    \vspace*{1cm}
    \includegraphics[width=0.3\textwidth]{example-image-a} % Placeholder for a logo
    
    \vspace{1.5cm}
    
    \Huge
    \textbf{Cybersecurity Posture Assessment Report}
    
    \vspace{1.5cm}
    
    \Large
    Prepared for: \textbf{[Organization Name]}
    
    \vspace{2cm}
    
    \large
    \today
    
    \vfill
    
    \normalsize
    \textit{This report contains sensitive information and should be handled with care. Distribution is restricted to authorized personnel only.}
    
\end{titlepage}

\tableofcontents
\newpage

% --- Section 1: Executive Summary ---
\section{Executive Summary}

This report details the findings of a cybersecurity posture assessment for \textbf{[Organization Name]}. The assessment combined a review of organizational security controls, an external network vulnerability scan, and an analysis of pre-existing risks.

The overall security posture presents a notable contrast. On one hand, the organization's external network perimeter appears highly secure. A network scan against the designated target IP address revealed no open ports, indicating a robust and well-configured firewall that effectively minimizes the external attack surface. This is a commendable security practice.

On the other hand, a critical administrative gap was identified in the security control review. The organization currently lacks a formal Employee Acceptable Use Policy (AUP). This absence represents a \textbf{High} risk, as it fails to establish clear guidelines for employees regarding the use of company assets, potentially leading to unintentional security incidents, insider threats, and legal liabilities.

Key recommendations focus on mitigating this policy gap by immediately developing and implementing a comprehensive AUP. Continued vigilance in maintaining the strong network perimeter is also advised.

% --- Section 2: Organizational Information ---
\section{Organizational Information}

This section provides a summary of the organizational details used as a basis for this assessment. The data has been anonymized as requested.

\begin{table}[h!]
\centering
\begin{tabular}{@{}ll@{}}
\toprule
\textbf{Attribute} & \textbf{Value} \\ \midrule
Organization Name & \textbf{[Organization Name]} \\
Primary Domain & \texttt{[Domain]} \\
External IP Scanned & \texttt{[Client IP]} \\ \bottomrule
\end{tabular}
\caption{Client Organizational Details.}
\end{table}

% --- Section 3: Security Control Review ---
\section{Security Control Review}

A review of administrative and technical security controls was conducted via a standardized questionnaire. The responses provide insight into the organization's internal security policies and practices.

\begin{table}[h!]
\centering
\begin{tabular}{@{}p{0.75\linewidth}c@{}}
\toprule
\textbf{Control Question} & \textbf{Response} \\ \midrule
Do you require MFA to access email? & \response{Yes} \\
Do you require MFA to log into computers? & \response{Yes} \\
Do you require MFA to access sensitive data systems? & \response{Yes} \\
Does your organization have an employee acceptable use policy? & \response{No} \\
Does your organization do security awareness training for new employees? & \response{Yes} \\
Does your organization do security awareness training for all employees at least once per year? & \response{Yes} \\ \bottomrule
\end{tabular}
\caption{Security Control Questionnaire Responses.}
\end{table}

\subsection*{Analysis of Controls}
The organization demonstrates a strong commitment to identity and access management, with comprehensive enforcement of Multi-Factor Authentication (MFA) across critical systems. The security awareness training program appears robust, covering both onboarding and annual refreshers.

However, the lack of an Employee Acceptable Use Policy (AUP) is a critical deficiency. An AUP is a foundational governance document that sets clear expectations for all employees on the proper use of corporate technology and data resources. Its absence creates ambiguity and increases the risk of both malicious and accidental insider threats.

% --- Section 4: Technical Scan Results ---
\section{Technical Scan Results}

An external network scan was performed to identify open ports, running services, and potential vulnerabilities visible from the public internet.

\begin{itemize}
    \item \textbf{Target IP Address:} \texttt{[Target IP]}
    \item \textbf{Scan Date:} Data not provided in scan results.
    \item \textbf{Scanner Used:} Nmap
\end{itemize}

\subsection*{Summary of Findings}
The scan determined that the target host is online and responsive. However, \textbf{no open TCP or UDP ports were discovered}. All scanned ports were found to be in a `closed` state.

\paragraph{Interpretation:} This result is highly positive from a security perspective. It indicates that a strong, default-deny firewall policy is in place, which significantly reduces the external attack surface. This configuration prevents unauthorized external actors from discovering and potentially exploiting network services.

% --- Section 5: Risk Assessment ---
\section{Risk Assessment}

This section synthesizes findings from the security control review, technical scan, and pre-existing risk data to provide a consolidated view of the current risk landscape. The pre-existing risk list provided was empty.

\begin{table}[h!]
\centering
\begin{tabular}{@{}p{0.25\linewidth}p{0.5\linewidth}l@{}}
\toprule
\textbf{Risk Name} & \textbf{Overview} & \textbf{Severity} \\ \midrule
\textbf{Lack of Acceptable Use Policy (AUP)} & The organization does not have a formal, documented policy defining the acceptable use of company assets, including computers, networks, and data. This leads to inconsistent user behavior and a lack of enforceable security standards. & \textbf{High} \\
\bottomrule
\end{tabular}
\caption{Consolidated Risk Register.}
\end{table}

% --- Section 6: Recommendations ---
\section{Recommendations}

Based on the analysis, the following actions are recommended to improve the security posture of \textbf{[Organization Name]}.

\subsection*{High Priority Recommendations}

\subsubsection*{1. Develop and Implement an Acceptable Use Policy (AUP)}
\begin{itemize}
    \item \textbf{Action:} Draft a comprehensive AUP that clearly defines rules and responsibilities for all employees and contractors. The policy should cover topics such as data handling, internet usage, email security, use of personal devices (BYOD), and consequences for non-compliance.
    \item \textbf{Impact:} An AUP is a critical administrative control that reduces insider threats, provides a legal framework for enforcing security rules, helps meet regulatory compliance requirements, and fosters a security-conscious culture.
    \item \textbf{Timeline:} Immediate (within 30 days).
\end{itemize}

\subsection*{Informational Recommendations}

\subsubsection*{2. Maintain Strong Network Perimeter Security}
\begin{itemize}
    \item \textbf{Action:} The current restrictive firewall configuration is excellent. Continue to maintain and regularly audit this configuration. Ensure a formal change management process is used for any modifications to firewall rules.
    \item \textbf{Impact:} Preserves the strong defensive posture against external network-based attacks and prevents accidental exposure of internal services.
    \item \textbf{Timeline:} Ongoing.
\end{itemize}

\end{document}
```