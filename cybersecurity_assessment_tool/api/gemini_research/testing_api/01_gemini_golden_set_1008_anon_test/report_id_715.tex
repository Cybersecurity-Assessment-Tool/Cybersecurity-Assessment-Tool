```latex
\documentclass[12pt]{article}

% Preamble: Required Packages
\usepackage[margin=1in]{geometry}
\usepackage{pifont} % For checkmarks and crosses
\usepackage{booktabs} % For professional tables
\usepackage{hyperref} % For hyperlinks
\usepackage{url} % For URL formatting
\usepackage{seqsplit} % To split long strings without breaking
\usepackage[T1]{fontenc}

% Document Metadata
\title{Cybersecurity Posture Assessment Report}
\author{Cybersecurity Analyst}
\date{\today}

\begin{document}

\maketitle
\thispagestyle{empty}
\newpage
\tableofcontents
\newpage

% --- 1. Executive Overview ---
\section{Executive Overview}

This report details the findings of a cybersecurity assessment for \textbf{[Organization Name]}, conducted on \today. The assessment combined a review of organizational security controls, an external network scan, and an analysis of pre-existing risks.

While the organization demonstrates a foundational security posture with established employee acceptable use policies and security awareness training programs, several critical and high-risk vulnerabilities were identified that require immediate attention.

Key findings include:
\begin{itemize}
    \item \textbf{Critical Gaps in Access Control:} Multi-Factor Authentication (MFA) is not enforced for accessing corporate email or for logging into company computers. This exposes the organization to significant risk from credential theft and unauthorized access.
    \item \textbf{Exposed Management Service:} The external network scan revealed an open Secure Shell (SSH) port (22/TCP). Exposing management services directly to the internet creates a substantial attack surface.
    \item \textbf{No Pre-existing Risks Logged:} The provided data indicated no previously documented vulnerabilities, suggesting a potential gap in continuous risk tracking.
\end{itemize}

Immediate remediation of the identified access control and network exposure issues is strongly recommended to reduce the likelihood of a security breach.

% --- 2. Organizational Information ---
\section{Organizational Information}

The following information was used as the basis for this assessment. Due to the anonymized nature of the input data, placeholders have been used where necessary.

\begin{table}[h!]
\centering
\begin{tabular}{@{}ll@{}}
\toprule
\textbf{Attribute} & \textbf{Value} \\ \midrule
Organization Name & \textbf{[Organization Name]} \\
Primary Domain & \texttt{[Domain]} \\
External IP Address & \texttt{[Client IP]} \\ \bottomrule
\end{tabular}
\caption{Client Organizational Data}
\label{tab:org_data}
\end{table}

% --- 3. Security Control Review (Questionnaire) ---
\section{Security Control Review (Questionnaire)}

A review of the organization's security controls was conducted via a standardized questionnaire. The responses highlight gaps in fundamental identity and access management practices.

\begin{table}[h!]
\centering
\begin{tabular}{@{}p{0.6\linewidth}cp{0.2\linewidth}@{}}
\toprule
\textbf{Control Question} & \textbf{Response} & \textbf{Assessment} \\ \midrule
Do you require MFA to access email? & \ding{55} & \textbf{Critical Gap} \\
Do you require MFA to log into computers? & \ding{55} & \textbf{High Risk} \\
Do you require MFA to access sensitive data systems? & \ding{51} & Best Practice \\
Does your organization have an employee acceptable use policy? & \ding{51} & Best Practice \\
Does your organization do security awareness training for new employees? & \ding{51} & Best Practice \\
Does your organization do security awareness training for all employees at least once per year? & \ding{51} & Best Practice \\ \bottomrule
\end{tabular}
\caption{Security Control Questionnaire Results}
\label{tab:controls}
\end{table}

% --- 4. Technical Scan Results ---
\section{Technical Scan Results}

An external network vulnerability scan was performed against the organization's provided IP address.

\subsection{Nmap Scan}
\begin{itemize}
    \item \textbf{Target IP:} \texttt{[Target IP]}
    \item \textbf{Scan Date:} [Scan Date]
    \item \textbf{Status:} Host is Up
\end{itemize}

The scan identified the following open port accessible from the public internet.

\begin{table}[h!]
\centering
\begin{tabular}{@{}lllll@{}}
\toprule
\textbf{Port} & \textbf{State} & \textbf{Service} & \textbf{Product} & \textbf{Version} \\ \midrule
22/tcp & open & ssh & \textit{n/a} & \textit{n/a} \\ \bottomrule
\end{tabular}
\caption{Open Ports Detected on \texttt{[Client IP]}}
\label{tab:nmap_results}
\end{table}

\textbf{Analysis:} The presence of an open SSH port (22) indicates that a remote management interface is exposed to the internet. While necessary for administration, direct exposure is highly discouraged. The scan was unable to fingerprint the specific version of the SSH service, which prevents an automated check for known vulnerabilities.

% --- 5. Consolidated Risk Assessment ---
\section{Consolidated Risk Assessment}

The following table synthesizes findings from the security control review and the technical scan into a prioritized list of risks.

\begin{table}[h!]
\centering
\begin{tabular}{@{}p{0.1\linewidth}p{0.6\linewidth}l@{}}
\toprule
\textbf{Risk ID} & \textbf{Description} & \textbf{Severity} \\ \midrule
RISK-001 & \textbf{Lack of MFA for Email Access:} Corporate email is a primary target for phishing and account takeover. Without MFA, a single compromised password grants an attacker full access. & \textbf{Critical} \\
\addlinespace
RISK-002 & \textbf{Lack of MFA for Endpoint Login:} The absence of MFA on computer logins allows an attacker with stolen credentials to easily gain access to an endpoint and the corporate network. & \textbf{High} \\
\addlinespace
RISK-003 & \textbf{Exposed SSH Management Interface:} The SSH service on port 22 is open to the internet, making it a target for brute-force attacks and exploitation of potential vulnerabilities. & \textbf{High} \\ \bottomrule
\end{tabular}
\caption{Summary of Identified Risks}
\label{tab:risks}
\end{table}

% --- 6. Recommendations ---
\section{Recommendations}

The following actions are recommended to mitigate the identified risks and improve the overall security posture of \textbf{[Organization Name]}.

\subsection{Remediation for RISK-001 (Critical)}
\begin{itemize}
    \item \textbf{Immediate Action:} Enforce mandatory Multi-Factor Authentication (MFA) for all user accounts, including administrative and service accounts, across the email platform (e.g., Microsoft 365, Google Workspace).
    \item \textbf{Policy Update:} Update the access control policy to require MFA for all externally facing services.
\end{itemize}

\subsection{Remediation for RISK-002 (High)}
\begin{itemize}
    \item \textbf{Short-Term Action:} Implement and enforce MFA for all remote and on-premise computer logins. Solutions like Windows Hello for Business, Duo, or other third-party tools can be used.
    \item \textbf{User Training:} Communicate the change to all employees and provide training on how to use the new MFA system for computer access.
\end{itemize}

\subsection{Remediation for RISK-003 (High)}
\begin{itemize}
    \item \textbf{Immediate Action:} Restrict access to the SSH port (22) using a firewall. Access should be limited to specific, trusted IP addresses (IP whitelisting).
    \item \textbf{Best Practice:} If remote access is required from dynamic locations, implement a Virtual Private Network (VPN) or a bastion host (jump box) and require administrators to connect through it before accessing SSH.
    \item \textbf{Configuration Hardening:} Ensure the SSH service is configured securely by disabling password-based authentication and enforcing the use of strong cryptographic keys (e.g., ED25519).
\end{itemize}

\end{document}
```