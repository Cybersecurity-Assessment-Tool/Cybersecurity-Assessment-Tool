```latex
\documentclass[12pt]{article}

% Preamble: Required Packages
\usepackage[margin=1in]{geometry}
\usepackage{pifont} % For checkmarks and crosses
\usepackage{booktabs} % For professional tables
\usepackage{hyperref} % For hyperlinks and metadata
\usepackage{url} % For URL formatting
\usepackage{seqsplit} % For splitting long strings in tt font
\usepackage{graphicx}
\usepackage{xcolor}

% Hyperref Setup
\hypersetup{
    colorlinks=true,
    linkcolor=blue,
    filecolor=magenta,      
    urlcolor=cyan,
    pdftitle={Cybersecurity Posture Assessment Report},
    pdfauthor={Cybersecurity Analysis Team},
    pdfsubject={Security Assessment},
    pdfkeywords={Security, Analysis, Report},
    bookmarks=true
}

% Document Title Block
\title{Cybersecurity Posture Assessment Report}
\author{Cybersecurity Analysis Team}
\date{\today}

\begin{document}

\maketitle
\thispagestyle{empty}
\newpage

\tableofcontents
\newpage

% --- Section 1: Executive Overview ---
\section{Executive Overview}
This report provides a comprehensive analysis of the cybersecurity posture for \textbf{[Organization Name]}. The assessment is based on a synthesis of network scan data, a review of organizational security controls, and an evaluation of pre-existing risks.

The analysis has identified several critical and high-risk vulnerabilities that require immediate attention. Key findings include:
\begin{itemize}
    \item \textbf{A critically vulnerable public-facing service:} An outdated FTP server (\texttt{vsftpd 2.3.4}) was discovered on the external IP address. This version is known to be susceptible to a remote code execution backdoor (CVE-2011-2523). Furthermore, the service is dangerously misconfigured to allow anonymous public login.
    \item \textbf{Significant gaps in access control:} Multi-Factor Authentication (MFA) is not enforced for accessing email or other sensitive data systems, leaving these critical assets vulnerable to credential compromise.
    \item \textbf{Deficiencies in security governance:} The organization lacks a formal Acceptable Use Policy (AUP) and does not provide security awareness training to new employees, creating a weak human firewall.
    \item \textbf{Systemic lifecycle management issues:} The continued use of an end-of-life operating system (Windows 7) on workstations, coupled with the outdated FTP server, indicates a systemic issue with patching and asset lifecycle management.
\end{itemize}

The combination of these findings presents a high likelihood of a security breach. Immediate and decisive action is required to remediate these issues and mitigate the associated risks. This report outlines specific, actionable recommendations to improve the organization's security posture.

% --- Section 2: Organizational Information ---
\section{Organizational Information}
This section provides a summary of the organizational details used for this assessment. Due to the anonymized nature of the provided data, placeholders have been used where necessary.

\begin{tabular}{@{}ll}
    \toprule
    \textbf{Attribute} & \textbf{Value} \\
    \midrule
    Organization Name & \textbf{[Organization Name]} \\
    Primary Email Domain & \texttt{[Domain]} \\
    External IP Address Scanned & \texttt{[Client IP]} \\
    \bottomrule
\end{tabular}

% --- Section 3: Security Control Review ---
\section{Security Control Review}
A review of the organization's security controls was conducted via a questionnaire. The responses indicate significant gaps in foundational security practices. A "No" response, marked with a \ding{55}, highlights a missing control and a potential area of high risk.

\begin{table}[h!]
\centering
\caption{Security Controls Questionnaire Results}
\begin{tabular}{@{}lc@{}}
    \toprule
    \textbf{Control Question} & \textbf{Response} \\
    \midrule
    Do you require MFA to access email? & \textcolor{red}{\ding{55}} \\
    Do you require MFA to log into computers? & \textcolor{green}{\ding{51}} \\
    Do you require MFA to access sensitive data systems? & \textcolor{red}{\ding{55}} \\
    Does your organization have an employee acceptable use policy? & \textcolor{red}{\ding{55}} \\
    Does your organization do security awareness training for new employees? & \textcolor{red}{\ding{55}} \\
    Does your organization do security awareness training for all employees at least once per year? & \textcolor{green}{\ding{51}} \\
    \bottomrule
\end{tabular}
\end{table}

\subsection*{Analysis of Control Gaps}
The lack of MFA for email and sensitive data systems is a critical weakness. Email is a primary target for account takeover attacks, which can lead to business email compromise (BEC), data exfiltration, and further network intrusion. The absence of an Acceptable Use Policy and security training for new hires means that employees are not equipped with the necessary guidance to protect company assets from day one, increasing the risk of insider threats and human error.

% --- Section 4: Technical Scan Results ---
\section{Technical Scan Results}
An external network scan was performed against the target IP address \texttt{[Target IP]}. The scan identified one open port with a critically vulnerable service.

\subsection{Port 21/tcp (FTP) - Open}
The File Transfer Protocol (FTP) port was found to be open and accessible from the public internet.
\begin{itemize}
    \item \textbf{Service:} FTP (File Transfer Protocol)
    \item \textbf{Product:} \seqsplit{\texttt{vsftpd 2.3.4}}
\end{itemize}

\subsubsection*{Finding 1: Critical Vulnerability - Outdated Service (CVE-2011-2523)}
The running version, \texttt{vsftpd 2.3.4}, is a dangerously outdated version released in 2011. This specific version contains a backdoor that allows an unauthenticated remote attacker to execute arbitrary commands with root privileges on the server. This is a critical risk that could lead to a full system compromise.

\subsubsection*{Finding 2: Critical Misconfiguration - Anonymous FTP Login}
The scan confirmed that anonymous FTP login is allowed. This configuration permits any user on the internet to connect to the server and potentially access, upload, or download files without any authentication. This could lead to sensitive data exposure, malware distribution, or be used as a pivot point for further attacks into the internal network.

\subsubsection*{Finding 3: High Risk - Insecure Protocol}
FTP is an inherently insecure protocol that transmits all data, including user credentials, in cleartext. Any attacker able to monitor network traffic could easily capture login information. Secure alternatives like SFTP (SSH File Transfer Protocol) or FTPS (FTP over SSL/TLS) should be used instead.

% --- Section 5: Consolidated Risk Assessment ---
\section{Consolidated Risk Assessment}
The following table synthesizes findings from the security control review, technical scan, and pre-existing risk data into a prioritized list.

\begin{table}[h!]
\centering
\caption{Summary of Identified Risks}
\begin{tabular}{@{}p{0.3\linewidth}p{0.5\linewidth}p{0.15\linewidth}@{}}
    \toprule
    \textbf{Risk Name} & \textbf{Description} & \textbf{Severity} \\
    \midrule
    \textbf{Vulnerable Public FTP Server} & An outdated version of vsftpd (2.3.4) is exposed, which contains a critical backdoor vulnerability and is misconfigured to allow anonymous login. & \textbf{CRITICAL} \\
    \textbf{Lack of Multi-Factor Authentication} & MFA is not enforced on email or sensitive data systems, exposing the organization to account takeover and data breach risks. & \textbf{HIGH} \\
    \textbf{Deficient Employee Policies \& Training} & The absence of an Acceptable Use Policy and security training for new hires creates a significant risk from insider threats and human error. & \textbf{HIGH} \\
    \textbf{Outdated Workstation Operating Systems} & Workstations are running Windows 7, an end-of-life OS that no longer receives security updates, leaving them vulnerable to exploitation. & \textbf{MEDIUM} \\
    \bottomrule
\end{tabular}
\end{table}

% --- Section 6: Recommendations ---
\section{Recommendations}
The following actions are recommended to address the identified risks. They are prioritized based on severity.

\subsection{Remediate Vulnerable FTP Server (CRITICAL)}
\begin{itemize}
    \item \textbf{Immediate Action:} Take the FTP service on \texttt{[Target IP]} offline immediately. This is the most effective way to eliminate the risk from the backdoor vulnerability and anonymous access.
    \item \textbf{Short-Term Action:} If FTP is a business necessity, replace \texttt{vsftpd 2.3.4} with a modern, secure file transfer solution like SFTP (provided by OpenSSH). Ensure any new solution is configured securely, with anonymous access disabled and access restricted to authorized IP addresses only.
    \item \textbf{Preventative Action:} Implement a vulnerability management program to regularly scan for and remediate outdated software on all external and internal systems.
\end{itemize}

\subsection{Implement Comprehensive MFA (HIGH)}
\begin{itemize}
    \item \textbf{Immediate Action:} Begin a project to enable and enforce MFA for all users on all email accounts (e.g., Office 365, Google Workspace).
    \item \textbf{Short-Term Action:} Extend the MFA requirement to all systems that store or process sensitive data, as well as any remote access solutions (e.g., VPNs).
\end{itemize}

\subsection{Establish Foundational Security Policies (HIGH)}
\begin{itemize}
    \item \textbf{Immediate Action:} Develop and implement a formal Acceptable Use Policy (AUP) that all employees must read and acknowledge. This policy should define the rules for using company IT assets.
    \item \textbf{Short-Term Action:} Integrate mandatory security awareness training into the onboarding process for all new employees. This ensures a baseline level of security knowledge from their first day.
\end{itemize}

\subsection{Modernize Operating Systems (MEDIUM)}
\begin{itemize}
    \item \textbf{Action:} Execute the plan to upgrade all remaining Windows 7 workstations to a supported operating system, such as Windows 10 or Windows 11. This will ensure they receive critical security patches from Microsoft.
\end{itemize}

\end{document}
```