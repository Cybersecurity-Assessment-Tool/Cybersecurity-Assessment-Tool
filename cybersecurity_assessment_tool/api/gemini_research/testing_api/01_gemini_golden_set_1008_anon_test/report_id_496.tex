```latex
\documentclass[12pt]{article}

% Preamble: Required Packages
\usepackage[a4paper, margin=1in]{geometry}
\usepackage{pifont} % For checkmarks and crosses
\usepackage{booktabs} % For professional tables
\usepackage{hyperref} % For clickable links and ToC
\usepackage{url} % For formatting URLs
\usepackage{seqsplit} % For splitting long text strings in tt font
\usepackage{graphicx} % For potential logos
\usepackage[utf8]{inputenc}

% Document Metadata
\title{Cybersecurity Posture Assessment Report}
\author{Cybersecurity Analysis Division}
\date{\today}

\hypersetup{
    colorlinks=true,
    linkcolor=blue,
    filecolor=magenta,      
    urlcolor=cyan,
    pdftitle={Cybersecurity Posture Assessment Report},
    pdfpagemode=FullScreen,
}

\begin{document}

\maketitle
\thispagestyle{empty}
\newpage

\tableofcontents
\thispagestyle{empty}
\newpage

\setcounter{page}{1}

% --- SECTION 1: EXECUTIVE SUMMARY ---
\section{Executive Summary}

This report provides a comprehensive analysis of the cybersecurity posture for \textbf{[Organization Name]}. The assessment was conducted by correlating data from an external network scan, a security controls questionnaire, and a review of pre-existing risks.

The external network scan of the target IP address (\texttt{[Target IP]}) revealed a strong perimeter security posture, with no open ports detected. This significantly reduces the external attack surface and is a commendable finding.

However, the security controls review identified several critical administrative and procedural gaps. The most severe findings are the absence of Multi-Factor Authentication (MFA) for accessing email and sensitive data systems. These gaps expose the organization to significant risks, including business email compromise, data breaches, and unauthorized access to critical assets. Additionally, the lack of a formal Employee Acceptable Use Policy represents a high-risk governance gap.

This report details these findings and provides actionable recommendations to mitigate the identified risks and strengthen the organization's overall security framework. Priority should be given to implementing MFA across all critical systems.

% --- SECTION 2: ORGANIZATIONAL INFORMATION ---
\section{Organizational Information}

This section outlines the basic information used as the basis for this assessment. Due to the anonymized nature of the input data, placeholders are used where necessary.

\begin{itemize}
    \item \textbf{Organization Name:} \textbf{[Organization Name]}
    \item \textbf{Primary Email Domain:} \texttt{[Domain]}
    \item \textbf{Client External IP:} \texttt{[Client IP]}
\end{itemize}

% --- SECTION 3: SECURITY CONTROL REVIEW ---
\section{Security Control Review}

The following table summarizes the organization's responses to a security controls questionnaire. The assessment column highlights areas that align with best practices and identifies significant gaps that require immediate attention.

\begin{table}[h!]
\centering
\caption{Security Controls Questionnaire Analysis}
\label{tab:controls}
\begin{tabular}{p{0.6\linewidth} c p{0.2\linewidth}}
\toprule
\textbf{Control Question} & \textbf{Response} & \textbf{Assessment} \\
\midrule
Do you require MFA to access email? & \ding{55} No & \textbf{Critical Gap} \\
Do you require MFA to log into computers? & \ding{51} Yes & Aligns with best practice \\
Do you require MFA to access sensitive data systems? & \ding{55} No & \textbf{Critical Gap} \\
Does your organization have an employee acceptable use policy? & \ding{55} No & \textbf{High Risk} \\
Does your organization do security awareness training for new employees? & \ding{51} Yes & Aligns with best practice \\
Does your organization do security awareness training for all employees at least once per year? & \ding{51} Yes & Aligns with best practice \\
\bottomrule
\end{tabular}
\end{table}

% --- SECTION 4: TECHNICAL SCAN RESULTS ---
\section{Technical Scan Results}

An external network vulnerability scan was performed to identify potential weaknesses in the organization's internet-facing infrastructure.

\begin{itemize}
    \item \textbf{Target IP Address:} \texttt{[Target IP]}
    \item \textbf{Scan Date:} [Scan Date Not Provided]
\end{itemize}

\subsection{Summary of Findings}
The scan results were positive, indicating a well-configured network perimeter. No open ports were discovered on the target system. All 1000 most common TCP ports were found to be in a \textbf{`closed`} state, meaning they are accessible but have no application listening on them. This configuration effectively denies external attackers common entry points into the network.

% --- SECTION 5: RISK ASSESSMENT ---
\section{Risk Assessment}

This section synthesizes findings from all data sources into a prioritized list of identified risks. The risks below are derived from the gaps identified in the Security Control Review, as no active technical vulnerabilities or pre-existing risks were found.

\begin{table}[h!]
\centering
\caption{Identified Risks}
\label{tab:risks}
\begin{tabular}{p{0.1\linewidth} p{0.3\linewidth} p{0.15\linewidth} p{0.35\linewidth}}
\toprule
\textbf{Risk ID} & \textbf{Risk Name} & \textbf{Severity} & \textbf{Overview} \\
\midrule
RISK-001 & Lack of MFA for Email Access & \textbf{Critical} & Without MFA, email accounts are vulnerable to takeover via credential stuffing or phishing. This can lead to Business Email Compromise (BEC), data exfiltration, and further internal network compromise. \\
\addlinespace
RISK-002 & Lack of MFA for Sensitive Data Systems & \textbf{Critical} & Access to systems holding sensitive corporate or customer data is protected only by a password. A single compromised credential could lead to a major data breach, resulting in severe financial and reputational damage. \\
\addlinespace
RISK-003 & Missing Employee Acceptable Use Policy (AUP) & \textbf{High} & The absence of a formal AUP creates ambiguity regarding the proper use of company assets. It weakens the organization's legal standing in cases of insider misuse and hinders the consistent enforcement of security standards. \\
\bottomrule
\end{tabular}
\end{table}

% --- SECTION 6: RECOMMENDATIONS ---
\section{Recommendations}

The following actions are recommended to mitigate the identified risks and improve the overall security posture of \textbf{[Organization Name]}. Recommendations are prioritized based on the severity of the associated risk.

\begin{itemize}
    \item \textbf{[Critical] Remediate RISK-001: Implement MFA for Email} \\
    Immediately deploy and enforce a mandatory Multi-Factor Authentication policy for all user access to the email system (\texttt{[Domain]}). This is the single most effective control to prevent unauthorized account access and Business Email Compromise.
    
    \item \textbf{[Critical] Remediate RISK-002: Implement MFA for Sensitive Systems} \\
    Conduct an inventory of all systems that process or store sensitive data. Prioritize these systems for the deployment of mandatory MFA. This control should be a prerequisite for any privileged or remote access.
    
    \item \textbf{[High] Remediate RISK-003: Develop and Implement an AUP} \\
    Develop a comprehensive Employee Acceptable Use Policy that clearly defines the rules for using company networks, devices, and data. This policy should be formally acknowledged by all new and existing employees and integrated into the security awareness training program.
    
    \item \textbf{[Informational] Maintain Strong Perimeter Security} \\
    Continue the current practice of maintaining a minimal external network footprint. Regularly scan external IP addresses to ensure no unnecessary ports are opened and that all configurations remain secure.
\end{itemize}

\end{document}
```