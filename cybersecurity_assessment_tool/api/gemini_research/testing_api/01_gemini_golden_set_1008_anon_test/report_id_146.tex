```latex
\documentclass[12pt]{article}

% --- PACKAGES ---
\usepackage[margin=1in]{geometry}
\usepackage{pifont} % For checkmarks and crosses
\usepackage{booktabs} % For professional tables
\usepackage{hyperref} % For clickable links
\usepackage{url} % For URL formatting
\usepackage{seqsplit} % For splitting long strings
\usepackage[table]{xcolor} % For table cell colors

% --- DOCUMENT METADATA ---
\title{Cybersecurity Posture Assessment Report}
\author{Cybersecurity Analyst}
\date{\today}

% --- HYPERREF SETUP ---
\hypersetup{
    colorlinks=true,
    linkcolor=black,
    urlcolor=blue,
    pdftitle={Cybersecurity Posture Assessment Report},
    pdfauthor={Cybersecurity Analyst},
    pdfsubject={Security Analysis},
    pdfkeywords={Security, Assessment, Network Scan, Risk}
}

% --- DOCUMENT START ---
\begin{document}

\maketitle
\thispagestyle{empty}
\newpage

\tableofcontents
\newpage

% --- SECTION 1: EXECUTIVE OVERVIEW ---
\section{Executive Overview}
This report provides a comprehensive cybersecurity assessment for \textbf{[Organization Name]}, based on an analysis of network scan data, organizational security controls, and pre-existing risk information.

The assessment reveals several critical and high-severity risks that require immediate attention. While the organization has implemented some positive security controls, such as mandatory Multi-Factor Authentication (MFA) for email and a security awareness training program, significant gaps exist.

Key findings include a publicly exposed and dangerously misconfigured FTP server running a vulnerable, outdated service. This poses an immediate threat of unauthorized access and data breach. Furthermore, the absence of MFA for computer and sensitive system access, coupled with the lack of a formal Acceptable Use Policy, significantly increases the risk of insider threats and unauthorized actions. The continued use of an unsupported operating system (Windows 7) further compounds the organization's risk profile.

Urgent remediation is recommended to address these vulnerabilities and strengthen the overall security posture.

% --- SECTION 2: ORGANIZATIONAL INFORMATION ---
\section{Organizational Information}
This section details the information provided by the client for the scope of this assessment.

\begin{tabular}{@{}ll}
\toprule
\textbf{Attribute} & \textbf{Value} \\
\midrule
Organization Name & \textbf{[Organization Name]} \\
Primary Domain & \texttt{[Domain]} \\
External IP Address Scanned & \texttt{[Client IP]} \\
\bottomrule
\end{tabular}

% --- SECTION 3: SECURITY CONTROL REVIEW ---
\section{Security Control Review}
The following table summarizes the organization's responses to a security controls questionnaire. Gaps identified by a "No" answer (\ding{55}) represent significant weaknesses in the current security framework.

\begin{table}[h!]
\centering
\caption{Security Controls Questionnaire Analysis}
\begin{tabular}{@{}p{0.7\linewidth}c}
\toprule
\textbf{Control Question} & \textbf{Status} \\
\midrule
Do you require MFA to access email? & \ding{51} \\
\rowcolor{red!15} Do you require MFA to log into computers? & \ding{55} \\
\rowcolor{red!15} Do you require MFA to access sensitive data systems? & \ding{55} \\
\rowcolor{orange!20} Does your organization have an employee acceptable use policy? & \ding{55} \\
Do you require security awareness training for new employees? & \ding{51} \\
Do you require security awareness training for all employees at least once per year? & \ding{51} \\
\bottomrule
\end{tabular}
\end{table}

\subsection*{Analysis of Gaps}
\begin{itemize}
    \item \textbf{Lack of MFA on Computers \& Sensitive Systems:} This is a critical vulnerability. Without MFA, compromised credentials (e.g., from a phishing attack) are sufficient for an attacker to gain access to internal systems and sensitive data, bypassing a primary defense layer.
    \item \textbf{No Acceptable Use Policy (AUP):} The absence of a formal AUP creates ambiguity regarding safe and acceptable use of company resources. It weakens the organization's ability to enforce security policies and hold employees accountable for risky behavior.
\end{itemize}

% --- SECTION 4: TECHNICAL SCAN RESULTS ---
\section{Technical Scan Results}
An external network scan was performed to identify exposed services and potential vulnerabilities.

\begin{itemize}
    \item \textbf{Target IP Address:} \texttt{[Target IP]}
    \item \textbf{Scan Date:} Scan data processed on \today
\end{itemize}

\begin{table}[h!]
\centering
\caption{Open Ports and Services Detected}
\begin{tabular}{@{}lllll@{}}
\toprule
\textbf{Port} & \textbf{State} & \textbf{Service} & \textbf{Version} & \textbf{Notes} \\
\midrule
\rowcolor{red!15} 21/tcp & Open & ftp & vsftpd 2.3.4 & Anonymous FTP login is allowed. \\
\bottomrule
\end{tabular}
\end{table}

\subsection*{Technical Findings Analysis}
The scan identified one open port with a critical misconfiguration and a severe vulnerability:
\begin{itemize}
    \item \textbf{Anonymous FTP Access:} The FTP server allows any user on the internet to log in without a password. This is a severe misconfiguration that can lead to unauthorized file uploads (e.g., malware) or downloads of sensitive information.
    \item \textbf{Vulnerable Service Version:} The identified service, \textbf{vsftpd 2.3.4}, is a dangerously outdated version. This specific version is known to contain a critical backdoor vulnerability (\textbf{CVE-2011-2523}), which allows an attacker to execute arbitrary commands on the server with root privileges. The combination of anonymous access and a known remote code execution (RCE) vulnerability presents an extreme and immediate risk to the organization.
\end{itemize}

% --- SECTION 5: CORRELATED RISK ASSESSMENT ---
\section{Correlated Risk Assessment}
This section synthesizes findings from all data sources into a prioritized list of risks.

\begin{table}[h!]
\centering
\caption{Summary of Identified Risks}
\begin{tabular}{@{}p{0.3\linewidth}p{0.5\linewidth}l@{}}
\toprule
\textbf{Risk Name} & \textbf{Description} & \textbf{Severity} \\
\midrule
\rowcolor{red!25} \textbf{Exposed Vulnerable FTP Server} & An outdated FTP server (vsftpd 2.3.4) with a known RCE backdoor is publicly exposed and allows anonymous login. & \textbf{Critical} \\
\rowcolor{red!15} \textbf{No MFA on Endpoints/Systems} & Lack of MFA on computers and sensitive data systems allows for account takeover via stolen credentials. & \textbf{Critical} \\
\rowcolor{orange!20} \textbf{Lack of Acceptable Use Policy} & No formal policy defining proper use of IT assets, increasing risk of misuse and insider threat. & \textbf{High} \\
\rowcolor{yellow!25} \textbf{Outdated Windows Policy} & Workstations are running Windows 7, an unsupported OS that no longer receives security updates. & \textbf{Medium} \\
\bottomrule
\end{tabular}
\end{table}

% --- SECTION 6: RECOMMENDATIONS ---
\section{Recommendations}
The following actionable steps are recommended to mitigate the identified risks. Recommendations are prioritized based on severity.

\subsection*{Immediate Actions (To be completed within 72 hours)}
\begin{enumerate}
    \item \textbf{Remediate Exposed FTP Server:}
    \begin{itemize}
        \item \textbf{Option A (Preferred):} Decommission the FTP server immediately. If FTP is not essential for business operations, removing the service entirely is the most secure option.
        \item \textbf{Option B (If Service is Required):} If the service must remain, take the following steps:
        \begin{enumerate}
            \item Immediately disable anonymous FTP access.
            \item Place the server behind a firewall and restrict access to only known, trusted IP addresses.
            \item Upgrade the vsftpd service to the latest stable version to patch the RCE vulnerability.
            \item Consider replacing FTP with a more secure file transfer protocol like SFTP (SSH File Transfer Protocol).
        \end{enumerate}
    \end{itemize}
\end{enumerate}

\subsection*{High-Priority Actions (To be completed within 30 days)}
\begin{enumerate}
    \setcounter{enumi}{1} % Continue numbering from previous list
    \item \textbf{Implement Multi-Factor Authentication (MFA):}
    \begin{itemize}
        \item Deploy a robust MFA solution for all user logins to workstations, laptops, servers, and any systems containing sensitive data.
        \item Enforce this policy for all employees and contractors, including privileged accounts.
    \end{itemize}
    \item \textbf{Develop and Implement an Acceptable Use Policy (AUP):}
    \begin{itemize}
        \item Draft a clear and comprehensive AUP that outlines the rules for using company technology and data.
        \item Require all employees to read and formally acknowledge the policy.
        \item Integrate the AUP into the new-hire onboarding process.
    \end{itemize}
\end{enumerate}

\subsection*{Medium-Priority Actions (To be completed within 90 days)}
\begin{enumerate}
    \setcounter{enumi}{3} % Continue numbering
    \item \textbf{Upgrade Outdated Operating Systems:}
    \begin{itemize}
        \item Accelerate the existing plan to upgrade all workstations from Windows 7 to a modern, supported operating system (e.g., Windows 10/11).
        \item Isolate any remaining Windows 7 machines from the main network until they can be upgraded or decommissioned.
    \end{itemize}
\end{enumerate}

\end{document}
```