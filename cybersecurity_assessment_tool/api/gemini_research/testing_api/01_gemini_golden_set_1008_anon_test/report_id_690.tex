```latex
\documentclass[12pt]{article}

% --- PACKAGES ---
\usepackage[margin=1in]{geometry}
\usepackage{pifont} % For checkmarks and crosses
\usepackage{booktabs} % For professional tables
\usepackage{hyperref} % For clickable links
\usepackage{url} % For URL formatting
\usepackage{seqsplit} % For splitting long strings
\usepackage{xcolor} % For colors

% --- DOCUMENT INFORMATION ---
\title{Cybersecurity Posture Assessment Report \\ for \\ \textbf{[Organization Name]}}
\author{Cybersecurity Analysis Division}
\date{\today}

% --- HYPERREF SETUP ---
\hypersetup{
    colorlinks=true,
    linkcolor=blue,
    filecolor=magenta,      
    urlcolor=cyan,
    pdftitle={Cybersecurity Posture Assessment Report},
    pdfpagemode=FullScreen,
}

\begin{document}

\maketitle
\thispagestyle{empty}
\newpage

\tableofcontents
\newpage

% --- EXECUTIVE SUMMARY ---
\section{Executive Summary}

This report provides a comprehensive analysis of the cybersecurity posture for \textbf{[Organization Name]}, based on data collected from network scans, a security controls questionnaire, and a review of pre-existing risk documentation.

The assessment reveals a mixed security posture. While the organization has implemented some critical controls, such as Multi-Factor Authentication (MFA) for computer and sensitive system access, there are significant and high-risk gaps in foundational security policies and user-facing controls.

\textbf{Key Findings:}
\begin{itemize}
    \item \textbf{Critical Risk - Lack of Email MFA:} The absence of mandatory MFA for email access represents a critical vulnerability. Email accounts are a primary target for attackers, and a compromise could lead to Business Email Compromise (BEC), data breaches, and further network intrusion.
    \item \textbf{High Risk - Policy Gaps:} The organization lacks a formal employee Acceptable Use Policy (AUP) and does not conduct mandatory annual security awareness training for all employees. These gaps increase the risk of insider threats, accidental data loss, and successful social engineering attacks.
    \item \textbf{Positive Technical Finding:} A network scan of the external IP address \texttt{[Client IP]} revealed no open, vulnerable services. Specifically, Port 80 (HTTP), which was listed as an existing risk, was found to be \textbf{closed}. This suggests that a previously identified vulnerability may have been successfully remediated.
\end{itemize}

Immediate action is required to address the identified critical and high-risk gaps. Recommendations are provided in Section \ref{sec:recommendations} to guide remediation efforts and strengthen the organization's overall defensive capabilities.

\newpage

% --- ORGANIZATIONAL INFORMATION ---
\section{Organizational Information}

This section details the organizational data used as the basis for this assessment. Due to the anonymized nature of the input data, placeholders have been used where necessary.

\begin{tabular}{@{}ll}
    \toprule
    \textbf{Attribute} & \textbf{Value} \\
    \midrule
    Organization Name & \textbf{[Organization Name]} \\
    Primary Email Domain & \texttt{[Domain]} \\
    External IP Address Scanned & \texttt{[Client IP]} \\
    Target IP Address Scanned & \texttt{[Target IP]} \\
    \bottomrule
\end{tabular}

% --- SECURITY CONTROL REVIEW ---
\section{Security Control Review}

The following table summarizes the organization's responses to a security controls questionnaire. A green checkmark (\ding{51}) indicates a positive control is in place, while a red cross (\ding{55}) indicates a gap that introduces risk.

\begin{table}[h!]
\centering
\begin{tabular}{@{}lc@{}}
    \toprule
    \textbf{Security Control Question} & \textbf{Status} \\
    \midrule
    Do you require MFA to access email? & \textcolor{red}{\ding{55}} \\
    Do you require MFA to log into computers? & \textcolor{green}{\ding{51}} \\
    Do you require MFA to access sensitive data systems? & \textcolor{green}{\ding{51}} \\
    Does your organization have an employee acceptable use policy? & \textcolor{red}{\ding{55}} \\
    Does your organization do security awareness training for new employees? & \textcolor{green}{\ding{51}} \\
    Does your organization do security awareness training for all employees at least once per year? & \textcolor{red}{\ding{55}} \\
    \bottomrule
\end{tabular}
\caption{Security Controls Questionnaire Results}
\end{table}

\subsection{Analysis of Control Gaps}
The questionnaire reveals three significant control gaps:
\begin{enumerate}
    \item \textbf{No MFA for Email:} This is the most critical finding. Email is the gateway to an organization's data and a primary vector for phishing and account takeover attacks.
    \item \textbf{No Acceptable Use Policy (AUP):} Without a formal AUP, there are no clear, enforceable rules for how employees should use company assets. This can lead to risky behavior and complicates disciplinary action in the event of a policy violation.
    \item \textbf{No Annual Security Awareness Training:} While training new hires is a good first step, the threat landscape evolves constantly. Without annual reinforcement, employee knowledge becomes outdated, making them more susceptible to modern attack techniques.
\end{enumerate}

% --- TECHNICAL SCAN RESULTS ---
\section{Technical Scan Results}

An external network scan was performed on the target IP address \texttt{[Target IP]}. The scan was conducted to identify open ports and exposed services that could be exploited by an external attacker.

\subsection{Scan Summary}
\begin{itemize}
    \item \textbf{Scanner Used:} Nmap
    \item \textbf{Target IP:} \texttt{[Target IP]}
    \item \textbf{Host Status:} Up
\end{itemize}

\subsection{Port Scan Details}
The scan results indicate a secure external posture with no open ports detected. The status of a key port is detailed below.

\begin{table}[h!]
\centering
\begin{tabular}{@{}llll@{}}
    \toprule
    \textbf{Port} & \textbf{Protocol} & \textbf{State} & \textbf{Analysis} \\
    \midrule
    80 & TCP & \textbf{Closed} & Port 80 (HTTP) was found to be closed. \\
    & & & This is a positive security finding, as it prevents \\
    & & & unencrypted web traffic. This result contradicts \\
    & & & a pre-existing risk, suggesting remediation has occurred. \\
    \bottomrule
\end{tabular}
\caption{Network Port Scan Results}
\end{table}

% --- RISK ASSESSMENT ---
\section{Correlated Risk Assessment}

This section synthesizes findings from the security questionnaire, technical scans, and pre-existing risk data into a prioritized list of current risks.

\begin{table}[h!]
\centering
\begin{tabular}{@{}p{0.3\linewidth}p{0.5\linewidth}l@{}}
    \toprule
    \textbf{Risk Name} & \textbf{Overview} & \textbf{Severity} \\
    \midrule
    \textbf{Lack of MFA on Email} & The absence of MFA on email accounts makes them highly vulnerable to phishing and credential stuffing attacks, which can lead to Business Email Compromise (BEC) and data exfiltration. & \textbf{Critical} \\
    \addlinespace
    \textbf{No Acceptable Use Policy} & Without a formal AUP, employees lack clear guidance on the secure use of company assets, increasing the likelihood of unintentional security incidents and insider threats. & High \\
    \addlinespace
    \textbf{Inadequate Annual Security Training} & Failing to provide annual security training for all employees allows security knowledge to decay, making the organization more susceptible to social engineering and phishing attacks. & High \\
    \addlinespace
    \textbf{Unencrypted Web Server} & \textit{(From pre-existing risk list)} An open Port 80 allows for unencrypted HTTP traffic, posing a risk of data interception. \textbf{Note:} The current scan found this port to be closed, indicating this risk may be remediated. & Medium \\
    \bottomrule
\end{tabular}
\caption{Summary of Identified Risks}
\end{table}

% --- RECOMMENDATIONS ---
\section{Recommendations}
\label{sec:recommendations}

The following actionable recommendations are provided to mitigate the identified risks and improve the overall security posture of \textbf{[Organization Name]}.

\subsection{Immediate Priority (0-30 Days)}
\begin{enumerate}
    \item \textbf{Enforce MFA for all Email Accounts:}
    \begin{itemize}
        \item \textbf{Action:} Immediately enable and enforce MFA for all user mailboxes.
        \item \textbf{Justification:} This is the single most effective control to prevent email account takeovers and mitigate the risk of BEC. It directly addresses the most critical vulnerability identified in this assessment.
    \end{itemize}
\end{enumerate}

\subsection{High Priority (30-90 Days)}
\begin{enumerate}
    \item \textbf{Develop and Implement an Acceptable Use Policy (AUP):}
    \begin{itemize}
        \item \textbf{Action:} Draft a formal AUP that clearly defines the rules for using company networks, devices, and data. Ensure all employees read and acknowledge the policy.
        \item \textbf{Justification:} An AUP establishes a baseline for secure behavior, reduces legal liability, and provides a framework for enforcing security standards.
    \end{itemize}
    \item \textbf{Establish a Mandatory Annual Security Awareness Training Program:}
    \begin{itemize}
        \item \textbf{Action:} Procure or develop a security awareness training module and schedule mandatory annual sessions for all employees. Track completion to ensure compliance.
        \item \textbf{Justification:} Regular training reinforces security concepts, updates staff on new threats, and strengthens the human element of your defense.
    \end{itemize}
\end{enumerate}

\subsection{Medium Priority (Ongoing)}
\begin{enumerate}
    \item \textbf{Validate and Close Remediated Risks:}
    \begin{itemize}
        \item \textbf{Action:} Formally confirm that the "Unencrypted Web Server" risk (related to Port 80) has been intentionally remediated. Update risk registers and documentation to reflect the current state.
        \item \textbf{Justification:} Maintaining an accurate risk register is crucial for focusing resources on current, relevant threats.
    \end{itemize}
\end{enumerate}

\end{document}
```