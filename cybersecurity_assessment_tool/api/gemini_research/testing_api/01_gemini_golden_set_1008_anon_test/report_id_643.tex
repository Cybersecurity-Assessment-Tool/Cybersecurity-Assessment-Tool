```latex
\documentclass[12pt]{article}

% --- PACKAGES ---
\usepackage[margin=1in]{geometry}
\usepackage{pifont} % For check and cross marks
\usepackage{booktabs} % For professional tables
\usepackage{hyperref} % For hyperlinks
\usepackage{url}      % For URL formatting
\usepackage{seqsplit} % To split long strings in texttt
\usepackage[T1]{fontenc}

% --- DOCUMENT METADATA ---
\title{Cybersecurity Posture Assessment Report}
\author{Cybersecurity Analysis Division}
\date{\today}

% --- HYPERREF SETUP ---
\hypersetup{
    colorlinks=true,
    linkcolor=black,
    urlcolor=blue,
    pdftitle={Cybersecurity Posture Assessment Report},
    pdfauthor={Cybersecurity Analysis Division},
    pdfsubject={Security Analysis},
    pdfkeywords={Cybersecurity, Risk, Assessment, Scan}
}

% --- BEGIN DOCUMENT ---
\begin{document}

\maketitle
\hrule
\vspace{1em}

% ===================================================================
% SECTION 1: EXECUTIVE OVERVIEW
% ===================================================================
\section*{Executive Overview}

This report provides a cybersecurity posture assessment for \textbf{[Organization Name]}, based on a combination of self-reported organizational controls, an external network scan, and a review of pre-existing risks.

The analysis indicates a strong foundation in identity and access management, with commendable implementation of Multi-Factor Authentication (MFA) across key systems. However, two primary areas of concern have been identified that require immediate attention.

First, a critical gap exists in the organization's security awareness program, specifically the lack of mandatory annual training for all employees. This oversight significantly increases the risk of human-factor security incidents, such as phishing and social engineering attacks.

Second, the external network scan revealed an exposed Secure Shell (SSH) service (port 22) on the network perimeter. While potentially necessary for remote administration, an internet-facing SSH port is a high-value target for attackers and must be rigorously secured and justified.

No pre-existing vulnerabilities were reported. This report details these findings and provides actionable recommendations to mitigate the identified risks and enhance the overall security posture.

\vspace{2em}

% ===================================================================
% SECTION 2: ORGANIZATIONAL INFORMATION
% ===================================================================
\section{Organizational Information}

The following details were used as the basis for this assessment. Due to the anonymized nature of the input data, placeholders have been used where necessary.

\begin{tabular}{@{}ll}
    \toprule
    \textbf{Attribute} & \textbf{Value} \\
    \midrule
    Organization Name & \textbf{[Organization Name]} \\
    Primary Domain & \texttt{[Domain]} \\
    External IP Address & \texttt{[Client IP]} \\
    \bottomrule
\end{tabular}

\vspace{2em}

% ===================================================================
% SECTION 3: SECURITY CONTROL REVIEW (QUESTIONNAIRE)
% ===================================================================
\section{Security Control Review}

The following table summarizes the organization's self-reported security controls. A checkmark (\ding{51}) indicates a positive control is in place, while a cross (\ding{55}) indicates a potential security gap.

\begin{table}[h!]
\centering
\begin{tabular}{@{}p{0.8\linewidth}c}
    \toprule
    \textbf{Control Question} & \textbf{Status} \\
    \midrule
    Do you require MFA to access email? & \ding{51} \\
    Do you require MFA to log into computers? & \ding{51} \\
    Do you require MFA to access sensitive data systems? & \ding{51} \\
    Does your organization have an employee acceptable use policy? & \ding{51} \\
    Does your organization do security awareness training for new employees? & \ding{51} \\
    \textbf{Does your organization do security awareness training for all employees at least once per year?} & \textbf{\ding{55}} \\
    \bottomrule
\end{tabular}
\caption{Organizational Security Control Questionnaire Results}
\end{table}

\paragraph{Analysis:} The organization has implemented strong access controls with widespread MFA adoption. The primary weakness identified is the lack of ongoing, annual security awareness training for all staff. This is a critical gap, as the threat landscape evolves continuously, and employee knowledge must be refreshed to defend against modern phishing and social engineering tactics.

\vspace{2em}

% ===================================================================
% SECTION 4: TECHNICAL SCAN RESULTS
% ===================================================================
\section{Technical Scan Results}

An external network scan was performed to identify exposed services on the organization's perimeter.

\begin{itemize}
    \item \textbf{Target IP Address:} \texttt{[Target IP]}
    \item \textbf{Scan Date:} Not specified in scan data.
    \item \textbf{Status:} Host is online and responsive.
\end{itemize}

\begin{table}[h!]
\centering
\begin{tabular}{@{}llll@{}}
    \toprule
    \textbf{Port} & \textbf{State} & \textbf{Service} & \textbf{Analysis / Notes} \\
    \midrule
    22/tcp & open & ssh & The Secure Shell service is exposed to the internet. \\
           &        &     & No version information was available from the scan. \\
           &        &     & This port is a common target for brute-force and \\
           &        &     & credential stuffing attacks. \\
    \bottomrule
\end{tabular}
\caption{Open Ports Detected on \texttt{[Target IP]}}
\end{table}

\paragraph{Analysis:} The scan identified a single open port: 22/TCP, which is universally used for SSH. Exposing administrative services like SSH directly to the internet is a significant risk. Without proper controls (e.g., IP whitelisting, key-based authentication, fail2ban), this service can be subjected to automated attacks attempting to guess credentials. The lack of service version information prevents a direct check for known vulnerabilities, but the exposure itself constitutes a notable risk.

\vspace{2em}

% ===================================================================
% SECTION 5: RISK ASSESSMENT SUMMARY
% ===================================================================
\section{Risk Assessment}

The following table correlates findings from the security control review and the technical scan into a prioritized list of risks.

\begin{table}[h!]
\centering
\begin{tabular}{@{}p{0.25\linewidth}p{0.55\linewidth}l@{}}
    \toprule
    \textbf{Risk Name} & \textbf{Overview} & \textbf{Severity} \\
    \midrule
    \textbf{Lack of Annual Security Awareness Training} & Employees are not receiving regular, updated training on current cyber threats. This increases susceptibility to phishing, malware, and social engineering, making the human element the weakest link in the security chain. & \textbf{High} \\
    \addlinespace
    \textbf{Exposed SSH Service (Port 22)} & The SSH management port is open to the entire internet, creating a direct vector for unauthorized access attempts. Attackers can use automated tools to brute-force passwords or exploit potential vulnerabilities in the SSH server software. & \textbf{High} \\
    \bottomrule
\end{tabular}
\caption{Summary of Identified Risks}
\end{table}

\vspace{2em}

% ===================================================================
% SECTION 6: RECOMMENDATIONS
% ===================================================================
\section{Recommendations}

The following actionable recommendations are provided to address the identified risks.

\subsection*{High Priority}

\begin{enumerate}
    \item \textbf{Implement a Mandatory Annual Security Awareness Training Program:}
    \begin{itemize}
        \item \textbf{Action:} Procure and deploy a security awareness training platform or develop an in-house program. The training must be mandatory for all employees, including management and executives.
        \item \textbf{Content:} Training should cover modern threats such as phishing, ransomware, business email compromise (BEC), and secure data handling.
        \item \textbf{Validation:} Track completion for all employees and consider periodic phishing simulations to test effectiveness.
        \item \textbf{Justification:} This directly mitigates the risk of human error, which is a primary vector in most security breaches.
    \end{itemize}
    \vspace{1em}
    \item \textbf{Secure or Remediate Exposed SSH Service:}
    \begin{itemize}
        \item \textbf{Action 1 (Investigate):} Immediately determine the business justification for this externally accessible SSH port.
        \item \textbf{Action 2 (Remediate if Unnecessary):} If there is no valid business need, block port 22 at the network firewall immediately.
        \item \textbf{Action 3 (Secure if Necessary):} If external access is required, implement the following hardening measures:
        \begin{itemize}
            \item \textbf{Firewall ACLs:} Restrict access to a whitelist of trusted source IP addresses. Deny all other connections.
            \item \textbf{Disable Password Authentication:} Configure the SSH server to only allow public key authentication, which is significantly more resistant to brute-force attacks.
            \item \textbf{Use a Bastion Host:} If multiple administrators require access, route all connections through a hardened bastion host or VPN instead of exposing the service directly.
            \item \textbf{Monitor Logs:} Ensure all SSH login attempts (successful and failed) are logged and monitored for suspicious activity.
        \end{itemize}
        \item \textbf{Justification:} This significantly reduces the external attack surface and protects a critical administrative entry point from compromise.
    \end{itemize}
\end{enumerate}

\end{document}
```