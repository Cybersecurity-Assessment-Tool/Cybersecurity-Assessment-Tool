```latex
\documentclass[12pt]{article}

% Preamble: Required Packages
\usepackage[margin=1in]{geometry}
\usepackage{pifont} % For checkmarks and crosses
\usepackage{booktabs} % For professional tables
\usepackage{hyperref} % For clickable links
\usepackage{url} % For URL formatting
\usepackage{seqsplit} % For splitting long strings
\usepackage{graphicx} % For logo (optional)
\usepackage{xcolor} % For colors

% Document Information
\title{Cybersecurity Posture Assessment Report}
\author{Cybersecurity Analyst}
\date{November 22, 2025}

% Hyperref Setup
\hypersetup{
    colorlinks=true,
    linkcolor=blue,
    filecolor=magenta,      
    urlcolor=cyan,
    pdftitle={Cybersecurity Posture Assessment Report},
    pdfpagemode=FullScreen,
}

\begin{document}

\maketitle
\thispagestyle{empty}
\newpage

\tableofcontents
\newpage

% --- 1. Executive Summary ---
\section{Executive Summary}

This report details the findings of a cybersecurity posture assessment conducted for \textbf{[Organization Name]} on November 22, 2025. The assessment combined a review of organizational security controls, an external network scan, and an analysis of pre-existing risks.

The analysis revealed several critical and high-risk gaps in the organization's security posture. Most notably, there is a lack of multi-factor authentication (MFA) for sensitive data systems, a complete absence of an employee security awareness training program, and no formal acceptable use policy. These policy and procedural deficiencies create a significant risk of security incidents originating from human error or insider threats.

From a technical perspective, the external scan identified a web server running an outdated version of Nginx (\texttt{1.18.0}). Outdated software is a common vector for attacks, as it often contains publicly known vulnerabilities. When combined with the identified policy gaps, this technical finding elevates the organization's overall risk profile.

Urgent remediation is recommended, focusing on the implementation of MFA, the development of a security awareness program, and the patching of vulnerable software to mitigate these risks effectively.

% --- 2. Organizational Information ---
\section{Organizational Information}

The following details were used as the basis for this assessment. Due to the anonymized nature of the provided data, placeholders have been used where necessary.

\begin{itemize}
    \item \textbf{Organization Name:} \textbf{[Organization Name]}
    \item \textbf{Primary Domain:} \texttt{[Domain]}
    \item \textbf{Assessed External IP:} \texttt{[Client IP]}
\end{itemize}

% --- 3. Security Control Review ---
\section{Security Control Review}

A security questionnaire was completed to evaluate the organization's current policies and controls. The responses indicate significant gaps in foundational security practices. A "No" response (\ding{55}) highlights a control that is not in place and represents a potential risk.

\begin{table}[h!]
\centering
\caption{Organizational Security Controls Questionnaire}
\begin{tabular}{p{0.8\textwidth} c}
\toprule
\textbf{Security Control Question} & \textbf{Response} \\
\midrule
Do you require MFA to access email? & \ding{51} \\
Do you require MFA to log into computers? & \ding{51} \\
\textbf{Do you require MFA to access sensitive data systems?} & \textcolor{red}{\ding{55}} \\
\textbf{Does your organization have an employee acceptable use policy?} & \textcolor{red}{\ding{55}} \\
\textbf{Does your organization do security awareness training for new employees?} & \textcolor{red}{\ding{55}} \\
\textbf{Does your organization do security awareness training for all employees at least once per year?} & \textcolor{red}{\ding{55}} \\
\bottomrule
\end{tabular}
\end{table}

% --- 4. Technical Scan Results ---
\section{Technical Scan Results}

An external network scan was performed against the target IP address to identify open ports and exposed services.

\begin{itemize}
    \item \textbf{Target IP:} \texttt{[Target IP]}
    \item \textbf{Scan Date:} November 22, 2025
\end{itemize}

The scan revealed one open port, which is detailed in the table below.

\begin{table}[h!]
\centering
\caption{Open Ports and Services}
\begin{tabular}{l l l l}
\toprule
\textbf{Port} & \textbf{State} & \textbf{Service} & \textbf{Product / Version} \\
\midrule
443/TCP & Open & HTTPS & Nginx / 1.18.0 \\
\bottomrule
\end{tabular}
\end{table}

\subsection{Analysis of Technical Findings}
The scan identified an Nginx web server, version \textbf{1.18.0}. This version was released in April 2020 and is now considered outdated. Current stable versions are significantly newer. Running outdated software introduces unnecessary risk, as numerous vulnerabilities have been discovered and patched in subsequent releases. This server should be considered vulnerable to known exploits.

% --- 5. Risk Assessment Summary ---
\section{Risk Assessment Summary}

The following table synthesizes findings from the security control review and the technical scan. No pre-existing vulnerabilities were reported. The identified risks are prioritized by severity.

\begin{table}[h!]
\centering
\caption{Identified Risks and Severity}
\begin{tabular}{p{0.1\textwidth} p{0.25\textwidth} p{0.45\textwidth} p{0.1\textwidth}}
\toprule
\textbf{Risk ID} & \textbf{Risk Name} & \textbf{Description} & \textbf{Severity} \\
\midrule
RISK-001 & Inadequate MFA on Sensitive Systems & The absence of MFA on systems containing sensitive data allows a single compromised password to lead to a major data breach. & \textbf{Critical} \\
\addlinespace
RISK-002 & Lack of Security Awareness Program & Without training, employees are highly susceptible to phishing, social engineering, and other common attacks, making them the weakest link in the security chain. & High \\
\addlinespace
RISK-003 & Missing Acceptable Use Policy (AUP) & The lack of a formal AUP means there are no clear rules for employees regarding the use of company assets, increasing the risk of misuse and insider threats. & High \\
\addlinespace
RISK-004 & Outdated Web Server Software & The Nginx server is running an old version with known vulnerabilities, making it a prime target for automated attacks and unauthorized access. & High \\
\bottomrule
\end{tabular}
\end{table}

% --- 6. Recommendations ---
\section{Recommendations}

To address the identified risks and improve the overall security posture of \textbf{[Organization Name]}, the following actions are recommended with high priority.

\begin{description}
    \item[For RISK-001:] \textbf{Implement Mandatory MFA.} Immediately deploy and enforce a robust MFA solution across all sensitive data systems, including databases, financial applications, and administrative portals.
    
    \item[For RISK-002:] \textbf{Establish a Security Awareness Program.} Develop and implement a mandatory security awareness training program for all employees. The program should be conducted upon hiring and repeated annually. It should cover topics such as phishing identification, password hygiene, and incident reporting.
    
    \item[For RISK-003:] \textbf{Develop and Enforce an AUP.} Draft a comprehensive Acceptable Use Policy that clearly defines the rules for using company networks, devices, and data. This policy should be communicated to all employees and formally acknowledged.
    
    \item[For RISK-004:] \textbf{Patch Outdated Software.} Plan and execute an upgrade of the Nginx web server on host \texttt{[Target IP]} to the latest stable version. Implement a continuous vulnerability management program to ensure all public-facing software is kept up-to-date.
\end{description}

% --- 7. Conclusion ---
\section{Conclusion}

The assessment for \textbf{[Organization Name]} has identified significant, yet remediable, weaknesses in both policy and technical security controls. The combination of a vulnerable external service and a lack of internal security culture presents a high degree of risk. By implementing the recommendations outlined in this report, the organization can substantially reduce its attack surface and build a more resilient security posture.

\end{document}
```