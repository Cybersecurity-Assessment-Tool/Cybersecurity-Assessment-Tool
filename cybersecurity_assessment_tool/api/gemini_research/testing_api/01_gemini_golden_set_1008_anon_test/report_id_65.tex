```latex
\documentclass[12pt]{article}

% ----------------------------------------------------------------------
% PREAMBLE
% ----------------------------------------------------------------------
\usepackage[margin=1in]{geometry}
\usepackage{pifont} % For checkmarks and crosses (dingbats)
\usepackage{booktabs} % For professional-looking tables
\usepackage{hyperref} % For clickable links and references
\usepackage{url} % For formatting URLs
\usepackage{seqsplit} % For splitting long strings in tt font
\usepackage{graphicx} % For potential logos
\usepackage{xcolor} % For colors

% Define colors for severity
\definecolor{criticalred}{HTML}{D12727}
\definecolor{highorange}{HTML}{E96A00}
\definecolor{mediumyellow}{HTML}{F2AC00}
\definecolor{lowblue}{HTML}{0073E6}

% Hyperref setup
\hypersetup{
    colorlinks=true,
    linkcolor=blue,
    filecolor=magenta,      
    urlcolor=cyan,
    pdftitle={Cybersecurity Assessment Report},
    pdfpagemode=FullScreen,
}

% Checkmark and Cross definitions
\newcommand{\cmark}{\ding{51}}
\newcommand{\xmark}{\ding{55}}

% ----------------------------------------------------------------------
% DOCUMENT START
% ----------------------------------------------------------------------
\begin{document}

% ----------------------------------------------------------------------
% TITLE PAGE
% ----------------------------------------------------------------------
\begin{titlepage}
    \centering
    \vspace*{1cm}
    
    \Huge
    \textbf{Cybersecurity Assessment Report}
    
    \vspace{1.5cm}
    
    \Large
    Prepared for:
    
    \vspace{0.5cm}
    
    \Huge
    \textbf{[Organization Name]}
    
    \vspace{2cm}
    
    \Large
    \textbf{Report Date:} \today
    
    \vfill
    
    \large
    \textbf{CONFIDENTIAL}
    \vspace{0.5cm}
    
    \normalsize
    This document contains sensitive information. Access, distribution, and use are restricted to authorized personnel only.
    
\end{titlepage}

\tableofcontents
\newpage

% ----------------------------------------------------------------------
% 1. EXECUTIVE SUMMARY
% ----------------------------------------------------------------------
\section{Executive Summary}

This report details the findings of a cybersecurity assessment conducted for \textbf{[Organization Name]}. The assessment combined an external network scan, a review of existing risk documentation, and an analysis of organizational security controls based on a questionnaire.

The assessment identified a \textbf{Critical} risk: a publicly exposed Remote Desktop Protocol (RDP) service on the organization's external network. This vulnerability is actively targeted by threat actors for initial access, often leading to ransomware deployment and data breaches. This technical finding directly validates a pre-existing documented risk, confirming it remains an active and unmitigated threat.

Furthermore, a \textbf{High} risk was identified in the organization's security processes. While annual security training is in place, there is a critical gap in providing mandatory security awareness training for new employees during their onboarding. This gap increases the organization's susceptibility to social engineering attacks, as new hires may be unaware of internal security policies and common threats.

Immediate remediation of the exposed RDP service is strongly recommended to prevent a potential compromise. Concurrently, implementing a security training module into the employee onboarding process is crucial for long-term risk reduction.

% ----------------------------------------------------------------------
% 2. ORGANIZATIONAL INFORMATION
% ----------------------------------------------------------------------
\section{Organizational Information}

This assessment pertains to the following entity and its associated assets. The information below has been anonymized as per reporting protocols.

\begin{itemize}
    \item \textbf{Organization Name:} \textbf{[Organization Name]}
    \item \textbf{Primary Domain:} \texttt{[Domain]}
    \item \textbf{Assessed External IP:} \texttt{[Client IP]}
\end{itemize}

% ----------------------------------------------------------------------
% 3. SECURITY CONTROL REVIEW
% ----------------------------------------------------------------------
\section{Security Control Review}

The following table summarizes the organization's current security posture based on the provided questionnaire. "No" answers indicate potential gaps in security controls that may increase risk.

\begin{table}[h!]
\centering
\caption{Security Controls Questionnaire Analysis}
\label{tab:controls}
\begin{tabular}{p{0.6\linewidth} c p{0.25\linewidth}}
\toprule
\textbf{Control Question} & \textbf{Status} & \textbf{Analyst Note} \\
\midrule
Do you require MFA to access email? & \cmark & Good practice. \\
Do you require MFA to log into computers? & \cmark & Good practice. \\
Do you require MFA to access sensitive data systems? & \cmark & Excellent control. \\
Does your organization have an employee acceptable use policy? & \cmark & Foundational policy is in place. \\
\textbf{Does your organization do security awareness training for new employees?} & \textbf{\xmark} & \textbf{High Risk Gap.} New hires are a common target for attackers. \\
Does your organization do security awareness training for all employees at least once per year? & \cmark & Good, but does not cover the initial onboarding risk. \\
\bottomrule
\end{tabular}
\end{table}

The primary finding from this review is the lack of a mandatory security training program for new hires. This oversight leaves the organization vulnerable, as new employees are not immediately equipped with the knowledge to identify and report security threats like phishing.

% ----------------------------------------------------------------------
% 4. TECHNICAL SCAN RESULTS
% ----------------------------------------------------------------------
\section{Technical Scan Results}

An external network scan was performed to identify open ports and exposed services on the organization's perimeter.

\begin{itemize}
    \item \textbf{Target IP Address:} \texttt{[Target IP]}
    \item \textbf{Scan Date:} [Scan Date]
\end{itemize}

\subsection{Open Ports and Services}
The scan revealed the following open port, which is accessible from the public internet.

\begin{table}[h!]
\centering
\caption{Exposed Services on \texttt{[Target IP]}}
\label{tab:ports}
\begin{tabular}{c c l l}
\toprule
\textbf{Port} & \textbf{Protocol} & \textbf{Service Name} & \textbf{Finding} \\
\midrule
3389 & TCP & \texttt{ms-wbt-server} & \textbf{Critical.} Remote Desktop Protocol (RDP) is exposed. \\
\bottomrule
\end{tabular}
\end{table}

\textbf{Analysis:} The exposure of RDP (port 3389) is a significant and immediate threat. This service is a primary target for brute-force password attacks and exploitation of known vulnerabilities (e.g., BlueKeep). Successful exploitation provides an attacker with direct, interactive access to the internal network.

% ----------------------------------------------------------------------
% 5. CORRELATED RISK ASSESSMENT
% ----------------------------------------------------------------------
\section{Correlated Risk Assessment}

This section synthesizes findings from the security control review, the technical scan, and pre-existing risk documentation into a consolidated list of key risks.

\begin{table}[h!]
\centering
\caption{Summary of Identified Risks}
\label{tab:risks}
\begin{tabular}{p{0.2\linewidth} p{0.5\linewidth} p{0.15\linewidth}}
\toprule
\textbf{Risk Name} & \textbf{Description} & \textbf{Severity} \\
\midrule
\textbf{Public RDP Exposure} & The network scan confirmed that the Remote Desktop Protocol (RDP) service on \texttt{[Target IP]} is exposed to the public internet. This aligns with the pre-existing documented risk and presents an immediate and severe threat of unauthorized access. & \textcolor{criticalred}{\textbf{Critical (9.0)}} \\
\addlinespace
\textbf{Lack of New Employee Security Training} & The organization does not provide security awareness training during the employee onboarding process. This administrative gap creates a persistent vulnerability, as untrained employees are more likely to fall victim to social engineering attacks, mishandle data, or use weak credentials. & \textcolor{highorange}{\textbf{High}} \\
\bottomrule
\end{tabular}
\end{table}

% ----------------------------------------------------------------------
% 6. RECOMMENDATIONS
% ----------------------------------------------------------------------
\section{Recommendations}

The following actions are recommended to mitigate the identified risks. Recommendations are prioritized based on severity.

\subsection{Remediation for Public RDP Exposure (Immediate Priority)}
This critical risk requires immediate attention to prevent a potential network compromise.

\begin{itemize}
    \item \textbf{Immediate Action:} Block all inbound traffic to TCP port 3389 on the external firewall for the host at \texttt{[Client IP]}. This will immediately remove the public exposure.
    \item \textbf{Long-Term Solution:} Decommission public-facing RDP entirely. Implement a secure remote access solution, such as a Virtual Private Network (VPN) with multi-factor authentication (MFA) or a Zero Trust Network Access (ZTNA) gateway. This provides authenticated and encrypted access for authorized users without exposing services directly to the internet.
\end{itemize}

\subsection{Remediation for Security Training Gap (High Priority)}
This process gap should be addressed to strengthen the human element of the organization's security posture.

\begin{itemize}
    \item \textbf{Immediate Action:} Develop and mandate a security awareness training module as a required step in the new employee onboarding process. This training should be completed before the employee is granted access to sensitive systems.
    \item \textbf{Content Suggestions:} The training should cover, at a minimum:
    \begin{itemize}
        \item Phishing and social engineering awareness.
        \item Password security and MFA policies.
        \item The organization's Acceptable Use Policy.
        \item Procedures for reporting security incidents.
    \end{itemize}
\end{itemize}

% ----------------------------------------------------------------------
% 7. CONCLUSION
% ----------------------------------------------------------------------
\section{Conclusion}

The assessment for \textbf{[Organization Name]} has identified one critical technical vulnerability and one high-risk administrative gap. The public exposure of the Remote Desktop Protocol service poses an immediate and substantial threat that must be remediated without delay.

By implementing the recommendations outlined in this report—namely, securing remote access infrastructure and enhancing the employee onboarding process with security training—the organization can significantly improve its security posture and reduce its risk of a cyber incident.

% ----------------------------------------------------------------------
% DOCUMENT END
% ----------------------------------------------------------------------
\end{document}
```