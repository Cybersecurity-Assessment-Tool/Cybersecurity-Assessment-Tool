```latex
\documentclass[12pt]{article}

% Preamble: Required Packages
\usepackage[margin=1in]{geometry}
\usepackage{pifont} % For checkmarks and crosses
\usepackage{booktabs} % For professional tables
\usepackage{hyperref} % For clickable links and references
\usepackage{url} % For formatting URLs
\usepackage{seqsplit} % For splitting long strings in tt font
\usepackage{graphicx}
\usepackage{xcolor}

% Hyperref Setup
\hypersetup{
    colorlinks=true,
    linkcolor=black,
    filecolor=magenta,      
    urlcolor=blue,
    pdftitle={Cybersecurity Assessment Report},
    pdfpagemode=FullScreen,
}

% Document Title and Author
\title{Cybersecurity Assessment Report \\ \large For \textbf{[Organization Name]}}
\author{Cybersecurity Analyst}
\date{November 22, 2025}

\begin{document}

\maketitle
\thispagestyle{empty}
\newpage

\tableofcontents
\newpage

% --- 1. Executive Overview ---
\section{Executive Overview}
This report details the findings of a cybersecurity assessment conducted on November 22, 2025. The assessment combined a technical network scan, a review of organizational security controls, and an analysis of pre-existing risks to evaluate the overall security posture of \textbf{[Organization Name]}.

The organization has implemented several foundational security controls, including Multi-Factor Authentication (MFA) for email and computer access. However, the assessment identified critical deficiencies that expose the organization to significant risk. 

Key findings include:
\begin{itemize}
    \item \textbf{Critical Control Gap:} Multi-Factor Authentication is not enforced for accessing sensitive data systems. This represents a severe risk, as a single compromised credential could lead to a major data breach.
    \item \textbf{High-Risk Technical Vulnerability:} The external-facing web server is running an outdated version of Nginx (1.18.0), which is known to have multiple security vulnerabilities.
    \item \textbf{High-Risk Procedural Gap:} The organization does not provide mandatory, recurring security awareness training for all employees, increasing susceptibility to phishing and social engineering attacks.
\end{itemize}

Immediate remediation of these issues is strongly recommended to reduce the risk of unauthorized access, data exfiltration, and service disruption.

% --- 2. Organizational Information ---
\section{Organizational Information}
The following information was used as the basis for this assessment. Due to the anonymized nature of the provided data, placeholders have been used where necessary.

\begin{itemize}
    \item \textbf{Organization Name:} \textbf{[Organization Name]}
    \item \textbf{Primary Domain:} \texttt{[Domain]}
    \item \textbf{External IP Scanned:} \texttt{[Client IP]}
\end{itemize}

% --- 3. Security Control Review ---
\section{Security Control Review (Questionnaire)}
An administrative and procedural control review was conducted based on a security questionnaire. The responses indicate gaps in critical areas of the security program. A "No" response (\ding{55}) signifies a deviation from security best practices and a potential risk.

\begin{table}[h!]
\centering
\caption{Security Control Questionnaire Responses}
\begin{tabular}{p{0.8\textwidth} c}
\toprule
\textbf{Control Question} & \textbf{Response} \\
\midrule
Do you require MFA to access email? & \ding{51} \\
Do you require MFA to log into computers? & \ding{51} \\
\textbf{Do you require MFA to access sensitive data systems?} & \textcolor{red}{\ding{55}} \\
Does your organization have an employee acceptable use policy? & \ding{51} \\
Does your organization do security awareness training for new employees? & \ding{51} \\
\textbf{Does your organization do security awareness training for all employees at least once per year?} & \textcolor{red}{\ding{55}} \\
\bottomrule
\end{tabular}
\end{table}

% --- 4. Technical Scan Results ---
\section{Technical Scan Results}
An external network scan was performed to identify open ports and exposed services.

\begin{itemize}
    \item \textbf{Target IP:} \texttt{[Target IP]}
    \item \textbf{Scan Date:} November 22, 2025
\end{itemize}

The following open ports were discovered:

\begin{table}[h!]
\centering
\caption{Open Port Analysis}
\begin{tabular}{l l l l}
\toprule
\textbf{Port} & \textbf{State} & \textbf{Service} & \textbf{Product \& Version} \\
\midrule
443/tcp & Open & https & Nginx 1.18.0 \\
\bottomrule
\end{tabular}
\end{table}

\subsection*{Analysis}
The scan identified a single open port (443/TCP) running an Nginx web server, version 1.18.0. This version was released in April 2020 and is now considered outdated. It is missing years of critical security patches and is vulnerable to numerous publicly disclosed exploits (Common Vulnerabilities and Exposures - CVEs). An unpatched, internet-facing web server presents a high-risk entry point for attackers.

% --- 5. Risk Assessment Summary ---
\section{Risk Assessment Summary}
The following table synthesizes findings from the security control review and technical scan into a prioritized list of risks. No pre-existing vulnerabilities were reported in the input data.

\begin{table}[h!]
\centering
\caption{Identified Risks}
\begin{tabular}{p{0.1\textwidth} p{0.25\textwidth} p{0.4\textwidth} p{0.1\textwidth}}
\toprule
\textbf{Risk ID} & \textbf{Risk Name} & \textbf{Description} & \textbf{Severity} \\
\midrule
RISK-001 & Lack of MFA on Sensitive Systems & The absence of MFA on systems containing sensitive data allows an attacker with compromised credentials to gain direct access to the organization's most valuable assets. & \textbf{Critical} \\
\addlinespace
RISK-002 & Outdated Web Server Software & The public-facing Nginx 1.18.0 server is unpatched against numerous known vulnerabilities, which could be exploited to achieve remote code execution or service disruption. & \textbf{High} \\
\addlinespace
RISK-003 & Inadequate Security Awareness Training & Without mandatory annual training, employees are more likely to fall victim to phishing or social engineering, which is a primary vector for initial network compromise. & \textbf{High} \\
\bottomrule
\end{tabular}
\end{table}

% --- 6. Recommendations ---
\section{Recommendations}
Based on the risks identified in this assessment, the following prioritized actions are recommended to improve the security posture of \textbf{[Organization Name]}.

\begin{enumerate}
    \item \textbf{[Critical] Implement MFA on Sensitive Systems (RISK-001):}
    Immediately prioritize and enforce MFA for all user accounts (including administrative and service accounts) that have access to sensitive data repositories, databases, and critical applications. This is the single most effective control to mitigate the risk of a credential-based breach.

    \item \textbf{[High] Upgrade Web Server Software (RISK-002):}
    Plan and execute an upgrade of the Nginx web server from version 1.18.0 to the latest stable version. Establish a patch management policy to ensure all internet-facing systems are updated in a timely manner to protect against newly discovered vulnerabilities.

    \item \textbf{[High] Establish an Annual Security Training Program (RISK-003):}
    Develop and implement a mandatory security awareness training program for all employees, to be completed annually. The training should cover topics such as phishing identification, password hygiene, and acceptable use policies to create a more resilient human firewall.
\end{enumerate}

\end{document}
```